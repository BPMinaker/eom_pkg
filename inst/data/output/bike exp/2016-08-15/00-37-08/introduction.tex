\chapter{Introduction}
Replace this text with the introduction of your report.
\section{Geometry Diagram}
 The system geometry is shown in the following diagram.
\begin{figure}[hbtp]
\begin{center}
% Sketch output, version 0.3 (build 7d, Tue May 28 07:32:40 2013)
% Output language: PGF/TikZ,LaTeX
\begin{tikzpicture}[line join=round]
\filldraw[fill opacity=0.5,fill=gray!20](1.447,1.918)--(1.259,1.952)--(1.104,1.526)--(1.272,1.438)--cycle;
\filldraw[fill opacity=0.5,fill=gray!20](1.58,1.88)--(1.447,1.918)--(1.272,1.438)--(1.397,1.378)--cycle;
\filldraw[fill opacity=0.5,fill=gray!20](1.699,1.843)--(1.58,1.88)--(1.397,1.378)--(1.509,1.324)--cycle;
\filldraw[fill opacity=0.5,fill=gray!20](1.799,1.809)--(1.699,1.843)--(1.509,1.324)--(1.606,1.279)--cycle;
\filldraw[fill opacity=0.5,fill=gray!20](1.878,1.777)--(1.799,1.809)--(1.606,1.279)--(1.684,1.244)--cycle;
\filldraw[fill opacity=0.5,fill=gray!20](1.932,1.75)--(1.878,1.777)--(1.684,1.244)--(1.739,1.221)--cycle;
\filldraw[fill opacity=0.5,fill=gray!20](1.96,1.729)--(1.932,1.75)--(1.739,1.221)--(1.771,1.21)--cycle;
\filldraw[fill opacity=0.5,fill=gray!20](1.933,1.706)--(1.447,1.918)--(1.272,1.438)--(1.758,1.226)--cycle;
\filldraw[fill opacity=0.5,fill=gray!20](1.272,1.438)--(1.104,1.526)--(.857,1.147)--(.994,1.012)--cycle;
\filldraw[fill opacity=0.5,fill=gray!20](1.397,1.378)--(1.272,1.438)--(.994,1.012)--(1.106,.932)--cycle;
\filldraw[fill opacity=0.5,fill=gray!20](1.509,1.324)--(1.397,1.378)--(1.106,.932)--(1.208,.863)--cycle;
\filldraw[fill opacity=0.5,fill=gray!20](1.606,1.279)--(1.509,1.324)--(1.208,.863)--(1.299,.809)--cycle;
\filldraw[fill opacity=0.5,fill=gray!20](1.684,1.244)--(1.606,1.279)--(1.299,.809)--(1.375,.771)--cycle;
\filldraw[fill opacity=0.5,fill=gray!20](1.739,1.221)--(1.684,1.244)--(1.375,.771)--(1.433,.751)--cycle;
\filldraw[fill opacity=0.5,fill=gray!20](1.771,1.21)--(1.739,1.221)--(1.433,.751)--(1.47,.749)--cycle;
\filldraw[fill opacity=0.5,fill=gray!20](1.758,1.226)--(1.272,1.438)--(.994,1.012)--(1.48,.8)--cycle;
\filldraw[fill opacity=0.5,fill=gray!20](1.507,2.418)--(1.312,2.397)--(1.259,1.952)--(1.447,1.918)--cycle;
\filldraw[fill opacity=0.5,fill=gray!20](1.642,2.404)--(1.507,2.418)--(1.447,1.918)--(1.58,1.88)--cycle;
\filldraw[fill opacity=0.5,fill=gray!20](1.763,2.385)--(1.642,2.404)--(1.58,1.88)--(1.699,1.843)--cycle;
\filldraw[fill opacity=0.5,fill=gray!20](1.865,2.361)--(1.763,2.385)--(1.699,1.843)--(1.799,1.809)--cycle;
\filldraw[fill opacity=0.5,fill=gray!20](1.944,2.334)--(1.865,2.361)--(1.799,1.809)--(1.878,1.777)--cycle;
\filldraw[fill opacity=0.5,fill=gray!20](1.998,2.303)--(1.944,2.334)--(1.878,1.777)--(1.932,1.75)--cycle;
\filldraw[fill opacity=0.5,fill=gray!20](2.025,2.271)--(1.998,2.303)--(1.932,1.75)--(1.96,1.729)--cycle;
\filldraw[fill opacity=0.5,fill=gray!20](1.993,2.206)--(1.507,2.418)--(1.447,1.918)--(1.933,1.706)--cycle;
\filldraw[fill opacity=0.5,fill=gray!20](.994,1.012)--(.857,1.147)--(.535,.843)--(.632,.669)--cycle;
\filldraw[fill opacity=0.5,fill=gray!20](1.106,.932)--(.994,1.012)--(.632,.669)--(.726,.573)--cycle;
\filldraw[fill opacity=0.5,fill=gray!20](1.208,.863)--(1.106,.932)--(.726,.573)--(.816,.492)--cycle;
\filldraw[fill opacity=0.5,fill=gray!20](1.299,.809)--(1.208,.863)--(.816,.492)--(.899,.431)--cycle;
\filldraw[fill opacity=0.5,fill=gray!20](1.375,.771)--(1.299,.809)--(.899,.431)--(.972,.39)--cycle;
\filldraw[fill opacity=0.5,fill=gray!20](1.433,.751)--(1.375,.771)--(.972,.39)--(1.032,.373)--cycle;
\filldraw[fill opacity=0.5,fill=gray!20](1.48,.8)--(.994,1.012)--(.632,.669)--(1.118,.457)--cycle;
\filldraw[fill opacity=0.5,fill=gray!20](1.447,2.906)--(1.259,2.831)--(1.312,2.397)--(1.507,2.418)--cycle;
\filldraw[fill opacity=0.5,fill=gray!20](1.58,2.915)--(1.447,2.906)--(1.507,2.418)--(1.642,2.404)--cycle;
\filldraw[fill opacity=0.5,fill=gray!20](1.699,2.913)--(1.58,2.915)--(1.642,2.404)--(1.763,2.385)--cycle;
\filldraw[fill opacity=0.5,fill=gray!20](1.799,2.9)--(1.699,2.913)--(1.763,2.385)--(1.865,2.361)--cycle;
\filldraw[fill opacity=0.5,fill=gray!20](1.878,2.875)--(1.799,2.9)--(1.865,2.361)--(1.944,2.334)--cycle;
\filldraw[fill opacity=0.5,fill=gray!20](1.932,2.841)--(1.878,2.875)--(1.944,2.334)--(1.998,2.303)--cycle;
\filldraw[fill opacity=0.5,fill=gray!20](1.933,2.694)--(1.447,2.906)--(1.507,2.418)--(1.993,2.206)--cycle;
\filldraw[fill opacity=0.5,fill=gray!20](1.259,1.952)--(1.442,1.873)--(1.286,1.446)--(1.104,1.526)--cycle;
\filldraw[fill opacity=0.5,fill=gray!20](1.104,1.526)--(1.286,1.446)--(1.039,1.068)--(.857,1.147)--cycle;
\filldraw[fill opacity=0.5,fill=gray!20](1.312,2.397)--(1.495,2.318)--(1.442,1.873)--(1.259,1.952)--cycle;
\filldraw[fill opacity=0.5,fill=gray!20](.857,1.147)--(1.039,1.068)--(.717,.763)--(.535,.843)--cycle;
\filldraw[fill opacity=0.5,fill=gray!20](.632,.669)--(.535,.843)--(.16,.632)--(.21,.433)--cycle;
\filldraw[fill opacity=0.5,fill=gray!20](.726,.573)--(.632,.669)--(.21,.433)--(.284,.325)--cycle;
\filldraw[fill opacity=0.5,fill=gray!20](.816,.492)--(.726,.573)--(.284,.325)--(.359,.236)--cycle;
\filldraw[fill opacity=0.5,fill=gray!20](.899,.431)--(.816,.492)--(.359,.236)--(.433,.169)--cycle;
\filldraw[fill opacity=0.5,fill=gray!20](.972,.39)--(.899,.431)--(.433,.169)--(.503,.127)--cycle;
\filldraw[fill opacity=0.5,fill=gray!20](1.032,.373)--(.972,.39)--(.503,.127)--(.567,.111)--cycle;
\filldraw[fill opacity=0.5,fill=gray!20](1.118,.457)--(.632,.669)--(.21,.433)--(.696,.221)--cycle;
\filldraw[fill opacity=0.5,fill=gray!20](1.259,2.831)--(1.442,2.751)--(1.495,2.318)--(1.312,2.397)--cycle;
\filldraw[fill opacity=0.5,fill=gray!20](1.272,3.347)--(1.104,3.223)--(1.259,2.831)--(1.447,2.906)--cycle;
\filldraw[fill opacity=0.5,fill=gray!20](1.397,3.377)--(1.272,3.348)--(1.447,2.906)--(1.58,2.915)--cycle;
\filldraw[fill opacity=0.5,fill=gray!20](1.509,3.391)--(1.397,3.377)--(1.58,2.915)--(1.699,2.913)--cycle;
\filldraw[fill opacity=0.5,fill=gray!20](1.606,3.387)--(1.509,3.391)--(1.699,2.913)--(1.799,2.9)--cycle;
\filldraw[fill opacity=0.5,fill=gray!20](1.684,3.366)--(1.606,3.387)--(1.799,2.9)--(1.878,2.875)--cycle;
\filldraw[fill opacity=0.5,fill=gray!20](1.758,3.135)--(1.272,3.347)--(1.447,2.906)--(1.933,2.694)--cycle;
\filldraw[fill opacity=0.5,fill=gray!20](.535,.843)--(.717,.763)--(.342,.553)--(.16,.632)--cycle;
\filldraw[fill opacity=0.5,fill=gray!20](1.104,3.223)--(1.286,3.143)--(1.442,2.751)--(1.259,2.831)--cycle;
\filldraw[fill opacity=0.5,fill=gray!20](.21,.433)--(.16,.632)--(-.243,.53)--(-.243,.318)--cycle;
\filldraw[fill opacity=0.5,fill=gray!20](.284,.325)--(.21,.433)--(-.243,.318)--(-.19,.205)--cycle;
\filldraw[fill opacity=0.5,fill=gray!20](.359,.236)--(.284,.325)--(-.19,.205)--(-.131,.112)--cycle;
\filldraw[fill opacity=0.5,fill=gray!20,draw=none](.042,.091)--(.416,.185)--(.359,.236)--(-.013,.142)--cycle;
\draw(.416,.185)--(.359,.236)--(-.013,.142);
\filldraw[fill opacity=0.5,fill=gray!20,draw=none](-.295,.118)--(-.131,.112)--(-.19,.205)--(-.354,.211)--cycle;
\draw(-.295,.118)--(-.131,.112)--(-.19,.205)--(-.354,.211);
\filldraw[fill opacity=0.5,fill=gray!20,draw=none](.042,.091)--(-.013,.142)--(-.131,.112)--(-.082,.059)--cycle;
\draw(-.013,.142)--(-.131,.112)--(-.082,.059);
\filldraw[fill opacity=0.8,fill=gray!20,draw=none](-.209,.04)--(-.205,.064)--(.048,.055)--(.049,.05)--(.046,.031)--cycle;
\draw(-.209,.04)--(-.205,.064);
\draw(.048,.055)--(.049,.05)--(.046,.031);
\filldraw[fill opacity=0.5,fill=gray!20,draw=none](.097,.084)--(.433,.169)--(.416,.185)--(.082,.101)--cycle;
\draw(.097,.084)--(.433,.169)--(.416,.185);
\filldraw[fill opacity=0.8,fill=gray!20,draw=none](-.209,.04)--(.046,.031)--(.039,-.006)--(.011,-.055)--(-.032,-.091)--(-.082,-.107)--(-.132,-.102)--(-.174,-.075)--(-.203,-.032)--(-.213,.021)--cycle;
\draw(.046,.031)--(.039,-.006)--(.011,-.055)--(-.032,-.091)--(-.082,-.107)--(-.132,-.102)--(-.174,-.075)--(-.203,-.032)--(-.213,.021)--(-.209,.04);
\filldraw[fill opacity=0.5,fill=gray!20,draw=none](.102,.081)--(.443,.164)--(.433,.169)--(.097,.084)--cycle;
\draw(.443,.164)--(.433,.169)--(.097,.084);
\filldraw[fill opacity=0.5,fill=gray!20,draw=none](.102,.081)--(.164,.041)--(.503,.127)--(.443,.164)--cycle;
\draw(.164,.041)--(.503,.127)--(.443,.164);
\filldraw[fill opacity=0.5,fill=gray!20,draw=none](.226,.027)--(.558,.113)--(.503,.127)--(.164,.041)--cycle;
\draw(.558,.113)--(.503,.127)--(.164,.041);
\filldraw[fill opacity=0.8,fill=gray!20,draw=none](.099,.056)--(.103,.099)--(.117,.139)--(.139,.161)--(.164,.161)--(.189,.139)--(.21,.099)--(.224,.046)--(.226,.027)--cycle;
\draw(.099,.056)--(.103,.099)--(.117,.139)--(.139,.161)--(.164,.161)--(.189,.139)--(.21,.099)--(.224,.046)--(.226,.027);
\filldraw[fill opacity=0.8,fill=gray!20,draw=none](.125,-.077)--(.117,-.063)--(.103,-.01)--(.102,0)--(.226,.025)--(.229,-.006)--cycle;
\draw(.125,-.077)--(.117,-.063)--(.103,-.01)--(.102,0);
\draw(.226,.025)--(.229,-.006);
\filldraw[fill opacity=0.8,fill=gray!20,draw=none](.051,-.011)--(.039,-.006)--(.049,.05)--(.065,.043)--cycle;
\draw(.051,-.011)--(.039,-.006)--(.049,.05)--(.065,.043);
\filldraw[fill opacity=0.8,fill=gray!20,draw=none](.036,-.066)--(.011,-.055)--(.039,-.006)--(.06,-.015)--cycle;
\draw(.036,-.066)--(.011,-.055)--(.039,-.006)--(.06,-.015);
\filldraw[fill opacity=0.5,fill=gray!20,draw=none](.059,.075)--(.097,.084)--(.082,.101)--(.042,.091)--cycle;
\draw(.059,.075)--(.097,.084);
\filldraw[fill opacity=0.8,fill=gray!20,draw=none](.057,.076)--(.065,.043)--(.049,.05)--(.043,.084)--cycle;
\draw(.065,.043)--(.049,.05)--(.043,.084);
\filldraw[fill opacity=0.8,fill=gray!20,draw=none](.061,-.01)--(.053,-.012)--(.051,-.011)--(.065,.043)--cycle;
\draw(.053,-.012)--(.051,-.011);
\filldraw[fill opacity=0.5,fill=gray!20,draw=none](-.192,.047)--(-.132,.045)--(-.116,.096)--(-.131,.112)--(-.249,.116)--cycle;
\draw(-.192,.047)--(-.132,.045);
\draw(-.116,.096)--(-.131,.112)--(-.249,.116);
\filldraw[fill opacity=0.8,fill=gray!20,draw=none](-.068,.059)--(-.068,.177)--(-.032,.173)--(.011,.147)--(.039,.104)--(.048,.055)--cycle;
\draw(-.068,.177)--(-.032,.173)--(.011,.147)--(.039,.104)--(.048,.055);
\filldraw[fill opacity=0.8,fill=gray!20,draw=none](.057,.076)--(.043,.084)--(.039,.104)--(.051,.099)--cycle;
\draw(.043,.084)--(.039,.104)--(.051,.099);
\filldraw[fill opacity=0.8,fill=gray!20,draw=none](.065,.043)--(.098,.046)--(.103,-.01)--(.06,-.015)--cycle;
\draw(.065,.043)--(.098,.046)--(.103,-.01)--(.06,-.015);
\filldraw[fill opacity=0.8,fill=gray!20,draw=none](.102,0)--(.098,.046)--(.099,.056)--(.226,.027)--(.226,.025)--cycle;
\draw(.102,0)--(.098,.046)--(.099,.056);
\draw(.226,.027)--(.226,.025);
\filldraw[fill opacity=0.8,fill=gray!20,draw=none](.063,.072)--(.065,.043)--(.057,.076)--cycle;
\filldraw[fill opacity=0.5,fill=gray!20,draw=none](-.132,.045)--(-.098,.044)--(-.077,.054)--(-.116,.096)--cycle;
\draw(-.132,.045)--(-.098,.044);
\draw(-.077,.054)--(-.116,.096);
\filldraw[fill opacity=0.8,fill=gray!20,draw=none](-.068,.059)--(-.205,.064)--(-.203,.077)--(-.174,.127)--(-.132,.162)--(-.082,.179)--(-.068,.177)--cycle;
\draw(-.205,.064)--(-.203,.077)--(-.174,.127)--(-.132,.162)--(-.082,.179)--(-.068,.177);
\filldraw[fill opacity=0.5,fill=gray!20,draw=none](-.055,.046)--(.046,.086)--(.042,.091)--(-.082,.059)--(-.067,.043)--cycle;
\draw(-.082,.059)--(-.067,.043)--(-.055,.046);
\filldraw[fill opacity=0.8,fill=gray!20,draw=none](.107,.144)--(.116,.158)--(.139,.161)--(.122,.144)--cycle;
\draw(.116,.158)--(.139,.161)--(.122,.144);
\filldraw[fill opacity=0.8,fill=gray!20,draw=none](.072,.074)--(.102,.081)--(.098,.046)--(.089,.045)--cycle;
\draw(.102,.081)--(.098,.046)--(.089,.045);
\filldraw[fill opacity=0.8,fill=gray!20,draw=none](.07,-.042)--(.096,-.011)--(.103,-.01)--(.114,-.052)--cycle;
\draw(.096,-.011)--(.103,-.01)--(.114,-.052);
\filldraw[fill opacity=0.8,fill=gray!20,draw=none](.035,.209)--(.016,.159)--(-.01,.164)--(-.01,.21)--cycle;
\draw(-.01,.164)--(-.01,.21)--(.035,.209);
\filldraw[fill opacity=0.5,fill=gray!20,draw=none](.125,.04)--(.134,.034)--(.164,.041)--(.097,.084)--(.091,.083)--cycle;
\draw(.134,.034)--(.164,.041);
\draw(.097,.084)--(.091,.083);
\filldraw[fill opacity=0.5,fill=gray!20,draw=none](.125,.04)--(.091,.083)--(.07,.078)--cycle;
\draw(.091,.083)--(.07,.078);
\filldraw[fill opacity=0.5,fill=gray!20,draw=none](.063,.072)--(.074,.075)--(.07,.078)--(.059,.075)--cycle;
\draw(.07,.078)--(.059,.075);
\filldraw[fill opacity=0.8,fill=gray!20,draw=none](.072,.074)--(.06,.094)--(.103,.099)--(.102,.081)--cycle;
\draw(.06,.094)--(.103,.099)--(.102,.081);
\filldraw[fill opacity=0.8,fill=gray!20,draw=none](.026,.157)--(.019,.166)--(.035,.209)--(.046,.209)--(.046,.153)--cycle;
\draw(.035,.209)--(.046,.209)--(.046,.153);
\filldraw[fill opacity=0.8,fill=gray!20,draw=none](.056,.208)--(.07,.162)--(.064,.154)--(.046,.153)--(.046,.209)--cycle;
\draw(.046,.153)--(.046,.209)--(.056,.208);
\filldraw[fill opacity=0.8,fill=gray!20,draw=none](-.01,-.1)--(-.032,-.091)--(.011,-.055)--(.036,-.066)--cycle;
\draw(-.01,-.1)--(-.032,-.091)--(.011,-.055)--(.036,-.066);
\filldraw[fill opacity=0.5,fill=gray!20,draw=none](-.055,.046)--(.059,.075)--(.046,.086)--cycle;
\draw(-.055,.046)--(.059,.075);
\filldraw[fill opacity=0.8,fill=gray!20,draw=none](-.065,-.114)--(-.082,-.107)--(-.032,-.091)--(-.01,-.1)--cycle;
\draw(-.065,-.114)--(-.082,-.107)--(-.032,-.091)--(-.01,-.1);
\filldraw[fill opacity=0.8,fill=gray!20,draw=none](-.01,.21)--(-.01,.164)--(-.065,.197)--(-.065,.207)--cycle;
\draw(-.065,.197)--(-.065,.207)--(-.01,.21)--(-.01,.164);
\filldraw[fill opacity=0.8,fill=gray!20,draw=none](-.004,.205)--(.035,.209)--(-.01,.21)--(-.016,.21)--cycle;
\draw(.035,.209)--(-.01,.21)--(-.016,.21);
\filldraw[fill opacity=0.8,fill=gray!20,draw=none](.06,.094)--(.039,.104)--(.011,.147)--(.046,.131)--cycle;
\draw(.06,.094)--(.039,.104)--(.011,.147)--(.046,.131);
\filldraw[fill opacity=0.8,fill=gray!20,draw=none](.141,.161)--(.139,.161)--(.131,.16)--cycle;
\draw(.141,.161)--(.139,.161)--(.131,.16);
\filldraw[fill opacity=0.8,fill=gray!20,draw=none](.07,.162)--(.056,.208)--(.096,.203)--(.096,.195)--cycle;
\draw(.056,.208)--(.096,.203)--(.096,.195);
\filldraw[fill opacity=0.8,fill=gray!20,draw=none](.099,.202)--(.096,.195)--(.096,.203)--cycle;
\draw(.096,.195)--(.096,.203)--(.099,.202);
\filldraw[fill opacity=0.8,fill=gray!20,draw=none](-.063,.199)--(-.036,.187)--(.099,.202)--(.096,.203)--(.046,.209)--(.035,.209)--cycle;
\draw(.099,.202)--(.096,.203)--(.046,.209)--(.035,.209);
\filldraw[fill opacity=0.8,fill=gray!20,draw=none](.066,.133)--(.117,.139)--(.103,.099)--(.06,.094)--cycle;
\draw(.066,.133)--(.117,.139)--(.103,.099)--(.06,.094);
\filldraw[fill opacity=0.8,fill=gray!20,draw=none](.061,.15)--(.046,.131)--(.046,.153)--cycle;
\draw(.046,.131)--(.046,.153);
\filldraw[fill opacity=0.8,fill=gray!20,draw=none](.075,.146)--(.122,.144)--(.117,.139)--(.046,.131)--cycle;
\draw(.122,.144)--(.117,.139)--(.046,.131);
\filldraw[fill opacity=0.8,fill=gray!20,draw=none](.107,.144)--(.082,.146)--(.096,.156)--(.116,.158)--cycle;
\draw(.096,.156)--(.116,.158);
\filldraw[fill opacity=0.8,fill=gray!20,draw=none](.131,.157)--(.131,.053)--(.096,.079)--(.096,.156)--cycle;
\draw(.131,.157)--(.131,.053);
\draw(.096,.079)--(.096,.156);
\filldraw[fill opacity=0.8,fill=gray!20,draw=none](.082,.146)--(.075,.146)--(.096,.156)--cycle;
\filldraw[fill opacity=0.8,fill=gray!20,draw=none](.063,.072)--(.057,.076)--(.051,.099)--(.06,.094)--cycle;
\draw(.051,.099)--(.06,.094);
\filldraw[fill opacity=0.8,fill=gray!20,draw=none](-.05,.045)--(.072,.074)--(.089,.045)--(-.065,.029)--cycle;
\draw(.089,.045)--(-.065,.029);
\filldraw[fill opacity=0.5,fill=gray!20,draw=none](-.05,.045)--(.063,.072)--(.059,.075)--(-.055,.046)--(-.065,.042)--cycle;
\draw(.059,.075)--(-.055,.046);
\filldraw[fill opacity=0.8,fill=gray!20,draw=none](.013,.15)--(.019,.166)--(.046,.131)--cycle;
\filldraw[fill opacity=0.8,fill=gray!20,draw=none](.123,.158)--(.154,.16)--(.157,.161)--(.141,.161)--(.131,.16)--(.116,.158)--cycle;
\draw(.157,.161)--(.141,.161);
\draw(.131,.16)--(.116,.158);
\filldraw[fill opacity=0.8,fill=gray!20,draw=none](.073,.155)--(.07,.162)--(.096,.195)--(.096,.156)--cycle;
\draw(.096,.195)--(.096,.156);
\filldraw[fill opacity=0.8,fill=gray!20,draw=none](.046,.131)--(.066,.133)--(.06,.094)--(-.01,.087)--cycle;
\draw(.046,.131)--(.066,.133);
\draw(.06,.094)--(-.01,.087);
\filldraw[fill opacity=0.8,fill=gray!20,draw=none](.07,.162)--(.096,.079)--(.046,.131)--cycle;
\filldraw[fill opacity=0.8,fill=gray!20,draw=none](.026,.157)--(.046,.153)--(.046,.131)--cycle;
\draw(.046,.153)--(.046,.131);
\filldraw[fill opacity=0.8,fill=gray!20,draw=none](.016,.159)--(.013,.15)--(-.01,.164)--cycle;
\filldraw[fill opacity=0.8,fill=gray!20,draw=none](.046,.131)--(.011,.147)--(-.032,.173)--(.013,.154)--cycle;
\draw(.046,.131)--(.011,.147)--(-.032,.173)--(.013,.154);
\filldraw[fill opacity=0.8,fill=gray!20,draw=none](.073,.155)--(.096,.156)--(.096,.079)--cycle;
\draw(.096,.156)--(.096,.079);
\filldraw[fill opacity=0.8,fill=gray!20,draw=none](.064,.154)--(.061,.15)--(.046,.153)--cycle;
\filldraw[fill opacity=0.8,fill=gray!20,draw=none](.075,.146)--(.048,.147)--(.049,.151)--(.096,.156)--cycle;
\draw(.049,.151)--(.096,.156);
\filldraw[fill opacity=0.8,fill=gray!20,draw=none](.105,.154)--(.131,.157)--(.123,.158)--(.096,.156)--(.082,.155)--cycle;
\draw(.105,.154)--(.131,.157);
\draw(.096,.156)--(.082,.155);
\filldraw[fill opacity=0.8,fill=gray!20,draw=none](.123,.158)--(.116,.158)--(.096,.156)--cycle;
\draw(.116,.158)--(.096,.156);
\filldraw[fill opacity=0.8,fill=gray!20,draw=none](.131,.157)--(.154,.16)--(.123,.158)--cycle;
\draw(.131,.157)--(.154,.16);
\filldraw[fill opacity=0.8,fill=gray!20,draw=none](.131,.183)--(.131,.157)--(.096,.156)--cycle;
\draw(.131,.183)--(.131,.157);
\filldraw[fill opacity=0.8,fill=gray!20,draw=none](.099,.202)--(.131,.193)--(.131,.183)--(.096,.156)--(.096,.195)--cycle;
\draw(.099,.202)--(.131,.193)--(.131,.183);
\draw(.096,.156)--(.096,.195);
\filldraw[fill opacity=0.8,fill=gray!20,draw=none](-.05,.045)--(-.01,.087)--(.06,.094)--(.072,.074)--cycle;
\draw(-.01,.087)--(.06,.094);
\filldraw[fill opacity=0.8,fill=gray!20,draw=none](.146,.134)--(.146,.057)--(.131,.053)--(.131,.157)--cycle;
\draw(.146,.134)--(.146,.057);
\draw(.131,.053)--(.131,.157);
\filldraw[fill opacity=0.8,fill=gray!20,draw=none](.048,.147)--(.075,.146)--(.046,.131)--cycle;
\filldraw[fill opacity=0.8,fill=gray!20,draw=none](.099,.129)--(.146,.134)--(.154,.16)--(.067,.15)--cycle;
\draw(.099,.129)--(.146,.134);
\draw(.154,.16)--(.067,.15);
\filldraw[fill opacity=0.8,fill=gray!20,draw=none](.146,.181)--(.146,.134)--(.131,.157)--(.131,.193)--cycle;
\draw(.131,.157)--(.131,.193)--(.146,.181)--(.146,.134);
\filldraw[fill opacity=0.8,fill=gray!20,draw=none](.14,.171)--(.146,.181)--(.131,.193)--(.099,.202)--(.076,.199)--cycle;
\draw(.14,.171)--(.146,.181)--(.131,.193)--(.099,.202);
\filldraw[fill opacity=0.8,fill=gray!20,draw=none](.154,.16)--(.164,.161)--(.157,.161)--cycle;
\draw(.154,.16)--(.164,.161)--(.157,.161);
\filldraw[fill opacity=0.8,fill=gray!20,draw=none](-.007,.174)--(.105,.186)--(.076,.199)--(-.036,.187)--cycle;
\filldraw[fill opacity=0.8,fill=gray!20,draw=none](.146,.134)--(.189,.139)--(.164,.161)--(.154,.16)--cycle;
\draw(.146,.134)--(.189,.139)--(.164,.161)--(.154,.16);
\filldraw[fill opacity=0.8,fill=gray!20,draw=none](.07,-.042)--(.055,-.038)--(.06,-.015)--(.096,-.011)--cycle;
\draw(.06,-.015)--(.096,-.011);
\filldraw[fill opacity=0.8,fill=gray!20,draw=none](.224,-.063)--(.21,-.103)--(.189,-.125)--(.164,-.125)--(.139,-.103)--(.125,-.077)--(.229,-.006)--(.229,-.011)--cycle;
\draw(.229,-.006)--(.229,-.011)--(.224,-.063)--(.21,-.103)--(.189,-.125)--(.164,-.125)--(.139,-.103)--(.125,-.077);
\filldraw[fill opacity=0.8,fill=gray!20,draw=none](.07,-.042)--(.114,-.052)--(.117,-.063)--(.046,-.071)--cycle;
\draw(.114,-.052)--(.117,-.063)--(.046,-.071);
\filldraw[fill opacity=0.8,fill=gray!20,draw=none](-.121,-.107)--(-.132,-.102)--(-.082,-.107)--(-.065,-.114)--cycle;
\draw(-.121,-.107)--(-.132,-.102)--(-.082,-.107)--(-.065,-.114);
\filldraw[fill opacity=0.8,fill=gray!20,draw=none](-.01,-.122)--(-.01,-.147)--(-.05,-.149)--(-.065,-.14)--(-.065,-.114)--cycle;
\draw(-.01,-.122)--(-.01,-.147)--(-.05,-.149);
\draw(-.065,-.14)--(-.065,-.114);
\filldraw[fill opacity=0.8,fill=gray!20,draw=none](.046,-.101)--(.046,-.148)--(-.01,-.147)--(-.01,-.121)--cycle;
\draw(.046,-.101)--(.046,-.148)--(-.01,-.147)--(-.01,-.121);
\filldraw[fill opacity=0.8,fill=gray!20,draw=none](.146,.057)--(.06,.094)--(.046,.131)--(.139,.091)--cycle;
\draw(.146,.057)--(.06,.094);
\draw(.046,.131)--(.139,.091);
\filldraw[fill opacity=0.8,fill=gray!20,draw=none](-.01,-.1)--(-.01,-.122)--(-.065,-.114)--cycle;
\draw(-.01,-.1)--(-.01,-.122);
\filldraw[fill opacity=0.8,fill=gray!20,draw=none](.131,-.104)--(.139,-.103)--(.141,-.105)--cycle;
\draw(.131,-.104)--(.139,-.103)--(.141,-.105);
\filldraw[fill opacity=0.8,fill=gray!20,draw=none](.046,-.071)--(.117,-.063)--(.139,-.103)--(.131,-.104)--cycle;
\draw(.046,-.071)--(.117,-.063)--(.139,-.103)--(.131,-.104);
\filldraw[fill opacity=0.8,fill=gray!20,draw=none](.046,-.071)--(.131,-.104)--(.096,-.108)--cycle;
\draw(.131,-.104)--(.096,-.108);
\filldraw[fill opacity=0.8,fill=gray!20,draw=none](.096,-.031)--(.075,-.092)--(.046,-.071)--cycle;
\filldraw[fill opacity=0.8,fill=gray!20,draw=none](-.01,-.1)--(.036,-.066)--(.046,-.071)--cycle;
\draw(.036,-.066)--(.046,-.071);
\filldraw[fill opacity=0.8,fill=gray!20,draw=none](.021,-.11)--(-.01,-.121)--(-.01,-.023)--cycle;
\draw(-.01,-.121)--(-.01,-.023);
\filldraw[fill opacity=0.8,fill=gray!20,draw=none](.073,-.101)--(.056,-.15)--(.046,-.148)--(.046,-.092)--cycle;
\draw(.056,-.15)--(.046,-.148)--(.046,-.092);
\filldraw[fill opacity=0.8,fill=gray!20,draw=none](-.018,-.135)--(-.065,-.114)--(-.01,-.1)--(.025,-.116)--cycle;
\draw(-.018,-.135)--(-.065,-.114);
\draw(-.01,-.1)--(.025,-.116);
\filldraw[fill opacity=0.8,fill=gray!20,draw=none](-.05,-.149)--(-.065,-.15)--(-.065,-.14)--cycle;
\draw(-.05,-.149)--(-.065,-.15)--(-.065,-.14);
\filldraw[fill opacity=0.8,fill=gray!20,draw=none](.096,-.108)--(.131,-.104)--(.141,-.105)--(.164,-.125)--(.154,-.126)--cycle;
\draw(.096,-.108)--(.131,-.104);
\draw(.141,-.105)--(.164,-.125)--(.154,-.126);
\filldraw[fill opacity=0.8,fill=gray!20,draw=none](.075,-.092)--(.096,-.031)--(.096,-.108)--cycle;
\draw(.096,-.031)--(.096,-.108);
\filldraw[fill opacity=0.8,fill=gray!20,draw=none](.075,-.092)--(.096,-.108)--(.096,-.154)--(.056,-.15)--cycle;
\draw(.096,-.108)--(.096,-.154)--(.056,-.15);
\filldraw[fill opacity=0.8,fill=gray!20,draw=none](-.09,-.124)--(-.065,-.114)--(-.065,-.14)--cycle;
\draw(-.065,-.114)--(-.065,-.14);
\filldraw[fill opacity=0.8,fill=gray!20](.01,-.21)--(-.046,-.209)--(-.096,-.203)--(-.131,-.193)--(-.146,-.181)--(-.139,-.168)--(-.111,-.158)--(-.065,-.15)--(-.01,-.147)--(.046,-.148)--(.096,-.154)--(.131,-.164)--(.146,-.176)--(.139,-.189)--(.111,-.2)--(.065,-.207)--cycle;
\filldraw[fill opacity=0.8,fill=gray!20,draw=none](-.065,-.114)--(-.09,-.124)--(-.111,-.111)--cycle;
\filldraw[fill opacity=0.8,fill=gray!20,draw=none](-.09,-.124)--(-.065,-.14)--(-.065,-.15)--(-.111,-.158)--(-.111,-.133)--cycle;
\draw(-.065,-.14)--(-.065,-.15)--(-.111,-.158)--(-.111,-.133);
\filldraw[fill opacity=0.8,fill=gray!20,draw=none](.096,-.108)--(.123,-.116)--(.131,-.129)--cycle;
\filldraw[fill opacity=0.8,fill=gray!20,draw=none](.123,-.116)--(.154,-.126)--(.131,-.129)--cycle;
\draw(.154,-.126)--(.131,-.129);
\filldraw[fill opacity=0.8,fill=gray!20,draw=none](.131,-.059)--(.131,-.129)--(.096,-.108)--(.096,-.056)--cycle;
\draw(.131,-.059)--(.131,-.129);
\draw(.096,-.108)--(.096,-.056);
\filldraw[fill opacity=0.8,fill=gray!20,draw=none](.122,.002)--(.131,.004)--(.131,-.059)--(.096,-.056)--(.096,-.031)--cycle;
\draw(.131,.004)--(.131,-.059);
\draw(.096,-.056)--(.096,-.031);
\filldraw[fill opacity=0.8,fill=gray!20,draw=none](.159,.093)--(.21,.099)--(.189,.139)--(.146,.134)--cycle;
\draw(.159,.093)--(.21,.099)--(.189,.139)--(.146,.134);
\filldraw[fill opacity=0.8,fill=gray!20,draw=none](.122,.002)--(.061,-.01)--(.065,.043)--(.131,.014)--cycle;
\draw(.065,.043)--(.131,.014);
\filldraw[fill opacity=0.8,fill=gray!20,draw=none](.118,.038)--(.131,.014)--(.065,.043)--(.063,.072)--cycle;
\draw(.131,.014)--(.065,.043);
\filldraw[fill opacity=0.8,fill=gray!20,draw=none](.096,-.031)--(.06,-.015)--(.06,-.011)--(.122,.002)--cycle;
\draw(.096,-.031)--(.06,-.015);
\filldraw[fill opacity=0.5,fill=gray!20,draw=none](.118,.038)--(.131,.033)--(.125,.04)--(.074,.075)--(.063,.072)--cycle;
\filldraw[fill opacity=0.8,fill=gray!20,draw=none](.118,.038)--(.063,.072)--(.06,.094)--(.096,.079)--cycle;
\draw(.06,.094)--(.096,.079);
\filldraw[fill opacity=0.5,fill=gray!20,draw=none](.118,.038)--(.063,.072)--(.043,.067)--cycle;
\filldraw[fill opacity=0.8,fill=gray!20,draw=none](.06,-.015)--(.053,-.012)--(.06,-.011)--cycle;
\draw(.06,-.015)--(.053,-.012);
\filldraw[fill opacity=0.5,fill=gray!20,draw=none](.055,.063)--(.043,.067)--(-.05,.045)--cycle;
\filldraw[fill opacity=0.8,fill=gray!20,draw=none](-.065,.029)--(.065,.043)--(.06,-.015)--(-.01,-.023)--cycle;
\draw(-.065,.029)--(.065,.043);
\draw(.06,-.015)--(-.01,-.023);
\filldraw[fill opacity=0.8,fill=gray!20,draw=none](-.01,.087)--(-.01,-.048)--(-.065,-.044)--(-.065,.029)--cycle;
\draw(-.01,.087)--(-.01,-.048);
\draw(-.065,-.044)--(-.065,.029);
\filldraw[fill opacity=0.5,fill=gray!20,draw=none](.096,.047)--(.055,.063)--(.019,.057)--cycle;
\filldraw[fill opacity=0.8,fill=gray!20,draw=none](.055,-.038)--(.06,-.015)--(.068,-.018)--cycle;
\draw(.06,-.015)--(.068,-.018);
\filldraw[fill opacity=0.8,fill=gray!20,draw=none](.055,-.038)--(.045,-.052)--(.045,-.048)--(.06,-.015)--cycle;
\filldraw[fill opacity=0.8,fill=gray!20,draw=none](.055,-.038)--(-.01,-.023)--(.06,-.015)--cycle;
\draw(-.01,-.023)--(.06,-.015);
\filldraw[fill opacity=0.8,fill=gray!20,draw=none](.046,-.071)--(.036,-.066)--(.045,-.048)--cycle;
\draw(.046,-.071)--(.036,-.066);
\filldraw[fill opacity=0.8,fill=gray!20,draw=none](.055,-.038)--(.07,-.042)--(.046,-.071)--cycle;
\filldraw[fill opacity=0.8,fill=gray!20,draw=none](.046,-.071)--(.045,-.052)--(.068,-.018)--(.084,-.025)--cycle;
\draw(.068,-.018)--(.084,-.025);
\filldraw[fill opacity=0.8,fill=gray!20,draw=none](-.01,-.048)--(-.01,-.1)--(-.065,-.114)--(-.065,-.044)--cycle;
\draw(-.01,-.048)--(-.01,-.1);
\draw(-.065,-.114)--(-.065,-.044);
\filldraw[fill opacity=0.8,fill=gray!20,draw=none](.046,-.071)--(.013,-.088)--(-.01,-.023)--cycle;
\filldraw[fill opacity=0.8,fill=gray!20,draw=none](-.111,-.111)--(-.095,-.11)--(-.065,-.114)--cycle;
\filldraw[fill opacity=0.8,fill=gray!20,draw=none](-.111,-.111)--(-.121,-.107)--(-.095,-.11)--cycle;
\draw(-.111,-.111)--(-.121,-.107);
\filldraw[fill opacity=0.8,fill=gray!20,draw=none](-.065,.029)--(-.065,-.114)--(-.111,-.111)--(-.111,-.034)--cycle;
\draw(-.065,.029)--(-.065,-.114);
\draw(-.111,-.111)--(-.111,-.034);
\filldraw[fill opacity=0.8,fill=gray!20,draw=none](-.111,-.034)--(-.111,-.111)--(-.122,-.103)--cycle;
\draw(-.111,-.034)--(-.111,-.111);
\filldraw[fill opacity=0.8,fill=gray!20,draw=none](-.196,.014)--(-.065,.029)--(-.01,-.023)--(-.181,-.041)--cycle;
\draw(-.196,.014)--(-.065,.029);
\draw(-.01,-.023)--(-.181,-.041);
\filldraw[fill opacity=0.8,fill=gray!20,draw=none](.013,.15)--(.046,.131)--(-.01,.087)--cycle;
\filldraw[fill opacity=0.8,fill=gray!20,draw=none](.013,.15)--(-.01,.087)--(-.01,.164)--cycle;
\draw(-.01,.087)--(-.01,.164);
\filldraw[fill opacity=0.8,fill=gray!20,draw=none](-.01,.164)--(-.01,.087)--(-.065,.067)--(-.065,.171)--cycle;
\draw(-.01,.164)--(-.01,.087);
\draw(-.065,.067)--(-.065,.171);
\filldraw[fill opacity=0.8,fill=gray!20,draw=none](-.065,.171)--(-.065,.067)--(-.111,.076)--(-.111,.153)--cycle;
\draw(-.065,.171)--(-.065,.067);
\draw(-.111,.076)--(-.111,.153);
\filldraw[fill opacity=0.8,fill=gray!20,draw=none](.025,.148)--(.048,.147)--(.046,.131)--cycle;
\filldraw[fill opacity=0.8,fill=gray!20,draw=none](.025,.148)--(.049,.151)--(.048,.147)--cycle;
\draw(.025,.148)--(.049,.151);
\filldraw[fill opacity=0.8,fill=gray!20,draw=none](.035,.147)--(.105,.154)--(.082,.155)--(.025,.148)--cycle;
\draw(.035,.147)--(.105,.154);
\draw(.082,.155)--(.025,.148);
\filldraw[fill opacity=0.8,fill=gray!20,draw=none](-.007,.174)--(.05,.149)--(.065,.15)--(.111,.158)--(.139,.168)--(.14,.171)--(.105,.186)--cycle;
\draw(.05,.149)--(.065,.15)--(.111,.158)--(.139,.168)--(.14,.171);
\filldraw[fill opacity=0.8,fill=gray!20,draw=none](.139,.091)--(.146,.083)--(.146,.134)--cycle;
\draw(.146,.083)--(.146,.134);
\filldraw[fill opacity=0.8,fill=gray!20,draw=none](.151,.104)--(.154,.108)--(.146,.134)--(.132,.133)--cycle;
\draw(.146,.134)--(.132,.133);
\filldraw[fill opacity=0.5,fill=gray!20,draw=none](-.098,.044)--(-.067,.043)--(-.077,.054)--cycle;
\draw(-.098,.044)--(-.067,.043)--(-.077,.054);
\filldraw[fill opacity=0.5,fill=gray!20,draw=none](-.055,.046)--(-.067,.043)--(-.065,.042)--cycle;
\draw(-.055,.046)--(-.067,.043)--(-.065,.042);
\filldraw[fill opacity=0.8,fill=gray!20,draw=none](-.01,.087)--(-.054,.041)--(-.065,.042)--(-.065,.067)--cycle;
\draw(-.065,.042)--(-.065,.067);
\filldraw[fill opacity=0.8,fill=gray!20,draw=none](-.054,.041)--(-.065,.029)--(-.065,.042)--cycle;
\draw(-.065,.029)--(-.065,.042);
\filldraw[fill opacity=0.8,fill=gray!20,draw=none](-.075,.039)--(-.065,.029)--(-.16,.018)--cycle;
\draw(-.065,.029)--(-.16,.018);
\filldraw[fill opacity=0.8,fill=gray!20,draw=none](-.065,.042)--(-.065,.029)--(-.078,.042)--cycle;
\draw(-.065,.042)--(-.065,.029);
\filldraw[fill opacity=0.8,fill=gray!20,draw=none](-.188,.045)--(-.064,.041)--(-.16,.018)--(-.196,.014)--cycle;
\draw(-.16,.018)--(-.196,.014);
\filldraw[fill opacity=0.8,fill=gray!20,draw=none](-.111,-.034)--(-.122,-.103)--(-.139,-.091)--cycle;
\filldraw[fill opacity=0.8,fill=gray!20,draw=none](-.111,-.034)--(-.01,-.023)--(.055,-.038)--(.046,-.071)--(-.139,-.091)--cycle;
\draw(-.111,-.034)--(-.01,-.023);
\draw(.046,-.071)--(-.139,-.091);
\filldraw[fill opacity=0.8,fill=gray!20,draw=none](.046,-.071)--(.084,-.025)--(.096,-.031)--cycle;
\draw(.084,-.025)--(.096,-.031);
\filldraw[fill opacity=0.8,fill=gray!20,draw=none](.096,-.006)--(.096,-.031)--(.046,-.071)--(.046,.018)--cycle;
\draw(.096,-.006)--(.096,-.031);
\draw(.046,-.071)--(.046,.018);
\filldraw[fill opacity=0.8,fill=gray!20,draw=none](.046,.053)--(.046,-.071)--(-.01,-.023)--(-.01,.052)--cycle;
\draw(.046,.053)--(.046,-.071);
\draw(-.01,-.023)--(-.01,.052);
\filldraw[fill opacity=0.8,fill=gray!20,draw=none](.096,.029)--(.096,-.006)--(.046,.018)--(.046,.044)--cycle;
\draw(.096,.029)--(.096,-.006);
\draw(.046,.018)--(.046,.044);
\filldraw[fill opacity=0.8,fill=gray!20,draw=none](.096,.047)--(.096,.029)--(.046,.044)--(.046,.053)--cycle;
\draw(.096,.047)--(.096,.029);
\draw(.046,.044)--(.046,.053);
\filldraw[fill opacity=0.8,fill=gray!20,draw=none](-.064,.041)--(-.05,.045)--(-.054,.041)--cycle;
\filldraw[fill opacity=0.8,fill=gray!20,draw=none](-.075,.039)--(-.064,.041)--(-.054,.041)--(-.065,.029)--cycle;
\filldraw[fill opacity=0.5,fill=gray!20,draw=none](.131,.033)--(.134,.034)--(.125,.04)--cycle;
\draw(.131,.033)--(.134,.034);
\filldraw[fill opacity=0.8,fill=gray!20,draw=none](.131,.033)--(.131,.014)--(.122,.002)--(.096,-.003)--(.096,.036)--cycle;
\draw(.131,.033)--(.131,.014);
\draw(.096,-.003)--(.096,.036);
\filldraw[fill opacity=0.8,fill=gray!20,draw=none](.131,.053)--(.131,.033)--(.096,.036)--(.096,.079)--cycle;
\draw(.131,.053)--(.131,.033);
\draw(.096,.036)--(.096,.079);
\filldraw[fill opacity=0.8,fill=gray!20,draw=none](.141,.158)--(.143,.116)--(.146,.134)--(.146,.171)--cycle;
\draw(.146,.134)--(.146,.171);
\filldraw[fill opacity=0.8,fill=gray!20,draw=none](.139,.168)--(.139,.153)--(.146,.171)--(.146,.181)--cycle;
\draw(.146,.171)--(.146,.181)--(.139,.168)--(.139,.153);
\filldraw[fill opacity=0.8,fill=gray!20,draw=none](.112,-.129)--(.131,-.154)--(.131,-.164)--(.099,-.155)--(.096,-.146)--(.096,-.13)--cycle;
\draw(.131,-.154)--(.131,-.164)--(.099,-.155);
\draw(.096,-.146)--(.096,-.13);
\filldraw[fill opacity=0.8,fill=gray!20,draw=none](.15,-.128)--(.131,-.129)--(.154,-.126)--cycle;
\draw(.131,-.129)--(.154,-.126);
\filldraw[fill opacity=0.8,fill=gray!20,draw=none](.131,-.129)--(.131,-.154)--(.096,-.108)--cycle;
\draw(.131,-.129)--(.131,-.154);
\filldraw[fill opacity=0.8,fill=gray!20,draw=none](.131,-.129)--(.15,-.128)--(.146,-.13)--cycle;
\filldraw[fill opacity=0.8,fill=gray!20,draw=none](.128,.032)--(.12,.034)--(.118,.038)--cycle;
\filldraw[fill opacity=0.8,fill=gray!20,draw=none](.137,.03)--(.131,.014)--(.131,.033)--cycle;
\draw(.131,.014)--(.131,.033);
\filldraw[fill opacity=0.8,fill=gray!20,draw=none](.128,.032)--(.136,.027)--(.131,.014)--(.12,.034)--cycle;
\filldraw[fill opacity=0.5,fill=gray!20,draw=none](.128,.032)--(.118,.038)--(.096,.047)--(.019,.057)--(-.05,.045)--(-.064,.041)--(0,0)--cycle;
\draw(-.064,.041)--(0,0)--(.128,.032);
\filldraw[fill opacity=0.8,fill=gray!20,draw=none](.046,.131)--(.046,.053)--(-.01,.052)--(-.01,.087)--cycle;
\draw(.046,.131)--(.046,.053);
\draw(-.01,.052)--(-.01,.087);
\filldraw[fill opacity=0.8,fill=gray!20,draw=none](.096,.079)--(.096,.047)--(.046,.053)--(.046,.131)--cycle;
\draw(.096,.079)--(.096,.047);
\draw(.046,.053)--(.046,.131);
\filldraw[fill opacity=0.8,fill=gray!20,draw=none](-.122,.136)--(-.111,.138)--(-.111,.076)--(-.139,.111)--cycle;
\draw(-.111,.138)--(-.111,.076);
\filldraw[fill opacity=0.8,fill=gray!20,draw=none](-.139,.111)--(.046,.131)--(-.01,.087)--(-.111,.076)--cycle;
\draw(-.139,.111)--(.046,.131);
\draw(-.01,.087)--(-.111,.076);
\filldraw[fill opacity=0.5,fill=gray!20,draw=none](-.135,.04)--(-.065,.042)--(-.067,.043)--(-.132,.045)--cycle;
\draw(-.065,.042)--(-.067,.043)--(-.132,.045);
\filldraw[fill opacity=0.8,fill=gray!20,draw=none](-.065,.067)--(-.065,.042)--(-.078,.042)--(-.111,.076)--cycle;
\draw(-.065,.067)--(-.065,.042);
\filldraw[fill opacity=0.8,fill=gray!20,draw=none](-.078,.042)--(-.111,.076)--(-.01,.087)--(-.05,.045)--(-.064,.041)--cycle;
\draw(-.111,.076)--(-.01,.087);
\filldraw[fill opacity=0.8,fill=gray!20,draw=none](.143,.116)--(.132,.133)--(.075,.127)--cycle;
\draw(.132,.133)--(.075,.127);
\filldraw[fill opacity=0.8,fill=gray!20,draw=none](.139,.091)--(.143,.116)--(.075,.127)--cycle;
\filldraw[fill opacity=0.8,fill=gray!20,draw=none](.141,.154)--(.139,.153)--(.139,.113)--(.143,.116)--cycle;
\draw(.139,.153)--(.139,.113);
\filldraw[fill opacity=0.8,fill=gray!20,draw=none](.133,.005)--(.146,-.052)--(.146,-.13)--(.131,-.129)--(.131,.004)--cycle;
\draw(.146,-.052)--(.146,-.13);
\draw(.131,-.129)--(.131,.004);
\filldraw[fill opacity=0.8,fill=gray!20,draw=none](.146,-.052)--(.096,-.031)--(.122,.002)--(.133,.005)--cycle;
\draw(.146,-.052)--(.096,-.031);
\filldraw[fill opacity=0.8,fill=gray!20,draw=none](.122,.002)--(.096,-.031)--(.096,-.003)--cycle;
\draw(.096,-.031)--(.096,-.003);
\filldraw[fill opacity=0.8,fill=gray!20,draw=none](.141,.154)--(.141,.158)--(.139,.153)--cycle;
\filldraw[fill opacity=0.8,fill=gray!20,draw=none](.111,.158)--(.111,.111)--(.139,.113)--(.139,.168)--cycle;
\draw(.139,.113)--(.139,.168)--(.111,.158)--(.111,.111);
\filldraw[fill opacity=0.5,fill=gray!20](.696,.221)--(.21,.433)--(-.243,.318)--(.243,.106)--cycle;
\filldraw[fill opacity=0.5,fill=gray!20](.16,.632)--(.342,.553)--(-.061,.451)--(-.243,.53)--cycle;
\filldraw[fill opacity=0.5,fill=gray!20](.994,3.713)--(.857,3.548)--(1.104,3.223)--(1.272,3.347)--cycle;
\filldraw[fill opacity=0.5,fill=gray!20](1.106,3.76)--(.994,3.713)--(1.272,3.348)--(1.397,3.377)--cycle;
\filldraw[fill opacity=0.5,fill=gray!20](1.208,3.786)--(1.106,3.76)--(1.397,3.377)--(1.509,3.391)--cycle;
\filldraw[fill opacity=0.5,fill=gray!20](1.299,3.79)--(1.208,3.786)--(1.509,3.391)--(1.606,3.387)--cycle;
\filldraw[fill opacity=0.5,fill=gray!20](1.48,3.5)--(.994,3.713)--(1.272,3.347)--(1.758,3.135)--cycle;
\filldraw[fill opacity=0.5,fill=gray!20](.857,3.548)--(1.039,3.468)--(1.286,3.143)--(1.104,3.223)--cycle;
\filldraw[fill opacity=0.5,fill=gray!20](-.243,.318)--(-.243,.53)--(-.646,.544)--(-.696,.334)--cycle;
\filldraw[fill opacity=0.5,fill=gray!20](-.243,.53)--(-.061,.451)--(-.463,.465)--(-.646,.544)--cycle;
\filldraw[fill opacity=0.5,fill=gray!20](.632,3.976)--(.535,3.782)--(.857,3.548)--(.994,3.713)--cycle;
\filldraw[fill opacity=0.5,fill=gray!20](.535,3.782)--(.717,3.703)--(1.039,3.468)--(.857,3.548)--cycle;
\filldraw[fill opacity=0.5,fill=gray!20](-.696,.334)--(-.646,.544)--(-1.021,.673)--(-1.118,.479)--cycle;
\filldraw[fill opacity=0.5,fill=gray!20](-.646,.544)--(-.463,.465)--(-.838,.593)--(-1.021,.673)--cycle;
\filldraw[fill opacity=0.8,fill=gray!20,draw=none](.06,2.106)--(.039,2.116)--(.049,2.171)--(.07,2.162)--cycle;
\draw(.06,2.106)--(.039,2.116)--(.049,2.171)--(.07,2.162);
\filldraw[fill opacity=0.8,fill=gray!20,draw=none](.032,2.057)--(.011,2.066)--(.039,2.116)--(.06,2.106)--cycle;
\draw(.032,2.057)--(.011,2.066)--(.039,2.116)--(.06,2.106);
\filldraw[fill opacity=0.8,fill=gray!20](-.203,2.199)--(-.174,2.248)--(-.132,2.284)--(-.082,2.3)--(-.032,2.295)--(.011,2.268)--(.039,2.225)--(.049,2.171)--(.039,2.116)--(.011,2.066)--(-.032,2.031)--(-.082,2.014)--(-.132,2.02)--(-.174,2.046)--(-.203,2.089)--(-.213,2.143)--cycle;
\filldraw[fill opacity=0.8,fill=gray!20,draw=none](-.011,2.021)--(-.032,2.031)--(.011,2.066)--(.032,2.057)--cycle;
\draw(-.011,2.021)--(-.032,2.031)--(.011,2.066)--(.032,2.057);
\filldraw[fill opacity=0.8,fill=gray!20,draw=none](.07,2.162)--(.049,2.171)--(.039,2.225)--(.06,2.216)--cycle;
\draw(.07,2.162)--(.049,2.171)--(.039,2.225)--(.06,2.216);
\filldraw[fill opacity=0.8,fill=gray!20,draw=none](-.061,2.005)--(-.082,2.014)--(-.032,2.031)--(-.011,2.021)--cycle;
\draw(-.061,2.005)--(-.082,2.014)--(-.032,2.031)--(-.011,2.021);
\filldraw[fill opacity=0.8,fill=gray!20,draw=none](.06,2.216)--(.039,2.225)--(.011,2.268)--(.032,2.259)--cycle;
\draw(.06,2.216)--(.039,2.225)--(.011,2.268)--(.032,2.259);
\filldraw[fill opacity=0.8,fill=gray!20,draw=none](-.111,2.01)--(-.132,2.02)--(-.082,2.014)--(-.061,2.005)--cycle;
\draw(-.111,2.01)--(-.132,2.02)--(-.082,2.014)--(-.061,2.005);
\filldraw[fill opacity=0.8,fill=gray!20,draw=none](.032,2.259)--(.011,2.268)--(-.032,2.295)--(-.011,2.285)--cycle;
\draw(.032,2.259)--(.011,2.268)--(-.032,2.295)--(-.011,2.285);
\filldraw[fill opacity=0.8,fill=gray!20,draw=none](-.153,2.037)--(-.174,2.046)--(-.132,2.02)--(-.111,2.01)--cycle;
\draw(-.153,2.037)--(-.174,2.046)--(-.132,2.02)--(-.111,2.01);
\filldraw[fill opacity=0.8,fill=gray!20,draw=none](-.011,2.285)--(-.032,2.295)--(-.082,2.3)--(-.061,2.291)--cycle;
\draw(-.011,2.285)--(-.032,2.295)--(-.082,2.3)--(-.061,2.291);
\filldraw[fill opacity=0.8,fill=gray!20,draw=none](-.182,2.08)--(-.203,2.089)--(-.174,2.046)--(-.153,2.037)--cycle;
\draw(-.182,2.08)--(-.203,2.089)--(-.174,2.046)--(-.153,2.037);
\filldraw[fill opacity=0.8,fill=gray!20,draw=none](-.061,2.291)--(-.082,2.3)--(-.132,2.284)--(-.111,2.274)--cycle;
\draw(-.061,2.291)--(-.082,2.3)--(-.132,2.284)--(-.111,2.274);
\filldraw[fill opacity=0.8,fill=gray!20,draw=none](-.188,2.116)--(-.197,2.136)--(-.213,2.143)--(-.203,2.089)--(-.182,2.08)--cycle;
\draw(-.197,2.136)--(-.213,2.143)--(-.203,2.089)--(-.182,2.08);
\filldraw[fill opacity=0.8,fill=gray!20,draw=none](-.111,2.274)--(-.132,2.284)--(-.174,2.248)--(-.153,2.239)--cycle;
\draw(-.111,2.274)--(-.132,2.284)--(-.174,2.248)--(-.153,2.239);
\filldraw[fill opacity=0.8,fill=gray!20,draw=none](-.186,2.171)--(-.184,2.19)--(-.203,2.199)--(-.213,2.143)--(-.204,2.139)--cycle;
\draw(-.184,2.19)--(-.203,2.199)--(-.213,2.143)--(-.204,2.139);
\filldraw[fill opacity=0.8,fill=gray!20,draw=none](-.171,2.208)--(-.153,2.239)--(-.174,2.248)--(-.203,2.199)--(-.194,2.195)--cycle;
\draw(-.153,2.239)--(-.174,2.248)--(-.203,2.199)--(-.194,2.195);
\filldraw[fill opacity=0.8,fill=gray!20,draw=none](-.186,2.164)--(-.186,2.171)--(-.204,2.139)--(-.192,2.134)--cycle;
\draw(-.204,2.139)--(-.192,2.134);
\filldraw[fill opacity=0.8,fill=gray!20,draw=none](-.171,2.208)--(-.194,2.195)--(-.182,2.189)--cycle;
\draw(-.194,2.195)--(-.182,2.189);
\filldraw[fill opacity=0.8,fill=gray!20,draw=none](-.188,2.116)--(-.192,2.134)--(-.197,2.136)--cycle;
\draw(-.192,2.134)--(-.197,2.136);
\filldraw[fill opacity=0.8,fill=gray!20,draw=none](-.186,2.164)--(-.182,2.189)--(-.184,2.19)--cycle;
\draw(-.182,2.189)--(-.184,2.19);
\filldraw[fill opacity=0.8,fill=gray!20](-1.378,2.07)--(-1.381,2.126)--(-1.27,2.131)--(-1.27,2.074)--cycle;
\filldraw[fill opacity=0.8,fill=gray!20](-1.368,2.014)--(-1.378,2.07)--(-1.27,2.074)--(-1.272,2.018)--cycle;
\filldraw[fill opacity=0.8,fill=gray!20,draw=none](-1.202,2.094)--(-1.213,2.07)--(-1.27,2.074)--(-1.27,2.131)--(-1.197,2.126)--cycle;
\draw(-1.213,2.07)--(-1.27,2.074)--(-1.27,2.131)--(-1.197,2.126);
\filldraw[fill opacity=0.8,fill=gray!20,draw=none](-1.233,2.015)--(-1.272,2.018)--(-1.27,2.074)--(-1.206,2.07)--cycle;
\draw(-1.233,2.015)--(-1.272,2.018)--(-1.27,2.074)--(-1.206,2.07);
\filldraw[fill opacity=0.8,fill=gray!20](-1.381,2.126)--(-1.378,2.18)--(-1.27,2.185)--(-1.27,2.131)--cycle;
\filldraw[fill opacity=0.8,fill=gray!20,draw=none](-1.2,2.143)--(-1.213,2.127)--(-1.27,2.131)--(-1.27,2.185)--(-1.206,2.181)--cycle;
\draw(-1.213,2.127)--(-1.27,2.131)--(-1.27,2.185)--(-1.206,2.181);
\filldraw[fill opacity=0.8,fill=gray!20,draw=none](-1.279,1.966)--(-1.353,1.962)--(-1.368,2.014)--(-1.272,2.018)--(-1.274,1.968)--cycle;
\draw(-1.279,1.966)--(-1.353,1.962)--(-1.368,2.014)--(-1.272,2.018)--(-1.274,1.968);
\filldraw[fill opacity=0.8,fill=gray!20,draw=none](-1.237,2.011)--(-1.274,1.968)--(-1.272,2.018)--(-1.254,2.017)--cycle;
\draw(-1.274,1.968)--(-1.272,2.018)--(-1.254,2.017);
\filldraw[fill opacity=0.8,fill=gray!20,draw=none](-1.237,2.011)--(-1.254,2.017)--(-1.233,2.015)--cycle;
\draw(-1.254,2.017)--(-1.233,2.015);
\filldraw[fill opacity=0.8,fill=gray!20,draw=none](-1.202,2.094)--(-1.206,2.07)--(-1.213,2.07)--cycle;
\draw(-1.206,2.07)--(-1.213,2.07);
\filldraw[fill opacity=0.8,fill=gray!20](-1.442,1.999)--(-1.46,2.054)--(-1.378,2.07)--(-1.368,2.014)--cycle;
\filldraw[fill opacity=0.8,fill=gray!20](-1.46,2.054)--(-1.466,2.11)--(-1.381,2.126)--(-1.378,2.07)--cycle;
\filldraw[fill opacity=0.8,fill=gray!20,draw=none](-1.2,2.143)--(-1.197,2.126)--(-1.213,2.127)--cycle;
\draw(-1.197,2.126)--(-1.213,2.127);
\filldraw[fill opacity=0.8,fill=gray!20,draw=none](-1.378,2.18)--(-1.371,2.213)--(-1.271,2.215)--(-1.27,2.185)--cycle;
\draw(-1.271,2.215)--(-1.27,2.185)--(-1.378,2.18)--(-1.371,2.213);
\filldraw[fill opacity=0.8,fill=gray!20,draw=none](-1.226,2.216)--(-1.206,2.181)--(-1.27,2.185)--(-1.271,2.215)--cycle;
\draw(-1.206,2.181)--(-1.27,2.185)--(-1.271,2.215);
\filldraw[fill opacity=0.8,fill=gray!20](-1.466,2.11)--(-1.46,2.165)--(-1.378,2.18)--(-1.381,2.126)--cycle;
\filldraw[fill opacity=0.8,fill=gray!20,draw=none](-1.409,1.951)--(-1.414,1.953)--(-1.442,1.999)--(-1.368,2.014)--(-1.353,1.962)--cycle;
\draw(-1.414,1.953)--(-1.442,1.999)--(-1.368,2.014)--(-1.353,1.962)--(-1.409,1.951);
\filldraw[fill opacity=0.8,fill=gray!20,draw=none](-1.344,1.944)--(-1.353,1.962)--(-1.279,1.966)--cycle;
\draw(-1.344,1.944)--(-1.353,1.962)--(-1.279,1.966);
\filldraw[fill opacity=0.8,fill=gray!20,draw=none](-1.226,2.216)--(-1.271,2.215)--(-1.272,2.232)--(-1.233,2.229)--cycle;
\draw(-1.271,2.215)--(-1.272,2.232)--(-1.233,2.229);
\filldraw[fill opacity=0.8,fill=gray!20,draw=none](-1.371,2.213)--(-1.368,2.228)--(-1.272,2.232)--(-1.271,2.215)--cycle;
\draw(-1.371,2.213)--(-1.368,2.228)--(-1.272,2.232)--(-1.271,2.215);
\filldraw[fill opacity=0.8,fill=gray!20](-1.46,2.165)--(-1.442,2.214)--(-1.368,2.228)--(-1.378,2.18)--cycle;
\filldraw[fill opacity=0.8,fill=gray!20,draw=none](-1.409,1.951)--(-1.353,1.962)--(-1.344,1.944)--cycle;
\draw(-1.409,1.951)--(-1.353,1.962)--(-1.344,1.944);
\filldraw[fill opacity=0.8,fill=gray!20,draw=none](-1.368,2.228)--(-1.353,2.265)--(-1.279,2.269)--(-1.274,2.267)--(-1.272,2.232)--cycle;
\draw(-1.274,2.267)--(-1.272,2.232)--(-1.368,2.228)--(-1.353,2.265)--(-1.279,2.269);
\filldraw[fill opacity=0.8,fill=gray!20,draw=none](-1.233,2.229)--(-1.272,2.232)--(-1.274,2.267)--cycle;
\draw(-1.233,2.229)--(-1.272,2.232)--(-1.274,2.267);
\filldraw[fill opacity=0.8,fill=gray!20,draw=none](-1.442,2.214)--(-1.414,2.252)--(-1.409,2.255)--(-1.353,2.265)--(-1.368,2.228)--cycle;
\draw(-1.409,2.255)--(-1.353,2.265)--(-1.368,2.228)--(-1.442,2.214)--(-1.414,2.252);
\filldraw[fill opacity=0.8,fill=gray!20,draw=none](-1.455,1.991)--(-1.479,2.036)--(-1.48,2.04)--(-1.46,2.054)--(-1.442,1.999)--cycle;
\draw(-1.48,2.04)--(-1.46,2.054)--(-1.442,1.999)--(-1.455,1.991);
\filldraw[fill opacity=0.8,fill=gray!20,draw=none](-1.481,2.04)--(-1.482,2.043)--(-1.489,2.082)--(-1.489,2.09)--(-1.489,2.095)--(-1.466,2.11)--(-1.46,2.054)--cycle;
\draw(-1.489,2.095)--(-1.466,2.11)--(-1.46,2.054)--(-1.481,2.04);
\filldraw[fill opacity=0.8,fill=gray!20,draw=none](-1.455,1.991)--(-1.442,1.999)--(-1.414,1.953)--cycle;
\draw(-1.455,1.991)--(-1.442,1.999)--(-1.414,1.953);
\filldraw[fill opacity=0.8,fill=gray!20,draw=none](-1.476,2.103)--(-1.479,2.124)--(-1.466,2.117)--(-1.466,2.11)--cycle;
\draw(-1.466,2.117)--(-1.466,2.11)--(-1.476,2.103);
\filldraw[fill opacity=0.8,fill=gray!20,draw=none](-1.279,2.269)--(-1.353,2.265)--(-1.344,2.276)--cycle;
\draw(-1.279,2.269)--(-1.353,2.265)--(-1.344,2.276);
\filldraw[fill opacity=0.8,fill=gray!20,draw=none](-1.479,2.124)--(-1.482,2.143)--(-1.48,2.151)--(-1.46,2.165)--(-1.466,2.117)--cycle;
\draw(-1.48,2.151)--(-1.46,2.165)--(-1.466,2.117);
\filldraw[fill opacity=0.8,fill=gray!20,draw=none](-1.465,2.161)--(-1.467,2.179)--(-1.455,2.205)--(-1.442,2.214)--(-1.46,2.165)--cycle;
\draw(-1.455,2.205)--(-1.442,2.214)--(-1.46,2.165)--(-1.465,2.161);
\filldraw[fill opacity=0.8,fill=gray!20,draw=none](-1.409,2.255)--(-1.344,2.276)--(-1.353,2.265)--cycle;
\draw(-1.344,2.276)--(-1.353,2.265)--(-1.409,2.255);
\filldraw[fill opacity=0.8,fill=gray!20,draw=none](-1.479,2.124)--(-1.476,2.103)--(-1.491,2.094)--(-1.485,2.127)--cycle;
\draw(-1.476,2.103)--(-1.491,2.094);
\filldraw[fill opacity=0.8,fill=gray!20,draw=none](-1.465,2.161)--(-1.482,2.151)--(-1.467,2.179)--cycle;
\draw(-1.465,2.161)--(-1.482,2.151);
\filldraw[fill opacity=0.8,fill=gray!20,draw=none](-1.455,2.205)--(-1.414,2.252)--(-1.442,2.214)--cycle;
\draw(-1.414,2.252)--(-1.442,2.214)--(-1.455,2.205);
\filldraw[fill opacity=0.8,fill=gray!20,draw=none](-1.479,2.036)--(-1.482,2.04)--(-1.48,2.04)--cycle;
\draw(-1.482,2.04)--(-1.48,2.04);
\filldraw[fill opacity=0.8,fill=gray!20,draw=none](-1.481,2.04)--(-1.482,2.04)--(-1.482,2.043)--cycle;
\draw(-1.481,2.04)--(-1.482,2.04);
\filldraw[fill opacity=0.8,fill=gray!20,draw=none](-1.479,2.124)--(-1.485,2.127)--(-1.484,2.137)--(-1.482,2.143)--cycle;
\filldraw[fill opacity=0.8,fill=gray!20,draw=none](-1.489,2.09)--(-1.49,2.095)--(-1.489,2.095)--cycle;
\draw(-1.49,2.095)--(-1.489,2.095);
\filldraw[fill opacity=0.8,fill=gray!20,draw=none](-1.489,2.082)--(-1.491,2.094)--(-1.49,2.095)--cycle;
\draw(-1.491,2.094)--(-1.49,2.095);
\filldraw[fill opacity=0.8,fill=gray!20,draw=none](-1.484,2.137)--(-1.482,2.151)--(-1.48,2.151)--cycle;
\draw(-1.482,2.151)--(-1.48,2.151);
\filldraw[fill opacity=0.5,fill=gray!20](1.442,2.751)--(1.286,3.143)--(1.039,3.468)--(.717,3.703)--(.342,3.831)--(-.061,3.845)--(-.463,3.743)--(-.838,3.533)--(-1.161,3.228)--(-1.408,2.85)--(-1.563,2.423)--(-1.616,1.978)--(-1.563,1.545)--(-1.408,1.152)--(-1.161,.828)--(-.838,.593)--(-.463,.465)--(-.061,.451)--(.342,.553)--(.717,.763)--(1.039,1.068)--(1.286,1.446)--(1.442,1.873)--(1.495,2.318)--cycle;
\filldraw[fill opacity=0.5,fill=gray!20](-.19,.205)--(-.243,.318)--(-.696,.334)--(-.665,.221)--cycle;
\filldraw[fill opacity=0.5,fill=gray!20,draw=none](-.295,.118)--(-.354,.211)--(-.665,.221)--(-.621,.129)--cycle;
\draw(-.354,.211)--(-.665,.221)--(-.621,.129)--(-.295,.118);
\filldraw[fill opacity=0.8,fill=gray!20,draw=none](.013,.154)--(-.032,.173)--(-.082,.179)--(-.033,.157)--cycle;
\draw(.013,.154)--(-.032,.173)--(-.082,.179)--(-.033,.157);
\filldraw[fill opacity=0.8,fill=gray!20,draw=none](-.018,.151)--(-.082,.179)--(-.132,.162)--(-.075,.137)--cycle;
\draw(-.018,.151)--(-.082,.179)--(-.132,.162)--(-.075,.137);
\filldraw[fill opacity=0.8,fill=gray!20,draw=none](-.01,.164)--(-.065,.171)--(-.065,.197)--cycle;
\draw(-.065,.171)--(-.065,.197);
\filldraw[fill opacity=0.8,fill=gray!20,draw=none](.097,.116)--(.139,.091)--(.046,.131)--(.013,.154)--(.075,.127)--cycle;
\draw(.139,.091)--(.046,.131);
\draw(.013,.154)--(.075,.127);
\filldraw[fill opacity=0.8,fill=gray!20,draw=none](-.075,.137)--(.025,.148)--(.046,.131)--(-.139,.111)--cycle;
\draw(-.075,.137)--(.025,.148);
\draw(.046,.131)--(-.139,.111);
\filldraw[fill opacity=0.8,fill=gray!20,draw=none](-.134,.135)--(-.128,.161)--(-.132,.162)--(-.166,.134)--cycle;
\draw(-.128,.161)--(-.132,.162)--(-.166,.134);
\filldraw[fill opacity=0.8,fill=gray!20,draw=none](-.065,.207)--(-.065,.171)--(-.111,.175)--(-.111,.2)--cycle;
\draw(-.111,.175)--(-.111,.2)--(-.065,.207)--(-.065,.171);
\filldraw[fill opacity=0.8,fill=gray!20,draw=none](-.063,.199)--(-.004,.205)--(-.016,.21)--(-.065,.207)--(-.078,.205)--cycle;
\draw(-.016,.21)--(-.065,.207)--(-.078,.205);
\filldraw[fill opacity=0.8,fill=gray!20,draw=none](-.122,.136)--(-.111,.153)--(-.111,.138)--cycle;
\draw(-.111,.153)--(-.111,.138);
\filldraw[fill opacity=0.8,fill=gray!20,draw=none](-.075,.137)--(-.128,.161)--(-.134,.135)--cycle;
\draw(-.075,.137)--(-.128,.161);
\filldraw[fill opacity=0.8,fill=gray!20,draw=none](-.065,.171)--(-.111,.153)--(-.111,.175)--cycle;
\draw(-.111,.153)--(-.111,.175);
\filldraw[fill opacity=0.8,fill=gray!20,draw=none](-.134,.135)--(-.166,.134)--(-.174,.127)--(-.139,.111)--cycle;
\draw(-.166,.134)--(-.174,.127)--(-.139,.111);
\filldraw[fill opacity=0.5,fill=gray!20,draw=none](-.05,.045)--(-.065,.042)--(-.064,.041)--cycle;
\draw(-.065,.042)--(-.064,.041);
\filldraw[fill opacity=0.8,fill=gray!20,draw=none](-.149,.115)--(-.174,.127)--(-.203,.077)--(-.181,.068)--cycle;
\draw(-.149,.115)--(-.174,.127)--(-.203,.077)--(-.181,.068);
\filldraw[fill opacity=0.8,fill=gray!20,draw=none](-.111,.2)--(-.111,.153)--(-.139,.133)--(-.139,.189)--cycle;
\draw(-.139,.133)--(-.139,.189)--(-.111,.2)--(-.111,.153);
\filldraw[fill opacity=0.8,fill=gray!20,draw=none](-.078,.042)--(-.065,.029)--(-.111,-.034)--(-.111,.041)--cycle;
\draw(-.111,-.034)--(-.111,.041);
\filldraw[fill opacity=0.8,fill=gray!20,draw=none](.046,-.071)--(.019,-.104)--(.013,-.088)--cycle;
\filldraw[fill opacity=0.8,fill=gray!20,draw=none](.013,-.11)--(-.01,-.1)--(.046,-.071)--cycle;
\draw(.013,-.11)--(-.01,-.1);
\filldraw[fill opacity=0.8,fill=gray!20,draw=none](.046,-.071)--(.046,-.101)--(.021,-.11)--(.019,-.104)--cycle;
\draw(.046,-.071)--(.046,-.101);
\filldraw[fill opacity=0.8,fill=gray!20,draw=none](-.075,-.127)--(0,-.143)--(-.035,-.147)--cycle;
\draw(0,-.143)--(-.035,-.147);
\filldraw[fill opacity=0.8,fill=gray!20,draw=none](-.164,-.062)--(-.111,-.034)--(-.139,-.091)--cycle;
\filldraw[fill opacity=0.8,fill=gray!20,draw=none](-.164,-.062)--(-.181,-.041)--(-.111,-.034)--cycle;
\draw(-.181,-.041)--(-.111,-.034);
\filldraw[fill opacity=0.8,fill=gray!20,draw=none](-.111,.041)--(-.111,-.034)--(-.139,-.091)--(-.139,.033)--cycle;
\draw(-.111,.041)--(-.111,-.034);
\draw(-.139,-.091)--(-.139,.033);
\filldraw[fill opacity=0.5,fill=gray!20,draw=none](-.188,.045)--(-.133,.043)--(-.132,.045)--(-.192,.047)--cycle;
\draw(-.132,.045)--(-.192,.047);
\filldraw[fill opacity=0.8,fill=gray!20,draw=none](-.078,.042)--(-.188,.045)--(-.181,.068)--(-.111,.076)--cycle;
\draw(-.181,.068)--(-.111,.076);
\filldraw[fill opacity=0.5,fill=gray!20,draw=none](-.154,.029)--(-.111,.041)--(-.164,.039)--cycle;
\filldraw[fill opacity=0.5,fill=gray!20,draw=none](-.188,.045)--(-.179,.038)--(-.135,.04)--(-.133,.043)--cycle;
\filldraw[fill opacity=0.8,fill=gray!20,draw=none](-.141,.132)--(-.142,.139)--(-.14,.186)--(-.139,.189)--(-.139,.133)--cycle;
\draw(-.14,.186)--(-.139,.189)--(-.139,.133);
\filldraw[fill opacity=0.8,fill=gray!20,draw=none](-.121,.178)--(-.036,.187)--(-.078,.205)--(-.111,.2)--(-.139,.189)--(-.14,.186)--cycle;
\draw(-.078,.205)--(-.111,.2)--(-.139,.189)--(-.14,.186);
\filldraw[fill opacity=0.8,fill=gray!20,draw=none](0,.143)--(.025,.148)--(-.075,.137)--cycle;
\draw(.025,.148)--(-.075,.137);
\filldraw[fill opacity=0.8,fill=gray!20,draw=none](.025,.148)--(.013,.154)--(-.033,.157)--(0,.143)--cycle;
\draw(.025,.148)--(.013,.154);
\draw(-.033,.157)--(0,.143);
\filldraw[fill opacity=0.8,fill=gray!20,draw=none](-.046,.071)--(.139,.091)--(.075,.127)--(-.025,.116)--cycle;
\draw(-.046,.071)--(.139,.091);
\draw(.075,.127)--(-.025,.116);
\filldraw[fill opacity=0.8,fill=gray!20,draw=none](-.02,.103)--(-.025,.116)--(-.075,.137)--(-.134,.135)--(-.139,.111)--(-.046,.071)--cycle;
\draw(-.025,.116)--(-.075,.137);
\draw(-.139,.111)--(-.046,.071);
\filldraw[fill opacity=0.8,fill=gray!20,draw=none](-.111,.153)--(-.139,.111)--(-.139,.133)--cycle;
\draw(-.139,.111)--(-.139,.133);
\filldraw[fill opacity=0.8,fill=gray!20,draw=none](.075,.127)--(.025,.148)--(0,.143)--cycle;
\draw(.075,.127)--(.025,.148);
\filldraw[fill opacity=0.8,fill=gray!20,draw=none](0,.143)--(.035,.147)--(.025,.148)--cycle;
\draw(0,.143)--(.035,.147);
\filldraw[fill opacity=0.8,fill=gray!20,draw=none](-.164,.086)--(-.139,.111)--(-.134,.073)--(-.167,.069)--cycle;
\draw(-.134,.073)--(-.167,.069);
\filldraw[fill opacity=0.8,fill=gray!20,draw=none](-.164,.086)--(-.167,.069)--(-.181,.068)--cycle;
\draw(-.167,.069)--(-.181,.068);
\filldraw[fill opacity=0.8,fill=gray!20,draw=none](-.149,.115)--(-.181,.068)--(-.174,.065)--cycle;
\draw(-.181,.068)--(-.174,.065);
\filldraw[fill opacity=0.8,fill=gray!20,draw=none](-.139,.014)--(-.139,-.091)--(-.146,-.057)--(-.146,.01)--cycle;
\draw(-.139,.014)--(-.139,-.091);
\draw(-.146,-.057)--(-.146,.01);
\filldraw[fill opacity=0.8,fill=gray!20,draw=none](-.139,.033)--(-.139,.014)--(-.146,.01)--(-.146,.02)--cycle;
\draw(-.139,.033)--(-.139,.014);
\draw(-.146,.01)--(-.146,.02);
\filldraw[fill opacity=0.8,fill=gray!20,draw=none](.099,-.155)--(.096,-.154)--(.096,-.146)--cycle;
\draw(.099,-.155)--(.096,-.154)--(.096,-.146);
\filldraw[fill opacity=0.8,fill=gray!20,draw=none](.112,-.129)--(.096,-.13)--(.096,-.108)--cycle;
\draw(.096,-.13)--(.096,-.108);
\filldraw[fill opacity=0.8,fill=gray!20,draw=none](.025,-.116)--(.096,-.108)--(.131,-.129)--(.035,-.139)--cycle;
\draw(.025,-.116)--(.096,-.108);
\draw(.131,-.129)--(.035,-.139);
\filldraw[fill opacity=0.8,fill=gray!20,draw=none](.146,-.13)--(.146,-.176)--(.131,-.164)--(.131,-.154)--cycle;
\draw(.146,-.13)--(.146,-.176)--(.131,-.164)--(.131,-.154);
\filldraw[fill opacity=0.5,fill=gray!20,draw=none](-.139,.033)--(-.146,.02)--(-.137,.01)--(-.128,.004)--(0,0)--(-.065,.042)--(-.111,.041)--cycle;
\draw(-.128,.004)--(0,0)--(-.065,.042);
\filldraw[fill opacity=0.8,fill=gray!20,draw=none](-.139,.111)--(-.111,.076)--(-.134,.073)--cycle;
\draw(-.111,.076)--(-.134,.073);
\filldraw[fill opacity=0.8,fill=gray!20,draw=none](-.111,.076)--(-.111,.041)--(-.139,.033)--(-.139,.111)--cycle;
\draw(-.111,.076)--(-.111,.041);
\draw(-.139,.033)--(-.139,.111);
\filldraw[fill opacity=0.8,fill=gray!20,draw=none](-.193,.048)--(-.206,.057)--(-.213,.021)--(-.208,.019)--cycle;
\draw(-.206,.057)--(-.213,.021)--(-.208,.019);
\filldraw[fill opacity=0.5,fill=gray!20,draw=none](-.161,.027)--(-.154,.029)--(-.164,.039)--(-.179,.038)--cycle;
\filldraw[fill opacity=0.5,fill=gray!20,draw=none](-.146,.02)--(-.139,.033)--(-.154,.029)--cycle;
\filldraw[fill opacity=0.8,fill=gray!20,draw=none](-.139,.111)--(-.139,.033)--(-.146,.02)--(-.146,.052)--cycle;
\draw(-.139,.111)--(-.139,.033);
\draw(-.146,.02)--(-.146,.052);
\filldraw[fill opacity=0.5,fill=gray!20,draw=none](-.146,.02)--(-.154,.029)--(-.161,.027)--(-.147,.018)--cycle;
\filldraw[fill opacity=0.5,fill=gray!20,draw=none](-.204,.048)--(-.201,.048)--(-.224,.087)--(-.249,.116)--(-.276,.117)--cycle;
\draw(-.249,.116)--(-.276,.117);
\filldraw[fill opacity=0.5,fill=gray!20,draw=none](-.245,.049)--(-.204,.048)--(-.276,.117)--(-.337,.119)--cycle;
\draw(-.245,.049)--(-.204,.048);
\draw(-.276,.117)--(-.337,.119);
\filldraw[fill opacity=0.8,fill=gray!20,draw=none](-.137,.01)--(-.146,.02)--(-.146,.01)--(-.136,.006)--cycle;
\draw(-.146,.02)--(-.146,.01);
\filldraw[fill opacity=0.8,fill=gray!20,draw=none](-.146,.052)--(-.146,.02)--(-.137,.01)--cycle;
\draw(-.146,.052)--(-.146,.02);
\filldraw[fill opacity=0.5,fill=gray!20,draw=none](-.146,.02)--(-.147,.018)--(-.137,.01)--cycle;
\filldraw[fill opacity=0.8,fill=gray!20,draw=none](-.137,.01)--(-.146,.052)--(-.203,.077)--(-.206,.057)--cycle;
\draw(-.146,.052)--(-.203,.077)--(-.206,.057);
\filldraw[fill opacity=0.5,fill=gray!20,draw=none](-.201,.048)--(-.192,.048)--(-.224,.087)--cycle;
\filldraw[fill opacity=0.8,fill=gray!20,draw=none](-.136,.006)--(-.146,.01)--(-.146,-.057)--(-.131,-.014)--cycle;
\draw(-.146,.01)--(-.146,-.057);
\filldraw[fill opacity=0.8,fill=gray!20,draw=none](-.136,.006)--(-.207,.022)--(-.208,.019)--(-.131,-.014)--cycle;
\draw(-.208,.019)--(-.131,-.014);
\filldraw[fill opacity=0.8,fill=gray!20,draw=none](-.136,.006)--(-.137,.01)--(-.193,.048)--(-.207,.022)--cycle;
\filldraw[fill opacity=0.8,fill=gray!20,draw=none](-.075,-.127)--(-.111,-.111)--(-.065,-.114)--(0,-.143)--cycle;
\draw(-.075,-.127)--(-.111,-.111);
\draw(-.065,-.114)--(0,-.143);
\filldraw[fill opacity=0.8,fill=gray!20,draw=none](.075,-.092)--(.073,-.101)--(.046,-.092)--(.046,-.071)--cycle;
\draw(.046,-.092)--(.046,-.071);
\filldraw[fill opacity=0.8,fill=gray!20,draw=none](.046,-.071)--(.096,-.108)--(.025,-.116)--cycle;
\draw(.096,-.108)--(.025,-.116);
\filldraw[fill opacity=0.8,fill=gray!20,draw=none](.15,-.128)--(.154,-.126)--(.164,-.125)--(.189,-.125)--(.166,-.128)--cycle;
\draw(.154,-.126)--(.164,-.125)--(.189,-.125)--(.166,-.128);
\filldraw[fill opacity=0.8,fill=gray!20,draw=none](.146,-.13)--(.131,-.154)--(.131,-.129)--cycle;
\draw(.131,-.154)--(.131,-.129);
\filldraw[fill opacity=0.8,fill=gray!20,draw=none](.035,-.139)--(.131,-.129)--(.146,-.13)--(.075,-.137)--cycle;
\draw(.035,-.139)--(.131,-.129);
\draw(.146,-.13)--(.075,-.137);
\filldraw[fill opacity=0.8,fill=gray!20,draw=none](.15,-.128)--(.166,-.128)--(.146,-.13)--cycle;
\draw(.166,-.128)--(.146,-.13);
\filldraw[fill opacity=0.8,fill=gray!20,draw=none](.075,-.137)--(.013,-.11)--(.046,-.071)--(.139,-.111)--cycle;
\draw(.075,-.137)--(.013,-.11);
\draw(.046,-.071)--(.139,-.111);
\filldraw[fill opacity=0.8,fill=gray!20,draw=none](-.139,-.091)--(.046,-.071)--(.025,-.116)--(-.075,-.127)--cycle;
\draw(-.139,-.091)--(.046,-.071);
\draw(.025,-.116)--(-.075,-.127);
\filldraw[fill opacity=0.8,fill=gray!20,draw=none](-.131,-.014)--(-.213,.021)--(-.204,-.027)--(-.199,-.034)--(-.096,-.079)--cycle;
\draw(-.131,-.014)--(-.213,.021)--(-.204,-.027);
\draw(-.199,-.034)--(-.096,-.079);
\filldraw[fill opacity=0.8,fill=gray!20,draw=none](-.218,.044)--(-.199,.043)--(-.196,.014)--(-.222,.011)--cycle;
\draw(-.196,.014)--(-.222,.011);
\filldraw[fill opacity=0.8,fill=gray!20,draw=none](-.199,.043)--(-.188,.043)--(-.196,.014)--cycle;
\filldraw[fill opacity=0.5,fill=gray!20,draw=none](-.204,.048)--(-.204,.048)--(-.192,.047)--(-.192,.048)--cycle;
\draw(-.204,.048)--(-.192,.047);
\filldraw[fill opacity=0.5,fill=gray!20,draw=none](-.188,.045)--(-.192,.047)--(-.567,.06)--(-.558,.055)--cycle;
\draw(-.192,.047)--(-.567,.06)--(-.558,.055);
\filldraw[fill opacity=0.8,fill=gray!20,draw=none](-.226,.044)--(-.224,.063)--(-.181,.068)--(-.188,.043)--cycle;
\draw(-.226,.044)--(-.224,.063)--(-.181,.068);
\filldraw[fill opacity=0.8,fill=gray!20,draw=none](-.226,.044)--(-.218,.044)--(-.222,.011)--(-.229,.011)--cycle;
\draw(-.222,.011)--(-.229,.011)--(-.226,.044);
\filldraw[fill opacity=0.8,fill=gray!20,draw=none](-.166,-.081)--(-.132,-.102)--(-.075,-.127)--cycle;
\draw(-.166,-.081)--(-.132,-.102)--(-.075,-.127);
\filldraw[fill opacity=0.8,fill=gray!20,draw=none](0,-.143)--(-.018,-.135)--(.025,-.116)--cycle;
\draw(0,-.143)--(-.018,-.135);
\filldraw[fill opacity=0.8,fill=gray!20,draw=none](-.075,-.127)--(.025,-.116)--(.035,-.139)--(0,-.143)--cycle;
\draw(-.075,-.127)--(.025,-.116);
\draw(.035,-.139)--(0,-.143);
\filldraw[fill opacity=0.5,fill=gray!20,draw=none](-.213,.037)--(-.179,.038)--(-.188,.045)--(-.226,.046)--cycle;
\filldraw[fill opacity=0.8,fill=gray!20,draw=none](-.078,.042)--(-.111,.041)--(-.111,.076)--cycle;
\draw(-.111,.041)--(-.111,.076);
\filldraw[fill opacity=0.8,fill=gray!20,draw=none](-.046,.071)--(-.149,.115)--(-.174,.065)--(-.096,.031)--cycle;
\draw(-.046,.071)--(-.149,.115);
\draw(-.174,.065)--(-.096,.031);
\filldraw[fill opacity=0.8,fill=gray!20,draw=none](-.167,.094)--(-.139,.111)--(-.181,.068)--cycle;
\filldraw[fill opacity=0.8,fill=gray!20,draw=none](-.21,.103)--(-.159,.109)--(-.181,.068)--(-.224,.063)--cycle;
\draw(-.181,.068)--(-.224,.063)--(-.21,.103)--(-.159,.109);
\filldraw[fill opacity=0.5,fill=gray!20,draw=none](-.168,.025)--(-.161,.027)--(-.179,.038)--(-.188,.038)--cycle;
\filldraw[fill opacity=0.8,fill=gray!20,draw=none](-.22,.012)--(-.196,.014)--(-.181,-.041)--cycle;
\draw(-.22,.012)--(-.196,.014);
\filldraw[fill opacity=0.5,fill=gray!20,draw=none](-.164,.006)--(-.134,.005)--(-.188,.038)--(-.213,.037)--cycle;
\draw(-.164,.006)--(-.134,.005);
\filldraw[fill opacity=0.8,fill=gray!20,draw=none](-.189,.125)--(-.146,.13)--(-.159,.109)--(-.21,.103)--cycle;
\draw(-.159,.109)--(-.21,.103)--(-.189,.125)--(-.146,.13);
\filldraw[fill opacity=0.8,fill=gray!20,draw=none](-.171,.01)--(-.171,.125)--(-.189,.125)--(-.21,.103)--(-.224,.063)--(-.226,.046)--cycle;
\draw(-.171,.125)--(-.189,.125)--(-.21,.103)--(-.224,.063)--(-.226,.046);
\filldraw[fill opacity=0.8,fill=gray!20,draw=none](-.167,.094)--(-.159,.109)--(-.139,.111)--cycle;
\draw(-.159,.109)--(-.139,.111);
\filldraw[fill opacity=0.8,fill=gray!20,draw=none](-.143,.122)--(-.139,.111)--(-.159,.109)--cycle;
\draw(-.139,.111)--(-.159,.109);
\filldraw[fill opacity=0.8,fill=gray!20,draw=none](-.143,.131)--(-.141,.132)--(-.139,.111)--(-.146,.052)--cycle;
\filldraw[fill opacity=0.8,fill=gray!20,draw=none](-.141,.132)--(-.139,.133)--(-.139,.111)--cycle;
\draw(-.139,.133)--(-.139,.111);
\filldraw[fill opacity=0.8,fill=gray!20,draw=none](-.143,.122)--(-.075,.137)--(-.139,.111)--cycle;
\filldraw[fill opacity=0.8,fill=gray!20,draw=none](-.146,.13)--(-.143,.122)--(-.159,.109)--cycle;
\filldraw[fill opacity=0.8,fill=gray!20,draw=none](0,.143)--(-.018,.151)--(-.075,.137)--(-.025,.116)--cycle;
\draw(0,.143)--(-.018,.151);
\draw(-.075,.137)--(-.025,.116);
\filldraw[fill opacity=0.8,fill=gray!20,draw=none](-.143,.122)--(-.132,.131)--(-.075,.137)--cycle;
\draw(-.132,.131)--(-.075,.137);
\filldraw[fill opacity=0.8,fill=gray!20,draw=none](-.143,.131)--(-.146,.052)--(-.146,.13)--cycle;
\draw(-.146,.052)--(-.146,.13);
\filldraw[fill opacity=0.8,fill=gray!20,draw=none](-.146,.13)--(-.132,.131)--(-.143,.122)--cycle;
\draw(-.146,.13)--(-.132,.131);
\filldraw[fill opacity=0.8,fill=gray!20,draw=none](-.096,.108)--(.099,.129)--(.067,.15)--(-.131,.129)--cycle;
\draw(-.096,.108)--(.099,.129);
\draw(.067,.15)--(-.131,.129);
\filldraw[fill opacity=0.8,fill=gray!20,draw=none](-.14,.186)--(-.143,.131)--(-.146,.13)--(-.146,.176)--cycle;
\draw(-.146,.13)--(-.146,.176)--(-.14,.186);
\filldraw[fill opacity=0.8,fill=gray!20,draw=none](.05,.149)--(-.036,.187)--(-.145,.175)--(-.131,.164)--(-.096,.154)--(-.046,.148)--(.01,.147)--cycle;
\draw(-.145,.175)--(-.131,.164)--(-.096,.154)--(-.046,.148)--(.01,.147)--(.05,.149);
\filldraw[fill opacity=0.8,fill=gray!20,draw=none](-.143,.131)--(-.142,.139)--(-.141,.132)--cycle;
\filldraw[fill opacity=0.8,fill=gray!20,draw=none](-.131,.129)--(-.035,.139)--(-.075,.137)--(-.146,.13)--cycle;
\draw(-.131,.129)--(-.035,.139);
\draw(-.075,.137)--(-.146,.13);
\filldraw[fill opacity=0.8,fill=gray!20,draw=none](-.035,.139)--(0,.143)--(-.075,.137)--cycle;
\draw(-.035,.139)--(0,.143);
\filldraw[fill opacity=0.8,fill=gray!20,draw=none](-.121,.178)--(-.14,.186)--(-.146,.176)--(-.145,.175)--cycle;
\draw(-.14,.186)--(-.146,.176)--(-.145,.175);
\filldraw[fill opacity=0.5,fill=gray!20,draw=none](-.096,-.002)--(-.046,-.007)--(.01,-.012)--(.05,-.014)--(.064,-.015)--(0,0)--(-.131,.005)--cycle;
\draw(.064,-.015)--(0,0)--(-.131,.005);
\filldraw[fill opacity=0.8,fill=gray!20,draw=none](-.128,.004)--(-.118,.002)--(-.096,.031)--(-.146,.052)--(-.137,.01)--cycle;
\draw(-.096,.031)--(-.146,.052);
\filldraw[fill opacity=0.8,fill=gray!20,draw=none](-.135,.168)--(-.145,.175)--(-.146,.168)--(-.146,.052)--(-.131,.024)--(-.131,.154)--cycle;
\draw(-.135,.168)--(-.145,.175);
\draw(-.146,.168)--(-.146,.052);
\draw(-.131,.024)--(-.131,.154);
\filldraw[fill opacity=0.5,fill=gray!20,draw=none](.118,.038)--(.128,.032)--(.131,.033)--cycle;
\draw(.128,.032)--(.131,.033);
\filldraw[fill opacity=0.8,fill=gray!20,draw=none](.087,.121)--(.075,.127)--(0,.143)--(.033,.128)--cycle;
\draw(.087,.121)--(.075,.127);
\draw(0,.143)--(.033,.128);
\filldraw[fill opacity=0.8,fill=gray!20,draw=none](.065,.114)--(0,.143)--(-.025,.116)--(.01,.1)--cycle;
\draw(.065,.114)--(0,.143);
\draw(-.025,.116)--(.01,.1);
\filldraw[fill opacity=0.8,fill=gray!20,draw=none](.139,.113)--(.139,.091)--(.143,.116)--cycle;
\draw(.139,.113)--(.139,.091);
\filldraw[fill opacity=0.8,fill=gray!20,draw=none](.139,.091)--(.159,.093)--(.143,.116)--cycle;
\draw(.139,.091)--(.159,.093);
\filldraw[fill opacity=0.8,fill=gray!20,draw=none](-.186,.125)--(-.164,.125)--(-.131,.129)--(-.146,.13)--(-.181,.126)--cycle;
\draw(-.186,.125)--(-.164,.125)--(-.131,.129);
\draw(-.146,.13)--(-.181,.126);
\filldraw[fill opacity=0.8,fill=gray!20,draw=none](-.145,.175)--(-.146,.176)--(-.146,.168)--cycle;
\draw(-.145,.175)--(-.146,.176)--(-.146,.168);
\filldraw[fill opacity=0.8,fill=gray!20,draw=none](.097,.116)--(.075,.127)--(.087,.121)--cycle;
\draw(.075,.127)--(.087,.121);
\filldraw[fill opacity=0.8,fill=gray!20,draw=none](-.128,.004)--(-.137,.01)--(-.136,.006)--cycle;
\filldraw[fill opacity=0.5,fill=gray!20,draw=none](-.137,.01)--(-.131,.005)--(-.128,.004)--cycle;
\draw(-.131,.005)--(-.128,.004);
\filldraw[fill opacity=0.8,fill=gray!20,draw=none](-.12,-.001)--(-.128,.004)--(-.136,.006)--(-.131,-.014)--cycle;
\filldraw[fill opacity=0.8,fill=gray!20,draw=none](-.186,.125)--(-.181,.126)--(-.189,.125)--cycle;
\draw(-.181,.126)--(-.189,.125)--(-.186,.125);
\filldraw[fill opacity=0.8,fill=gray!20,draw=none](-.131,-.014)--(-.131,-.053)--(-.096,-.079)--(-.096,.031)--cycle;
\draw(-.131,-.014)--(-.131,-.053);
\draw(-.096,-.079)--(-.096,.031);
\filldraw[fill opacity=0.8,fill=gray!20,draw=none](-.106,0)--(-.128,.004)--(-.109,-.008)--cycle;
\filldraw[fill opacity=0.8,fill=gray!20,draw=none](-.131,.154)--(-.131,-.014)--(-.096,.031)--(-.096,.108)--cycle;
\draw(-.131,.154)--(-.131,-.014);
\draw(-.096,.031)--(-.096,.108);
\filldraw[fill opacity=0.5,fill=gray!20,draw=none](.15,.031)--(.199,.02)--(.226,.027)--(.164,.041)--(.154,.039)--cycle;
\draw(.164,.041)--(.154,.039);
\filldraw[fill opacity=0.5,fill=gray!20,draw=none](.15,.031)--(.154,.039)--(.134,.034)--cycle;
\draw(.154,.039)--(.134,.034);
\filldraw[fill opacity=0.8,fill=gray!20,draw=none](.122,.002)--(.131,.014)--(.131,.004)--cycle;
\draw(.131,.014)--(.131,.004);
\filldraw[fill opacity=0.8,fill=gray!20,draw=none](.181,.041)--(.201,.044)--(.196,.097)--(.159,.093)--cycle;
\draw(.181,.041)--(.201,.044);
\draw(.196,.097)--(.159,.093);
\filldraw[fill opacity=0.8,fill=gray!20,draw=none](.111,.111)--(.139,.091)--(.139,.113)--cycle;
\draw(.139,.091)--(.139,.113);
\filldraw[fill opacity=0.8,fill=gray!20,draw=none](.097,.116)--(.129,.1)--(.135,.095)--(.139,.091)--cycle;
\filldraw[fill opacity=0.8,fill=gray!20,draw=none](0,-.143)--(.035,-.139)--(.075,-.137)--cycle;
\draw(0,-.143)--(.035,-.139);
\filldraw[fill opacity=0.8,fill=gray!20,draw=none](.018,-.151)--(0,-.143)--(.025,-.116)--(.075,-.137)--cycle;
\draw(.018,-.151)--(0,-.143);
\draw(.025,-.116)--(.075,-.137);
\filldraw[fill opacity=0.8,fill=gray!20,draw=none](-.137,.01)--(-.136,.006)--(-.131,.005)--cycle;
\filldraw[fill opacity=0.5,fill=gray!20,draw=none](-.137,.01)--(-.161,.027)--(-.168,.025)--(-.145,.011)--(-.131,.005)--cycle;
\filldraw[fill opacity=0.8,fill=gray!20,draw=none](-.146,.052)--(-.137,.01)--(-.131,.005)--(-.131,.024)--cycle;
\draw(-.131,.005)--(-.131,.024);
\filldraw[fill opacity=0.8,fill=gray!20,draw=none](-.136,.006)--(-.131,-.014)--(-.131,.005)--cycle;
\draw(-.131,-.014)--(-.131,.005);
\filldraw[fill opacity=0.5,fill=gray!20,draw=none](-.145,.011)--(-.134,.005)--(-.131,.005)--cycle;
\draw(-.134,.005)--(-.131,.005);
\filldraw[fill opacity=0.8,fill=gray!20,draw=none](-.109,-.008)--(-.12,-.001)--(-.131,-.014)--(-.113,-.022)--cycle;
\draw(-.131,-.014)--(-.113,-.022);
\filldraw[fill opacity=0.5,fill=gray!20,draw=none](-.096,-.002)--(-.131,.005)--(-.154,.005)--cycle;
\draw(-.131,.005)--(-.154,.005);
\filldraw[fill opacity=0.8,fill=gray!20,draw=none](-.096,-.079)--(-.203,-.032)--(-.174,-.075)--(-.046,-.131)--cycle;
\draw(-.096,-.079)--(-.203,-.032)--(-.174,-.075)--(-.046,-.131);
\filldraw[fill opacity=0.8,fill=gray!20,draw=none](-.096,-.02)--(-.096,-.079)--(-.086,-.11)--(-.046,-.131)--(-.046,-.016)--cycle;
\draw(-.096,-.02)--(-.096,-.079);
\draw(-.046,-.131)--(-.046,-.016);
\filldraw[fill opacity=0.8,fill=gray!20,draw=none](-.075,.092)--(-.096,.031)--(-.096,-.02)--(-.046,-.016)--(-.046,.071)--cycle;
\draw(-.096,.031)--(-.096,-.02);
\draw(-.046,-.016)--(-.046,.071);
\filldraw[fill opacity=0.8,fill=gray!20,draw=none](-.063,-.012)--(-.063,-.01)--(-.106,0)--(-.109,-.008)--(-.095,-.017)--cycle;
\filldraw[fill opacity=0.5,fill=gray!20,draw=none](-.102,-.009)--(-.074,-.01)--(-.12,.001)--(-.154,.005)--(-.164,.006)--cycle;
\draw(-.154,.005)--(-.164,.006);
\filldraw[fill opacity=0.8,fill=gray!20,draw=none](-.171,.01)--(-.164,.006)--(-.141,.001)--(-.141,.105)--(-.164,.125)--(-.171,.125)--cycle;
\draw(-.141,.105)--(-.164,.125)--(-.171,.125);
\filldraw[fill opacity=0.5,fill=gray!20,draw=none](.137,.03)--(.147,.027)--(.15,.031)--(.134,.034)--(.131,.033)--cycle;
\draw(.134,.034)--(.131,.033);
\filldraw[fill opacity=0.8,fill=gray!20,draw=none](.146,.057)--(.137,.03)--(.131,.033)--(.131,.053)--cycle;
\draw(.131,.033)--(.131,.053);
\filldraw[fill opacity=0.8,fill=gray!20,draw=none](.133,.005)--(.131,.014)--(.147,.007)--cycle;
\draw(.131,.014)--(.147,.007);
\filldraw[fill opacity=0.5,fill=gray!20,draw=none](.137,.03)--(.131,.033)--(.128,.032)--cycle;
\draw(.131,.033)--(.128,.032);
\filldraw[fill opacity=0.8,fill=gray!20,draw=none](.137,.03)--(.128,.032)--(.118,.038)--(.096,.079)--(.146,.057)--cycle;
\draw(.096,.079)--(.146,.057);
\filldraw[fill opacity=0.8,fill=gray!20,draw=none](.129,.1)--(.097,.116)--(.087,.121)--(.111,.111)--cycle;
\draw(.087,.121)--(.111,.111);
\filldraw[fill opacity=0.8,fill=gray!20,draw=none](.139,-.111)--(.046,-.071)--(.096,-.031)--(.146,-.052)--cycle;
\draw(.139,-.111)--(.046,-.071);
\draw(.096,-.031)--(.146,-.052);
\filldraw[fill opacity=0.8,fill=gray!20,draw=none](.104,.114)--(.087,.121)--(.033,.128)--(.052,.12)--cycle;
\draw(.104,.114)--(.087,.121);
\draw(.033,.128)--(.052,.12);
\filldraw[fill opacity=0.8,fill=gray!20,draw=none](-.035,-.147)--(0,-.143)--(-.025,-.148)--cycle;
\draw(-.035,-.147)--(0,-.143);
\filldraw[fill opacity=0.8,fill=gray!20,draw=none](-.025,-.148)--(-.075,-.127)--(0,-.143)--cycle;
\draw(-.025,-.148)--(-.075,-.127);
\filldraw[fill opacity=0.8,fill=gray!20,draw=none](-.046,-.131)--(-.174,-.075)--(-.166,-.081)--(-.075,-.127)--(-.025,-.148)--cycle;
\draw(-.046,-.131)--(-.174,-.075)--(-.166,-.081);
\draw(-.075,-.127)--(-.025,-.148);
\filldraw[fill opacity=0.8,fill=gray!20,draw=none](-.111,-.111)--(-.111,-.158)--(-.139,-.168)--(-.139,-.091)--cycle;
\draw(-.111,-.111)--(-.111,-.158)--(-.139,-.168)--(-.139,-.091);
\filldraw[fill opacity=0.8,fill=gray!20,draw=none](-.09,-.124)--(-.111,-.133)--(-.111,-.111)--cycle;
\draw(-.111,-.133)--(-.111,-.111);
\filldraw[fill opacity=0.8,fill=gray!20,draw=none](-.135,.168)--(-.131,.154)--(-.131,.164)--cycle;
\draw(-.131,.154)--(-.131,.164)--(-.135,.168);
\filldraw[fill opacity=0.8,fill=gray!20,draw=none](-.116,.106)--(-.096,.108)--(-.131,.129)--cycle;
\draw(-.116,.106)--(-.096,.108);
\filldraw[fill opacity=0.8,fill=gray!20,draw=none](-.157,.119)--(-.139,.103)--(-.116,.106)--(-.131,.129)--(-.154,.126)--cycle;
\draw(-.157,.119)--(-.139,.103)--(-.116,.106);
\draw(-.131,.129)--(-.154,.126);
\filldraw[fill opacity=0.8,fill=gray!20,draw=none](.136,.027)--(.148,.02)--(.148,.008)--(.147,.007)--(.131,.014)--cycle;
\draw(.147,.007)--(.131,.014);
\filldraw[fill opacity=0.8,fill=gray!20,draw=none](.147,.027)--(.148,.02)--(.128,.032)--cycle;
\filldraw[fill opacity=0.5,fill=gray!20,draw=none](.137,.03)--(.146,.025)--(.147,.027)--cycle;
\filldraw[fill opacity=0.8,fill=gray!20,draw=none](-.099,.155)--(-.131,.164)--(-.131,.154)--(-.096,.108)--(-.096,.146)--cycle;
\draw(-.099,.155)--(-.131,.164)--(-.131,.154);
\draw(-.096,.108)--(-.096,.146);
\filldraw[fill opacity=0.8,fill=gray!20,draw=none](.137,.03)--(.146,.025)--(.146,.007)--(.131,.004)--(.131,.014)--cycle;
\draw(.146,.025)--(.146,.007);
\draw(.131,.004)--(.131,.014);
\filldraw[fill opacity=0.8,fill=gray!20,draw=none](.133,.005)--(.122,.002)--(.131,.014)--cycle;
\filldraw[fill opacity=0.8,fill=gray!20,draw=none](.145,-.046)--(.133,.005)--(.147,.007)--(.196,-.014)--cycle;
\draw(.147,.007)--(.196,-.014);
\filldraw[fill opacity=0.8,fill=gray!20,draw=none](.181,-.068)--(.146,-.052)--(.145,-.046)--(.196,-.014)--cycle;
\draw(.181,-.068)--(.146,-.052);
\filldraw[fill opacity=0.8,fill=gray!20,draw=none](.133,.005)--(.146,.007)--(.146,-.052)--cycle;
\draw(.146,.007)--(.146,-.052);
\filldraw[fill opacity=0.8,fill=gray!20,draw=none](.148,.02)--(.163,.011)--(.148,.008)--cycle;
\filldraw[fill opacity=0.8,fill=gray!20,draw=none](.147,.027)--(.188,.017)--(.188,.016)--(.163,.011)--(.148,.02)--cycle;
\filldraw[fill opacity=0.8,fill=gray!20,draw=none](.111,-.001)--(.111,-.004)--(.139,.004)--(.139,.013)--cycle;
\draw(.111,-.001)--(.111,-.004);
\draw(.139,.004)--(.139,.013);
\filldraw[fill opacity=0.5,fill=gray!20,draw=none](.078,-.011)--(.135,.006)--(.146,.025)--(.137,.03)--(.128,.032)--(0,0)--(.064,-.015)--cycle;
\draw(.128,.032)--(0,0)--(.064,-.015);
\filldraw[fill opacity=0.5,fill=gray!20,draw=none](.147,.027)--(.188,.017)--(.199,.02)--(.15,.031)--cycle;
\filldraw[fill opacity=0.5,fill=gray!20,draw=none](-.104,-.003)--(-.077,-.005)--(-.096,-.002)--(-.12,.001)--cycle;
\filldraw[fill opacity=0.8,fill=gray!20,draw=none](.01,.023)--(.111,.034)--(.139,.091)--(-.046,.071)--cycle;
\draw(.01,.023)--(.111,.034);
\draw(.139,.091)--(-.046,.071);
\filldraw[fill opacity=0.8,fill=gray!20,draw=none](-.065,-.037)--(-.109,-.008)--(-.113,-.022)--(-.065,-.043)--cycle;
\draw(-.113,-.022)--(-.065,-.043);
\filldraw[fill opacity=0.8,fill=gray!20,draw=none](.065,-.029)--(.111,.034)--(.01,.023)--cycle;
\draw(.111,.034)--(.01,.023);
\filldraw[fill opacity=0.8,fill=gray!20,draw=none](.065,.114)--(.065,.01)--(.111,.034)--(.111,.111)--cycle;
\draw(.065,.114)--(.065,.01);
\draw(.111,.034)--(.111,.111);
\filldraw[fill opacity=0.8,fill=gray!20,draw=none](-.157,.119)--(-.154,.126)--(-.164,.125)--cycle;
\draw(-.154,.126)--(-.164,.125)--(-.157,.119);
\filldraw[fill opacity=0.8,fill=gray!20,draw=none](.095,.11)--(.111,.111)--(.104,.114)--(.052,.12)--(.065,.114)--cycle;
\draw(.111,.111)--(.104,.114);
\draw(.052,.12)--(.065,.114);
\filldraw[fill opacity=0.8,fill=gray!20,draw=none](.065,.15)--(.065,.114)--(.111,.111)--(.111,.158)--cycle;
\draw(.111,.111)--(.111,.158)--(.065,.15)--(.065,.114);
\filldraw[fill opacity=0.8,fill=gray!20,draw=none](.01,.1)--(.01,-.012)--(.065,-.015)--(.065,.14)--cycle;
\draw(.01,.1)--(.01,-.012);
\draw(.065,-.015)--(.065,.14);
\filldraw[fill opacity=0.8,fill=gray!20,draw=none](-.075,.092)--(-.096,.108)--(-.096,.031)--cycle;
\draw(-.096,.108)--(-.096,.031);
\filldraw[fill opacity=0.8,fill=gray!20,draw=none](-.117,.063)--(-.046,.071)--(-.025,.116)--(-.139,.103)--cycle;
\draw(-.025,.116)--(-.139,.103)--(-.117,.063)--(-.046,.071);
\filldraw[fill opacity=0.8,fill=gray!20,draw=none](-.075,.092)--(-.07,.107)--(-.096,.146)--(-.096,.108)--cycle;
\draw(-.096,.146)--(-.096,.108);
\filldraw[fill opacity=0.8,fill=gray!20,draw=none](-.099,.155)--(-.096,.146)--(-.096,.154)--cycle;
\draw(-.096,.146)--(-.096,.154)--(-.099,.155);
\filldraw[fill opacity=0.8,fill=gray!20,draw=none](-.096,.154)--(-.096,.146)--(-.046,.071)--(-.046,.148)--cycle;
\draw(-.046,.071)--(-.046,.148)--(-.096,.154)--(-.096,.146);
\filldraw[fill opacity=0.8,fill=gray!20,draw=none](.095,.11)--(.065,.114)--(.077,.109)--cycle;
\draw(.065,.114)--(.077,.109);
\filldraw[fill opacity=0.8,fill=gray!20,draw=none](-.046,.071)--(.01,.048)--(.01,.1)--cycle;
\draw(.01,.048)--(.01,.1);
\filldraw[fill opacity=0.8,fill=gray!20,draw=none](-.002,.062)--(.032,.091)--(-.025,.116)--cycle;
\draw(-.002,.062)--(.032,.091)--(-.025,.116);
\filldraw[fill opacity=0.8,fill=gray!20,draw=none](-.013,.088)--(-.02,.103)--(-.046,.071)--cycle;
\filldraw[fill opacity=0.8,fill=gray!20,draw=none](-.035,.148)--(-.046,.148)--(-.046,.071)--(-.013,.088)--cycle;
\draw(-.035,.148)--(-.046,.148)--(-.046,.071);
\filldraw[fill opacity=0.8,fill=gray!20,draw=none](.065,.114)--(.01,.1)--(.02,.096)--cycle;
\draw(.01,.1)--(.02,.096);
\filldraw[fill opacity=0.8,fill=gray!20,draw=none](.082,.107)--(.065,.114)--(.02,.096)--(.032,.091)--cycle;
\draw(.02,.096)--(.032,.091)--(.082,.107)--(.065,.114);
\filldraw[fill opacity=0.8,fill=gray!20,draw=none](.01,.147)--(.01,.1)--(.065,.14)--(.065,.15)--cycle;
\draw(.065,.14)--(.065,.15)--(.01,.147)--(.01,.1);
\filldraw[fill opacity=0.8,fill=gray!20,draw=none](-.035,.148)--(-.013,.088)--(.01,.1)--(.01,.147)--cycle;
\draw(.01,.1)--(.01,.147)--(-.035,.148);
\filldraw[fill opacity=0.5,fill=gray!20](.243,.106)--(-.243,.318)--(-.696,.334)--(-.21,.122)--cycle;
\filldraw[fill opacity=0.5,fill=gray!20](1.47,.749)--(1.433,.751)--(1.032,.373)--(1.078,.378)--cycle;
\filldraw[fill opacity=0.5,fill=gray!20](1.96,2.799)--(1.932,2.841)--(1.998,2.303)--(2.025,2.271)--cycle;
\filldraw[fill opacity=0.5,fill=gray!20](1.739,3.329)--(1.684,3.366)--(1.878,2.875)--(1.932,2.841)--cycle;
\filldraw[fill opacity=0.5,fill=gray!20](.726,4.036)--(.632,3.977)--(.994,3.713)--(1.106,3.76)--cycle;
\filldraw[fill opacity=0.5,fill=gray!20](.816,4.071)--(.726,4.036)--(1.106,3.76)--(1.208,3.786)--cycle;
\filldraw[fill opacity=0.5,fill=gray!20](1.118,3.764)--(.632,3.976)--(.994,3.713)--(1.48,3.5)--cycle;
\filldraw[fill opacity=0.5,fill=gray!20](1.777,1.212)--(1.771,1.21)--(1.47,.749)--(1.486,.766)--cycle;
\filldraw[fill opacity=0.5,fill=gray!20](1.961,1.714)--(1.96,1.729)--(1.771,1.21)--(1.777,1.212)--cycle;
\filldraw[fill opacity=0.5,fill=gray!20](2.023,2.238)--(2.025,2.271)--(1.96,1.729)--(1.961,1.714)--cycle;
\filldraw[fill opacity=0.5,fill=gray!20](1.078,.378)--(1.032,.373)--(.567,.111)--(.621,.122)--cycle;
\filldraw[fill opacity=0.8,fill=gray!20,draw=none](.153,2.004)--(.032,2.057)--(.06,2.106)--(.182,2.053)--cycle;
\draw(.153,2.004)--(.032,2.057);
\draw(.06,2.106)--(.182,2.053);
\filldraw[fill opacity=0.8,fill=gray!20,draw=none](.182,2.053)--(.06,2.106)--(.07,2.162)--(.192,2.109)--cycle;
\draw(.182,2.053)--(.06,2.106);
\draw(.07,2.162)--(.192,2.109);
\filldraw[fill opacity=0.8,fill=gray!20,draw=none](.111,1.968)--(-.011,2.021)--(.032,2.057)--(.153,2.004)--cycle;
\draw(.111,1.968)--(-.011,2.021);
\draw(.032,2.057)--(.153,2.004);
\filldraw[fill opacity=0.8,fill=gray!20,draw=none](.192,2.109)--(.07,2.162)--(.06,2.216)--(.182,2.163)--cycle;
\draw(.192,2.109)--(.07,2.162);
\draw(.06,2.216)--(.182,2.163);
\filldraw[fill opacity=0.8,fill=gray!20,draw=none](.061,1.952)--(-.061,2.005)--(-.011,2.021)--(.111,1.968)--cycle;
\draw(.061,1.952)--(-.061,2.005);
\draw(-.011,2.021)--(.111,1.968);
\filldraw[fill opacity=0.8,fill=gray!20,draw=none](.182,2.163)--(.06,2.216)--(.032,2.259)--(.153,2.206)--cycle;
\draw(.182,2.163)--(.06,2.216);
\draw(.032,2.259)--(.153,2.206);
\filldraw[fill opacity=0.8,fill=gray!20,draw=none](-.066,1.991)--(-.111,2.01)--(-.061,2.005)--(-.032,1.993)--cycle;
\draw(-.066,1.991)--(-.111,2.01);
\draw(-.061,2.005)--(-.032,1.993);
\filldraw[fill opacity=0.8,fill=gray!20,draw=none](.153,2.206)--(.032,2.259)--(-.011,2.285)--(.111,2.232)--cycle;
\draw(.153,2.206)--(.032,2.259);
\draw(-.011,2.285)--(.111,2.232);
\filldraw[fill opacity=0.8,fill=gray!20,draw=none](-.099,2.01)--(-.14,2.031)--(-.153,2.037)--(-.111,2.01)--(-.061,1.989)--cycle;
\draw(-.14,2.031)--(-.153,2.037);
\draw(-.111,2.01)--(-.061,1.989);
\filldraw[fill opacity=0.8,fill=gray!20,draw=none](-.189,2.132)--(-.188,2.143)--(-.059,2.147)--(-.058,2.096)--(-.178,2.092)--cycle;
\draw(-.188,2.143)--(-.059,2.147)--(-.058,2.096)--(-.178,2.092);
\filldraw[fill opacity=0.8,fill=gray!20,draw=none](-.171,2.075)--(-.178,2.092)--(-.058,2.096)--(-.048,2.048)--(-.145,2.045)--cycle;
\draw(-.178,2.092)--(-.058,2.096)--(-.048,2.048)--(-.145,2.045);
\filldraw[fill opacity=0.8,fill=gray!20](.051,2.048)--(.036,2.012)--(.015,1.992)--(-.009,1.992)--(-.03,2.012)--(-.048,2.048)--(-.058,2.096)--(-.059,2.147)--(-.051,2.195)--(-.036,2.231)--(-.015,2.251)--(.009,2.251)--(.03,2.231)--(.048,2.195)--(.058,2.147)--(.059,2.096)--cycle;
\filldraw[fill opacity=0.8,fill=gray!20,draw=none](-.176,2.187)--(-.173,2.191)--(-.051,2.195)--(-.059,2.147)--(-.188,2.143)--cycle;
\draw(-.173,2.191)--(-.051,2.195)--(-.059,2.147)--(-.188,2.143);
\filldraw[fill opacity=0.8,fill=gray!20,draw=none](.111,2.232)--(-.011,2.285)--(-.061,2.291)--(.061,2.238)--cycle;
\draw(.111,2.232)--(-.011,2.285);
\draw(-.061,2.291)--(.061,2.238);
\filldraw[fill opacity=0.8,fill=gray!20,draw=none](-.099,2.01)--(-.122,2.023)--(-.14,2.031)--cycle;
\draw(-.122,2.023)--(-.14,2.031);
\filldraw[fill opacity=0.8,fill=gray!20,draw=none](-.145,2.045)--(-.048,2.048)--(-.03,2.012)--(-.108,2.01)--cycle;
\draw(-.145,2.045)--(-.048,2.048)--(-.03,2.012)--(-.108,2.01);
\filldraw[fill opacity=0.8,fill=gray!20,draw=none](-.136,2.228)--(-.036,2.231)--(-.051,2.195)--(-.173,2.191)--cycle;
\draw(-.136,2.228)--(-.036,2.231)--(-.051,2.195)--(-.173,2.191);
\filldraw[fill opacity=0.8,fill=gray!20,draw=none](-.157,2.043)--(-.153,2.037)--(-.14,2.031)--cycle;
\draw(-.153,2.037)--(-.14,2.031);
\filldraw[fill opacity=0.8,fill=gray!20,draw=none](-.039,1.979)--(-.066,1.991)--(-.032,1.993)--(.006,1.976)--cycle;
\draw(-.039,1.979)--(-.066,1.991);
\draw(-.032,1.993)--(.006,1.976);
\filldraw[fill opacity=0.8,fill=gray!20,draw=none](-.099,2.01)--(-.03,2.012)--(-.009,1.992)--(-.061,1.991)--cycle;
\draw(-.099,2.01)--(-.03,2.012)--(-.009,1.992)--(-.061,1.991);
\filldraw[fill opacity=0.8,fill=gray!20,draw=none](-.062,2.009)--(-.071,2.032)--(-.182,2.08)--(-.157,2.043)--(-.14,2.031)--(-.063,1.997)--cycle;
\draw(-.071,2.032)--(-.182,2.08);
\draw(-.14,2.031)--(-.063,1.997);
\filldraw[fill opacity=0.8,fill=gray!20,draw=none](-.006,2.267)--(-.061,2.291)--(-.111,2.274)--(-.067,2.255)--cycle;
\draw(-.006,2.267)--(-.061,2.291);
\draw(-.111,2.274)--(-.067,2.255);
\filldraw[fill opacity=0.8,fill=gray!20,draw=none](-.188,2.116)--(-.182,2.08)--(-.171,2.075)--cycle;
\draw(-.182,2.08)--(-.171,2.075);
\filldraw[fill opacity=0.8,fill=gray!20,draw=none](-.084,2.249)--(-.067,2.255)--(-.111,2.274)--(-.142,2.248)--cycle;
\draw(-.067,2.255)--(-.111,2.274);
\filldraw[fill opacity=0.8,fill=gray!20,draw=none](-.084,2.249)--(-.142,2.248)--(-.153,2.239)--(-.133,2.23)--cycle;
\draw(-.153,2.239)--(-.133,2.23);
\filldraw[fill opacity=0.8,fill=gray!20,draw=none](-.1,2.248)--(-.015,2.251)--(-.036,2.231)--(-.136,2.228)--cycle;
\draw(-.1,2.248)--(-.015,2.251)--(-.036,2.231)--(-.136,2.228);
\filldraw[fill opacity=0.8,fill=gray!20,draw=none](-.069,2.045)--(-.073,2.082)--(-.192,2.134)--(-.188,2.116)--(-.171,2.075)--(-.071,2.032)--cycle;
\draw(-.073,2.082)--(-.192,2.134);
\draw(-.171,2.075)--(-.071,2.032);
\filldraw[fill opacity=0.8,fill=gray!20,draw=none](-.186,2.164)--(-.192,2.134)--(-.189,2.132)--cycle;
\draw(-.192,2.134)--(-.189,2.132);
\filldraw[fill opacity=0.8,fill=gray!20,draw=none](-1.189,2.068)--(-1.179,2.114)--(-.192,2.143)--(-.178,2.092)--(-1.188,2.062)--cycle;
\draw(-1.179,2.114)--(-.192,2.143);
\draw(-.178,2.092)--(-1.188,2.062);
\filldraw[fill opacity=0.8,fill=gray!20,draw=none](-.171,2.075)--(-.185,2.092)--(-.178,2.092)--cycle;
\draw(-.185,2.092)--(-.178,2.092);
\filldraw[fill opacity=0.8,fill=gray!20,draw=none](-1.176,2.022)--(-1.179,2.063)--(-.185,2.092)--(-.145,2.045)--(-1.174,2.015)--cycle;
\draw(-1.179,2.063)--(-.185,2.092);
\draw(-.145,2.045)--(-1.174,2.015);
\filldraw[fill opacity=0.8,fill=gray!20,draw=none](-.143,2.234)--(-.153,2.239)--(-.182,2.189)--(-.176,2.187)--cycle;
\draw(-.143,2.234)--(-.153,2.239);
\draw(-.182,2.189)--(-.176,2.187);
\filldraw[fill opacity=0.8,fill=gray!20,draw=none](-.07,2.092)--(-.066,2.139)--(-.182,2.189)--(-.186,2.164)--(-.189,2.132)--(-.073,2.082)--cycle;
\draw(-.066,2.139)--(-.182,2.189);
\draw(-.189,2.132)--(-.073,2.082);
\filldraw[fill opacity=0.8,fill=gray!20,draw=none](-1.188,2.125)--(-1.172,2.162)--(-.175,2.191)--(-.188,2.143)--(-1.19,2.114)--cycle;
\draw(-1.172,2.162)--(-.175,2.191);
\draw(-.188,2.143)--(-1.19,2.114);
\filldraw[fill opacity=0.8,fill=gray!20,draw=none](-.189,2.132)--(-.192,2.143)--(-.188,2.143)--cycle;
\draw(-.192,2.143)--(-.188,2.143);
\filldraw[fill opacity=0.8,fill=gray!20,draw=none](-1.174,2.015)--(-.145,2.045)--(-.108,2.01)--(-1.132,1.979)--cycle;
\draw(-1.174,2.015)--(-.145,2.045);
\draw(-.108,2.01)--(-1.132,1.979);
\filldraw[fill opacity=0.8,fill=gray!20,draw=none](-.064,2.144)--(-.052,2.192)--(-.053,2.195)--(-.143,2.234)--(-.176,2.187)--(-.066,2.139)--cycle;
\draw(-.053,2.195)--(-.143,2.234);
\draw(-.176,2.187)--(-.066,2.139);
\filldraw[fill opacity=0.8,fill=gray!20,draw=none](-.176,2.187)--(-.175,2.191)--(-.173,2.191)--cycle;
\draw(-.175,2.191)--(-.173,2.191);
\filldraw[fill opacity=0.8,fill=gray!20,draw=none](-.034,1.977)--(-.039,1.979)--(.006,1.976)--(.032,1.965)--cycle;
\draw(-.034,1.977)--(-.039,1.979);
\draw(.006,1.976)--(.032,1.965);
\filldraw[fill opacity=0.8,fill=gray!20,draw=none](-.054,1.99)--(-.063,1.997)--(-.122,2.023)--(-.065,1.991)--cycle;
\draw(-.063,1.997)--(-.122,2.023);
\filldraw[fill opacity=0.8,fill=gray!20,draw=none](-.137,2.228)--(-.136,2.228)--(-.173,2.191)--(-.202,2.19)--cycle;
\draw(-.137,2.228)--(-.136,2.228);
\draw(-.173,2.191)--(-.202,2.19);
\filldraw[fill opacity=0.8,fill=gray!20,draw=none](-1.142,1.979)--(-.099,2.01)--(-.061,1.991)--(-1.087,1.96)--cycle;
\draw(-1.142,1.979)--(-.099,2.01);
\draw(-.061,1.991)--(-1.087,1.96);
\filldraw[fill opacity=0.8,fill=gray!20,draw=none](-.054,1.99)--(-.065,1.991)--(-.061,1.989)--(-.039,1.979)--cycle;
\draw(-.061,1.989)--(-.039,1.979);
\filldraw[fill opacity=0.8,fill=gray!20,draw=none](-.061,1.991)--(-.009,1.992)--(.015,1.992)--(-.054,1.99)--cycle;
\draw(-.061,1.991)--(-.009,1.992)--(.015,1.992)--(-.054,1.99);
\filldraw[fill opacity=0.8,fill=gray!20,draw=none](-.052,2.249)--(-.047,2.249)--(.009,2.251)--(-.015,2.251)--(-.084,2.249)--cycle;
\draw(-.047,2.249)--(.009,2.251)--(-.015,2.251)--(-.084,2.249);
\filldraw[fill opacity=0.8,fill=gray!20,draw=none](-1.168,2.18)--(-1.153,2.198)--(-.137,2.228)--(-.202,2.19)--(-1.17,2.162)--cycle;
\draw(-1.153,2.198)--(-.137,2.228);
\draw(-.202,2.19)--(-1.17,2.162);
\filldraw[fill opacity=0.8,fill=gray!20,draw=none](-1.155,2.2)--(-1.117,2.218)--(-.1,2.248)--(-.136,2.228)--(-1.156,2.198)--cycle;
\draw(-1.117,2.218)--(-.1,2.248);
\draw(-.136,2.228)--(-1.156,2.198);
\filldraw[fill opacity=0.8,fill=gray!20,draw=none](-.039,2.229)--(-.048,2.247)--(-.067,2.255)--(-.133,2.23)--(-.053,2.195)--cycle;
\draw(-.048,2.247)--(-.067,2.255);
\draw(-.133,2.23)--(-.053,2.195);
\filldraw[fill opacity=0.8,fill=gray!20,draw=none](-.826,1.968)--(-.061,1.991)--(-.054,1.99)--(-.647,1.973)--cycle;
\draw(-.826,1.968)--(-.061,1.991);
\draw(-.054,1.99)--(-.647,1.973);
\filldraw[fill opacity=0.8,fill=gray!20,draw=none](-.052,2.249)--(-.084,2.249)--(-.103,2.248)--cycle;
\draw(-.084,2.249)--(-.103,2.248);
\filldraw[fill opacity=0.8,fill=gray!20,draw=none](.011,1.957)--(-.034,1.977)--(.032,1.965)--(.061,1.952)--cycle;
\draw(.011,1.957)--(-.034,1.977);
\draw(.032,1.965)--(.061,1.952);
\filldraw[fill opacity=0.8,fill=gray!20,draw=none](-.032,1.984)--(-.045,1.99)--(-.054,1.99)--(-.039,1.979)--(.011,1.957)--cycle;
\draw(-.032,1.984)--(-.045,1.99);
\draw(-.039,1.979)--(.011,1.957);
\filldraw[fill opacity=0.8,fill=gray!20,draw=none](-.041,2.26)--(-.067,2.255)--(-.048,2.247)--cycle;
\draw(-.067,2.255)--(-.048,2.247);
\filldraw[fill opacity=0.8,fill=gray!20,draw=none](.032,2.25)--(-.006,2.267)--(-.041,2.26)--(-.047,2.249)--cycle;
\draw(.032,2.25)--(-.006,2.267);
\filldraw[fill opacity=0.8,fill=gray!20,draw=none](-1.274,1.966)--(-1.274,1.968)--(-1.233,2.015)--(-1.178,2.011)--(-1.198,1.96)--cycle;
\draw(-1.233,2.015)--(-1.178,2.011)--(-1.198,1.96)--(-1.274,1.966)--(-1.274,1.968);
\filldraw[fill opacity=0.8,fill=gray!20,draw=none](-1.006,1.962)--(.015,1.992)--(.036,2.012)--(-.913,1.984)--cycle;
\draw(-1.006,1.962)--(.015,1.992)--(.036,2.012)--(-.913,1.984);
\filldraw[fill opacity=0.8,fill=gray!20,draw=none](-1.17,2.178)--(-1.168,2.18)--(-1.17,2.162)--(-1.177,2.161)--cycle;
\draw(-1.17,2.162)--(-1.177,2.161);
\filldraw[fill opacity=0.8,fill=gray!20,draw=none](-1.053,2.219)--(-.049,2.249)--(-.052,2.249)--(-.103,2.248)--(-1.117,2.218)--cycle;
\draw(-1.053,2.219)--(-.049,2.249);
\draw(-.103,2.248)--(-1.117,2.218);
\filldraw[fill opacity=0.8,fill=gray!20,draw=none](-.978,2.201)--(.03,2.231)--(.009,2.251)--(-1.053,2.219)--cycle;
\draw(-.978,2.201)--(.03,2.231)--(.009,2.251)--(-1.053,2.219);
\filldraw[fill opacity=0.8,fill=gray!20,draw=none](-.052,2.249)--(-.049,2.249)--(-.047,2.249)--cycle;
\draw(-.049,2.249)--(-.047,2.249);
\filldraw[fill opacity=0.8,fill=gray!20,draw=none](-.017,2.249)--(-.047,2.249)--(-.048,2.247)--(-.034,2.241)--cycle;
\draw(-.048,2.247)--(-.034,2.241);
\filldraw[fill opacity=0.8,fill=gray!20,draw=none](-.054,1.99)--(-.045,1.99)--(-.063,1.997)--cycle;
\draw(-.045,1.99)--(-.063,1.997);
\filldraw[fill opacity=0.8,fill=gray!20,draw=none](-.039,2.229)--(-.034,2.241)--(-.048,2.247)--cycle;
\draw(-.034,2.241)--(-.048,2.247);
\filldraw[fill opacity=0.8,fill=gray!20,draw=none](-.062,2.009)--(.036,2.012)--(.051,2.048)--(-.076,2.044)--cycle;
\draw(-.062,2.009)--(.036,2.012)--(.051,2.048)--(-.076,2.044);
\filldraw[fill opacity=0.8,fill=gray!20,draw=none](-.9,2.167)--(.048,2.195)--(.03,2.231)--(-.963,2.202)--cycle;
\draw(-.9,2.167)--(.048,2.195)--(.03,2.231)--(-.963,2.202);
\filldraw[fill opacity=0.8,fill=gray!20,draw=none](.061,2.238)--(.032,2.25)--(-.017,2.249)--(-.034,2.241)--(.011,2.221)--cycle;
\draw(.061,2.238)--(.032,2.25);
\draw(-.034,2.241)--(.011,2.221);
\filldraw[fill opacity=0.8,fill=gray!20,draw=none](-.06,2.027)--(-.061,2.027)--(-.063,1.997)--(-.032,1.984)--cycle;
\draw(-.06,2.027)--(-.061,2.027);
\draw(-.063,1.997)--(-.032,1.984);
\filldraw[fill opacity=0.8,fill=gray!20,draw=none](-.039,2.229)--(-.021,2.195)--(.011,2.221)--(-.034,2.241)--cycle;
\draw(.011,2.221)--(-.034,2.241);
\filldraw[fill opacity=0.8,fill=gray!20,draw=none](-.069,2.045)--(.051,2.048)--(.059,2.096)--(-.074,2.092)--cycle;
\draw(-.069,2.045)--(.051,2.048)--(.059,2.096)--(-.074,2.092);
\filldraw[fill opacity=0.8,fill=gray!20,draw=none](-.064,2.144)--(.058,2.147)--(.048,2.195)--(-.052,2.192)--cycle;
\draw(-.064,2.144)--(.058,2.147)--(.048,2.195)--(-.052,2.192);
\filldraw[fill opacity=0.8,fill=gray!20,draw=none](-.021,2.195)--(-.039,2.229)--(-.053,2.195)--(-.032,2.186)--cycle;
\draw(-.053,2.195)--(-.032,2.186);
\filldraw[fill opacity=0.8,fill=gray!20,draw=none](-.062,2.009)--(-.061,2.027)--(-.071,2.032)--cycle;
\draw(-.061,2.027)--(-.071,2.032);
\filldraw[fill opacity=0.8,fill=gray!20,draw=none](-1.049,1.98)--(-.062,2.009)--(-.076,2.044)--(-1.014,2.017)--cycle;
\draw(-1.049,1.98)--(-.062,2.009);
\draw(-.076,2.044)--(-1.014,2.017);
\filldraw[fill opacity=0.8,fill=gray!20,draw=none](-.066,2.139)--(-.07,2.092)--(.059,2.096)--(.058,2.147)--(-.064,2.144)--cycle;
\draw(-.07,2.092)--(.059,2.096)--(.058,2.147)--(-.064,2.144);
\filldraw[fill opacity=0.8,fill=gray!20,draw=none](-.067,2.063)--(-.071,2.032)--(-.06,2.027)--cycle;
\draw(-.071,2.032)--(-.06,2.027);
\filldraw[fill opacity=0.8,fill=gray!20,draw=none](-.05,2.155)--(-.032,2.186)--(-.053,2.195)--cycle;
\draw(-.032,2.186)--(-.053,2.195);
\filldraw[fill opacity=0.8,fill=gray!20,draw=none](-.05,2.155)--(-.052,2.192)--(-.065,2.139)--(-.06,2.136)--cycle;
\draw(-.065,2.139)--(-.06,2.136);
\filldraw[fill opacity=0.8,fill=gray!20,draw=none](-.093,2.143)--(-.064,2.144)--(-.052,2.192)--(-.054,2.192)--cycle;
\draw(-.093,2.143)--(-.064,2.144);
\draw(-.052,2.192)--(-.054,2.192);
\filldraw[fill opacity=0.8,fill=gray!20,draw=none](-.848,2.12)--(-.093,2.143)--(-.054,2.192)--(-.88,2.167)--cycle;
\draw(-.848,2.12)--(-.093,2.143);
\draw(-.054,2.192)--(-.88,2.167);
\filldraw[fill opacity=0.8,fill=gray!20,draw=none](-1.176,2.022)--(-1.19,2.062)--(-1.179,2.063)--cycle;
\draw(-1.19,2.062)--(-1.179,2.063);
\filldraw[fill opacity=0.8,fill=gray!20,draw=none](-1.189,2.068)--(-1.188,2.062)--(-1.19,2.062)--cycle;
\draw(-1.188,2.062)--(-1.19,2.062);
\filldraw[fill opacity=0.8,fill=gray!20,draw=none](-1.233,2.015)--(-1.206,2.07)--(-1.166,2.067)--(-1.178,2.011)--cycle;
\draw(-1.206,2.07)--(-1.166,2.067)--(-1.178,2.011)--(-1.233,2.015);
\filldraw[fill opacity=0.8,fill=gray!20,draw=none](-1.014,2.017)--(-.069,2.045)--(-.074,2.092)--(-.989,2.065)--cycle;
\draw(-1.014,2.017)--(-.069,2.045);
\draw(-.074,2.092)--(-.989,2.065);
\filldraw[fill opacity=0.8,fill=gray!20,draw=none](-.069,2.045)--(-.067,2.063)--(-.07,2.081)--(-.073,2.082)--cycle;
\draw(-.07,2.081)--(-.073,2.082);
\filldraw[fill opacity=0.8,fill=gray!20,draw=none](-.065,2.111)--(-.073,2.082)--(-.07,2.081)--cycle;
\draw(-.073,2.082)--(-.07,2.081);
\filldraw[fill opacity=0.8,fill=gray!20,draw=none](-.064,2.144)--(-.066,2.139)--(-.065,2.139)--cycle;
\draw(-.066,2.139)--(-.065,2.139);
\filldraw[fill opacity=0.8,fill=gray!20,draw=none](-1.189,2.068)--(-1.206,2.07)--(-1.197,2.126)--(-1.195,2.126)--cycle;
\draw(-1.189,2.068)--(-1.206,2.07);
\draw(-1.197,2.126)--(-1.195,2.126);
\filldraw[fill opacity=0.8,fill=gray!20,draw=none](-1.227,2.113)--(-1.179,2.114)--(-1.19,2.062)--(-1.335,2.058)--cycle;
\draw(-1.227,2.113)--(-1.179,2.114);
\draw(-1.19,2.062)--(-1.335,2.058);
\filldraw[fill opacity=0.8,fill=gray!20,draw=none](-1.188,2.125)--(-1.19,2.114)--(-1.194,2.114)--cycle;
\draw(-1.19,2.114)--(-1.194,2.114);
\filldraw[fill opacity=0.8,fill=gray!20,draw=none](-1.189,2.068)--(-1.195,2.126)--(-1.162,2.124)--(-1.166,2.067)--cycle;
\draw(-1.195,2.126)--(-1.162,2.124)--(-1.166,2.067)--(-1.189,2.068);
\filldraw[fill opacity=0.8,fill=gray!20,draw=none](-1.175,2.015)--(-1.177,2.018)--(-1.176,2.022)--cycle;
\draw(-1.177,2.018)--(-1.176,2.022);
\filldraw[fill opacity=0.8,fill=gray!20,draw=none](-1.175,2.015)--(-1.175,2.01)--(-1.178,2.011)--(-1.177,2.018)--cycle;
\draw(-1.175,2.01)--(-1.178,2.011)--(-1.177,2.018);
\filldraw[fill opacity=0.8,fill=gray!20,draw=none](-.43,2.081)--(-.088,2.091)--(-.064,2.144)--(-.295,2.137)--cycle;
\draw(-.43,2.081)--(-.088,2.091);
\draw(-.064,2.144)--(-.295,2.137);
\filldraw[fill opacity=0.8,fill=gray!20,draw=none](-.07,2.092)--(-.065,2.111)--(-.063,2.12)--(-.065,2.139)--(-.066,2.139)--cycle;
\draw(-.065,2.139)--(-.066,2.139);
\filldraw[fill opacity=0.8,fill=gray!20,draw=none](-.066,2.139)--(-.088,2.091)--(-.07,2.092)--cycle;
\draw(-.088,2.091)--(-.07,2.092);
\filldraw[fill opacity=0.8,fill=gray!20,draw=none](-.063,2.12)--(-.06,2.136)--(-.065,2.139)--cycle;
\draw(-.06,2.136)--(-.065,2.139);
\filldraw[fill opacity=0.8,fill=gray!20,draw=none](-1.257,2.06)--(-1.19,2.062)--(-1.176,2.022)--(-1.175,2.015)--(-1.319,2.011)--cycle;
\draw(-1.257,2.06)--(-1.19,2.062);
\draw(-1.175,2.015)--(-1.319,2.011);
\filldraw[fill opacity=0.8,fill=gray!20,draw=none](-1.176,2.022)--(-1.174,2.015)--(-1.175,2.015)--cycle;
\draw(-1.174,2.015)--(-1.175,2.015);
\filldraw[fill opacity=0.8,fill=gray!20,draw=none](-.989,2.065)--(-.43,2.081)--(-.295,2.137)--(-.978,2.117)--cycle;
\draw(-.989,2.065)--(-.43,2.081);
\draw(-.295,2.137)--(-.978,2.117);
\filldraw[fill opacity=0.8,fill=gray!20,draw=none](-1.332,1.919)--(-1.344,1.944)--(-1.279,1.966)--(-1.274,1.966)--(-1.277,1.922)--cycle;
\draw(-1.279,1.966)--(-1.274,1.966)--(-1.277,1.922)--(-1.332,1.919)--(-1.344,1.944);
\filldraw[fill opacity=0.8,fill=gray!20,draw=none](-1.137,1.959)--(-.826,1.968)--(-.647,1.973)--(-1.096,1.959)--cycle;
\draw(-1.137,1.959)--(-.826,1.968);
\draw(-.647,1.973)--(-1.096,1.959);
\filldraw[fill opacity=0.8,fill=gray!20,draw=none](-1.188,2.125)--(-1.197,2.126)--(-1.206,2.181)--(-1.182,2.179)--cycle;
\draw(-1.188,2.125)--(-1.197,2.126);
\draw(-1.206,2.181)--(-1.182,2.179);
\filldraw[fill opacity=0.8,fill=gray!20,draw=none](-1.334,2.157)--(-1.172,2.162)--(-1.194,2.114)--(-1.342,2.109)--cycle;
\draw(-1.194,2.114)--(-1.342,2.109)--(-1.334,2.157)--(-1.172,2.162);
\filldraw[fill opacity=0.8,fill=gray!20,draw=none](-1.17,2.178)--(-1.177,2.161)--(-1.184,2.161)--cycle;
\draw(-1.177,2.161)--(-1.184,2.161);
\filldraw[fill opacity=0.8,fill=gray!20,draw=none](-1.188,2.125)--(-1.182,2.179)--(-1.166,2.178)--(-1.162,2.124)--cycle;
\draw(-1.182,2.179)--(-1.166,2.178)--(-1.162,2.124)--(-1.188,2.125);
\filldraw[fill opacity=0.8,fill=gray!20,draw=none](-1.279,1.966)--(-1.274,1.968)--(-1.274,1.966)--cycle;
\draw(-1.274,1.968)--(-1.274,1.966)--(-1.279,1.966);
\filldraw[fill opacity=0.8,fill=gray!20,draw=none](-1.274,1.957)--(-1.274,1.966)--(-1.198,1.96)--(-1.199,1.958)--cycle;
\draw(-1.274,1.957)--(-1.274,1.966)--(-1.198,1.96)--(-1.199,1.958);
\filldraw[fill opacity=0.8,fill=gray!20,draw=none](-1.174,2.015)--(-1.132,1.979)--(-1.164,1.978)--cycle;
\draw(-1.132,1.979)--(-1.164,1.978);
\filldraw[fill opacity=0.8,fill=gray!20,draw=none](-1.277,1.922)--(-1.274,1.957)--(-1.199,1.958)--(-1.223,1.918)--cycle;
\draw(-1.199,1.958)--(-1.223,1.918)--(-1.277,1.922)--(-1.274,1.957);
\filldraw[fill opacity=0.8,fill=gray!20,draw=none](-1.198,1.96)--(-1.185,1.993)--(-1.137,1.959)--(-1.144,1.947)--cycle;
\draw(-1.137,1.959)--(-1.144,1.947)--(-1.198,1.96)--(-1.185,1.993);
\filldraw[fill opacity=0.8,fill=gray!20,draw=none](-1.138,1.964)--(-1.155,1.979)--(-1.142,1.979)--(-1.141,1.979)--cycle;
\draw(-1.155,1.979)--(-1.142,1.979);
\filldraw[fill opacity=0.8,fill=gray!20,draw=none](-1.138,1.964)--(-1.141,1.979)--(-1.087,1.96)--(-1.132,1.959)--cycle;
\draw(-1.087,1.96)--(-1.132,1.959);
\filldraw[fill opacity=0.8,fill=gray!20,draw=none](-1.213,1.916)--(-1.223,1.918)--(-1.198,1.96)--(-1.144,1.947)--(-1.157,1.935)--cycle;
\draw(-1.213,1.916)--(-1.223,1.918)--(-1.198,1.96)--(-1.144,1.947)--(-1.157,1.935);
\filldraw[fill opacity=0.8,fill=gray!20,draw=none](-1.298,1.975)--(-1.155,1.979)--(-1.138,1.964)--(-1.137,1.959)--(-1.198,1.957)--cycle;
\draw(-1.298,1.975)--(-1.155,1.979);
\draw(-1.137,1.959)--(-1.198,1.957);
\filldraw[fill opacity=0.8,fill=gray!20,draw=none](-1.138,1.964)--(-1.132,1.959)--(-1.137,1.959)--cycle;
\draw(-1.132,1.959)--(-1.137,1.959);
\filldraw[fill opacity=0.8,fill=gray!20,draw=none](-1.185,1.993)--(-1.178,2.011)--(-1.123,1.998)--(-1.121,1.983)--(-1.137,1.959)--cycle;
\draw(-1.185,1.993)--(-1.178,2.011)--(-1.123,1.998);
\draw(-1.121,1.983)--(-1.137,1.959);
\filldraw[fill opacity=0.8,fill=gray!20,draw=none](-1.173,2.01)--(-1.175,2.01)--(-1.175,2.015)--cycle;
\draw(-1.173,2.01)--(-1.175,2.01);
\filldraw[fill opacity=0.8,fill=gray!20,draw=none](-1.139,2.028)--(-1.171,2.045)--(-1.166,2.067)--(-1.157,2.064)--cycle;
\draw(-1.171,2.045)--(-1.166,2.067)--(-1.157,2.064);
\filldraw[fill opacity=0.8,fill=gray!20](-1.166,2.067)--(-1.162,2.124)--(-1.087,2.105)--(-1.093,2.049)--cycle;
\filldraw[fill opacity=0.8,fill=gray!20,draw=none](-1.254,2.013)--(-1.174,2.015)--(-1.164,1.978)--(-1.298,1.975)--cycle;
\draw(-1.254,2.013)--(-1.174,2.015);
\draw(-1.164,1.978)--(-1.298,1.975);
\filldraw[fill opacity=0.8,fill=gray!20,draw=none](-1.139,2.028)--(-1.123,1.998)--(-1.173,2.01)--(-1.175,2.015)--(-1.176,2.022)--(-1.171,2.045)--cycle;
\draw(-1.123,1.998)--(-1.173,2.01);
\draw(-1.176,2.022)--(-1.171,2.045);
\filldraw[fill opacity=0.8,fill=gray!20](-1.162,2.124)--(-1.166,2.178)--(-1.093,2.16)--(-1.087,2.105)--cycle;
\filldraw[fill opacity=0.8,fill=gray!20,draw=none](-1.123,1.998)--(-1.157,2.064)--(-1.093,2.049)--(-1.113,1.995)--cycle;
\draw(-1.157,2.064)--(-1.093,2.049)--(-1.113,1.995)--(-1.123,1.998);
\filldraw[fill opacity=0.8,fill=gray!20,draw=none](-1.342,2.109)--(-1.227,2.113)--(-1.335,2.058)--(-1.341,2.058)--cycle;
\draw(-1.335,2.058)--(-1.341,2.058)--(-1.342,2.109)--(-1.227,2.113);
\filldraw[fill opacity=0.8,fill=gray!20,draw=none](-1.17,2.178)--(-1.206,2.181)--(-1.226,2.216)--(-1.176,2.217)--(-1.168,2.183)--cycle;
\draw(-1.17,2.178)--(-1.206,2.181);
\draw(-1.176,2.217)--(-1.168,2.183);
\filldraw[fill opacity=0.8,fill=gray!20,draw=none](-1.166,2.198)--(-1.153,2.198)--(-1.168,2.18)--cycle;
\draw(-1.166,2.198)--(-1.153,2.198);
\filldraw[fill opacity=0.8,fill=gray!20,draw=none](-1.155,2.2)--(-1.156,2.198)--(-1.161,2.198)--cycle;
\draw(-1.156,2.198)--(-1.161,2.198);
\filldraw[fill opacity=0.8,fill=gray!20,draw=none](-1.147,2.217)--(-1.155,2.2)--(-1.161,2.198)--(-1.166,2.198)--cycle;
\draw(-1.161,2.198)--(-1.166,2.198);
\filldraw[fill opacity=0.8,fill=gray!20,draw=none](-1.147,2.217)--(-1.117,2.218)--(-1.155,2.2)--cycle;
\draw(-1.147,2.217)--(-1.117,2.218);
\filldraw[fill opacity=0.8,fill=gray!20,draw=none](-1.166,2.178)--(-1.17,2.193)--(-1.147,2.218)--(-1.113,2.209)--(-1.093,2.16)--cycle;
\draw(-1.147,2.218)--(-1.113,2.209)--(-1.093,2.16)--(-1.166,2.178)--(-1.17,2.193);
\filldraw[fill opacity=0.8,fill=gray!20,draw=none](-1.17,2.193)--(-1.176,2.217)--(-1.147,2.218)--cycle;
\draw(-1.17,2.193)--(-1.176,2.217);
\filldraw[fill opacity=0.8,fill=gray!20,draw=none](-1.319,2.193)--(-1.166,2.198)--(-1.168,2.18)--(-1.184,2.161)--(-1.334,2.157)--cycle;
\draw(-1.184,2.161)--(-1.334,2.157)--(-1.319,2.193)--(-1.166,2.198);
\filldraw[fill opacity=0.8,fill=gray!20,draw=none](-1.17,2.178)--(-1.168,2.183)--(-1.166,2.178)--cycle;
\draw(-1.168,2.183)--(-1.166,2.178)--(-1.17,2.178);
\filldraw[fill opacity=0.8,fill=gray!20,draw=none](-1.335,2.058)--(-1.257,2.06)--(-1.319,2.011)--cycle;
\draw(-1.335,2.058)--(-1.257,2.06);
\filldraw[fill opacity=0.8,fill=gray!20,draw=none](-1.375,1.911)--(-1.413,1.951)--(-1.409,1.951)--(-1.344,1.944)--(-1.332,1.919)--cycle;
\draw(-1.344,1.944)--(-1.332,1.919)--(-1.375,1.911)--(-1.413,1.951)--(-1.409,1.951);
\filldraw[fill opacity=0.8,fill=gray!20,draw=none](-1.226,2.216)--(-1.233,2.229)--(-1.178,2.225)--(-1.176,2.217)--cycle;
\draw(-1.233,2.229)--(-1.178,2.225)--(-1.176,2.217);
\filldraw[fill opacity=0.8,fill=gray!20,draw=none](-1.233,2.229)--(-1.274,2.267)--(-1.274,2.269)--(-1.198,2.263)--(-1.178,2.225)--cycle;
\draw(-1.274,2.267)--(-1.274,2.269)--(-1.198,2.263)--(-1.178,2.225)--(-1.233,2.229);
\filldraw[fill opacity=0.8,fill=gray!20](-1.309,1.888)--(-1.332,1.919)--(-1.277,1.922)--(-1.28,1.889)--cycle;
\filldraw[fill opacity=0.8,fill=gray!20](-1.28,1.889)--(-1.277,1.922)--(-1.223,1.918)--(-1.252,1.887)--cycle;
\filldraw[fill opacity=0.8,fill=gray!20,draw=none](-1.298,2.213)--(-1.147,2.217)--(-1.166,2.198)--(-1.319,2.193)--cycle;
\draw(-1.166,2.198)--(-1.319,2.193)--(-1.298,2.213)--(-1.147,2.217);
\filldraw[fill opacity=0.8,fill=gray!20,draw=none](-1.319,2.011)--(-1.254,2.013)--(-1.298,1.975)--cycle;
\draw(-1.319,2.011)--(-1.254,2.013);
\filldraw[fill opacity=0.8,fill=gray!20](-1.331,1.883)--(-1.375,1.911)--(-1.332,1.919)--(-1.309,1.888)--cycle;
\filldraw[fill opacity=0.8,fill=gray!20,draw=none](-1.176,2.217)--(-1.178,2.225)--(-1.147,2.218)--cycle;
\draw(-1.176,2.217)--(-1.178,2.225)--(-1.147,2.218);
\filldraw[fill opacity=0.8,fill=gray!20](-1.252,1.887)--(-1.223,1.918)--(-1.185,1.909)--(-1.232,1.882)--cycle;
\filldraw[fill opacity=0.8,fill=gray!20,draw=none](-1.145,2.217)--(-1.178,2.225)--(-1.198,2.263)--(-1.144,2.25)--(-1.119,2.217)--cycle;
\draw(-1.145,2.217)--(-1.178,2.225)--(-1.198,2.263)--(-1.144,2.25)--(-1.119,2.217);
\filldraw[fill opacity=0.8,fill=gray!20,draw=none](-1.438,1.934)--(-1.457,1.958)--(-1.464,1.985)--(-1.455,1.991)--(-1.414,1.953)--(-1.413,1.951)--cycle;
\draw(-1.464,1.985)--(-1.455,1.991);
\draw(-1.414,1.953)--(-1.413,1.951)--(-1.438,1.934)--(-1.457,1.958);
\filldraw[fill opacity=0.8,fill=gray!20,draw=none](-1.455,1.991)--(-1.464,1.985)--(-1.485,2.037)--(-1.482,2.04)--cycle;
\draw(-1.455,1.991)--(-1.464,1.985);
\draw(-1.485,2.037)--(-1.482,2.04);
\filldraw[fill opacity=0.8,fill=gray!20,draw=none](-1.308,2.015)--(-1.322,2.01)--(-1.336,2.058)--(-1.328,2.06)--cycle;
\draw(-1.336,2.058)--(-1.328,2.06)--(-1.308,2.015)--(-1.322,2.01);
\filldraw[fill opacity=0.8,fill=gray!20,draw=none](-1.328,2.06)--(-1.336,2.058)--(-1.342,2.109)--cycle;
\draw(-1.342,2.109)--(-1.328,2.06)--(-1.336,2.058);
\filldraw[fill opacity=0.8,fill=gray!20,draw=none](-1.298,1.975)--(-1.252,1.967)--(-1.275,1.955)--cycle;
\filldraw[fill opacity=0.8,fill=gray!20,draw=none](-1.252,1.967)--(-1.198,1.957)--(-1.275,1.955)--cycle;
\draw(-1.198,1.957)--(-1.275,1.955);
\filldraw[fill opacity=0.8,fill=gray!20,draw=none](-1.224,2.058)--(-1.22,2.013)--(-1.225,1.978)--(-1.238,1.96)--(-1.259,1.96)--(-1.284,1.98)--(-1.308,2.015)--(-1.328,2.06)--(-1.342,2.109)--cycle;
\draw(-1.224,2.058)--(-1.22,2.013)--(-1.225,1.978)--(-1.238,1.96)--(-1.259,1.96)--(-1.284,1.98)--(-1.308,2.015)--(-1.328,2.06)--(-1.342,2.109);
\filldraw[fill opacity=0.8,fill=gray!20,draw=none](-1.409,1.951)--(-1.413,1.951)--(-1.414,1.953)--cycle;
\draw(-1.409,1.951)--(-1.413,1.951)--(-1.414,1.953);
\filldraw[fill opacity=0.8,fill=gray!20,draw=none](-1.213,1.916)--(-1.157,1.935)--(-1.185,1.909)--cycle;
\draw(-1.157,1.935)--(-1.185,1.909)--(-1.213,1.916);
\filldraw[fill opacity=0.8,fill=gray!20,draw=none](-1.279,2.269)--(-1.274,2.269)--(-1.274,2.267)--cycle;
\draw(-1.279,2.269)--(-1.274,2.269)--(-1.274,2.267);
\filldraw[fill opacity=0.8,fill=gray!20,draw=none](-1.483,2.008)--(-1.609,1.966)--(-1.611,1.976)--(-1.609,2.02)--(-1.487,2.061)--cycle;
\draw(-1.483,2.008)--(-1.609,1.966);
\draw(-1.609,2.02)--(-1.487,2.061);
\filldraw[fill opacity=0.8,fill=gray!20,draw=none](-1.484,2.051)--(-1.482,2.043)--(-1.482,2.04)--(-1.49,2.034)--(-1.491,2.037)--cycle;
\draw(-1.482,2.04)--(-1.49,2.034);
\filldraw[fill opacity=0.8,fill=gray!20,draw=none](-1.49,2.088)--(-1.482,2.043)--(-1.491,2.075)--cycle;
\filldraw[fill opacity=0.8,fill=gray!20,draw=none](-1.491,2.075)--(-1.487,2.061)--(-1.517,2.051)--(-1.603,2.049)--(-1.599,2.069)--cycle;
\draw(-1.487,2.061)--(-1.517,2.051);
\filldraw[fill opacity=0.8,fill=gray!20,draw=none](-1.491,2.075)--(-1.484,2.051)--(-1.491,2.037)--(-1.493,2.064)--cycle;
\filldraw[fill opacity=0.8,fill=gray!20,draw=none](-1.336,2.058)--(-1.398,2.037)--(-1.486,2.047)--(-1.487,2.061)--(-1.434,2.078)--cycle;
\draw(-1.336,2.058)--(-1.398,2.037);
\draw(-1.487,2.061)--(-1.434,2.078);
\filldraw[fill opacity=0.8,fill=gray!20,draw=none](-1.398,2.037)--(-1.483,2.008)--(-1.486,2.047)--cycle;
\draw(-1.398,2.037)--(-1.483,2.008);
\filldraw[fill opacity=0.8,fill=gray!20,draw=none](-1.469,1.982)--(-1.484,2.008)--(-1.474,2.011)--cycle;
\draw(-1.484,2.008)--(-1.474,2.011);
\filldraw[fill opacity=0.8,fill=gray!20,draw=none](-1.464,1.985)--(-1.468,1.982)--(-1.49,2.034)--(-1.485,2.037)--cycle;
\draw(-1.464,1.985)--(-1.468,1.982);
\draw(-1.49,2.034)--(-1.485,2.037);
\filldraw[fill opacity=0.8,fill=gray!20](-1.393,1.899)--(-1.438,1.934)--(-1.413,1.951)--(-1.375,1.911)--cycle;
\filldraw[fill opacity=0.8,fill=gray!20,draw=none](-1.411,1.909)--(-1.431,1.927)--(-1.438,1.934)--(-1.416,1.917)--cycle;
\draw(-1.431,1.927)--(-1.438,1.934)--(-1.416,1.917);
\filldraw[fill opacity=0.8,fill=gray!20,draw=none](-1.284,1.98)--(-1.433,1.929)--(-1.461,1.963)--(-1.308,2.015)--cycle;
\draw(-1.461,1.963)--(-1.308,2.015)--(-1.284,1.98)--(-1.433,1.929);
\filldraw[fill opacity=0.8,fill=gray!20,draw=none](-1.459,1.964)--(-1.469,1.982)--(-1.464,1.985)--cycle;
\draw(-1.469,1.982)--(-1.464,1.985);
\filldraw[fill opacity=0.8,fill=gray!20,draw=none](-1.322,2.01)--(-1.459,1.964)--(-1.469,1.982)--(-1.469,1.985)--(-1.336,2.058)--cycle;
\draw(-1.322,2.01)--(-1.459,1.964);
\filldraw[fill opacity=0.8,fill=gray!20,draw=none](-1.469,1.985)--(-1.474,2.011)--(-1.336,2.058)--cycle;
\draw(-1.474,2.011)--(-1.336,2.058);
\filldraw[fill opacity=0.8,fill=gray!20,draw=none](-1.341,2.058)--(-1.335,2.058)--(-1.319,2.011)--(-1.331,2.01)--cycle;
\draw(-1.319,2.011)--(-1.331,2.01)--(-1.341,2.058)--(-1.335,2.058);
\filldraw[fill opacity=0.8,fill=gray!20,draw=none](-1.123,1.998)--(-1.113,1.995)--(-1.121,1.983)--cycle;
\draw(-1.123,1.998)--(-1.113,1.995)--(-1.121,1.983);
\filldraw[fill opacity=0.8,fill=gray!20,draw=none](-1.336,2.058)--(-1.434,2.078)--(-1.342,2.109)--cycle;
\draw(-1.434,2.078)--(-1.342,2.109);
\filldraw[fill opacity=0.8,fill=gray!20,draw=none](-1.279,2.269)--(-1.344,2.276)--(-1.332,2.291)--(-1.277,2.293)--(-1.274,2.269)--cycle;
\draw(-1.344,2.276)--(-1.332,2.291)--(-1.277,2.293)--(-1.274,2.269)--(-1.279,2.269);
\filldraw[fill opacity=0.8,fill=gray!20](-1.274,2.269)--(-1.277,2.293)--(-1.223,2.289)--(-1.198,2.263)--cycle;
\filldraw[fill opacity=0.8,fill=gray!20,draw=none](-1.342,2.109)--(-1.435,2.113)--(-1.339,2.157)--(-1.331,2.157)--cycle;
\draw(-1.339,2.157)--(-1.331,2.157)--(-1.342,2.109)--(-1.435,2.113);
\filldraw[fill opacity=0.8,fill=gray!20,draw=none](-1.331,2.01)--(-1.319,2.011)--(-1.298,1.975)--(-1.314,1.974)--cycle;
\draw(-1.298,1.975)--(-1.314,1.974)--(-1.331,2.01)--(-1.319,2.011);
\filldraw[fill opacity=0.5,fill=gray!20,draw=none](-1.745,1.624)--(-1.719,1.613)--(-1.646,1.947)--(-1.76,2.041)--(-1.798,2.058)--cycle;
\draw(-1.76,2.041)--(-1.798,2.058)--(-1.745,1.624)--(-1.719,1.613);
\filldraw[fill opacity=0.5,fill=gray!20](-1.933,1.549)--(-1.745,1.624)--(-1.798,2.058)--(-1.993,2.037)--cycle;
\filldraw[fill opacity=0.8,fill=gray!20,draw=none](-1.259,1.96)--(-1.275,1.955)--(-1.363,1.953)--(-1.284,1.98)--cycle;
\draw(-1.363,1.953)--(-1.284,1.98)--(-1.259,1.96)--(-1.275,1.955);
\filldraw[fill opacity=0.8,fill=gray!20,draw=none](-1.314,1.974)--(-1.298,1.975)--(-1.275,1.955)--(-1.292,1.954)--cycle;
\draw(-1.275,1.955)--(-1.292,1.954)--(-1.314,1.974)--(-1.298,1.975);
\filldraw[fill opacity=0.8,fill=gray!20,draw=none](-1.238,1.96)--(-1.253,1.955)--(-1.275,1.955)--(-1.259,1.96)--cycle;
\draw(-1.275,1.955)--(-1.259,1.96)--(-1.238,1.96)--(-1.253,1.955);
\filldraw[fill opacity=0.8,fill=gray!20,draw=none](-1.224,2.058)--(-1.342,2.109)--(-1.341,2.058)--(-1.331,2.01)--(-1.314,1.974)--(-1.292,1.954)--(-1.268,1.954)--(-1.248,1.974)--(-1.232,2.01)--cycle;
\draw(-1.342,2.109)--(-1.341,2.058)--(-1.331,2.01)--(-1.314,1.974)--(-1.292,1.954)--(-1.268,1.954)--(-1.248,1.974)--(-1.232,2.01)--(-1.224,2.058);
\filldraw[fill opacity=0.8,fill=gray!20](-1.224,2.058)--(-1.235,2.01)--(-1.253,1.974)--(-1.276,1.954)--(-1.299,1.954)--(-1.321,1.974)--(-1.336,2.01)--(-1.344,2.058)--(-1.342,2.109)--(-1.331,2.157)--(-1.313,2.193)--(-1.291,2.213)--(-1.267,2.213)--(-1.245,2.193)--(-1.23,2.157)--(-1.222,2.109)--cycle;
\filldraw[fill opacity=0.8,fill=gray!20,draw=none](-1.49,2.088)--(-1.491,2.075)--(-1.494,2.086)--(-1.495,2.091)--(-1.491,2.094)--cycle;
\draw(-1.495,2.091)--(-1.491,2.094);
\filldraw[fill opacity=0.8,fill=gray!20,draw=none](-1.494,2.086)--(-1.342,2.109)--(-1.434,2.078)--(-1.491,2.075)--cycle;
\draw(-1.342,2.109)--(-1.434,2.078);
\filldraw[fill opacity=0.8,fill=gray!20,draw=none](-1.491,2.075)--(-1.434,2.078)--(-1.487,2.061)--cycle;
\draw(-1.434,2.078)--(-1.487,2.061);
\filldraw[fill opacity=0.8,fill=gray!20,draw=none](-1.494,2.086)--(-1.491,2.075)--(-1.599,2.069)--(-1.599,2.07)--cycle;
\filldraw[fill opacity=0.8,fill=gray!20,draw=none](-1.491,2.075)--(-1.493,2.064)--(-1.494,2.086)--cycle;
\filldraw[fill opacity=0.8,fill=gray!20,draw=none](-1.344,2.058)--(-1.493,2.064)--(-1.494,2.086)--(-1.342,2.109)--cycle;
\draw(-1.342,2.109)--(-1.344,2.058)--(-1.493,2.064);
\filldraw[fill opacity=0.8,fill=gray!20,draw=none](-1.493,2.098)--(-1.484,2.137)--(-1.491,2.094)--(-1.492,2.093)--cycle;
\draw(-1.491,2.094)--(-1.492,2.093);
\filldraw[fill opacity=0.8,fill=gray!20,draw=none](-1.493,2.098)--(-1.492,2.093)--(-1.495,2.091)--cycle;
\draw(-1.492,2.093)--(-1.495,2.091);
\filldraw[fill opacity=0.8,fill=gray!20,draw=none](-1.494,2.086)--(-1.495,2.097)--(-1.435,2.113)--(-1.342,2.109)--cycle;
\draw(-1.435,2.113)--(-1.342,2.109);
\filldraw[fill opacity=0.8,fill=gray!20,draw=none](-1.444,2.218)--(-1.455,2.205)--(-1.47,2.195)--(-1.467,2.2)--(-1.452,2.219)--cycle;
\draw(-1.455,2.205)--(-1.47,2.195);
\draw(-1.467,2.2)--(-1.452,2.219);
\filldraw[fill opacity=0.8,fill=gray!20,draw=none](-1.467,2.179)--(-1.469,2.189)--(-1.464,2.199)--(-1.455,2.205)--cycle;
\draw(-1.464,2.199)--(-1.455,2.205);
\filldraw[fill opacity=0.8,fill=gray!20,draw=none](-1.467,2.179)--(-1.482,2.151)--(-1.486,2.147)--(-1.469,2.189)--cycle;
\draw(-1.482,2.151)--(-1.486,2.147);
\filldraw[fill opacity=0.8,fill=gray!20,draw=none](-1.331,2.157)--(-1.339,2.157)--(-1.328,2.194)--(-1.313,2.193)--cycle;
\draw(-1.328,2.194)--(-1.313,2.193)--(-1.331,2.157)--(-1.339,2.157);
\filldraw[fill opacity=0.8,fill=gray!20,draw=none](-1.444,2.218)--(-1.452,2.219)--(-1.438,2.237)--(-1.413,2.254)--(-1.414,2.252)--cycle;
\draw(-1.452,2.219)--(-1.438,2.237)--(-1.413,2.254)--(-1.414,2.252);
\filldraw[fill opacity=0.8,fill=gray!20,draw=none](-1.313,2.193)--(-1.468,2.2)--(-1.44,2.219)--(-1.291,2.213)--cycle;
\draw(-1.44,2.219)--(-1.291,2.213)--(-1.313,2.193)--(-1.468,2.2);
\filldraw[fill opacity=0.8,fill=gray!20,draw=none](-1.224,2.058)--(-1.225,2.109)--(-1.235,2.157)--(-1.253,2.193)--(-1.275,2.213)--(-1.298,2.213)--(-1.319,2.193)--(-1.334,2.157)--(-1.342,2.109)--cycle;
\draw(-1.224,2.058)--(-1.225,2.109)--(-1.235,2.157)--(-1.253,2.193)--(-1.275,2.213)--(-1.298,2.213)--(-1.319,2.193)--(-1.334,2.157)--(-1.342,2.109);
\filldraw[fill opacity=0.8,fill=gray!20,draw=none](-1.495,2.097)--(-1.496,2.116)--(-1.435,2.113)--cycle;
\draw(-1.496,2.116)--(-1.435,2.113);
\filldraw[fill opacity=0.8,fill=gray!20,draw=none](-1.489,2.116)--(-1.597,2.12)--(-1.587,2.168)--(-1.485,2.163)--cycle;
\draw(-1.489,2.116)--(-1.597,2.12);
\draw(-1.587,2.168)--(-1.485,2.163);
\filldraw[fill opacity=0.8,fill=gray!20,draw=none](-1.484,2.137)--(-1.486,2.127)--(-1.492,2.131)--(-1.491,2.144)--(-1.482,2.151)--cycle;
\draw(-1.491,2.144)--(-1.482,2.151);
\filldraw[fill opacity=0.8,fill=gray!20,draw=none](-1.486,2.127)--(-1.493,2.098)--(-1.495,2.105)--(-1.492,2.131)--cycle;
\filldraw[fill opacity=0.8,fill=gray!20,draw=none](-1.435,2.113)--(-1.489,2.116)--(-1.488,2.128)--(-1.401,2.16)--(-1.339,2.157)--cycle;
\draw(-1.435,2.113)--(-1.489,2.116);
\draw(-1.401,2.16)--(-1.339,2.157);
\filldraw[fill opacity=0.8,fill=gray!20,draw=none](-1.488,2.128)--(-1.485,2.163)--(-1.401,2.16)--cycle;
\draw(-1.485,2.163)--(-1.401,2.16);
\filldraw[fill opacity=0.8,fill=gray!20,draw=none](-1.48,2.163)--(-1.493,2.164)--(-1.472,2.194)--cycle;
\draw(-1.48,2.163)--(-1.493,2.164);
\filldraw[fill opacity=0.8,fill=gray!20,draw=none](-1.339,2.157)--(-1.444,2.177)--(-1.38,2.196)--(-1.328,2.194)--cycle;
\draw(-1.38,2.196)--(-1.328,2.194);
\filldraw[fill opacity=0.8,fill=gray!20,draw=none](-1.478,2.167)--(-1.483,2.164)--(-1.469,2.196)--(-1.464,2.199)--cycle;
\draw(-1.469,2.196)--(-1.464,2.199);
\filldraw[fill opacity=0.8,fill=gray!20,draw=none](-1.339,2.157)--(-1.48,2.163)--(-1.475,2.183)--cycle;
\draw(-1.339,2.157)--(-1.48,2.163);
\filldraw[fill opacity=0.8,fill=gray!20,draw=none](-1.224,2.058)--(-1.342,2.109)--(-1.347,2.155)--(-1.342,2.189)--(-1.328,2.207)--(-1.307,2.207)--(-1.283,2.188)--(-1.258,2.153)--(-1.238,2.107)--cycle;
\draw(-1.342,2.109)--(-1.347,2.155)--(-1.342,2.189)--(-1.328,2.207)--(-1.307,2.207)--(-1.283,2.188)--(-1.258,2.153)--(-1.238,2.107)--(-1.224,2.058);
\filldraw[fill opacity=0.8,fill=gray!20,draw=none](-1.469,1.982)--(-1.466,1.97)--(-1.494,2.005)--(-1.484,2.008)--cycle;
\draw(-1.494,2.005)--(-1.484,2.008);
\filldraw[fill opacity=0.8,fill=gray!20,draw=none](-1.477,2.003)--(-1.482,1.999)--(-1.493,2.028)--(-1.493,2.032)--(-1.49,2.034)--cycle;
\draw(-1.482,1.999)--(-1.493,2.028);
\draw(-1.493,2.032)--(-1.49,2.034);
\filldraw[fill opacity=0.8,fill=gray!20,draw=none](-1.493,2.064)--(-1.49,2.034)--(-1.495,2.031)--(-1.496,2.04)--cycle;
\draw(-1.49,2.034)--(-1.495,2.031)--(-1.496,2.04);
\filldraw[fill opacity=0.8,fill=gray!20,draw=none](-1.493,2.028)--(-1.495,2.031)--(-1.493,2.032)--cycle;
\draw(-1.493,2.028)--(-1.495,2.031)--(-1.493,2.032);
\filldraw[fill opacity=0.8,fill=gray!20,draw=none](-1.336,2.01)--(-1.488,2.017)--(-1.494,2.03)--(-1.502,2.065)--(-1.344,2.058)--cycle;
\draw(-1.502,2.065)--(-1.344,2.058)--(-1.336,2.01)--(-1.488,2.017);
\filldraw[fill opacity=0.8,fill=gray!20,draw=none](-1.494,2.086)--(-1.493,2.064)--(-1.599,2.069)--cycle;
\draw(-1.493,2.064)--(-1.599,2.069);
\filldraw[fill opacity=0.8,fill=gray!20,draw=none](-1.495,2.097)--(-1.494,2.086)--(-1.599,2.069)--(-1.599,2.07)--cycle;
\filldraw[fill opacity=0.8,fill=gray!20,draw=none](-1.493,2.098)--(-1.495,2.091)--(-1.496,2.091)--(-1.495,2.105)--cycle;
\draw(-1.495,2.091)--(-1.496,2.091);
\filldraw[fill opacity=0.8,fill=gray!20,draw=none](-1.494,2.086)--(-1.496,2.091)--(-1.495,2.091)--cycle;
\draw(-1.496,2.091)--(-1.495,2.091);
\filldraw[fill opacity=0.8,fill=gray!20,draw=none](-1.495,2.097)--(-1.599,2.07)--(-1.598,2.092)--(-1.519,2.117)--(-1.496,2.116)--cycle;
\draw(-1.519,2.117)--(-1.496,2.116);
\filldraw[fill opacity=0.8,fill=gray!20,draw=none](-1.342,2.109)--(-1.494,2.086)--(-1.499,2.103)--(-1.347,2.155)--cycle;
\draw(-1.499,2.103)--(-1.347,2.155)--(-1.342,2.109);
\filldraw[fill opacity=0.8,fill=gray!20,draw=none](-1.119,2.217)--(-1.053,2.219)--(-1.117,2.218)--(-1.122,2.218)--cycle;
\draw(-1.119,2.217)--(-1.053,2.219);
\draw(-1.117,2.218)--(-1.122,2.218);
\filldraw[fill opacity=0.8,fill=gray!20,draw=none](-1.095,1.976)--(-1.112,1.959)--(-1.006,1.962)--(-.913,1.984)--(-1.091,1.979)--cycle;
\draw(-1.112,1.959)--(-1.006,1.962);
\draw(-.913,1.984)--(-1.091,1.979);
\filldraw[fill opacity=0.8,fill=gray!20,draw=none](-1.275,2.213)--(-1.121,2.218)--(-1.122,2.218)--(-1.298,2.213)--cycle;
\draw(-1.122,2.218)--(-1.298,2.213)--(-1.275,2.213);
\filldraw[fill opacity=0.8,fill=gray!20,draw=none](-1.092,1.979)--(-1.095,1.976)--(-1.113,1.995)--(-1.093,2.049)--(-1.071,2.025)--(-1.089,1.984)--cycle;
\draw(-1.095,1.976)--(-1.113,1.995)--(-1.093,2.049)--(-1.071,2.025)--(-1.089,1.984);
\filldraw[fill opacity=0.8,fill=gray!20,draw=none](-1.095,1.976)--(-1.119,1.959)--(-1.112,1.959)--cycle;
\draw(-1.119,1.959)--(-1.112,1.959);
\filldraw[fill opacity=0.8,fill=gray!20,draw=none](-1.109,1.959)--(-1.137,1.959)--(-1.113,1.995)--(-1.094,1.975)--cycle;
\draw(-1.137,1.959)--(-1.113,1.995)--(-1.094,1.975);
\filldraw[fill opacity=0.8,fill=gray!20](-1.093,2.049)--(-1.087,2.105)--(-1.064,2.081)--(-1.071,2.025)--cycle;
\filldraw[fill opacity=0.8,fill=gray!20,draw=none](-1.198,1.957)--(-1.137,1.959)--(-1.097,1.959)--(-1.097,1.959)--(-1.253,1.955)--cycle;
\draw(-1.198,1.957)--(-1.137,1.959);
\draw(-1.097,1.959)--(-1.253,1.955);
\filldraw[fill opacity=0.8,fill=gray!20,draw=none](-1.144,1.947)--(-1.137,1.959)--(-1.105,1.959)--(-1.128,1.93)--cycle;
\draw(-1.105,1.959)--(-1.128,1.93)--(-1.144,1.947)--(-1.137,1.959);
\filldraw[fill opacity=0.8,fill=gray!20](-1.283,1.869)--(-1.309,1.888)--(-1.28,1.889)--(-1.283,1.869)--cycle;
\filldraw[fill opacity=0.8,fill=gray!20](-1.283,1.869)--(-1.28,1.889)--(-1.252,1.887)--(-1.283,1.869)--cycle;
\filldraw[fill opacity=0.8,fill=gray!20](-1.185,1.909)--(-1.144,1.947)--(-1.128,1.93)--(-1.174,1.897)--cycle;
\filldraw[fill opacity=0.8,fill=gray!20](-1.34,1.877)--(-1.393,1.899)--(-1.375,1.911)--(-1.331,1.883)--cycle;
\filldraw[fill opacity=0.8,fill=gray!20](-1.283,1.869)--(-1.331,1.883)--(-1.309,1.888)--(-1.283,1.869)--cycle;
\filldraw[fill opacity=0.8,fill=gray!20](-1.087,2.105)--(-1.093,2.16)--(-1.071,2.136)--(-1.064,2.081)--cycle;
\filldraw[fill opacity=0.8,fill=gray!20,draw=none](-1.477,2.003)--(-1.468,1.982)--(-1.473,1.979)--(-1.482,1.999)--cycle;
\draw(-1.468,1.982)--(-1.473,1.979)--(-1.482,1.999);
\filldraw[fill opacity=0.8,fill=gray!20,draw=none](-1.466,1.97)--(-1.465,1.962)--(-1.591,1.92)--(-1.609,1.966)--(-1.494,2.005)--cycle;
\draw(-1.465,1.962)--(-1.591,1.92);
\draw(-1.609,1.966)--(-1.494,2.005);
\filldraw[fill opacity=0.8,fill=gray!20,draw=none](-1.466,1.97)--(-1.469,1.982)--(-1.459,1.964)--(-1.461,1.963)--cycle;
\draw(-1.459,1.964)--(-1.461,1.963);
\filldraw[fill opacity=0.8,fill=gray!20,draw=none](-1.459,1.964)--(-1.457,1.958)--(-1.473,1.979)--(-1.469,1.982)--cycle;
\draw(-1.457,1.958)--(-1.473,1.979)--(-1.469,1.982);
\filldraw[fill opacity=0.8,fill=gray!20,draw=none](-1.321,1.974)--(-1.468,1.981)--(-1.488,2.017)--(-1.336,2.01)--cycle;
\draw(-1.488,2.017)--(-1.336,2.01)--(-1.321,1.974)--(-1.468,1.981);
\filldraw[fill opacity=0.8,fill=gray!20,draw=none](-1.145,2.217)--(-1.119,2.217)--(-1.113,2.209)--cycle;
\draw(-1.119,2.217)--(-1.113,2.209)--(-1.145,2.217);
\filldraw[fill opacity=0.8,fill=gray!20](-1.283,1.869)--(-1.252,1.887)--(-1.232,1.882)--(-1.283,1.869)--cycle;
\filldraw[fill opacity=0.8,fill=gray!20](-1.198,2.263)--(-1.223,2.289)--(-1.185,2.28)--(-1.144,2.25)--cycle;
\filldraw[fill opacity=0.8,fill=gray!20,draw=none](-1.494,2.086)--(-1.599,2.07)--(-1.499,2.103)--cycle;
\draw(-1.599,2.07)--(-1.499,2.103);
\filldraw[fill opacity=0.8,fill=gray!20,draw=none](-1.598,2.092)--(-1.597,2.12)--(-1.519,2.117)--cycle;
\draw(-1.597,2.12)--(-1.519,2.117);
\filldraw[fill opacity=0.8,fill=gray!20,draw=none](-1.492,2.131)--(-1.495,2.105)--(-1.498,2.12)--(-1.496,2.133)--cycle;
\draw(-1.498,2.12)--(-1.496,2.133);
\filldraw[fill opacity=0.8,fill=gray!20,draw=none](-1.492,2.131)--(-1.496,2.133)--(-1.495,2.142)--(-1.491,2.144)--cycle;
\draw(-1.496,2.133)--(-1.495,2.142)--(-1.491,2.144);
\filldraw[fill opacity=0.8,fill=gray!20,draw=none](-1.347,2.155)--(-1.495,2.105)--(-1.498,2.12)--(-1.492,2.138)--(-1.342,2.189)--cycle;
\draw(-1.492,2.138)--(-1.342,2.189)--(-1.347,2.155)--(-1.495,2.105);
\filldraw[fill opacity=0.8,fill=gray!20](-1.232,1.882)--(-1.185,1.909)--(-1.174,1.897)--(-1.226,1.876)--cycle;
\filldraw[fill opacity=0.8,fill=gray!20,draw=none](-1.275,2.213)--(-1.119,2.217)--(-1.121,2.218)--cycle;
\draw(-1.275,2.213)--(-1.119,2.217);
\filldraw[fill opacity=0.8,fill=gray!20,draw=none](-1.102,2.198)--(-.978,2.201)--(-1.053,2.219)--(-1.116,2.217)--cycle;
\draw(-1.102,2.198)--(-.978,2.201);
\draw(-1.053,2.219)--(-1.116,2.217);
\filldraw[fill opacity=0.8,fill=gray!20,draw=none](-1.409,2.255)--(-1.413,2.254)--(-1.375,2.282)--(-1.332,2.291)--(-1.344,2.276)--cycle;
\draw(-1.409,2.255)--(-1.413,2.254)--(-1.375,2.282)--(-1.332,2.291)--(-1.344,2.276);
\filldraw[fill opacity=0.8,fill=gray!20,draw=none](-1.061,2.114)--(-.848,2.12)--(-.88,2.167)--(-1.083,2.161)--cycle;
\draw(-1.061,2.114)--(-.848,2.12);
\draw(-.88,2.167)--(-1.083,2.161);
\filldraw[fill opacity=0.8,fill=gray!20,draw=none](-1.093,2.16)--(-1.113,2.209)--(-1.103,2.198)--(-1.079,2.153)--(-1.071,2.136)--cycle;
\draw(-1.079,2.153)--(-1.071,2.136)--(-1.093,2.16)--(-1.113,2.209)--(-1.103,2.198);
\filldraw[fill opacity=0.8,fill=gray!20,draw=none](-1.073,2.162)--(-.9,2.167)--(-.963,2.202)--(-1.102,2.198)--cycle;
\draw(-1.073,2.162)--(-.9,2.167);
\draw(-.963,2.202)--(-1.102,2.198);
\filldraw[fill opacity=0.8,fill=gray!20](-1.283,1.869)--(-1.34,1.877)--(-1.331,1.883)--(-1.283,1.869)--cycle;
\filldraw[fill opacity=0.8,fill=gray!20,draw=none](-1.292,1.954)--(-1.198,1.957)--(-1.253,1.955)--(-1.268,1.954)--cycle;
\draw(-1.253,1.955)--(-1.268,1.954)--(-1.292,1.954)--(-1.198,1.957);
\filldraw[fill opacity=0.8,fill=gray!20](-1.283,1.869)--(-1.232,1.882)--(-1.226,1.876)--(-1.283,1.869)--cycle;
\filldraw[fill opacity=0.8,fill=gray!20,draw=none](-1.097,1.959)--(-1.096,1.959)--(-1.097,1.959)--cycle;
\draw(-1.096,1.959)--(-1.097,1.959);
\filldraw[fill opacity=0.8,fill=gray!20](-1.332,2.291)--(-1.309,2.302)--(-1.28,2.303)--(-1.277,2.293)--cycle;
\filldraw[fill opacity=0.8,fill=gray!20](-1.277,2.293)--(-1.28,2.303)--(-1.252,2.301)--(-1.223,2.289)--cycle;
\filldraw[fill opacity=0.8,fill=gray!20,draw=none](-1.466,1.97)--(-1.461,1.963)--(-1.465,1.962)--cycle;
\draw(-1.461,1.963)--(-1.465,1.962);
\filldraw[fill opacity=0.8,fill=gray!20,draw=none](-1.299,1.954)--(-1.455,1.961)--(-1.468,1.981)--(-1.321,1.974)--cycle;
\draw(-1.468,1.981)--(-1.321,1.974)--(-1.299,1.954)--(-1.455,1.961);
\filldraw[fill opacity=0.8,fill=gray!20,draw=none](-1.103,2.199)--(-1.113,2.209)--(-1.119,2.217)--cycle;
\draw(-1.103,2.199)--(-1.113,2.209)--(-1.119,2.217);
\filldraw[fill opacity=0.8,fill=gray!20,draw=none](-1.253,2.193)--(-1.102,2.198)--(-1.116,2.217)--(-1.275,2.213)--cycle;
\draw(-1.116,2.217)--(-1.275,2.213)--(-1.253,2.193)--(-1.102,2.198);
\filldraw[fill opacity=0.8,fill=gray!20](-1.375,2.282)--(-1.331,2.297)--(-1.309,2.302)--(-1.332,2.291)--cycle;
\filldraw[fill opacity=0.8,fill=gray!20,draw=none](-1.381,1.887)--(-1.411,1.909)--(-1.416,1.917)--(-1.393,1.899)--cycle;
\draw(-1.416,1.917)--(-1.393,1.899)--(-1.381,1.887)--(-1.411,1.909);
\filldraw[fill opacity=0.8,fill=gray!20,draw=none](-1.431,1.927)--(-1.444,1.942)--(-1.438,1.934)--cycle;
\draw(-1.444,1.942)--(-1.438,1.934)--(-1.431,1.927);
\filldraw[fill opacity=0.8,fill=gray!20,draw=none](-1.275,1.955)--(-1.351,1.93)--(-1.514,1.902)--(-1.363,1.953)--cycle;
\draw(-1.275,1.955)--(-1.351,1.93);
\draw(-1.514,1.902)--(-1.363,1.953);
\filldraw[fill opacity=0.8,fill=gray!20,draw=none](-1.253,1.955)--(-1.351,1.93)--(-1.275,1.955)--cycle;
\draw(-1.351,1.93)--(-1.275,1.955);
\filldraw[fill opacity=0.8,fill=gray!20,draw=none](-1.268,1.954)--(-1.253,1.955)--(-1.236,1.974)--(-1.248,1.974)--cycle;
\draw(-1.236,1.974)--(-1.248,1.974)--(-1.268,1.954)--(-1.253,1.955);
\filldraw[fill opacity=0.8,fill=gray!20,draw=none](-1.225,1.978)--(-1.236,1.974)--(-1.239,1.973)--(-1.257,1.954)--(-1.238,1.96)--cycle;
\draw(-1.257,1.954)--(-1.238,1.96)--(-1.225,1.978)--(-1.236,1.974);
\filldraw[fill opacity=0.8,fill=gray!20,draw=none](-1.253,1.955)--(-1.119,1.959)--(-1.091,1.979)--(-1.236,1.974)--cycle;
\draw(-1.253,1.955)--(-1.119,1.959);
\draw(-1.091,1.979)--(-1.236,1.974);
\filldraw[fill opacity=0.8,fill=gray!20,draw=none](-1.103,2.199)--(-1.119,2.217)--(-1.144,2.25)--(-1.128,2.233)--(-1.093,2.188)--cycle;
\draw(-1.119,2.217)--(-1.144,2.25)--(-1.128,2.233)--(-1.093,2.188)--(-1.103,2.199);
\filldraw[fill opacity=0.8,fill=gray!20,draw=none](-1.478,2.167)--(-1.486,2.147)--(-1.491,2.144)--(-1.483,2.164)--cycle;
\draw(-1.486,2.147)--(-1.491,2.144);
\filldraw[fill opacity=0.8,fill=gray!20,draw=none](-1.342,2.189)--(-1.492,2.138)--(-1.473,2.159)--(-1.328,2.207)--cycle;
\draw(-1.473,2.159)--(-1.328,2.207)--(-1.342,2.189)--(-1.492,2.138);
\filldraw[fill opacity=0.8,fill=gray!20,draw=none](-1.414,2.252)--(-1.413,2.254)--(-1.409,2.255)--cycle;
\draw(-1.414,2.252)--(-1.413,2.254)--(-1.409,2.255);
\filldraw[fill opacity=0.8,fill=gray!20](-1.334,1.871)--(-1.381,1.887)--(-1.393,1.899)--(-1.34,1.877)--cycle;
\filldraw[fill opacity=0.8,fill=gray!20](-1.283,1.869)--(-1.334,1.871)--(-1.34,1.877)--(-1.283,1.869)--cycle;
\filldraw[fill opacity=0.8,fill=gray!20](-1.438,2.237)--(-1.393,2.271)--(-1.375,2.282)--(-1.413,2.254)--cycle;
\filldraw[fill opacity=0.8,fill=gray!20](-1.283,1.869)--(-1.314,1.866)--(-1.334,1.871)--(-1.283,1.869)--cycle;
\filldraw[fill opacity=0.8,fill=gray!20](-1.283,1.869)--(-1.287,1.864)--(-1.314,1.866)--(-1.283,1.869)--cycle;
\filldraw[fill opacity=0.8,fill=gray!20](-1.283,1.869)--(-1.258,1.865)--(-1.287,1.864)--(-1.283,1.869)--cycle;
\filldraw[fill opacity=0.8,fill=gray!20](-1.283,1.869)--(-1.236,1.87)--(-1.258,1.865)--(-1.283,1.869)--cycle;
\filldraw[fill opacity=0.8,fill=gray!20](-1.283,1.869)--(-1.226,1.876)--(-1.236,1.87)--(-1.283,1.869)--cycle;
\filldraw[fill opacity=0.8,fill=gray!20](-1.223,2.289)--(-1.252,2.301)--(-1.232,2.296)--(-1.185,2.28)--cycle;
\filldraw[fill opacity=0.8,fill=gray!20](-1.226,1.876)--(-1.174,1.897)--(-1.192,1.885)--(-1.236,1.87)--cycle;
\filldraw[fill opacity=0.8,fill=gray!20,draw=none](-1.433,1.929)--(-1.434,1.929)--(-1.472,1.96)--(-1.461,1.963)--cycle;
\draw(-1.433,1.929)--(-1.434,1.929);
\draw(-1.472,1.96)--(-1.461,1.963);
\filldraw[fill opacity=0.8,fill=gray!20,draw=none](-1.432,1.916)--(-1.351,1.93)--(-1.411,1.909)--cycle;
\draw(-1.351,1.93)--(-1.411,1.909);
\filldraw[fill opacity=0.8,fill=gray!20,draw=none](-1.411,1.909)--(-1.422,1.917)--(-1.431,1.927)--cycle;
\draw(-1.411,1.909)--(-1.422,1.917)--(-1.431,1.927);
\filldraw[fill opacity=0.8,fill=gray!20,draw=none](-1.431,1.927)--(-1.422,1.917)--(-1.453,1.958)--(-1.456,1.961)--(-1.458,1.96)--(-1.444,1.942)--cycle;
\draw(-1.431,1.927)--(-1.422,1.917)--(-1.453,1.958)--(-1.456,1.961);
\draw(-1.458,1.96)--(-1.444,1.942);
\filldraw[fill opacity=0.8,fill=gray!20,draw=none](-1.456,1.961)--(-1.473,1.979)--(-1.458,1.96)--cycle;
\draw(-1.456,1.961)--(-1.473,1.979)--(-1.458,1.96);
\filldraw[fill opacity=0.8,fill=gray!20,draw=none](-1.434,1.929)--(-1.569,1.884)--(-1.591,1.92)--(-1.472,1.96)--cycle;
\draw(-1.434,1.929)--(-1.569,1.884);
\draw(-1.591,1.92)--(-1.472,1.96);
\filldraw[fill opacity=0.8,fill=gray!20,draw=none](-1.092,1.979)--(-1.049,1.98)--(-1.014,2.017)--(-1.071,2.015)--cycle;
\draw(-1.092,1.979)--(-1.049,1.98);
\draw(-1.014,2.017)--(-1.071,2.015);
\filldraw[fill opacity=0.8,fill=gray!20,draw=none](-1.456,1.961)--(-1.455,1.961)--(-1.473,2.007)--(-1.495,2.031)--(-1.473,1.979)--cycle;
\draw(-1.455,1.961)--(-1.473,2.007)--(-1.495,2.031)--(-1.473,1.979)--(-1.456,1.961);
\filldraw[fill opacity=0.8,fill=gray!20](-1.174,1.897)--(-1.128,1.93)--(-1.154,1.913)--(-1.192,1.885)--cycle;
\filldraw[fill opacity=0.8,fill=gray!20](-1.144,2.25)--(-1.185,2.28)--(-1.174,2.268)--(-1.128,2.233)--cycle;
\filldraw[fill opacity=0.8,fill=gray!20,draw=none](-1.483,2.164)--(-1.486,2.162)--(-1.473,2.193)--(-1.469,2.196)--cycle;
\draw(-1.486,2.162)--(-1.473,2.193)--(-1.469,2.196);
\filldraw[fill opacity=0.8,fill=gray!20,draw=none](-1.444,2.177)--(-1.475,2.183)--(-1.472,2.194)--(-1.468,2.2)--(-1.38,2.196)--cycle;
\draw(-1.468,2.2)--(-1.38,2.196);
\filldraw[fill opacity=0.8,fill=gray!20,draw=none](-1.109,1.959)--(-1.094,1.975)--(-1.093,1.974)--(-1.105,1.959)--cycle;
\draw(-1.094,1.975)--(-1.093,1.974)--(-1.105,1.959);
\filldraw[fill opacity=0.8,fill=gray!20,draw=none](-1.291,2.213)--(-1.44,2.219)--(-1.422,2.22)--(-1.267,2.213)--cycle;
\draw(-1.422,2.22)--(-1.267,2.213)--(-1.291,2.213)--(-1.44,2.219);
\filldraw[fill opacity=0.8,fill=gray!20,draw=none](-1.494,2.086)--(-1.493,2.064)--(-1.496,2.04)--(-1.502,2.086)--(-1.496,2.091)--cycle;
\draw(-1.496,2.04)--(-1.502,2.086)--(-1.496,2.091);
\filldraw[fill opacity=0.8,fill=gray!20](-1.473,2.007)--(-1.48,2.062)--(-1.502,2.086)--(-1.495,2.031)--cycle;
\filldraw[fill opacity=0.8,fill=gray!20,draw=none](-1.128,1.93)--(-1.093,1.974)--(-1.112,1.962)--(-1.152,1.915)--(-1.154,1.913)--cycle;
\draw(-1.152,1.915)--(-1.154,1.913)--(-1.128,1.93)--(-1.093,1.974)--(-1.112,1.962);
\filldraw[fill opacity=0.8,fill=gray!20](-1.314,1.866)--(-1.344,1.878)--(-1.381,1.887)--(-1.334,1.871)--cycle;
\filldraw[fill opacity=0.8,fill=gray!20,draw=none](-1.095,1.976)--(-1.091,1.98)--(-1.093,1.974)--cycle;
\draw(-1.091,1.98)--(-1.093,1.974)--(-1.095,1.976);
\filldraw[fill opacity=0.8,fill=gray!20,draw=none](-1.455,1.961)--(-1.456,1.961)--(-1.473,1.981)--(-1.468,1.981)--cycle;
\draw(-1.455,1.961)--(-1.456,1.961);
\draw(-1.473,1.981)--(-1.468,1.981);
\filldraw[fill opacity=0.8,fill=gray!20,draw=none](-1.456,1.961)--(-1.557,1.966)--(-1.581,1.986)--(-1.473,1.981)--cycle;
\draw(-1.456,1.961)--(-1.557,1.966);
\draw(-1.581,1.986)--(-1.473,1.981);
\filldraw[fill opacity=0.8,fill=gray!20,draw=none](-1.276,1.954)--(-1.423,1.961)--(-1.455,1.961)--(-1.299,1.954)--cycle;
\draw(-1.455,1.961)--(-1.299,1.954)--(-1.276,1.954)--(-1.423,1.961);
\filldraw[fill opacity=0.8,fill=gray!20](-1.393,2.271)--(-1.34,2.291)--(-1.331,2.297)--(-1.375,2.282)--cycle;
\filldraw[fill opacity=0.8,fill=gray!20,draw=none](-1.468,1.981)--(-1.473,1.981)--(-1.502,2.018)--(-1.488,2.017)--cycle;
\draw(-1.468,1.981)--(-1.473,1.981);
\draw(-1.502,2.018)--(-1.488,2.017);
\filldraw[fill opacity=0.8,fill=gray!20](-1.236,1.87)--(-1.192,1.885)--(-1.234,1.877)--(-1.258,1.865)--cycle;
\filldraw[fill opacity=0.5,fill=gray!20,draw=none](-1.642,1.965)--(-1.719,1.613)--(-1.605,1.563)--(-1.617,1.962)--(-1.62,1.965)--cycle;
\draw(-1.719,1.613)--(-1.605,1.563);
\filldraw[fill opacity=0.8,fill=gray!20,draw=none](-1.248,1.974)--(-1.236,1.974)--(-1.226,2.011)--(-1.232,2.01)--cycle;
\draw(-1.226,2.011)--(-1.232,2.01)--(-1.248,1.974)--(-1.236,1.974);
\filldraw[fill opacity=0.8,fill=gray!20,draw=none](-1.22,2.013)--(-1.227,2.009)--(-1.239,1.973)--(-1.225,1.978)--cycle;
\draw(-1.239,1.973)--(-1.225,1.978)--(-1.22,2.013);
\filldraw[fill opacity=0.8,fill=gray!20,draw=none](-1.236,1.974)--(-1.092,1.979)--(-1.089,1.984)--(-1.081,2.015)--(-1.226,2.011)--cycle;
\draw(-1.236,1.974)--(-1.092,1.979);
\draw(-1.081,2.015)--(-1.226,2.011);
\filldraw[fill opacity=0.8,fill=gray!20,draw=none](-1.092,1.979)--(-1.089,1.984)--(-1.091,1.98)--cycle;
\draw(-1.089,1.984)--(-1.091,1.98);
\filldraw[fill opacity=0.8,fill=gray!20,draw=none](-1.101,1.969)--(-1.093,1.974)--(-1.071,2.025)--(-1.079,2.02)--cycle;
\draw(-1.101,1.969)--(-1.093,1.974)--(-1.071,2.025)--(-1.079,2.02);
\filldraw[fill opacity=0.8,fill=gray!20,draw=none](-1.232,2.01)--(-1.081,2.015)--(-1.079,2.02)--(-1.076,2.062)--(-1.224,2.058)--cycle;
\draw(-1.076,2.062)--(-1.224,2.058)--(-1.232,2.01)--(-1.081,2.015);
\filldraw[fill opacity=0.8,fill=gray!20,draw=none](-1.081,2.041)--(-1.085,2.017)--(-1.071,2.025)--(-1.064,2.081)--(-1.073,2.075)--cycle;
\draw(-1.085,2.017)--(-1.071,2.025)--(-1.064,2.081)--(-1.073,2.075);
\filldraw[fill opacity=0.8,fill=gray!20,draw=none](-1.079,2.02)--(-1.079,2.015)--(-1.014,2.017)--(-.989,2.065)--(-1.061,2.063)--cycle;
\draw(-1.079,2.015)--(-1.014,2.017);
\draw(-.989,2.065)--(-1.061,2.063);
\filldraw[fill opacity=0.8,fill=gray!20,draw=none](-1.089,1.984)--(-1.071,2.015)--(-1.081,2.015)--cycle;
\draw(-1.071,2.015)--(-1.081,2.015);
\filldraw[fill opacity=0.8,fill=gray!20,draw=none](-1.235,2.157)--(-1.077,2.161)--(-1.077,2.166)--(-1.102,2.198)--(-1.253,2.193)--cycle;
\draw(-1.102,2.198)--(-1.253,2.193)--(-1.235,2.157)--(-1.077,2.161);
\filldraw[fill opacity=0.8,fill=gray!20,draw=none](-1.103,2.198)--(-1.093,2.188)--(-1.079,2.153)--cycle;
\draw(-1.103,2.198)--(-1.093,2.188)--(-1.079,2.153);
\filldraw[fill opacity=0.8,fill=gray!20,draw=none](-1.075,2.114)--(-1.073,2.075)--(-1.064,2.081)--(-1.071,2.136)--(-1.073,2.135)--cycle;
\draw(-1.073,2.075)--(-1.064,2.081)--(-1.071,2.136)--(-1.073,2.135);
\filldraw[fill opacity=0.8,fill=gray!20,draw=none](-1.073,2.075)--(-1.072,2.062)--(-.989,2.065)--(-.978,2.117)--(-1.063,2.114)--cycle;
\draw(-1.072,2.062)--(-.989,2.065);
\draw(-.978,2.117)--(-1.063,2.114);
\filldraw[fill opacity=0.8,fill=gray!20](-1.309,2.302)--(-1.283,2.298)--(-1.283,2.298)--(-1.28,2.303)--cycle;
\filldraw[fill opacity=0.8,fill=gray!20](-1.28,2.303)--(-1.283,2.298)--(-1.283,2.298)--(-1.252,2.301)--cycle;
\filldraw[fill opacity=0.8,fill=gray!20,draw=none](-1.473,1.981)--(-1.575,1.985)--(-1.591,2.022)--(-1.502,2.018)--cycle;
\draw(-1.473,1.981)--(-1.575,1.985);
\draw(-1.591,2.022)--(-1.502,2.018);
\filldraw[fill opacity=0.5,fill=gray!20](-1.59,1.232)--(-1.408,1.152)--(-1.563,1.545)--(-1.745,1.624)--cycle;
\filldraw[fill opacity=0.5,fill=gray!20](-1.758,1.108)--(-1.59,1.232)--(-1.745,1.624)--(-1.933,1.549)--cycle;
\filldraw[fill opacity=0.5,fill=gray!20,draw=none](-1.605,1.563)--(-1.563,1.545)--(-1.616,1.978)--(-1.618,1.979)--cycle;
\draw(-1.605,1.563)--(-1.563,1.545)--(-1.616,1.978)--(-1.618,1.979);
\filldraw[fill opacity=0.8,fill=gray!20,draw=none](-1.483,2.164)--(-1.491,2.144)--(-1.495,2.142)--(-1.486,2.162)--cycle;
\draw(-1.491,2.144)--(-1.495,2.142)--(-1.486,2.162);
\filldraw[fill opacity=0.8,fill=gray!20,draw=none](-1.328,2.207)--(-1.473,2.159)--(-1.459,2.156)--(-1.307,2.207)--cycle;
\draw(-1.459,2.156)--(-1.307,2.207)--(-1.328,2.207)--(-1.473,2.159);
\filldraw[fill opacity=0.8,fill=gray!20](-1.185,2.28)--(-1.232,2.296)--(-1.226,2.29)--(-1.174,2.268)--cycle;
\filldraw[fill opacity=0.8,fill=gray!20](-1.331,2.297)--(-1.283,2.298)--(-1.283,2.298)--(-1.309,2.302)--cycle;
\filldraw[fill opacity=0.8,fill=gray!20](-1.287,1.864)--(-1.29,1.874)--(-1.344,1.878)--(-1.314,1.866)--cycle;
\filldraw[fill opacity=0.8,fill=gray!20,draw=none](-1.473,2.118)--(-1.459,2.156)--(-1.465,2.185)--(-1.473,2.193)--(-1.495,2.142)--cycle;
\draw(-1.465,2.185)--(-1.473,2.193)--(-1.495,2.142)--(-1.473,2.118)--(-1.459,2.156);
\filldraw[fill opacity=0.8,fill=gray!20](-1.252,2.301)--(-1.283,2.298)--(-1.283,2.298)--(-1.232,2.296)--cycle;
\filldraw[fill opacity=0.8,fill=gray!20,draw=none](-1.253,1.955)--(-1.388,1.91)--(-1.392,1.909)--(-1.411,1.909)--(-1.351,1.93)--cycle;
\draw(-1.253,1.955)--(-1.388,1.91);
\draw(-1.411,1.909)--(-1.351,1.93);
\filldraw[fill opacity=0.8,fill=gray!20](-1.258,1.865)--(-1.234,1.877)--(-1.29,1.874)--(-1.287,1.864)--cycle;
\filldraw[fill opacity=0.8,fill=gray!20,draw=none](-1.344,1.878)--(-1.369,1.904)--(-1.389,1.909)--(-1.389,1.893)--(-1.381,1.887)--cycle;
\draw(-1.389,1.893)--(-1.381,1.887)--(-1.344,1.878)--(-1.369,1.904)--(-1.389,1.909);
\filldraw[fill opacity=0.8,fill=gray!20,draw=none](-1.112,1.962)--(-1.101,1.969)--(-1.079,2.02)--(-1.085,2.017)--cycle;
\draw(-1.112,1.962)--(-1.101,1.969);
\draw(-1.079,2.02)--(-1.085,2.017);
\filldraw[fill opacity=0.8,fill=gray!20,draw=none](-1.432,1.916)--(-1.411,1.909)--(-1.545,1.865)--(-1.569,1.884)--(-1.514,1.902)--cycle;
\draw(-1.411,1.909)--(-1.545,1.865);
\draw(-1.569,1.884)--(-1.514,1.902);
\filldraw[fill opacity=0.8,fill=gray!20,draw=none](-1.389,1.909)--(-1.422,1.917)--(-1.389,1.893)--cycle;
\draw(-1.389,1.909)--(-1.422,1.917)--(-1.389,1.893);
\filldraw[fill opacity=0.8,fill=gray!20,draw=none](-1.47,2.195)--(-1.473,2.193)--(-1.467,2.2)--cycle;
\draw(-1.47,2.195)--(-1.473,2.193)--(-1.467,2.2);
\filldraw[fill opacity=0.8,fill=gray!20](-1.453,2.172)--(-1.422,2.22)--(-1.438,2.237)--(-1.473,2.193)--cycle;
\filldraw[fill opacity=0.8,fill=gray!20,draw=none](-1.077,2.161)--(-1.073,2.162)--(-1.077,2.166)--cycle;
\draw(-1.077,2.161)--(-1.073,2.162);
\filldraw[fill opacity=0.8,fill=gray!20,draw=none](-1.468,2.2)--(-1.569,2.204)--(-1.547,2.224)--(-1.44,2.219)--cycle;
\draw(-1.468,2.2)--(-1.569,2.204);
\draw(-1.547,2.224)--(-1.44,2.219);
\filldraw[fill opacity=0.8,fill=gray!20,draw=none](-1.253,1.974)--(-1.405,1.981)--(-1.423,1.961)--(-1.276,1.954)--cycle;
\draw(-1.423,1.961)--(-1.276,1.954)--(-1.253,1.974)--(-1.405,1.981);
\filldraw[fill opacity=0.8,fill=gray!20,draw=none](-1.239,1.973)--(-1.28,1.955)--(-1.257,1.954)--cycle;
\filldraw[fill opacity=0.8,fill=gray!20,draw=none](-1.28,1.955)--(-1.367,1.917)--(-1.257,1.954)--cycle;
\draw(-1.367,1.917)--(-1.257,1.954);
\filldraw[fill opacity=0.8,fill=gray!20](-1.34,2.291)--(-1.283,2.298)--(-1.283,2.298)--(-1.331,2.297)--cycle;
\filldraw[fill opacity=0.8,fill=gray!20,draw=none](-1.192,1.885)--(-1.154,1.913)--(-1.158,1.913)--(-1.222,1.891)--(-1.234,1.877)--cycle;
\draw(-1.222,1.891)--(-1.234,1.877)--(-1.192,1.885)--(-1.154,1.913)--(-1.158,1.913);
\filldraw[fill opacity=0.8,fill=gray!20,draw=none](-1.079,2.02)--(-1.081,2.015)--(-1.079,2.015)--cycle;
\draw(-1.081,2.015)--(-1.079,2.015);
\filldraw[fill opacity=0.5,fill=gray!20](-1.993,2.037)--(-1.798,2.058)--(-1.745,2.503)--(-1.933,2.537)--cycle;
\filldraw[fill opacity=0.8,fill=gray!20,draw=none](-1.267,2.213)--(-1.287,2.214)--(-1.264,2.194)--(-1.245,2.193)--cycle;
\draw(-1.264,2.194)--(-1.245,2.193)--(-1.267,2.213)--(-1.287,2.214);
\filldraw[fill opacity=0.8,fill=gray!20,draw=none](-1.225,2.109)--(-1.075,2.114)--(-1.073,2.135)--(-1.074,2.142)--(-1.083,2.161)--(-1.235,2.157)--cycle;
\draw(-1.083,2.161)--(-1.235,2.157)--(-1.225,2.109)--(-1.075,2.114);
\filldraw[fill opacity=0.8,fill=gray!20](-1.232,2.296)--(-1.283,2.298)--(-1.283,2.298)--(-1.226,2.29)--cycle;
\filldraw[fill opacity=0.8,fill=gray!20,draw=none](-1.079,2.02)--(-1.061,2.063)--(-1.076,2.062)--cycle;
\draw(-1.061,2.063)--(-1.076,2.062);
\filldraw[fill opacity=0.8,fill=gray!20](-1.422,2.22)--(-1.381,2.258)--(-1.393,2.271)--(-1.438,2.237)--cycle;
\filldraw[fill opacity=0.8,fill=gray!20,draw=none](-1.085,2.128)--(-1.071,2.136)--(-1.093,2.188)--(-1.112,2.176)--cycle;
\draw(-1.085,2.128)--(-1.071,2.136)--(-1.093,2.188)--(-1.112,2.176);
\filldraw[fill opacity=0.8,fill=gray!20,draw=none](-1.069,2.114)--(-1.061,2.114)--(-1.074,2.142)--cycle;
\draw(-1.069,2.114)--(-1.061,2.114);
\filldraw[fill opacity=0.8,fill=gray!20,draw=none](-1.389,1.909)--(-1.523,1.864)--(-1.545,1.865)--(-1.411,1.909)--cycle;
\draw(-1.389,1.909)--(-1.523,1.864);
\draw(-1.545,1.865)--(-1.411,1.909);
\filldraw[fill opacity=0.8,fill=gray!20,draw=none](-1.224,2.058)--(-1.076,2.062)--(-1.073,2.075)--(-1.075,2.114)--(-1.225,2.109)--cycle;
\draw(-1.075,2.114)--(-1.225,2.109)--(-1.224,2.058)--(-1.076,2.062);
\filldraw[fill opacity=0.8,fill=gray!20,draw=none](-1.078,2.09)--(-1.073,2.135)--(-1.085,2.128)--cycle;
\draw(-1.073,2.135)--(-1.085,2.128);
\filldraw[fill opacity=0.8,fill=gray!20,draw=none](-1.075,2.114)--(-1.069,2.114)--(-1.073,2.135)--cycle;
\draw(-1.075,2.114)--(-1.069,2.114);
\filldraw[fill opacity=0.5,fill=gray!20,draw=none](-1.606,2.129)--(-1.791,2.125)--(-1.798,2.058)--(-1.616,1.978)--(-1.609,2.042)--cycle;
\draw(-1.791,2.125)--(-1.798,2.058)--(-1.616,1.978)--(-1.609,2.042);
\filldraw[fill opacity=0.8,fill=gray!20,draw=none](-1.498,2.12)--(-1.495,2.105)--(-1.503,2.102)--cycle;
\draw(-1.495,2.105)--(-1.503,2.102);
\filldraw[fill opacity=0.8,fill=gray!20,draw=none](-1.495,2.105)--(-1.496,2.091)--(-1.502,2.086)--(-1.498,2.12)--cycle;
\draw(-1.496,2.091)--(-1.502,2.086)--(-1.498,2.12);
\filldraw[fill opacity=0.8,fill=gray!20,draw=none](-1.498,2.12)--(-1.503,2.102)--(-1.599,2.07)--(-1.581,2.109)--(-1.501,2.136)--cycle;
\draw(-1.503,2.102)--(-1.599,2.07);
\draw(-1.581,2.109)--(-1.501,2.136);
\filldraw[fill opacity=0.8,fill=gray!20](-1.48,2.062)--(-1.473,2.118)--(-1.495,2.142)--(-1.502,2.086)--cycle;
\filldraw[fill opacity=0.8,fill=gray!20,draw=none](-1.498,2.12)--(-1.501,2.136)--(-1.492,2.138)--cycle;
\draw(-1.501,2.136)--(-1.492,2.138);
\filldraw[fill opacity=0.5,fill=gray!20,draw=none](-1.606,2.129)--(-1.609,2.042)--(-1.598,2.129)--cycle;
\draw(-1.609,2.042)--(-1.598,2.129);
\filldraw[fill opacity=0.8,fill=gray!20,draw=none](-1.492,2.138)--(-1.48,2.156)--(-1.473,2.159)--cycle;
\draw(-1.48,2.156)--(-1.473,2.159);
\filldraw[fill opacity=0.8,fill=gray!20,draw=none](-1.492,2.138)--(-1.581,2.109)--(-1.558,2.13)--(-1.48,2.156)--cycle;
\draw(-1.492,2.138)--(-1.581,2.109);
\draw(-1.558,2.13)--(-1.48,2.156);
\filldraw[fill opacity=0.8,fill=gray!20,draw=none](-1.307,2.207)--(-1.459,2.156)--(-1.428,2.139)--(-1.283,2.188)--cycle;
\draw(-1.428,2.139)--(-1.283,2.188)--(-1.307,2.207)--(-1.459,2.156);
\filldraw[fill opacity=0.8,fill=gray!20,draw=none](-1.287,2.214)--(-1.422,2.22)--(-1.395,2.2)--(-1.264,2.194)--cycle;
\draw(-1.287,2.214)--(-1.422,2.22);
\draw(-1.395,2.2)--(-1.264,2.194);
\filldraw[fill opacity=0.8,fill=gray!20,draw=none](-1.073,2.075)--(-1.063,2.114)--(-1.075,2.114)--cycle;
\draw(-1.063,2.114)--(-1.075,2.114);
\filldraw[fill opacity=0.8,fill=gray!20,draw=none](-1.389,1.909)--(-1.379,1.924)--(-1.388,1.942)--(-1.453,1.958)--(-1.422,1.917)--cycle;
\draw(-1.379,1.924)--(-1.388,1.942)--(-1.453,1.958)--(-1.422,1.917)--(-1.389,1.909);
\filldraw[fill opacity=0.8,fill=gray!20](-1.381,2.258)--(-1.334,2.285)--(-1.34,2.291)--(-1.393,2.271)--cycle;
\filldraw[fill opacity=0.8,fill=gray!20,draw=none](-1.112,2.176)--(-1.093,2.188)--(-1.128,2.233)--(-1.154,2.216)--(-1.152,2.214)--cycle;
\draw(-1.112,2.176)--(-1.093,2.188)--(-1.128,2.233)--(-1.154,2.216)--(-1.152,2.214);
\filldraw[fill opacity=0.8,fill=gray!20,draw=none](-1.076,2.062)--(-1.072,2.062)--(-1.073,2.075)--cycle;
\draw(-1.076,2.062)--(-1.072,2.062);
\filldraw[fill opacity=0.8,fill=gray!20,draw=none](-1.075,2.114)--(-1.078,2.09)--(-1.076,2.073)--(-1.073,2.075)--cycle;
\draw(-1.076,2.073)--(-1.073,2.075);
\filldraw[fill opacity=0.8,fill=gray!20,draw=none](-1.081,2.041)--(-1.073,2.075)--(-1.076,2.073)--cycle;
\draw(-1.073,2.075)--(-1.076,2.073);
\filldraw[fill opacity=0.8,fill=gray!20](-1.334,2.285)--(-1.283,2.298)--(-1.283,2.298)--(-1.34,2.291)--cycle;
\filldraw[fill opacity=0.8,fill=gray!20](-1.128,2.233)--(-1.174,2.268)--(-1.192,2.256)--(-1.154,2.216)--cycle;
\filldraw[fill opacity=0.8,fill=gray!20,draw=none](-1.456,1.961)--(-1.453,1.958)--(-1.455,1.961)--cycle;
\draw(-1.456,1.961)--(-1.453,1.958)--(-1.455,1.961);
\filldraw[fill opacity=0.8,fill=gray!20](-1.258,2.279)--(-1.283,2.298)--(-1.283,2.298)--(-1.287,2.278)--cycle;
\filldraw[fill opacity=0.8,fill=gray!20](-1.287,2.278)--(-1.283,2.298)--(-1.283,2.298)--(-1.314,2.28)--cycle;
\filldraw[fill opacity=0.8,fill=gray!20](-1.236,2.284)--(-1.283,2.298)--(-1.283,2.298)--(-1.258,2.279)--cycle;
\filldraw[fill opacity=0.8,fill=gray!20](-1.314,2.28)--(-1.283,2.298)--(-1.283,2.298)--(-1.334,2.285)--cycle;
\filldraw[fill opacity=0.8,fill=gray!20](-1.226,2.29)--(-1.283,2.298)--(-1.283,2.298)--(-1.236,2.284)--cycle;
\filldraw[fill opacity=0.8,fill=gray!20](-1.29,1.874)--(-1.292,1.898)--(-1.369,1.904)--(-1.344,1.878)--cycle;
\filldraw[fill opacity=0.8,fill=gray!20](-1.174,2.268)--(-1.226,2.29)--(-1.236,2.284)--(-1.192,2.256)--cycle;
\filldraw[fill opacity=0.8,fill=gray!20,draw=none](-1.158,1.913)--(-1.154,1.913)--(-1.152,1.915)--cycle;
\draw(-1.158,1.913)--(-1.154,1.913)--(-1.152,1.915);
\filldraw[fill opacity=0.8,fill=gray!20,draw=none](-1.459,2.156)--(-1.453,2.172)--(-1.465,2.185)--cycle;
\draw(-1.459,2.156)--(-1.453,2.172)--(-1.465,2.185);
\filldraw[fill opacity=0.8,fill=gray!20,draw=none](-1.44,2.219)--(-1.547,2.224)--(-1.523,2.224)--(-1.422,2.22)--cycle;
\draw(-1.44,2.219)--(-1.547,2.224);
\draw(-1.523,2.224)--(-1.422,2.22);
\filldraw[fill opacity=0.8,fill=gray!20,draw=none](-1.234,1.877)--(-1.222,1.891)--(-1.287,1.898)--(-1.292,1.898)--(-1.29,1.874)--cycle;
\draw(-1.287,1.898)--(-1.292,1.898)--(-1.29,1.874)--(-1.234,1.877)--(-1.222,1.891);
\filldraw[fill opacity=0.8,fill=gray!20,draw=none](-1.473,2.159)--(-1.558,2.13)--(-1.534,2.131)--(-1.459,2.156)--cycle;
\draw(-1.473,2.159)--(-1.558,2.13);
\draw(-1.534,2.131)--(-1.459,2.156);
\filldraw[fill opacity=0.8,fill=gray!20,draw=none](-1.236,1.974)--(-1.375,1.928)--(-1.403,1.905)--(-1.367,1.917)--cycle;
\draw(-1.236,1.974)--(-1.375,1.928);
\draw(-1.403,1.905)--(-1.367,1.917);
\filldraw[fill opacity=0.8,fill=gray!20,draw=none](-1.407,1.947)--(-1.404,1.99)--(-1.473,2.007)--(-1.453,1.958)--cycle;
\draw(-1.404,1.99)--(-1.473,2.007)--(-1.453,1.958)--(-1.407,1.947);
\filldraw[fill opacity=0.5,fill=gray!20,draw=none](-1.646,1.947)--(-1.637,1.987)--(-1.76,2.041)--cycle;
\draw(-1.637,1.987)--(-1.76,2.041);
\filldraw[fill opacity=0.8,fill=gray!20,draw=none](-1.517,2.051)--(-1.609,2.02)--(-1.603,2.049)--cycle;
\draw(-1.517,2.051)--(-1.609,2.02);
\filldraw[fill opacity=0.8,fill=gray!20,draw=none](-1.495,2.017)--(-1.591,2.022)--(-1.599,2.069)--(-1.504,2.065)--cycle;
\draw(-1.495,2.017)--(-1.591,2.022);
\draw(-1.599,2.069)--(-1.504,2.065);
\filldraw[fill opacity=0.8,fill=gray!20,draw=none](-1.488,2.017)--(-1.495,2.017)--(-1.5,2.043)--cycle;
\draw(-1.488,2.017)--(-1.495,2.017);
\filldraw[fill opacity=0.5,fill=gray!20,draw=none](-1.642,1.965)--(-1.62,1.965)--(-1.632,1.985)--(-1.637,1.987)--cycle;
\draw(-1.632,1.985)--(-1.637,1.987);
\filldraw[fill opacity=0.5,fill=gray!20,draw=none](-1.617,1.962)--(-1.618,1.979)--(-1.632,1.985)--cycle;
\draw(-1.618,1.979)--(-1.632,1.985);
\filldraw[fill opacity=0.8,fill=gray!20,draw=none](-1.609,1.966)--(-1.798,1.903)--(-1.808,1.953)--(-1.619,2.016)--cycle;
\draw(-1.609,1.966)--(-1.798,1.903);
\draw(-1.808,1.953)--(-1.619,2.016);
\filldraw[fill opacity=0.8,fill=gray!20,draw=none](-1.599,2.069)--(-1.609,2.02)--(-1.619,2.016)--(-1.621,2.063)--(-1.601,2.069)--cycle;
\draw(-1.609,2.02)--(-1.619,2.016);
\draw(-1.621,2.063)--(-1.601,2.069);
\filldraw[fill opacity=0.8,fill=gray!20,draw=none](-1.611,1.976)--(-1.619,2.016)--(-1.609,2.02)--cycle;
\draw(-1.619,2.016)--(-1.609,2.02);
\filldraw[fill opacity=0.8,fill=gray!20,draw=none](-1.619,2.017)--(-1.619,2.016)--(-1.808,1.953)--(-1.809,1.956)--(-1.799,2.003)--(-1.75,2.019)--cycle;
\draw(-1.619,2.016)--(-1.808,1.953);
\draw(-1.799,2.003)--(-1.75,2.019);
\filldraw[fill opacity=0.8,fill=gray!20,draw=none](-1.619,2.017)--(-1.75,2.019)--(-1.621,2.063)--cycle;
\draw(-1.75,2.019)--(-1.621,2.063);
\filldraw[fill opacity=0.5,fill=gray!20,draw=none](-1.791,2.125)--(-1.598,2.129)--(-1.57,2.363)--(-1.605,2.441)--(-1.745,2.503)--cycle;
\draw(-1.598,2.129)--(-1.57,2.363);
\draw(-1.605,2.441)--(-1.745,2.503)--(-1.791,2.125);
\filldraw[fill opacity=0.8,fill=gray!20,draw=none](-1.599,2.069)--(-1.79,2.077)--(-1.788,2.129)--(-1.597,2.12)--cycle;
\draw(-1.599,2.069)--(-1.79,2.077);
\draw(-1.788,2.129)--(-1.597,2.12);
\filldraw[fill opacity=0.8,fill=gray!20,draw=none](-1.783,2.037)--(-1.79,2.077)--(-1.601,2.069)--cycle;
\draw(-1.79,2.077)--(-1.601,2.069);
\filldraw[fill opacity=0.8,fill=gray!20,draw=none](-1.599,2.07)--(-1.799,2.003)--(-1.791,2.019)--cycle;
\draw(-1.599,2.07)--(-1.799,2.003);
\filldraw[fill opacity=0.8,fill=gray!20,draw=none](-1.603,2.069)--(-1.791,2.019)--(-1.783,2.037)--cycle;
\filldraw[fill opacity=0.8,fill=gray!20,draw=none](-1.591,2.022)--(-1.782,2.03)--(-1.783,2.037)--(-1.601,2.069)--(-1.599,2.069)--cycle;
\draw(-1.591,2.022)--(-1.782,2.03);
\draw(-1.601,2.069)--(-1.599,2.069);
\filldraw[fill opacity=0.8,fill=gray!20,draw=none](-1.575,1.985)--(-1.766,1.994)--(-1.782,2.03)--(-1.591,2.022)--cycle;
\draw(-1.575,1.985)--(-1.766,1.994);
\draw(-1.782,2.03)--(-1.591,2.022);
\filldraw[fill opacity=0.8,fill=gray!20,draw=none](-1.235,2.01)--(-1.384,2.017)--(-1.401,1.998)--(-1.408,1.981)--(-1.253,1.974)--cycle;
\draw(-1.408,1.981)--(-1.253,1.974)--(-1.235,2.01)--(-1.384,2.017);
\filldraw[fill opacity=0.8,fill=gray!20,draw=none](-1.227,2.009)--(-1.28,1.987)--(-1.239,1.973)--cycle;
\filldraw[fill opacity=0.8,fill=gray!20,draw=none](-1.28,1.987)--(-1.373,1.947)--(-1.375,1.941)--(-1.375,1.928)--(-1.239,1.973)--cycle;
\draw(-1.375,1.928)--(-1.239,1.973);
\filldraw[fill opacity=0.8,fill=gray!20,draw=none](-1.423,1.961)--(-1.536,1.966)--(-1.557,1.966)--(-1.455,1.961)--cycle;
\draw(-1.423,1.961)--(-1.536,1.966);
\draw(-1.557,1.966)--(-1.455,1.961);
\filldraw[fill opacity=0.8,fill=gray!20,draw=none](-1.493,2.164)--(-1.587,2.168)--(-1.569,2.204)--(-1.468,2.2)--cycle;
\draw(-1.493,2.164)--(-1.587,2.168);
\draw(-1.569,2.204)--(-1.468,2.2);
\filldraw[fill opacity=0.8,fill=gray!20,draw=none](-1.587,2.168)--(-1.778,2.176)--(-1.76,2.213)--(-1.569,2.204)--cycle;
\draw(-1.587,2.168)--(-1.778,2.176);
\draw(-1.76,2.213)--(-1.569,2.204);
\filldraw[fill opacity=0.8,fill=gray!20,draw=none](-1.569,2.204)--(-1.76,2.213)--(-1.738,2.232)--(-1.547,2.224)--cycle;
\draw(-1.569,2.204)--(-1.76,2.213);
\draw(-1.738,2.232)--(-1.547,2.224);
\filldraw[fill opacity=0.8,fill=gray!20,draw=none](-1.245,2.193)--(-1.264,2.194)--(-1.244,2.157)--(-1.23,2.157)--cycle;
\draw(-1.244,2.157)--(-1.23,2.157)--(-1.245,2.193)--(-1.264,2.194);
\filldraw[fill opacity=0.8,fill=gray!20,draw=none](-1.388,1.91)--(-1.389,1.909)--(-1.392,1.909)--cycle;
\draw(-1.388,1.91)--(-1.389,1.909);
\filldraw[fill opacity=0.8,fill=gray!20,draw=none](-1.389,1.909)--(-1.369,1.904)--(-1.379,1.924)--cycle;
\draw(-1.389,1.909)--(-1.369,1.904)--(-1.379,1.924);
\filldraw[fill opacity=0.8,fill=gray!20](-1.344,2.249)--(-1.314,2.28)--(-1.334,2.285)--(-1.381,2.258)--cycle;
\filldraw[fill opacity=0.8,fill=gray!20,draw=none](-1.599,2.069)--(-1.601,2.069)--(-1.599,2.07)--cycle;
\draw(-1.601,2.069)--(-1.599,2.07);
\filldraw[fill opacity=0.8,fill=gray!20,draw=none](-1.599,2.07)--(-1.603,2.069)--(-1.783,2.037)--(-1.78,2.042)--(-1.581,2.109)--cycle;
\draw(-1.78,2.042)--(-1.581,2.109);
\filldraw[fill opacity=0.8,fill=gray!20,draw=none](-1.283,2.188)--(-1.428,2.139)--(-1.409,2.102)--(-1.258,2.153)--cycle;
\draw(-1.409,2.102)--(-1.258,2.153)--(-1.283,2.188)--(-1.428,2.139);
\filldraw[fill opacity=0.8,fill=gray!20,draw=none](-1.264,2.194)--(-1.395,2.2)--(-1.385,2.163)--(-1.244,2.157)--cycle;
\draw(-1.264,2.194)--(-1.395,2.2);
\draw(-1.385,2.163)--(-1.244,2.157);
\filldraw[fill opacity=0.8,fill=gray!20](-1.192,2.256)--(-1.236,2.284)--(-1.258,2.279)--(-1.234,2.248)--cycle;
\filldraw[fill opacity=0.8,fill=gray!20,draw=none](-1.375,1.941)--(-1.388,1.942)--(-1.375,1.917)--cycle;
\draw(-1.375,1.941)--(-1.388,1.942)--(-1.375,1.917);
\filldraw[fill opacity=0.8,fill=gray!20,draw=none](-1.375,1.928)--(-1.424,1.912)--(-1.408,1.903)--(-1.403,1.905)--cycle;
\draw(-1.375,1.928)--(-1.424,1.912);
\draw(-1.408,1.903)--(-1.403,1.905);
\filldraw[fill opacity=0.8,fill=gray!20](-1.4,1.989)--(-1.404,2.044)--(-1.48,2.062)--(-1.473,2.007)--cycle;
\filldraw[fill opacity=0.8,fill=gray!20,draw=none](-1.424,1.912)--(-1.508,1.883)--(-1.523,1.864)--(-1.408,1.903)--cycle;
\draw(-1.424,1.912)--(-1.508,1.883);
\draw(-1.523,1.864)--(-1.408,1.903);
\filldraw[fill opacity=0.8,fill=gray!20](-1.369,2.207)--(-1.344,2.249)--(-1.381,2.258)--(-1.422,2.22)--cycle;
\filldraw[fill opacity=0.8,fill=gray!20,draw=none](-1.407,1.947)--(-1.388,1.942)--(-1.4,1.989)--(-1.404,1.99)--cycle;
\draw(-1.407,1.947)--(-1.388,1.942)--(-1.4,1.989)--(-1.404,1.99);
\filldraw[fill opacity=0.8,fill=gray!20,draw=none](-1.373,1.947)--(-1.22,2.013)--(-1.366,1.964)--cycle;
\draw(-1.22,2.013)--(-1.366,1.964);
\filldraw[fill opacity=0.5,fill=gray!20,draw=none](-1.806,1.86)--(-1.794,1.488)--(-1.933,1.549)--(-1.99,2.011)--cycle;
\draw(-1.794,1.488)--(-1.933,1.549)--(-1.99,2.011);
\filldraw[fill opacity=0.5,fill=gray!20](-1.961,1.494)--(-1.933,1.549)--(-1.993,2.037)--(-2.023,2.005)--cycle;
\filldraw[fill opacity=0.5,fill=gray!20,draw=none](-1.807,1.898)--(-1.806,1.86)--(-1.99,2.011)--(-1.993,2.037)--(-1.856,1.977)--cycle;
\draw(-1.99,2.011)--(-1.993,2.037)--(-1.856,1.977);
\filldraw[fill opacity=0.5,fill=gray!20](-1.96,1.444)--(-1.961,1.494)--(-2.023,2.005)--(-2.025,1.972)--cycle;
\filldraw[fill opacity=0.8,fill=gray!20,draw=none](-1.591,1.92)--(-1.78,1.856)--(-1.798,1.903)--(-1.609,1.966)--cycle;
\draw(-1.591,1.92)--(-1.78,1.856);
\draw(-1.798,1.903)--(-1.609,1.966);
\filldraw[fill opacity=0.8,fill=gray!20,draw=none](-1.569,1.884)--(-1.758,1.821)--(-1.78,1.856)--(-1.591,1.92)--cycle;
\draw(-1.569,1.884)--(-1.758,1.821);
\draw(-1.78,1.856)--(-1.591,1.92);
\filldraw[fill opacity=0.8,fill=gray!20,draw=none](-1.224,2.058)--(-1.227,2.01)--(-1.22,2.013)--cycle;
\draw(-1.227,2.01)--(-1.22,2.013)--(-1.224,2.058);
\filldraw[fill opacity=0.8,fill=gray!20](-1.29,2.245)--(-1.287,2.278)--(-1.314,2.28)--(-1.344,2.249)--cycle;
\filldraw[fill opacity=0.8,fill=gray!20,draw=none](-1.292,1.898)--(-1.293,1.9)--(-1.333,1.938)--(-1.375,1.941)--(-1.375,1.917)--(-1.369,1.904)--cycle;
\draw(-1.333,1.938)--(-1.375,1.941);
\draw(-1.375,1.917)--(-1.369,1.904)--(-1.292,1.898)--(-1.293,1.9);
\filldraw[fill opacity=0.8,fill=gray!20,draw=none](-1.494,2.03)--(-1.5,2.043)--(-1.504,2.065)--(-1.502,2.065)--cycle;
\draw(-1.504,2.065)--(-1.502,2.065);
\filldraw[fill opacity=0.8,fill=gray!20,draw=none](-1.224,2.058)--(-1.382,2.037)--(-1.384,2.017)--(-1.235,2.01)--cycle;
\draw(-1.384,2.017)--(-1.235,2.01)--(-1.224,2.058);
\filldraw[fill opacity=0.8,fill=gray!20,draw=none](-1.224,2.058)--(-1.369,2.009)--(-1.372,1.987)--(-1.366,1.964)--(-1.227,2.01)--cycle;
\draw(-1.224,2.058)--(-1.369,2.009);
\draw(-1.366,1.964)--(-1.227,2.01);
\filldraw[fill opacity=0.8,fill=gray!20,draw=none](-1.597,2.12)--(-1.788,2.129)--(-1.778,2.176)--(-1.587,2.168)--cycle;
\draw(-1.597,2.12)--(-1.788,2.129);
\draw(-1.778,2.176)--(-1.587,2.168);
\filldraw[fill opacity=0.8,fill=gray!20,draw=none](-1.581,2.109)--(-1.78,2.042)--(-1.758,2.063)--(-1.558,2.13)--cycle;
\draw(-1.581,2.109)--(-1.78,2.042);
\draw(-1.758,2.063)--(-1.558,2.13);
\filldraw[fill opacity=0.8,fill=gray!20,draw=none](-1.23,2.157)--(-1.244,2.157)--(-1.23,2.11)--(-1.222,2.109)--cycle;
\draw(-1.23,2.11)--(-1.222,2.109)--(-1.23,2.157)--(-1.244,2.157);
\filldraw[fill opacity=0.8,fill=gray!20,draw=none](-1.404,2.044)--(-1.402,2.078)--(-1.409,2.103)--(-1.473,2.118)--(-1.48,2.062)--cycle;
\draw(-1.409,2.103)--(-1.473,2.118)--(-1.48,2.062)--(-1.404,2.044)--(-1.402,2.078);
\filldraw[fill opacity=0.8,fill=gray!20](-1.234,2.248)--(-1.258,2.279)--(-1.287,2.278)--(-1.29,2.245)--cycle;
\filldraw[fill opacity=0.8,fill=gray!20,draw=none](-1.158,2.216)--(-1.154,2.216)--(-1.192,2.256)--(-1.234,2.248)--(-1.222,2.223)--cycle;
\draw(-1.158,2.216)--(-1.154,2.216)--(-1.192,2.256)--(-1.234,2.248)--(-1.222,2.223);
\filldraw[fill opacity=0.8,fill=gray!20](-1.388,2.156)--(-1.369,2.207)--(-1.422,2.22)--(-1.453,2.172)--cycle;
\filldraw[fill opacity=0.8,fill=gray!20,draw=none](-1.422,2.22)--(-1.436,2.22)--(-1.405,2.2)--(-1.395,2.2)--cycle;
\draw(-1.422,2.22)--(-1.436,2.22);
\draw(-1.405,2.2)--(-1.395,2.2);
\filldraw[fill opacity=0.8,fill=gray!20,draw=none](-1.405,1.981)--(-1.508,1.985)--(-1.53,1.966)--(-1.423,1.961)--cycle;
\draw(-1.405,1.981)--(-1.508,1.985);
\draw(-1.53,1.966)--(-1.423,1.961);
\filldraw[fill opacity=0.8,fill=gray!20,draw=none](-1.557,1.966)--(-1.745,1.974)--(-1.766,1.994)--(-1.581,1.986)--cycle;
\draw(-1.557,1.966)--(-1.745,1.974);
\draw(-1.766,1.994)--(-1.581,1.986);
\filldraw[fill opacity=0.8,fill=gray!20,draw=none](-1.258,2.153)--(-1.409,2.102)--(-1.408,2.087)--(-1.386,2.057)--(-1.238,2.107)--cycle;
\draw(-1.386,2.057)--(-1.238,2.107)--(-1.258,2.153)--(-1.409,2.102);
\filldraw[fill opacity=0.8,fill=gray!20,draw=none](-1.244,2.157)--(-1.385,2.163)--(-1.385,2.157)--(-1.371,2.116)--(-1.23,2.11)--cycle;
\draw(-1.244,2.157)--(-1.385,2.163);
\draw(-1.371,2.116)--(-1.23,2.11);
\filldraw[fill opacity=0.8,fill=gray!20](-1.4,2.1)--(-1.388,2.156)--(-1.453,2.172)--(-1.473,2.118)--cycle;
\filldraw[fill opacity=0.8,fill=gray!20,draw=none](-1.459,2.156)--(-1.534,2.131)--(-1.513,2.11)--(-1.428,2.139)--cycle;
\draw(-1.459,2.156)--(-1.534,2.131);
\draw(-1.513,2.11)--(-1.428,2.139);
\filldraw[fill opacity=0.8,fill=gray!20,draw=none](-1.436,2.22)--(-1.523,2.224)--(-1.502,2.204)--(-1.405,2.2)--cycle;
\draw(-1.436,2.22)--(-1.523,2.224);
\draw(-1.502,2.204)--(-1.405,2.2);
\filldraw[fill opacity=0.8,fill=gray!20,draw=none](-1.558,2.13)--(-1.758,2.063)--(-1.734,2.064)--(-1.534,2.131)--cycle;
\draw(-1.558,2.13)--(-1.758,2.063);
\draw(-1.734,2.064)--(-1.534,2.131);
\filldraw[fill opacity=0.8,fill=gray!20,draw=none](-1.222,2.109)--(-1.23,2.11)--(-1.224,2.058)--cycle;
\draw(-1.224,2.058)--(-1.222,2.109)--(-1.23,2.11);
\filldraw[fill opacity=0.8,fill=gray!20,draw=none](-1.238,2.107)--(-1.386,2.057)--(-1.384,2.042)--(-1.376,2.025)--(-1.316,2.027)--(-1.224,2.058)--cycle;
\draw(-1.316,2.027)--(-1.224,2.058)--(-1.238,2.107)--(-1.386,2.057);
\filldraw[fill opacity=0.8,fill=gray!20,draw=none](-1.23,2.11)--(-1.371,2.116)--(-1.374,2.099)--(-1.371,2.064)--(-1.224,2.058)--cycle;
\draw(-1.23,2.11)--(-1.371,2.116);
\draw(-1.371,2.064)--(-1.224,2.058);
\filldraw[fill opacity=0.8,fill=gray!20,draw=none](-1.381,2.042)--(-1.382,2.037)--(-1.224,2.058)--(-1.317,2.062)--(-1.378,2.048)--cycle;
\draw(-1.224,2.058)--(-1.317,2.062);
\filldraw[fill opacity=0.8,fill=gray!20,draw=none](-1.287,1.898)--(-1.293,1.9)--(-1.292,1.898)--cycle;
\draw(-1.293,1.9)--(-1.292,1.898)--(-1.287,1.898);
\filldraw[fill opacity=0.8,fill=gray!20,draw=none](-1.152,2.214)--(-1.154,2.216)--(-1.158,2.216)--cycle;
\draw(-1.152,2.214)--(-1.154,2.216)--(-1.158,2.216);
\filldraw[fill opacity=0.8,fill=gray!20,draw=none](-1.375,1.941)--(-1.381,1.926)--(-1.375,1.928)--cycle;
\draw(-1.381,1.926)--(-1.375,1.928);
\filldraw[fill opacity=0.8,fill=gray!20,draw=none](-1.36,1.94)--(-1.372,1.987)--(-1.4,1.989)--(-1.388,1.942)--cycle;
\draw(-1.372,1.987)--(-1.4,1.989)--(-1.388,1.942)--(-1.36,1.94);
\filldraw[fill opacity=0.8,fill=gray!20,draw=none](-1.375,1.941)--(-1.375,1.946)--(-1.399,1.935)--(-1.388,1.923)--(-1.381,1.926)--cycle;
\draw(-1.388,1.923)--(-1.381,1.926);
\filldraw[fill opacity=0.8,fill=gray!20,draw=none](-1.399,1.935)--(-1.506,1.889)--(-1.508,1.883)--(-1.388,1.923)--cycle;
\draw(-1.508,1.883)--(-1.388,1.923);
\filldraw[fill opacity=0.8,fill=gray!20](-1.292,2.201)--(-1.29,2.245)--(-1.344,2.249)--(-1.369,2.207)--cycle;
\filldraw[fill opacity=0.8,fill=gray!20,draw=none](-1.373,1.947)--(-1.375,1.946)--(-1.375,1.941)--cycle;
\filldraw[fill opacity=0.8,fill=gray!20,draw=none](-1.287,2.202)--(-1.222,2.223)--(-1.234,2.248)--(-1.29,2.245)--(-1.292,2.201)--cycle;
\draw(-1.222,2.223)--(-1.234,2.248)--(-1.29,2.245)--(-1.292,2.201)--(-1.287,2.202);
\filldraw[fill opacity=0.8,fill=gray!20,draw=none](-1.373,1.947)--(-1.366,1.964)--(-1.5,1.919)--(-1.506,1.889)--cycle;
\draw(-1.366,1.964)--(-1.5,1.919);
\filldraw[fill opacity=0.8,fill=gray!20,draw=none](-1.36,1.94)--(-1.333,1.938)--(-1.36,1.987)--(-1.372,1.987)--cycle;
\draw(-1.36,1.94)--(-1.333,1.938);
\draw(-1.36,1.987)--(-1.372,1.987);
\filldraw[fill opacity=0.8,fill=gray!20,draw=none](-1.395,2.2)--(-1.405,2.2)--(-1.387,2.164)--(-1.385,2.163)--cycle;
\draw(-1.395,2.2)--(-1.405,2.2);
\draw(-1.387,2.164)--(-1.385,2.163);
\filldraw[fill opacity=0.8,fill=gray!20,draw=none](-1.428,2.139)--(-1.435,2.137)--(-1.409,2.102)--cycle;
\draw(-1.428,2.139)--(-1.435,2.137);
\filldraw[fill opacity=0.8,fill=gray!20,draw=none](-1.405,2.2)--(-1.427,2.201)--(-1.401,2.164)--(-1.387,2.164)--cycle;
\draw(-1.405,2.2)--(-1.427,2.201);
\draw(-1.401,2.164)--(-1.387,2.164);
\filldraw[fill opacity=0.8,fill=gray!20,draw=none](-1.435,2.137)--(-1.513,2.11)--(-1.497,2.073)--(-1.409,2.102)--cycle;
\draw(-1.435,2.137)--(-1.513,2.11);
\draw(-1.497,2.073)--(-1.409,2.102);
\filldraw[fill opacity=0.8,fill=gray!20,draw=none](-1.401,1.998)--(-1.416,1.981)--(-1.408,1.981)--cycle;
\draw(-1.416,1.981)--(-1.408,1.981);
\filldraw[fill opacity=0.8,fill=gray!20,draw=none](-1.36,1.987)--(-1.361,1.99)--(-1.384,2.042)--(-1.404,2.044)--(-1.4,1.989)--cycle;
\draw(-1.384,2.042)--(-1.404,2.044)--(-1.4,1.989)--(-1.36,1.987);
\filldraw[fill opacity=0.8,fill=gray!20,draw=none](-1.401,1.998)--(-1.393,2.017)--(-1.49,2.022)--(-1.508,1.985)--(-1.416,1.981)--cycle;
\draw(-1.393,2.017)--(-1.49,2.022);
\draw(-1.508,1.985)--(-1.416,1.981);
\filldraw[fill opacity=0.8,fill=gray!20,draw=none](-1.372,1.987)--(-1.374,1.961)--(-1.366,1.964)--cycle;
\draw(-1.374,1.961)--(-1.366,1.964);
\filldraw[fill opacity=0.8,fill=gray!20,draw=none](-1.384,2.017)--(-1.393,2.017)--(-1.401,1.998)--cycle;
\draw(-1.384,2.017)--(-1.393,2.017);
\filldraw[fill opacity=0.8,fill=gray!20,draw=none](-1.376,2.025)--(-1.369,2.009)--(-1.316,2.027)--cycle;
\draw(-1.369,2.009)--(-1.316,2.027);
\filldraw[fill opacity=0.8,fill=gray!20,draw=none](-1.427,2.201)--(-1.502,2.204)--(-1.486,2.168)--(-1.401,2.164)--cycle;
\draw(-1.427,2.201)--(-1.502,2.204);
\draw(-1.486,2.168)--(-1.401,2.164);
\filldraw[fill opacity=0.8,fill=gray!20,draw=none](-1.369,2.009)--(-1.394,2.001)--(-1.407,1.95)--(-1.374,1.961)--cycle;
\draw(-1.369,2.009)--(-1.394,2.001);
\draw(-1.407,1.95)--(-1.374,1.961);
\filldraw[fill opacity=0.8,fill=gray!20,draw=none](-1.333,2.152)--(-1.293,2.199)--(-1.292,2.201)--(-1.369,2.207)--(-1.388,2.156)--cycle;
\draw(-1.293,2.199)--(-1.292,2.201)--(-1.369,2.207)--(-1.388,2.156)--(-1.333,2.152);
\filldraw[fill opacity=0.8,fill=gray!20,draw=none](-1.378,2.048)--(-1.317,2.062)--(-1.371,2.064)--cycle;
\draw(-1.317,2.062)--(-1.371,2.064);
\filldraw[fill opacity=0.8,fill=gray!20,draw=none](-1.394,2.001)--(-1.492,1.968)--(-1.502,1.918)--(-1.407,1.95)--cycle;
\draw(-1.394,2.001)--(-1.492,1.968);
\draw(-1.502,1.918)--(-1.407,1.95);
\filldraw[fill opacity=0.8,fill=gray!20,draw=none](-1.361,1.99)--(-1.369,2.041)--(-1.384,2.042)--cycle;
\draw(-1.369,2.041)--(-1.384,2.042);
\filldraw[fill opacity=0.8,fill=gray!20,draw=none](-1.402,2.078)--(-1.4,2.1)--(-1.409,2.103)--cycle;
\draw(-1.402,2.078)--(-1.4,2.1)--(-1.409,2.103);
\filldraw[fill opacity=0.8,fill=gray!20,draw=none](-1.536,1.966)--(-1.721,1.974)--(-1.745,1.974)--(-1.557,1.966)--cycle;
\draw(-1.536,1.966)--(-1.721,1.974);
\draw(-1.745,1.974)--(-1.557,1.966);
\filldraw[fill opacity=0.8,fill=gray!20,draw=none](-1.508,1.985)--(-1.699,1.994)--(-1.721,1.974)--(-1.53,1.966)--cycle;
\draw(-1.508,1.985)--(-1.699,1.994);
\draw(-1.721,1.974)--(-1.53,1.966);
\filldraw[fill opacity=0.8,fill=gray!20,draw=none](-1.49,2.022)--(-1.492,2.022)--(-1.696,1.999)--(-1.699,1.994)--(-1.508,1.985)--cycle;
\draw(-1.49,2.022)--(-1.492,2.022);
\draw(-1.699,1.994)--(-1.508,1.985);
\filldraw[fill opacity=0.8,fill=gray!20,draw=none](-1.409,2.102)--(-1.497,2.073)--(-1.49,2.023)--(-1.406,2.05)--cycle;
\draw(-1.409,2.102)--(-1.497,2.073);
\draw(-1.49,2.023)--(-1.406,2.05);
\filldraw[fill opacity=0.8,fill=gray!20,draw=none](-1.382,2.037)--(-1.394,2.035)--(-1.399,2.018)--(-1.384,2.017)--cycle;
\draw(-1.399,2.018)--(-1.384,2.017);
\filldraw[fill opacity=0.8,fill=gray!20,draw=none](-1.385,2.157)--(-1.385,2.163)--(-1.385,2.159)--cycle;
\filldraw[fill opacity=0.8,fill=gray!20,draw=none](-1.385,2.159)--(-1.385,2.163)--(-1.387,2.164)--cycle;
\draw(-1.385,2.163)--(-1.387,2.164);
\filldraw[fill opacity=0.8,fill=gray!20,draw=none](-1.374,2.099)--(-1.362,2.154)--(-1.388,2.156)--(-1.4,2.1)--cycle;
\draw(-1.362,2.154)--(-1.388,2.156)--(-1.4,2.1)--(-1.374,2.099);
\filldraw[fill opacity=0.8,fill=gray!20,draw=none](-1.385,2.159)--(-1.387,2.164)--(-1.394,2.164)--(-1.48,2.132)--(-1.478,2.12)--(-1.392,2.117)--cycle;
\draw(-1.387,2.164)--(-1.394,2.164);
\draw(-1.478,2.12)--(-1.392,2.117);
\filldraw[fill opacity=0.8,fill=gray!20,draw=none](-1.376,2.025)--(-1.382,2.025)--(-1.38,2.005)--(-1.369,2.009)--cycle;
\draw(-1.38,2.005)--(-1.369,2.009);
\filldraw[fill opacity=0.8,fill=gray!20,draw=none](-1.384,2.042)--(-1.39,2.1)--(-1.4,2.1)--(-1.404,2.044)--cycle;
\draw(-1.39,2.1)--(-1.4,2.1)--(-1.404,2.044)--(-1.384,2.042);
\filldraw[fill opacity=0.8,fill=gray!20,draw=none](-1.385,2.157)--(-1.385,2.159)--(-1.386,2.156)--(-1.384,2.138)--cycle;
\filldraw[fill opacity=0.8,fill=gray!20,draw=none](-1.385,2.157)--(-1.384,2.138)--(-1.383,2.127)--(-1.376,2.116)--(-1.371,2.116)--cycle;
\draw(-1.376,2.116)--(-1.371,2.116);
\filldraw[fill opacity=0.8,fill=gray!20,draw=none](-1.386,2.156)--(-1.392,2.117)--(-1.382,2.116)--cycle;
\draw(-1.392,2.117)--(-1.382,2.116);
\filldraw[fill opacity=0.8,fill=gray!20,draw=none](-1.408,2.087)--(-1.406,2.05)--(-1.386,2.057)--cycle;
\draw(-1.406,2.05)--(-1.386,2.057);
\filldraw[fill opacity=0.8,fill=gray!20,draw=none](-1.534,2.131)--(-1.734,2.064)--(-1.713,2.044)--(-1.513,2.11)--cycle;
\draw(-1.534,2.131)--(-1.734,2.064);
\draw(-1.713,2.044)--(-1.513,2.11);
\filldraw[fill opacity=0.8,fill=gray!20,draw=none](-1.48,2.132)--(-1.394,2.164)--(-1.486,2.168)--cycle;
\draw(-1.394,2.164)--(-1.486,2.168);
\filldraw[fill opacity=0.8,fill=gray!20,draw=none](-1.394,2.035)--(-1.49,2.022)--(-1.49,2.022)--(-1.399,2.018)--cycle;
\draw(-1.49,2.022)--(-1.399,2.018);
\filldraw[fill opacity=0.8,fill=gray!20,draw=none](-1.382,2.037)--(-1.49,2.022)--(-1.49,2.022)--(-1.376,2.025)--cycle;
\filldraw[fill opacity=0.8,fill=gray!20,draw=none](-1.382,2.025)--(-1.49,2.022)--(-1.49,2)--(-1.399,1.999)--(-1.38,2.005)--cycle;
\draw(-1.399,1.999)--(-1.38,2.005);
\filldraw[fill opacity=0.8,fill=gray!20,draw=none](-1.287,2.202)--(-1.292,2.201)--(-1.293,2.199)--cycle;
\draw(-1.287,2.202)--(-1.292,2.201)--(-1.293,2.199);
\filldraw[fill opacity=0.8,fill=gray!20,draw=none](-1.49,2.022)--(-1.382,2.037)--(-1.39,2.056)--(-1.49,2.023)--cycle;
\draw(-1.39,2.056)--(-1.49,2.023);
\filldraw[fill opacity=0.8,fill=gray!20,draw=none](-1.384,2.036)--(-1.382,2.037)--(-1.381,2.042)--cycle;
\filldraw[fill opacity=0.8,fill=gray!20,draw=none](-1.384,2.042)--(-1.369,2.041)--(-1.36,2.097)--(-1.39,2.1)--cycle;
\draw(-1.384,2.042)--(-1.369,2.041);
\draw(-1.36,2.097)--(-1.39,2.1);
\filldraw[fill opacity=0.8,fill=gray!20,draw=none](-1.386,2.057)--(-1.39,2.056)--(-1.384,2.042)--cycle;
\draw(-1.386,2.057)--(-1.39,2.056);
\filldraw[fill opacity=0.8,fill=gray!20,draw=none](-1.384,2.036)--(-1.381,2.042)--(-1.38,2.047)--(-1.391,2.045)--(-1.394,2.035)--cycle;
\filldraw[fill opacity=0.8,fill=gray!20,draw=none](-1.381,2.042)--(-1.378,2.048)--(-1.38,2.047)--cycle;
\filldraw[fill opacity=0.8,fill=gray!20,draw=none](-1.374,2.099)--(-1.36,2.097)--(-1.333,2.152)--(-1.362,2.154)--cycle;
\draw(-1.374,2.099)--(-1.36,2.097);
\draw(-1.333,2.152)--(-1.362,2.154);
\filldraw[fill opacity=0.8,fill=gray!20,draw=none](-1.391,2.045)--(-1.378,2.048)--(-1.371,2.064)--(-1.386,2.065)--cycle;
\draw(-1.371,2.064)--(-1.386,2.065);
\filldraw[fill opacity=0.8,fill=gray!20,draw=none](-1.394,2.035)--(-1.391,2.045)--(-1.49,2.022)--(-1.49,2.022)--cycle;
\filldraw[fill opacity=0.8,fill=gray!20,draw=none](-1.371,2.116)--(-1.376,2.116)--(-1.374,2.099)--cycle;
\draw(-1.371,2.116)--(-1.376,2.116);
\filldraw[fill opacity=0.8,fill=gray!20,draw=none](-1.49,2)--(-1.492,1.968)--(-1.399,1.999)--cycle;
\draw(-1.492,1.968)--(-1.399,1.999);
\filldraw[fill opacity=0.8,fill=gray!20,draw=none](-1.376,2.116)--(-1.401,2.117)--(-1.379,2.065)--(-1.371,2.064)--cycle;
\draw(-1.376,2.116)--(-1.401,2.117);
\draw(-1.379,2.065)--(-1.371,2.064);
\filldraw[fill opacity=0.8,fill=gray!20,draw=none](-1.513,2.11)--(-1.713,2.044)--(-1.697,2.006)--(-1.497,2.073)--cycle;
\draw(-1.513,2.11)--(-1.713,2.044);
\draw(-1.697,2.006)--(-1.497,2.073);
\filldraw[fill opacity=0.8,fill=gray!20,draw=none](-1.497,2.073)--(-1.697,2.006)--(-1.696,1.999)--(-1.492,2.022)--(-1.49,2.023)--cycle;
\draw(-1.497,2.073)--(-1.697,2.006);
\draw(-1.492,2.022)--(-1.49,2.023);
\filldraw[fill opacity=0.8,fill=gray!20,draw=none](-1.49,2.022)--(-1.391,2.045)--(-1.386,2.065)--(-1.48,2.069)--cycle;
\draw(-1.386,2.065)--(-1.48,2.069);
\filldraw[fill opacity=0.8,fill=gray!20,draw=none](-1.383,2.127)--(-1.382,2.116)--(-1.376,2.116)--cycle;
\draw(-1.382,2.116)--(-1.376,2.116);
\filldraw[fill opacity=0.8,fill=gray!20,draw=none](-1.401,2.117)--(-1.478,2.12)--(-1.48,2.069)--(-1.379,2.065)--cycle;
\draw(-1.401,2.117)--(-1.478,2.12);
\draw(-1.48,2.069)--(-1.379,2.065);
\filldraw[fill opacity=0.5,fill=gray!20](1.616,2.265)--(1.563,1.82)--(1.408,1.393)--(1.161,1.015)--(.838,.71)--(.463,.5)--(.061,.398)--(-.342,.412)--(-.717,.54)--(-1.039,.775)--(-1.286,1.099)--(-1.442,1.492)--(-1.495,1.925)--(-1.442,2.37)--(-1.286,2.797)--(-1.039,3.175)--(-.717,3.48)--(-.342,3.69)--(.061,3.792)--(.463,3.778)--(.838,3.65)--(1.161,3.415)--(1.408,3.09)--(1.563,2.698)--cycle;
\filldraw[fill opacity=0.5,fill=gray!20](1.563,1.82)--(1.745,1.74)--(1.59,1.314)--(1.408,1.393)--cycle;
\filldraw[fill opacity=0.5,fill=gray!20](1.408,1.393)--(1.59,1.314)--(1.343,.935)--(1.161,1.015)--cycle;
\filldraw[fill opacity=0.5,fill=gray!20](1.486,.766)--(1.47,.749)--(1.078,.378)--(1.107,.407)--cycle;
\filldraw[fill opacity=0.5,fill=gray!20](1.616,2.265)--(1.798,2.185)--(1.745,1.74)--(1.563,1.82)--cycle;
\filldraw[fill opacity=0.5,fill=gray!20](1.375,3.772)--(1.299,3.79)--(1.606,3.387)--(1.684,3.366)--cycle;
\filldraw[fill opacity=0.5,fill=gray!20](1.161,1.015)--(1.343,.935)--(1.021,.63)--(.838,.71)--cycle;
\filldraw[fill opacity=0.5,fill=gray!20](1.961,2.749)--(1.96,2.799)--(2.025,2.271)--(2.023,2.238)--cycle;
\filldraw[fill opacity=0.5,fill=gray!20](-.665,.221)--(-.696,.334)--(-1.118,.478)--(-1.107,.373)--cycle;
\filldraw[fill opacity=0.5,fill=gray!20](-.21,.122)--(-.696,.334)--(-1.118,.479)--(-.632,.266)--cycle;
\filldraw[fill opacity=0.5,fill=gray!20](1.771,3.276)--(1.739,3.329)--(1.932,2.841)--(1.96,2.799)--cycle;
\filldraw[fill opacity=0.5,fill=gray!20](1.563,2.698)--(1.745,2.619)--(1.798,2.185)--(1.616,2.265)--cycle;
\filldraw[fill opacity=0.5,fill=gray!20](1.933,1.706)--(1.961,1.714)--(1.777,1.212)--(1.758,1.226)--cycle;
\filldraw[fill opacity=0.5,fill=gray!20](1.758,1.226)--(1.777,1.212)--(1.486,.766)--(1.48,.8)--cycle;
\filldraw[fill opacity=0.5,fill=gray!20,draw=none](-.378,.054)--(-.245,.049)--(-.337,.119)--(-.513,.125)--cycle;
\draw(-.378,.054)--(-.245,.049);
\draw(-.337,.119)--(-.513,.125);
\filldraw[fill opacity=0.5,fill=gray!20](1.993,2.206)--(2.023,2.238)--(1.961,1.714)--(1.933,1.706)--cycle;
\filldraw[fill opacity=0.5,fill=gray!20](1.107,.407)--(1.078,.378)--(.621,.122)--(.665,.159)--cycle;
\filldraw[fill opacity=0.5,fill=gray!20](.838,.71)--(1.021,.63)--(.646,.42)--(.463,.5)--cycle;
\filldraw[fill opacity=0.8,fill=gray!20,draw=none](-.204,-.027)--(-.203,-.032)--(-.199,-.034)--cycle;
\draw(-.204,-.027)--(-.203,-.032)--(-.199,-.034);
\filldraw[fill opacity=0.5,fill=gray!20,draw=none](.226,.027)--(.23,.026)--(.567,.111)--(.558,.113)--cycle;
\draw(.23,.026)--(.567,.111)--(.558,.113);
\filldraw[fill opacity=0.5,fill=gray!20](1.433,3.732)--(1.375,3.772)--(1.684,3.366)--(1.739,3.329)--cycle;
\filldraw[fill opacity=0.5,fill=gray!20](.899,4.081)--(.816,4.071)--(1.208,3.786)--(1.299,3.79)--cycle;
\filldraw[fill opacity=0.5,fill=gray!20](1.48,.8)--(1.486,.766)--(1.107,.407)--(1.118,.457)--cycle;
\filldraw[fill opacity=0.5,fill=gray!20](-.621,.129)--(-.665,.221)--(-1.107,.373)--(-1.078,.286)--cycle;
\filldraw[fill opacity=0.8,fill=gray!20,draw=none](.163,.011)--(.196,-.01)--(.196,-.014)--(.147,.007)--cycle;
\draw(.196,-.014)--(.147,.007);
\filldraw[fill opacity=0.8,fill=gray!20,draw=none](.139,.004)--(.139,-.111)--(.146,-.052)--(.146,.007)--cycle;
\draw(.139,.004)--(.139,-.111);
\draw(.146,-.052)--(.146,.007);
\filldraw[fill opacity=0.8,fill=gray!20,draw=none](.188,.016)--(.195,-.009)--(.163,.011)--cycle;
\filldraw[fill opacity=0.8,fill=gray!20,draw=none](.111,-.004)--(.111,-.026)--(.139,-.006)--(.139,.004)--cycle;
\draw(.111,-.004)--(.111,-.026);
\draw(.139,-.006)--(.139,.004);
\filldraw[fill opacity=0.8,fill=gray!20,draw=none](.188,.017)--(.196,-.014)--(.229,-.011)--(.226,.027)--cycle;
\draw(.196,-.014)--(.229,-.011)--(.226,.027);
\filldraw[fill opacity=0.8,fill=gray!20,draw=none](.147,.027)--(.137,.03)--(.146,.057)--cycle;
\filldraw[fill opacity=0.8,fill=gray!20,draw=none](.137,.03)--(.146,.057)--(.146,.025)--cycle;
\draw(.146,.057)--(.146,.025);
\filldraw[fill opacity=0.8,fill=gray!20,draw=none](.111,-.026)--(.111,-.076)--(.139,-.111)--(.139,-.006)--cycle;
\draw(.111,-.026)--(.111,-.076);
\draw(.139,-.111)--(.139,-.006);
\filldraw[fill opacity=0.8,fill=gray!20,draw=none](.133,.003)--(.132,-.021)--(.196,-.014)--(.188,.017)--cycle;
\draw(.132,-.021)--(.196,-.014);
\filldraw[fill opacity=0.8,fill=gray!20,draw=none](.188,.016)--(.192,.016)--(.193,.016)--(.196,-.01)--(.195,-.009)--cycle;
\filldraw[fill opacity=0.8,fill=gray!20,draw=none](.143,-.122)--(.146,-.13)--(.146,-.052)--cycle;
\draw(.146,-.13)--(.146,-.052);
\filldraw[fill opacity=0.8,fill=gray!20,draw=none](.139,-.111)--(.145,-.084)--(.146,-.052)--cycle;
\filldraw[fill opacity=0.8,fill=gray!20,draw=none](.193,.017)--(.205,.019)--(.209,-.004)--(.199,-.012)--(.196,-.01)--cycle;
\draw(.205,.019)--(.209,-.004);
\filldraw[fill opacity=0.8,fill=gray!20,draw=none](.065,-.015)--(.065,-.029)--(.111,-.076)--(.111,-.001)--cycle;
\draw(.065,-.015)--(.065,-.029);
\draw(.111,-.076)--(.111,-.001);
\filldraw[fill opacity=0.8,fill=gray!20,draw=none](-.177,-.069)--(-.164,-.062)--(-.139,-.091)--cycle;
\filldraw[fill opacity=0.8,fill=gray!20,draw=none](-.224,-.046)--(-.217,-.045)--(-.139,-.091)--(-.21,-.099)--cycle;
\draw(-.139,-.091)--(-.21,-.099)--(-.224,-.046)--(-.217,-.045);
\filldraw[fill opacity=0.8,fill=gray!20,draw=none](.139,-.111)--(.143,-.122)--(.145,-.084)--cycle;
\filldraw[fill opacity=0.8,fill=gray!20,draw=none](.122,.103)--(.111,.034)--(.111,-.001)--(.139,.013)--(.139,.091)--cycle;
\draw(.111,.034)--(.111,-.001);
\draw(.139,.013)--(.139,.091);
\filldraw[fill opacity=0.5,fill=gray!20,draw=none](.078,-.011)--(.133,.003)--(.135,.006)--cycle;
\filldraw[fill opacity=0.8,fill=gray!20,draw=none](.078,-.011)--(.111,-.001)--(.111,.034)--cycle;
\draw(.111,-.001)--(.111,.034);
\filldraw[fill opacity=0.8,fill=gray!20,draw=none](.101,-.025)--(.132,-.021)--(.134,.036)--(.111,.034)--cycle;
\draw(.101,-.025)--(.132,-.021);
\draw(.134,.036)--(.111,.034);
\filldraw[fill opacity=0.8,fill=gray!20,draw=none](.164,-.086)--(.139,-.111)--(.146,-.052)--cycle;
\filldraw[fill opacity=0.8,fill=gray!20,draw=none](.164,-.086)--(.146,-.052)--(.181,-.068)--cycle;
\draw(.146,-.052)--(.181,-.068);
\filldraw[fill opacity=0.8,fill=gray!20,draw=none](.134,-.073)--(.181,-.068)--(.22,-.012)--(.132,-.021)--cycle;
\draw(.134,-.073)--(.181,-.068);
\draw(.22,-.012)--(.132,-.021);
\filldraw[fill opacity=0.8,fill=gray!20,draw=none](.209,-.004)--(.21,-.007)--(.208,-.017)--(.199,-.012)--cycle;
\draw(.209,-.004)--(.21,-.007);
\filldraw[fill opacity=0.8,fill=gray!20,draw=none](.078,-.011)--(.065,-.029)--(.101,-.025)--(.104,-.004)--cycle;
\draw(.065,-.029)--(.101,-.025);
\filldraw[fill opacity=0.8,fill=gray!20,draw=none](.096,-.009)--(.096,-.013)--(.141,-.014)--(.21,-.007)--(.205,.019)--cycle;
\draw(.21,-.007)--(.205,.019);
\filldraw[fill opacity=0.5,fill=gray!20](.621,.122)--(.567,.111)--(.067,-.015)--(.131,-.002)--cycle;
\filldraw[fill opacity=0.8,fill=gray!20,draw=none](-.139,-.091)--(-.139,-.168)--(-.146,-.181)--(-.146,-.134)--cycle;
\draw(-.139,-.091)--(-.139,-.168)--(-.146,-.181)--(-.146,-.134);
\filldraw[fill opacity=0.8,fill=gray!20,draw=none](-.222,.011)--(-.22,.012)--(-.181,-.041)--(-.215,-.045)--cycle;
\draw(-.222,.011)--(-.22,.012);
\draw(-.181,-.041)--(-.215,-.045);
\filldraw[fill opacity=0.8,fill=gray!20,draw=none](-.164,-.062)--(-.177,-.069)--(-.217,-.045)--(-.181,-.041)--cycle;
\draw(-.217,-.045)--(-.181,-.041);
\filldraw[fill opacity=0.8,fill=gray!20,draw=none](-.206,-.107)--(-.21,-.099)--(-.139,-.091)--(-.075,-.127)--cycle;
\draw(-.206,-.107)--(-.21,-.099)--(-.139,-.091);
\filldraw[fill opacity=0.8,fill=gray!20,draw=none](-.206,-.107)--(-.075,-.127)--(-.189,-.139)--cycle;
\draw(-.075,-.127)--(-.189,-.139)--(-.206,-.107);
\filldraw[fill opacity=0.8,fill=gray!20,draw=none](-.139,-.091)--(-.146,-.134)--(-.146,-.083)--cycle;
\draw(-.146,-.134)--(-.146,-.083);
\filldraw[fill opacity=0.8,fill=gray!20,draw=none](-.139,-.091)--(-.146,-.083)--(-.146,-.057)--cycle;
\draw(-.146,-.083)--(-.146,-.057);
\filldraw[fill opacity=0.8,fill=gray!20,draw=none](-.146,-.134)--(-.075,-.127)--(-.035,-.147)--(-.131,-.157)--cycle;
\draw(-.146,-.134)--(-.075,-.127);
\draw(-.035,-.147)--(-.131,-.157);
\filldraw[fill opacity=0.8,fill=gray!20,draw=none](-.025,-.148)--(0,-.143)--(.018,-.151)--cycle;
\draw(0,-.143)--(.018,-.151);
\filldraw[fill opacity=0.8,fill=gray!20,draw=none](0,-.143)--(.075,-.137)--(-.025,-.148)--cycle;
\draw(.075,-.137)--(-.025,-.148);
\filldraw[fill opacity=0.8,fill=gray!20,draw=none](-.146,-.156)--(-.146,-.171)--(-.132,-.185)--(-.131,-.183)--(-.131,-.157)--cycle;
\draw(-.146,-.156)--(-.146,-.171);
\draw(-.131,-.183)--(-.131,-.157);
\filldraw[fill opacity=0.8,fill=gray!20,draw=none](-.146,-.134)--(-.146,-.156)--(-.131,-.157)--cycle;
\draw(-.146,-.134)--(-.146,-.156);
\filldraw[fill opacity=0.8,fill=gray!20,draw=none](-.146,-.134)--(-.131,-.157)--(-.154,-.16)--cycle;
\draw(-.131,-.157)--(-.154,-.16);
\filldraw[fill opacity=0.8,fill=gray!20,draw=none](-.146,-.119)--(-.146,-.134)--(-.131,-.157)--(-.131,-.129)--cycle;
\draw(-.146,-.119)--(-.146,-.134);
\draw(-.131,-.157)--(-.131,-.129);
\filldraw[fill opacity=0.8,fill=gray!20,draw=none](-.131,-.157)--(-.131,-.183)--(-.096,-.156)--cycle;
\draw(-.131,-.157)--(-.131,-.183);
\filldraw[fill opacity=0.8,fill=gray!20,draw=none](-.154,-.16)--(-.035,-.147)--(-.025,-.148)--(-.096,-.156)--cycle;
\draw(-.154,-.16)--(-.035,-.147);
\draw(-.025,-.148)--(-.096,-.156);
\filldraw[fill opacity=0.8,fill=gray!20,draw=none](-.131,-.129)--(-.131,-.157)--(-.096,-.156)--(-.096,-.141)--cycle;
\draw(-.131,-.129)--(-.131,-.157);
\draw(-.096,-.156)--(-.096,-.141);
\filldraw[fill opacity=0.8,fill=gray!20,draw=none](-.131,-.087)--(-.131,-.129)--(-.096,-.141)--(-.096,-.104)--cycle;
\draw(-.131,-.087)--(-.131,-.129);
\draw(-.096,-.141)--(-.096,-.104);
\filldraw[fill opacity=0.8,fill=gray!20,draw=none](-.164,-.161)--(-.154,-.16)--(-.096,-.156)--(-.139,-.161)--cycle;
\draw(-.096,-.156)--(-.139,-.161)--(-.164,-.161)--(-.154,-.16);
\filldraw[fill opacity=0.8,fill=gray!20,draw=none](-.096,-.141)--(-.096,-.156)--(-.075,-.146)--cycle;
\draw(-.096,-.141)--(-.096,-.156);
\filldraw[fill opacity=0.8,fill=gray!20,draw=none](-.096,-.104)--(-.096,-.141)--(-.075,-.146)--(-.046,-.131)--cycle;
\draw(-.096,-.104)--(-.096,-.141);
\filldraw[fill opacity=0.8,fill=gray!20](-.139,-.161)--(.189,-.125)--(.21,-.103)--(-.117,-.139)--cycle;
\filldraw[fill opacity=0.5,fill=gray!20](.21,4.121)--(.16,3.911)--(.535,3.782)--(.632,3.976)--cycle;
\filldraw[fill opacity=0.5,fill=gray!20](.696,3.909)--(.21,4.121)--(.632,3.976)--(1.118,3.764)--cycle;
\filldraw[fill opacity=0.5,fill=gray!20](.284,4.188)--(.21,4.121)--(.632,3.977)--(.726,4.036)--cycle;
\filldraw[fill opacity=0.8,fill=gray!20,draw=none](-.229,.011)--(-.222,.011)--(-.215,-.045)--(-.224,-.046)--cycle;
\draw(-.215,-.045)--(-.224,-.046)--(-.229,.011)--(-.222,.011);
\filldraw[fill opacity=0.5,fill=gray!20](.16,3.911)--(.342,3.831)--(.717,3.703)--(.535,3.782)--cycle;
\filldraw[fill opacity=0.8,fill=gray!20,draw=none](-.189,-.139)--(-.146,-.134)--(-.154,-.16)--(-.164,-.161)--cycle;
\draw(-.154,-.16)--(-.164,-.161)--(-.189,-.139)--(-.146,-.134);
\filldraw[fill opacity=0.8,fill=gray!20,draw=none](-.146,-.083)--(-.146,-.119)--(-.131,-.129)--(-.131,-.087)--cycle;
\draw(-.146,-.083)--(-.146,-.119);
\draw(-.131,-.129)--(-.131,-.087);
\filldraw[fill opacity=0.8,fill=gray!20,draw=none](-.146,-.057)--(-.146,-.083)--(-.131,-.087)--(-.131,-.053)--cycle;
\draw(-.146,-.057)--(-.146,-.083);
\draw(-.131,-.087)--(-.131,-.053);
\filldraw[fill opacity=0.8,fill=gray!20,draw=none](-.131,-.053)--(-.131,-.087)--(-.096,-.104)--(-.096,-.079)--cycle;
\draw(-.131,-.053)--(-.131,-.087);
\draw(-.096,-.104)--(-.096,-.079);
\filldraw[fill opacity=0.8,fill=gray!20,draw=none](-.103,-.094)--(-.225,-.04)--(-.224,-.046)--(-.21,-.099)--(-.189,-.139)--(-.164,-.161)--(-.139,-.161)--(-.117,-.139)--(-.103,-.099)--cycle;
\draw(-.225,-.04)--(-.224,-.046)--(-.21,-.099)--(-.189,-.139)--(-.164,-.161)--(-.139,-.161)--(-.117,-.139)--(-.103,-.099)--(-.103,-.094);
\filldraw[fill opacity=0.8,fill=gray!20,draw=none](-.146,-.057)--(-.131,-.053)--(-.131,-.014)--cycle;
\draw(-.131,-.053)--(-.131,-.014);
\filldraw[fill opacity=0.8,fill=gray!20,draw=none](-.103,-.094)--(-.102,-.089)--(-.195,.011)--(-.228,.025)--(-.229,.011)--(-.225,-.04)--cycle;
\draw(-.103,-.094)--(-.102,-.089);
\draw(-.228,.025)--(-.229,.011)--(-.225,-.04);
\filldraw[fill opacity=0.8,fill=gray!20,draw=none](.143,-.122)--(.14,-.186)--(.146,-.176)--(.146,-.13)--cycle;
\draw(.14,-.186)--(.146,-.176)--(.146,-.13);
\filldraw[fill opacity=0.5,fill=gray!20](1.933,2.694)--(1.961,2.749)--(2.023,2.238)--(1.993,2.206)--cycle;
\filldraw[fill opacity=0.5,fill=gray!20,draw=none](-.378,.054)--(-.513,.125)--(-.621,.129)--(-.567,.06)--cycle;
\draw(-.513,.125)--(-.621,.129)--(-.567,.06)--(-.378,.054);
\filldraw[fill opacity=0.5,fill=gray!20](1.777,3.211)--(1.771,3.276)--(1.96,2.799)--(1.961,2.749)--cycle;
\filldraw[fill opacity=0.5,fill=gray!20](1.408,3.09)--(1.59,3.011)--(1.745,2.619)--(1.563,2.698)--cycle;
\filldraw[fill opacity=0.8,fill=gray!20,draw=none](-.195,.011)--(-.226,.044)--(-.228,.025)--cycle;
\draw(-.226,.044)--(-.228,.025);
\filldraw[fill opacity=0.8,fill=gray!20,draw=none](-.065,-.043)--(-.131,-.014)--(-.096,-.079)--(-.06,-.094)--cycle;
\draw(-.065,-.043)--(-.131,-.014);
\draw(-.096,-.079)--(-.06,-.094);
\filldraw[fill opacity=0.8,fill=gray!20,draw=none](-.102,-.089)--(-.098,-.046)--(-.099,-.036)--(-.226,.046)--(-.226,.044)--cycle;
\draw(-.102,-.089)--(-.098,-.046)--(-.099,-.036);
\draw(-.226,.046)--(-.226,.044);
\filldraw[fill opacity=0.5,fill=gray!20,draw=none](-.336,.033)--(-.213,.037)--(-.226,.046)--(-.37,.05)--cycle;
\filldraw[fill opacity=0.8,fill=gray!20,draw=none](-.146,-.171)--(-.146,-.181)--(-.131,-.193)--(-.131,-.186)--cycle;
\draw(-.146,-.171)--(-.146,-.181)--(-.131,-.193)--(-.131,-.186);
\filldraw[fill opacity=0.8,fill=gray!20,draw=none](.192,.016)--(.188,.016)--(.188,.017)--cycle;
\filldraw[fill opacity=0.5,fill=gray!20,draw=none](.147,.027)--(.146,.025)--(.171,.011)--(.192,.016)--cycle;
\draw(.171,.011)--(.192,.016);
\filldraw[fill opacity=0.8,fill=gray!20,draw=none](.188,.017)--(.147,.027)--(.146,.057)--(.181,.041)--cycle;
\draw(.146,.057)--(.181,.041);
\filldraw[fill opacity=0.8,fill=gray!20,draw=none](.139,-.111)--(.139,-.189)--(.14,-.186)--(.143,-.122)--cycle;
\draw(.139,-.111)--(.139,-.189)--(.14,-.186);
\filldraw[fill opacity=0.8,fill=gray!20,draw=none](-.096,-.079)--(-.096,-.104)--(-.086,-.11)--cycle;
\draw(-.096,-.079)--(-.096,-.104);
\filldraw[fill opacity=0.8,fill=gray!20,draw=none](-.055,-.11)--(-.096,-.079)--(-.046,-.131)--cycle;
\filldraw[fill opacity=0.8,fill=gray!20,draw=none](-.055,-.11)--(-.06,-.094)--(-.096,-.079)--cycle;
\draw(-.06,-.094)--(-.096,-.079);
\filldraw[fill opacity=0.8,fill=gray!20,draw=none](-.046,-.016)--(-.046,-.131)--(.01,-.087)--(.01,-.015)--cycle;
\draw(-.046,-.016)--(-.046,-.131);
\draw(.01,-.087)--(.01,-.015);
\filldraw[fill opacity=0.8,fill=gray!20](-.117,-.139)--(.21,-.103)--(.224,-.063)--(-.103,-.099)--cycle;
\filldraw[fill opacity=0.8,fill=gray!20,draw=none](.139,.091)--(.139,.004)--(.146,.007)--(.146,.083)--cycle;
\draw(.139,.091)--(.139,.004);
\draw(.146,.007)--(.146,.083);
\filldraw[fill opacity=0.5,fill=gray!20,draw=none](-.169,.006)--(-.164,.006)--(-.213,.037)--(-.224,.037)--cycle;
\draw(-.169,.006)--(-.164,.006);
\filldraw[fill opacity=0.5,fill=gray!20,draw=none](.133,.003)--(.169,.012)--(.146,.025)--cycle;
\filldraw[fill opacity=0.8,fill=gray!20,draw=none](.111,-.153)--(.111,-.2)--(.139,-.189)--(.139,-.133)--cycle;
\draw(.111,-.153)--(.111,-.2)--(.139,-.189)--(.139,-.133);
\filldraw[fill opacity=0.8,fill=gray!20,draw=none](.065,-.171)--(.065,-.207)--(.106,-.2)--(.111,-.191)--(.111,-.175)--cycle;
\draw(.065,-.171)--(.065,-.207)--(.106,-.2);
\draw(.111,-.191)--(.111,-.175);
\filldraw[fill opacity=0.8,fill=gray!20,draw=none](.065,-.171)--(.111,-.175)--(.111,-.153)--cycle;
\draw(.111,-.175)--(.111,-.153);
\filldraw[fill opacity=0.8,fill=gray!20,draw=none](.082,-.179)--(.018,-.151)--(.075,-.137)--(.132,-.162)--cycle;
\draw(.075,-.137)--(.132,-.162)--(.082,-.179)--(.018,-.151);
\filldraw[fill opacity=0.8,fill=gray!20,draw=none](.111,-.153)--(.139,-.133)--(.139,-.111)--cycle;
\draw(.139,-.133)--(.139,-.111);
\filldraw[fill opacity=0.8,fill=gray!20,draw=none](.132,-.162)--(.075,-.137)--(.139,-.111)--(.174,-.127)--cycle;
\draw(.139,-.111)--(.174,-.127)--(.132,-.162)--(.075,-.137);
\filldraw[fill opacity=0.5,fill=gray!20](1.745,1.74)--(1.933,1.706)--(1.758,1.226)--(1.59,1.314)--cycle;
\filldraw[fill opacity=0.5,fill=gray!20](1.59,1.314)--(1.758,1.226)--(1.48,.8)--(1.343,.935)--cycle;
\filldraw[fill opacity=0.5,fill=gray!20,draw=none](-.169,.006)--(-.224,.037)--(-.518,.027)--(-.503,.017)--cycle;
\draw(-.518,.027)--(-.503,.017)--(-.169,.006);
\filldraw[fill opacity=0.8,fill=gray!20,draw=none](.197,.02)--(.226,.027)--(.224,.046)--(.181,.041)--cycle;
\draw(.226,.027)--(.224,.046)--(.181,.041);
\filldraw[fill opacity=0.8,fill=gray!20,draw=none](.164,.062)--(.159,.093)--(.139,.091)--cycle;
\draw(.159,.093)--(.139,.091);
\filldraw[fill opacity=0.8,fill=gray!20,draw=none](.164,.062)--(.181,.041)--(.159,.093)--cycle;
\filldraw[fill opacity=0.8,fill=gray!20,draw=none](.203,.032)--(.146,.057)--(.139,.091)--(.174,.075)--cycle;
\draw(.139,.091)--(.174,.075)--(.203,.032)--(.146,.057);
\filldraw[fill opacity=0.8,fill=gray!20,draw=none](.01,-.186)--(.01,-.201)--(.065,-.2)--(.065,-.171)--cycle;
\draw(.01,-.186)--(.01,-.201);
\draw(.065,-.2)--(.065,-.171);
\filldraw[fill opacity=0.8,fill=gray!20,draw=none](.01,-.164)--(.036,-.179)--(.065,-.171)--cycle;
\filldraw[fill opacity=0.8,fill=gray!20,draw=none](.032,-.173)--(-.025,-.148)--(.018,-.151)--(.082,-.179)--cycle;
\draw(.018,-.151)--(.082,-.179)--(.032,-.173)--(-.025,-.148);
\filldraw[fill opacity=0.5,fill=gray!20](.972,4.065)--(.899,4.081)--(1.299,3.79)--(1.375,3.772)--cycle;
\filldraw[fill opacity=0.5,fill=gray!20](1.118,.457)--(1.107,.407)--(.665,.159)--(.696,.221)--cycle;
\filldraw[fill opacity=0.5,fill=gray!20](-.567,.06)--(-.621,.129)--(-1.078,.286)--(-1.032,.22)--cycle;
\filldraw[fill opacity=0.5,fill=gray!20](1.798,2.185)--(1.993,2.206)--(1.933,1.706)--(1.745,1.74)--cycle;
\filldraw[fill opacity=0.8,fill=gray!20,draw=none](-.132,-.185)--(-.131,-.186)--(-.131,-.183)--cycle;
\draw(-.131,-.186)--(-.131,-.183);
\filldraw[fill opacity=0.8,fill=gray!20,draw=none](.164,-.086)--(.168,-.094)--(.139,-.111)--cycle;
\filldraw[fill opacity=0.8,fill=gray!20,draw=none](.168,-.094)--(.18,-.117)--(.174,-.127)--(.139,-.111)--cycle;
\draw(.18,-.117)--(.174,-.127)--(.139,-.111);
\filldraw[fill opacity=0.8,fill=gray!20,draw=none](.111,-.101)--(.111,-.153)--(.139,-.111)--cycle;
\draw(.111,-.101)--(.111,-.153);
\filldraw[fill opacity=0.8,fill=gray!20,draw=none](.111,-.076)--(.111,-.101)--(.139,-.111)--cycle;
\draw(.111,-.076)--(.111,-.101);
\filldraw[fill opacity=0.5,fill=gray!20](1.47,3.672)--(1.433,3.732)--(1.739,3.329)--(1.771,3.276)--cycle;
\filldraw[fill opacity=0.8,fill=gray!20,draw=none](-.131,-.186)--(-.131,-.193)--(-.096,-.203)--(-.096,-.193)--cycle;
\draw(-.131,-.186)--(-.131,-.193)--(-.096,-.203)--(-.096,-.193);
\filldraw[fill opacity=0.8,fill=gray!20,draw=none](-.131,-.183)--(-.131,-.186)--(-.096,-.193)--(-.096,-.156)--cycle;
\draw(-.131,-.183)--(-.131,-.186);
\draw(-.096,-.193)--(-.096,-.156);
\filldraw[fill opacity=0.8,fill=gray!20,draw=none](.217,-.064)--(.224,-.063)--(.225,-.058)--cycle;
\draw(.217,-.064)--(.224,-.063)--(.225,-.058);
\filldraw[fill opacity=0.8,fill=gray!20,draw=none](.181,-.068)--(.217,-.064)--(.225,-.058)--(.229,-.011)--(.22,-.012)--cycle;
\draw(.181,-.068)--(.217,-.064);
\draw(.225,-.058)--(.229,-.011)--(.22,-.012);
\filldraw[fill opacity=0.5,fill=gray!20](.463,.5)--(.646,.42)--(.243,.318)--(.061,.398)--cycle;
\filldraw[fill opacity=0.8,fill=gray!20,draw=none](.201,.044)--(.224,.046)--(.21,.099)--(.196,.097)--cycle;
\draw(.201,.044)--(.224,.046)--(.21,.099)--(.196,.097);
\filldraw[fill opacity=0.8,fill=gray!20,draw=none](.168,-.094)--(.164,-.086)--(.181,-.068)--(.199,-.076)--cycle;
\draw(.181,-.068)--(.199,-.076);
\filldraw[fill opacity=0.5,fill=gray!20](.359,4.228)--(.284,4.188)--(.726,4.036)--(.816,4.071)--cycle;
\filldraw[fill opacity=0.5,fill=gray!20](1.343,.935)--(1.48,.8)--(1.118,.457)--(1.021,.63)--cycle;
\filldraw[fill opacity=0.8,fill=gray!20,draw=none](.01,-.112)--(.01,-.164)--(.065,-.171)--(.065,-.102)--cycle;
\draw(.01,-.112)--(.01,-.164);
\draw(.065,-.171)--(.065,-.102);
\filldraw[fill opacity=0.8,fill=gray!20,draw=none](-.046,-.131)--(.01,-.164)--(.01,-.112)--cycle;
\draw(.01,-.164)--(.01,-.112);
\filldraw[fill opacity=0.8,fill=gray!20,draw=none](-.002,-.152)--(-.025,-.148)--(.032,-.173)--cycle;
\draw(-.025,-.148)--(.032,-.173)--(-.002,-.152);
\filldraw[fill opacity=0.8,fill=gray!20,draw=none](-.002,-.152)--(-.011,-.147)--(-.046,-.131)--(-.025,-.148)--cycle;
\draw(-.002,-.152)--(-.011,-.147)--(-.046,-.131);
\filldraw[fill opacity=0.8,fill=gray!20,draw=none](.151,.104)--(.159,.093)--(.154,.108)--cycle;
\filldraw[fill opacity=0.5,fill=gray!20,draw=none](.199,.02)--(.204,.019)--(.23,.026)--(.226,.027)--cycle;
\draw(.204,.019)--(.23,.026);
\filldraw[fill opacity=0.5,fill=gray!20,draw=none](.188,.017)--(.192,.016)--(.204,.019)--(.199,.02)--cycle;
\draw(.192,.016)--(.204,.019);
\filldraw[fill opacity=0.8,fill=gray!20,draw=none](.188,.017)--(.197,.02)--(.181,.041)--cycle;
\filldraw[fill opacity=0.8,fill=gray!20,draw=none](.133,.003)--(.188,.017)--(.181,.041)--(.134,.036)--cycle;
\draw(.181,.041)--(.134,.036);
\filldraw[fill opacity=0.8,fill=gray!20,draw=none](.205,.019)--(.192,.016)--(.188,.017)--(.181,.041)--(.203,.032)--cycle;
\draw(.181,.041)--(.203,.032)--(.205,.019);
\filldraw[fill opacity=0.8,fill=gray!20,draw=none](.192,.016)--(.193,.017)--(.193,.016)--cycle;
\filldraw[fill opacity=0.5,fill=gray!20,draw=none](.171,.011)--(.169,.012)--(.064,-.015)--(.067,-.015)--cycle;
\draw(.064,-.015)--(.067,-.015)--(.171,.011);
\filldraw[fill opacity=0.8,fill=gray!20,draw=none](.134,.036)--(.167,.04)--(.164,.062)--(.139,.091)--cycle;
\draw(.134,.036)--(.167,.04);
\filldraw[fill opacity=0.8,fill=gray!20,draw=none](.111,.034)--(.134,.036)--(.139,.091)--cycle;
\draw(.111,.034)--(.134,.036);
\filldraw[fill opacity=0.8,fill=gray!20,draw=none](.134,.097)--(.166,.081)--(.174,.075)--(.139,.091)--cycle;
\draw(.166,.081)--(.174,.075)--(.139,.091);
\filldraw[fill opacity=0.8,fill=gray!20,draw=none](.164,.062)--(.167,.04)--(.181,.041)--cycle;
\draw(.167,.04)--(.181,.041);
\filldraw[fill opacity=0.8,fill=gray!20,draw=none](.01,-.026)--(.01,-.087)--(.065,-.029)--(.065,-.015)--cycle;
\draw(.01,-.026)--(.01,-.087);
\draw(.065,-.029)--(.065,-.015);
\filldraw[fill opacity=0.8,fill=gray!20,draw=none](-.103,-.099)--(.134,-.073)--(.132,-.021)--(-.098,-.046)--cycle;
\draw(.132,-.021)--(-.098,-.046)--(-.103,-.099)--(.134,-.073);
\filldraw[fill opacity=0.8,fill=gray!20,draw=none](.122,.103)--(.111,.111)--(.111,.034)--cycle;
\draw(.111,.111)--(.111,.034);
\filldraw[fill opacity=0.8,fill=gray!20,draw=none](.129,.1)--(.134,.097)--(.135,.095)--cycle;
\filldraw[fill opacity=0.8,fill=gray!20,draw=none](.078,-.011)--(.104,-.004)--(.111,.034)--cycle;
\filldraw[fill opacity=0.8,fill=gray!20,draw=none](.065,.01)--(.065,-.015)--(.078,-.011)--(.111,.034)--cycle;
\draw(.065,.01)--(.065,-.015);
\filldraw[fill opacity=0.8,fill=gray!20,draw=none](.134,.097)--(.129,.1)--(.111,.111)--(.128,.103)--cycle;
\draw(.111,.111)--(.128,.103);
\filldraw[fill opacity=0.8,fill=gray!20,draw=none](.166,.081)--(.134,.097)--(.128,.103)--(.132,.102)--cycle;
\draw(.128,.103)--(.132,.102)--(.166,.081);
\filldraw[fill opacity=0.8,fill=gray!20,draw=none](.01,-.012)--(.01,-.026)--(.065,-.015)--cycle;
\draw(.01,-.012)--(.01,-.026);
\filldraw[fill opacity=0.8,fill=gray!20,draw=none](.023,-.013)--(.03,-.032)--(.065,-.029)--(.05,-.014)--cycle;
\draw(.03,-.032)--(.065,-.029);
\filldraw[fill opacity=0.8,fill=gray!20,draw=none](-.046,-.007)--(-.046,-.016)--(.01,-.015)--(.01,-.012)--cycle;
\draw(-.046,-.007)--(-.046,-.016);
\draw(.01,-.015)--(.01,-.012);
\filldraw[fill opacity=0.5,fill=gray!20,draw=none](-.043,-.011)--(-.038,-.012)--(.067,-.015)--(.064,-.015)--cycle;
\draw(-.038,-.012)--(.067,-.015)--(.064,-.015);
\filldraw[fill opacity=0.8,fill=gray!20,draw=none](.023,-.013)--(.05,-.014)--(.01,.023)--cycle;
\filldraw[fill opacity=0.8,fill=gray!20,draw=none](0,.053)--(-.046,.071)--(-.046,-.007)--(.01,-.012)--(.01,.023)--cycle;
\draw(-.046,.071)--(-.046,-.007);
\draw(.01,-.012)--(.01,.023);
\filldraw[fill opacity=0.5,fill=gray!20,draw=none](-.046,-.007)--(-.096,-.002)--(-.077,-.005)--cycle;
\filldraw[fill opacity=0.5,fill=gray!20,draw=none](-.104,-.003)--(-.074,-.01)--(-.043,-.011)--(-.077,-.005)--cycle;
\filldraw[fill opacity=0.5,fill=gray!20,draw=none](-.019,-.011)--(-.006,-.012)--(.05,-.014)--cycle;
\filldraw[fill opacity=0.5,fill=gray!20,draw=none](-.046,-.007)--(-.077,-.005)--(-.043,-.011)--(-.006,-.012)--cycle;
\filldraw[fill opacity=0.8,fill=gray!20,draw=none](-.091,-.004)--(-.063,-.01)--(-.06,.015)--(-.084,.025)--cycle;
\draw(-.06,.015)--(-.084,.025);
\filldraw[fill opacity=0.5,fill=gray!20,draw=none](-.046,-.007)--(-.019,-.011)--(.01,-.012)--cycle;
\filldraw[fill opacity=0.8,fill=gray!20,draw=none](-.072,-.01)--(.023,-.013)--(.01,.023)--(-.06,.015)--cycle;
\draw(.01,.023)--(-.06,.015);
\filldraw[fill opacity=0.8,fill=gray!20,draw=none](0,.053)--(.01,.023)--(.01,.048)--cycle;
\draw(.01,.023)--(.01,.048);
\filldraw[fill opacity=0.8,fill=gray!20,draw=none](-.118,.002)--(-.091,-.004)--(-.084,.025)--(-.096,.031)--cycle;
\draw(-.084,.025)--(-.096,.031);
\filldraw[fill opacity=0.8,fill=gray!20,draw=none](-.06,.015)--(.01,.023)--(-.046,.071)--cycle;
\draw(-.06,.015)--(.01,.023);
\filldraw[fill opacity=0.8,fill=gray!20,draw=none](-.042,-.008)--(.082,-.012)--(.205,.019)--(.203,.032)--(.174,.075)--(.132,.102)--(.082,.107)--(.032,.091)--(-.011,.055)--(-.039,.006)--cycle;
\draw(.205,.019)--(.203,.032)--(.174,.075)--(.132,.102)--(.082,.107)--(.032,.091)--(-.011,.055)--(-.039,.006)--(-.042,-.008);
\filldraw[fill opacity=0.5,fill=gray!20](.665,.159)--(.621,.122)--(.131,-.002)--(.19,.039)--cycle;
\filldraw[fill opacity=0.5,fill=gray!20,draw=none](-.336,.033)--(-.37,.05)--(-.558,.055)--(-.518,.027)--cycle;
\draw(-.558,.055)--(-.518,.027);
\filldraw[fill opacity=0.8,fill=gray!20,draw=none](.181,-.068)--(.196,-.014)--(.208,-.019)--cycle;
\draw(.196,-.014)--(.208,-.019);
\filldraw[fill opacity=0.8,fill=gray!20,draw=none](.203,-.077)--(.181,-.068)--(.208,-.019)--(.213,-.021)--cycle;
\draw(.208,-.019)--(.213,-.021)--(.203,-.077)--(.181,-.068);
\filldraw[fill opacity=0.5,fill=gray!20](-1.021,.673)--(-.838,.593)--(-1.161,.828)--(-1.343,.907)--cycle;
\filldraw[fill opacity=0.5,fill=gray!20](-1.118,.479)--(-1.021,.673)--(-1.343,.907)--(-1.48,.742)--cycle;
\filldraw[fill opacity=0.8,fill=gray!20,draw=none](.18,-.117)--(.168,-.094)--(.199,-.076)--(.203,-.077)--cycle;
\draw(.199,-.076)--(.203,-.077)--(.18,-.117);
\filldraw[fill opacity=0.8,fill=gray!20,draw=none](.106,-.2)--(.111,-.2)--(.111,-.191)--cycle;
\draw(.106,-.2)--(.111,-.2)--(.111,-.191);
\filldraw[fill opacity=0.5,fill=gray!20](-.632,.266)--(-1.118,.479)--(-1.48,.742)--(-.994,.53)--cycle;
\filldraw[fill opacity=0.5,fill=gray!20](-1.107,.373)--(-1.118,.478)--(-1.48,.742)--(-1.486,.649)--cycle;
\filldraw[fill opacity=0.8,fill=gray!20,draw=none](.196,-.01)--(.212,-.02)--(.213,-.021)--(.196,-.014)--cycle;
\draw(.212,-.02)--(.213,-.021)--(.196,-.014);
\filldraw[fill opacity=0.8,fill=gray!20,draw=none](.065,-.102)--(.065,-.171)--(.111,-.153)--(.111,-.101)--cycle;
\draw(.065,-.102)--(.065,-.171);
\draw(.111,-.153)--(.111,-.101);
\filldraw[fill opacity=0.8,fill=gray!20,draw=none](-.075,-.146)--(-.096,-.156)--(-.096,-.203)--(-.046,-.209)--(-.046,-.153)--cycle;
\draw(-.096,-.156)--(-.096,-.203)--(-.046,-.209)--(-.046,-.153);
\filldraw[fill opacity=0.8,fill=gray!20,draw=none](.065,-.067)--(.065,-.102)--(.111,-.101)--(.111,-.076)--cycle;
\draw(.065,-.067)--(.065,-.102);
\draw(.111,-.101)--(.111,-.076);
\filldraw[fill opacity=0.8,fill=gray!20,draw=none](.065,-.029)--(.065,-.067)--(.111,-.076)--cycle;
\draw(.065,-.029)--(.065,-.067);
\filldraw[fill opacity=0.8,fill=gray!20,draw=none](-.164,.006)--(-.123,-.021)--(-.1,-.031)--(-.102,-.009)--cycle;
\draw(-.1,-.031)--(-.102,-.009);
\filldraw[fill opacity=0.5,fill=gray!20,draw=none](-.112,-.009)--(-.097,-.01)--(-.164,.006)--(-.169,.006)--cycle;
\draw(-.112,-.009)--(-.097,-.01);
\draw(-.164,.006)--(-.169,.006);
\filldraw[fill opacity=0.8,fill=gray!20,draw=none](.01,-.201)--(.01,-.21)--(.065,-.207)--(.065,-.2)--cycle;
\draw(.01,-.201)--(.01,-.21)--(.065,-.207)--(.065,-.2);
\filldraw[fill opacity=0.5,fill=gray!20,draw=none](-.112,-.009)--(-.169,.006)--(-.503,.017)--(-.433,.002)--cycle;
\draw(-.169,.006)--(-.503,.017)--(-.433,.002)--(-.112,-.009);
\filldraw[fill opacity=0.8,fill=gray!20,draw=none](.01,-.087)--(.01,-.112)--(.065,-.102)--(.065,-.067)--cycle;
\draw(.01,-.087)--(.01,-.112);
\draw(.065,-.102)--(.065,-.067);
\filldraw[fill opacity=0.8,fill=gray!20,draw=none](-.046,-.131)--(.01,-.112)--(.01,-.087)--cycle;
\draw(.01,-.112)--(.01,-.087);
\filldraw[fill opacity=0.8,fill=gray!20,draw=none](-.04,-.098)--(-.039,-.104)--(-.011,-.147)--(.032,-.173)--(.082,-.179)--(.132,-.162)--(.174,-.127)--(.203,-.077)--(.204,-.072)--cycle;
\draw(-.04,-.098)--(-.039,-.104)--(-.011,-.147)--(.032,-.173)--(.082,-.179)--(.132,-.162)--(.174,-.127)--(.203,-.077)--(.204,-.072);
\filldraw[fill opacity=0.8,fill=gray!20,draw=none](.21,-.007)--(.212,-.02)--(.208,-.017)--cycle;
\draw(.21,-.007)--(.212,-.02);
\filldraw[fill opacity=0.8,fill=gray!20,draw=none](.01,-.087)--(.065,-.067)--(.065,-.029)--cycle;
\draw(.065,-.067)--(.065,-.029);
\filldraw[fill opacity=0.8,fill=gray!20,draw=none](-.063,-.012)--(-.095,-.017)--(-.065,-.037)--cycle;
\filldraw[fill opacity=0.8,fill=gray!20,draw=none](.023,-.013)--(-.074,-.01)--(-.065,-.043)--(.03,-.032)--cycle;
\draw(-.065,-.043)--(.03,-.032);
\filldraw[fill opacity=0.8,fill=gray!20,draw=none](.022,-.027)--(-.048,-.045)--(-.049,-.05)--(-.04,-.098)--(.204,-.072)--(.213,-.021)--(.21,-.007)--cycle;
\draw(-.048,-.045)--(-.049,-.05)--(-.04,-.098);
\draw(.204,-.072)--(.213,-.021)--(.21,-.007);
\filldraw[fill opacity=0.8,fill=gray!20,draw=none](-.075,-.146)--(-.046,-.153)--(-.046,-.131)--cycle;
\draw(-.046,-.153)--(-.046,-.131);
\filldraw[fill opacity=0.5,fill=gray!20](1.745,2.619)--(1.933,2.694)--(1.993,2.206)--(1.798,2.185)--cycle;
\filldraw[fill opacity=0.8,fill=gray!20,draw=none](-.046,-.153)--(-.046,-.209)--(.01,-.21)--(.01,-.164)--cycle;
\draw(-.046,-.153)--(-.046,-.209)--(.01,-.21)--(.01,-.164);
\filldraw[fill opacity=0.5,fill=gray!20](1.758,3.135)--(1.777,3.211)--(1.961,2.749)--(1.933,2.694)--cycle;
\filldraw[fill opacity=0.8,fill=gray!20,draw=none](.01,-.164)--(.01,-.186)--(.036,-.179)--cycle;
\draw(.01,-.164)--(.01,-.186);
\filldraw[fill opacity=0.8,fill=gray!20,draw=none](-.046,-.131)--(-.046,-.153)--(.01,-.164)--cycle;
\draw(-.046,-.131)--(-.046,-.153);
\filldraw[fill opacity=0.8,fill=gray!20,draw=none](-.043,-.102)--(-.06,-.094)--(-.046,-.131)--cycle;
\draw(-.043,-.102)--(-.06,-.094);
\filldraw[fill opacity=0.8,fill=gray!20,draw=none](-.039,-.104)--(-.043,-.102)--(-.046,-.131)--(-.011,-.147)--cycle;
\draw(-.046,-.131)--(-.011,-.147)--(-.039,-.104)--(-.043,-.102);
\filldraw[fill opacity=0.8,fill=gray!20,draw=none](-.017,.045)--(-.096,.031)--(-.039,.006)--cycle;
\draw(-.096,.031)--(-.039,.006)--(-.017,.045);
\filldraw[fill opacity=0.8,fill=gray!20,draw=none](-.055,.038)--(-.046,.071)--(-.096,.031)--cycle;
\filldraw[fill opacity=0.8,fill=gray!20,draw=none](.132,.102)--(.111,.111)--(.077,.109)--(.082,.107)--cycle;
\draw(.077,.109)--(.082,.107)--(.132,.102)--(.111,.111);
\filldraw[fill opacity=0.5,fill=gray!20](1.161,3.415)--(1.343,3.335)--(1.59,3.011)--(1.408,3.09)--cycle;
\filldraw[fill opacity=0.8,fill=gray!20,draw=none](-.141,.001)--(-.102,-.009)--(-.103,.01)--(-.117,.063)--(-.139,.103)--(-.141,.105)--cycle;
\draw(-.102,-.009)--(-.103,.01)--(-.117,.063)--(-.139,.103)--(-.141,.105);
\filldraw[fill opacity=0.5,fill=gray!20](-.503,.017)--(-.567,.06)--(-1.032,.22)--(-.972,.178)--cycle;
\filldraw[fill opacity=0.8,fill=gray!20,draw=none](-.074,-.01)--(-.072,-.01)--(-.06,.015)--(-.08,.013)--cycle;
\draw(-.06,.015)--(-.08,.013);
\filldraw[fill opacity=0.8,fill=gray!20,draw=none](-.063,-.01)--(-.058,-.011)--(-.042,-.008)--(-.039,.006)--(-.06,.015)--cycle;
\draw(-.042,-.008)--(-.039,.006)--(-.06,.015);
\filldraw[fill opacity=0.8,fill=gray!20,draw=none](-.102,-.009)--(-.074,-.01)--(-.08,.013)--(-.103,.01)--cycle;
\draw(-.08,.013)--(-.103,.01)--(-.102,-.009);
\filldraw[fill opacity=0.5,fill=gray!20,draw=none](-.033,.004)--(.131,-.002)--(.067,-.015)--(-.097,-.01)--cycle;
\draw(-.033,.004)--(.131,-.002)--(.067,-.015)--(-.097,-.01);
\filldraw[fill opacity=0.8,fill=gray!20,draw=none](.096,-.013)--(.096,-.019)--(.141,-.014)--cycle;
\filldraw[fill opacity=0.5,fill=gray!20](1.032,4.023)--(.972,4.065)--(1.375,3.772)--(1.433,3.732)--cycle;
\filldraw[fill opacity=0.8,fill=gray!20,draw=none](-.07,.107)--(-.075,.092)--(-.046,.071)--cycle;
\filldraw[fill opacity=0.5,fill=gray!20](1.021,.63)--(1.118,.457)--(.696,.221)--(.646,.42)--cycle;
\filldraw[fill opacity=0.8,fill=gray!20,draw=none](.096,-.009)--(.059,-.018)--(.096,-.019)--cycle;
\filldraw[fill opacity=0.8,fill=gray!20,draw=none](.059,-.018)--(.022,-.027)--(.096,-.019)--cycle;
\filldraw[fill opacity=0.5,fill=gray!20](.433,4.241)--(.359,4.228)--(.816,4.071)--(.899,4.081)--cycle;
\filldraw[fill opacity=0.8,fill=gray!20,draw=none](-.103,.01)--(-.06,.015)--(-.046,.071)--(-.117,.063)--cycle;
\draw(-.046,.071)--(-.117,.063)--(-.103,.01)--(-.06,.015);
\filldraw[fill opacity=0.8,fill=gray!20,draw=none](-.013,.088)--(-.046,.071)--(-.036,.066)--cycle;
\draw(-.046,.071)--(-.036,.066);
\filldraw[fill opacity=0.8,fill=gray!20,draw=none](-.055,.038)--(-.017,.045)--(-.011,.055)--(-.046,.071)--cycle;
\draw(-.017,.045)--(-.011,.055)--(-.046,.071);
\filldraw[fill opacity=0.5,fill=gray!20](1.486,3.594)--(1.47,3.672)--(1.771,3.276)--(1.777,3.211)--cycle;
\filldraw[fill opacity=0.8,fill=gray!20,draw=none](-.123,-.021)--(-.099,-.036)--(-.1,-.031)--cycle;
\draw(-.099,-.036)--(-.1,-.031);
\filldraw[fill opacity=0.5,fill=gray!20](-1.078,.286)--(-1.107,.373)--(-1.486,.649)--(-1.47,.571)--cycle;
\filldraw[fill opacity=0.8,fill=gray!20,draw=none](-.054,-.048)--(-.065,-.043)--(-.06,-.094)--cycle;
\draw(-.054,-.048)--(-.065,-.043);
\filldraw[fill opacity=0.8,fill=gray!20,draw=none](-.04,-.098)--(-.049,-.05)--(-.054,-.048)--(-.06,-.094)--(-.043,-.102)--cycle;
\draw(-.04,-.098)--(-.049,-.05)--(-.054,-.048);
\draw(-.06,-.094)--(-.043,-.102);
\filldraw[fill opacity=0.5,fill=gray!20,draw=none](-.102,-.009)--(-.097,-.01)--(-.038,-.012)--(-.043,-.011)--cycle;
\draw(-.097,-.01)--(-.038,-.012);
\filldraw[fill opacity=0.8,fill=gray!20,draw=none](-.042,-.008)--(-.048,-.045)--(.082,-.012)--cycle;
\draw(-.042,-.008)--(-.048,-.045);
\filldraw[fill opacity=0.8,fill=gray!20,draw=none](-.002,.062)--(-.013,.088)--(-.036,.066)--(-.011,.055)--cycle;
\draw(-.036,.066)--(-.011,.055)--(-.002,.062);
\filldraw[fill opacity=0.5,fill=gray!20](-.243,3.925)--(-.061,3.845)--(.342,3.831)--(.16,3.911)--cycle;
\filldraw[fill opacity=0.5,fill=gray!20](-.243,4.137)--(-.243,3.925)--(.16,3.911)--(.21,4.121)--cycle;
\filldraw[fill opacity=0.5,fill=gray!20](.696,.221)--(.665,.159)--(.19,.039)--(.243,.106)--cycle;
\filldraw[fill opacity=0.8,fill=gray!20,draw=none](-.057,-.011)--(-.063,-.01)--(-.065,-.037)--(-.064,-.038)--cycle;
\filldraw[fill opacity=0.5,fill=gray!20](.243,3.925)--(-.243,4.137)--(.21,4.121)--(.696,3.909)--cycle;
\filldraw[fill opacity=0.8,fill=gray!20,draw=none](-.064,-.038)--(-.065,-.037)--(-.065,-.043)--cycle;
\filldraw[fill opacity=0.5,fill=gray!20](-.19,4.204)--(-.243,4.137)--(.21,4.121)--(.284,4.188)--cycle;
\filldraw[fill opacity=0.8,fill=gray!20,draw=none](-.102,-.009)--(-.098,-.046)--(-.065,-.043)--(-.074,-.01)--cycle;
\draw(-.102,-.009)--(-.098,-.046)--(-.065,-.043);
\filldraw[fill opacity=0.8,fill=gray!20,draw=none](-.049,-.048)--(-.064,-.038)--(-.065,-.043)--(-.049,-.05)--cycle;
\draw(-.065,-.043)--(-.049,-.05)--(-.049,-.048);
\filldraw[fill opacity=0.8,fill=gray!20,draw=none](-.057,-.011)--(-.064,-.038)--(-.049,-.048)--(-.043,-.014)--cycle;
\draw(-.049,-.048)--(-.043,-.014);
\filldraw[fill opacity=0.8,fill=gray!20,draw=none](-.058,-.011)--(-.043,-.014)--(-.042,-.008)--cycle;
\draw(-.043,-.014)--(-.042,-.008);
\filldraw[fill opacity=0.8,fill=gray!20,draw=none](-.04,-.098)--(-.043,-.102)--(-.039,-.104)--cycle;
\draw(-.043,-.102)--(-.039,-.104)--(-.04,-.098);
\filldraw[fill opacity=0.5,fill=gray!20](.061,.398)--(.243,.318)--(-.16,.332)--(-.342,.412)--cycle;
\filldraw[fill opacity=0.5,fill=gray!20,draw=none](-.033,.004)--(-.097,-.01)--(-.433,.002)--(-.359,.015)--cycle;
\draw(-.097,-.01)--(-.433,.002)--(-.359,.015)--(-.033,.004);
\filldraw[fill opacity=0.5,fill=gray!20](1.59,3.011)--(1.758,3.135)--(1.933,2.694)--(1.745,2.619)--cycle;
\filldraw[fill opacity=0.5,fill=gray!20](-.433,.002)--(-.503,.017)--(-.972,.178)--(-.899,.162)--cycle;
\filldraw[fill opacity=0.5,fill=gray!20](.19,.039)--(.131,-.002)--(-.359,.015)--(-.284,.055)--cycle;
\filldraw[fill opacity=0.5,fill=gray!20](-1.343,.907)--(-1.161,.828)--(-1.408,1.152)--(-1.59,1.232)--cycle;
\filldraw[fill opacity=0.5,fill=gray!20](-1.48,.742)--(-1.343,.907)--(-1.59,1.232)--(-1.758,1.108)--cycle;
\filldraw[fill opacity=0.5,fill=gray!20](1.078,3.957)--(1.032,4.023)--(1.433,3.732)--(1.47,3.672)--cycle;
\filldraw[fill opacity=0.5,fill=gray!20](-1.032,.22)--(-1.078,.286)--(-1.47,.571)--(-1.432,.511)--cycle;
\filldraw[fill opacity=0.5,fill=gray!20](.503,4.225)--(.433,4.241)--(.899,4.081)--(.972,4.065)--cycle;
\filldraw[fill opacity=0.5,fill=gray!20](1.48,3.5)--(1.486,3.594)--(1.777,3.211)--(1.758,3.135)--cycle;
\filldraw[fill opacity=0.5,fill=gray!20](-.994,.53)--(-1.48,.742)--(-1.758,1.108)--(-1.272,.895)--cycle;
\filldraw[fill opacity=0.5,fill=gray!20](-1.486,.649)--(-1.48,.742)--(-1.759,1.107)--(-1.777,1.032)--cycle;
\filldraw[fill opacity=0.5,fill=gray!20](-.131,4.245)--(-.19,4.204)--(.284,4.188)--(.359,4.228)--cycle;
\filldraw[fill opacity=0.5,fill=gray!20](.646,.42)--(.696,.221)--(.243,.106)--(.243,.318)--cycle;
\filldraw[fill opacity=0.5,fill=gray!20](.838,3.65)--(1.021,3.57)--(1.343,3.335)--(1.161,3.415)--cycle;
\filldraw[fill opacity=0.5,fill=gray!20](-.646,3.823)--(-.463,3.743)--(-.061,3.845)--(-.243,3.925)--cycle;
\filldraw[fill opacity=0.5,fill=gray!20](-.696,4.022)--(-.646,3.823)--(-.243,3.925)--(-.243,4.137)--cycle;
\filldraw[fill opacity=0.5,fill=gray!20](-.359,.015)--(-.433,.002)--(-.899,.162)--(-.816,.171)--cycle;
\filldraw[fill opacity=0.5,fill=gray!20](.243,.106)--(.19,.039)--(-.284,.055)--(-.21,.122)--cycle;
\filldraw[fill opacity=0.5,fill=gray!20](1.107,3.87)--(1.078,3.957)--(1.47,3.672)--(1.486,3.594)--cycle;
\filldraw[fill opacity=0.5,fill=gray!20](-.972,.178)--(-1.032,.22)--(-1.432,.511)--(-1.375,.471)--cycle;
\filldraw[fill opacity=0.5,fill=gray!20](.567,4.183)--(.503,4.225)--(.972,4.065)--(1.032,4.023)--cycle;
\filldraw[fill opacity=0.5,fill=gray!20](-1.47,.571)--(-1.486,.649)--(-1.777,1.032)--(-1.771,.966)--cycle;
\filldraw[fill opacity=0.5,fill=gray!20](1.343,3.335)--(1.48,3.5)--(1.758,3.135)--(1.59,3.011)--cycle;
\filldraw[fill opacity=0.5,fill=gray!20](-.342,.412)--(-.16,.332)--(-.535,.461)--(-.717,.54)--cycle;
\filldraw[fill opacity=0.5,fill=gray!20](-.067,4.258)--(-.131,4.245)--(.359,4.228)--(.433,4.241)--cycle;
\filldraw[fill opacity=0.5,fill=gray!20](-.21,3.81)--(-.696,4.022)--(-.243,4.137)--(.243,3.925)--cycle;
\filldraw[fill opacity=0.5,fill=gray!20](-.665,4.084)--(-.696,4.022)--(-.243,4.137)--(-.19,4.204)--cycle;
\filldraw[fill opacity=0.8,fill=gray!20,draw=none](.174,1.995)--(.153,2.004)--(.182,2.053)--(.203,2.044)--cycle;
\draw(.182,2.053)--(.203,2.044)--(.174,1.995)--(.153,2.004);
\filldraw[fill opacity=0.8,fill=gray!20,draw=none](.203,2.044)--(.182,2.053)--(.192,2.109)--(.213,2.1)--cycle;
\draw(.192,2.109)--(.213,2.1)--(.203,2.044)--(.182,2.053);
\filldraw[fill opacity=0.5,fill=gray!20](.243,.318)--(.243,.106)--(-.21,.122)--(-.16,.332)--cycle;
\filldraw[fill opacity=0.5,fill=gray!20](-.284,.055)--(-.359,.015)--(-.816,.171)--(-.726,.207)--cycle;
\filldraw[fill opacity=0.8,fill=gray!20,draw=none](.132,1.959)--(.111,1.968)--(.153,2.004)--(.174,1.995)--cycle;
\draw(.153,2.004)--(.174,1.995)--(.132,1.959)--(.111,1.968);
\filldraw[fill opacity=0.5,fill=gray!20](-1.272,.895)--(-1.758,1.108)--(-1.933,1.549)--(-1.447,1.337)--cycle;
\filldraw[fill opacity=0.5,fill=gray!20](-1.777,1.032)--(-1.759,1.107)--(-1.933,1.549)--(-1.961,1.494)--cycle;
\filldraw[fill opacity=0.8,fill=gray!20,draw=none](.213,2.1)--(.192,2.109)--(.182,2.163)--(.203,2.154)--cycle;
\draw(.182,2.163)--(.203,2.154)--(.213,2.1)--(.192,2.109);
\filldraw[fill opacity=0.5,fill=gray!20](1.118,3.764)--(1.107,3.87)--(1.486,3.594)--(1.48,3.5)--cycle;
\filldraw[fill opacity=0.5,fill=gray!20](-.899,.162)--(-.972,.178)--(-1.375,.471)--(-1.299,.453)--cycle;
\filldraw[fill opacity=0.8,fill=gray!20,draw=none](.082,1.943)--(.061,1.952)--(.111,1.968)--(.132,1.959)--cycle;
\draw(.111,1.968)--(.132,1.959)--(.082,1.943)--(.061,1.952);
\filldraw[fill opacity=0.5,fill=gray!20](.621,4.114)--(.567,4.183)--(1.032,4.023)--(1.078,3.957)--cycle;
\filldraw[fill opacity=0.5,fill=gray!20](.463,3.778)--(.646,3.699)--(1.021,3.57)--(.838,3.65)--cycle;
\filldraw[fill opacity=0.5,fill=gray!20](-1.432,.511)--(-1.47,.571)--(-1.771,.966)--(-1.739,.914)--cycle;
\filldraw[fill opacity=0.8,fill=gray!20,draw=none](.203,2.154)--(.182,2.163)--(.153,2.206)--(.174,2.197)--cycle;
\draw(.153,2.206)--(.174,2.197)--(.203,2.154)--(.182,2.163);
\filldraw[fill opacity=0.5,fill=gray!20](-.621,4.121)--(-.665,4.084)--(-.19,4.204)--(-.131,4.245)--cycle;
\filldraw[fill opacity=0.5,fill=gray!20](0,4.243)--(-.067,4.258)--(.433,4.241)--(.503,4.225)--cycle;
\filldraw[fill opacity=0.8,fill=gray!20,draw=none](.032,1.948)--(.011,1.957)--(.061,1.952)--(.082,1.943)--cycle;
\draw(.061,1.952)--(.082,1.943)--(.032,1.948)--(.011,1.957);
\filldraw[fill opacity=0.8,fill=gray!20,draw=none](.174,2.197)--(.153,2.206)--(.111,2.232)--(.132,2.223)--cycle;
\draw(.111,2.232)--(.132,2.223)--(.174,2.197)--(.153,2.206);
\filldraw[fill opacity=0.8,fill=gray!20](-.049,2.071)--(-.039,2.018)--(-.011,1.975)--(.032,1.948)--(.082,1.943)--(.132,1.959)--(.174,1.995)--(.203,2.044)--(.213,2.1)--(.203,2.154)--(.174,2.197)--(.132,2.223)--(.082,2.229)--(.032,2.212)--(-.011,2.177)--(-.039,2.127)--cycle;
\filldraw[fill opacity=0.5,fill=gray!20](-1.021,3.612)--(-.838,3.533)--(-.463,3.743)--(-.646,3.823)--cycle;
\filldraw[fill opacity=0.5,fill=gray!20](-1.118,3.786)--(-1.021,3.612)--(-.646,3.823)--(-.696,4.022)--cycle;
\filldraw[fill opacity=0.8,fill=gray!20,draw=none](-.011,1.975)--(-.032,1.984)--(.011,1.957)--(.032,1.948)--cycle;
\draw(.011,1.957)--(.032,1.948)--(-.011,1.975)--(-.032,1.984);
\filldraw[fill opacity=0.8,fill=gray!20,draw=none](.132,2.223)--(.111,2.232)--(.061,2.238)--(.082,2.229)--cycle;
\draw(.061,2.238)--(.082,2.229)--(.132,2.223)--(.111,2.232);
\filldraw[fill opacity=0.8,fill=gray!20,draw=none](-.039,2.018)--(-.06,2.027)--(-.032,1.984)--(-.011,1.975)--cycle;
\draw(-.032,1.984)--(-.011,1.975)--(-.039,2.018)--(-.06,2.027);
\filldraw[fill opacity=0.8,fill=gray!20,draw=none](.082,2.229)--(.061,2.238)--(.011,2.221)--(.032,2.212)--cycle;
\draw(.011,2.221)--(.032,2.212)--(.082,2.229)--(.061,2.238);
\filldraw[fill opacity=0.5,fill=gray!20](-.632,3.573)--(-1.118,3.786)--(-.696,4.022)--(-.21,3.81)--cycle;
\filldraw[fill opacity=0.5,fill=gray!20](-1.107,3.836)--(-1.118,3.786)--(-.696,4.022)--(-.665,4.084)--cycle;
\filldraw[fill opacity=0.8,fill=gray!20,draw=none](-.049,2.071)--(-.07,2.081)--(-.06,2.027)--(-.039,2.018)--cycle;
\draw(-.06,2.027)--(-.039,2.018)--(-.049,2.071)--(-.07,2.081);
\filldraw[fill opacity=0.5,fill=gray!20](-.717,.54)--(-.535,.461)--(-.857,.695)--(-1.039,.775)--cycle;
\filldraw[fill opacity=0.5,fill=gray!20](-.21,.122)--(-.284,.055)--(-.726,.207)--(-.632,.266)--cycle;
\filldraw[fill opacity=0.5,fill=gray!20](1.021,3.57)--(1.118,3.764)--(1.48,3.5)--(1.343,3.335)--cycle;
\filldraw[fill opacity=0.8,fill=gray!20,draw=none](.032,2.212)--(.011,2.221)--(-.032,2.186)--(-.011,2.177)--cycle;
\draw(-.032,2.186)--(-.011,2.177)--(.032,2.212)--(.011,2.221);
\filldraw[fill opacity=0.5,fill=gray!20](-1.771,.966)--(-1.777,1.032)--(-1.961,1.494)--(-1.96,1.444)--cycle;
\filldraw[fill opacity=0.8,fill=gray!20,draw=none](-.039,2.127)--(-.06,2.136)--(-.07,2.081)--(-.049,2.071)--cycle;
\draw(-.07,2.081)--(-.049,2.071)--(-.039,2.127)--(-.06,2.136);
\filldraw[fill opacity=0.8,fill=gray!20,draw=none](-.011,2.177)--(-.032,2.186)--(-.06,2.136)--(-.039,2.127)--cycle;
\draw(-.06,2.136)--(-.039,2.127)--(-.011,2.177)--(-.032,2.186);
\filldraw[fill opacity=0.5,fill=gray!20](-.816,.171)--(-.899,.162)--(-1.299,.453)--(-1.208,.457)--cycle;
\filldraw[fill opacity=0.5,fill=gray!20](.665,4.021)--(.621,4.114)--(1.078,3.957)--(1.107,3.87)--cycle;
\filldraw[fill opacity=0.5,fill=gray!20](-1.375,.471)--(-1.432,.511)--(-1.739,.914)--(-1.684,.877)--cycle;
\filldraw[fill opacity=0.5,fill=gray!20](-.567,4.132)--(-.621,4.121)--(-.131,4.245)--(-.067,4.258)--cycle;
\filldraw[fill opacity=0.5,fill=gray!20](.067,4.2)--(0,4.243)--(.503,4.225)--(.567,4.183)--cycle;
\filldraw[fill opacity=0.5,fill=gray!20](-1.343,3.308)--(-1.161,3.228)--(-.838,3.533)--(-1.021,3.612)--cycle;
\filldraw[fill opacity=0.5,fill=gray!20](-1.48,3.443)--(-1.343,3.308)--(-1.021,3.612)--(-1.118,3.786)--cycle;
\filldraw[fill opacity=0.5,fill=gray!20](-.16,.332)--(-.21,.122)--(-.632,.266)--(-.535,.461)--cycle;
\filldraw[fill opacity=0.5,fill=gray!20](.061,3.792)--(.243,3.712)--(.646,3.699)--(.463,3.778)--cycle;
\filldraw[fill opacity=0.5,fill=gray!20](-1.078,3.865)--(-1.107,3.836)--(-.665,4.084)--(-.621,4.121)--cycle;
\filldraw[fill opacity=0.5,fill=gray!20](.696,3.909)--(.665,4.021)--(1.107,3.87)--(1.118,3.764)--cycle;
\filldraw[fill opacity=0.5,fill=gray!20](-1.739,.914)--(-1.771,.966)--(-1.96,1.444)--(-1.932,1.401)--cycle;
\filldraw[fill opacity=0.5,fill=gray!20](-.726,.207)--(-.816,.171)--(-1.208,.457)--(-1.105,.483)--cycle;
\filldraw[fill opacity=0.5,fill=gray!20](-.994,3.23)--(-1.48,3.443)--(-1.118,3.786)--(-.632,3.573)--cycle;
\filldraw[fill opacity=0.5,fill=gray!20](-1.486,3.477)--(-1.48,3.443)--(-1.118,3.786)--(-1.107,3.836)--cycle;
\filldraw[fill opacity=0.5,fill=gray!20](-1.59,2.929)--(-1.408,2.85)--(-1.161,3.228)--(-1.343,3.308)--cycle;
\filldraw[fill opacity=0.5,fill=gray!20](-1.758,3.017)--(-1.59,2.929)--(-1.343,3.308)--(-1.48,3.443)--cycle;
\filldraw[fill opacity=0.5,fill=gray!20](-1.299,.453)--(-1.375,.471)--(-1.684,.877)--(-1.606,.856)--cycle;
\filldraw[fill opacity=0.5,fill=gray!20](-.503,4.116)--(-.567,4.132)--(-.067,4.258)--(0,4.243)--cycle;
\filldraw[fill opacity=0.5,fill=gray!20](.131,4.131)--(.067,4.2)--(.567,4.183)--(.621,4.114)--cycle;
\filldraw[fill opacity=0.5,fill=gray!20](-1.745,2.503)--(-1.563,2.423)--(-1.408,2.85)--(-1.59,2.929)--cycle;
\filldraw[fill opacity=0.5,fill=gray!20](-1.933,2.537)--(-1.745,2.503)--(-1.59,2.929)--(-1.758,3.017)--cycle;
\filldraw[fill opacity=0.5,fill=gray!20,draw=none](-1.57,2.363)--(-1.563,2.423)--(-1.605,2.441)--cycle;
\draw(-1.57,2.363)--(-1.563,2.423)--(-1.605,2.441);
\filldraw[fill opacity=0.5,fill=gray!20,draw=none](-1.806,1.86)--(-1.786,1.844)--(-1.704,1.449)--(-1.794,1.488)--cycle;
\draw(-1.704,1.449)--(-1.794,1.488);
\filldraw[fill opacity=0.8,fill=gray!20,draw=none](-1.545,1.865)--(-1.734,1.801)--(-1.758,1.821)--(-1.569,1.884)--cycle;
\draw(-1.545,1.865)--(-1.734,1.801);
\draw(-1.758,1.821)--(-1.569,1.884);
\filldraw[fill opacity=0.8,fill=gray!20,draw=none](-1.523,1.864)--(-1.713,1.801)--(-1.734,1.801)--(-1.545,1.865)--cycle;
\draw(-1.523,1.864)--(-1.713,1.801);
\draw(-1.734,1.801)--(-1.545,1.865);
\filldraw[fill opacity=0.8,fill=gray!20,draw=none](-1.547,2.224)--(-1.738,2.232)--(-1.714,2.232)--(-1.523,2.224)--cycle;
\draw(-1.547,2.224)--(-1.738,2.232);
\draw(-1.714,2.232)--(-1.523,2.224);
\filldraw[fill opacity=0.8,fill=gray!20,draw=none](-1.508,1.883)--(-1.697,1.82)--(-1.713,1.801)--(-1.523,1.864)--cycle;
\draw(-1.508,1.883)--(-1.697,1.82);
\draw(-1.713,1.801)--(-1.523,1.864);
\filldraw[fill opacity=0.8,fill=gray!20,draw=none](-1.523,2.224)--(-1.714,2.232)--(-1.693,2.213)--(-1.502,2.204)--cycle;
\draw(-1.523,2.224)--(-1.714,2.232);
\draw(-1.693,2.213)--(-1.502,2.204);
\filldraw[fill opacity=0.8,fill=gray!20,draw=none](-1.5,1.919)--(-1.689,1.855)--(-1.697,1.82)--(-1.508,1.883)--cycle;
\draw(-1.5,1.919)--(-1.689,1.855);
\draw(-1.697,1.82)--(-1.508,1.883);
\filldraw[fill opacity=0.8,fill=gray!20,draw=none](-1.502,2.204)--(-1.693,2.213)--(-1.677,2.176)--(-1.486,2.168)--cycle;
\draw(-1.502,2.204)--(-1.693,2.213);
\draw(-1.677,2.176)--(-1.486,2.168);
\filldraw[fill opacity=0.8,fill=gray!20,draw=none](-1.492,1.968)--(-1.691,1.902)--(-1.689,1.855)--(-1.502,1.918)--cycle;
\draw(-1.492,1.968)--(-1.691,1.902);
\draw(-1.689,1.855)--(-1.502,1.918);
\filldraw[fill opacity=0.8,fill=gray!20,draw=none](-1.48,2.132)--(-1.486,2.168)--(-1.572,2.172)--(-1.672,2.144)--(-1.669,2.129)--(-1.508,2.122)--cycle;
\draw(-1.486,2.168)--(-1.572,2.172);
\draw(-1.669,2.129)--(-1.508,2.122);
\filldraw[fill opacity=0.8,fill=gray!20,draw=none](-1.49,2.022)--(-1.492,2.022)--(-1.49,2.022)--cycle;
\draw(-1.492,2.022)--(-1.49,2.022);
\filldraw[fill opacity=0.8,fill=gray!20,draw=none](-1.49,2.022)--(-1.49,2.023)--(-1.492,2.022)--cycle;
\draw(-1.49,2.023)--(-1.492,2.022);
\filldraw[fill opacity=0.8,fill=gray!20,draw=none](-1.49,2.022)--(-1.492,2.022)--(-1.49,2.022)--cycle;
\filldraw[fill opacity=0.8,fill=gray!20,draw=none](-1.49,2.022)--(-1.492,2.022)--(-1.554,2.001)--(-1.49,2)--cycle;
\draw(-1.492,2.022)--(-1.554,2.001);
\filldraw[fill opacity=0.8,fill=gray!20,draw=none](-1.49,2.022)--(-1.48,2.069)--(-1.67,2.077)--(-1.681,2.03)--(-1.492,2.022)--cycle;
\draw(-1.48,2.069)--(-1.67,2.077);
\draw(-1.681,2.03)--(-1.492,2.022);
\filldraw[fill opacity=0.8,fill=gray!20,draw=none](-1.49,2)--(-1.554,2.001)--(-1.641,1.972)--(-1.499,1.966)--(-1.492,1.968)--cycle;
\draw(-1.554,2.001)--(-1.641,1.972);
\draw(-1.499,1.966)--(-1.492,1.968);
\filldraw[fill opacity=0.8,fill=gray!20,draw=none](-1.492,2.022)--(-1.688,2.016)--(-1.696,1.999)--cycle;
\filldraw[fill opacity=0.8,fill=gray!20,draw=none](-1.492,2.022)--(-1.681,2.03)--(-1.688,2.016)--cycle;
\draw(-1.492,2.022)--(-1.681,2.03);
\filldraw[fill opacity=0.8,fill=gray!20,draw=none](-1.696,1.999)--(-1.69,1.956)--(-1.492,2.022)--cycle;
\draw(-1.69,1.956)--(-1.492,2.022);
\filldraw[fill opacity=0.8,fill=gray!20,draw=none](-1.641,1.972)--(-1.69,1.956)--(-1.691,1.905)--(-1.691,1.902)--(-1.499,1.966)--cycle;
\draw(-1.641,1.972)--(-1.69,1.956);
\draw(-1.691,1.902)--(-1.499,1.966);
\filldraw[fill opacity=0.8,fill=gray!20,draw=none](-1.48,2.132)--(-1.508,2.122)--(-1.478,2.12)--cycle;
\draw(-1.508,2.122)--(-1.478,2.12);
\filldraw[fill opacity=0.8,fill=gray!20,draw=none](-1.478,2.12)--(-1.669,2.129)--(-1.67,2.077)--(-1.48,2.069)--cycle;
\draw(-1.478,2.12)--(-1.669,2.129);
\draw(-1.67,2.077)--(-1.48,2.069);
\filldraw[fill opacity=0.8,fill=gray!20,draw=none](-1.572,2.172)--(-1.677,2.176)--(-1.672,2.144)--cycle;
\draw(-1.572,2.172)--(-1.677,2.176);
\filldraw[fill opacity=0.5,fill=gray!20](-1.507,1.824)--(-1.993,2.037)--(-1.933,2.537)--(-1.447,2.325)--cycle;
\filldraw[fill opacity=0.5,fill=gray!20](-2.023,2.005)--(-1.993,2.037)--(-1.933,2.537)--(-1.961,2.529)--cycle;
\filldraw[fill opacity=0.5,fill=gray!20](-1.039,.775)--(-.857,.695)--(-1.104,1.02)--(-1.286,1.099)--cycle;
\filldraw[fill opacity=0.5,fill=gray!20](.646,3.699)--(.696,3.909)--(1.118,3.764)--(1.021,3.57)--cycle;
\filldraw[fill opacity=0.5,fill=gray!20](-1.032,3.87)--(-1.078,3.865)--(-.621,4.121)--(-.567,4.132)--cycle;
\filldraw[fill opacity=0.5,fill=gray!20](-1.039,3.175)--(-.857,3.095)--(-.535,3.4)--(-.717,3.48)--cycle;
\filldraw[fill opacity=0.5,fill=gray!20](-1.286,2.797)--(-1.104,2.717)--(-.857,3.095)--(-1.039,3.175)--cycle;
\filldraw[fill opacity=0.5,fill=gray!20](-1.272,2.805)--(-1.758,3.017)--(-1.48,3.443)--(-.994,3.23)--cycle;
\filldraw[fill opacity=0.5,fill=gray!20](-1.777,3.031)--(-1.759,3.017)--(-1.48,3.443)--(-1.486,3.477)--cycle;
\filldraw[fill opacity=0.5,fill=gray!20](-.632,.266)--(-.726,.207)--(-1.105,.483)--(-.994,.53)--cycle;
\filldraw[fill opacity=0.5,fill=gray!20](-1.684,.877)--(-1.739,.914)--(-1.932,1.401)--(-1.878,1.367)--cycle;
\filldraw[fill opacity=0.8,fill=gray!20,draw=none](-1.293,1.9)--(-1.295,1.935)--(-1.333,1.938)--cycle;
\draw(-1.293,1.9)--(-1.295,1.935)--(-1.333,1.938);
\filldraw[fill opacity=0.8,fill=gray!20,draw=none](-1.333,1.938)--(-1.295,1.935)--(-1.296,1.982)--(-1.36,1.987)--cycle;
\draw(-1.333,1.938)--(-1.295,1.935)--(-1.296,1.982)--(-1.36,1.987);
\filldraw[fill opacity=0.8,fill=gray!20,draw=none](-1.287,2.202)--(-1.214,2.205)--(-1.222,2.223)--cycle;
\draw(-1.287,2.202)--(-1.214,2.205)--(-1.222,2.223);
\filldraw[fill opacity=0.8,fill=gray!20,draw=none](-1.198,2.154)--(-1.214,2.205)--(-1.287,2.202)--(-1.293,2.199)--(-1.295,2.149)--cycle;
\draw(-1.293,2.199)--(-1.295,2.149)--(-1.198,2.154)--(-1.214,2.205)--(-1.287,2.202);
\filldraw[fill opacity=0.8,fill=gray!20](-1.188,1.987)--(-1.185,2.041)--(-1.296,2.036)--(-1.296,1.982)--cycle;
\filldraw[fill opacity=0.8,fill=gray!20,draw=none](-1.36,1.987)--(-1.296,1.982)--(-1.296,2.036)--(-1.369,2.041)--cycle;
\draw(-1.36,1.987)--(-1.296,1.982)--(-1.296,2.036)--(-1.369,2.041);
\filldraw[fill opacity=0.8,fill=gray!20,draw=none](-1.333,2.152)--(-1.295,2.149)--(-1.293,2.199)--cycle;
\draw(-1.333,2.152)--(-1.295,2.149)--(-1.293,2.199);
\filldraw[fill opacity=0.8,fill=gray!20](-1.188,2.098)--(-1.198,2.154)--(-1.295,2.149)--(-1.296,2.093)--cycle;
\filldraw[fill opacity=0.8,fill=gray!20,draw=none](-1.369,2.041)--(-1.296,2.036)--(-1.296,2.093)--(-1.36,2.097)--cycle;
\draw(-1.369,2.041)--(-1.296,2.036)--(-1.296,2.093)--(-1.36,2.097);
\filldraw[fill opacity=0.8,fill=gray!20](-1.185,2.041)--(-1.188,2.098)--(-1.296,2.093)--(-1.296,2.036)--cycle;
\filldraw[fill opacity=0.8,fill=gray!20,draw=none](-1.36,2.097)--(-1.296,2.093)--(-1.295,2.149)--(-1.333,2.152)--cycle;
\draw(-1.36,2.097)--(-1.296,2.093)--(-1.295,2.149)--(-1.333,2.152);
\filldraw[fill opacity=0.5,fill=gray!20](-1.495,1.925)--(-1.312,1.846)--(-1.259,2.291)--(-1.442,2.37)--cycle;
\filldraw[fill opacity=0.5,fill=gray!20](-1.442,2.37)--(-1.259,2.291)--(-1.104,2.717)--(-1.286,2.797)--cycle;
\filldraw[fill opacity=0.5,fill=gray!20](-1.447,2.325)--(-1.933,2.537)--(-1.758,3.017)--(-1.272,2.805)--cycle;
\filldraw[fill opacity=0.5,fill=gray!20](-1.961,2.529)--(-1.933,2.537)--(-1.759,3.017)--(-1.777,3.031)--cycle;
\filldraw[fill opacity=0.5,fill=gray!20](.19,4.038)--(.131,4.131)--(.621,4.114)--(.665,4.021)--cycle;
\filldraw[fill opacity=0.5,fill=gray!20](-1.47,3.494)--(-1.486,3.477)--(-1.107,3.836)--(-1.078,3.865)--cycle;
\filldraw[fill opacity=0.5,fill=gray!20](-1.208,.457)--(-1.299,.453)--(-1.606,.856)--(-1.509,.852)--cycle;
\filldraw[fill opacity=0.5,fill=gray!20](-.433,4.074)--(-.503,4.116)--(0,4.243)--(.067,4.2)--cycle;
\filldraw[fill opacity=0.5,fill=gray!20](-.342,3.69)--(-.16,3.611)--(.243,3.712)--(.061,3.792)--cycle;
\filldraw[fill opacity=0.5,fill=gray!20](-1.932,1.401)--(-1.96,1.444)--(-2.025,1.972)--(-1.998,1.94)--cycle;
\filldraw[fill opacity=0.5,fill=gray!20](-2.025,1.972)--(-2.023,2.005)--(-1.961,2.529)--(-1.96,2.514)--cycle;
\filldraw[fill opacity=0.5,fill=gray!20](-.535,.461)--(-.632,.266)--(-.994,.53)--(-.857,.695)--cycle;
\filldraw[fill opacity=0.5,fill=gray!20](-.972,3.853)--(-1.032,3.87)--(-.567,4.132)--(-.503,4.116)--cycle;
\filldraw[fill opacity=0.5,fill=gray!20](.243,3.925)--(.19,4.038)--(.665,4.021)--(.696,3.909)--cycle;
\filldraw[fill opacity=0.5,fill=gray!20](-1.771,3.033)--(-1.777,3.031)--(-1.486,3.477)--(-1.47,3.494)--cycle;
\filldraw[fill opacity=0.5,fill=gray!20](-1.105,.483)--(-1.208,.457)--(-1.509,.852)--(-1.397,.865)--cycle;
\filldraw[fill opacity=0.5,fill=gray!20](-1.606,.856)--(-1.684,.877)--(-1.878,1.367)--(-1.799,1.343)--cycle;
\filldraw[fill opacity=0.5,fill=gray!20](-1.286,1.099)--(-1.104,1.02)--(-1.259,1.412)--(-1.442,1.492)--cycle;
\filldraw[fill opacity=0.5,fill=gray!20](-.359,4.007)--(-.433,4.074)--(.067,4.2)--(.131,4.131)--cycle;
\filldraw[fill opacity=0.5,fill=gray!20](-1.432,3.492)--(-1.47,3.494)--(-1.078,3.865)--(-1.032,3.87)--cycle;
\filldraw[fill opacity=0.5,fill=gray!20](-1.96,2.514)--(-1.961,2.529)--(-1.777,3.031)--(-1.771,3.033)--cycle;
\filldraw[fill opacity=0.5,fill=gray!20](.243,3.712)--(.243,3.925)--(.696,3.909)--(.646,3.699)--cycle;
\filldraw[fill opacity=0.5,fill=gray!20,draw=none](-1.808,1.928)--(-1.807,1.898)--(-1.856,1.977)--(-1.82,1.961)--cycle;
\draw(-1.856,1.977)--(-1.82,1.961);
\filldraw[fill opacity=0.5,fill=gray!20,draw=none](-1.807,1.898)--(-1.808,1.928)--(-1.798,1.903)--(-1.792,1.876)--cycle;
\filldraw[fill opacity=0.5,fill=gray!20,draw=none](-1.798,1.903)--(-1.82,1.961)--(-1.809,1.956)--cycle;
\draw(-1.82,1.961)--(-1.809,1.956);
\filldraw[fill opacity=0.8,fill=gray!20,draw=none](-1.798,1.903)--(-1.977,1.843)--(-1.987,1.893)--(-1.808,1.953)--cycle;
\draw(-1.798,1.903)--(-1.977,1.843);
\draw(-1.987,1.893)--(-1.808,1.953);
\filldraw[fill opacity=0.8,fill=gray!20,draw=none](-1.809,1.956)--(-1.809,1.953)--(-1.987,1.893)--(-1.989,1.934)--(-1.988,1.94)--(-1.81,1.999)--cycle;
\draw(-1.809,1.953)--(-1.987,1.893);
\draw(-1.988,1.94)--(-1.81,1.999);
\filldraw[fill opacity=0.8,fill=gray!20,draw=none](-1.803,1.848)--(-1.962,1.795)--(-1.977,1.843)--(-1.811,1.898)--cycle;
\draw(-1.803,1.848)--(-1.962,1.795);
\draw(-1.977,1.843)--(-1.811,1.898);
\filldraw[fill opacity=0.8,fill=gray!20,draw=none](-1.788,1.81)--(-1.943,1.759)--(-1.962,1.795)--(-1.803,1.848)--cycle;
\draw(-1.788,1.81)--(-1.943,1.759);
\draw(-1.962,1.795)--(-1.803,1.848);
\filldraw[fill opacity=0.5,fill=gray!20,draw=none](-1.586,1.397)--(-1.704,1.449)--(-1.809,1.956)--(-1.691,1.905)--cycle;
\draw(-1.586,1.397)--(-1.704,1.449);
\draw(-1.809,1.956)--(-1.691,1.905);
\filldraw[fill opacity=0.8,fill=gray!20,draw=none](-1.734,1.801)--(-1.923,1.738)--(-1.943,1.759)--(-1.758,1.821)--cycle;
\draw(-1.734,1.801)--(-1.923,1.738);
\draw(-1.943,1.759)--(-1.758,1.821);
\filldraw[fill opacity=0.5,fill=gray!20](-1.878,1.367)--(-1.932,1.401)--(-1.998,1.94)--(-1.944,1.909)--cycle;
\filldraw[fill opacity=0.5,fill=gray!20](-1.998,1.94)--(-2.025,1.972)--(-1.96,2.514)--(-1.932,2.492)--cycle;
\filldraw[fill opacity=0.5,fill=gray!20](-.717,3.48)--(-.535,3.4)--(-.16,3.611)--(-.342,3.69)--cycle;
\filldraw[fill opacity=0.5,fill=gray!20](-.994,.53)--(-1.105,.483)--(-1.397,.865)--(-1.272,.895)--cycle;
\filldraw[fill opacity=0.8,fill=gray!20,draw=none](-1.085,2.017)--(-1.076,2.073)--(-1.1,2.057)--(-1.106,2.003)--cycle;
\draw(-1.076,2.073)--(-1.1,2.057)--(-1.106,2.003)--(-1.085,2.017);
\filldraw[fill opacity=0.8,fill=gray!20,draw=none](-1.112,1.962)--(-1.085,2.017)--(-1.106,2.003)--(-1.124,1.954)--cycle;
\draw(-1.085,2.017)--(-1.106,2.003)--(-1.124,1.954)--(-1.112,1.962);
\filldraw[fill opacity=0.8,fill=gray!20,draw=none](-1.112,1.962)--(-1.124,1.954)--(-1.152,1.915)--cycle;
\draw(-1.112,1.962)--(-1.124,1.954)--(-1.152,1.915);
\filldraw[fill opacity=0.5,fill=gray!20,draw=none](-1.806,1.86)--(-1.807,1.898)--(-1.792,1.876)--(-1.786,1.844)--cycle;
\filldraw[fill opacity=0.8,fill=gray!20,draw=none](-1.076,2.073)--(-1.085,2.128)--(-1.106,2.114)--(-1.1,2.057)--cycle;
\draw(-1.085,2.128)--(-1.106,2.114)--(-1.1,2.057)--(-1.076,2.073);
\filldraw[fill opacity=0.8,fill=gray!20,draw=none](-1.158,1.913)--(-1.152,1.915)--(-1.124,1.954)--(-1.198,1.939)--(-1.214,1.902)--cycle;
\draw(-1.152,1.915)--(-1.124,1.954)--(-1.198,1.939)--(-1.214,1.902)--(-1.158,1.913);
\filldraw[fill opacity=0.8,fill=gray!20,draw=none](-1.158,1.913)--(-1.214,1.902)--(-1.222,1.891)--cycle;
\draw(-1.158,1.913)--(-1.214,1.902)--(-1.222,1.891);
\filldraw[fill opacity=0.5,fill=gray!20](-.899,3.812)--(-.972,3.853)--(-.503,4.116)--(-.433,4.074)--cycle;
\filldraw[fill opacity=0.5,fill=gray!20](-1.509,.852)--(-1.606,.856)--(-1.799,1.343)--(-1.698,1.33)--cycle;
\filldraw[fill opacity=0.8,fill=gray!20,draw=none](-1.085,2.128)--(-1.112,2.176)--(-1.124,2.168)--(-1.106,2.114)--cycle;
\draw(-1.112,2.176)--(-1.124,2.168)--(-1.106,2.114)--(-1.085,2.128);
\filldraw[fill opacity=0.5,fill=gray!20](-.284,3.918)--(-.359,4.007)--(.131,4.131)--(.19,4.038)--cycle;
\filldraw[fill opacity=0.5,fill=gray!20](-1.739,3.022)--(-1.771,3.033)--(-1.47,3.494)--(-1.432,3.492)--cycle;
\filldraw[fill opacity=0.8,fill=gray!20](-1.124,1.954)--(-1.106,2.003)--(-1.188,1.987)--(-1.198,1.939)--cycle;
\filldraw[fill opacity=0.8,fill=gray!20,draw=none](-1.222,1.891)--(-1.214,1.902)--(-1.287,1.898)--cycle;
\draw(-1.222,1.891)--(-1.214,1.902)--(-1.287,1.898);
\filldraw[fill opacity=0.8,fill=gray!20,draw=none](-1.78,1.856)--(-1.803,1.848)--(-1.811,1.898)--(-1.798,1.903)--cycle;
\draw(-1.78,1.856)--(-1.803,1.848);
\draw(-1.811,1.898)--(-1.798,1.903);
\filldraw[fill opacity=0.8,fill=gray!20,draw=none](-1.112,2.176)--(-1.152,2.214)--(-1.124,2.168)--cycle;
\draw(-1.152,2.214)--(-1.124,2.168)--(-1.112,2.176);
\filldraw[fill opacity=0.8,fill=gray!20,draw=none](-1.809,1.956)--(-1.808,1.953)--(-1.809,1.953)--cycle;
\draw(-1.808,1.953)--(-1.809,1.953);
\filldraw[fill opacity=0.8,fill=gray!20,draw=none](-1.809,1.956)--(-1.81,1.999)--(-1.799,2.003)--cycle;
\draw(-1.81,1.999)--(-1.799,2.003);
\filldraw[fill opacity=0.5,fill=gray!20](-.857,.695)--(-.994,.53)--(-1.272,.895)--(-1.104,1.02)--cycle;
\filldraw[fill opacity=0.8,fill=gray!20,draw=none](-1.758,1.821)--(-1.788,1.81)--(-1.803,1.848)--(-1.78,1.856)--cycle;
\draw(-1.758,1.821)--(-1.788,1.81);
\draw(-1.803,1.848)--(-1.78,1.856);
\filldraw[fill opacity=0.5,fill=gray!20](-1.375,3.471)--(-1.432,3.492)--(-1.032,3.87)--(-.972,3.853)--cycle;
\filldraw[fill opacity=0.8,fill=gray!20,draw=none](-1.287,1.898)--(-1.214,1.902)--(-1.198,1.939)--(-1.295,1.935)--(-1.293,1.9)--cycle;
\draw(-1.287,1.898)--(-1.214,1.902)--(-1.198,1.939)--(-1.295,1.935)--(-1.293,1.9);
\filldraw[fill opacity=0.8,fill=gray!20](-1.106,2.003)--(-1.1,2.057)--(-1.185,2.041)--(-1.188,1.987)--cycle;
\filldraw[fill opacity=0.5,fill=gray!20](-1.442,1.492)--(-1.259,1.412)--(-1.312,1.846)--(-1.495,1.925)--cycle;
\filldraw[fill opacity=0.8,fill=gray!20,draw=none](-1.783,2.039)--(-1.812,2.031)--(-1.965,2.038)--(-1.97,2.085)--(-1.79,2.077)--cycle;
\draw(-1.812,2.031)--(-1.965,2.038);
\draw(-1.97,2.085)--(-1.79,2.077);
\filldraw[fill opacity=0.8,fill=gray!20,draw=none](-1.783,2.039)--(-1.783,2.037)--(-1.812,2.031)--cycle;
\filldraw[fill opacity=0.8,fill=gray!20,draw=none](-1.799,2.003)--(-1.81,1.999)--(-1.802,2.035)--(-1.78,2.042)--cycle;
\draw(-1.799,2.003)--(-1.81,1.999);
\draw(-1.802,2.035)--(-1.78,2.042);
\filldraw[fill opacity=0.5,fill=gray!20](-1.932,2.492)--(-1.96,2.514)--(-1.771,3.033)--(-1.739,3.022)--cycle;
\filldraw[fill opacity=0.5,fill=gray!20,draw=none](-1.982,1.967)--(-1.988,1.94)--(-1.99,1.935)--(-1.998,1.94)--(-1.997,1.951)--cycle;
\draw(-1.99,1.935)--(-1.998,1.94)--(-1.997,1.951);
\filldraw[fill opacity=0.5,fill=gray!20,draw=none](-1.979,2.081)--(-1.972,2.058)--(-1.975,1.974)--(-1.997,1.951)--(-1.984,2.055)--cycle;
\draw(-1.997,1.951)--(-1.984,2.055);
\filldraw[fill opacity=0.5,fill=gray!20,draw=none](-1.982,1.967)--(-1.975,1.974)--(-1.975,1.97)--(-1.988,1.94)--cycle;
\filldraw[fill opacity=0.5,fill=gray!20,draw=none](-1.968,2.137)--(-1.984,2.055)--(-1.964,2.225)--cycle;
\draw(-1.984,2.055)--(-1.964,2.225);
\filldraw[fill opacity=0.8,fill=gray!20,draw=none](-1.788,2.129)--(-1.968,2.137)--(-1.959,2.184)--(-1.778,2.176)--cycle;
\draw(-1.788,2.129)--(-1.968,2.137);
\draw(-1.959,2.184)--(-1.778,2.176);
\filldraw[fill opacity=0.8,fill=gray!20,draw=none](-1.79,2.077)--(-1.97,2.085)--(-1.968,2.137)--(-1.788,2.129)--cycle;
\draw(-1.79,2.077)--(-1.97,2.085);
\draw(-1.968,2.137)--(-1.788,2.129);
\filldraw[fill opacity=0.8,fill=gray!20,draw=none](-1.778,2.176)--(-1.959,2.184)--(-1.943,2.221)--(-1.76,2.213)--cycle;
\draw(-1.778,2.176)--(-1.959,2.184);
\draw(-1.943,2.221)--(-1.76,2.213);
\filldraw[fill opacity=0.5,fill=gray!20,draw=none](-1.964,1.998)--(-1.972,1.978)--(-1.975,1.974)--(-1.974,1.999)--cycle;
\filldraw[fill opacity=0.5,fill=gray!20,draw=none](-1.972,1.978)--(-1.975,1.97)--(-1.975,1.974)--cycle;
\filldraw[fill opacity=0.8,fill=gray!20,draw=none](-1.81,1.999)--(-1.994,1.938)--(-1.993,1.971)--(-1.802,2.035)--cycle;
\draw(-1.81,1.999)--(-1.994,1.938);
\draw(-1.993,1.971)--(-1.802,2.035);
\filldraw[fill opacity=0.8,fill=gray!20,draw=none](-1.124,2.168)--(-1.152,2.214)--(-1.158,2.216)--(-1.214,2.205)--(-1.198,2.154)--cycle;
\draw(-1.158,2.216)--(-1.214,2.205)--(-1.198,2.154)--(-1.124,2.168)--(-1.152,2.214);
\filldraw[fill opacity=0.8,fill=gray!20](-1.1,2.057)--(-1.106,2.114)--(-1.188,2.098)--(-1.185,2.041)--cycle;
\filldraw[fill opacity=0.8,fill=gray!20,draw=none](-1.158,2.216)--(-1.222,2.223)--(-1.214,2.205)--cycle;
\draw(-1.222,2.223)--(-1.214,2.205)--(-1.158,2.216);
\filldraw[fill opacity=0.8,fill=gray!20](-1.106,2.114)--(-1.124,2.168)--(-1.198,2.154)--(-1.188,2.098)--cycle;
\filldraw[fill opacity=0.8,fill=gray!20,draw=none](-1.783,2.037)--(-1.782,2.03)--(-1.812,2.031)--cycle;
\draw(-1.782,2.03)--(-1.812,2.031);
\filldraw[fill opacity=0.8,fill=gray!20,draw=none](-1.766,1.994)--(-1.779,1.994)--(-1.802,2.031)--(-1.782,2.03)--cycle;
\draw(-1.766,1.994)--(-1.779,1.994);
\draw(-1.802,2.031)--(-1.782,2.03);
\filldraw[fill opacity=0.8,fill=gray!20,draw=none](-1.78,2.042)--(-1.802,2.035)--(-1.786,2.054)--(-1.758,2.063)--cycle;
\draw(-1.78,2.042)--(-1.802,2.035);
\draw(-1.786,2.054)--(-1.758,2.063);
\filldraw[fill opacity=0.8,fill=gray!20](-1.198,1.939)--(-1.188,1.987)--(-1.296,1.982)--(-1.295,1.935)--cycle;
\filldraw[fill opacity=0.8,fill=gray!20,draw=none](-1.713,1.801)--(-1.898,1.739)--(-1.923,1.738)--(-1.734,1.801)--cycle;
\draw(-1.713,1.801)--(-1.898,1.739);
\draw(-1.923,1.738)--(-1.734,1.801);
\filldraw[fill opacity=0.8,fill=gray!20,draw=none](-1.72,1.812)--(-1.878,1.759)--(-1.898,1.739)--(-1.743,1.791)--cycle;
\draw(-1.72,1.812)--(-1.878,1.759);
\draw(-1.898,1.739)--(-1.743,1.791);
\filldraw[fill opacity=0.8,fill=gray!20,draw=none](-1.689,1.855)--(-1.868,1.795)--(-1.878,1.759)--(-1.697,1.82)--cycle;
\draw(-1.689,1.855)--(-1.868,1.795);
\draw(-1.878,1.759)--(-1.697,1.82);
\filldraw[fill opacity=0.8,fill=gray!20,draw=none](-1.691,1.901)--(-1.869,1.842)--(-1.868,1.795)--(-1.702,1.851)--cycle;
\draw(-1.691,1.901)--(-1.869,1.842);
\draw(-1.868,1.795)--(-1.702,1.851);
\filldraw[fill opacity=0.8,fill=gray!20,draw=none](-1.69,1.956)--(-1.879,1.892)--(-1.879,1.886)--(-1.869,1.842)--(-1.691,1.901)--cycle;
\draw(-1.69,1.956)--(-1.879,1.892);
\draw(-1.869,1.842)--(-1.691,1.901);
\filldraw[fill opacity=0.5,fill=gray!20](-1.799,1.343)--(-1.878,1.367)--(-1.944,1.909)--(-1.865,1.882)--cycle;
\filldraw[fill opacity=0.8,fill=gray!20,draw=none](-1.812,2.031)--(-1.864,2.014)--(-1.942,2.001)--cycle;
\draw(-1.812,2.031)--(-1.864,2.014);
\filldraw[fill opacity=0.8,fill=gray!20,draw=none](-1.864,2.014)--(-1.972,1.978)--(-1.953,1.998)--(-1.942,2.001)--cycle;
\draw(-1.864,2.014)--(-1.972,1.978);
\draw(-1.953,1.998)--(-1.942,2.001);
\filldraw[fill opacity=0.5,fill=gray!20,draw=none](-1.957,2.015)--(-1.959,2.009)--(-1.973,2.025)--(-1.972,2.058)--cycle;
\filldraw[fill opacity=0.5,fill=gray!20,draw=none](-1.959,2.009)--(-1.964,1.998)--(-1.974,1.999)--(-1.973,2.025)--cycle;
\filldraw[fill opacity=0.5,fill=gray!20,draw=none](-1.964,1.998)--(-1.959,2.009)--(-1.953,2.002)--(-1.951,1.998)--cycle;
\filldraw[fill opacity=0.5,fill=gray!20,draw=none](-1.959,2.009)--(-1.957,2.015)--(-1.953,2.002)--cycle;
\filldraw[fill opacity=0.8,fill=gray!20,draw=none](-1.779,1.994)--(-1.97,2.003)--(-1.971,2.013)--(-1.812,2.031)--(-1.802,2.031)--cycle;
\draw(-1.779,1.994)--(-1.97,2.003);
\draw(-1.812,2.031)--(-1.802,2.031);
\filldraw[fill opacity=0.8,fill=gray!20,draw=none](-1.971,2.013)--(-1.975,2.038)--(-1.812,2.031)--cycle;
\draw(-1.975,2.038)--(-1.812,2.031);
\filldraw[fill opacity=0.8,fill=gray!20,draw=none](-1.802,2.035)--(-1.812,2.031)--(-1.942,2.001)--(-1.786,2.054)--cycle;
\draw(-1.802,2.035)--(-1.812,2.031);
\draw(-1.942,2.001)--(-1.786,2.054);
\filldraw[fill opacity=0.8,fill=gray!20,draw=none](-2.427,1.691)--(-2.428,1.691)--(-2.428,1.694)--cycle;
\draw(-2.427,1.691)--(-2.428,1.691)--(-2.428,1.694);
\filldraw[fill opacity=0.8,fill=gray!20,draw=none](-2.452,1.652)--(-2.493,1.65)--(-2.5,1.688)--(-2.428,1.691)--(-2.428,1.685)--cycle;
\draw(-2.452,1.652)--(-2.493,1.65)--(-2.5,1.688)--(-2.428,1.691)--(-2.428,1.685);
\filldraw[fill opacity=0.8,fill=gray!20,draw=none](-2.492,1.688)--(-2.454,1.728)--(-2.437,1.728)--(-2.428,1.703)--(-2.428,1.691)--cycle;
\draw(-2.454,1.728)--(-2.437,1.728);
\draw(-2.428,1.703)--(-2.428,1.691)--(-2.492,1.688);
\filldraw[fill opacity=0.8,fill=gray!20,draw=none](-2.427,1.691)--(-2.428,1.694)--(-2.428,1.703)--cycle;
\draw(-2.428,1.694)--(-2.428,1.703);
\filldraw[fill opacity=0.8,fill=gray!20,draw=none](-2.451,1.738)--(-2.437,1.728)--(-2.454,1.728)--cycle;
\draw(-2.437,1.728)--(-2.454,1.728);
\filldraw[fill opacity=0.8,fill=gray!20,draw=none](-2.451,1.738)--(-2.454,1.728)--(-2.502,1.726)--(-2.5,1.762)--(-2.49,1.762)--cycle;
\draw(-2.454,1.728)--(-2.502,1.726)--(-2.5,1.762)--(-2.49,1.762);
\filldraw[fill opacity=0.8,fill=gray!20,draw=none](-2.492,1.688)--(-2.5,1.688)--(-2.502,1.726)--(-2.454,1.728)--cycle;
\draw(-2.492,1.688)--(-2.5,1.688)--(-2.502,1.726)--(-2.454,1.728);
\filldraw[fill opacity=0.8,fill=gray!20,draw=none](-1.977,1.843)--(-2.482,1.674)--(-2.496,1.723)--(-1.987,1.893)--cycle;
\draw(-1.977,1.843)--(-2.482,1.674)--(-2.496,1.723)--(-1.987,1.893);
\filldraw[fill opacity=0.8,fill=gray!20,draw=none](-1.962,1.795)--(-1.971,1.792)--(-1.982,1.841)--(-1.977,1.843)--cycle;
\draw(-1.962,1.795)--(-1.971,1.792);
\draw(-1.982,1.841)--(-1.977,1.843);
\filldraw[fill opacity=0.8,fill=gray!20,draw=none](-1.971,1.792)--(-2.41,1.645)--(-2.426,1.692)--(-1.982,1.841)--cycle;
\draw(-1.971,1.792)--(-2.41,1.645);
\draw(-2.426,1.692)--(-1.982,1.841);
\filldraw[fill opacity=0.5,fill=gray!20,draw=none](-1.939,2.28)--(-1.953,2.21)--(-1.965,2.197)--(-1.964,2.225)--(-1.956,2.298)--cycle;
\draw(-1.964,2.225)--(-1.956,2.298);
\filldraw[fill opacity=0.5,fill=gray!20,draw=none](-1.935,2.24)--(-1.943,2.221)--(-1.953,2.21)--(-1.947,2.24)--cycle;
\filldraw[fill opacity=0.5,fill=gray!20,draw=none](-1.924,2.241)--(-1.943,2.221)--(-1.935,2.24)--cycle;
\filldraw[fill opacity=0.8,fill=gray!20,draw=none](-1.76,2.213)--(-1.953,2.221)--(-1.941,2.241)--(-1.738,2.232)--cycle;
\draw(-1.76,2.213)--(-1.953,2.221);
\draw(-1.941,2.241)--(-1.738,2.232);
\filldraw[fill opacity=0.8,fill=gray!20,draw=none](-1.987,1.893)--(-1.99,1.892)--(-1.994,1.938)--(-1.989,1.939)--cycle;
\draw(-1.987,1.893)--(-1.99,1.892);
\draw(-1.994,1.938)--(-1.989,1.939);
\filldraw[fill opacity=0.8,fill=gray!20,draw=none](-2.451,1.738)--(-2.49,1.762)--(-2.444,1.764)--cycle;
\draw(-2.49,1.762)--(-2.444,1.764);
\filldraw[fill opacity=0.8,fill=gray!20,draw=none](-2.456,1.783)--(-2.444,1.764)--(-2.473,1.763)--cycle;
\draw(-2.444,1.764)--(-2.473,1.763);
\filldraw[fill opacity=0.8,fill=gray!20,draw=none](-2.456,1.783)--(-2.473,1.763)--(-2.5,1.762)--(-2.493,1.793)--(-2.463,1.795)--cycle;
\draw(-2.473,1.763)--(-2.5,1.762)--(-2.493,1.793)--(-2.463,1.795);
\filldraw[fill opacity=0.8,fill=gray!20](-2.555,1.751)--(-2.542,1.784)--(-2.493,1.793)--(-2.5,1.762)--cycle;
\filldraw[fill opacity=0.8,fill=gray!20,draw=none](-1.99,1.892)--(-2.496,1.723)--(-2.5,1.768)--(-1.994,1.938)--cycle;
\draw(-1.99,1.892)--(-2.496,1.723)--(-2.5,1.768)--(-1.994,1.938);
\filldraw[fill opacity=0.5,fill=gray!20,draw=none](-1.97,1.982)--(-1.946,1.982)--(-1.938,1.959)--(-1.944,1.909)--(-1.99,1.935)--cycle;
\draw(-1.938,1.959)--(-1.944,1.909)--(-1.99,1.935);
\filldraw[fill opacity=0.5,fill=gray!20,draw=none](-1.97,1.982)--(-1.964,1.998)--(-1.951,1.998)--(-1.946,1.982)--cycle;
\filldraw[fill opacity=0.8,fill=gray!20,draw=none](-1.972,1.978)--(-1.993,1.971)--(-1.988,1.986)--(-1.953,1.998)--cycle;
\draw(-1.972,1.978)--(-1.993,1.971);
\draw(-1.988,1.986)--(-1.953,1.998);
\filldraw[fill opacity=0.8,fill=gray!20,draw=none](-1.949,1.997)--(-1.942,2.001)--(-1.951,1.998)--cycle;
\filldraw[fill opacity=0.8,fill=gray!20,draw=none](-1.951,1.998)--(-1.942,2.001)--(-1.953,1.998)--cycle;
\draw(-1.942,2.001)--(-1.953,1.998);
\filldraw[fill opacity=0.8,fill=gray!20,draw=none](-1.951,1.998)--(-1.953,1.998)--(-1.971,1.992)--(-1.968,1.991)--cycle;
\draw(-1.953,1.998)--(-1.971,1.992);
\filldraw[fill opacity=0.5,fill=gray!20,draw=none](-1.951,1.998)--(-1.953,2.002)--(-1.949,1.997)--cycle;
\filldraw[fill opacity=0.5,fill=gray!20,draw=none](-1.946,1.982)--(-1.951,1.998)--(-1.949,1.997)--(-1.936,1.982)--cycle;
\filldraw[fill opacity=0.8,fill=gray!20,draw=none](-1.949,1.997)--(-1.951,1.998)--(-1.968,1.991)--(-1.964,1.989)--cycle;
\filldraw[fill opacity=0.8,fill=gray!20,draw=none](-1.745,1.974)--(-1.962,1.984)--(-1.97,2.003)--(-1.766,1.994)--cycle;
\draw(-1.745,1.974)--(-1.962,1.984);
\draw(-1.97,2.003)--(-1.766,1.994);
\filldraw[fill opacity=0.5,fill=gray!20](-.21,3.81)--(-.284,3.918)--(.19,4.038)--(.243,3.925)--cycle;
\filldraw[fill opacity=0.5,fill=gray!20](-1.397,.865)--(-1.509,.852)--(-1.698,1.33)--(-1.58,1.328)--cycle;
\filldraw[fill opacity=0.8,fill=gray!20,draw=none](-1.943,1.759)--(-1.96,1.753)--(-1.971,1.792)--(-1.962,1.795)--cycle;
\draw(-1.943,1.759)--(-1.96,1.753);
\draw(-1.971,1.792)--(-1.962,1.795);
\filldraw[fill opacity=0.5,fill=gray!20](-.816,3.751)--(-.899,3.812)--(-.433,4.074)--(-.359,4.007)--cycle;
\filldraw[fill opacity=0.8,fill=gray!20,draw=none](-2.413,1.644)--(-2.416,1.618)--(-2.43,1.619)--(-2.429,1.653)--(-2.422,1.653)--cycle;
\draw(-2.416,1.618)--(-2.43,1.619)--(-2.429,1.653)--(-2.422,1.653);
\filldraw[fill opacity=0.8,fill=gray!20](-2.483,1.616)--(-2.493,1.65)--(-2.429,1.653)--(-2.43,1.619)--cycle;
\filldraw[fill opacity=0.8,fill=gray!20,draw=none](-2.383,1.615)--(-2.416,1.618)--(-2.413,1.644)--cycle;
\draw(-2.383,1.615)--(-2.416,1.618);
\filldraw[fill opacity=0.8,fill=gray!20,draw=none](-2.431,1.614)--(-2.43,1.619)--(-2.383,1.615)--(-2.383,1.611)--cycle;
\draw(-2.431,1.614)--(-2.43,1.619)--(-2.383,1.615);
\filldraw[fill opacity=0.8,fill=gray!20](-2.469,1.588)--(-2.483,1.616)--(-2.43,1.619)--(-2.432,1.589)--cycle;
\filldraw[fill opacity=0.8,fill=gray!20,draw=none](-2.432,1.589)--(-2.431,1.614)--(-2.383,1.611)--(-2.383,1.61)--(-2.396,1.587)--cycle;
\draw(-2.383,1.61)--(-2.396,1.587)--(-2.432,1.589)--(-2.431,1.614);
\filldraw[fill opacity=0.8,fill=gray!20,draw=none](-1.96,1.753)--(-2.437,1.593)--(-2.461,1.628)--(-1.971,1.792)--cycle;
\draw(-1.96,1.753)--(-2.437,1.593)--(-2.461,1.628)--(-1.971,1.792);
\filldraw[fill opacity=0.8,fill=gray!20,draw=none](-1.758,2.063)--(-1.786,2.054)--(-1.765,2.053)--(-1.734,2.064)--cycle;
\draw(-1.758,2.063)--(-1.786,2.054);
\draw(-1.765,2.053)--(-1.734,2.064);
\filldraw[fill opacity=0.8,fill=gray!20,draw=none](-1.989,1.934)--(-1.989,1.939)--(-1.988,1.94)--cycle;
\draw(-1.989,1.939)--(-1.988,1.94);
\filldraw[fill opacity=0.5,fill=gray!20,draw=none](-1.979,2.081)--(-1.968,2.137)--(-1.972,2.058)--cycle;
\filldraw[fill opacity=0.8,fill=gray!20,draw=none](-2.456,1.783)--(-2.463,1.795)--(-2.445,1.795)--cycle;
\draw(-2.463,1.795)--(-2.445,1.795);
\filldraw[fill opacity=0.8,fill=gray!20,draw=none](-2.445,1.795)--(-2.491,1.793)--(-2.443,1.82)--cycle;
\draw(-2.445,1.795)--(-2.491,1.793);
\filldraw[fill opacity=0.8,fill=gray!20,draw=none](-2.491,1.793)--(-2.493,1.793)--(-2.483,1.818)--(-2.443,1.82)--cycle;
\draw(-2.491,1.793)--(-2.493,1.793)--(-2.483,1.818)--(-2.443,1.82);
\filldraw[fill opacity=0.8,fill=gray!20](-2.542,1.784)--(-2.523,1.81)--(-2.483,1.818)--(-2.493,1.793)--cycle;
\filldraw[fill opacity=0.8,fill=gray!20,draw=none](-1.994,1.938)--(-2.5,1.768)--(-2.495,1.802)--(-1.993,1.971)--cycle;
\draw(-1.994,1.938)--(-2.5,1.768)--(-2.495,1.802)--(-1.993,1.971);
\filldraw[fill opacity=0.5,fill=gray!20,draw=none](-1.949,1.997)--(-1.934,1.997)--(-1.936,1.982)--cycle;
\draw(-1.934,1.997)--(-1.936,1.982);
\filldraw[fill opacity=0.8,fill=gray!20,draw=none](-1.786,2.054)--(-1.942,2.001)--(-1.964,1.989)--(-1.961,1.988)--(-1.765,2.053)--cycle;
\draw(-1.786,2.054)--(-1.942,2.001);
\draw(-1.961,1.988)--(-1.765,2.053);
\filldraw[fill opacity=0.5,fill=gray!20](-.16,3.611)--(-.21,3.81)--(.243,3.925)--(.243,3.712)--cycle;
\filldraw[fill opacity=0.8,fill=gray!20,draw=none](-1.923,1.738)--(-1.949,1.729)--(-1.96,1.753)--(-1.943,1.759)--cycle;
\draw(-1.923,1.738)--(-1.949,1.729);
\draw(-1.96,1.753)--(-1.943,1.759);
\filldraw[fill opacity=0.5,fill=gray!20,draw=none](-1.925,2.264)--(-1.935,2.24)--(-1.947,2.24)--(-1.939,2.28)--cycle;
\filldraw[fill opacity=0.5,fill=gray!20,draw=none](-1.914,2.251)--(-1.924,2.241)--(-1.935,2.24)--(-1.925,2.264)--cycle;
\filldraw[fill opacity=0.5,fill=gray!20,draw=none](-1.924,2.241)--(-1.914,2.251)--(-1.905,2.241)--cycle;
\filldraw[fill opacity=0.8,fill=gray!20,draw=none](-1.738,2.232)--(-1.941,2.241)--(-1.931,2.242)--(-1.714,2.232)--cycle;
\draw(-1.738,2.232)--(-1.941,2.241);
\draw(-1.931,2.242)--(-1.714,2.232);
\filldraw[fill opacity=0.5,fill=gray!20,draw=none](-1.953,2.21)--(-1.959,2.184)--(-1.967,2.165)--(-1.965,2.197)--cycle;
\filldraw[fill opacity=0.5,fill=gray!20,draw=none](-1.959,2.184)--(-1.968,2.137)--(-1.967,2.165)--cycle;
\filldraw[fill opacity=0.8,fill=gray!20,draw=none](-1.968,2.137)--(-1.971,2.137)--(-1.963,2.184)--(-1.959,2.184)--cycle;
\draw(-1.968,2.137)--(-1.971,2.137);
\draw(-1.963,2.184)--(-1.959,2.184);
\filldraw[fill opacity=0.5,fill=gray!20,draw=none](-1.943,2.221)--(-1.924,2.241)--(-1.905,2.241)--(-1.921,2.102)--(-1.957,2.015)--(-1.972,2.058)--(-1.968,2.137)--(-1.959,2.184)--cycle;
\draw(-1.905,2.241)--(-1.921,2.102);
\filldraw[fill opacity=0.8,fill=gray!20,draw=none](-1.97,2.085)--(-1.975,2.086)--(-1.971,2.137)--(-1.968,2.137)--cycle;
\draw(-1.97,2.085)--(-1.975,2.086);
\draw(-1.971,2.137)--(-1.968,2.137);
\filldraw[fill opacity=0.8,fill=gray!20,draw=none](-1.721,1.974)--(-1.734,1.975)--(-1.755,1.974)--(-1.745,1.974)--cycle;
\draw(-1.721,1.974)--(-1.734,1.975);
\draw(-1.755,1.974)--(-1.745,1.974);
\filldraw[fill opacity=0.8,fill=gray!20,draw=none](-1.965,2.038)--(-1.975,2.038)--(-1.975,2.086)--(-1.97,2.085)--cycle;
\draw(-1.965,2.038)--(-1.975,2.038);
\draw(-1.975,2.086)--(-1.97,2.085);
\filldraw[fill opacity=0.8,fill=gray!20](-3.309,2.998)--(-3.154,1.786)--(-3.254,1.752)--(-3.409,2.965)--cycle;
\filldraw[fill opacity=0.8,fill=gray!20,draw=none](-3.077,2.168)--(-3.055,1.995)--(-3.177,1.968)--(-3.202,2.162)--cycle;
\draw(-3.077,2.168)--(-3.055,1.995);
\draw(-3.177,1.968)--(-3.202,2.162);
\filldraw[fill opacity=0.8,fill=gray!20,draw=none](-2.942,2.113)--(-2.938,2.083)--(-3.016,2.024)--(-3.058,2.025)--(-3.074,2.145)--cycle;
\draw(-2.942,2.113)--(-2.938,2.083);
\draw(-3.058,2.025)--(-3.074,2.145);
\filldraw[fill opacity=0.8,fill=gray!20,draw=none](-2.9,2.221)--(-2.87,2.181)--(-2.938,2.083)--(-2.952,2.198)--cycle;
\draw(-2.938,2.083)--(-2.952,2.198);
\filldraw[fill opacity=0.8,fill=gray!20,draw=none](-2.952,2.198)--(-2.942,2.113)--(-3.074,2.145)--(-3.089,2.263)--cycle;
\draw(-2.952,2.198)--(-2.942,2.113);
\draw(-3.074,2.145)--(-3.089,2.263);
\filldraw[fill opacity=0.8,fill=gray!20,draw=none](-3.089,2.263)--(-3.077,2.168)--(-3.202,2.162)--(-3.218,2.29)--cycle;
\draw(-3.089,2.263)--(-3.077,2.168);
\draw(-3.202,2.162)--(-3.218,2.29);
\filldraw[fill opacity=0.8,fill=gray!20,draw=none](-2.922,2.106)--(-2.891,2.15)--(-2.85,2.119)--(-2.854,2.079)--cycle;
\filldraw[fill opacity=0.8,fill=gray!20,draw=none](-2.968,2.166)--(-2.964,2.134)--(-3.061,2.061)--(-3.081,2.221)--cycle;
\draw(-2.968,2.166)--(-2.964,2.134);
\draw(-3.061,2.061)--(-3.081,2.221);
\filldraw[fill opacity=0.8,fill=gray!20,draw=none](-2.891,2.15)--(-2.87,2.181)--(-2.847,2.149)--(-2.85,2.119)--cycle;
\filldraw[fill opacity=0.8,fill=gray!20,draw=none](-3.081,2.221)--(-3.069,2.126)--(-3.176,2.139)--(-3.189,2.243)--cycle;
\draw(-3.081,2.221)--(-3.069,2.126);
\draw(-3.176,2.139)--(-3.189,2.243);
\filldraw[fill opacity=0.8,fill=gray!20,draw=none](-2.998,2.109)--(-2.964,2.134)--(-2.959,2.099)--cycle;
\draw(-2.964,2.134)--(-2.959,2.099);
\filldraw[fill opacity=0.8,fill=gray!20,draw=none](-3.111,2.437)--(-3.089,2.263)--(-3.218,2.29)--(-3.243,2.484)--cycle;
\draw(-3.111,2.437)--(-3.089,2.263);
\draw(-3.218,2.29)--(-3.243,2.484);
\filldraw[fill opacity=0.8,fill=gray!20,draw=none](-3.208,2.392)--(-3.163,2.039)--(-3.254,2.07)--(-3.291,2.361)--cycle;
\draw(-3.208,2.392)--(-3.163,2.039);
\draw(-3.254,2.07)--(-3.291,2.361);
\filldraw[fill opacity=0.8,fill=gray!20,draw=none](-2.922,2.106)--(-2.854,2.079)--(-2.859,2.02)--(-2.93,2.021)--(-2.938,2.083)--cycle;
\draw(-2.93,2.021)--(-2.938,2.083);
\filldraw[fill opacity=0.8,fill=gray!20,draw=none](-2.951,2.153)--(-2.91,2.122)--(-2.913,2.082)--(-2.959,2.099)--(-2.964,2.134)--cycle;
\draw(-2.959,2.099)--(-2.964,2.134);
\filldraw[fill opacity=0.8,fill=gray!20,draw=none](-2.85,2.119)--(-2.847,2.149)--(-2.833,2.13)--(-2.83,2.103)--cycle;
\draw(-2.833,2.13)--(-2.83,2.103);
\filldraw[fill opacity=0.8,fill=gray!20,draw=none](-2.945,2.282)--(-2.9,2.221)--(-2.952,2.198)--(-2.963,2.282)--cycle;
\draw(-2.952,2.198)--(-2.963,2.282);
\filldraw[fill opacity=0.8,fill=gray!20,draw=none](-2.963,2.282)--(-2.952,2.198)--(-3.089,2.263)--(-3.092,2.287)--cycle;
\draw(-2.963,2.282)--(-2.952,2.198);
\draw(-3.089,2.263)--(-3.092,2.287);
\filldraw[fill opacity=0.8,fill=gray!20,draw=none](-2.93,2.183)--(-2.964,2.134)--(-2.978,2.249)--cycle;
\draw(-2.964,2.134)--(-2.978,2.249);
\filldraw[fill opacity=0.8,fill=gray!20,draw=none](-2.83,2.123)--(-2.823,2.095)--(-2.83,2.103)--(-2.833,2.13)--cycle;
\draw(-2.83,2.103)--(-2.833,2.13);
\filldraw[fill opacity=0.8,fill=gray!20,draw=none](-2.854,2.203)--(-2.845,2.175)--(-2.847,2.149)--(-2.87,2.181)--cycle;
\filldraw[fill opacity=0.8,fill=gray!20,draw=none](-2.847,2.149)--(-2.845,2.175)--(-2.835,2.146)--(-2.833,2.13)--cycle;
\draw(-2.835,2.146)--(-2.833,2.13);
\filldraw[fill opacity=0.8,fill=gray!20,draw=none](-2.974,2.217)--(-2.968,2.166)--(-3.041,2.202)--cycle;
\draw(-2.974,2.217)--(-2.968,2.166);
\filldraw[fill opacity=0.8,fill=gray!20,draw=none](-2.83,2.123)--(-2.833,2.13)--(-2.835,2.146)--cycle;
\draw(-2.833,2.13)--(-2.835,2.146);
\filldraw[fill opacity=0.8,fill=gray!20,draw=none](-2.951,2.153)--(-2.93,2.183)--(-2.907,2.152)--(-2.91,2.122)--cycle;
\filldraw[fill opacity=0.8,fill=gray!20,draw=none](-3.306,2.156)--(-3.29,2.036)--(-3.313,2.037)--(-3.367,2.102)--(-3.37,2.132)--cycle;
\draw(-3.306,2.156)--(-3.29,2.036);
\draw(-3.367,2.102)--(-3.37,2.132);
\filldraw[fill opacity=0.8,fill=gray!20,draw=none](-2.91,2.122)--(-2.907,2.152)--(-2.888,2.126)--(-2.878,2.097)--cycle;
\filldraw[fill opacity=0.8,fill=gray!20,draw=none](-2.914,2.206)--(-2.905,2.178)--(-2.907,2.152)--(-2.93,2.183)--cycle;
\filldraw[fill opacity=0.8,fill=gray!20,draw=none](-2.907,2.152)--(-2.905,2.178)--(-2.888,2.126)--cycle;
\filldraw[fill opacity=0.8,fill=gray!20,draw=none](-3.321,2.274)--(-3.306,2.156)--(-3.37,2.132)--(-3.381,2.217)--cycle;
\draw(-3.321,2.274)--(-3.306,2.156);
\draw(-3.37,2.132)--(-3.381,2.217);
\filldraw[fill opacity=0.8,fill=gray!20,draw=none](-3.275,2.229)--(-3.254,2.07)--(-3.321,2.15)--(-3.325,2.182)--cycle;
\draw(-3.275,2.229)--(-3.254,2.07);
\draw(-3.321,2.15)--(-3.325,2.182);
\filldraw[fill opacity=0.8,fill=gray!20,draw=none](-3.292,2.213)--(-3.325,2.182)--(-3.331,2.233)--cycle;
\draw(-3.325,2.182)--(-3.331,2.233);
\filldraw[fill opacity=0.8,fill=gray!20,draw=none](-3.298,2.122)--(-3.316,2.115)--(-3.321,2.15)--cycle;
\draw(-3.316,2.115)--(-3.321,2.15);
\filldraw[fill opacity=0.8,fill=gray!20,draw=none](-3.394,2.204)--(-3.395,2.243)--(-3.381,2.216)--(-3.367,2.102)--cycle;
\draw(-3.381,2.216)--(-3.367,2.102);
\filldraw[fill opacity=0.8,fill=gray!20,draw=none](-3.322,2.1)--(-3.329,2.14)--(-3.326,2.17)--(-3.321,2.15)--(-3.316,2.115)--cycle;
\draw(-3.321,2.15)--(-3.316,2.115);
\filldraw[fill opacity=0.8,fill=gray!20,draw=none](-3.382,2.102)--(-3.389,2.143)--(-3.386,2.172)--(-3.373,2.125)--cycle;
\filldraw[fill opacity=0.8,fill=gray!20,draw=none](-3.324,2.297)--(-3.321,2.274)--(-3.381,2.217)--(-3.392,2.301)--cycle;
\draw(-3.324,2.297)--(-3.321,2.274);
\draw(-3.381,2.217)--(-3.392,2.301);
\filldraw[fill opacity=0.8,fill=gray!20,draw=none](-3.335,2.201)--(-3.336,2.265)--(-3.321,2.15)--cycle;
\draw(-3.336,2.265)--(-3.321,2.15);
\filldraw[fill opacity=0.8,fill=gray!20,draw=none](-3.329,2.14)--(-3.334,2.171)--(-3.335,2.201)--(-3.326,2.17)--cycle;
\filldraw[fill opacity=0.8,fill=gray!20,draw=none](-3.389,2.143)--(-3.394,2.173)--(-3.394,2.204)--(-3.386,2.172)--cycle;
\filldraw[fill opacity=0.8,fill=gray!20,draw=none](-3.331,2.117)--(-3.334,2.145)--(-3.334,2.171)--(-3.329,2.14)--cycle;
\filldraw[fill opacity=0.8,fill=gray!20,draw=none](-3.335,2.201)--(-3.334,2.171)--(-3.339,2.197)--(-3.341,2.225)--cycle;
\filldraw[fill opacity=0.8,fill=gray!20,draw=none](-3.334,2.145)--(-3.339,2.197)--(-3.334,2.171)--cycle;
\filldraw[fill opacity=0.8,fill=gray!20,draw=none](-3.394,2.173)--(-3.389,2.143)--(-3.39,2.128)--(-3.394,2.155)--cycle;
\draw(-3.39,2.128)--(-3.394,2.155);
\filldraw[fill opacity=0.8,fill=gray!20,draw=none](-3.394,2.204)--(-3.394,2.173)--(-3.398,2.2)--(-3.401,2.227)--cycle;
\filldraw[fill opacity=0.8,fill=gray!20,draw=none](-3.398,2.2)--(-3.394,2.173)--(-3.394,2.155)--(-3.396,2.17)--cycle;
\draw(-3.394,2.155)--(-3.396,2.17);
\filldraw[fill opacity=0.8,fill=gray!20,draw=none](-3.389,2.143)--(-3.382,2.102)--(-3.386,2.093)--(-3.39,2.128)--cycle;
\draw(-3.386,2.093)--(-3.39,2.128);
\filldraw[fill opacity=0.8,fill=gray!20,draw=none](-3.387,2.12)--(-3.392,2.148)--(-3.394,2.155)--(-3.39,2.128)--cycle;
\draw(-3.394,2.155)--(-3.39,2.128);
\filldraw[fill opacity=0.8,fill=gray!20,draw=none](-3.392,2.148)--(-3.396,2.17)--(-3.394,2.155)--cycle;
\draw(-3.396,2.17)--(-3.394,2.155);
\filldraw[fill opacity=0.8,fill=gray!20](-3.797,2.262)--(-3.768,2.312)--(-3.726,2.347)--(-3.676,2.364)--(-3.626,2.358)--(-3.583,2.332)--(-3.555,2.289)--(-3.545,2.235)--(-3.555,2.179)--(-3.583,2.13)--(-3.626,2.094)--(-3.676,2.078)--(-3.726,2.083)--(-3.768,2.11)--(-3.797,2.153)--(-3.807,2.206)--cycle;
\filldraw[fill opacity=0.8,fill=gray!20,draw=none](-3.535,2.154)--(-3.553,2.116)--(-3.583,2.13)--(-3.555,2.179)--(-3.532,2.169)--cycle;
\draw(-3.553,2.116)--(-3.583,2.13)--(-3.555,2.179)--(-3.532,2.169);
\filldraw[fill opacity=0.8,fill=gray!20,draw=none](-3.526,2.205)--(-3.535,2.171)--(-3.555,2.179)--(-3.545,2.235)--(-3.524,2.226)--cycle;
\draw(-3.535,2.171)--(-3.555,2.179)--(-3.545,2.235)--(-3.524,2.226);
\filldraw[fill opacity=0.8,fill=gray!20,draw=none](-3.526,2.205)--(-3.529,2.168)--(-3.535,2.171)--cycle;
\draw(-3.529,2.168)--(-3.535,2.171);
\filldraw[fill opacity=0.8,fill=gray!20,draw=none](-3.535,2.154)--(-3.532,2.169)--(-3.529,2.168)--cycle;
\draw(-3.532,2.169)--(-3.529,2.168);
\filldraw[fill opacity=0.8,fill=gray!20,draw=none](-1.975,2.086)--(-3.655,2.159)--(-3.653,2.211)--(-1.971,2.137)--cycle;
\draw(-1.975,2.086)--(-3.655,2.159)--(-3.653,2.211)--(-1.971,2.137);
\filldraw[fill opacity=0.8,fill=gray!20,draw=none](-2.879,2.28)--(-2.865,2.237)--(-2.9,2.221)--(-2.945,2.282)--cycle;
\filldraw[fill opacity=0.8,fill=gray!20,draw=none](-2.865,2.237)--(-2.854,2.203)--(-2.87,2.181)--(-2.9,2.221)--cycle;
\filldraw[fill opacity=0.8,fill=gray!20,draw=none](-3.064,2.396)--(-2.967,2.313)--(-2.963,2.282)--(-3.092,2.287)--(-3.107,2.407)--cycle;
\draw(-2.967,2.313)--(-2.963,2.282);
\draw(-3.092,2.287)--(-3.107,2.407);
\filldraw[fill opacity=0.8,fill=gray!20,draw=none](-3.02,2.285)--(-2.978,2.249)--(-2.974,2.217)--(-3.041,2.202)--(-3.081,2.221)--(-3.09,2.288)--cycle;
\draw(-2.978,2.249)--(-2.974,2.217);
\draw(-3.081,2.221)--(-3.09,2.288);
\filldraw[fill opacity=0.8,fill=gray!20,draw=none](-3.09,2.288)--(-3.081,2.221)--(-3.189,2.243)--(-3.195,2.292)--cycle;
\draw(-3.09,2.288)--(-3.081,2.221);
\draw(-3.189,2.243)--(-3.195,2.292);
\filldraw[fill opacity=0.8,fill=gray!20,draw=none](-2.844,2.218)--(-2.843,2.202)--(-2.845,2.175)--(-2.854,2.203)--cycle;
\draw(-2.844,2.218)--(-2.843,2.202);
\filldraw[fill opacity=0.8,fill=gray!20,draw=none](-2.848,2.245)--(-2.844,2.218)--(-2.854,2.203)--(-2.865,2.237)--cycle;
\draw(-2.848,2.245)--(-2.844,2.218);
\filldraw[fill opacity=0.8,fill=gray!20,draw=none](-2.843,2.223)--(-2.843,2.202)--(-2.844,2.218)--cycle;
\draw(-2.843,2.202)--(-2.844,2.218);
\filldraw[fill opacity=0.8,fill=gray!20,draw=none](-2.945,2.282)--(-2.963,2.282)--(-2.967,2.313)--cycle;
\draw(-2.963,2.282)--(-2.967,2.313);
\filldraw[fill opacity=0.8,fill=gray!20,draw=none](-2.901,2.345)--(-2.879,2.28)--(-2.945,2.282)--(-2.967,2.313)--(-2.975,2.374)--cycle;
\draw(-2.967,2.313)--(-2.975,2.374);
\filldraw[fill opacity=0.8,fill=gray!20,draw=none](-2.939,2.283)--(-2.925,2.239)--(-2.96,2.224)--(-2.978,2.249)--(-2.983,2.284)--cycle;
\draw(-2.978,2.249)--(-2.983,2.284);
\filldraw[fill opacity=0.8,fill=gray!20,draw=none](-2.925,2.239)--(-2.914,2.206)--(-2.93,2.183)--(-2.96,2.224)--cycle;
\filldraw[fill opacity=0.8,fill=gray!20,draw=none](-2.843,2.249)--(-2.843,2.223)--(-2.844,2.218)--(-2.848,2.245)--cycle;
\draw(-2.844,2.218)--(-2.848,2.245);
\filldraw[fill opacity=0.8,fill=gray!20,draw=none](-2.9,2.226)--(-2.905,2.178)--(-2.914,2.206)--cycle;
\filldraw[fill opacity=0.8,fill=gray!20,draw=none](-2.898,2.251)--(-2.9,2.226)--(-2.914,2.206)--(-2.925,2.239)--cycle;
\filldraw[fill opacity=0.8,fill=gray!20,draw=none](-3.32,2.298)--(-3.283,2.296)--(-3.275,2.229)--(-3.292,2.213)--(-3.331,2.233)--(-3.336,2.265)--cycle;
\draw(-3.283,2.296)--(-3.275,2.229);
\draw(-3.331,2.233)--(-3.336,2.265);
\filldraw[fill opacity=0.8,fill=gray!20,draw=none](-3.395,2.243)--(-3.396,2.302)--(-3.392,2.301)--(-3.381,2.216)--cycle;
\draw(-3.392,2.301)--(-3.381,2.216);
\filldraw[fill opacity=0.8,fill=gray!20,draw=none](-3.343,2.447)--(-3.324,2.297)--(-3.392,2.301)--(-3.396,2.331)--cycle;
\draw(-3.343,2.447)--(-3.324,2.297);
\draw(-3.392,2.301)--(-3.396,2.331);
\filldraw[fill opacity=0.8,fill=gray!20,draw=none](-3.335,2.24)--(-3.344,2.258)--(-3.348,2.301)--(-3.34,2.3)--(-3.336,2.265)--cycle;
\draw(-3.34,2.3)--(-3.336,2.265);
\filldraw[fill opacity=0.8,fill=gray!20,draw=none](-3.335,2.24)--(-3.335,2.201)--(-3.341,2.225)--(-3.344,2.258)--cycle;
\filldraw[fill opacity=0.8,fill=gray!20,draw=none](-3.396,2.302)--(-3.395,2.243)--(-3.404,2.26)--(-3.408,2.304)--cycle;
\filldraw[fill opacity=0.8,fill=gray!20,draw=none](-3.395,2.243)--(-3.394,2.204)--(-3.401,2.227)--(-3.404,2.26)--cycle;
\filldraw[fill opacity=0.8,fill=gray!20,draw=none](-3.341,2.225)--(-3.339,2.197)--(-3.347,2.245)--cycle;
\filldraw[fill opacity=0.8,fill=gray!20,draw=none](-3.344,2.258)--(-3.341,2.225)--(-3.347,2.245)--(-3.351,2.271)--cycle;
\filldraw[fill opacity=0.8,fill=gray!20,draw=none](-3.401,2.227)--(-3.398,2.2)--(-3.403,2.227)--(-3.405,2.242)--cycle;
\draw(-3.403,2.227)--(-3.405,2.242);
\filldraw[fill opacity=0.8,fill=gray!20,draw=none](-3.404,2.26)--(-3.401,2.227)--(-3.405,2.242)--(-3.408,2.269)--cycle;
\draw(-3.405,2.242)--(-3.408,2.269);
\filldraw[fill opacity=0.8,fill=gray!20,draw=none](-3.405,2.242)--(-3.403,2.227)--(-3.404,2.248)--cycle;
\draw(-3.405,2.242)--(-3.403,2.227);
\filldraw[fill opacity=0.8,fill=gray!20,draw=none](-3.408,2.304)--(-3.404,2.26)--(-3.408,2.269)--(-3.413,2.304)--cycle;
\draw(-3.408,2.269)--(-3.413,2.304);
\filldraw[fill opacity=0.8,fill=gray!20,draw=none](-3.408,2.269)--(-3.405,2.242)--(-3.404,2.248)--(-3.406,2.274)--cycle;
\draw(-3.408,2.269)--(-3.405,2.242);
\filldraw[fill opacity=0.8,fill=gray!20,draw=none](-3.526,2.233)--(-3.527,2.227)--(-3.545,2.235)--(-3.555,2.289)--(-3.539,2.281)--cycle;
\draw(-3.527,2.227)--(-3.545,2.235)--(-3.555,2.289)--(-3.539,2.281);
\filldraw[fill opacity=0.8,fill=gray!20,draw=none](-3.526,2.233)--(-3.524,2.226)--(-3.527,2.227)--cycle;
\draw(-3.524,2.226)--(-3.527,2.227);
\filldraw[fill opacity=0.8,fill=gray!20,draw=none](-3.523,2.253)--(-3.52,2.224)--(-3.524,2.226)--(-3.526,2.233)--cycle;
\draw(-3.52,2.224)--(-3.524,2.226);
\filldraw[fill opacity=0.8,fill=gray!20,draw=none](-3.526,2.205)--(-3.524,2.226)--(-3.52,2.224)--cycle;
\draw(-3.524,2.226)--(-3.52,2.224);
\filldraw[fill opacity=0.8,fill=gray!20,draw=none](-3.523,2.253)--(-3.526,2.233)--(-3.539,2.281)--(-3.525,2.276)--cycle;
\draw(-3.539,2.281)--(-3.525,2.276);
\filldraw[fill opacity=0.8,fill=gray!20,draw=none](-3.629,2.258)--(-3.653,2.211)--(-3.697,2.248)--cycle;
\draw(-3.653,2.211)--(-3.697,2.248);
\filldraw[fill opacity=0.8,fill=gray!20,draw=none](-3.745,2.301)--(-3.71,2.325)--(-3.629,2.258)--(-3.697,2.248)--(-3.751,2.292)--cycle;
\draw(-3.71,2.325)--(-3.629,2.258);
\draw(-3.697,2.248)--(-3.751,2.292);
\filldraw[fill opacity=0.8,fill=gray!20,draw=none](-1.971,2.137)--(-3.653,2.211)--(-3.642,2.258)--(-1.963,2.184)--cycle;
\draw(-1.971,2.137)--(-3.653,2.211)--(-3.642,2.258)--(-1.963,2.184);
\filldraw[fill opacity=0.8,fill=gray!20,draw=none](-1.697,1.82)--(-1.72,1.812)--(-1.743,1.791)--(-1.713,1.801)--cycle;
\draw(-1.697,1.82)--(-1.72,1.812);
\draw(-1.743,1.791)--(-1.713,1.801);
\filldraw[fill opacity=0.8,fill=gray!20,draw=none](-2.383,1.611)--(-2.382,1.611)--(-2.383,1.61)--cycle;
\draw(-2.382,1.611)--(-2.383,1.61);
\filldraw[fill opacity=0.8,fill=gray!20,draw=none](-2.39,1.585)--(-2.383,1.61)--(-2.382,1.611)--(-2.359,1.592)--cycle;
\draw(-2.383,1.61)--(-2.382,1.611);
\filldraw[fill opacity=0.8,fill=gray!20,draw=none](-2.39,1.585)--(-2.396,1.587)--(-2.383,1.61)--cycle;
\draw(-2.39,1.585)--(-2.396,1.587)--(-2.383,1.61);
\filldraw[fill opacity=0.8,fill=gray!20,draw=none](-2.434,1.573)--(-2.432,1.589)--(-2.396,1.587)--(-2.4,1.583)--cycle;
\draw(-2.434,1.573)--(-2.432,1.589)--(-2.396,1.587)--(-2.4,1.583);
\filldraw[fill opacity=0.8,fill=gray!20,draw=none](-2.4,1.583)--(-2.396,1.587)--(-2.391,1.585)--cycle;
\draw(-2.4,1.583)--(-2.396,1.587)--(-2.391,1.585);
\filldraw[fill opacity=0.8,fill=gray!20,draw=none](-2.416,1.566)--(-2.4,1.583)--(-2.391,1.585)--(-2.371,1.58)--(-2.403,1.563)--cycle;
\draw(-2.391,1.585)--(-2.371,1.58)--(-2.403,1.563)--(-2.416,1.566)--(-2.4,1.583);
\filldraw[fill opacity=0.8,fill=gray!20,draw=none](-2.434,1.567)--(-2.434,1.573)--(-2.4,1.583)--(-2.416,1.566)--cycle;
\draw(-2.4,1.583)--(-2.416,1.566)--(-2.434,1.567)--(-2.434,1.573);
\filldraw[fill opacity=0.8,fill=gray!20,draw=none](-2.39,1.585)--(-2.359,1.592)--(-2.371,1.58)--cycle;
\draw(-2.359,1.592)--(-2.371,1.58)--(-2.39,1.585);
\filldraw[fill opacity=0.8,fill=gray!20,draw=none](-1.949,1.729)--(-2.413,1.574)--(-2.437,1.593)--(-1.96,1.753)--cycle;
\draw(-1.949,1.729)--(-2.413,1.574)--(-2.437,1.593)--(-1.96,1.753);
\filldraw[fill opacity=0.8,fill=gray!20,draw=none](-3.016,2.024)--(-2.938,2.083)--(-2.93,2.021)--cycle;
\draw(-2.938,2.083)--(-2.93,2.021);
\filldraw[fill opacity=0.8,fill=gray!20,draw=none](-2.859,2.02)--(-2.854,2.079)--(-2.825,2.068)--(-2.819,2.019)--cycle;
\draw(-2.825,2.068)--(-2.819,2.019);
\filldraw[fill opacity=0.8,fill=gray!20,draw=none](-2.998,2.109)--(-2.959,2.099)--(-2.949,2.023)--(-3.056,2.026)--(-3.061,2.061)--cycle;
\draw(-2.959,2.099)--(-2.949,2.023);
\draw(-3.056,2.026)--(-3.061,2.061);
\filldraw[fill opacity=0.8,fill=gray!20,draw=none](-3.069,2.126)--(-3.061,2.061)--(-3.163,2.039)--(-3.176,2.139)--cycle;
\draw(-3.069,2.126)--(-3.061,2.061);
\draw(-3.163,2.039)--(-3.176,2.139);
\filldraw[fill opacity=0.8,fill=gray!20,draw=none](-2.913,2.082)--(-2.919,2.022)--(-2.949,2.023)--(-2.959,2.099)--cycle;
\draw(-2.949,2.023)--(-2.959,2.099);
\filldraw[fill opacity=0.8,fill=gray!20,draw=none](-2.854,2.079)--(-2.85,2.119)--(-2.83,2.103)--(-2.825,2.068)--cycle;
\draw(-2.83,2.103)--(-2.825,2.068);
\filldraw[fill opacity=0.8,fill=gray!20,draw=none](-2.919,2.022)--(-2.913,2.082)--(-2.868,2.064)--(-2.859,2.039)--(-2.857,2.021)--cycle;
\draw(-2.859,2.039)--(-2.857,2.021);
\filldraw[fill opacity=0.8,fill=gray!20,draw=none](-2.823,2.095)--(-2.815,2.062)--(-2.825,2.068)--(-2.83,2.103)--cycle;
\draw(-2.825,2.068)--(-2.83,2.103);
\filldraw[fill opacity=0.8,fill=gray!20,draw=none](-2.806,2.075)--(-2.794,2.049)--(-2.815,2.062)--(-2.823,2.095)--cycle;
\filldraw[fill opacity=0.8,fill=gray!20,draw=none](-2.806,2.075)--(-2.823,2.095)--(-2.83,2.123)--cycle;
\filldraw[fill opacity=0.8,fill=gray!20,draw=none](-2.913,2.082)--(-2.91,2.122)--(-2.878,2.097)--(-2.868,2.064)--cycle;
\filldraw[fill opacity=0.8,fill=gray!20,draw=none](-2.868,2.064)--(-2.878,2.097)--(-2.865,2.087)--(-2.862,2.062)--cycle;
\draw(-2.865,2.087)--(-2.862,2.062);
\filldraw[fill opacity=0.8,fill=gray!20,draw=none](-2.857,2.077)--(-2.844,2.051)--(-2.862,2.062)--(-2.865,2.087)--cycle;
\draw(-2.862,2.062)--(-2.865,2.087);
\filldraw[fill opacity=0.8,fill=gray!20,draw=none](-2.878,2.097)--(-2.888,2.126)--(-2.867,2.097)--(-2.865,2.087)--cycle;
\draw(-2.867,2.097)--(-2.865,2.087);
\filldraw[fill opacity=0.8,fill=gray!20,draw=none](-2.857,2.077)--(-2.865,2.087)--(-2.867,2.097)--cycle;
\draw(-2.865,2.087)--(-2.867,2.097);
\filldraw[fill opacity=0.8,fill=gray!20,draw=none](-3.298,2.122)--(-3.254,2.07)--(-3.25,2.035)--(-3.307,2.039)--(-3.316,2.115)--cycle;
\draw(-3.254,2.07)--(-3.25,2.035);
\draw(-3.307,2.039)--(-3.316,2.115);
\filldraw[fill opacity=0.8,fill=gray!20,draw=none](-3.313,2.037)--(-3.359,2.04)--(-3.367,2.102)--cycle;
\draw(-3.359,2.04)--(-3.367,2.102);
\filldraw[fill opacity=0.8,fill=gray!20,draw=none](-3.372,2.042)--(-3.382,2.102)--(-3.373,2.125)--(-3.367,2.102)--(-3.359,2.04)--cycle;
\draw(-3.367,2.102)--(-3.359,2.04);
\filldraw[fill opacity=0.8,fill=gray!20,draw=none](-3.312,2.039)--(-3.322,2.1)--(-3.316,2.115)--(-3.307,2.039)--cycle;
\draw(-3.316,2.115)--(-3.307,2.039);
\filldraw[fill opacity=0.8,fill=gray!20,draw=none](-3.328,2.084)--(-3.322,2.1)--(-3.312,2.039)--(-3.324,2.041)--(-3.326,2.059)--cycle;
\draw(-3.324,2.041)--(-3.326,2.059);
\filldraw[fill opacity=0.8,fill=gray!20,draw=none](-3.382,2.102)--(-3.372,2.042)--(-3.379,2.043)--(-3.386,2.093)--cycle;
\draw(-3.379,2.043)--(-3.386,2.093);
\filldraw[fill opacity=0.8,fill=gray!20,draw=none](-3.328,2.084)--(-3.331,2.117)--(-3.329,2.14)--(-3.322,2.1)--cycle;
\filldraw[fill opacity=0.8,fill=gray!20,draw=none](-3.331,2.117)--(-3.328,2.084)--(-3.329,2.083)--(-3.332,2.108)--cycle;
\draw(-3.329,2.083)--(-3.332,2.108);
\filldraw[fill opacity=0.8,fill=gray!20,draw=none](-3.334,2.145)--(-3.331,2.117)--(-3.332,2.108)--(-3.334,2.118)--cycle;
\draw(-3.332,2.108)--(-3.334,2.118);
\filldraw[fill opacity=0.8,fill=gray!20,draw=none](-3.321,2.072)--(-3.328,2.098)--(-3.332,2.108)--(-3.329,2.083)--cycle;
\draw(-3.332,2.108)--(-3.329,2.083);
\filldraw[fill opacity=0.8,fill=gray!20,draw=none](-3.381,2.087)--(-3.387,2.12)--(-3.39,2.128)--(-3.386,2.093)--cycle;
\draw(-3.39,2.128)--(-3.386,2.093);
\filldraw[fill opacity=0.8,fill=gray!20,draw=none](-3.328,2.098)--(-3.334,2.118)--(-3.332,2.108)--cycle;
\draw(-3.334,2.118)--(-3.332,2.108);
\filldraw[fill opacity=0.8,fill=gray!20,draw=none](-3.372,2.074)--(-3.379,2.1)--(-3.387,2.12)--(-3.381,2.087)--cycle;
\filldraw[fill opacity=0.8,fill=gray!20,draw=none](-3.379,2.1)--(-3.392,2.148)--(-3.387,2.12)--cycle;
\filldraw[fill opacity=0.8,fill=gray!20,draw=none](-3.549,2.108)--(-3.576,2.072)--(-3.626,2.094)--(-3.583,2.13)--(-3.546,2.114)--cycle;
\draw(-3.576,2.072)--(-3.626,2.094)--(-3.583,2.13)--(-3.546,2.114);
\filldraw[fill opacity=0.8,fill=gray!20,draw=none](-3.545,2.113)--(-3.553,2.116)--(-3.535,2.154)--cycle;
\draw(-3.545,2.113)--(-3.553,2.116);
\filldraw[fill opacity=0.8,fill=gray!20,draw=none](-3.549,2.108)--(-3.546,2.114)--(-3.545,2.113)--cycle;
\draw(-3.546,2.114)--(-3.545,2.113);
\filldraw[fill opacity=0.8,fill=gray!20,draw=none](-1.975,2.038)--(-3.647,2.112)--(-3.655,2.159)--(-1.975,2.086)--cycle;
\draw(-1.975,2.038)--(-3.647,2.112)--(-3.655,2.159)--(-1.975,2.086);
\filldraw[fill opacity=0.5,fill=gray!20,draw=none](-1.914,1.982)--(-1.936,1.982)--(-1.934,1.997)--cycle;
\draw(-1.936,1.982)--(-1.934,1.997);
\filldraw[fill opacity=0.8,fill=gray!20,draw=none](-1.734,1.975)--(-1.951,1.984)--(-1.962,1.984)--(-1.755,1.974)--cycle;
\draw(-1.734,1.975)--(-1.951,1.984);
\draw(-1.962,1.984)--(-1.755,1.974);
\filldraw[fill opacity=0.5,fill=gray!20](-1.684,2.998)--(-1.739,3.022)--(-1.432,3.492)--(-1.375,3.471)--cycle;
\filldraw[fill opacity=0.5,fill=gray!20,draw=none](-1.939,2.28)--(-1.956,2.298)--(-1.932,2.492)--(-1.9,2.476)--cycle;
\draw(-1.956,2.298)--(-1.932,2.492)--(-1.9,2.476);
\filldraw[fill opacity=0.8,fill=gray!20,draw=none](-1.734,2.064)--(-1.765,2.053)--(-1.741,2.034)--(-1.713,2.044)--cycle;
\draw(-1.734,2.064)--(-1.765,2.053);
\draw(-1.741,2.034)--(-1.713,2.044);
\filldraw[fill opacity=0.5,fill=gray!20,draw=none](-1.943,2.221)--(-1.959,2.184)--(-1.953,2.21)--cycle;
\filldraw[fill opacity=0.8,fill=gray!20,draw=none](-1.959,2.184)--(-1.963,2.184)--(-1.953,2.221)--(-1.943,2.221)--cycle;
\draw(-1.959,2.184)--(-1.963,2.184);
\draw(-1.953,2.221)--(-1.943,2.221);
\filldraw[fill opacity=0.8,fill=gray!20,draw=none](-2.852,2.28)--(-2.848,2.245)--(-2.865,2.237)--(-2.879,2.28)--cycle;
\draw(-2.852,2.28)--(-2.848,2.245);
\filldraw[fill opacity=0.8,fill=gray!20,draw=none](-2.983,2.284)--(-2.978,2.249)--(-3.02,2.285)--cycle;
\draw(-2.983,2.284)--(-2.978,2.249);
\filldraw[fill opacity=0.8,fill=gray!20,draw=none](-2.83,2.26)--(-2.843,2.223)--(-2.843,2.249)--cycle;
\filldraw[fill opacity=0.8,fill=gray!20,draw=none](-2.843,2.28)--(-2.843,2.249)--(-2.848,2.245)--(-2.852,2.28)--cycle;
\draw(-2.848,2.245)--(-2.852,2.28);
\filldraw[fill opacity=0.8,fill=gray!20,draw=none](-2.823,2.279)--(-2.83,2.26)--(-2.843,2.249)--(-2.843,2.28)--cycle;
\filldraw[fill opacity=0.8,fill=gray!20,draw=none](-2.895,2.282)--(-2.898,2.251)--(-2.925,2.239)--(-2.939,2.283)--cycle;
\filldraw[fill opacity=0.8,fill=gray!20,draw=none](-2.9,2.226)--(-2.898,2.251)--(-2.887,2.256)--(-2.886,2.247)--cycle;
\draw(-2.887,2.256)--(-2.886,2.247);
\filldraw[fill opacity=0.8,fill=gray!20,draw=none](-2.898,2.251)--(-2.895,2.282)--(-2.89,2.282)--(-2.887,2.256)--cycle;
\draw(-2.89,2.282)--(-2.887,2.256);
\filldraw[fill opacity=0.8,fill=gray!20,draw=none](-2.881,2.262)--(-2.886,2.247)--(-2.887,2.256)--cycle;
\draw(-2.886,2.247)--(-2.887,2.256);
\filldraw[fill opacity=0.8,fill=gray!20,draw=none](-2.859,2.329)--(-2.852,2.28)--(-2.879,2.28)--(-2.901,2.345)--cycle;
\draw(-2.859,2.329)--(-2.852,2.28);
\filldraw[fill opacity=0.8,fill=gray!20,draw=none](-2.874,2.282)--(-2.881,2.262)--(-2.887,2.256)--(-2.89,2.282)--cycle;
\draw(-2.887,2.256)--(-2.89,2.282);
\filldraw[fill opacity=0.8,fill=gray!20,draw=none](-3.32,2.298)--(-3.336,2.265)--(-3.34,2.3)--cycle;
\draw(-3.336,2.265)--(-3.34,2.3);
\filldraw[fill opacity=0.8,fill=gray!20,draw=none](-3.348,2.301)--(-3.344,2.258)--(-3.351,2.271)--(-3.356,2.302)--cycle;
\filldraw[fill opacity=0.8,fill=gray!20,draw=none](-3.351,2.271)--(-3.347,2.245)--(-3.353,2.267)--(-3.354,2.277)--cycle;
\draw(-3.353,2.267)--(-3.354,2.277);
\filldraw[fill opacity=0.8,fill=gray!20,draw=none](-3.356,2.302)--(-3.351,2.271)--(-3.354,2.277)--(-3.357,2.302)--cycle;
\draw(-3.354,2.277)--(-3.357,2.302);
\filldraw[fill opacity=0.8,fill=gray!20,draw=none](-3.352,2.283)--(-3.354,2.277)--(-3.353,2.267)--cycle;
\draw(-3.354,2.277)--(-3.353,2.267);
\filldraw[fill opacity=0.8,fill=gray!20,draw=none](-3.354,2.36)--(-3.348,2.301)--(-3.356,2.302)--(-3.36,2.325)--(-3.362,2.34)--cycle;
\draw(-3.36,2.325)--(-3.362,2.34)--(-3.354,2.36);
\filldraw[fill opacity=0.8,fill=gray!20,draw=none](-3.352,2.283)--(-3.351,2.303)--(-3.357,2.302)--(-3.354,2.277)--cycle;
\draw(-3.357,2.302)--(-3.354,2.277);
\filldraw[fill opacity=0.8,fill=gray!20,draw=none](-3.406,2.274)--(-3.404,2.248)--(-3.402,2.285)--cycle;
\filldraw[fill opacity=0.8,fill=gray!20,draw=none](-3.413,2.304)--(-3.408,2.269)--(-3.406,2.274)--(-3.409,2.304)--cycle;
\draw(-3.413,2.304)--(-3.408,2.269);
\filldraw[fill opacity=0.8,fill=gray!20,draw=none](-3.396,2.302)--(-3.408,2.304)--(-3.414,2.368)--(-3.404,2.393)--(-3.396,2.331)--cycle;
\draw(-3.404,2.393)--(-3.396,2.331);
\filldraw[fill opacity=0.8,fill=gray!20,draw=none](-3.409,2.304)--(-3.406,2.274)--(-3.402,2.285)--(-3.401,2.305)--cycle;
\filldraw[fill opacity=0.8,fill=gray!20,draw=none](-3.546,2.31)--(-3.527,2.279)--(-3.527,2.277)--(-3.555,2.289)--(-3.569,2.31)--cycle;
\draw(-3.527,2.277)--(-3.555,2.289)--(-3.569,2.31);
\filldraw[fill opacity=0.8,fill=gray!20,draw=none](-3.523,2.253)--(-3.525,2.276)--(-3.52,2.274)--cycle;
\draw(-3.525,2.276)--(-3.52,2.274);
\filldraw[fill opacity=0.8,fill=gray!20,draw=none](-3.527,2.279)--(-3.525,2.276)--(-3.527,2.277)--cycle;
\draw(-3.525,2.276)--(-3.527,2.277);
\filldraw[fill opacity=0.8,fill=gray!20,draw=none](-3.527,2.29)--(-3.52,2.274)--(-3.525,2.276)--(-3.527,2.279)--cycle;
\draw(-3.52,2.274)--(-3.525,2.276);
\filldraw[fill opacity=0.8,fill=gray!20,draw=none](-3.536,2.311)--(-3.527,2.29)--(-3.527,2.279)--(-3.546,2.31)--cycle;
\filldraw[fill opacity=0.8,fill=gray!20,draw=none](-3.647,2.284)--(-3.6,2.293)--(-3.629,2.258)--cycle;
\filldraw[fill opacity=0.8,fill=gray!20,draw=none](-3.694,2.333)--(-3.682,2.335)--(-3.629,2.258)--(-3.71,2.325)--cycle;
\draw(-3.629,2.258)--(-3.71,2.325);
\filldraw[fill opacity=0.8,fill=gray!20,draw=none](-3.647,2.284)--(-3.682,2.335)--(-3.655,2.34)--(-3.6,2.293)--cycle;
\draw(-3.655,2.34)--(-3.6,2.293);
\filldraw[fill opacity=0.8,fill=gray!20,draw=none](-1.963,2.184)--(-3.642,2.258)--(-3.624,2.294)--(-1.953,2.221)--cycle;
\draw(-1.963,2.184)--(-3.642,2.258)--(-3.624,2.294)--(-1.953,2.221);
\filldraw[fill opacity=0.8,fill=gray!20,draw=none](-2.443,1.82)--(-2.462,1.831)--(-2.442,1.836)--(-2.432,1.837)--(-2.431,1.83)--cycle;
\draw(-2.442,1.836)--(-2.432,1.837)--(-2.431,1.83);
\filldraw[fill opacity=0.8,fill=gray!20,draw=none](-2.462,1.831)--(-2.443,1.82)--(-2.483,1.818)--(-2.475,1.828)--cycle;
\draw(-2.443,1.82)--(-2.483,1.818)--(-2.475,1.828);
\filldraw[fill opacity=0.8,fill=gray!20,draw=none](-2.431,1.83)--(-2.432,1.837)--(-2.427,1.836)--cycle;
\draw(-2.431,1.83)--(-2.432,1.837)--(-2.427,1.836);
\filldraw[fill opacity=0.8,fill=gray!20,draw=none](-2.428,1.838)--(-2.427,1.836)--(-2.432,1.837)--(-2.433,1.838)--cycle;
\draw(-2.427,1.836)--(-2.432,1.837)--(-2.433,1.838);
\filldraw[fill opacity=0.8,fill=gray!20,draw=none](-1.993,1.971)--(-2.495,1.802)--(-2.481,1.821)--(-1.988,1.986)--cycle;
\draw(-1.993,1.971)--(-2.495,1.802)--(-2.481,1.821)--(-1.988,1.986);
\filldraw[fill opacity=0.5,fill=gray!20](-1.299,3.434)--(-1.375,3.471)--(-.972,3.853)--(-.899,3.812)--cycle;
\filldraw[fill opacity=0.8,fill=gray!20,draw=none](-3.055,1.995)--(-3.031,1.81)--(-3.154,1.786)--(-3.177,1.968)--cycle;
\draw(-3.055,1.995)--(-3.031,1.81)--(-3.154,1.786)--(-3.177,1.968);
\filldraw[fill opacity=0.8,fill=gray!20,draw=none](-2.864,1.809)--(-2.921,1.804)--(-3.037,1.808)--(-3.031,1.81)--(-2.911,1.82)--cycle;
\draw(-3.037,1.808)--(-3.031,1.81)--(-2.911,1.82);
\filldraw[fill opacity=0.8,fill=gray!20,draw=none](-2.913,1.89)--(-2.904,1.82)--(-3.031,1.81)--(-3.037,1.86)--cycle;
\draw(-2.913,1.89)--(-2.904,1.82)--(-3.031,1.81)--(-3.037,1.86);
\filldraw[fill opacity=0.8,fill=gray!20,draw=none](-2.921,1.804)--(-3.159,1.784)--(-3.154,1.786)--(-3.037,1.808)--cycle;
\draw(-3.159,1.784)--(-3.154,1.786)--(-3.037,1.808);
\filldraw[fill opacity=0.8,fill=gray!20,draw=none](-3.016,2.024)--(-2.93,2.021)--(-2.913,1.89)--(-3.037,1.86)--(-3.055,1.995)--cycle;
\draw(-2.93,2.021)--(-2.913,1.89);
\draw(-3.037,1.86)--(-3.055,1.995);
\filldraw[fill opacity=0.8,fill=gray!20,draw=none](-2.931,1.881)--(-2.866,1.371)--(-2.961,1.283)--(-3.002,1.6)--cycle;
\draw(-2.931,1.881)--(-2.866,1.371);
\draw(-2.961,1.283)--(-3.002,1.6);
\filldraw[fill opacity=0.8,fill=gray!20,draw=none](-3.129,1.775)--(-3.058,1.218)--(-3.142,1.19)--(-3.213,1.747)--cycle;
\draw(-3.129,1.775)--(-3.058,1.218)--(-3.142,1.19)--(-3.213,1.747);
\filldraw[fill opacity=0.8,fill=gray!20,draw=none](-3.265,1.715)--(-3.249,1.753)--(-3.159,1.784)--(-3.151,1.785)--cycle;
\draw(-3.249,1.753)--(-3.159,1.784);
\filldraw[fill opacity=0.8,fill=gray!20,draw=none](-3.236,1.733)--(-3.195,1.608)--(-3.142,1.19)--(-3.194,1.158)--(-3.265,1.715)--cycle;
\draw(-3.195,1.608)--(-3.142,1.19)--(-3.194,1.158)--(-3.265,1.715);
\filldraw[fill opacity=0.8,fill=gray!20,draw=none](-2.864,1.809)--(-2.911,1.82)--(-2.904,1.82)--(-2.793,1.816)--(-2.789,1.815)--cycle;
\draw(-2.911,1.82)--(-2.904,1.82)--(-2.793,1.816)--(-2.789,1.815);
\filldraw[fill opacity=0.8,fill=gray!20,draw=none](-2.869,1.909)--(-2.877,1.819)--(-2.904,1.82)--(-2.913,1.89)--cycle;
\draw(-2.877,1.819)--(-2.904,1.82)--(-2.913,1.89);
\filldraw[fill opacity=0.8,fill=gray!20,draw=none](-2.869,1.909)--(-2.809,1.937)--(-2.793,1.816)--(-2.877,1.819)--cycle;
\draw(-2.809,1.937)--(-2.793,1.816)--(-2.877,1.819);
\filldraw[fill opacity=0.8,fill=gray!20,draw=none](-2.859,2.02)--(-2.869,1.909)--(-2.913,1.89)--(-2.93,2.021)--cycle;
\draw(-2.913,1.89)--(-2.93,2.021);
\filldraw[fill opacity=0.8,fill=gray!20,draw=none](-2.829,1.8)--(-2.778,1.402)--(-2.885,1.523)--(-2.921,1.804)--cycle;
\draw(-2.829,1.8)--(-2.778,1.402);
\draw(-2.885,1.523)--(-2.921,1.804);
\filldraw[fill opacity=0.8,fill=gray!20,draw=none](-3.027,1.795)--(-3.002,1.6)--(-3.095,1.504)--(-3.129,1.775)--cycle;
\draw(-3.027,1.795)--(-3.002,1.6);
\draw(-3.095,1.504)--(-3.129,1.775);
\filldraw[fill opacity=0.8,fill=gray!20,draw=none](-2.935,1.909)--(-2.931,1.881)--(-3.002,1.6)--(-3.038,1.885)--cycle;
\draw(-2.935,1.909)--(-2.931,1.881);
\draw(-3.002,1.6)--(-3.038,1.885);
\filldraw[fill opacity=0.8,fill=gray!20,draw=none](-2.422,1.653)--(-2.429,1.657)--(-2.428,1.691)--(-2.359,1.686)--(-2.367,1.649)--cycle;
\draw(-2.429,1.657)--(-2.428,1.691)--(-2.359,1.686)--(-2.367,1.649)--(-2.422,1.653);
\filldraw[fill opacity=0.8,fill=gray!20,draw=none](-2.452,1.652)--(-2.428,1.685)--(-2.429,1.653)--cycle;
\draw(-2.428,1.685)--(-2.429,1.653)--(-2.452,1.652);
\filldraw[fill opacity=0.8,fill=gray!20,draw=none](-2.422,1.653)--(-2.429,1.653)--(-2.429,1.657)--cycle;
\draw(-2.422,1.653)--(-2.429,1.653)--(-2.429,1.657);
\filldraw[fill opacity=0.8,fill=gray!20,draw=none](-2.413,1.644)--(-2.422,1.653)--(-2.412,1.652)--cycle;
\draw(-2.422,1.653)--(-2.412,1.652);
\filldraw[fill opacity=0.8,fill=gray!20,draw=none](-2.41,1.645)--(-2.461,1.628)--(-2.482,1.674)--(-2.426,1.692)--cycle;
\draw(-2.41,1.645)--(-2.461,1.628)--(-2.482,1.674)--(-2.426,1.692);
\filldraw[fill opacity=0.8,fill=gray!20](-3.003,.941)--(-3.007,.998)--(-2.895,1.003)--(-2.896,.946)--cycle;
\filldraw[fill opacity=0.8,fill=gray!20](-3.007,.998)--(-3.003,1.052)--(-2.896,1.057)--(-2.895,1.003)--cycle;
\filldraw[fill opacity=0.8,fill=gray!20](-2.896,1.057)--(-2.897,1.104)--(-2.804,1.097)--(-2.792,1.05)--cycle;
\filldraw[fill opacity=0.8,fill=gray!20](-3.003,1.052)--(-2.994,1.1)--(-2.897,1.104)--(-2.896,1.057)--cycle;
\filldraw[fill opacity=0.8,fill=gray!20](-3.086,1.036)--(-3.067,1.086)--(-2.994,1.1)--(-3.003,1.052)--cycle;
\filldraw[fill opacity=0.8,fill=gray!20](-2.994,1.1)--(-2.978,1.137)--(-2.899,1.141)--(-2.897,1.104)--cycle;
\filldraw[fill opacity=0.8,fill=gray!20,draw=none](-3.313,2.037)--(-3.287,2.005)--(-3.254,1.752)--(-3.317,1.713)--(-3.359,2.04)--cycle;
\draw(-3.287,2.005)--(-3.254,1.752)--(-3.317,1.713)--(-3.359,2.04);
\filldraw[fill opacity=0.8,fill=gray!20,draw=none](-3.232,1.893)--(-3.195,1.608)--(-3.289,1.897)--(-3.292,1.925)--cycle;
\draw(-3.232,1.893)--(-3.195,1.608);
\draw(-3.289,1.897)--(-3.292,1.925);
\filldraw[fill opacity=0.8,fill=gray!20,draw=none](-3.262,1.717)--(-3.151,1.785)--(-2.921,1.804)--(-2.874,1.802)--cycle;
\filldraw[fill opacity=0.8,fill=gray!20,draw=none](-3.163,2.039)--(-3.129,1.775)--(-3.213,1.747)--(-3.254,2.07)--cycle;
\draw(-3.163,2.039)--(-3.129,1.775);
\draw(-3.213,1.747)--(-3.254,2.07);
\filldraw[fill opacity=0.8,fill=gray!20,draw=none](-3.016,2.024)--(-3.055,1.995)--(-3.058,2.025)--cycle;
\draw(-3.055,1.995)--(-3.058,2.025);
\filldraw[fill opacity=0.8,fill=gray!20,draw=none](-3.061,2.061)--(-3.038,1.885)--(-3.143,1.879)--(-3.163,2.039)--cycle;
\draw(-3.061,2.061)--(-3.038,1.885);
\draw(-3.143,1.879)--(-3.163,2.039);
\filldraw[fill opacity=0.8,fill=gray!20,draw=none](-2.815,2.062)--(-2.804,2.018)--(-2.819,2.019)--(-2.825,2.068)--cycle;
\draw(-2.819,2.019)--(-2.825,2.068);
\filldraw[fill opacity=0.8,fill=gray!20,draw=none](-2.794,2.049)--(-2.779,2.018)--(-2.804,2.018)--(-2.815,2.062)--cycle;
\filldraw[fill opacity=0.8,fill=gray!20,draw=none](-2.776,2.038)--(-2.794,2.049)--(-2.806,2.075)--cycle;
\filldraw[fill opacity=0.8,fill=gray!20,draw=none](-2.844,2.051)--(-2.829,2.02)--(-2.855,2.021)--(-2.859,2.039)--(-2.862,2.062)--cycle;
\draw(-2.859,2.039)--(-2.862,2.062);
\filldraw[fill opacity=0.8,fill=gray!20,draw=none](-2.776,2.038)--(-2.759,2.018)--(-2.779,2.018)--(-2.794,2.049)--cycle;
\filldraw[fill opacity=0.8,fill=gray!20,draw=none](-2.868,2.064)--(-2.862,2.062)--(-2.859,2.039)--cycle;
\draw(-2.862,2.062)--(-2.859,2.039);
\filldraw[fill opacity=0.8,fill=gray!20,draw=none](-2.826,2.04)--(-2.844,2.051)--(-2.857,2.077)--cycle;
\filldraw[fill opacity=0.8,fill=gray!20,draw=none](-2.826,2.04)--(-2.81,2.02)--(-2.829,2.02)--(-2.844,2.051)--cycle;
\filldraw[fill opacity=0.8,fill=gray!20,draw=none](-3.313,2.042)--(-3.321,2.072)--(-3.329,2.083)--(-3.326,2.059)--(-3.323,2.041)--cycle;
\draw(-3.329,2.083)--(-3.326,2.059);
\filldraw[fill opacity=0.8,fill=gray!20,draw=none](-3.373,2.043)--(-3.381,2.087)--(-3.386,2.093)--(-3.379,2.043)--cycle;
\draw(-3.386,2.093)--(-3.379,2.043);
\filldraw[fill opacity=0.8,fill=gray!20,draw=none](-3.328,2.084)--(-3.326,2.059)--(-3.329,2.083)--cycle;
\draw(-3.326,2.059)--(-3.329,2.083);
\filldraw[fill opacity=0.8,fill=gray!20,draw=none](-3.313,2.062)--(-3.328,2.098)--(-3.321,2.072)--cycle;
\filldraw[fill opacity=0.8,fill=gray!20,draw=none](-3.363,2.044)--(-3.372,2.074)--(-3.381,2.087)--(-3.373,2.043)--cycle;
\filldraw[fill opacity=0.8,fill=gray!20,draw=none](-3.305,2.042)--(-3.313,2.062)--(-3.321,2.072)--(-3.313,2.042)--cycle;
\filldraw[fill opacity=0.8,fill=gray!20,draw=none](-3.364,2.064)--(-3.379,2.1)--(-3.372,2.074)--cycle;
\filldraw[fill opacity=0.8,fill=gray!20,draw=none](-3.356,2.044)--(-3.364,2.064)--(-3.372,2.074)--(-3.363,2.044)--cycle;
\filldraw[fill opacity=0.8,fill=gray!20,draw=none](-3.585,2.054)--(-3.624,2.055)--(-3.676,2.078)--(-3.626,2.094)--(-3.564,2.067)--cycle;
\draw(-3.624,2.055)--(-3.676,2.078)--(-3.626,2.094)--(-3.564,2.067);
\filldraw[fill opacity=0.8,fill=gray!20,draw=none](-3.562,2.072)--(-3.568,2.083)--(-3.549,2.108)--cycle;
\filldraw[fill opacity=0.8,fill=gray!20,draw=none](-3.562,2.072)--(-3.564,2.067)--(-3.576,2.072)--(-3.568,2.083)--cycle;
\draw(-3.564,2.067)--(-3.576,2.072);
\filldraw[fill opacity=0.8,fill=gray!20,draw=none](-1.97,2.003)--(-3.632,2.076)--(-3.647,2.112)--(-1.975,2.038)--cycle;
\draw(-1.97,2.003)--(-3.632,2.076)--(-3.647,2.112)--(-1.975,2.038);
\filldraw[fill opacity=0.8,fill=gray!20,draw=none](-1.765,2.053)--(-1.961,1.988)--(-1.96,1.984)--(-1.897,1.982)--(-1.741,2.034)--cycle;
\draw(-1.765,2.053)--(-1.961,1.988);
\draw(-1.897,1.982)--(-1.741,2.034);
\filldraw[fill opacity=0.8,fill=gray!20,draw=none](-1.898,1.739)--(-1.906,1.736)--(-1.923,1.738)--cycle;
\draw(-1.898,1.739)--(-1.906,1.736);
\filldraw[fill opacity=0.8,fill=gray!20,draw=none](-1.906,1.736)--(-1.941,1.725)--(-1.949,1.729)--(-1.923,1.738)--cycle;
\draw(-1.906,1.736)--(-1.941,1.725);
\draw(-1.949,1.729)--(-1.923,1.738);
\filldraw[fill opacity=0.5,fill=gray!20](-1.878,2.466)--(-1.932,2.492)--(-1.739,3.022)--(-1.684,2.998)--cycle;
\filldraw[fill opacity=0.8,fill=gray!20,draw=none](-1.953,2.221)--(-2.796,2.258)--(-2.788,2.278)--(-1.941,2.241)--cycle;
\draw(-1.953,2.221)--(-2.796,2.258);
\draw(-2.788,2.278)--(-1.941,2.241);
\filldraw[fill opacity=0.5,fill=gray!20,draw=none](-1.879,2.221)--(-1.86,2.184)--(-1.85,2.137)--(-1.85,2.085)--(-1.861,2.038)--(-1.882,2.002)--(-1.883,2)--(-1.934,1.997)--(-1.905,2.241)--cycle;
\draw(-1.934,1.997)--(-1.905,2.241);
\filldraw[fill opacity=0.8,fill=gray!20,draw=none](-1.941,2.241)--(-2.788,2.278)--(-2.776,2.279)--(-1.931,2.242)--cycle;
\draw(-1.941,2.241)--(-2.788,2.278);
\draw(-2.776,2.279)--(-1.931,2.242);
\filldraw[fill opacity=0.8,fill=gray!20,draw=none](-1.964,1.989)--(-1.971,1.992)--(-1.994,1.984)--(-1.972,1.984)--cycle;
\draw(-1.971,1.992)--(-1.994,1.984);
\filldraw[fill opacity=0.8,fill=gray!20,draw=none](-1.964,1.989)--(-1.972,1.984)--(-2.743,2.018)--(-2.757,2.037)--(-1.97,2.003)--cycle;
\draw(-1.972,1.984)--(-2.743,2.018);
\draw(-2.757,2.037)--(-1.97,2.003);
\filldraw[fill opacity=0.8,fill=gray!20,draw=none](-2.743,2.018)--(-2.776,2.038)--(-2.757,2.037)--cycle;
\draw(-2.776,2.038)--(-2.757,2.037);
\filldraw[fill opacity=0.8,fill=gray!20,draw=none](-2.796,2.258)--(-2.83,2.26)--(-2.807,2.279)--(-2.788,2.278)--cycle;
\draw(-2.796,2.258)--(-2.83,2.26);
\draw(-2.807,2.279)--(-2.788,2.278);
\filldraw[fill opacity=0.8,fill=gray!20,draw=none](-2.788,2.278)--(-2.807,2.279)--(-2.776,2.279)--cycle;
\draw(-2.788,2.278)--(-2.807,2.279);
\filldraw[fill opacity=0.8,fill=gray!20,draw=none](-2.843,2.223)--(-2.815,2.303)--(-2.776,2.279)--(-2.743,2.018)--(-2.776,2.038)--(-2.806,2.075)--(-2.83,2.123)--(-2.835,2.146)--(-2.843,2.202)--cycle;
\draw(-2.776,2.279)--(-2.743,2.018);
\draw(-2.835,2.146)--(-2.843,2.202);
\filldraw[fill opacity=0.8,fill=gray!20,draw=none](-3.098,2.353)--(-3.09,2.288)--(-3.195,2.292)--(-3.208,2.392)--cycle;
\draw(-3.098,2.353)--(-3.09,2.288);
\draw(-3.195,2.292)--(-3.208,2.392);
\filldraw[fill opacity=0.8,fill=gray!20,draw=none](-3.02,2.285)--(-3.09,2.288)--(-3.098,2.353)--cycle;
\draw(-3.09,2.288)--(-3.098,2.353);
\filldraw[fill opacity=0.8,fill=gray!20,draw=none](-3.168,3.024)--(-3.017,2.702)--(-2.967,2.313)--(-3.111,2.437)--(-3.186,3.023)--cycle;
\draw(-3.017,2.702)--(-2.967,2.313);
\draw(-3.111,2.437)--(-3.186,3.023)--(-3.168,3.024);
\filldraw[fill opacity=0.8,fill=gray!20,draw=none](-2.992,2.361)--(-2.983,2.284)--(-3.02,2.285)--(-3.098,2.353)--(-3.103,2.387)--cycle;
\draw(-2.992,2.361)--(-2.983,2.284);
\draw(-3.098,2.353)--(-3.103,2.387);
\filldraw[fill opacity=0.8,fill=gray!20,draw=none](-2.96,2.348)--(-2.939,2.283)--(-2.983,2.284)--(-2.992,2.361)--cycle;
\draw(-2.983,2.284)--(-2.992,2.361);
\filldraw[fill opacity=0.8,fill=gray!20,draw=none](-2.843,2.202)--(-2.835,2.146)--(-2.845,2.175)--cycle;
\draw(-2.843,2.202)--(-2.835,2.146);
\filldraw[fill opacity=0.8,fill=gray!20,draw=none](-2.743,2.018)--(-2.759,2.018)--(-2.776,2.038)--cycle;
\filldraw[fill opacity=0.8,fill=gray!20,draw=none](-2.743,2.018)--(-2.793,2.02)--(-2.808,2.039)--(-2.776,2.038)--cycle;
\draw(-2.743,2.018)--(-2.793,2.02);
\draw(-2.808,2.039)--(-2.776,2.038);
\filldraw[fill opacity=0.8,fill=gray!20,draw=none](-2.843,2.32)--(-2.843,2.28)--(-2.852,2.28)--(-2.859,2.329)--cycle;
\draw(-2.852,2.28)--(-2.859,2.329);
\filldraw[fill opacity=0.8,fill=gray!20,draw=none](-2.793,2.02)--(-2.826,2.04)--(-2.808,2.039)--cycle;
\draw(-2.826,2.04)--(-2.808,2.039);
\filldraw[fill opacity=0.8,fill=gray!20,draw=none](-2.815,2.303)--(-2.823,2.279)--(-2.843,2.28)--(-2.843,2.32)--cycle;
\filldraw[fill opacity=0.8,fill=gray!20,draw=none](-2.895,2.282)--(-2.939,2.283)--(-2.96,2.348)--(-2.896,2.323)--(-2.893,2.305)--cycle;
\draw(-2.896,2.323)--(-2.893,2.305);
\filldraw[fill opacity=0.8,fill=gray!20,draw=none](-2.895,2.282)--(-2.893,2.305)--(-2.89,2.282)--cycle;
\draw(-2.893,2.305)--(-2.89,2.282);
\filldraw[fill opacity=0.8,fill=gray!20,draw=none](-2.893,2.322)--(-2.866,2.305)--(-2.874,2.282)--(-2.89,2.282)--(-2.893,2.305)--cycle;
\draw(-2.89,2.282)--(-2.893,2.305);
\filldraw[fill opacity=0.8,fill=gray!20,draw=none](-2.83,2.26)--(-2.846,2.26)--(-2.839,2.281)--(-2.807,2.279)--cycle;
\draw(-2.83,2.26)--(-2.846,2.26);
\draw(-2.839,2.281)--(-2.807,2.279);
\filldraw[fill opacity=0.8,fill=gray!20,draw=none](-2.846,2.26)--(-2.881,2.262)--(-2.857,2.281)--(-2.839,2.281)--cycle;
\draw(-2.846,2.26)--(-2.881,2.262);
\draw(-2.857,2.281)--(-2.839,2.281);
\filldraw[fill opacity=0.8,fill=gray!20,draw=none](-2.78,2.279)--(-2.807,2.279)--(-2.839,2.281)--(-2.827,2.281)--(-2.781,2.279)--cycle;
\draw(-2.807,2.279)--(-2.839,2.281);
\draw(-2.827,2.281)--(-2.781,2.279);
\filldraw[fill opacity=0.8,fill=gray!20,draw=none](-2.881,2.262)--(-2.857,2.281)--(-2.827,2.281)--(-2.793,2.02)--(-2.826,2.04)--(-2.857,2.077)--(-2.867,2.097)--(-2.886,2.247)--cycle;
\draw(-2.827,2.281)--(-2.793,2.02);
\draw(-2.867,2.097)--(-2.886,2.247);
\filldraw[fill opacity=0.8,fill=gray!20,draw=none](-2.762,2.26)--(-2.747,2.223)--(-2.734,2.176)--(-2.73,2.153)--(-2.723,2.096)--(-2.721,2.076)--(-2.723,2.039)--(-2.731,2.018)--(-2.743,2.018)--(-2.776,2.279)--cycle;
\draw(-2.73,2.153)--(-2.723,2.096);
\draw(-2.743,2.018)--(-2.776,2.279);
\filldraw[fill opacity=0.8,fill=gray!20,draw=none](-2.857,2.281)--(-2.881,2.262)--(-2.874,2.282)--cycle;
\filldraw[fill opacity=0.8,fill=gray!20,draw=none](-3.32,2.298)--(-3.291,2.361)--(-3.283,2.296)--cycle;
\draw(-3.291,2.361)--(-3.283,2.296);
\filldraw[fill opacity=0.8,fill=gray!20,draw=none](-3.327,2.384)--(-3.296,2.396)--(-3.291,2.361)--(-3.32,2.298)--(-3.34,2.3)--(-3.349,2.371)--cycle;
\draw(-3.296,2.396)--(-3.291,2.361);
\draw(-3.34,2.3)--(-3.349,2.371)--(-3.327,2.384);
\filldraw[fill opacity=0.8,fill=gray!20,draw=none](-3.396,2.302)--(-3.396,2.331)--(-3.392,2.301)--cycle;
\draw(-3.396,2.331)--(-3.392,2.301);
\filldraw[fill opacity=0.8,fill=gray!20,draw=none](-3.348,2.301)--(-3.354,2.36)--(-3.349,2.371)--(-3.34,2.3)--cycle;
\draw(-3.354,2.36)--(-3.349,2.371)--(-3.34,2.3);
\filldraw[fill opacity=0.8,fill=gray!20,draw=none](-2.881,2.262)--(-3.352,2.283)--(-3.344,2.303)--(-2.857,2.281)--cycle;
\draw(-2.881,2.262)--(-3.352,2.283);
\draw(-3.344,2.303)--(-2.857,2.281);
\filldraw[fill opacity=0.8,fill=gray!20,draw=none](-2.812,2.262)--(-2.797,2.225)--(-2.792,2.206)--(-2.773,2.056)--(-2.774,2.041)--(-2.781,2.021)--(-2.793,2.02)--(-2.827,2.281)--cycle;
\draw(-2.792,2.206)--(-2.773,2.056);
\draw(-2.793,2.02)--(-2.827,2.281);
\filldraw[fill opacity=0.8,fill=gray!20,draw=none](-2.857,2.281)--(-2.874,2.282)--(-2.866,2.305)--(-2.845,2.292)--cycle;
\filldraw[fill opacity=0.8,fill=gray!20,draw=none](-2.857,2.281)--(-2.845,2.292)--(-2.827,2.281)--cycle;
\filldraw[fill opacity=0.8,fill=gray!20,draw=none](-2.839,2.281)--(-3.344,2.303)--(-3.332,2.303)--(-2.827,2.281)--cycle;
\draw(-2.839,2.281)--(-3.344,2.303);
\draw(-3.332,2.303)--(-2.827,2.281);
\filldraw[fill opacity=0.8,fill=gray!20,draw=none](-2.855,2.021)--(-2.857,2.021)--(-2.859,2.039)--cycle;
\draw(-2.857,2.021)--(-2.859,2.039);
\filldraw[fill opacity=0.8,fill=gray!20,draw=none](-2.793,2.02)--(-2.81,2.02)--(-2.826,2.04)--cycle;
\filldraw[fill opacity=0.8,fill=gray!20,draw=none](-2.793,2.02)--(-3.299,2.042)--(-3.313,2.062)--(-2.826,2.04)--cycle;
\draw(-2.793,2.02)--(-3.299,2.042);
\draw(-3.313,2.062)--(-2.826,2.04);
\filldraw[fill opacity=0.8,fill=gray!20,draw=none](-3.347,2.245)--(-3.339,2.197)--(-3.334,2.145)--(-3.334,2.118)--(-3.353,2.267)--cycle;
\draw(-3.334,2.118)--(-3.353,2.267);
\filldraw[fill opacity=0.8,fill=gray!20,draw=none](-3.349,2.326)--(-3.353,2.267)--(-3.334,2.118)--(-3.328,2.098)--(-3.313,2.062)--(-3.299,2.042)--(-3.332,2.303)--cycle;
\draw(-3.353,2.267)--(-3.334,2.118);
\draw(-3.299,2.042)--(-3.332,2.303);
\filldraw[fill opacity=0.8,fill=gray!20,draw=none](-3.398,2.2)--(-3.396,2.17)--(-3.403,2.227)--cycle;
\draw(-3.396,2.17)--(-3.403,2.227);
\filldraw[fill opacity=0.8,fill=gray!20,draw=none](-3.323,2.041)--(-3.326,2.059)--(-3.324,2.041)--cycle;
\draw(-3.326,2.059)--(-3.324,2.041);
\filldraw[fill opacity=0.8,fill=gray!20,draw=none](-3.313,2.062)--(-3.305,2.042)--(-3.299,2.042)--cycle;
\filldraw[fill opacity=0.8,fill=gray!20,draw=none](-3.364,2.064)--(-3.356,2.044)--(-3.349,2.044)--cycle;
\filldraw[fill opacity=0.8,fill=gray!20,draw=none](-3.299,2.042)--(-3.349,2.044)--(-3.382,2.065)--(-3.313,2.062)--cycle;
\draw(-3.299,2.042)--(-3.349,2.044);
\draw(-3.382,2.065)--(-3.313,2.062);
\filldraw[fill opacity=0.8,fill=gray!20,draw=none](-3.414,2.368)--(-3.408,2.304)--(-3.413,2.304)--(-3.419,2.354)--cycle;
\draw(-3.413,2.304)--(-3.419,2.354);
\filldraw[fill opacity=0.8,fill=gray!20,draw=none](-3.356,2.302)--(-3.357,2.302)--(-3.36,2.325)--cycle;
\draw(-3.357,2.302)--(-3.36,2.325);
\filldraw[fill opacity=0.8,fill=gray!20,draw=none](-3.351,2.303)--(-3.349,2.326)--(-3.355,2.334)--(-3.361,2.339)--(-3.36,2.325)--(-3.357,2.302)--cycle;
\draw(-3.355,2.334)--(-3.361,2.339);
\draw(-3.36,2.325)--(-3.357,2.302);
\filldraw[fill opacity=0.8,fill=gray!20,draw=none](-3.419,2.354)--(-3.413,2.304)--(-3.409,2.304)--(-3.412,2.345)--cycle;
\draw(-3.419,2.354)--(-3.413,2.304);
\filldraw[fill opacity=0.8,fill=gray!20,draw=none](-3.546,2.31)--(-3.569,2.31)--(-3.583,2.332)--(-3.55,2.317)--cycle;
\draw(-3.569,2.31)--(-3.583,2.332)--(-3.55,2.317);
\filldraw[fill opacity=0.8,fill=gray!20,draw=none](-3.412,2.345)--(-3.409,2.304)--(-3.401,2.305)--(-3.4,2.328)--cycle;
\filldraw[fill opacity=0.8,fill=gray!20,draw=none](-3.352,2.283)--(-3.402,2.285)--(-3.395,2.305)--(-3.344,2.303)--cycle;
\draw(-3.352,2.283)--(-3.402,2.285);
\draw(-3.395,2.305)--(-3.344,2.303);
\filldraw[fill opacity=0.8,fill=gray!20,draw=none](-3.344,2.303)--(-3.413,2.306)--(-3.397,2.306)--(-3.371,2.305)--(-3.332,2.303)--cycle;
\draw(-3.344,2.303)--(-3.413,2.306);
\draw(-3.371,2.305)--(-3.332,2.303);
\filldraw[fill opacity=0.8,fill=gray!20,draw=none](-3.395,2.305)--(-3.402,2.285)--(-3.404,2.248)--(-3.403,2.227)--(-3.396,2.17)--(-3.392,2.148)--(-3.379,2.1)--(-3.364,2.064)--(-3.349,2.044)--(-3.383,2.306)--cycle;
\draw(-3.403,2.227)--(-3.396,2.17);
\draw(-3.349,2.044)--(-3.383,2.306);
\filldraw[fill opacity=0.8,fill=gray!20,draw=none](-3.268,2.246)--(-3.299,2.283)--(-3.332,2.303)--(-3.299,2.042)--(-3.268,2.042)--(-3.245,2.062)--(-3.24,2.077)--(-3.259,2.226)--cycle;
\draw(-3.332,2.303)--(-3.299,2.042);
\draw(-3.24,2.077)--(-3.259,2.226);
\filldraw[fill opacity=0.8,fill=gray!20,draw=none](-3.395,2.305)--(-3.401,2.305)--(-3.402,2.285)--cycle;
\filldraw[fill opacity=0.8,fill=gray!20,draw=none](-3.402,2.285)--(-3.503,2.289)--(-3.527,2.307)--(-3.527,2.311)--(-3.395,2.305)--cycle;
\draw(-3.402,2.285)--(-3.503,2.289);
\draw(-3.527,2.311)--(-3.395,2.305);
\filldraw[fill opacity=0.8,fill=gray!20,draw=none](-3.536,2.311)--(-3.583,2.332)--(-3.626,2.358)--(-3.571,2.334)--cycle;
\draw(-3.536,2.311)--(-3.583,2.332)--(-3.626,2.358)--(-3.571,2.334);
\filldraw[fill opacity=0.8,fill=gray!20,draw=none](-3.395,2.305)--(-3.39,2.316)--(-3.4,2.328)--(-3.401,2.305)--cycle;
\filldraw[fill opacity=0.8,fill=gray!20,draw=none](-3.296,2.2)--(-3.319,2.248)--(-3.35,2.285)--(-3.383,2.306)--(-3.349,2.044)--(-3.319,2.044)--(-3.295,2.064)--(-3.283,2.1)--(-3.283,2.121)--(-3.29,2.178)--cycle;
\draw(-3.383,2.306)--(-3.349,2.044);
\draw(-3.283,2.121)--(-3.29,2.178);
\filldraw[fill opacity=0.8,fill=gray!20,draw=none](-3.503,2.289)--(-3.527,2.29)--(-3.527,2.307)--cycle;
\draw(-3.503,2.289)--(-3.527,2.29);
\filldraw[fill opacity=0.8,fill=gray!20,draw=none](-3.444,2.254)--(-3.426,2.232)--(-3.52,2.274)--(-3.536,2.311)--(-3.476,2.285)--cycle;
\draw(-3.426,2.232)--(-3.52,2.274);
\draw(-3.536,2.311)--(-3.476,2.285);
\filldraw[fill opacity=0.8,fill=gray!20,draw=none](-3.39,2.316)--(-3.395,2.305)--(-3.383,2.306)--cycle;
\filldraw[fill opacity=0.8,fill=gray!20,draw=none](-3.413,2.306)--(-3.426,2.306)--(-3.397,2.306)--cycle;
\draw(-3.413,2.306)--(-3.426,2.306);
\filldraw[fill opacity=0.8,fill=gray!20,draw=none](-3.536,2.311)--(-3.546,2.31)--(-3.55,2.317)--(-3.536,2.311)--cycle;
\draw(-3.55,2.317)--(-3.536,2.311);
\filldraw[fill opacity=0.8,fill=gray!20,draw=none](-3.527,2.307)--(-3.532,2.311)--(-3.527,2.311)--cycle;
\draw(-3.532,2.311)--(-3.527,2.311);
\filldraw[fill opacity=0.8,fill=gray!20,draw=none](-3.529,2.312)--(-3.508,2.299)--(-3.535,2.311)--cycle;
\draw(-3.508,2.299)--(-3.535,2.311);
\filldraw[fill opacity=0.8,fill=gray!20,draw=none](-3.527,2.307)--(-3.527,2.29)--(-3.567,2.292)--(-3.57,2.313)--(-3.532,2.311)--cycle;
\draw(-3.527,2.29)--(-3.567,2.292);
\draw(-3.57,2.313)--(-3.532,2.311);
\filldraw[fill opacity=0.8,fill=gray!20,draw=none](-3.384,2.306)--(-3.397,2.306)--(-3.426,2.306)--(-3.532,2.311)--(-3.529,2.312)--(-3.505,2.311)--(-3.391,2.306)--cycle;
\draw(-3.426,2.306)--(-3.532,2.311);
\draw(-3.505,2.311)--(-3.391,2.306);
\filldraw[fill opacity=0.8,fill=gray!20,draw=none](-3.567,2.292)--(-3.6,2.293)--(-3.57,2.313)--cycle;
\draw(-3.567,2.292)--(-3.6,2.293);
\filldraw[fill opacity=0.8,fill=gray!20,draw=none](-3.611,2.338)--(-3.595,2.334)--(-3.545,2.292)--(-3.581,2.278)--(-3.6,2.293)--cycle;
\draw(-3.595,2.334)--(-3.545,2.292)--(-3.581,2.278)--(-3.6,2.293);
\filldraw[fill opacity=0.8,fill=gray!20,draw=none](-3.558,2.328)--(-3.534,2.311)--(-3.535,2.311)--(-3.536,2.311)--(-3.571,2.334)--(-3.566,2.332)--cycle;
\draw(-3.535,2.311)--(-3.536,2.311);
\draw(-3.571,2.334)--(-3.566,2.332);
\filldraw[fill opacity=0.8,fill=gray!20,draw=none](-3.558,2.328)--(-3.527,2.312)--(-3.534,2.311)--cycle;
\filldraw[fill opacity=0.8,fill=gray!20,draw=none](-3.624,2.341)--(-3.611,2.338)--(-3.6,2.293)--(-3.655,2.34)--cycle;
\draw(-3.6,2.293)--(-3.655,2.34);
\filldraw[fill opacity=0.8,fill=gray!20,draw=none](-3.6,2.293)--(-3.624,2.294)--(-3.601,2.314)--(-3.57,2.313)--cycle;
\draw(-3.6,2.293)--(-3.624,2.294)--(-3.601,2.314)--(-3.57,2.313);
\filldraw[fill opacity=0.8,fill=gray!20,draw=none](-3.532,2.311)--(-3.537,2.311)--(-3.54,2.312)--(-3.529,2.312)--cycle;
\draw(-3.532,2.311)--(-3.537,2.311);
\filldraw[fill opacity=0.8,fill=gray!20,draw=none](-3.537,2.311)--(-3.57,2.313)--(-3.54,2.312)--cycle;
\draw(-3.537,2.311)--(-3.57,2.313);
\filldraw[fill opacity=0.8,fill=gray!20,draw=none](-3.543,2.313)--(-3.517,2.291)--(-3.545,2.292)--(-3.57,2.313)--cycle;
\draw(-3.543,2.313)--(-3.517,2.291)--(-3.545,2.292)--(-3.57,2.313);
\filldraw[fill opacity=0.8,fill=gray!20,draw=none](-3.543,2.313)--(-3.57,2.313)--(-3.595,2.334)--cycle;
\draw(-3.57,2.313)--(-3.595,2.334);
\filldraw[fill opacity=0.8,fill=gray!20,draw=none](-3.54,2.312)--(-3.57,2.313)--(-3.601,2.314)--(-3.578,2.314)--(-3.543,2.313)--cycle;
\draw(-3.57,2.313)--(-3.601,2.314)--(-3.578,2.314)--(-3.543,2.313);
\filldraw[fill opacity=0.8,fill=gray!20,draw=none](-3.529,2.312)--(-3.527,2.312)--(-3.487,2.291)--(-3.476,2.285)--(-3.508,2.299)--cycle;
\draw(-3.476,2.285)--(-3.508,2.299);
\filldraw[fill opacity=0.8,fill=gray!20,draw=none](-3.408,2.154)--(-3.417,2.119)--(-3.529,2.168)--(-3.526,2.205)--(-3.52,2.224)--(-3.406,2.174)--cycle;
\draw(-3.417,2.119)--(-3.529,2.168);
\draw(-3.52,2.224)--(-3.406,2.174);
\filldraw[fill opacity=0.8,fill=gray!20,draw=none](-3.414,2.205)--(-3.406,2.174)--(-3.52,2.224)--(-3.523,2.253)--(-3.52,2.274)--(-3.426,2.232)--cycle;
\draw(-3.406,2.174)--(-3.52,2.224);
\draw(-3.52,2.274)--(-3.426,2.232);
\filldraw[fill opacity=0.8,fill=gray!20,draw=none](-3.653,2.211)--(-3.619,2.249)--(-3.581,2.278)--(-3.545,2.292)--(-3.517,2.291)--(-3.5,2.273)--(-3.498,2.242)--(-3.51,2.202)--(-3.535,2.159)--cycle;
\draw(-3.653,2.211)--(-3.619,2.249)--(-3.581,2.278)--(-3.545,2.292)--(-3.517,2.291)--(-3.5,2.273)--(-3.498,2.242)--(-3.51,2.202)--(-3.535,2.159);
\filldraw[fill opacity=0.8,fill=gray!20,draw=none](-3.525,2.293)--(-3.5,2.273)--(-3.517,2.291)--(-3.543,2.313)--cycle;
\draw(-3.525,2.293)--(-3.5,2.273)--(-3.517,2.291)--(-3.543,2.313);
\filldraw[fill opacity=0.8,fill=gray!20,draw=none](-1.714,2.232)--(-3.578,2.314)--(-3.556,2.294)--(-1.693,2.213)--cycle;
\draw(-1.714,2.232)--(-3.578,2.314)--(-3.556,2.294)--(-1.693,2.213);
\filldraw[fill opacity=0.5,fill=gray!20,draw=none](-1.949,1.997)--(-1.953,2.002)--(-1.957,2.015)--(-1.921,2.102)--(-1.934,1.997)--cycle;
\draw(-1.921,2.102)--(-1.934,1.997);
\filldraw[fill opacity=0.8,fill=gray!20,draw=none](-1.699,1.994)--(-1.718,1.995)--(-1.734,1.975)--(-1.721,1.974)--cycle;
\draw(-1.699,1.994)--(-1.718,1.995);
\draw(-1.734,1.975)--(-1.721,1.974);
\filldraw[fill opacity=0.8,fill=gray!20,draw=none](-2.365,1.575)--(-2.359,1.592)--(-2.337,1.592)--cycle;
\filldraw[fill opacity=0.8,fill=gray!20,draw=none](-2.365,1.575)--(-2.366,1.575)--(-2.371,1.58)--(-2.359,1.592)--cycle;
\draw(-2.366,1.575)--(-2.371,1.58)--(-2.359,1.592);
\filldraw[fill opacity=0.8,fill=gray!20,draw=none](-1.941,1.725)--(-2.392,1.573)--(-2.413,1.574)--(-1.949,1.729)--cycle;
\draw(-1.941,1.725)--(-2.392,1.573)--(-2.413,1.574)--(-1.949,1.729);
\filldraw[fill opacity=0.5,fill=gray!20,draw=none](-1.893,2.34)--(-1.925,2.264)--(-1.939,2.28)--(-1.9,2.476)--(-1.878,2.466)--cycle;
\draw(-1.9,2.476)--(-1.878,2.466)--(-1.893,2.34);
\filldraw[fill opacity=0.8,fill=gray!20,draw=none](-1.741,2.034)--(-1.897,1.982)--(-1.729,1.995)--(-1.719,1.998)--cycle;
\draw(-1.741,2.034)--(-1.897,1.982);
\draw(-1.729,1.995)--(-1.719,1.998);
\filldraw[fill opacity=0.8,fill=gray!20,draw=none](-1.964,1.989)--(-1.962,1.984)--(-1.972,1.984)--cycle;
\draw(-1.962,1.984)--(-1.972,1.984);
\filldraw[fill opacity=0.8,fill=gray!20,draw=none](-1.964,1.989)--(-1.972,1.984)--(-1.961,1.988)--cycle;
\draw(-1.972,1.984)--(-1.961,1.988);
\filldraw[fill opacity=0.5,fill=gray!20,draw=none](-1.883,2)--(-1.908,1.982)--(-1.914,1.982)--(-1.934,1.997)--cycle;
\filldraw[fill opacity=0.8,fill=gray!20,draw=none](-1.96,1.984)--(-1.961,1.988)--(-1.972,1.984)--cycle;
\draw(-1.961,1.988)--(-1.972,1.984);
\filldraw[fill opacity=0.8,fill=gray!20,draw=none](-3.349,2.044)--(-3.55,2.053)--(-3.557,2.064)--(-3.555,2.072)--(-3.382,2.065)--cycle;
\draw(-3.349,2.044)--(-3.55,2.053);
\draw(-3.555,2.072)--(-3.382,2.065);
\filldraw[fill opacity=0.8,fill=gray!20,draw=none](-3.557,2.064)--(-3.562,2.072)--(-3.555,2.072)--cycle;
\draw(-3.562,2.072)--(-3.555,2.072);
\filldraw[fill opacity=0.8,fill=gray!20,draw=none](-3.465,2.071)--(-3.516,2.053)--(-3.531,2.053)--(-3.564,2.067)--(-3.549,2.108)--(-3.545,2.113)--(-3.458,2.075)--cycle;
\draw(-3.531,2.053)--(-3.564,2.067);
\draw(-3.545,2.113)--(-3.458,2.075);
\filldraw[fill opacity=0.8,fill=gray!20,draw=none](-3.557,2.064)--(-3.55,2.053)--(-3.559,2.054)--cycle;
\draw(-3.55,2.053)--(-3.559,2.054);
\filldraw[fill opacity=0.8,fill=gray!20,draw=none](-3.585,2.054)--(-3.564,2.067)--(-3.531,2.053)--cycle;
\draw(-3.564,2.067)--(-3.531,2.053);
\filldraw[fill opacity=0.8,fill=gray!20,draw=none](-3.557,2.064)--(-3.559,2.054)--(-3.61,2.056)--(-3.632,2.076)--(-3.562,2.072)--cycle;
\draw(-3.559,2.054)--(-3.61,2.056)--(-3.632,2.076)--(-3.562,2.072);
\filldraw[fill opacity=0.8,fill=gray!20,draw=none](-1.718,1.995)--(-3.564,2.076)--(-3.587,2.056)--(-1.734,1.975)--cycle;
\draw(-1.718,1.995)--(-3.564,2.076)--(-3.587,2.056)--(-1.734,1.975);
\filldraw[fill opacity=0.5,fill=gray!20,draw=none](-1.47,1.347)--(-1.586,1.397)--(-1.691,1.905)--(-1.681,1.901)--cycle;
\draw(-1.47,1.347)--(-1.586,1.397);
\draw(-1.691,1.905)--(-1.681,1.901);
\filldraw[fill opacity=0.8,fill=gray!20,draw=none](-1.713,2.044)--(-1.741,2.034)--(-1.719,1.998)--(-1.697,2.006)--cycle;
\draw(-1.713,2.044)--(-1.741,2.034);
\draw(-1.719,1.998)--(-1.697,2.006);
\filldraw[fill opacity=0.5,fill=gray!20](-1.698,1.33)--(-1.799,1.343)--(-1.865,1.882)--(-1.763,1.858)--cycle;
\filldraw[fill opacity=0.5,fill=gray!20](-1.272,.895)--(-1.397,.865)--(-1.58,1.328)--(-1.447,1.337)--cycle;
\filldraw[fill opacity=0.8,fill=gray!20,draw=none](-1.994,1.984)--(-2.117,1.943)--(-2.034,1.963)--(-1.972,1.984)--cycle;
\draw(-1.994,1.984)--(-2.117,1.943);
\draw(-2.034,1.963)--(-1.972,1.984);
\filldraw[fill opacity=0.5,fill=gray!20,draw=none](-1.946,1.982)--(-1.936,1.982)--(-1.938,1.959)--cycle;
\draw(-1.936,1.982)--(-1.938,1.959);
\filldraw[fill opacity=0.5,fill=gray!20](-.726,3.67)--(-.816,3.751)--(-.359,4.007)--(-.284,3.918)--cycle;
\filldraw[fill opacity=0.5,fill=gray!20](-1.104,1.02)--(-1.272,.895)--(-1.447,1.337)--(-1.259,1.412)--cycle;
\filldraw[fill opacity=0.8,fill=gray!20,draw=none](-1.878,1.759)--(-1.894,1.754)--(-1.906,1.736)--(-1.898,1.739)--cycle;
\draw(-1.878,1.759)--(-1.894,1.754);
\draw(-1.906,1.736)--(-1.898,1.739);
\filldraw[fill opacity=0.8,fill=gray!20,draw=none](-2.888,2.126)--(-2.905,2.178)--(-2.9,2.226)--(-2.886,2.247)--(-2.867,2.097)--cycle;
\draw(-2.886,2.247)--(-2.867,2.097);
\filldraw[fill opacity=0.8,fill=gray!20,draw=none](-3.516,2.257)--(-3.498,2.242)--(-3.5,2.273)--(-3.525,2.293)--cycle;
\draw(-3.516,2.257)--(-3.498,2.242)--(-3.5,2.273)--(-3.525,2.293);
\filldraw[fill opacity=0.8,fill=gray!20,draw=none](-1.693,2.213)--(-3.556,2.294)--(-3.541,2.258)--(-1.677,2.176)--cycle;
\draw(-1.693,2.213)--(-3.556,2.294)--(-3.541,2.258)--(-1.677,2.176);
\filldraw[fill opacity=0.8,fill=gray!20,draw=none](-1.691,1.902)--(-1.691,1.901)--(-1.702,1.851)--(-1.689,1.855)--cycle;
\draw(-1.691,1.902)--(-1.691,1.901);
\draw(-1.702,1.851)--(-1.689,1.855);
\filldraw[fill opacity=0.8,fill=gray!20,draw=none](-2.382,1.611)--(-2.38,1.615)--(-2.344,1.606)--(-2.359,1.592)--cycle;
\draw(-2.382,1.611)--(-2.38,1.615)--(-2.344,1.606)--(-2.359,1.592);
\filldraw[fill opacity=0.8,fill=gray!20](-2.344,1.606)--(-2.323,1.638)--(-2.31,1.624)--(-2.333,1.595)--cycle;
\filldraw[fill opacity=0.8,fill=gray!20,draw=none](-2.359,1.592)--(-2.344,1.606)--(-2.333,1.595)--(-2.337,1.592)--cycle;
\draw(-2.359,1.592)--(-2.344,1.606)--(-2.333,1.595)--(-2.337,1.592);
\filldraw[fill opacity=0.8,fill=gray!20,draw=none](-2.337,1.592)--(-2.333,1.595)--(-2.335,1.594)--cycle;
\draw(-2.337,1.592)--(-2.333,1.595)--(-2.335,1.594);
\filldraw[fill opacity=0.8,fill=gray!20,draw=none](-2.33,1.608)--(-2.335,1.594)--(-2.333,1.595)--(-2.31,1.624)--(-2.325,1.614)--cycle;
\draw(-2.335,1.594)--(-2.333,1.595)--(-2.31,1.624)--(-2.325,1.614);
\filldraw[fill opacity=0.8,fill=gray!20,draw=none](-1.894,1.754)--(-2.378,1.592)--(-2.392,1.573)--(-1.906,1.736)--cycle;
\draw(-1.894,1.754)--(-2.378,1.592)--(-2.392,1.573)--(-1.906,1.736);
\filldraw[fill opacity=0.8,fill=gray!20,draw=none](-1.696,1.999)--(-1.697,2.006)--(-1.729,1.995)--cycle;
\draw(-1.697,2.006)--(-1.729,1.995);
\filldraw[fill opacity=0.8,fill=gray!20,draw=none](-1.681,2.03)--(-1.711,2.031)--(-1.718,1.995)--(-1.699,1.994)--cycle;
\draw(-1.681,2.03)--(-1.711,2.031);
\draw(-1.718,1.995)--(-1.699,1.994);
\filldraw[fill opacity=0.8,fill=gray!20,draw=none](-1.696,1.996)--(-1.729,1.995)--(-1.889,1.941)--(-1.879,1.892)--(-1.69,1.956)--cycle;
\draw(-1.729,1.995)--(-1.889,1.941);
\draw(-1.879,1.892)--(-1.69,1.956);
\filldraw[fill opacity=0.8,fill=gray!20,draw=none](-1.897,1.982)--(-1.908,1.978)--(-1.889,1.941)--(-1.781,1.977)--cycle;
\draw(-1.897,1.982)--(-1.908,1.978);
\draw(-1.889,1.941)--(-1.781,1.977);
\filldraw[fill opacity=0.5,fill=gray!20,draw=none](-1.86,1.924)--(-1.865,1.882)--(-1.944,1.909)--(-1.936,1.982)--cycle;
\draw(-1.86,1.924)--(-1.865,1.882)--(-1.944,1.909)--(-1.936,1.982);
\filldraw[fill opacity=0.5,fill=gray!20,draw=none](-1.914,2.251)--(-1.925,2.264)--(-1.893,2.34)--(-1.902,2.264)--cycle;
\draw(-1.893,2.34)--(-1.902,2.264);
\filldraw[fill opacity=0.8,fill=gray!20,draw=none](-1.711,2.031)--(-1.861,2.038)--(-1.882,2.002)--(-1.718,1.995)--cycle;
\draw(-1.711,2.031)--(-1.861,2.038);
\draw(-1.882,2.002)--(-1.718,1.995);
\filldraw[fill opacity=0.8,fill=gray!20,draw=none](-1.696,1.996)--(-1.696,1.999)--(-1.729,1.995)--cycle;
\filldraw[fill opacity=0.8,fill=gray!20,draw=none](-1.914,1.982)--(-1.908,1.978)--(-1.897,1.982)--cycle;
\draw(-1.908,1.978)--(-1.897,1.982);
\filldraw[fill opacity=0.5,fill=gray!20,draw=none](-1.908,1.982)--(-1.922,1.972)--(-1.936,1.982)--cycle;
\filldraw[fill opacity=0.8,fill=gray!20,draw=none](-1.897,1.982)--(-1.781,1.977)--(-1.729,1.995)--cycle;
\draw(-1.781,1.977)--(-1.729,1.995);
\filldraw[fill opacity=0.8,fill=gray!20,draw=none](-1.691,1.905)--(-1.691,1.901)--(-1.691,1.902)--cycle;
\draw(-1.691,1.901)--(-1.691,1.902);
\filldraw[fill opacity=0.8,fill=gray!20,draw=none](-1.951,1.984)--(-2.074,1.989)--(-2.034,1.987)--(-1.962,1.984)--cycle;
\draw(-1.951,1.984)--(-2.074,1.989);
\draw(-2.034,1.987)--(-1.962,1.984);
\filldraw[fill opacity=0.8,fill=gray!20,draw=none](-2.792,2.178)--(-2.787,2.126)--(-2.779,2.078)--(-2.773,2.056)--(-2.792,2.206)--cycle;
\draw(-2.773,2.056)--(-2.792,2.206);
\filldraw[fill opacity=0.8,fill=gray!20,draw=none](-3.225,2.098)--(-3.221,2.145)--(-3.238,2.198)--(-3.259,2.226)--(-3.24,2.077)--cycle;
\draw(-3.259,2.226)--(-3.24,2.077);
\filldraw[fill opacity=0.8,fill=gray!20,draw=none](-3.52,2.21)--(-3.51,2.202)--(-3.498,2.242)--(-3.516,2.257)--cycle;
\draw(-3.52,2.21)--(-3.51,2.202)--(-3.498,2.242)--(-3.516,2.257);
\filldraw[fill opacity=0.8,fill=gray!20,draw=none](-1.677,2.176)--(-3.541,2.258)--(-3.533,2.211)--(-1.669,2.129)--cycle;
\draw(-1.677,2.176)--(-3.541,2.258)--(-3.533,2.211)--(-1.669,2.129);
\filldraw[fill opacity=0.5,fill=gray!20,draw=none](-1.914,2.251)--(-1.902,2.264)--(-1.905,2.241)--cycle;
\draw(-1.902,2.264)--(-1.905,2.241);
\filldraw[fill opacity=0.8,fill=gray!20,draw=none](-2.445,1.795)--(-2.443,1.82)--(-2.43,1.821)--(-2.429,1.796)--cycle;
\draw(-2.443,1.82)--(-2.43,1.821)--(-2.429,1.796)--(-2.445,1.795);
\filldraw[fill opacity=0.8,fill=gray!20](-2.429,1.796)--(-2.43,1.821)--(-2.38,1.817)--(-2.367,1.792)--cycle;
\filldraw[fill opacity=0.8,fill=gray!20,draw=none](-2.43,1.821)--(-2.431,1.83)--(-2.427,1.836)--(-2.396,1.834)--(-2.38,1.817)--cycle;
\draw(-2.427,1.836)--(-2.396,1.834)--(-2.38,1.817)--(-2.43,1.821)--(-2.431,1.83);
\filldraw[fill opacity=0.8,fill=gray!20,draw=none](-2.443,1.82)--(-2.431,1.83)--(-2.43,1.821)--cycle;
\draw(-2.431,1.83)--(-2.43,1.821)--(-2.443,1.82);
\filldraw[fill opacity=0.8,fill=gray!20](-2.38,1.817)--(-2.396,1.834)--(-2.371,1.828)--(-2.344,1.808)--cycle;
\filldraw[fill opacity=0.8,fill=gray!20,draw=none](-2.427,1.836)--(-2.428,1.838)--(-2.411,1.84)--(-2.396,1.834)--cycle;
\draw(-2.411,1.84)--(-2.396,1.834)--(-2.427,1.836);
\filldraw[fill opacity=0.8,fill=gray!20,draw=none](-2.396,1.834)--(-2.411,1.84)--(-2.405,1.839)--(-2.403,1.839)--(-2.371,1.828)--cycle;
\draw(-2.405,1.839)--(-2.403,1.839)--(-2.371,1.828)--(-2.396,1.834)--(-2.411,1.84);
\filldraw[fill opacity=0.8,fill=gray!20](-2.367,1.792)--(-2.38,1.817)--(-2.344,1.808)--(-2.323,1.781)--cycle;
\filldraw[fill opacity=0.8,fill=gray!20](-2.323,1.781)--(-2.344,1.808)--(-2.333,1.797)--(-2.31,1.767)--cycle;
\filldraw[fill opacity=0.8,fill=gray!20](-2.344,1.808)--(-2.371,1.828)--(-2.364,1.82)--(-2.333,1.797)--cycle;
\filldraw[fill opacity=0.8,fill=gray!20,draw=none](-2.371,1.828)--(-2.403,1.839)--(-2.402,1.838)--(-2.391,1.831)--(-2.364,1.82)--cycle;
\draw(-2.391,1.831)--(-2.364,1.82)--(-2.371,1.828)--(-2.403,1.839)--(-2.402,1.838);
\filldraw[fill opacity=0.8,fill=gray!20,draw=none](-2.402,1.838)--(-2.403,1.839)--(-2.405,1.839)--cycle;
\draw(-2.402,1.838)--(-2.403,1.839)--(-2.405,1.839);
\filldraw[fill opacity=0.8,fill=gray!20,draw=none](-2.374,1.813)--(-2.364,1.82)--(-2.391,1.831)--(-2.383,1.819)--cycle;
\draw(-2.374,1.813)--(-2.364,1.82)--(-2.391,1.831);
\filldraw[fill opacity=0.8,fill=gray!20,draw=none](-1.96,1.984)--(-1.972,1.984)--(-2.461,1.82)--(-2.436,1.801)--(-1.953,1.963)--cycle;
\draw(-1.972,1.984)--(-2.461,1.82)--(-2.436,1.801)--(-1.953,1.963);
\filldraw[fill opacity=0.8,fill=gray!20,draw=none](-2.727,2.124)--(-2.723,2.096)--(-2.73,2.153)--cycle;
\draw(-2.723,2.096)--(-2.73,2.153);
\filldraw[fill opacity=0.8,fill=gray!20,draw=none](-3.426,2.107)--(-3.458,2.075)--(-3.545,2.113)--(-3.535,2.154)--(-3.529,2.168)--(-3.417,2.119)--cycle;
\draw(-3.458,2.075)--(-3.545,2.113);
\draw(-3.529,2.168)--(-3.417,2.119);
\filldraw[fill opacity=0.8,fill=gray!20,draw=none](-3.28,2.148)--(-3.29,2.178)--(-3.283,2.121)--cycle;
\draw(-3.29,2.178)--(-3.283,2.121);
\filldraw[fill opacity=0.8,fill=gray!20,draw=none](-1.67,2.077)--(-3.535,2.159)--(-3.546,2.112)--(-1.681,2.03)--cycle;
\draw(-1.67,2.077)--(-3.535,2.159)--(-3.546,2.112)--(-1.681,2.03);
\filldraw[fill opacity=0.5,fill=gray!20,draw=none](-1.908,1.982)--(-1.853,1.982)--(-1.86,1.924)--(-1.922,1.972)--cycle;
\draw(-1.853,1.982)--(-1.86,1.924);
\filldraw[fill opacity=0.8,fill=gray!20,draw=none](-1.914,1.982)--(-1.96,1.984)--(-1.953,1.963)--(-1.908,1.978)--cycle;
\draw(-1.953,1.963)--(-1.908,1.978);
\filldraw[fill opacity=0.8,fill=gray!20,draw=none](-2.117,1.943)--(-2.481,1.821)--(-2.461,1.82)--(-2.034,1.963)--cycle;
\draw(-2.117,1.943)--(-2.481,1.821)--(-2.461,1.82)--(-2.034,1.963);
\filldraw[fill opacity=0.5,fill=gray!20,draw=none](-1.447,1.337)--(-1.47,1.347)--(-1.681,1.901)--(-1.507,1.824)--cycle;
\draw(-1.681,1.901)--(-1.507,1.824)--(-1.447,1.337)--(-1.47,1.347);
\filldraw[fill opacity=0.8,fill=gray!20,draw=none](-1.868,1.795)--(-1.889,1.788)--(-1.894,1.754)--(-1.878,1.759)--cycle;
\draw(-1.868,1.795)--(-1.889,1.788);
\draw(-1.894,1.754)--(-1.878,1.759);
\filldraw[fill opacity=0.8,fill=gray!20](-2.367,1.649)--(-2.359,1.686)--(-2.31,1.674)--(-2.323,1.638)--cycle;
\filldraw[fill opacity=0.8,fill=gray!20](-2.38,1.615)--(-2.367,1.649)--(-2.323,1.638)--(-2.344,1.606)--cycle;
\filldraw[fill opacity=0.8,fill=gray!20](-2.323,1.638)--(-2.31,1.674)--(-2.295,1.658)--(-2.31,1.624)--cycle;
\filldraw[fill opacity=0.8,fill=gray!20,draw=none](-2.325,1.614)--(-2.31,1.624)--(-2.295,1.658)--(-2.318,1.644)--cycle;
\draw(-2.325,1.614)--(-2.31,1.624)--(-2.295,1.658)--(-2.318,1.644);
\filldraw[fill opacity=0.8,fill=gray!20,draw=none](-1.889,1.788)--(-2.373,1.626)--(-2.378,1.592)--(-1.894,1.754)--cycle;
\draw(-1.889,1.788)--(-2.373,1.626)--(-2.378,1.592)--(-1.894,1.754);
\filldraw[fill opacity=0.8,fill=gray!20,draw=none](-3.535,2.159)--(-3.51,2.202)--(-3.52,2.21)--cycle;
\draw(-3.535,2.159)--(-3.51,2.202)--(-3.52,2.21);
\filldraw[fill opacity=0.8,fill=gray!20,draw=none](-1.669,2.129)--(-3.533,2.211)--(-3.535,2.159)--(-1.67,2.077)--cycle;
\draw(-1.669,2.129)--(-3.533,2.211)--(-3.535,2.159)--(-1.67,2.077);
\filldraw[fill opacity=0.5,fill=gray!20,draw=none](-1.89,2.241)--(-1.879,2.221)--(-1.905,2.241)--cycle;
\filldraw[fill opacity=0.5,fill=gray!20,draw=none](-1.811,2.332)--(-1.822,2.24)--(-1.905,2.241)--(-1.902,2.264)--cycle;
\draw(-1.811,2.332)--(-1.822,2.24);
\draw(-1.905,2.241)--(-1.902,2.264);
\filldraw[fill opacity=0.8,fill=gray!20,draw=none](-2.444,1.764)--(-2.456,1.783)--(-2.445,1.795)--(-2.429,1.796)--(-2.428,1.765)--cycle;
\draw(-2.445,1.795)--(-2.429,1.796)--(-2.428,1.765)--(-2.444,1.764);
\filldraw[fill opacity=0.8,fill=gray!20](-2.428,1.765)--(-2.429,1.796)--(-2.367,1.792)--(-2.359,1.76)--cycle;
\filldraw[fill opacity=0.8,fill=gray!20](-2.359,1.76)--(-2.367,1.792)--(-2.323,1.781)--(-2.31,1.748)--cycle;
\filldraw[fill opacity=0.8,fill=gray!20](-2.31,1.748)--(-2.323,1.781)--(-2.31,1.767)--(-2.295,1.732)--cycle;
\filldraw[fill opacity=0.8,fill=gray!20](-2.31,1.767)--(-2.333,1.797)--(-2.35,1.786)--(-2.331,1.753)--cycle;
\filldraw[fill opacity=0.8,fill=gray!20,draw=none](-2.333,1.797)--(-2.364,1.82)--(-2.374,1.813)--(-2.364,1.8)--(-2.35,1.786)--cycle;
\draw(-2.364,1.8)--(-2.35,1.786)--(-2.333,1.797)--(-2.364,1.82)--(-2.374,1.813);
\filldraw[fill opacity=0.8,fill=gray!20,draw=none](-2.36,1.784)--(-2.35,1.786)--(-2.364,1.8)--cycle;
\draw(-2.36,1.784)--(-2.35,1.786)--(-2.364,1.8);
\filldraw[fill opacity=0.8,fill=gray!20,draw=none](-1.908,1.978)--(-2.436,1.801)--(-2.412,1.766)--(-1.889,1.941)--cycle;
\draw(-1.908,1.978)--(-2.436,1.801)--(-2.412,1.766)--(-1.889,1.941);
\filldraw[fill opacity=0.5,fill=gray!20,draw=none](-1.883,2)--(-1.85,2.003)--(-1.853,1.982)--(-1.908,1.982)--cycle;
\draw(-1.85,2.003)--(-1.853,1.982);
\filldraw[fill opacity=0.5,fill=gray!20](-1.606,2.964)--(-1.684,2.998)--(-1.375,3.471)--(-1.299,3.434)--cycle;
\filldraw[fill opacity=0.5,fill=gray!20,draw=none](-1.811,2.332)--(-1.902,2.264)--(-1.878,2.466)--(-1.799,2.434)--cycle;
\draw(-1.902,2.264)--(-1.878,2.466)--(-1.799,2.434)--(-1.811,2.332);
\filldraw[fill opacity=0.8,fill=gray!20,draw=none](-2.427,1.691)--(-2.428,1.703)--(-2.428,1.729)--(-2.356,1.724)--(-2.359,1.686)--cycle;
\draw(-2.428,1.703)--(-2.428,1.729)--(-2.356,1.724)--(-2.359,1.686)--(-2.427,1.691);
\filldraw[fill opacity=0.8,fill=gray!20,draw=none](-2.437,1.728)--(-2.428,1.729)--(-2.428,1.703)--cycle;
\draw(-2.437,1.728)--(-2.428,1.729)--(-2.428,1.703);
\filldraw[fill opacity=0.5,fill=gray!20](-1.208,3.379)--(-1.299,3.434)--(-.899,3.812)--(-.816,3.751)--cycle;
\filldraw[fill opacity=0.8,fill=gray!20](-2.897,.89)--(-2.896,.946)--(-2.792,.939)--(-2.804,.883)--cycle;
\filldraw[fill opacity=0.8,fill=gray!20](-2.896,.946)--(-2.895,1.003)--(-2.788,.995)--(-2.792,.939)--cycle;
\filldraw[fill opacity=0.8,fill=gray!20](-2.994,.885)--(-3.003,.941)--(-2.896,.946)--(-2.897,.89)--cycle;
\filldraw[fill opacity=0.8,fill=gray!20,draw=none](-2.437,1.728)--(-2.451,1.738)--(-2.444,1.764)--(-2.428,1.765)--(-2.428,1.729)--cycle;
\draw(-2.444,1.764)--(-2.428,1.765)--(-2.428,1.729)--(-2.437,1.728);
\filldraw[fill opacity=0.8,fill=gray!20](-2.428,1.729)--(-2.428,1.765)--(-2.359,1.76)--(-2.356,1.724)--cycle;
\filldraw[fill opacity=0.8,fill=gray!20,draw=none](-2.074,1.989)--(-3.587,2.056)--(-3.61,2.056)--(-2.034,1.987)--cycle;
\draw(-2.074,1.989)--(-3.587,2.056)--(-3.61,2.056)--(-2.034,1.987);
\filldraw[fill opacity=0.8,fill=gray!20](-2.895,1.003)--(-2.896,1.057)--(-2.792,1.05)--(-2.788,.995)--cycle;
\filldraw[fill opacity=0.8,fill=gray!20,draw=none](-2.383,1.615)--(-2.413,1.644)--(-2.412,1.652)--(-2.367,1.649)--(-2.38,1.615)--cycle;
\draw(-2.412,1.652)--(-2.367,1.649)--(-2.38,1.615)--(-2.383,1.615);
\filldraw[fill opacity=0.8,fill=gray!20](-2.899,.838)--(-2.897,.89)--(-2.804,.883)--(-2.823,.832)--cycle;
\filldraw[fill opacity=0.8,fill=gray!20](-2.978,.834)--(-2.994,.885)--(-2.897,.89)--(-2.899,.838)--cycle;
\filldraw[fill opacity=0.8,fill=gray!20](-2.542,1.641)--(-2.555,1.677)--(-2.5,1.688)--(-2.493,1.65)--cycle;
\filldraw[fill opacity=0.8,fill=gray!20](-2.555,1.677)--(-2.559,1.715)--(-2.502,1.726)--(-2.5,1.688)--cycle;
\filldraw[fill opacity=0.8,fill=gray!20](-2.359,1.686)--(-2.356,1.724)--(-2.306,1.711)--(-2.31,1.674)--cycle;
\filldraw[fill opacity=0.8,fill=gray!20](-2.31,1.674)--(-2.306,1.711)--(-2.29,1.695)--(-2.295,1.658)--cycle;
\filldraw[fill opacity=0.8,fill=gray!20](-2.306,1.711)--(-2.31,1.748)--(-2.295,1.732)--(-2.29,1.695)--cycle;
\filldraw[fill opacity=0.8,fill=gray!20,draw=none](-2.318,1.644)--(-2.318,1.644)--(-2.295,1.658)--(-2.29,1.695)--(-2.314,1.679)--(-2.317,1.655)--cycle;
\draw(-2.318,1.644)--(-2.295,1.658)--(-2.29,1.695)--(-2.314,1.679)--(-2.317,1.655);
\filldraw[fill opacity=0.8,fill=gray!20](-2.356,1.724)--(-2.359,1.76)--(-2.31,1.748)--(-2.306,1.711)--cycle;
\filldraw[fill opacity=0.8,fill=gray!20](-2.29,1.695)--(-2.295,1.732)--(-2.319,1.717)--(-2.314,1.679)--cycle;
\filldraw[fill opacity=0.8,fill=gray!20,draw=none](-2.317,1.655)--(-2.314,1.679)--(-2.321,1.678)--cycle;
\draw(-2.317,1.655)--(-2.314,1.679)--(-2.321,1.678);
\filldraw[fill opacity=0.8,fill=gray!20,draw=none](-2.333,1.686)--(-2.321,1.678)--(-2.314,1.679)--(-2.319,1.717)--(-2.331,1.715)--cycle;
\draw(-2.321,1.678)--(-2.314,1.679)--(-2.319,1.717)--(-2.331,1.715);
\filldraw[fill opacity=0.8,fill=gray!20,draw=none](-1.869,1.842)--(-2.378,1.671)--(-2.373,1.626)--(-1.868,1.795)--cycle;
\draw(-1.869,1.842)--(-2.378,1.671)--(-2.373,1.626)--(-1.868,1.795);
\filldraw[fill opacity=0.8,fill=gray!20](-2.559,1.715)--(-2.555,1.751)--(-2.5,1.762)--(-2.502,1.726)--cycle;
\filldraw[fill opacity=0.5,fill=gray!20,draw=none](-1.89,2.241)--(-1.872,2.24)--(-1.865,2.21)--(-1.879,2.221)--cycle;
\filldraw[fill opacity=0.8,fill=gray!20,draw=none](-2.885,1.2)--(-3.117,1.199)--(-3.058,1.218)--(-2.956,1.238)--(-2.85,1.247)--(-2.776,1.244)--(-2.765,1.226)--cycle;
\draw(-3.117,1.199)--(-3.058,1.218)--(-2.956,1.238)--(-2.85,1.247)--(-2.776,1.244);
\filldraw[fill opacity=0.8,fill=gray!20,draw=none](-2.866,1.371)--(-2.854,1.278)--(-2.879,1.245)--(-2.956,1.238)--(-2.961,1.283)--cycle;
\draw(-2.866,1.371)--(-2.854,1.278);
\draw(-2.879,1.245)--(-2.956,1.238)--(-2.961,1.283);
\filldraw[fill opacity=0.8,fill=gray!20,draw=none](-3.002,1.6)--(-2.956,1.238)--(-3.058,1.218)--(-3.095,1.504)--cycle;
\draw(-3.002,1.6)--(-2.956,1.238)--(-3.058,1.218)--(-3.095,1.504);
\filldraw[fill opacity=0.8,fill=gray!20](-2.523,1.608)--(-2.542,1.641)--(-2.493,1.65)--(-2.483,1.616)--cycle;
\filldraw[fill opacity=0.5,fill=gray!20](-1.58,1.328)--(-1.698,1.33)--(-1.763,1.858)--(-1.642,1.838)--cycle;
\filldraw[fill opacity=0.8,fill=gray!20](-3.086,.926)--(-3.092,.982)--(-3.007,.998)--(-3.003,.941)--cycle;
\filldraw[fill opacity=0.8,fill=gray!20](-3.067,.871)--(-3.086,.926)--(-3.003,.941)--(-2.994,.885)--cycle;
\filldraw[fill opacity=0.8,fill=gray!20,draw=none](-2.383,1.611)--(-2.383,1.615)--(-2.38,1.615)--(-2.382,1.611)--cycle;
\draw(-2.383,1.615)--(-2.38,1.615)--(-2.382,1.611);
\filldraw[fill opacity=0.8,fill=gray!20,draw=none](-2.461,1.679)--(-2.449,1.693)--(-2.427,1.675)--(-2.449,1.649)--cycle;
\draw(-2.461,1.679)--(-2.449,1.693)--(-2.427,1.675)--(-2.449,1.649);
\filldraw[fill opacity=0.8,fill=gray!20,draw=none](-2.469,1.711)--(-2.449,1.693)--(-2.461,1.679)--cycle;
\draw(-2.469,1.711)--(-2.449,1.693)--(-2.461,1.679);
\filldraw[fill opacity=0.8,fill=gray!20,draw=none](-2.389,1.666)--(-2.381,1.655)--(-2.381,1.65)--(-2.39,1.652)--(-2.406,1.661)--(-2.427,1.675)--(-2.449,1.693)--(-2.469,1.711)--(-2.404,1.683)--cycle;
\draw(-2.404,1.683)--(-2.389,1.666)--(-2.381,1.655)--(-2.381,1.65)--(-2.39,1.652)--(-2.406,1.661)--(-2.427,1.675)--(-2.449,1.693)--(-2.469,1.711);
\filldraw[fill opacity=0.8,fill=gray!20,draw=none](-2.449,1.649)--(-2.427,1.675)--(-2.406,1.661)--(-2.434,1.627)--cycle;
\draw(-2.449,1.649)--(-2.427,1.675)--(-2.406,1.661)--(-2.434,1.627);
\filldraw[fill opacity=0.8,fill=gray!20,draw=none](-2.434,1.627)--(-2.406,1.661)--(-2.39,1.652)--(-2.421,1.615)--cycle;
\draw(-2.434,1.627)--(-2.406,1.661)--(-2.39,1.652)--(-2.421,1.615);
\filldraw[fill opacity=0.8,fill=gray!20,draw=none](-2.421,1.615)--(-2.39,1.652)--(-2.381,1.65)--(-2.409,1.616)--cycle;
\draw(-2.421,1.615)--(-2.39,1.652)--(-2.381,1.65)--(-2.409,1.616);
\filldraw[fill opacity=0.8,fill=gray!20,draw=none](-2.409,1.616)--(-2.381,1.65)--(-2.381,1.655)--(-2.402,1.629)--cycle;
\draw(-2.409,1.616)--(-2.381,1.65)--(-2.381,1.655)--(-2.402,1.629);
\filldraw[fill opacity=0.8,fill=gray!20,draw=none](-2.402,1.629)--(-2.381,1.655)--(-2.389,1.666)--(-2.4,1.652)--cycle;
\draw(-2.402,1.629)--(-2.381,1.655)--(-2.389,1.666)--(-2.4,1.652);
\filldraw[fill opacity=0.8,fill=gray!20,draw=none](-2.4,1.652)--(-2.389,1.666)--(-2.404,1.683)--cycle;
\draw(-2.4,1.652)--(-2.389,1.666)--(-2.404,1.683);
\filldraw[fill opacity=0.8,fill=gray!20](-2.391,1.72)--(-2.412,1.766)--(-2.436,1.801)--(-2.461,1.82)--(-2.481,1.821)--(-2.495,1.802)--(-2.5,1.768)--(-2.496,1.723)--(-2.482,1.674)--(-2.461,1.628)--(-2.437,1.593)--(-2.413,1.574)--(-2.392,1.573)--(-2.378,1.592)--(-2.373,1.626)--(-2.378,1.671)--cycle;
\filldraw[fill opacity=0.8,fill=gray!20,draw=none](-1.889,1.941)--(-1.9,1.937)--(-1.88,1.892)--(-1.879,1.892)--cycle;
\draw(-1.889,1.941)--(-1.9,1.937);
\draw(-1.88,1.892)--(-1.879,1.892);
\filldraw[fill opacity=0.5,fill=gray!20](-.632,3.573)--(-.726,3.67)--(-.284,3.918)--(-.21,3.81)--cycle;
\filldraw[fill opacity=0.5,fill=gray!20](-.535,3.4)--(-.632,3.573)--(-.21,3.81)--(-.16,3.611)--cycle;
\filldraw[fill opacity=0.8,fill=gray!20](-3.092,.982)--(-3.086,1.036)--(-3.003,1.052)--(-3.007,.998)--cycle;
\filldraw[fill opacity=0.8,fill=gray!20,draw=none](-2.854,1.278)--(-2.85,1.247)--(-2.879,1.245)--cycle;
\draw(-2.854,1.278)--(-2.85,1.247)--(-2.879,1.245);
\filldraw[fill opacity=0.8,fill=gray!20](-2.295,1.732)--(-2.31,1.767)--(-2.331,1.753)--(-2.319,1.717)--cycle;
\filldraw[fill opacity=0.8,fill=gray!20,draw=none](-2.343,1.751)--(-2.331,1.753)--(-2.35,1.786)--(-2.36,1.784)--cycle;
\draw(-2.343,1.751)--(-2.331,1.753)--(-2.35,1.786)--(-2.36,1.784);
\filldraw[fill opacity=0.8,fill=gray!20,draw=none](-2.347,1.735)--(-2.331,1.715)--(-2.319,1.717)--(-2.331,1.753)--(-2.343,1.751)--cycle;
\draw(-2.331,1.715)--(-2.319,1.717)--(-2.331,1.753)--(-2.343,1.751);
\filldraw[fill opacity=0.8,fill=gray!20,draw=none](-1.9,1.937)--(-2.412,1.766)--(-2.391,1.72)--(-1.88,1.892)--cycle;
\draw(-1.9,1.937)--(-2.412,1.766)--(-2.391,1.72)--(-1.88,1.892);
\filldraw[fill opacity=0.5,fill=gray!20](-1.799,2.434)--(-1.878,2.466)--(-1.684,2.998)--(-1.606,2.964)--cycle;
\filldraw[fill opacity=0.8,fill=gray!20](-3.038,.823)--(-3.067,.871)--(-2.994,.885)--(-2.978,.834)--cycle;
\filldraw[fill opacity=0.8,fill=gray!20](-2.498,1.582)--(-2.523,1.608)--(-2.483,1.616)--(-2.469,1.588)--cycle;
\filldraw[fill opacity=0.5,fill=gray!20,draw=none](-1.848,2.027)--(-1.85,2.003)--(-1.883,2)--cycle;
\draw(-1.848,2.027)--(-1.85,2.003);
\filldraw[fill opacity=0.8,fill=gray!20,draw=none](-2.778,1.402)--(-2.758,1.244)--(-2.85,1.247)--(-2.885,1.523)--cycle;
\draw(-2.778,1.402)--(-2.758,1.244)--(-2.85,1.247)--(-2.885,1.523);
\filldraw[fill opacity=0.5,fill=gray!20,draw=none](-1.86,2.184)--(-1.879,2.221)--(-1.865,2.21)--cycle;
\filldraw[fill opacity=0.5,fill=gray!20,draw=none](-1.861,2.038)--(-1.868,2.012)--(-1.882,2.002)--cycle;
\filldraw[fill opacity=0.8,fill=gray!20,draw=none](-1.861,2.038)--(-3.546,2.112)--(-3.564,2.076)--(-1.882,2.002)--cycle;
\draw(-1.861,2.038)--(-3.546,2.112)--(-3.564,2.076)--(-1.882,2.002);
\filldraw[fill opacity=0.8,fill=gray!20](-2.958,.791)--(-2.978,.834)--(-2.899,.838)--(-2.902,.794)--cycle;
\filldraw[fill opacity=0.8,fill=gray!20](-2.902,.794)--(-2.899,.838)--(-2.823,.832)--(-2.848,.79)--cycle;
\filldraw[fill opacity=0.8,fill=gray!20](-2.792,.939)--(-2.788,.995)--(-2.712,.977)--(-2.719,.921)--cycle;
\filldraw[fill opacity=0.8,fill=gray!20](-2.804,.883)--(-2.792,.939)--(-2.719,.921)--(-2.739,.867)--cycle;
\filldraw[fill opacity=0.8,fill=gray!20,draw=none](-1.879,1.886)--(-1.88,1.838)--(-1.869,1.842)--cycle;
\draw(-1.88,1.838)--(-1.869,1.842);
\filldraw[fill opacity=0.8,fill=gray!20,draw=none](-1.879,1.886)--(-1.88,1.892)--(-2.391,1.72)--(-2.378,1.671)--(-1.88,1.838)--cycle;
\draw(-1.88,1.892)--(-2.391,1.72)--(-2.378,1.671)--(-1.88,1.838);
\filldraw[fill opacity=0.8,fill=gray!20](-2.788,.995)--(-2.792,1.05)--(-2.719,1.032)--(-2.712,.977)--cycle;
\filldraw[fill opacity=0.8,fill=gray!20](-2.823,.832)--(-2.804,.883)--(-2.739,.867)--(-2.77,.819)--cycle;
\filldraw[fill opacity=0.8,fill=gray!20,draw=none](-1.879,1.892)--(-1.88,1.892)--(-1.879,1.886)--cycle;
\draw(-1.879,1.892)--(-1.88,1.892);
\filldraw[fill opacity=0.5,fill=gray!20,draw=none](-1.872,2.24)--(-1.822,2.24)--(-1.829,2.182)--(-1.865,2.21)--cycle;
\draw(-1.822,2.24)--(-1.829,2.182);
\filldraw[fill opacity=0.8,fill=gray!20](-2.454,1.566)--(-2.469,1.588)--(-2.432,1.589)--(-2.434,1.567)--cycle;
\filldraw[fill opacity=0.8,fill=gray!20](-3,.783)--(-3.038,.823)--(-2.978,.834)--(-2.958,.791)--cycle;
\filldraw[fill opacity=0.8,fill=gray!20](-2.468,1.564)--(-2.498,1.582)--(-2.469,1.588)--(-2.454,1.566)--cycle;
\filldraw[fill opacity=0.5,fill=gray!20,draw=none](-1.86,2.184)--(-1.865,2.21)--(-1.829,2.182)--(-1.835,2.137)--cycle;
\draw(-1.829,2.182)--(-1.835,2.137);
\filldraw[fill opacity=0.5,fill=gray!20,draw=none](-1.861,2.038)--(-1.842,2.073)--(-1.848,2.027)--(-1.868,2.012)--cycle;
\draw(-1.842,2.073)--(-1.848,2.027);
\filldraw[fill opacity=0.8,fill=gray!20,draw=none](-2.897,1.104)--(-2.898,1.113)--(-2.857,1.138)--(-2.823,1.135)--(-2.804,1.097)--cycle;
\draw(-2.857,1.138)--(-2.823,1.135)--(-2.804,1.097)--(-2.897,1.104)--(-2.898,1.113);
\filldraw[fill opacity=0.8,fill=gray!20](-2.848,.79)--(-2.823,.832)--(-2.77,.819)--(-2.81,.781)--cycle;
\filldraw[fill opacity=0.8,fill=gray!20](-2.792,1.05)--(-2.804,1.097)--(-2.739,1.081)--(-2.719,1.032)--cycle;
\filldraw[fill opacity=0.8,fill=gray!20,draw=none](-2.898,1.113)--(-2.899,1.141)--(-2.857,1.138)--cycle;
\draw(-2.898,1.113)--(-2.899,1.141)--(-2.857,1.138);
\filldraw[fill opacity=0.8,fill=gray!20,draw=none](-3.039,1.167)--(-3.151,1.185)--(-3.142,1.19)--(-3.117,1.199)--(-2.885,1.2)--cycle;
\draw(-3.151,1.185)--(-3.142,1.19)--(-3.117,1.199);
\filldraw[fill opacity=0.8,fill=gray!20,draw=none](-2.565,1.631)--(-2.565,1.631)--(-2.561,1.629)--(-2.55,1.614)--(-2.556,1.607)--cycle;
\draw(-2.565,1.631)--(-2.565,1.631);
\draw(-2.55,1.614)--(-2.556,1.607);
\filldraw[fill opacity=0.8,fill=gray!20,draw=none](-2.565,1.631)--(-2.564,1.633)--(-2.561,1.629)--cycle;
\draw(-2.565,1.631)--(-2.564,1.633);
\filldraw[fill opacity=0.8,fill=gray!20,draw=none](-2.566,1.634)--(-2.564,1.633)--(-2.565,1.631)--cycle;
\draw(-2.564,1.633)--(-2.565,1.631);
\filldraw[fill opacity=0.8,fill=gray!20](-2.563,1.627)--(-2.578,1.662)--(-2.555,1.677)--(-2.542,1.641)--cycle;
\filldraw[fill opacity=0.8,fill=gray!20](-2.578,1.662)--(-2.583,1.699)--(-2.559,1.715)--(-2.555,1.677)--cycle;
\filldraw[fill opacity=0.8,fill=gray!20,draw=none](-2.754,1.369)--(-2.55,1.614)--(-2.532,1.602)--(-2.528,1.598)--(-2.747,1.335)--cycle;
\draw(-2.754,1.369)--(-2.55,1.614);
\draw(-2.528,1.598)--(-2.747,1.335);
\filldraw[fill opacity=0.8,fill=gray!20,draw=none](-2.55,1.614)--(-2.546,1.619)--(-2.532,1.602)--cycle;
\draw(-2.55,1.614)--(-2.546,1.619);
\filldraw[fill opacity=0.8,fill=gray!20,draw=none](-2.561,1.629)--(-2.546,1.619)--(-2.55,1.614)--cycle;
\draw(-2.546,1.619)--(-2.55,1.614);
\filldraw[fill opacity=0.8,fill=gray!20](-2.54,1.597)--(-2.563,1.627)--(-2.542,1.641)--(-2.523,1.608)--cycle;
\filldraw[fill opacity=0.5,fill=gray!20,draw=none](-1.849,1.878)--(-1.865,1.882)--(-1.851,1.997)--cycle;
\draw(-1.849,1.878)--(-1.865,1.882)--(-1.851,1.997);
\filldraw[fill opacity=0.5,fill=gray!20,draw=none](-1.85,2.137)--(-1.86,2.184)--(-1.849,2.165)--cycle;
\filldraw[fill opacity=0.5,fill=gray!20,draw=none](-1.85,2.085)--(-1.851,2.058)--(-1.861,2.038)--cycle;
\filldraw[fill opacity=0.8,fill=gray!20](-2.583,1.699)--(-2.578,1.736)--(-2.555,1.751)--(-2.559,1.715)--cycle;
\filldraw[fill opacity=0.8,fill=gray!20,draw=none](-2.747,1.335)--(-2.527,1.599)--(-2.506,1.58)--(-2.734,1.307)--cycle;
\draw(-2.747,1.335)--(-2.527,1.599);
\draw(-2.506,1.58)--(-2.734,1.307);
\filldraw[fill opacity=0.8,fill=gray!20,draw=none](-2.532,1.602)--(-2.527,1.599)--(-2.528,1.598)--cycle;
\draw(-2.527,1.599)--(-2.528,1.598);
\filldraw[fill opacity=0.8,fill=gray!20,draw=none](-2.506,1.579)--(-2.506,1.58)--(-2.504,1.578)--cycle;
\draw(-2.506,1.579)--(-2.506,1.58);
\filldraw[fill opacity=0.8,fill=gray!20,draw=none](-2.509,1.576)--(-2.506,1.579)--(-2.504,1.578)--(-2.499,1.571)--cycle;
\draw(-2.509,1.576)--(-2.506,1.579);
\filldraw[fill opacity=0.8,fill=gray!20,draw=none](-2.714,1.33)--(-2.509,1.576)--(-2.499,1.571)--(-2.49,1.56)--(-2.688,1.322)--cycle;
\draw(-2.714,1.33)--(-2.509,1.576);
\draw(-2.49,1.56)--(-2.688,1.322);
\filldraw[fill opacity=0.8,fill=gray!20](-2.51,1.574)--(-2.54,1.597)--(-2.523,1.608)--(-2.498,1.582)--cycle;
\filldraw[fill opacity=0.8,fill=gray!20](-2.934,.76)--(-2.958,.791)--(-2.902,.794)--(-2.905,.761)--cycle;
\filldraw[fill opacity=0.8,fill=gray!20](-2.905,.761)--(-2.902,.794)--(-2.848,.79)--(-2.877,.759)--cycle;
\filldraw[fill opacity=0.8,fill=gray!20,draw=none](-2.527,1.599)--(-2.461,1.679)--(-2.449,1.649)--(-2.506,1.58)--cycle;
\draw(-2.527,1.599)--(-2.461,1.679);
\draw(-2.449,1.649)--(-2.506,1.58);
\filldraw[fill opacity=0.8,fill=gray!20,draw=none](-2.532,1.602)--(-2.546,1.619)--(-2.469,1.711)--(-2.461,1.679)--(-2.527,1.599)--cycle;
\draw(-2.546,1.619)--(-2.469,1.711);
\draw(-2.461,1.679)--(-2.527,1.599);
\filldraw[fill opacity=0.8,fill=gray!20](-2.578,1.736)--(-2.563,1.77)--(-2.542,1.784)--(-2.555,1.751)--cycle;
\filldraw[fill opacity=0.8,fill=gray!20](-2.523,1.81)--(-2.498,1.83)--(-2.469,1.835)--(-2.483,1.818)--cycle;
\filldraw[fill opacity=0.8,fill=gray!20](-2.437,1.554)--(-2.454,1.566)--(-2.434,1.567)--(-2.437,1.554)--cycle;
\filldraw[fill opacity=0.8,fill=gray!20](-2.437,1.554)--(-2.434,1.567)--(-2.416,1.566)--(-2.437,1.554)--cycle;
\filldraw[fill opacity=0.8,fill=gray!20,draw=none](-2.751,1.201)--(-2.776,1.244)--(-2.758,1.244)--(-2.692,1.228)--(-2.664,1.203)--(-2.664,1.201)--cycle;
\draw(-2.776,1.244)--(-2.758,1.244)--(-2.692,1.228)--(-2.664,1.203)--(-2.664,1.201);
\filldraw[fill opacity=0.8,fill=gray!20,draw=none](-2.504,1.578)--(-2.485,1.565)--(-2.49,1.56)--cycle;
\draw(-2.485,1.565)--(-2.49,1.56);
\filldraw[fill opacity=0.8,fill=gray!20,draw=none](-2.486,1.565)--(-2.485,1.565)--(-2.483,1.563)--cycle;
\draw(-2.486,1.565)--(-2.485,1.565);
\filldraw[fill opacity=0.8,fill=gray!20](-2.474,1.56)--(-2.51,1.574)--(-2.498,1.582)--(-2.468,1.564)--cycle;
\filldraw[fill opacity=0.8,fill=gray!20](-3.067,1.086)--(-3.038,1.126)--(-2.978,1.137)--(-2.994,1.1)--cycle;
\filldraw[fill opacity=0.8,fill=gray!20](-2.437,1.554)--(-2.468,1.564)--(-2.454,1.566)--(-2.437,1.554)--cycle;
\filldraw[fill opacity=0.8,fill=gray!20,draw=none](-2.504,1.578)--(-2.506,1.58)--(-2.449,1.649)--(-2.434,1.627)--(-2.485,1.565)--cycle;
\draw(-2.506,1.58)--(-2.449,1.649);
\draw(-2.434,1.627)--(-2.485,1.565);
\filldraw[fill opacity=0.8,fill=gray!20,draw=none](-2.561,1.629)--(-2.564,1.633)--(-2.485,1.728)--(-2.469,1.711)--(-2.546,1.619)--cycle;
\draw(-2.564,1.633)--(-2.485,1.728)--(-2.469,1.711)--(-2.546,1.619);
\filldraw[fill opacity=0.8,fill=gray!20,draw=none](-2.469,1.711)--(-2.485,1.728)--(-2.493,1.739)--(-2.492,1.744)--(-2.483,1.742)--(-2.467,1.733)--(-2.446,1.719)--(-2.424,1.701)--(-2.404,1.683)--cycle;
\draw(-2.469,1.711)--(-2.485,1.728)--(-2.493,1.739)--(-2.492,1.744)--(-2.483,1.742)--(-2.467,1.733)--(-2.446,1.719)--(-2.424,1.701)--(-2.404,1.683);
\filldraw[fill opacity=0.5,fill=gray!20,draw=none](-1.735,2.092)--(-1.763,1.858)--(-1.849,1.878)--(-1.851,1.997)--(-1.848,2.027)--cycle;
\draw(-1.735,2.092)--(-1.763,1.858)--(-1.849,1.878);
\draw(-1.851,1.997)--(-1.848,2.027);
\filldraw[fill opacity=0.5,fill=gray!20,draw=none](-1.85,2.137)--(-1.849,2.165)--(-1.835,2.137)--(-1.84,2.091)--cycle;
\draw(-1.835,2.137)--(-1.84,2.091);
\filldraw[fill opacity=0.5,fill=gray!20,draw=none](-1.85,2.137)--(-1.844,2.111)--(-1.85,2.085)--cycle;
\filldraw[fill opacity=0.5,fill=gray!20,draw=none](-1.85,2.085)--(-1.844,2.111)--(-1.84,2.091)--(-1.842,2.073)--(-1.851,2.058)--cycle;
\draw(-1.84,2.091)--(-1.842,2.073);
\filldraw[fill opacity=0.8,fill=gray!20](-2.437,1.554)--(-2.416,1.566)--(-2.403,1.563)--(-2.437,1.554)--cycle;
\filldraw[fill opacity=0.8,fill=gray!20](-2.956,.755)--(-3,.783)--(-2.958,.791)--(-2.934,.76)--cycle;
\filldraw[fill opacity=0.8,fill=gray!20](-2.403,1.563)--(-2.371,1.58)--(-2.364,1.572)--(-2.399,1.559)--cycle;
\filldraw[fill opacity=0.8,fill=gray!20,draw=none](-2.718,1.273)--(-2.698,1.229)--(-2.758,1.244)--(-2.759,1.256)--cycle;
\draw(-2.698,1.229)--(-2.758,1.244)--(-2.759,1.256);
\filldraw[fill opacity=0.5,fill=gray!20,draw=none](-1.824,2.041)--(-1.848,2.027)--(-1.84,2.091)--cycle;
\draw(-1.848,2.027)--(-1.84,2.091);
\filldraw[fill opacity=0.8,fill=gray!20](-2.804,1.097)--(-2.823,1.135)--(-2.77,1.122)--(-2.739,1.081)--cycle;
\filldraw[fill opacity=0.8,fill=gray!20](-2.877,.759)--(-2.848,.79)--(-2.81,.781)--(-2.858,.754)--cycle;
\filldraw[fill opacity=0.8,fill=gray!20,draw=none](-2.726,1.291)--(-2.718,1.273)--(-2.759,1.256)--(-2.766,1.306)--cycle;
\draw(-2.759,1.256)--(-2.766,1.306);
\filldraw[fill opacity=0.8,fill=gray!20](-2.563,1.77)--(-2.54,1.799)--(-2.523,1.81)--(-2.542,1.784)--cycle;
\filldraw[fill opacity=0.8,fill=gray!20](-2.437,1.554)--(-2.474,1.56)--(-2.468,1.564)--(-2.437,1.554)--cycle;
\filldraw[fill opacity=0.8,fill=gray!20,draw=none](-3.112,.908)--(-3.107,.896)--(-3.111,.895)--cycle;
\draw(-3.107,.896)--(-3.111,.895);
\filldraw[fill opacity=0.8,fill=gray!20,draw=none](-3.107,.867)--(-3.109,.896)--(-3.107,.896)--(-3.1,.858)--cycle;
\draw(-3.109,.896)--(-3.107,.896);
\filldraw[fill opacity=0.8,fill=gray!20,draw=none](-3.107,.867)--(-3.108,.869)--(-3.116,.894)--(-3.109,.896)--cycle;
\draw(-3.116,.894)--(-3.109,.896);
\filldraw[fill opacity=0.8,fill=gray!20,draw=none](-3.128,.924)--(-3.13,.943)--(-3.127,.944)--(-3.112,.908)--(-3.111,.895)--(-3.115,.894)--cycle;
\draw(-3.13,.943)--(-3.127,.944);
\draw(-3.111,.895)--(-3.115,.894);
\filldraw[fill opacity=0.8,fill=gray!20,draw=none](-3.128,.924)--(-3.115,.894)--(-3.125,.892)--cycle;
\draw(-3.115,.894)--(-3.125,.892);
\filldraw[fill opacity=0.8,fill=gray!20,draw=none](-3.108,.869)--(-3.125,.892)--(-3.116,.894)--cycle;
\draw(-3.125,.892)--(-3.116,.894);
\filldraw[fill opacity=0.8,fill=gray!20,draw=none](-3.116,.906)--(-3.113,.876)--(-3.149,.925)--(-3.121,.923)--cycle;
\draw(-3.149,.925)--(-3.121,.923);
\filldraw[fill opacity=0.8,fill=gray!20,draw=none](-3.142,.915)--(-3.162,.869)--(-3.212,.873)--(-3.212,.93)--(-3.149,.925)--cycle;
\draw(-3.162,.869)--(-3.212,.873)--(-3.212,.93)--(-3.149,.925);
\filldraw[fill opacity=0.8,fill=gray!20](-3.31,.812)--(-3.32,.868)--(-3.212,.873)--(-3.214,.816)--cycle;
\filldraw[fill opacity=0.8,fill=gray!20,draw=none](-3.112,.865)--(-3.116,.906)--(-3.107,.88)--(-3.108,.865)--cycle;
\draw(-3.107,.88)--(-3.108,.865)--(-3.112,.865);
\filldraw[fill opacity=0.8,fill=gray!20,draw=none](-3.214,.816)--(-3.212,.873)--(-3.126,.866)--(-3.11,.856)--(-3.12,.809)--cycle;
\draw(-3.11,.856)--(-3.12,.809)--(-3.214,.816)--(-3.212,.873)--(-3.126,.866);
\filldraw[fill opacity=0.8,fill=gray!20,draw=none](-3.142,.915)--(-3.125,.892)--(-3.122,.866)--(-3.162,.869)--cycle;
\draw(-3.122,.866)--(-3.162,.869);
\filldraw[fill opacity=0.8,fill=gray!20,draw=none](-3.126,.866)--(-3.108,.865)--(-3.11,.856)--cycle;
\draw(-3.126,.866)--(-3.108,.865)--(-3.11,.856);
\filldraw[fill opacity=0.8,fill=gray!20,draw=none](-3.125,.892)--(-3.113,.876)--(-3.112,.865)--(-3.122,.866)--cycle;
\draw(-3.112,.865)--(-3.122,.866);
\filldraw[fill opacity=0.8,fill=gray!20,draw=none](-3.107,.865)--(-3.108,.865)--(-3.108,.869)--cycle;
\draw(-3.107,.865)--(-3.108,.865)--(-3.108,.869);
\filldraw[fill opacity=0.8,fill=gray!20,draw=none](-3.107,.867)--(-3.105,.865)--(-3.099,.848)--(-3.105,.846)--cycle;
\draw(-3.099,.848)--(-3.105,.846);
\filldraw[fill opacity=0.8,fill=gray!20,draw=none](-3.103,.864)--(-3.104,.864)--(-3.108,.871)--(-3.107,.88)--cycle;
\draw(-3.103,.864)--(-3.104,.864);
\draw(-3.108,.871)--(-3.107,.88);
\filldraw[fill opacity=0.8,fill=gray!20,draw=none](-3.105,.865)--(-3.1,.858)--(-3.098,.848)--(-3.099,.848)--cycle;
\draw(-3.098,.848)--(-3.099,.848);
\filldraw[fill opacity=0.8,fill=gray!20](-3.098,.851)--(-3.12,.903)--(-3.086,.926)--(-3.067,.871)--cycle;
\filldraw[fill opacity=0.8,fill=gray!20,draw=none](-3.127,.944)--(-3.115,.947)--(-3.112,.908)--cycle;
\draw(-3.127,.944)--(-3.115,.947);
\filldraw[fill opacity=0.8,fill=gray!20,draw=none](-3.12,.964)--(-3.115,.947)--(-3.121,.946)--cycle;
\draw(-3.115,.947)--(-3.121,.946);
\filldraw[fill opacity=0.8,fill=gray!20](-3.32,.868)--(-3.323,.925)--(-3.212,.93)--(-3.212,.873)--cycle;
\filldraw[fill opacity=0.8,fill=gray!20,draw=none](-3.133,.978)--(-3.131,.99)--(-3.127,.991)--(-3.12,.964)--(-3.121,.946)--(-3.124,.945)--cycle;
\draw(-3.131,.99)--(-3.127,.991);
\draw(-3.121,.946)--(-3.124,.945);
\filldraw[fill opacity=0.8,fill=gray!20,draw=none](-3.133,.978)--(-3.124,.945)--(-3.137,.942)--cycle;
\draw(-3.124,.945)--(-3.137,.942);
\filldraw[fill opacity=0.8,fill=gray!20,draw=none](-3.148,.968)--(-3.179,.927)--(-3.212,.93)--(-3.212,.984)--(-3.153,.979)--cycle;
\draw(-3.179,.927)--(-3.212,.93)--(-3.212,.984)--(-3.153,.979);
\filldraw[fill opacity=0.8,fill=gray!20,draw=none](-3.137,.942)--(-3.13,.943)--(-3.128,.924)--cycle;
\draw(-3.137,.942)--(-3.13,.943);
\filldraw[fill opacity=0.8,fill=gray!20,draw=none](-3.123,.961)--(-3.127,.924)--(-3.128,.924)--(-3.137,.942)--(-3.133,.978)--(-3.126,.977)--cycle;
\draw(-3.127,.924)--(-3.128,.924);
\draw(-3.133,.978)--(-3.126,.977);
\filldraw[fill opacity=0.8,fill=gray!20,draw=none](-3.123,.961)--(-3.117,.923)--(-3.127,.924)--cycle;
\draw(-3.117,.923)--(-3.127,.924);
\filldraw[fill opacity=0.8,fill=gray!20,draw=none](-3.116,.906)--(-3.121,.923)--(-3.117,.923)--cycle;
\draw(-3.121,.923)--(-3.117,.923);
\filldraw[fill opacity=0.8,fill=gray!20](-3.12,.903)--(-3.128,.958)--(-3.092,.982)--(-3.086,.926)--cycle;
\filldraw[fill opacity=0.8,fill=gray!20](-3.064,.806)--(-3.098,.851)--(-3.067,.871)--(-3.038,.823)--cycle;
\filldraw[fill opacity=0.8,fill=gray!20,draw=none](-2.462,1.831)--(-2.475,1.828)--(-2.469,1.835)--(-2.468,1.835)--cycle;
\draw(-2.475,1.828)--(-2.469,1.835)--(-2.468,1.835);
\filldraw[fill opacity=0.8,fill=gray!20,draw=none](-2.471,1.555)--(-2.483,1.563)--(-2.474,1.56)--cycle;
\draw(-2.483,1.563)--(-2.474,1.56)--(-2.471,1.555);
\filldraw[fill opacity=0.8,fill=gray!20](-2.437,1.554)--(-2.471,1.555)--(-2.474,1.56)--(-2.437,1.554)--cycle;
\filldraw[fill opacity=0.8,fill=gray!20,draw=none](-2.485,1.565)--(-2.434,1.627)--(-2.421,1.615)--(-2.473,1.552)--cycle;
\draw(-2.485,1.565)--(-2.434,1.627);
\draw(-2.421,1.615)--(-2.473,1.552);
\filldraw[fill opacity=0.8,fill=gray!20](-2.437,1.554)--(-2.403,1.563)--(-2.399,1.559)--(-2.437,1.554)--cycle;
\filldraw[fill opacity=0.5,fill=gray!20,draw=none](-1.827,2.181)--(-1.731,2.124)--(-1.735,2.092)--(-1.824,2.041)--(-1.84,2.091)--(-1.831,2.167)--cycle;
\draw(-1.731,2.124)--(-1.735,2.092);
\draw(-1.84,2.091)--(-1.831,2.167);
\filldraw[fill opacity=0.8,fill=gray!20,draw=none](-2.462,1.831)--(-2.468,1.835)--(-2.442,1.836)--cycle;
\draw(-2.468,1.835)--(-2.442,1.836);
\filldraw[fill opacity=0.8,fill=gray!20,draw=none](-2.569,1.647)--(-2.57,1.643)--(-2.566,1.634)--cycle;
\draw(-2.57,1.643)--(-2.566,1.634);
\filldraw[fill opacity=0.8,fill=gray!20,draw=none](-2.569,1.647)--(-2.572,1.656)--(-2.577,1.661)--(-2.57,1.643)--cycle;
\draw(-2.577,1.661)--(-2.57,1.643);
\filldraw[fill opacity=0.8,fill=gray!20,draw=none](-2.576,1.638)--(-2.493,1.739)--(-2.485,1.728)--(-2.564,1.633)--cycle;
\draw(-2.576,1.638)--(-2.493,1.739)--(-2.485,1.728)--(-2.564,1.633);
\filldraw[fill opacity=0.5,fill=gray!20](-1.259,1.412)--(-1.447,1.337)--(-1.507,1.824)--(-1.312,1.846)--cycle;
\filldraw[fill opacity=0.8,fill=gray!20](-2.469,1.835)--(-2.454,1.842)--(-2.434,1.843)--(-2.432,1.837)--cycle;
\filldraw[fill opacity=0.8,fill=gray!20,draw=none](-3.095,.847)--(-3.097,.849)--(-3.097,.849)--(-3.095,.847)--cycle;
\draw(-3.097,.849)--(-3.095,.847);
\filldraw[fill opacity=0.8,fill=gray!20,draw=none](-3.097,.849)--(-3.098,.851)--(-3.097,.849)--cycle;
\draw(-3.098,.851)--(-3.097,.849);
\filldraw[fill opacity=0.8,fill=gray!20,draw=none](-3.107,.896)--(-2.958,.931)--(-2.941,.884)--(-3.098,.848)--cycle;
\draw(-3.107,.896)--(-2.958,.931)--(-2.941,.884)--(-3.098,.848);
\filldraw[fill opacity=0.8,fill=gray!20,draw=none](-3.112,.908)--(-3.115,.947)--(-2.968,.981)--(-2.958,.931)--(-3.107,.896)--cycle;
\draw(-3.115,.947)--(-2.968,.981)--(-2.958,.931)--(-3.107,.896);
\filldraw[fill opacity=0.8,fill=gray!20,draw=none](-2.899,1.141)--(-2.902,1.165)--(-2.858,1.162)--(-2.843,1.156)--(-2.823,1.135)--cycle;
\draw(-2.843,1.156)--(-2.823,1.135)--(-2.899,1.141)--(-2.902,1.165)--(-2.858,1.162);
\filldraw[fill opacity=0.8,fill=gray!20](-2.978,1.137)--(-2.958,1.163)--(-2.902,1.165)--(-2.899,1.141)--cycle;
\filldraw[fill opacity=0.8,fill=gray!20,draw=none](-2.858,1.162)--(-2.902,1.165)--(-2.904,1.17)--cycle;
\draw(-2.858,1.162)--(-2.902,1.165)--(-2.904,1.17);
\filldraw[fill opacity=0.8,fill=gray!20,draw=none](-2.858,1.162)--(-2.848,1.161)--(-2.843,1.156)--cycle;
\draw(-2.858,1.162)--(-2.848,1.161)--(-2.843,1.156);
\filldraw[fill opacity=0.8,fill=gray!20,draw=none](-2.858,1.162)--(-2.87,1.164)--(-2.852,1.163)--(-2.848,1.161)--cycle;
\draw(-2.852,1.163)--(-2.848,1.161)--(-2.858,1.162);
\filldraw[fill opacity=0.8,fill=gray!20,draw=none](-2.948,1.077)--(-2.951,1.089)--(-2.887,1.167)--(-2.853,1.163)--(-2.915,1.088)--cycle;
\draw(-2.951,1.089)--(-2.887,1.167);
\draw(-2.853,1.163)--(-2.915,1.088);
\filldraw[fill opacity=0.8,fill=gray!20,draw=none](-2.823,1.135)--(-2.848,1.161)--(-2.835,1.158)--(-2.798,1.143)--(-2.77,1.122)--cycle;
\draw(-2.798,1.143)--(-2.77,1.122)--(-2.823,1.135)--(-2.848,1.161)--(-2.835,1.158);
\filldraw[fill opacity=0.8,fill=gray!20,draw=none](-2.941,1.057)--(-2.853,1.163)--(-2.835,1.158)--(-2.827,1.155)--(-2.929,1.033)--cycle;
\draw(-2.941,1.057)--(-2.853,1.163);
\draw(-2.827,1.155)--(-2.929,1.033);
\filldraw[fill opacity=0.8,fill=gray!20,draw=none](-2.827,1.155)--(-2.808,1.15)--(-2.798,1.143)--cycle;
\draw(-2.808,1.15)--(-2.798,1.143);
\filldraw[fill opacity=0.8,fill=gray!20,draw=none](-2.929,1.033)--(-2.827,1.155)--(-2.808,1.15)--(-2.915,1.02)--cycle;
\draw(-2.929,1.033)--(-2.827,1.155);
\draw(-2.808,1.15)--(-2.915,1.02);
\filldraw[fill opacity=0.8,fill=gray!20,draw=none](-2.948,1.077)--(-2.915,1.088)--(-2.941,1.057)--cycle;
\draw(-2.915,1.088)--(-2.941,1.057);
\filldraw[fill opacity=0.8,fill=gray!20,draw=none](-2.915,1.02)--(-2.808,1.15)--(-2.802,1.144)--(-2.904,1.022)--cycle;
\draw(-2.915,1.02)--(-2.808,1.15);
\draw(-2.802,1.144)--(-2.904,1.022);
\filldraw[fill opacity=0.8,fill=gray!20,draw=none](-2.699,1.281)--(-2.692,1.228)--(-2.698,1.229)--(-2.718,1.273)--cycle;
\draw(-2.699,1.281)--(-2.692,1.228)--(-2.698,1.229);
\filldraw[fill opacity=0.8,fill=gray!20,draw=none](-2.755,1.402)--(-2.565,1.631)--(-2.556,1.607)--(-2.754,1.369)--cycle;
\draw(-2.755,1.402)--(-2.565,1.631);
\draw(-2.556,1.607)--(-2.754,1.369);
\filldraw[fill opacity=0.8,fill=gray!20,draw=none](-2.694,1.271)--(-2.697,1.265)--(-2.699,1.281)--cycle;
\draw(-2.697,1.265)--(-2.699,1.281);
\filldraw[fill opacity=0.8,fill=gray!20,draw=none](-2.717,1.287)--(-2.486,1.565)--(-2.483,1.563)--(-2.473,1.552)--(-2.699,1.281)--cycle;
\draw(-2.717,1.287)--(-2.486,1.565);
\draw(-2.473,1.552)--(-2.699,1.281);
\filldraw[fill opacity=0.8,fill=gray!20,draw=none](-2.73,1.454)--(-2.576,1.638)--(-2.566,1.634)--(-2.565,1.631)--(-2.741,1.419)--cycle;
\draw(-2.73,1.454)--(-2.576,1.638);
\draw(-2.565,1.631)--(-2.741,1.419);
\filldraw[fill opacity=0.8,fill=gray!20,draw=none](-2.699,1.281)--(-2.47,1.555)--(-2.463,1.551)--(-2.692,1.277)--cycle;
\draw(-2.699,1.281)--(-2.47,1.555);
\draw(-2.463,1.551)--(-2.692,1.277);
\filldraw[fill opacity=0.8,fill=gray!20,draw=none](-2.709,1.484)--(-2.587,1.631)--(-2.576,1.638)--(-2.73,1.454)--cycle;
\draw(-2.709,1.484)--(-2.587,1.631);
\draw(-2.576,1.638)--(-2.73,1.454);
\filldraw[fill opacity=0.8,fill=gray!20,draw=none](-2.734,1.307)--(-2.726,1.291)--(-2.766,1.306)--(-2.771,1.349)--cycle;
\draw(-2.766,1.306)--(-2.771,1.349);
\filldraw[fill opacity=0.8,fill=gray!20,draw=none](-2.904,1.022)--(-2.802,1.144)--(-2.803,1.144)--(-2.807,1.143)--(-2.895,1.037)--cycle;
\draw(-2.904,1.022)--(-2.802,1.144);
\draw(-2.807,1.143)--(-2.895,1.037);
\filldraw[fill opacity=0.8,fill=gray!20,draw=none](-2.821,1.162)--(-2.777,1.215)--(-2.757,1.211)--(-2.808,1.15)--cycle;
\draw(-2.821,1.162)--(-2.777,1.215);
\draw(-2.757,1.211)--(-2.808,1.15);
\filldraw[fill opacity=0.8,fill=gray!20,draw=none](-2.887,1.167)--(-2.853,1.207)--(-2.811,1.213)--(-2.853,1.163)--cycle;
\draw(-2.887,1.167)--(-2.853,1.207);
\draw(-2.811,1.213)--(-2.853,1.163);
\filldraw[fill opacity=0.8,fill=gray!20,draw=none](-2.852,1.165)--(-2.811,1.213)--(-2.777,1.215)--(-2.821,1.163)--cycle;
\draw(-2.852,1.165)--(-2.811,1.213);
\draw(-2.777,1.215)--(-2.821,1.163);
\filldraw[fill opacity=0.8,fill=gray!20,draw=none](-2.853,1.163)--(-2.852,1.165)--(-2.835,1.158)--cycle;
\draw(-2.853,1.163)--(-2.852,1.165);
\filldraw[fill opacity=0.8,fill=gray!20,draw=none](-2.835,1.158)--(-2.852,1.165)--(-2.821,1.163)--(-2.827,1.155)--cycle;
\draw(-2.821,1.163)--(-2.827,1.155);
\filldraw[fill opacity=0.8,fill=gray!20,draw=none](-2.827,1.155)--(-2.821,1.162)--(-2.808,1.15)--cycle;
\draw(-2.827,1.155)--(-2.821,1.162);
\filldraw[fill opacity=0.8,fill=gray!20,draw=none](-2.842,1.16)--(-2.852,1.163)--(-2.877,1.173)--(-2.858,1.168)--(-2.81,1.152)--cycle;
\draw(-2.852,1.163)--(-2.877,1.173)--(-2.858,1.168)--(-2.81,1.152)--(-2.842,1.16);
\filldraw[fill opacity=0.8,fill=gray!20,draw=none](-2.893,1.053)--(-2.895,1.037)--(-2.869,1.068)--cycle;
\draw(-2.895,1.037)--(-2.869,1.068);
\filldraw[fill opacity=0.8,fill=gray!20](-2.849,.883)--(-2.858,.848)--(-2.874,.829)--(-2.895,.829)--(-2.919,.849)--(-2.941,.884)--(-2.958,.931)--(-2.968,.981)--(-2.968,1.028)--(-2.96,1.063)--(-2.943,1.082)--(-2.922,1.082)--(-2.898,1.062)--(-2.876,1.027)--(-2.859,.98)--(-2.85,.93)--cycle;
\filldraw[fill opacity=0.8,fill=gray!20,draw=none](-3.12,.964)--(-3.127,.991)--(-3.117,.993)--cycle;
\draw(-3.127,.991)--(-3.117,.993);
\filldraw[fill opacity=0.8,fill=gray!20,draw=none](-3.123,.961)--(-3.126,.977)--(-3.122,.977)--cycle;
\draw(-3.126,.977)--(-3.122,.977);
\filldraw[fill opacity=0.8,fill=gray!20](-3.323,.925)--(-3.32,.979)--(-3.212,.984)--(-3.212,.93)--cycle;
\filldraw[fill opacity=0.8,fill=gray!20,draw=none](-3.121,1.01)--(-3.117,.993)--(-3.127,.991)--cycle;
\draw(-3.117,.993)--(-3.127,.991);
\filldraw[fill opacity=0.8,fill=gray!20,draw=none](-3.138,.924)--(-3.179,.927)--(-3.14,.978)--(-3.133,.978)--cycle;
\draw(-3.138,.924)--(-3.179,.927);
\draw(-3.14,.978)--(-3.133,.978);
\filldraw[fill opacity=0.8,fill=gray!20,draw=none](-3.135,.989)--(-3.14,.978)--(-3.212,.984)--(-3.214,1.03)--(-3.145,1.025)--cycle;
\draw(-3.14,.978)--(-3.212,.984)--(-3.214,1.03)--(-3.145,1.025);
\filldraw[fill opacity=0.8,fill=gray!20,draw=none](-3.148,.968)--(-3.153,.979)--(-3.14,.978)--cycle;
\draw(-3.153,.979)--(-3.14,.978);
\filldraw[fill opacity=0.8,fill=gray!20,draw=none](-3.135,.989)--(-3.133,.978)--(-3.14,.978)--cycle;
\draw(-3.133,.978)--(-3.14,.978);
\filldraw[fill opacity=0.8,fill=gray!20,draw=none](-3.284,.911)--(-3.23,.967)--(-3.131,.99)--(-3.137,.942)--(-3.284,.908)--cycle;
\draw(-3.23,.967)--(-3.131,.99);
\draw(-3.137,.942)--(-3.284,.908)--(-3.284,.911);
\filldraw[fill opacity=0.8,fill=gray!20,draw=none](-3.122,1.018)--(-3.121,1.01)--(-3.127,.991)--(-3.135,.989)--cycle;
\draw(-3.127,.991)--(-3.135,.989);
\filldraw[fill opacity=0.8,fill=gray!20,draw=none](-3.122,.999)--(-3.122,.977)--(-3.128,.977)--cycle;
\draw(-3.122,.977)--(-3.128,.977);
\filldraw[fill opacity=0.8,fill=gray!20](-3.128,.958)--(-3.12,1.014)--(-3.086,1.036)--(-3.092,.982)--cycle;
\filldraw[fill opacity=0.8,fill=gray!20](-3.018,.771)--(-3.064,.806)--(-3.038,.823)--(-3,.783)--cycle;
\filldraw[fill opacity=0.8,fill=gray!20,draw=none](-3.12,.964)--(-3.117,.993)--(-2.968,1.028)--(-2.968,.981)--(-3.115,.947)--cycle;
\draw(-3.117,.993)--(-2.968,1.028)--(-2.968,.981)--(-3.115,.947);
\filldraw[fill opacity=0.8,fill=gray!20](-2.498,1.83)--(-2.468,1.84)--(-2.454,1.842)--(-2.469,1.835)--cycle;
\filldraw[fill opacity=0.8,fill=gray!20,draw=none](-2.428,1.838)--(-2.433,1.838)--(-2.434,1.843)--(-2.433,1.843)--cycle;
\draw(-2.433,1.838)--(-2.434,1.843)--(-2.433,1.843);
\filldraw[fill opacity=0.5,fill=gray!20](-1.105,3.311)--(-1.208,3.379)--(-.816,3.751)--(-.726,3.67)--cycle;
\filldraw[fill opacity=0.5,fill=gray!20,draw=none](-1.827,2.181)--(-1.831,2.167)--(-1.829,2.182)--cycle;
\draw(-1.831,2.167)--(-1.829,2.182);
\filldraw[fill opacity=0.5,fill=gray!20](-1.447,1.337)--(-1.58,1.328)--(-1.642,1.838)--(-1.507,1.824)--cycle;
\filldraw[fill opacity=0.8,fill=gray!20,draw=none](-2.365,1.575)--(-2.365,1.574)--(-2.366,1.575)--cycle;
\draw(-2.365,1.574)--(-2.366,1.575);
\filldraw[fill opacity=0.8,fill=gray!20](-2.54,1.799)--(-2.51,1.822)--(-2.498,1.83)--(-2.523,1.81)--cycle;
\filldraw[fill opacity=0.8,fill=gray!20,draw=none](-2.471,1.555)--(-2.502,1.566)--(-2.51,1.574)--(-2.483,1.563)--cycle;
\draw(-2.471,1.555)--(-2.502,1.566)--(-2.51,1.574)--(-2.483,1.563);
\filldraw[fill opacity=0.8,fill=gray!20,draw=none](-2.365,1.574)--(-2.365,1.575)--(-2.337,1.592)--(-2.364,1.572)--cycle;
\draw(-2.337,1.592)--(-2.364,1.572)--(-2.365,1.574);
\filldraw[fill opacity=0.5,fill=gray!20,draw=none](-1.731,2.124)--(-1.829,2.182)--(-1.799,2.434)--(-1.698,2.4)--cycle;
\draw(-1.829,2.182)--(-1.799,2.434)--(-1.698,2.4)--(-1.731,2.124);
\filldraw[fill opacity=0.8,fill=gray!20,draw=none](-2.437,1.554)--(-2.457,1.552)--(-2.46,1.553)--(-2.456,1.555)--(-2.437,1.554)--cycle;
\draw(-2.456,1.555)--(-2.437,1.554)--(-2.437,1.554)--(-2.457,1.552)--(-2.46,1.553);
\filldraw[fill opacity=0.8,fill=gray!20](-2.502,1.566)--(-2.529,1.586)--(-2.54,1.597)--(-2.51,1.574)--cycle;
\filldraw[fill opacity=0.8,fill=gray!20](-2.437,1.554)--(-2.439,1.551)--(-2.457,1.552)--(-2.437,1.554)--cycle;
\filldraw[fill opacity=0.8,fill=gray!20](-2.437,1.554)--(-2.42,1.552)--(-2.439,1.551)--(-2.437,1.554)--cycle;
\filldraw[fill opacity=0.8,fill=gray!20](-2.437,1.554)--(-2.405,1.554)--(-2.42,1.552)--(-2.437,1.554)--cycle;
\filldraw[fill opacity=0.8,fill=gray!20](-2.437,1.554)--(-2.399,1.559)--(-2.405,1.554)--(-2.437,1.554)--cycle;
\filldraw[fill opacity=0.8,fill=gray!20](-2.739,.867)--(-2.719,.921)--(-2.697,.897)--(-2.719,.846)--cycle;
\filldraw[fill opacity=0.8,fill=gray!20](-2.77,.819)--(-2.739,.867)--(-2.719,.846)--(-2.754,.802)--cycle;
\filldraw[fill opacity=0.8,fill=gray!20,draw=none](-2.428,1.838)--(-2.433,1.843)--(-2.416,1.842)--(-2.411,1.84)--cycle;
\draw(-2.433,1.843)--(-2.416,1.842)--(-2.411,1.84);
\filldraw[fill opacity=0.8,fill=gray!20,draw=none](-3.061,.803)--(-3.078,.825)--(-3.081,.829)--(-3.064,.806)--cycle;
\draw(-3.081,.829)--(-3.064,.806)--(-3.061,.803);
\filldraw[fill opacity=0.8,fill=gray!20,draw=none](-3.078,.825)--(-3.095,.847)--(-3.095,.847)--(-3.081,.829)--cycle;
\draw(-3.095,.847)--(-3.081,.829);
\filldraw[fill opacity=0.8,fill=gray!20,draw=none](-3.069,.814)--(-3.097,.849)--(-3.078,.825)--cycle;
\filldraw[fill opacity=0.8,fill=gray!20,draw=none](-3.054,.795)--(-3.064,.806)--(-3.055,.8)--cycle;
\draw(-3.054,.795)--(-3.064,.806)--(-3.055,.8);
\filldraw[fill opacity=0.8,fill=gray!20,draw=none](-3.053,.795)--(-3.069,.814)--(-3.078,.825)--(-3.061,.803)--(-3.054,.795)--cycle;
\draw(-3.061,.803)--(-3.054,.795);
\filldraw[fill opacity=0.8,fill=gray!20,draw=none](-3.087,.81)--(-3.096,.815)--(-3.099,.834)--cycle;
\filldraw[fill opacity=0.8,fill=gray!20,draw=none](-3.097,.848)--(-2.941,.884)--(-2.919,.849)--(-3.088,.809)--cycle;
\draw(-3.097,.848)--(-2.941,.884)--(-2.919,.849)--(-3.088,.809);
\filldraw[fill opacity=0.8,fill=gray!20](-2.719,.921)--(-2.712,.977)--(-2.689,.953)--(-2.697,.897)--cycle;
\filldraw[fill opacity=0.8,fill=gray!20](-2.529,1.586)--(-2.55,1.613)--(-2.563,1.627)--(-2.54,1.597)--cycle;
\filldraw[fill opacity=0.8,fill=gray!20](-2.399,1.559)--(-2.364,1.572)--(-2.376,1.564)--(-2.405,1.554)--cycle;
\filldraw[fill opacity=0.8,fill=gray!20,draw=none](-3.039,1.167)--(-3.206,1.13)--(-3.194,1.158)--(-3.151,1.185)--cycle;
\draw(-3.206,1.13)--(-3.194,1.158)--(-3.151,1.185);
\filldraw[fill opacity=0.8,fill=gray!20,draw=none](-2.734,1.307)--(-2.771,1.349)--(-2.778,1.402)--cycle;
\draw(-2.771,1.349)--(-2.778,1.402);
\filldraw[fill opacity=0.8,fill=gray!20,draw=none](-2.46,1.553)--(-2.471,1.555)--(-2.456,1.555)--cycle;
\draw(-2.46,1.553)--(-2.471,1.555)--(-2.456,1.555);
\filldraw[fill opacity=0.8,fill=gray!20,draw=none](-2.46,1.553)--(-2.46,1.553)--(-2.462,1.554)--(-2.471,1.555)--cycle;
\draw(-2.46,1.553)--(-2.462,1.554);
\draw(-2.471,1.555)--(-2.46,1.553);
\filldraw[fill opacity=0.8,fill=gray!20,draw=none](-2.46,1.555)--(-2.462,1.554)--(-2.46,1.553)--cycle;
\draw(-2.462,1.554)--(-2.46,1.553);
\filldraw[fill opacity=0.8,fill=gray!20,draw=none](-2.47,1.555)--(-2.421,1.615)--(-2.409,1.616)--(-2.463,1.551)--cycle;
\draw(-2.47,1.555)--(-2.421,1.615);
\draw(-2.409,1.616)--(-2.463,1.551);
\filldraw[fill opacity=0.8,fill=gray!20](-3.12,1.014)--(-3.098,1.065)--(-3.067,1.086)--(-3.086,1.036)--cycle;
\filldraw[fill opacity=0.8,fill=gray!20](-3.038,1.126)--(-3,1.154)--(-2.958,1.163)--(-2.978,1.137)--cycle;
\filldraw[fill opacity=0.8,fill=gray!20](-2.909,.741)--(-2.934,.76)--(-2.905,.761)--(-2.909,.741)--cycle;
\filldraw[fill opacity=0.8,fill=gray!20](-2.909,.741)--(-2.905,.761)--(-2.877,.759)--(-2.909,.741)--cycle;
\filldraw[fill opacity=0.8,fill=gray!20](-2.81,.781)--(-2.77,.819)--(-2.754,.802)--(-2.799,.768)--cycle;
\filldraw[fill opacity=0.8,fill=gray!20](-2.965,.749)--(-3.018,.771)--(-3,.783)--(-2.956,.755)--cycle;
\filldraw[fill opacity=0.5,fill=gray!20](-1.509,2.919)--(-1.606,2.964)--(-1.299,3.434)--(-1.208,3.379)--cycle;
\filldraw[fill opacity=0.8,fill=gray!20,draw=none](-2.569,1.652)--(-2.492,1.744)--(-2.493,1.739)--(-2.569,1.647)--cycle;
\draw(-2.569,1.652)--(-2.492,1.744)--(-2.493,1.739)--(-2.569,1.647);
\filldraw[fill opacity=0.8,fill=gray!20,draw=none](-2.55,1.613)--(-2.563,1.646)--(-2.572,1.656)--(-2.566,1.634)--(-2.563,1.627)--cycle;
\draw(-2.566,1.634)--(-2.563,1.627)--(-2.55,1.613)--(-2.563,1.646)--(-2.572,1.656);
\filldraw[fill opacity=0.8,fill=gray!20,draw=none](-2.364,1.572)--(-2.337,1.592)--(-2.354,1.581)--(-2.376,1.564)--cycle;
\draw(-2.354,1.581)--(-2.376,1.564)--(-2.364,1.572)--(-2.337,1.592);
\filldraw[fill opacity=0.8,fill=gray!20](-2.909,.741)--(-2.956,.755)--(-2.934,.76)--(-2.909,.741)--cycle;
\filldraw[fill opacity=0.8,fill=gray!20](-2.712,.977)--(-2.719,1.032)--(-2.697,1.008)--(-2.689,.953)--cycle;
\filldraw[fill opacity=0.8,fill=gray!20,draw=none](-2.337,1.592)--(-2.335,1.594)--(-2.35,1.584)--(-2.354,1.581)--cycle;
\draw(-2.335,1.594)--(-2.35,1.584)--(-2.354,1.581);
\filldraw[fill opacity=0.8,fill=gray!20,draw=none](-3.122,1.018)--(-3.135,.989)--(-3.145,1.025)--(-3.122,1.024)--cycle;
\draw(-3.145,1.025)--(-3.122,1.024);
\filldraw[fill opacity=0.8,fill=gray!20,draw=none](-3.123,1.024)--(-3.214,1.03)--(-3.214,1.031)--(-3.13,1.042)--(-3.121,1.025)--cycle;
\draw(-3.123,1.024)--(-3.214,1.03)--(-3.214,1.031);
\draw(-3.13,1.042)--(-3.121,1.025);
\filldraw[fill opacity=0.8,fill=gray!20,draw=none](-3.12,1.013)--(-3.12,1.014)--(-3.121,1.01)--cycle;
\draw(-3.12,1.013)--(-3.12,1.014)--(-3.121,1.01);
\filldraw[fill opacity=0.8,fill=gray!20,draw=none](-3.12,1.013)--(-3.112,1.028)--(-3.117,1.023)--(-3.12,1.014)--cycle;
\draw(-3.117,1.023)--(-3.12,1.014)--(-3.12,1.013);
\filldraw[fill opacity=0.8,fill=gray!20,draw=none](-3.112,1.028)--(-3.099,1.049)--(-3.105,1.049)--(-3.117,1.023)--cycle;
\draw(-3.105,1.049)--(-3.117,1.023);
\filldraw[fill opacity=0.8,fill=gray!20,draw=none](-3.122,1.018)--(-3.122,1.024)--(-3.12,1.024)--cycle;
\draw(-3.122,1.024)--(-3.12,1.024);
\filldraw[fill opacity=0.8,fill=gray!20,draw=none](-3.123,1.024)--(-3.121,1.025)--(-3.12,1.024)--cycle;
\draw(-3.121,1.025)--(-3.12,1.024)--(-3.123,1.024);
\filldraw[fill opacity=0.8,fill=gray!20,draw=none](-3.121,1.025)--(-3.12,1.024)--(-3.121,1.025)--cycle;
\draw(-3.12,1.024)--(-3.121,1.025);
\filldraw[fill opacity=0.8,fill=gray!20,draw=none](-3.124,1.025)--(-2.96,1.063)--(-2.968,1.028)--(-3.117,.993)--cycle;
\draw(-3.124,1.025)--(-2.96,1.063)--(-2.968,1.028)--(-3.117,.993);
\filldraw[fill opacity=0.8,fill=gray!20,draw=none](-2.572,1.656)--(-2.578,1.662)--(-2.577,1.661)--cycle;
\draw(-2.572,1.656)--(-2.578,1.662)--(-2.577,1.661);
\filldraw[fill opacity=0.8,fill=gray!20](-2.909,.741)--(-2.877,.759)--(-2.858,.754)--(-2.909,.741)--cycle;
\filldraw[fill opacity=0.8,fill=gray!20,draw=none](-2.572,1.656)--(-2.572,1.687)--(-2.583,1.699)--(-2.578,1.662)--cycle;
\draw(-2.572,1.687)--(-2.583,1.699)--(-2.578,1.662)--(-2.572,1.656);
\filldraw[fill opacity=0.8,fill=gray!20,draw=none](-2.462,1.554)--(-2.477,1.56)--(-2.502,1.566)--(-2.471,1.555)--cycle;
\draw(-2.462,1.554)--(-2.477,1.56)--(-2.502,1.566)--(-2.471,1.555);
\filldraw[fill opacity=0.8,fill=gray!20](-2.858,.754)--(-2.81,.781)--(-2.799,.768)--(-2.852,.748)--cycle;
\filldraw[fill opacity=0.8,fill=gray!20,draw=none](-2.346,1.587)--(-2.335,1.594)--(-2.33,1.608)--cycle;
\draw(-2.346,1.587)--(-2.335,1.594);
\filldraw[fill opacity=0.8,fill=gray!20](-2.51,1.822)--(-2.474,1.835)--(-2.468,1.84)--(-2.498,1.83)--cycle;
\filldraw[fill opacity=0.8,fill=gray!20,draw=none](-2.747,1.335)--(-2.778,1.402)--(-2.798,1.561)--cycle;
\draw(-2.778,1.402)--(-2.798,1.561);
\filldraw[fill opacity=0.8,fill=gray!20](-2.405,1.554)--(-2.376,1.564)--(-2.404,1.559)--(-2.42,1.552)--cycle;
\filldraw[fill opacity=0.8,fill=gray!20](-2.568,1.683)--(-2.563,1.72)--(-2.578,1.736)--(-2.583,1.699)--cycle;
\filldraw[fill opacity=0.8,fill=gray!20,draw=none](-2.411,1.84)--(-2.416,1.842)--(-2.405,1.839)--cycle;
\draw(-2.411,1.84)--(-2.416,1.842)--(-2.405,1.839);
\filldraw[fill opacity=0.8,fill=gray!20](-2.434,1.843)--(-2.437,1.84)--(-2.437,1.84)--(-2.416,1.842)--cycle;
\filldraw[fill opacity=0.8,fill=gray!20](-2.454,1.842)--(-2.437,1.84)--(-2.437,1.84)--(-2.434,1.843)--cycle;
\filldraw[fill opacity=0.8,fill=gray!20](-2.719,1.032)--(-2.739,1.081)--(-2.719,1.06)--(-2.697,1.008)--cycle;
\filldraw[fill opacity=0.8,fill=gray!20,draw=none](-2.46,1.553)--(-2.457,1.552)--(-2.46,1.553)--cycle;
\draw(-2.46,1.553)--(-2.457,1.552)--(-2.46,1.553);
\filldraw[fill opacity=0.8,fill=gray!20,draw=none](-2.587,1.631)--(-2.569,1.652)--(-2.569,1.647)--(-2.576,1.638)--cycle;
\draw(-2.587,1.631)--(-2.569,1.652);
\draw(-2.569,1.647)--(-2.576,1.638);
\filldraw[fill opacity=0.8,fill=gray!20](-3.098,1.065)--(-3.064,1.109)--(-3.038,1.126)--(-3.067,1.086)--cycle;
\filldraw[fill opacity=0.8,fill=gray!20](-2.468,1.84)--(-2.437,1.84)--(-2.437,1.84)--(-2.454,1.842)--cycle;
\filldraw[fill opacity=0.8,fill=gray!20,draw=none](-2.92,1.17)--(-2.754,1.369)--(-2.747,1.335)--(-2.884,1.171)--cycle;
\draw(-2.92,1.17)--(-2.754,1.369);
\draw(-2.747,1.335)--(-2.884,1.171);
\filldraw[fill opacity=0.8,fill=gray!20,draw=none](-2.808,1.15)--(-2.757,1.211)--(-2.752,1.204)--(-2.796,1.151)--cycle;
\draw(-2.808,1.15)--(-2.757,1.211);
\draw(-2.752,1.204)--(-2.796,1.151);
\filldraw[fill opacity=0.8,fill=gray!20](-2.739,1.081)--(-2.77,1.122)--(-2.754,1.105)--(-2.719,1.06)--cycle;
\filldraw[fill opacity=0.8,fill=gray!20](-2.77,1.122)--(-2.81,1.152)--(-2.799,1.14)--(-2.754,1.105)--cycle;
\filldraw[fill opacity=0.8,fill=gray!20,draw=none](-2.808,1.15)--(-2.796,1.151)--(-2.802,1.144)--cycle;
\draw(-2.796,1.151)--(-2.802,1.144);
\filldraw[fill opacity=0.8,fill=gray!20,draw=none](-2.842,1.16)--(-2.848,1.161)--(-2.852,1.163)--cycle;
\draw(-2.842,1.16)--(-2.848,1.161)--(-2.852,1.163);
\filldraw[fill opacity=0.8,fill=gray!20,draw=none](-2.835,1.158)--(-2.827,1.155)--(-2.827,1.155)--cycle;
\draw(-2.827,1.155)--(-2.827,1.155);
\filldraw[fill opacity=0.8,fill=gray!20,draw=none](-2.827,1.155)--(-2.835,1.158)--(-2.81,1.152)--(-2.808,1.15)--cycle;
\draw(-2.835,1.158)--(-2.81,1.152)--(-2.808,1.15);
\filldraw[fill opacity=0.8,fill=gray!20](-2.81,1.152)--(-2.858,1.168)--(-2.852,1.162)--(-2.799,1.14)--cycle;
\filldraw[fill opacity=0.8,fill=gray!20,draw=none](-2.803,1.144)--(-2.801,1.145)--(-2.752,1.204)--(-2.765,1.193)--(-2.805,1.144)--cycle;
\draw(-2.801,1.145)--(-2.752,1.204);
\draw(-2.765,1.193)--(-2.805,1.144);
\filldraw[fill opacity=0.8,fill=gray!20,draw=none](-2.885,1.2)--(-2.765,1.226)--(-2.751,1.201)--cycle;
\filldraw[fill opacity=0.8,fill=gray!20](-2.909,.741)--(-2.965,.749)--(-2.956,.755)--(-2.909,.741)--cycle;
\filldraw[fill opacity=0.8,fill=gray!20,draw=none](-3.014,.767)--(-3.029,.775)--(-3.047,.789)--(-3.054,.795)--(-3.055,.8)--(-3.018,.771)--cycle;
\draw(-3.029,.775)--(-3.047,.789)--(-3.054,.795);
\draw(-3.055,.8)--(-3.018,.771)--(-3.014,.767);
\filldraw[fill opacity=0.8,fill=gray!20,draw=none](-3.13,.784)--(-3.12,.809)--(-3.06,.795)--(-3.056,.792)--cycle;
\draw(-3.13,.784)--(-3.12,.809)--(-3.06,.795);
\filldraw[fill opacity=0.8,fill=gray!20,draw=none](-3.102,.749)--(-3.139,.759)--(-3.13,.784)--(-3.056,.792)--(-3.074,.764)--cycle;
\draw(-3.102,.749)--(-3.139,.759)--(-3.13,.784);
\draw(-3.056,.792)--(-3.074,.764);
\filldraw[fill opacity=0.8,fill=gray!20,draw=none](-3.087,.81)--(-3.082,.8)--(-3.095,.803)--(-3.096,.815)--cycle;
\draw(-3.082,.8)--(-3.095,.803);
\filldraw[fill opacity=0.8,fill=gray!20,draw=none](-3.069,.814)--(-3.053,.795)--(-3.051,.793)--cycle;
\filldraw[fill opacity=0.8,fill=gray!20,draw=none](-3.06,.795)--(-3.082,.8)--(-3.087,.81)--cycle;
\draw(-3.06,.795)--(-3.082,.8);
\filldraw[fill opacity=0.8,fill=gray!20,draw=none](-3.053,.795)--(-3.054,.795)--(-3.053,.795)--cycle;
\draw(-3.054,.795)--(-3.053,.795);
\filldraw[fill opacity=0.8,fill=gray!20,draw=none](-3.053,.795)--(-3.053,.795)--(-3.047,.789)--(-3.051,.793)--cycle;
\draw(-3.053,.795)--(-3.047,.789)--(-3.051,.793);
\filldraw[fill opacity=0.8,fill=gray!20,draw=none](-3.088,.809)--(-2.919,.849)--(-2.895,.829)--(-3.097,.782)--cycle;
\draw(-3.088,.809)--(-2.919,.849)--(-2.895,.829)--(-3.097,.782);
\filldraw[fill opacity=0.8,fill=gray!20,draw=none](-2.46,1.555)--(-2.46,1.559)--(-2.477,1.56)--(-2.462,1.554)--cycle;
\draw(-2.46,1.559)--(-2.477,1.56)--(-2.462,1.554);
\filldraw[fill opacity=0.8,fill=gray!20,draw=none](-2.683,1.287)--(-2.409,1.616)--(-2.402,1.629)--(-2.671,1.306)--cycle;
\draw(-2.683,1.287)--(-2.409,1.616);
\draw(-2.402,1.629)--(-2.671,1.306);
\filldraw[fill opacity=0.8,fill=gray!20,draw=none](-2.439,1.551)--(-2.441,1.557)--(-2.446,1.558)--(-2.46,1.555)--(-2.46,1.553)--(-2.457,1.552)--cycle;
\draw(-2.46,1.553)--(-2.457,1.552)--(-2.439,1.551)--(-2.441,1.557)--(-2.446,1.558);
\filldraw[fill opacity=0.8,fill=gray!20](-2.563,1.72)--(-2.55,1.756)--(-2.563,1.77)--(-2.578,1.736)--cycle;
\filldraw[fill opacity=0.8,fill=gray!20,draw=none](-2.405,1.839)--(-2.416,1.842)--(-2.437,1.84)--(-2.437,1.84)--(-2.405,1.839)--cycle;
\draw(-2.405,1.839)--(-2.416,1.842)--(-2.437,1.84)--(-2.437,1.84)--(-2.405,1.839);
\filldraw[fill opacity=0.8,fill=gray!20,draw=none](-2.853,1.207)--(-2.747,1.335)--(-2.734,1.307)--(-2.811,1.213)--cycle;
\draw(-2.853,1.207)--(-2.747,1.335);
\draw(-2.734,1.307)--(-2.811,1.213);
\filldraw[fill opacity=0.8,fill=gray!20,draw=none](-2.734,1.307)--(-2.714,1.33)--(-2.688,1.322)--(-2.717,1.287)--cycle;
\draw(-2.734,1.307)--(-2.714,1.33);
\draw(-2.688,1.322)--(-2.717,1.287);
\filldraw[fill opacity=0.8,fill=gray!20,draw=none](-2.749,1.431)--(-2.73,1.454)--(-2.741,1.419)--(-2.755,1.402)--cycle;
\draw(-2.749,1.431)--(-2.73,1.454);
\draw(-2.741,1.419)--(-2.755,1.402);
\filldraw[fill opacity=0.8,fill=gray!20,draw=none](-2.729,1.46)--(-2.709,1.484)--(-2.73,1.454)--(-2.733,1.45)--cycle;
\draw(-2.729,1.46)--(-2.709,1.484);
\draw(-2.73,1.454)--(-2.733,1.45);
\filldraw[fill opacity=0.8,fill=gray!20,draw=none](-2.721,1.456)--(-2.563,1.646)--(-2.581,1.638)--(-2.729,1.46)--cycle;
\draw(-2.721,1.456)--(-2.563,1.646);
\draw(-2.581,1.638)--(-2.729,1.46);
\filldraw[fill opacity=0.8,fill=gray!20,draw=none](-2.737,1.449)--(-2.729,1.46)--(-2.733,1.45)--(-2.749,1.431)--cycle;
\draw(-2.737,1.449)--(-2.729,1.46);
\draw(-2.733,1.45)--(-2.749,1.431);
\filldraw[fill opacity=0.8,fill=gray!20,draw=none](-2.869,1.909)--(-2.859,2.02)--(-2.819,2.019)--(-2.809,1.937)--cycle;
\draw(-2.819,2.019)--(-2.809,1.937);
\filldraw[fill opacity=0.8,fill=gray!20,draw=none](-2.779,1.798)--(-2.829,1.8)--(-2.864,1.809)--(-2.797,1.814)--cycle;
\filldraw[fill opacity=0.8,fill=gray!20,draw=none](-2.721,1.456)--(-2.729,1.46)--(-2.737,1.449)--cycle;
\draw(-2.729,1.46)--(-2.737,1.449);
\filldraw[fill opacity=0.8,fill=gray!20,draw=none](-2.763,1.784)--(-2.829,1.8)--(-2.779,1.798)--cycle;
\filldraw[fill opacity=0.8,fill=gray!20,draw=none](-2.839,1.956)--(-2.726,1.494)--(-2.699,1.281)--(-2.718,1.273)--(-2.747,1.335)--(-2.798,1.561)--(-2.848,1.948)--cycle;
\draw(-2.726,1.494)--(-2.699,1.281);
\draw(-2.798,1.561)--(-2.848,1.948);
\filldraw[fill opacity=0.8,fill=gray!20](-2.42,1.552)--(-2.404,1.559)--(-2.441,1.557)--(-2.439,1.551)--cycle;
\filldraw[fill opacity=0.8,fill=gray!20](-2.477,1.56)--(-2.494,1.577)--(-2.529,1.586)--(-2.502,1.566)--cycle;
\filldraw[fill opacity=0.8,fill=gray!20](-2.909,.741)--(-2.858,.754)--(-2.852,.748)--(-2.909,.741)--cycle;
\filldraw[fill opacity=0.8,fill=gray!20,draw=none](-2.811,1.213)--(-2.734,1.307)--(-2.717,1.287)--(-2.777,1.215)--cycle;
\draw(-2.811,1.213)--(-2.734,1.307);
\draw(-2.717,1.287)--(-2.777,1.215);
\filldraw[fill opacity=0.8,fill=gray!20,draw=none](-2.977,1.173)--(-3.034,1.167)--(-2.897,1.198)--cycle;
\filldraw[fill opacity=0.8,fill=gray!20,draw=none](-2.935,1.186)--(-2.755,1.402)--(-2.754,1.369)--(-2.897,1.198)--cycle;
\draw(-2.935,1.186)--(-2.755,1.402);
\draw(-2.754,1.369)--(-2.897,1.198);
\filldraw[fill opacity=0.8,fill=gray!20,draw=none](-2.563,1.646)--(-2.483,1.742)--(-2.492,1.744)--(-2.581,1.638)--cycle;
\draw(-2.563,1.646)--(-2.483,1.742)--(-2.492,1.744)--(-2.581,1.638);
\filldraw[fill opacity=0.8,fill=gray!20](-2.55,1.756)--(-2.529,1.788)--(-2.54,1.799)--(-2.563,1.77)--cycle;
\filldraw[fill opacity=0.5,fill=gray!20](-1.698,2.4)--(-1.799,2.434)--(-1.606,2.964)--(-1.509,2.919)--cycle;
\filldraw[fill opacity=0.8,fill=gray!20](-2.474,1.835)--(-2.437,1.84)--(-2.437,1.84)--(-2.468,1.84)--cycle;
\filldraw[fill opacity=0.8,fill=gray!20,draw=none](-2.958,1.163)--(-2.94,1.171)--(-2.903,1.169)--(-2.902,1.165)--cycle;
\draw(-2.903,1.169)--(-2.902,1.165)--(-2.958,1.163)--(-2.94,1.171);
\filldraw[fill opacity=0.8,fill=gray!20,draw=none](-2.572,1.656)--(-2.563,1.646)--(-2.568,1.683)--(-2.572,1.687)--cycle;
\draw(-2.572,1.656)--(-2.563,1.646)--(-2.568,1.683)--(-2.572,1.687);
\filldraw[fill opacity=0.8,fill=gray!20,draw=none](-2.405,1.839)--(-2.405,1.839)--(-2.403,1.839)--cycle;
\draw(-2.405,1.839)--(-2.403,1.839)--(-2.405,1.839);
\filldraw[fill opacity=0.8,fill=gray!20](-2.376,1.564)--(-2.35,1.584)--(-2.39,1.576)--(-2.404,1.559)--cycle;
\filldraw[fill opacity=0.8,fill=gray!20,draw=none](-2.46,1.555)--(-2.446,1.558)--(-2.46,1.559)--cycle;
\draw(-2.446,1.558)--(-2.46,1.559);
\filldraw[fill opacity=0.8,fill=gray!20,draw=none](-3.254,.763)--(-3.306,.812)--(-3.214,.816)--(-3.216,.764)--cycle;
\draw(-3.306,.812)--(-3.214,.816)--(-3.216,.764)--(-3.254,.763);
\filldraw[fill opacity=0.8,fill=gray!20](-3.216,.764)--(-3.214,.816)--(-3.12,.809)--(-3.139,.759)--cycle;
\filldraw[fill opacity=0.8,fill=gray!20,draw=none](-2.405,1.839)--(-2.437,1.84)--(-2.437,1.84)--(-2.408,1.836)--cycle;
\draw(-2.405,1.839)--(-2.437,1.84)--(-2.437,1.84)--(-2.408,1.836);
\filldraw[fill opacity=0.8,fill=gray!20,draw=none](-2.402,1.838)--(-2.405,1.839)--(-2.408,1.836)--(-2.399,1.834)--cycle;
\draw(-2.408,1.836)--(-2.399,1.834)--(-2.402,1.838);
\filldraw[fill opacity=0.8,fill=gray!20](-2.529,1.788)--(-2.502,1.814)--(-2.51,1.822)--(-2.54,1.799)--cycle;
\filldraw[fill opacity=0.8,fill=gray!20,draw=none](-2.402,1.838)--(-2.399,1.834)--(-2.391,1.831)--cycle;
\draw(-2.402,1.838)--(-2.399,1.834)--(-2.391,1.831);
\filldraw[fill opacity=0.8,fill=gray!20,draw=none](-2.346,1.587)--(-2.33,1.608)--(-2.328,1.612)--(-2.331,1.61)--(-2.35,1.584)--cycle;
\draw(-2.328,1.612)--(-2.331,1.61)--(-2.35,1.584)--(-2.346,1.587);
\filldraw[fill opacity=0.8,fill=gray!20,draw=none](-2.949,1.169)--(-2.935,1.186)--(-2.897,1.198)--(-2.914,1.176)--cycle;
\draw(-2.949,1.169)--(-2.935,1.186);
\draw(-2.897,1.198)--(-2.914,1.176);
\filldraw[fill opacity=0.8,fill=gray!20,draw=none](-3,1.154)--(-2.963,1.167)--(-2.942,1.17)--(-2.958,1.163)--cycle;
\draw(-2.942,1.17)--(-2.958,1.163)--(-3,1.154)--(-2.963,1.167);
\filldraw[fill opacity=0.8,fill=gray!20,draw=none](-3.047,1.052)--(-2.949,1.169)--(-2.914,1.176)--(-3.013,1.058)--cycle;
\draw(-3.047,1.052)--(-2.949,1.169);
\draw(-2.914,1.176)--(-3.013,1.058);
\filldraw[fill opacity=0.8,fill=gray!20,draw=none](-2.94,1.171)--(-2.934,1.174)--(-2.905,1.175)--(-2.903,1.169)--cycle;
\draw(-2.94,1.171)--(-2.934,1.174)--(-2.905,1.175)--(-2.903,1.169);
\filldraw[fill opacity=0.8,fill=gray!20,draw=none](-3.013,1.058)--(-2.92,1.17)--(-2.884,1.171)--(-2.967,1.07)--cycle;
\draw(-3.013,1.058)--(-2.92,1.17);
\draw(-2.884,1.171)--(-2.967,1.07);
\filldraw[fill opacity=0.8,fill=gray!20,draw=none](-2.967,1.07)--(-2.951,1.089)--(-2.948,1.077)--cycle;
\draw(-2.967,1.07)--(-2.951,1.089);
\filldraw[fill opacity=0.8,fill=gray!20,draw=none](-3.063,1.054)--(-3.062,1.055)--(-3.039,1.062)--(-3.047,1.052)--cycle;
\draw(-3.063,1.054)--(-3.062,1.055);
\draw(-3.039,1.062)--(-3.047,1.052);
\filldraw[fill opacity=0.8,fill=gray!20,draw=none](-3.099,1.049)--(-3.112,1.028)--(-3.092,1.047)--cycle;
\filldraw[fill opacity=0.8,fill=gray!20,draw=none](-3.11,1.043)--(-3.121,1.025)--(-3.125,1.043)--cycle;
\filldraw[fill opacity=0.8,fill=gray!20,draw=none](-3.124,1.04)--(-2.943,1.082)--(-2.96,1.063)--(-3.118,1.026)--cycle;
\draw(-3.124,1.04)--(-2.943,1.082)--(-2.96,1.063)--(-3.118,1.026);
\filldraw[fill opacity=0.8,fill=gray!20,draw=none](-2.777,1.215)--(-2.717,1.287)--(-2.699,1.281)--(-2.757,1.211)--cycle;
\draw(-2.777,1.215)--(-2.717,1.287);
\draw(-2.699,1.281)--(-2.757,1.211);
\filldraw[fill opacity=0.8,fill=gray!20,draw=none](-2.961,1.175)--(-2.977,1.173)--(-2.935,1.186)--cycle;
\filldraw[fill opacity=0.8,fill=gray!20,draw=none](-2.963,1.174)--(-2.749,1.431)--(-2.755,1.402)--(-2.935,1.186)--cycle;
\draw(-2.963,1.174)--(-2.749,1.431);
\draw(-2.755,1.402)--(-2.935,1.186);
\filldraw[fill opacity=0.8,fill=gray!20](-2.494,1.577)--(-2.506,1.602)--(-2.55,1.613)--(-2.529,1.586)--cycle;
\filldraw[fill opacity=0.8,fill=gray!20](-2.502,1.814)--(-2.471,1.831)--(-2.474,1.835)--(-2.51,1.822)--cycle;
\filldraw[fill opacity=0.8,fill=gray!20](-3.402,.852)--(-3.408,.908)--(-3.323,.925)--(-3.32,.868)--cycle;
\filldraw[fill opacity=0.8,fill=gray!20](-3.384,.798)--(-3.402,.852)--(-3.32,.868)--(-3.31,.812)--cycle;
\filldraw[fill opacity=0.8,fill=gray!20,draw=none](-2.694,1.271)--(-2.688,1.26)--(-2.69,1.226)--(-2.692,1.228)--(-2.697,1.265)--cycle;
\draw(-2.69,1.226)--(-2.692,1.228)--(-2.697,1.265);
\filldraw[fill opacity=0.5,fill=gray!20,draw=none](-1.698,2.109)--(-1.615,2.07)--(-1.642,1.838)--(-1.763,1.858)--(-1.735,2.092)--cycle;
\draw(-1.615,2.07)--(-1.642,1.838)--(-1.763,1.858)--(-1.735,2.092);
\filldraw[fill opacity=0.8,fill=gray!20](-2.909,.741)--(-2.959,.743)--(-2.965,.749)--(-2.909,.741)--cycle;
\filldraw[fill opacity=0.8,fill=gray!20](-2.959,.743)--(-3.007,.759)--(-3.018,.771)--(-2.965,.749)--cycle;
\filldraw[fill opacity=0.8,fill=gray!20](-3.064,1.109)--(-3.018,1.143)--(-3,1.154)--(-3.038,1.126)--cycle;
\filldraw[fill opacity=0.8,fill=gray!20,draw=none](-2.46,1.559)--(-2.46,1.56)--(-2.467,1.573)--(-2.471,1.575)--(-2.494,1.577)--(-2.477,1.56)--cycle;
\draw(-2.471,1.575)--(-2.494,1.577)--(-2.477,1.56)--(-2.46,1.559);
\filldraw[fill opacity=0.8,fill=gray!20,draw=none](-2.936,1.177)--(-2.897,1.198)--(-2.885,1.2)--(-2.751,1.201)--(-2.748,1.196)--cycle;
\filldraw[fill opacity=0.8,fill=gray!20,draw=none](-2.719,1.451)--(-2.689,1.282)--(-2.694,1.271)--(-2.699,1.281)--(-2.721,1.456)--cycle;
\draw(-2.699,1.281)--(-2.721,1.456);
\filldraw[fill opacity=0.8,fill=gray!20,draw=none](-2.757,1.211)--(-2.699,1.281)--(-2.692,1.277)--(-2.752,1.204)--cycle;
\draw(-2.757,1.211)--(-2.699,1.281);
\draw(-2.692,1.277)--(-2.752,1.204);
\filldraw[fill opacity=0.8,fill=gray!20,draw=none](-2.712,1.438)--(-2.703,1.41)--(-2.695,1.376)--(-2.689,1.34)--(-2.686,1.309)--(-2.687,1.286)--(-2.689,1.282)--(-2.719,1.451)--cycle;
\filldraw[fill opacity=0.8,fill=gray!20,draw=none](-2.87,1.164)--(-2.904,1.17)--(-2.905,1.175)--(-2.877,1.173)--(-2.852,1.163)--cycle;
\draw(-2.904,1.17)--(-2.905,1.175)--(-2.877,1.173)--(-2.852,1.163);
\filldraw[fill opacity=0.8,fill=gray!20,draw=none](-3.028,1.096)--(-2.972,1.162)--(-2.943,1.177)--(-3.025,1.078)--cycle;
\draw(-3.028,1.096)--(-2.972,1.162);
\draw(-2.943,1.177)--(-3.025,1.078);
\filldraw[fill opacity=0.8,fill=gray!20,draw=none](-2.963,1.167)--(-2.956,1.169)--(-2.934,1.174)--(-2.942,1.17)--cycle;
\draw(-2.963,1.167)--(-2.956,1.169)--(-2.934,1.174)--(-2.942,1.17);
\filldraw[fill opacity=0.8,fill=gray!20](-2.934,1.174)--(-2.909,1.17)--(-2.909,1.17)--(-2.905,1.175)--cycle;
\filldraw[fill opacity=0.8,fill=gray!20](-2.905,1.175)--(-2.909,1.17)--(-2.909,1.17)--(-2.877,1.173)--cycle;
\filldraw[fill opacity=0.8,fill=gray!20](-2.956,1.169)--(-2.909,1.17)--(-2.909,1.17)--(-2.934,1.174)--cycle;
\filldraw[fill opacity=0.8,fill=gray!20](-2.877,1.173)--(-2.909,1.17)--(-2.909,1.17)--(-2.858,1.168)--cycle;
\filldraw[fill opacity=0.8,fill=gray!20](-2.965,1.163)--(-2.909,1.17)--(-2.909,1.17)--(-2.956,1.169)--cycle;
\filldraw[fill opacity=0.8,fill=gray!20,draw=none](-2.803,1.144)--(-2.802,1.144)--(-2.801,1.145)--cycle;
\draw(-2.802,1.144)--(-2.801,1.145);
\filldraw[fill opacity=0.8,fill=gray!20](-2.858,1.168)--(-2.909,1.17)--(-2.909,1.17)--(-2.852,1.162)--cycle;
\filldraw[fill opacity=0.8,fill=gray!20,draw=none](-2.803,1.144)--(-2.805,1.144)--(-2.807,1.143)--cycle;
\draw(-2.805,1.144)--(-2.807,1.143);
\filldraw[fill opacity=0.8,fill=gray!20,draw=none](-2.799,1.151)--(-2.765,1.193)--(-2.793,1.181)--(-2.812,1.157)--cycle;
\draw(-2.799,1.151)--(-2.765,1.193);
\draw(-2.793,1.181)--(-2.812,1.157);
\filldraw[fill opacity=0.8,fill=gray!20](-2.959,1.157)--(-2.909,1.17)--(-2.909,1.17)--(-2.965,1.163)--cycle;
\filldraw[fill opacity=0.8,fill=gray!20](-2.94,1.152)--(-2.909,1.17)--(-2.909,1.17)--(-2.959,1.157)--cycle;
\filldraw[fill opacity=0.8,fill=gray!20](-2.912,1.15)--(-2.909,1.17)--(-2.909,1.17)--(-2.94,1.152)--cycle;
\filldraw[fill opacity=0.8,fill=gray!20](-2.883,1.151)--(-2.909,1.17)--(-2.909,1.17)--(-2.912,1.15)--cycle;
\filldraw[fill opacity=0.8,fill=gray!20](-2.861,1.156)--(-2.909,1.17)--(-2.909,1.17)--(-2.883,1.151)--cycle;
\filldraw[fill opacity=0.8,fill=gray!20](-2.852,1.162)--(-2.909,1.17)--(-2.909,1.17)--(-2.861,1.156)--cycle;
\filldraw[fill opacity=0.8,fill=gray!20,draw=none](-2.893,1.053)--(-2.869,1.068)--(-2.807,1.143)--(-2.826,1.141)--(-2.891,1.063)--cycle;
\draw(-2.869,1.068)--(-2.807,1.143);
\draw(-2.826,1.141)--(-2.891,1.063);
\filldraw[fill opacity=0.8,fill=gray!20,draw=none](-2.802,1.137)--(-2.799,1.14)--(-2.852,1.162)--(-2.861,1.156)--(-2.84,1.143)--cycle;
\draw(-2.802,1.137)--(-2.799,1.14)--(-2.852,1.162)--(-2.861,1.156)--(-2.84,1.143);
\filldraw[fill opacity=0.8,fill=gray!20,draw=none](-2.831,1.169)--(-2.875,1.16)--(-2.891,1.157)--(-2.884,1.17)--(-2.858,1.185)--(-2.748,1.196)--(-2.747,1.195)--cycle;
\filldraw[fill opacity=0.8,fill=gray!20,draw=none](-2.765,1.193)--(-2.402,1.629)--(-2.4,1.652)--(-2.793,1.181)--cycle;
\draw(-2.765,1.193)--(-2.402,1.629);
\draw(-2.4,1.652)--(-2.793,1.181);
\filldraw[fill opacity=0.8,fill=gray!20](-3.32,.979)--(-3.31,1.026)--(-3.214,1.03)--(-3.212,.984)--cycle;
\filldraw[fill opacity=0.8,fill=gray!20](-2.909,.741)--(-2.94,.738)--(-2.959,.743)--(-2.909,.741)--cycle;
\filldraw[fill opacity=0.8,fill=gray!20](-2.909,.741)--(-2.912,.736)--(-2.94,.738)--(-2.909,.741)--cycle;
\filldraw[fill opacity=0.8,fill=gray!20](-2.909,.741)--(-2.883,.737)--(-2.912,.736)--(-2.909,.741)--cycle;
\filldraw[fill opacity=0.8,fill=gray!20](-2.909,.741)--(-2.861,.742)--(-2.883,.737)--(-2.909,.741)--cycle;
\filldraw[fill opacity=0.8,fill=gray!20](-2.909,.741)--(-2.852,.748)--(-2.861,.742)--(-2.909,.741)--cycle;
\filldraw[fill opacity=0.8,fill=gray!20](-2.471,1.831)--(-2.437,1.84)--(-2.437,1.84)--(-2.474,1.835)--cycle;
\filldraw[fill opacity=0.5,fill=gray!20](-.857,3.095)--(-.994,3.23)--(-.632,3.573)--(-.535,3.4)--cycle;
\filldraw[fill opacity=0.8,fill=gray!20](-3.408,.908)--(-3.402,.963)--(-3.32,.979)--(-3.323,.925)--cycle;
\filldraw[fill opacity=0.8,fill=gray!20,draw=none](-3.254,.763)--(-3.294,.761)--(-3.31,.812)--(-3.306,.812)--cycle;
\draw(-3.254,.763)--(-3.294,.761)--(-3.31,.812)--(-3.306,.812);
\filldraw[fill opacity=0.8,fill=gray!20,draw=none](-3.128,.924)--(-3.138,.924)--(-3.137,.942)--cycle;
\draw(-3.128,.924)--(-3.138,.924);
\filldraw[fill opacity=0.8,fill=gray!20](-3.355,.749)--(-3.384,.798)--(-3.31,.812)--(-3.294,.761)--cycle;
\filldraw[fill opacity=0.8,fill=gray!20](-2.457,1.828)--(-2.437,1.84)--(-2.437,1.84)--(-2.471,1.831)--cycle;
\filldraw[fill opacity=0.8,fill=gray!20](-2.439,1.827)--(-2.437,1.84)--(-2.437,1.84)--(-2.457,1.828)--cycle;
\filldraw[fill opacity=0.8,fill=gray!20,draw=none](-2.408,1.836)--(-2.437,1.84)--(-2.437,1.84)--(-2.413,1.833)--cycle;
\draw(-2.408,1.836)--(-2.437,1.84)--(-2.437,1.84)--(-2.413,1.833);
\filldraw[fill opacity=0.8,fill=gray!20,draw=none](-2.413,1.833)--(-2.437,1.84)--(-2.437,1.84)--(-2.421,1.828)--cycle;
\draw(-2.413,1.833)--(-2.437,1.84)--(-2.437,1.84)--(-2.421,1.828);
\filldraw[fill opacity=0.8,fill=gray!20,draw=none](-2.423,1.827)--(-2.421,1.828)--(-2.437,1.84)--(-2.437,1.84)--(-2.439,1.827)--cycle;
\draw(-2.421,1.828)--(-2.437,1.84)--(-2.437,1.84)--(-2.439,1.827)--(-2.423,1.827);
\filldraw[fill opacity=0.8,fill=gray!20,draw=none](-2.33,1.608)--(-2.325,1.614)--(-2.328,1.612)--cycle;
\draw(-2.325,1.614)--(-2.328,1.612);
\filldraw[fill opacity=0.8,fill=gray!20,draw=none](-2.399,1.834)--(-2.408,1.836)--(-2.413,1.833)--(-2.405,1.83)--cycle;
\draw(-2.413,1.833)--(-2.405,1.83)--(-2.399,1.834)--(-2.408,1.836);
\filldraw[fill opacity=0.8,fill=gray!20,draw=none](-2.383,1.819)--(-2.391,1.831)--(-2.399,1.834)--(-2.404,1.831)--cycle;
\draw(-2.391,1.831)--(-2.399,1.834)--(-2.404,1.831);
\filldraw[fill opacity=0.8,fill=gray!20](-2.852,.748)--(-2.799,.768)--(-2.817,.757)--(-2.861,.742)--cycle;
\filldraw[fill opacity=0.8,fill=gray!20](-2.35,1.584)--(-2.331,1.61)--(-2.38,1.601)--(-2.39,1.576)--cycle;
\filldraw[fill opacity=0.8,fill=gray!20,draw=none](-3.229,.749)--(-3.254,.763)--(-3.216,.764)--(-3.217,.75)--cycle;
\draw(-3.254,.763)--(-3.216,.764)--(-3.217,.75);
\filldraw[fill opacity=0.8,fill=gray!20,draw=none](-3.229,.749)--(-3.288,.747)--(-3.294,.761)--(-3.254,.763)--cycle;
\draw(-3.288,.747)--(-3.294,.761)--(-3.254,.763);
\filldraw[fill opacity=0.8,fill=gray!20,draw=none](-3.3,.76)--(-3.294,.761)--(-3.292,.755)--cycle;
\draw(-3.3,.76)--(-3.294,.761)--(-3.292,.755);
\filldraw[fill opacity=0.8,fill=gray!20](-3.295,.687)--(-3.301,.724)--(-3.229,.728)--(-3.23,.69)--cycle;
\filldraw[fill opacity=0.8,fill=gray!20,draw=none](-3.239,.727)--(-3.272,.726)--(-3.303,.748)--(-3.303,.753)--cycle;
\draw(-3.239,.727)--(-3.272,.726);
\draw(-3.303,.748)--(-3.303,.753);
\filldraw[fill opacity=0.8,fill=gray!20,draw=none](-3.272,.726)--(-3.301,.724)--(-3.303,.748)--cycle;
\draw(-3.272,.726)--(-3.301,.724)--(-3.303,.748);
\filldraw[fill opacity=0.8,fill=gray!20](-3.344,.678)--(-3.356,.714)--(-3.301,.724)--(-3.295,.687)--cycle;
\filldraw[fill opacity=0.8,fill=gray!20,draw=none](-3.349,.744)--(-3.355,.749)--(-3.3,.76)--(-3.292,.755)--cycle;
\draw(-3.349,.744)--(-3.355,.749)--(-3.3,.76);
\filldraw[fill opacity=0.8,fill=gray!20,draw=none](-3.324,.72)--(-3.358,.747)--(-3.301,.726)--(-3.301,.724)--cycle;
\draw(-3.301,.726)--(-3.301,.724)--(-3.324,.72);
\filldraw[fill opacity=0.8,fill=gray!20,draw=none](-3.36,.748)--(-3.405,.784)--(-3.384,.798)--(-3.356,.751)--cycle;
\draw(-3.405,.784)--(-3.384,.798)--(-3.356,.751);
\filldraw[fill opacity=0.8,fill=gray!20,draw=none](-3.358,.747)--(-3.36,.748)--(-3.356,.751)--(-3.355,.749)--cycle;
\draw(-3.356,.751)--(-3.355,.749)--(-3.358,.747);
\filldraw[fill opacity=0.8,fill=gray!20,draw=none](-3.358,.747)--(-3.324,.72)--(-3.356,.714)--(-3.36,.747)--cycle;
\draw(-3.324,.72)--(-3.356,.714)--(-3.36,.747);
\filldraw[fill opacity=0.8,fill=gray!20,draw=none](-3.363,.709)--(-3.381,.734)--(-3.36,.748)--(-3.356,.714)--cycle;
\draw(-3.36,.748)--(-3.356,.714)--(-3.363,.709);
\filldraw[fill opacity=0.8,fill=gray!20](-3.335,.698)--(-3.38,.732)--(-3.355,.749)--(-3.317,.71)--cycle;
\filldraw[fill opacity=0.8,fill=gray!20,draw=none](-3.358,.747)--(-3.36,.748)--(-3.36,.751)--(-3.303,.762)--(-3.301,.726)--cycle;
\draw(-3.36,.748)--(-3.36,.751)--(-3.303,.762)--(-3.301,.726);
\filldraw[fill opacity=0.8,fill=gray!20,draw=none](-3.317,.71)--(-3.349,.744)--(-3.292,.755)--(-3.274,.718)--cycle;
\draw(-3.292,.755)--(-3.274,.718)--(-3.317,.71)--(-3.349,.744);
\filldraw[fill opacity=0.8,fill=gray!20,draw=none](-3.239,.727)--(-3.303,.753)--(-3.303,.762)--(-3.229,.766)--(-3.229,.728)--cycle;
\draw(-3.303,.753)--(-3.303,.762)--(-3.229,.766)--(-3.229,.728)--(-3.239,.727);
\filldraw[fill opacity=0.8,fill=gray!20,draw=none](-3.238,.866)--(-3.241,.832)--(-3.289,.851)--(-3.287,.872)--cycle;
\draw(-3.238,.866)--(-3.241,.832);
\draw(-3.289,.851)--(-3.287,.872);
\filldraw[fill opacity=0.8,fill=gray!20,draw=none](-3.238,.866)--(-3.287,.872)--(-3.284,.908)--cycle;
\draw(-3.287,.872)--(-3.284,.908);
\filldraw[fill opacity=0.8,fill=gray!20,draw=none](-3.284,.908)--(-3.137,.942)--(-3.128,.924)--(-3.125,.892)--(-3.274,.857)--cycle;
\draw(-3.125,.892)--(-3.274,.857)--(-3.284,.908)--(-3.137,.942);
\filldraw[fill opacity=0.8,fill=gray!20,draw=none](-3.099,.834)--(-3.095,.803)--(-3.12,.809)--(-3.11,.856)--cycle;
\draw(-3.095,.803)--(-3.12,.809)--(-3.11,.856);
\filldraw[fill opacity=0.8,fill=gray!20,draw=none](-3.13,.808)--(-3.145,.812)--(-3.146,.837)--(-3.097,.848)--(-3.088,.812)--cycle;
\draw(-3.146,.837)--(-3.097,.848);
\filldraw[fill opacity=0.8,fill=gray!20,draw=none](-3.099,.834)--(-3.11,.856)--(-3.108,.865)--(-3.103,.864)--cycle;
\draw(-3.11,.856)--(-3.108,.865)--(-3.103,.864);
\filldraw[fill opacity=0.8,fill=gray!20,draw=none](-3.188,.851)--(-3.187,.878)--(-3.125,.892)--(-3.107,.867)--(-3.105,.846)--(-3.173,.83)--cycle;
\draw(-3.187,.878)--(-3.125,.892);
\draw(-3.105,.846)--(-3.173,.83);
\filldraw[fill opacity=0.8,fill=gray!20,draw=none](-3.038,1.885)--(-3.027,1.795)--(-3.129,1.775)--(-3.143,1.879)--cycle;
\draw(-3.038,1.885)--(-3.027,1.795);
\draw(-3.129,1.775)--(-3.143,1.879);
\filldraw[fill opacity=0.8,fill=gray!20,draw=none](-2.544,1.641)--(-2.467,1.733)--(-2.483,1.742)--(-2.563,1.646)--cycle;
\draw(-2.544,1.641)--(-2.467,1.733)--(-2.483,1.742)--(-2.563,1.646);
\filldraw[fill opacity=0.8,fill=gray!20,draw=none](-2.688,1.26)--(-2.684,1.251)--(-2.678,1.215)--(-2.69,1.226)--cycle;
\draw(-2.678,1.215)--(-2.69,1.226);
\filldraw[fill opacity=0.8,fill=gray!20,draw=none](-2.325,1.614)--(-2.318,1.644)--(-2.319,1.643)--(-2.331,1.61)--cycle;
\draw(-2.318,1.644)--(-2.319,1.643)--(-2.331,1.61)--(-2.325,1.614);
\filldraw[fill opacity=0.8,fill=gray!20](-2.404,1.559)--(-2.39,1.576)--(-2.443,1.573)--(-2.441,1.557)--cycle;
\filldraw[fill opacity=0.8,fill=gray!20,draw=none](-3.183,.724)--(-3.229,.728)--(-3.229,.743)--(-3.217,.743)--cycle;
\draw(-3.183,.724)--(-3.229,.728)--(-3.229,.743);
\filldraw[fill opacity=0.8,fill=gray!20,draw=none](-3.23,.69)--(-3.229,.728)--(-3.183,.724)--(-3.163,.709)--(-3.168,.686)--cycle;
\draw(-3.163,.709)--(-3.168,.686)--(-3.23,.69)--(-3.229,.728)--(-3.183,.724);
\filldraw[fill opacity=0.8,fill=gray!20,draw=none](-3.183,.724)--(-3.217,.743)--(-3.159,.738)--(-3.16,.723)--cycle;
\draw(-3.159,.738)--(-3.16,.723)--(-3.183,.724);
\filldraw[fill opacity=0.8,fill=gray!20,draw=none](-3.183,.724)--(-3.16,.723)--(-3.163,.709)--cycle;
\draw(-3.183,.724)--(-3.16,.723)--(-3.163,.709);
\filldraw[fill opacity=0.8,fill=gray!20,draw=none](-3.154,.734)--(-3.136,.717)--(-3.16,.723)--(-3.159,.735)--cycle;
\draw(-3.136,.717)--(-3.16,.723)--(-3.159,.735);
\filldraw[fill opacity=0.8,fill=gray!20,draw=none](-3.154,.734)--(-3.159,.735)--(-3.159,.738)--cycle;
\draw(-3.159,.735)--(-3.159,.738);
\filldraw[fill opacity=0.8,fill=gray!20](-3.218,.72)--(-3.216,.764)--(-3.139,.759)--(-3.165,.716)--cycle;
\filldraw[fill opacity=0.8,fill=gray!20,draw=none](-3.104,.864)--(-3.107,.865)--(-3.108,.869)--(-3.108,.871)--cycle;
\draw(-3.104,.864)--(-3.107,.865);
\draw(-3.108,.869)--(-3.108,.871);
\filldraw[fill opacity=0.8,fill=gray!20,draw=none](-2.46,1.56)--(-2.46,1.559)--(-2.459,1.559)--cycle;
\draw(-2.46,1.559)--(-2.459,1.559);
\filldraw[fill opacity=0.8,fill=gray!20,draw=none](-3.116,.906)--(-3.117,.923)--(-3.104,.922)--(-3.107,.88)--cycle;
\draw(-3.117,.923)--(-3.104,.922)--(-3.107,.88);
\filldraw[fill opacity=0.8,fill=gray!20,draw=none](-2.46,1.56)--(-2.459,1.559)--(-2.441,1.557)--(-2.443,1.573)--(-2.453,1.574)--cycle;
\draw(-2.459,1.559)--(-2.441,1.557)--(-2.443,1.573)--(-2.453,1.574);
\filldraw[fill opacity=0.8,fill=gray!20,draw=none](-3.064,.91)--(-2.941,1.057)--(-2.929,1.033)--(-3.046,.892)--cycle;
\draw(-3.064,.91)--(-2.941,1.057);
\draw(-2.929,1.033)--(-3.046,.892);
\filldraw[fill opacity=0.8,fill=gray!20,draw=none](-3.035,.905)--(-2.929,1.033)--(-2.915,1.02)--(-3.031,.881)--cycle;
\draw(-3.035,.905)--(-2.929,1.033);
\draw(-2.915,1.02)--(-3.031,.881);
\filldraw[fill opacity=0.8,fill=gray!20,draw=none](-3.065,.912)--(-3.072,.945)--(-3.053,.922)--(-3.064,.91)--cycle;
\draw(-3.053,.922)--(-3.064,.91);
\filldraw[fill opacity=0.8,fill=gray!20,draw=none](-3.065,.912)--(-3.064,.91)--(-3.065,.909)--cycle;
\draw(-3.064,.91)--(-3.065,.909);
\filldraw[fill opacity=0.8,fill=gray!20,draw=none](-3.065,.909)--(-3.064,.91)--(-3.046,.892)--(-3.05,.886)--cycle;
\draw(-3.065,.909)--(-3.064,.91);
\draw(-3.046,.892)--(-3.05,.886);
\filldraw[fill opacity=0.8,fill=gray!20,draw=none](-3.05,.886)--(-3.035,.905)--(-3.031,.881)--cycle;
\draw(-3.05,.886)--(-3.035,.905);
\filldraw[fill opacity=0.8,fill=gray!20,draw=none](-3.103,.864)--(-3.107,.88)--(-3.104,.922)--(-3.029,.903)--(-3.035,.847)--cycle;
\draw(-3.107,.88)--(-3.104,.922)--(-3.029,.903)--(-3.035,.847)--(-3.103,.864);
\filldraw[fill opacity=0.8,fill=gray!20,draw=none](-3.117,.923)--(-3.123,.961)--(-3.122,.977)--(-3.108,.976)--(-3.104,.922)--cycle;
\draw(-3.122,.977)--(-3.108,.976)--(-3.104,.922)--(-3.117,.923);
\filldraw[fill opacity=0.8,fill=gray!20,draw=none](-2.508,1.608)--(-2.503,1.606)--(-2.424,1.701)--(-2.446,1.719)--(-2.523,1.627)--cycle;
\draw(-2.503,1.606)--(-2.424,1.701)--(-2.446,1.719)--(-2.523,1.627);
\filldraw[fill opacity=0.8,fill=gray!20,draw=none](-2.506,1.602)--(-2.51,1.617)--(-2.544,1.642)--(-2.563,1.646)--(-2.55,1.613)--cycle;
\draw(-2.544,1.642)--(-2.563,1.646)--(-2.55,1.613)--(-2.506,1.602)--(-2.51,1.617);
\filldraw[fill opacity=0.8,fill=gray!20,draw=none](-3.007,1.15)--(-2.987,1.155)--(-2.979,1.158)--(-2.969,1.165)--(-3,1.154)--cycle;
\draw(-2.987,1.155)--(-2.979,1.158);
\draw(-2.969,1.165)--(-3,1.154)--(-3.007,1.15);
\filldraw[fill opacity=0.8,fill=gray!20,draw=none](-2.972,1.162)--(-2.963,1.174)--(-2.935,1.186)--(-2.943,1.177)--cycle;
\draw(-2.972,1.162)--(-2.963,1.174);
\draw(-2.935,1.186)--(-2.943,1.177);
\filldraw[fill opacity=0.8,fill=gray!20,draw=none](-2.936,1.177)--(-2.961,1.175)--(-2.935,1.186)--(-2.897,1.198)--cycle;
\filldraw[fill opacity=0.8,fill=gray!20,draw=none](-2.471,1.575)--(-2.467,1.573)--(-2.4,1.652)--(-2.404,1.683)--(-2.481,1.59)--cycle;
\draw(-2.467,1.573)--(-2.4,1.652);
\draw(-2.404,1.683)--(-2.481,1.59);
\filldraw[fill opacity=0.8,fill=gray!20,draw=none](-2.46,1.56)--(-2.453,1.574)--(-2.468,1.575)--cycle;
\draw(-2.453,1.574)--(-2.468,1.575);
\filldraw[fill opacity=0.8,fill=gray!20,draw=none](-3.095,.848)--(-3.103,.861)--(-3.103,.864)--(-3.035,.847)--(-3.054,.797)--cycle;
\draw(-3.103,.864)--(-3.035,.847)--(-3.054,.797);
\filldraw[fill opacity=0.8,fill=gray!20,draw=none](-3.072,.945)--(-2.967,1.07)--(-2.948,1.077)--(-2.941,1.057)--(-3.053,.922)--cycle;
\draw(-3.072,.945)--(-2.967,1.07);
\draw(-2.941,1.057)--(-3.053,.922);
\filldraw[fill opacity=0.8,fill=gray!20,draw=none](-3.07,.989)--(-3.013,1.058)--(-2.967,1.07)--(-3.072,.945)--cycle;
\draw(-3.07,.989)--(-3.013,1.058);
\draw(-2.967,1.07)--(-3.072,.945);
\filldraw[fill opacity=0.8,fill=gray!20,draw=none](-3.085,.917)--(-3.104,.922)--(-3.108,.976)--(-3.086,.97)--cycle;
\draw(-3.085,.917)--(-3.104,.922)--(-3.108,.976)--(-3.086,.97);
\filldraw[fill opacity=0.8,fill=gray!20,draw=none](-3.078,.937)--(-3.072,.945)--(-3.065,.912)--cycle;
\draw(-3.078,.937)--(-3.072,.945);
\filldraw[fill opacity=0.8,fill=gray!20,draw=none](-3.072,.914)--(-3.085,.917)--(-3.086,.97)--cycle;
\draw(-3.072,.914)--(-3.085,.917);
\filldraw[fill opacity=0.8,fill=gray!20,draw=none](-3.118,.932)--(-3.086,.971)--(-3.072,.945)--(-3.113,.895)--cycle;
\draw(-3.118,.932)--(-3.086,.971);
\draw(-3.072,.945)--(-3.113,.895);
\filldraw[fill opacity=0.8,fill=gray!20,draw=none](-3.086,.971)--(-3.07,.989)--(-3.072,.945)--cycle;
\draw(-3.086,.971)--(-3.07,.989);
\filldraw[fill opacity=0.8,fill=gray!20,draw=none](-3.072,.914)--(-3.086,.97)--(-3.035,.958)--(-3.029,.903)--cycle;
\draw(-3.086,.97)--(-3.035,.958)--(-3.029,.903)--(-3.072,.914);
\filldraw[fill opacity=0.8,fill=gray!20,draw=none](-3.113,.895)--(-3.078,.937)--(-3.065,.912)--(-3.065,.909)--(-3.1,.867)--cycle;
\draw(-3.113,.895)--(-3.078,.937);
\draw(-3.065,.909)--(-3.1,.867);
\filldraw[fill opacity=0.8,fill=gray!20,draw=none](-3.1,.867)--(-3.065,.909)--(-3.05,.886)--(-3.087,.842)--cycle;
\draw(-3.1,.867)--(-3.065,.909);
\draw(-3.05,.886)--(-3.087,.842);
\filldraw[fill opacity=0.8,fill=gray!20,draw=none](-3.097,.849)--(-3.069,.814)--(-3.051,.793)--(-3.079,.83)--(-3.098,.851)--cycle;
\draw(-3.051,.793)--(-3.079,.83)--(-3.098,.851);
\filldraw[fill opacity=0.8,fill=gray!20,draw=none](-3.081,.832)--(-3.087,.842)--(-3.05,.886)--(-3.031,.881)--(-3.076,.827)--cycle;
\draw(-3.087,.842)--(-3.05,.886);
\draw(-3.031,.881)--(-3.076,.827);
\filldraw[fill opacity=0.8,fill=gray!20](-3.079,.83)--(-3.098,.879)--(-3.12,.903)--(-3.098,.851)--cycle;
\filldraw[fill opacity=0.8,fill=gray!20](-2.799,.768)--(-2.754,.802)--(-2.779,.785)--(-2.817,.757)--cycle;
\filldraw[fill opacity=0.8,fill=gray!20,draw=none](-2.986,.808)--(-2.895,.829)--(-2.874,.829)--(-2.901,.823)--cycle;
\draw(-2.986,.808)--(-2.895,.829)--(-2.874,.829)--(-2.901,.823);
\filldraw[fill opacity=0.8,fill=gray!20,draw=none](-2.976,1.163)--(-2.981,1.172)--(-2.977,1.173)--(-2.962,1.175)--cycle;
\filldraw[fill opacity=0.8,fill=gray!20,draw=none](-2.976,1.163)--(-2.737,1.449)--(-2.749,1.431)--(-2.963,1.174)--cycle;
\draw(-2.976,1.163)--(-2.737,1.449);
\draw(-2.749,1.431)--(-2.963,1.174);
\filldraw[fill opacity=0.8,fill=gray!20,draw=none](-3.095,.848)--(-3.102,.856)--(-3.103,.861)--cycle;
\filldraw[fill opacity=0.8,fill=gray!20,draw=none](-3.265,1.715)--(-3.294,1.697)--(-3.314,1.715)--(-3.254,1.752)--(-3.249,1.753)--cycle;
\draw(-3.314,1.715)--(-3.254,1.752)--(-3.249,1.753);
\filldraw[fill opacity=0.8,fill=gray!20,draw=none](-3.278,1.684)--(-3.294,1.697)--(-3.265,1.715)--cycle;
\filldraw[fill opacity=0.8,fill=gray!20,draw=none](-3.294,1.928)--(-3.289,1.897)--(-3.194,1.158)--(-3.207,1.127)--(-3.315,1.969)--cycle;
\draw(-3.289,1.897)--(-3.194,1.158)--(-3.207,1.127)--(-3.315,1.969);
\filldraw[fill opacity=0.8,fill=gray!20,draw=none](-3.06,.795)--(-3.087,.81)--(-3.099,.834)--(-3.102,.856)--(-3.054,.797)--(-3.055,.793)--cycle;
\draw(-3.054,.797)--(-3.055,.793)--(-3.06,.795);
\filldraw[fill opacity=0.8,fill=gray!20,draw=none](-2.405,1.83)--(-2.413,1.833)--(-2.421,1.828)--(-2.42,1.828)--cycle;
\draw(-2.421,1.828)--(-2.42,1.828)--(-2.405,1.83)--(-2.413,1.833);
\filldraw[fill opacity=0.8,fill=gray!20](-2.477,1.807)--(-2.457,1.828)--(-2.471,1.831)--(-2.502,1.814)--cycle;
\filldraw[fill opacity=0.8,fill=gray!20,draw=none](-3.217,.743)--(-3.229,.743)--(-3.229,.749)--cycle;
\draw(-3.229,.743)--(-3.229,.749);
\filldraw[fill opacity=0.8,fill=gray!20,draw=none](-3.274,.718)--(-3.288,.747)--(-3.217,.75)--(-3.218,.72)--cycle;
\draw(-3.217,.75)--(-3.218,.72)--(-3.274,.718)--(-3.288,.747);
\filldraw[fill opacity=0.8,fill=gray!20](-3.402,.963)--(-3.384,1.012)--(-3.31,1.026)--(-3.32,.979)--cycle;
\filldraw[fill opacity=0.8,fill=gray!20,draw=none](-3.014,.767)--(-3.007,.759)--(-3.029,.775)--cycle;
\draw(-3.014,.767)--(-3.007,.759)--(-3.029,.775);
\filldraw[fill opacity=0.8,fill=gray!20,draw=none](-2.383,1.819)--(-2.404,1.831)--(-2.405,1.83)--(-2.382,1.816)--cycle;
\draw(-2.404,1.831)--(-2.405,1.83)--(-2.382,1.816);
\filldraw[fill opacity=0.8,fill=gray!20,draw=none](-2.829,1.8)--(-2.921,1.804)--(-2.864,1.809)--cycle;
\filldraw[fill opacity=0.8,fill=gray!20,draw=none](-2.928,1.912)--(-2.848,1.948)--(-2.829,1.8)--(-2.921,1.804)--(-2.931,1.881)--cycle;
\draw(-2.848,1.948)--(-2.829,1.8);
\draw(-2.921,1.804)--(-2.931,1.881);
\filldraw[fill opacity=0.8,fill=gray!20,draw=none](-3.217,.743)--(-3.229,.749)--(-3.229,.766)--(-3.157,.761)--(-3.159,.738)--cycle;
\draw(-3.229,.749)--(-3.229,.766)--(-3.157,.761)--(-3.159,.738);
\filldraw[fill opacity=0.8,fill=gray!20,draw=none](-3.122,.977)--(-3.122,.999)--(-3.119,1.012)--(-3.118,1.014)--(-3.108,.976)--cycle;
\draw(-3.118,1.014)--(-3.108,.976)--(-3.122,.977);
\filldraw[fill opacity=0.8,fill=gray!20,draw=none](-3.079,1.013)--(-3.047,1.052)--(-3.013,1.058)--(-3.072,.987)--cycle;
\draw(-3.079,1.013)--(-3.047,1.052);
\draw(-3.013,1.058)--(-3.072,.987);
\filldraw[fill opacity=0.8,fill=gray!20,draw=none](-3.093,.997)--(-3.079,1.013)--(-3.072,.987)--(-3.086,.971)--cycle;
\draw(-3.093,.997)--(-3.079,1.013);
\draw(-3.072,.987)--(-3.086,.971);
\filldraw[fill opacity=0.8,fill=gray!20,draw=none](-3.094,.972)--(-3.108,.976)--(-3.111,.986)--(-3.093,1.017)--(-3.092,1.017)--cycle;
\draw(-3.094,.972)--(-3.108,.976)--(-3.111,.986);
\draw(-3.093,1.017)--(-3.092,1.017);
\filldraw[fill opacity=0.8,fill=gray!20,draw=none](-3.111,.986)--(-3.118,1.014)--(-3.11,1.021)--(-3.093,1.017)--cycle;
\draw(-3.111,.986)--(-3.118,1.014);
\draw(-3.11,1.021)--(-3.093,1.017);
\filldraw[fill opacity=0.8,fill=gray!20,draw=none](-3.092,1.019)--(-3.063,1.054)--(-3.047,1.052)--(-3.088,1.002)--cycle;
\draw(-3.092,1.019)--(-3.063,1.054);
\draw(-3.047,1.052)--(-3.088,1.002);
\filldraw[fill opacity=0.8,fill=gray!20,draw=none](-3.135,.989)--(-3.122,1.018)--(-3.122,.999)--(-3.128,.977)--(-3.133,.978)--cycle;
\draw(-3.128,.977)--(-3.133,.978);
\filldraw[fill opacity=0.8,fill=gray!20,draw=none](-3.12,.95)--(-3.117,.968)--(-3.104,.984)--(-3.107,.945)--(-3.118,.932)--cycle;
\draw(-3.117,.968)--(-3.104,.984);
\draw(-3.107,.945)--(-3.118,.932);
\filldraw[fill opacity=0.8,fill=gray!20](-3.098,.879)--(-3.105,.934)--(-3.128,.958)--(-3.12,.903)--cycle;
\filldraw[fill opacity=0.8,fill=gray!20](-2.754,.802)--(-2.719,.846)--(-2.75,.825)--(-2.779,.785)--cycle;
\filldraw[fill opacity=0.8,fill=gray!20,draw=none](-2.383,1.81)--(-2.382,1.816)--(-2.405,1.83)--(-2.42,1.828)--(-2.404,1.806)--cycle;
\draw(-2.382,1.816)--(-2.405,1.83)--(-2.42,1.828)--(-2.404,1.806)--(-2.383,1.81);
\filldraw[fill opacity=0.8,fill=gray!20,draw=none](-2.348,1.607)--(-2.331,1.61)--(-2.321,1.638)--cycle;
\draw(-2.348,1.607)--(-2.331,1.61)--(-2.321,1.638);
\filldraw[fill opacity=0.5,fill=gray!20](-.994,3.23)--(-1.105,3.311)--(-.726,3.67)--(-.632,3.573)--cycle;
\filldraw[fill opacity=0.8,fill=gray!20,draw=none](-3.229,.766)--(-3.229,.802)--(-3.183,.798)--(-3.158,.77)--(-3.157,.761)--cycle;
\draw(-3.158,.77)--(-3.157,.761)--(-3.229,.766)--(-3.229,.802)--(-3.183,.798);
\filldraw[fill opacity=0.8,fill=gray!20,draw=none](-3.28,.763)--(-3.259,.8)--(-3.229,.802)--(-3.229,.766)--cycle;
\draw(-3.259,.8)--(-3.229,.802)--(-3.229,.766)--(-3.28,.763);
\filldraw[fill opacity=0.8,fill=gray!20,draw=none](-2.704,1.449)--(-2.544,1.641)--(-2.563,1.646)--(-2.721,1.456)--cycle;
\draw(-2.704,1.449)--(-2.544,1.641);
\draw(-2.563,1.646)--(-2.721,1.456);
\filldraw[fill opacity=0.8,fill=gray!20,draw=none](-2.526,1.623)--(-2.446,1.719)--(-2.467,1.733)--(-2.544,1.641)--cycle;
\draw(-2.526,1.623)--(-2.446,1.719)--(-2.467,1.733)--(-2.544,1.641);
\filldraw[fill opacity=0.8,fill=gray!20](-2.515,1.634)--(-2.517,1.67)--(-2.568,1.683)--(-2.563,1.646)--cycle;
\filldraw[fill opacity=0.8,fill=gray!20,draw=none](-2.692,1.277)--(-2.683,1.287)--(-2.671,1.306)--(-2.687,1.286)--cycle;
\draw(-2.692,1.277)--(-2.683,1.287);
\draw(-2.671,1.306)--(-2.687,1.286);
\filldraw[fill opacity=0.8,fill=gray!20,draw=none](-2.689,1.282)--(-2.688,1.274)--(-2.688,1.26)--(-2.694,1.271)--cycle;
\filldraw[fill opacity=0.8,fill=gray!20](-2.94,.738)--(-2.969,.75)--(-3.007,.759)--(-2.959,.743)--cycle;
\filldraw[fill opacity=0.8,fill=gray!20](-3.232,.655)--(-3.23,.69)--(-3.168,.686)--(-3.181,.652)--cycle;
\filldraw[fill opacity=0.8,fill=gray!20](-3.284,.653)--(-3.295,.687)--(-3.23,.69)--(-3.232,.655)--cycle;
\filldraw[fill opacity=0.8,fill=gray!20,draw=none](-2.374,1.813)--(-2.383,1.819)--(-2.382,1.816)--(-2.376,1.812)--cycle;
\draw(-2.382,1.816)--(-2.376,1.812)--(-2.374,1.813);
\filldraw[fill opacity=0.8,fill=gray!20](-2.494,1.779)--(-2.477,1.807)--(-2.502,1.814)--(-2.529,1.788)--cycle;
\filldraw[fill opacity=0.8,fill=gray!20,draw=none](-2.949,2.023)--(-2.935,1.909)--(-3.038,1.885)--(-3.056,2.026)--cycle;
\draw(-2.949,2.023)--(-2.935,1.909);
\draw(-3.038,1.885)--(-3.056,2.026);
\filldraw[fill opacity=0.8,fill=gray!20](-3.384,1.012)--(-3.355,1.052)--(-3.294,1.064)--(-3.31,1.026)--cycle;
\filldraw[fill opacity=0.8,fill=gray!20,draw=none](-3.275,.956)--(-3.323,.968)--(-3.258,.994)--cycle;
\draw(-3.275,.956)--(-3.323,.968);
\filldraw[fill opacity=0.8,fill=gray!20,draw=none](-3.439,.888)--(-3.438,.893)--(-3.41,.958)--(-3.402,.963)--(-3.408,.908)--cycle;
\draw(-3.41,.958)--(-3.402,.963)--(-3.408,.908)--(-3.439,.888);
\filldraw[fill opacity=0.8,fill=gray!20,draw=none](-3.418,.953)--(-3.414,.968)--(-3.392,1.006)--(-3.384,1.012)--(-3.402,.963)--cycle;
\draw(-3.392,1.006)--(-3.384,1.012)--(-3.402,.963)--(-3.418,.953);
\filldraw[fill opacity=0.8,fill=gray!20,draw=none](-3.438,.893)--(-3.425,.948)--(-3.41,.958)--cycle;
\draw(-3.425,.948)--(-3.41,.958);
\filldraw[fill opacity=0.8,fill=gray!20,draw=none](-3.418,.953)--(-3.425,.948)--(-3.414,.968)--cycle;
\draw(-3.418,.953)--(-3.425,.948);
\filldraw[fill opacity=0.8,fill=gray!20,draw=none](-3.306,.945)--(-3.319,.967)--(-3.275,.956)--cycle;
\draw(-3.319,.967)--(-3.275,.956);
\filldraw[fill opacity=0.8,fill=gray!20,draw=none](-3.303,.976)--(-3.376,1.022)--(-3.258,.994)--cycle;
\draw(-3.376,1.022)--(-3.258,.994);
\filldraw[fill opacity=0.8,fill=gray!20,draw=none](-3.276,.989)--(-3.124,1.025)--(-3.122,1.018)--(-3.135,.989)--(-3.285,.954)--cycle;
\draw(-3.135,.989)--(-3.285,.954)--(-3.276,.989)--(-3.124,1.025);
\filldraw[fill opacity=0.8,fill=gray!20](-2.861,.742)--(-2.817,.757)--(-2.86,.748)--(-2.883,.737)--cycle;
\filldraw[fill opacity=0.8,fill=gray!20](-3.31,1.026)--(-3.294,1.064)--(-3.216,1.067)--(-3.214,1.03)--cycle;
\filldraw[fill opacity=0.8,fill=gray!20,draw=none](-3.214,1.031)--(-3.216,1.067)--(-3.139,1.062)--(-3.13,1.042)--cycle;
\draw(-3.214,1.031)--(-3.216,1.067)--(-3.139,1.062)--(-3.13,1.042);
\filldraw[fill opacity=0.8,fill=gray!20,draw=none](-3.165,.698)--(-3.16,.723)--(-3.13,.715)--cycle;
\draw(-3.165,.698)--(-3.16,.723)--(-3.13,.715);
\filldraw[fill opacity=0.8,fill=gray!20,draw=none](-3.168,.686)--(-3.165,.698)--(-3.13,.715)--(-3.111,.711)--(-3.125,.675)--cycle;
\draw(-3.13,.715)--(-3.111,.711)--(-3.125,.675)--(-3.168,.686)--(-3.165,.698);
\filldraw[fill opacity=0.8,fill=gray!20,draw=none](-3.13,.715)--(-3.136,.717)--(-3.154,.734)--(-3.11,.723)--cycle;
\draw(-3.13,.715)--(-3.136,.717);
\filldraw[fill opacity=0.8,fill=gray!20,draw=none](-3.13,.715)--(-3.11,.723)--(-3.111,.711)--cycle;
\draw(-3.11,.723)--(-3.111,.711)--(-3.13,.715);
\filldraw[fill opacity=0.8,fill=gray!20](-3.165,.716)--(-3.139,.759)--(-3.086,.746)--(-3.127,.707)--cycle;
\filldraw[fill opacity=0.8,fill=gray!20,draw=none](-2.374,1.813)--(-2.376,1.812)--(-2.364,1.8)--cycle;
\draw(-2.374,1.813)--(-2.376,1.812)--(-2.364,1.8);
\filldraw[fill opacity=0.8,fill=gray!20,draw=none](-3.188,.851)--(-3.205,.874)--(-3.187,.878)--cycle;
\draw(-3.205,.874)--(-3.187,.878);
\filldraw[fill opacity=0.8,fill=gray!20,draw=none](-2.499,1.602)--(-2.481,1.59)--(-2.404,1.683)--(-2.424,1.701)--(-2.503,1.606)--cycle;
\draw(-2.481,1.59)--(-2.404,1.683)--(-2.424,1.701)--(-2.503,1.606);
\filldraw[fill opacity=0.8,fill=gray!20,draw=none](-2.468,1.575)--(-2.47,1.6)--(-2.506,1.602)--(-2.494,1.577)--cycle;
\draw(-2.47,1.6)--(-2.506,1.602)--(-2.494,1.577)--(-2.468,1.575);
\filldraw[fill opacity=0.8,fill=gray!20](-2.441,1.805)--(-2.439,1.827)--(-2.457,1.828)--(-2.477,1.807)--cycle;
\filldraw[fill opacity=0.8,fill=gray!20,draw=none](-2.423,1.827)--(-2.42,1.828)--(-2.421,1.828)--cycle;
\draw(-2.423,1.827)--(-2.42,1.828)--(-2.421,1.828);
\filldraw[fill opacity=0.8,fill=gray!20,draw=none](-3.28,.763)--(-3.303,.762)--(-3.301,.798)--(-3.259,.8)--cycle;
\draw(-3.28,.763)--(-3.303,.762)--(-3.301,.798)--(-3.259,.8);
\filldraw[fill opacity=0.8,fill=gray!20,draw=none](-3.09,1.009)--(-3.088,1.002)--(-3.104,.984)--cycle;
\draw(-3.088,1.002)--(-3.104,.984);
\filldraw[fill opacity=0.8,fill=gray!20,draw=none](-3.079,.969)--(-3.094,.972)--(-3.092,1.005)--cycle;
\draw(-3.079,.969)--(-3.094,.972);
\filldraw[fill opacity=0.8,fill=gray!20,draw=none](-3.104,.984)--(-3.093,.997)--(-3.086,.971)--(-3.107,.945)--cycle;
\draw(-3.104,.984)--(-3.093,.997);
\draw(-3.086,.971)--(-3.107,.945);
\filldraw[fill opacity=0.8,fill=gray!20,draw=none](-3.087,1.029)--(-3.069,1.05)--(-3.063,1.054)--(-3.071,1.044)--(-3.092,1.019)--cycle;
\draw(-3.087,1.029)--(-3.069,1.05);
\draw(-3.071,1.044)--(-3.092,1.019);
\filldraw[fill opacity=0.8,fill=gray!20,draw=none](-3.093,1.017)--(-3.087,1.029)--(-3.074,1.032)--(-3.055,1.008)--cycle;
\draw(-3.074,1.032)--(-3.055,1.008)--(-3.093,1.017);
\filldraw[fill opacity=0.8,fill=gray!20,draw=none](-3.11,.998)--(-3.092,1.019)--(-3.09,1.009)--(-3.104,.984)--(-3.117,.968)--cycle;
\draw(-3.11,.998)--(-3.092,1.019);
\draw(-3.104,.984)--(-3.117,.968);
\filldraw[fill opacity=0.8,fill=gray!20,draw=none](-3.087,1.029)--(-3.093,1.017)--(-3.11,1.021)--(-3.104,1.025)--cycle;
\draw(-3.093,1.017)--(-3.11,1.021);
\filldraw[fill opacity=0.8,fill=gray!20,draw=none](-3.108,1.001)--(-3.096,1.018)--(-3.087,1.029)--(-3.092,1.019)--(-3.11,.998)--cycle;
\draw(-3.096,1.018)--(-3.087,1.029);
\draw(-3.092,1.019)--(-3.11,.998);
\filldraw[fill opacity=0.8,fill=gray!20,draw=none](-3.105,.934)--(-3.098,.99)--(-3.12,1.013)--(-3.121,1.01)--(-3.128,.958)--cycle;
\draw(-3.121,1.01)--(-3.128,.958)--(-3.105,.934)--(-3.098,.99)--(-3.12,1.013);
\filldraw[fill opacity=0.8,fill=gray!20](-2.719,.846)--(-2.697,.897)--(-2.732,.875)--(-2.75,.825)--cycle;
\filldraw[fill opacity=0.8,fill=gray!20,draw=none](-2.383,1.81)--(-2.376,1.812)--(-2.382,1.816)--cycle;
\draw(-2.383,1.81)--(-2.376,1.812)--(-2.382,1.816);
\filldraw[fill opacity=0.8,fill=gray!20](-2.517,1.67)--(-2.515,1.708)--(-2.563,1.72)--(-2.568,1.683)--cycle;
\filldraw[fill opacity=0.8,fill=gray!20](-2.404,1.806)--(-2.42,1.828)--(-2.439,1.827)--(-2.441,1.805)--cycle;
\filldraw[fill opacity=0.5,fill=gray!20,draw=none](-1.698,2.109)--(-1.735,2.092)--(-1.731,2.124)--cycle;
\draw(-1.735,2.092)--(-1.731,2.124);
\filldraw[fill opacity=0.8,fill=gray!20,draw=none](-3.079,.969)--(-3.092,1.005)--(-3.092,1.017)--(-3.055,1.008)--(-3.035,.958)--cycle;
\draw(-3.092,1.017)--(-3.055,1.008)--(-3.035,.958)--(-3.079,.969);
\filldraw[fill opacity=0.8,fill=gray!20,draw=none](-3.087,1.029)--(-3.104,1.025)--(-3.082,1.039)--cycle;
\filldraw[fill opacity=0.8,fill=gray!20,draw=none](-3.105,1.034)--(-3.112,1.028)--(-3.12,1.013)--(-3.119,1.012)--cycle;
\draw(-3.12,1.013)--(-3.119,1.012);
\filldraw[fill opacity=0.8,fill=gray!20,draw=none](-3.108,1.029)--(-3.119,1.012)--(-3.108,1.001)--(-3.096,1.018)--cycle;
\draw(-3.119,1.012)--(-3.108,1.001);
\filldraw[fill opacity=0.8,fill=gray!20,draw=none](-3.125,1.043)--(-3.121,1.025)--(-3.121,1.025)--(-3.13,1.042)--cycle;
\draw(-3.121,1.025)--(-3.13,1.042);
\filldraw[fill opacity=0.8,fill=gray!20,draw=none](-3.139,1.037)--(-3.124,1.04)--(-3.118,1.026)--(-3.124,1.025)--cycle;
\draw(-3.139,1.037)--(-3.124,1.04);
\draw(-3.118,1.026)--(-3.124,1.025);
\filldraw[fill opacity=0.8,fill=gray!20,draw=none](-3.086,1.047)--(-3.092,1.047)--(-3.105,1.034)--(-3.108,1.029)--(-3.096,1.018)--(-3.081,1.041)--cycle;
\filldraw[fill opacity=0.8,fill=gray!20,draw=none](-3.11,1.021)--(-3.12,1.024)--(-3.121,1.025)--(-3.11,1.043)--(-3.083,1.044)--(-3.08,1.041)--cycle;
\draw(-3.11,1.021)--(-3.12,1.024);
\draw(-3.083,1.044)--(-3.08,1.041);
\filldraw[fill opacity=0.8,fill=gray!20,draw=none](-3.236,1.014)--(-3.139,1.037)--(-3.124,1.025)--(-3.215,1.004)--cycle;
\draw(-3.236,1.014)--(-3.139,1.037);
\draw(-3.124,1.025)--(-3.215,1.004);
\filldraw[fill opacity=0.8,fill=gray!20,draw=none](-3.063,1.054)--(-3.063,1.054)--(-3.063,1.054)--(-3.071,1.044)--cycle;
\draw(-3.063,1.054)--(-3.071,1.044);
\filldraw[fill opacity=0.8,fill=gray!20,draw=none](-3.121,.984)--(-3.11,.998)--(-3.117,.968)--(-3.121,.963)--cycle;
\draw(-3.121,.984)--(-3.11,.998);
\draw(-3.117,.968)--(-3.121,.963);
\filldraw[fill opacity=0.8,fill=gray!20,draw=none](-3.133,.974)--(-3.102,1.011)--(-3.11,.998)--(-3.163,.934)--cycle;
\draw(-3.133,.974)--(-3.102,1.011);
\draw(-3.11,.998)--(-3.163,.934);
\filldraw[fill opacity=0.8,fill=gray!20,draw=none](-3.074,1.032)--(-3.059,1.05)--(-3.069,1.05)--(-3.079,1.038)--cycle;
\draw(-3.074,1.032)--(-3.059,1.05);
\draw(-3.069,1.05)--(-3.079,1.038);
\filldraw[fill opacity=0.8,fill=gray!20,draw=none](-3.086,1.047)--(-3.081,1.041)--(-3.079,1.044)--(-3.08,1.046)--cycle;
\draw(-3.079,1.044)--(-3.08,1.046);
\filldraw[fill opacity=0.8,fill=gray!20,draw=none](-3.145,1.03)--(-2.922,1.082)--(-2.943,1.082)--(-3.124,1.04)--cycle;
\draw(-3.145,1.03)--(-2.922,1.082)--(-2.943,1.082)--(-3.124,1.04);
\filldraw[fill opacity=0.8,fill=gray!20,draw=none](-3.145,.812)--(-3.19,.826)--(-3.146,.837)--cycle;
\draw(-3.19,.826)--(-3.146,.837);
\filldraw[fill opacity=0.8,fill=gray!20,draw=none](-2.788,1.954)--(-2.752,1.806)--(-2.793,1.816)--(-2.809,1.937)--cycle;
\draw(-2.752,1.806)--(-2.793,1.816)--(-2.809,1.937);
\filldraw[fill opacity=0.8,fill=gray!20,draw=none](-2.467,1.573)--(-2.468,1.575)--(-2.471,1.575)--cycle;
\draw(-2.468,1.575)--(-2.471,1.575);
\filldraw[fill opacity=0.8,fill=gray!20,draw=none](-2.468,1.571)--(-2.467,1.573)--(-2.471,1.575)--cycle;
\draw(-2.468,1.571)--(-2.467,1.573);
\filldraw[fill opacity=0.8,fill=gray!20,draw=none](-2.793,1.181)--(-2.468,1.571)--(-2.471,1.575)--(-2.485,1.585)--(-2.831,1.169)--cycle;
\draw(-2.793,1.181)--(-2.468,1.571);
\draw(-2.485,1.585)--(-2.831,1.169);
\filldraw[fill opacity=0.8,fill=gray!20,draw=none](-2.348,1.607)--(-2.321,1.638)--(-2.319,1.643)--(-2.373,1.632)--(-2.38,1.601)--cycle;
\draw(-2.321,1.638)--(-2.319,1.643)--(-2.373,1.632)--(-2.38,1.601)--(-2.348,1.607);
\filldraw[fill opacity=0.8,fill=gray!20,draw=none](-2.318,1.643)--(-2.318,1.644)--(-2.318,1.644)--cycle;
\draw(-2.318,1.643)--(-2.318,1.644);
\filldraw[fill opacity=0.8,fill=gray!20,draw=none](-3.241,.832)--(-3.243,.811)--(-3.292,.816)--(-3.289,.851)--cycle;
\draw(-3.241,.832)--(-3.243,.811);
\draw(-3.292,.816)--(-3.289,.851);
\filldraw[fill opacity=0.8,fill=gray!20,draw=none](-3.217,.871)--(-3.205,.874)--(-3.173,.83)--(-3.19,.826)--cycle;
\draw(-3.217,.871)--(-3.205,.874);
\draw(-3.173,.83)--(-3.19,.826);
\filldraw[fill opacity=0.8,fill=gray!20,draw=none](-3.238,.866)--(-3.217,.871)--(-3.19,.826)--cycle;
\draw(-3.238,.866)--(-3.217,.871);
\filldraw[fill opacity=0.8,fill=gray!20,draw=none](-3.226,.823)--(-3.23,.82)--(-3.242,.825)--(-3.238,.866)--cycle;
\draw(-3.242,.825)--(-3.238,.866);
\filldraw[fill opacity=0.8,fill=gray!20,draw=none](-3.23,.82)--(-3.226,.823)--(-3.225,.818)--cycle;
\filldraw[fill opacity=0.8,fill=gray!20,draw=none](-3.184,.9)--(-3.192,.803)--(-3.225,.818)--(-3.238,.866)--(-3.235,.906)--cycle;
\draw(-3.238,.866)--(-3.235,.906)--(-3.184,.9)--(-3.192,.803);
\filldraw[fill opacity=0.8,fill=gray!20,draw=none](-3.18,.819)--(-3.168,.811)--(-3.192,.803)--(-3.192,.808)--cycle;
\draw(-3.192,.803)--(-3.192,.808);
\filldraw[fill opacity=0.8,fill=gray!20,draw=none](-3.18,.819)--(-3.192,.808)--(-3.19,.826)--cycle;
\draw(-3.192,.808)--(-3.19,.826);
\filldraw[fill opacity=0.8,fill=gray!20,draw=none](-3.217,.832)--(-3.226,.838)--(-3.202,.836)--(-3.19,.826)--(-3.2,.824)--cycle;
\draw(-3.19,.826)--(-3.2,.824);
\filldraw[fill opacity=0.8,fill=gray!20,draw=none](-3.171,.815)--(-3.18,.819)--(-3.19,.826)--(-3.182,.824)--cycle;
\filldraw[fill opacity=0.8,fill=gray!20,draw=none](-3.15,.806)--(-3.171,.815)--(-3.182,.824)--(-3.13,.808)--cycle;
\filldraw[fill opacity=0.8,fill=gray!20,draw=none](-3.176,.823)--(-3.18,.819)--(-3.19,.826)--(-3.189,.836)--cycle;
\draw(-3.19,.826)--(-3.189,.836);
\filldraw[fill opacity=0.8,fill=gray!20,draw=none](-3.18,.819)--(-3.195,.825)--(-3.19,.826)--cycle;
\draw(-3.195,.825)--(-3.19,.826);
\filldraw[fill opacity=0.8,fill=gray!20,draw=none](-3.166,.897)--(-3.165,.875)--(-3.176,.823)--(-3.189,.836)--(-3.184,.9)--cycle;
\draw(-3.189,.836)--(-3.184,.9)--(-3.166,.897);
\filldraw[fill opacity=0.8,fill=gray!20,draw=none](-3.168,.811)--(-3.204,.823)--(-3.195,.825)--(-3.18,.819)--cycle;
\draw(-3.204,.823)--(-3.195,.825);
\filldraw[fill opacity=0.8,fill=gray!20,draw=none](-3.217,.832)--(-3.2,.824)--(-3.204,.823)--cycle;
\draw(-3.2,.824)--(-3.204,.823);
\filldraw[fill opacity=0.8,fill=gray!20,draw=none](-3.163,.809)--(-3.168,.811)--(-3.18,.819)--(-3.171,.815)--cycle;
\filldraw[fill opacity=0.8,fill=gray!20,draw=none](-3.155,.806)--(-3.163,.809)--(-3.171,.815)--(-3.157,.809)--cycle;
\filldraw[fill opacity=0.8,fill=gray!20,draw=none](-3.18,.819)--(-3.176,.823)--(-3.164,.812)--(-3.168,.811)--cycle;
\filldraw[fill opacity=0.8,fill=gray!20,draw=none](-3.176,.823)--(-3.162,.835)--(-3.152,.816)--(-3.164,.812)--cycle;
\filldraw[fill opacity=0.8,fill=gray!20,draw=none](-3.165,.875)--(-3.162,.835)--(-3.176,.823)--cycle;
\filldraw[fill opacity=0.8,fill=gray!20,draw=none](-3.175,.829)--(-3.19,.845)--(-3.165,.875)--(-3.162,.836)--(-3.17,.826)--cycle;
\draw(-3.19,.845)--(-3.165,.875);
\draw(-3.162,.836)--(-3.17,.826);
\filldraw[fill opacity=0.8,fill=gray!20,draw=none](-3.152,.816)--(-3.167,.822)--(-3.17,.826)--(-3.162,.836)--cycle;
\draw(-3.17,.826)--(-3.162,.836);
\filldraw[fill opacity=0.8,fill=gray!20,draw=none](-3.167,.822)--(-3.171,.825)--(-3.17,.826)--cycle;
\draw(-3.171,.825)--(-3.17,.826);
\filldraw[fill opacity=0.8,fill=gray!20,draw=none](-3.175,.829)--(-3.17,.826)--(-3.171,.825)--cycle;
\draw(-3.17,.826)--(-3.171,.825);
\filldraw[fill opacity=0.8,fill=gray!20](-3.229,.802)--(-3.23,.833)--(-3.168,.828)--(-3.16,.797)--cycle;
\filldraw[fill opacity=0.8,fill=gray!20,draw=none](-3.23,.82)--(-3.243,.811)--(-3.242,.825)--cycle;
\draw(-3.243,.811)--(-3.242,.825);
\filldraw[fill opacity=0.8,fill=gray!20,draw=none](-3.291,.824)--(-3.292,.816)--(-3.296,.822)--cycle;
\draw(-3.291,.824)--(-3.292,.816);
\filldraw[fill opacity=0.8,fill=gray!20,draw=none](-3.287,.872)--(-3.291,.824)--(-3.296,.822)--(-3.328,.862)--(-3.328,.866)--cycle;
\draw(-3.287,.872)--(-3.291,.824);
\draw(-3.328,.862)--(-3.328,.866);
\filldraw[fill opacity=0.8,fill=gray!20](-3.301,.798)--(-3.295,.83)--(-3.23,.833)--(-3.229,.802)--cycle;
\filldraw[fill opacity=0.8,fill=gray!20,draw=none](-3.23,.82)--(-3.225,.818)--(-3.207,.754)--(-3.247,.758)--(-3.243,.811)--cycle;
\draw(-3.207,.754)--(-3.247,.758)--(-3.243,.811);
\filldraw[fill opacity=0.8,fill=gray!20,draw=none](-3.243,.811)--(-3.247,.758)--(-3.297,.76)--(-3.292,.816)--cycle;
\draw(-3.243,.811)--(-3.247,.758)--(-3.297,.76)--(-3.292,.816);
\filldraw[fill opacity=0.8,fill=gray!20](-3.25,.686)--(-3.274,.718)--(-3.218,.72)--(-3.222,.687)--cycle;
\filldraw[fill opacity=0.8,fill=gray!20](-3.222,.687)--(-3.218,.72)--(-3.165,.716)--(-3.194,.685)--cycle;
\filldraw[fill opacity=0.8,fill=gray!20,draw=none](-3.154,.734)--(-3.159,.738)--(-3.157,.761)--(-3.107,.748)--(-3.11,.723)--cycle;
\draw(-3.159,.738)--(-3.157,.761)--(-3.107,.748)--(-3.11,.723);
\filldraw[fill opacity=0.8,fill=gray!20,draw=none](-3.321,.759)--(-3.36,.751)--(-3.356,.788)--(-3.324,.794)--cycle;
\draw(-3.321,.759)--(-3.36,.751)--(-3.356,.788)--(-3.324,.794);
\filldraw[fill opacity=0.8,fill=gray!20](-3.272,.682)--(-3.317,.71)--(-3.274,.718)--(-3.25,.686)--cycle;
\filldraw[fill opacity=0.8,fill=gray!20](-3.181,.652)--(-3.168,.686)--(-3.125,.675)--(-3.145,.643)--cycle;
\filldraw[fill opacity=0.8,fill=gray!20](-3.194,.685)--(-3.165,.716)--(-3.127,.707)--(-3.174,.68)--cycle;
\filldraw[fill opacity=0.8,fill=gray!20,draw=none](-3.358,.747)--(-3.38,.732)--(-3.415,.777)--(-3.405,.784)--cycle;
\draw(-3.358,.747)--(-3.38,.732)--(-3.415,.777)--(-3.405,.784);
\filldraw[fill opacity=0.8,fill=gray!20,draw=none](-3.358,.747)--(-3.36,.747)--(-3.36,.748)--cycle;
\draw(-3.36,.747)--(-3.36,.748);
\filldraw[fill opacity=0.8,fill=gray!20,draw=none](-3.381,.734)--(-3.383,.736)--(-3.36,.751)--(-3.36,.748)--cycle;
\draw(-3.383,.736)--(-3.36,.751)--(-3.36,.748);
\filldraw[fill opacity=0.8,fill=gray!20,draw=none](-3.383,.736)--(-3.384,.738)--(-3.379,.773)--(-3.356,.788)--(-3.36,.751)--cycle;
\draw(-3.384,.738)--(-3.379,.773)--(-3.356,.788)--(-3.36,.751)--(-3.383,.736);
\filldraw[fill opacity=0.8,fill=gray!20,draw=none](-3.25,.73)--(-3.236,.747)--(-3.201,.744)--(-3.228,.712)--cycle;
\draw(-3.201,.744)--(-3.228,.712)--(-3.25,.73)--(-3.236,.747);
\filldraw[fill opacity=0.8,fill=gray!20,draw=none](-3.228,.712)--(-3.201,.744)--(-3.172,.74)--(-3.208,.697)--cycle;
\draw(-3.172,.74)--(-3.208,.697)--(-3.228,.712)--(-3.201,.744);
\filldraw[fill opacity=0.8,fill=gray!20,draw=none](-3.271,.748)--(-3.236,.747)--(-3.25,.73)--cycle;
\draw(-3.236,.747)--(-3.25,.73)--(-3.271,.748);
\filldraw[fill opacity=0.8,fill=gray!20,draw=none](-3.208,.697)--(-3.172,.74)--(-3.153,.734)--(-3.192,.688)--cycle;
\draw(-3.153,.734)--(-3.192,.688)--(-3.208,.697)--(-3.172,.74);
\filldraw[fill opacity=0.8,fill=gray!20,draw=none](-3.192,.688)--(-3.153,.734)--(-3.147,.729)--(-3.183,.686)--cycle;
\draw(-3.147,.729)--(-3.183,.686)--(-3.192,.688)--(-3.153,.734);
\filldraw[fill opacity=0.8,fill=gray!20,draw=none](-3.205,.72)--(-3.271,.748)--(-3.25,.73)--(-3.228,.712)--(-3.208,.697)--(-3.192,.688)--(-3.183,.686)--(-3.182,.692)--(-3.19,.703)--cycle;
\draw(-3.271,.748)--(-3.25,.73)--(-3.228,.712)--(-3.208,.697)--(-3.192,.688)--(-3.183,.686)--(-3.182,.692)--(-3.19,.703)--(-3.205,.72);
\filldraw[fill opacity=0.8,fill=gray!20,draw=none](-3.106,.705)--(-3.111,.711)--(-3.11,.723)--cycle;
\draw(-3.106,.705)--(-3.111,.711)--(-3.11,.723);
\filldraw[fill opacity=0.8,fill=gray!20,draw=none](-3.119,.691)--(-3.111,.711)--(-3.104,.703)--cycle;
\draw(-3.119,.691)--(-3.111,.711)--(-3.104,.703);
\filldraw[fill opacity=0.8,fill=gray!20,draw=none](-3.11,.723)--(-3.105,.727)--(-3.091,.713)--(-3.097,.709)--cycle;
\draw(-3.11,.723)--(-3.105,.727);
\draw(-3.091,.713)--(-3.097,.709);
\filldraw[fill opacity=0.8,fill=gray!20,draw=none](-3.092,.713)--(-3.105,.727)--(-3.086,.746)--(-3.078,.737)--cycle;
\draw(-3.105,.727)--(-3.086,.746)--(-3.078,.737);
\filldraw[fill opacity=0.8,fill=gray!20,draw=none](-3.097,.709)--(-3.11,.723)--(-3.107,.748)--(-3.092,.732)--(-3.095,.71)--cycle;
\draw(-3.11,.723)--(-3.107,.748)--(-3.092,.732)--(-3.095,.71);
\filldraw[fill opacity=0.8,fill=gray!20,draw=none](-3.104,.703)--(-3.106,.705)--(-3.11,.723)--(-3.097,.709)--cycle;
\draw(-3.104,.703)--(-3.106,.705);
\filldraw[fill opacity=0.8,fill=gray!20,draw=none](-3.127,.707)--(-3.11,.723)--(-3.097,.709)--(-3.115,.695)--cycle;
\draw(-3.097,.709)--(-3.115,.695)--(-3.127,.707)--(-3.11,.723);
\filldraw[fill opacity=0.8,fill=gray!20,draw=none](-3.183,.686)--(-3.147,.729)--(-3.155,.724)--(-3.182,.692)--cycle;
\draw(-3.155,.724)--(-3.182,.692)--(-3.183,.686)--(-3.147,.729);
\filldraw[fill opacity=0.8,fill=gray!20,draw=none](-3.182,.692)--(-3.155,.724)--(-3.175,.721)--(-3.19,.703)--cycle;
\draw(-3.175,.721)--(-3.19,.703)--(-3.182,.692)--(-3.155,.724);
\filldraw[fill opacity=0.8,fill=gray!20,draw=none](-3.19,.703)--(-3.175,.721)--(-3.205,.72)--cycle;
\draw(-3.205,.72)--(-3.19,.703)--(-3.175,.721);
\filldraw[fill opacity=0.8,fill=gray!20](-3.228,.71)--(-3.279,.715)--(-3.324,.724)--(-3.355,.733)--(-3.369,.743)--(-3.363,.752)--(-3.337,.757)--(-3.297,.76)--(-3.247,.758)--(-3.197,.752)--(-3.152,.744)--(-3.121,.734)--(-3.107,.725)--(-3.113,.716)--(-3.139,.71)--(-3.179,.708)--cycle;
\filldraw[fill opacity=0.8,fill=gray!20,draw=none](-2.318,1.643)--(-2.318,1.644)--(-2.318,1.646)--(-2.319,1.643)--cycle;
\draw(-2.318,1.646)--(-2.319,1.643)--(-2.318,1.643);
\filldraw[fill opacity=0.8,fill=gray!20,draw=none](-3.004,1.129)--(-2.985,1.152)--(-2.965,1.171)--(-3.028,1.096)--cycle;
\draw(-3.004,1.129)--(-2.985,1.152);
\draw(-2.965,1.171)--(-3.028,1.096);
\filldraw[fill opacity=0.8,fill=gray!20,draw=none](-2.979,1.158)--(-2.965,1.163)--(-2.956,1.169)--(-2.969,1.165)--cycle;
\draw(-2.979,1.158)--(-2.965,1.163)--(-2.956,1.169)--(-2.969,1.165);
\filldraw[fill opacity=0.8,fill=gray!20,draw=none](-2.979,1.146)--(-2.959,1.157)--(-2.965,1.163)--(-2.987,1.155)--cycle;
\draw(-2.979,1.146)--(-2.959,1.157)--(-2.965,1.163)--(-2.987,1.155);
\filldraw[fill opacity=0.8,fill=gray!20,draw=none](-2.985,1.152)--(-2.976,1.163)--(-2.963,1.174)--(-2.965,1.171)--cycle;
\draw(-2.985,1.152)--(-2.976,1.163);
\draw(-2.963,1.174)--(-2.965,1.171);
\filldraw[fill opacity=0.8,fill=gray!20,draw=none](-2.956,1.167)--(-2.973,1.158)--(-2.976,1.163)--(-2.962,1.175)--(-2.958,1.175)--cycle;
\filldraw[fill opacity=0.8,fill=gray!20,draw=none](-2.956,1.167)--(-2.958,1.175)--(-2.936,1.177)--cycle;
\filldraw[fill opacity=0.8,fill=gray!20,draw=none](-2.96,1.141)--(-2.947,1.144)--(-2.94,1.152)--(-2.959,1.157)--(-2.979,1.146)--cycle;
\draw(-2.947,1.144)--(-2.94,1.152)--(-2.959,1.157)--(-2.979,1.146);
\filldraw[fill opacity=0.8,fill=gray!20,draw=none](-2.954,1.16)--(-2.951,1.152)--(-2.971,1.156)--(-2.973,1.158)--(-2.966,1.162)--cycle;
\filldraw[fill opacity=0.8,fill=gray!20,draw=none](-2.971,1.156)--(-2.721,1.456)--(-2.737,1.449)--(-2.976,1.163)--cycle;
\draw(-2.971,1.156)--(-2.721,1.456);
\draw(-2.737,1.449)--(-2.976,1.163);
\filldraw[fill opacity=0.8,fill=gray!20,draw=none](-2.318,1.644)--(-2.317,1.655)--(-2.318,1.646)--cycle;
\draw(-2.317,1.655)--(-2.318,1.646);
\filldraw[fill opacity=0.8,fill=gray!20,draw=none](-2.36,1.784)--(-2.364,1.8)--(-2.376,1.812)--(-2.404,1.806)--(-2.39,1.778)--cycle;
\draw(-2.364,1.8)--(-2.376,1.812)--(-2.404,1.806)--(-2.39,1.778)--(-2.36,1.784);
\filldraw[fill opacity=0.8,fill=gray!20,draw=none](-3.321,.759)--(-3.324,.794)--(-3.301,.798)--(-3.303,.762)--cycle;
\draw(-3.324,.794)--(-3.301,.798)--(-3.303,.762)--(-3.321,.759);
\filldraw[fill opacity=0.8,fill=gray!20](-2.506,1.745)--(-2.494,1.779)--(-2.529,1.788)--(-2.55,1.756)--cycle;
\filldraw[fill opacity=0.8,fill=gray!20,draw=none](-2.319,1.643)--(-2.318,1.646)--(-2.372,1.652)--(-2.373,1.632)--cycle;
\draw(-2.372,1.652)--(-2.373,1.632)--(-2.319,1.643)--(-2.318,1.646);
\filldraw[fill opacity=0.8,fill=gray!20,draw=none](-2.39,1.576)--(-2.38,1.601)--(-2.437,1.598)--(-2.444,1.589)--(-2.443,1.573)--cycle;
\draw(-2.444,1.589)--(-2.443,1.573)--(-2.39,1.576)--(-2.38,1.601)--(-2.437,1.598);
\filldraw[fill opacity=0.8,fill=gray!20,draw=none](-2.318,1.646)--(-2.317,1.655)--(-2.321,1.678)--(-2.371,1.668)--(-2.372,1.652)--cycle;
\draw(-2.318,1.646)--(-2.317,1.655);
\draw(-2.321,1.678)--(-2.371,1.668)--(-2.372,1.652);
\filldraw[fill opacity=0.8,fill=gray!20](-3.324,.645)--(-3.344,.678)--(-3.295,.687)--(-3.284,.653)--cycle;
\filldraw[fill opacity=0.8,fill=gray!20,draw=none](-2.752,1.204)--(-2.692,1.277)--(-2.687,1.286)--(-2.765,1.193)--cycle;
\draw(-2.752,1.204)--(-2.692,1.277);
\draw(-2.687,1.286)--(-2.765,1.193);
\filldraw[fill opacity=0.8,fill=gray!20,draw=none](-2.779,1.798)--(-2.797,1.814)--(-2.789,1.815)--(-2.715,1.797)--(-2.713,1.795)--cycle;
\draw(-2.789,1.815)--(-2.715,1.797)--(-2.713,1.795);
\filldraw[fill opacity=0.8,fill=gray!20,draw=none](-2.688,1.274)--(-2.684,1.251)--(-2.688,1.26)--cycle;
\filldraw[fill opacity=0.8,fill=gray!20,draw=none](-2.471,1.575)--(-2.481,1.59)--(-2.485,1.585)--cycle;
\draw(-2.481,1.59)--(-2.485,1.585);
\filldraw[fill opacity=0.5,fill=gray!20,draw=none](-1.615,2.07)--(-1.731,2.124)--(-1.698,2.4)--(-1.58,2.363)--cycle;
\draw(-1.731,2.124)--(-1.698,2.4)--(-1.58,2.363)--(-1.615,2.07);
\filldraw[fill opacity=0.8,fill=gray!20](-2.515,1.708)--(-2.506,1.745)--(-2.55,1.756)--(-2.563,1.72)--cycle;
\filldraw[fill opacity=0.8,fill=gray!20,draw=none](-2.468,1.575)--(-2.453,1.574)--(-2.444,1.589)--(-2.444,1.598)--(-2.47,1.6)--cycle;
\draw(-2.468,1.575)--(-2.453,1.574);
\draw(-2.444,1.589)--(-2.444,1.598)--(-2.47,1.6);
\filldraw[fill opacity=0.8,fill=gray!20,draw=none](-3.192,.803)--(-3.197,.752)--(-3.207,.754)--(-3.225,.818)--cycle;
\draw(-3.192,.803)--(-3.197,.752)--(-3.207,.754);
\filldraw[fill opacity=0.8,fill=gray!20,draw=none](-3.136,.777)--(-3.124,.752)--(-3.157,.761)--(-3.158,.776)--cycle;
\draw(-3.124,.752)--(-3.157,.761)--(-3.158,.776);
\filldraw[fill opacity=0.8,fill=gray!20,draw=none](-3.183,.798)--(-3.16,.797)--(-3.158,.77)--cycle;
\draw(-3.183,.798)--(-3.16,.797)--(-3.158,.77);
\filldraw[fill opacity=0.8,fill=gray!20,draw=none](-3.136,.777)--(-3.158,.776)--(-3.16,.797)--(-3.143,.792)--cycle;
\draw(-3.158,.776)--(-3.16,.797)--(-3.143,.792);
\filldraw[fill opacity=0.8,fill=gray!20,draw=none](-3.15,.797)--(-3.147,.796)--(-3.143,.792)--(-3.16,.797)--(-3.16,.798)--cycle;
\draw(-3.143,.792)--(-3.16,.797)--(-3.16,.798);
\filldraw[fill opacity=0.8,fill=gray!20,draw=none](-3.15,.797)--(-3.16,.798)--(-3.161,.802)--cycle;
\draw(-3.16,.798)--(-3.161,.802);
\filldraw[fill opacity=0.8,fill=gray!20,draw=none](-3.15,.806)--(-3.13,.808)--(-3.116,.803)--(-3.134,.799)--cycle;
\draw(-3.116,.803)--(-3.134,.799);
\filldraw[fill opacity=0.8,fill=gray!20,draw=none](-3.148,.814)--(-3.154,.845)--(-3.113,.895)--(-3.1,.867)--(-3.144,.813)--cycle;
\draw(-3.154,.845)--(-3.113,.895);
\draw(-3.1,.867)--(-3.144,.813);
\filldraw[fill opacity=0.8,fill=gray!20,draw=none](-3.147,.796)--(-3.161,.802)--(-3.163,.809)--cycle;
\draw(-3.161,.802)--(-3.163,.809);
\filldraw[fill opacity=0.8,fill=gray!20,draw=none](-3.155,.806)--(-3.157,.809)--(-3.134,.799)--(-3.135,.798)--cycle;
\draw(-3.134,.799)--(-3.135,.798);
\filldraw[fill opacity=0.8,fill=gray!20,draw=none](-3.168,.811)--(-3.155,.806)--(-3.148,.795)--cycle;
\filldraw[fill opacity=0.8,fill=gray!20,draw=none](-3.148,.797)--(-3.149,.797)--(-3.163,.809)--(-3.164,.812)--cycle;
\draw(-3.163,.809)--(-3.164,.812);
\filldraw[fill opacity=0.8,fill=gray!20,draw=none](-3.13,.808)--(-3.088,.812)--(-3.088,.809)--(-3.116,.803)--cycle;
\draw(-3.088,.809)--(-3.116,.803);
\filldraw[fill opacity=0.8,fill=gray!20,draw=none](-3.14,.794)--(-3.144,.796)--(-3.116,.803)--(-3.091,.802)--(-3.097,.783)--cycle;
\draw(-3.144,.796)--(-3.116,.803);
\filldraw[fill opacity=0.8,fill=gray!20,draw=none](-3.155,.806)--(-3.135,.798)--(-3.148,.795)--cycle;
\draw(-3.135,.798)--(-3.148,.795);
\filldraw[fill opacity=0.8,fill=gray!20,draw=none](-3.148,.814)--(-3.152,.816)--(-3.162,.836)--(-3.154,.845)--cycle;
\draw(-3.162,.836)--(-3.154,.845);
\filldraw[fill opacity=0.8,fill=gray!20,draw=none](-3.148,.797)--(-3.146,.795)--(-3.147,.796)--(-3.149,.797)--cycle;
\filldraw[fill opacity=0.8,fill=gray!20,draw=none](-3.152,.816)--(-3.148,.814)--(-3.147,.809)--(-3.148,.808)--cycle;
\draw(-3.147,.809)--(-3.148,.808);
\filldraw[fill opacity=0.8,fill=gray!20,draw=none](-3.148,.814)--(-3.144,.813)--(-3.147,.809)--cycle;
\draw(-3.144,.813)--(-3.147,.809);
\filldraw[fill opacity=0.8,fill=gray!20,draw=none](-3.147,.808)--(-3.148,.808)--(-3.147,.809)--cycle;
\draw(-3.148,.808)--(-3.147,.809);
\filldraw[fill opacity=0.8,fill=gray!20,draw=none](-3.146,.818)--(-3.152,.744)--(-3.197,.752)--(-3.192,.803)--cycle;
\draw(-3.146,.818)--(-3.152,.744)--(-3.197,.752)--(-3.192,.803);
\filldraw[fill opacity=0.8,fill=gray!20](-3.271,.624)--(-3.284,.653)--(-3.232,.655)--(-3.234,.626)--cycle;
\filldraw[fill opacity=0.8,fill=gray!20](-3.234,.626)--(-3.232,.655)--(-3.181,.652)--(-3.198,.623)--cycle;
\filldraw[fill opacity=0.8,fill=gray!20,draw=none](-2.928,1.912)--(-2.931,1.881)--(-2.935,1.909)--cycle;
\draw(-2.931,1.881)--(-2.935,1.909);
\filldraw[fill opacity=0.8,fill=gray!20,draw=none](-3.122,.999)--(-3.122,1.018)--(-3.12,1.024)--(-3.118,1.015)--cycle;
\draw(-3.12,1.024)--(-3.118,1.015);
\filldraw[fill opacity=0.8,fill=gray!20,draw=none](-2.51,1.617)--(-2.515,1.634)--(-2.544,1.642)--cycle;
\draw(-2.51,1.617)--(-2.515,1.634)--(-2.544,1.642);
\filldraw[fill opacity=0.8,fill=gray!20,draw=none](-2.687,1.43)--(-2.526,1.623)--(-2.544,1.641)--(-2.704,1.449)--cycle;
\draw(-2.687,1.43)--(-2.526,1.623);
\draw(-2.544,1.641)--(-2.704,1.449);
\filldraw[fill opacity=0.8,fill=gray!20,draw=none](-2.453,1.574)--(-2.443,1.573)--(-2.444,1.589)--cycle;
\draw(-2.453,1.574)--(-2.443,1.573)--(-2.444,1.589);
\filldraw[fill opacity=0.8,fill=gray!20](-2.912,.736)--(-2.915,.746)--(-2.969,.75)--(-2.94,.738)--cycle;
\filldraw[fill opacity=0.8,fill=gray!20](-2.697,.897)--(-2.689,.953)--(-2.725,.929)--(-2.732,.875)--cycle;
\filldraw[fill opacity=0.8,fill=gray!20,draw=none](-3.007,1.15)--(-3.018,1.143)--(-2.987,1.155)--cycle;
\draw(-3.007,1.15)--(-3.018,1.143)--(-2.987,1.155);
\filldraw[fill opacity=0.8,fill=gray!20,draw=none](-3.116,.803)--(-3.088,.809)--(-3.091,.802)--cycle;
\draw(-3.116,.803)--(-3.088,.809);
\filldraw[fill opacity=0.8,fill=gray!20,draw=none](-3.062,1.055)--(-3.028,1.096)--(-3.025,1.078)--(-3.039,1.062)--cycle;
\draw(-3.062,1.055)--(-3.028,1.096);
\draw(-3.025,1.078)--(-3.039,1.062);
\filldraw[fill opacity=0.8,fill=gray!20,draw=none](-2.804,2.018)--(-2.788,1.954)--(-2.809,1.937)--(-2.819,2.019)--cycle;
\draw(-2.809,1.937)--(-2.819,2.019);
\filldraw[fill opacity=0.8,fill=gray!20,draw=none](-2.763,1.784)--(-2.726,1.494)--(-2.799,1.793)--cycle;
\draw(-2.763,1.784)--(-2.726,1.494);
\filldraw[fill opacity=0.8,fill=gray!20,draw=none](-2.891,1.157)--(-2.917,1.154)--(-2.95,1.152)--(-2.951,1.152)--(-2.956,1.167)--(-2.936,1.177)--(-2.878,1.183)--cycle;
\filldraw[fill opacity=0.8,fill=gray!20,draw=none](-2.954,1.16)--(-2.966,1.162)--(-2.956,1.167)--cycle;
\filldraw[fill opacity=0.8,fill=gray!20,draw=none](-2.712,1.438)--(-2.704,1.449)--(-2.721,1.456)--cycle;
\draw(-2.712,1.438)--(-2.704,1.449);
\filldraw[fill opacity=0.8,fill=gray!20,draw=none](-3.119,1.012)--(-3.118,1.015)--(-3.118,1.014)--cycle;
\draw(-3.118,1.015)--(-3.118,1.014);
\filldraw[fill opacity=0.8,fill=gray!20,draw=none](-3.118,1.014)--(-3.118,1.015)--(-3.115,1.022)--(-3.11,1.021)--cycle;
\draw(-3.118,1.014)--(-3.118,1.015);
\draw(-3.115,1.022)--(-3.11,1.021);
\filldraw[fill opacity=0.8,fill=gray!20,draw=none](-3.118,1.015)--(-3.12,1.024)--(-3.115,1.022)--cycle;
\draw(-3.118,1.015)--(-3.12,1.024)--(-3.115,1.022);
\filldraw[fill opacity=0.8,fill=gray!20,draw=none](-3.226,.838)--(-3.261,.861)--(-3.238,.866)--(-3.202,.836)--cycle;
\draw(-3.261,.861)--(-3.238,.866);
\filldraw[fill opacity=0.8,fill=gray!20,draw=none](-3.235,.906)--(-3.238,.866)--(-3.284,.908)--cycle;
\draw(-3.284,.908)--(-3.235,.906)--(-3.238,.866);
\filldraw[fill opacity=0.8,fill=gray!20,draw=none](-3.091,1.013)--(-3.088,1.015)--(-3.074,1.032)--(-3.079,1.038)--(-3.096,1.018)--cycle;
\draw(-3.088,1.015)--(-3.074,1.032);
\draw(-3.079,1.038)--(-3.096,1.018);
\filldraw[fill opacity=0.8,fill=gray!20,draw=none](-3.087,1.029)--(-3.082,1.039)--(-3.08,1.041)--(-3.074,1.032)--cycle;
\draw(-3.08,1.041)--(-3.074,1.032);
\filldraw[fill opacity=0.8,fill=gray!20,draw=none](-3.044,1.04)--(-3.017,1.035)--(-2.898,1.062)--(-2.922,1.082)--(-3.059,1.05)--cycle;
\draw(-3.017,1.035)--(-2.898,1.062)--(-2.922,1.082)--(-3.059,1.05);
\filldraw[fill opacity=0.8,fill=gray!20,draw=none](-2.703,1.41)--(-2.687,1.43)--(-2.704,1.449)--(-2.712,1.438)--cycle;
\draw(-2.703,1.41)--(-2.687,1.43);
\draw(-2.704,1.449)--(-2.712,1.438);
\filldraw[fill opacity=0.8,fill=gray!20,draw=none](-2.762,1.783)--(-2.702,1.418)--(-2.708,1.429)--(-2.73,1.523)--(-2.763,1.784)--cycle;
\draw(-2.73,1.523)--(-2.763,1.784);
\filldraw[fill opacity=0.8,fill=gray!20,draw=none](-2.492,1.601)--(-2.508,1.608)--(-2.506,1.602)--cycle;
\draw(-2.508,1.608)--(-2.506,1.602)--(-2.492,1.601);
\filldraw[fill opacity=0.8,fill=gray!20,draw=none](-2.689,1.34)--(-2.485,1.586)--(-2.499,1.602)--(-2.504,1.605)--(-2.695,1.376)--cycle;
\draw(-2.689,1.34)--(-2.485,1.586);
\draw(-2.504,1.605)--(-2.695,1.376);
\filldraw[fill opacity=0.8,fill=gray!20,draw=none](-2.884,1.17)--(-2.878,1.183)--(-2.858,1.185)--cycle;
\filldraw[fill opacity=0.8,fill=gray!20,draw=none](-2.695,1.376)--(-2.681,1.393)--(-2.682,1.421)--(-2.687,1.43)--(-2.703,1.41)--cycle;
\draw(-2.695,1.376)--(-2.681,1.393);
\draw(-2.687,1.43)--(-2.703,1.41);
\filldraw[fill opacity=0.8,fill=gray!20,draw=none](-2.762,1.783)--(-2.738,1.761)--(-2.733,1.745)--(-2.704,1.52)--(-2.692,1.377)--(-2.698,1.395)--cycle;
\draw(-2.733,1.745)--(-2.704,1.52);
\filldraw[fill opacity=0.8,fill=gray!20,draw=none](-2.704,1.52)--(-2.684,1.361)--(-2.686,1.309)--cycle;
\draw(-2.704,1.52)--(-2.684,1.361);
\filldraw[fill opacity=0.8,fill=gray!20,draw=none](-2.731,1.598)--(-2.712,1.447)--(-2.684,1.361)--(-2.733,1.745)--cycle;
\draw(-2.731,1.598)--(-2.712,1.447);
\draw(-2.684,1.361)--(-2.733,1.745);
\filldraw[fill opacity=0.8,fill=gray!20,draw=none](-2.735,1.759)--(-2.733,1.745)--(-2.738,1.761)--cycle;
\draw(-2.735,1.759)--(-2.733,1.745);
\filldraw[fill opacity=0.8,fill=gray!20,draw=none](-2.748,1.728)--(-2.731,1.598)--(-2.733,1.745)--(-2.735,1.759)--cycle;
\draw(-2.748,1.728)--(-2.731,1.598);
\draw(-2.733,1.745)--(-2.735,1.759);
\filldraw[fill opacity=0.8,fill=gray!20,draw=none](-2.977,1.173)--(-3.187,1.109)--(-3.207,1.127)--(-3.206,1.13)--(-3.034,1.167)--cycle;
\draw(-3.187,1.109)--(-3.207,1.127)--(-3.206,1.13);
\filldraw[fill opacity=0.8,fill=gray!20](-3.047,1.092)--(-3.007,1.13)--(-3.018,1.143)--(-3.064,1.109)--cycle;
\filldraw[fill opacity=0.8,fill=gray!20,draw=none](-3.059,1.063)--(-3.004,1.129)--(-3.028,1.096)--(-3.063,1.054)--cycle;
\draw(-3.059,1.063)--(-3.004,1.129);
\draw(-3.028,1.096)--(-3.063,1.054);
\filldraw[fill opacity=0.8,fill=gray!20,draw=none](-3.035,1.079)--(-2.979,1.146)--(-2.976,1.163)--(-3.059,1.063)--cycle;
\draw(-3.035,1.079)--(-2.979,1.146);
\draw(-2.976,1.163)--(-3.059,1.063);
\filldraw[fill opacity=0.8,fill=gray!20,draw=none](-3.014,1.138)--(-2.979,1.146)--(-2.987,1.155)--(-3.018,1.143)--cycle;
\draw(-2.987,1.155)--(-3.018,1.143)--(-3.014,1.138);
\filldraw[fill opacity=0.8,fill=gray!20,draw=none](-2.979,1.146)--(-2.971,1.156)--(-2.976,1.163)--cycle;
\draw(-2.979,1.146)--(-2.971,1.156);
\filldraw[fill opacity=0.8,fill=gray!20,draw=none](-2.975,1.163)--(-2.973,1.158)--(-3.111,1.086)--(-3.113,1.086)--(-3.179,1.102)--(-3.187,1.109)--(-3.001,1.166)--cycle;
\draw(-3.111,1.086)--(-3.113,1.086)--(-3.179,1.102)--(-3.187,1.109);
\filldraw[fill opacity=0.8,fill=gray!20,draw=none](-3.25,1.658)--(-3.232,1.516)--(-3.127,1.09)--(-3.113,1.086)--(-3.184,1.643)--cycle;
\draw(-3.25,1.658)--(-3.232,1.516);
\draw(-3.127,1.09)--(-3.113,1.086)--(-3.184,1.643);
\filldraw[fill opacity=0.8,fill=gray!20,draw=none](-2.8,1.696)--(-2.775,1.501)--(-2.731,1.598)--(-2.748,1.728)--cycle;
\draw(-2.8,1.696)--(-2.775,1.501);
\draw(-2.731,1.598)--(-2.748,1.728);
\filldraw[fill opacity=0.8,fill=gray!20,draw=none](-2.95,1.152)--(-2.712,1.438)--(-2.721,1.456)--(-2.971,1.156)--cycle;
\draw(-2.95,1.152)--(-2.712,1.438);
\draw(-2.721,1.456)--(-2.971,1.156);
\filldraw[fill opacity=0.8,fill=gray!20,draw=none](-2.917,1.154)--(-2.703,1.41)--(-2.712,1.438)--(-2.95,1.152)--cycle;
\draw(-2.917,1.154)--(-2.703,1.41);
\draw(-2.712,1.438)--(-2.95,1.152);
\filldraw[fill opacity=0.8,fill=gray!20,draw=none](-2.891,1.063)--(-2.823,1.144)--(-2.854,1.141)--(-2.892,1.096)--cycle;
\draw(-2.891,1.063)--(-2.823,1.144);
\draw(-2.854,1.141)--(-2.892,1.096);
\filldraw[fill opacity=0.8,fill=gray!20,draw=none](-2.809,1.161)--(-2.793,1.181)--(-2.831,1.169)--(-2.849,1.148)--cycle;
\draw(-2.809,1.161)--(-2.793,1.181);
\draw(-2.831,1.169)--(-2.849,1.148);
\filldraw[fill opacity=0.8,fill=gray!20,draw=none](-2.831,1.169)--(-2.747,1.195)--(-2.744,1.188)--cycle;
\filldraw[fill opacity=0.8,fill=gray!20](-2.754,1.105)--(-2.799,1.14)--(-2.817,1.128)--(-2.779,1.088)--cycle;
\filldraw[fill opacity=0.8,fill=gray!20,draw=none](-2.807,1.143)--(-2.805,1.144)--(-2.817,1.152)--(-2.822,1.146)--cycle;
\draw(-2.807,1.143)--(-2.805,1.144);
\draw(-2.817,1.152)--(-2.822,1.146);
\filldraw[fill opacity=0.8,fill=gray!20,draw=none](-2.807,1.143)--(-2.822,1.146)--(-2.826,1.141)--cycle;
\draw(-2.822,1.146)--(-2.826,1.141);
\filldraw[fill opacity=0.8,fill=gray!20,draw=none](-2.875,1.16)--(-2.695,1.376)--(-2.703,1.41)--(-2.917,1.154)--cycle;
\draw(-2.875,1.16)--(-2.695,1.376);
\draw(-2.703,1.41)--(-2.917,1.154);
\filldraw[fill opacity=0.8,fill=gray!20,draw=none](-2.831,1.169)--(-2.689,1.34)--(-2.695,1.376)--(-2.875,1.16)--cycle;
\draw(-2.831,1.169)--(-2.689,1.34);
\draw(-2.695,1.376)--(-2.875,1.16);
\filldraw[fill opacity=0.8,fill=gray!20,draw=none](-2.802,1.137)--(-2.84,1.143)--(-2.817,1.128)--cycle;
\draw(-2.84,1.143)--(-2.817,1.128)--(-2.802,1.137);
\filldraw[fill opacity=0.8,fill=gray!20,draw=none](-2.835,1.139)--(-2.861,1.156)--(-2.883,1.151)--(-2.878,1.144)--cycle;
\draw(-2.835,1.139)--(-2.861,1.156)--(-2.883,1.151)--(-2.878,1.144);
\filldraw[fill opacity=0.8,fill=gray!20,draw=none](-2.892,1.096)--(-2.854,1.141)--(-2.878,1.144)--(-2.889,1.143)--(-2.898,1.131)--cycle;
\draw(-2.892,1.096)--(-2.854,1.141);
\draw(-2.889,1.143)--(-2.898,1.131);
\filldraw[fill opacity=0.8,fill=gray!20,draw=none](-2.876,1.141)--(-2.883,1.151)--(-2.912,1.15)--(-2.913,1.145)--cycle;
\draw(-2.876,1.141)--(-2.883,1.151)--(-2.912,1.15)--(-2.913,1.145);
\filldraw[fill opacity=0.8,fill=gray!20,draw=none](-2.878,1.144)--(-2.849,1.148)--(-2.831,1.169)--(-2.875,1.16)--(-2.887,1.145)--cycle;
\draw(-2.849,1.148)--(-2.831,1.169);
\draw(-2.875,1.16)--(-2.887,1.145);
\filldraw[fill opacity=0.8,fill=gray!20,draw=none](-2.852,1.163)--(-2.831,1.169)--(-2.744,1.188)--(-2.738,1.179)--cycle;
\filldraw[fill opacity=0.8,fill=gray!20,draw=none](-2.823,1.144)--(-2.809,1.161)--(-2.849,1.148)--(-2.854,1.141)--cycle;
\draw(-2.823,1.144)--(-2.809,1.161);
\draw(-2.849,1.148)--(-2.854,1.141);
\filldraw[fill opacity=0.8,fill=gray!20,draw=none](-2.996,1.098)--(-2.95,1.152)--(-2.971,1.156)--(-3.035,1.079)--cycle;
\draw(-2.996,1.098)--(-2.95,1.152);
\draw(-2.971,1.156)--(-3.035,1.079);
\filldraw[fill opacity=0.8,fill=gray!20,draw=none](-2.829,1.126)--(-2.817,1.128)--(-2.835,1.139)--(-2.878,1.144)--(-2.876,1.141)--cycle;
\draw(-2.829,1.126)--(-2.817,1.128)--(-2.835,1.139);
\draw(-2.878,1.144)--(-2.876,1.141);
\filldraw[fill opacity=0.8,fill=gray!20,draw=none](-2.831,1.169)--(-2.852,1.163)--(-2.875,1.16)--cycle;
\filldraw[fill opacity=0.8,fill=gray!20,draw=none](-2.96,1.141)--(-2.979,1.146)--(-3.007,1.13)--cycle;
\draw(-2.979,1.146)--(-3.007,1.13);
\filldraw[fill opacity=0.8,fill=gray!20,draw=none](-2.995,1.059)--(-2.926,1.142)--(-2.956,1.145)--(-3.033,1.053)--cycle;
\draw(-2.995,1.059)--(-2.926,1.142);
\draw(-2.956,1.145)--(-3.033,1.053);
\filldraw[fill opacity=0.8,fill=gray!20,draw=none](-2.969,1.121)--(-2.952,1.14)--(-2.96,1.141)--(-3.007,1.13)--cycle;
\draw(-3.007,1.13)--(-2.969,1.121)--(-2.952,1.14);
\filldraw[fill opacity=0.8,fill=gray!20,draw=none](-2.913,1.145)--(-2.912,1.15)--(-2.94,1.152)--(-2.954,1.137)--cycle;
\draw(-2.913,1.145)--(-2.912,1.15)--(-2.94,1.152)--(-2.954,1.137);
\filldraw[fill opacity=0.8,fill=gray!20,draw=none](-2.878,1.144)--(-2.854,1.141)--(-2.849,1.148)--cycle;
\draw(-2.854,1.141)--(-2.849,1.148);
\filldraw[fill opacity=0.8,fill=gray!20,draw=none](-2.926,1.142)--(-2.917,1.154)--(-2.95,1.152)--(-2.956,1.145)--cycle;
\draw(-2.926,1.142)--(-2.917,1.154);
\draw(-2.95,1.152)--(-2.956,1.145);
\filldraw[fill opacity=0.8,fill=gray!20,draw=none](-2.805,1.144)--(-2.799,1.151)--(-2.812,1.157)--(-2.817,1.152)--cycle;
\draw(-2.805,1.144)--(-2.799,1.151);
\draw(-2.812,1.157)--(-2.817,1.152);
\filldraw[fill opacity=0.8,fill=gray!20,draw=none](-2.878,1.144)--(-2.887,1.145)--(-2.889,1.143)--cycle;
\draw(-2.887,1.145)--(-2.889,1.143);
\filldraw[fill opacity=0.8,fill=gray!20,draw=none](-2.951,1.068)--(-2.875,1.16)--(-2.917,1.154)--(-2.995,1.059)--cycle;
\draw(-2.951,1.068)--(-2.875,1.16);
\draw(-2.917,1.154)--(-2.995,1.059);
\filldraw[fill opacity=0.8,fill=gray!20,draw=none](-2.877,1.155)--(-2.852,1.163)--(-2.738,1.179)--(-2.728,1.161)--cycle;
\filldraw[fill opacity=0.8,fill=gray!20](-2.779,1.088)--(-2.817,1.128)--(-2.86,1.12)--(-2.839,1.077)--cycle;
\filldraw[fill opacity=0.8,fill=gray!20,draw=none](-2.926,1.148)--(-2.943,1.151)--(-2.938,1.153)--(-2.898,1.154)--cycle;
\filldraw[fill opacity=0.8,fill=gray!20,draw=none](-2.852,1.163)--(-2.877,1.155)--(-2.892,1.155)--(-2.891,1.157)--cycle;
\filldraw[fill opacity=0.8,fill=gray!20,draw=none](-2.891,1.157)--(-2.892,1.155)--(-2.917,1.154)--cycle;
\filldraw[fill opacity=0.8,fill=gray!20,draw=none](-3.19,.802)--(-3.239,.815)--(-3.204,.823)--(-3.168,.811)--(-3.16,.805)--cycle;
\draw(-3.239,.815)--(-3.204,.823);
\filldraw[fill opacity=0.8,fill=gray!20,draw=none](-2.96,1.141)--(-2.952,1.14)--(-2.947,1.144)--cycle;
\draw(-2.952,1.14)--(-2.947,1.144);
\filldraw[fill opacity=0.8,fill=gray!20,draw=none](-3.031,.881)--(-2.915,1.02)--(-2.904,1.022)--(-3.01,.894)--cycle;
\draw(-3.031,.881)--(-2.915,1.02);
\draw(-2.904,1.022)--(-3.01,.894);
\filldraw[fill opacity=0.8,fill=gray!20,draw=none](-2.884,1.668)--(-2.849,1.397)--(-2.775,1.501)--(-2.8,1.696)--cycle;
\draw(-2.884,1.668)--(-2.849,1.397);
\draw(-2.775,1.501)--(-2.8,1.696);
\filldraw[fill opacity=0.8,fill=gray!20,draw=none](-2.986,1.648)--(-2.952,1.384)--(-2.877,1.611)--(-2.884,1.668)--cycle;
\draw(-2.986,1.648)--(-2.952,1.384);
\draw(-2.877,1.611)--(-2.884,1.668);
\filldraw[fill opacity=0.8,fill=gray!20,draw=none](-2.88,1.653)--(-3.241,1.651)--(-3.278,1.684)--(-3.265,1.715)--(-3.262,1.717)--(-2.874,1.802)--(-2.829,1.8)--(-2.681,1.765)--(-2.696,1.729)--(-2.759,1.691)--(-2.859,1.657)--cycle;
\draw(-2.681,1.765)--(-2.696,1.729)--(-2.759,1.691)--(-2.859,1.657)--(-2.88,1.653);
\filldraw[fill opacity=0.8,fill=gray!20,draw=none](-2.686,1.309)--(-2.684,1.361)--(-2.664,1.203)--(-2.678,1.215)--(-2.679,1.223)--cycle;
\draw(-2.684,1.361)--(-2.664,1.203)--(-2.678,1.215);
\filldraw[fill opacity=0.8,fill=gray!20,draw=none](-2.686,1.309)--(-2.679,1.223)--(-2.688,1.274)--cycle;
\filldraw[fill opacity=0.8,fill=gray!20](-2.883,.737)--(-2.86,.748)--(-2.915,.746)--(-2.912,.736)--cycle;
\filldraw[fill opacity=0.5,fill=gray!20](-1.312,1.846)--(-1.507,1.824)--(-1.447,2.325)--(-1.259,2.291)--cycle;
\filldraw[fill opacity=0.8,fill=gray!20,draw=none](-3.296,.822)--(-3.333,.808)--(-3.328,.862)--cycle;
\draw(-3.333,.808)--(-3.328,.862);
\filldraw[fill opacity=0.8,fill=gray!20,draw=none](-3.332,.816)--(-3.333,.808)--(-3.341,.821)--cycle;
\draw(-3.332,.816)--(-3.333,.808);
\filldraw[fill opacity=0.8,fill=gray!20,draw=none](-3.335,.839)--(-3.33,.842)--(-3.332,.816)--(-3.34,.82)--cycle;
\draw(-3.33,.842)--(-3.332,.816);
\filldraw[fill opacity=0.8,fill=gray!20,draw=none](-3.338,.826)--(-3.34,.82)--(-3.341,.821)--(-3.342,.822)--cycle;
\filldraw[fill opacity=0.8,fill=gray!20](-3.356,.788)--(-3.344,.821)--(-3.295,.83)--(-3.301,.798)--cycle;
\filldraw[fill opacity=0.8,fill=gray!20,draw=none](-3.296,.822)--(-3.292,.816)--(-3.297,.76)--(-3.337,.757)--(-3.333,.808)--cycle;
\draw(-3.292,.816)--(-3.297,.76)--(-3.337,.757)--(-3.333,.808);
\filldraw[fill opacity=0.8,fill=gray!20,draw=none](-2.919,2.022)--(-2.928,1.912)--(-2.935,1.909)--(-2.949,2.023)--cycle;
\draw(-2.935,1.909)--(-2.949,2.023);
\filldraw[fill opacity=0.5,fill=gray!20](-1.397,2.865)--(-1.509,2.919)--(-1.208,3.379)--(-1.105,3.311)--cycle;
\filldraw[fill opacity=0.8,fill=gray!20](-3.299,.619)--(-3.324,.645)--(-3.284,.653)--(-3.271,.624)--cycle;
\filldraw[fill opacity=0.8,fill=gray!20](-2.969,.75)--(-2.994,.776)--(-3.047,.789)--(-3.007,.759)--cycle;
\filldraw[fill opacity=0.8,fill=gray!20,draw=none](-3.029,.784)--(-3.029,.785)--(-3.051,.793)--(-3.047,.789)--cycle;
\draw(-3.051,.793)--(-3.047,.789)--(-3.029,.784);
\filldraw[fill opacity=0.8,fill=gray!20,draw=none](-3.029,.785)--(-3.023,.794)--(-3.051,.793)--cycle;
\filldraw[fill opacity=0.8,fill=gray!20,draw=none](-3.055,.779)--(-3.056,.792)--(-3.053,.792)--(-3.047,.785)--cycle;
\draw(-3.053,.792)--(-3.047,.785);
\filldraw[fill opacity=0.8,fill=gray!20,draw=none](-3.074,.764)--(-3.056,.792)--(-3.055,.779)--cycle;
\draw(-3.074,.764)--(-3.056,.792);
\filldraw[fill opacity=0.8,fill=gray!20,draw=none](-3.047,.785)--(-3.053,.792)--(-3.048,.791)--(-3.044,.787)--cycle;
\draw(-3.047,.785)--(-3.053,.792);
\filldraw[fill opacity=0.8,fill=gray!20,draw=none](-3.044,.787)--(-3.048,.791)--(-3.042,.79)--cycle;
\filldraw[fill opacity=0.8,fill=gray!20,draw=none](-3.095,.783)--(-2.986,.808)--(-2.901,.823)--(-3.041,.79)--cycle;
\draw(-3.095,.783)--(-2.986,.808);
\draw(-2.901,.823)--(-3.041,.79);
\filldraw[fill opacity=0.8,fill=gray!20,draw=none](-2.485,1.586)--(-2.481,1.59)--(-2.499,1.602)--cycle;
\draw(-2.485,1.586)--(-2.481,1.59);
\filldraw[fill opacity=0.8,fill=gray!20,draw=none](-2.687,1.286)--(-2.688,1.274)--(-2.689,1.282)--cycle;
\filldraw[fill opacity=0.8,fill=gray!20,draw=none](-2.334,1.676)--(-2.321,1.678)--(-2.333,1.686)--cycle;
\draw(-2.334,1.676)--(-2.321,1.678);
\filldraw[fill opacity=0.8,fill=gray!20,draw=none](-3.099,1.049)--(-3.093,1.059)--(-3.098,1.065)--(-3.105,1.049)--cycle;
\draw(-3.093,1.059)--(-3.098,1.065)--(-3.105,1.049);
\filldraw[fill opacity=0.8,fill=gray!20,draw=none](-2.928,1.912)--(-2.919,2.022)--(-2.857,2.021)--(-2.848,1.948)--cycle;
\draw(-2.857,2.021)--(-2.848,1.948);
\filldraw[fill opacity=0.8,fill=gray!20,draw=none](-2.378,1.744)--(-2.343,1.751)--(-2.36,1.784)--cycle;
\draw(-2.378,1.744)--(-2.343,1.751);
\filldraw[fill opacity=0.8,fill=gray!20,draw=none](-2.378,1.744)--(-2.36,1.784)--(-2.39,1.778)--(-2.38,1.744)--cycle;
\draw(-2.36,1.784)--(-2.39,1.778)--(-2.38,1.744)--(-2.378,1.744);
\filldraw[fill opacity=0.8,fill=gray!20](-2.443,1.775)--(-2.441,1.805)--(-2.477,1.807)--(-2.494,1.779)--cycle;
\filldraw[fill opacity=0.8,fill=gray!20,draw=none](-3.217,.832)--(-3.23,.833)--(-3.231,.838)--cycle;
\draw(-3.217,.832)--(-3.23,.833)--(-3.231,.838);
\filldraw[fill opacity=0.8,fill=gray!20,draw=none](-3.268,.841)--(-3.274,.857)--(-3.261,.861)--(-3.226,.838)--cycle;
\draw(-3.268,.841)--(-3.274,.857)--(-3.261,.861);
\filldraw[fill opacity=0.8,fill=gray!20,draw=none](-3.264,.856)--(-3.261,.869)--(-3.275,.861)--(-3.274,.857)--(-3.27,.846)--cycle;
\draw(-3.275,.861)--(-3.274,.857)--(-3.27,.846);
\filldraw[fill opacity=0.8,fill=gray!20,draw=none](-3.264,.856)--(-3.27,.846)--(-3.268,.841)--cycle;
\draw(-3.27,.846)--(-3.268,.841);
\filldraw[fill opacity=0.8,fill=gray!20](-3.295,.83)--(-3.284,.855)--(-3.232,.857)--(-3.23,.833)--cycle;
\filldraw[fill opacity=0.8,fill=gray!20,draw=none](-3.165,.875)--(-3.125,.924)--(-3.113,.895)--(-3.162,.836)--cycle;
\draw(-3.165,.875)--(-3.125,.924);
\draw(-3.113,.895)--(-3.162,.836);
\filldraw[fill opacity=0.8,fill=gray!20,draw=none](-3.147,.808)--(-3.147,.809)--(-3.104,.861)--(-3.088,.841)--(-3.124,.798)--cycle;
\draw(-3.147,.809)--(-3.104,.861);
\draw(-3.088,.841)--(-3.124,.798);
\filldraw[fill opacity=0.8,fill=gray!20,draw=none](-3.162,.835)--(-3.143,.853)--(-3.146,.818)--(-3.152,.816)--cycle;
\draw(-3.143,.853)--(-3.146,.818);
\filldraw[fill opacity=0.8,fill=gray!20,draw=none](-3.14,.794)--(-3.146,.795)--(-3.147,.796)--(-3.144,.796)--cycle;
\draw(-3.147,.796)--(-3.144,.796);
\filldraw[fill opacity=0.8,fill=gray!20,draw=none](-3.143,.79)--(-3.134,.742)--(-3.138,.74)--(-3.152,.744)--(-3.148,.792)--cycle;
\draw(-3.138,.74)--(-3.152,.744)--(-3.148,.792);
\filldraw[fill opacity=0.8,fill=gray!20,draw=none](-3.14,.796)--(-3.148,.797)--(-3.147,.808)--cycle;
\draw(-3.148,.797)--(-3.147,.808);
\filldraw[fill opacity=0.8,fill=gray!20,draw=none](-3.151,.804)--(-3.145,.797)--(-3.148,.797)--(-3.164,.812)--(-3.164,.814)--cycle;
\draw(-3.164,.812)--(-3.164,.814);
\filldraw[fill opacity=0.8,fill=gray!20,draw=none](-3.19,.802)--(-3.16,.805)--(-3.148,.795)--(-3.158,.793)--cycle;
\draw(-3.148,.795)--(-3.158,.793);
\filldraw[fill opacity=0.8,fill=gray!20,draw=none](-3.144,.793)--(-3.143,.79)--(-3.148,.792)--(-3.148,.795)--cycle;
\draw(-3.148,.792)--(-3.148,.795);
\filldraw[fill opacity=0.8,fill=gray!20,draw=none](-3.147,.796)--(-3.146,.795)--(-3.143,.792)--cycle;
\filldraw[fill opacity=0.8,fill=gray!20,draw=none](-3.146,.795)--(-3.148,.795)--(-3.147,.796)--cycle;
\draw(-3.148,.795)--(-3.147,.796);
\filldraw[fill opacity=0.8,fill=gray!20,draw=none](-3.142,.794)--(-3.141,.792)--(-3.143,.792)--(-3.146,.795)--cycle;
\draw(-3.141,.792)--(-3.143,.792);
\filldraw[fill opacity=0.8,fill=gray!20,draw=none](-3.121,.778)--(-3.136,.777)--(-3.143,.792)--(-3.135,.79)--(-3.12,.783)--cycle;
\draw(-3.143,.792)--(-3.135,.79);
\filldraw[fill opacity=0.8,fill=gray!20,draw=none](-3.142,.794)--(-3.158,.793)--(-3.148,.795)--(-3.146,.795)--cycle;
\draw(-3.158,.793)--(-3.148,.795);
\filldraw[fill opacity=0.8,fill=gray!20,draw=none](-3.145,.797)--(-3.142,.794)--(-3.146,.795)--(-3.148,.797)--cycle;
\filldraw[fill opacity=0.8,fill=gray!20,draw=none](-3.142,.794)--(-3.146,.795)--(-3.14,.794)--cycle;
\filldraw[fill opacity=0.8,fill=gray!20,draw=none](-3.151,.804)--(-3.164,.814)--(-3.167,.822)--cycle;
\draw(-3.164,.814)--(-3.167,.822);
\filldraw[fill opacity=0.8,fill=gray!20,draw=none](-3.236,.747)--(-3.171,.825)--(-3.167,.822)--(-3.151,.804)--(-3.201,.744)--cycle;
\draw(-3.236,.747)--(-3.171,.825);
\draw(-3.151,.804)--(-3.201,.744);
\filldraw[fill opacity=0.8,fill=gray!20,draw=none](-3.268,.841)--(-3.226,.838)--(-3.204,.823)--(-3.257,.811)--cycle;
\draw(-3.204,.823)--(-3.257,.811)--(-3.268,.841);
\filldraw[fill opacity=0.8,fill=gray!20,draw=none](-3.284,.908)--(-3.287,.872)--(-3.328,.866)--cycle;
\draw(-3.284,.908)--(-3.287,.872);
\filldraw[fill opacity=0.8,fill=gray!20,draw=none](-3.261,.869)--(-3.254,.894)--(-3.284,.908)--(-3.275,.861)--cycle;
\draw(-3.284,.908)--(-3.275,.861);
\filldraw[fill opacity=0.8,fill=gray!20,draw=none](-3.186,.83)--(-3.205,.831)--(-3.206,.856)--(-3.204,.855)--cycle;
\draw(-3.186,.83)--(-3.205,.831);
\draw(-3.206,.856)--(-3.204,.855);
\filldraw[fill opacity=0.8,fill=gray!20,draw=none](-3.165,.875)--(-3.166,.897)--(-3.16,.896)--cycle;
\draw(-3.166,.897)--(-3.16,.896);
\filldraw[fill opacity=0.8,fill=gray!20,draw=none](-3.19,.846)--(-3.197,.871)--(-3.177,.895)--(-3.16,.896)--(-3.165,.875)--(-3.19,.846)--cycle;
\draw(-3.197,.871)--(-3.177,.895);
\draw(-3.165,.875)--(-3.19,.846);
\filldraw[fill opacity=0.8,fill=gray!20,draw=none](-3.194,.841)--(-3.19,.845)--(-3.175,.829)--cycle;
\draw(-3.194,.841)--(-3.19,.845);
\filldraw[fill opacity=0.8,fill=gray!20,draw=none](-3.19,.846)--(-3.19,.846)--(-3.19,.845)--cycle;
\draw(-3.19,.846)--(-3.19,.845);
\filldraw[fill opacity=0.8,fill=gray!20,draw=none](-3.205,.831)--(-3.217,.832)--(-3.231,.838)--(-3.232,.857)--(-3.206,.856)--cycle;
\draw(-3.205,.831)--(-3.217,.832);
\draw(-3.231,.838)--(-3.232,.857)--(-3.206,.856);
\filldraw[fill opacity=0.8,fill=gray!20,draw=none](-3.242,.783)--(-3.229,.798)--(-3.19,.802)--(-3.215,.772)--cycle;
\draw(-3.242,.783)--(-3.229,.798);
\draw(-3.19,.802)--(-3.215,.772);
\filldraw[fill opacity=0.8,fill=gray!20,draw=none](-3.121,.784)--(-3.137,.791)--(-3.14,.794)--(-3.124,.79)--cycle;
\filldraw[fill opacity=0.8,fill=gray!20,draw=none](-3.06,.795)--(-3.055,.793)--(-3.056,.792)--cycle;
\draw(-3.06,.795)--(-3.055,.793)--(-3.056,.792);
\filldraw[fill opacity=0.8,fill=gray!20,draw=none](-3.126,.758)--(-3.136,.777)--(-3.121,.778)--cycle;
\filldraw[fill opacity=0.8,fill=gray!20,draw=none](-3.201,.744)--(-3.16,.794)--(-3.143,.775)--(-3.172,.74)--cycle;
\draw(-3.201,.744)--(-3.16,.794);
\draw(-3.143,.775)--(-3.172,.74);
\filldraw[fill opacity=0.8,fill=gray!20,draw=none](-3.116,.75)--(-3.124,.752)--(-3.126,.758)--(-3.121,.778)--(-3.111,.779)--(-3.108,.757)--cycle;
\draw(-3.116,.75)--(-3.124,.752);
\draw(-3.111,.779)--(-3.108,.757);
\filldraw[fill opacity=0.8,fill=gray!20,draw=none](-3.141,.776)--(-3.143,.79)--(-3.134,.786)--(-3.129,.777)--cycle;
\filldraw[fill opacity=0.8,fill=gray!20,draw=none](-3.143,.79)--(-3.144,.793)--(-3.136,.789)--(-3.134,.786)--cycle;
\filldraw[fill opacity=0.8,fill=gray!20,draw=none](-3.159,.793)--(-3.132,.788)--(-3.143,.775)--cycle;
\draw(-3.132,.788)--(-3.143,.775);
\filldraw[fill opacity=0.8,fill=gray!20,draw=none](-3.141,.776)--(-3.129,.777)--(-3.119,.758)--(-3.119,.754)--(-3.134,.742)--cycle;
\draw(-3.119,.758)--(-3.119,.754);
\filldraw[fill opacity=0.8,fill=gray!20,draw=none](-3.24,.747)--(-3.245,.778)--(-3.242,.783)--(-3.215,.772)--(-3.236,.747)--cycle;
\draw(-3.245,.778)--(-3.242,.783);
\draw(-3.215,.772)--(-3.236,.747);
\filldraw[fill opacity=0.8,fill=gray!20,draw=none](-3.121,.778)--(-3.12,.783)--(-3.111,.779)--cycle;
\filldraw[fill opacity=0.8,fill=gray!20,draw=none](-3.134,.786)--(-3.132,.788)--(-3.121,.78)--cycle;
\draw(-3.134,.786)--(-3.132,.788);
\filldraw[fill opacity=0.8,fill=gray!20,draw=none](-3.134,.786)--(-3.136,.789)--(-3.117,.778)--cycle;
\filldraw[fill opacity=0.8,fill=gray!20,draw=none](-3.129,.777)--(-3.134,.786)--(-3.117,.778)--cycle;
\filldraw[fill opacity=0.8,fill=gray!20,draw=none](-3.144,.774)--(-3.134,.786)--(-3.121,.78)--(-3.117,.778)--cycle;
\draw(-3.144,.774)--(-3.134,.786);
\filldraw[fill opacity=0.8,fill=gray!20,draw=none](-3.127,.773)--(-3.129,.777)--(-3.117,.778)--cycle;
\filldraw[fill opacity=0.8,fill=gray!20,draw=none](-3.127,.773)--(-3.117,.778)--(-3.119,.758)--cycle;
\draw(-3.117,.778)--(-3.119,.758);
\filldraw[fill opacity=0.8,fill=gray!20,draw=none](-3.165,.737)--(-3.172,.74)--(-3.144,.774)--(-3.117,.778)--(-3.141,.749)--cycle;
\draw(-3.172,.74)--(-3.144,.774);
\draw(-3.117,.778)--(-3.141,.749);
\filldraw[fill opacity=0.8,fill=gray!20,draw=none](-3.235,.775)--(-3.159,.793)--(-3.136,.789)--(-3.117,.778)--(-3.212,.756)--cycle;
\draw(-3.117,.778)--(-3.212,.756)--(-3.235,.775)--(-3.159,.793);
\filldraw[fill opacity=0.8,fill=gray!20,draw=none](-3.136,.789)--(-3.144,.793)--(-3.142,.794)--(-3.127,.787)--cycle;
\filldraw[fill opacity=0.8,fill=gray!20,draw=none](-3.111,.78)--(-3.121,.784)--(-3.124,.79)--(-3.115,.788)--cycle;
\filldraw[fill opacity=0.8,fill=gray!20,draw=none](-3.111,.78)--(-3.115,.788)--(-3.097,.783)--(-3.097,.782)--(-3.109,.779)--cycle;
\draw(-3.097,.782)--(-3.109,.779);
\filldraw[fill opacity=0.8,fill=gray!20,draw=none](-3.119,.787)--(-3.123,.799)--(-3.088,.841)--(-3.081,.832)--(-3.077,.826)--(-3.111,.785)--cycle;
\draw(-3.123,.799)--(-3.088,.841);
\draw(-3.077,.826)--(-3.111,.785);
\filldraw[fill opacity=0.8,fill=gray!20,draw=none](-3.023,.794)--(-3.013,.812)--(-3.013,.814)--(-3.023,.816)--(-3.054,.797)--(-3.051,.793)--cycle;
\draw(-3.013,.814)--(-3.023,.816);
\draw(-3.054,.797)--(-3.051,.793);
\filldraw[fill opacity=0.8,fill=gray!20,draw=none](-3.151,.804)--(-3.148,.808)--(-3.124,.798)--(-3.131,.79)--cycle;
\draw(-3.151,.804)--(-3.148,.808);
\draw(-3.124,.798)--(-3.131,.79);
\filldraw[fill opacity=0.8,fill=gray!20,draw=none](-3.131,.79)--(-3.123,.799)--(-3.119,.787)--cycle;
\draw(-3.131,.79)--(-3.123,.799);
\filldraw[fill opacity=0.8,fill=gray!20,draw=none](-3.145,.797)--(-3.144,.793)--(-3.148,.795)--(-3.148,.797)--cycle;
\draw(-3.148,.795)--(-3.148,.797);
\filldraw[fill opacity=0.8,fill=gray!20,draw=none](-3.14,.796)--(-3.145,.797)--(-3.151,.804)--cycle;
\filldraw[fill opacity=0.8,fill=gray!20,draw=none](-3.137,.791)--(-3.142,.794)--(-3.14,.794)--cycle;
\filldraw[fill opacity=0.8,fill=gray!20,draw=none](-3.14,.796)--(-3.13,.789)--(-3.141,.792)--(-3.145,.797)--cycle;
\draw(-3.13,.789)--(-3.141,.792);
\filldraw[fill opacity=0.8,fill=gray!20,draw=none](-3.136,.789)--(-3.144,.793)--(-3.145,.797)--(-3.14,.796)--cycle;
\filldraw[fill opacity=0.8,fill=gray!20,draw=none](-3.136,.789)--(-3.159,.793)--(-3.158,.793)--(-3.144,.793)--cycle;
\draw(-3.159,.793)--(-3.158,.793);
\filldraw[fill opacity=0.8,fill=gray!20,draw=none](-3.12,.783)--(-3.135,.79)--(-3.119,.787)--cycle;
\draw(-3.135,.79)--(-3.119,.787);
\filldraw[fill opacity=0.8,fill=gray!20,draw=none](-3.159,.793)--(-3.16,.794)--(-3.151,.804)--(-3.131,.79)--(-3.132,.788)--cycle;
\draw(-3.16,.794)--(-3.151,.804);
\draw(-3.131,.79)--(-3.132,.788);
\filldraw[fill opacity=0.8,fill=gray!20,draw=none](-3.119,.787)--(-3.13,.789)--(-3.14,.796)--(-3.115,.795)--cycle;
\draw(-3.119,.787)--(-3.13,.789);
\filldraw[fill opacity=0.8,fill=gray!20,draw=none](-3.136,.789)--(-3.127,.787)--(-3.121,.784)--(-3.117,.778)--cycle;
\filldraw[fill opacity=0.8,fill=gray!20,draw=none](-3.121,.784)--(-3.117,.783)--(-3.117,.778)--cycle;
\filldraw[fill opacity=0.8,fill=gray!20,draw=none](-3.117,.782)--(-3.12,.783)--(-3.119,.787)--(-3.111,.785)--cycle;
\draw(-3.119,.787)--(-3.111,.785);
\filldraw[fill opacity=0.8,fill=gray!20,draw=none](-3.132,.788)--(-3.131,.79)--(-3.119,.787)--(-3.117,.778)--(-3.117,.778)--cycle;
\draw(-3.132,.788)--(-3.131,.79);
\draw(-3.117,.778)--(-3.117,.778);
\filldraw[fill opacity=0.8,fill=gray!20,draw=none](-3.119,.787)--(-3.115,.795)--(-3.111,.785)--cycle;
\draw(-3.115,.795)--(-3.111,.785)--(-3.119,.787);
\filldraw[fill opacity=0.8,fill=gray!20,draw=none](-3.117,.783)--(-3.111,.78)--(-3.111,.779)--(-3.117,.778)--cycle;
\draw(-3.111,.779)--(-3.117,.778);
\filldraw[fill opacity=0.8,fill=gray!20,draw=none](-3.117,.782)--(-3.111,.785)--(-3.111,.779)--cycle;
\draw(-3.111,.785)--(-3.111,.779);
\filldraw[fill opacity=0.8,fill=gray!20,draw=none](-3.119,.787)--(-3.111,.785)--(-3.117,.778)--cycle;
\draw(-3.111,.785)--(-3.117,.778);
\filldraw[fill opacity=0.8,fill=gray!20,draw=none](-3.136,.789)--(-3.14,.796)--(-3.115,.795)--(-3.117,.778)--cycle;
\draw(-3.115,.795)--(-3.117,.778);
\filldraw[fill opacity=0.8,fill=gray!20,draw=none](-3.056,.792)--(-3.055,.793)--(-3.053,.792)--cycle;
\draw(-3.056,.792)--(-3.055,.793)--(-3.053,.792);
\filldraw[fill opacity=0.8,fill=gray!20,draw=none](-3.053,.792)--(-3.055,.793)--(-3.054,.797)--(-3.048,.791)--cycle;
\draw(-3.053,.792)--(-3.055,.793)--(-3.054,.797);
\filldraw[fill opacity=0.8,fill=gray!20,draw=none](-3.016,.824)--(-3.049,.792)--(-3.054,.797)--(-3.035,.847)--(-3.015,.825)--cycle;
\draw(-3.054,.797)--(-3.035,.847)--(-3.015,.825);
\filldraw[fill opacity=0.8,fill=gray!20,draw=none](-3.023,.816)--(-3.065,.826)--(-3.069,.817)--(-3.054,.797)--cycle;
\draw(-3.023,.816)--(-3.065,.826);
\draw(-3.069,.817)--(-3.054,.797);
\filldraw[fill opacity=0.8,fill=gray!20,draw=none](-3.102,.777)--(-3.095,.783)--(-3.041,.79)--(-3.099,.777)--cycle;
\draw(-3.041,.79)--(-3.099,.777);
\filldraw[fill opacity=0.8,fill=gray!20,draw=none](-3.102,.777)--(-3.104,.777)--(-3.106,.778)--(-3.099,.782)--(-3.095,.783)--cycle;
\draw(-3.099,.782)--(-3.095,.783);
\filldraw[fill opacity=0.8,fill=gray!20,draw=none](-3.106,.778)--(-3.109,.779)--(-3.099,.782)--cycle;
\draw(-3.109,.779)--(-3.099,.782);
\filldraw[fill opacity=0.8,fill=gray!20,draw=none](-3.111,.78)--(-3.109,.779)--(-3.111,.779)--cycle;
\draw(-3.109,.779)--(-3.111,.779);
\filldraw[fill opacity=0.8,fill=gray!20,draw=none](-3.108,.781)--(-3.106,.791)--(-3.076,.827)--(-3.071,.821)--(-3.106,.779)--cycle;
\draw(-3.106,.791)--(-3.076,.827);
\draw(-3.071,.821)--(-3.106,.779);
\filldraw[fill opacity=0.8,fill=gray!20,draw=none](-3.108,.757)--(-3.11,.769)--(-3.106,.778)--(-3.104,.777)--(-3.102,.774)--cycle;
\draw(-3.108,.757)--(-3.11,.769);
\draw(-3.104,.777)--(-3.102,.774);
\filldraw[fill opacity=0.8,fill=gray!20,draw=none](-3.106,.778)--(-3.11,.769)--(-3.111,.779)--cycle;
\draw(-3.11,.769)--(-3.111,.779);
\filldraw[fill opacity=0.8,fill=gray!20,draw=none](-3.015,.825)--(-3.035,.847)--(-3.034,.86)--(-3.025,.875)--(-3.009,.859)--(-3.011,.838)--cycle;
\draw(-3.015,.825)--(-3.035,.847)--(-3.034,.86);
\draw(-3.009,.859)--(-3.011,.838);
\filldraw[fill opacity=0.8,fill=gray!20,draw=none](-3.062,.831)--(-3.065,.826)--(-3.019,.815)--(-3.016,.822)--(-3.022,.846)--cycle;
\draw(-3.065,.826)--(-3.019,.815);
\draw(-3.016,.822)--(-3.022,.846);
\filldraw[fill opacity=0.8,fill=gray!20,draw=none](-3.029,.803)--(-3.042,.79)--(-3.048,.791)--(-3.049,.792)--(-3.016,.824)--cycle;
\filldraw[fill opacity=0.8,fill=gray!20,draw=none](-3.104,.861)--(-3.1,.867)--(-3.087,.842)--(-3.088,.841)--cycle;
\draw(-3.104,.861)--(-3.1,.867);
\draw(-3.087,.842)--(-3.088,.841);
\filldraw[fill opacity=0.8,fill=gray!20,draw=none](-3.034,.86)--(-3.031,.881)--(-3.025,.875)--cycle;
\draw(-3.034,.86)--(-3.031,.881);
\filldraw[fill opacity=0.8,fill=gray!20,draw=none](-3.065,.826)--(-3.071,.828)--(-3.076,.827)--(-3.069,.817)--cycle;
\draw(-3.065,.826)--(-3.071,.828);
\draw(-3.076,.827)--(-3.069,.817);
\filldraw[fill opacity=0.8,fill=gray!20,draw=none](-3.062,.831)--(-3.071,.828)--(-3.065,.826)--cycle;
\draw(-3.071,.828)--(-3.065,.826);
\filldraw[fill opacity=0.8,fill=gray!20,draw=none](-3.076,.827)--(-3.031,.881)--(-3.01,.894)--(-3.071,.821)--cycle;
\draw(-3.076,.827)--(-3.031,.881);
\draw(-3.01,.894)--(-3.071,.821);
\filldraw[fill opacity=0.8,fill=gray!20,draw=none](-3.106,.778)--(-3.106,.778)--(-3.111,.779)--(-3.109,.779)--cycle;
\draw(-3.111,.779)--(-3.109,.779);
\filldraw[fill opacity=0.8,fill=gray!20,draw=none](-3.108,.777)--(-3.103,.776)--(-3.119,.754)--(-3.119,.758)--cycle;
\draw(-3.119,.754)--(-3.119,.758);
\filldraw[fill opacity=0.8,fill=gray!20,draw=none](-3.071,.828)--(-3.062,.831)--(-3.053,.846)--(-3.052,.868)--(-3.098,.879)--(-3.079,.83)--cycle;
\draw(-3.052,.868)--(-3.098,.879)--(-3.079,.83)--(-3.071,.828);
\filldraw[fill opacity=0.8,fill=gray!20,draw=none](-3.109,.775)--(-3.119,.758)--(-3.117,.778)--cycle;
\draw(-3.119,.758)--(-3.117,.778);
\filldraw[fill opacity=0.8,fill=gray!20,draw=none](-3.106,.778)--(-3.107,.777)--(-3.117,.778)--(-3.111,.779)--cycle;
\draw(-3.117,.778)--(-3.111,.779);
\filldraw[fill opacity=0.8,fill=gray!20,draw=none](-3.106,.778)--(-3.111,.779)--(-3.111,.785)--(-3.106,.779)--cycle;
\draw(-3.111,.779)--(-3.111,.785)--(-3.106,.779);
\filldraw[fill opacity=0.8,fill=gray!20,draw=none](-3.111,.785)--(-3.106,.791)--(-3.108,.781)--cycle;
\draw(-3.111,.785)--(-3.106,.791);
\filldraw[fill opacity=0.8,fill=gray!20,draw=none](-3.109,.782)--(-3.111,.785)--(-3.115,.795)--cycle;
\draw(-3.109,.782)--(-3.111,.785)--(-3.115,.795);
\filldraw[fill opacity=0.8,fill=gray!20,draw=none](-3.141,.749)--(-3.111,.785)--(-3.108,.781)--(-3.11,.774)--(-3.118,.764)--cycle;
\draw(-3.141,.749)--(-3.111,.785);
\draw(-3.11,.774)--(-3.118,.764);
\filldraw[fill opacity=0.8,fill=gray!20,draw=none](-3.109,.777)--(-3.107,.777)--(-3.11,.774)--cycle;
\draw(-3.107,.777)--(-3.11,.774);
\filldraw[fill opacity=0.8,fill=gray!20,draw=none](-3.108,.777)--(-3.109,.775)--(-3.117,.778)--cycle;
\filldraw[fill opacity=0.8,fill=gray!20,draw=none](-3.108,.882)--(-3.115,.795)--(-3.14,.796)--(-3.147,.808)--(-3.139,.892)--cycle;
\draw(-3.147,.808)--(-3.139,.892)--(-3.108,.882)--(-3.115,.795);
\filldraw[fill opacity=0.8,fill=gray!20,draw=none](-3.094,.873)--(-3.101,.788)--(-3.105,.781)--(-3.115,.795)--(-3.108,.882)--cycle;
\draw(-3.115,.795)--(-3.108,.882)--(-3.094,.873)--(-3.101,.788);
\filldraw[fill opacity=0.8,fill=gray!20,draw=none](-3.105,.784)--(-3.106,.779)--(-3.109,.782)--(-3.115,.795)--(-3.125,.818)--(-3.111,.804)--(-3.109,.797)--cycle;
\draw(-3.106,.779)--(-3.109,.782);
\draw(-3.115,.795)--(-3.125,.818)--(-3.111,.804)--(-3.109,.797);
\filldraw[fill opacity=0.8,fill=gray!20,draw=none](-3.109,.777)--(-3.108,.781)--(-3.106,.779)--(-3.107,.777)--cycle;
\draw(-3.106,.779)--(-3.107,.777);
\filldraw[fill opacity=0.8,fill=gray!20,draw=none](-3.105,.781)--(-3.108,.777)--(-3.117,.778)--(-3.115,.795)--cycle;
\draw(-3.117,.778)--(-3.115,.795);
\filldraw[fill opacity=0.8,fill=gray!20,draw=none](-3.212,.756)--(-3.117,.778)--(-3.107,.777)--(-3.114,.773)--(-3.19,.755)--cycle;
\draw(-3.114,.773)--(-3.19,.755)--(-3.212,.756)--(-3.117,.778);
\filldraw[fill opacity=0.8,fill=gray!20,draw=none](-3.088,.841)--(-3.087,.842)--(-3.081,.832)--cycle;
\draw(-3.088,.841)--(-3.087,.842);
\filldraw[fill opacity=0.8,fill=gray!20,draw=none](-3.025,.875)--(-3.031,.881)--(-3.029,.903)--(-3.017,.891)--cycle;
\draw(-3.031,.881)--(-3.029,.903)--(-3.017,.891);
\filldraw[fill opacity=0.8,fill=gray!20,draw=none](-3.099,.787)--(-3.072,.819)--(-3.059,.84)--(-3.102,.788)--cycle;
\draw(-3.099,.787)--(-3.072,.819);
\draw(-3.059,.84)--(-3.102,.788);
\filldraw[fill opacity=0.8,fill=gray!20,draw=none](-3.106,.778)--(-3.104,.777)--(-3.106,.778)--cycle;
\filldraw[fill opacity=0.8,fill=gray!20,draw=none](-3.106,.778)--(-3.106,.779)--(-3.104,.777)--cycle;
\draw(-3.106,.779)--(-3.104,.777);
\filldraw[fill opacity=0.8,fill=gray!20,draw=none](-3.106,.778)--(-3.104,.777)--(-3.107,.777)--cycle;
\filldraw[fill opacity=0.8,fill=gray!20,draw=none](-3.118,.764)--(-3.099,.787)--(-3.102,.788)--(-3.114,.774)--cycle;
\draw(-3.118,.764)--(-3.099,.787);
\draw(-3.102,.788)--(-3.114,.774);
\filldraw[fill opacity=0.8,fill=gray!20,draw=none](-3.167,.822)--(-3.152,.816)--(-3.148,.808)--(-3.151,.804)--cycle;
\draw(-3.148,.808)--(-3.151,.804);
\filldraw[fill opacity=0.8,fill=gray!20,draw=none](-3.14,.796)--(-3.151,.804)--(-3.167,.822)--(-3.168,.828)--(-3.125,.818)--(-3.115,.795)--cycle;
\draw(-3.167,.822)--(-3.168,.828)--(-3.125,.818)--(-3.115,.795);
\filldraw[fill opacity=0.8,fill=gray!20,draw=none](-3.102,.777)--(-3.102,.776)--(-3.104,.777)--(-3.106,.779)--(-3.105,.781)--cycle;
\draw(-3.104,.777)--(-3.106,.779);
\filldraw[fill opacity=0.8,fill=gray!20,draw=none](-3.107,.777)--(-3.104,.777)--(-3.102,.776)--(-3.103,.776)--(-3.114,.773)--cycle;
\draw(-3.103,.776)--(-3.114,.773);
\filldraw[fill opacity=0.8,fill=gray!20,draw=none](-3.102,.776)--(-3.102,.774)--(-3.104,.777)--cycle;
\draw(-3.102,.774)--(-3.104,.777);
\filldraw[fill opacity=0.8,fill=gray!20,draw=none](-3.108,.777)--(-3.101,.788)--(-3.103,.776)--cycle;
\draw(-3.101,.788)--(-3.103,.776);
\filldraw[fill opacity=0.8,fill=gray!20,draw=none](-3.081,.832)--(-3.076,.827)--(-3.077,.826)--cycle;
\draw(-3.076,.827)--(-3.077,.826);
\filldraw[fill opacity=0.8,fill=gray!20,draw=none](-3.01,.894)--(-2.904,1.022)--(-2.895,1.037)--(-2.99,.923)--cycle;
\draw(-3.01,.894)--(-2.904,1.022);
\draw(-2.895,1.037)--(-2.99,.923);
\filldraw[fill opacity=0.8,fill=gray!20,draw=none](-3.021,.881)--(-3.01,.894)--(-2.99,.923)--(-3.018,.889)--cycle;
\draw(-3.021,.881)--(-3.01,.894);
\draw(-2.99,.923)--(-3.018,.889);
\filldraw[fill opacity=0.8,fill=gray!20,draw=none](-3.017,.911)--(-3.018,.892)--(-3.029,.903)--(-3.035,.958)--(-3.02,.942)--cycle;
\draw(-3.018,.892)--(-3.029,.903)--(-3.035,.958)--(-3.02,.942);
\filldraw[fill opacity=0.8,fill=gray!20,draw=none](-3.071,.828)--(-3.079,.83)--(-3.076,.827)--cycle;
\draw(-3.071,.828)--(-3.079,.83)--(-3.076,.827);
\filldraw[fill opacity=0.8,fill=gray!20,draw=none](-3.104,.777)--(-3.102,.777)--(-3.102,.776)--cycle;
\filldraw[fill opacity=0.8,fill=gray!20,draw=none](-3.104,.779)--(-3.105,.781)--(-3.105,.784)--cycle;
\filldraw[fill opacity=0.8,fill=gray!20,draw=none](-3.072,.819)--(-3.025,.875)--(-3.018,.888)--(-3.059,.84)--cycle;
\draw(-3.072,.819)--(-3.025,.875);
\draw(-3.018,.888)--(-3.059,.84);
\filldraw[fill opacity=0.8,fill=gray!20,draw=none](-3.114,.774)--(-3.102,.788)--(-3.102,.794)--(-3.111,.798)--(-3.13,.776)--cycle;
\draw(-3.114,.774)--(-3.102,.788);
\draw(-3.111,.798)--(-3.13,.776);
\filldraw[fill opacity=0.8,fill=gray!20,draw=none](-3.102,.777)--(-3.104,.779)--(-3.105,.784)--(-3.105,.788)--(-3.104,.786)--cycle;
\draw(-3.105,.788)--(-3.104,.786);
\filldraw[fill opacity=0.8,fill=gray!20,draw=none](-3.102,.777)--(-3.102,.777)--(-3.102,.776)--(-3.102,.776)--cycle;
\filldraw[fill opacity=0.8,fill=gray!20,draw=none](-3.104,.786)--(-3.105,.783)--(-3.103,.776)--(-3.102,.777)--cycle;
\draw(-3.103,.776)--(-3.102,.777);
\filldraw[fill opacity=0.8,fill=gray!20,draw=none](-3.102,.777)--(-3.102,.777)--(-3.104,.786)--(-3.102,.782)--cycle;
\draw(-3.104,.786)--(-3.102,.782);
\filldraw[fill opacity=0.8,fill=gray!20,draw=none](-3.102,.794)--(-3.104,.786)--(-3.102,.777)--(-3.102,.786)--cycle;
\draw(-3.102,.777)--(-3.102,.786);
\filldraw[fill opacity=0.8,fill=gray!20,draw=none](-3.025,.875)--(-3.021,.881)--(-3.018,.889)--(-3.018,.888)--cycle;
\draw(-3.025,.875)--(-3.021,.881);
\draw(-3.018,.889)--(-3.018,.888);
\filldraw[fill opacity=0.8,fill=gray!20,draw=none](-3.025,.875)--(-3.017,.891)--(-3.006,.879)--(-3.009,.859)--cycle;
\draw(-3.017,.891)--(-3.006,.879)--(-3.009,.859);
\filldraw[fill opacity=0.8,fill=gray!20,draw=none](-3.011,.819)--(-3.012,.86)--(-3.025,.861)--(-3.013,.814)--cycle;
\draw(-3.012,.86)--(-3.025,.861)--(-3.013,.814);
\filldraw[fill opacity=0.8,fill=gray!20,draw=none](-3.022,.81)--(-3.029,.803)--(-3.016,.824)--(-3.014,.825)--(-3.014,.824)--cycle;
\draw(-3.014,.825)--(-3.014,.824);
\filldraw[fill opacity=0.8,fill=gray!20,draw=none](-3.102,.794)--(-3.102,.786)--(-3.101,.798)--cycle;
\draw(-3.102,.786)--(-3.101,.798);
\filldraw[fill opacity=0.8,fill=gray!20,draw=none](-3.102,.794)--(-3.102,.788)--(-3.098,.792)--cycle;
\draw(-3.102,.788)--(-3.098,.792);
\filldraw[fill opacity=0.8,fill=gray!20,draw=none](-3.105,.784)--(-3.109,.797)--(-3.105,.788)--cycle;
\draw(-3.109,.797)--(-3.105,.788);
\filldraw[fill opacity=0.8,fill=gray!20](-3.125,.818)--(-3.145,.845)--(-3.135,.834)--(-3.111,.804)--cycle;
\filldraw[fill opacity=0.8,fill=gray!20,draw=none](-3.107,.792)--(-3.111,.804)--(-3.114,.802)--cycle;
\draw(-3.107,.792)--(-3.111,.804)--(-3.114,.802);
\filldraw[fill opacity=0.8,fill=gray!20,draw=none](-3.102,.794)--(-3.102,.809)--(-3.111,.798)--cycle;
\draw(-3.102,.809)--(-3.111,.798);
\filldraw[fill opacity=0.8,fill=gray!20,draw=none](-3.102,.809)--(-3.103,.788)--(-3.102,.794)--cycle;
\filldraw[fill opacity=0.8,fill=gray!20,draw=none](-3.052,.869)--(-3.017,.911)--(-3.02,.942)--(-3.052,.903)--cycle;
\draw(-3.052,.869)--(-3.017,.911);
\draw(-3.02,.942)--(-3.052,.903);
\filldraw[fill opacity=0.8,fill=gray!20,draw=none](-3.052,.868)--(-3.048,.885)--(-3.057,.922)--(-3.105,.934)--(-3.098,.879)--cycle;
\draw(-3.057,.922)--(-3.105,.934)--(-3.098,.879)--(-3.052,.868);
\filldraw[fill opacity=0.8,fill=gray!20,draw=none](-3.13,.776)--(-3.11,.799)--(-3.128,.812)--(-3.163,.77)--cycle;
\draw(-3.13,.776)--(-3.11,.799);
\draw(-3.128,.812)--(-3.163,.77);
\filldraw[fill opacity=0.8,fill=gray!20,draw=none](-3.12,.787)--(-3.105,.783)--(-3.104,.786)--(-3.107,.792)--(-3.114,.802)--(-3.128,.793)--cycle;
\draw(-3.104,.786)--(-3.107,.792);
\draw(-3.114,.802)--(-3.128,.793);
\filldraw[fill opacity=0.8,fill=gray!20,draw=none](-3.103,.794)--(-3.106,.786)--(-3.105,.783)--(-3.103,.788)--cycle;
\filldraw[fill opacity=0.8,fill=gray!20,draw=none](-3.105,.783)--(-3.102,.782)--(-3.104,.786)--cycle;
\draw(-3.102,.782)--(-3.104,.786);
\filldraw[fill opacity=0.8,fill=gray!20,draw=none](-3.16,.896)--(-3.147,.897)--(-3.165,.875)--cycle;
\draw(-3.147,.897)--(-3.165,.875);
\filldraw[fill opacity=0.8,fill=gray!20,draw=none](-3.165,.875)--(-3.16,.896)--(-3.139,.892)--(-3.143,.853)--(-3.162,.835)--cycle;
\draw(-3.16,.896)--(-3.139,.892)--(-3.143,.853);
\filldraw[fill opacity=0.8,fill=gray!20](-3.168,.828)--(-3.181,.854)--(-3.145,.845)--(-3.125,.818)--cycle;
\filldraw[fill opacity=0.8,fill=gray!20,draw=none](-3.117,.8)--(-3.111,.804)--(-3.135,.834)--(-3.142,.829)--cycle;
\draw(-3.117,.8)--(-3.111,.804)--(-3.135,.834)--(-3.142,.829);
\filldraw[fill opacity=0.8,fill=gray!20,draw=none](-3.11,.799)--(-3.102,.809)--(-3.099,.847)--(-3.128,.812)--cycle;
\draw(-3.11,.799)--(-3.102,.809);
\draw(-3.099,.847)--(-3.128,.812);
\filldraw[fill opacity=0.8,fill=gray!20,draw=none](-3.105,.786)--(-3.103,.794)--(-3.102,.809)--(-3.102,.82)--cycle;
\filldraw[fill opacity=0.8,fill=gray!20,draw=none](-3.246,.792)--(-3.229,.798)--(-3.245,.778)--cycle;
\draw(-3.229,.798)--(-3.245,.778);
\filldraw[fill opacity=0.8,fill=gray!20,draw=none](-3.246,.792)--(-3.207,.806)--(-3.158,.793)--(-3.235,.775)--cycle;
\draw(-3.158,.793)--(-3.235,.775)--(-3.246,.792);
\filldraw[fill opacity=0.8,fill=gray!20](-3.181,.854)--(-3.198,.871)--(-3.172,.865)--(-3.145,.845)--cycle;
\filldraw[fill opacity=0.8,fill=gray!20](-3.145,.845)--(-3.172,.865)--(-3.165,.857)--(-3.135,.834)--cycle;
\filldraw[fill opacity=0.8,fill=gray!20,draw=none](-3.105,.786)--(-3.102,.82)--(-3.102,.843)--(-3.107,.79)--(-3.106,.786)--cycle;
\draw(-3.102,.843)--(-3.107,.79);
\filldraw[fill opacity=0.8,fill=gray!20,draw=none](-3.182,.765)--(-3.163,.77)--(-3.124,.817)--(-3.147,.832)--(-3.186,.785)--cycle;
\draw(-3.163,.77)--(-3.124,.817);
\draw(-3.147,.832)--(-3.186,.785);
\filldraw[fill opacity=0.8,fill=gray!20](-3.19,.755)--(-2.874,.829)--(-2.858,.848)--(-3.174,.774)--cycle;
\filldraw[fill opacity=0.8,fill=gray!20,draw=none](-2.673,1.401)--(-2.505,1.604)--(-2.523,1.627)--(-2.687,1.43)--cycle;
\draw(-2.673,1.401)--(-2.505,1.604);
\draw(-2.523,1.627)--(-2.687,1.43);
\filldraw[fill opacity=0.8,fill=gray!20](-3.198,.623)--(-3.181,.652)--(-3.145,.643)--(-3.172,.617)--cycle;
\filldraw[fill opacity=0.8,fill=gray!20,draw=none](-3.087,1.053)--(-3.06,1.073)--(-3.047,1.092)--(-3.064,1.109)--(-3.098,1.065)--cycle;
\draw(-3.06,1.073)--(-3.047,1.092)--(-3.064,1.109)--(-3.098,1.065)--(-3.087,1.053);
\filldraw[fill opacity=0.8,fill=gray!20](-2.689,.953)--(-2.697,1.008)--(-2.732,.985)--(-2.725,.929)--cycle;
\filldraw[fill opacity=0.8,fill=gray!20,draw=none](-2.499,1.602)--(-2.503,1.606)--(-2.504,1.605)--cycle;
\draw(-2.503,1.606)--(-2.504,1.605);
\filldraw[fill opacity=0.8,fill=gray!20,draw=none](-3.125,.924)--(-3.118,.932)--(-3.113,.895)--cycle;
\draw(-3.125,.924)--(-3.118,.932);
\filldraw[fill opacity=0.8,fill=gray!20,draw=none](-2.492,1.601)--(-2.47,1.6)--(-2.464,1.631)--(-2.515,1.634)--(-2.508,1.608)--cycle;
\draw(-2.492,1.601)--(-2.47,1.6);
\draw(-2.464,1.631)--(-2.515,1.634)--(-2.508,1.608);
\filldraw[fill opacity=0.8,fill=gray!20,draw=none](-2.508,1.608)--(-2.505,1.604)--(-2.503,1.606)--cycle;
\draw(-2.505,1.604)--(-2.503,1.606);
\filldraw[fill opacity=0.8,fill=gray!20,draw=none](-2.975,1.163)--(-3.001,1.166)--(-2.981,1.172)--cycle;
\filldraw[fill opacity=0.8,fill=gray!20](-2.39,1.778)--(-2.404,1.806)--(-2.441,1.805)--(-2.443,1.775)--cycle;
\filldraw[fill opacity=0.8,fill=gray!20,draw=none](-2.719,1.451)--(-2.721,1.456)--(-2.726,1.494)--cycle;
\draw(-2.721,1.456)--(-2.726,1.494);
\filldraw[fill opacity=0.8,fill=gray!20,draw=none](-3.437,.829)--(-3.441,.859)--(-3.415,.904)--(-3.408,.908)--(-3.402,.852)--cycle;
\draw(-3.415,.904)--(-3.408,.908)--(-3.402,.852)--(-3.437,.829)--(-3.441,.859);
\filldraw[fill opacity=0.8,fill=gray!20,draw=none](-3.441,.859)--(-3.443,.873)--(-3.441,.887)--(-3.415,.904)--cycle;
\draw(-3.441,.859)--(-3.443,.873);
\draw(-3.441,.887)--(-3.415,.904);
\filldraw[fill opacity=0.8,fill=gray!20,draw=none](-3.328,.866)--(-3.44,.893)--(-3.444,.947)--(-3.284,.908)--cycle;
\draw(-3.328,.866)--(-3.44,.893);
\draw(-3.444,.947)--(-3.284,.908);
\filldraw[fill opacity=0.8,fill=gray!20,draw=none](-3.26,.897)--(-3.291,.872)--(-3.284,.908)--cycle;
\draw(-3.291,.872)--(-3.284,.908);
\filldraw[fill opacity=0.8,fill=gray!20,draw=none](-3.293,.857)--(-3.328,.866)--(-3.284,.908)--cycle;
\draw(-3.284,.908)--(-3.293,.857)--(-3.328,.866);
\filldraw[fill opacity=0.8,fill=gray!20,draw=none](-3.284,.908)--(-3.328,.866)--(-3.324,.905)--cycle;
\draw(-3.328,.866)--(-3.324,.905)--(-3.284,.908);
\filldraw[fill opacity=0.8,fill=gray!20,draw=none](-3.335,.839)--(-3.328,.866)--(-3.33,.842)--cycle;
\draw(-3.328,.866)--(-3.33,.842);
\filldraw[fill opacity=0.8,fill=gray!20,draw=none](-3.328,.866)--(-3.335,.839)--(-3.348,.831)--(-3.35,.835)--cycle;
\filldraw[fill opacity=0.8,fill=gray!20,draw=none](-3.293,.845)--(-3.35,.86)--(-3.348,.871)--(-3.293,.857)--cycle;
\draw(-3.348,.871)--(-3.293,.857)--(-3.293,.845);
\filldraw[fill opacity=0.8,fill=gray!20,draw=none](-3.437,.829)--(-3.402,.852)--(-3.398,.841)--cycle;
\draw(-3.437,.829)--(-3.402,.852)--(-3.398,.841);
\filldraw[fill opacity=0.8,fill=gray!20,draw=none](-3.415,.777)--(-3.437,.829)--(-3.398,.841)--(-3.384,.798)--cycle;
\draw(-3.398,.841)--(-3.384,.798)--(-3.415,.777)--(-3.437,.829);
\filldraw[fill opacity=0.8,fill=gray!20,draw=none](-3.35,.86)--(-3.335,.856)--(-3.355,.828)--(-3.356,.828)--cycle;
\filldraw[fill opacity=0.8,fill=gray!20,draw=none](-3.35,.86)--(-3.442,.884)--(-3.438,.893)--(-3.348,.871)--cycle;
\draw(-3.438,.893)--(-3.348,.871);
\filldraw[fill opacity=0.8,fill=gray!20,draw=none](-3.377,.867)--(-3.35,.86)--(-3.356,.828)--(-3.359,.827)--(-3.374,.831)--cycle;
\draw(-3.359,.827)--(-3.374,.831);
\filldraw[fill opacity=0.8,fill=gray!20,draw=none](-3.324,.905)--(-3.328,.866)--(-3.35,.835)--(-3.355,.841)--(-3.35,.9)--cycle;
\draw(-3.355,.841)--(-3.35,.9)--(-3.324,.905)--(-3.328,.866);
\filldraw[fill opacity=0.8,fill=gray!20,draw=none](-3.166,.897)--(-3.184,.9)--(-3.235,.906)--(-3.284,.908)--(-3.324,.905)--(-3.34,.902)--cycle;
\draw(-3.166,.897)--(-3.184,.9)--(-3.235,.906)--(-3.284,.908)--(-3.324,.905)--(-3.34,.902);
\filldraw[fill opacity=0.8,fill=gray!20,draw=none](-2.751,1.201)--(-2.664,1.201)--(-2.677,1.172)--(-2.719,1.146)--cycle;
\draw(-2.664,1.201)--(-2.677,1.172)--(-2.719,1.146);
\filldraw[fill opacity=0.8,fill=gray!20](-2.817,.757)--(-2.779,.785)--(-2.839,.774)--(-2.86,.748)--cycle;
\filldraw[fill opacity=0.8,fill=gray!20,draw=none](-3.12,.95)--(-3.118,.932)--(-3.123,.925)--cycle;
\draw(-3.118,.932)--(-3.123,.925);
\filldraw[fill opacity=0.8,fill=gray!20,draw=none](-2.347,1.735)--(-2.343,1.751)--(-2.357,1.748)--cycle;
\draw(-2.343,1.751)--(-2.357,1.748);
\filldraw[fill opacity=0.8,fill=gray!20,draw=none](-2.347,1.735)--(-2.353,1.71)--(-2.331,1.715)--cycle;
\draw(-2.353,1.71)--(-2.331,1.715);
\filldraw[fill opacity=0.8,fill=gray!20,draw=none](-2.333,1.686)--(-2.331,1.715)--(-2.366,1.708)--cycle;
\draw(-2.331,1.715)--(-2.366,1.708);
\filldraw[fill opacity=0.8,fill=gray!20](-2.38,1.601)--(-2.373,1.632)--(-2.445,1.629)--(-2.444,1.598)--cycle;
\filldraw[fill opacity=0.8,fill=gray!20,draw=none](-3.249,.797)--(-3.257,.811)--(-3.239,.815)--(-3.226,.812)--cycle;
\draw(-3.249,.797)--(-3.257,.811)--(-3.239,.815);
\filldraw[fill opacity=0.8,fill=gray!20,draw=none](-3.246,.792)--(-3.249,.797)--(-3.226,.812)--(-3.207,.806)--cycle;
\draw(-3.246,.792)--(-3.249,.797);
\filldraw[fill opacity=0.8,fill=gray!20,draw=none](-3.177,.895)--(-3.16,.896)--(-3.166,.897)--cycle;
\draw(-3.16,.896)--(-3.166,.897);
\filldraw[fill opacity=0.8,fill=gray!20,draw=none](-3.237,.893)--(-3.177,.895)--(-3.166,.897)--(-3.261,.9)--cycle;
\filldraw[fill opacity=0.8,fill=gray!20,draw=none](-3.177,.895)--(-3.121,.963)--(-3.12,.95)--(-3.123,.925)--(-3.147,.897)--cycle;
\draw(-3.177,.895)--(-3.121,.963);
\draw(-3.123,.925)--(-3.147,.897);
\filldraw[fill opacity=0.8,fill=gray!20,draw=none](-2.334,1.676)--(-2.333,1.686)--(-2.366,1.708)--(-2.373,1.706)--(-2.371,1.668)--cycle;
\draw(-2.366,1.708)--(-2.373,1.706)--(-2.371,1.668)--(-2.334,1.676);
\filldraw[fill opacity=0.8,fill=gray!20](-3.255,.603)--(-3.271,.624)--(-3.234,.626)--(-3.236,.604)--cycle;
\filldraw[fill opacity=0.8,fill=gray!20](-3.236,.604)--(-3.234,.626)--(-3.198,.623)--(-3.217,.603)--cycle;
\filldraw[fill opacity=0.8,fill=gray!20,draw=none](-2.437,1.598)--(-2.444,1.598)--(-2.444,1.589)--cycle;
\draw(-2.437,1.598)--(-2.444,1.598)--(-2.444,1.589);
\filldraw[fill opacity=0.8,fill=gray!20,draw=none](-2.712,1.438)--(-2.719,1.451)--(-2.721,1.465)--cycle;
\filldraw[fill opacity=0.8,fill=gray!20,draw=none](-2.47,1.6)--(-2.444,1.598)--(-2.445,1.629)--(-2.464,1.631)--cycle;
\draw(-2.47,1.6)--(-2.444,1.598)--(-2.445,1.629)--(-2.464,1.631);
\filldraw[fill opacity=0.8,fill=gray!20,draw=none](-2.708,1.429)--(-2.712,1.438)--(-2.721,1.465)--(-2.726,1.494)--(-2.73,1.523)--cycle;
\draw(-2.726,1.494)--(-2.73,1.523);
\filldraw[fill opacity=0.8,fill=gray!20,draw=none](-3.257,.896)--(-3.285,.855)--(-3.293,.847)--(-3.293,.857)--(-3.291,.872)--(-3.26,.897)--cycle;
\draw(-3.293,.847)--(-3.293,.857)--(-3.291,.872);
\filldraw[fill opacity=0.8,fill=gray!20,draw=none](-3.285,.855)--(-3.293,.843)--(-3.293,.847)--cycle;
\draw(-3.293,.843)--(-3.293,.847);
\filldraw[fill opacity=0.8,fill=gray!20,draw=none](-3.335,.839)--(-3.338,.826)--(-3.342,.822)--(-3.348,.831)--cycle;
\filldraw[fill opacity=0.8,fill=gray!20,draw=none](-3.316,.851)--(-3.293,.845)--(-3.293,.843)--(-3.335,.833)--cycle;
\draw(-3.293,.845)--(-3.293,.843);
\filldraw[fill opacity=0.8,fill=gray!20,draw=none](-3.348,.831)--(-3.342,.822)--(-3.343,.822)--(-3.354,.828)--cycle;
\filldraw[fill opacity=0.8,fill=gray!20,draw=none](-3.35,.835)--(-3.348,.831)--(-3.354,.828)--(-3.355,.828)--cycle;
\filldraw[fill opacity=0.8,fill=gray!20,draw=none](-3.335,.856)--(-3.316,.851)--(-3.335,.833)--(-3.355,.828)--cycle;
\filldraw[fill opacity=0.8,fill=gray!20,draw=none](-3.342,.822)--(-3.341,.821)--(-3.343,.822)--cycle;
\filldraw[fill opacity=0.8,fill=gray!20](-3.344,.821)--(-3.324,.847)--(-3.284,.855)--(-3.295,.83)--cycle;
\filldraw[fill opacity=0.8,fill=gray!20](-2.697,1.008)--(-2.719,1.06)--(-2.75,1.04)--(-2.732,.985)--cycle;
\filldraw[fill opacity=0.8,fill=gray!20,draw=none](-2.353,1.71)--(-2.347,1.735)--(-2.357,1.748)--(-2.38,1.744)--(-2.373,1.706)--cycle;
\draw(-2.357,1.748)--(-2.38,1.744)--(-2.373,1.706)--(-2.353,1.71);
\filldraw[fill opacity=0.8,fill=gray!20,draw=none](-3.215,.871)--(-3.206,.856)--(-3.222,.857)--cycle;
\draw(-3.206,.856)--(-3.222,.857);
\filldraw[fill opacity=0.8,fill=gray!20,draw=none](-3.208,.858)--(-3.197,.871)--(-3.19,.846)--cycle;
\draw(-3.208,.858)--(-3.197,.871);
\filldraw[fill opacity=0.8,fill=gray!20,draw=none](-3.215,.871)--(-3.222,.857)--(-3.232,.857)--(-3.234,.874)--(-3.216,.872)--cycle;
\draw(-3.222,.857)--(-3.232,.857)--(-3.234,.874)--(-3.216,.872);
\filldraw[fill opacity=0.8,fill=gray!20,draw=none](-3.214,.872)--(-3.197,.892)--(-3.177,.895)--(-3.202,.866)--cycle;
\draw(-3.214,.872)--(-3.197,.892);
\draw(-3.177,.895)--(-3.202,.866);
\filldraw[fill opacity=0.8,fill=gray!20,draw=none](-3.242,.838)--(-3.214,.872)--(-3.202,.866)--(-3.247,.812)--cycle;
\draw(-3.242,.838)--(-3.214,.872);
\draw(-3.202,.866)--(-3.247,.812);
\filldraw[fill opacity=0.8,fill=gray!20,draw=none](-3.206,.856)--(-3.215,.871)--(-3.214,.872)--(-3.198,.871)--(-3.181,.854)--cycle;
\draw(-3.214,.872)--(-3.198,.871)--(-3.181,.854)--(-3.206,.856);
\filldraw[fill opacity=0.8,fill=gray!20,draw=none](-3.246,.792)--(-3.247,.812)--(-3.208,.858)--(-3.19,.846)--(-3.19,.845)--(-3.229,.798)--cycle;
\draw(-3.247,.812)--(-3.208,.858);
\draw(-3.19,.845)--(-3.229,.798);
\filldraw[fill opacity=0.8,fill=gray!20,draw=none](-3.186,.83)--(-3.204,.855)--(-3.181,.854)--(-3.168,.828)--cycle;
\draw(-3.204,.855)--(-3.181,.854)--(-3.168,.828)--(-3.186,.83);
\filldraw[fill opacity=0.8,fill=gray!20,draw=none](-3.102,.749)--(-3.074,.764)--(-3.086,.746)--cycle;
\draw(-3.074,.764)--(-3.086,.746)--(-3.102,.749);
\filldraw[fill opacity=0.8,fill=gray!20,draw=none](-3.12,.95)--(-3.121,.963)--(-3.117,.968)--cycle;
\draw(-3.121,.963)--(-3.117,.968);
\filldraw[fill opacity=0.8,fill=gray!20](-3.27,.6)--(-3.299,.619)--(-3.271,.624)--(-3.255,.603)--cycle;
\filldraw[fill opacity=0.8,fill=gray!20,draw=none](-3.284,.908)--(-3.357,.926)--(-3.275,.956)--cycle;
\draw(-3.284,.908)--(-3.357,.926);
\filldraw[fill opacity=0.8,fill=gray!20,draw=none](-3.284,.911)--(-3.285,.954)--(-3.23,.967)--cycle;
\draw(-3.284,.911)--(-3.285,.954)--(-3.23,.967);
\filldraw[fill opacity=0.8,fill=gray!20,draw=none](-3.229,.798)--(-3.194,.841)--(-3.175,.829)--(-3.171,.825)--(-3.19,.802)--cycle;
\draw(-3.229,.798)--(-3.194,.841);
\draw(-3.171,.825)--(-3.19,.802);
\filldraw[fill opacity=0.8,fill=gray!20](-3.294,1.064)--(-3.274,1.089)--(-3.218,1.091)--(-3.216,1.067)--cycle;
\filldraw[fill opacity=0.8,fill=gray!20](-3.216,1.067)--(-3.218,1.091)--(-3.165,1.087)--(-3.139,1.062)--cycle;
\filldraw[fill opacity=0.8,fill=gray!20,draw=none](-3.099,1.049)--(-3.092,1.047)--(-3.087,1.053)--(-3.093,1.059)--cycle;
\draw(-3.087,1.053)--(-3.093,1.059);
\filldraw[fill opacity=0.8,fill=gray!20,draw=none](-3.284,.908)--(-3.275,.956)--(-3.266,.954)--cycle;
\draw(-3.275,.956)--(-3.266,.954)--(-3.284,.908);
\filldraw[fill opacity=0.8,fill=gray!20,draw=none](-3.236,1.014)--(-3.215,1.004)--(-3.258,.994)--cycle;
\draw(-3.215,1.004)--(-3.258,.994);
\filldraw[fill opacity=0.8,fill=gray!20,draw=none](-3.249,.892)--(-3.237,.893)--(-3.261,.9)--(-3.266,.9)--cycle;
\filldraw[fill opacity=0.8,fill=gray!20,draw=none](-3.26,.897)--(-3.284,.908)--(-3.266,.954)--(-3.241,.989)--(-3.214,1.008)--(-3.189,1.008)--(-3.169,.989)--(-3.165,.975)--cycle;
\draw(-3.284,.908)--(-3.266,.954)--(-3.241,.989)--(-3.214,1.008)--(-3.189,1.008)--(-3.169,.989)--(-3.165,.975);
\filldraw[fill opacity=0.8,fill=gray!20,draw=none](-3.266,.954)--(-3.275,.956)--(-3.258,.994)--(-3.241,.989)--cycle;
\draw(-3.258,.994)--(-3.241,.989)--(-3.266,.954)--(-3.275,.956);
\filldraw[fill opacity=0.8,fill=gray!20,draw=none](-3.241,.989)--(-3.258,.994)--(-3.277,1.007)--(-3.236,1.014)--(-3.214,1.008)--cycle;
\draw(-3.236,1.014)--(-3.214,1.008)--(-3.241,.989)--(-3.258,.994);
\filldraw[fill opacity=0.8,fill=gray!20,draw=none](-3.193,.895)--(-3.249,.892)--(-3.245,.89)--(-3.208,.886)--cycle;
\filldraw[fill opacity=0.8,fill=gray!20,draw=none](-3.219,.879)--(-3.208,.886)--(-3.245,.89)--cycle;
\filldraw[fill opacity=0.8,fill=gray!20,draw=none](-3.219,.879)--(-3.26,.897)--(-3.165,.975)--(-3.16,.962)--cycle;
\draw(-3.165,.975)--(-3.16,.962);
\filldraw[fill opacity=0.8,fill=gray!20,draw=none](-3.397,1.003)--(-3.384,1.028)--(-3.379,1.019)--(-3.384,1.012)--cycle;
\draw(-3.379,1.019)--(-3.384,1.012)--(-3.397,1.003);
\filldraw[fill opacity=0.8,fill=gray!20,draw=none](-3.384,1.028)--(-3.382,1.033)--(-3.38,1.035)--(-3.371,1.041)--(-3.367,1.035)--(-3.379,1.019)--cycle;
\draw(-3.382,1.033)--(-3.38,1.035)--(-3.371,1.041);
\draw(-3.367,1.035)--(-3.379,1.019);
\filldraw[fill opacity=0.8,fill=gray!20,draw=none](-3.371,1.041)--(-3.355,1.052)--(-3.367,1.035)--cycle;
\draw(-3.371,1.041)--(-3.355,1.052)--(-3.367,1.035);
\filldraw[fill opacity=0.8,fill=gray!20](-3.731,1.034)--(-3.703,1.083)--(-3.661,1.119)--(-3.61,1.135)--(-3.56,1.13)--(-3.518,1.103)--(-3.489,1.06)--(-3.479,1.006)--(-3.489,.951)--(-3.518,.901)--(-3.56,.866)--(-3.61,.849)--(-3.661,.855)--(-3.703,.881)--(-3.731,.924)--(-3.741,.978)--cycle;
\filldraw[fill opacity=0.8,fill=gray!20,draw=none](-3.371,1.041)--(-3.369,1.044)--(-3.343,1.063)--(-3.327,1.073)--(-3.355,1.052)--cycle;
\draw(-3.369,1.044)--(-3.343,1.063);
\draw(-3.327,1.073)--(-3.355,1.052)--(-3.371,1.041);
\filldraw[fill opacity=0.8,fill=gray!20,draw=none](-3.258,.994)--(-3.376,1.022)--(-3.386,1.03)--(-3.379,1.035)--(-3.359,1.044)--(-3.309,1.032)--cycle;
\draw(-3.258,.994)--(-3.376,1.022);
\draw(-3.359,1.044)--(-3.309,1.032);
\filldraw[fill opacity=0.8,fill=gray!20,draw=none](-3.26,1.008)--(-3.236,1.014)--(-3.258,.994)--(-3.276,.989)--cycle;
\draw(-3.258,.994)--(-3.276,.989)--(-3.26,1.008)--(-3.236,1.014);
\filldraw[fill opacity=0.8,fill=gray!20,draw=none](-3.254,.894)--(-3.225,.997)--(-3.238,1.008)--(-3.26,1.008)--(-3.276,.989)--(-3.285,.954)--(-3.284,.908)--cycle;
\draw(-3.225,.997)--(-3.238,1.008)--(-3.26,1.008)--(-3.276,.989)--(-3.285,.954)--(-3.284,.908);
\filldraw[fill opacity=0.8,fill=gray!20,draw=none](-3.264,.856)--(-3.253,.873)--(-3.234,.874)--(-3.232,.857)--cycle;
\draw(-3.253,.873)--(-3.234,.874)--(-3.232,.857)--(-3.264,.856);
\filldraw[fill opacity=0.8,fill=gray!20,draw=none](-3.253,.873)--(-3.239,.88)--(-3.236,.88)--(-3.234,.874)--cycle;
\draw(-3.239,.88)--(-3.236,.88)--(-3.234,.874)--(-3.253,.873);
\filldraw[fill opacity=0.8,fill=gray!20,draw=none](-3.249,.886)--(-3.258,.877)--(-3.268,.841)--(-3.257,.811)--(-3.246,.792)--cycle;
\draw(-3.268,.841)--(-3.257,.811)--(-3.246,.792);
\filldraw[fill opacity=0.8,fill=gray!20](-2.444,1.741)--(-2.443,1.775)--(-2.494,1.779)--(-2.506,1.745)--cycle;
\filldraw[fill opacity=0.8,fill=gray!20](-2.445,1.629)--(-2.445,1.665)--(-2.517,1.67)--(-2.515,1.634)--cycle;
\filldraw[fill opacity=0.8,fill=gray!20](-3.217,.603)--(-3.198,.623)--(-3.172,.617)--(-3.204,.6)--cycle;
\filldraw[fill opacity=0.8,fill=gray!20,draw=none](-3.214,.877)--(-3.205,.888)--(-3.197,.892)--(-3.212,.875)--cycle;
\draw(-3.214,.877)--(-3.205,.888);
\draw(-3.197,.892)--(-3.212,.875);
\filldraw[fill opacity=0.8,fill=gray!20,draw=none](-3.193,.895)--(-3.205,.888)--(-3.202,.885)--(-3.095,.872)--(-3.094,.873)--(-3.108,.882)--(-3.139,.892)--(-3.16,.896)--cycle;
\draw(-3.095,.872)--(-3.094,.873)--(-3.108,.882)--(-3.139,.892)--(-3.16,.896);
\filldraw[fill opacity=0.8,fill=gray!20,draw=none](-3.197,.892)--(-3.121,.984)--(-3.121,.963)--(-3.177,.895)--cycle;
\draw(-3.197,.892)--(-3.121,.984);
\draw(-3.121,.963)--(-3.177,.895);
\filldraw[fill opacity=0.8,fill=gray!20,draw=none](-3.125,1.043)--(-3.13,1.042)--(-3.139,1.062)--(-3.129,1.059)--cycle;
\draw(-3.13,1.042)--(-3.139,1.062)--(-3.129,1.059);
\filldraw[fill opacity=0.8,fill=gray!20](-3.364,.664)--(-3.379,.699)--(-3.356,.714)--(-3.344,.678)--cycle;
\filldraw[fill opacity=0.8,fill=gray!20](-3.341,.634)--(-3.364,.664)--(-3.344,.678)--(-3.324,.645)--cycle;
\filldraw[fill opacity=0.8,fill=gray!20](-2.719,1.06)--(-2.754,1.105)--(-2.779,1.088)--(-2.75,1.04)--cycle;
\filldraw[fill opacity=0.8,fill=gray!20,draw=none](-3.358,.801)--(-3.345,.819)--(-3.344,.821)--(-3.356,.788)--cycle;
\draw(-3.345,.819)--(-3.344,.821)--(-3.356,.788);
\filldraw[fill opacity=0.8,fill=gray!20,draw=none](-3.378,.773)--(-3.358,.801)--(-3.356,.788)--cycle;
\draw(-3.356,.788)--(-3.378,.773);
\filldraw[fill opacity=0.8,fill=gray!20,draw=none](-3.345,.819)--(-3.343,.822)--(-3.344,.821)--cycle;
\draw(-3.343,.822)--(-3.344,.821)--(-3.345,.819);
\filldraw[fill opacity=0.8,fill=gray!20,draw=none](-3.341,.821)--(-3.333,.808)--(-3.337,.757)--(-3.363,.752)--(-3.356,.829)--cycle;
\draw(-3.333,.808)--(-3.337,.757)--(-3.363,.752)--(-3.356,.829);
\filldraw[fill opacity=0.5,fill=gray!20](-1.507,1.824)--(-1.642,1.838)--(-1.58,2.363)--(-1.447,2.325)--cycle;
\filldraw[fill opacity=0.8,fill=gray!20](-2.38,1.744)--(-2.39,1.778)--(-2.443,1.775)--(-2.444,1.741)--cycle;
\filldraw[fill opacity=0.8,fill=gray!20,draw=none](-3.119,.754)--(-3.121,.734)--(-3.138,.74)--cycle;
\draw(-3.119,.754)--(-3.121,.734)--(-3.138,.74);
\filldraw[fill opacity=0.5,fill=gray!20](-1.58,2.363)--(-1.698,2.4)--(-1.509,2.919)--(-1.397,2.865)--cycle;
\filldraw[fill opacity=0.8,fill=gray!20](-2.373,1.632)--(-2.371,1.668)--(-2.445,1.665)--(-2.445,1.629)--cycle;
\filldraw[fill opacity=0.8,fill=gray!20,draw=none](-2.702,1.418)--(-2.698,1.395)--(-2.712,1.438)--cycle;
\filldraw[fill opacity=0.8,fill=gray!20,draw=none](-2.759,1.978)--(-2.73,1.918)--(-2.715,1.797)--(-2.752,1.806)--(-2.788,1.954)--cycle;
\draw(-2.73,1.918)--(-2.715,1.797)--(-2.752,1.806);
\filldraw[fill opacity=0.8,fill=gray!20](-3.311,.611)--(-3.341,.634)--(-3.324,.645)--(-3.299,.619)--cycle;
\filldraw[fill opacity=0.8,fill=gray!20,draw=none](-3.11,1.043)--(-3.125,1.043)--(-3.129,1.059)--(-3.104,1.053)--cycle;
\draw(-3.129,1.059)--(-3.104,1.053);
\filldraw[fill opacity=0.8,fill=gray!20,draw=none](-3.086,.746)--(-3.074,.764)--(-3.055,.779)--(-3.053,.75)--(-3.07,.728)--cycle;
\draw(-3.053,.75)--(-3.07,.728)--(-3.086,.746)--(-3.074,.764);
\filldraw[fill opacity=0.8,fill=gray!20,draw=none](-3.345,.819)--(-3.355,.813)--(-3.344,.824)--(-3.337,.829)--(-3.343,.822)--cycle;
\draw(-3.345,.819)--(-3.355,.813);
\draw(-3.337,.829)--(-3.343,.822);
\filldraw[fill opacity=0.8,fill=gray!20,draw=none](-3.344,.824)--(-3.335,.833)--(-3.337,.829)--cycle;
\draw(-3.335,.833)--(-3.337,.829);
\filldraw[fill opacity=0.8,fill=gray!20,draw=none](-3.355,.828)--(-3.293,.843)--(-3.292,.811)--(-3.356,.826)--cycle;
\draw(-3.293,.843)--(-3.292,.811)--(-3.356,.826);
\filldraw[fill opacity=0.8,fill=gray!20,draw=none](-3.24,.747)--(-3.271,.748)--(-3.245,.778)--cycle;
\draw(-3.271,.748)--(-3.245,.778);
\filldraw[fill opacity=0.8,fill=gray!20,draw=none](-3.264,.856)--(-3.284,.855)--(-3.283,.857)--(-3.264,.872)--(-3.253,.873)--cycle;
\draw(-3.264,.856)--(-3.284,.855)--(-3.283,.857);
\draw(-3.264,.872)--(-3.253,.873);
\filldraw[fill opacity=0.8,fill=gray!20,draw=none](-3.286,.764)--(-3.247,.812)--(-3.245,.778)--(-3.271,.748)--cycle;
\draw(-3.245,.778)--(-3.271,.748)--(-3.286,.764)--(-3.247,.812);
\filldraw[fill opacity=0.8,fill=gray!20,draw=none](-3.214,.876)--(-3.214,.877)--(-3.212,.875)--(-3.214,.872)--cycle;
\draw(-3.214,.876)--(-3.214,.877);
\draw(-3.212,.875)--(-3.214,.872);
\filldraw[fill opacity=0.8,fill=gray!20,draw=none](-3.229,.873)--(-3.212,.877)--(-3.198,.871)--cycle;
\draw(-3.212,.877)--(-3.198,.871)--(-3.229,.873);
\filldraw[fill opacity=0.8,fill=gray!20,draw=none](-3.215,.871)--(-3.216,.872)--(-3.214,.872)--cycle;
\draw(-3.216,.872)--(-3.214,.872);
\filldraw[fill opacity=0.8,fill=gray!20,draw=none](-3.234,.853)--(-3.214,.876)--(-3.214,.872)--(-3.242,.838)--cycle;
\draw(-3.234,.853)--(-3.214,.876);
\draw(-3.214,.872)--(-3.242,.838);
\filldraw[fill opacity=0.8,fill=gray!20,draw=none](-3.186,.785)--(-3.196,.818)--(-3.209,.842)--(-3.222,.854)--(-3.234,.853)--(-3.247,.83)--(-3.246,.792)--(-3.235,.775)--(-3.212,.756)--(-3.19,.755)--(-3.182,.765)--cycle;
\draw(-3.246,.792)--(-3.235,.775)--(-3.212,.756)--(-3.19,.755)--(-3.182,.765);
\filldraw[fill opacity=0.8,fill=gray!20,draw=none](-3.234,.853)--(-3.248,.851)--(-3.247,.83)--cycle;
\filldraw[fill opacity=0.8,fill=gray!20,draw=none](-3.283,.781)--(-3.275,.799)--(-3.242,.838)--(-3.247,.812)--(-3.269,.785)--cycle;
\draw(-3.275,.799)--(-3.242,.838);
\draw(-3.247,.812)--(-3.269,.785);
\filldraw[fill opacity=0.8,fill=gray!20,draw=none](-3.289,.769)--(-3.283,.781)--(-3.269,.785)--(-3.286,.764)--cycle;
\draw(-3.269,.785)--(-3.286,.764)--(-3.289,.769);
\filldraw[fill opacity=0.8,fill=gray!20,draw=none](-3.249,.886)--(-3.249,.892)--(-3.254,.894)--(-3.258,.877)--cycle;
\filldraw[fill opacity=0.8,fill=gray!20,draw=none](-3.214,.876)--(-3.229,.873)--(-3.234,.874)--(-3.235,.879)--(-3.217,.879)--(-3.214,.878)--cycle;
\draw(-3.229,.873)--(-3.234,.874)--(-3.235,.879);
\draw(-3.217,.879)--(-3.214,.878);
\filldraw[fill opacity=0.8,fill=gray!20,draw=none](-3.255,.827)--(-3.234,.853)--(-3.242,.838)--(-3.275,.799)--cycle;
\draw(-3.255,.827)--(-3.234,.853);
\draw(-3.242,.838)--(-3.275,.799);
\filldraw[fill opacity=0.8,fill=gray!20](-3.198,.871)--(-3.217,.879)--(-3.204,.876)--(-3.172,.865)--cycle;
\filldraw[fill opacity=0.8,fill=gray!20,draw=none](-3.222,.854)--(-3.204,.876)--(-3.214,.877)--(-3.234,.853)--cycle;
\draw(-3.222,.854)--(-3.204,.876);
\draw(-3.214,.877)--(-3.234,.853);
\filldraw[fill opacity=0.8,fill=gray!20,draw=none](-3.222,.854)--(-3.249,.879)--(-3.248,.851)--cycle;
\filldraw[fill opacity=0.8,fill=gray!20,draw=none](-3.246,.876)--(-3.253,.873)--(-3.264,.872)--(-3.26,.874)--cycle;
\draw(-3.253,.873)--(-3.264,.872);
\filldraw[fill opacity=0.8,fill=gray!20,draw=none](-3.235,.879)--(-3.236,.88)--(-3.217,.879)--cycle;
\draw(-3.235,.879)--(-3.236,.88)--(-3.217,.879);
\filldraw[fill opacity=0.8,fill=gray!20,draw=none](-3.096,1.018)--(-3.108,1.001)--(-3.098,.99)--(-3.09,1.012)--cycle;
\draw(-3.108,1.001)--(-3.098,.99)--(-3.09,1.012);
\filldraw[fill opacity=0.8,fill=gray!20,draw=none](-3.057,.922)--(-3.056,.941)--(-3.063,.977)--(-3.067,.982)--(-3.098,.99)--(-3.105,.934)--cycle;
\draw(-3.067,.982)--(-3.098,.99)--(-3.105,.934)--(-3.057,.922);
\filldraw[fill opacity=0.8,fill=gray!20,draw=none](-3.067,.982)--(-3.071,1.007)--(-3.09,1.012)--(-3.098,.99)--cycle;
\draw(-3.09,1.012)--(-3.098,.99)--(-3.067,.982);
\filldraw[fill opacity=0.8,fill=gray!20,draw=none](-3.23,.817)--(-3.189,.865)--(-3.191,.868)--(-3.204,.876)--(-3.239,.834)--cycle;
\draw(-3.23,.817)--(-3.189,.865);
\draw(-3.204,.876)--(-3.239,.834);
\filldraw[fill opacity=0.8,fill=gray!20,draw=none](-3.216,.855)--(-3.228,.878)--(-3.241,.881)--(-3.249,.879)--(-3.222,.854)--cycle;
\filldraw[fill opacity=0.8,fill=gray!20,draw=none](-3.246,.876)--(-3.26,.874)--(-3.243,.88)--(-3.239,.88)--cycle;
\draw(-3.243,.88)--(-3.239,.88);
\filldraw[fill opacity=0.8,fill=gray!20,draw=none](-3.229,.883)--(-3.245,.89)--(-3.249,.886)--(-3.249,.879)--cycle;
\filldraw[fill opacity=0.8,fill=gray!20,draw=none](-3.249,.886)--(-3.245,.89)--(-3.249,.892)--cycle;
\filldraw[fill opacity=0.8,fill=gray!20,draw=none](-3.236,.88)--(-3.236,.879)--(-3.228,.878)--(-3.217,.879)--cycle;
\draw(-3.228,.878)--(-3.217,.879)--(-3.236,.88)--(-3.236,.879);
\filldraw[fill opacity=0.8,fill=gray!20,draw=none](-3.228,.878)--(-3.231,.883)--(-3.241,.881)--cycle;
\filldraw[fill opacity=0.8,fill=gray!20,draw=none](-3.239,.834)--(-3.222,.854)--(-3.234,.853)--(-3.255,.827)--cycle;
\draw(-3.239,.834)--(-3.222,.854);
\draw(-3.234,.853)--(-3.255,.827);
\filldraw[fill opacity=0.8,fill=gray!20,draw=none](-3.243,.88)--(-3.236,.879)--(-3.236,.88)--cycle;
\draw(-3.236,.879)--(-3.236,.88)--(-3.243,.88);
\filldraw[fill opacity=0.8,fill=gray!20,draw=none](-3.214,.876)--(-3.214,.878)--(-3.212,.877)--cycle;
\draw(-3.214,.878)--(-3.212,.877);
\filldraw[fill opacity=0.8,fill=gray!20,draw=none](-3.217,.879)--(-3.228,.878)--(-3.224,.876)--(-3.204,.876)--cycle;
\draw(-3.224,.876)--(-3.204,.876)--(-3.217,.879)--(-3.228,.878);
\filldraw[fill opacity=0.8,fill=gray!20,draw=none](-3.236,.879)--(-3.236,.879)--(-3.23,.877)--(-3.228,.878)--cycle;
\draw(-3.236,.879)--(-3.236,.879);
\draw(-3.23,.877)--(-3.228,.878);
\filldraw[fill opacity=0.8,fill=gray!20,draw=none](-3.032,.97)--(-3.02,.942)--(-3.035,.958)--(-3.055,1.008)--(-3.044,.996)--cycle;
\draw(-3.02,.942)--(-3.035,.958)--(-3.055,1.008)--(-3.044,.996);
\filldraw[fill opacity=0.8,fill=gray!20,draw=none](-3.056,1.011)--(-3.044,.996)--(-3.055,1.008)--(-3.074,1.032)--cycle;
\draw(-3.044,.996)--(-3.055,1.008)--(-3.074,1.032);
\filldraw[fill opacity=0.8,fill=gray!20,draw=none](-3.204,.876)--(-3.224,.876)--(-3.22,.874)--(-3.2,.871)--cycle;
\draw(-3.22,.874)--(-3.2,.871)--(-3.204,.876)--(-3.224,.876);
\filldraw[fill opacity=0.8,fill=gray!20,draw=none](-3.2,.871)--(-3.22,.874)--(-3.212,.869)--(-3.206,.867)--cycle;
\draw(-3.212,.869)--(-3.206,.867)--(-3.2,.871)--(-3.22,.874);
\filldraw[fill opacity=0.8,fill=gray!20,draw=none](-3.23,.881)--(-3.21,.867)--(-3.213,.871)--(-3.221,.88)--(-3.229,.883)--(-3.231,.883)--cycle;
\filldraw[fill opacity=0.8,fill=gray!20,draw=none](-3.23,.881)--(-3.229,.879)--(-3.208,.866)--cycle;
\filldraw[fill opacity=0.8,fill=gray!20,draw=none](-3.209,.867)--(-3.229,.879)--(-3.216,.855)--(-3.201,.856)--cycle;
\filldraw[fill opacity=0.8,fill=gray!20,draw=none](-3.228,.878)--(-3.23,.877)--(-3.227,.876)--(-3.224,.876)--cycle;
\draw(-3.228,.878)--(-3.23,.877);
\draw(-3.227,.876)--(-3.224,.876);
\filldraw[fill opacity=0.8,fill=gray!20,draw=none](-3.216,.855)--(-3.222,.854)--(-3.209,.842)--cycle;
\filldraw[fill opacity=0.8,fill=gray!20,draw=none](-3.224,.876)--(-3.227,.876)--(-3.223,.875)--(-3.22,.874)--cycle;
\draw(-3.224,.876)--(-3.227,.876);
\draw(-3.223,.875)--(-3.22,.874);
\filldraw[fill opacity=0.8,fill=gray!20,draw=none](-3.169,.851)--(-3.143,.828)--(-3.135,.834)--(-3.165,.857)--(-3.17,.853)--cycle;
\draw(-3.143,.828)--(-3.135,.834)--(-3.165,.857)--(-3.17,.853);
\filldraw[fill opacity=0.8,fill=gray!20,draw=none](-3.17,.853)--(-3.165,.857)--(-3.195,.869)--cycle;
\draw(-3.17,.853)--(-3.165,.857)--(-3.195,.869);
\filldraw[fill opacity=0.8,fill=gray!20,draw=none](-3.22,.874)--(-3.223,.875)--(-3.218,.871)--(-3.212,.869)--cycle;
\draw(-3.22,.874)--(-3.223,.875);
\draw(-3.218,.871)--(-3.212,.869);
\filldraw[fill opacity=0.8,fill=gray!20,draw=none](-3.196,.818)--(-3.165,.855)--(-3.189,.866)--(-3.209,.842)--cycle;
\draw(-3.196,.818)--(-3.165,.855);
\draw(-3.189,.866)--(-3.209,.842);
\filldraw[fill opacity=0.8,fill=gray!20,draw=none](-3.17,.853)--(-3.195,.869)--(-3.2,.871)--(-3.206,.867)--(-3.177,.849)--cycle;
\draw(-3.195,.869)--(-3.2,.871)--(-3.206,.867)--(-3.177,.849)--(-3.17,.853);
\filldraw[fill opacity=0.8,fill=gray!20,draw=none](-3.208,.856)--(-3.216,.855)--(-3.209,.842)--(-3.202,.835)--cycle;
\filldraw[fill opacity=0.8,fill=gray!20,draw=none](-3.212,.866)--(-3.206,.867)--(-3.218,.871)--cycle;
\draw(-3.212,.866)--(-3.206,.867)--(-3.218,.871);
\filldraw[fill opacity=0.8,fill=gray!20,draw=none](-3.213,.871)--(-3.219,.879)--(-3.221,.88)--cycle;
\filldraw[fill opacity=0.8,fill=gray!20](-3.172,.865)--(-3.204,.876)--(-3.2,.871)--(-3.165,.857)--cycle;
\filldraw[fill opacity=0.8,fill=gray!20,draw=none](-3.165,.855)--(-3.151,.872)--(-3.18,.877)--(-3.189,.866)--cycle;
\draw(-3.165,.855)--(-3.151,.872);
\draw(-3.18,.877)--(-3.189,.866);
\filldraw[fill opacity=0.8,fill=gray!20,draw=none](-3.192,.846)--(-3.177,.849)--(-3.206,.867)--(-3.209,.867)--cycle;
\draw(-3.192,.846)--(-3.177,.849)--(-3.206,.867)--(-3.209,.867);
\filldraw[fill opacity=0.8,fill=gray!20,draw=none](-3.213,.871)--(-3.207,.865)--(-3.195,.857)--(-3.184,.858)--(-3.184,.864)--(-3.219,.879)--cycle;
\filldraw[fill opacity=0.8,fill=gray!20,draw=none](-3.202,.835)--(-3.209,.842)--(-3.196,.818)--cycle;
\filldraw[fill opacity=0.8,fill=gray!20,draw=none](-3.229,.778)--(-3.196,.818)--(-3.209,.842)--(-3.23,.817)--cycle;
\draw(-3.229,.778)--(-3.196,.818);
\draw(-3.209,.842)--(-3.23,.817);
\filldraw[fill opacity=0.8,fill=gray!20,draw=none](-3.21,.867)--(-3.208,.866)--(-3.213,.871)--cycle;
\filldraw[fill opacity=0.8,fill=gray!20,draw=none](-3.169,.851)--(-3.17,.853)--(-3.171,.853)--cycle;
\draw(-3.17,.853)--(-3.171,.853);
\filldraw[fill opacity=0.8,fill=gray!20,draw=none](-3.232,.759)--(-3.208,.759)--(-3.147,.832)--(-3.169,.851)--(-3.229,.778)--cycle;
\draw(-3.208,.759)--(-3.147,.832);
\draw(-3.169,.851)--(-3.229,.778);
\filldraw[fill opacity=0.8,fill=gray!20,draw=none](-3.148,.825)--(-3.169,.851)--(-3.171,.853)--(-3.177,.849)--(-3.152,.822)--cycle;
\draw(-3.171,.853)--(-3.177,.849)--(-3.152,.822)--(-3.148,.825);
\filldraw[fill opacity=0.8,fill=gray!20,draw=none](-3.147,.832)--(-3.116,.869)--(-3.151,.872)--(-3.169,.851)--cycle;
\draw(-3.147,.832)--(-3.116,.869);
\draw(-3.151,.872)--(-3.169,.851);
\filldraw[fill opacity=0.8,fill=gray!20,draw=none](-3.169,.851)--(-3.148,.825)--(-3.143,.828)--cycle;
\draw(-3.148,.825)--(-3.143,.828);
\filldraw[fill opacity=0.8,fill=gray!20,draw=none](-3.176,.818)--(-3.198,.857)--(-3.208,.856)--(-3.202,.835)--(-3.173,.808)--cycle;
\filldraw[fill opacity=0.8,fill=gray!20,draw=none](-3.209,.867)--(-3.201,.856)--(-3.199,.857)--(-3.208,.866)--cycle;
\filldraw[fill opacity=0.8,fill=gray!20,draw=none](-3.173,.808)--(-3.202,.835)--(-3.196,.818)--(-3.174,.775)--(-3.168,.797)--cycle;
\draw(-3.174,.775)--(-3.168,.797);
\filldraw[fill opacity=0.8,fill=gray!20,draw=none](-3.2,.844)--(-3.192,.846)--(-3.209,.867)--(-3.212,.866)--cycle;
\draw(-3.2,.844)--(-3.192,.846);
\draw(-3.209,.867)--(-3.212,.866);
\filldraw[fill opacity=0.8,fill=gray!20,draw=none](-3.196,.818)--(-3.18,.767)--(-3.174,.774)--(-3.174,.775)--cycle;
\draw(-3.18,.767)--(-3.174,.774)--(-3.174,.775);
\filldraw[fill opacity=0.8,fill=gray!20,draw=none](-3.207,.865)--(-3.199,.857)--(-3.195,.857)--cycle;
\filldraw[fill opacity=0.8,fill=gray!20,draw=none](-3.152,.822)--(-3.177,.849)--(-3.2,.844)--(-3.193,.818)--(-3.192,.815)--cycle;
\draw(-3.193,.818)--(-3.192,.815)--(-3.152,.822)--(-3.177,.849)--(-3.2,.844);
\filldraw[fill opacity=0.8,fill=gray!20,draw=none](-3.192,.857)--(-3.198,.857)--(-3.176,.818)--cycle;
\filldraw[fill opacity=0.8,fill=gray!20,draw=none](-3.18,.797)--(-3.136,.797)--(-3.152,.822)--(-3.192,.815)--(-3.187,.799)--cycle;
\draw(-3.136,.797)--(-3.152,.822)--(-3.192,.815)--(-3.187,.799);
\filldraw[fill opacity=0.8,fill=gray!20,draw=none](-3.166,.856)--(-3.174,.86)--(-3.192,.857)--(-3.176,.818)--(-3.169,.805)--(-3.167,.803)--(-3.166,.81)--cycle;
\draw(-3.167,.803)--(-3.166,.81)--(-3.166,.856);
\filldraw[fill opacity=0.8,fill=gray!20,draw=none](-3.184,.858)--(-3.174,.86)--(-3.184,.864)--cycle;
\filldraw[fill opacity=0.8,fill=gray!20,draw=none](-3.192,.815)--(-3.192,.815)--(-3.193,.818)--cycle;
\draw(-3.192,.815)--(-3.192,.815)--(-3.193,.818);
\filldraw[fill opacity=0.8,fill=gray!20,draw=none](-3.182,.765)--(-3.186,.785)--(-3.21,.757)--cycle;
\draw(-3.186,.785)--(-3.21,.757);
\filldraw[fill opacity=0.8,fill=gray!20,draw=none](-3.186,.785)--(-3.182,.765)--(-3.18,.767)--cycle;
\draw(-3.182,.765)--(-3.18,.767);
\filldraw[fill opacity=0.8,fill=gray!20,draw=none](-3.192,.799)--(-3.187,.799)--(-3.192,.815)--(-3.192,.815)--cycle;
\draw(-3.187,.799)--(-3.192,.815)--(-3.192,.815);
\filldraw[fill opacity=0.8,fill=gray!20,draw=none](-3.257,.896)--(-3.166,.856)--(-3.185,.81)--(-3.209,.774)--(-3.236,.755)--(-3.261,.756)--(-3.281,.775)--(-3.292,.811)--(-3.293,.843)--cycle;
\draw(-3.166,.856)--(-3.185,.81)--(-3.209,.774)--(-3.236,.755)--(-3.261,.756)--(-3.281,.775)--(-3.292,.811)--(-3.293,.843);
\filldraw[fill opacity=0.8,fill=gray!20](-2.445,1.703)--(-2.444,1.741)--(-2.506,1.745)--(-2.515,1.708)--cycle;
\filldraw[fill opacity=0.8,fill=gray!20](-2.445,1.665)--(-2.445,1.703)--(-2.515,1.708)--(-2.517,1.67)--cycle;
\filldraw[fill opacity=0.8,fill=gray!20,draw=none](-3.063,1.054)--(-3.062,1.055)--(-3.063,1.054)--cycle;
\filldraw[fill opacity=0.8,fill=gray!20](-3.355,1.052)--(-3.317,1.081)--(-3.274,1.089)--(-3.294,1.064)--cycle;
\filldraw[fill opacity=0.8,fill=gray!20](-3.225,.668)--(-3.25,.686)--(-3.222,.687)--(-3.225,.668)--cycle;
\filldraw[fill opacity=0.8,fill=gray!20](-3.225,.668)--(-3.222,.687)--(-3.194,.685)--(-3.225,.668)--cycle;
\filldraw[fill opacity=0.8,fill=gray!20](-3.282,.676)--(-3.335,.698)--(-3.317,.71)--(-3.272,.682)--cycle;
\filldraw[fill opacity=0.8,fill=gray!20,draw=none](-3.069,1.05)--(-3.059,1.063)--(-3.062,1.055)--(-3.063,1.054)--cycle;
\draw(-3.069,1.05)--(-3.059,1.063);
\filldraw[fill opacity=0.8,fill=gray!20,draw=none](-3.116,.75)--(-3.108,.757)--(-3.107,.748)--cycle;
\draw(-3.108,.757)--(-3.107,.748)--(-3.116,.75);
\filldraw[fill opacity=0.8,fill=gray!20,draw=none](-3.125,.675)--(-3.119,.691)--(-3.104,.703)--(-3.097,.695)--(-3.111,.661)--cycle;
\draw(-3.104,.703)--(-3.097,.695)--(-3.111,.661)--(-3.125,.675)--(-3.119,.691);
\filldraw[fill opacity=0.8,fill=gray!20](-3.145,.643)--(-3.125,.675)--(-3.111,.661)--(-3.135,.631)--cycle;
\filldraw[fill opacity=0.8,fill=gray!20](-2.915,.746)--(-2.918,.77)--(-2.994,.776)--(-2.969,.75)--cycle;
\filldraw[fill opacity=0.8,fill=gray!20](-2.779,.785)--(-2.75,.825)--(-2.824,.811)--(-2.839,.774)--cycle;
\filldraw[fill opacity=0.8,fill=gray!20](-3.225,.668)--(-3.272,.682)--(-3.25,.686)--(-3.225,.668)--cycle;
\filldraw[fill opacity=0.5,fill=gray!20](-1.104,2.717)--(-1.272,2.805)--(-.994,3.23)--(-.857,3.095)--cycle;
\filldraw[fill opacity=0.8,fill=gray!20,draw=none](-2.703,1.41)--(-2.698,1.395)--(-2.695,1.376)--cycle;
\filldraw[fill opacity=0.8,fill=gray!20,draw=none](-2.698,1.395)--(-2.692,1.377)--(-2.689,1.34)--cycle;
\filldraw[fill opacity=0.8,fill=gray!20,draw=none](-2.712,1.447)--(-2.677,1.172)--(-2.664,1.203)--(-2.684,1.361)--cycle;
\draw(-2.712,1.447)--(-2.677,1.172)--(-2.664,1.203)--(-2.684,1.361);
\filldraw[fill opacity=0.8,fill=gray!20](-2.373,1.706)--(-2.38,1.744)--(-2.444,1.741)--(-2.445,1.703)--cycle;
\filldraw[fill opacity=0.8,fill=gray!20,draw=none](-3.285,.855)--(-3.283,.857)--(-3.284,.855)--cycle;
\draw(-3.283,.857)--(-3.284,.855)--(-3.285,.855);
\filldraw[fill opacity=0.8,fill=gray!20](-3.172,.617)--(-3.145,.643)--(-3.135,.631)--(-3.165,.609)--cycle;
\filldraw[fill opacity=0.8,fill=gray!20](-3.238,.591)--(-3.255,.603)--(-3.236,.604)--(-3.238,.591)--cycle;
\filldraw[fill opacity=0.8,fill=gray!20](-3.238,.591)--(-3.236,.604)--(-3.217,.603)--(-3.238,.591)--cycle;
\filldraw[fill opacity=0.8,fill=gray!20](-2.371,1.668)--(-2.373,1.706)--(-2.445,1.703)--(-2.445,1.665)--cycle;
\filldraw[fill opacity=0.8,fill=gray!20,draw=none](-3.285,.855)--(-3.324,.847)--(-3.299,.866)--(-3.271,.872)--(-3.283,.857)--cycle;
\draw(-3.285,.855)--(-3.324,.847)--(-3.299,.866)--(-3.271,.872)--(-3.283,.857);
\filldraw[fill opacity=0.8,fill=gray!20](-3.276,.596)--(-3.311,.611)--(-3.299,.619)--(-3.27,.6)--cycle;
\filldraw[fill opacity=0.8,fill=gray!20,draw=none](-3.092,1.047)--(-3.08,1.046)--(-3.087,1.053)--cycle;
\draw(-3.08,1.046)--(-3.087,1.053);
\filldraw[fill opacity=0.8,fill=gray!20](-3.139,1.062)--(-3.165,1.087)--(-3.127,1.078)--(-3.086,1.049)--cycle;
\filldraw[fill opacity=0.8,fill=gray!20](-3.225,.668)--(-3.194,.685)--(-3.174,.68)--(-3.225,.668)--cycle;
\filldraw[fill opacity=0.8,fill=gray!20,draw=none](-2.681,1.393)--(-2.673,1.401)--(-2.682,1.421)--cycle;
\draw(-2.681,1.393)--(-2.673,1.401);
\filldraw[fill opacity=0.8,fill=gray!20,draw=none](-3.029,.784)--(-3.027,.784)--(-3.029,.785)--cycle;
\draw(-3.029,.784)--(-3.027,.784);
\filldraw[fill opacity=0.8,fill=gray!20](-2.86,.748)--(-2.839,.774)--(-2.918,.77)--(-2.915,.746)--cycle;
\filldraw[fill opacity=0.8,fill=gray!20,draw=none](-3.283,.857)--(-3.271,.872)--(-3.264,.872)--cycle;
\draw(-3.283,.857)--(-3.271,.872)--(-3.264,.872);
\filldraw[fill opacity=0.8,fill=gray!20,draw=none](-3.029,.785)--(-3.027,.784)--(-2.994,.776)--(-3.004,.795)--(-3.023,.794)--cycle;
\draw(-3.027,.784)--(-2.994,.776)--(-3.004,.795);
\filldraw[fill opacity=0.8,fill=gray!20](-3.238,.591)--(-3.27,.6)--(-3.255,.603)--(-3.238,.591)--cycle;
\filldraw[fill opacity=0.8,fill=gray!20,draw=none](-3.205,.72)--(-3.225,.738)--(-3.234,.745)--(-3.289,.769)--(-3.286,.764)--(-3.271,.748)--cycle;
\draw(-3.205,.72)--(-3.225,.738)--(-3.234,.745);
\draw(-3.289,.769)--(-3.286,.764)--(-3.271,.748);
\filldraw[fill opacity=0.8,fill=gray!20,draw=none](-3.107,.748)--(-3.108,.757)--(-3.102,.774)--(-3.097,.769)--(-3.092,.732)--cycle;
\draw(-3.102,.774)--(-3.097,.769)--(-3.092,.732)--(-3.107,.748)--(-3.108,.757);
\filldraw[fill opacity=0.8,fill=gray!20](-3.174,.68)--(-3.127,.707)--(-3.115,.695)--(-3.168,.674)--cycle;
\filldraw[fill opacity=0.8,fill=gray!20,draw=none](-3.363,.709)--(-3.379,.699)--(-3.384,.732)--(-3.381,.734)--cycle;
\draw(-3.363,.709)--(-3.379,.699)--(-3.384,.732);
\filldraw[fill opacity=0.8,fill=gray!20,draw=none](-3.053,.846)--(-3.018,.888)--(-3.018,.892)--(-3.017,.911)--(-3.052,.869)--cycle;
\draw(-3.053,.846)--(-3.018,.888);
\draw(-3.017,.911)--(-3.052,.869);
\filldraw[fill opacity=0.8,fill=gray!20,draw=none](-3.062,.831)--(-3.022,.846)--(-3.025,.861)--(-3.042,.866)--cycle;
\draw(-3.022,.846)--(-3.025,.861)--(-3.042,.866);
\filldraw[fill opacity=0.8,fill=gray!20,draw=none](-3.097,.869)--(-3.102,.82)--(-3.102,.794)--(-3.101,.798)--(-3.094,.873)--cycle;
\draw(-3.101,.798)--(-3.094,.873)--(-3.097,.869);
\filldraw[fill opacity=0.8,fill=gray!20,draw=none](-3.102,.794)--(-3.098,.792)--(-3.059,.84)--(-3.052,.869)--(-3.102,.809)--cycle;
\draw(-3.098,.792)--(-3.059,.84);
\draw(-3.052,.869)--(-3.102,.809);
\filldraw[fill opacity=0.8,fill=gray!20,draw=none](-3.019,.815)--(-3.013,.814)--(-3.016,.822)--cycle;
\draw(-3.019,.815)--(-3.013,.814)--(-3.016,.822);
\filldraw[fill opacity=0.8,fill=gray!20,draw=none](-3.102,.809)--(-3.052,.869)--(-3.052,.903)--(-3.099,.847)--cycle;
\draw(-3.102,.809)--(-3.052,.869);
\draw(-3.052,.903)--(-3.099,.847);
\filldraw[fill opacity=0.8,fill=gray!20,draw=none](-3.014,.824)--(-3.015,.825)--(-3.011,.838)--(-3.013,.827)--cycle;
\draw(-3.014,.824)--(-3.015,.825);
\draw(-3.011,.838)--(-3.013,.827);
\filldraw[fill opacity=0.8,fill=gray!20,draw=none](-3.014,.833)--(-3.013,.827)--(-3.006,.879)--(-3.01,.876)--cycle;
\draw(-3.013,.827)--(-3.006,.879)--(-3.01,.876);
\filldraw[fill opacity=0.8,fill=gray!20,draw=none](-3.011,.819)--(-2.998,.849)--(-3.001,.86)--(-3.012,.86)--cycle;
\draw(-3.001,.86)--(-3.012,.86);
\filldraw[fill opacity=0.8,fill=gray!20,draw=none](-3.016,.824)--(-3.015,.825)--(-3.014,.825)--cycle;
\draw(-3.015,.825)--(-3.014,.825);
\filldraw[fill opacity=0.8,fill=gray!20,draw=none](-3.018,.888)--(-2.895,1.037)--(-2.893,1.053)--(-2.906,1.044)--(-3.011,.918)--cycle;
\draw(-3.018,.888)--(-2.895,1.037);
\draw(-2.906,1.044)--(-3.011,.918);
\filldraw[fill opacity=0.8,fill=gray!20,draw=none](-3.011,.918)--(-2.906,1.044)--(-2.948,1.029)--(-3.02,.942)--cycle;
\draw(-3.011,.918)--(-2.906,1.044);
\draw(-2.948,1.029)--(-3.02,.942);
\filldraw[fill opacity=0.8,fill=gray!20,draw=none](-3.018,.892)--(-3.011,.918)--(-3.017,.911)--cycle;
\draw(-3.011,.918)--(-3.017,.911);
\filldraw[fill opacity=0.8,fill=gray!20,draw=none](-3.017,.911)--(-3.011,.918)--(-3.02,.942)--cycle;
\draw(-3.017,.911)--(-3.011,.918);
\filldraw[fill opacity=0.8,fill=gray!20,draw=none](-3.018,.892)--(-3.017,.939)--(-3.013,.935)--(-3.006,.879)--cycle;
\draw(-3.017,.939)--(-3.013,.935)--(-3.006,.879)--(-3.018,.892);
\filldraw[fill opacity=0.8,fill=gray!20,draw=none](-3.017,.911)--(-3.02,.942)--(-3.017,.939)--cycle;
\draw(-3.02,.942)--(-3.017,.939);
\filldraw[fill opacity=0.8,fill=gray!20,draw=none](-3.012,.86)--(-3.01,.876)--(-3.017,.915)--(-3.03,.916)--(-3.025,.861)--cycle;
\draw(-3.017,.915)--(-3.03,.916)--(-3.025,.861)--(-3.012,.86);
\filldraw[fill opacity=0.8,fill=gray!20,draw=none](-3.016,.845)--(-3.014,.833)--(-3.01,.876)--(-3.014,.874)--cycle;
\draw(-3.01,.876)--(-3.014,.874);
\filldraw[fill opacity=0.8,fill=gray!20,draw=none](-3.124,.817)--(-3.099,.847)--(-3.097,.869)--(-3.116,.869)--(-3.147,.832)--cycle;
\draw(-3.124,.817)--(-3.099,.847);
\draw(-3.116,.869)--(-3.147,.832);
\filldraw[fill opacity=0.8,fill=gray!20,draw=none](-3.097,.869)--(-3.1,.864)--(-3.102,.843)--(-3.102,.82)--cycle;
\draw(-3.097,.869)--(-3.1,.864)--(-3.102,.843);
\filldraw[fill opacity=0.8,fill=gray!20,draw=none](-3.128,.793)--(-3.117,.8)--(-3.142,.829)--(-3.152,.822)--(-3.136,.797)--cycle;
\draw(-3.128,.793)--(-3.117,.8);
\draw(-3.142,.829)--(-3.152,.822)--(-3.136,.797);
\filldraw[fill opacity=0.8,fill=gray!20,draw=none](-3.126,.858)--(-3.129,.818)--(-3.107,.79)--(-3.1,.864)--cycle;
\draw(-3.107,.79)--(-3.1,.864)--(-3.126,.858)--(-3.129,.818);
\filldraw[fill opacity=0.8,fill=gray!20](-3.174,.774)--(-2.858,.848)--(-2.849,.883)--(-3.166,.81)--cycle;
\filldraw[fill opacity=0.8,fill=gray!20,draw=none](-2.728,1.776)--(-2.763,1.784)--(-2.779,1.798)--(-2.72,1.796)--cycle;
\filldraw[fill opacity=0.8,fill=gray!20,draw=none](-3.014,1.138)--(-3.007,1.13)--(-2.979,1.146)--cycle;
\draw(-3.014,1.138)--(-3.007,1.13)--(-2.979,1.146);
\filldraw[fill opacity=0.8,fill=gray!20,draw=none](-3.414,.968)--(-3.406,.997)--(-3.392,1.006)--cycle;
\draw(-3.406,.997)--(-3.392,1.006);
\filldraw[fill opacity=0.8,fill=gray!20,draw=none](-3.386,1.025)--(-3.397,1.003)--(-3.406,.997)--cycle;
\draw(-3.397,1.003)--(-3.406,.997);
\filldraw[fill opacity=0.8,fill=gray!20,draw=none](-3.303,.976)--(-3.323,.968)--(-3.411,.99)--(-3.386,1.025)--(-3.376,1.022)--cycle;
\draw(-3.323,.968)--(-3.411,.99);
\draw(-3.386,1.025)--(-3.376,1.022);
\filldraw[fill opacity=0.8,fill=gray!20,draw=none](-3.408,.989)--(-3.414,.968)--(-3.425,.948)--(-3.426,.947)--cycle;
\draw(-3.425,.948)--(-3.426,.947);
\filldraw[fill opacity=0.8,fill=gray!20,draw=none](-3.306,.945)--(-3.357,.926)--(-3.426,.943)--(-3.426,.947)--(-3.408,.989)--(-3.319,.967)--cycle;
\draw(-3.357,.926)--(-3.426,.943);
\draw(-3.408,.989)--(-3.319,.967);
\filldraw[fill opacity=0.8,fill=gray!20](-3.238,.591)--(-3.217,.603)--(-3.204,.6)--(-3.238,.591)--cycle;
\filldraw[fill opacity=0.8,fill=gray!20,draw=none](-3.359,.804)--(-3.358,.806)--(-3.358,.801)--cycle;
\draw(-3.358,.806)--(-3.358,.801);
\filldraw[fill opacity=0.8,fill=gray!20,draw=none](-3.36,.81)--(-3.358,.808)--(-3.358,.806)--(-3.359,.804)--cycle;
\draw(-3.358,.808)--(-3.358,.806);
\filldraw[fill opacity=0.8,fill=gray!20,draw=none](-3.36,.81)--(-3.356,.826)--(-3.358,.808)--cycle;
\draw(-3.356,.826)--(-3.358,.808);
\filldraw[fill opacity=0.8,fill=gray!20,draw=none](-3.354,.807)--(-3.359,.804)--(-3.36,.81)--(-3.345,.819)--cycle;
\draw(-3.36,.81)--(-3.345,.819);
\filldraw[fill opacity=0.8,fill=gray!20,draw=none](-3.359,.804)--(-3.358,.801)--(-3.378,.773)--(-3.379,.773)--(-3.367,.8)--cycle;
\draw(-3.378,.773)--(-3.379,.773)--(-3.367,.8);
\filldraw[fill opacity=0.8,fill=gray!20,draw=none](-3.284,.778)--(-3.289,.769)--(-3.294,.776)--(-3.292,.778)--cycle;
\draw(-3.289,.769)--(-3.294,.776)--(-3.292,.778);
\filldraw[fill opacity=0.8,fill=gray!20,draw=none](-3.359,.804)--(-3.358,.801)--(-3.361,.769)--(-3.366,.775)--(-3.364,.796)--cycle;
\draw(-3.358,.801)--(-3.361,.769);
\draw(-3.366,.775)--(-3.364,.796);
\filldraw[fill opacity=0.8,fill=gray!20,draw=none](-3.354,.807)--(-3.358,.801)--(-3.359,.804)--cycle;
\filldraw[fill opacity=0.8,fill=gray!20,draw=none](-3.283,.781)--(-3.284,.778)--(-3.292,.778)--(-3.292,.779)--cycle;
\draw(-3.292,.778)--(-3.292,.779);
\filldraw[fill opacity=0.8,fill=gray!20,draw=none](-3.36,.81)--(-3.344,.824)--(-3.355,.813)--(-3.36,.81)--cycle;
\draw(-3.355,.813)--(-3.36,.81);
\filldraw[fill opacity=0.8,fill=gray!20,draw=none](-3.359,.804)--(-3.367,.8)--(-3.365,.806)--(-3.363,.808)--(-3.36,.81)--cycle;
\draw(-3.367,.8)--(-3.365,.806);
\draw(-3.363,.808)--(-3.36,.81);
\filldraw[fill opacity=0.8,fill=gray!20,draw=none](-3.36,.81)--(-3.359,.804)--(-3.364,.796)--cycle;
\filldraw[fill opacity=0.8,fill=gray!20,draw=none](-3.36,.81)--(-3.36,.81)--(-3.363,.808)--cycle;
\draw(-3.36,.81)--(-3.363,.808);
\filldraw[fill opacity=0.8,fill=gray!20,draw=none](-3.361,.769)--(-3.363,.752)--(-3.369,.743)--(-3.366,.775)--cycle;
\draw(-3.361,.769)--(-3.363,.752)--(-3.369,.743)--(-3.366,.775);
\filldraw[fill opacity=0.8,fill=gray!20,draw=none](-3.359,.776)--(-3.364,.796)--(-3.366,.775)--cycle;
\draw(-3.364,.796)--(-3.366,.775);
\filldraw[fill opacity=0.8,fill=gray!20,draw=none](-3.365,.778)--(-3.365,.796)--(-3.366,.798)--(-3.368,.798)--(-3.378,.776)--cycle;
\draw(-3.368,.798)--(-3.378,.776);
\filldraw[fill opacity=0.8,fill=gray!20,draw=none](-3.366,.798)--(-3.367,.8)--(-3.368,.798)--cycle;
\draw(-3.367,.8)--(-3.368,.798);
\filldraw[fill opacity=0.8,fill=gray!20,draw=none](-3.365,.796)--(-3.365,.806)--(-3.367,.8)--cycle;
\draw(-3.365,.806)--(-3.367,.8);
\filldraw[fill opacity=0.8,fill=gray!20,draw=none](-3.281,.775)--(-3.395,.803)--(-3.356,.826)--(-3.292,.811)--cycle;
\draw(-3.356,.826)--(-3.292,.811)--(-3.281,.775)--(-3.395,.803);
\filldraw[fill opacity=0.8,fill=gray!20,draw=none](-3.108,1.001)--(-3.102,1.011)--(-3.096,1.018)--cycle;
\draw(-3.102,1.011)--(-3.096,1.018);
\filldraw[fill opacity=0.8,fill=gray!20,draw=none](-3.103,.776)--(-3.107,.725)--(-3.121,.734)--(-3.119,.754)--cycle;
\draw(-3.103,.776)--(-3.107,.725)--(-3.121,.734)--(-3.119,.754);
\filldraw[fill opacity=0.8,fill=gray!20](-3.204,.6)--(-3.172,.617)--(-3.165,.609)--(-3.2,.595)--cycle;
\filldraw[fill opacity=0.8,fill=gray!20,draw=none](-2.779,2.018)--(-2.759,1.978)--(-2.788,1.954)--(-2.804,2.018)--cycle;
\filldraw[fill opacity=0.8,fill=gray!20,draw=none](-3.165,.737)--(-3.141,.749)--(-3.153,.734)--cycle;
\draw(-3.141,.749)--(-3.153,.734);
\filldraw[fill opacity=0.8,fill=gray!20,draw=none](-3.258,.892)--(-3.249,.892)--(-3.266,.9)--(-3.336,.902)--cycle;
\filldraw[fill opacity=0.8,fill=gray!20,draw=none](-3.11,1.043)--(-3.104,1.053)--(-3.086,1.049)--(-3.083,1.044)--cycle;
\draw(-3.104,1.053)--(-3.086,1.049)--(-3.083,1.044);
\filldraw[fill opacity=0.8,fill=gray!20,draw=none](-2.81,1.98)--(-2.781,1.92)--(-2.763,1.784)--(-2.799,1.793)--(-2.839,1.956)--cycle;
\draw(-2.781,1.92)--(-2.763,1.784);
\filldraw[fill opacity=0.8,fill=gray!20,draw=none](-3.205,.888)--(-3.133,.974)--(-3.163,.934)--(-3.197,.892)--cycle;
\draw(-3.205,.888)--(-3.133,.974);
\draw(-3.163,.934)--(-3.197,.892);
\filldraw[fill opacity=0.8,fill=gray!20,draw=none](-3.055,.779)--(-3.047,.785)--(-3.035,.772)--(-3.053,.75)--cycle;
\draw(-3.047,.785)--(-3.035,.772)--(-3.053,.75);
\filldraw[fill opacity=0.8,fill=gray!20,draw=none](-3.033,1.053)--(-2.996,1.098)--(-3.035,1.079)--(-3.059,1.05)--cycle;
\draw(-3.033,1.053)--(-2.996,1.098);
\draw(-3.035,1.079)--(-3.059,1.05);
\filldraw[fill opacity=0.8,fill=gray!20](-2.994,1.079)--(-2.969,1.121)--(-3.007,1.13)--(-3.047,1.092)--cycle;
\filldraw[fill opacity=0.8,fill=gray!20,draw=none](-3.059,1.05)--(-3.035,1.079)--(-3.059,1.063)--(-3.069,1.05)--cycle;
\draw(-3.059,1.05)--(-3.035,1.079);
\draw(-3.059,1.063)--(-3.069,1.05);
\filldraw[fill opacity=0.8,fill=gray!20,draw=none](-3.071,1.007)--(-3.034,1.052)--(-3.034,1.053)--(-3.059,1.05)--(-3.09,1.012)--cycle;
\draw(-3.071,1.007)--(-3.034,1.052);
\draw(-3.059,1.05)--(-3.09,1.012);
\filldraw[fill opacity=0.8,fill=gray!20,draw=none](-3.034,1.052)--(-3.033,1.053)--(-3.034,1.053)--cycle;
\draw(-3.034,1.052)--(-3.033,1.053);
\filldraw[fill opacity=0.8,fill=gray!20,draw=none](-3.091,1.013)--(-3.096,1.018)--(-3.133,.974)--cycle;
\draw(-3.096,1.018)--(-3.133,.974);
\filldraw[fill opacity=0.8,fill=gray!20,draw=none](-3.081,1.041)--(-3.096,1.018)--(-3.09,1.012)--(-3.08,1.041)--cycle;
\draw(-3.09,1.012)--(-3.08,1.041);
\filldraw[fill opacity=0.8,fill=gray!20,draw=none](-3.204,.876)--(-3.199,.882)--(-3.205,.888)--(-3.214,.877)--cycle;
\draw(-3.204,.876)--(-3.199,.882);
\draw(-3.205,.888)--(-3.214,.877);
\filldraw[fill opacity=0.8,fill=gray!20,draw=none](-3.017,1.035)--(-3.016,1.034)--(-2.995,1.059)--(-3.033,1.053)--(-3.034,1.052)--cycle;
\draw(-3.016,1.034)--(-2.995,1.059);
\draw(-3.033,1.053)--(-3.034,1.052);
\filldraw[fill opacity=0.8,fill=gray!20,draw=none](-3.017,1.035)--(-3.034,1.052)--(-3.044,1.04)--cycle;
\draw(-3.034,1.052)--(-3.044,1.04);
\filldraw[fill opacity=0.8,fill=gray!20,draw=none](-3.054,1.048)--(-3.004,1.054)--(-2.994,1.079)--(-3.047,1.092)--(-3.06,1.073)--cycle;
\draw(-3.004,1.054)--(-2.994,1.079)--(-3.047,1.092)--(-3.06,1.073);
\filldraw[fill opacity=0.8,fill=gray!20,draw=none](-2.915,1.117)--(-2.913,1.145)--(-2.954,1.137)--(-2.969,1.121)--cycle;
\draw(-2.954,1.137)--(-2.969,1.121)--(-2.915,1.117)--(-2.913,1.145);
\filldraw[fill opacity=0.8,fill=gray!20,draw=none](-3.02,.942)--(-2.948,1.029)--(-2.986,1.025)--(-3.026,.978)--cycle;
\draw(-3.02,.942)--(-2.948,1.029);
\draw(-2.986,1.025)--(-3.026,.978);
\filldraw[fill opacity=0.8,fill=gray!20,draw=none](-3.016,1.034)--(-2.938,1.012)--(-2.876,1.027)--(-2.898,1.062)--(-3.017,1.035)--cycle;
\draw(-2.938,1.012)--(-2.876,1.027)--(-2.898,1.062)--(-3.017,1.035);
\filldraw[fill opacity=0.8,fill=gray!20,draw=none](-2.948,1.029)--(-2.892,1.096)--(-2.898,1.131)--(-2.986,1.025)--cycle;
\draw(-2.948,1.029)--(-2.892,1.096);
\draw(-2.898,1.131)--(-2.986,1.025);
\filldraw[fill opacity=0.8,fill=gray!20](-2.918,1.073)--(-2.915,1.117)--(-2.969,1.121)--(-2.994,1.079)--cycle;
\filldraw[fill opacity=0.8,fill=gray!20,draw=none](-2.973,1.158)--(-2.933,1.09)--(-3.021,1.082)--(-3.111,1.086)--cycle;
\draw(-2.933,1.09)--(-3.021,1.082)--(-3.111,1.086);
\filldraw[fill opacity=0.8,fill=gray!20](-3.238,.591)--(-3.276,.596)--(-3.27,.6)--(-3.238,.591)--cycle;
\filldraw[fill opacity=0.8,fill=gray!20,draw=none](-3.087,1.053)--(-3.08,1.046)--(-3.078,1.046)--(-3.06,1.073)--cycle;
\draw(-3.087,1.053)--(-3.08,1.046);
\draw(-3.078,1.046)--(-3.06,1.073);
\filldraw[fill opacity=0.8,fill=gray!20](-3.225,.668)--(-3.282,.676)--(-3.272,.682)--(-3.225,.668)--cycle;
\filldraw[fill opacity=0.8,fill=gray!20,draw=none](-2.855,2.021)--(-2.839,1.956)--(-2.848,1.948)--(-2.857,2.021)--cycle;
\draw(-2.848,1.948)--(-2.857,2.021);
\filldraw[fill opacity=0.8,fill=gray!20,draw=none](-3.312,2.039)--(-3.294,1.928)--(-3.315,1.969)--(-3.324,2.041)--cycle;
\draw(-3.315,1.969)--(-3.324,2.041);
\filldraw[fill opacity=0.8,fill=gray!20,draw=none](-3.312,1.976)--(-3.315,1.969)--(-3.207,1.127)--(-3.179,1.102)--(-3.232,1.516)--cycle;
\draw(-3.315,1.969)--(-3.207,1.127)--(-3.179,1.102)--(-3.232,1.516);
\filldraw[fill opacity=0.8,fill=gray!20,draw=none](-3.214,1.008)--(-3.236,1.014)--(-3.213,1.014)--(-3.189,1.008)--cycle;
\draw(-3.213,1.014)--(-3.189,1.008)--(-3.214,1.008)--(-3.236,1.014);
\filldraw[fill opacity=0.8,fill=gray!20,draw=none](-3.238,1.008)--(-3.145,1.03)--(-3.124,1.04)--(-3.26,1.008)--cycle;
\draw(-3.124,1.04)--(-3.26,1.008)--(-3.238,1.008)--(-3.145,1.03);
\filldraw[fill opacity=0.8,fill=gray!20](-3.238,.591)--(-3.204,.6)--(-3.2,.595)--(-3.238,.591)--cycle;
\filldraw[fill opacity=0.8,fill=gray!20,draw=none](-3.153,.734)--(-3.141,.749)--(-3.118,.764)--(-3.147,.729)--cycle;
\draw(-3.153,.734)--(-3.141,.749);
\draw(-3.118,.764)--(-3.147,.729);
\filldraw[fill opacity=0.8,fill=gray!20](-3.225,.668)--(-3.174,.68)--(-3.168,.674)--(-3.225,.668)--cycle;
\filldraw[fill opacity=0.8,fill=gray!20,draw=none](-3.356,.828)--(-3.356,.826)--(-3.359,.827)--cycle;
\draw(-3.356,.826)--(-3.359,.827);
\filldraw[fill opacity=0.8,fill=gray!20,draw=none](-3.395,.803)--(-3.422,.81)--(-3.428,.82)--(-3.359,.827)--(-3.356,.826)--cycle;
\draw(-3.395,.803)--(-3.422,.81);
\draw(-3.359,.827)--(-3.356,.826);
\filldraw[fill opacity=0.8,fill=gray!20,draw=none](-3.428,.82)--(-3.436,.846)--(-3.359,.827)--cycle;
\draw(-3.436,.846)--(-3.359,.827);
\filldraw[fill opacity=0.8,fill=gray!20,draw=none](-3.355,.837)--(-3.356,.826)--(-3.36,.81)--(-3.361,.811)--cycle;
\draw(-3.355,.837)--(-3.356,.826);
\filldraw[fill opacity=0.8,fill=gray!20,draw=none](-3.344,.824)--(-3.363,.808)--(-3.364,.807)--(-3.341,.836)--(-3.324,.847)--(-3.335,.833)--cycle;
\draw(-3.363,.808)--(-3.364,.807)--(-3.341,.836)--(-3.324,.847)--(-3.335,.833);
\filldraw[fill opacity=0.5,fill=gray!20](-1.259,2.291)--(-1.447,2.325)--(-1.272,2.805)--(-1.104,2.717)--cycle;
\filldraw[fill opacity=0.8,fill=gray!20,draw=none](-2.969,1.151)--(-2.971,1.156)--(-2.95,1.152)--cycle;
\filldraw[fill opacity=0.8,fill=gray!20,draw=none](-3.047,.785)--(-3.044,.787)--(-3.034,.776)--(-3.035,.772)--cycle;
\draw(-3.034,.776)--(-3.035,.772)--(-3.047,.785);
\filldraw[fill opacity=0.8,fill=gray!20,draw=none](-3.064,2.396)--(-3.107,2.407)--(-3.111,2.437)--cycle;
\draw(-3.107,2.407)--(-3.111,2.437);
\filldraw[fill opacity=0.8,fill=gray!20](-2.75,.825)--(-2.732,.875)--(-2.814,.859)--(-2.824,.811)--cycle;
\filldraw[fill opacity=0.8,fill=gray!20,draw=none](-3.29,2.036)--(-3.287,2.005)--(-3.313,2.037)--cycle;
\draw(-3.29,2.036)--(-3.287,2.005);
\filldraw[fill opacity=0.8,fill=gray!20](-3.274,1.089)--(-3.25,1.1)--(-3.222,1.101)--(-3.218,1.091)--cycle;
\filldraw[fill opacity=0.8,fill=gray!20](-3.218,1.091)--(-3.222,1.101)--(-3.194,1.099)--(-3.165,1.087)--cycle;
\filldraw[fill opacity=0.8,fill=gray!20,draw=none](-3.236,1.733)--(-3.265,1.715)--(-3.289,1.897)--cycle;
\draw(-3.265,1.715)--(-3.289,1.897);
\filldraw[fill opacity=0.8,fill=gray!20,draw=none](-3.023,.794)--(-3.004,.795)--(-3.013,.812)--cycle;
\draw(-3.004,.795)--(-3.013,.812);
\filldraw[fill opacity=0.8,fill=gray!20,draw=none](-3.264,.872)--(-3.271,.872)--(-3.255,.879)--(-3.243,.88)--cycle;
\draw(-3.264,.872)--(-3.271,.872)--(-3.255,.879)--(-3.243,.88);
\filldraw[fill opacity=0.8,fill=gray!20,draw=none](-3.278,.764)--(-3.294,.777)--(-3.294,.776)--(-3.289,.769)--cycle;
\draw(-3.294,.777)--(-3.294,.776)--(-3.289,.769);
\filldraw[fill opacity=0.8,fill=gray!20,draw=none](-3.044,.787)--(-3.042,.79)--(-3.028,.788)--(-3.034,.776)--cycle;
\draw(-3.028,.788)--(-3.034,.776);
\filldraw[fill opacity=0.8,fill=gray!20](-3.299,.866)--(-3.27,.876)--(-3.255,.879)--(-3.271,.872)--cycle;
\filldraw[fill opacity=0.8,fill=gray!20,draw=none](-3.092,.713)--(-3.078,.737)--(-3.07,.728)--(-3.091,.713)--cycle;
\draw(-3.078,.737)--(-3.07,.728)--(-3.091,.713);
\filldraw[fill opacity=0.8,fill=gray!20,draw=none](-2.893,1.053)--(-2.891,1.063)--(-2.906,1.044)--cycle;
\draw(-2.891,1.063)--(-2.906,1.044);
\filldraw[fill opacity=0.8,fill=gray!20,draw=none](-2.901,2.345)--(-2.975,2.374)--(-3.017,2.702)--cycle;
\draw(-2.975,2.374)--(-3.017,2.702);
\filldraw[fill opacity=0.8,fill=gray!20,draw=none](-3.25,2.035)--(-3.232,1.893)--(-3.292,1.925)--(-3.307,2.039)--cycle;
\draw(-3.25,2.035)--(-3.232,1.893);
\draw(-3.292,1.925)--(-3.307,2.039);
\filldraw[fill opacity=0.8,fill=gray!20,draw=none](-3.104,.703)--(-3.097,.709)--(-3.095,.707)--(-3.097,.695)--cycle;
\draw(-3.095,.707)--(-3.097,.695)--(-3.104,.703);
\filldraw[fill opacity=0.8,fill=gray!20,draw=none](-3.35,.835)--(-3.355,.828)--(-3.356,.829)--(-3.355,.841)--cycle;
\draw(-3.356,.829)--(-3.355,.841);
\filldraw[fill opacity=0.8,fill=gray!20,draw=none](-2.829,2.02)--(-2.81,1.98)--(-2.839,1.956)--(-2.855,2.021)--cycle;
\filldraw[fill opacity=0.8,fill=gray!20](-3.341,.836)--(-3.311,.859)--(-3.299,.866)--(-3.324,.847)--cycle;
\filldraw[fill opacity=0.8,fill=gray!20](-3.238,.591)--(-3.272,.592)--(-3.276,.596)--(-3.238,.591)--cycle;
\filldraw[fill opacity=0.8,fill=gray!20](-3.272,.592)--(-3.303,.603)--(-3.311,.611)--(-3.276,.596)--cycle;
\filldraw[fill opacity=0.8,fill=gray!20,draw=none](-3.186,3.023)--(-3.111,2.437)--(-3.243,2.484)--(-3.309,2.998)--cycle;
\draw(-3.243,2.484)--(-3.309,2.998)--(-3.186,3.023)--(-3.111,2.437);
\filldraw[fill opacity=0.8,fill=gray!20,draw=none](-3.283,.781)--(-3.292,.779)--(-3.275,.799)--cycle;
\draw(-3.292,.779)--(-3.275,.799);
\filldraw[fill opacity=0.8,fill=gray!20,draw=none](-3.353,.896)--(-3.35,.9)--(-3.355,.837)--(-3.361,.811)--(-3.363,.812)--(-3.356,.887)--cycle;
\draw(-3.353,.896)--(-3.35,.9)--(-3.355,.837);
\draw(-3.363,.812)--(-3.356,.887);
\filldraw[fill opacity=0.8,fill=gray!20,draw=none](-3.353,.896)--(-3.356,.887)--(-3.356,.891)--cycle;
\draw(-3.356,.887)--(-3.356,.891)--(-3.353,.896);
\filldraw[fill opacity=0.8,fill=gray!20,draw=none](-3.258,.892)--(-3.336,.902)--(-3.34,.902)--(-3.35,.9)--(-3.356,.891)--(-3.352,.888)--cycle;
\draw(-3.34,.902)--(-3.35,.9)--(-3.356,.891)--(-3.352,.888);
\filldraw[fill opacity=0.8,fill=gray!20,draw=none](-3.445,.885)--(-3.377,.867)--(-3.374,.831)--(-3.436,.846)--cycle;
\draw(-3.374,.831)--(-3.436,.846);
\filldraw[fill opacity=0.8,fill=gray!20](-3.311,.859)--(-3.276,.872)--(-3.27,.876)--(-3.299,.866)--cycle;
\filldraw[fill opacity=0.8,fill=gray!20,draw=none](-3.243,.88)--(-3.255,.879)--(-3.238,.877)--(-3.238,.877)--(-3.236,.879)--cycle;
\draw(-3.243,.88)--(-3.255,.879)--(-3.238,.877)--(-3.238,.877)--(-3.236,.879);
\filldraw[fill opacity=0.8,fill=gray!20](-3.27,.876)--(-3.238,.877)--(-3.238,.877)--(-3.255,.879)--cycle;
\filldraw[fill opacity=0.8,fill=gray!20,draw=none](-3.236,.879)--(-3.238,.877)--(-3.238,.877)--(-3.23,.877)--cycle;
\draw(-3.236,.879)--(-3.238,.877)--(-3.238,.877)--(-3.23,.877);
\filldraw[fill opacity=0.8,fill=gray!20](-3.276,.872)--(-3.238,.877)--(-3.238,.877)--(-3.27,.876)--cycle;
\filldraw[fill opacity=0.8,fill=gray!20,draw=none](-3.23,.877)--(-3.238,.877)--(-3.238,.877)--(-3.227,.876)--cycle;
\draw(-3.23,.877)--(-3.238,.877)--(-3.238,.877)--(-3.227,.876);
\filldraw[fill opacity=0.8,fill=gray!20](-3.303,.85)--(-3.272,.868)--(-3.276,.872)--(-3.311,.859)--cycle;
\filldraw[fill opacity=0.8,fill=gray!20,draw=none](-3.227,.876)--(-3.238,.877)--(-3.238,.877)--(-3.223,.875)--cycle;
\draw(-3.227,.876)--(-3.238,.877)--(-3.238,.877)--(-3.223,.875);
\filldraw[fill opacity=0.8,fill=gray!20](-3.272,.868)--(-3.238,.877)--(-3.238,.877)--(-3.276,.872)--cycle;
\filldraw[fill opacity=0.8,fill=gray!20,draw=none](-3.223,.875)--(-3.238,.877)--(-3.238,.877)--(-3.218,.871)--cycle;
\draw(-3.223,.875)--(-3.238,.877)--(-3.238,.877)--(-3.218,.871);
\filldraw[fill opacity=0.8,fill=gray!20,draw=none](-3.212,.866)--(-3.218,.871)--(-3.238,.877)--(-3.238,.877)--(-3.221,.864)--cycle;
\draw(-3.218,.871)--(-3.238,.877)--(-3.238,.877)--(-3.221,.864)--(-3.212,.866);
\filldraw[fill opacity=0.8,fill=gray!20](-3.221,.864)--(-3.238,.877)--(-3.238,.877)--(-3.24,.864)--cycle;
\filldraw[fill opacity=0.8,fill=gray!20](-3.259,.865)--(-3.238,.877)--(-3.238,.877)--(-3.272,.868)--cycle;
\filldraw[fill opacity=0.8,fill=gray!20](-3.24,.864)--(-3.238,.877)--(-3.238,.877)--(-3.259,.865)--cycle;
\filldraw[fill opacity=0.8,fill=gray!20](-3.278,.844)--(-3.259,.865)--(-3.272,.868)--(-3.303,.85)--cycle;
\filldraw[fill opacity=0.8,fill=gray!20,draw=none](-3.2,.844)--(-3.212,.866)--(-3.221,.864)--(-3.205,.843)--cycle;
\draw(-3.212,.866)--(-3.221,.864)--(-3.205,.843)--(-3.2,.844);
\filldraw[fill opacity=0.8,fill=gray!20](-3.242,.842)--(-3.24,.864)--(-3.259,.865)--(-3.278,.844)--cycle;
\filldraw[fill opacity=0.8,fill=gray!20](-3.205,.843)--(-3.221,.864)--(-3.24,.864)--(-3.242,.842)--cycle;
\filldraw[fill opacity=0.8,fill=gray!20,draw=none](-3.166,.856)--(-3.245,.89)--(-3.258,.892)--(-3.352,.888)--(-3.342,.881)--(-3.311,.872)--(-3.266,.863)--(-3.215,.858)--cycle;
\draw(-3.352,.888)--(-3.342,.881)--(-3.311,.872)--(-3.266,.863)--(-3.215,.858)--(-3.166,.856);
\filldraw[fill opacity=0.8,fill=gray!20,draw=none](-3.426,.943)--(-3.426,.947)--(-3.425,.948)--cycle;
\draw(-3.426,.947)--(-3.425,.948);
\filldraw[fill opacity=0.8,fill=gray!20,draw=none](-3.426,.943)--(-3.448,.948)--(-3.453,.995)--(-3.451,1)--(-3.422,.992)--cycle;
\draw(-3.426,.943)--(-3.448,.948);
\draw(-3.451,1)--(-3.422,.992);
\filldraw[fill opacity=0.8,fill=gray!20,draw=none](-3.426,.943)--(-3.436,.903)--(-3.436,.941)--(-3.426,.947)--cycle;
\draw(-3.436,.941)--(-3.426,.947);
\filldraw[fill opacity=0.8,fill=gray!20,draw=none](-3.426,.947)--(-3.422,.992)--(-3.408,.989)--cycle;
\draw(-3.422,.992)--(-3.408,.989);
\filldraw[fill opacity=0.8,fill=gray!20,draw=none](-3.414,.974)--(-3.426,.947)--(-3.436,.941)--(-3.436,.942)--(-3.418,.984)--(-3.415,.989)--cycle;
\draw(-3.426,.947)--(-3.436,.941);
\draw(-3.436,.942)--(-3.418,.984);
\filldraw[fill opacity=0.8,fill=gray!20,draw=none](-3.442,.884)--(-3.484,.895)--(-3.482,.904)--(-3.438,.893)--cycle;
\draw(-3.482,.904)--(-3.438,.893);
\filldraw[fill opacity=0.8,fill=gray!20,draw=none](-3.44,.893)--(-3.469,.9)--(-3.466,.94)--(-3.461,.951)--(-3.444,.947)--cycle;
\draw(-3.44,.893)--(-3.469,.9);
\draw(-3.461,.951)--(-3.444,.947);
\filldraw[fill opacity=0.8,fill=gray!20,draw=none](-3.436,.903)--(-3.439,.888)--(-3.444,.885)--(-3.437,.94)--(-3.436,.941)--cycle;
\draw(-3.439,.888)--(-3.444,.885)--(-3.437,.94)--(-3.436,.941);
\filldraw[fill opacity=0.8,fill=gray!20,draw=none](-3.443,.873)--(-3.444,.885)--(-3.441,.887)--cycle;
\draw(-3.443,.873)--(-3.444,.885)--(-3.441,.887);
\filldraw[fill opacity=0.8,fill=gray!20,draw=none](-3.416,.801)--(-3.422,.81)--(-3.395,.803)--cycle;
\draw(-3.422,.81)--(-3.395,.803);
\filldraw[fill opacity=0.8,fill=gray!20](-3.415,.806)--(-3.421,.86)--(-3.444,.885)--(-3.437,.829)--cycle;
\filldraw[fill opacity=0.8,fill=gray!20,draw=none](-3.464,.889)--(-3.445,.885)--(-3.436,.846)--(-3.457,.851)--cycle;
\draw(-3.436,.846)--(-3.457,.851);
\filldraw[fill opacity=0.8,fill=gray!20,draw=none](-3.411,.99)--(-3.426,.993)--(-3.448,1.04)--(-3.386,1.025)--cycle;
\draw(-3.411,.99)--(-3.426,.993);
\draw(-3.448,1.04)--(-3.386,1.025);
\filldraw[fill opacity=0.8,fill=gray!20,draw=none](-3.408,.989)--(-3.414,.974)--(-3.415,.989)--(-3.414,.992)--(-3.406,.997)--cycle;
\draw(-3.414,.992)--(-3.406,.997);
\filldraw[fill opacity=0.8,fill=gray!20,draw=none](-3.448,.948)--(-3.465,.952)--(-3.453,.995)--cycle;
\draw(-3.448,.948)--(-3.465,.952);
\filldraw[fill opacity=0.8,fill=gray!20,draw=none](-3.466,.94)--(-3.465,.952)--(-3.461,.951)--cycle;
\draw(-3.465,.952)--(-3.461,.951);
\filldraw[fill opacity=0.8,fill=gray!20,draw=none](-3.425,.864)--(-3.421,.921)--(-3.422,.925)--(-3.437,.94)--(-3.444,.885)--cycle;
\draw(-3.422,.925)--(-3.437,.94)--(-3.444,.885)--(-3.425,.864);
\filldraw[fill opacity=0.8,fill=gray!20,draw=none](-3.436,.941)--(-3.437,.94)--(-3.436,.942)--cycle;
\draw(-3.436,.941)--(-3.437,.94)--(-3.436,.942);
\filldraw[fill opacity=0.8,fill=gray!20,draw=none](-3.421,.924)--(-3.4,.957)--(-3.397,.966)--(-3.436,.942)--(-3.437,.94)--cycle;
\draw(-3.4,.957)--(-3.397,.966);
\draw(-3.436,.942)--(-3.437,.94)--(-3.421,.924);
\filldraw[fill opacity=0.8,fill=gray!20,draw=none](-3.397,.966)--(-3.395,.97)--(-3.415,.992)--(-3.436,.942)--cycle;
\draw(-3.397,.966)--(-3.395,.97)--(-3.415,.992)--(-3.436,.942);
\filldraw[fill opacity=0.8,fill=gray!20,draw=none](-3.421,.921)--(-3.421,.924)--(-3.422,.925)--cycle;
\draw(-3.421,.924)--(-3.422,.925);
\filldraw[fill opacity=0.8,fill=gray!20,draw=none](-3.466,.94)--(-3.469,.9)--(-3.482,.904)--cycle;
\draw(-3.469,.9)--(-3.482,.904);
\filldraw[fill opacity=0.8,fill=gray!20,draw=none](-3.486,.887)--(-3.481,.894)--(-3.464,.889)--(-3.457,.851)--(-3.491,.86)--cycle;
\draw(-3.457,.851)--(-3.491,.86);
\filldraw[fill opacity=0.8,fill=gray!20,draw=none](-3.356,.828)--(-3.355,.828)--(-3.356,.826)--cycle;
\filldraw[fill opacity=0.8,fill=gray!20,draw=none](-3.363,.805)--(-3.361,.811)--(-3.36,.81)--(-3.364,.796)--(-3.364,.798)--cycle;
\draw(-3.364,.796)--(-3.364,.798);
\filldraw[fill opacity=0.8,fill=gray!20,draw=none](-3.365,.798)--(-3.356,.798)--(-3.364,.807)--(-3.365,.806)--cycle;
\draw(-3.356,.798)--(-3.364,.807)--(-3.365,.806);
\filldraw[fill opacity=0.8,fill=gray!20,draw=none](-3.363,.805)--(-3.361,.811)--(-3.364,.807)--cycle;
\draw(-3.361,.811)--(-3.364,.807)--(-3.363,.805);
\filldraw[fill opacity=0.8,fill=gray!20,draw=none](-3.363,.805)--(-3.356,.798)--(-3.35,.795)--(-3.33,.825)--(-3.341,.836)--(-3.361,.811)--cycle;
\draw(-3.363,.805)--(-3.356,.798);
\draw(-3.35,.795)--(-3.33,.825)--(-3.341,.836)--(-3.361,.811);
\filldraw[fill opacity=0.8,fill=gray!20,draw=none](-3.363,.805)--(-3.362,.811)--(-3.361,.811)--cycle;
\filldraw[fill opacity=0.8,fill=gray!20,draw=none](-3.362,.811)--(-3.364,.798)--(-3.363,.812)--cycle;
\draw(-3.364,.798)--(-3.363,.812);
\filldraw[fill opacity=0.8,fill=gray!20,draw=none](-3.356,.891)--(-3.364,.801)--(-3.356,.798)--(-3.35,.797)--(-3.342,.881)--cycle;
\draw(-3.35,.797)--(-3.342,.881)--(-3.356,.891)--(-3.364,.801);
\filldraw[fill opacity=0.8,fill=gray!20,draw=none](-3.486,.887)--(-3.484,.895)--(-3.481,.894)--cycle;
\filldraw[fill opacity=0.8,fill=gray!20,draw=none](-3.294,.777)--(-3.309,.767)--(-3.351,.778)--(-3.364,.796)--(-3.296,.779)--cycle;
\draw(-3.309,.767)--(-3.351,.778);
\draw(-3.364,.796)--(-3.296,.779);
\filldraw[fill opacity=0.8,fill=gray!20](-3.33,.825)--(-3.303,.85)--(-3.311,.859)--(-3.341,.836)--cycle;
\filldraw[fill opacity=0.8,fill=gray!20](-3.295,.816)--(-3.278,.844)--(-3.303,.85)--(-3.33,.825)--cycle;
\filldraw[fill opacity=0.8,fill=gray!20,draw=none](-3.337,.789)--(-3.313,.784)--(-3.306,.787)--(-3.295,.816)--(-3.33,.825)--(-3.35,.795)--cycle;
\draw(-3.337,.789)--(-3.313,.784);
\draw(-3.306,.787)--(-3.295,.816)--(-3.33,.825)--(-3.35,.795);
\filldraw[fill opacity=0.8,fill=gray!20,draw=none](-3.359,.776)--(-3.366,.775)--(-3.369,.743)--(-3.355,.733)--(-3.353,.754)--cycle;
\draw(-3.366,.775)--(-3.369,.743)--(-3.355,.733)--(-3.353,.754);
\filldraw[fill opacity=0.8,fill=gray!20,draw=none](-3.364,.796)--(-3.359,.776)--(-3.351,.778)--cycle;
\filldraw[fill opacity=0.8,fill=gray!20,draw=none](-3.351,.778)--(-3.356,.779)--(-3.365,.796)--(-3.364,.796)--cycle;
\draw(-3.351,.778)--(-3.356,.779);
\draw(-3.365,.796)--(-3.364,.796);
\filldraw[fill opacity=0.8,fill=gray!20,draw=none](-3.364,.798)--(-3.364,.796)--(-3.351,.778)--(-3.35,.797)--cycle;
\draw(-3.364,.798)--(-3.364,.796);
\draw(-3.351,.778)--(-3.35,.797);
\filldraw[fill opacity=0.8,fill=gray!20,draw=none](-3.337,.789)--(-3.35,.795)--(-3.351,.778)--(-3.319,.776)--(-3.319,.778)--cycle;
\draw(-3.35,.795)--(-3.351,.778);
\draw(-3.319,.776)--(-3.319,.778);
\filldraw[fill opacity=0.8,fill=gray!20,draw=none](-3.321,.776)--(-3.313,.784)--(-3.351,.793)--(-3.356,.779)--cycle;
\draw(-3.313,.784)--(-3.351,.793)--(-3.356,.779);
\filldraw[fill opacity=0.8,fill=gray!20,draw=none](-3.342,.881)--(-3.35,.797)--(-3.337,.789)--(-3.319,.781)--(-3.311,.872)--cycle;
\draw(-3.319,.781)--(-3.311,.872)--(-3.342,.881)--(-3.35,.797);
\filldraw[fill opacity=0.8,fill=gray!20,draw=none](-3.425,.864)--(-3.421,.86)--(-3.417,.899)--(-3.421,.921)--cycle;
\draw(-3.425,.864)--(-3.421,.86)--(-3.417,.899);
\filldraw[fill opacity=0.8,fill=gray!20,draw=none](-3.294,.777)--(-3.296,.779)--(-3.293,.778)--cycle;
\draw(-3.296,.779)--(-3.293,.778);
\filldraw[fill opacity=0.8,fill=gray!20,draw=none](-3.294,.777)--(-3.293,.781)--(-3.255,.827)--(-3.275,.799)--(-3.292,.779)--cycle;
\draw(-3.294,.777)--(-3.293,.781)--(-3.255,.827);
\draw(-3.275,.799)--(-3.292,.779);
\filldraw[fill opacity=0.8,fill=gray!20,draw=none](-3.284,.779)--(-3.239,.834)--(-3.255,.827)--(-3.293,.781)--cycle;
\draw(-3.255,.827)--(-3.293,.781)--(-3.284,.779)--(-3.239,.834);
\filldraw[fill opacity=0.8,fill=gray!20,draw=none](-3.268,.77)--(-3.23,.817)--(-3.239,.834)--(-3.284,.779)--cycle;
\draw(-3.239,.834)--(-3.284,.779)--(-3.268,.77)--(-3.23,.817);
\filldraw[fill opacity=0.8,fill=gray!20,draw=none](-3.252,.759)--(-3.247,.758)--(-3.245,.759)--(-3.229,.778)--(-3.23,.817)--(-3.268,.77)--cycle;
\draw(-3.245,.759)--(-3.229,.778);
\draw(-3.23,.817)--(-3.268,.77)--(-3.252,.759);
\filldraw[fill opacity=0.8,fill=gray!20](-3.244,.812)--(-3.242,.842)--(-3.278,.844)--(-3.295,.816)--cycle;
\filldraw[fill opacity=0.8,fill=gray!20,draw=none](-3.192,.815)--(-3.193,.818)--(-3.205,.843)--(-3.242,.842)--(-3.244,.812)--cycle;
\draw(-3.193,.818)--(-3.205,.843)--(-3.242,.842)--(-3.244,.812)--(-3.192,.815);
\filldraw[fill opacity=0.8,fill=gray!20,draw=none](-3.365,.806)--(-3.364,.807)--(-3.363,.808)--cycle;
\draw(-3.365,.806)--(-3.364,.807)--(-3.363,.808);
\filldraw[fill opacity=0.8,fill=gray!20,draw=none](-3.365,.798)--(-3.365,.796)--(-3.356,.779)--(-3.351,.793)--(-3.356,.798)--cycle;
\draw(-3.356,.779)--(-3.351,.793)--(-3.356,.798);
\filldraw[fill opacity=0.8,fill=gray!20,draw=none](-3.365,.796)--(-3.365,.778)--(-3.356,.779)--cycle;
\filldraw[fill opacity=0.8,fill=gray!20,draw=none](-3.356,.779)--(-3.409,.792)--(-3.416,.801)--(-3.395,.803)--(-3.365,.796)--cycle;
\draw(-3.356,.779)--(-3.409,.792);
\draw(-3.395,.803)--(-3.365,.796);
\filldraw[fill opacity=0.8,fill=gray!20,draw=none](-3.364,.801)--(-3.364,.798)--(-3.356,.798)--cycle;
\draw(-3.364,.801)--(-3.364,.798);
\filldraw[fill opacity=0.8,fill=gray!20,draw=none](-3.356,.798)--(-3.351,.793)--(-3.35,.795)--cycle;
\draw(-3.356,.798)--(-3.351,.793)--(-3.35,.795);
\filldraw[fill opacity=0.8,fill=gray!20,draw=none](-3.373,.795)--(-3.353,.81)--(-3.345,.831)--(-3.346,.842)--(-3.421,.86)--(-3.415,.806)--cycle;
\draw(-3.345,.831)--(-3.346,.842)--(-3.421,.86)--(-3.415,.806)--(-3.373,.795);
\filldraw[fill opacity=0.8,fill=gray!20,draw=none](-3.41,.858)--(-3.394,.909)--(-3.417,.899)--(-3.421,.86)--cycle;
\draw(-3.417,.899)--(-3.421,.86)--(-3.41,.858);
\filldraw[fill opacity=0.8,fill=gray!20,draw=none](-3.2,.844)--(-3.205,.843)--(-3.193,.818)--cycle;
\draw(-3.2,.844)--(-3.205,.843)--(-3.193,.818);
\filldraw[fill opacity=0.8,fill=gray!20,draw=none](-3.263,.758)--(-3.294,.777)--(-3.293,.778)--(-3.281,.775)--cycle;
\draw(-3.293,.778)--(-3.281,.775)--(-3.263,.758);
\filldraw[fill opacity=0.8,fill=gray!20,draw=none](-3.294,.777)--(-3.292,.779)--(-3.294,.776)--cycle;
\draw(-3.292,.779)--(-3.294,.776)--(-3.294,.777);
\filldraw[fill opacity=0.8,fill=gray!20,draw=none](-3.278,.764)--(-3.263,.758)--(-3.294,.777)--(-3.294,.777)--cycle;
\draw(-3.294,.777)--(-3.294,.777);
\filldraw[fill opacity=0.8,fill=gray!20,draw=none](-3.394,.909)--(-3.41,.858)--(-3.346,.842)--(-3.342,.899)--(-3.39,.911)--cycle;
\draw(-3.41,.858)--(-3.346,.842)--(-3.342,.899)--(-3.39,.911);
\filldraw[fill opacity=0.8,fill=gray!20,draw=none](-3.297,.783)--(-3.261,.787)--(-3.245,.794)--(-3.244,.812)--(-3.295,.816)--(-3.306,.787)--cycle;
\draw(-3.245,.794)--(-3.244,.812)--(-3.295,.816)--(-3.306,.787);
\filldraw[fill opacity=0.8,fill=gray!20,draw=none](-3.263,.758)--(-3.259,.756)--(-3.252,.759)--(-3.268,.77)--(-3.284,.779)--(-3.293,.781)--(-3.294,.777)--cycle;
\draw(-3.252,.759)--(-3.268,.77)--(-3.284,.779)--(-3.293,.781)--(-3.294,.777);
\filldraw[fill opacity=0.8,fill=gray!20,draw=none](-3.294,.777)--(-3.263,.758)--(-3.261,.756)--(-3.309,.767)--cycle;
\draw(-3.263,.758)--(-3.261,.756)--(-3.309,.767);
\filldraw[fill opacity=0.8,fill=gray!20,draw=none](-3.311,.872)--(-3.317,.8)--(-3.296,.783)--(-3.273,.79)--(-3.266,.863)--cycle;
\draw(-3.273,.79)--(-3.266,.863)--(-3.311,.872)--(-3.317,.8);
\filldraw[fill opacity=0.8,fill=gray!20,draw=none](-3.272,.792)--(-3.273,.79)--(-3.27,.791)--cycle;
\draw(-3.272,.792)--(-3.273,.79);
\filldraw[fill opacity=0.8,fill=gray!20,draw=none](-3.192,.799)--(-3.192,.815)--(-3.244,.812)--(-3.245,.794)--cycle;
\draw(-3.192,.815)--(-3.244,.812)--(-3.245,.794);
\filldraw[fill opacity=0.8,fill=gray!20,draw=none](-3.266,.863)--(-3.272,.792)--(-3.27,.791)--(-3.219,.818)--(-3.215,.858)--cycle;
\draw(-3.219,.818)--(-3.215,.858)--(-3.266,.863)--(-3.272,.792);
\filldraw[fill opacity=0.8,fill=gray!20,draw=none](-3.335,.841)--(-3.346,.842)--(-3.345,.831)--cycle;
\draw(-3.335,.841)--(-3.346,.842)--(-3.345,.831);
\filldraw[fill opacity=0.8,fill=gray!20,draw=none](-3.209,.885)--(-3.12,.86)--(-3.1,.864)--(-3.095,.872)--(-3.208,.886)--cycle;
\draw(-3.12,.86)--(-3.1,.864)--(-3.095,.872);
\filldraw[fill opacity=0.8,fill=gray!20,draw=none](-3.199,.882)--(-3.088,1.015)--(-3.133,.974)--(-3.205,.888)--cycle;
\draw(-3.199,.882)--(-3.088,1.015);
\draw(-3.133,.974)--(-3.205,.888);
\filldraw[fill opacity=0.8,fill=gray!20,draw=none](-3.205,.888)--(-3.208,.886)--(-3.202,.885)--cycle;
\filldraw[fill opacity=0.8,fill=gray!20,draw=none](-3.189,.866)--(-3.094,.98)--(-3.091,1.003)--(-3.097,1.005)--(-3.204,.876)--cycle;
\draw(-3.189,.866)--(-3.094,.98);
\draw(-3.097,1.005)--(-3.204,.876);
\filldraw[fill opacity=0.8,fill=gray!20,draw=none](-3.209,.885)--(-3.219,.879)--(-3.166,.856)--(-3.126,.858)--(-3.12,.86)--cycle;
\draw(-3.166,.856)--(-3.126,.858)--(-3.12,.86);
\filldraw[fill opacity=0.8,fill=gray!20,draw=none](-3.231,.885)--(-3.248,.916)--(-3.254,.894)--cycle;
\filldraw[fill opacity=0.8,fill=gray!20,draw=none](-3.249,.892)--(-3.258,.892)--(-3.245,.89)--cycle;
\filldraw[fill opacity=0.8,fill=gray!20,draw=none](-3.225,.871)--(-3.308,.891)--(-3.329,.845)--(-3.24,.823)--cycle;
\draw(-3.225,.871)--(-3.308,.891);
\draw(-3.329,.845)--(-3.24,.823);
\filldraw[fill opacity=0.8,fill=gray!20,draw=none](-3.335,.841)--(-3.306,.862)--(-3.309,.896)--(-3.342,.899)--(-3.346,.842)--cycle;
\draw(-3.309,.896)--(-3.342,.899)--(-3.346,.842)--(-3.335,.841);
\filldraw[fill opacity=0.8,fill=gray!20,draw=none](-3.343,.835)--(-3.339,.848)--(-3.366,.854)--(-3.386,.844)--(-3.408,.823)--(-3.373,.815)--cycle;
\draw(-3.339,.848)--(-3.366,.854);
\draw(-3.408,.823)--(-3.373,.815);
\filldraw[fill opacity=0.8,fill=gray!20,draw=none](-3.308,.891)--(-3.326,.895)--(-3.331,.846)--(-3.329,.845)--cycle;
\draw(-3.308,.891)--(-3.326,.895);
\draw(-3.331,.846)--(-3.329,.845);
\filldraw[fill opacity=0.8,fill=gray!20,draw=none](-3.326,.895)--(-3.343,.9)--(-3.35,.89)--(-3.36,.853)--(-3.331,.846)--cycle;
\draw(-3.326,.895)--(-3.343,.9);
\draw(-3.36,.853)--(-3.331,.846);
\filldraw[fill opacity=0.8,fill=gray!20,draw=none](-3.366,.854)--(-3.373,.856)--(-3.386,.844)--cycle;
\draw(-3.366,.854)--(-3.373,.856);
\filldraw[fill opacity=0.8,fill=gray!20,draw=none](-3.35,.89)--(-3.373,.856)--(-3.36,.853)--cycle;
\draw(-3.373,.856)--(-3.36,.853);
\filldraw[fill opacity=0.8,fill=gray!20](-3.354,.83)--(-3.518,.901)--(-3.489,.951)--(-3.326,.879)--cycle;
\filldraw[fill opacity=0.8,fill=gray!20,draw=none](-3.393,.911)--(-3.39,.911)--(-3.366,.93)--(-3.397,.966)--(-3.4,.957)--cycle;
\draw(-3.393,.911)--(-3.39,.911);
\draw(-3.397,.966)--(-3.4,.957);
\filldraw[fill opacity=0.8,fill=gray!20,draw=none](-3.366,.93)--(-3.336,.956)--(-3.395,.97)--(-3.397,.966)--cycle;
\draw(-3.336,.956)--(-3.395,.97)--(-3.397,.966);
\filldraw[fill opacity=0.8,fill=gray!20,draw=none](-3.151,.872)--(-3.065,.975)--(-3.071,1.007)--(-3.18,.877)--cycle;
\draw(-3.151,.872)--(-3.065,.975);
\draw(-3.071,1.007)--(-3.18,.877);
\filldraw[fill opacity=0.8,fill=gray!20,draw=none](-3.166,.856)--(-3.219,.879)--(-3.16,.962)--(-3.158,.953)--(-3.157,.906)--cycle;
\draw(-3.16,.962)--(-3.158,.953)--(-3.157,.906)--(-3.166,.856);
\filldraw[fill opacity=0.8,fill=gray!20,draw=none](-3.231,.885)--(-3.166,.856)--(-3.176,.906)--(-3.193,.953)--(-3.215,.989)--(-3.225,.997)--(-3.248,.916)--cycle;
\draw(-3.166,.856)--(-3.176,.906)--(-3.193,.953)--(-3.215,.989)--(-3.225,.997);
\filldraw[fill opacity=0.8,fill=gray!20,draw=none](-3.052,.868)--(-3.025,.861)--(-3.03,.916)--(-3.04,.918)--cycle;
\draw(-3.052,.868)--(-3.025,.861)--(-3.03,.916)--(-3.04,.918);
\filldraw[fill opacity=0.8,fill=gray!20,draw=none](-3.065,.975)--(-3.046,.998)--(-3.062,1.018)--(-3.071,1.007)--cycle;
\draw(-3.065,.975)--(-3.046,.998);
\draw(-3.062,1.018)--(-3.071,1.007);
\filldraw[fill opacity=0.8,fill=gray!20,draw=none](-3.067,.982)--(-3.064,.982)--(-3.061,1.004)--(-3.071,1.007)--cycle;
\draw(-3.067,.982)--(-3.064,.982);
\filldraw[fill opacity=0.8,fill=gray!20,draw=none](-3.044,.996)--(-3.028,.976)--(-2.986,1.025)--(-3.016,1.034)--(-3.045,.999)--cycle;
\draw(-3.028,.976)--(-2.986,1.025);
\draw(-3.016,1.034)--(-3.045,.999);
\filldraw[fill opacity=0.8,fill=gray!20,draw=none](-3.056,.941)--(-3.032,.97)--(-3.044,.996)--(-3.046,.998)--(-3.063,.977)--cycle;
\draw(-3.056,.941)--(-3.032,.97);
\draw(-3.046,.998)--(-3.063,.977);
\filldraw[fill opacity=0.8,fill=gray!20,draw=none](-3.048,1.002)--(-3.059,1.015)--(-3.064,.982)--(-3.055,.98)--cycle;
\draw(-3.064,.982)--(-3.055,.98);
\filldraw[fill opacity=0.8,fill=gray!20,draw=none](-3.02,.942)--(-3.026,.978)--(-3.032,.97)--cycle;
\draw(-3.026,.978)--(-3.032,.97);
\filldraw[fill opacity=0.8,fill=gray!20,draw=none](-3.032,.97)--(-3.028,.976)--(-3.044,.996)--cycle;
\draw(-3.032,.97)--(-3.028,.976);
\filldraw[fill opacity=0.8,fill=gray!20,draw=none](-3.02,.942)--(-3.037,.987)--(-3.031,.976)--(-3.013,.935)--cycle;
\draw(-3.031,.976)--(-3.013,.935)--(-3.02,.942);
\filldraw[fill opacity=0.8,fill=gray!20,draw=none](-3.032,.97)--(-3.025,.955)--(-3.02,.942)--cycle;
\filldraw[fill opacity=0.8,fill=gray!20,draw=none](-3.058,.897)--(-3.02,.942)--(-3.032,.97)--(-3.056,.941)--cycle;
\draw(-3.058,.897)--(-3.02,.942);
\draw(-3.032,.97)--(-3.056,.941);
\filldraw[fill opacity=0.8,fill=gray!20,draw=none](-3.031,.97)--(-3.025,.955)--(-3.044,.996)--(-3.043,.995)--cycle;
\draw(-3.044,.996)--(-3.043,.995);
\filldraw[fill opacity=0.8,fill=gray!20,draw=none](-3.047,.951)--(-3.041,.976)--(-3.067,.982)--cycle;
\draw(-3.041,.976)--(-3.067,.982);
\filldraw[fill opacity=0.8,fill=gray!20,draw=none](-3.037,.987)--(-3.031,.97)--(-3.043,.995)--(-3.038,.99)--cycle;
\draw(-3.043,.995)--(-3.038,.99);
\filldraw[fill opacity=0.8,fill=gray!20,draw=none](-3.116,.869)--(-3.056,.941)--(-3.063,.977)--(-3.151,.872)--cycle;
\draw(-3.116,.869)--(-3.056,.941);
\draw(-3.063,.977)--(-3.151,.872);
\filldraw[fill opacity=0.8,fill=gray!20,draw=none](-3.053,.921)--(-3.047,.951)--(-3.063,.977)--cycle;
\filldraw[fill opacity=0.8,fill=gray!20,draw=none](-3.06,.894)--(-3.058,.897)--(-3.056,.941)--(-3.06,.937)--cycle;
\draw(-3.06,.894)--(-3.058,.897);
\draw(-3.056,.941)--(-3.06,.937);
\filldraw[fill opacity=0.8,fill=gray!20,draw=none](-3.082,.868)--(-3.06,.894)--(-3.06,.937)--(-3.116,.869)--cycle;
\draw(-3.082,.868)--(-3.06,.894);
\draw(-3.06,.937)--(-3.116,.869);
\filldraw[fill opacity=0.8,fill=gray!20,draw=none](-3.057,.922)--(-3.053,.921)--(-3.056,.941)--cycle;
\draw(-3.057,.922)--(-3.053,.921);
\filldraw[fill opacity=0.8,fill=gray!20,draw=none](-3.031,.976)--(-3.038,.99)--(-3.048,1.002)--(-3.055,.98)--(-3.03,.974)--cycle;
\draw(-3.055,.98)--(-3.03,.974);
\filldraw[fill opacity=0.8,fill=gray!20,draw=none](-3.053,.921)--(-3.03,.916)--(-3.027,.958)--(-3.03,.974)--(-3.041,.976)--cycle;
\draw(-3.053,.921)--(-3.03,.916)--(-3.027,.958);
\draw(-3.03,.974)--(-3.041,.976);
\filldraw[fill opacity=0.8,fill=gray!20,draw=none](-3.032,.982)--(-3.038,.99)--(-3.031,.976)--cycle;
\filldraw[fill opacity=0.8,fill=gray!20,draw=none](-3.037,.987)--(-3.037,.989)--(-3.035,.987)--(-3.031,.976)--cycle;
\draw(-3.037,.989)--(-3.035,.987)--(-3.031,.976);
\filldraw[fill opacity=0.8,fill=gray!20,draw=none](-3.017,.915)--(-3.018,.932)--(-3.027,.958)--(-3.03,.916)--cycle;
\draw(-3.027,.958)--(-3.03,.916)--(-3.017,.915);
\filldraw[fill opacity=0.8,fill=gray!20,draw=none](-3.048,.885)--(-3.044,.904)--(-3.053,.921)--(-3.057,.922)--cycle;
\draw(-3.053,.921)--(-3.057,.922);
\filldraw[fill opacity=0.8,fill=gray!20,draw=none](-3.044,.904)--(-3.04,.918)--(-3.053,.921)--cycle;
\draw(-3.04,.918)--(-3.053,.921);
\filldraw[fill opacity=0.8,fill=gray!20,draw=none](-3.176,.906)--(-3.166,.909)--(-3.175,.957)--(-3.193,.953)--cycle;
\draw(-3.175,.957)--(-3.193,.953)--(-3.176,.906)--(-3.166,.909);
\filldraw[fill opacity=0.8,fill=gray!20,draw=none](-3.166,.856)--(-3.166,.909)--(-3.176,.906)--cycle;
\draw(-3.166,.909)--(-3.176,.906)--(-3.166,.856);
\filldraw[fill opacity=0.8,fill=gray!20,draw=none](-3.394,.909)--(-3.39,.911)--(-3.393,.911)--cycle;
\draw(-3.39,.911)--(-3.393,.911);
\filldraw[fill opacity=0.8,fill=gray!20,draw=none](-3.366,.93)--(-3.39,.911)--(-3.342,.899)--(-3.341,.901)--cycle;
\draw(-3.39,.911)--(-3.342,.899)--(-3.341,.901);
\filldraw[fill opacity=0.8,fill=gray!20,draw=none](-3.366,.93)--(-3.341,.901)--(-3.33,.954)--(-3.336,.956)--cycle;
\draw(-3.341,.901)--(-3.33,.954)--(-3.336,.956);
\filldraw[fill opacity=0.8,fill=gray!20,draw=none](-3.166,.909)--(-3.283,.938)--(-3.301,.933)--(-3.31,.891)--(-3.239,.874)--cycle;
\draw(-3.166,.909)--(-3.283,.938);
\draw(-3.31,.891)--(-3.239,.874);
\filldraw[fill opacity=0.8,fill=gray!20,draw=none](-3.34,.899)--(-3.309,.896)--(-3.299,.941)--(-3.302,.952)--(-3.33,.954)--(-3.341,.901)--cycle;
\draw(-3.34,.899)--(-3.309,.896);
\draw(-3.302,.952)--(-3.33,.954)--(-3.341,.901);
\filldraw[fill opacity=0.8,fill=gray!20,draw=none](-3.34,.899)--(-3.341,.901)--(-3.342,.899)--cycle;
\draw(-3.341,.901)--(-3.342,.899)--(-3.34,.899);
\filldraw[fill opacity=0.8,fill=gray!20,draw=none](-3.301,.933)--(-3.299,.941)--(-3.321,.947)--(-3.332,.924)--cycle;
\draw(-3.299,.941)--(-3.321,.947);
\filldraw[fill opacity=0.8,fill=gray!20,draw=none](-3.301,.933)--(-3.332,.924)--(-3.335,.917)--(-3.33,.896)--(-3.31,.891)--cycle;
\draw(-3.33,.896)--(-3.31,.891);
\filldraw[fill opacity=0.8,fill=gray!20,draw=none](-3.335,.917)--(-3.321,.947)--(-3.342,.952)--(-3.343,.947)--cycle;
\draw(-3.321,.947)--(-3.342,.952);
\filldraw[fill opacity=0.8,fill=gray!20,draw=none](-3.335,.917)--(-3.343,.9)--(-3.33,.896)--cycle;
\draw(-3.343,.9)--(-3.33,.896);
\filldraw[fill opacity=0.8,fill=gray!20,draw=none](-3.335,.917)--(-3.343,.947)--(-3.347,.901)--(-3.343,.9)--cycle;
\draw(-3.347,.901)--(-3.343,.9);
\filldraw[fill opacity=0.8,fill=gray!20,draw=none](-3.343,.9)--(-3.347,.901)--(-3.35,.89)--cycle;
\draw(-3.343,.9)--(-3.347,.901);
\filldraw[fill opacity=0.8,fill=gray!20](-3.326,.879)--(-3.489,.951)--(-3.479,1.006)--(-3.316,.935)--cycle;
\filldraw[fill opacity=0.8,fill=gray!20](-3.238,.591)--(-3.206,.591)--(-3.221,.588)--(-3.238,.591)--cycle;
\filldraw[fill opacity=0.8,fill=gray!20](-3.238,.591)--(-3.2,.595)--(-3.206,.591)--(-3.238,.591)--cycle;
\filldraw[fill opacity=0.8,fill=gray!20](-3.238,.591)--(-3.221,.588)--(-3.24,.588)--(-3.238,.591)--cycle;
\filldraw[fill opacity=0.8,fill=gray!20](-3.238,.591)--(-3.24,.588)--(-3.259,.589)--(-3.238,.591)--cycle;
\filldraw[fill opacity=0.8,fill=gray!20](-3.238,.591)--(-3.259,.589)--(-3.272,.592)--(-3.238,.591)--cycle;
\filldraw[fill opacity=0.8,fill=gray!20](-3.303,.603)--(-3.33,.623)--(-3.341,.634)--(-3.311,.611)--cycle;
\filldraw[fill opacity=0.8,fill=gray!20,draw=none](-2.692,1.777)--(-2.715,1.797)--(-2.73,1.918)--cycle;
\draw(-2.692,1.777)--(-2.715,1.797)--(-2.73,1.918);
\filldraw[fill opacity=0.8,fill=gray!20,draw=none](-3.034,.789)--(-3.042,.79)--(-3.022,.81)--cycle;
\filldraw[fill opacity=0.8,fill=gray!20,draw=none](-3.409,.792)--(-3.415,.806)--(-3.437,.829)--(-3.417,.782)--cycle;
\draw(-3.409,.792)--(-3.415,.806)--(-3.437,.829)--(-3.417,.782);
\filldraw[fill opacity=0.8,fill=gray!20,draw=none](-3.428,.82)--(-3.442,.848)--(-3.436,.846)--cycle;
\draw(-3.442,.848)--(-3.436,.846);
\filldraw[fill opacity=0.8,fill=gray!20,draw=none](-3.428,.82)--(-3.454,.817)--(-3.511,.831)--(-3.508,.843)--(-3.488,.859)--(-3.442,.848)--cycle;
\draw(-3.454,.817)--(-3.511,.831);
\draw(-3.488,.859)--(-3.442,.848);
\filldraw[fill opacity=0.8,fill=gray!20,draw=none](-3.486,.887)--(-3.491,.86)--(-3.503,.863)--cycle;
\draw(-3.491,.86)--(-3.503,.863);
\filldraw[fill opacity=0.8,fill=gray!20,draw=none](-3.422,.81)--(-3.454,.817)--(-3.428,.82)--cycle;
\draw(-3.422,.81)--(-3.454,.817);
\filldraw[fill opacity=0.8,fill=gray!20,draw=none](-3.508,.843)--(-3.503,.863)--(-3.488,.859)--cycle;
\draw(-3.503,.863)--(-3.488,.859);
\filldraw[fill opacity=0.8,fill=gray!20,draw=none](-3.416,.801)--(-3.439,.799)--(-3.498,.814)--(-3.511,.831)--(-3.422,.81)--cycle;
\draw(-3.439,.799)--(-3.498,.814);
\draw(-3.511,.831)--(-3.422,.81);
\filldraw[fill opacity=0.8,fill=gray!20,draw=none](-3.251,.759)--(-3.259,.756)--(-3.236,.755)--(-3.249,.758)--cycle;
\draw(-3.259,.756)--(-3.236,.755)--(-3.249,.758);
\filldraw[fill opacity=0.8,fill=gray!20,draw=none](-3.232,.759)--(-3.249,.758)--(-3.236,.755)--cycle;
\draw(-3.249,.758)--(-3.236,.755)--(-3.232,.759);
\filldraw[fill opacity=0.8,fill=gray!20,draw=none](-3.234,.745)--(-3.229,.778)--(-3.248,.756)--cycle;
\draw(-3.229,.778)--(-3.248,.756)--(-3.234,.745);
\filldraw[fill opacity=0.8,fill=gray!20](-3.323,.686)--(-3.364,.715)--(-3.38,.732)--(-3.335,.698)--cycle;
\filldraw[fill opacity=0.8,fill=gray!20,draw=none](-3.369,.721)--(-3.364,.757)--(-3.379,.773)--(-3.384,.738)--cycle;
\draw(-3.369,.721)--(-3.364,.757)--(-3.379,.773)--(-3.384,.738);
\filldraw[fill opacity=0.8,fill=gray!20,draw=none](-3.364,.757)--(-3.365,.778)--(-3.378,.776)--(-3.379,.773)--cycle;
\draw(-3.378,.776)--(-3.379,.773)--(-3.364,.757);
\filldraw[fill opacity=0.8,fill=gray!20,draw=none](-3.377,.79)--(-3.373,.795)--(-3.415,.806)--(-3.409,.792)--cycle;
\draw(-3.373,.795)--(-3.415,.806)--(-3.409,.792);
\filldraw[fill opacity=0.8,fill=gray!20,draw=none](-3.349,.792)--(-3.337,.789)--(-3.35,.795)--cycle;
\draw(-3.349,.792)--(-3.337,.789);
\filldraw[fill opacity=0.8,fill=gray!20,draw=none](-3.349,.792)--(-3.35,.795)--(-3.351,.793)--cycle;
\draw(-3.35,.795)--(-3.351,.793)--(-3.349,.792);
\filldraw[fill opacity=0.8,fill=gray!20,draw=none](-3.35,.797)--(-3.35,.795)--(-3.337,.789)--cycle;
\draw(-3.35,.797)--(-3.35,.795);
\filldraw[fill opacity=0.8,fill=gray!20,draw=none](-3.409,.792)--(-3.439,.799)--(-3.416,.801)--cycle;
\draw(-3.409,.792)--(-3.439,.799);
\filldraw[fill opacity=0.8,fill=gray!20,draw=none](-3.359,.811)--(-3.404,.822)--(-3.449,.817)--(-3.466,.812)--(-3.377,.79)--cycle;
\draw(-3.359,.811)--(-3.404,.822);
\draw(-3.466,.812)--(-3.377,.79);
\filldraw[fill opacity=0.8,fill=gray!20,draw=none](-3.404,.822)--(-3.42,.826)--(-3.449,.817)--cycle;
\draw(-3.404,.822)--(-3.42,.826);
\filldraw[fill opacity=0.8,fill=gray!20,draw=none](-3.386,.844)--(-3.42,.826)--(-3.408,.823)--cycle;
\draw(-3.42,.826)--(-3.408,.823);
\filldraw[fill opacity=0.8,fill=gray!20](-3.397,.794)--(-3.56,.866)--(-3.518,.901)--(-3.354,.83)--cycle;
\filldraw[fill opacity=0.8,fill=gray!20](-3.2,.595)--(-3.165,.609)--(-3.177,.601)--(-3.206,.591)--cycle;
\filldraw[fill opacity=0.8,fill=gray!20](-3.33,.623)--(-3.351,.65)--(-3.364,.664)--(-3.341,.634)--cycle;
\filldraw[fill opacity=0.5,fill=gray!20](-1.272,2.805)--(-1.397,2.865)--(-1.105,3.311)--(-.994,3.23)--cycle;
\filldraw[fill opacity=0.8,fill=gray!20,draw=none](-2.728,1.776)--(-2.72,1.796)--(-2.713,1.795)--(-2.68,1.767)--(-2.681,1.765)--cycle;
\draw(-2.713,1.795)--(-2.68,1.767)--(-2.681,1.765);
\filldraw[fill opacity=0.8,fill=gray!20](-3.317,1.081)--(-3.272,1.096)--(-3.25,1.1)--(-3.274,1.089)--cycle;
\filldraw[fill opacity=0.8,fill=gray!20,draw=none](-2.966,1.147)--(-2.969,1.151)--(-2.95,1.152)--(-2.939,1.15)--cycle;
\filldraw[fill opacity=0.8,fill=gray!20,draw=none](-2.906,1.044)--(-2.891,1.063)--(-2.892,1.096)--(-2.948,1.029)--cycle;
\draw(-2.906,1.044)--(-2.891,1.063);
\draw(-2.892,1.096)--(-2.948,1.029);
\filldraw[fill opacity=0.8,fill=gray!20,draw=none](-2.918,.77)--(-2.92,.807)--(-3.011,.814)--(-3.013,.812)--(-2.994,.776)--cycle;
\draw(-3.013,.812)--(-2.994,.776)--(-2.918,.77)--(-2.92,.807)--(-3.011,.814);
\filldraw[fill opacity=0.8,fill=gray!20,draw=none](-3.059,.84)--(-3.053,.846)--(-3.052,.869)--cycle;
\draw(-3.059,.84)--(-3.053,.846);
\filldraw[fill opacity=0.8,fill=gray!20,draw=none](-3.053,.846)--(-3.042,.866)--(-3.052,.868)--cycle;
\draw(-3.042,.866)--(-3.052,.868);
\filldraw[fill opacity=0.8,fill=gray!20,draw=none](-3.099,.847)--(-3.082,.868)--(-3.097,.869)--cycle;
\draw(-3.099,.847)--(-3.082,.868);
\filldraw[fill opacity=0.8,fill=gray!20,draw=none](-3.01,.876)--(-3.006,.879)--(-3.013,.935)--(-3.019,.931)--cycle;
\draw(-3.01,.876)--(-3.006,.879)--(-3.013,.935)--(-3.019,.931);
\filldraw[fill opacity=0.8,fill=gray!20,draw=none](-3.012,.86)--(-3.001,.86)--(-2.997,.895)--(-3.005,.914)--(-3.007,.914)--cycle;
\draw(-3.012,.86)--(-3.001,.86);
\draw(-3.005,.914)--(-3.007,.914);
\filldraw[fill opacity=0.8,fill=gray!20,draw=none](-3.019,.89)--(-3.014,.874)--(-3.01,.876)--(-3.019,.931)--(-3.021,.929)--cycle;
\draw(-3.014,.874)--(-3.01,.876);
\draw(-3.019,.931)--(-3.021,.929);
\filldraw[fill opacity=0.8,fill=gray!20,draw=none](-3.01,.876)--(-3.007,.914)--(-3.017,.915)--cycle;
\draw(-3.007,.914)--(-3.017,.915);
\filldraw[fill opacity=0.8,fill=gray!20,draw=none](-3.166,.856)--(-3.129,.818)--(-3.126,.858)--cycle;
\draw(-3.129,.818)--(-3.126,.858)--(-3.166,.856);
\filldraw[fill opacity=0.8,fill=gray!20](-3.166,.81)--(-2.849,.883)--(-2.85,.93)--(-3.166,.856)--cycle;
\filldraw[fill opacity=0.8,fill=gray!20,draw=none](-3.386,1.03)--(-3.376,1.022)--(-3.392,1.026)--cycle;
\draw(-3.376,1.022)--(-3.392,1.026);
\filldraw[fill opacity=0.8,fill=gray!20,draw=none](-3.386,1.025)--(-3.406,.997)--(-3.415,.992)--(-3.382,1.033)--cycle;
\draw(-3.406,.997)--(-3.415,.992)--(-3.382,1.033);
\filldraw[fill opacity=0.8,fill=gray!20,draw=none](-3.426,.993)--(-3.453,1)--(-3.449,1.04)--(-3.448,1.04)--cycle;
\draw(-3.426,.993)--(-3.453,1);
\draw(-3.449,1.04)--(-3.448,1.04);
\filldraw[fill opacity=0.8,fill=gray!20,draw=none](-3.453,.995)--(-3.453,1)--(-3.451,1)--cycle;
\draw(-3.453,1)--(-3.451,1);
\filldraw[fill opacity=0.8,fill=gray!20,draw=none](-3.418,.984)--(-3.415,.992)--(-3.414,.992)--cycle;
\draw(-3.418,.984)--(-3.415,.992)--(-3.414,.992);
\filldraw[fill opacity=0.8,fill=gray!20,draw=none](-3.387,1.03)--(-3.392,1.026)--(-3.448,1.04)--(-3.448,1.042)--(-3.435,1.063)--cycle;
\draw(-3.392,1.026)--(-3.448,1.04);
\filldraw[fill opacity=0.8,fill=gray!20](-3.395,.97)--(-3.364,1.018)--(-3.38,1.035)--(-3.415,.992)--cycle;
\filldraw[fill opacity=0.8,fill=gray!20,draw=none](-3.448,1.04)--(-3.449,1.04)--(-3.448,1.042)--cycle;
\draw(-3.448,1.04)--(-3.449,1.04);
\filldraw[fill opacity=0.8,fill=gray!20,draw=none](-3.277,1.007)--(-3.309,1.032)--(-3.236,1.014)--cycle;
\draw(-3.309,1.032)--(-3.236,1.014);
\filldraw[fill opacity=0.8,fill=gray!20,draw=none](-3.193,.953)--(-3.175,.957)--(-3.191,.994)--(-3.215,.989)--cycle;
\draw(-3.191,.994)--(-3.215,.989)--(-3.193,.953)--(-3.175,.957);
\filldraw[fill opacity=0.8,fill=gray!20,draw=none](-3.33,.954)--(-3.312,1.002)--(-3.313,1.006)--(-3.364,1.018)--(-3.395,.97)--cycle;
\draw(-3.313,1.006)--(-3.364,1.018)--(-3.395,.97)--(-3.33,.954)--(-3.312,1.002);
\filldraw[fill opacity=0.8,fill=gray!20,draw=none](-3.302,.952)--(-3.303,.989)--(-3.312,1.002)--(-3.33,.954)--cycle;
\draw(-3.312,1.002)--(-3.33,.954)--(-3.302,.952);
\filldraw[fill opacity=0.8,fill=gray!20,draw=none](-3.371,1.041)--(-3.38,1.035)--(-3.369,1.044)--cycle;
\draw(-3.371,1.041)--(-3.38,1.035)--(-3.369,1.044);
\filldraw[fill opacity=0.8,fill=gray!20,draw=none](-3.379,1.035)--(-3.386,1.03)--(-3.387,1.03)--(-3.384,1.033)--cycle;
\filldraw[fill opacity=0.8,fill=gray!20,draw=none](-3.387,1.03)--(-3.435,1.063)--(-3.367,1.046)--cycle;
\draw(-3.435,1.063)--(-3.367,1.046);
\filldraw[fill opacity=0.8,fill=gray!20,draw=none](-3.236,1.014)--(-3.359,1.044)--(-3.336,1.044)--(-3.213,1.014)--cycle;
\draw(-3.236,1.014)--(-3.359,1.044);
\draw(-3.336,1.044)--(-3.213,1.014);
\filldraw[fill opacity=0.8,fill=gray!20,draw=none](-3.369,1.024)--(-3.361,1.021)--(-3.336,1.044)--(-3.369,1.044)--(-3.38,1.035)--cycle;
\draw(-3.361,1.021)--(-3.336,1.044);
\draw(-3.369,1.044)--(-3.38,1.035)--(-3.369,1.024);
\filldraw[fill opacity=0.8,fill=gray!20,draw=none](-3.336,1.044)--(-3.343,1.063)--(-3.369,1.044)--cycle;
\draw(-3.343,1.063)--(-3.369,1.044);
\filldraw[fill opacity=0.8,fill=gray!20,draw=none](-3.379,1.035)--(-3.384,1.033)--(-3.367,1.046)--(-3.364,1.045)--cycle;
\draw(-3.367,1.046)--(-3.364,1.045);
\filldraw[fill opacity=0.8,fill=gray!20,draw=none](-3.379,1.035)--(-3.364,1.045)--(-3.359,1.044)--cycle;
\draw(-3.364,1.045)--(-3.359,1.044);
\filldraw[fill opacity=0.8,fill=gray!20,draw=none](-3.369,1.024)--(-3.364,1.018)--(-3.361,1.021)--cycle;
\draw(-3.369,1.024)--(-3.364,1.018)--(-3.361,1.021);
\filldraw[fill opacity=0.8,fill=gray!20,draw=none](-3.313,1.006)--(-3.314,1.024)--(-3.336,1.044)--(-3.364,1.018)--cycle;
\draw(-3.336,1.044)--(-3.364,1.018)--(-3.313,1.006);
\filldraw[fill opacity=0.8,fill=gray!20,draw=none](-3.338,1.012)--(-3.39,1.043)--(-3.367,1.007)--(-3.364,1.004)--(-3.344,.999)--cycle;
\draw(-3.364,1.004)--(-3.344,.999);
\filldraw[fill opacity=0.8,fill=gray!20,draw=none](-3.308,.99)--(-3.348,1)--(-3.345,.98)--(-3.317,.946)--(-3.315,.945)--cycle;
\draw(-3.308,.99)--(-3.348,1);
\draw(-3.317,.946)--(-3.315,.945);
\filldraw[fill opacity=0.8,fill=gray!20,draw=none](-3.345,.98)--(-3.348,1)--(-3.364,1.004)--cycle;
\draw(-3.348,1)--(-3.364,1.004);
\filldraw[fill opacity=0.8,fill=gray!20,draw=none](-3.345,.98)--(-3.342,.952)--(-3.317,.946)--cycle;
\draw(-3.342,.952)--(-3.317,.946);
\filldraw[fill opacity=0.8,fill=gray!20,draw=none](-3.345,.98)--(-3.364,1.004)--(-3.366,1.004)--(-3.344,.953)--(-3.342,.952)--cycle;
\draw(-3.364,1.004)--(-3.366,1.004);
\draw(-3.344,.953)--(-3.342,.952);
\filldraw[fill opacity=0.8,fill=gray!20,draw=none](-3.367,1.007)--(-3.366,1.004)--(-3.364,1.004)--cycle;
\draw(-3.366,1.004)--(-3.364,1.004);
\filldraw[fill opacity=0.8,fill=gray!20,draw=none](-3.343,.947)--(-3.342,.952)--(-3.344,.953)--cycle;
\draw(-3.342,.952)--(-3.344,.953);
\filldraw[fill opacity=0.8,fill=gray!20](-3.316,.935)--(-3.479,1.006)--(-3.489,1.06)--(-3.326,.989)--cycle;
\filldraw[fill opacity=0.8,fill=gray!20](-3.165,.609)--(-3.135,.631)--(-3.152,.62)--(-3.177,.601)--cycle;
\filldraw[fill opacity=0.8,fill=gray!20](-3.351,.65)--(-3.364,.683)--(-3.379,.699)--(-3.364,.664)--cycle;
\filldraw[fill opacity=0.8,fill=gray!20](-3.225,.668)--(-3.276,.669)--(-3.282,.676)--(-3.225,.668)--cycle;
\filldraw[fill opacity=0.8,fill=gray!20](-3.276,.669)--(-3.323,.686)--(-3.335,.698)--(-3.282,.676)--cycle;
\filldraw[fill opacity=0.8,fill=gray!20,draw=none](-3.119,.763)--(-3.118,.764)--(-3.114,.774)--(-3.135,.748)--cycle;
\draw(-3.119,.763)--(-3.118,.764);
\draw(-3.114,.774)--(-3.135,.748);
\filldraw[fill opacity=0.8,fill=gray!20,draw=none](-3.147,.729)--(-3.119,.763)--(-3.135,.748)--(-3.155,.724)--cycle;
\draw(-3.147,.729)--(-3.119,.763);
\draw(-3.135,.748)--(-3.155,.724);
\filldraw[fill opacity=0.8,fill=gray!20,draw=none](-2.943,1.151)--(-2.95,1.152)--(-2.938,1.153)--cycle;
\filldraw[fill opacity=0.8,fill=gray!20](-3.165,1.087)--(-3.194,1.099)--(-3.174,1.094)--(-3.127,1.078)--cycle;
\filldraw[fill opacity=0.8,fill=gray!20](-3.225,.668)--(-3.178,.668)--(-3.2,.664)--(-3.225,.668)--cycle;
\filldraw[fill opacity=0.8,fill=gray!20](-3.225,.668)--(-3.168,.674)--(-3.178,.668)--(-3.225,.668)--cycle;
\filldraw[fill opacity=0.8,fill=gray!20](-3.225,.668)--(-3.2,.664)--(-3.228,.663)--(-3.225,.668)--cycle;
\filldraw[fill opacity=0.8,fill=gray!20](-3.225,.668)--(-3.228,.663)--(-3.256,.665)--(-3.225,.668)--cycle;
\filldraw[fill opacity=0.8,fill=gray!20](-3.225,.668)--(-3.256,.665)--(-3.276,.669)--(-3.225,.668)--cycle;
\filldraw[fill opacity=0.8,fill=gray!20,draw=none](-3.073,1.043)--(-3.078,1.046)--(-3.079,1.044)--cycle;
\draw(-3.078,1.046)--(-3.079,1.044)--(-3.073,1.043);
\filldraw[fill opacity=0.8,fill=gray!20,draw=none](-3.046,.998)--(-3.045,.999)--(-3.058,1.023)--(-3.062,1.018)--cycle;
\draw(-3.046,.998)--(-3.045,.999);
\draw(-3.058,1.023)--(-3.062,1.018);
\filldraw[fill opacity=0.8,fill=gray!20,draw=none](-3.059,1.015)--(-3.058,1.026)--(-3.064,1.037)--(-3.073,1.043)--(-3.079,1.044)--(-3.08,1.041)--cycle;
\draw(-3.073,1.043)--(-3.079,1.044)--(-3.08,1.041);
\filldraw[fill opacity=0.8,fill=gray!20,draw=none](-3.045,.999)--(-3.016,1.034)--(-3.017,1.035)--(-3.044,1.04)--(-3.058,1.023)--cycle;
\draw(-3.045,.999)--(-3.016,1.034);
\draw(-3.044,1.04)--(-3.058,1.023);
\filldraw[fill opacity=0.8,fill=gray!20,draw=none](-3.064,1.037)--(-3.058,1.026)--(-3.057,1.033)--cycle;
\filldraw[fill opacity=0.8,fill=gray!20,draw=none](-3.027,.976)--(-3.004,.946)--(-2.859,.98)--(-2.876,1.027)--(-3.028,.991)--cycle;
\draw(-3.004,.946)--(-2.859,.98)--(-2.876,1.027)--(-3.028,.991);
\filldraw[fill opacity=0.8,fill=gray!20,draw=none](-3.047,1.028)--(-3.057,1.033)--(-3.058,1.026)--(-3.038,.99)--(-3.025,.974)--(-3.023,.982)--cycle;
\draw(-3.025,.974)--(-3.023,.982);
\filldraw[fill opacity=0.8,fill=gray!20,draw=none](-3.076,1.036)--(-3.065,1.024)--(-3.032,1.031)--(-3.059,1.05)--(-3.078,1.045)--cycle;
\draw(-3.065,1.024)--(-3.032,1.031);
\draw(-3.059,1.05)--(-3.078,1.045);
\filldraw[fill opacity=0.8,fill=gray!20,draw=none](-3.08,1.046)--(-3.079,1.044)--(-3.078,1.046)--cycle;
\draw(-3.08,1.046)--(-3.079,1.044)--(-3.078,1.046);
\filldraw[fill opacity=0.8,fill=gray!20](-2.839,.774)--(-2.824,.811)--(-2.92,.807)--(-2.918,.77)--cycle;
\filldraw[fill opacity=0.8,fill=gray!20](-2.732,.875)--(-2.725,.929)--(-2.81,.913)--(-2.814,.859)--cycle;
\filldraw[fill opacity=0.8,fill=gray!20,draw=none](-3.037,.988)--(-3.074,1.032)--(-3.083,1.044)--(-3.08,1.042)--(-3.036,.988)--(-3.035,.987)--cycle;
\draw(-3.074,1.032)--(-3.083,1.044);
\draw(-3.036,.988)--(-3.035,.987)--(-3.037,.988);
\filldraw[fill opacity=0.8,fill=gray!20](-3.135,.631)--(-3.111,.661)--(-3.132,.647)--(-3.152,.62)--cycle;
\filldraw[fill opacity=0.8,fill=gray!20,draw=none](-3.381,.734)--(-3.384,.732)--(-3.384,.736)--(-3.383,.736)--cycle;
\draw(-3.384,.732)--(-3.384,.736)--(-3.383,.736);
\filldraw[fill opacity=0.8,fill=gray!20](-3.364,.683)--(-3.369,.72)--(-3.384,.736)--(-3.379,.699)--cycle;
\filldraw[fill opacity=0.8,fill=gray!20](-3.168,.674)--(-3.115,.695)--(-3.133,.683)--(-3.178,.668)--cycle;
\filldraw[fill opacity=0.8,fill=gray!20](-3.364,.715)--(-3.395,.756)--(-3.415,.777)--(-3.38,.732)--cycle;
\filldraw[fill opacity=0.8,fill=gray!20,draw=none](-2.759,2.018)--(-2.74,1.994)--(-2.759,1.978)--(-2.779,2.018)--cycle;
\filldraw[fill opacity=0.8,fill=gray!20,draw=none](-3.054,1.048)--(-3.06,1.073)--(-3.078,1.046)--cycle;
\draw(-3.06,1.073)--(-3.078,1.046);
\filldraw[fill opacity=0.8,fill=gray!20,draw=none](-3.081,1.041)--(-3.08,1.041)--(-3.079,1.044)--cycle;
\draw(-3.08,1.041)--(-3.079,1.044);
\filldraw[fill opacity=0.8,fill=gray!20,draw=none](-3.395,.756)--(-3.409,.792)--(-3.417,.782)--(-3.415,.777)--cycle;
\draw(-3.417,.782)--(-3.415,.777)--(-3.395,.756)--(-3.409,.792);
\filldraw[fill opacity=0.8,fill=gray!20,draw=none](-3.498,.814)--(-3.515,.818)--(-3.52,.834)--(-3.511,.831)--cycle;
\draw(-3.498,.814)--(-3.515,.818);
\draw(-3.52,.834)--(-3.511,.831);
\filldraw[fill opacity=0.8,fill=gray!20,draw=none](-3.259,.756)--(-3.251,.759)--(-3.328,.778)--(-3.393,.788)--(-3.261,.756)--cycle;
\draw(-3.251,.759)--(-3.328,.778);
\draw(-3.393,.788)--(-3.261,.756)--(-3.259,.756);
\filldraw[fill opacity=0.8,fill=gray!20,draw=none](-3.359,.776)--(-3.353,.754)--(-3.351,.778)--cycle;
\draw(-3.353,.754)--(-3.351,.778);
\filldraw[fill opacity=0.8,fill=gray!20,draw=none](-3.365,.778)--(-3.364,.757)--(-3.356,.779)--cycle;
\draw(-3.364,.757)--(-3.356,.779);
\filldraw[fill opacity=0.8,fill=gray!20,draw=none](-3.351,.778)--(-3.355,.733)--(-3.324,.724)--(-3.319,.776)--cycle;
\draw(-3.351,.778)--(-3.355,.733)--(-3.324,.724)--(-3.319,.776);
\filldraw[fill opacity=0.8,fill=gray!20,draw=none](-3.317,.8)--(-3.319,.776)--(-3.296,.783)--cycle;
\draw(-3.317,.8)--(-3.319,.776);
\filldraw[fill opacity=0.8,fill=gray!20,draw=none](-3.303,.782)--(-3.297,.783)--(-3.306,.787)--(-3.308,.782)--cycle;
\draw(-3.306,.787)--(-3.308,.782)--(-3.303,.782);
\filldraw[fill opacity=0.8,fill=gray!20,draw=none](-3.508,.843)--(-3.511,.831)--(-3.52,.834)--cycle;
\draw(-3.511,.831)--(-3.52,.834);
\filldraw[fill opacity=0.8,fill=gray!20,draw=none](-3.328,.778)--(-3.421,.801)--(-3.492,.812)--(-3.393,.788)--cycle;
\draw(-3.328,.778)--(-3.421,.801);
\draw(-3.492,.812)--(-3.393,.788);
\filldraw[fill opacity=0.8,fill=gray!20,draw=none](-3.397,.767)--(-3.377,.79)--(-3.409,.792)--(-3.401,.771)--cycle;
\draw(-3.409,.792)--(-3.401,.771);
\filldraw[fill opacity=0.8,fill=gray!20,draw=none](-3.421,.801)--(-3.475,.814)--(-3.515,.818)--(-3.492,.812)--cycle;
\draw(-3.421,.801)--(-3.475,.814);
\draw(-3.515,.818)--(-3.492,.812);
\filldraw[fill opacity=0.8,fill=gray!20,draw=none](-3.449,.817)--(-3.475,.814)--(-3.466,.812)--cycle;
\draw(-3.475,.814)--(-3.466,.812);
\filldraw[fill opacity=0.8,fill=gray!20](-3.447,.778)--(-3.61,.849)--(-3.56,.866)--(-3.397,.794)--cycle;
\filldraw[fill opacity=0.8,fill=gray!20](-3.259,.589)--(-3.278,.597)--(-3.303,.603)--(-3.272,.592)--cycle;
\filldraw[fill opacity=0.8,fill=gray!20,draw=none](-3.061,1.004)--(-3.059,1.015)--(-3.08,1.041)--(-3.09,1.012)--cycle;
\draw(-3.08,1.041)--(-3.09,1.012);
\filldraw[fill opacity=0.8,fill=gray!20,draw=none](-3.094,.98)--(-3.071,1.007)--(-3.09,1.012)--cycle;
\draw(-3.094,.98)--(-3.071,1.007);
\filldraw[fill opacity=0.8,fill=gray!20,draw=none](-3.091,1.003)--(-3.09,1.012)--(-3.097,1.005)--cycle;
\draw(-3.09,1.012)--(-3.097,1.005);
\filldraw[fill opacity=0.8,fill=gray!20,draw=none](-3.294,1.697)--(-3.331,1.675)--(-3.332,1.676)--(-3.317,1.713)--(-3.314,1.715)--cycle;
\draw(-3.331,1.675)--(-3.332,1.676)--(-3.317,1.713)--(-3.314,1.715);
\filldraw[fill opacity=0.8,fill=gray!20,draw=none](-3.044,1.04)--(-3.032,1.031)--(-3.017,1.035)--cycle;
\draw(-3.032,1.031)--(-3.017,1.035);
\filldraw[fill opacity=0.8,fill=gray!20,draw=none](-2.986,1.025)--(-2.951,1.068)--(-2.995,1.059)--(-3.016,1.034)--cycle;
\draw(-2.986,1.025)--(-2.951,1.068);
\draw(-2.995,1.059)--(-3.016,1.034);
\filldraw[fill opacity=0.8,fill=gray!20,draw=none](-3.03,.996)--(-3.028,.991)--(-2.987,1.001)--(-3.017,1.035)--(-3.032,1.031)--cycle;
\draw(-3.028,.991)--(-2.987,1.001);
\draw(-3.017,1.035)--(-3.032,1.031);
\filldraw[fill opacity=0.8,fill=gray!20,draw=none](-3.017,1.035)--(-3.016,1.034)--(-3.016,1.034)--cycle;
\draw(-3.016,1.034)--(-3.016,1.034);
\filldraw[fill opacity=0.8,fill=gray!20,draw=none](-3.016,1.034)--(-2.987,1.001)--(-2.938,1.012)--cycle;
\draw(-2.987,1.001)--(-2.938,1.012);
\filldraw[fill opacity=0.8,fill=gray!20,draw=none](-3.073,1.043)--(-3.013,1.028)--(-3.004,1.054)--(-3.078,1.046)--cycle;
\draw(-3.073,1.043)--(-3.013,1.028)--(-3.004,1.054);
\filldraw[fill opacity=0.8,fill=gray!20,draw=none](-3.076,1.036)--(-3.078,1.045)--(-3.083,1.044)--cycle;
\draw(-3.078,1.045)--(-3.083,1.044);
\filldraw[fill opacity=0.8,fill=gray!20,draw=none](-3.383,.736)--(-3.384,.736)--(-3.384,.738)--cycle;
\draw(-3.383,.736)--(-3.384,.736)--(-3.384,.738);
\filldraw[fill opacity=0.8,fill=gray!20,draw=none](-3.107,.776)--(-3.105,.783)--(-3.12,.787)--cycle;
\filldraw[fill opacity=0.8,fill=gray!20,draw=none](-3.155,.724)--(-3.114,.774)--(-3.13,.776)--(-3.175,.721)--cycle;
\draw(-3.155,.724)--(-3.114,.774);
\draw(-3.13,.776)--(-3.175,.721);
\filldraw[fill opacity=0.8,fill=gray!20,draw=none](-3.109,.766)--(-3.107,.776)--(-3.12,.787)--(-3.132,.79)--(-3.127,.775)--cycle;
\draw(-3.132,.79)--(-3.127,.775);
\filldraw[fill opacity=0.8,fill=gray!20,draw=none](-3.107,.79)--(-3.113,.716)--(-3.107,.725)--(-3.103,.776)--cycle;
\draw(-3.107,.79)--(-3.113,.716)--(-3.107,.725)--(-3.103,.776);
\filldraw[fill opacity=0.8,fill=gray!20](-3.206,.591)--(-3.177,.601)--(-3.205,.596)--(-3.221,.588)--cycle;
\filldraw[fill opacity=0.8,fill=gray!20,draw=none](-3.102,.776)--(-3.102,.777)--(-3.099,.777)--(-3.102,.776)--cycle;
\draw(-3.099,.777)--(-3.102,.776);
\filldraw[fill opacity=0.8,fill=gray!20,draw=none](-2.962,1.14)--(-2.966,1.147)--(-2.939,1.15)--(-2.926,1.148)--cycle;
\filldraw[fill opacity=0.8,fill=gray!20,draw=none](-2.86,1.12)--(-2.876,1.141)--(-2.889,1.143)--(-2.913,1.139)--(-2.915,1.117)--cycle;
\draw(-2.913,1.139)--(-2.915,1.117)--(-2.86,1.12)--(-2.876,1.141);
\filldraw[fill opacity=0.8,fill=gray!20,draw=none](-2.829,1.126)--(-2.876,1.141)--(-2.86,1.12)--cycle;
\draw(-2.876,1.141)--(-2.86,1.12)--(-2.829,1.126);
\filldraw[fill opacity=0.8,fill=gray!20,draw=none](-3.083,1.044)--(-3.086,1.049)--(-3.08,1.042)--cycle;
\draw(-3.083,1.044)--(-3.086,1.049)--(-3.08,1.042);
\filldraw[fill opacity=0.8,fill=gray!20,draw=none](-2.889,1.143)--(-2.913,1.145)--(-2.913,1.139)--cycle;
\draw(-2.913,1.145)--(-2.913,1.139);
\filldraw[fill opacity=0.8,fill=gray!20,draw=none](-3.165,1)--(-3.072,1.022)--(-3.076,1.036)--(-3.083,1.044)--(-3.185,1.021)--cycle;
\draw(-3.165,1)--(-3.072,1.022);
\draw(-3.083,1.044)--(-3.185,1.021);
\filldraw[fill opacity=0.8,fill=gray!20,draw=none](-3.175,.721)--(-3.13,.776)--(-3.163,.77)--(-3.205,.72)--cycle;
\draw(-3.175,.721)--(-3.13,.776);
\draw(-3.163,.77)--(-3.205,.72);
\filldraw[fill opacity=0.8,fill=gray!20,draw=none](-3.127,.775)--(-3.134,.764)--(-3.139,.71)--(-3.113,.716)--(-3.109,.766)--cycle;
\draw(-3.134,.764)--(-3.139,.71)--(-3.113,.716)--(-3.109,.766);
\filldraw[fill opacity=0.8,fill=gray!20,draw=none](-3.114,.698)--(-3.095,.71)--(-3.092,.732)--(-3.116,.716)--(-3.118,.7)--cycle;
\draw(-3.095,.71)--(-3.092,.732)--(-3.116,.716)--(-3.118,.7);
\filldraw[fill opacity=0.8,fill=gray!20](-3.115,.695)--(-3.07,.728)--(-3.095,.712)--(-3.133,.683)--cycle;
\filldraw[fill opacity=0.8,fill=gray!20](-3.111,.661)--(-3.097,.695)--(-3.12,.68)--(-3.132,.647)--cycle;
\filldraw[fill opacity=0.8,fill=gray!20,draw=none](-3.369,.72)--(-3.369,.721)--(-3.384,.738)--(-3.384,.736)--cycle;
\draw(-3.384,.738)--(-3.384,.736)--(-3.369,.72)--(-3.369,.721);
\filldraw[fill opacity=0.8,fill=gray!20,draw=none](-2.738,1.761)--(-2.763,1.784)--(-2.781,1.92)--cycle;
\draw(-2.763,1.784)--(-2.781,1.92);
\filldraw[fill opacity=0.8,fill=gray!20,draw=none](-2.74,1.994)--(-2.73,1.918)--(-2.759,1.978)--cycle;
\draw(-2.74,1.994)--(-2.73,1.918);
\filldraw[fill opacity=0.8,fill=gray!20,draw=none](-3.386,1.03)--(-3.392,1.026)--(-3.387,1.03)--cycle;
\filldraw[fill opacity=0.8,fill=gray!20,draw=none](-3.221,.994)--(-3.202,.992)--(-3.191,.994)--(-3.213,1.014)--(-3.238,1.008)--cycle;
\draw(-3.202,.992)--(-3.191,.994);
\draw(-3.213,1.014)--(-3.238,1.008)--(-3.221,.994);
\filldraw[fill opacity=0.8,fill=gray!20,draw=none](-3.359,1.044)--(-3.435,1.063)--(-3.439,1.069)--(-3.437,1.069)--(-3.336,1.044)--cycle;
\draw(-3.359,1.044)--(-3.435,1.063);
\draw(-3.437,1.069)--(-3.336,1.044);
\filldraw[fill opacity=0.8,fill=gray!20,draw=none](-3.448,1.042)--(-3.452,1.067)--(-3.435,1.063)--cycle;
\draw(-3.452,1.067)--(-3.435,1.063);
\filldraw[fill opacity=0.8,fill=gray!20,draw=none](-3.221,.994)--(-3.215,.989)--(-3.202,.992)--cycle;
\draw(-3.221,.994)--(-3.215,.989)--(-3.202,.992);
\filldraw[fill opacity=0.8,fill=gray!20,draw=none](-3.435,1.063)--(-3.45,1.066)--(-3.449,1.067)--(-3.439,1.069)--cycle;
\draw(-3.435,1.063)--(-3.45,1.066);
\filldraw[fill opacity=0.8,fill=gray!20,draw=none](-3.45,1.066)--(-3.452,1.067)--(-3.449,1.067)--cycle;
\draw(-3.45,1.066)--(-3.452,1.067);
\filldraw[fill opacity=0.8,fill=gray!20,draw=none](-3.362,1.051)--(-3.437,1.069)--(-3.433,1.066)--(-3.39,1.043)--(-3.322,1.026)--cycle;
\draw(-3.362,1.051)--(-3.437,1.069);
\draw(-3.39,1.043)--(-3.322,1.026);
\filldraw[fill opacity=0.8,fill=gray!20,draw=none](-3.314,1.024)--(-3.39,1.043)--(-3.313,.998)--cycle;
\draw(-3.314,1.024)--(-3.39,1.043);
\filldraw[fill opacity=0.8,fill=gray!20,draw=none](-3.433,1.066)--(-3.406,1.047)--(-3.39,1.043)--cycle;
\draw(-3.406,1.047)--(-3.39,1.043);
\filldraw[fill opacity=0.8,fill=gray!20,draw=none](-3.338,1.012)--(-3.344,.999)--(-3.292,.986)--cycle;
\draw(-3.344,.999)--(-3.292,.986);
\filldraw[fill opacity=0.8,fill=gray!20,draw=none](-3.367,1.007)--(-3.39,1.043)--(-3.406,1.047)--cycle;
\draw(-3.39,1.043)--(-3.406,1.047);
\filldraw[fill opacity=0.8,fill=gray!20](-3.326,.989)--(-3.489,1.06)--(-3.518,1.103)--(-3.354,1.032)--cycle;
\filldraw[fill opacity=0.8,fill=gray!20,draw=none](-3.102,.776)--(-3.102,.776)--(-3.102,.774)--cycle;
\filldraw[fill opacity=0.8,fill=gray!20,draw=none](-3.102,.774)--(-3.102,.782)--(-3.097,.769)--cycle;
\draw(-3.102,.782)--(-3.097,.769)--(-3.102,.774);
\filldraw[fill opacity=0.8,fill=gray!20,draw=none](-3.08,1.042)--(-3.086,1.049)--(-3.127,1.078)--(-3.115,1.066)--(-3.078,1.038)--cycle;
\draw(-3.08,1.042)--(-3.086,1.049)--(-3.127,1.078)--(-3.115,1.066)--(-3.078,1.038);
\filldraw[fill opacity=0.8,fill=gray!20,draw=none](-3.353,1.931)--(-3.317,1.713)--(-3.332,1.676)--(-3.369,1.961)--cycle;
\draw(-3.317,1.713)--(-3.332,1.676)--(-3.369,1.961);
\filldraw[fill opacity=0.8,fill=gray!20,draw=none](-3.353,1.931)--(-3.342,1.908)--(-3.317,1.713)--cycle;
\draw(-3.342,1.908)--(-3.317,1.713);
\filldraw[fill opacity=0.8,fill=gray!20,draw=none](-3.064,1.037)--(-3.066,1.041)--(-3.073,1.043)--cycle;
\draw(-3.066,1.041)--(-3.073,1.043);
\filldraw[fill opacity=0.8,fill=gray!20,draw=none](-2.926,1.148)--(-2.898,1.154)--(-2.728,1.161)--(-2.719,1.146)--(-2.729,1.139)--(-2.775,1.124)--cycle;
\draw(-2.719,1.146)--(-2.729,1.139)--(-2.775,1.124);
\filldraw[fill opacity=0.8,fill=gray!20,draw=none](-3.102,.776)--(-3.102,.776)--(-3.103,.776)--cycle;
\draw(-3.102,.776)--(-3.103,.776);
\filldraw[fill opacity=0.8,fill=gray!20,draw=none](-3.097,.709)--(-3.095,.71)--(-3.095,.707)--cycle;
\draw(-3.095,.71)--(-3.095,.707);
\filldraw[fill opacity=0.8,fill=gray!20,draw=none](-3.034,.98)--(-3.035,.987)--(-3.037,.989)--(-3.04,.988)--cycle;
\draw(-3.037,.989)--(-3.04,.988);
\filldraw[fill opacity=0.8,fill=gray!20,draw=none](-3.169,.989)--(-3.314,1.024)--(-3.313,.998)--(-3.292,.986)--(-3.158,.953)--cycle;
\draw(-3.292,.986)--(-3.158,.953)--(-3.169,.989)--(-3.314,1.024);
\filldraw[fill opacity=0.8,fill=gray!20,draw=none](-3.158,.953)--(-3.175,.957)--(-3.215,.921)--(-3.157,.906)--cycle;
\draw(-3.215,.921)--(-3.157,.906)--(-3.158,.953)--(-3.175,.957);
\filldraw[fill opacity=0.8,fill=gray!20,draw=none](-3.115,.921)--(-3.026,.941)--(-3.034,.98)--(-3.04,.988)--(-3.175,.957)--cycle;
\draw(-3.115,.921)--(-3.026,.941);
\draw(-3.04,.988)--(-3.175,.957);
\filldraw[fill opacity=0.8,fill=gray!20,draw=none](-3.021,.929)--(-3.013,.935)--(-3.035,.987)--(-3.043,.981)--cycle;
\draw(-3.021,.929)--(-3.013,.935)--(-3.035,.987)--(-3.043,.981);
\filldraw[fill opacity=0.8,fill=gray!20,draw=none](-3.018,.932)--(-3.012,.914)--(-3.005,.914)--(-3.009,.945)--(-3.02,.961)--cycle;
\draw(-3.012,.914)--(-3.005,.914);
\filldraw[fill opacity=0.8,fill=gray!20,draw=none](-2.995,.896)--(-2.85,.93)--(-2.859,.98)--(-3.015,.944)--cycle;
\draw(-2.995,.896)--(-2.85,.93)--(-2.859,.98)--(-3.015,.944);
\filldraw[fill opacity=0.8,fill=gray!20,draw=none](-3.034,.789)--(-3.022,.81)--(-3.017,.815)--(-3.028,.788)--cycle;
\draw(-3.017,.815)--(-3.028,.788);
\filldraw[fill opacity=0.8,fill=gray!20](-2.75,1.04)--(-2.779,1.088)--(-2.839,1.077)--(-2.824,1.026)--cycle;
\filldraw[fill opacity=0.8,fill=gray!20](-2.725,.929)--(-2.732,.985)--(-2.814,.97)--(-2.81,.913)--cycle;
\filldraw[fill opacity=0.8,fill=gray!20,draw=none](-2.948,3.029)--(-2.859,2.329)--(-2.901,2.345)--(-3.017,2.702)--(-3.059,3.033)--cycle;
\draw(-3.017,2.702)--(-3.059,3.033)--(-2.948,3.029)--(-2.859,2.329);
\filldraw[fill opacity=0.8,fill=gray!20,draw=none](-3.111,2.451)--(-3.098,2.353)--(-3.208,2.392)--(-3.213,2.431)--cycle;
\draw(-3.208,2.392)--(-3.213,2.431)--(-3.111,2.451)--(-3.098,2.353);
\filldraw[fill opacity=0.8,fill=gray!20,draw=none](-3.097,.695)--(-3.095,.71)--(-3.118,.695)--(-3.12,.68)--cycle;
\draw(-3.118,.695)--(-3.12,.68)--(-3.097,.695)--(-3.095,.71);
\filldraw[fill opacity=0.8,fill=gray!20](-3.24,.588)--(-3.242,.594)--(-3.278,.597)--(-3.259,.589)--cycle;
\filldraw[fill opacity=0.8,fill=gray!20,draw=none](-3.343,1.063)--(-3.335,1.069)--(-3.317,1.081)--(-3.327,1.073)--cycle;
\draw(-3.343,1.063)--(-3.335,1.069)--(-3.317,1.081)--(-3.327,1.073);
\filldraw[fill opacity=0.8,fill=gray!20](-3.07,.728)--(-3.035,.772)--(-3.066,.752)--(-3.095,.712)--cycle;
\filldraw[fill opacity=0.8,fill=gray!20,draw=none](-3.064,1.037)--(-3.057,1.033)--(-3.056,1.039)--(-3.066,1.041)--cycle;
\draw(-3.056,1.039)--(-3.066,1.041);
\filldraw[fill opacity=0.8,fill=gray!20](-3.221,.588)--(-3.205,.596)--(-3.242,.594)--(-3.24,.588)--cycle;
\filldraw[fill opacity=0.5,fill=gray!20](-1.447,2.325)--(-1.58,2.363)--(-1.397,2.865)--(-1.272,2.805)--cycle;
\filldraw[fill opacity=0.8,fill=gray!20](-2.839,1.077)--(-2.86,1.12)--(-2.915,1.117)--(-2.918,1.073)--cycle;
\filldraw[fill opacity=0.8,fill=gray!20,draw=none](-2.915,1.091)--(-2.933,1.09)--(-2.962,1.14)--(-2.926,1.148)--(-2.775,1.124)--(-2.813,1.111)--cycle;
\draw(-2.775,1.124)--(-2.813,1.111)--(-2.915,1.091)--(-2.933,1.09);
\filldraw[fill opacity=0.8,fill=gray!20](-3.278,.597)--(-3.295,.614)--(-3.33,.623)--(-3.303,.603)--cycle;
\filldraw[fill opacity=0.8,fill=gray!20,draw=none](-2.696,1.887)--(-2.68,1.767)--(-2.692,1.777)--(-2.73,1.918)--(-2.74,1.994)--cycle;
\draw(-2.696,1.887)--(-2.68,1.767)--(-2.692,1.777);
\draw(-2.73,1.918)--(-2.74,1.994);
\filldraw[fill opacity=0.8,fill=gray!20](-3.256,.665)--(-3.286,.676)--(-3.323,.686)--(-3.276,.669)--cycle;
\filldraw[fill opacity=0.8,fill=gray!20](-3.497,.783)--(-3.661,.855)--(-3.61,.849)--(-3.447,.778)--cycle;
\filldraw[fill opacity=0.8,fill=gray!20](-2.732,.985)--(-2.75,1.04)--(-2.824,1.026)--(-2.814,.97)--cycle;
\filldraw[fill opacity=0.8,fill=gray!20,draw=none](-3.072,1.022)--(-3.065,1.024)--(-3.076,1.036)--cycle;
\draw(-3.072,1.022)--(-3.065,1.024);
\filldraw[fill opacity=0.8,fill=gray!20,draw=none](-3.092,.732)--(-3.097,.769)--(-3.111,.76)--(-3.117,.725)--(-3.116,.716)--cycle;
\draw(-3.117,.725)--(-3.116,.716)--(-3.092,.732)--(-3.097,.769)--(-3.111,.76);
\filldraw[fill opacity=0.8,fill=gray!20,draw=none](-3.013,.812)--(-3.013,.812)--(-3.013,.814)--cycle;
\draw(-3.013,.812)--(-3.013,.814);
\filldraw[fill opacity=0.8,fill=gray!20](-3.335,1.069)--(-3.282,1.09)--(-3.272,1.096)--(-3.317,1.081)--cycle;
\filldraw[fill opacity=0.8,fill=gray!20,draw=none](-3.005,2.46)--(-2.992,2.361)--(-3.103,2.387)--(-3.111,2.451)--cycle;
\draw(-3.103,2.387)--(-3.111,2.451)--(-3.005,2.46)--(-2.992,2.361);
\filldraw[fill opacity=0.8,fill=gray!20,draw=none](-2.81,2.02)--(-2.79,1.997)--(-2.81,1.98)--(-2.829,2.02)--cycle;
\filldraw[fill opacity=0.8,fill=gray!20,draw=none](-3.011,.814)--(-3.013,.814)--(-3.013,.812)--cycle;
\draw(-3.011,.814)--(-3.013,.814)--(-3.013,.812);
\filldraw[fill opacity=0.8,fill=gray!20,draw=none](-2.743,2.018)--(-2.74,1.994)--(-2.759,2.018)--cycle;
\draw(-2.743,2.018)--(-2.74,1.994);
\filldraw[fill opacity=0.8,fill=gray!20](-3.178,.668)--(-3.133,.683)--(-3.176,.675)--(-3.2,.664)--cycle;
\filldraw[fill opacity=0.8,fill=gray!20](-3.177,.601)--(-3.152,.62)--(-3.192,.613)--(-3.205,.596)--cycle;
\filldraw[fill opacity=0.8,fill=gray!20,draw=none](-3.011,.814)--(-2.992,.825)--(-2.998,.849)--(-3.013,.814)--cycle;
\draw(-3.013,.814)--(-3.011,.814);
\filldraw[fill opacity=0.8,fill=gray!20,draw=none](-3.011,.814)--(-2.92,.807)--(-2.921,.854)--(-2.942,.855)--cycle;
\draw(-3.011,.814)--(-2.92,.807)--(-2.921,.854)--(-2.942,.855);
\filldraw[fill opacity=0.8,fill=gray!20,draw=none](-3.03,.996)--(-3.032,1.031)--(-3.047,1.028)--cycle;
\draw(-3.032,1.031)--(-3.047,1.028);
\filldraw[fill opacity=0.8,fill=gray!20,draw=none](-3.057,1.033)--(-3.017,1.011)--(-3.013,1.028)--(-3.056,1.039)--cycle;
\draw(-3.017,1.011)--(-3.013,1.028)--(-3.056,1.039);
\filldraw[fill opacity=0.8,fill=gray!20,draw=none](-3.022,.81)--(-3.014,.824)--(-3.013,.824)--(-3.017,.815)--cycle;
\draw(-3.014,.824)--(-3.013,.824)--(-3.017,.815);
\filldraw[fill opacity=0.8,fill=gray!20,draw=none](-3.035,.772)--(-3.017,.815)--(-3.019,.82)--(-3.048,.801)--(-3.066,.752)--cycle;
\draw(-3.019,.82)--(-3.048,.801)--(-3.066,.752)--(-3.035,.772)--(-3.017,.815);
\filldraw[fill opacity=0.8,fill=gray!20,draw=none](-3.466,.94)--(-3.482,.904)--(-3.597,.932)--(-3.588,.982)--(-3.465,.952)--cycle;
\draw(-3.482,.904)--(-3.597,.932)--(-3.588,.982)--(-3.465,.952);
\filldraw[fill opacity=0.8,fill=gray!20,draw=none](-3.551,1.024)--(-3.588,.982)--(-3.607,1.023)--cycle;
\draw(-3.588,.982)--(-3.607,1.023);
\filldraw[fill opacity=0.8,fill=gray!20,draw=none](-3.637,1.108)--(-3.596,1.119)--(-3.551,1.024)--(-3.607,1.023)--(-3.644,1.103)--cycle;
\draw(-3.596,1.119)--(-3.551,1.024);
\draw(-3.607,1.023)--(-3.644,1.103);
\filldraw[fill opacity=0.8,fill=gray!20,draw=none](-3.453,.995)--(-3.465,.952)--(-3.588,.982)--(-3.569,1.028)--(-3.453,1)--cycle;
\draw(-3.465,.952)--(-3.588,.982)--(-3.569,1.028)--(-3.453,1);
\filldraw[fill opacity=0.8,fill=gray!20,draw=none](-3.038,.99)--(-3.058,1.026)--(-3.059,1.015)--cycle;
\filldraw[fill opacity=0.8,fill=gray!20,draw=none](-2.997,2.46)--(-2.96,2.348)--(-2.992,2.361)--(-3.005,2.46)--cycle;
\draw(-2.992,2.361)--(-3.005,2.46)--(-2.997,2.46);
\filldraw[fill opacity=0.8,fill=gray!20,draw=none](-3.488,1.102)--(-3.483,1.093)--(-3.509,1.107)--cycle;
\draw(-3.488,1.102)--(-3.483,1.093);
\filldraw[fill opacity=0.8,fill=gray!20,draw=none](-3.454,1.091)--(-3.446,1.074)--(-3.483,1.093)--(-3.488,1.102)--cycle;
\draw(-3.454,1.091)--(-3.446,1.074);
\draw(-3.483,1.093)--(-3.488,1.102);
\filldraw[fill opacity=0.8,fill=gray!20,draw=none](-3.439,1.069)--(-3.44,1.07)--(-3.437,1.069)--cycle;
\draw(-3.44,1.07)--(-3.437,1.069);
\filldraw[fill opacity=0.8,fill=gray!20,draw=none](-3.433,1.066)--(-3.437,1.069)--(-3.44,1.07)--cycle;
\draw(-3.437,1.069)--(-3.44,1.07);
\filldraw[fill opacity=0.8,fill=gray!20,draw=none](-3.443,1.083)--(-3.438,1.073)--(-3.446,1.074)--(-3.454,1.091)--cycle;
\draw(-3.443,1.083)--(-3.438,1.073);
\draw(-3.446,1.074)--(-3.454,1.091);
\filldraw[fill opacity=0.8,fill=gray!20,draw=none](-3.448,1.081)--(-3.438,1.073)--(-3.443,1.083)--cycle;
\draw(-3.438,1.073)--(-3.443,1.083);
\filldraw[fill opacity=0.8,fill=gray!20](-3.354,1.032)--(-3.518,1.103)--(-3.56,1.13)--(-3.397,1.058)--cycle;
\filldraw[fill opacity=0.8,fill=gray!20,draw=none](-3.107,.776)--(-3.098,.768)--(-3.097,.769)--(-3.102,.782)--(-3.105,.783)--cycle;
\draw(-3.098,.768)--(-3.097,.769)--(-3.102,.782);
\filldraw[fill opacity=0.8,fill=gray!20](-3.25,1.1)--(-3.225,1.096)--(-3.225,1.096)--(-3.222,1.101)--cycle;
\filldraw[fill opacity=0.8,fill=gray!20](-3.222,1.101)--(-3.225,1.096)--(-3.225,1.096)--(-3.194,1.099)--cycle;
\filldraw[fill opacity=0.8,fill=gray!20,draw=none](-3.056,1.011)--(-3.037,.988)--(-3.044,.996)--cycle;
\draw(-3.037,.988)--(-3.044,.996);
\filldraw[fill opacity=0.8,fill=gray!20,draw=none](-2.87,3.01)--(-2.843,2.805)--(-2.843,2.32)--(-2.859,2.329)--(-2.948,3.029)--cycle;
\draw(-2.859,2.329)--(-2.948,3.029)--(-2.87,3.01)--(-2.843,2.805);
\filldraw[fill opacity=0.8,fill=gray!20,draw=none](-2.79,1.997)--(-2.781,1.92)--(-2.81,1.98)--cycle;
\draw(-2.79,1.997)--(-2.781,1.92);
\filldraw[fill opacity=0.8,fill=gray!20,draw=none](-3.111,.76)--(-3.098,.768)--(-3.107,.776)--cycle;
\draw(-3.111,.76)--(-3.098,.768);
\filldraw[fill opacity=0.8,fill=gray!20,draw=none](-3.044,.996)--(-3.045,.999)--(-3.046,.998)--cycle;
\draw(-3.045,.999)--(-3.046,.998);
\filldraw[fill opacity=0.8,fill=gray!20,draw=none](-3.155,.962)--(-3.037,.989)--(-3.045,1.023)--(-3.047,1.028)--(-3.165,1)--cycle;
\draw(-3.155,.962)--(-3.037,.989);
\draw(-3.047,1.028)--(-3.165,1);
\filldraw[fill opacity=0.8,fill=gray!20,draw=none](-3.08,1.042)--(-3.08,1.042)--(-3.07,1.031)--(-3.036,.988)--cycle;
\draw(-3.08,1.042)--(-3.07,1.031)--(-3.036,.988);
\filldraw[fill opacity=0.8,fill=gray!20](-3.127,1.078)--(-3.174,1.094)--(-3.168,1.088)--(-3.115,1.066)--cycle;
\filldraw[fill opacity=0.8,fill=gray!20,draw=none](-3.259,.756)--(-3.234,.745)--(-3.248,.756)--(-3.252,.759)--cycle;
\draw(-3.234,.745)--(-3.248,.756)--(-3.252,.759);
\filldraw[fill opacity=0.8,fill=gray!20,draw=none](-3.08,1.042)--(-3.078,1.038)--(-3.07,1.031)--cycle;
\draw(-3.078,1.038)--(-3.07,1.031)--(-3.08,1.042);
\filldraw[fill opacity=0.8,fill=gray!20](-3.295,.614)--(-3.308,.639)--(-3.351,.65)--(-3.33,.623)--cycle;
\filldraw[fill opacity=0.8,fill=gray!20](-2.824,.811)--(-2.814,.859)--(-2.921,.854)--(-2.92,.807)--cycle;
\filldraw[fill opacity=0.8,fill=gray!20](-3.272,1.096)--(-3.225,1.096)--(-3.225,1.096)--(-3.25,1.1)--cycle;
\filldraw[fill opacity=0.8,fill=gray!20,draw=none](-3.484,.895)--(-3.597,.924)--(-3.597,.932)--(-3.482,.904)--cycle;
\draw(-3.597,.924)--(-3.597,.932)--(-3.482,.904);
\filldraw[fill opacity=0.8,fill=gray!20,draw=none](-3.014,.824)--(-3.013,.827)--(-3.013,.824)--cycle;
\draw(-3.013,.827)--(-3.013,.824)--(-3.014,.824);
\filldraw[fill opacity=0.8,fill=gray!20,draw=none](-3.017,.815)--(-3.013,.824)--(-3.019,.82)--cycle;
\draw(-3.017,.815)--(-3.013,.824)--(-3.019,.82);
\filldraw[fill opacity=0.8,fill=gray!20,draw=none](-3.047,1.028)--(-3.023,.982)--(-3.017,1.011)--cycle;
\draw(-3.023,.982)--(-3.017,1.011);
\filldraw[fill opacity=0.8,fill=gray!20](-3.194,1.099)--(-3.225,1.096)--(-3.225,1.096)--(-3.174,1.094)--cycle;
\filldraw[fill opacity=0.8,fill=gray!20](-3.228,.663)--(-3.232,.672)--(-3.286,.676)--(-3.256,.665)--cycle;
\filldraw[fill opacity=0.8,fill=gray!20,draw=none](-3.018,.821)--(-3.013,.824)--(-3.013,.827)--(-3.016,.845)--cycle;
\draw(-3.018,.821)--(-3.013,.824)--(-3.013,.827);
\filldraw[fill opacity=0.8,fill=gray!20,draw=none](-2.731,2.018)--(-2.713,2.019)--(-2.696,1.887)--(-2.74,1.994)--cycle;
\draw(-2.713,2.019)--(-2.696,1.887);
\filldraw[fill opacity=0.8,fill=gray!20,draw=none](-2.731,2.018)--(-2.74,1.994)--(-2.743,2.018)--cycle;
\draw(-2.74,1.994)--(-2.743,2.018);
\filldraw[fill opacity=0.8,fill=gray!20,draw=none](-2.992,.825)--(-2.942,.855)--(-2.994,.859)--(-2.998,.849)--cycle;
\draw(-2.942,.855)--(-2.994,.859);
\filldraw[fill opacity=0.8,fill=gray!20,draw=none](-2.96,2.348)--(-2.997,2.46)--(-2.913,2.456)--(-2.896,2.323)--cycle;
\draw(-2.997,2.46)--(-2.913,2.456)--(-2.896,2.323);
\filldraw[fill opacity=0.8,fill=gray!20,draw=none](-2.815,2.303)--(-2.843,2.32)--(-2.843,2.805)--(-2.789,2.379)--cycle;
\draw(-2.843,2.805)--(-2.789,2.379);
\filldraw[fill opacity=0.8,fill=gray!20,draw=none](-3.213,2.431)--(-3.208,2.392)--(-3.291,2.361)--(-3.297,2.403)--cycle;
\draw(-3.291,2.361)--(-3.297,2.403)--(-3.213,2.431)--(-3.208,2.392);
\filldraw[fill opacity=0.8,fill=gray!20,draw=none](-2.893,2.452)--(-2.893,2.305)--(-2.913,2.456)--cycle;
\draw(-2.893,2.305)--(-2.913,2.456)--(-2.893,2.452);
\filldraw[fill opacity=0.8,fill=gray!20,draw=none](-2.893,2.322)--(-2.893,2.452)--(-2.847,2.441)--(-2.84,2.381)--(-2.866,2.305)--cycle;
\draw(-2.893,2.452)--(-2.847,2.441)--(-2.84,2.381);
\filldraw[fill opacity=0.8,fill=gray!20,draw=none](-2.866,2.305)--(-2.84,2.381)--(-2.827,2.281)--cycle;
\draw(-2.84,2.381)--(-2.827,2.281);
\filldraw[fill opacity=0.8,fill=gray!20,draw=none](-3.327,2.384)--(-3.297,2.403)--(-3.296,2.396)--cycle;
\draw(-3.327,2.384)--(-3.297,2.403)--(-3.296,2.396);
\filldraw[fill opacity=0.8,fill=gray!20,draw=none](-2.815,2.303)--(-2.789,2.379)--(-2.776,2.279)--cycle;
\draw(-2.789,2.379)--(-2.776,2.279);
\filldraw[fill opacity=0.8,fill=gray!20,draw=none](-2.837,2.431)--(-2.819,2.415)--(-2.796,2.241)--(-2.812,2.262)--(-2.83,2.305)--(-2.84,2.381)--cycle;
\draw(-2.837,2.431)--(-2.819,2.415)--(-2.796,2.241);
\draw(-2.83,2.305)--(-2.84,2.381);
\filldraw[fill opacity=0.8,fill=gray!20,draw=none](-2.837,2.431)--(-2.84,2.381)--(-2.847,2.441)--cycle;
\draw(-2.84,2.381)--(-2.847,2.441)--(-2.837,2.431);
\filldraw[fill opacity=0.8,fill=gray!20,draw=none](-2.968,2.324)--(-2.884,2.352)--(-2.832,2.384)--(-2.819,2.415)--(-2.847,2.441)--(-2.913,2.456)--(-3.005,2.46)--(-3.111,2.451)--(-3.213,2.431)--(-3.297,2.403)--(-3.327,2.384)--(-3.355,2.334)--(-3.334,2.315)--(-3.268,2.299)--(-3.176,2.295)--(-3.07,2.304)--cycle;
\draw(-3.355,2.334)--(-3.334,2.315)--(-3.268,2.299)--(-3.176,2.295)--(-3.07,2.304)--(-2.968,2.324)--(-2.884,2.352)--(-2.832,2.384)--(-2.819,2.415)--(-2.847,2.441)--(-2.913,2.456)--(-3.005,2.46)--(-3.111,2.451)--(-3.213,2.431)--(-3.297,2.403)--(-3.327,2.384);
\filldraw[fill opacity=0.8,fill=gray!20,draw=none](-3.484,.895)--(-3.486,.887)--(-3.503,.863)--(-3.596,.885)--(-3.597,.924)--cycle;
\draw(-3.503,.863)--(-3.596,.885)--(-3.597,.924);
\filldraw[fill opacity=0.8,fill=gray!20,draw=none](-3.037,.989)--(-3.03,.991)--(-3.03,.996)--(-3.045,1.023)--cycle;
\draw(-3.037,.989)--(-3.03,.991);
\filldraw[fill opacity=0.8,fill=gray!20](-3.2,.664)--(-3.176,.675)--(-3.232,.672)--(-3.228,.663)--cycle;
\filldraw[fill opacity=0.8,fill=gray!20,draw=none](-3.554,1.052)--(-3.511,1.056)--(-3.551,1.024)--cycle;
\filldraw[fill opacity=0.8,fill=gray!20,draw=none](-3.579,1.121)--(-3.56,1.118)--(-3.551,1.024)--(-3.596,1.119)--cycle;
\draw(-3.551,1.024)--(-3.596,1.119);
\filldraw[fill opacity=0.8,fill=gray!20,draw=none](-3.554,1.052)--(-3.56,1.118)--(-3.538,1.114)--(-3.511,1.056)--cycle;
\draw(-3.538,1.114)--(-3.511,1.056);
\filldraw[fill opacity=0.8,fill=gray!20,draw=none](-3.453,1)--(-3.569,1.028)--(-3.544,1.064)--(-3.449,1.04)--cycle;
\draw(-3.453,1)--(-3.569,1.028)--(-3.544,1.064)--(-3.449,1.04);
\filldraw[fill opacity=0.8,fill=gray!20,draw=none](-2.754,1.907)--(-2.735,1.759)--(-2.738,1.761)--(-2.781,1.92)--(-2.79,1.997)--cycle;
\draw(-2.754,1.907)--(-2.735,1.759);
\draw(-2.781,1.92)--(-2.79,1.997);
\filldraw[fill opacity=0.8,fill=gray!20](-3.242,.594)--(-3.244,.61)--(-3.295,.614)--(-3.278,.597)--cycle;
\filldraw[fill opacity=0.8,fill=gray!20,draw=none](-2.775,1.501)--(-2.729,1.139)--(-2.677,1.172)--(-2.731,1.598)--cycle;
\draw(-2.775,1.501)--(-2.729,1.139)--(-2.677,1.172)--(-2.731,1.598);
\filldraw[fill opacity=0.8,fill=gray!20,draw=none](-3.189,.865)--(-3.189,.866)--(-3.191,.868)--cycle;
\draw(-3.189,.865)--(-3.189,.866);
\filldraw[fill opacity=0.8,fill=gray!20](-3.286,.676)--(-3.311,.702)--(-3.364,.715)--(-3.323,.686)--cycle;
\filldraw[fill opacity=0.8,fill=gray!20](-3.152,.62)--(-3.132,.647)--(-3.181,.638)--(-3.192,.613)--cycle;
\filldraw[fill opacity=0.8,fill=gray!20](-3.539,.81)--(-3.703,.881)--(-3.661,.855)--(-3.497,.783)--cycle;
\filldraw[fill opacity=0.8,fill=gray!20,draw=none](-3.189,1.008)--(-3.213,1.014)--(-3.232,1.004)--(-3.169,.989)--cycle;
\draw(-3.232,1.004)--(-3.169,.989)--(-3.189,1.008)--(-3.213,1.014);
\filldraw[fill opacity=0.8,fill=gray!20](-3.205,.596)--(-3.192,.613)--(-3.244,.61)--(-3.242,.594)--cycle;
\filldraw[fill opacity=0.8,fill=gray!20,draw=none](-3.191,.994)--(-3.165,1)--(-3.185,1.021)--(-3.213,1.014)--cycle;
\draw(-3.191,.994)--(-3.165,1);
\draw(-3.185,1.021)--(-3.213,1.014);
\filldraw[fill opacity=0.8,fill=gray!20,draw=none](-2.793,2.02)--(-2.79,1.997)--(-2.81,2.02)--cycle;
\draw(-2.793,2.02)--(-2.79,1.997);
\filldraw[fill opacity=0.8,fill=gray!20](-3.308,.639)--(-3.316,.671)--(-3.364,.683)--(-3.351,.65)--cycle;
\filldraw[fill opacity=0.8,fill=gray!20,draw=none](-3.019,.82)--(-3.018,.821)--(-3.016,.845)--(-3.021,.869)--(-3.042,.856)--(-3.044,.839)--cycle;
\draw(-3.019,.82)--(-3.018,.821);
\draw(-3.021,.869)--(-3.042,.856)--(-3.044,.839);
\filldraw[fill opacity=0.8,fill=gray!20](-2.92,1.021)--(-2.918,1.073)--(-2.994,1.079)--(-3.013,1.028)--cycle;
\filldraw[fill opacity=0.8,fill=gray!20](-3.282,1.09)--(-3.225,1.096)--(-3.225,1.096)--(-3.272,1.096)--cycle;
\filldraw[fill opacity=0.8,fill=gray!20,draw=none](-2.998,.849)--(-2.994,.859)--(-3.001,.86)--cycle;
\draw(-2.994,.859)--(-3.001,.86);
\filldraw[fill opacity=0.8,fill=gray!20,draw=none](-3.019,.82)--(-3.044,.839)--(-3.048,.801)--cycle;
\draw(-3.044,.839)--(-3.048,.801)--(-3.019,.82);
\filldraw[fill opacity=0.8,fill=gray!20,draw=none](-3.509,1.107)--(-3.497,1.1)--(-3.504,1.076)--(-3.522,1.079)--(-3.538,1.114)--cycle;
\draw(-3.522,1.079)--(-3.538,1.114);
\filldraw[fill opacity=0.8,fill=gray!20,draw=none](-3.504,1.076)--(-3.511,1.056)--(-3.522,1.079)--cycle;
\draw(-3.511,1.056)--(-3.522,1.079);
\filldraw[fill opacity=0.8,fill=gray!20,draw=none](-3.504,1.076)--(-3.474,1.072)--(-3.511,1.056)--cycle;
\filldraw[fill opacity=0.8,fill=gray!20,draw=none](-3.504,1.076)--(-3.497,1.1)--(-3.483,1.093)--(-3.474,1.072)--cycle;
\draw(-3.483,1.093)--(-3.474,1.072);
\filldraw[fill opacity=0.8,fill=gray!20,draw=none](-3.448,1.042)--(-3.449,1.04)--(-3.544,1.064)--(-3.517,1.083)--(-3.452,1.067)--cycle;
\draw(-3.449,1.04)--(-3.544,1.064)--(-3.517,1.083)--(-3.452,1.067);
\filldraw[fill opacity=0.8,fill=gray!20,draw=none](-3.449,1.067)--(-3.452,1.067)--(-3.474,1.072)--(-3.445,1.071)--cycle;
\draw(-3.452,1.067)--(-3.474,1.072);
\filldraw[fill opacity=0.8,fill=gray!20,draw=none](-3.439,1.069)--(-3.449,1.067)--(-3.445,1.071)--(-3.44,1.07)--cycle;
\draw(-3.445,1.071)--(-3.44,1.07);
\filldraw[fill opacity=0.8,fill=gray!20,draw=none](-3.445,1.071)--(-3.423,1.024)--(-3.453,1.029)--(-3.474,1.072)--cycle;
\draw(-3.445,1.071)--(-3.423,1.024)--(-3.453,1.029)--(-3.474,1.072);
\filldraw[fill opacity=0.8,fill=gray!20,draw=none](-3.466,1.084)--(-3.445,1.071)--(-3.474,1.072)--cycle;
\filldraw[fill opacity=0.8,fill=gray!20,draw=none](-3.466,1.084)--(-3.474,1.072)--(-3.483,1.093)--cycle;
\draw(-3.474,1.072)--(-3.483,1.093);
\filldraw[fill opacity=0.8,fill=gray!20,draw=none](-3.474,1.072)--(-3.517,1.083)--(-3.492,1.083)--(-3.445,1.071)--cycle;
\draw(-3.474,1.072)--(-3.517,1.083)--(-3.492,1.083)--(-3.445,1.071);
\filldraw[fill opacity=0.8,fill=gray!20,draw=none](-3.428,1.052)--(-3.408,1.009)--(-3.423,1.024)--(-3.445,1.071)--cycle;
\draw(-3.428,1.052)--(-3.408,1.009)--(-3.423,1.024)--(-3.445,1.071);
\filldraw[fill opacity=0.8,fill=gray!20,draw=none](-3.466,1.084)--(-3.446,1.074)--(-3.445,1.071)--cycle;
\draw(-3.446,1.074)--(-3.445,1.071);
\filldraw[fill opacity=0.8,fill=gray!20,draw=none](-3.474,1.072)--(-3.453,1.029)--(-3.495,1.023)--(-3.511,1.056)--cycle;
\draw(-3.474,1.072)--(-3.453,1.029)--(-3.495,1.023)--(-3.511,1.056);
\filldraw[fill opacity=0.8,fill=gray!20,draw=none](-3.422,1.059)--(-3.433,1.066)--(-3.44,1.07)--(-3.492,1.083)--(-3.472,1.063)--(-3.428,1.052)--cycle;
\draw(-3.44,1.07)--(-3.492,1.083)--(-3.472,1.063)--(-3.428,1.052);
\filldraw[fill opacity=0.8,fill=gray!20,draw=none](-3.438,1.073)--(-3.428,1.052)--(-3.445,1.071)--(-3.446,1.074)--cycle;
\draw(-3.438,1.073)--(-3.428,1.052);
\draw(-3.445,1.071)--(-3.446,1.074);
\filldraw[fill opacity=0.8,fill=gray!20,draw=none](-3.588,.982)--(-3.542,1.006)--(-3.495,1.023)--(-3.453,1.029)--(-3.423,1.024)--(-3.408,1.009)--(-3.412,.986)--(-3.434,.959)--(-3.47,.931)--cycle;
\draw(-3.588,.982)--(-3.542,1.006)--(-3.495,1.023)--(-3.453,1.029)--(-3.423,1.024)--(-3.408,1.009)--(-3.412,.986)--(-3.434,.959)--(-3.47,.931);
\filldraw[fill opacity=0.8,fill=gray!20,draw=none](-3.428,1.019)--(-3.412,.986)--(-3.408,1.009)--(-3.428,1.052)--cycle;
\draw(-3.428,1.019)--(-3.412,.986)--(-3.408,1.009)--(-3.428,1.052);
\filldraw[fill opacity=0.8,fill=gray!20,draw=none](-3.396,1.037)--(-3.406,1.047)--(-3.472,1.063)--(-3.461,1.028)--(-3.428,1.019)--cycle;
\draw(-3.406,1.047)--(-3.472,1.063)--(-3.461,1.028)--(-3.428,1.019);
\filldraw[fill opacity=0.8,fill=gray!20,draw=none](-3.448,1.081)--(-3.452,1.079)--(-3.45,1.067)--(-3.428,1.019)--(-3.428,1.052)--(-3.438,1.073)--cycle;
\draw(-3.45,1.067)--(-3.428,1.019);
\draw(-3.428,1.052)--(-3.438,1.073);
\filldraw[fill opacity=0.8,fill=gray!20,draw=none](-3.508,.843)--(-3.52,.834)--(-3.585,.85)--(-3.596,.885)--(-3.503,.863)--cycle;
\draw(-3.52,.834)--(-3.585,.85)--(-3.596,.885)--(-3.503,.863);
\filldraw[fill opacity=0.8,fill=gray!20,draw=none](-3.511,1.056)--(-3.495,1.023)--(-3.542,1.006)--(-3.551,1.024)--cycle;
\draw(-3.511,1.056)--(-3.495,1.023)--(-3.542,1.006)--(-3.551,1.024);
\filldraw[fill opacity=0.8,fill=gray!20,draw=none](-3.551,1.024)--(-3.542,1.006)--(-3.588,.982)--cycle;
\draw(-3.551,1.024)--(-3.542,1.006)--(-3.588,.982);
\filldraw[fill opacity=0.8,fill=gray!20,draw=none](-3.515,.818)--(-3.565,.83)--(-3.585,.85)--(-3.52,.834)--cycle;
\draw(-3.515,.818)--(-3.565,.83)--(-3.585,.85)--(-3.52,.834);
\filldraw[fill opacity=0.8,fill=gray!20](-3.46,.981)--(-3.461,1.028)--(-3.472,1.063)--(-3.492,1.083)--(-3.517,1.083)--(-3.544,1.064)--(-3.569,1.028)--(-3.588,.982)--(-3.597,.932)--(-3.596,.885)--(-3.585,.85)--(-3.565,.83)--(-3.54,.83)--(-3.513,.849)--(-3.488,.884)--(-3.47,.931)--cycle;
\filldraw[fill opacity=0.8,fill=gray!20,draw=none](-3.442,.977)--(-3.434,.959)--(-3.412,.986)--(-3.428,1.019)--cycle;
\draw(-3.442,.977)--(-3.434,.959)--(-3.412,.986)--(-3.428,1.019);
\filldraw[fill opacity=0.8,fill=gray!20,draw=none](-3.358,.985)--(-3.366,1.004)--(-3.461,1.028)--(-3.46,.981)--(-3.442,.977)--cycle;
\draw(-3.366,1.004)--(-3.461,1.028)--(-3.46,.981)--(-3.442,.977);
\filldraw[fill opacity=0.8,fill=gray!20,draw=none](-3.478,1.077)--(-3.487,1.071)--(-3.442,.977)--(-3.428,1.019)--(-3.455,1.078)--cycle;
\draw(-3.487,1.071)--(-3.442,.977);
\draw(-3.428,1.019)--(-3.455,1.078);
\filldraw[fill opacity=0.8,fill=gray!20,draw=none](-3.452,1.079)--(-3.455,1.078)--(-3.45,1.067)--cycle;
\draw(-3.455,1.078)--(-3.45,1.067);
\filldraw[fill opacity=0.8,fill=gray!20](-3.397,1.058)--(-3.56,1.13)--(-3.61,1.135)--(-3.447,1.064)--cycle;
\filldraw[fill opacity=0.8,fill=gray!20,draw=none](-3.205,.72)--(-3.163,.77)--(-3.21,.757)--(-3.225,.738)--cycle;
\draw(-3.21,.757)--(-3.225,.738)--(-3.205,.72)--(-3.163,.77);
\filldraw[fill opacity=0.8,fill=gray!20,draw=none](-3.128,.793)--(-3.136,.797)--(-3.132,.79)--cycle;
\draw(-3.136,.797)--(-3.132,.79)--(-3.128,.793);
\filldraw[fill opacity=0.8,fill=gray!20,draw=none](-3.132,.79)--(-3.136,.797)--(-3.186,.797)--(-3.181,.781)--cycle;
\draw(-3.186,.797)--(-3.181,.781)--(-3.132,.79)--(-3.136,.797);
\filldraw[fill opacity=0.8,fill=gray!20,draw=none](-3.174,.765)--(-3.179,.708)--(-3.139,.71)--(-3.134,.764)--cycle;
\draw(-3.174,.765)--(-3.179,.708)--(-3.139,.71)--(-3.134,.764);
\filldraw[fill opacity=0.8,fill=gray!20,draw=none](-3.162,.692)--(-3.118,.7)--(-3.116,.716)--(-3.172,.705)--(-3.173,.696)--cycle;
\draw(-3.118,.7)--(-3.116,.716)--(-3.172,.705)--(-3.173,.696);
\filldraw[fill opacity=0.8,fill=gray!20](-3.133,.683)--(-3.095,.712)--(-3.156,.7)--(-3.176,.675)--cycle;
\filldraw[fill opacity=0.8,fill=gray!20,draw=none](-3.037,.987)--(-3.038,.99)--(-3.037,.989)--cycle;
\draw(-3.038,.99)--(-3.037,.989);
\filldraw[fill opacity=0.8,fill=gray!20,draw=none](-3.12,.787)--(-3.128,.793)--(-3.132,.79)--cycle;
\draw(-3.128,.793)--(-3.132,.79);
\filldraw[fill opacity=0.8,fill=gray!20,draw=none](-3.157,.747)--(-3.143,.749)--(-3.127,.775)--(-3.132,.79)--(-3.181,.781)--(-3.18,.771)--cycle;
\draw(-3.157,.747)--(-3.143,.749);
\draw(-3.127,.775)--(-3.132,.79)--(-3.181,.781)--(-3.18,.771);
\filldraw[fill opacity=0.8,fill=gray!20,draw=none](-3.119,.792)--(-3.131,.795)--(-3.133,.778)--(-3.109,.766)--(-3.108,.772)--cycle;
\draw(-3.131,.795)--(-3.133,.778);
\draw(-3.109,.766)--(-3.108,.772);
\filldraw[fill opacity=0.8,fill=gray!20,draw=none](-3.111,.76)--(-3.109,.766)--(-3.127,.775)--(-3.12,.754)--cycle;
\draw(-3.127,.775)--(-3.12,.754)--(-3.111,.76);
\filldraw[fill opacity=0.8,fill=gray!20](-3.174,1.094)--(-3.225,1.096)--(-3.225,1.096)--(-3.168,1.088)--cycle;
\filldraw[fill opacity=0.8,fill=gray!20,draw=none](-3.119,.792)--(-3.108,.772)--(-3.107,.79)--cycle;
\draw(-3.108,.772)--(-3.107,.79);
\filldraw[fill opacity=0.8,fill=gray!20,draw=none](-3.353,1.931)--(-3.372,2.042)--(-3.359,2.04)--(-3.342,1.908)--cycle;
\draw(-3.359,2.04)--(-3.342,1.908);
\filldraw[fill opacity=0.8,fill=gray!20,draw=none](-2.997,.895)--(-2.995,.896)--(-3.015,.944)--(-3.018,.943)--cycle;
\draw(-2.997,.895)--(-2.995,.896);
\draw(-3.015,.944)--(-3.018,.943);
\filldraw[fill opacity=0.8,fill=gray!20,draw=none](-3.001,.86)--(-2.994,.859)--(-2.955,.91)--(-2.995,.913)--cycle;
\draw(-3.001,.86)--(-2.994,.859);
\draw(-2.955,.91)--(-2.995,.913);
\filldraw[fill opacity=0.8,fill=gray!20,draw=none](-3.213,1.014)--(-3.362,1.051)--(-3.322,1.026)--(-3.232,1.004)--cycle;
\draw(-3.213,1.014)--(-3.362,1.051);
\draw(-3.322,1.026)--(-3.232,1.004);
\filldraw[fill opacity=0.8,fill=gray!20](-2.824,1.026)--(-2.839,1.077)--(-2.918,1.073)--(-2.92,1.021)--cycle;
\filldraw[fill opacity=0.8,fill=gray!20,draw=none](-3.016,.845)--(-3.014,.874)--(-3.021,.869)--cycle;
\draw(-3.014,.874)--(-3.021,.869);
\filldraw[fill opacity=0.8,fill=gray!20,draw=none](-2.994,.859)--(-2.921,.854)--(-2.922,.908)--(-2.955,.91)--cycle;
\draw(-2.994,.859)--(-2.921,.854)--(-2.922,.908)--(-2.955,.91);
\filldraw[fill opacity=0.8,fill=gray!20,draw=none](-3.232,1.516)--(-3.179,1.102)--(-3.127,1.09)--cycle;
\draw(-3.232,1.516)--(-3.179,1.102)--(-3.127,1.09);
\filldraw[fill opacity=0.8,fill=gray!20,draw=none](-3.129,.818)--(-3.131,.795)--(-3.107,.79)--cycle;
\draw(-3.129,.818)--(-3.131,.795);
\filldraw[fill opacity=0.8,fill=gray!20](-3.132,.647)--(-3.12,.68)--(-3.175,.669)--(-3.181,.638)--cycle;
\filldraw[fill opacity=0.8,fill=gray!20,draw=none](-2.781,2.021)--(-2.768,2.021)--(-2.754,1.907)--(-2.79,1.997)--cycle;
\draw(-2.768,2.021)--(-2.754,1.907);
\filldraw[fill opacity=0.8,fill=gray!20,draw=none](-2.781,2.021)--(-2.79,1.997)--(-2.793,2.02)--cycle;
\draw(-2.79,1.997)--(-2.793,2.02);
\filldraw[fill opacity=0.8,fill=gray!20,draw=none](-3.03,.996)--(-3.03,.991)--(-3.028,.991)--cycle;
\draw(-3.03,.991)--(-3.028,.991);
\filldraw[fill opacity=0.8,fill=gray!20,draw=none](-3.331,.675)--(-3.367,.719)--(-3.369,.72)--(-3.364,.683)--cycle;
\draw(-3.367,.719)--(-3.369,.72)--(-3.364,.683)--(-3.331,.675);
\filldraw[fill opacity=0.8,fill=gray!20](-2.814,.859)--(-2.81,.913)--(-2.922,.908)--(-2.921,.854)--cycle;
\filldraw[fill opacity=0.8,fill=gray!20,draw=none](-3.018,.932)--(-3.02,.961)--(-3.026,.969)--(-3.027,.958)--cycle;
\draw(-3.026,.969)--(-3.027,.958);
\filldraw[fill opacity=0.8,fill=gray!20,draw=none](-3.027,.973)--(-3.018,.943)--(-3.004,.946)--(-3.027,.976)--cycle;
\draw(-3.018,.943)--(-3.004,.946);
\filldraw[fill opacity=0.8,fill=gray!20,draw=none](-3.032,.982)--(-3.031,.976)--(-3.029,.973)--(-3.025,.972)--(-3.025,.974)--cycle;
\draw(-3.029,.973)--(-3.025,.972)--(-3.025,.974);
\filldraw[fill opacity=0.8,fill=gray!20,draw=none](-3.027,.976)--(-3.028,.991)--(-3.037,.989)--cycle;
\draw(-3.028,.991)--(-3.037,.989);
\filldraw[fill opacity=0.8,fill=gray!20,draw=none](-3.111,.76)--(-3.12,.754)--(-3.117,.725)--cycle;
\draw(-3.111,.76)--(-3.12,.754)--(-3.117,.725);
\filldraw[fill opacity=0.8,fill=gray!20,draw=none](-3.278,1.684)--(-3.294,1.645)--(-3.298,1.646)--(-3.331,1.675)--(-3.294,1.697)--cycle;
\draw(-3.294,1.645)--(-3.298,1.646)--(-3.331,1.675);
\filldraw[fill opacity=0.8,fill=gray!20,draw=none](-3.031,.976)--(-3.03,.974)--(-3.029,.973)--cycle;
\draw(-3.03,.974)--(-3.029,.973);
\filldraw[fill opacity=0.8,fill=gray!20](-2.921,.965)--(-2.92,1.021)--(-3.013,1.028)--(-3.025,.972)--cycle;
\filldraw[fill opacity=0.8,fill=gray!20,draw=none](-3.336,1.044)--(-3.323,1.057)--(-3.335,1.069)--(-3.343,1.063)--cycle;
\draw(-3.336,1.044)--(-3.323,1.057)--(-3.335,1.069)--(-3.343,1.063);
\filldraw[fill opacity=0.8,fill=gray!20,draw=none](-3.168,3.024)--(-3.059,3.033)--(-3.017,2.702)--cycle;
\draw(-3.168,3.024)--(-3.059,3.033)--(-3.017,2.702);
\filldraw[fill opacity=0.8,fill=gray!20,draw=none](-3.114,.698)--(-3.118,.7)--(-3.118,.695)--cycle;
\draw(-3.118,.7)--(-3.118,.695);
\filldraw[fill opacity=0.8,fill=gray!20,draw=none](-3.352,.717)--(-3.345,.752)--(-3.364,.757)--(-3.369,.721)--cycle;
\draw(-3.345,.752)--(-3.364,.757)--(-3.369,.721);
\filldraw[fill opacity=0.8,fill=gray!20,draw=none](-3.232,.759)--(-3.209,.774)--(-3.264,.788)--(-3.308,.781)--(-3.32,.776)--(-3.249,.758)--cycle;
\draw(-3.232,.759)--(-3.209,.774)--(-3.264,.788);
\draw(-3.32,.776)--(-3.249,.758);
\filldraw[fill opacity=0.8,fill=gray!20,draw=none](-3.282,.771)--(-3.319,.776)--(-3.324,.724)--(-3.279,.715)--(-3.275,.765)--cycle;
\draw(-3.319,.776)--(-3.324,.724)--(-3.279,.715)--(-3.275,.765);
\filldraw[fill opacity=0.8,fill=gray!20,draw=none](-3.352,.717)--(-3.318,.709)--(-3.316,.745)--(-3.345,.752)--cycle;
\draw(-3.318,.709)--(-3.316,.745)--(-3.345,.752);
\filldraw[fill opacity=0.8,fill=gray!20,draw=none](-3.252,.759)--(-3.248,.756)--(-3.246,.758)--cycle;
\draw(-3.252,.759)--(-3.248,.756)--(-3.246,.758);
\filldraw[fill opacity=0.8,fill=gray!20,draw=none](-3.232,.759)--(-3.234,.745)--(-3.225,.738)--(-3.208,.759)--cycle;
\draw(-3.234,.745)--(-3.225,.738)--(-3.208,.759);
\filldraw[fill opacity=0.8,fill=gray!20,draw=none](-3.247,.758)--(-3.246,.758)--(-3.245,.759)--cycle;
\draw(-3.246,.758)--(-3.245,.759);
\filldraw[fill opacity=0.8,fill=gray!20,draw=none](-3.251,.759)--(-3.249,.758)--(-3.251,.759)--cycle;
\draw(-3.249,.758)--(-3.251,.759);
\filldraw[fill opacity=0.8,fill=gray!20,draw=none](-3.268,.769)--(-3.275,.765)--(-3.279,.715)--(-3.228,.71)--(-3.224,.761)--cycle;
\draw(-3.275,.765)--(-3.279,.715)--(-3.228,.71)--(-3.224,.761);
\filldraw[fill opacity=0.8,fill=gray!20,draw=none](-3.315,.707)--(-3.247,.702)--(-3.246,.74)--(-3.312,.745)--(-3.317,.734)--(-3.318,.709)--cycle;
\draw(-3.315,.707)--(-3.247,.702)--(-3.246,.74)--(-3.312,.745);
\draw(-3.317,.734)--(-3.318,.709);
\filldraw[fill opacity=0.8,fill=gray!20,draw=none](-3.31,.703)--(-3.247,.698)--(-3.247,.702)--(-3.315,.707)--cycle;
\draw(-3.247,.698)--(-3.247,.702)--(-3.315,.707);
\filldraw[fill opacity=0.8,fill=gray!20](-3.311,.702)--(-3.33,.74)--(-3.395,.756)--(-3.364,.715)--cycle;
\filldraw[fill opacity=0.8,fill=gray!20,draw=none](-3.027,.958)--(-3.026,.967)--(-3.029,.973)--(-3.03,.974)--cycle;
\draw(-3.027,.958)--(-3.026,.967);
\draw(-3.029,.973)--(-3.03,.974);
\filldraw[fill opacity=0.8,fill=gray!20,draw=none](-3.026,.941)--(-3.025,.942)--(-3.027,.969)--(-3.035,.987)--cycle;
\draw(-3.026,.941)--(-3.025,.942);
\filldraw[fill opacity=0.8,fill=gray!20,draw=none](-3.026,.967)--(-3.026,.969)--(-3.027,.973)--(-3.029,.973)--cycle;
\draw(-3.026,.967)--(-3.026,.969);
\draw(-3.027,.973)--(-3.029,.973);
\filldraw[fill opacity=0.8,fill=gray!20,draw=none](-3.027,.969)--(-3.027,.976)--(-3.035,.987)--cycle;
\filldraw[fill opacity=0.8,fill=gray!20,draw=none](-3.475,.814)--(-3.54,.83)--(-3.565,.83)--(-3.515,.818)--cycle;
\draw(-3.475,.814)--(-3.54,.83)--(-3.565,.83)--(-3.515,.818);
\filldraw[fill opacity=0.8,fill=gray!20](-3.568,.853)--(-3.731,.924)--(-3.703,.881)--(-3.539,.81)--cycle;
\filldraw[fill opacity=0.8,fill=gray!20](-3.035,.987)--(-3.07,1.031)--(-3.095,1.015)--(-3.066,.966)--cycle;
\filldraw[fill opacity=0.8,fill=gray!20](-3.323,1.057)--(-3.276,1.083)--(-3.282,1.09)--(-3.335,1.069)--cycle;
\filldraw[fill opacity=0.8,fill=gray!20](-3.244,.61)--(-3.246,.635)--(-3.308,.639)--(-3.295,.614)--cycle;
\filldraw[fill opacity=0.8,fill=gray!20,draw=none](-3.038,.886)--(-3.019,.89)--(-3.022,.942)--(-3.025,.942)--cycle;
\draw(-3.038,.886)--(-3.019,.89);
\draw(-3.022,.942)--(-3.025,.942);
\filldraw[fill opacity=0.8,fill=gray!20,draw=none](-3.157,.906)--(-3.166,.909)--(-3.239,.874)--(-3.166,.856)--cycle;
\draw(-3.239,.874)--(-3.166,.856)--(-3.157,.906)--(-3.166,.909);
\filldraw[fill opacity=0.8,fill=gray!20,draw=none](-3.09,.874)--(-3.038,.886)--(-3.026,.939)--(-3.026,.941)--(-3.166,.909)--cycle;
\draw(-3.09,.874)--(-3.038,.886);
\draw(-3.026,.941)--(-3.166,.909);
\filldraw[fill opacity=0.8,fill=gray!20,draw=none](-3.014,.874)--(-3.028,.925)--(-3.048,.912)--(-3.042,.856)--cycle;
\draw(-3.028,.925)--(-3.048,.912)--(-3.042,.856)--(-3.014,.874);
\filldraw[fill opacity=0.8,fill=gray!20,draw=none](-3.294,1.928)--(-3.292,1.925)--(-3.289,1.897)--cycle;
\draw(-3.292,1.925)--(-3.289,1.897);
\filldraw[fill opacity=0.8,fill=gray!20,draw=none](-2.997,.895)--(-2.995,.913)--(-3.005,.914)--cycle;
\draw(-2.995,.913)--(-3.005,.914);
\filldraw[fill opacity=0.8,fill=gray!20,draw=none](-3.018,.932)--(-3.017,.915)--(-3.012,.914)--cycle;
\draw(-3.017,.915)--(-3.012,.914);
\filldraw[fill opacity=0.8,fill=gray!20,draw=none](-3.019,.89)--(-2.997,.895)--(-3.018,.943)--(-3.022,.942)--cycle;
\draw(-3.019,.89)--(-2.997,.895);
\draw(-3.018,.943)--(-3.022,.942);
\filldraw[fill opacity=0.8,fill=gray!20,draw=none](-3.367,.719)--(-3.369,.721)--(-3.369,.72)--cycle;
\draw(-3.369,.721)--(-3.369,.72)--(-3.367,.719);
\filldraw[fill opacity=0.8,fill=gray!20,draw=none](-3.367,.719)--(-3.319,.707)--(-3.318,.709)--(-3.369,.721)--cycle;
\draw(-3.367,.719)--(-3.319,.707)--(-3.318,.709);
\filldraw[fill opacity=0.8,fill=gray!20,draw=none](-3.026,.969)--(-3.025,.972)--(-3.027,.973)--cycle;
\draw(-3.026,.969)--(-3.025,.972)--(-3.027,.973);
\filldraw[fill opacity=0.8,fill=gray!20,draw=none](-3.027,.973)--(-3.025,.942)--(-3.018,.943)--cycle;
\draw(-3.025,.942)--(-3.018,.943);
\filldraw[fill opacity=0.8,fill=gray!20,draw=none](-3.009,.945)--(-3.011,.971)--(-3.025,.972)--(-3.026,.969)--cycle;
\draw(-3.011,.971)--(-3.025,.972)--(-3.026,.969);
\filldraw[fill opacity=0.8,fill=gray!20,draw=none](-3.331,.675)--(-3.316,.671)--(-3.319,.707)--(-3.367,.719)--cycle;
\draw(-3.331,.675)--(-3.316,.671)--(-3.319,.707)--(-3.367,.719);
\filldraw[fill opacity=0.8,fill=gray!20,draw=none](-3.005,.914)--(-2.995,.913)--(-2.991,.922)--(-3.009,.945)--cycle;
\draw(-3.005,.914)--(-2.995,.913);
\filldraw[fill opacity=0.8,fill=gray!20,draw=none](-3.147,.675)--(-3.12,.68)--(-3.118,.695)--cycle;
\draw(-3.147,.675)--(-3.12,.68)--(-3.118,.695);
\filldraw[fill opacity=0.8,fill=gray!20](-3.192,.613)--(-3.181,.638)--(-3.246,.635)--(-3.244,.61)--cycle;
\filldraw[fill opacity=0.8,fill=gray!20,draw=none](-3.009,.945)--(-2.991,.922)--(-2.972,.968)--(-3.011,.971)--cycle;
\draw(-2.972,.968)--(-3.011,.971);
\filldraw[fill opacity=0.8,fill=gray!20](-3.07,1.031)--(-3.115,1.066)--(-3.133,1.054)--(-3.095,1.015)--cycle;
\filldraw[fill opacity=0.8,fill=gray!20](-3.276,1.083)--(-3.225,1.096)--(-3.225,1.096)--(-3.282,1.09)--cycle;
\filldraw[fill opacity=0.8,fill=gray!20,draw=none](-3.316,.745)--(-3.309,.775)--(-3.356,.779)--(-3.364,.757)--cycle;
\draw(-3.356,.779)--(-3.364,.757)--(-3.316,.745)--(-3.309,.775);
\filldraw[fill opacity=0.8,fill=gray!20](-2.814,.97)--(-2.824,1.026)--(-2.92,1.021)--(-2.921,.965)--cycle;
\filldraw[fill opacity=0.8,fill=gray!20,draw=none](-2.995,.913)--(-2.922,.908)--(-2.921,.965)--(-2.972,.968)--cycle;
\draw(-2.995,.913)--(-2.922,.908)--(-2.921,.965)--(-2.972,.968);
\filldraw[fill opacity=0.8,fill=gray!20,draw=none](-3.421,.924)--(-3.415,.917)--(-3.4,.957)--cycle;
\draw(-3.421,.924)--(-3.415,.917)--(-3.4,.957);
\filldraw[fill opacity=0.8,fill=gray!20,draw=none](-3.421,.921)--(-3.417,.899)--(-3.415,.917)--(-3.421,.924)--cycle;
\draw(-3.417,.899)--(-3.415,.917)--(-3.421,.924);
\filldraw[fill opacity=0.8,fill=gray!20,draw=none](-3.393,.911)--(-3.4,.957)--(-3.415,.917)--cycle;
\draw(-3.4,.957)--(-3.415,.917)--(-3.393,.911);
\filldraw[fill opacity=0.8,fill=gray!20,draw=none](-3.47,.931)--(-3.434,.959)--(-3.442,.977)--cycle;
\draw(-3.47,.931)--(-3.434,.959)--(-3.442,.977);
\filldraw[fill opacity=0.8,fill=gray!20,draw=none](-3.343,.947)--(-3.344,.953)--(-3.46,.981)--(-3.47,.931)--(-3.347,.901)--cycle;
\draw(-3.344,.953)--(-3.46,.981)--(-3.47,.931)--(-3.347,.901);
\filldraw[fill opacity=0.8,fill=gray!20,draw=none](-3.521,1.058)--(-3.527,1.052)--(-3.47,.931)--(-3.442,.977)--(-3.487,1.071)--cycle;
\draw(-3.527,1.052)--(-3.47,.931);
\draw(-3.442,.977)--(-3.487,1.071);
\filldraw[fill opacity=0.8,fill=gray!20](-3.447,1.064)--(-3.61,1.135)--(-3.661,1.119)--(-3.497,1.047)--cycle;
\filldraw[fill opacity=0.8,fill=gray!20,draw=none](-3.294,1.928)--(-3.312,2.039)--(-3.307,2.039)--(-3.292,1.925)--cycle;
\draw(-3.307,2.039)--(-3.292,1.925);
\filldraw[fill opacity=0.8,fill=gray!20,draw=none](-3.176,.675)--(-3.162,.692)--(-3.173,.696)--(-3.234,.693)--(-3.232,.672)--cycle;
\draw(-3.234,.693)--(-3.232,.672)--(-3.176,.675)--(-3.162,.692);
\filldraw[fill opacity=0.8,fill=gray!20,draw=none](-3.147,.675)--(-3.118,.695)--(-3.118,.7)--(-3.173,.69)--(-3.175,.669)--cycle;
\draw(-3.118,.695)--(-3.118,.7);
\draw(-3.173,.69)--(-3.175,.669)--(-3.147,.675);
\filldraw[fill opacity=0.8,fill=gray!20](-2.81,.913)--(-2.814,.97)--(-2.921,.965)--(-2.922,.908)--cycle;
\filldraw[fill opacity=0.8,fill=gray!20,draw=none](-3.019,.89)--(-3.021,.929)--(-3.028,.925)--cycle;
\draw(-3.021,.929)--(-3.028,.925);
\filldraw[fill opacity=0.8,fill=gray!20](-3.168,1.088)--(-3.225,1.096)--(-3.225,1.096)--(-3.178,1.082)--cycle;
\filldraw[fill opacity=0.8,fill=gray!20](-3.115,1.066)--(-3.168,1.088)--(-3.178,1.082)--(-3.133,1.054)--cycle;
\filldraw[fill opacity=0.8,fill=gray!20](-3.178,1.082)--(-3.225,1.096)--(-3.225,1.096)--(-3.2,1.078)--cycle;
\filldraw[fill opacity=0.8,fill=gray!20,draw=none](-3.232,.672)--(-3.234,.693)--(-3.245,.697)--(-3.311,.702)--(-3.286,.676)--cycle;
\draw(-3.245,.697)--(-3.311,.702)--(-3.286,.676)--(-3.232,.672)--(-3.234,.693);
\filldraw[fill opacity=0.8,fill=gray!20](-3.256,1.079)--(-3.225,1.096)--(-3.225,1.096)--(-3.276,1.083)--cycle;
\filldraw[fill opacity=0.8,fill=gray!20](-3.2,1.078)--(-3.225,1.096)--(-3.225,1.096)--(-3.228,1.077)--cycle;
\filldraw[fill opacity=0.8,fill=gray!20](-3.228,1.077)--(-3.225,1.096)--(-3.225,1.096)--(-3.256,1.079)--cycle;
\filldraw[fill opacity=0.8,fill=gray!20,draw=none](-3.021,.929)--(-3.043,.981)--(-3.066,.966)--(-3.048,.912)--cycle;
\draw(-3.043,.981)--(-3.066,.966)--(-3.048,.912)--(-3.021,.929);
\filldraw[fill opacity=0.8,fill=gray!20](-3.116,.716)--(-3.12,.754)--(-3.175,.743)--(-3.172,.705)--cycle;
\filldraw[fill opacity=0.8,fill=gray!20](-3.095,.712)--(-3.066,.752)--(-3.14,.738)--(-3.156,.7)--cycle;
\filldraw[fill opacity=0.8,fill=gray!20,draw=none](-3.026,.939)--(-3.025,.942)--(-3.026,.941)--cycle;
\draw(-3.025,.942)--(-3.026,.941);
\filldraw[fill opacity=0.8,fill=gray!20,draw=none](-3.16,.98)--(-3.165,1)--(-3.191,.994)--cycle;
\draw(-3.165,1)--(-3.191,.994);
\filldraw[fill opacity=0.8,fill=gray!20,draw=none](-3.651,1.114)--(-3.661,1.119)--(-3.703,1.083)--(-3.676,1.071)--cycle;
\draw(-3.651,1.114)--(-3.661,1.119)--(-3.703,1.083)--(-3.676,1.071);
\filldraw[fill opacity=0.8,fill=gray!20,draw=none](-3.644,1.103)--(-3.638,1.109)--(-3.651,1.114)--(-3.667,1.087)--cycle;
\draw(-3.638,1.109)--(-3.651,1.114);
\filldraw[fill opacity=0.8,fill=gray!20,draw=none](-3.69,1.04)--(-3.677,1.068)--(-3.676,1.071)--(-3.693,1.079)--cycle;
\draw(-3.676,1.071)--(-3.693,1.079);
\filldraw[fill opacity=0.8,fill=gray!20,draw=none](-3.644,1.103)--(-3.667,1.087)--(-3.676,1.071)--(-3.676,1.071)--cycle;
\draw(-3.676,1.071)--(-3.676,1.071);
\filldraw[fill opacity=0.8,fill=gray!20,draw=none](-3.677,1.068)--(-3.676,1.071)--(-3.676,1.071)--cycle;
\draw(-3.676,1.071)--(-3.676,1.071);
\filldraw[fill opacity=0.8,fill=gray!20,draw=none](-3.646,1.106)--(-3.588,.982)--(-3.624,.954)--(-3.687,1.089)--cycle;
\draw(-3.646,1.106)--(-3.588,.982)--(-3.624,.954)--(-3.687,1.089);
\filldraw[fill opacity=0.8,fill=gray!20,draw=none](-3.515,.907)--(-3.562,.89)--(-3.604,.884)--(-3.634,.888)--(-3.649,.903)--(-3.645,.927)--(-3.624,.954)--(-3.588,.982)--(-3.47,.931)--cycle;
\draw(-3.47,.931)--(-3.515,.907)--(-3.562,.89)--(-3.604,.884)--(-3.634,.888)--(-3.649,.903)--(-3.645,.927)--(-3.624,.954)--(-3.588,.982);
\filldraw[fill opacity=0.8,fill=gray!20,draw=none](-3.127,.775)--(-3.133,.778)--(-3.134,.764)--cycle;
\draw(-3.133,.778)--(-3.134,.764);
\filldraw[fill opacity=0.8,fill=gray!20,draw=none](-3.143,.749)--(-3.12,.754)--(-3.127,.775)--cycle;
\draw(-3.143,.749)--(-3.12,.754)--(-3.127,.775);
\filldraw[fill opacity=0.8,fill=gray!20,draw=none](-2.724,2.019)--(-2.731,2.018)--(-2.723,2.039)--cycle;
\filldraw[fill opacity=0.8,fill=gray!20,draw=none](-3.38,.753)--(-3.397,.767)--(-3.399,.765)--(-3.395,.756)--cycle;
\draw(-3.399,.765)--(-3.395,.756)--(-3.38,.753);
\filldraw[fill opacity=0.8,fill=gray!20,draw=none](-3.677,1.068)--(-3.624,.954)--(-3.645,.927)--(-3.689,1.02)--cycle;
\draw(-3.677,1.068)--(-3.624,.954)--(-3.645,.927)--(-3.689,1.02);
\filldraw[fill opacity=0.8,fill=gray!20,draw=none](-3.689,1.015)--(-3.689,1.02)--(-3.645,.927)--(-3.649,.903)--(-3.679,.968)--cycle;
\draw(-3.689,1.02)--(-3.645,.927)--(-3.649,.903)--(-3.679,.968);
\filldraw[fill opacity=0.8,fill=gray!20,draw=none](-3.679,.968)--(-3.649,.903)--(-3.634,.888)--(-3.659,.941)--cycle;
\draw(-3.679,.968)--(-3.649,.903)--(-3.634,.888)--(-3.659,.941);
\filldraw[fill opacity=0.8,fill=gray!20,draw=none](-3.659,.941)--(-3.634,.888)--(-3.604,.884)--(-3.634,.948)--cycle;
\draw(-3.659,.941)--(-3.634,.888)--(-3.604,.884)--(-3.634,.948);
\filldraw[fill opacity=0.8,fill=gray!20,draw=none](-3.61,.981)--(-3.634,.948)--(-3.604,.884)--(-3.562,.89)--(-3.606,.984)--cycle;
\draw(-3.634,.948)--(-3.604,.884)--(-3.562,.89)--(-3.606,.984);
\filldraw[fill opacity=0.8,fill=gray!20](-3.578,.906)--(-3.741,.978)--(-3.731,.924)--(-3.568,.853)--cycle;
\filldraw[fill opacity=0.8,fill=gray!20,draw=none](-3.337,.789)--(-3.319,.778)--(-3.319,.781)--cycle;
\draw(-3.319,.778)--(-3.319,.781);
\filldraw[fill opacity=0.8,fill=gray!20](-3.246,.635)--(-3.246,.666)--(-3.316,.671)--(-3.308,.639)--cycle;
\filldraw[fill opacity=0.8,fill=gray!20,draw=none](-3.432,2.95)--(-3.409,2.965)--(-3.343,2.447)--(-3.396,2.331)--(-3.446,2.721)--cycle;
\draw(-3.432,2.95)--(-3.409,2.965)--(-3.343,2.447);
\draw(-3.396,2.331)--(-3.446,2.721);
\filldraw[fill opacity=0.8,fill=gray!20,draw=none](-3.397,.767)--(-3.401,.771)--(-3.399,.765)--cycle;
\draw(-3.401,.771)--(-3.399,.765);
\filldraw[fill opacity=0.8,fill=gray!20,draw=none](-2.705,1.799)--(-2.696,1.729)--(-2.68,1.767)--(-2.696,1.887)--cycle;
\draw(-2.705,1.799)--(-2.696,1.729)--(-2.68,1.767)--(-2.696,1.887);
\filldraw[fill opacity=0.8,fill=gray!20,draw=none](-2.761,1.833)--(-2.748,1.728)--(-2.735,1.759)--(-2.754,1.907)--cycle;
\draw(-2.761,1.833)--(-2.748,1.728);
\draw(-2.735,1.759)--(-2.754,1.907);
\filldraw[fill opacity=0.8,fill=gray!20,draw=none](-3.175,.957)--(-3.155,.962)--(-3.16,.98)--(-3.191,.994)--cycle;
\draw(-3.175,.957)--(-3.155,.962);
\filldraw[fill opacity=0.8,fill=gray!20,draw=none](-3.176,.818)--(-3.173,.808)--(-3.169,.805)--cycle;
\filldraw[fill opacity=0.8,fill=gray!20,draw=none](-3.169,.805)--(-3.173,.808)--(-3.168,.797)--(-3.167,.802)--cycle;
\draw(-3.168,.797)--(-3.167,.802);
\filldraw[fill opacity=0.8,fill=gray!20,draw=none](-3.169,.805)--(-3.167,.802)--(-3.167,.803)--cycle;
\draw(-3.167,.802)--(-3.167,.803);
\filldraw[fill opacity=0.8,fill=gray!20,draw=none](-3.169,.821)--(-3.174,.765)--(-3.134,.764)--(-3.129,.818)--cycle;
\draw(-3.169,.821)--(-3.174,.765);
\draw(-3.134,.764)--(-3.129,.818);
\filldraw[fill opacity=0.8,fill=gray!20,draw=none](-3.321,.776)--(-3.32,.776)--(-3.308,.781)--(-3.308,.782)--(-3.313,.784)--cycle;
\draw(-3.308,.781)--(-3.308,.782)--(-3.313,.784);
\filldraw[fill opacity=0.8,fill=gray!20,draw=none](-3.313,.784)--(-3.308,.782)--(-3.306,.787)--cycle;
\draw(-3.313,.784)--(-3.308,.782)--(-3.306,.787);
\filldraw[fill opacity=0.8,fill=gray!20,draw=none](-3.282,.792)--(-3.359,.811)--(-3.377,.79)--(-3.32,.776)--cycle;
\draw(-3.282,.792)--(-3.359,.811);
\draw(-3.377,.79)--(-3.32,.776);
\filldraw[fill opacity=0.8,fill=gray!20,draw=none](-3.397,.767)--(-3.38,.753)--(-3.351,.745)--(-3.338,.774)--(-3.342,.788)--(-3.377,.79)--cycle;
\draw(-3.38,.753)--(-3.351,.745);
\draw(-3.338,.774)--(-3.342,.788);
\filldraw[fill opacity=0.8,fill=gray!20](-3.181,.638)--(-3.175,.669)--(-3.246,.666)--(-3.246,.635)--cycle;
\filldraw[fill opacity=0.8,fill=gray!20](-3.286,1.047)--(-3.256,1.079)--(-3.276,1.083)--(-3.323,1.057)--cycle;
\filldraw[fill opacity=0.8,fill=gray!20,draw=none](-3.185,.81)--(-3.24,.823)--(-3.273,.79)--(-3.209,.774)--cycle;
\draw(-3.273,.79)--(-3.209,.774)--(-3.185,.81)--(-3.24,.823);
\filldraw[fill opacity=0.8,fill=gray!20,draw=none](-3.166,.856)--(-3.169,.821)--(-3.129,.818)--cycle;
\draw(-3.166,.856)--(-3.169,.821);
\filldraw[fill opacity=0.8,fill=gray!20,draw=none](-3.606,.984)--(-3.562,.89)--(-3.515,.907)--(-3.568,1.021)--cycle;
\draw(-3.606,.984)--(-3.562,.89)--(-3.515,.907)--(-3.568,1.021);
\filldraw[fill opacity=0.8,fill=gray!20](-3.568,.962)--(-3.731,1.034)--(-3.741,.978)--(-3.578,.906)--cycle;
\filldraw[fill opacity=0.8,fill=gray!20](-3.133,1.054)--(-3.178,1.082)--(-3.2,1.078)--(-3.176,1.046)--cycle;
\filldraw[fill opacity=0.8,fill=gray!20,draw=none](-2.78,2.279)--(-2.781,2.279)--(-2.776,2.279)--cycle;
\draw(-2.781,2.279)--(-2.776,2.279);
\filldraw[fill opacity=0.8,fill=gray!20](-3.066,.752)--(-3.048,.801)--(-3.13,.785)--(-3.14,.738)--cycle;
\filldraw[fill opacity=0.8,fill=gray!20,draw=none](-3.166,.856)--(-3.09,.874)--(-3.166,.909)--cycle;
\draw(-3.166,.856)--(-3.09,.874);
\filldraw[fill opacity=0.8,fill=gray!20,draw=none](-3.69,1.04)--(-3.693,1.079)--(-3.703,1.083)--(-3.731,1.034)--(-3.7,1.02)--cycle;
\draw(-3.693,1.079)--(-3.703,1.083)--(-3.731,1.034)--(-3.7,1.02);
\filldraw[fill opacity=0.8,fill=gray!20,draw=none](-3.269,.668)--(-3.315,.707)--(-3.319,.707)--(-3.316,.671)--cycle;
\draw(-3.315,.707)--(-3.319,.707)--(-3.316,.671)--(-3.269,.668);
\filldraw[fill opacity=0.8,fill=gray!20,draw=none](-3.261,.787)--(-3.297,.783)--(-3.295,.781)--(-3.278,.78)--cycle;
\draw(-3.295,.781)--(-3.278,.78);
\filldraw[fill opacity=0.8,fill=gray!20,draw=none](-3.296,.783)--(-3.264,.788)--(-3.273,.79)--cycle;
\draw(-3.264,.788)--(-3.273,.79);
\filldraw[fill opacity=0.8,fill=gray!20,draw=none](-3.275,.765)--(-3.25,.778)--(-3.295,.781)--cycle;
\draw(-3.25,.778)--(-3.295,.781);
\filldraw[fill opacity=0.8,fill=gray!20,draw=none](-3.261,.787)--(-3.278,.78)--(-3.25,.778)--(-3.245,.779)--(-3.245,.789)--cycle;
\draw(-3.278,.78)--(-3.25,.778);
\draw(-3.245,.779)--(-3.245,.789);
\filldraw[fill opacity=0.8,fill=gray!20,draw=none](-3.319,.776)--(-3.274,.77)--(-3.273,.79)--cycle;
\draw(-3.274,.77)--(-3.273,.79);
\filldraw[fill opacity=0.8,fill=gray!20,draw=none](-2.762,2.26)--(-2.776,2.279)--(-2.779,2.302)--cycle;
\draw(-2.776,2.279)--(-2.779,2.302);
\filldraw[fill opacity=0.8,fill=gray!20,draw=none](-3.32,.776)--(-3.309,.775)--(-3.308,.781)--cycle;
\draw(-3.309,.775)--(-3.308,.781);
\filldraw[fill opacity=0.8,fill=gray!20,draw=none](-2.775,2.021)--(-2.781,2.021)--(-2.774,2.041)--cycle;
\filldraw[fill opacity=0.8,fill=gray!20,draw=none](-3.372,2.042)--(-3.353,1.931)--(-3.369,1.961)--(-3.379,2.043)--cycle;
\draw(-3.369,1.961)--(-3.379,2.043);
\filldraw[fill opacity=0.8,fill=gray!20,draw=none](-3.162,.692)--(-3.173,.696)--(-3.173,.69)--cycle;
\draw(-3.173,.696)--(-3.173,.69);
\filldraw[fill opacity=0.8,fill=gray!20,draw=none](-3.239,.693)--(-3.173,.696)--(-3.172,.705)--(-3.247,.702)--(-3.247,.699)--cycle;
\draw(-3.173,.696)--(-3.172,.705)--(-3.247,.702)--(-3.247,.699);
\filldraw[fill opacity=0.8,fill=gray!20,draw=none](-3.162,.692)--(-3.156,.7)--(-3.181,.699)--cycle;
\draw(-3.162,.692)--(-3.156,.7)--(-3.181,.699);
\filldraw[fill opacity=0.8,fill=gray!20,draw=none](-3.302,2.002)--(-3.312,1.976)--(-3.232,1.516)--(-3.286,1.942)--cycle;
\draw(-3.232,1.516)--(-3.286,1.942);
\filldraw[fill opacity=0.8,fill=gray!20,draw=none](-3.278,1.684)--(-3.214,1.627)--(-3.22,1.627)--(-3.294,1.645)--cycle;
\draw(-3.214,1.627)--(-3.22,1.627)--(-3.294,1.645);
\filldraw[fill opacity=0.8,fill=gray!20,draw=none](-2.724,2.019)--(-2.723,2.039)--(-2.717,2.054)--(-2.713,2.019)--cycle;
\draw(-2.717,2.054)--(-2.713,2.019);
\filldraw[fill opacity=0.8,fill=gray!20,draw=none](-3.362,1.979)--(-3.369,1.961)--(-3.332,1.676)--(-3.305,1.652)--cycle;
\draw(-3.369,1.961)--(-3.332,1.676)--(-3.305,1.652);
\filldraw[fill opacity=0.8,fill=gray!20,draw=none](-3.18,.797)--(-3.187,.799)--(-3.186,.797)--cycle;
\draw(-3.187,.799)--(-3.186,.797);
\filldraw[fill opacity=0.8,fill=gray!20,draw=none](-3.192,.799)--(-3.192,.78)--(-3.181,.781)--(-3.187,.799)--cycle;
\draw(-3.192,.78)--(-3.181,.781)--(-3.187,.799);
\filldraw[fill opacity=0.8,fill=gray!20,draw=none](-3.192,.786)--(-3.192,.799)--(-3.245,.794)--(-3.245,.789)--cycle;
\draw(-3.245,.794)--(-3.245,.789);
\filldraw[fill opacity=0.8,fill=gray!20,draw=none](-3.242,.778)--(-3.192,.78)--(-3.192,.786)--(-3.245,.789)--(-3.245,.779)--cycle;
\draw(-3.242,.778)--(-3.192,.78);
\draw(-3.245,.789)--(-3.245,.779);
\filldraw[fill opacity=0.8,fill=gray!20,draw=none](-3.194,.768)--(-3.18,.771)--(-3.181,.781)--(-3.192,.78)--cycle;
\draw(-3.18,.771)--(-3.181,.781)--(-3.192,.78);
\filldraw[fill opacity=0.8,fill=gray!20,draw=none](-3.224,.76)--(-3.194,.768)--(-3.192,.78)--(-3.242,.778)--cycle;
\draw(-3.192,.78)--(-3.242,.778);
\filldraw[fill opacity=0.8,fill=gray!20,draw=none](-3.166,.856)--(-3.219,.818)--(-3.185,.81)--cycle;
\draw(-3.219,.818)--(-3.185,.81)--(-3.166,.856);
\filldraw[fill opacity=0.8,fill=gray!20,draw=none](-3.219,.818)--(-3.228,.71)--(-3.179,.708)--(-3.169,.821)--cycle;
\draw(-3.219,.818)--(-3.228,.71)--(-3.179,.708)--(-3.169,.821);
\filldraw[fill opacity=0.8,fill=gray!20,draw=none](-3.222,.667)--(-3.175,.669)--(-3.173,.69)--cycle;
\draw(-3.222,.667)--(-3.175,.669)--(-3.173,.69);
\filldraw[fill opacity=0.8,fill=gray!20,draw=none](-2.78,2.542)--(-2.74,2.23)--(-2.762,2.26)--(-2.779,2.302)--(-2.789,2.379)--cycle;
\draw(-2.78,2.542)--(-2.74,2.23);
\draw(-2.779,2.302)--(-2.789,2.379);
\filldraw[fill opacity=0.8,fill=gray!20,draw=none](-3.304,1.016)--(-3.286,1.047)--(-3.323,1.057)--(-3.336,1.044)--cycle;
\draw(-3.304,1.016)--(-3.286,1.047)--(-3.323,1.057)--(-3.336,1.044);
\filldraw[fill opacity=0.8,fill=gray!20,draw=none](-3.157,.747)--(-3.18,.771)--(-3.175,.743)--cycle;
\draw(-3.18,.771)--(-3.175,.743)--(-3.157,.747);
\filldraw[fill opacity=0.8,fill=gray!20,draw=none](-3.166,.909)--(-3.115,.921)--(-3.175,.957)--cycle;
\draw(-3.166,.909)--(-3.115,.921);
\filldraw[fill opacity=0.8,fill=gray!20,draw=none](-3.312,.745)--(-3.316,.745)--(-3.317,.734)--cycle;
\draw(-3.312,.745)--(-3.316,.745)--(-3.317,.734);
\filldraw[fill opacity=0.8,fill=gray!20](-3.172,.705)--(-3.175,.743)--(-3.246,.74)--(-3.247,.702)--cycle;
\filldraw[fill opacity=0.8,fill=gray!20](-3.234,.697)--(-3.236,.733)--(-3.33,.74)--(-3.311,.702)--cycle;
\filldraw[fill opacity=0.8,fill=gray!20](-3.232,1.044)--(-3.228,1.077)--(-3.256,1.079)--(-3.286,1.047)--cycle;
\filldraw[fill opacity=0.8,fill=gray!20,draw=none](-3.315,.707)--(-3.318,.709)--(-3.319,.707)--cycle;
\draw(-3.318,.709)--(-3.319,.707)--(-3.315,.707);
\filldraw[fill opacity=0.8,fill=gray!20,draw=none](-3.246,.74)--(-3.295,.781)--(-3.308,.782)--(-3.316,.745)--cycle;
\draw(-3.295,.781)--(-3.308,.782)--(-3.316,.745)--(-3.246,.74);
\filldraw[fill opacity=0.8,fill=gray!20,draw=none](-3.296,.783)--(-3.273,.79)--(-3.282,.792)--(-3.308,.781)--cycle;
\draw(-3.273,.79)--(-3.282,.792);
\filldraw[fill opacity=0.8,fill=gray!20,draw=none](-2.849,1.397)--(-2.813,1.111)--(-2.729,1.139)--(-2.775,1.501)--cycle;
\draw(-2.849,1.397)--(-2.813,1.111)--(-2.729,1.139)--(-2.775,1.501);
\filldraw[fill opacity=0.8,fill=gray!20,draw=none](-3.31,.703)--(-3.269,.668)--(-3.246,.666)--(-3.247,.698)--cycle;
\draw(-3.269,.668)--(-3.246,.666)--(-3.247,.698);
\filldraw[fill opacity=0.8,fill=gray!20,draw=none](-3.297,.783)--(-3.303,.782)--(-3.295,.781)--cycle;
\draw(-3.303,.782)--(-3.295,.781);
\filldraw[fill opacity=0.8,fill=gray!20,draw=none](-3.222,.667)--(-3.173,.69)--(-3.173,.696)--(-3.247,.693)--(-3.246,.666)--cycle;
\draw(-3.173,.69)--(-3.173,.696);
\draw(-3.247,.693)--(-3.246,.666)--(-3.222,.667);
\filldraw[fill opacity=0.8,fill=gray!20](-3.176,1.046)--(-3.2,1.078)--(-3.228,1.077)--(-3.232,1.044)--cycle;
\filldraw[fill opacity=0.8,fill=gray!20,draw=none](-3.175,.957)--(-3.308,.99)--(-3.315,.945)--(-3.215,.921)--cycle;
\draw(-3.175,.957)--(-3.308,.99);
\draw(-3.315,.945)--(-3.215,.921);
\filldraw[fill opacity=0.8,fill=gray!20,draw=none](-3.173,.696)--(-3.181,.699)--(-3.234,.697)--(-3.234,.693)--cycle;
\draw(-3.181,.699)--(-3.234,.697)--(-3.234,.693);
\filldraw[fill opacity=0.8,fill=gray!20,draw=none](-2.733,2.022)--(-2.705,1.799)--(-2.696,1.887)--(-2.713,2.019)--cycle;
\draw(-2.733,2.022)--(-2.705,1.799);
\draw(-2.696,1.887)--(-2.713,2.019);
\filldraw[fill opacity=0.8,fill=gray!20](-3.095,1.015)--(-3.133,1.054)--(-3.176,1.046)--(-3.156,1.003)--cycle;
\filldraw[fill opacity=0.8,fill=gray!20,draw=none](-3.194,.768)--(-3.197,.742)--(-3.175,.743)--(-3.18,.771)--cycle;
\draw(-3.197,.742)--(-3.175,.743)--(-3.18,.771);
\filldraw[fill opacity=0.8,fill=gray!20](-3.048,.801)--(-3.042,.856)--(-3.127,.839)--(-3.13,.785)--cycle;
\filldraw[fill opacity=0.8,fill=gray!20](-3.156,.7)--(-3.14,.738)--(-3.236,.733)--(-3.234,.697)--cycle;
\filldraw[fill opacity=0.8,fill=gray!20,draw=none](-3.394,.909)--(-3.393,.911)--(-3.415,.917)--(-3.417,.899)--cycle;
\draw(-3.393,.911)--(-3.415,.917)--(-3.417,.899);
\filldraw[fill opacity=0.8,fill=gray!20,draw=none](-3.282,.771)--(-3.275,.765)--(-3.274,.77)--cycle;
\draw(-3.275,.765)--(-3.274,.77);
\filldraw[fill opacity=0.8,fill=gray!20,draw=none](-3.637,1.108)--(-3.638,1.109)--(-3.644,1.103)--cycle;
\draw(-3.637,1.108)--(-3.638,1.109);
\filldraw[fill opacity=0.8,fill=gray!20,draw=none](-3.598,1.125)--(-3.596,1.119)--(-3.646,1.106)--(-3.658,1.133)--cycle;
\draw(-3.598,1.125)--(-3.596,1.119);
\draw(-3.646,1.106)--(-3.658,1.133);
\filldraw[fill opacity=0.8,fill=gray!20,draw=none](-3.579,1.121)--(-3.596,1.119)--(-3.598,1.125)--cycle;
\draw(-3.596,1.119)--(-3.598,1.125);
\filldraw[fill opacity=0.8,fill=gray!20,draw=none](-3.261,.787)--(-3.245,.789)--(-3.245,.794)--cycle;
\draw(-3.245,.789)--(-3.245,.794);
\filldraw[fill opacity=0.8,fill=gray!20,draw=none](-3.197,.742)--(-3.194,.768)--(-3.246,.755)--(-3.246,.74)--cycle;
\draw(-3.246,.755)--(-3.246,.74)--(-3.197,.742);
\filldraw[fill opacity=0.8,fill=gray!20,draw=none](-3.509,1.107)--(-3.538,1.114)--(-3.545,1.127)--cycle;
\draw(-3.538,1.114)--(-3.545,1.127);
\filldraw[fill opacity=0.8,fill=gray!20,draw=none](-3.56,1.118)--(-3.562,1.138)--(-3.545,1.127)--(-3.538,1.114)--cycle;
\draw(-3.545,1.127)--(-3.538,1.114);
\filldraw[fill opacity=0.8,fill=gray!20,draw=none](-3.561,1.124)--(-3.56,1.118)--(-3.579,1.121)--cycle;
\filldraw[fill opacity=0.8,fill=gray!20](-7.83,1.834)--(-7.801,1.884)--(-7.759,1.919)--(-7.709,1.936)--(-7.658,1.93)--(-7.616,1.904)--(-7.588,1.861)--(-7.578,1.807)--(-7.588,1.751)--(-7.616,1.702)--(-7.658,1.666)--(-7.709,1.65)--(-7.759,1.655)--(-7.801,1.682)--(-7.83,1.725)--(-7.839,1.779)--cycle;
\filldraw[fill opacity=0.8,fill=gray!20](-3.283,3.13)--(-3.286,3.186)--(-3.175,3.191)--(-3.175,3.134)--cycle;
\filldraw[fill opacity=0.8,fill=gray!20](-3.286,3.186)--(-3.283,3.24)--(-3.175,3.245)--(-3.175,3.191)--cycle;
\filldraw[fill opacity=0.8,fill=gray!20](-3.365,3.114)--(-3.371,3.17)--(-3.286,3.186)--(-3.283,3.13)--cycle;
\filldraw[fill opacity=0.8,fill=gray!20](-3.371,3.17)--(-3.365,3.225)--(-3.283,3.24)--(-3.286,3.186)--cycle;
\filldraw[fill opacity=0.8,fill=gray!20,draw=none](-3.629,2.258)--(-3.619,2.249)--(-3.653,2.211)--cycle;
\draw(-3.629,2.258)--(-3.619,2.249)--(-3.653,2.211);
\filldraw[fill opacity=0.8,fill=gray!20,draw=none](-3.6,2.293)--(-3.581,2.278)--(-3.619,2.249)--(-3.629,2.258)--cycle;
\draw(-3.6,2.293)--(-3.581,2.278)--(-3.619,2.249)--(-3.629,2.258);
\filldraw[fill opacity=0.8,fill=gray!20](-3.533,2.211)--(-3.541,2.258)--(-3.556,2.294)--(-3.578,2.314)--(-3.601,2.314)--(-3.624,2.294)--(-3.642,2.258)--(-3.653,2.211)--(-3.655,2.159)--(-3.647,2.112)--(-3.632,2.076)--(-3.61,2.056)--(-3.587,2.056)--(-3.564,2.076)--(-3.546,2.112)--(-3.535,2.159)--cycle;
\filldraw[fill opacity=0.8,fill=gray!20,draw=none](-3.694,2.333)--(-3.726,2.347)--(-3.768,2.312)--(-3.753,2.305)--cycle;
\draw(-3.694,2.333)--(-3.726,2.347)--(-3.768,2.312)--(-3.753,2.305);
\filldraw[fill opacity=0.8,fill=gray!20,draw=none](-3.759,2.308)--(-3.768,2.312)--(-3.797,2.262)--(-3.771,2.251)--cycle;
\draw(-3.759,2.308)--(-3.768,2.312)--(-3.797,2.262)--(-3.771,2.251);
\filldraw[fill opacity=0.8,fill=gray!20,draw=none](-3.746,2.302)--(-3.759,2.308)--(-3.771,2.251)--(-3.77,2.25)--cycle;
\draw(-3.746,2.302)--(-3.759,2.308);
\draw(-3.771,2.251)--(-3.77,2.25);
\filldraw[fill opacity=0.8,fill=gray!20,draw=none](-3.601,2.278)--(-3.594,2.29)--(-3.694,2.333)--(-3.753,2.305)--(-3.629,2.251)--cycle;
\draw(-3.594,2.29)--(-3.694,2.333);
\draw(-3.753,2.305)--(-3.629,2.251);
\filldraw[fill opacity=0.8,fill=gray!20,draw=none](-3.624,2.341)--(-3.676,2.364)--(-3.726,2.347)--(-3.704,2.338)--cycle;
\draw(-3.624,2.341)--(-3.676,2.364)--(-3.726,2.347)--(-3.704,2.338);
\filldraw[fill opacity=0.8,fill=gray!20,draw=none](-4.314,2.7)--(-4.257,2.652)--(-4.191,2.547)--(-4.226,2.576)--cycle;
\draw(-4.314,2.7)--(-4.257,2.652);
\draw(-4.191,2.547)--(-4.226,2.576);
\filldraw[fill opacity=0.8,fill=gray!20,draw=none](-4.303,2.821)--(-4.274,2.796)--(-4.225,2.688)--(-4.332,2.778)--cycle;
\draw(-4.303,2.821)--(-4.274,2.796);
\draw(-4.225,2.688)--(-4.332,2.778);
\filldraw[fill opacity=0.8,fill=gray!20,draw=none](-4.381,2.819)--(-4.315,2.764)--(-4.257,2.652)--(-4.314,2.7)--cycle;
\draw(-4.381,2.819)--(-4.315,2.764);
\draw(-4.257,2.652)--(-4.314,2.7);
\filldraw[fill opacity=0.8,fill=gray!20,draw=none](-4.355,2.734)--(-4.314,2.7)--(-4.226,2.576)--cycle;
\draw(-4.355,2.734)--(-4.314,2.7);
\filldraw[fill opacity=0.8,fill=gray!20,draw=none](-4.327,2.722)--(-4.314,2.7)--(-4.337,2.719)--cycle;
\draw(-4.314,2.7)--(-4.337,2.719);
\filldraw[fill opacity=0.8,fill=gray!20,draw=none](-4.391,2.792)--(-4.373,2.805)--(-4.327,2.722)--(-4.337,2.719)--(-4.355,2.734)--cycle;
\draw(-4.337,2.719)--(-4.355,2.734);
\filldraw[fill opacity=0.8,fill=gray!20,draw=none](-4.316,2.802)--(-4.332,2.778)--(-4.349,2.792)--cycle;
\draw(-4.332,2.778)--(-4.349,2.792);
\filldraw[fill opacity=0.8,fill=gray!20,draw=none](-4.445,2.824)--(-4.447,2.88)--(-4.425,2.881)--(-4.427,2.824)--cycle;
\draw(-4.447,2.88)--(-4.425,2.881)--(-4.427,2.824)--(-4.445,2.824);
\filldraw[fill opacity=0.8,fill=gray!20,draw=none](-4.341,2.852)--(-4.303,2.821)--(-4.316,2.802)--(-4.349,2.792)--(-4.383,2.82)--cycle;
\draw(-4.341,2.852)--(-4.303,2.821);
\draw(-4.349,2.792)--(-4.383,2.82);
\filldraw[fill opacity=0.8,fill=gray!20,draw=none](-4.376,2.849)--(-4.419,2.832)--(-4.426,2.834)--(-4.425,2.881)--(-4.37,2.877)--cycle;
\draw(-4.426,2.834)--(-4.425,2.881)--(-4.37,2.877);
\filldraw[fill opacity=0.8,fill=gray!20,draw=none](-4.447,2.88)--(-4.461,2.936)--(-4.425,2.938)--(-4.425,2.881)--cycle;
\draw(-4.461,2.936)--(-4.425,2.938)--(-4.425,2.881)--(-4.447,2.88);
\filldraw[fill opacity=0.8,fill=gray!20,draw=none](-4.378,2.877)--(-4.425,2.881)--(-4.425,2.929)--cycle;
\draw(-4.378,2.877)--(-4.425,2.881)--(-4.425,2.929);
\filldraw[fill opacity=0.8,fill=gray!20,draw=none](-4.339,2.896)--(-4.336,2.874)--(-4.378,2.877)--(-4.425,2.929)--(-4.425,2.938)--(-4.366,2.934)--cycle;
\draw(-4.336,2.874)--(-4.378,2.877);
\draw(-4.425,2.929)--(-4.425,2.938)--(-4.366,2.934);
\filldraw[fill opacity=0.8,fill=gray!20,draw=none](-4.381,2.945)--(-4.389,2.935)--(-4.425,2.938)--(-4.425,2.968)--cycle;
\draw(-4.389,2.935)--(-4.425,2.938)--(-4.425,2.968);
\filldraw[fill opacity=0.8,fill=gray!20,draw=none](-4.452,2.937)--(-4.44,2.964)--(-4.425,2.968)--(-4.425,2.938)--cycle;
\draw(-4.425,2.968)--(-4.425,2.938)--(-4.452,2.937);
\filldraw[fill opacity=0.8,fill=gray!20,draw=none](-4.382,2.822)--(-4.383,2.821)--(-4.427,2.824)--(-4.426,2.834)--cycle;
\draw(-4.383,2.821)--(-4.427,2.824)--(-4.426,2.834);
\filldraw[fill opacity=0.8,fill=gray!20,draw=none](-4.335,2.865)--(-4.376,2.849)--(-4.37,2.877)--(-4.349,2.875)--cycle;
\draw(-4.37,2.877)--(-4.349,2.875);
\filldraw[fill opacity=0.8,fill=gray!20,draw=none](-4.332,2.86)--(-4.292,2.838)--(-4.303,2.821)--(-4.34,2.852)--cycle;
\draw(-4.303,2.821)--(-4.34,2.852);
\filldraw[fill opacity=0.8,fill=gray!20,draw=none](-4.376,2.849)--(-4.382,2.822)--(-4.419,2.832)--cycle;
\filldraw[fill opacity=0.8,fill=gray!20,draw=none](-4.395,2.798)--(-4.406,2.796)--(-4.405,2.823)--(-4.383,2.821)--cycle;
\draw(-4.405,2.823)--(-4.383,2.821);
\filldraw[fill opacity=0.8,fill=gray!20,draw=none](-4.406,2.796)--(-4.428,2.791)--(-4.427,2.824)--(-4.405,2.823)--cycle;
\draw(-4.428,2.791)--(-4.427,2.824)--(-4.405,2.823);
\filldraw[fill opacity=0.8,fill=gray!20,draw=none](-4.391,2.792)--(-4.395,2.798)--(-4.383,2.821)--(-4.381,2.819)--(-4.373,2.805)--cycle;
\draw(-4.383,2.821)--(-4.381,2.819);
\filldraw[fill opacity=0.8,fill=gray!20,draw=none](-4.335,2.865)--(-4.302,2.861)--(-4.292,2.838)--(-4.339,2.864)--cycle;
\filldraw[fill opacity=0.8,fill=gray!20,draw=none](-4.344,2.866)--(-4.332,2.86)--(-4.34,2.852)--(-4.36,2.868)--cycle;
\draw(-4.34,2.852)--(-4.36,2.868);
\filldraw[fill opacity=0.8,fill=gray!20,draw=none](-4.35,2.86)--(-4.341,2.852)--(-4.383,2.82)--cycle;
\draw(-4.35,2.86)--(-4.341,2.852);
\filldraw[fill opacity=0.8,fill=gray!20,draw=none](-4.376,2.849)--(-4.335,2.865)--(-4.328,2.86)--(-4.361,2.82)--(-4.38,2.821)--(-4.382,2.822)--cycle;
\draw(-4.361,2.82)--(-4.38,2.821);
\filldraw[fill opacity=0.8,fill=gray!20,draw=none](-4.382,2.822)--(-4.38,2.821)--(-4.383,2.821)--cycle;
\draw(-4.38,2.821)--(-4.383,2.821);
\filldraw[fill opacity=0.8,fill=gray!20,draw=none](-4.386,2.8)--(-4.395,2.798)--(-4.383,2.821)--(-4.361,2.82)--cycle;
\draw(-4.383,2.821)--(-4.361,2.82);
\filldraw[fill opacity=0.8,fill=gray!20,draw=none](-4.36,2.868)--(-4.35,2.86)--(-4.383,2.82)--(-4.433,2.863)--cycle;
\draw(-4.36,2.868)--(-4.35,2.86);
\draw(-4.383,2.82)--(-4.433,2.863);
\filldraw[fill opacity=0.8,fill=gray!20,draw=none](-4.395,2.798)--(-4.417,2.834)--(-4.408,2.842)--(-4.383,2.821)--cycle;
\draw(-4.408,2.842)--(-4.383,2.821);
\filldraw[fill opacity=0.8,fill=gray!20,draw=none](-4.307,2.872)--(-4.322,2.899)--(-4.311,2.901)--(-4.251,2.863)--(-4.276,2.858)--cycle;
\draw(-4.322,2.899)--(-4.311,2.901);
\draw(-4.251,2.863)--(-4.276,2.858);
\filldraw[fill opacity=0.8,fill=gray!20,draw=none](-4.339,2.896)--(-4.322,2.873)--(-4.336,2.874)--cycle;
\draw(-4.322,2.873)--(-4.336,2.874);
\filldraw[fill opacity=0.8,fill=gray!20,draw=none](-4.353,2.893)--(-4.322,2.899)--(-4.307,2.872)--cycle;
\draw(-4.353,2.893)--(-4.322,2.899);
\filldraw[fill opacity=0.8,fill=gray!20,draw=none](-4.335,2.865)--(-4.349,2.875)--(-4.322,2.873)--(-4.322,2.871)--cycle;
\draw(-4.349,2.875)--(-4.322,2.873);
\filldraw[fill opacity=0.8,fill=gray!20,draw=none](-4.321,2.871)--(-4.335,2.865)--(-4.344,2.866)--(-4.382,2.887)--cycle;
\filldraw[fill opacity=0.8,fill=gray!20,draw=none](-4.364,2.817)--(-4.361,2.82)--(-4.333,2.818)--(-4.336,2.811)--cycle;
\draw(-4.361,2.82)--(-4.333,2.818)--(-4.336,2.811);
\filldraw[fill opacity=0.8,fill=gray!20,draw=none](-4.361,2.82)--(-4.328,2.86)--(-4.324,2.858)--(-4.333,2.818)--cycle;
\draw(-4.324,2.858)--(-4.333,2.818)--(-4.361,2.82);
\filldraw[fill opacity=0.8,fill=gray!20,draw=none](-4.386,2.8)--(-4.364,2.817)--(-4.336,2.811)--cycle;
\filldraw[fill opacity=0.8,fill=gray!20,draw=none](-4.344,2.866)--(-4.36,2.868)--(-4.382,2.887)--cycle;
\draw(-4.36,2.868)--(-4.382,2.887);
\filldraw[fill opacity=0.8,fill=gray!20,draw=none](-4.342,2.932)--(-4.343,2.941)--(-4.322,2.899)--(-4.327,2.898)--cycle;
\draw(-4.322,2.899)--(-4.327,2.898);
\filldraw[fill opacity=0.8,fill=gray!20,draw=none](-4.339,2.896)--(-4.366,2.934)--(-4.342,2.932)--cycle;
\draw(-4.366,2.934)--(-4.342,2.932);
\filldraw[fill opacity=0.8,fill=gray!20,draw=none](-4.359,2.973)--(-4.348,2.951)--(-4.35,2.932)--(-4.389,2.935)--cycle;
\draw(-4.35,2.932)--(-4.389,2.935);
\filldraw[fill opacity=0.8,fill=gray!20,draw=none](-4.411,2.887)--(-4.382,2.887)--(-4.36,2.868)--(-4.433,2.863)--(-4.438,2.867)--cycle;
\draw(-4.382,2.887)--(-4.36,2.868);
\draw(-4.433,2.863)--(-4.438,2.867);
\filldraw[fill opacity=0.8,fill=gray!20,draw=none](-4.331,2.905)--(-4.327,2.898)--(-4.337,2.896)--cycle;
\draw(-4.327,2.898)--(-4.337,2.896);
\filldraw[fill opacity=0.8,fill=gray!20,draw=none](-4.347,2.962)--(-4.34,2.97)--(-4.335,2.949)--(-4.349,2.946)--cycle;
\draw(-4.335,2.949)--(-4.349,2.946);
\filldraw[fill opacity=0.8,fill=gray!20,draw=none](-4.342,2.932)--(-4.349,2.946)--(-4.346,2.946)--(-4.343,2.941)--cycle;
\draw(-4.349,2.946)--(-4.346,2.946);
\filldraw[fill opacity=0.8,fill=gray!20,draw=none](-4.348,2.951)--(-4.338,2.932)--(-4.35,2.932)--cycle;
\draw(-4.338,2.932)--(-4.35,2.932);
\filldraw[fill opacity=0.8,fill=gray!20,draw=none](-4.382,2.912)--(-4.396,2.936)--(-4.349,2.946)--(-4.331,2.905)--(-4.337,2.896)--(-4.353,2.893)--cycle;
\draw(-4.396,2.936)--(-4.349,2.946);
\draw(-4.337,2.896)--(-4.353,2.893);
\filldraw[fill opacity=0.8,fill=gray!20,draw=none](-4.331,2.916)--(-4.346,2.946)--(-4.3,2.956)--(-4.251,2.913)--(-4.311,2.901)--cycle;
\draw(-4.346,2.946)--(-4.3,2.956);
\draw(-4.251,2.913)--(-4.311,2.901);
\filldraw[fill opacity=0.8,fill=gray!20,draw=none](-4.331,2.916)--(-4.311,2.901)--(-4.322,2.899)--cycle;
\draw(-4.311,2.901)--(-4.322,2.899);
\filldraw[fill opacity=0.8,fill=gray!20,draw=none](-4.329,2.882)--(-4.339,2.896)--(-4.342,2.924)--(-4.331,2.919)--(-4.319,2.897)--(-4.32,2.882)--cycle;
\draw(-4.319,2.897)--(-4.32,2.882);
\filldraw[fill opacity=0.8,fill=gray!20,draw=none](-4.322,2.871)--(-4.361,2.881)--(-4.368,2.89)--(-4.353,2.893)--(-4.322,2.879)--cycle;
\draw(-4.368,2.89)--(-4.353,2.893);
\filldraw[fill opacity=0.8,fill=gray!20,draw=none](-4.382,2.912)--(-4.353,2.893)--(-4.368,2.89)--cycle;
\draw(-4.353,2.893)--(-4.368,2.89);
\filldraw[fill opacity=0.8,fill=gray!20,draw=none](-4.4,2.883)--(-4.368,2.89)--(-4.332,2.847)--(-4.351,2.843)--cycle;
\draw(-4.4,2.883)--(-4.368,2.89);
\draw(-4.332,2.847)--(-4.351,2.843);
\filldraw[fill opacity=0.8,fill=gray!20,draw=none](-4.407,2.893)--(-4.432,2.929)--(-4.396,2.936)--(-4.368,2.89)--(-4.382,2.887)--cycle;
\draw(-4.432,2.929)--(-4.396,2.936);
\draw(-4.368,2.89)--(-4.382,2.887);
\filldraw[fill opacity=0.8,fill=gray!20,draw=none](-4.363,2.931)--(-4.324,2.899)--(-4.346,2.877)--(-4.382,2.887)--(-4.413,2.913)--cycle;
\draw(-4.363,2.931)--(-4.324,2.899);
\draw(-4.382,2.887)--(-4.413,2.913);
\filldraw[fill opacity=0.8,fill=gray!20,draw=none](-4.359,2.973)--(-4.381,2.945)--(-4.425,2.968)--(-4.425,2.992)--(-4.366,2.987)--cycle;
\draw(-4.425,2.968)--(-4.425,2.992)--(-4.366,2.987);
\filldraw[fill opacity=0.8,fill=gray!20,draw=none](-4.34,2.97)--(-4.315,2.999)--(-4.254,3.012)--(-4.297,2.956)--(-4.335,2.949)--cycle;
\draw(-4.315,2.999)--(-4.254,3.012);
\draw(-4.297,2.956)--(-4.335,2.949);
\filldraw[fill opacity=0.8,fill=gray!20,draw=none](-4.346,2.993)--(-4.349,2.986)--(-4.399,2.99)--(-4.365,3.009)--cycle;
\draw(-4.349,2.986)--(-4.399,2.99);
\filldraw[fill opacity=0.8,fill=gray!20,draw=none](-4.348,2.951)--(-4.366,2.987)--(-4.344,2.986)--cycle;
\draw(-4.366,2.987)--(-4.344,2.986);
\filldraw[fill opacity=0.8,fill=gray!20](-3.283,3.24)--(-3.273,3.288)--(-3.177,3.292)--(-3.175,3.245)--cycle;
\filldraw[fill opacity=0.8,fill=gray!20](-3.365,3.225)--(-3.347,3.274)--(-3.273,3.288)--(-3.283,3.24)--cycle;
\filldraw[fill opacity=0.8,fill=gray!20,draw=none](-4.346,2.993)--(-4.315,2.999)--(-4.34,2.97)--cycle;
\draw(-4.346,2.993)--(-4.315,2.999);
\filldraw[fill opacity=0.8,fill=gray!20,draw=none](-4.334,3.007)--(-4.327,3.011)--(-4.298,3.013)--(-4.315,2.999)--(-4.335,2.995)--cycle;
\draw(-4.315,2.999)--(-4.335,2.995);
\filldraw[fill opacity=0.8,fill=gray!20,draw=none](-4.343,3.001)--(-4.334,3.007)--(-4.335,2.995)--(-4.346,2.993)--cycle;
\draw(-4.335,2.995)--(-4.346,2.993);
\filldraw[fill opacity=0.8,fill=gray!20,draw=none](-4.349,2.986)--(-4.343,3.001)--(-4.325,3)--(-4.321,2.984)--cycle;
\draw(-4.325,3)--(-4.321,2.984)--(-4.349,2.986);
\filldraw[fill opacity=0.8,fill=gray!20,draw=none](-4.41,2.935)--(-4.358,2.991)--(-4.346,2.993)--(-4.344,2.986)--(-4.347,2.962)--(-4.364,2.943)--(-4.407,2.934)--cycle;
\draw(-4.358,2.991)--(-4.346,2.993);
\draw(-4.364,2.943)--(-4.407,2.934);
\filldraw[fill opacity=0.8,fill=gray!20,draw=none](-4.404,2.892)--(-4.396,2.898)--(-4.382,2.887)--cycle;
\draw(-4.396,2.898)--(-4.382,2.887);
\filldraw[fill opacity=0.8,fill=gray!20,draw=none](-4.411,2.887)--(-4.404,2.892)--(-4.382,2.887)--cycle;
\filldraw[fill opacity=0.8,fill=gray!20,draw=none](-4.448,2.883)--(-4.447,2.88)--(-4.46,2.879)--cycle;
\draw(-4.447,2.88)--(-4.46,2.879);
\filldraw[fill opacity=0.8,fill=gray!20,draw=none](-4.445,2.824)--(-4.456,2.823)--(-4.497,2.878)--(-4.447,2.88)--cycle;
\draw(-4.445,2.824)--(-4.456,2.823);
\draw(-4.497,2.878)--(-4.447,2.88);
\filldraw[fill opacity=0.8,fill=gray!20,draw=none](-4.448,2.883)--(-4.46,2.879)--(-4.533,2.876)--(-4.536,2.933)--(-4.461,2.936)--cycle;
\draw(-4.46,2.879)--(-4.533,2.876)--(-4.536,2.933)--(-4.461,2.936);
\filldraw[fill opacity=0.8,fill=gray!20,draw=none](-4.439,2.901)--(-4.404,2.892)--(-4.438,2.867)--(-4.457,2.882)--cycle;
\draw(-4.438,2.867)--(-4.457,2.882);
\filldraw[fill opacity=0.8,fill=gray!20,draw=none](-4.407,2.893)--(-4.382,2.887)--(-4.4,2.883)--cycle;
\draw(-4.382,2.887)--(-4.4,2.883);
\filldraw[fill opacity=0.8,fill=gray!20,draw=none](-4.423,2.879)--(-4.4,2.883)--(-4.351,2.843)--cycle;
\draw(-4.423,2.879)--(-4.4,2.883);
\filldraw[fill opacity=0.8,fill=gray!20,draw=none](-4.44,2.898)--(-4.439,2.901)--(-4.407,2.893)--(-4.4,2.883)--(-4.423,2.879)--cycle;
\draw(-4.4,2.883)--(-4.423,2.879);
\filldraw[fill opacity=0.8,fill=gray!20,draw=none](-4.439,2.901)--(-4.431,2.927)--(-4.407,2.893)--cycle;
\filldraw[fill opacity=0.8,fill=gray!20,draw=none](-4.439,2.901)--(-4.422,2.92)--(-4.396,2.898)--(-4.404,2.892)--cycle;
\draw(-4.422,2.92)--(-4.396,2.898);
\filldraw[fill opacity=0.8,fill=gray!20,draw=none](-4.41,2.935)--(-4.43,2.945)--(-4.387,2.984)--(-4.358,2.991)--cycle;
\draw(-4.387,2.984)--(-4.358,2.991);
\filldraw[fill opacity=0.8,fill=gray!20,draw=none](-4.452,2.937)--(-4.536,2.933)--(-4.536,2.941)--(-4.44,2.964)--cycle;
\draw(-4.452,2.937)--(-4.536,2.933)--(-4.536,2.941);
\filldraw[fill opacity=0.8,fill=gray!20,draw=none](-4.464,2.929)--(-4.46,2.939)--(-4.441,2.951)--(-4.407,2.934)--(-4.463,2.923)--cycle;
\draw(-4.407,2.934)--(-4.463,2.923);
\filldraw[fill opacity=0.8,fill=gray!20,draw=none](-4.363,2.931)--(-4.413,2.913)--(-4.429,2.926)--cycle;
\draw(-4.413,2.913)--(-4.429,2.926);
\filldraw[fill opacity=0.8,fill=gray!20,draw=none](-4.439,2.901)--(-4.46,2.92)--(-4.461,2.923)--(-4.432,2.929)--(-4.431,2.927)--cycle;
\draw(-4.461,2.923)--(-4.432,2.929);
\filldraw[fill opacity=0.8,fill=gray!20,draw=none](-4.49,2.952)--(-4.425,2.988)--(-4.425,2.968)--cycle;
\draw(-4.425,2.988)--(-4.425,2.968);
\filldraw[fill opacity=0.8,fill=gray!20,draw=none](-4.43,2.985)--(-4.49,2.952)--(-4.536,2.941)--(-4.535,2.948)--(-4.508,2.988)--(-4.427,2.992)--cycle;
\draw(-4.536,2.941)--(-4.535,2.948);
\draw(-4.508,2.988)--(-4.427,2.992);
\filldraw[fill opacity=0.8,fill=gray!20,draw=none](-4.427,2.948)--(-4.433,2.956)--(-4.387,2.984)--cycle;
\filldraw[fill opacity=0.8,fill=gray!20,draw=none](-4.421,2.949)--(-4.433,2.956)--(-4.387,2.984)--(-4.414,2.948)--cycle;
\draw(-4.387,2.984)--(-4.414,2.948);
\filldraw[fill opacity=0.8,fill=gray!20,draw=none](-4.439,2.901)--(-4.445,2.903)--(-4.46,2.92)--cycle;
\filldraw[fill opacity=0.8,fill=gray!20,draw=none](-4.443,2.902)--(-4.426,2.924)--(-4.422,2.92)--(-4.439,2.901)--cycle;
\draw(-4.426,2.924)--(-4.422,2.92);
\filldraw[fill opacity=0.8,fill=gray!20,draw=none](-4.424,2.933)--(-4.403,2.963)--(-4.368,2.923)--(-4.381,2.906)--cycle;
\draw(-4.424,2.933)--(-4.403,2.963);
\draw(-4.368,2.923)--(-4.381,2.906);
\filldraw[fill opacity=0.8,fill=gray!20,draw=none](-4.457,2.906)--(-4.439,2.901)--(-4.457,2.882)--(-4.465,2.889)--cycle;
\draw(-4.457,2.882)--(-4.465,2.889);
\filldraw[fill opacity=0.8,fill=gray!20,draw=none](-4.457,2.906)--(-4.46,2.92)--(-4.444,2.901)--cycle;
\filldraw[fill opacity=0.8,fill=gray!20,draw=none](-4.443,2.902)--(-4.457,2.906)--(-4.429,2.926)--(-4.426,2.924)--cycle;
\draw(-4.429,2.926)--(-4.426,2.924);
\filldraw[fill opacity=0.8,fill=gray!20,draw=none](-4.433,2.956)--(-4.427,2.948)--(-4.43,2.945)--(-4.441,2.951)--cycle;
\filldraw[fill opacity=0.8,fill=gray!20,draw=none](-4.448,2.923)--(-4.45,2.924)--(-4.453,2.952)--(-4.414,2.948)--(-4.429,2.926)--cycle;
\draw(-4.414,2.948)--(-4.429,2.926);
\filldraw[fill opacity=0.8,fill=gray!20,draw=none](-4.439,2.912)--(-4.424,2.933)--(-4.381,2.906)--cycle;
\draw(-4.439,2.912)--(-4.424,2.933);
\filldraw[fill opacity=0.8,fill=gray!20,draw=none](-4.421,2.949)--(-4.453,2.952)--(-4.454,2.961)--(-4.447,2.964)--cycle;
\filldraw[fill opacity=0.8,fill=gray!20,draw=none](-4.347,2.962)--(-4.349,2.946)--(-4.364,2.943)--cycle;
\draw(-4.349,2.946)--(-4.364,2.943);
\filldraw[fill opacity=0.8,fill=gray!20,draw=none](-4.379,2.935)--(-4.403,2.963)--(-4.396,2.972)--(-4.351,2.947)--(-4.363,2.931)--cycle;
\draw(-4.403,2.963)--(-4.396,2.972);
\draw(-4.351,2.947)--(-4.363,2.931);
\filldraw[fill opacity=0.8,fill=gray!20,draw=none](-4.379,2.935)--(-4.363,2.931)--(-4.368,2.923)--cycle;
\draw(-4.363,2.931)--(-4.368,2.923);
\filldraw[fill opacity=0.8,fill=gray!20,draw=none](-4.429,2.943)--(-4.393,2.956)--(-4.363,2.931)--(-4.42,2.927)--cycle;
\draw(-4.393,2.956)--(-4.363,2.931);
\filldraw[fill opacity=0.8,fill=gray!20,draw=none](-4.429,2.943)--(-4.42,2.927)--(-4.429,2.926)--(-4.443,2.937)--cycle;
\draw(-4.429,2.926)--(-4.443,2.937);
\filldraw[fill opacity=0.8,fill=gray!20,draw=none](-4.457,2.906)--(-4.449,2.923)--(-4.429,2.926)--cycle;
\filldraw[fill opacity=0.8,fill=gray!20,draw=none](-4.449,2.923)--(-4.443,2.937)--(-4.429,2.926)--cycle;
\draw(-4.443,2.937)--(-4.429,2.926);
\filldraw[fill opacity=0.8,fill=gray!20,draw=none](-4.449,2.923)--(-4.429,2.926)--(-4.435,2.918)--cycle;
\draw(-4.429,2.926)--(-4.435,2.918);
\filldraw[fill opacity=0.8,fill=gray!20,draw=none](-4.449,2.923)--(-4.435,2.918)--(-4.439,2.912)--cycle;
\draw(-4.435,2.918)--(-4.439,2.912);
\filldraw[fill opacity=0.8,fill=gray!20,draw=none](-4.444,2.924)--(-4.449,2.923)--(-4.45,2.924)--(-4.443,2.926)--cycle;
\draw(-4.45,2.924)--(-4.443,2.926);
\filldraw[fill opacity=0.8,fill=gray!20,draw=none](-4.46,2.92)--(-4.463,2.923)--(-4.461,2.923)--cycle;
\draw(-4.463,2.923)--(-4.461,2.923);
\filldraw[fill opacity=0.8,fill=gray!20,draw=none](-4.476,2.927)--(-4.458,2.93)--(-4.449,2.924)--(-4.468,2.917)--cycle;
\draw(-4.449,2.924)--(-4.468,2.917);
\filldraw[fill opacity=0.8,fill=gray!20,draw=none](-4.46,2.939)--(-4.452,2.944)--(-4.449,2.923)--(-4.459,2.932)--cycle;
\filldraw[fill opacity=0.8,fill=gray!20,draw=none](-4.452,2.932)--(-4.447,2.941)--(-4.443,2.937)--(-4.449,2.924)--cycle;
\draw(-4.447,2.941)--(-4.443,2.937);
\filldraw[fill opacity=0.8,fill=gray!20,draw=none](-4.452,2.931)--(-4.446,2.932)--(-4.446,2.925)--(-4.449,2.924)--cycle;
\draw(-4.446,2.925)--(-4.449,2.924);
\filldraw[fill opacity=0.8,fill=gray!20,draw=none](-4.446,2.932)--(-4.439,2.934)--(-4.436,2.929)--(-4.446,2.925)--cycle;
\draw(-4.436,2.929)--(-4.446,2.925);
\filldraw[fill opacity=0.8,fill=gray!20,draw=none](-4.434,2.926)--(-4.444,2.924)--(-4.443,2.926)--(-4.436,2.929)--cycle;
\draw(-4.443,2.926)--(-4.436,2.929);
\filldraw[fill opacity=0.8,fill=gray!20,draw=none](-4.424,2.929)--(-4.434,2.926)--(-4.436,2.929)--(-4.425,2.933)--cycle;
\draw(-4.436,2.929)--(-4.425,2.933)--(-4.424,2.929);
\filldraw[fill opacity=0.8,fill=gray!20,draw=none](-4.439,2.934)--(-4.427,2.936)--(-4.425,2.933)--(-4.436,2.929)--cycle;
\draw(-4.427,2.936)--(-4.425,2.933)--(-4.436,2.929);
\filldraw[fill opacity=0.8,fill=gray!20](-2.462,5.755)--(-2.465,5.803)--(-2.372,5.807)--(-2.372,5.759)--cycle;
\filldraw[fill opacity=0.8,fill=gray!20](-2.465,5.803)--(-2.462,5.848)--(-2.372,5.852)--(-2.372,5.807)--cycle;
\filldraw[fill opacity=0.8,fill=gray!20](-2.53,5.742)--(-2.536,5.789)--(-2.465,5.803)--(-2.462,5.755)--cycle;
\filldraw[fill opacity=0.8,fill=gray!20](-2.515,5.697)--(-2.53,5.742)--(-2.462,5.755)--(-2.454,5.709)--cycle;
\filldraw[fill opacity=0.8,fill=gray!20](-2.536,5.789)--(-2.53,5.835)--(-2.462,5.848)--(-2.465,5.803)--cycle;
\filldraw[fill opacity=0.8,fill=gray!20,draw=none](-4.43,2.985)--(-4.427,2.992)--(-4.425,2.992)--(-4.425,2.988)--cycle;
\draw(-4.427,2.992)--(-4.425,2.992)--(-4.425,2.988);
\filldraw[fill opacity=0.8,fill=gray!20,draw=none](-4.361,3.021)--(-4.356,3.015)--(-4.399,2.99)--(-4.425,2.992)--(-4.426,3.037)--(-4.416,3.038)--(-4.398,3.037)--cycle;
\draw(-4.399,2.99)--(-4.425,2.992)--(-4.426,3.037);
\draw(-4.416,3.038)--(-4.398,3.037);
\filldraw[fill opacity=0.8,fill=gray!20,draw=none](-4.427,2.992)--(-4.425,2.998)--(-4.425,2.992)--cycle;
\draw(-4.425,2.998)--(-4.425,2.992)--(-4.427,2.992);
\filldraw[fill opacity=0.8,fill=gray!20,draw=none](-4.427,2.992)--(-4.508,2.988)--(-4.485,3.024)--(-4.472,3.029)--(-4.426,3.037)--(-4.425,2.998)--cycle;
\draw(-4.427,2.992)--(-4.508,2.988);
\draw(-4.426,3.037)--(-4.425,2.998);
\filldraw[fill opacity=0.8,fill=gray!20,draw=none](-4.338,3.011)--(-4.346,2.993)--(-4.365,3.009)--(-4.352,3.017)--cycle;
\filldraw[fill opacity=0.8,fill=gray!20,draw=none](-4.398,3.037)--(-4.416,3.038)--(-4.406,3.039)--cycle;
\draw(-4.398,3.037)--(-4.416,3.038);
\filldraw[fill opacity=0.8,fill=gray!20,draw=none](-4.374,3.035)--(-4.398,3.037)--(-4.406,3.039)--(-4.384,3.042)--cycle;
\draw(-4.374,3.035)--(-4.398,3.037);
\filldraw[fill opacity=0.8,fill=gray!20,draw=none](-4.339,3.021)--(-4.351,3.035)--(-4.341,3.05)--(-4.269,3.062)--(-4.286,3.039)--cycle;
\draw(-4.351,3.035)--(-4.341,3.05);
\draw(-4.269,3.062)--(-4.286,3.039);
\filldraw[fill opacity=0.8,fill=gray!20,draw=none](-4.361,3.021)--(-4.398,3.037)--(-4.374,3.035)--cycle;
\draw(-4.398,3.037)--(-4.374,3.035);
\filldraw[fill opacity=0.8,fill=gray!20,draw=none](-4.442,2.993)--(-4.41,3.037)--(-4.384,3.042)--(-4.374,3.035)--(-4.361,3.021)--(-4.368,3.012)--cycle;
\draw(-4.442,2.993)--(-4.41,3.037);
\draw(-4.361,3.021)--(-4.368,3.012);
\filldraw[fill opacity=0.8,fill=gray!20,draw=none](-4.338,3.011)--(-4.387,2.984)--(-4.368,3.012)--(-4.352,3.017)--cycle;
\draw(-4.387,2.984)--(-4.368,3.012);
\filldraw[fill opacity=0.8,fill=gray!20,draw=none](-4.405,2.978)--(-4.429,2.996)--(-4.368,3.012)--(-4.387,2.984)--cycle;
\draw(-4.368,3.012)--(-4.387,2.984);
\filldraw[fill opacity=0.8,fill=gray!20,draw=none](-4.405,2.978)--(-4.447,2.964)--(-4.455,2.968)--(-4.455,2.974)--(-4.442,2.993)--(-4.429,2.996)--cycle;
\draw(-4.455,2.974)--(-4.442,2.993);
\filldraw[fill opacity=0.8,fill=gray!20,draw=none](-4.472,3.029)--(-4.448,3.038)--(-4.427,3.039)--(-4.426,3.037)--cycle;
\draw(-4.448,3.038)--(-4.427,3.039)--(-4.426,3.037);
\filldraw[fill opacity=0.8,fill=gray!20,draw=none](-4.426,3.037)--(-4.427,3.039)--(-4.416,3.038)--cycle;
\draw(-4.426,3.037)--(-4.427,3.039)--(-4.416,3.038);
\filldraw[fill opacity=0.8,fill=gray!20,draw=none](-4.384,3.042)--(-4.416,3.038)--(-4.427,3.039)--(-4.428,3.07)--cycle;
\draw(-4.416,3.038)--(-4.427,3.039)--(-4.428,3.07);
\filldraw[fill opacity=0.8,fill=gray!20,draw=none](-4.477,3.036)--(-4.455,3.074)--(-4.429,3.075)--(-4.427,3.039)--cycle;
\draw(-4.455,3.074)--(-4.429,3.075)--(-4.427,3.039)--(-4.477,3.036);
\filldraw[fill opacity=0.8,fill=gray!20,draw=none](-4.485,3.024)--(-4.477,3.036)--(-4.448,3.038)--cycle;
\draw(-4.477,3.036)--(-4.448,3.038);
\filldraw[fill opacity=0.8,fill=gray!20](-2.454,5.709)--(-2.462,5.755)--(-2.372,5.759)--(-2.373,5.712)--cycle;
\filldraw[fill opacity=0.8,fill=gray!20](-2.441,5.666)--(-2.454,5.709)--(-2.373,5.712)--(-2.375,5.669)--cycle;
\filldraw[fill opacity=0.8,fill=gray!20,draw=none](-4.41,3.037)--(-2.524,5.678)--(-2.494,5.669)--(-2.481,5.654)--(-4.341,3.05)--cycle;
\draw(-4.41,3.037)--(-2.524,5.678);
\draw(-2.481,5.654)--(-4.341,3.05);
\filldraw[fill opacity=0.8,fill=gray!20,draw=none](-4.4,3.052)--(-4.428,3.07)--(-4.429,3.075)--(-4.416,3.074)--cycle;
\draw(-4.428,3.07)--(-4.429,3.075)--(-4.416,3.074);
\filldraw[fill opacity=0.8,fill=gray!20,draw=none](-4.396,3.061)--(-4.409,3.107)--(-2.564,5.689)--(-2.516,5.69)--(-4.394,3.06)--cycle;
\draw(-4.409,3.107)--(-2.564,5.689);
\draw(-2.516,5.69)--(-4.394,3.06);
\filldraw[fill opacity=0.8,fill=gray!20,draw=none](-4.423,3.075)--(-4.429,3.075)--(-4.43,3.088)--cycle;
\draw(-4.423,3.075)--(-4.429,3.075)--(-4.43,3.088);
\filldraw[fill opacity=0.8,fill=gray!20,draw=none](-4.455,3.074)--(-4.443,3.099)--(-4.44,3.099)--(-4.43,3.088)--(-4.429,3.075)--cycle;
\draw(-4.443,3.099)--(-4.44,3.099);
\draw(-4.43,3.088)--(-4.429,3.075)--(-4.455,3.074);
\filldraw[fill opacity=0.8,fill=gray!20,draw=none](-4.486,2.986)--(-4.488,2.996)--(-4.467,3.025)--(-4.401,3.053)--(-4.4,3.052)--(-4.446,2.987)--cycle;
\draw(-4.488,2.996)--(-4.467,3.025);
\draw(-4.4,3.052)--(-4.446,2.987);
\filldraw[fill opacity=0.8,fill=gray!20,draw=none](-4.46,2.939)--(-4.464,2.953)--(-4.453,2.952)--(-4.452,2.944)--cycle;
\filldraw[fill opacity=0.8,fill=gray!20,draw=none](-4.453,2.952)--(-4.47,2.954)--(-4.455,2.974)--cycle;
\draw(-4.47,2.954)--(-4.455,2.974);
\filldraw[fill opacity=0.8,fill=gray!20,draw=none](-4.46,2.939)--(-4.453,2.957)--(-4.441,2.951)--cycle;
\filldraw[fill opacity=0.8,fill=gray!20,draw=none](-4.445,2.942)--(-4.441,2.938)--(-4.443,2.937)--(-4.456,2.948)--cycle;
\draw(-4.443,2.937)--(-4.456,2.948);
\filldraw[fill opacity=0.8,fill=gray!20,draw=none](-4.481,2.961)--(-4.486,2.986)--(-4.446,2.987)--(-4.47,2.954)--cycle;
\draw(-4.446,2.987)--(-4.47,2.954);
\filldraw[fill opacity=0.8,fill=gray!20,draw=none](-4.47,2.975)--(-4.441,3.002)--(-4.34,3.01)--(-4.387,2.984)--(-4.464,2.969)--cycle;
\draw(-4.387,2.984)--(-4.464,2.969);
\filldraw[fill opacity=0.8,fill=gray!20,draw=none](-4.433,2.956)--(-4.447,2.964)--(-4.387,2.984)--cycle;
\filldraw[fill opacity=0.8,fill=gray!20,draw=none](-4.433,2.956)--(-4.448,2.972)--(-4.387,2.984)--cycle;
\draw(-4.448,2.972)--(-4.387,2.984);
\filldraw[fill opacity=0.8,fill=gray!20,draw=none](-4.447,2.964)--(-4.454,2.961)--(-4.455,2.968)--cycle;
\filldraw[fill opacity=0.8,fill=gray!20,draw=none](-4.433,2.956)--(-4.441,2.951)--(-4.47,2.965)--(-4.47,2.968)--(-4.448,2.972)--cycle;
\draw(-4.47,2.968)--(-4.448,2.972);
\filldraw[fill opacity=0.8,fill=gray!20,draw=none](-4.452,2.951)--(-4.437,2.958)--(-4.429,2.943)--(-4.441,2.938)--cycle;
\filldraw[fill opacity=0.8,fill=gray!20,draw=none](-4.444,2.941)--(-4.441,2.936)--(-4.447,2.932)--cycle;
\draw(-4.444,2.941)--(-4.441,2.936)--(-4.447,2.932);
\filldraw[fill opacity=0.8,fill=gray!20,draw=none](-4.438,2.932)--(-4.447,2.932)--(-4.441,2.936)--cycle;
\draw(-4.447,2.932)--(-4.441,2.936)--(-4.438,2.932);
\filldraw[fill opacity=0.8,fill=gray!20,draw=none](-4.452,2.932)--(-4.456,2.948)--(-4.447,2.941)--cycle;
\draw(-4.456,2.948)--(-4.447,2.941);
\filldraw[fill opacity=0.8,fill=gray!20,draw=none](-4.432,2.947)--(-4.424,2.929)--(-4.425,2.933)--(-4.432,2.947)--cycle;
\draw(-4.424,2.929)--(-4.425,2.933)--(-4.432,2.947);
\filldraw[fill opacity=0.8,fill=gray!20,draw=none](-4.438,2.932)--(-4.439,2.934)--(-4.444,2.941)--cycle;
\draw(-4.438,2.932)--(-4.439,2.934);
\filldraw[fill opacity=0.8,fill=gray!20,draw=none](-4.405,2.956)--(-3.566,1.173)--(-3.561,1.124)--(-3.579,1.121)--(-3.598,1.125)--(-4.452,2.94)--cycle;
\draw(-3.598,1.125)--(-4.452,2.94)--(-4.405,2.956)--(-3.566,1.173);
\filldraw[fill opacity=0.8,fill=gray!20,draw=none](-4.424,2.929)--(-4.407,2.885)--(-7.596,1.752)--(-7.614,1.8)--(-4.462,2.919)--cycle;
\draw(-4.424,2.929)--(-4.407,2.885)--(-7.596,1.752)--(-7.614,1.8)--(-4.462,2.919);
\filldraw[fill opacity=0.8,fill=gray!20,draw=none](-4.463,2.954)--(-4.456,2.948)--(-4.45,2.923)--(-4.465,2.889)--(-4.497,2.916)--cycle;
\draw(-4.465,2.889)--(-4.497,2.916)--(-4.463,2.954)--(-4.456,2.948);
\filldraw[fill opacity=0.8,fill=gray!20,draw=none](-4.433,2.949)--(-4.452,2.931)--(-4.467,2.966)--(-4.445,2.974)--cycle;
\draw(-4.467,2.966)--(-4.445,2.974)--(-4.433,2.949);
\filldraw[fill opacity=0.8,fill=gray!20,draw=none](-3.273,.79)--(-3.274,.77)--(-3.224,.761)--(-3.221,.795)--cycle;
\draw(-3.273,.79)--(-3.274,.77);
\draw(-3.224,.761)--(-3.221,.795);
\filldraw[fill opacity=0.8,fill=gray!20,draw=none](-3.275,.765)--(-3.246,.74)--(-3.246,.778)--(-3.25,.778)--cycle;
\draw(-3.246,.74)--(-3.246,.778)--(-3.25,.778);
\filldraw[fill opacity=0.8,fill=gray!20](-7.707,1.066)--(-7.678,1.116)--(-7.636,1.151)--(-7.586,1.168)--(-7.536,1.162)--(-7.493,1.136)--(-7.465,1.093)--(-7.455,1.039)--(-7.465,.984)--(-7.493,.934)--(-7.536,.898)--(-7.586,.882)--(-7.636,.888)--(-7.678,.914)--(-7.707,.957)--(-7.717,1.011)--cycle;
\filldraw[fill opacity=0.8,fill=gray!20,draw=none](-4.399,2.791)--(-4.395,2.798)--(-4.355,2.734)--cycle;
\filldraw[fill opacity=0.8,fill=gray!20,draw=none](-4.454,2.771)--(-4.445,2.824)--(-4.427,2.824)--(-4.429,2.772)--cycle;
\draw(-4.445,2.824)--(-4.427,2.824)--(-4.429,2.772)--(-4.454,2.771);
\filldraw[fill opacity=0.8,fill=gray!20,draw=none](-4.448,2.808)--(-4.456,2.823)--(-4.445,2.824)--cycle;
\draw(-4.456,2.823)--(-4.445,2.824);
\filldraw[fill opacity=0.8,fill=gray!20,draw=none](-4.406,2.785)--(-4.424,2.772)--(-4.429,2.772)--(-4.428,2.791)--(-4.406,2.796)--cycle;
\draw(-4.424,2.772)--(-4.429,2.772)--(-4.428,2.791);
\filldraw[fill opacity=0.8,fill=gray!20,draw=none](-4.399,2.791)--(-4.406,2.785)--(-4.406,2.796)--(-4.395,2.798)--cycle;
\filldraw[fill opacity=0.8,fill=gray!20,draw=none](-4.399,2.791)--(-4.395,2.798)--(-4.386,2.8)--cycle;
\filldraw[fill opacity=0.8,fill=gray!20,draw=none](-4.594,2.907)--(-4.577,2.878)--(-7.403,1.874)--(-7.496,1.89)--(-4.671,2.894)--cycle;
\draw(-4.577,2.878)--(-7.403,1.874);
\draw(-7.496,1.89)--(-4.671,2.894);
\filldraw[fill opacity=0.8,fill=gray!20,draw=none](-4.399,2.791)--(-4.427,2.826)--(-4.418,2.837)--(-4.395,2.798)--cycle;
\filldraw[fill opacity=0.8,fill=gray!20,draw=none](-4.429,2.769)--(-4.429,2.772)--(-4.424,2.772)--cycle;
\draw(-4.429,2.769)--(-4.429,2.772)--(-4.424,2.772);
\filldraw[fill opacity=0.8,fill=gray!20,draw=none](-4.417,2.834)--(-4.436,2.865)--(-4.408,2.842)--cycle;
\draw(-4.436,2.865)--(-4.408,2.842);
\filldraw[fill opacity=0.8,fill=gray!20,draw=none](-4.427,2.826)--(-4.461,2.869)--(-4.457,2.882)--(-4.436,2.865)--(-4.418,2.837)--cycle;
\draw(-4.457,2.882)--(-4.436,2.865);
\filldraw[fill opacity=0.8,fill=gray!20,draw=none](-4.588,2.865)--(-4.589,2.876)--(-4.57,2.901)--(-4.533,2.879)--(-4.533,2.876)--cycle;
\draw(-4.533,2.879)--(-4.533,2.876)--(-4.588,2.865);
\filldraw[fill opacity=0.8,fill=gray!20,draw=none](-4.6,2.919)--(-4.601,2.92)--(-4.536,2.933)--(-4.533,2.879)--cycle;
\draw(-4.601,2.92)--(-4.536,2.933)--(-4.533,2.879);
\filldraw[fill opacity=0.8,fill=gray!20,draw=none](-4.519,2.92)--(-4.476,2.927)--(-4.472,2.922)--(-4.471,2.916)--(-4.557,2.886)--cycle;
\draw(-4.471,2.916)--(-4.557,2.886);
\filldraw[fill opacity=0.8,fill=gray!20,draw=none](-4.446,2.89)--(-4.444,2.901)--(-4.423,2.879)--cycle;
\filldraw[fill opacity=0.8,fill=gray!20,draw=none](-4.456,2.823)--(-4.523,2.82)--(-4.533,2.876)--(-4.497,2.878)--cycle;
\draw(-4.456,2.823)--(-4.523,2.82)--(-4.533,2.876)--(-4.497,2.878);
\filldraw[fill opacity=0.8,fill=gray!20,draw=none](-4.49,2.848)--(-4.52,2.833)--(-4.523,2.844)--(-4.49,2.855)--cycle;
\draw(-4.523,2.844)--(-4.49,2.855);
\filldraw[fill opacity=0.8,fill=gray!20,draw=none](-4.489,2.834)--(-4.493,2.83)--(-4.499,2.84)--(-4.489,2.845)--cycle;
\draw(-4.499,2.84)--(-4.489,2.845);
\filldraw[fill opacity=0.8,fill=gray!20,draw=none](-4.599,2.906)--(-4.671,2.894)--(-4.601,2.919)--cycle;
\draw(-4.671,2.894)--(-4.601,2.919);
\filldraw[fill opacity=0.8,fill=gray!20,draw=none](-4.582,2.886)--(-4.596,2.911)--(-4.59,2.913)--(-4.57,2.901)--cycle;
\filldraw[fill opacity=0.8,fill=gray!20,draw=none](-4.582,2.886)--(-4.565,2.912)--(-4.519,2.92)--(-4.557,2.886)--(-4.577,2.878)--cycle;
\draw(-4.557,2.886)--(-4.577,2.878);
\filldraw[fill opacity=0.8,fill=gray!20,draw=none](-4.602,2.92)--(-4.619,2.936)--(-4.618,2.943)--(-4.617,2.944)--(-4.536,2.941)--(-4.536,2.933)--cycle;
\draw(-4.619,2.936)--(-4.618,2.943);
\draw(-4.536,2.941)--(-4.536,2.933)--(-4.602,2.92);
\filldraw[fill opacity=0.8,fill=gray!20,draw=none](-4.615,2.914)--(-4.661,2.897)--(-4.619,2.934)--cycle;
\draw(-4.615,2.914)--(-4.661,2.897);
\filldraw[fill opacity=0.8,fill=gray!20,draw=none](-4.594,2.907)--(-4.599,2.906)--(-4.601,2.919)--cycle;
\filldraw[fill opacity=0.8,fill=gray!20,draw=none](-4.601,2.919)--(-4.615,2.914)--(-4.619,2.934)--(-4.618,2.935)--cycle;
\draw(-4.601,2.919)--(-4.615,2.914);
\filldraw[fill opacity=0.8,fill=gray!20,draw=none](-4.62,2.904)--(-4.621,2.916)--(-4.602,2.92)--(-4.59,2.913)--cycle;
\draw(-4.62,2.904)--(-4.621,2.916)--(-4.602,2.92);
\filldraw[fill opacity=0.8,fill=gray!20,draw=none](-4.46,2.939)--(-4.464,2.929)--(-4.465,2.936)--cycle;
\filldraw[fill opacity=0.8,fill=gray!20,draw=none](-4.458,2.93)--(-4.594,2.907)--(-4.601,2.919)--(-4.494,2.957)--cycle;
\draw(-4.601,2.919)--(-4.494,2.957);
\filldraw[fill opacity=0.8,fill=gray!20,draw=none](-4.472,2.922)--(-4.468,2.917)--(-4.471,2.916)--cycle;
\draw(-4.468,2.917)--(-4.471,2.916);
\filldraw[fill opacity=0.8,fill=gray!20,draw=none](-4.48,2.927)--(-4.47,2.921)--(-4.469,2.918)--(-4.537,2.876)--(-4.548,2.898)--cycle;
\draw(-4.469,2.918)--(-4.537,2.876);
\filldraw[fill opacity=0.8,fill=gray!20,draw=none](-4.47,2.859)--(-4.475,2.854)--(-4.476,2.86)--(-4.469,2.863)--cycle;
\draw(-4.476,2.86)--(-4.469,2.863);
\filldraw[fill opacity=0.8,fill=gray!20,draw=none](-4.461,2.869)--(-4.461,2.87)--(-4.459,2.884)--(-4.457,2.882)--cycle;
\draw(-4.459,2.884)--(-4.457,2.882);
\filldraw[fill opacity=0.8,fill=gray!20,draw=none](-4.474,2.853)--(-4.489,2.834)--(-4.489,2.845)--(-4.475,2.855)--cycle;
\draw(-4.489,2.845)--(-4.475,2.855);
\filldraw[fill opacity=0.8,fill=gray!20,draw=none](-4.472,2.856)--(-4.489,2.845)--(-4.49,2.848)--cycle;
\draw(-4.472,2.856)--(-4.489,2.845);
\filldraw[fill opacity=0.8,fill=gray!20,draw=none](-4.475,2.854)--(-4.49,2.837)--(-4.49,2.855)--(-4.476,2.86)--cycle;
\draw(-4.49,2.855)--(-4.476,2.86);
\filldraw[fill opacity=0.8,fill=gray!20,draw=none](-4.474,2.89)--(-4.471,2.86)--(-4.472,2.856)--(-4.475,2.855)--(-4.499,2.9)--(-4.492,2.904)--cycle;
\draw(-4.499,2.9)--(-4.492,2.904);
\filldraw[fill opacity=0.8,fill=gray!20,draw=none](-4.439,2.901)--(-4.44,2.898)--(-4.445,2.903)--cycle;
\filldraw[fill opacity=0.8,fill=gray!20,draw=none](-4.47,2.859)--(-4.469,2.863)--(-4.466,2.864)--cycle;
\draw(-4.469,2.863)--(-4.466,2.864);
\filldraw[fill opacity=0.8,fill=gray!20,draw=none](-4.461,2.87)--(-4.466,2.876)--(-4.469,2.893)--(-4.459,2.884)--cycle;
\draw(-4.469,2.893)--(-4.459,2.884);
\filldraw[fill opacity=0.8,fill=gray!20,draw=none](-4.468,2.885)--(-4.466,2.876)--(-4.471,2.86)--(-4.474,2.89)--cycle;
\filldraw[fill opacity=0.8,fill=gray!20,draw=none](-4.446,2.89)--(-4.455,2.895)--(-4.457,2.906)--(-4.444,2.901)--cycle;
\filldraw[fill opacity=0.8,fill=gray!20,draw=none](-4.459,2.896)--(-4.466,2.876)--(-4.474,2.915)--(-4.467,2.919)--cycle;
\draw(-4.474,2.915)--(-4.467,2.919);
\filldraw[fill opacity=0.8,fill=gray!20,draw=none](-4.468,2.885)--(-4.474,2.89)--(-4.476,2.914)--(-4.474,2.915)--cycle;
\draw(-4.476,2.914)--(-4.474,2.915);
\filldraw[fill opacity=0.8,fill=gray!20,draw=none](-4.474,2.89)--(-4.492,2.904)--(-4.476,2.914)--cycle;
\draw(-4.492,2.904)--(-4.476,2.914);
\filldraw[fill opacity=0.8,fill=gray!20,draw=none](-4.466,2.876)--(-4.497,2.916)--(-4.469,2.893)--cycle;
\draw(-4.497,2.916)--(-4.469,2.893);
\filldraw[fill opacity=0.8,fill=gray!20,draw=none](-4.449,2.923)--(-4.462,2.919)--(-4.45,2.924)--cycle;
\draw(-4.462,2.919)--(-4.45,2.924);
\filldraw[fill opacity=0.8,fill=gray!20,draw=none](-4.455,2.895)--(-4.459,2.897)--(-4.468,2.922)--(-4.463,2.923)--(-4.46,2.92)--cycle;
\draw(-4.468,2.922)--(-4.463,2.923);
\filldraw[fill opacity=0.8,fill=gray!20,draw=none](-4.448,2.913)--(-4.449,2.923)--(-4.439,2.912)--cycle;
\filldraw[fill opacity=0.8,fill=gray!20,draw=none](-4.454,2.912)--(-4.459,2.896)--(-4.467,2.919)--(-4.457,2.926)--cycle;
\draw(-4.467,2.919)--(-4.457,2.926);
\filldraw[fill opacity=0.8,fill=gray!20,draw=none](-4.47,2.921)--(-4.467,2.919)--(-4.469,2.918)--cycle;
\draw(-4.467,2.919)--(-4.469,2.918);
\filldraw[fill opacity=0.8,fill=gray!20,draw=none](-4.459,2.897)--(-4.497,2.916)--(-4.468,2.922)--cycle;
\draw(-4.497,2.916)--(-4.468,2.922);
\filldraw[fill opacity=0.8,fill=gray!20,draw=none](-4.448,2.913)--(-4.454,2.913)--(-4.457,2.926)--(-4.449,2.923)--cycle;
\filldraw[fill opacity=0.8,fill=gray!20,draw=none](-4.472,2.931)--(-4.465,2.936)--(-4.463,2.923)--(-4.47,2.921)--cycle;
\draw(-4.463,2.923)--(-4.47,2.921);
\filldraw[fill opacity=0.8,fill=gray!20,draw=none](-4.448,2.923)--(-4.449,2.923)--(-4.45,2.924)--cycle;
\filldraw[fill opacity=0.8,fill=gray!20,draw=none](-4.458,2.93)--(-4.452,2.931)--(-4.449,2.924)--cycle;
\filldraw[fill opacity=0.8,fill=gray!20,draw=none](-4.452,2.932)--(-4.449,2.924)--(-4.45,2.923)--cycle;
\filldraw[fill opacity=0.8,fill=gray!20,draw=none](-4.457,2.926)--(-4.459,2.932)--(-4.449,2.923)--cycle;
\filldraw[fill opacity=0.8,fill=gray!20,draw=none](-4.454,2.912)--(-4.457,2.926)--(-4.447,2.932)--cycle;
\draw(-4.457,2.926)--(-4.447,2.932);
\filldraw[fill opacity=0.8,fill=gray!20,draw=none](-4.452,2.94)--(-3.598,1.125)--(-3.658,1.133)--(-4.497,2.916)--cycle;
\draw(-3.658,1.133)--(-4.497,2.916)--(-4.452,2.94)--(-3.598,1.125);
\filldraw[fill opacity=0.8,fill=gray!20,draw=none](-4.497,2.916)--(-4.439,2.912)--(-4.458,2.886)--cycle;
\draw(-4.439,2.912)--(-4.458,2.886)--(-4.497,2.916);
\filldraw[fill opacity=0.8,fill=gray!20,draw=none](-3.268,.769)--(-3.274,.77)--(-3.275,.765)--cycle;
\draw(-3.274,.77)--(-3.275,.765);
\filldraw[fill opacity=0.8,fill=gray!20,draw=none](-3.351,.745)--(-3.33,.74)--(-3.338,.774)--cycle;
\draw(-3.351,.745)--(-3.33,.74)--(-3.338,.774);
\filldraw[fill opacity=0.8,fill=gray!20,draw=none](-2.775,2.021)--(-2.774,2.041)--(-2.772,2.046)--(-2.768,2.021)--cycle;
\draw(-2.772,2.046)--(-2.768,2.021);
\filldraw[fill opacity=0.8,fill=gray!20,draw=none](-2.769,2.021)--(-2.785,2.024)--(-2.761,1.833)--(-2.754,1.907)--(-2.765,1.998)--cycle;
\draw(-2.785,2.024)--(-2.761,1.833);
\draw(-2.754,1.907)--(-2.765,1.998);
\filldraw[fill opacity=0.8,fill=gray!20,draw=none](-2.721,2.076)--(-2.719,2.05)--(-2.723,2.039)--cycle;
\filldraw[fill opacity=0.8,fill=gray!20,draw=none](-3.215,.858)--(-3.219,.818)--(-3.169,.821)--(-3.166,.856)--cycle;
\draw(-3.169,.821)--(-3.166,.856)--(-3.215,.858)--(-3.219,.818);
\filldraw[fill opacity=0.8,fill=gray!20,draw=none](-3.092,1.639)--(-3.021,1.082)--(-2.915,1.091)--(-2.986,1.648)--cycle;
\draw(-3.092,1.639)--(-3.021,1.082)--(-2.915,1.091)--(-2.986,1.648);
\filldraw[fill opacity=0.8,fill=gray!20,draw=none](-2.88,1.653)--(-2.976,1.634)--(-3.234,1.645)--(-3.241,1.651)--cycle;
\draw(-2.88,1.653)--(-2.976,1.634);
\filldraw[fill opacity=0.8,fill=gray!20,draw=none](-3.2,1.765)--(-3.113,1.086)--(-3.021,1.082)--(-3.088,1.613)--cycle;
\draw(-3.2,1.765)--(-3.113,1.086)--(-3.021,1.082)--(-3.088,1.613);
\filldraw[fill opacity=0.8,fill=gray!20,draw=none](-2.812,2.262)--(-2.827,2.281)--(-2.83,2.305)--cycle;
\draw(-2.827,2.281)--(-2.83,2.305);
\filldraw[fill opacity=0.8,fill=gray!20,draw=none](-2.858,3)--(-2.835,2.979)--(-2.78,2.542)--(-2.789,2.379)--(-2.843,2.805)--cycle;
\draw(-2.858,3)--(-2.835,2.979)--(-2.78,2.542);
\draw(-2.789,2.379)--(-2.843,2.805);
\filldraw[fill opacity=0.8,fill=gray!20,draw=none](-2.858,3)--(-2.843,2.805)--(-2.87,3.01)--cycle;
\draw(-2.843,2.805)--(-2.87,3.01)--(-2.858,3);
\filldraw[fill opacity=0.8,fill=gray!20,draw=none](-3.472,2.926)--(-3.446,2.721)--(-3.414,2.368)--(-3.419,2.354)--(-3.487,2.889)--cycle;
\draw(-3.419,2.354)--(-3.487,2.889)--(-3.472,2.926)--(-3.446,2.721);
\filldraw[fill opacity=0.8,fill=gray!20,draw=none](-3.432,2.95)--(-3.446,2.721)--(-3.472,2.926)--cycle;
\draw(-3.446,2.721)--(-3.472,2.926)--(-3.432,2.95);
\filldraw[fill opacity=0.8,fill=gray!20](-3.014,2.87)--(-2.914,2.904)--(-2.851,2.942)--(-2.835,2.979)--(-2.87,3.01)--(-2.948,3.029)--(-3.059,3.033)--(-3.186,3.023)--(-3.309,2.998)--(-3.409,2.965)--(-3.472,2.926)--(-3.487,2.889)--(-3.453,2.859)--(-3.375,2.84)--(-3.264,2.835)--(-3.137,2.846)--cycle;
\filldraw[fill opacity=0.8,fill=gray!20,draw=none](-3.234,.693)--(-3.234,.697)--(-3.245,.697)--cycle;
\draw(-3.234,.693)--(-3.234,.697)--(-3.245,.697);
\filldraw[fill opacity=0.8,fill=gray!20,draw=none](-3.272,.791)--(-3.276,.791)--(-3.273,.79)--cycle;
\draw(-3.276,.791)--(-3.273,.79);
\filldraw[fill opacity=0.8,fill=gray!20,draw=none](-3.273,.79)--(-3.221,.795)--(-3.219,.818)--cycle;
\draw(-3.221,.795)--(-3.219,.818);
\filldraw[fill opacity=0.8,fill=gray!20,draw=none](-3.239,.693)--(-3.247,.699)--(-3.247,.693)--cycle;
\draw(-3.247,.699)--(-3.247,.693);
\filldraw[fill opacity=0.8,fill=gray!20,draw=none](-3.25,.778)--(-3.246,.778)--(-3.245,.779)--cycle;
\draw(-3.25,.778)--(-3.246,.778)--(-3.245,.779);
\filldraw[fill opacity=0.8,fill=gray!20,draw=none](-3.272,.791)--(-3.24,.823)--(-3.339,.848)--(-3.353,.81)--(-3.276,.791)--cycle;
\draw(-3.24,.823)--(-3.339,.848);
\draw(-3.353,.81)--(-3.276,.791);
\filldraw[fill opacity=0.8,fill=gray!20,draw=none](-3.224,.76)--(-3.242,.778)--(-3.246,.778)--(-3.246,.755)--cycle;
\draw(-3.242,.778)--(-3.246,.778)--(-3.246,.755);
\filldraw[fill opacity=0.8,fill=gray!20,draw=none](-3.52,1.171)--(-3.488,1.102)--(-3.509,1.107)--(-3.545,1.127)--(-3.645,1.34)--cycle;
\draw(-3.52,1.171)--(-3.488,1.102);
\draw(-3.545,1.127)--(-3.645,1.34);
\filldraw[fill opacity=0.8,fill=gray!20,draw=none](-3.562,1.138)--(-3.566,1.173)--(-3.545,1.127)--cycle;
\draw(-3.566,1.173)--(-3.545,1.127);
\filldraw[fill opacity=0.8,fill=gray!20,draw=none](-3.242,.778)--(-3.245,.779)--(-3.246,.778)--cycle;
\draw(-3.245,.779)--(-3.246,.778)--(-3.242,.778);
\filldraw[fill opacity=0.8,fill=gray!20,draw=none](-3.637,1.108)--(-3.644,1.103)--(-3.646,1.106)--cycle;
\draw(-3.644,1.103)--(-3.646,1.106);
\filldraw[fill opacity=0.8,fill=gray!20,draw=none](-3.566,1.023)--(-3.568,1.021)--(-3.515,.907)--(-3.47,.931)--(-3.527,1.052)--cycle;
\draw(-3.568,1.021)--(-3.515,.907)--(-3.47,.931)--(-3.527,1.052);
\filldraw[fill opacity=0.8,fill=gray!20,draw=none](-3.497,1.047)--(-3.637,1.108)--(-3.644,1.103)--(-3.676,1.071)--(-3.539,1.012)--cycle;
\draw(-3.676,1.071)--(-3.539,1.012)--(-3.497,1.047)--(-3.637,1.108);
\filldraw[fill opacity=0.8,fill=gray!20,draw=none](-4.448,2.808)--(-4.454,2.771)--(-4.484,2.77)--(-4.517,2.8)--(-4.523,2.82)--(-4.456,2.823)--cycle;
\draw(-4.454,2.771)--(-4.484,2.77);
\draw(-4.517,2.8)--(-4.523,2.82)--(-4.456,2.823);
\filldraw[fill opacity=0.8,fill=gray!20,draw=none](-4.461,2.749)--(-4.464,2.751)--(-4.454,2.771)--(-4.429,2.772)--(-4.429,2.769)--cycle;
\draw(-4.454,2.771)--(-4.429,2.772)--(-4.429,2.769);
\filldraw[fill opacity=0.8,fill=gray!20,draw=none](-4.464,2.751)--(-4.484,2.77)--(-4.454,2.771)--cycle;
\draw(-4.484,2.77)--(-4.454,2.771);
\filldraw[fill opacity=0.8,fill=gray!20,draw=none](-4.464,2.751)--(-4.466,2.746)--(-4.49,2.731)--(-4.507,2.769)--(-4.484,2.77)--cycle;
\draw(-4.49,2.731)--(-4.507,2.769)--(-4.484,2.77);
\filldraw[fill opacity=0.8,fill=gray!20,draw=none](-4.461,2.749)--(-4.466,2.746)--(-4.464,2.751)--cycle;
\filldraw[fill opacity=0.8,fill=gray!20,draw=none](-4.493,2.83)--(-4.511,2.806)--(-4.518,2.828)--(-4.499,2.84)--cycle;
\draw(-4.518,2.828)--(-4.499,2.84);
\filldraw[fill opacity=0.8,fill=gray!20,draw=none](-4.49,2.848)--(-4.489,2.845)--(-4.518,2.828)--(-4.52,2.833)--cycle;
\draw(-4.489,2.845)--(-4.518,2.828);
\filldraw[fill opacity=0.8,fill=gray!20,draw=none](-4.501,2.842)--(-4.495,2.832)--(-4.514,2.811)--(-4.52,2.833)--cycle;
\filldraw[fill opacity=0.8,fill=gray!20,draw=none](-4.538,2.817)--(-4.559,2.834)--(-4.561,2.852)--(-4.562,2.87)--(-4.533,2.876)--(-4.523,2.82)--cycle;
\draw(-4.562,2.87)--(-4.533,2.876)--(-4.523,2.82)--(-4.538,2.817);
\filldraw[fill opacity=0.8,fill=gray!20,draw=none](-4.538,2.817)--(-4.557,2.813)--(-4.559,2.834)--cycle;
\draw(-4.538,2.817)--(-4.557,2.813);
\filldraw[fill opacity=0.8,fill=gray!20,draw=none](-4.56,2.812)--(-4.564,2.81)--(-4.574,2.825)--(-4.565,2.828)--cycle;
\draw(-4.574,2.825)--(-4.565,2.828);
\filldraw[fill opacity=0.8,fill=gray!20,draw=none](-4.56,2.812)--(-4.565,2.828)--(-4.56,2.83)--cycle;
\draw(-4.565,2.828)--(-4.56,2.83);
\filldraw[fill opacity=0.8,fill=gray!20,draw=none](-4.559,2.834)--(-4.557,2.813)--(-4.56,2.813)--(-4.579,2.85)--cycle;
\draw(-4.557,2.813)--(-4.56,2.813);
\filldraw[fill opacity=0.8,fill=gray!20,draw=none](-4.52,2.833)--(-4.56,2.812)--(-4.56,2.83)--(-4.523,2.844)--cycle;
\draw(-4.56,2.83)--(-4.523,2.844);
\filldraw[fill opacity=0.8,fill=gray!20,draw=none](-4.49,2.848)--(-4.52,2.833)--(-4.537,2.876)--(-4.503,2.897)--cycle;
\draw(-4.537,2.876)--(-4.503,2.897);
\filldraw[fill opacity=0.8,fill=gray!20,draw=none](-4.501,2.842)--(-4.49,2.848)--(-4.49,2.837)--(-4.495,2.832)--cycle;
\filldraw[fill opacity=0.8,fill=gray!20,draw=none](-4.484,2.77)--(-4.507,2.769)--(-4.517,2.8)--cycle;
\draw(-4.484,2.77)--(-4.507,2.769)--(-4.517,2.8);
\filldraw[fill opacity=0.8,fill=gray!20,draw=none](-4.489,2.823)--(-4.487,2.791)--(-4.488,2.788)--(-4.504,2.778)--(-4.511,2.806)--(-4.493,2.83)--cycle;
\draw(-4.488,2.788)--(-4.504,2.778);
\filldraw[fill opacity=0.8,fill=gray!20,draw=none](-4.475,2.855)--(-4.49,2.848)--(-4.503,2.897)--(-4.499,2.9)--cycle;
\draw(-4.503,2.897)--(-4.499,2.9);
\filldraw[fill opacity=0.8,fill=gray!20,draw=none](-4.489,2.823)--(-4.493,2.83)--(-4.489,2.834)--cycle;
\filldraw[fill opacity=0.8,fill=gray!20,draw=none](-4.473,2.847)--(-4.482,2.812)--(-4.489,2.823)--(-4.489,2.834)--(-4.474,2.853)--cycle;
\filldraw[fill opacity=0.8,fill=gray!20,draw=none](-4.537,2.817)--(-4.523,2.82)--(-4.517,2.8)--cycle;
\draw(-4.537,2.817)--(-4.523,2.82)--(-4.517,2.8);
\filldraw[fill opacity=0.8,fill=gray!20,draw=none](-4.515,2.767)--(-4.547,2.799)--(-4.545,2.816)--(-4.537,2.817)--(-4.517,2.8)--(-4.507,2.769)--cycle;
\draw(-4.545,2.816)--(-4.537,2.817);
\draw(-4.517,2.8)--(-4.507,2.769)--(-4.515,2.767);
\filldraw[fill opacity=0.8,fill=gray!20,draw=none](-4.511,2.806)--(-4.521,2.801)--(-4.537,2.816)--(-4.518,2.828)--cycle;
\draw(-4.537,2.816)--(-4.518,2.828);
\filldraw[fill opacity=0.8,fill=gray!20,draw=none](-4.511,2.806)--(-4.517,2.798)--(-4.521,2.801)--cycle;
\filldraw[fill opacity=0.8,fill=gray!20,draw=none](-4.506,2.788)--(-4.517,2.798)--(-4.511,2.806)--cycle;
\filldraw[fill opacity=0.8,fill=gray!20,draw=none](-4.49,2.823)--(-4.491,2.803)--(-4.51,2.796)--(-4.514,2.811)--(-4.495,2.832)--cycle;
\draw(-4.491,2.803)--(-4.51,2.796);
\filldraw[fill opacity=0.8,fill=gray!20,draw=none](-4.49,2.823)--(-4.495,2.832)--(-4.49,2.837)--cycle;
\filldraw[fill opacity=0.8,fill=gray!20,draw=none](-4.482,2.812)--(-4.487,2.791)--(-4.489,2.823)--cycle;
\filldraw[fill opacity=0.8,fill=gray!20,draw=none](-4.474,2.848)--(-4.489,2.804)--(-4.491,2.803)--(-4.49,2.837)--(-4.475,2.854)--cycle;
\draw(-4.489,2.804)--(-4.491,2.803);
\filldraw[fill opacity=0.8,fill=gray!20,draw=none](-4.474,2.853)--(-4.475,2.855)--(-4.472,2.856)--cycle;
\draw(-4.475,2.855)--(-4.472,2.856);
\filldraw[fill opacity=0.8,fill=gray!20,draw=none](-4.473,2.847)--(-4.474,2.853)--(-4.472,2.856)--(-4.471,2.857)--cycle;
\draw(-4.472,2.856)--(-4.471,2.857);
\filldraw[fill opacity=0.8,fill=gray!20,draw=none](-4.471,2.857)--(-4.472,2.856)--(-4.471,2.86)--cycle;
\draw(-4.471,2.857)--(-4.472,2.856);
\filldraw[fill opacity=0.8,fill=gray!20,draw=none](-4.474,2.848)--(-4.475,2.854)--(-4.47,2.859)--cycle;
\filldraw[fill opacity=0.8,fill=gray!20,draw=none](-4.491,2.8)--(-4.503,2.784)--(-4.509,2.79)--(-4.51,2.796)--(-4.491,2.803)--cycle;
\draw(-4.51,2.796)--(-4.491,2.803);
\filldraw[fill opacity=0.8,fill=gray!20,draw=none](-4.491,2.8)--(-4.491,2.803)--(-4.489,2.804)--cycle;
\draw(-4.491,2.803)--(-4.489,2.804);
\filldraw[fill opacity=0.8,fill=gray!20,draw=none](-4.49,2.933)--(-4.502,2.918)--(-4.548,2.898)--(-4.557,2.916)--(-4.512,2.945)--cycle;
\draw(-4.557,2.916)--(-4.512,2.945);
\filldraw[fill opacity=0.8,fill=gray!20,draw=none](-4.617,2.944)--(-4.568,2.98)--(-4.533,2.987)--(-4.536,2.941)--cycle;
\draw(-4.568,2.98)--(-4.533,2.987)--(-4.536,2.941);
\filldraw[fill opacity=0.8,fill=gray!20,draw=none](-4.535,2.948)--(-4.533,2.987)--(-4.508,2.988)--cycle;
\draw(-4.535,2.948)--(-4.533,2.987)--(-4.508,2.988);
\filldraw[fill opacity=0.8,fill=gray!20,draw=none](-4.527,2.875)--(-4.535,2.902)--(-4.497,2.916)--cycle;
\filldraw[fill opacity=0.8,fill=gray!20,draw=none](-4.617,2.944)--(-4.618,2.944)--(-4.615,2.971)--(-4.568,2.98)--cycle;
\draw(-4.618,2.944)--(-4.615,2.971)--(-4.568,2.98);
\filldraw[fill opacity=0.8,fill=gray!20,draw=none](-4.619,2.934)--(-4.661,2.897)--(-7.505,1.887)--(-7.504,1.891)--(-7.433,1.948)--(-4.621,2.947)--cycle;
\draw(-4.661,2.897)--(-7.505,1.887);
\draw(-7.433,1.948)--(-4.621,2.947);
\filldraw[fill opacity=0.8,fill=gray!20,draw=none](-4.564,2.81)--(-4.774,2.702)--(-7.455,1.75)--(-7.494,1.788)--(-4.574,2.825)--cycle;
\draw(-4.774,2.702)--(-7.455,1.75);
\draw(-7.494,1.788)--(-4.574,2.825);
\filldraw[fill opacity=0.8,fill=gray!20,draw=none](-5.081,2.571)--(-5.35,2.456)--(-5.494,2.405)--(-5.352,2.497)--(-4.952,2.639)--cycle;
\draw(-5.35,2.456)--(-5.494,2.405);
\draw(-5.352,2.497)--(-4.952,2.639);
\filldraw[fill opacity=0.8,fill=gray!20,draw=none](-5.257,2.482)--(-5.511,2.374)--(-5.565,2.355)--(-5.509,2.394)--(-5.35,2.456)--(-5.195,2.511)--cycle;
\draw(-5.511,2.374)--(-5.565,2.355);
\draw(-5.35,2.456)--(-5.195,2.511);
\filldraw[fill opacity=0.8,fill=gray!20,draw=none](-5.195,2.511)--(-5.35,2.456)--(-5.081,2.571)--cycle;
\draw(-5.195,2.511)--(-5.35,2.456);
\filldraw[fill opacity=0.8,fill=gray!20,draw=none](-5.257,2.482)--(-5.195,2.511)--(-5.16,2.524)--cycle;
\draw(-5.195,2.511)--(-5.16,2.524);
\filldraw[fill opacity=0.8,fill=gray!20,draw=none](-5.16,2.524)--(-5.195,2.511)--(-5.081,2.571)--(-4.913,2.643)--cycle;
\draw(-5.16,2.524)--(-5.195,2.511);
\filldraw[fill opacity=0.8,fill=gray!20,draw=none](-5.081,2.571)--(-4.952,2.639)--(-4.774,2.702)--cycle;
\draw(-4.952,2.639)--(-4.774,2.702);
\filldraw[fill opacity=0.8,fill=gray!20,draw=none](-4.561,2.852)--(-4.563,2.87)--(-4.562,2.87)--cycle;
\draw(-4.563,2.87)--(-4.562,2.87);
\filldraw[fill opacity=0.8,fill=gray!20,draw=none](-4.559,2.834)--(-4.579,2.85)--(-4.588,2.865)--(-4.563,2.87)--cycle;
\draw(-4.588,2.865)--(-4.563,2.87);
\filldraw[fill opacity=0.8,fill=gray!20,draw=none](-4.582,2.886)--(-4.594,2.907)--(-4.565,2.912)--cycle;
\filldraw[fill opacity=0.8,fill=gray!20,draw=none](-4.582,2.886)--(-4.589,2.876)--(-4.611,2.883)--(-4.619,2.893)--(-4.62,2.904)--(-4.596,2.911)--cycle;
\draw(-4.619,2.893)--(-4.62,2.904);
\filldraw[fill opacity=0.8,fill=gray!20,draw=none](-4.56,2.802)--(-5.076,2.482)--(-5.284,2.413)--(-4.584,2.847)--cycle;
\draw(-4.56,2.802)--(-5.076,2.482);
\draw(-5.284,2.413)--(-4.584,2.847);
\filldraw[fill opacity=0.8,fill=gray!20,draw=none](-4.597,2.864)--(-4.585,2.846)--(-4.913,2.643)--(-5.16,2.524)--(-5.257,2.482)--(-4.617,2.879)--cycle;
\draw(-4.585,2.846)--(-4.913,2.643);
\draw(-5.257,2.482)--(-4.617,2.879);
\filldraw[fill opacity=0.8,fill=gray!20,draw=none](-4.597,2.864)--(-4.579,2.85)--(-4.585,2.846)--cycle;
\draw(-4.579,2.85)--(-4.585,2.846);
\filldraw[fill opacity=0.8,fill=gray!20,draw=none](-4.565,2.812)--(-4.597,2.806)--(-4.615,2.86)--(-4.592,2.865)--cycle;
\draw(-4.565,2.812)--(-4.597,2.806)--(-4.615,2.86)--(-4.592,2.865);
\filldraw[fill opacity=0.8,fill=gray!20,draw=none](-4.579,2.85)--(-4.588,2.856)--(-4.592,2.865)--(-4.588,2.865)--cycle;
\draw(-4.592,2.865)--(-4.588,2.865);
\filldraw[fill opacity=0.8,fill=gray!20,draw=none](-4.597,2.864)--(-4.617,2.879)--(-4.611,2.883)--cycle;
\draw(-4.617,2.879)--(-4.611,2.883);
\filldraw[fill opacity=0.8,fill=gray!20,draw=none](-4.588,2.865)--(-4.599,2.863)--(-4.589,2.876)--cycle;
\draw(-4.588,2.865)--(-4.599,2.863);
\filldraw[fill opacity=0.8,fill=gray!20,draw=none](-4.611,2.883)--(-4.589,2.876)--(-4.598,2.865)--cycle;
\filldraw[fill opacity=0.8,fill=gray!20,draw=none](-4.548,2.898)--(-4.537,2.876)--(-4.579,2.85)--(-4.597,2.864)--(-4.605,2.874)--cycle;
\draw(-4.537,2.876)--(-4.579,2.85);
\filldraw[fill opacity=0.8,fill=gray!20,draw=none](-4.531,2.89)--(-4.527,2.875)--(-4.542,2.882)--cycle;
\draw(-4.527,2.875)--(-4.542,2.882);
\filldraw[fill opacity=0.8,fill=gray!20,draw=none](-4.497,2.916)--(-3.646,1.106)--(-3.687,1.089)--(-4.533,2.888)--cycle;
\draw(-3.687,1.089)--(-4.533,2.888)--(-4.497,2.916)--(-3.646,1.106);
\filldraw[fill opacity=0.8,fill=gray!20,draw=none](-3.166,.856)--(-3.225,.871)--(-3.24,.823)--(-3.219,.818)--cycle;
\draw(-3.166,.856)--(-3.225,.871);
\draw(-3.24,.823)--(-3.219,.818);
\filldraw[fill opacity=0.8,fill=gray!20,draw=none](-3.377,.79)--(-3.342,.788)--(-3.373,.795)--cycle;
\draw(-3.342,.788)--(-3.373,.795);
\filldraw[fill opacity=0.8,fill=gray!20](-3.066,.966)--(-3.095,1.015)--(-3.156,1.003)--(-3.14,.952)--cycle;
\filldraw[fill opacity=0.8,fill=gray!20](-3.042,.856)--(-3.048,.912)--(-3.13,.896)--(-3.127,.839)--cycle;
\filldraw[fill opacity=0.8,fill=gray!20,draw=none](-7.54,1.698)--(-7.551,1.665)--(-7.58,1.665)--(-7.58,1.686)--cycle;
\draw(-7.58,1.665)--(-7.58,1.686);
\filldraw[fill opacity=0.8,fill=gray!20,draw=none](-7.531,1.727)--(-7.54,1.698)--(-7.58,1.686)--cycle;
\filldraw[fill opacity=0.8,fill=gray!20,draw=none](-7.514,1.73)--(-7.496,1.69)--(-7.496,1.593)--(-7.531,1.603)--(-7.531,1.727)--cycle;
\draw(-7.496,1.69)--(-7.496,1.593)--(-7.531,1.603)--(-7.531,1.727);
\filldraw[fill opacity=0.8,fill=gray!20,draw=none](-7.551,1.665)--(-7.54,1.698)--(-7.531,1.701)--(-7.531,1.665)--cycle;
\draw(-7.531,1.701)--(-7.531,1.665);
\filldraw[fill opacity=0.8,fill=gray!20,draw=none](-7.54,1.698)--(-7.531,1.727)--(-7.531,1.701)--cycle;
\draw(-7.531,1.727)--(-7.531,1.701);
\filldraw[fill opacity=0.8,fill=gray!20,draw=none](-7.484,1.676)--(-7.481,1.664)--(-7.481,1.581)--(-7.496,1.593)--(-7.496,1.675)--cycle;
\draw(-7.481,1.664)--(-7.481,1.581)--(-7.496,1.593)--(-7.496,1.675);
\filldraw[fill opacity=0.8,fill=gray!20,draw=none](-7.514,1.73)--(-7.496,1.714)--(-7.496,1.69)--cycle;
\draw(-7.496,1.714)--(-7.496,1.69);
\filldraw[fill opacity=0.8,fill=gray!20,draw=none](-7.49,1.696)--(-7.484,1.676)--(-7.496,1.675)--(-7.496,1.703)--cycle;
\draw(-7.496,1.675)--(-7.496,1.703);
\filldraw[fill opacity=0.8,fill=gray!20,draw=none](-7.49,1.696)--(-7.496,1.703)--(-7.496,1.714)--cycle;
\draw(-7.496,1.703)--(-7.496,1.714);
\filldraw[fill opacity=0.8,fill=gray!20,draw=none](-7.479,1.7)--(-7.504,1.691)--(-7.516,1.72)--(-7.515,1.729)--(-7.514,1.729)--cycle;
\draw(-7.479,1.7)--(-7.504,1.691);
\draw(-7.515,1.729)--(-7.514,1.729);
\filldraw[fill opacity=0.8,fill=gray!20,draw=none](-7.47,1.703)--(-7.479,1.7)--(-7.514,1.729)--(-7.497,1.735)--cycle;
\draw(-7.47,1.703)--(-7.479,1.7);
\draw(-7.514,1.729)--(-7.497,1.735);
\filldraw[fill opacity=0.8,fill=gray!20,draw=none](-7.516,1.72)--(-7.504,1.691)--(-7.523,1.684)--cycle;
\draw(-7.504,1.691)--(-7.523,1.684);
\filldraw[fill opacity=0.8,fill=gray!20,draw=none](-7.477,1.699)--(-7.481,1.682)--(-7.49,1.696)--(-7.479,1.7)--cycle;
\draw(-7.49,1.696)--(-7.479,1.7);
\filldraw[fill opacity=0.8,fill=gray!20,draw=none](-7.481,1.705)--(-7.481,1.682)--(-7.49,1.696)--(-7.496,1.714)--(-7.496,1.733)--cycle;
\draw(-7.481,1.705)--(-7.481,1.682);
\draw(-7.496,1.714)--(-7.496,1.733);
\filldraw[fill opacity=0.8,fill=gray!20,draw=none](-7.481,1.682)--(-7.481,1.676)--(-7.484,1.676)--(-7.49,1.696)--cycle;
\draw(-7.481,1.682)--(-7.481,1.676);
\filldraw[fill opacity=0.8,fill=gray!20,draw=none](-7.481,1.676)--(-7.481,1.664)--(-7.484,1.676)--cycle;
\draw(-7.481,1.676)--(-7.481,1.664);
\filldraw[fill opacity=0.8,fill=gray!20,draw=none](-7.481,1.682)--(-7.483,1.674)--(-7.501,1.667)--(-7.523,1.684)--(-7.49,1.696)--cycle;
\draw(-7.483,1.674)--(-7.501,1.667);
\draw(-7.523,1.684)--(-7.49,1.696);
\filldraw[fill opacity=0.8,fill=gray!20,draw=none](-7.283,1.745)--(-7.436,1.69)--(-7.479,1.7)--(-7.206,1.797)--cycle;
\draw(-7.283,1.745)--(-7.436,1.69);
\draw(-7.479,1.7)--(-7.206,1.797);
\filldraw[fill opacity=0.8,fill=gray!20,draw=none](-7.477,1.699)--(-7.436,1.69)--(-7.476,1.676)--(-7.481,1.682)--cycle;
\draw(-7.436,1.69)--(-7.476,1.676);
\filldraw[fill opacity=0.8,fill=gray!20](-7.616,1.547)--(-7.673,1.548)--(-7.723,1.554)--(-7.758,1.564)--(-7.773,1.576)--(-7.766,1.589)--(-7.737,1.6)--(-7.692,1.607)--(-7.637,1.61)--(-7.58,1.609)--(-7.531,1.603)--(-7.496,1.593)--(-7.481,1.581)--(-7.488,1.568)--(-7.516,1.558)--(-7.561,1.55)--cycle;
\filldraw[fill opacity=0.8,fill=gray!20,draw=none](-7.485,1.658)--(-7.488,1.646)--(-7.488,1.568)--(-7.481,1.581)--(-7.481,1.646)--cycle;
\draw(-7.488,1.646)--(-7.488,1.568)--(-7.481,1.581)--(-7.481,1.646);
\filldraw[fill opacity=0.8,fill=gray!20,draw=none](-7.482,1.675)--(-7.485,1.658)--(-7.481,1.646)--(-7.481,1.676)--cycle;
\draw(-7.481,1.646)--(-7.481,1.676);
\filldraw[fill opacity=0.8,fill=gray!20,draw=none](-7.482,1.675)--(-7.481,1.676)--(-7.481,1.679)--cycle;
\draw(-7.481,1.676)--(-7.481,1.679);
\filldraw[fill opacity=0.8,fill=gray!20,draw=none](-7.485,1.672)--(-7.501,1.667)--(-7.483,1.674)--cycle;
\draw(-7.501,1.667)--(-7.483,1.674);
\filldraw[fill opacity=0.8,fill=gray!20,draw=none](-7.485,1.672)--(-7.482,1.675)--(-7.481,1.679)--(-7.481,1.705)--cycle;
\draw(-7.481,1.679)--(-7.481,1.705);
\filldraw[fill opacity=0.8,fill=gray!20,draw=none](-7.481,1.682)--(-7.476,1.676)--(-7.483,1.674)--cycle;
\draw(-7.476,1.676)--(-7.483,1.674);
\filldraw[fill opacity=0.8,fill=gray!20,draw=none](-7.485,1.672)--(-7.486,1.66)--(-7.485,1.658)--(-7.482,1.675)--cycle;
\filldraw[fill opacity=0.8,fill=gray!20,draw=none](-7.446,1.682)--(-7.452,1.68)--(-7.485,1.672)--(-7.483,1.674)--(-7.479,1.675)--cycle;
\draw(-7.446,1.682)--(-7.452,1.68);
\draw(-7.483,1.674)--(-7.479,1.675);
\filldraw[fill opacity=0.8,fill=gray!20,draw=none](-7.364,1.712)--(-7.446,1.682)--(-7.479,1.675)--(-7.283,1.745)--cycle;
\draw(-7.364,1.712)--(-7.446,1.682);
\draw(-7.479,1.675)--(-7.283,1.745);
\filldraw[fill opacity=0.8,fill=gray!20,draw=none](-7.486,1.66)--(-7.488,1.646)--(-7.485,1.658)--cycle;
\filldraw[fill opacity=0.8,fill=gray!20,draw=none](-7.519,1.741)--(-7.514,1.73)--(-7.531,1.727)--cycle;
\filldraw[fill opacity=0.8,fill=gray!20,draw=none](-7.497,1.735)--(-7.514,1.729)--(-7.519,1.741)--(-7.52,1.779)--(-7.519,1.779)--cycle;
\draw(-7.497,1.735)--(-7.514,1.729);
\draw(-7.52,1.779)--(-7.519,1.779);
\filldraw[fill opacity=0.8,fill=gray!20,draw=none](-7.455,1.75)--(-7.497,1.735)--(-7.519,1.779)--(-7.494,1.788)--cycle;
\draw(-7.455,1.75)--(-7.497,1.735);
\draw(-7.519,1.779)--(-7.494,1.788);
\filldraw[fill opacity=0.8,fill=gray!20,draw=none](-7.519,1.741)--(-7.496,1.771)--(-7.496,1.733)--(-7.514,1.73)--cycle;
\draw(-7.496,1.771)--(-7.496,1.733);
\filldraw[fill opacity=0.8,fill=gray!20,draw=none](-7.514,1.729)--(-7.519,1.727)--(-7.519,1.741)--cycle;
\draw(-7.514,1.729)--(-7.519,1.727);
\filldraw[fill opacity=0.8,fill=gray!20,draw=none](-7.516,1.72)--(-7.519,1.727)--(-7.515,1.729)--cycle;
\draw(-7.519,1.727)--(-7.515,1.729);
\filldraw[fill opacity=0.8,fill=gray!20,draw=none](-7.514,1.73)--(-7.496,1.733)--(-7.496,1.714)--cycle;
\draw(-7.496,1.733)--(-7.496,1.714);
\filldraw[fill opacity=0.8,fill=gray!20,draw=none](-7.475,1.71)--(-7.497,1.735)--(-7.472,1.744)--cycle;
\draw(-7.497,1.735)--(-7.472,1.744);
\filldraw[fill opacity=0.8,fill=gray!20,draw=none](-5.494,2.405)--(-7.47,1.703)--(-7.475,1.71)--(-7.472,1.744)--(-5.352,2.497)--cycle;
\draw(-5.494,2.405)--(-7.47,1.703);
\draw(-7.472,1.744)--(-5.352,2.497);
\filldraw[fill opacity=0.8,fill=gray!20,draw=none](-7.481,1.705)--(-7.496,1.733)--(-7.496,1.771)--cycle;
\draw(-7.496,1.733)--(-7.496,1.771);
\filldraw[fill opacity=0.8,fill=gray!20,draw=none](-5.509,2.394)--(-6.084,2.171)--(-7.283,1.745)--(-7.206,1.797)--(-5.494,2.405)--cycle;
\draw(-6.084,2.171)--(-7.283,1.745);
\draw(-7.206,1.797)--(-5.494,2.405);
\filldraw[fill opacity=0.8,fill=gray!20,draw=none](-6.862,1.89)--(-7.364,1.712)--(-7.283,1.745)--(-6.084,2.171)--cycle;
\draw(-6.862,1.89)--(-7.364,1.712);
\draw(-7.283,1.745)--(-6.084,2.171);
\filldraw[fill opacity=0.8,fill=gray!20,draw=none](-7.409,1.706)--(-7.44,1.687)--(-7.446,1.682)--(-7.364,1.712)--cycle;
\draw(-7.446,1.682)--(-7.364,1.712);
\filldraw[fill opacity=0.8,fill=gray!20,draw=none](-7.419,1.705)--(-7.44,1.687)--(-7.409,1.706)--cycle;
\filldraw[fill opacity=0.8,fill=gray!20](-7.424,1.68)--(-7.588,1.751)--(-7.578,1.807)--(-7.414,1.736)--cycle;
\filldraw[fill opacity=0.8,fill=gray!20,draw=none](-7.44,1.687)--(-7.452,1.68)--(-7.446,1.682)--cycle;
\draw(-7.452,1.68)--(-7.446,1.682);
\filldraw[fill opacity=0.8,fill=gray!20](-7.452,1.631)--(-7.616,1.702)--(-7.588,1.751)--(-7.424,1.68)--cycle;
\filldraw[fill opacity=0.8,fill=gray!20,draw=none](-3.461,1.105)--(-3.454,1.091)--(-3.488,1.102)--(-3.52,1.171)--cycle;
\draw(-3.461,1.105)--(-3.454,1.091);
\draw(-3.488,1.102)--(-3.52,1.171);
\filldraw[fill opacity=0.8,fill=gray!20](-3.048,.912)--(-3.066,.966)--(-3.14,.952)--(-3.13,.896)--cycle;
\filldraw[fill opacity=0.8,fill=gray!20,draw=none](-3.373,.795)--(-3.342,.788)--(-3.344,.817)--cycle;
\draw(-3.373,.795)--(-3.342,.788)--(-3.344,.817);
\filldraw[fill opacity=0.8,fill=gray!20,draw=none](-7.601,1.662)--(-7.637,1.619)--(-7.637,1.657)--cycle;
\draw(-7.637,1.619)--(-7.637,1.657);
\filldraw[fill opacity=0.8,fill=gray!20,draw=none](-7.625,1.633)--(-7.601,1.662)--(-7.58,1.665)--(-7.58,1.624)--cycle;
\draw(-7.58,1.665)--(-7.58,1.624);
\filldraw[fill opacity=0.8,fill=gray!20,draw=none](-7.551,1.665)--(-7.57,1.608)--(-7.58,1.609)--(-7.58,1.665)--cycle;
\draw(-7.57,1.608)--(-7.58,1.609)--(-7.58,1.665);
\filldraw[fill opacity=0.8,fill=gray!20,draw=none](-7.58,1.686)--(-7.601,1.662)--(-7.637,1.657)--cycle;
\filldraw[fill opacity=0.8,fill=gray!20,draw=none](-7.551,1.665)--(-7.531,1.665)--(-7.531,1.603)--(-7.57,1.608)--cycle;
\draw(-7.531,1.665)--(-7.531,1.603)--(-7.57,1.608);
\filldraw[fill opacity=0.8,fill=gray!20,draw=none](-7.601,1.662)--(-7.58,1.686)--(-7.58,1.665)--cycle;
\draw(-7.58,1.686)--(-7.58,1.665);
\filldraw[fill opacity=0.8,fill=gray!20,draw=none](-7.498,1.617)--(-7.516,1.566)--(-7.516,1.558)--(-7.488,1.568)--(-7.488,1.624)--cycle;
\draw(-7.516,1.566)--(-7.516,1.558)--(-7.488,1.568)--(-7.488,1.624);
\filldraw[fill opacity=0.8,fill=gray!20,draw=none](-7.498,1.617)--(-7.488,1.624)--(-7.488,1.646)--cycle;
\draw(-7.488,1.624)--(-7.488,1.646);
\filldraw[fill opacity=0.8,fill=gray!20,draw=none](-7.516,1.604)--(-7.498,1.617)--(-7.488,1.646)--cycle;
\filldraw[fill opacity=0.8,fill=gray!20,draw=none](-7.516,1.604)--(-7.516,1.566)--(-7.498,1.617)--cycle;
\draw(-7.516,1.604)--(-7.516,1.566);
\filldraw[fill opacity=0.8,fill=gray!20](-7.495,1.595)--(-7.658,1.666)--(-7.616,1.702)--(-7.452,1.631)--cycle;
\filldraw[fill opacity=0.8,fill=gray!20,draw=none](-2.747,2.223)--(-2.762,2.26)--(-2.754,2.249)--cycle;
\filldraw[fill opacity=0.8,fill=gray!20](-3.236,.733)--(-3.238,.78)--(-3.342,.788)--(-3.33,.74)--cycle;
\filldraw[fill opacity=0.8,fill=gray!20](-3.234,1)--(-3.232,1.044)--(-3.286,1.047)--(-3.311,1.005)--cycle;
\filldraw[fill opacity=0.8,fill=gray!20,draw=none](-7.529,1.83)--(-7.533,1.828)--(-7.534,1.837)--cycle;
\draw(-7.529,1.83)--(-7.533,1.828);
\filldraw[fill opacity=0.8,fill=gray!20,draw=none](-7.531,1.853)--(-7.527,1.86)--(-7.497,1.841)--(-7.529,1.83)--(-7.534,1.837)--(-7.534,1.842)--cycle;
\draw(-7.497,1.841)--(-7.529,1.83);
\filldraw[fill opacity=0.8,fill=gray!20,draw=none](-7.496,1.81)--(-7.496,1.771)--(-7.531,1.836)--cycle;
\draw(-7.496,1.81)--(-7.496,1.771);
\filldraw[fill opacity=0.8,fill=gray!20,draw=none](-7.403,1.874)--(-7.497,1.841)--(-7.527,1.86)--(-7.512,1.884)--(-7.496,1.89)--cycle;
\draw(-7.403,1.874)--(-7.497,1.841);
\draw(-7.512,1.884)--(-7.496,1.89);
\filldraw[fill opacity=0.8,fill=gray!20,draw=none](-7.534,1.848)--(-7.54,1.867)--(-7.531,1.862)--(-7.531,1.853)--cycle;
\draw(-7.531,1.862)--(-7.531,1.853);
\filldraw[fill opacity=0.8,fill=gray!20,draw=none](-7.529,1.861)--(-7.527,1.86)--(-7.531,1.853)--cycle;
\filldraw[fill opacity=0.8,fill=gray!20,draw=none](-7.529,1.861)--(-7.496,1.844)--(-7.496,1.81)--(-7.531,1.836)--(-7.531,1.853)--cycle;
\draw(-7.496,1.844)--(-7.496,1.81);
\draw(-7.531,1.836)--(-7.531,1.853);
\filldraw[fill opacity=0.8,fill=gray!20,draw=none](-7.481,1.814)--(-7.496,1.771)--(-7.496,1.81)--cycle;
\draw(-7.496,1.771)--(-7.496,1.81);
\filldraw[fill opacity=0.8,fill=gray!20,draw=none](-7.481,1.876)--(-7.481,1.814)--(-7.496,1.81)--(-7.496,1.886)--cycle;
\draw(-7.481,1.876)--(-7.481,1.814);
\draw(-7.496,1.81)--(-7.496,1.886);
\filldraw[fill opacity=0.8,fill=gray!20](-7.414,1.736)--(-7.578,1.807)--(-7.588,1.861)--(-7.424,1.789)--cycle;
\filldraw[fill opacity=0.8,fill=gray!20,draw=none](-3.539,1.012)--(-3.676,1.071)--(-3.7,1.02)--(-3.568,.962)--cycle;
\draw(-3.7,1.02)--(-3.568,.962)--(-3.539,1.012)--(-3.676,1.071);
\filldraw[fill opacity=0.8,fill=gray!20,draw=none](-3.691,1.097)--(-3.677,1.068)--(-3.689,1.02)--(-3.69,1.022)--cycle;
\draw(-3.691,1.097)--(-3.677,1.068);
\draw(-3.689,1.02)--(-3.69,1.022);
\filldraw[fill opacity=0.8,fill=gray!20,draw=none](-2.769,2.021)--(-2.765,1.998)--(-2.768,2.021)--cycle;
\draw(-2.765,1.998)--(-2.768,2.021);
\filldraw[fill opacity=0.8,fill=gray!20,draw=none](-2.773,2.056)--(-2.772,2.046)--(-2.774,2.041)--cycle;
\draw(-2.773,2.056)--(-2.772,2.046);
\filldraw[fill opacity=0.8,fill=gray!20,draw=none](-2.721,2.076)--(-2.721,2.081)--(-2.717,2.054)--(-2.719,2.05)--cycle;
\draw(-2.721,2.081)--(-2.717,2.054);
\filldraw[fill opacity=0.8,fill=gray!20,draw=none](-3.313,1.006)--(-3.311,1.005)--(-3.304,1.016)--(-3.314,1.024)--cycle;
\draw(-3.313,1.006)--(-3.311,1.005)--(-3.304,1.016);
\filldraw[fill opacity=0.8,fill=gray!20,draw=none](-2.744,2.107)--(-2.733,2.022)--(-2.713,2.019)--(-2.717,2.054)--cycle;
\draw(-2.744,2.107)--(-2.733,2.022);
\draw(-2.713,2.019)--(-2.717,2.054);
\filldraw[fill opacity=0.8,fill=gray!20](-3.156,1.003)--(-3.176,1.046)--(-3.232,1.044)--(-3.234,1)--cycle;
\filldraw[fill opacity=0.8,fill=gray!20,draw=none](-3.327,2.384)--(-3.349,2.371)--(-3.362,2.34)--(-3.355,2.334)--cycle;
\draw(-3.327,2.384)--(-3.349,2.371)--(-3.362,2.34)--(-3.355,2.334);
\filldraw[fill opacity=0.8,fill=gray!20,draw=none](-7.662,1.639)--(-7.637,1.635)--(-7.637,1.61)--(-7.692,1.607)--(-7.692,1.617)--cycle;
\draw(-7.637,1.635)--(-7.637,1.61)--(-7.692,1.607)--(-7.692,1.617);
\filldraw[fill opacity=0.8,fill=gray!20,draw=none](-7.625,1.633)--(-7.58,1.624)--(-7.58,1.609)--(-7.637,1.61)--(-7.637,1.619)--cycle;
\draw(-7.58,1.624)--(-7.58,1.609)--(-7.637,1.61)--(-7.637,1.619);
\filldraw[fill opacity=0.8,fill=gray!20,draw=none](-7.662,1.639)--(-7.637,1.657)--(-7.637,1.635)--cycle;
\draw(-7.637,1.657)--(-7.637,1.635);
\filldraw[fill opacity=0.8,fill=gray!20,draw=none](-7.637,1.657)--(-7.662,1.639)--(-7.692,1.643)--cycle;
\filldraw[fill opacity=0.8,fill=gray!20,draw=none](-7.662,1.639)--(-7.692,1.617)--(-7.692,1.643)--cycle;
\draw(-7.692,1.617)--(-7.692,1.643);
\filldraw[fill opacity=0.8,fill=gray!20,draw=none](-7.561,1.56)--(-7.561,1.55)--(-7.521,1.557)--(-7.516,1.566)--(-7.516,1.604)--cycle;
\draw(-7.561,1.56)--(-7.561,1.55)--(-7.521,1.557);
\draw(-7.516,1.566)--(-7.516,1.604);
\filldraw[fill opacity=0.8,fill=gray!20,draw=none](-7.561,1.586)--(-7.537,1.584)--(-7.516,1.604)--cycle;
\filldraw[fill opacity=0.8,fill=gray!20,draw=none](-7.561,1.586)--(-7.561,1.56)--(-7.537,1.584)--cycle;
\draw(-7.561,1.586)--(-7.561,1.56);
\filldraw[fill opacity=0.8,fill=gray!20](-7.545,1.579)--(-7.709,1.65)--(-7.658,1.666)--(-7.495,1.595)--cycle;
\filldraw[fill opacity=0.8,fill=gray!20,draw=none](-2.774,2.052)--(-2.769,2.021)--(-2.768,2.021)--(-2.772,2.046)--cycle;
\draw(-2.768,2.021)--(-2.772,2.046);
\filldraw[fill opacity=0.8,fill=gray!20](-3.14,.738)--(-3.13,.785)--(-3.238,.78)--(-3.236,.733)--cycle;
\filldraw[fill opacity=0.8,fill=gray!20,draw=none](-2.723,2.096)--(-2.721,2.081)--(-2.721,2.076)--cycle;
\draw(-2.723,2.096)--(-2.721,2.081);
\filldraw[fill opacity=0.8,fill=gray!20,draw=none](-2.725,2.096)--(-2.722,2.063)--(-2.717,2.054)--(-2.721,2.081)--cycle;
\draw(-2.717,2.054)--(-2.721,2.081);
\filldraw[fill opacity=0.8,fill=gray!20,draw=none](-3.77,1.265)--(-3.691,1.097)--(-3.69,1.022)--(-3.727,1.102)--cycle;
\draw(-3.77,1.265)--(-3.691,1.097);
\draw(-3.69,1.022)--(-3.727,1.102);
\filldraw[fill opacity=0.8,fill=gray!20,draw=none](-2.769,2.021)--(-2.774,2.052)--(-2.794,2.09)--(-2.785,2.024)--cycle;
\draw(-2.794,2.09)--(-2.785,2.024);
\filldraw[fill opacity=0.8,fill=gray!20,draw=none](-7.54,1.867)--(-7.534,1.848)--(-7.537,1.843)--(-7.58,1.888)--cycle;
\filldraw[fill opacity=0.8,fill=gray!20,draw=none](-7.551,1.903)--(-7.54,1.867)--(-7.58,1.888)--(-7.58,1.91)--cycle;
\draw(-7.58,1.888)--(-7.58,1.91);
\filldraw[fill opacity=0.8,fill=gray!20,draw=none](-7.534,1.848)--(-7.534,1.837)--(-7.537,1.843)--cycle;
\filldraw[fill opacity=0.8,fill=gray!20,draw=none](-7.531,1.853)--(-7.534,1.842)--(-7.534,1.848)--cycle;
\filldraw[fill opacity=0.8,fill=gray!20,draw=none](-7.534,1.848)--(-7.531,1.836)--(-7.537,1.843)--cycle;
\filldraw[fill opacity=0.8,fill=gray!20,draw=none](-7.536,1.865)--(-7.534,1.848)--(-7.537,1.843)--(-7.555,1.869)--(-7.548,1.872)--cycle;
\draw(-7.555,1.869)--(-7.548,1.872);
\filldraw[fill opacity=0.8,fill=gray!20,draw=none](-7.534,1.848)--(-7.531,1.853)--(-7.531,1.836)--cycle;
\draw(-7.531,1.853)--(-7.531,1.836);
\filldraw[fill opacity=0.8,fill=gray!20,draw=none](-7.524,1.88)--(-7.531,1.853)--(-7.531,1.869)--cycle;
\draw(-7.531,1.853)--(-7.531,1.869);
\filldraw[fill opacity=0.8,fill=gray!20,draw=none](-7.529,1.861)--(-7.531,1.853)--(-7.534,1.848)--(-7.536,1.865)--cycle;
\filldraw[fill opacity=0.8,fill=gray!20,draw=none](-7.54,1.867)--(-7.551,1.903)--(-7.531,1.898)--(-7.531,1.862)--cycle;
\draw(-7.531,1.898)--(-7.531,1.862);
\filldraw[fill opacity=0.8,fill=gray!20,draw=none](-7.527,1.86)--(-7.529,1.861)--(-7.524,1.88)--(-7.512,1.884)--cycle;
\draw(-7.524,1.88)--(-7.512,1.884);
\filldraw[fill opacity=0.8,fill=gray!20,draw=none](-7.516,1.909)--(-7.524,1.88)--(-7.531,1.869)--(-7.531,1.898)--cycle;
\draw(-7.531,1.869)--(-7.531,1.898);
\filldraw[fill opacity=0.8,fill=gray!20,draw=none](-7.529,1.861)--(-7.536,1.865)--(-7.537,1.876)--(-7.524,1.88)--cycle;
\draw(-7.537,1.876)--(-7.524,1.88);
\filldraw[fill opacity=0.8,fill=gray!20,draw=none](-7.516,1.893)--(-7.504,1.891)--(-7.512,1.884)--(-7.524,1.88)--cycle;
\draw(-7.512,1.884)--(-7.524,1.88);
\filldraw[fill opacity=0.8,fill=gray!20,draw=none](-7.505,1.887)--(-7.512,1.884)--(-7.504,1.891)--cycle;
\draw(-7.505,1.887)--(-7.512,1.884);
\filldraw[fill opacity=0.8,fill=gray!20,draw=none](-7.551,1.903)--(-7.57,1.965)--(-7.531,1.96)--(-7.531,1.898)--cycle;
\draw(-7.57,1.965)--(-7.531,1.96)--(-7.531,1.898);
\filldraw[fill opacity=0.8,fill=gray!20,draw=none](-7.505,1.953)--(-7.516,1.909)--(-7.531,1.898)--(-7.531,1.96)--cycle;
\draw(-7.531,1.898)--(-7.531,1.96)--(-7.505,1.953);
\filldraw[fill opacity=0.8,fill=gray!20,draw=none](-7.524,1.88)--(-7.516,1.909)--(-7.496,1.925)--(-7.496,1.924)--cycle;
\draw(-7.496,1.925)--(-7.496,1.924);
\filldraw[fill opacity=0.8,fill=gray!20,draw=none](-7.516,1.909)--(-7.505,1.953)--(-7.496,1.95)--(-7.496,1.925)--cycle;
\draw(-7.505,1.953)--(-7.496,1.95)--(-7.496,1.925);
\filldraw[fill opacity=0.8,fill=gray!20,draw=none](-7.504,1.891)--(-7.533,1.895)--(-7.53,1.914)--(-7.495,1.926)--cycle;
\draw(-7.53,1.914)--(-7.495,1.926);
\filldraw[fill opacity=0.8,fill=gray!20,draw=none](-7.504,1.891)--(-7.495,1.926)--(-7.433,1.948)--cycle;
\draw(-7.495,1.926)--(-7.433,1.948);
\filldraw[fill opacity=0.8,fill=gray!20,draw=none](-7.529,1.861)--(-7.524,1.88)--(-7.496,1.924)--(-7.496,1.844)--cycle;
\draw(-7.496,1.924)--(-7.496,1.844);
\filldraw[fill opacity=0.8,fill=gray!20,draw=none](-7.536,1.865)--(-7.548,1.872)--(-7.537,1.876)--cycle;
\draw(-7.548,1.872)--(-7.537,1.876);
\filldraw[fill opacity=0.8,fill=gray!20,draw=none](-7.548,1.872)--(-7.555,1.869)--(-7.559,1.879)--cycle;
\draw(-7.548,1.872)--(-7.555,1.869);
\filldraw[fill opacity=0.8,fill=gray!20,draw=none](-7.533,1.895)--(-7.537,1.876)--(-7.548,1.872)--(-7.559,1.879)--(-7.568,1.9)--cycle;
\draw(-7.537,1.876)--(-7.548,1.872);
\filldraw[fill opacity=0.8,fill=gray!20,draw=none](-7.482,1.841)--(-7.488,1.848)--(-7.481,1.814)--cycle;
\filldraw[fill opacity=0.8,fill=gray!20,draw=none](-7.482,1.841)--(-7.481,1.814)--(-7.481,1.84)--cycle;
\draw(-7.481,1.814)--(-7.481,1.84);
\filldraw[fill opacity=0.8,fill=gray!20,draw=none](-7.57,1.965)--(-7.551,1.903)--(-7.58,1.91)--(-7.58,1.966)--cycle;
\draw(-7.58,1.91)--(-7.58,1.966)--(-7.57,1.965);
\filldraw[fill opacity=0.8,fill=gray!20,draw=none](-7.486,1.942)--(-7.433,1.948)--(-7.495,1.926)--cycle;
\draw(-7.433,1.948)--(-7.495,1.926);
\filldraw[fill opacity=0.8,fill=gray!20,draw=none](-7.495,1.926)--(-7.486,1.942)--(-7.481,1.938)--(-7.481,1.876)--(-7.496,1.886)--(-7.496,1.924)--cycle;
\draw(-7.486,1.942)--(-7.481,1.938)--(-7.481,1.876);
\draw(-7.496,1.886)--(-7.496,1.924);
\filldraw[fill opacity=0.8,fill=gray!20,draw=none](-7.486,1.928)--(-7.482,1.841)--(-7.481,1.84)--(-7.481,1.938)--cycle;
\draw(-7.481,1.84)--(-7.481,1.938)--(-7.486,1.928);
\filldraw[fill opacity=0.8,fill=gray!20,draw=none](-7.482,1.841)--(-7.486,1.928)--(-7.488,1.926)--(-7.488,1.848)--cycle;
\draw(-7.486,1.928)--(-7.488,1.926)--(-7.488,1.848);
\filldraw[fill opacity=0.8,fill=gray!20](-7.424,1.789)--(-7.588,1.861)--(-7.616,1.904)--(-7.452,1.833)--cycle;
\filldraw[fill opacity=0.8,fill=gray!20,draw=none](-2.779,2.078)--(-2.774,2.052)--(-2.772,2.046)--(-2.773,2.056)--cycle;
\draw(-2.772,2.046)--(-2.773,2.056);
\filldraw[fill opacity=0.8,fill=gray!20,draw=none](-2.747,2.223)--(-2.754,2.249)--(-2.74,2.23)--(-2.735,2.195)--cycle;
\draw(-2.74,2.23)--(-2.735,2.195);
\filldraw[fill opacity=0.8,fill=gray!20,draw=none](-2.727,2.124)--(-2.725,2.096)--(-2.721,2.081)--(-2.723,2.096)--cycle;
\draw(-2.721,2.081)--(-2.723,2.096);
\filldraw[fill opacity=0.8,fill=gray!20,draw=none](-2.922,1.147)--(-2.915,1.091)--(-2.813,1.111)--(-2.872,1.573)--cycle;
\draw(-2.922,1.147)--(-2.915,1.091)--(-2.813,1.111)--(-2.872,1.573);
\filldraw[fill opacity=0.8,fill=gray!20,draw=none](-2.797,2.225)--(-2.812,2.262)--(-2.804,2.251)--cycle;
\filldraw[fill opacity=0.8,fill=gray!20,draw=none](-3.353,.81)--(-3.344,.817)--(-3.345,.831)--cycle;
\draw(-3.344,.817)--(-3.345,.831);
\filldraw[fill opacity=0.8,fill=gray!20,draw=none](-3.345,.831)--(-3.363,.812)--(-3.353,.81)--cycle;
\draw(-3.363,.812)--(-3.353,.81);
\filldraw[fill opacity=0.8,fill=gray!20,draw=none](-3.312,1.002)--(-3.311,1.005)--(-3.313,1.006)--cycle;
\draw(-3.312,1.002)--(-3.311,1.005)--(-3.313,1.006);
\filldraw[fill opacity=0.8,fill=gray!20,draw=none](-3.606,2.338)--(-3.566,2.332)--(-3.626,2.358)--(-3.676,2.364)--(-3.641,2.349)--cycle;
\draw(-3.566,2.332)--(-3.626,2.358)--(-3.676,2.364)--(-3.641,2.349);
\filldraw[fill opacity=0.8,fill=gray!20,draw=none](-4.257,2.652)--(-4.173,2.582)--(-4.127,2.494)--(-4.191,2.547)--cycle;
\draw(-4.257,2.652)--(-4.173,2.582);
\draw(-4.127,2.494)--(-4.191,2.547);
\filldraw[fill opacity=0.8,fill=gray!20,draw=none](-4.101,2.449)--(-4.169,2.518)--(-4.191,2.547)--(-4.127,2.494)--cycle;
\draw(-4.191,2.547)--(-4.127,2.494);
\filldraw[fill opacity=0.8,fill=gray!20,draw=none](-4.315,2.764)--(-4.225,2.688)--(-4.173,2.582)--(-4.257,2.652)--cycle;
\draw(-4.315,2.764)--(-4.225,2.688);
\draw(-4.173,2.582)--(-4.257,2.652);
\filldraw[fill opacity=0.8,fill=gray!20,draw=none](-4.169,2.518)--(-4.101,2.449)--(-4.094,2.437)--(-4.129,2.466)--cycle;
\draw(-4.094,2.437)--(-4.129,2.466);
\filldraw[fill opacity=0.8,fill=gray!20,draw=none](-4.226,2.576)--(-4.169,2.518)--(-4.129,2.466)--cycle;
\filldraw[fill opacity=0.8,fill=gray!20,draw=none](-4.169,2.518)--(-4.226,2.576)--(-4.191,2.547)--cycle;
\draw(-4.226,2.576)--(-4.191,2.547);
\filldraw[fill opacity=0.8,fill=gray!20,draw=none](-4.292,2.838)--(-4.274,2.796)--(-4.303,2.821)--cycle;
\draw(-4.274,2.796)--(-4.303,2.821);
\filldraw[fill opacity=0.8,fill=gray!20,draw=none](-4.311,2.901)--(-4.251,2.913)--(-4.243,2.906)--(-4.232,2.867)--(-4.251,2.863)--cycle;
\draw(-4.311,2.901)--(-4.251,2.913);
\draw(-4.232,2.867)--(-4.251,2.863);
\filldraw[fill opacity=0.8,fill=gray!20,draw=none](-4.276,2.858)--(-4.232,2.867)--(-4.231,2.853)--cycle;
\draw(-4.276,2.858)--(-4.232,2.867);
\filldraw[fill opacity=0.8,fill=gray!20,draw=none](-3.541,2.278)--(-3.516,2.257)--(-3.525,2.293)--cycle;
\draw(-3.541,2.278)--(-3.516,2.257);
\filldraw[fill opacity=0.8,fill=gray!20,draw=none](-3.572,2.303)--(-3.564,2.315)--(-3.624,2.341)--(-3.704,2.338)--(-3.594,2.29)--cycle;
\draw(-3.564,2.315)--(-3.624,2.341);
\draw(-3.704,2.338)--(-3.594,2.29);
\filldraw[fill opacity=0.8,fill=gray!20,draw=none](-4.292,2.838)--(-4.181,2.779)--(-3.725,2.398)--(-3.682,2.335)--(-3.715,2.33)--(-4.274,2.796)--cycle;
\draw(-4.181,2.779)--(-3.725,2.398);
\draw(-3.715,2.33)--(-4.274,2.796);
\filldraw[fill opacity=0.8,fill=gray!20,draw=none](-4.276,2.858)--(-4.231,2.853)--(-4.23,2.845)--cycle;
\filldraw[fill opacity=0.8,fill=gray!20,draw=none](-4.342,2.924)--(-4.342,2.932)--(-4.338,2.932)--(-4.331,2.919)--cycle;
\draw(-4.342,2.932)--(-4.338,2.932);
\filldraw[fill opacity=0.8,fill=gray!20,draw=none](-4.338,2.932)--(-4.317,2.93)--(-4.319,2.897)--cycle;
\draw(-4.338,2.932)--(-4.317,2.93)--(-4.319,2.897);
\filldraw[fill opacity=0.8,fill=gray!20,draw=none](-4.321,2.871)--(-4.303,2.866)--(-4.302,2.861)--(-4.328,2.864)--cycle;
\filldraw[fill opacity=0.8,fill=gray!20,draw=none](-4.322,2.871)--(-4.322,2.879)--(-4.307,2.872)--(-4.303,2.866)--cycle;
\filldraw[fill opacity=0.8,fill=gray!20,draw=none](-4.293,2.842)--(-4.303,2.866)--(-4.276,2.858)--(-4.257,2.843)--cycle;
\draw(-4.276,2.858)--(-4.257,2.843);
\filldraw[fill opacity=0.8,fill=gray!20,draw=none](-4.321,2.871)--(-4.328,2.864)--(-4.335,2.865)--cycle;
\filldraw[fill opacity=0.8,fill=gray!20,draw=none](-4.335,2.865)--(-4.339,2.864)--(-4.344,2.866)--cycle;
\filldraw[fill opacity=0.8,fill=gray!20,draw=none](-4.328,2.86)--(-4.335,2.865)--(-4.322,2.871)--(-4.322,2.867)--cycle;
\draw(-4.322,2.871)--(-4.322,2.867);
\filldraw[fill opacity=0.8,fill=gray!20,draw=none](-4.322,2.871)--(-4.303,2.866)--(-4.297,2.854)--(-4.32,2.849)--cycle;
\draw(-4.297,2.854)--(-4.32,2.849);
\filldraw[fill opacity=0.8,fill=gray!20,draw=none](-4.292,2.838)--(-4.293,2.842)--(-4.257,2.843)--(-4.181,2.779)--cycle;
\draw(-4.257,2.843)--(-4.181,2.779);
\filldraw[fill opacity=0.8,fill=gray!20,draw=none](-4.279,2.826)--(-4.312,2.834)--(-4.32,2.849)--(-4.297,2.854)--cycle;
\draw(-4.32,2.849)--(-4.297,2.854);
\filldraw[fill opacity=0.8,fill=gray!20,draw=none](-4.322,2.873)--(-4.329,2.882)--(-4.32,2.882)--(-4.321,2.873)--cycle;
\draw(-4.32,2.882)--(-4.321,2.873)--(-4.322,2.873);
\filldraw[fill opacity=0.8,fill=gray!20,draw=none](-4.322,2.873)--(-4.321,2.873)--(-4.322,2.871)--cycle;
\draw(-4.322,2.873)--(-4.321,2.873)--(-4.322,2.871);
\filldraw[fill opacity=0.8,fill=gray!20,draw=none](-3.606,2.338)--(-3.641,2.349)--(-3.624,2.341)--cycle;
\draw(-3.641,2.349)--(-3.624,2.341);
\filldraw[fill opacity=0.8,fill=gray!20,draw=none](-4.194,2.835)--(-3.614,2.35)--(-3.611,2.338)--(-3.667,2.349)--(-4.257,2.843)--cycle;
\draw(-4.194,2.835)--(-3.614,2.35);
\draw(-3.667,2.349)--(-4.257,2.843);
\filldraw[fill opacity=0.8,fill=gray!20,draw=none](-4.194,2.835)--(-4.257,2.843)--(-4.276,2.858)--cycle;
\draw(-4.257,2.843)--(-4.276,2.858);
\filldraw[fill opacity=0.8,fill=gray!20,draw=none](-4.279,2.826)--(-4.297,2.854)--(-4.276,2.858)--(-4.23,2.845)--(-4.229,2.828)--(-4.259,2.822)--cycle;
\draw(-4.297,2.854)--(-4.276,2.858);
\draw(-4.229,2.828)--(-4.259,2.822);
\filldraw[fill opacity=0.8,fill=gray!20,draw=none](-4.307,2.872)--(-4.276,2.858)--(-4.297,2.854)--cycle;
\draw(-4.276,2.858)--(-4.297,2.854);
\filldraw[fill opacity=0.8,fill=gray!20,draw=none](-4.322,2.87)--(-4.321,2.873)--(-4.319,2.873)--cycle;
\draw(-4.322,2.87)--(-4.321,2.873)--(-4.319,2.873);
\filldraw[fill opacity=0.8,fill=gray!20,draw=none](-4.312,2.885)--(-4.308,2.877)--(-4.32,2.87)--(-4.321,2.87)--cycle;
\filldraw[fill opacity=0.8,fill=gray!20,draw=none](-4.312,2.885)--(-4.319,2.873)--(-4.321,2.873)--(-4.319,2.897)--cycle;
\draw(-4.319,2.873)--(-4.321,2.873)--(-4.319,2.897);
\filldraw[fill opacity=0.8,fill=gray!20,draw=none](-4.324,2.899)--(-4.314,2.89)--(-4.312,2.885)--(-4.321,2.87)--(-4.346,2.877)--cycle;
\draw(-4.324,2.899)--(-4.314,2.89);
\filldraw[fill opacity=0.8,fill=gray!20,draw=none](-4.328,2.86)--(-4.322,2.867)--(-4.324,2.858)--cycle;
\draw(-4.322,2.867)--(-4.324,2.858);
\filldraw[fill opacity=0.8,fill=gray!20,draw=none](-4.303,2.866)--(-4.312,2.885)--(-4.31,2.887)--(-4.276,2.858)--cycle;
\draw(-4.31,2.887)--(-4.276,2.858);
\filldraw[fill opacity=0.8,fill=gray!20,draw=none](-4.308,2.877)--(-4.303,2.866)--(-4.32,2.87)--cycle;
\filldraw[fill opacity=0.8,fill=gray!20,draw=none](-4.309,2.828)--(-4.329,2.835)--(-4.324,2.858)--cycle;
\draw(-4.329,2.835)--(-4.324,2.858);
\filldraw[fill opacity=0.8,fill=gray!20,draw=none](-4.309,2.828)--(-4.312,2.834)--(-4.279,2.826)--(-4.274,2.819)--(-4.293,2.815)--cycle;
\draw(-4.274,2.819)--(-4.293,2.815);
\filldraw[fill opacity=0.8,fill=gray!20,draw=none](-4.277,2.832)--(-4.297,2.824)--(-4.309,2.828)--(-4.324,2.858)--(-4.322,2.87)--(-4.319,2.873)--(-4.301,2.868)--cycle;
\draw(-4.324,2.858)--(-4.322,2.87);
\draw(-4.319,2.873)--(-4.301,2.868);
\filldraw[fill opacity=0.8,fill=gray!20,draw=none](-4.361,2.881)--(-4.322,2.871)--(-4.32,2.849)--(-4.332,2.847)--cycle;
\draw(-4.32,2.849)--(-4.332,2.847);
\filldraw[fill opacity=0.8,fill=gray!20,draw=none](-4.309,2.828)--(-4.305,2.821)--(-4.32,2.814)--(-4.333,2.818)--(-4.329,2.835)--cycle;
\draw(-4.32,2.814)--(-4.333,2.818)--(-4.329,2.835);
\filldraw[fill opacity=0.8,fill=gray!20,draw=none](-4.336,2.811)--(-4.333,2.818)--(-4.328,2.816)--cycle;
\draw(-4.336,2.811)--(-4.333,2.818)--(-4.328,2.816);
\filldraw[fill opacity=0.8,fill=gray!20,draw=none](-4.336,2.811)--(-4.328,2.816)--(-4.32,2.814)--cycle;
\draw(-4.328,2.816)--(-4.32,2.814);
\filldraw[fill opacity=0.8,fill=gray!20,draw=none](-4.312,2.834)--(-4.318,2.835)--(-4.332,2.847)--(-4.32,2.849)--cycle;
\draw(-4.332,2.847)--(-4.32,2.849);
\filldraw[fill opacity=0.8,fill=gray!20](-3.273,3.074)--(-3.283,3.13)--(-3.175,3.134)--(-3.177,3.078)--cycle;
\filldraw[fill opacity=0.8,fill=gray!20](-3.258,3.022)--(-3.273,3.074)--(-3.177,3.078)--(-3.179,3.026)--cycle;
\filldraw[fill opacity=0.8,fill=gray!20](-3.347,3.059)--(-3.365,3.114)--(-3.283,3.13)--(-3.273,3.074)--cycle;
\filldraw[fill opacity=0.8,fill=gray!20](-3.318,3.011)--(-3.347,3.059)--(-3.273,3.074)--(-3.258,3.022)--cycle;
\filldraw[fill opacity=0.8,fill=gray!20,draw=none](-4.278,2.905)--(-4.256,2.886)--(-4.23,2.845)--(-4.276,2.858)--(-4.31,2.887)--cycle;
\draw(-4.278,2.905)--(-4.256,2.886);
\draw(-4.276,2.858)--(-4.31,2.887);
\filldraw[fill opacity=0.8,fill=gray!20](-3.182,2.982)--(-3.179,3.026)--(-3.103,3.02)--(-3.128,2.978)--cycle;
\filldraw[fill opacity=0.8,fill=gray!20](-3.237,2.979)--(-3.258,3.022)--(-3.179,3.026)--(-3.182,2.982)--cycle;
\filldraw[fill opacity=0.8,fill=gray!20](-3.28,2.971)--(-3.318,3.011)--(-3.258,3.022)--(-3.237,2.979)--cycle;
\filldraw[fill opacity=0.8,fill=gray!20,draw=none](-4.243,2.906)--(-4.205,2.873)--(-4.232,2.867)--cycle;
\draw(-4.205,2.873)--(-4.232,2.867);
\filldraw[fill opacity=0.8,fill=gray!20,draw=none](-4.232,2.867)--(-4.205,2.873)--(-4.169,2.84)--(-4.229,2.828)--cycle;
\draw(-4.232,2.867)--(-4.205,2.873);
\draw(-4.169,2.84)--(-4.229,2.828);
\filldraw[fill opacity=0.8,fill=gray!20,draw=none](-4.128,2.801)--(-4.055,2.74)--(-4.161,2.807)--(-4.194,2.835)--cycle;
\draw(-4.128,2.801)--(-4.055,2.74);
\draw(-4.161,2.807)--(-4.194,2.835);
\filldraw[fill opacity=0.8,fill=gray!20,draw=none](-4.194,2.835)--(-4.169,2.84)--(-4.15,2.819)--cycle;
\draw(-4.194,2.835)--(-4.169,2.84);
\filldraw[fill opacity=0.8,fill=gray!20,draw=none](-4.269,2.903)--(-4.318,2.913)--(-4.317,2.93)--(-4.273,2.919)--cycle;
\draw(-4.318,2.913)--(-4.317,2.93)--(-4.273,2.919);
\filldraw[fill opacity=0.8,fill=gray!20,draw=none](-4.266,2.9)--(-4.256,2.886)--(-4.274,2.901)--cycle;
\draw(-4.256,2.886)--(-4.274,2.901);
\filldraw[fill opacity=0.8,fill=gray!20,draw=none](-4.277,2.929)--(-4.273,2.919)--(-4.317,2.93)--(-4.32,2.97)--cycle;
\draw(-4.273,2.919)--(-4.317,2.93)--(-4.32,2.97);
\filldraw[fill opacity=0.8,fill=gray!20,draw=none](-4.269,2.903)--(-4.266,2.9)--(-4.274,2.901)--(-4.278,2.905)--cycle;
\draw(-4.274,2.901)--(-4.278,2.905);
\filldraw[fill opacity=0.8,fill=gray!20,draw=none](-4.269,2.903)--(-4.265,2.891)--(-4.282,2.864)--(-4.301,2.868)--(-4.319,2.897)--(-4.318,2.913)--cycle;
\draw(-4.282,2.864)--(-4.301,2.868);
\draw(-4.319,2.897)--(-4.318,2.913);
\filldraw[fill opacity=0.8,fill=gray!20,draw=none](-4.308,2.93)--(-4.278,2.905)--(-4.31,2.887)--(-4.349,2.92)--cycle;
\draw(-4.308,2.93)--(-4.278,2.905);
\draw(-4.31,2.887)--(-4.349,2.92);
\filldraw[fill opacity=0.8,fill=gray!20,draw=none](-4.318,2.835)--(-4.351,2.843)--(-4.332,2.847)--cycle;
\draw(-4.351,2.843)--(-4.332,2.847);
\filldraw[fill opacity=0.8,fill=gray!20,draw=none](-4.309,2.828)--(-4.318,2.835)--(-4.312,2.834)--cycle;
\filldraw[fill opacity=0.8,fill=gray!20,draw=none](-4.305,2.821)--(-4.351,2.843)--(-4.318,2.835)--(-4.309,2.828)--cycle;
\filldraw[fill opacity=0.8,fill=gray!20,draw=none](-4.305,2.821)--(-4.309,2.828)--(-4.293,2.815)--cycle;
\filldraw[fill opacity=0.8,fill=gray!20,draw=none](-4.309,2.828)--(-4.297,2.824)--(-4.305,2.821)--cycle;
\filldraw[fill opacity=0.8,fill=gray!20,draw=none](-4.344,2.986)--(-4.34,2.97)--(-4.347,2.962)--cycle;
\filldraw[fill opacity=0.8,fill=gray!20,draw=none](-4.338,2.932)--(-4.348,2.951)--(-4.344,2.986)--(-4.337,2.985)--(-4.32,2.97)--(-4.317,2.93)--cycle;
\draw(-4.344,2.986)--(-4.337,2.985);
\draw(-4.32,2.97)--(-4.317,2.93)--(-4.338,2.932);
\filldraw[fill opacity=0.8,fill=gray!20,draw=none](-4.343,3.001)--(-4.346,2.993)--(-4.358,2.991)--cycle;
\draw(-4.346,2.993)--(-4.358,2.991);
\filldraw[fill opacity=0.8,fill=gray!20,draw=none](-4.34,3.01)--(-4.327,3.011)--(-4.358,2.991)--(-4.387,2.984)--cycle;
\draw(-4.358,2.991)--(-4.387,2.984);
\filldraw[fill opacity=0.8,fill=gray!20,draw=none](-4.396,2.972)--(-4.387,2.984)--(-4.357,3.001)--(-4.33,2.977)--(-4.351,2.947)--cycle;
\draw(-4.396,2.972)--(-4.387,2.984);
\draw(-4.33,2.977)--(-4.351,2.947);
\filldraw[fill opacity=0.8,fill=gray!20,draw=none](-4.308,2.93)--(-4.349,2.92)--(-4.363,2.931)--cycle;
\draw(-4.349,2.92)--(-4.363,2.931);
\filldraw[fill opacity=0.8,fill=gray!20,draw=none](-4.324,2.923)--(-4.355,2.914)--(-4.368,2.923)--(-4.345,2.956)--cycle;
\draw(-4.368,2.923)--(-4.345,2.956);
\filldraw[fill opacity=0.8,fill=gray!20,draw=none](-4.324,2.923)--(-4.345,2.956)--(-4.33,2.977)--(-4.296,2.946)--(-4.309,2.928)--cycle;
\draw(-4.345,2.956)--(-4.33,2.977);
\draw(-4.296,2.946)--(-4.309,2.928);
\filldraw[fill opacity=0.8,fill=gray!20,draw=none](-4.389,2.997)--(-4.308,2.93)--(-4.363,2.931)--(-4.425,2.983)--cycle;
\draw(-4.363,2.931)--(-4.425,2.983)--(-4.389,2.997)--(-4.308,2.93);
\filldraw[fill opacity=0.8,fill=gray!20,draw=none](-4.355,2.914)--(-4.381,2.906)--(-4.368,2.923)--cycle;
\draw(-4.381,2.906)--(-4.368,2.923);
\filldraw[fill opacity=0.8,fill=gray!20,draw=none](-4.381,2.906)--(-4.355,2.914)--(-4.331,2.896)--cycle;
\filldraw[fill opacity=0.8,fill=gray!20,draw=none](-4.355,2.914)--(-4.324,2.923)--(-4.319,2.914)--(-4.331,2.896)--cycle;
\draw(-4.319,2.914)--(-4.331,2.896);
\filldraw[fill opacity=0.8,fill=gray!20,draw=none](-4.443,2.969)--(-4.425,2.983)--(-4.393,2.956)--(-4.429,2.943)--cycle;
\draw(-4.443,2.969)--(-4.425,2.983)--(-4.393,2.956);
\filldraw[fill opacity=0.8,fill=gray!20,draw=none](-4.363,2.963)--(-3.52,1.171)--(-3.645,1.34)--(-4.405,2.956)--cycle;
\draw(-3.645,1.34)--(-4.405,2.956)--(-4.363,2.963)--(-3.52,1.171);
\filldraw[fill opacity=0.8,fill=gray!20,draw=none](-3.765,2.304)--(-3.653,2.211)--(-3.678,2.168)--(-3.77,2.245)--cycle;
\draw(-3.765,2.304)--(-3.653,2.211)--(-3.678,2.168)--(-3.77,2.245);
\filldraw[fill opacity=0.8,fill=gray!20,draw=none](-3.569,2.121)--(-3.607,2.092)--(-3.643,2.078)--(-3.671,2.079)--(-3.688,2.097)--(-3.69,2.128)--(-3.678,2.168)--(-3.653,2.211)--(-3.535,2.159)--cycle;
\draw(-3.535,2.159)--(-3.569,2.121)--(-3.607,2.092)--(-3.643,2.078)--(-3.671,2.079)--(-3.688,2.097)--(-3.69,2.128)--(-3.678,2.168)--(-3.653,2.211);
\filldraw[fill opacity=0.8,fill=gray!20,draw=none](-3.633,2.241)--(-3.629,2.251)--(-3.746,2.302)--(-3.77,2.25)--(-3.66,2.203)--cycle;
\draw(-3.629,2.251)--(-3.746,2.302);
\draw(-3.77,2.25)--(-3.66,2.203);
\filldraw[fill opacity=0.8,fill=gray!20,draw=none](-4.274,2.796)--(-3.721,2.334)--(-3.745,2.301)--(-3.754,2.295)--(-4.225,2.688)--cycle;
\draw(-4.274,2.796)--(-3.721,2.334);
\draw(-3.754,2.295)--(-4.225,2.688);
\filldraw[fill opacity=0.8,fill=gray!20,draw=none](-3.766,2.192)--(-3.771,2.251)--(-3.788,2.258)--cycle;
\draw(-3.771,2.251)--(-3.788,2.258);
\filldraw[fill opacity=0.8,fill=gray!20,draw=none](-4.225,2.688)--(-3.765,2.304)--(-3.77,2.245)--(-4.173,2.582)--cycle;
\draw(-4.225,2.688)--(-3.765,2.304);
\draw(-3.77,2.245)--(-4.173,2.582);
\filldraw[fill opacity=0.8,fill=gray!20,draw=none](-4.173,2.582)--(-4.034,2.466)--(-4.011,2.397)--(-4.127,2.494)--cycle;
\draw(-4.173,2.582)--(-4.034,2.466);
\draw(-4.011,2.397)--(-4.127,2.494);
\filldraw[fill opacity=0.8,fill=gray!20,draw=none](-4.101,2.449)--(-4.127,2.494)--(-4.011,2.397)--(-4.013,2.369)--(-4.068,2.415)--cycle;
\draw(-4.127,2.494)--(-4.011,2.397);
\draw(-4.013,2.369)--(-4.068,2.415);
\filldraw[fill opacity=0.8,fill=gray!20,draw=none](-4.087,2.429)--(-4.094,2.437)--(-4.068,2.415)--(-4.02,2.374)--(-4.038,2.386)--cycle;
\draw(-4.094,2.437)--(-4.068,2.415);
\filldraw[fill opacity=0.8,fill=gray!20,draw=none](-4.101,2.449)--(-4.068,2.415)--(-4.094,2.437)--cycle;
\draw(-4.068,2.415)--(-4.094,2.437);
\filldraw[fill opacity=0.8,fill=gray!20,draw=none](-4.087,2.429)--(-4.129,2.466)--(-4.094,2.437)--cycle;
\draw(-4.129,2.466)--(-4.094,2.437);
\filldraw[fill opacity=0.8,fill=gray!20,draw=none](-4.087,2.429)--(-4.038,2.386)--(-4.079,2.42)--cycle;
\draw(-4.038,2.386)--(-4.079,2.42);
\filldraw[fill opacity=0.8,fill=gray!20,draw=none](-4.129,2.466)--(-4.087,2.429)--(-4.079,2.42)--cycle;
\filldraw[fill opacity=0.8,fill=gray!20,draw=none](-4.312,2.885)--(-4.301,2.868)--(-4.319,2.873)--cycle;
\draw(-4.301,2.868)--(-4.319,2.873);
\filldraw[fill opacity=0.8,fill=gray!20,draw=none](-4.312,2.885)--(-4.314,2.89)--(-4.31,2.887)--cycle;
\draw(-4.314,2.89)--(-4.31,2.887);
\filldraw[fill opacity=0.8,fill=gray!20,draw=none](-4.266,2.809)--(-4.274,2.819)--(-4.259,2.822)--(-4.23,2.82)--(-4.234,2.802)--cycle;
\draw(-4.274,2.819)--(-4.259,2.822);
\filldraw[fill opacity=0.8,fill=gray!20,draw=none](-4.279,2.826)--(-4.259,2.822)--(-4.274,2.819)--cycle;
\draw(-4.259,2.822)--(-4.274,2.819);
\filldraw[fill opacity=0.8,fill=gray!20,draw=none](-4.277,2.832)--(-4.301,2.868)--(-4.256,2.857)--(-4.253,2.842)--cycle;
\draw(-4.301,2.868)--(-4.256,2.857);
\filldraw[fill opacity=0.8,fill=gray!20,draw=none](-4.266,2.809)--(-4.293,2.815)--(-4.274,2.819)--cycle;
\draw(-4.293,2.815)--(-4.274,2.819);
\filldraw[fill opacity=0.8,fill=gray!20,draw=none](-4.293,2.815)--(-4.266,2.809)--(-4.257,2.798)--cycle;
\filldraw[fill opacity=0.8,fill=gray!20,draw=none](-4.297,2.824)--(-4.277,2.832)--(-4.264,2.813)--cycle;
\filldraw[fill opacity=0.8,fill=gray!20,draw=none](-4.266,2.809)--(-4.234,2.802)--(-4.257,2.798)--cycle;
\draw(-4.234,2.802)--(-4.257,2.798);
\filldraw[fill opacity=0.8,fill=gray!20,draw=none](-4.277,2.832)--(-4.253,2.842)--(-4.264,2.813)--cycle;
\draw(-4.253,2.842)--(-4.264,2.813);
\filldraw[fill opacity=0.8,fill=gray!20,draw=none](-4.324,2.923)--(-4.309,2.928)--(-4.319,2.914)--cycle;
\draw(-4.309,2.928)--(-4.319,2.914);
\filldraw[fill opacity=0.8,fill=gray!20,draw=none](-4.264,2.963)--(-3.409,3.136)--(-3.392,3.096)--(-3.392,3.088)--(-4.245,2.915)--cycle;
\draw(-4.264,2.963)--(-3.409,3.136);
\draw(-3.392,3.088)--(-4.245,2.915);
\filldraw[fill opacity=0.8,fill=gray!20,draw=none](-4.243,2.906)--(-4.245,2.915)--(-3.4,3.086)--(-3.381,3.046)--(-3.38,3.04)--(-4.205,2.873)--cycle;
\draw(-4.245,2.915)--(-3.4,3.086);
\draw(-3.38,3.04)--(-4.205,2.873);
\filldraw[fill opacity=0.8,fill=gray!20,draw=none](-4.263,2.917)--(-4.273,2.919)--(-4.277,2.929)--cycle;
\draw(-4.263,2.917)--(-4.273,2.919);
\filldraw[fill opacity=0.8,fill=gray!20,draw=none](-4.29,2.94)--(-4.309,2.928)--(-4.296,2.946)--cycle;
\draw(-4.309,2.928)--(-4.296,2.946);
\filldraw[fill opacity=0.8,fill=gray!20,draw=none](-4.281,2.918)--(-4.269,2.903)--(-4.278,2.905)--(-4.308,2.93)--cycle;
\draw(-4.278,2.905)--(-4.308,2.93);
\filldraw[fill opacity=0.8,fill=gray!20,draw=none](-4.307,2.929)--(-4.29,2.94)--(-4.287,2.936)--(-4.285,2.925)--cycle;
\filldraw[fill opacity=0.8,fill=gray!20,draw=none](-4.343,2.98)--(-4.302,2.946)--(-4.285,2.925)--(-4.308,2.93)--(-4.338,2.955)--cycle;
\draw(-4.343,2.98)--(-4.302,2.946);
\draw(-4.308,2.93)--(-4.338,2.955);
\filldraw[fill opacity=0.8,fill=gray!20,draw=none](-4.256,2.886)--(-4.194,2.835)--(-4.23,2.845)--cycle;
\draw(-4.256,2.886)--(-4.194,2.835);
\filldraw[fill opacity=0.8,fill=gray!20,draw=none](-4.269,2.903)--(-4.273,2.919)--(-4.263,2.917)--(-4.243,2.897)--cycle;
\draw(-4.273,2.919)--(-4.263,2.917);
\filldraw[fill opacity=0.8,fill=gray!20,draw=none](-4.273,2.922)--(-4.284,2.92)--(-4.308,2.93)--cycle;
\filldraw[fill opacity=0.8,fill=gray!20,draw=none](-4.309,2.901)--(-4.319,2.914)--(-4.309,2.928)--(-4.307,2.929)--(-4.285,2.925)--(-4.283,2.907)--cycle;
\draw(-4.319,2.914)--(-4.309,2.928);
\filldraw[fill opacity=0.8,fill=gray!20,draw=none](-4.309,2.901)--(-4.331,2.896)--(-4.319,2.914)--cycle;
\draw(-4.331,2.896)--(-4.319,2.914);
\filldraw[fill opacity=0.8,fill=gray!20,draw=none](-4.331,2.896)--(-4.309,2.901)--(-4.298,2.886)--cycle;
\filldraw[fill opacity=0.8,fill=gray!20,draw=none](-4.361,2.996)--(-4.343,2.98)--(-4.338,2.955)--(-4.389,2.997)--cycle;
\draw(-4.338,2.955)--(-4.389,2.997)--(-4.361,2.996)--(-4.343,2.98);
\filldraw[fill opacity=0.8,fill=gray!20,draw=none](-4.309,2.901)--(-4.283,2.907)--(-4.298,2.886)--cycle;
\draw(-4.283,2.907)--(-4.298,2.886);
\filldraw[fill opacity=0.8,fill=gray!20,draw=none](-4.332,2.958)--(-3.461,1.105)--(-3.52,1.171)--(-4.363,2.963)--cycle;
\draw(-3.52,1.171)--(-4.363,2.963)--(-4.332,2.958)--(-3.461,1.105);
\filldraw[fill opacity=0.8,fill=gray!20,draw=none](-3.303,.989)--(-3.303,1.005)--(-3.311,1.005)--(-3.312,1.002)--cycle;
\draw(-3.303,1.005)--(-3.311,1.005)--(-3.312,1.002);
\filldraw[fill opacity=0.8,fill=gray!20,draw=none](-3.238,.78)--(-3.238,.834)--(-3.259,.836)--(-3.344,.817)--(-3.342,.788)--cycle;
\draw(-3.344,.817)--(-3.342,.788)--(-3.238,.78)--(-3.238,.834)--(-3.259,.836);
\filldraw[fill opacity=0.8,fill=gray!20,draw=none](-2.734,2.176)--(-2.747,2.223)--(-2.739,2.204)--cycle;
\filldraw[fill opacity=0.8,fill=gray!20,draw=none](-3.689,1.015)--(-3.69,1.022)--(-3.689,1.02)--cycle;
\draw(-3.69,1.022)--(-3.689,1.02);
\filldraw[fill opacity=0.8,fill=gray!20,draw=none](-3.345,.831)--(-3.343,.835)--(-3.373,.815)--(-3.363,.812)--cycle;
\draw(-3.373,.815)--(-3.363,.812);
\filldraw[fill opacity=0.8,fill=gray!20,draw=none](-3.414,2.368)--(-3.446,2.721)--(-3.404,2.393)--cycle;
\draw(-3.446,2.721)--(-3.404,2.393);
\filldraw[fill opacity=0.8,fill=gray!20,draw=none](-3.303,.989)--(-3.277,.951)--(-3.236,.948)--(-3.234,1)--(-3.303,1.005)--cycle;
\draw(-3.277,.951)--(-3.236,.948)--(-3.234,1)--(-3.303,1.005);
\filldraw[fill opacity=0.8,fill=gray!20,draw=none](-7.717,1.633)--(-7.692,1.643)--(-7.692,1.617)--cycle;
\draw(-7.692,1.643)--(-7.692,1.617);
\filldraw[fill opacity=0.8,fill=gray!20,draw=none](-7.692,1.643)--(-7.717,1.633)--(-7.737,1.646)--cycle;
\filldraw[fill opacity=0.8,fill=gray!20,draw=none](-7.616,1.572)--(-7.616,1.547)--(-7.576,1.549)--(-7.561,1.56)--(-7.561,1.586)--cycle;
\draw(-7.616,1.572)--(-7.616,1.547)--(-7.576,1.549);
\draw(-7.561,1.56)--(-7.561,1.586);
\filldraw[fill opacity=0.8,fill=gray!20,draw=none](-7.616,1.593)--(-7.591,1.578)--(-7.561,1.586)--cycle;
\filldraw[fill opacity=0.8,fill=gray!20](-7.595,1.584)--(-7.759,1.655)--(-7.709,1.65)--(-7.545,1.579)--cycle;
\filldraw[fill opacity=0.8,fill=gray!20,draw=none](-3.727,1.102)--(-3.69,1.022)--(-3.689,1.015)--(-3.686,.983)--(-3.689,.988)--cycle;
\draw(-3.727,1.102)--(-3.69,1.022);
\draw(-3.686,.983)--(-3.689,.988);
\filldraw[fill opacity=0.8,fill=gray!20,draw=none](-2.779,2.078)--(-2.784,2.098)--(-2.781,2.066)--(-2.774,2.052)--cycle;
\filldraw[fill opacity=0.8,fill=gray!20,draw=none](-3.475,1.152)--(-3.443,1.083)--(-3.454,1.091)--(-3.461,1.105)--cycle;
\draw(-3.475,1.152)--(-3.443,1.083);
\draw(-3.454,1.091)--(-3.461,1.105);
\filldraw[fill opacity=0.8,fill=gray!20,draw=none](-2.812,1.786)--(-2.8,1.696)--(-2.748,1.728)--(-2.761,1.833)--cycle;
\draw(-2.812,1.786)--(-2.8,1.696);
\draw(-2.748,1.728)--(-2.761,1.833);
\filldraw[fill opacity=0.8,fill=gray!20,draw=none](-3.689,1.015)--(-3.679,.968)--(-3.686,.983)--cycle;
\draw(-3.679,.968)--(-3.686,.983);
\filldraw[fill opacity=0.8,fill=gray!20,draw=none](-3.422,1.059)--(-3.428,1.052)--(-3.406,1.047)--cycle;
\draw(-3.428,1.052)--(-3.406,1.047);
\filldraw[fill opacity=0.8,fill=gray!20,draw=none](-3.312,1.976)--(-3.323,2.041)--(-3.324,2.041)--(-3.315,1.969)--cycle;
\draw(-3.324,2.041)--(-3.315,1.969);
\filldraw[fill opacity=0.8,fill=gray!20](-3.14,.952)--(-3.156,1.003)--(-3.234,1)--(-3.236,.948)--cycle;
\filldraw[fill opacity=0.8,fill=gray!20,draw=none](-3.259,.836)--(-3.335,.841)--(-3.345,.831)--(-3.344,.817)--cycle;
\draw(-3.259,.836)--(-3.335,.841);
\draw(-3.345,.831)--(-3.344,.817);
\filldraw[fill opacity=0.8,fill=gray!20,draw=none](-3.302,.952)--(-3.277,.951)--(-3.303,.989)--cycle;
\draw(-3.302,.952)--(-3.277,.951);
\filldraw[fill opacity=0.8,fill=gray!20,draw=none](-3.449,.817)--(-3.42,.826)--(-3.513,.849)--(-3.54,.83)--(-3.475,.814)--cycle;
\draw(-3.42,.826)--(-3.513,.849)--(-3.54,.83)--(-3.475,.814);
\filldraw[fill opacity=0.8,fill=gray!20,draw=none](-3.362,1.979)--(-3.373,2.043)--(-3.379,2.043)--(-3.369,1.961)--cycle;
\draw(-3.379,2.043)--(-3.369,1.961);
\filldraw[fill opacity=0.8,fill=gray!20](-3.562,2.012)--(-3.726,2.083)--(-3.676,2.078)--(-3.512,2.006)--cycle;
\filldraw[fill opacity=0.8,fill=gray!20,draw=none](-3.585,2.054)--(-3.601,2.045)--(-3.624,2.055)--cycle;
\draw(-3.601,2.045)--(-3.624,2.055);
\filldraw[fill opacity=0.8,fill=gray!20,draw=none](-3.754,2.155)--(-3.807,2.206)--(-3.797,2.153)--(-3.745,2.13)--cycle;
\draw(-3.807,2.206)--(-3.797,2.153)--(-3.745,2.13);
\filldraw[fill opacity=0.8,fill=gray!20,draw=none](-3.766,2.192)--(-3.788,2.258)--(-3.797,2.262)--(-3.807,2.206)--(-3.766,2.189)--cycle;
\draw(-3.788,2.258)--(-3.797,2.262)--(-3.807,2.206)--(-3.766,2.189);
\filldraw[fill opacity=0.8,fill=gray!20,draw=none](-4.034,2.466)--(-3.678,2.168)--(-3.69,2.128)--(-4.011,2.397)--cycle;
\draw(-4.034,2.466)--(-3.678,2.168)--(-3.69,2.128)--(-4.011,2.397);
\filldraw[fill opacity=0.8,fill=gray!20,draw=none](-3.754,2.155)--(-3.766,2.189)--(-3.807,2.206)--cycle;
\draw(-3.766,2.189)--(-3.807,2.206);
\filldraw[fill opacity=0.8,fill=gray!20,draw=none](-3.74,2.141)--(-3.765,2.188)--(-3.766,2.189)--(-3.754,2.155)--cycle;
\draw(-3.765,2.188)--(-3.766,2.189);
\filldraw[fill opacity=0.8,fill=gray!20,draw=none](-3.74,2.141)--(-3.754,2.155)--(-3.745,2.13)--(-3.731,2.124)--cycle;
\draw(-3.745,2.13)--(-3.731,2.124);
\filldraw[fill opacity=0.8,fill=gray!20,draw=none](-3.766,2.192)--(-3.69,2.128)--(-3.688,2.097)--(-3.753,2.151)--cycle;
\draw(-3.766,2.192)--(-3.69,2.128)--(-3.688,2.097)--(-3.753,2.151);
\filldraw[fill opacity=0.8,fill=gray!20,draw=none](-3.74,2.141)--(-3.731,2.124)--(-3.717,2.118)--cycle;
\draw(-3.731,2.124)--(-3.717,2.118);
\filldraw[fill opacity=0.8,fill=gray!20,draw=none](-4.068,2.415)--(-3.755,2.153)--(-3.751,2.149)--(-3.732,2.13)--(-3.905,2.275)--cycle;
\draw(-4.068,2.415)--(-3.755,2.153);
\draw(-3.732,2.13)--(-3.905,2.275);
\filldraw[fill opacity=0.8,fill=gray!20,draw=none](-4.063,2.419)--(-4.079,2.42)--(-4.038,2.386)--cycle;
\draw(-4.079,2.42)--(-4.038,2.386);
\filldraw[fill opacity=0.8,fill=gray!20,draw=none](-4.079,2.42)--(-4.063,2.419)--(-4.084,2.446)--cycle;
\filldraw[fill opacity=0.8,fill=gray!20,draw=none](-4.265,2.891)--(-4.256,2.857)--(-4.282,2.864)--cycle;
\draw(-4.256,2.857)--(-4.282,2.864);
\filldraw[fill opacity=0.8,fill=gray!20,draw=none](-4.242,2.798)--(-4.257,2.798)--(-4.234,2.802)--cycle;
\draw(-4.257,2.798)--(-4.234,2.802);
\filldraw[fill opacity=0.8,fill=gray!20,draw=none](-4.257,2.798)--(-4.242,2.798)--(-4.248,2.795)--cycle;
\filldraw[fill opacity=0.8,fill=gray!20,draw=none](-4.252,2.804)--(-4.264,2.813)--(-4.253,2.822)--cycle;
\filldraw[fill opacity=0.8,fill=gray!20,draw=none](-4.15,2.819)--(-4.164,2.819)--(-4.194,2.835)--cycle;
\filldraw[fill opacity=0.8,fill=gray!20,draw=none](-4.15,2.819)--(-4.128,2.801)--(-4.164,2.819)--cycle;
\draw(-4.15,2.819)--(-4.128,2.801);
\filldraw[fill opacity=0.8,fill=gray!20,draw=none](-4.259,2.822)--(-4.194,2.835)--(-4.15,2.819)--(-4.16,2.817)--cycle;
\draw(-4.259,2.822)--(-4.194,2.835);
\draw(-4.15,2.819)--(-4.16,2.817);
\filldraw[fill opacity=0.8,fill=gray!20,draw=none](-4.253,2.822)--(-4.264,2.813)--(-4.253,2.842)--cycle;
\draw(-4.264,2.813)--(-4.253,2.842);
\filldraw[fill opacity=0.8,fill=gray!20,draw=none](-4.285,2.891)--(-4.298,2.886)--(-4.283,2.907)--cycle;
\draw(-4.298,2.886)--(-4.283,2.907);
\filldraw[fill opacity=0.8,fill=gray!20,draw=none](-4.298,2.886)--(-4.285,2.891)--(-4.286,2.877)--cycle;
\filldraw[fill opacity=0.8,fill=gray!20,draw=none](-4.318,2.943)--(-3.475,1.152)--(-3.461,1.105)--(-4.332,2.958)--cycle;
\draw(-3.461,1.105)--(-4.332,2.958)--(-4.318,2.943)--(-3.475,1.152);
\filldraw[fill opacity=0.8,fill=gray!20,draw=none](-2.725,2.096)--(-2.731,2.12)--(-2.731,2.081)--(-2.722,2.063)--cycle;
\filldraw[fill opacity=0.8,fill=gray!20,draw=none](-2.725,2.096)--(-2.727,2.124)--(-2.732,2.15)--(-2.731,2.12)--cycle;
\filldraw[fill opacity=0.8,fill=gray!20,draw=none](-7.601,1.914)--(-7.58,1.888)--(-7.637,1.921)--cycle;
\filldraw[fill opacity=0.8,fill=gray!20,draw=none](-7.58,1.91)--(-7.58,1.888)--(-7.601,1.914)--cycle;
\draw(-7.58,1.91)--(-7.58,1.888);
\filldraw[fill opacity=0.8,fill=gray!20,draw=none](-7.516,1.893)--(-7.524,1.88)--(-7.537,1.876)--(-7.533,1.895)--cycle;
\draw(-7.524,1.88)--(-7.537,1.876);
\filldraw[fill opacity=0.8,fill=gray!20,draw=none](-7.568,1.9)--(-7.559,1.879)--(-7.583,1.895)--(-7.569,1.9)--cycle;
\draw(-7.583,1.895)--(-7.569,1.9);
\filldraw[fill opacity=0.8,fill=gray!20,draw=none](-7.569,1.9)--(-7.583,1.895)--(-7.591,1.901)--cycle;
\draw(-7.569,1.9)--(-7.583,1.895);
\filldraw[fill opacity=0.8,fill=gray!20,draw=none](-7.498,1.869)--(-7.516,1.868)--(-7.488,1.848)--cycle;
\filldraw[fill opacity=0.8,fill=gray!20,draw=none](-7.499,1.922)--(-7.508,1.909)--(-7.512,1.899)--(-7.488,1.848)--(-7.488,1.926)--cycle;
\draw(-7.488,1.848)--(-7.488,1.926)--(-7.499,1.922);
\filldraw[fill opacity=0.8,fill=gray!20](-7.452,1.833)--(-7.616,1.904)--(-7.658,1.93)--(-7.495,1.859)--cycle;
\filldraw[fill opacity=0.8,fill=gray!20](-3.13,.785)--(-3.127,.839)--(-3.238,.834)--(-3.238,.78)--cycle;
\filldraw[fill opacity=0.8,fill=gray!20,draw=none](-2.732,2.15)--(-2.727,2.124)--(-2.73,2.153)--(-2.732,2.168)--cycle;
\draw(-2.73,2.153)--(-2.732,2.168);
\filldraw[fill opacity=0.8,fill=gray!20,draw=none](-2.734,2.176)--(-2.732,2.168)--(-2.73,2.153)--cycle;
\draw(-2.732,2.168)--(-2.73,2.153);
\filldraw[fill opacity=0.8,fill=gray!20,draw=none](-2.734,2.176)--(-2.739,2.204)--(-2.735,2.195)--(-2.732,2.168)--cycle;
\draw(-2.735,2.195)--(-2.732,2.168);
\filldraw[fill opacity=0.8,fill=gray!20,draw=none](-2.779,2.078)--(-2.787,2.126)--(-2.784,2.098)--cycle;
\filldraw[fill opacity=0.8,fill=gray!20,draw=none](-2.737,2.181)--(-2.732,2.15)--(-2.732,2.168)--(-2.735,2.195)--cycle;
\draw(-2.732,2.168)--(-2.735,2.195);
\filldraw[fill opacity=0.8,fill=gray!20,draw=none](-2.797,2.225)--(-2.804,2.251)--(-2.796,2.241)--(-2.793,2.215)--cycle;
\draw(-2.796,2.241)--(-2.793,2.215);
\filldraw[fill opacity=0.8,fill=gray!20,draw=none](-3.301,.933)--(-3.283,.938)--(-3.299,.941)--cycle;
\draw(-3.283,.938)--(-3.299,.941);
\filldraw[fill opacity=0.8,fill=gray!20,draw=none](-3.335,.841)--(-3.303,.839)--(-3.306,.862)--cycle;
\draw(-3.335,.841)--(-3.303,.839);
\filldraw[fill opacity=0.8,fill=gray!20,draw=none](-7.58,1.888)--(-7.58,1.686)--(-7.637,1.657)--(-7.637,1.921)--cycle;
\draw(-7.58,1.888)--(-7.58,1.686);
\draw(-7.637,1.657)--(-7.637,1.921);
\filldraw[fill opacity=0.8,fill=gray!20,draw=none](-7.647,1.925)--(-7.658,1.93)--(-7.709,1.936)--(-7.692,1.929)--cycle;
\draw(-7.647,1.925)--(-7.658,1.93)--(-7.709,1.936)--(-7.692,1.929);
\filldraw[fill opacity=0.8,fill=gray!20,draw=none](-7.637,1.921)--(-7.647,1.925)--(-7.692,1.929)--cycle;
\draw(-7.637,1.921)--(-7.647,1.925);
\filldraw[fill opacity=0.8,fill=gray!20,draw=none](-7.637,1.921)--(-7.637,1.657)--(-7.692,1.643)--(-7.692,1.954)--cycle;
\draw(-7.637,1.921)--(-7.637,1.657);
\draw(-7.692,1.643)--(-7.692,1.954);
\filldraw[fill opacity=0.8,fill=gray!20,draw=none](-3.353,2.004)--(-3.362,1.979)--(-3.305,1.652)--(-3.298,1.646)--(-3.336,1.945)--cycle;
\draw(-3.305,1.652)--(-3.298,1.646)--(-3.336,1.945);
\filldraw[fill opacity=0.8,fill=gray!20,draw=none](-2.765,1.742)--(-2.759,1.691)--(-2.696,1.729)--(-2.705,1.799)--cycle;
\draw(-2.765,1.742)--(-2.759,1.691)--(-2.696,1.729)--(-2.705,1.799);
\filldraw[fill opacity=0.8,fill=gray!20,draw=none](-3.286,1.942)--(-3.25,1.658)--(-3.184,1.643)--(-3.2,1.765)--cycle;
\draw(-3.286,1.942)--(-3.25,1.658);
\draw(-3.184,1.643)--(-3.2,1.765);
\filldraw[fill opacity=0.8,fill=gray!20,draw=none](-7.717,1.633)--(-7.692,1.617)--(-7.692,1.607)--(-7.737,1.6)--(-7.737,1.624)--cycle;
\draw(-7.692,1.617)--(-7.692,1.607)--(-7.737,1.6)--(-7.737,1.624);
\filldraw[fill opacity=0.8,fill=gray!20,draw=none](-2.743,2.221)--(-2.737,2.181)--(-2.735,2.195)--(-2.74,2.23)--cycle;
\draw(-2.735,2.195)--(-2.74,2.23);
\filldraw[fill opacity=0.8,fill=gray!20,draw=none](-3.302,2.002)--(-3.313,2.042)--(-3.323,2.041)--(-3.312,1.976)--cycle;
\filldraw[fill opacity=0.8,fill=gray!20,draw=none](-3.285,.895)--(-3.238,.891)--(-3.236,.948)--(-3.302,.952)--cycle;
\draw(-3.285,.895)--(-3.238,.891)--(-3.236,.948)--(-3.302,.952);
\filldraw[fill opacity=0.8,fill=gray!20,draw=none](-2.737,2.181)--(-2.744,2.107)--(-2.731,2.081)--(-2.732,2.15)--cycle;
\filldraw[fill opacity=0.8,fill=gray!20,draw=none](-2.797,2.225)--(-2.793,2.215)--(-2.792,2.206)--cycle;
\draw(-2.793,2.215)--(-2.792,2.206);
\filldraw[fill opacity=0.8,fill=gray!20,draw=none](-7.692,1.929)--(-7.692,1.643)--(-7.737,1.646)--(-7.737,1.91)--cycle;
\draw(-7.692,1.929)--(-7.692,1.643);
\draw(-7.737,1.646)--(-7.737,1.91);
\filldraw[fill opacity=0.8,fill=gray!20,draw=none](-7.722,1.921)--(-7.704,1.934)--(-7.709,1.936)--(-7.759,1.919)--(-7.747,1.914)--cycle;
\draw(-7.704,1.934)--(-7.709,1.936)--(-7.759,1.919)--(-7.747,1.914);
\filldraw[fill opacity=0.8,fill=gray!20,draw=none](-7.737,1.91)--(-7.737,1.646)--(-7.766,1.666)--(-7.766,1.868)--cycle;
\draw(-7.737,1.91)--(-7.737,1.646);
\draw(-7.766,1.666)--(-7.766,1.868);
\filldraw[fill opacity=0.8,fill=gray!20,draw=none](-7.728,1.65)--(-7.741,1.662)--(-7.71,1.675)--(-7.672,1.643)--(-7.698,1.632)--cycle;
\draw(-7.741,1.662)--(-7.71,1.675)--(-7.672,1.643)--(-7.698,1.632);
\filldraw[fill opacity=0.8,fill=gray!20,draw=none](-7.698,1.632)--(-7.672,1.643)--(-7.627,1.629)--(-7.661,1.614)--cycle;
\draw(-7.698,1.632)--(-7.672,1.643)--(-7.627,1.629)--(-7.661,1.614);
\filldraw[fill opacity=0.8,fill=gray!20,draw=none](-7.661,1.614)--(-7.627,1.629)--(-7.582,1.633)--(-7.641,1.607)--cycle;
\draw(-7.661,1.614)--(-7.627,1.629)--(-7.582,1.633)--(-7.641,1.607);
\filldraw[fill opacity=0.8,fill=gray!20,draw=none](-7.616,1.622)--(-7.641,1.608)--(-7.641,1.607)--(-7.582,1.633)--(-7.607,1.627)--cycle;
\draw(-7.641,1.607)--(-7.582,1.633);
\filldraw[fill opacity=0.8,fill=gray!20,draw=none](-7.531,1.766)--(-7.531,1.727)--(-7.58,1.686)--(-7.58,1.795)--cycle;
\draw(-7.531,1.766)--(-7.531,1.727);
\draw(-7.58,1.686)--(-7.58,1.795);
\filldraw[fill opacity=0.8,fill=gray!20,draw=none](-7.519,1.741)--(-7.534,1.774)--(-7.52,1.779)--cycle;
\draw(-7.534,1.774)--(-7.52,1.779);
\filldraw[fill opacity=0.8,fill=gray!20,draw=none](-7.496,1.771)--(-7.531,1.727)--(-7.531,1.836)--cycle;
\draw(-7.531,1.727)--(-7.531,1.836);
\filldraw[fill opacity=0.8,fill=gray!20,draw=none](-7.519,1.741)--(-7.519,1.727)--(-7.583,1.704)--(-7.596,1.752)--(-7.534,1.774)--cycle;
\draw(-7.519,1.727)--(-7.583,1.704)--(-7.596,1.752)--(-7.534,1.774);
\filldraw[fill opacity=0.8,fill=gray!20,draw=none](-7.537,1.843)--(-7.57,1.789)--(-7.58,1.795)--(-7.58,1.888)--cycle;
\draw(-7.58,1.795)--(-7.58,1.888);
\filldraw[fill opacity=0.8,fill=gray!20,draw=none](-7.534,1.837)--(-7.533,1.828)--(-7.614,1.8)--(-7.634,1.841)--(-7.555,1.869)--cycle;
\draw(-7.533,1.828)--(-7.614,1.8)--(-7.634,1.841)--(-7.555,1.869);
\filldraw[fill opacity=0.8,fill=gray!20,draw=none](-7.537,1.843)--(-7.531,1.836)--(-7.531,1.766)--(-7.57,1.789)--cycle;
\draw(-7.531,1.836)--(-7.531,1.766);
\filldraw[fill opacity=0.8,fill=gray!20,draw=none](-7.671,1.879)--(-7.667,1.882)--(-7.653,1.87)--(-7.634,1.841)--(-7.614,1.8)--(-7.596,1.752)--(-7.583,1.704)--(-7.576,1.665)--(-7.577,1.64)--(-7.583,1.636)--cycle;
\draw(-7.671,1.879)--(-7.667,1.882)--(-7.653,1.87)--(-7.634,1.841)--(-7.614,1.8)--(-7.596,1.752)--(-7.583,1.704)--(-7.576,1.665)--(-7.577,1.64)--(-7.583,1.636);
\filldraw[fill opacity=0.8,fill=gray!20,draw=none](-7.516,1.72)--(-7.523,1.684)--(-7.576,1.665)--(-7.583,1.704)--(-7.519,1.727)--cycle;
\draw(-7.523,1.684)--(-7.576,1.665)--(-7.583,1.704)--(-7.519,1.727);
\filldraw[fill opacity=0.8,fill=gray!20,draw=none](-7.559,1.879)--(-7.555,1.869)--(-7.634,1.841)--(-7.653,1.87)--(-7.583,1.895)--cycle;
\draw(-7.555,1.869)--(-7.634,1.841)--(-7.653,1.87)--(-7.583,1.895);
\filldraw[fill opacity=0.8,fill=gray!20,draw=none](-7.501,1.667)--(-7.577,1.64)--(-7.576,1.665)--(-7.523,1.684)--cycle;
\draw(-7.501,1.667)--(-7.577,1.64)--(-7.576,1.665)--(-7.523,1.684);
\filldraw[fill opacity=0.8,fill=gray!20,draw=none](-7.591,1.901)--(-7.583,1.895)--(-7.653,1.87)--(-7.667,1.882)--(-7.611,1.902)--cycle;
\draw(-7.583,1.895)--(-7.653,1.87)--(-7.667,1.882)--(-7.611,1.902);
\filldraw[fill opacity=0.8,fill=gray!20,draw=none](-7.616,1.613)--(-7.616,1.593)--(-7.561,1.586)--(-7.561,1.634)--cycle;
\draw(-7.616,1.613)--(-7.616,1.593);
\draw(-7.561,1.586)--(-7.561,1.634);
\filldraw[fill opacity=0.8,fill=gray!20,draw=none](-7.573,1.639)--(-7.582,1.633)--(-7.593,1.632)--cycle;
\draw(-7.573,1.639)--(-7.582,1.633)--(-7.593,1.632);
\filldraw[fill opacity=0.8,fill=gray!20,draw=none](-7.586,1.633)--(-7.577,1.64)--(-7.55,1.65)--cycle;
\draw(-7.586,1.633)--(-7.577,1.64)--(-7.55,1.65);
\filldraw[fill opacity=0.8,fill=gray!20,draw=none](-7.518,1.696)--(-7.543,1.657)--(-7.573,1.639)--(-7.593,1.632)--(-7.627,1.629)--(-7.672,1.643)--(-7.71,1.675)--(-7.736,1.72)--(-7.745,1.77)--(-7.736,1.818)--(-7.71,1.857)--(-7.68,1.876)--(-7.66,1.882)--(-7.627,1.886)--(-7.582,1.871)--(-7.543,1.839)--(-7.518,1.794)--(-7.509,1.744)--cycle;
\draw(-7.593,1.632)--(-7.627,1.629)--(-7.672,1.643)--(-7.71,1.675)--(-7.736,1.72)--(-7.745,1.77)--(-7.736,1.818)--(-7.71,1.857)--(-7.68,1.876);
\draw(-7.66,1.882)--(-7.627,1.886)--(-7.582,1.871)--(-7.543,1.839)--(-7.518,1.794)--(-7.509,1.744)--(-7.518,1.696)--(-7.543,1.657)--(-7.573,1.639);
\filldraw[fill opacity=0.8,fill=gray!20,draw=none](-7.657,1.763)--(-7.671,1.81)--(-7.677,1.849)--(-7.676,1.874)--(-7.671,1.879)--(-7.583,1.636)--(-7.586,1.633)--(-7.601,1.644)--(-7.62,1.673)--(-7.639,1.715)--cycle;
\draw(-7.583,1.636)--(-7.586,1.633)--(-7.601,1.644)--(-7.62,1.673)--(-7.639,1.715)--(-7.657,1.763)--(-7.671,1.81)--(-7.677,1.849)--(-7.676,1.874)--(-7.671,1.879);
\filldraw[fill opacity=0.8,fill=gray!20,draw=none](-7.481,1.814)--(-7.481,1.705)--(-7.496,1.771)--cycle;
\draw(-7.481,1.814)--(-7.481,1.705);
\filldraw[fill opacity=0.8,fill=gray!20,draw=none](-7.673,1.642)--(-7.673,1.626)--(-7.641,1.608)--(-7.616,1.622)--(-7.616,1.661)--cycle;
\draw(-7.673,1.642)--(-7.673,1.626);
\draw(-7.616,1.622)--(-7.616,1.661);
\filldraw[fill opacity=0.8,fill=gray!20,draw=none](-7.616,1.622)--(-7.641,1.608)--(-7.641,1.608)--cycle;
\filldraw[fill opacity=0.8,fill=gray!20,draw=none](-7.641,1.608)--(-7.637,1.605)--(-7.616,1.613)--(-7.616,1.622)--cycle;
\draw(-7.616,1.613)--(-7.616,1.622);
\filldraw[fill opacity=0.8,fill=gray!20,draw=none](-7.637,1.605)--(-7.616,1.593)--(-7.616,1.613)--cycle;
\draw(-7.616,1.593)--(-7.616,1.613);
\filldraw[fill opacity=0.8,fill=gray!20,draw=none](-7.585,1.633)--(-7.582,1.633)--(-7.543,1.657)--(-7.557,1.651)--cycle;
\draw(-7.582,1.633)--(-7.543,1.657)--(-7.557,1.651);
\filldraw[fill opacity=0.8,fill=gray!20,draw=none](-7.485,1.672)--(-7.488,1.668)--(-7.493,1.666)--(-7.586,1.633)--(-7.55,1.65)--(-7.501,1.667)--cycle;
\draw(-7.493,1.666)--(-7.586,1.633);
\draw(-7.55,1.65)--(-7.501,1.667);
\filldraw[fill opacity=0.8,fill=gray!20,draw=none](-7.488,1.69)--(-7.488,1.668)--(-7.485,1.672)--(-7.483,1.69)--cycle;
\draw(-7.488,1.69)--(-7.488,1.668);
\filldraw[fill opacity=0.8,fill=gray!20,draw=none](-7.488,1.668)--(-7.488,1.664)--(-7.486,1.66)--(-7.485,1.672)--cycle;
\draw(-7.488,1.668)--(-7.488,1.664);
\filldraw[fill opacity=0.8,fill=gray!20,draw=none](-7.558,1.659)--(-7.601,1.644)--(-7.586,1.633)--(-7.555,1.644)--cycle;
\draw(-7.558,1.659)--(-7.601,1.644)--(-7.586,1.633)--(-7.555,1.644);
\filldraw[fill opacity=0.8,fill=gray!20,draw=none](-7.547,1.683)--(-7.549,1.692)--(-7.555,1.696)--(-7.62,1.673)--(-7.601,1.644)--(-7.558,1.659)--cycle;
\draw(-7.555,1.696)--(-7.62,1.673)--(-7.601,1.644)--(-7.558,1.659);
\filldraw[fill opacity=0.8,fill=gray!20,draw=none](-7.616,1.661)--(-7.616,1.622)--(-7.592,1.636)--(-7.561,1.673)--cycle;
\draw(-7.616,1.661)--(-7.616,1.622);
\filldraw[fill opacity=0.8,fill=gray!20,draw=none](-7.641,1.611)--(-7.641,1.608)--(-7.607,1.627)--(-7.613,1.626)--cycle;
\filldraw[fill opacity=0.8,fill=gray!20](-7.637,1.61)--(-7.801,1.682)--(-7.759,1.655)--(-7.595,1.584)--cycle;
\filldraw[fill opacity=0.8,fill=gray!20](-3.13,.896)--(-3.14,.952)--(-3.236,.948)--(-3.238,.891)--cycle;
\filldraw[fill opacity=0.8,fill=gray!20,draw=none](-3.309,.896)--(-3.285,.895)--(-3.299,.941)--cycle;
\draw(-3.309,.896)--(-3.285,.895);
\filldraw[fill opacity=0.8,fill=gray!20](-3.127,.839)--(-3.13,.896)--(-3.238,.891)--(-3.238,.834)--cycle;
\filldraw[fill opacity=0.8,fill=gray!20,draw=none](-3.367,1.007)--(-3.396,1.037)--(-3.428,1.019)--(-3.366,1.004)--cycle;
\draw(-3.428,1.019)--(-3.366,1.004);
\filldraw[fill opacity=0.8,fill=gray!20,draw=none](-3.303,.839)--(-3.238,.834)--(-3.238,.891)--(-3.309,.896)--cycle;
\draw(-3.303,.839)--(-3.238,.834)--(-3.238,.891)--(-3.309,.896);
\filldraw[fill opacity=0.8,fill=gray!20,draw=none](-3.386,.844)--(-3.373,.856)--(-3.488,.884)--(-3.513,.849)--(-3.42,.826)--cycle;
\draw(-3.373,.856)--(-3.488,.884)--(-3.513,.849)--(-3.42,.826);
\filldraw[fill opacity=0.8,fill=gray!20,draw=none](-3.689,.988)--(-3.679,.968)--(-3.659,.941)--cycle;
\draw(-3.689,.988)--(-3.679,.968);
\filldraw[fill opacity=0.8,fill=gray!20,draw=none](-2.794,2.206)--(-2.792,2.178)--(-2.792,2.206)--(-2.793,2.215)--cycle;
\draw(-2.792,2.206)--(-2.793,2.215);
\filldraw[fill opacity=0.8,fill=gray!20,draw=none](-2.952,1.384)--(-2.922,1.147)--(-2.872,1.573)--(-2.877,1.611)--cycle;
\draw(-2.952,1.384)--(-2.922,1.147);
\draw(-2.872,1.573)--(-2.877,1.611);
\filldraw[fill opacity=0.8,fill=gray!20,draw=none](-2.797,2.239)--(-2.794,2.206)--(-2.793,2.215)--(-2.796,2.241)--cycle;
\draw(-2.793,2.215)--(-2.796,2.241);
\filldraw[fill opacity=0.8,fill=gray!20,draw=none](-2.818,2.683)--(-2.755,2.191)--(-2.74,2.23)--(-2.78,2.542)--cycle;
\draw(-2.818,2.683)--(-2.755,2.191);
\draw(-2.74,2.23)--(-2.78,2.542);
\filldraw[fill opacity=0.8,fill=gray!20,draw=none](-2.842,2.027)--(-2.812,1.786)--(-2.761,1.833)--(-2.785,2.024)--cycle;
\draw(-2.842,2.027)--(-2.812,1.786);
\draw(-2.761,1.833)--(-2.785,2.024);
\filldraw[fill opacity=0.8,fill=gray!20,draw=none](-4.497,2.724)--(-4.532,2.761)--(-4.533,2.764)--(-4.507,2.769)--(-4.487,2.726)--cycle;
\draw(-4.533,2.764)--(-4.507,2.769)--(-4.487,2.726)--(-4.497,2.724);
\filldraw[fill opacity=0.8,fill=gray!20,draw=none](-4.538,2.772)--(-4.541,2.768)--(-4.56,2.802)--(-4.553,2.806)--cycle;
\draw(-4.56,2.802)--(-4.553,2.806);
\filldraw[fill opacity=0.8,fill=gray!20,draw=none](-4.553,2.806)--(-4.56,2.802)--(-4.572,2.825)--cycle;
\draw(-4.553,2.806)--(-4.56,2.802);
\filldraw[fill opacity=0.8,fill=gray!20,draw=none](-4.56,2.812)--(-4.548,2.783)--(-4.564,2.81)--cycle;
\filldraw[fill opacity=0.8,fill=gray!20,draw=none](-4.549,2.783)--(-4.565,2.812)--(-4.545,2.816)--cycle;
\draw(-4.565,2.812)--(-4.545,2.816);
\filldraw[fill opacity=0.8,fill=gray!20,draw=none](-4.533,2.786)--(-4.515,2.767)--(-4.533,2.764)--cycle;
\draw(-4.515,2.767)--(-4.533,2.764);
\filldraw[fill opacity=0.8,fill=gray!20,draw=none](-4.533,2.786)--(-4.533,2.764)--(-4.534,2.764)--(-4.548,2.793)--(-4.547,2.799)--cycle;
\draw(-4.533,2.764)--(-4.534,2.764);
\filldraw[fill opacity=0.8,fill=gray!20,draw=none](-4.517,2.798)--(-4.538,2.772)--(-4.553,2.806)--(-4.537,2.816)--cycle;
\draw(-4.553,2.806)--(-4.537,2.816);
\filldraw[fill opacity=0.8,fill=gray!20,draw=none](-4.518,2.828)--(-4.553,2.806)--(-4.572,2.825)--(-4.584,2.847)--(-4.537,2.876)--cycle;
\draw(-4.518,2.828)--(-4.553,2.806);
\draw(-4.584,2.847)--(-4.537,2.876);
\filldraw[fill opacity=0.8,fill=gray!20,draw=none](-4.532,2.761)--(-4.535,2.759)--(-4.541,2.768)--(-4.54,2.769)--cycle;
\draw(-4.532,2.761)--(-4.535,2.759);
\filldraw[fill opacity=0.8,fill=gray!20,draw=none](-4.497,2.724)--(-4.511,2.721)--(-4.525,2.743)--(-4.532,2.761)--cycle;
\draw(-4.497,2.724)--(-4.511,2.721);
\filldraw[fill opacity=0.8,fill=gray!20,draw=none](-4.501,2.764)--(-4.516,2.737)--(-4.522,2.739)--(-4.535,2.759)--(-4.504,2.778)--cycle;
\draw(-4.535,2.759)--(-4.504,2.778);
\filldraw[fill opacity=0.8,fill=gray!20,draw=none](-4.525,2.743)--(-4.537,2.763)--(-4.533,2.764)--cycle;
\draw(-4.537,2.763)--(-4.533,2.764);
\filldraw[fill opacity=0.8,fill=gray!20,draw=none](-4.506,2.788)--(-4.504,2.778)--(-4.532,2.761)--(-4.54,2.769)--(-4.517,2.798)--cycle;
\draw(-4.504,2.778)--(-4.532,2.761);
\filldraw[fill opacity=0.8,fill=gray!20,draw=none](-4.534,2.764)--(-4.537,2.763)--(-4.549,2.783)--(-4.548,2.793)--cycle;
\draw(-4.534,2.764)--(-4.537,2.763);
\filldraw[fill opacity=0.8,fill=gray!20,draw=none](-4.56,2.812)--(-4.543,2.821)--(-4.522,2.802)--(-4.535,2.787)--(-4.548,2.783)--cycle;
\draw(-4.535,2.787)--(-4.548,2.783);
\filldraw[fill opacity=0.8,fill=gray!20,draw=none](-4.475,2.726)--(-4.487,2.726)--(-4.49,2.731)--(-4.466,2.746)--cycle;
\draw(-4.475,2.726)--(-4.487,2.726)--(-4.49,2.731);
\filldraw[fill opacity=0.8,fill=gray!20,draw=none](-4.543,2.821)--(-4.52,2.833)--(-4.514,2.811)--(-4.522,2.802)--cycle;
\filldraw[fill opacity=0.8,fill=gray!20,draw=none](-4.509,2.79)--(-4.507,2.777)--(-4.526,2.75)--(-4.546,2.779)--(-4.535,2.787)--(-4.514,2.795)--cycle;
\draw(-4.535,2.787)--(-4.514,2.795);
\filldraw[fill opacity=0.8,fill=gray!20,draw=none](-4.514,2.795)--(-4.535,2.787)--(-4.522,2.802)--cycle;
\draw(-4.514,2.795)--(-4.535,2.787);
\filldraw[fill opacity=0.8,fill=gray!20,draw=none](-4.51,2.796)--(-4.514,2.795)--(-4.522,2.802)--(-4.514,2.811)--cycle;
\draw(-4.51,2.796)--(-4.514,2.795);
\filldraw[fill opacity=0.8,fill=gray!20,draw=none](-4.482,2.719)--(-4.487,2.726)--(-4.475,2.726)--cycle;
\draw(-4.482,2.719)--(-4.487,2.726)--(-4.475,2.726);
\filldraw[fill opacity=0.8,fill=gray!20,draw=none](-4.509,2.79)--(-4.514,2.795)--(-4.51,2.796)--cycle;
\draw(-4.514,2.795)--(-4.51,2.796);
\filldraw[fill opacity=0.8,fill=gray!20,draw=none](-4.494,2.776)--(-4.501,2.764)--(-4.504,2.778)--(-4.499,2.781)--cycle;
\draw(-4.504,2.778)--(-4.499,2.781);
\filldraw[fill opacity=0.8,fill=gray!20,draw=none](-4.494,2.776)--(-4.499,2.781)--(-4.488,2.788)--cycle;
\draw(-4.499,2.781)--(-4.488,2.788);
\filldraw[fill opacity=0.8,fill=gray!20,draw=none](-4.487,2.716)--(-4.501,2.707)--(-4.511,2.721)--(-4.487,2.726)--(-4.486,2.724)--cycle;
\draw(-4.511,2.721)--(-4.487,2.726)--(-4.486,2.724);
\filldraw[fill opacity=0.8,fill=gray!20,draw=none](-4.487,2.789)--(-4.488,2.788)--(-4.487,2.791)--cycle;
\draw(-4.487,2.789)--(-4.488,2.788);
\filldraw[fill opacity=0.8,fill=gray!20,draw=none](-4.49,2.773)--(-4.494,2.776)--(-4.488,2.788)--(-4.487,2.789)--cycle;
\draw(-4.488,2.788)--(-4.487,2.789);
\filldraw[fill opacity=0.8,fill=gray!20,draw=none](-4.487,2.716)--(-4.486,2.724)--(-4.482,2.719)--cycle;
\draw(-4.486,2.724)--(-4.482,2.719);
\filldraw[fill opacity=0.8,fill=gray!20,draw=none](-4.503,2.784)--(-4.507,2.777)--(-4.509,2.79)--cycle;
\filldraw[fill opacity=0.8,fill=gray!20,draw=none](-4.49,2.773)--(-4.497,2.737)--(-4.501,2.764)--(-4.494,2.776)--cycle;
\filldraw[fill opacity=0.8,fill=gray!20,draw=none](-4.497,2.778)--(-4.503,2.784)--(-4.491,2.8)--cycle;
\filldraw[fill opacity=0.8,fill=gray!20,draw=none](-4.497,2.737)--(-4.497,2.735)--(-4.513,2.725)--(-4.518,2.733)--(-4.501,2.764)--cycle;
\draw(-4.497,2.735)--(-4.513,2.725);
\filldraw[fill opacity=0.8,fill=gray!20,draw=none](-4.497,2.778)--(-4.503,2.757)--(-4.505,2.757)--(-4.507,2.777)--(-4.503,2.784)--cycle;
\draw(-4.503,2.757)--(-4.505,2.757);
\filldraw[fill opacity=0.8,fill=gray!20,draw=none](-4.505,2.757)--(-4.525,2.749)--(-4.526,2.75)--(-4.507,2.777)--cycle;
\draw(-4.505,2.757)--(-4.525,2.749);
\filldraw[fill opacity=0.8,fill=gray!20,draw=none](-4.505,2.755)--(-4.516,2.738)--(-4.525,2.749)--(-4.505,2.757)--cycle;
\draw(-4.525,2.749)--(-4.505,2.757);
\filldraw[fill opacity=0.8,fill=gray!20,draw=none](-4.505,2.755)--(-4.505,2.757)--(-4.503,2.757)--cycle;
\draw(-4.505,2.757)--(-4.503,2.757);
\filldraw[fill opacity=0.8,fill=gray!20,draw=none](-4.528,2.872)--(-4.532,2.876)--(-4.527,2.875)--cycle;
\filldraw[fill opacity=0.8,fill=gray!20,draw=none](-4.64,2.94)--(-7.433,1.948)--(-7.325,2.003)--(-4.678,2.944)--cycle;
\draw(-4.64,2.94)--(-7.433,1.948);
\draw(-7.325,2.003)--(-4.678,2.944);
\filldraw[fill opacity=0.8,fill=gray!20,draw=none](-5.509,2.394)--(-5.494,2.405)--(-5.35,2.456)--cycle;
\draw(-5.494,2.405)--(-5.35,2.456);
\filldraw[fill opacity=0.8,fill=gray!20,draw=none](-4.541,2.768)--(-4.559,2.744)--(-7.358,1.009)--(-7.36,1.011)--(-7.301,1.102)--(-4.56,2.802)--cycle;
\draw(-4.559,2.744)--(-7.358,1.009);
\draw(-7.301,1.102)--(-4.56,2.802);
\filldraw[fill opacity=0.8,fill=gray!20,draw=none](-4.913,2.643)--(-5.584,2.227)--(-5.683,2.219)--(-5.298,2.457)--cycle;
\draw(-4.913,2.643)--(-5.584,2.227);
\draw(-5.683,2.219)--(-5.298,2.457);
\filldraw[fill opacity=0.8,fill=gray!20,draw=none](-5.557,2.355)--(-5.563,2.355)--(-5.511,2.374)--(-5.42,2.406)--cycle;
\draw(-5.511,2.374)--(-5.42,2.406);
\filldraw[fill opacity=0.8,fill=gray!20,draw=none](-5.42,2.406)--(-5.315,2.446)--(-5.443,2.367)--(-5.528,2.352)--(-5.5,2.369)--cycle;
\draw(-5.315,2.446)--(-5.443,2.367);
\draw(-5.528,2.352)--(-5.5,2.369);
\filldraw[fill opacity=0.8,fill=gray!20,draw=none](-5.257,2.482)--(-5.315,2.446)--(-5.42,2.406)--cycle;
\draw(-5.257,2.482)--(-5.315,2.446);
\filldraw[fill opacity=0.8,fill=gray!20,draw=none](-5.16,2.524)--(-5.298,2.457)--(-5.257,2.482)--cycle;
\draw(-5.298,2.457)--(-5.257,2.482);
\filldraw[fill opacity=0.8,fill=gray!20,draw=none](-4.768,2.638)--(-5.511,2.374)--(-5.16,2.524)--(-4.641,2.708)--cycle;
\draw(-4.768,2.638)--(-5.511,2.374);
\draw(-5.16,2.524)--(-4.641,2.708);
\filldraw[fill opacity=0.8,fill=gray!20,draw=none](-4.602,2.92)--(-4.621,2.916)--(-4.619,2.936)--cycle;
\draw(-4.602,2.92)--(-4.621,2.916)--(-4.619,2.936);
\filldraw[fill opacity=0.8,fill=gray!20,draw=none](-4.611,2.883)--(-4.618,2.885)--(-4.619,2.893)--cycle;
\draw(-4.618,2.885)--(-4.619,2.893);
\filldraw[fill opacity=0.8,fill=gray!20,draw=none](-4.615,2.881)--(-5.257,2.482)--(-5.42,2.406)--(-5.472,2.387)--(-4.631,2.908)--cycle;
\draw(-4.615,2.881)--(-5.257,2.482);
\draw(-5.472,2.387)--(-4.631,2.908);
\filldraw[fill opacity=0.8,fill=gray!20,draw=none](-5.563,2.355)--(-5.565,2.355)--(-5.511,2.374)--cycle;
\draw(-5.565,2.355)--(-5.511,2.374);
\filldraw[fill opacity=0.8,fill=gray!20,draw=none](-5.527,2.364)--(-5.555,2.354)--(-5.557,2.355)--(-5.472,2.387)--cycle;
\draw(-5.527,2.364)--(-5.555,2.354);
\filldraw[fill opacity=0.8,fill=gray!20,draw=none](-4.55,2.937)--(-4.601,2.919)--(-4.618,2.935)--(-4.593,2.957)--(-4.524,2.982)--cycle;
\draw(-4.55,2.937)--(-4.601,2.919);
\draw(-4.593,2.957)--(-4.524,2.982);
\filldraw[fill opacity=0.8,fill=gray!20,draw=none](-4.548,2.898)--(-4.605,2.874)--(-4.611,2.883)--(-4.557,2.916)--cycle;
\draw(-4.611,2.883)--(-4.557,2.916);
\filldraw[fill opacity=0.8,fill=gray!20,draw=none](-4.631,2.908)--(-4.661,2.89)--(-4.681,2.894)--(-4.649,2.914)--cycle;
\draw(-4.631,2.908)--(-4.661,2.89);
\draw(-4.681,2.894)--(-4.649,2.914);
\filldraw[fill opacity=0.8,fill=gray!20,draw=none](-4.654,2.872)--(-4.657,2.893)--(-4.621,2.916)--(-4.616,2.868)--cycle;
\draw(-4.654,2.872)--(-4.657,2.893)--(-4.621,2.916)--(-4.616,2.868);
\filldraw[fill opacity=0.8,fill=gray!20,draw=none](-4.602,2.889)--(-4.615,2.881)--(-4.631,2.908)--(-4.613,2.919)--cycle;
\draw(-4.602,2.889)--(-4.615,2.881);
\draw(-4.631,2.908)--(-4.613,2.919);
\filldraw[fill opacity=0.8,fill=gray!20,draw=none](-4.6,2.919)--(-4.602,2.92)--(-4.601,2.92)--cycle;
\draw(-4.602,2.92)--(-4.601,2.92);
\filldraw[fill opacity=0.8,fill=gray!20,draw=none](-4.542,2.955)--(-4.57,2.908)--(-4.602,2.889)--(-4.613,2.919)--(-4.56,2.953)--cycle;
\draw(-4.57,2.908)--(-4.602,2.889);
\draw(-4.613,2.919)--(-4.56,2.953);
\filldraw[fill opacity=0.8,fill=gray!20,draw=none](-4.633,2.909)--(-4.644,2.917)--(-4.618,2.943)--(-4.621,2.916)--cycle;
\draw(-4.618,2.943)--(-4.621,2.916)--(-4.633,2.909);
\filldraw[fill opacity=0.8,fill=gray!20,draw=none](-4.532,2.876)--(-4.536,2.879)--(-4.527,2.875)--cycle;
\draw(-4.536,2.879)--(-4.527,2.875);
\filldraw[fill opacity=0.8,fill=gray!20,draw=none](-4.528,2.872)--(-4.534,2.858)--(-4.545,2.872)--(-4.54,2.877)--(-4.532,2.876)--cycle;
\filldraw[fill opacity=0.8,fill=gray!20,draw=none](-4.532,2.876)--(-4.54,2.877)--(-4.537,2.88)--(-4.536,2.879)--cycle;
\draw(-4.537,2.88)--(-4.536,2.879);
\filldraw[fill opacity=0.8,fill=gray!20,draw=none](-4.54,2.877)--(-4.551,2.879)--(-4.561,2.892)--(-4.537,2.88)--cycle;
\draw(-4.561,2.892)--(-4.537,2.88);
\filldraw[fill opacity=0.8,fill=gray!20,draw=none](-4.545,2.872)--(-4.551,2.879)--(-4.54,2.877)--cycle;
\filldraw[fill opacity=0.8,fill=gray!20,draw=none](-4.545,2.872)--(-4.534,2.858)--(-4.535,2.855)--(-4.547,2.851)--(-4.552,2.865)--cycle;
\filldraw[fill opacity=0.8,fill=gray!20,draw=none](-4.545,2.872)--(-4.552,2.865)--(-4.556,2.88)--(-4.551,2.879)--cycle;
\filldraw[fill opacity=0.8,fill=gray!20,draw=none](-4.611,2.883)--(-4.598,2.865)--(-4.599,2.863)--(-4.615,2.86)--(-4.618,2.885)--cycle;
\draw(-4.599,2.863)--(-4.615,2.86)--(-4.618,2.885);
\filldraw[fill opacity=0.8,fill=gray!20,draw=none](-4.631,2.908)--(-4.649,2.914)--(-4.644,2.917)--cycle;
\draw(-4.649,2.914)--(-4.644,2.917);
\filldraw[fill opacity=0.8,fill=gray!20,draw=none](-4.633,2.909)--(-4.657,2.893)--(-4.655,2.905)--(-4.644,2.917)--cycle;
\draw(-4.633,2.909)--(-4.657,2.893)--(-4.655,2.905);
\filldraw[fill opacity=0.8,fill=gray!20,draw=none](-4.552,2.865)--(-4.568,2.848)--(-4.589,2.859)--(-4.571,2.883)--(-4.556,2.88)--cycle;
\draw(-4.568,2.848)--(-4.589,2.859);
\filldraw[fill opacity=0.8,fill=gray!20,draw=none](-4.552,2.865)--(-4.547,2.851)--(-4.563,2.846)--(-4.568,2.848)--cycle;
\draw(-4.563,2.846)--(-4.568,2.848);
\filldraw[fill opacity=0.8,fill=gray!20,draw=none](-4.536,2.883)--(-4.535,2.855)--(-4.547,2.851)--(-4.553,2.866)--cycle;
\filldraw[fill opacity=0.8,fill=gray!20,draw=none](-4.535,2.855)--(-4.534,2.837)--(-4.543,2.84)--(-4.547,2.851)--cycle;
\filldraw[fill opacity=0.8,fill=gray!20,draw=none](-4.534,2.837)--(-4.533,2.821)--(-4.535,2.819)--(-4.543,2.84)--cycle;
\draw(-4.533,2.821)--(-4.535,2.819);
\filldraw[fill opacity=0.8,fill=gray!20,draw=none](-4.563,2.846)--(-4.543,2.84)--(-4.535,2.819)--(-4.546,2.801)--(-4.568,2.844)--cycle;
\draw(-4.535,2.819)--(-4.546,2.801);
\filldraw[fill opacity=0.8,fill=gray!20,draw=none](-4.563,2.846)--(-4.547,2.851)--(-4.543,2.84)--cycle;
\filldraw[fill opacity=0.8,fill=gray!20,draw=none](-4.535,2.855)--(-4.543,2.836)--(-4.547,2.851)--cycle;
\filldraw[fill opacity=0.8,fill=gray!20,draw=none](-4.503,2.977)--(-4.481,2.961)--(-4.55,2.937)--(-4.524,2.982)--cycle;
\draw(-4.481,2.961)--(-4.55,2.937);
\filldraw[fill opacity=0.8,fill=gray!20,draw=none](-4.535,2.902)--(-4.531,2.89)--(-4.542,2.882)--(-4.562,2.892)--cycle;
\draw(-4.542,2.882)--(-4.562,2.892);
\filldraw[fill opacity=0.8,fill=gray!20,draw=none](-4.537,2.885)--(-4.536,2.883)--(-4.553,2.866)--(-4.555,2.869)--(-4.541,2.89)--cycle;
\draw(-4.555,2.869)--(-4.541,2.89)--(-4.537,2.885);
\filldraw[fill opacity=0.8,fill=gray!20,draw=none](-4.695,2.613)--(-7.352,.965)--(-7.347,1.016)--(-4.559,2.744)--cycle;
\draw(-4.695,2.613)--(-7.352,.965);
\draw(-7.347,1.016)--(-4.559,2.744);
\filldraw[fill opacity=0.8,fill=gray!20,draw=none](-4.86,2.481)--(-7.106,1.088)--(-7.021,1.171)--(-4.695,2.613)--cycle;
\draw(-4.86,2.481)--(-7.106,1.088);
\draw(-7.021,1.171)--(-4.695,2.613);
\filldraw[fill opacity=0.8,fill=gray!20,draw=none](-4.549,2.777)--(-4.641,2.708)--(-5.16,2.524)--(-4.913,2.643)--(-4.774,2.702)--(-4.549,2.782)--cycle;
\draw(-4.641,2.708)--(-5.16,2.524);
\draw(-4.774,2.702)--(-4.549,2.782);
\filldraw[fill opacity=0.8,fill=gray!20,draw=none](-4.564,2.81)--(-4.548,2.783)--(-4.774,2.702)--cycle;
\draw(-4.548,2.783)--(-4.774,2.702);
\filldraw[fill opacity=0.8,fill=gray!20,draw=none](-4.579,2.85)--(-4.56,2.813)--(-4.565,2.812)--(-4.588,2.856)--cycle;
\draw(-4.56,2.813)--(-4.565,2.812);
\filldraw[fill opacity=0.8,fill=gray!20,draw=none](-4.632,2.797)--(-4.65,2.837)--(-4.615,2.86)--(-4.601,2.818)--cycle;
\draw(-4.632,2.797)--(-4.65,2.837)--(-4.615,2.86)--(-4.601,2.818);
\filldraw[fill opacity=0.8,fill=gray!20,draw=none](-4.628,2.785)--(-4.632,2.797)--(-4.601,2.818)--(-4.597,2.806)--cycle;
\draw(-4.601,2.818)--(-4.597,2.806)--(-4.628,2.785)--(-4.632,2.797);
\filldraw[fill opacity=0.8,fill=gray!20,draw=none](-4.522,2.726)--(-4.526,2.717)--(-4.695,2.613)--(-4.559,2.744)--(-4.547,2.751)--cycle;
\draw(-4.526,2.717)--(-4.695,2.613);
\draw(-4.559,2.744)--(-4.547,2.751);
\filldraw[fill opacity=0.8,fill=gray!20,draw=none](-4.535,2.759)--(-4.559,2.744)--(-4.541,2.768)--cycle;
\draw(-4.535,2.759)--(-4.559,2.744);
\filldraw[fill opacity=0.8,fill=gray!20,draw=none](-4.552,2.747)--(-4.553,2.739)--(-4.559,2.737)--(-4.577,2.756)--(-4.563,2.766)--cycle;
\draw(-4.553,2.739)--(-4.559,2.737);
\filldraw[fill opacity=0.8,fill=gray!20,draw=none](-4.559,2.737)--(-4.641,2.708)--(-4.577,2.756)--cycle;
\draw(-4.559,2.737)--(-4.641,2.708);
\filldraw[fill opacity=0.8,fill=gray!20](-4.593,2.741)--(-4.628,2.785)--(-4.597,2.806)--(-4.567,2.757)--cycle;
\filldraw[fill opacity=0.8,fill=gray!20,draw=none](-4.481,2.961)--(-4.47,2.954)--(-4.477,2.944)--cycle;
\draw(-4.47,2.954)--(-4.477,2.944);
\filldraw[fill opacity=0.8,fill=gray!20,draw=none](-4.485,2.95)--(-4.494,2.957)--(-4.474,2.964)--cycle;
\draw(-4.494,2.957)--(-4.474,2.964);
\filldraw[fill opacity=0.8,fill=gray!20,draw=none](-4.51,2.96)--(-4.508,2.947)--(-4.57,2.908)--(-4.542,2.955)--cycle;
\draw(-4.508,2.947)--(-4.57,2.908);
\filldraw[fill opacity=0.8,fill=gray!20,draw=none](-4.535,2.902)--(-4.562,2.892)--(-4.581,2.902)--(-4.55,2.943)--(-4.545,2.94)--cycle;
\draw(-4.562,2.892)--(-4.581,2.902);
\draw(-4.55,2.943)--(-4.545,2.94);
\filldraw[fill opacity=0.8,fill=gray!20,draw=none](-4.549,2.914)--(-4.541,2.89)--(-4.547,2.881)--(-4.552,2.906)--cycle;
\draw(-4.549,2.914)--(-4.541,2.89)--(-4.547,2.881);
\filldraw[fill opacity=0.8,fill=gray!20,draw=none](-4.614,2.932)--(-4.613,2.919)--(-4.631,2.908)--(-4.644,2.917)--(-4.626,2.928)--cycle;
\draw(-4.613,2.919)--(-4.631,2.908);
\draw(-4.644,2.917)--(-4.626,2.928);
\filldraw[fill opacity=0.8,fill=gray!20,draw=none](-4.551,2.879)--(-4.571,2.883)--(-4.564,2.893)--(-4.561,2.892)--cycle;
\draw(-4.564,2.893)--(-4.561,2.892);
\filldraw[fill opacity=0.8,fill=gray!20,draw=none](-4.542,2.876)--(-4.542,2.876)--(-4.55,2.891)--(-4.55,2.894)--cycle;
\draw(-4.542,2.876)--(-4.55,2.891)--(-4.55,2.894);
\filldraw[fill opacity=0.8,fill=gray!20,draw=none](-4.571,2.883)--(-4.606,2.89)--(-4.594,2.909)--(-4.564,2.893)--cycle;
\draw(-4.594,2.909)--(-4.564,2.893);
\filldraw[fill opacity=0.8,fill=gray!20,draw=none](-4.548,2.888)--(-4.561,2.877)--(-4.55,2.891)--cycle;
\draw(-4.561,2.877)--(-4.55,2.891)--(-4.548,2.888);
\filldraw[fill opacity=0.8,fill=gray!20,draw=none](-4.65,2.837)--(-4.654,2.872)--(-4.616,2.868)--(-4.615,2.86)--cycle;
\draw(-4.616,2.868)--(-4.615,2.86)--(-4.65,2.837)--(-4.654,2.872);
\filldraw[fill opacity=0.8,fill=gray!20,draw=none](-4.589,2.859)--(-4.617,2.873)--(-4.606,2.89)--(-4.571,2.883)--cycle;
\draw(-4.589,2.859)--(-4.617,2.873);
\filldraw[fill opacity=0.8,fill=gray!20,draw=none](-4.549,2.914)--(-4.552,2.906)--(-4.556,2.925)--(-4.554,2.927)--cycle;
\draw(-4.556,2.925)--(-4.554,2.927)--(-4.549,2.914);
\filldraw[fill opacity=0.8,fill=gray!20,draw=none](-4.552,2.906)--(-4.55,2.891)--(-4.559,2.88)--cycle;
\draw(-4.552,2.906)--(-4.55,2.891)--(-4.559,2.88);
\filldraw[fill opacity=0.8,fill=gray!20,draw=none](-4.552,2.906)--(-4.547,2.881)--(-4.567,2.851)--cycle;
\draw(-4.547,2.881)--(-4.567,2.851);
\filldraw[fill opacity=0.8,fill=gray!20,draw=none](-4.547,2.851)--(-4.543,2.836)--(-4.563,2.846)--cycle;
\draw(-4.543,2.836)--(-4.563,2.846);
\filldraw[fill opacity=0.8,fill=gray!20,draw=none](-4.563,2.846)--(-4.568,2.844)--(-4.569,2.848)--cycle;
\filldraw[fill opacity=0.8,fill=gray!20,draw=none](-4.55,2.835)--(-4.562,2.835)--(-4.57,2.85)--(-4.543,2.836)--cycle;
\draw(-4.57,2.85)--(-4.543,2.836);
\filldraw[fill opacity=0.8,fill=gray!20,draw=none](-4.553,2.866)--(-4.547,2.851)--(-4.563,2.846)--(-4.569,2.848)--(-4.564,2.856)--cycle;
\draw(-4.569,2.848)--(-4.564,2.856);
\filldraw[fill opacity=0.8,fill=gray!20,draw=none](-4.553,2.866)--(-4.564,2.856)--(-4.555,2.869)--cycle;
\draw(-4.564,2.856)--(-4.555,2.869);
\filldraw[fill opacity=0.8,fill=gray!20,draw=none](-4.661,2.89)--(-5.472,2.387)--(-5.527,2.364)--(-5.526,2.37)--(-4.681,2.894)--cycle;
\draw(-4.661,2.89)--(-5.472,2.387);
\draw(-5.526,2.37)--(-4.681,2.894);
\filldraw[fill opacity=0.8,fill=gray!20,draw=none](-4.569,2.848)--(-4.593,2.855)--(-4.589,2.859)--(-4.57,2.85)--cycle;
\draw(-4.589,2.859)--(-4.57,2.85);
\filldraw[fill opacity=0.8,fill=gray!20,draw=none](-4.551,2.76)--(-4.554,2.773)--(-4.549,2.777)--cycle;
\filldraw[fill opacity=0.8,fill=gray!20,draw=none](-4.551,2.76)--(-4.552,2.747)--(-4.563,2.766)--(-4.554,2.773)--cycle;
\filldraw[fill opacity=0.8,fill=gray!20,draw=none](-4.537,2.763)--(-4.567,2.757)--(-4.597,2.806)--(-4.565,2.812)--cycle;
\draw(-4.537,2.763)--(-4.567,2.757)--(-4.597,2.806)--(-4.565,2.812);
\filldraw[fill opacity=0.8,fill=gray!20,draw=none](-4.525,2.743)--(-4.52,2.729)--(-4.522,2.726)--(-4.547,2.751)--(-4.535,2.759)--cycle;
\draw(-4.547,2.751)--(-4.535,2.759);
\filldraw[fill opacity=0.8,fill=gray!20,draw=none](-4.548,2.737)--(-4.553,2.727)--(-4.579,2.705)--(-4.768,2.638)--(-4.641,2.708)--(-4.548,2.741)--cycle;
\draw(-4.579,2.705)--(-4.768,2.638);
\draw(-4.641,2.708)--(-4.548,2.741);
\filldraw[fill opacity=0.8,fill=gray!20,draw=none](-4.546,2.742)--(-4.553,2.739)--(-4.551,2.76)--cycle;
\draw(-4.546,2.742)--(-4.553,2.739);
\filldraw[fill opacity=0.8,fill=gray!20,draw=none](-4.517,2.723)--(-4.518,2.722)--(-4.522,2.726)--(-4.52,2.729)--cycle;
\draw(-4.517,2.723)--(-4.518,2.722);
\filldraw[fill opacity=0.8,fill=gray!20,draw=none](-4.518,2.722)--(-4.526,2.717)--(-4.522,2.726)--cycle;
\draw(-4.518,2.722)--(-4.526,2.717);
\filldraw[fill opacity=0.8,fill=gray!20,draw=none](-4.548,2.737)--(-4.548,2.741)--(-4.546,2.742)--cycle;
\draw(-4.548,2.741)--(-4.546,2.742);
\filldraw[fill opacity=0.8,fill=gray!20,draw=none](-4.577,2.867)--(-4.559,2.88)--(-4.567,2.851)--(-4.569,2.848)--cycle;
\draw(-4.567,2.851)--(-4.569,2.848);
\filldraw[fill opacity=0.8,fill=gray!20,draw=none](-4.548,2.888)--(-4.541,2.874)--(-4.547,2.845)--(-4.561,2.877)--cycle;
\draw(-4.548,2.888)--(-4.541,2.874);
\filldraw[fill opacity=0.8,fill=gray!20,draw=none](-4.533,2.888)--(-3.77,1.265)--(-3.727,1.102)--(-4.554,2.86)--cycle;
\draw(-3.727,1.102)--(-4.554,2.86)--(-4.533,2.888)--(-3.77,1.265);
\filldraw[fill opacity=0.8,fill=gray!20,draw=none](-4.489,2.692)--(-4.492,2.694)--(-4.501,2.707)--(-4.487,2.716)--cycle;
\draw(-4.489,2.692)--(-4.492,2.694);
\filldraw[fill opacity=0.8,fill=gray!20,draw=none](-4.506,2.717)--(-4.513,2.725)--(-4.497,2.735)--cycle;
\draw(-4.513,2.725)--(-4.497,2.735);
\filldraw[fill opacity=0.8,fill=gray!20,draw=none](-4.496,2.736)--(-4.497,2.735)--(-4.497,2.737)--cycle;
\draw(-4.496,2.736)--(-4.497,2.735);
\filldraw[fill opacity=0.8,fill=gray!20,draw=none](-4.497,2.707)--(-4.506,2.717)--(-4.497,2.735)--(-4.496,2.736)--cycle;
\draw(-4.497,2.735)--(-4.496,2.736);
\filldraw[fill opacity=0.8,fill=gray!20,draw=none](-4.507,2.731)--(-4.509,2.73)--(-4.516,2.738)--(-4.505,2.755)--cycle;
\draw(-4.507,2.731)--(-4.509,2.73);
\filldraw[fill opacity=0.8,fill=gray!20,draw=none](-4.534,2.817)--(-4.541,2.792)--(-4.546,2.801)--(-4.535,2.819)--cycle;
\draw(-4.546,2.801)--(-4.535,2.819);
\filldraw[fill opacity=0.8,fill=gray!20,draw=none](-4.531,2.804)--(-4.536,2.809)--(-4.534,2.817)--cycle;
\filldraw[fill opacity=0.8,fill=gray!20,draw=none](-4.549,2.777)--(-4.549,2.782)--(-4.535,2.787)--cycle;
\draw(-4.549,2.782)--(-4.535,2.787);
\filldraw[fill opacity=0.8,fill=gray!20,draw=none](-4.531,2.804)--(-4.518,2.752)--(-4.518,2.751)--(-4.541,2.792)--(-4.536,2.809)--cycle;
\draw(-4.518,2.752)--(-4.518,2.751);
\filldraw[fill opacity=0.8,fill=gray!20,draw=none](-4.543,2.835)--(-4.55,2.835)--(-4.543,2.836)--cycle;
\filldraw[fill opacity=0.8,fill=gray!20,draw=none](-4.554,2.819)--(-4.55,2.835)--(-4.543,2.835)--(-4.542,2.831)--(-4.553,2.818)--cycle;
\draw(-4.542,2.831)--(-4.553,2.818);
\filldraw[fill opacity=0.8,fill=gray!20,draw=none](-4.529,2.775)--(-4.553,2.818)--(-4.542,2.831)--cycle;
\draw(-4.553,2.818)--(-4.542,2.831);
\filldraw[fill opacity=0.8,fill=gray!20,draw=none](-4.55,2.835)--(-4.547,2.845)--(-4.543,2.835)--cycle;
\filldraw[fill opacity=0.8,fill=gray!20,draw=none](-4.547,2.845)--(-4.554,2.819)--(-4.574,2.862)--(-4.561,2.877)--cycle;
\draw(-4.574,2.862)--(-4.561,2.877);
\filldraw[fill opacity=0.8,fill=gray!20,draw=none](-4.543,2.835)--(-4.543,2.805)--(-4.561,2.834)--(-4.55,2.835)--cycle;
\filldraw[fill opacity=0.8,fill=gray!20,draw=none](-4.508,2.73)--(-4.509,2.73)--(-4.507,2.731)--cycle;
\draw(-4.509,2.73)--(-4.507,2.731);
\filldraw[fill opacity=0.8,fill=gray!20,draw=none](-4.55,2.835)--(-4.561,2.834)--(-4.562,2.835)--cycle;
\filldraw[fill opacity=0.8,fill=gray!20,draw=none](-4.554,2.86)--(-3.727,1.102)--(-3.689,.988)--(-4.558,2.837)--cycle;
\draw(-3.689,.988)--(-4.558,2.837)--(-4.554,2.86)--(-3.727,1.102);
\filldraw[fill opacity=0.8,fill=gray!20,draw=none](-2.784,2.098)--(-2.791,2.122)--(-2.79,2.083)--(-2.781,2.066)--cycle;
\filldraw[fill opacity=0.8,fill=gray!20,draw=none](-2.784,2.098)--(-2.787,2.126)--(-2.791,2.153)--(-2.791,2.122)--cycle;
\filldraw[fill opacity=0.8,fill=gray!20,draw=none](-7.408,.933)--(-7.408,.897)--(-7.458,.897)--(-7.458,.919)--cycle;
\draw(-7.408,.933)--(-7.408,.897);
\draw(-7.458,.897)--(-7.458,.919);
\filldraw[fill opacity=0.8,fill=gray!20,draw=none](-7.408,.959)--(-7.408,.933)--(-7.458,.919)--cycle;
\draw(-7.408,.959)--(-7.408,.933);
\filldraw[fill opacity=0.8,fill=gray!20,draw=none](-7.379,.964)--(-7.373,.958)--(-7.373,.825)--(-7.408,.835)--(-7.408,.959)--cycle;
\draw(-7.373,.958)--(-7.373,.825)--(-7.408,.835)--(-7.408,.959);
\filldraw[fill opacity=0.8,fill=gray!20,draw=none](-7.366,.931)--(-7.358,.9)--(-7.358,.813)--(-7.373,.825)--(-7.373,.93)--cycle;
\draw(-7.358,.9)--(-7.358,.813)--(-7.373,.825)--(-7.373,.93);
\filldraw[fill opacity=0.8,fill=gray!20,draw=none](-7.372,.953)--(-7.366,.931)--(-7.373,.93)--(-7.373,.955)--cycle;
\draw(-7.373,.93)--(-7.373,.955);
\filldraw[fill opacity=0.8,fill=gray!20,draw=none](-7.372,.953)--(-7.373,.955)--(-7.373,.958)--cycle;
\draw(-7.373,.955)--(-7.373,.958);
\filldraw[fill opacity=0.8,fill=gray!20,draw=none](-7.369,.955)--(-7.384,.945)--(-7.385,.949)--(-7.379,.964)--cycle;
\draw(-7.369,.955)--(-7.384,.945);
\filldraw[fill opacity=0.8,fill=gray!20,draw=none](-7.385,.949)--(-7.384,.945)--(-7.387,.944)--cycle;
\draw(-7.384,.945)--(-7.387,.944);
\filldraw[fill opacity=0.8,fill=gray!20,draw=none](-7.36,.96)--(-7.369,.955)--(-7.379,.964)--(-7.374,.977)--cycle;
\draw(-7.36,.96)--(-7.369,.955);
\filldraw[fill opacity=0.8,fill=gray!20,draw=none](-7.354,.953)--(-7.358,.932)--(-7.372,.953)--(-7.369,.955)--cycle;
\draw(-7.372,.953)--(-7.369,.955);
\filldraw[fill opacity=0.8,fill=gray!20,draw=none](-7.354,.953)--(-7.369,.955)--(-7.352,.965)--cycle;
\draw(-7.369,.955)--(-7.352,.965);
\filldraw[fill opacity=0.8,fill=gray!20,draw=none](-7.358,.937)--(-7.358,.932)--(-7.372,.953)--(-7.373,.958)--(-7.373,.965)--cycle;
\draw(-7.358,.937)--(-7.358,.932);
\draw(-7.373,.958)--(-7.373,.965);
\filldraw[fill opacity=0.8,fill=gray!20,draw=none](-7.358,.932)--(-7.366,.931)--(-7.372,.953)--cycle;
\filldraw[fill opacity=0.8,fill=gray!20,draw=none](-7.358,.932)--(-7.358,.9)--(-7.366,.931)--cycle;
\draw(-7.358,.932)--(-7.358,.9);
\filldraw[fill opacity=0.8,fill=gray!20,draw=none](-7.36,.931)--(-7.367,.927)--(-7.387,.944)--(-7.384,.945)--cycle;
\draw(-7.36,.931)--(-7.367,.927);
\draw(-7.387,.944)--(-7.384,.945);
\filldraw[fill opacity=0.8,fill=gray!20,draw=none](-7.358,.932)--(-7.36,.931)--(-7.384,.945)--(-7.372,.953)--cycle;
\draw(-7.358,.932)--(-7.36,.931);
\draw(-7.384,.945)--(-7.372,.953);
\filldraw[fill opacity=0.8,fill=gray!20,draw=none](-7.354,.953)--(-7.329,.95)--(-7.358,.932)--cycle;
\draw(-7.329,.95)--(-7.358,.932);
\filldraw[fill opacity=0.8,fill=gray!20](-7.494,.779)--(-7.55,.78)--(-7.6,.787)--(-7.635,.796)--(-7.65,.808)--(-7.643,.821)--(-7.614,.832)--(-7.569,.839)--(-7.514,.843)--(-7.458,.841)--(-7.408,.835)--(-7.373,.825)--(-7.358,.813)--(-7.365,.801)--(-7.393,.79)--(-7.438,.782)--cycle;
\filldraw[fill opacity=0.8,fill=gray!20,draw=none](-7.365,.878)--(-7.365,.801)--(-7.358,.813)--(-7.358,.911)--cycle;
\draw(-7.365,.878)--(-7.365,.801)--(-7.358,.813)--(-7.358,.911);
\filldraw[fill opacity=0.8,fill=gray!20,draw=none](-7.359,.931)--(-7.361,.914)--(-7.358,.909)--(-7.358,.911)--(-7.358,.932)--cycle;
\draw(-7.358,.911)--(-7.358,.932);
\filldraw[fill opacity=0.8,fill=gray!20,draw=none](-7.358,.932)--(-7.359,.931)--(-7.367,.927)--(-7.36,.931)--cycle;
\draw(-7.367,.927)--(-7.36,.931);
\filldraw[fill opacity=0.8,fill=gray!20,draw=none](-7.358,.932)--(-7.36,.931)--(-7.358,.932)--cycle;
\draw(-7.36,.931)--(-7.358,.932);
\filldraw[fill opacity=0.8,fill=gray!20,draw=none](-7.357,.932)--(-7.358,.932)--(-7.358,.932)--cycle;
\filldraw[fill opacity=0.8,fill=gray!20,draw=none](-7.197,1.025)--(-7.306,.957)--(-7.352,.934)--(-7.357,.932)--(-7.358,.932)--(-7.106,1.088)--cycle;
\draw(-7.197,1.025)--(-7.306,.957);
\draw(-7.358,.932)--(-7.106,1.088);
\filldraw[fill opacity=0.8,fill=gray!20,draw=none](-7.361,.914)--(-7.365,.878)--(-7.358,.909)--cycle;
\filldraw[fill opacity=0.8,fill=gray!20,draw=none](-7.359,.931)--(-7.358,.932)--(-7.358,.937)--cycle;
\draw(-7.358,.932)--(-7.358,.937);
\filldraw[fill opacity=0.8,fill=gray!20,draw=none](-7.358,.932)--(-7.352,.934)--(-7.359,.931)--cycle;
\filldraw[fill opacity=0.8,fill=gray!20](-7.329,.863)--(-7.493,.934)--(-7.465,.984)--(-7.301,.912)--cycle;
\filldraw[fill opacity=0.8,fill=gray!20,draw=none](-7.394,.977)--(-7.379,.964)--(-7.408,.959)--cycle;
\filldraw[fill opacity=0.8,fill=gray!20,draw=none](-7.36,1.011)--(-7.364,1.005)--(-7.385,.992)--(-7.395,1.013)--(-7.394,1.045)--(-7.394,1.045)--cycle;
\draw(-7.364,1.005)--(-7.385,.992);
\draw(-7.394,1.045)--(-7.394,1.045);
\filldraw[fill opacity=0.8,fill=gray!20,draw=none](-7.385,.992)--(-7.396,.985)--(-7.395,1.013)--cycle;
\draw(-7.385,.992)--(-7.396,.985);
\filldraw[fill opacity=0.8,fill=gray!20,draw=none](-7.374,.977)--(-7.379,.964)--(-7.394,.977)--(-7.396,.985)--(-7.385,.992)--cycle;
\draw(-7.396,.985)--(-7.385,.992);
\filldraw[fill opacity=0.8,fill=gray!20,draw=none](-7.374,.977)--(-7.385,.992)--(-7.364,1.005)--cycle;
\draw(-7.385,.992)--(-7.364,1.005);
\filldraw[fill opacity=0.8,fill=gray!20,draw=none](-7.379,.964)--(-7.385,.949)--(-7.394,.977)--cycle;
\filldraw[fill opacity=0.8,fill=gray!20,draw=none](-7.352,.965)--(-7.36,.96)--(-7.374,.977)--(-7.364,1.005)--(-7.347,1.016)--cycle;
\draw(-7.352,.965)--(-7.36,.96);
\draw(-7.364,1.005)--(-7.347,1.016);
\filldraw[fill opacity=0.8,fill=gray!20,draw=none](-7.394,.977)--(-7.373,1.004)--(-7.373,.965)--(-7.379,.964)--cycle;
\draw(-7.373,1.004)--(-7.373,.965);
\filldraw[fill opacity=0.8,fill=gray!20,draw=none](-7.379,.964)--(-7.373,.965)--(-7.373,.958)--cycle;
\draw(-7.373,.965)--(-7.373,.958);
\filldraw[fill opacity=0.8,fill=gray!20,draw=none](-7.358,.937)--(-7.373,.965)--(-7.373,1.004)--cycle;
\draw(-7.373,.965)--(-7.373,1.004);
\filldraw[fill opacity=0.8,fill=gray!20,draw=none](-7.106,1.088)--(-7.329,.95)--(-7.354,.953)--(-7.352,.965)--(-7.021,1.171)--cycle;
\draw(-7.106,1.088)--(-7.329,.95);
\draw(-7.352,.965)--(-7.021,1.171);
\filldraw[fill opacity=0.8,fill=gray!20,draw=none](-7.306,.957)--(-7.327,.944)--(-7.352,.934)--cycle;
\draw(-7.306,.957)--(-7.327,.944);
\filldraw[fill opacity=0.8,fill=gray!20,draw=none](-7.301,.972)--(-7.327,.944)--(-7.313,.953)--cycle;
\draw(-7.327,.944)--(-7.313,.953);
\filldraw[fill opacity=0.8,fill=gray!20,draw=none](-7.249,1.003)--(-7.279,.991)--(-7.291,.983)--(-7.301,.972)--(-7.313,.953)--(-7.306,.957)--cycle;
\draw(-7.279,.991)--(-7.291,.983);
\draw(-7.313,.953)--(-7.306,.957);
\filldraw[fill opacity=0.8,fill=gray!20,draw=none](-5.028,2.37)--(-7.197,1.025)--(-7.106,1.088)--(-4.86,2.481)--cycle;
\draw(-5.028,2.37)--(-7.197,1.025);
\draw(-7.106,1.088)--(-4.86,2.481);
\filldraw[fill opacity=0.8,fill=gray!20,draw=none](-7.249,1.003)--(-7.306,.957)--(-7.197,1.025)--cycle;
\draw(-7.306,.957)--(-7.197,1.025);
\filldraw[fill opacity=0.8,fill=gray!20](-7.301,.912)--(-7.465,.984)--(-7.455,1.039)--(-7.291,.968)--cycle;
\filldraw[fill opacity=0.8,fill=gray!20,draw=none](-4.485,2.69)--(-4.489,2.692)--(-4.487,2.716)--(-4.482,2.719)--(-4.463,2.694)--cycle;
\draw(-4.482,2.719)--(-4.463,2.694)--(-4.485,2.69)--(-4.489,2.692);
\filldraw[fill opacity=0.8,fill=gray!20,draw=none](-4.438,2.728)--(-4.475,2.726)--(-4.466,2.746)--(-4.461,2.749)--cycle;
\draw(-4.438,2.728)--(-4.475,2.726);
\filldraw[fill opacity=0.8,fill=gray!20,draw=none](-4.478,2.714)--(-4.482,2.719)--(-4.475,2.726)--(-4.438,2.728)--(-4.432,2.722)--(-4.434,2.695)--cycle;
\draw(-4.478,2.714)--(-4.482,2.719);
\draw(-4.475,2.726)--(-4.438,2.728);
\draw(-4.432,2.722)--(-4.434,2.695);
\filldraw[fill opacity=0.8,fill=gray!20,draw=none](-4.463,2.727)--(-4.496,2.706)--(-4.497,2.707)--(-4.496,2.732)--cycle;
\draw(-4.463,2.727)--(-4.496,2.706);
\filldraw[fill opacity=0.8,fill=gray!20](-4.438,2.676)--(-4.485,2.69)--(-4.463,2.694)--(-4.438,2.676)--cycle;
\filldraw[fill opacity=0.8,fill=gray!20,draw=none](-4.469,2.752)--(-4.496,2.736)--(-4.497,2.737)--(-4.49,2.773)--cycle;
\draw(-4.469,2.752)--(-4.496,2.736);
\filldraw[fill opacity=0.8,fill=gray!20,draw=none](-4.46,2.743)--(-4.46,2.729)--(-4.463,2.727)--(-4.496,2.732)--(-4.496,2.736)--(-4.469,2.752)--cycle;
\draw(-4.46,2.729)--(-4.463,2.727);
\draw(-4.496,2.736)--(-4.469,2.752);
\filldraw[fill opacity=0.8,fill=gray!20,draw=none](-4.482,2.74)--(-4.507,2.731)--(-4.506,2.744)--cycle;
\draw(-4.482,2.74)--(-4.507,2.731);
\filldraw[fill opacity=0.8,fill=gray!20,draw=none](-4.487,2.689)--(-4.49,2.691)--(-4.489,2.692)--(-4.485,2.69)--cycle;
\draw(-4.489,2.692)--(-4.485,2.69)--(-4.487,2.689);
\filldraw[fill opacity=0.8,fill=gray!20,draw=none](-4.438,2.676)--(-4.481,2.682)--(-4.487,2.689)--(-4.485,2.69)--(-4.438,2.676)--cycle;
\draw(-4.487,2.689)--(-4.485,2.69)--(-4.438,2.676)--(-4.438,2.676)--(-4.481,2.682);
\filldraw[fill opacity=0.8,fill=gray!20,draw=none](-4.501,2.778)--(-4.513,2.759)--(-4.52,2.763)--(-4.531,2.804)--cycle;
\draw(-4.501,2.778)--(-4.513,2.759);
\filldraw[fill opacity=0.8,fill=gray!20,draw=none](-4.534,2.841)--(-4.527,2.857)--(-4.522,2.859)--(-4.516,2.856)--(-4.52,2.84)--(-4.533,2.821)--cycle;
\draw(-4.52,2.84)--(-4.533,2.821);
\filldraw[fill opacity=0.8,fill=gray!20,draw=none](-4.515,2.817)--(-4.518,2.793)--(-4.531,2.804)--(-4.534,2.817)--(-4.533,2.821)--(-4.525,2.833)--cycle;
\draw(-4.533,2.821)--(-4.525,2.833);
\filldraw[fill opacity=0.8,fill=gray!20,draw=none](-4.534,2.817)--(-4.535,2.819)--(-4.533,2.821)--cycle;
\draw(-4.535,2.819)--(-4.533,2.821);
\filldraw[fill opacity=0.8,fill=gray!20,draw=none](-4.531,2.829)--(-4.527,2.827)--(-4.52,2.816)--(-4.521,2.795)--(-4.536,2.808)--(-4.539,2.82)--cycle;
\filldraw[fill opacity=0.8,fill=gray!20,draw=none](-4.533,2.83)--(-4.531,2.829)--(-4.539,2.82)--(-4.542,2.831)--cycle;
\filldraw[fill opacity=0.8,fill=gray!20,draw=none](-4.512,2.787)--(-4.526,2.77)--(-4.529,2.775)--(-4.536,2.808)--cycle;
\draw(-4.512,2.787)--(-4.526,2.77);
\filldraw[fill opacity=0.8,fill=gray!20,draw=none](-4.533,2.804)--(-4.543,2.813)--(-4.543,2.832)--(-4.535,2.83)--cycle;
\filldraw[fill opacity=0.8,fill=gray!20,draw=none](-4.463,2.694)--(-4.478,2.714)--(-4.434,2.695)--cycle;
\draw(-4.434,2.695)--(-4.463,2.694)--(-4.478,2.714);
\filldraw[fill opacity=0.8,fill=gray!20,draw=none](-4.468,2.719)--(-4.489,2.705)--(-4.496,2.706)--(-4.46,2.729)--cycle;
\draw(-4.496,2.706)--(-4.46,2.729);
\filldraw[fill opacity=0.8,fill=gray!20,draw=none](-4.477,2.741)--(-4.504,2.73)--(-4.508,2.73)--(-4.507,2.731)--(-4.474,2.743)--cycle;
\draw(-4.507,2.731)--(-4.474,2.743);
\filldraw[fill opacity=0.8,fill=gray!20,draw=none](-4.483,2.764)--(-4.503,2.757)--(-4.497,2.778)--cycle;
\draw(-4.483,2.764)--(-4.503,2.757);
\filldraw[fill opacity=0.8,fill=gray!20,draw=none](-4.472,2.753)--(-4.474,2.743)--(-4.482,2.74)--(-4.506,2.744)--(-4.505,2.755)--(-4.503,2.757)--(-4.483,2.764)--cycle;
\draw(-4.474,2.743)--(-4.482,2.74);
\draw(-4.503,2.757)--(-4.483,2.764);
\filldraw[fill opacity=0.8,fill=gray!20,draw=none](-4.485,2.764)--(-4.476,2.74)--(-4.513,2.759)--(-4.501,2.778)--cycle;
\draw(-4.513,2.759)--(-4.501,2.778);
\filldraw[fill opacity=0.8,fill=gray!20,draw=none](-4.5,2.776)--(-4.495,2.756)--(-4.525,2.772)--(-4.512,2.787)--cycle;
\draw(-4.525,2.772)--(-4.512,2.787);
\filldraw[fill opacity=0.8,fill=gray!20,draw=none](-4.513,2.759)--(-4.518,2.752)--(-4.52,2.763)--cycle;
\draw(-4.513,2.759)--(-4.518,2.752);
\filldraw[fill opacity=0.8,fill=gray!20,draw=none](-4.477,2.741)--(-4.504,2.73)--(-4.517,2.75)--(-4.518,2.752)--(-4.513,2.759)--cycle;
\draw(-4.518,2.752)--(-4.513,2.759);
\filldraw[fill opacity=0.8,fill=gray!20,draw=none](-4.472,2.677)--(-4.476,2.677)--(-4.481,2.682)--(-4.471,2.68)--cycle;
\draw(-4.472,2.677)--(-4.476,2.677);
\draw(-4.481,2.682)--(-4.471,2.68);
\filldraw[fill opacity=0.8,fill=gray!20,draw=none](-4.477,2.741)--(-4.476,2.74)--(-4.473,2.73)--(-4.489,2.706)--(-4.504,2.73)--cycle;
\draw(-4.473,2.73)--(-4.489,2.706);
\filldraw[fill opacity=0.8,fill=gray!20,draw=none](-4.477,2.741)--(-4.489,2.733)--(-4.494,2.731)--(-4.501,2.729)--(-4.504,2.73)--cycle;
\draw(-4.494,2.731)--(-4.501,2.729);
\filldraw[fill opacity=0.8,fill=gray!20,draw=none](-4.495,2.756)--(-4.49,2.736)--(-4.494,2.731)--(-4.501,2.73)--(-4.526,2.77)--(-4.525,2.772)--cycle;
\draw(-4.49,2.736)--(-4.494,2.731);
\draw(-4.526,2.77)--(-4.525,2.772);
\filldraw[fill opacity=0.8,fill=gray!20,draw=none](-4.523,2.795)--(-4.543,2.805)--(-4.543,2.813)--cycle;
\draw(-4.523,2.795)--(-4.543,2.805);
\filldraw[fill opacity=0.8,fill=gray!20,draw=none](-4.506,2.777)--(-4.532,2.793)--(-4.54,2.801)--(-4.523,2.795)--cycle;
\filldraw[fill opacity=0.8,fill=gray!20,draw=none](-4.54,2.801)--(-4.543,2.805)--(-4.523,2.795)--cycle;
\draw(-4.543,2.805)--(-4.523,2.795);
\filldraw[fill opacity=0.8,fill=gray!20,draw=none](-4.506,2.777)--(-4.528,2.788)--(-4.532,2.793)--cycle;
\draw(-4.506,2.777)--(-4.528,2.788);
\filldraw[fill opacity=0.8,fill=gray!20,draw=none](-4.558,2.837)--(-3.689,.988)--(-3.659,.941)--(-4.544,2.822)--cycle;
\draw(-3.659,.941)--(-4.544,2.822)--(-4.558,2.837)--(-3.689,.988);
\filldraw[fill opacity=0.8,fill=gray!20,draw=none](-2.832,2.384)--(-2.809,2.208)--(-2.796,2.241)--(-2.819,2.415)--cycle;
\draw(-2.796,2.241)--(-2.819,2.415)--(-2.832,2.384)--(-2.809,2.208);
\filldraw[fill opacity=0.8,fill=gray!20,draw=none](-2.792,2.178)--(-2.791,2.153)--(-2.787,2.126)--cycle;
\filldraw[fill opacity=0.8,fill=gray!20,draw=none](-2.743,2.221)--(-2.755,2.191)--(-2.744,2.107)--(-2.737,2.181)--cycle;
\draw(-2.755,2.191)--(-2.744,2.107);
\filldraw[fill opacity=0.8,fill=gray!20,draw=none](-2.854,2.122)--(-2.842,2.027)--(-2.785,2.024)--(-2.794,2.09)--cycle;
\draw(-2.854,2.122)--(-2.842,2.027);
\draw(-2.785,2.024)--(-2.794,2.09);
\filldraw[fill opacity=0.8,fill=gray!20,draw=none](-2.802,2.027)--(-2.765,1.742)--(-2.705,1.799)--(-2.733,2.022)--cycle;
\draw(-2.802,2.027)--(-2.765,1.742);
\draw(-2.705,1.799)--(-2.733,2.022);
\filldraw[fill opacity=0.8,fill=gray!20,draw=none](-7.478,.894)--(-7.514,.851)--(-7.514,.889)--cycle;
\draw(-7.514,.851)--(-7.514,.889);
\filldraw[fill opacity=0.8,fill=gray!20,draw=none](-7.502,.865)--(-7.478,.894)--(-7.458,.897)--(-7.458,.857)--cycle;
\draw(-7.458,.897)--(-7.458,.857);
\filldraw[fill opacity=0.8,fill=gray!20,draw=none](-7.408,.897)--(-7.408,.835)--(-7.458,.841)--(-7.458,.897)--cycle;
\draw(-7.408,.897)--(-7.408,.835)--(-7.458,.841)--(-7.458,.897);
\filldraw[fill opacity=0.8,fill=gray!20,draw=none](-7.458,.919)--(-7.478,.894)--(-7.514,.889)--cycle;
\filldraw[fill opacity=0.8,fill=gray!20,draw=none](-7.478,.894)--(-7.458,.919)--(-7.458,.897)--cycle;
\draw(-7.458,.919)--(-7.458,.897);
\filldraw[fill opacity=0.8,fill=gray!20,draw=none](-7.375,.849)--(-7.393,.798)--(-7.393,.79)--(-7.365,.801)--(-7.365,.856)--cycle;
\draw(-7.393,.798)--(-7.393,.79)--(-7.365,.801)--(-7.365,.856);
\filldraw[fill opacity=0.8,fill=gray!20,draw=none](-7.375,.849)--(-7.365,.856)--(-7.365,.878)--cycle;
\draw(-7.365,.856)--(-7.365,.878);
\filldraw[fill opacity=0.8,fill=gray!20,draw=none](-7.393,.836)--(-7.375,.849)--(-7.365,.878)--cycle;
\filldraw[fill opacity=0.8,fill=gray!20,draw=none](-7.393,.836)--(-7.393,.798)--(-7.375,.849)--cycle;
\draw(-7.393,.836)--(-7.393,.798);
\filldraw[fill opacity=0.8,fill=gray!20](-7.372,.827)--(-7.536,.898)--(-7.493,.934)--(-7.329,.863)--cycle;
\filldraw[fill opacity=0.8,fill=gray!20,draw=none](-2.794,2.206)--(-2.796,2.183)--(-2.791,2.153)--(-2.792,2.178)--cycle;
\filldraw[fill opacity=0.8,fill=gray!20,draw=none](-3.221,1.928)--(-3.2,1.765)--(-3.088,1.613)--(-3.119,1.849)--cycle;
\draw(-3.221,1.928)--(-3.2,1.765);
\draw(-3.088,1.613)--(-3.119,1.849);
\filldraw[fill opacity=0.8,fill=gray!20,draw=none](-2.976,1.634)--(-2.982,1.633)--(-3.109,1.622)--(-3.214,1.627)--(-3.234,1.645)--cycle;
\draw(-2.976,1.634)--(-2.982,1.633)--(-3.109,1.622)--(-3.214,1.627);
\filldraw[fill opacity=0.8,fill=gray!20,draw=none](-3.336,1.945)--(-3.298,1.646)--(-3.22,1.627)--(-3.233,1.732)--cycle;
\draw(-3.336,1.945)--(-3.298,1.646)--(-3.22,1.627)--(-3.233,1.732);
\filldraw[fill opacity=0.8,fill=gray!20,draw=none](-2.797,2.239)--(-2.809,2.208)--(-2.801,2.141)--(-2.794,2.206)--cycle;
\draw(-2.809,2.208)--(-2.801,2.141);
\filldraw[fill opacity=0.8,fill=gray!20,draw=none](-3.35,.89)--(-3.347,.901)--(-3.47,.931)--(-3.488,.884)--(-3.373,.856)--cycle;
\draw(-3.347,.901)--(-3.47,.931)--(-3.488,.884)--(-3.373,.856);
\filldraw[fill opacity=0.8,fill=gray!20,draw=none](-3.358,.985)--(-3.442,.977)--(-3.344,.953)--cycle;
\draw(-3.442,.977)--(-3.344,.953);
\filldraw[fill opacity=0.8,fill=gray!20,draw=none](-7.395,1.013)--(-7.406,1.037)--(-7.394,1.045)--cycle;
\draw(-7.406,1.037)--(-7.394,1.045);
\filldraw[fill opacity=0.8,fill=gray!20,draw=none](-7.394,1.045)--(-7.394,1.045)--(-7.406,1.037)--(-7.415,1.071)--cycle;
\draw(-7.394,1.045)--(-7.406,1.037);
\filldraw[fill opacity=0.8,fill=gray!20,draw=none](-7.394,1.045)--(-7.394,1.045)--(-7.394,1.045)--cycle;
\draw(-7.394,1.045)--(-7.394,1.045);
\filldraw[fill opacity=0.8,fill=gray!20,draw=none](-7.334,1.051)--(-7.36,1.011)--(-7.394,1.045)--(-7.361,1.065)--cycle;
\draw(-7.394,1.045)--(-7.361,1.065);
\filldraw[fill opacity=0.8,fill=gray!20,draw=none](-7.361,1.065)--(-7.394,1.045)--(-7.394,1.045)--(-7.386,1.08)--cycle;
\draw(-7.361,1.065)--(-7.394,1.045);
\filldraw[fill opacity=0.8,fill=gray!20,draw=none](-7.386,1.08)--(-7.394,1.045)--(-7.415,1.071)--(-7.419,1.089)--(-7.411,1.094)--cycle;
\draw(-7.419,1.089)--(-7.411,1.094);
\filldraw[fill opacity=0.8,fill=gray!20,draw=none](-7.358,1.009)--(-7.364,1.005)--(-7.36,1.011)--cycle;
\draw(-7.358,1.009)--(-7.364,1.005);
\filldraw[fill opacity=0.8,fill=gray!20,draw=none](-7.373,1.042)--(-7.373,1.004)--(-7.408,1.068)--cycle;
\draw(-7.373,1.042)--(-7.373,1.004);
\filldraw[fill opacity=0.8,fill=gray!20,draw=none](-7.289,1.11)--(-7.361,1.065)--(-7.386,1.08)--(-7.379,1.114)--(-7.364,1.124)--cycle;
\draw(-7.289,1.11)--(-7.361,1.065);
\draw(-7.379,1.114)--(-7.364,1.124);
\filldraw[fill opacity=0.8,fill=gray!20,draw=none](-7.386,1.08)--(-7.411,1.094)--(-7.379,1.114)--cycle;
\draw(-7.411,1.094)--(-7.379,1.114);
\filldraw[fill opacity=0.8,fill=gray!20,draw=none](-7.364,1.124)--(-7.411,1.094)--(-7.42,1.1)--(-7.421,1.109)--(-7.41,1.131)--cycle;
\draw(-7.364,1.124)--(-7.411,1.094);
\filldraw[fill opacity=0.8,fill=gray!20,draw=none](-7.373,1.118)--(-7.373,1.042)--(-7.408,1.068)--(-7.408,1.13)--cycle;
\draw(-7.373,1.118)--(-7.373,1.042);
\draw(-7.408,1.068)--(-7.408,1.13);
\filldraw[fill opacity=0.8,fill=gray!20,draw=none](-7.358,1.046)--(-7.373,1.004)--(-7.373,1.042)--cycle;
\draw(-7.373,1.004)--(-7.373,1.042);
\filldraw[fill opacity=0.8,fill=gray!20,draw=none](-7.334,1.051)--(-7.361,1.065)--(-7.301,1.102)--cycle;
\draw(-7.361,1.065)--(-7.301,1.102);
\filldraw[fill opacity=0.8,fill=gray!20,draw=none](-5.076,2.482)--(-7.289,1.11)--(-7.364,1.124)--(-5.284,2.413)--cycle;
\draw(-5.076,2.482)--(-7.289,1.11);
\draw(-7.364,1.124)--(-5.284,2.413);
\filldraw[fill opacity=0.8,fill=gray!20,draw=none](-7.252,1.193)--(-7.364,1.124)--(-7.409,1.13)--(-7.408,1.135)--(-7.399,1.155)--(-7.342,1.19)--cycle;
\draw(-7.252,1.193)--(-7.364,1.124);
\draw(-7.399,1.155)--(-7.342,1.19);
\filldraw[fill opacity=0.8,fill=gray!20,draw=none](-7.371,1.187)--(-7.342,1.19)--(-7.399,1.155)--cycle;
\draw(-7.342,1.19)--(-7.399,1.155);
\filldraw[fill opacity=0.8,fill=gray!20,draw=none](-7.358,1.17)--(-7.358,1.046)--(-7.373,1.042)--(-7.373,1.182)--cycle;
\draw(-7.373,1.042)--(-7.373,1.182)--(-7.358,1.17)--(-7.358,1.046);
\filldraw[fill opacity=0.8,fill=gray!20,draw=none](-7.301,.972)--(-7.291,.983)--(-7.297,.98)--cycle;
\draw(-7.291,.983)--(-7.297,.98);
\filldraw[fill opacity=0.8,fill=gray!20,draw=none](-7.287,.99)--(-7.295,.99)--(-7.297,.98)--(-7.291,.983)--cycle;
\draw(-7.297,.98)--(-7.291,.983);
\filldraw[fill opacity=0.8,fill=gray!20](-7.291,.968)--(-7.455,1.039)--(-7.465,1.093)--(-7.301,1.021)--cycle;
\filldraw[fill opacity=0.8,fill=gray!20,draw=none](-7.717,1.633)--(-7.737,1.624)--(-7.737,1.646)--cycle;
\draw(-7.737,1.624)--(-7.737,1.646);
\filldraw[fill opacity=0.8,fill=gray!20,draw=none](-3.353,2.004)--(-3.363,2.044)--(-3.373,2.043)--(-3.362,1.979)--cycle;
\filldraw[fill opacity=0.8,fill=gray!20,draw=none](-3.561,1.304)--(-3.461,1.091)--(-3.448,1.081)--(-3.443,1.083)--(-3.475,1.152)--cycle;
\draw(-3.561,1.304)--(-3.461,1.091);
\draw(-3.443,1.083)--(-3.475,1.152);
\filldraw[fill opacity=0.8,fill=gray!20,draw=none](-7.766,1.752)--(-7.766,1.666)--(-7.773,1.7)--(-7.773,1.712)--cycle;
\draw(-7.766,1.752)--(-7.766,1.666);
\draw(-7.773,1.7)--(-7.773,1.712);
\filldraw[fill opacity=0.8,fill=gray!20,draw=none](-7.768,1.698)--(-7.771,1.704)--(-7.736,1.72)--(-7.71,1.675)--(-7.741,1.662)--cycle;
\draw(-7.771,1.704)--(-7.736,1.72)--(-7.71,1.675)--(-7.741,1.662);
\filldraw[fill opacity=0.8,fill=gray!20,draw=none](-7.723,1.788)--(-7.723,1.678)--(-7.673,1.626)--(-7.673,1.828)--cycle;
\draw(-7.723,1.788)--(-7.723,1.678);
\draw(-7.673,1.626)--(-7.673,1.828);
\filldraw[fill opacity=0.8,fill=gray!20,draw=none](-7.488,1.848)--(-7.488,1.69)--(-7.483,1.69)--(-7.481,1.705)--(-7.481,1.814)--cycle;
\draw(-7.488,1.848)--(-7.488,1.69);
\draw(-7.481,1.705)--(-7.481,1.814);
\filldraw[fill opacity=0.8,fill=gray!20,draw=none](-7.561,1.711)--(-7.586,1.733)--(-7.639,1.715)--(-7.62,1.673)--(-7.555,1.696)--cycle;
\draw(-7.586,1.733)--(-7.639,1.715)--(-7.62,1.673)--(-7.555,1.696);
\filldraw[fill opacity=0.8,fill=gray!20,draw=none](-7.673,1.685)--(-7.673,1.642)--(-7.616,1.661)--(-7.616,1.704)--cycle;
\draw(-7.673,1.685)--(-7.673,1.642);
\draw(-7.616,1.661)--(-7.616,1.704);
\filldraw[fill opacity=0.8,fill=gray!20,draw=none](-7.64,1.613)--(-7.641,1.612)--(-7.641,1.611)--(-7.613,1.626)--(-7.616,1.625)--(-7.624,1.622)--cycle;
\draw(-7.616,1.625)--(-7.624,1.622);
\filldraw[fill opacity=0.8,fill=gray!20](-7.666,1.654)--(-7.83,1.725)--(-7.801,1.682)--(-7.637,1.61)--cycle;
\filldraw[fill opacity=0.8,fill=gray!20,draw=none](-2.796,2.183)--(-2.801,2.141)--(-2.794,2.09)--(-2.79,2.083)--(-2.791,2.153)--cycle;
\draw(-2.801,2.141)--(-2.794,2.09);
\filldraw[fill opacity=0.8,fill=gray!20,draw=none](-7.54,.871)--(-7.514,.867)--(-7.514,.843)--(-7.569,.839)--(-7.569,.849)--cycle;
\draw(-7.514,.867)--(-7.514,.843)--(-7.569,.839)--(-7.569,.849);
\filldraw[fill opacity=0.8,fill=gray!20,draw=none](-7.502,.865)--(-7.458,.857)--(-7.458,.841)--(-7.514,.843)--(-7.514,.851)--cycle;
\draw(-7.458,.857)--(-7.458,.841)--(-7.514,.843)--(-7.514,.851);
\filldraw[fill opacity=0.8,fill=gray!20,draw=none](-7.54,.871)--(-7.514,.889)--(-7.514,.867)--cycle;
\draw(-7.514,.889)--(-7.514,.867);
\filldraw[fill opacity=0.8,fill=gray!20,draw=none](-7.514,.889)--(-7.54,.871)--(-7.569,.875)--cycle;
\filldraw[fill opacity=0.8,fill=gray!20,draw=none](-7.54,.871)--(-7.569,.849)--(-7.569,.875)--cycle;
\draw(-7.569,.849)--(-7.569,.875);
\filldraw[fill opacity=0.8,fill=gray!20,draw=none](-7.438,.792)--(-7.438,.782)--(-7.398,.789)--(-7.393,.798)--(-7.393,.836)--cycle;
\draw(-7.438,.792)--(-7.438,.782)--(-7.398,.789);
\draw(-7.393,.798)--(-7.393,.836);
\filldraw[fill opacity=0.8,fill=gray!20,draw=none](-7.438,.818)--(-7.414,.816)--(-7.393,.836)--cycle;
\filldraw[fill opacity=0.8,fill=gray!20,draw=none](-7.438,.818)--(-7.438,.792)--(-7.414,.816)--cycle;
\draw(-7.438,.818)--(-7.438,.792);
\filldraw[fill opacity=0.8,fill=gray!20](-7.422,.811)--(-7.586,.882)--(-7.536,.898)--(-7.372,.827)--cycle;
\filldraw[fill opacity=0.8,fill=gray!20,draw=none](-7.737,1.646)--(-7.737,1.6)--(-7.766,1.589)--(-7.766,1.644)--cycle;
\draw(-7.737,1.646)--(-7.737,1.6)--(-7.766,1.589)--(-7.766,1.644);
\filldraw[fill opacity=0.8,fill=gray!20,draw=none](-7.424,1.102)--(-7.429,1.091)--(-7.458,1.121)--cycle;
\filldraw[fill opacity=0.8,fill=gray!20,draw=none](-7.429,1.135)--(-7.421,1.109)--(-7.424,1.102)--(-7.458,1.121)--(-7.458,1.142)--cycle;
\draw(-7.458,1.121)--(-7.458,1.142);
\filldraw[fill opacity=0.8,fill=gray!20,draw=none](-7.415,1.071)--(-7.426,1.085)--(-7.419,1.089)--cycle;
\draw(-7.426,1.085)--(-7.419,1.089);
\filldraw[fill opacity=0.8,fill=gray!20,draw=none](-7.424,1.102)--(-7.42,1.1)--(-7.419,1.089)--(-7.426,1.085)--(-7.429,1.091)--cycle;
\draw(-7.419,1.089)--(-7.426,1.085);
\filldraw[fill opacity=0.8,fill=gray!20,draw=none](-7.411,1.094)--(-7.419,1.089)--(-7.42,1.1)--cycle;
\draw(-7.411,1.094)--(-7.419,1.089);
\filldraw[fill opacity=0.8,fill=gray!20,draw=none](-7.424,1.102)--(-7.417,1.099)--(-7.408,1.068)--(-7.429,1.091)--cycle;
\filldraw[fill opacity=0.8,fill=gray!20,draw=none](-7.424,1.102)--(-7.429,1.091)--(-7.444,1.115)--cycle;
\filldraw[fill opacity=0.8,fill=gray!20,draw=none](-7.421,1.109)--(-7.42,1.1)--(-7.424,1.102)--cycle;
\filldraw[fill opacity=0.8,fill=gray!20,draw=none](-7.421,1.109)--(-7.417,1.099)--(-7.424,1.102)--cycle;
\filldraw[fill opacity=0.8,fill=gray!20,draw=none](-7.417,1.099)--(-7.429,1.135)--(-7.408,1.13)--(-7.408,1.094)--cycle;
\draw(-7.408,1.13)--(-7.408,1.094);
\filldraw[fill opacity=0.8,fill=gray!20,draw=none](-7.422,1.132)--(-7.421,1.109)--(-7.424,1.102)--(-7.444,1.115)--(-7.45,1.123)--(-7.432,1.134)--cycle;
\draw(-7.45,1.123)--(-7.432,1.134);
\filldraw[fill opacity=0.8,fill=gray!20,draw=none](-7.417,1.099)--(-7.408,1.094)--(-7.408,1.068)--cycle;
\draw(-7.408,1.094)--(-7.408,1.068);
\filldraw[fill opacity=0.8,fill=gray!20,draw=none](-7.359,1.073)--(-7.365,1.08)--(-7.358,1.046)--cycle;
\filldraw[fill opacity=0.8,fill=gray!20,draw=none](-7.359,1.073)--(-7.358,1.046)--(-7.358,1.072)--cycle;
\draw(-7.358,1.046)--(-7.358,1.072);
\filldraw[fill opacity=0.8,fill=gray!20,draw=none](-7.364,1.16)--(-7.359,1.073)--(-7.358,1.072)--(-7.358,1.17)--cycle;
\draw(-7.358,1.072)--(-7.358,1.17)--(-7.364,1.16);
\filldraw[fill opacity=0.8,fill=gray!20,draw=none](-7.359,1.073)--(-7.364,1.16)--(-7.365,1.158)--(-7.365,1.08)--cycle;
\draw(-7.364,1.16)--(-7.365,1.158)--(-7.365,1.08);
\filldraw[fill opacity=0.8,fill=gray!20](-7.301,1.021)--(-7.465,1.093)--(-7.493,1.136)--(-7.329,1.065)--cycle;
\filldraw[fill opacity=0.8,fill=gray!20,draw=none](-7.737,1.646)--(-7.766,1.644)--(-7.766,1.666)--cycle;
\draw(-7.766,1.644)--(-7.766,1.666);
\filldraw[fill opacity=0.8,fill=gray!20,draw=none](-7.722,1.921)--(-7.692,1.929)--(-7.704,1.934)--cycle;
\draw(-7.692,1.929)--(-7.704,1.934);
\filldraw[fill opacity=0.8,fill=gray!20,draw=none](-3.67,.963)--(-3.659,.941)--(-3.647,.963)--cycle;
\draw(-3.67,.963)--(-3.659,.941);
\filldraw[fill opacity=0.8,fill=gray!20,draw=none](-3.659,.941)--(-3.634,.948)--(-3.644,.969)--cycle;
\draw(-3.634,.948)--(-3.644,.969);
\filldraw[fill opacity=0.8,fill=gray!20,draw=none](-3.305,2.042)--(-3.313,2.042)--(-3.302,2.002)--(-3.296,2.019)--cycle;
\filldraw[fill opacity=0.8,fill=gray!20,draw=none](-3.448,1.081)--(-3.461,1.091)--(-3.455,1.078)--cycle;
\draw(-3.461,1.091)--(-3.455,1.078);
\filldraw[fill opacity=0.8,fill=gray!20,draw=none](-7.728,1.852)--(-7.698,1.87)--(-7.672,1.881)--(-7.71,1.857)--(-7.741,1.844)--cycle;
\draw(-7.698,1.87)--(-7.672,1.881)--(-7.71,1.857)--(-7.741,1.844);
\filldraw[fill opacity=0.8,fill=gray!20,draw=none](-7.654,1.886)--(-7.667,1.882)--(-7.676,1.874)--(-7.667,1.877)--cycle;
\draw(-7.654,1.886)--(-7.667,1.882)--(-7.676,1.874)--(-7.667,1.877);
\filldraw[fill opacity=0.8,fill=gray!20,draw=none](-7.68,1.876)--(-7.672,1.881)--(-7.66,1.882)--cycle;
\draw(-7.68,1.876)--(-7.672,1.881)--(-7.66,1.882);
\filldraw[fill opacity=0.8,fill=gray!20,draw=none](-7.641,1.88)--(-7.627,1.886)--(-7.672,1.881)--(-7.683,1.876)--cycle;
\draw(-7.641,1.88)--(-7.627,1.886)--(-7.672,1.881)--(-7.683,1.876);
\filldraw[fill opacity=0.8,fill=gray!20,draw=none](-7.611,1.902)--(-7.654,1.886)--(-7.667,1.877)--(-7.615,1.896)--cycle;
\draw(-7.611,1.902)--(-7.654,1.886);
\draw(-7.667,1.877)--(-7.615,1.896);
\filldraw[fill opacity=0.8,fill=gray!20,draw=none](-7.646,1.885)--(-7.676,1.874)--(-7.677,1.849)--(-7.629,1.866)--cycle;
\draw(-7.646,1.885)--(-7.676,1.874)--(-7.677,1.849)--(-7.629,1.866);
\filldraw[fill opacity=0.8,fill=gray!20,draw=none](-7.611,1.858)--(-7.582,1.871)--(-7.627,1.886)--(-7.65,1.876)--cycle;
\draw(-7.611,1.858)--(-7.582,1.871)--(-7.627,1.886)--(-7.65,1.876);
\filldraw[fill opacity=0.8,fill=gray!20,draw=none](-7.615,1.896)--(-7.646,1.885)--(-7.629,1.866)--(-7.582,1.883)--cycle;
\draw(-7.615,1.896)--(-7.646,1.885);
\draw(-7.629,1.866)--(-7.582,1.883);
\filldraw[fill opacity=0.8,fill=gray!20,draw=none](-7.516,1.868)--(-7.516,1.786)--(-7.488,1.812)--(-7.488,1.848)--cycle;
\draw(-7.516,1.868)--(-7.516,1.786);
\draw(-7.488,1.812)--(-7.488,1.848);
\filldraw[fill opacity=0.8,fill=gray!20,draw=none](-7.618,1.87)--(-7.677,1.849)--(-7.671,1.81)--(-7.597,1.836)--cycle;
\draw(-7.618,1.87)--(-7.677,1.849)--(-7.671,1.81)--(-7.597,1.836);
\filldraw[fill opacity=0.8,fill=gray!20,draw=none](-7.582,1.822)--(-7.543,1.839)--(-7.582,1.871)--(-7.617,1.855)--cycle;
\draw(-7.582,1.822)--(-7.543,1.839)--(-7.582,1.871)--(-7.617,1.855);
\filldraw[fill opacity=0.8,fill=gray!20,draw=none](-7.532,1.869)--(-7.582,1.883)--(-7.618,1.87)--(-7.597,1.836)--(-7.543,1.855)--cycle;
\draw(-7.582,1.883)--(-7.618,1.87);
\draw(-7.597,1.836)--(-7.543,1.855);
\filldraw[fill opacity=0.8,fill=gray!20,draw=none](-7.535,1.733)--(-7.509,1.744)--(-7.518,1.794)--(-7.553,1.779)--cycle;
\draw(-7.535,1.733)--(-7.509,1.744)--(-7.518,1.794)--(-7.553,1.779);
\filldraw[fill opacity=0.8,fill=gray!20,draw=none](-7.573,1.797)--(-7.561,1.785)--(-7.553,1.779)--(-7.518,1.794)--(-7.543,1.839)--(-7.582,1.822)--cycle;
\draw(-7.553,1.779)--(-7.518,1.794)--(-7.543,1.839)--(-7.582,1.822);
\filldraw[fill opacity=0.8,fill=gray!20,draw=none](-7.532,1.869)--(-7.543,1.855)--(-7.561,1.811)--(-7.561,1.8)--(-7.553,1.779)--(-7.525,1.761)--(-7.516,1.786)--(-7.516,1.868)--cycle;
\draw(-7.561,1.811)--(-7.561,1.8);
\draw(-7.516,1.786)--(-7.516,1.868);
\filldraw[fill opacity=0.8,fill=gray!20,draw=none](-7.532,1.869)--(-7.543,1.855)--(-7.516,1.865)--cycle;
\draw(-7.543,1.855)--(-7.516,1.865);
\filldraw[fill opacity=0.8,fill=gray!20,draw=none](-7.532,1.869)--(-7.537,1.87)--(-7.543,1.855)--cycle;
\filldraw[fill opacity=0.8,fill=gray!20,draw=none](-7.573,1.797)--(-7.561,1.811)--(-7.543,1.855)--(-7.589,1.839)--cycle;
\draw(-7.543,1.855)--(-7.589,1.839);
\filldraw[fill opacity=0.8,fill=gray!20,draw=none](-7.537,1.87)--(-7.561,1.871)--(-7.561,1.811)--cycle;
\draw(-7.561,1.871)--(-7.561,1.811);
\filldraw[fill opacity=0.8,fill=gray!20,draw=none](-7.561,1.8)--(-7.561,1.7)--(-7.549,1.692)--(-7.535,1.733)--cycle;
\draw(-7.561,1.8)--(-7.561,1.7);
\filldraw[fill opacity=0.8,fill=gray!20,draw=none](-7.52,1.863)--(-7.543,1.855)--(-7.561,1.811)--cycle;
\draw(-7.52,1.863)--(-7.543,1.855);
\filldraw[fill opacity=0.8,fill=gray!20,draw=none](-7.641,1.845)--(-7.611,1.858)--(-7.65,1.876)--(-7.661,1.871)--cycle;
\draw(-7.641,1.845)--(-7.611,1.858);
\draw(-7.65,1.876)--(-7.661,1.871);
\filldraw[fill opacity=0.8,fill=gray!20,draw=none](-7.573,1.797)--(-7.589,1.839)--(-7.671,1.81)--(-7.657,1.763)--(-7.577,1.791)--cycle;
\draw(-7.589,1.839)--(-7.671,1.81)--(-7.657,1.763)--(-7.577,1.791);
\filldraw[fill opacity=0.8,fill=gray!20,draw=none](-7.575,1.799)--(-7.586,1.733)--(-7.561,1.711)--(-7.561,1.785)--cycle;
\draw(-7.561,1.711)--(-7.561,1.785);
\filldraw[fill opacity=0.8,fill=gray!20,draw=none](-7.575,1.799)--(-7.573,1.797)--(-7.582,1.822)--(-7.588,1.819)--cycle;
\draw(-7.582,1.822)--(-7.588,1.819);
\filldraw[fill opacity=0.8,fill=gray!20,draw=none](-7.562,1.804)--(-7.574,1.809)--(-7.575,1.799)--(-7.561,1.785)--(-7.561,1.8)--cycle;
\draw(-7.561,1.785)--(-7.561,1.8);
\filldraw[fill opacity=0.8,fill=gray!20,draw=none](-7.579,1.867)--(-7.561,1.8)--(-7.561,1.871)--cycle;
\draw(-7.561,1.8)--(-7.561,1.871);
\filldraw[fill opacity=0.8,fill=gray!20,draw=none](-7.574,1.809)--(-7.562,1.804)--(-7.57,1.832)--cycle;
\filldraw[fill opacity=0.8,fill=gray!20,draw=none](-7.486,1.854)--(-7.501,1.862)--(-7.516,1.865)--(-7.52,1.863)--(-7.561,1.811)--(-7.563,1.806)--(-7.563,1.804)--(-7.557,1.798)--(-7.507,1.816)--cycle;
\draw(-7.516,1.865)--(-7.52,1.863);
\draw(-7.557,1.798)--(-7.507,1.816);
\filldraw[fill opacity=0.8,fill=gray!20,draw=none](-7.495,1.859)--(-7.637,1.921)--(-7.692,1.929)--(-7.545,1.864)--cycle;
\draw(-7.692,1.929)--(-7.545,1.864)--(-7.495,1.859)--(-7.637,1.921);
\filldraw[fill opacity=0.8,fill=gray!20,draw=none](-7.601,1.914)--(-7.637,1.921)--(-7.637,1.959)--cycle;
\draw(-7.637,1.921)--(-7.637,1.959);
\filldraw[fill opacity=0.8,fill=gray!20,draw=none](-7.637,1.968)--(-7.637,1.921)--(-7.692,1.954)--(-7.692,1.964)--cycle;
\draw(-7.692,1.954)--(-7.692,1.964)--(-7.637,1.968)--(-7.637,1.921);
\filldraw[fill opacity=0.8,fill=gray!20,draw=none](-7.795,1.759)--(-7.839,1.779)--(-7.83,1.725)--(-7.81,1.717)--cycle;
\draw(-7.795,1.759)--(-7.839,1.779)--(-7.83,1.725)--(-7.81,1.717);
\filldraw[fill opacity=0.8,fill=gray!20,draw=none](-7.594,.865)--(-7.569,.875)--(-7.569,.849)--cycle;
\draw(-7.569,.875)--(-7.569,.849);
\filldraw[fill opacity=0.8,fill=gray!20,draw=none](-7.569,.875)--(-7.594,.865)--(-7.614,.878)--cycle;
\filldraw[fill opacity=0.8,fill=gray!20,draw=none](-7.494,.804)--(-7.494,.779)--(-7.454,.781)--(-7.438,.792)--(-7.438,.818)--cycle;
\draw(-7.494,.804)--(-7.494,.779)--(-7.454,.781);
\draw(-7.438,.792)--(-7.438,.818);
\filldraw[fill opacity=0.8,fill=gray!20,draw=none](-7.494,.825)--(-7.468,.81)--(-7.438,.818)--cycle;
\filldraw[fill opacity=0.8,fill=gray!20](-7.472,.816)--(-7.636,.888)--(-7.586,.882)--(-7.422,.811)--cycle;
\filldraw[fill opacity=0.8,fill=gray!20,draw=none](-7.58,1.966)--(-7.58,1.91)--(-7.601,1.914)--(-7.637,1.959)--(-7.637,1.968)--cycle;
\draw(-7.637,1.959)--(-7.637,1.968)--(-7.58,1.966)--(-7.58,1.91);
\filldraw[fill opacity=0.8,fill=gray!20,draw=none](-4.463,2.649)--(-3.67,.963)--(-3.647,.963)--(-3.644,.969)--(-4.425,2.631)--cycle;
\draw(-4.463,2.649)--(-3.67,.963);
\draw(-3.644,.969)--(-4.425,2.631);
\filldraw[fill opacity=0.8,fill=gray!20,draw=none](-4.401,2.691)--(-4.406,2.698)--(-4.129,2.466)--(-4.079,2.42)--(-4.391,2.681)--cycle;
\draw(-4.406,2.698)--(-4.129,2.466);
\draw(-4.079,2.42)--(-4.391,2.681);
\filldraw[fill opacity=0.8,fill=gray!20,draw=none](-4.415,2.727)--(-4.415,2.734)--(-4.226,2.576)--(-4.129,2.466)--(-4.407,2.698)--cycle;
\draw(-4.415,2.734)--(-4.226,2.576);
\draw(-4.129,2.466)--(-4.407,2.698);
\filldraw[fill opacity=0.8,fill=gray!20,draw=none](-4.391,2.681)--(-4.079,2.42)--(-4.084,2.446)--(-4.347,2.666)--cycle;
\draw(-4.391,2.681)--(-4.079,2.42);
\draw(-4.084,2.446)--(-4.347,2.666);
\filldraw[fill opacity=0.8,fill=gray!20,draw=none](-4.407,2.771)--(-4.405,2.776)--(-4.355,2.734)--(-4.226,2.576)--(-4.408,2.728)--cycle;
\draw(-4.405,2.776)--(-4.355,2.734);
\draw(-4.226,2.576)--(-4.408,2.728);
\filldraw[fill opacity=0.8,fill=gray!20,draw=none](-4.295,2.623)--(-4.084,2.446)--(-4.143,2.54)--(-4.237,2.619)--cycle;
\draw(-4.295,2.623)--(-4.084,2.446);
\draw(-4.143,2.54)--(-4.237,2.619);
\filldraw[fill opacity=0.8,fill=gray!20,draw=none](-4.395,2.786)--(-4.355,2.734)--(-4.405,2.776)--cycle;
\draw(-4.355,2.734)--(-4.405,2.776);
\filldraw[fill opacity=0.8,fill=gray!20,draw=none](-4.399,2.791)--(-4.395,2.786)--(-4.405,2.776)--(-4.406,2.777)--cycle;
\draw(-4.405,2.776)--(-4.406,2.777);
\filldraw[fill opacity=0.8,fill=gray!20,draw=none](-4.407,2.771)--(-4.406,2.777)--(-4.405,2.776)--cycle;
\draw(-4.406,2.777)--(-4.405,2.776);
\filldraw[fill opacity=0.8,fill=gray!20,draw=none](-4.41,2.771)--(-4.399,2.791)--(-4.386,2.8)--(-4.336,2.811)--(-4.352,2.767)--cycle;
\draw(-4.336,2.811)--(-4.352,2.767)--(-4.41,2.771);
\filldraw[fill opacity=0.8,fill=gray!20,draw=none](-4.41,2.771)--(-4.424,2.772)--(-4.399,2.791)--cycle;
\draw(-4.41,2.771)--(-4.424,2.772);
\filldraw[fill opacity=0.8,fill=gray!20,draw=none](-4.463,2.861)--(-4.461,2.869)--(-4.399,2.791)--(-4.406,2.777)--(-4.454,2.817)--cycle;
\draw(-4.406,2.777)--(-4.454,2.817);
\filldraw[fill opacity=0.8,fill=gray!20,draw=none](-4.42,2.728)--(-4.431,2.728)--(-4.429,2.769)--(-4.424,2.772)--(-4.407,2.771)--cycle;
\draw(-4.42,2.728)--(-4.431,2.728)--(-4.429,2.769);
\draw(-4.424,2.772)--(-4.407,2.771);
\filldraw[fill opacity=0.8,fill=gray!20,draw=none](-4.438,2.728)--(-4.461,2.749)--(-4.429,2.769)--(-4.431,2.728)--cycle;
\draw(-4.429,2.769)--(-4.431,2.728)--(-4.438,2.728);
\filldraw[fill opacity=0.8,fill=gray!20,draw=none](-4.292,2.807)--(-4.3,2.813)--(-4.293,2.815)--(-4.257,2.798)--(-4.27,2.795)--cycle;
\draw(-4.3,2.813)--(-4.293,2.815);
\draw(-4.257,2.798)--(-4.27,2.795);
\filldraw[fill opacity=0.8,fill=gray!20,draw=none](-4.293,2.808)--(-4.286,2.82)--(-4.264,2.813)--(-4.268,2.801)--cycle;
\draw(-4.264,2.813)--(-4.268,2.801)--(-4.293,2.808);
\filldraw[fill opacity=0.8,fill=gray!20,draw=none](-4.269,2.795)--(-4.27,2.795)--(-4.257,2.798)--(-4.253,2.796)--cycle;
\draw(-4.27,2.795)--(-4.257,2.798);
\filldraw[fill opacity=0.8,fill=gray!20,draw=none](-4.352,2.767)--(-4.336,2.811)--(-4.32,2.814)--(-4.292,2.807)--(-4.272,2.795)--(-4.299,2.754)--cycle;
\draw(-4.32,2.814)--(-4.292,2.807);
\draw(-4.272,2.795)--(-4.299,2.754)--(-4.352,2.767)--(-4.336,2.811);
\filldraw[fill opacity=0.8,fill=gray!20,draw=none](-4.301,2.813)--(-4.299,2.809)--(-4.32,2.814)--(-4.311,2.818)--cycle;
\draw(-4.299,2.809)--(-4.32,2.814);
\filldraw[fill opacity=0.8,fill=gray!20,draw=none](-4.301,2.813)--(-4.311,2.818)--(-4.305,2.821)--cycle;
\filldraw[fill opacity=0.8,fill=gray!20,draw=none](-4.399,2.833)--(-4.351,2.843)--(-4.305,2.821)--(-4.301,2.813)--(-4.318,2.809)--cycle;
\draw(-4.399,2.833)--(-4.351,2.843);
\draw(-4.301,2.813)--(-4.318,2.809);
\filldraw[fill opacity=0.8,fill=gray!20,draw=none](-4.426,2.845)--(-4.415,2.875)--(-4.351,2.843)--(-4.429,2.827)--cycle;
\draw(-4.351,2.843)--(-4.429,2.827);
\filldraw[fill opacity=0.8,fill=gray!20,draw=none](-4.428,2.825)--(-4.429,2.827)--(-4.399,2.833)--(-4.318,2.809)--(-4.385,2.796)--cycle;
\draw(-4.429,2.827)--(-4.399,2.833);
\draw(-4.318,2.809)--(-4.385,2.796);
\filldraw[fill opacity=0.8,fill=gray!20,draw=none](-4.37,2.829)--(-4.33,2.896)--(-4.298,2.886)--(-4.348,2.817)--cycle;
\draw(-4.298,2.886)--(-4.348,2.817)--(-4.37,2.829);
\filldraw[fill opacity=0.8,fill=gray!20,draw=none](-4.272,2.796)--(-4.27,2.795)--(-4.272,2.795)--cycle;
\draw(-4.27,2.795)--(-4.272,2.795);
\filldraw[fill opacity=0.8,fill=gray!20,draw=none](-4.269,2.795)--(-4.272,2.795)--(-4.27,2.795)--cycle;
\draw(-4.272,2.795)--(-4.27,2.795);
\filldraw[fill opacity=0.8,fill=gray!20,draw=none](-4.259,2.794)--(-4.269,2.795)--(-4.253,2.796)--(-4.249,2.795)--cycle;
\filldraw[fill opacity=0.8,fill=gray!20,draw=none](-4.272,2.795)--(-4.268,2.801)--(-4.261,2.794)--cycle;
\draw(-4.272,2.795)--(-4.268,2.801)--(-4.261,2.794);
\filldraw[fill opacity=0.8,fill=gray!20,draw=none](-4.464,2.803)--(-4.462,2.823)--(-4.406,2.777)--(-4.408,2.728)--(-4.457,2.769)--cycle;
\draw(-4.462,2.823)--(-4.406,2.777);
\draw(-4.408,2.728)--(-4.457,2.769);
\filldraw[fill opacity=0.8,fill=gray!20,draw=none](-4.415,2.727)--(-4.418,2.736)--(-4.415,2.734)--cycle;
\draw(-4.418,2.736)--(-4.415,2.734);
\filldraw[fill opacity=0.8,fill=gray!20,draw=none](-4.42,2.728)--(-4.407,2.771)--(-4.352,2.767)--(-4.377,2.724)--cycle;
\draw(-4.407,2.771)--(-4.352,2.767)--(-4.377,2.724)--(-4.42,2.728);
\filldraw[fill opacity=0.8,fill=gray!20,draw=none](-4.252,2.797)--(-4.256,2.793)--(-4.261,2.794)--(-4.268,2.801)--(-4.264,2.813)--(-4.252,2.804)--cycle;
\draw(-4.261,2.794)--(-4.268,2.801)--(-4.264,2.813);
\filldraw[fill opacity=0.8,fill=gray!20,draw=none](-4.29,2.768)--(-4.272,2.795)--(-4.261,2.794)--(-4.258,2.791)--cycle;
\draw(-4.29,2.768)--(-4.272,2.795);
\draw(-4.261,2.794)--(-4.258,2.791);
\filldraw[fill opacity=0.8,fill=gray!20,draw=none](-4.259,2.794)--(-4.319,2.785)--(-4.272,2.795)--(-4.269,2.795)--cycle;
\draw(-4.319,2.785)--(-4.272,2.795);
\filldraw[fill opacity=0.8,fill=gray!20,draw=none](-4.299,2.754)--(-4.29,2.768)--(-4.258,2.791)--(-4.252,2.785)--(-4.253,2.774)--(-4.283,2.737)--cycle;
\draw(-4.258,2.791)--(-4.252,2.785);
\draw(-4.253,2.774)--(-4.283,2.737)--(-4.299,2.754)--(-4.29,2.768);
\filldraw[fill opacity=0.8,fill=gray!20,draw=none](-4.256,2.793)--(-4.258,2.791)--(-4.261,2.794)--cycle;
\draw(-4.258,2.791)--(-4.261,2.794);
\filldraw[fill opacity=0.8,fill=gray!20,draw=none](-4.367,2.772)--(-4.37,2.775)--(-4.319,2.785)--(-4.259,2.794)--(-4.256,2.793)--(-4.358,2.773)--cycle;
\draw(-4.37,2.775)--(-4.319,2.785);
\draw(-4.256,2.793)--(-4.358,2.773);
\filldraw[fill opacity=0.8,fill=gray!20,draw=none](-4.336,2.814)--(-4.348,2.817)--(-4.298,2.886)--(-4.287,2.877)--cycle;
\draw(-4.336,2.814)--(-4.348,2.817)--(-4.298,2.886);
\filldraw[fill opacity=0.8,fill=gray!20,draw=none](-4.259,2.794)--(-4.255,2.793)--(-4.256,2.793)--cycle;
\draw(-4.255,2.793)--(-4.256,2.793);
\filldraw[fill opacity=0.8,fill=gray!20,draw=none](-4.259,2.794)--(-4.249,2.795)--(-4.248,2.795)--(-4.255,2.793)--cycle;
\draw(-4.248,2.795)--(-4.255,2.793);
\filldraw[fill opacity=0.8,fill=gray!20,draw=none](-4.256,2.793)--(-4.252,2.793)--(-4.252,2.785)--(-4.258,2.791)--cycle;
\draw(-4.252,2.785)--(-4.258,2.791);
\filldraw[fill opacity=0.8,fill=gray!20,draw=none](-4.305,2.821)--(-4.302,2.819)--(-4.3,2.813)--(-4.301,2.813)--cycle;
\draw(-4.3,2.813)--(-4.301,2.813);
\filldraw[fill opacity=0.8,fill=gray!20,draw=none](-4.302,2.819)--(-4.293,2.815)--(-4.3,2.813)--cycle;
\draw(-4.293,2.815)--(-4.3,2.813);
\filldraw[fill opacity=0.8,fill=gray!20,draw=none](-4.301,2.813)--(-4.3,2.813)--(-4.292,2.807)--cycle;
\draw(-4.301,2.813)--(-4.3,2.813);
\filldraw[fill opacity=0.8,fill=gray!20,draw=none](-4.293,2.808)--(-4.299,2.809)--(-4.305,2.821)--(-4.297,2.824)--(-4.286,2.82)--cycle;
\draw(-4.293,2.808)--(-4.299,2.809);
\filldraw[fill opacity=0.8,fill=gray!20,draw=none](-4.394,2.794)--(-4.301,2.813)--(-4.292,2.807)--(-4.272,2.795)--(-4.33,2.783)--cycle;
\draw(-4.394,2.794)--(-4.301,2.813);
\draw(-4.272,2.795)--(-4.33,2.783);
\filldraw[fill opacity=0.8,fill=gray!20,draw=none](-4.292,2.807)--(-4.272,2.796)--(-4.272,2.795)--cycle;
\filldraw[fill opacity=0.8,fill=gray!20,draw=none](-4.292,2.807)--(-4.268,2.801)--(-4.272,2.795)--cycle;
\draw(-4.292,2.807)--(-4.268,2.801)--(-4.272,2.795);
\filldraw[fill opacity=0.8,fill=gray!20,draw=none](-4.508,2.988)--(-4.533,2.987)--(-4.528,3.008)--(-4.485,3.024)--cycle;
\draw(-4.508,2.988)--(-4.533,2.987)--(-4.528,3.008);
\filldraw[fill opacity=0.8,fill=gray!20,draw=none](-4.426,2.845)--(-4.421,2.877)--(-4.415,2.875)--cycle;
\filldraw[fill opacity=0.8,fill=gray!20,draw=none](-4.463,2.861)--(-4.454,2.817)--(-4.47,2.83)--cycle;
\draw(-4.454,2.817)--(-4.47,2.83);
\filldraw[fill opacity=0.8,fill=gray!20,draw=none](-4.256,2.793)--(-4.248,2.795)--(-4.252,2.797)--cycle;
\draw(-4.256,2.793)--(-4.248,2.795);
\filldraw[fill opacity=0.8,fill=gray!20,draw=none](-4.312,2.782)--(-4.256,2.793)--(-4.252,2.797)--(-4.261,2.803)--cycle;
\draw(-4.312,2.782)--(-4.256,2.793);
\filldraw[fill opacity=0.8,fill=gray!20,draw=none](-4.252,2.797)--(-4.252,2.793)--(-4.256,2.793)--cycle;
\filldraw[fill opacity=0.8,fill=gray!20,draw=none](-4.464,2.803)--(-4.467,2.819)--(-4.466,2.826)--(-4.462,2.823)--cycle;
\draw(-4.466,2.826)--(-4.462,2.823);
\filldraw[fill opacity=0.8,fill=gray!20,draw=none](-4.447,2.87)--(-4.448,2.873)--(-4.423,2.879)--(-4.421,2.877)--(-4.425,2.853)--cycle;
\draw(-4.448,2.873)--(-4.423,2.879);
\filldraw[fill opacity=0.8,fill=gray!20,draw=none](-4.446,2.89)--(-4.423,2.879)--(-4.449,2.873)--cycle;
\draw(-4.423,2.879)--(-4.449,2.873);
\filldraw[fill opacity=0.8,fill=gray!20,draw=none](-4.445,2.862)--(-4.447,2.87)--(-4.425,2.853)--(-4.426,2.845)--(-4.432,2.83)--cycle;
\filldraw[fill opacity=0.8,fill=gray!20,draw=none](-4.299,2.787)--(-4.261,2.803)--(-4.268,2.807)--(-4.306,2.799)--cycle;
\draw(-4.268,2.807)--(-4.306,2.799);
\filldraw[fill opacity=0.8,fill=gray!20,draw=none](-4.336,2.814)--(-4.287,2.877)--(-4.286,2.877)--(-4.332,2.813)--cycle;
\draw(-4.286,2.877)--(-4.332,2.813)--(-4.336,2.814);
\filldraw[fill opacity=0.8,fill=gray!20,draw=none](-4.33,2.817)--(-4.332,2.814)--(-4.332,2.813)--(-4.286,2.877)--(-4.287,2.876)--cycle;
\draw(-4.332,2.814)--(-4.332,2.813)--(-4.286,2.877);
\filldraw[fill opacity=0.8,fill=gray!20,draw=none](-4.325,2.828)--(-4.311,2.844)--(-4.287,2.876)--(-4.298,2.869)--(-4.319,2.839)--cycle;
\draw(-4.298,2.869)--(-4.319,2.839);
\filldraw[fill opacity=0.8,fill=gray!20,draw=none](-4.321,2.92)--(-3.561,1.304)--(-3.475,1.152)--(-4.318,2.943)--cycle;
\draw(-3.475,1.152)--(-4.318,2.943)--(-4.321,2.92)--(-3.561,1.304);
\filldraw[fill opacity=0.8,fill=gray!20,draw=none](-7.771,1.759)--(-7.766,1.832)--(-7.766,1.752)--(-7.769,1.736)--cycle;
\draw(-7.766,1.832)--(-7.766,1.752);
\filldraw[fill opacity=0.8,fill=gray!20,draw=none](-7.782,1.754)--(-7.795,1.759)--(-7.81,1.717)--(-7.782,1.704)--cycle;
\draw(-7.782,1.754)--(-7.795,1.759);
\draw(-7.81,1.717)--(-7.782,1.704);
\filldraw[fill opacity=0.8,fill=gray!20,draw=none](-7.771,1.759)--(-7.769,1.736)--(-7.773,1.712)--(-7.773,1.726)--cycle;
\draw(-7.773,1.712)--(-7.773,1.726);
\filldraw[fill opacity=0.8,fill=gray!20,draw=none](-7.775,1.757)--(-7.745,1.77)--(-7.736,1.72)--(-7.771,1.704)--cycle;
\draw(-7.775,1.757)--(-7.745,1.77)--(-7.736,1.72)--(-7.771,1.704);
\filldraw[fill opacity=0.8,fill=gray!20,draw=none](-7.766,1.868)--(-7.766,1.832)--(-7.771,1.759)--(-7.773,1.783)--(-7.773,1.81)--cycle;
\draw(-7.766,1.868)--(-7.766,1.832);
\draw(-7.773,1.783)--(-7.773,1.81);
\filldraw[fill opacity=0.8,fill=gray!20,draw=none](-7.768,1.808)--(-7.741,1.844)--(-7.71,1.857)--(-7.736,1.818)--(-7.771,1.803)--cycle;
\draw(-7.741,1.844)--(-7.71,1.857)--(-7.736,1.818)--(-7.771,1.803);
\filldraw[fill opacity=0.8,fill=gray!20,draw=none](-7.661,1.871)--(-7.641,1.88)--(-7.683,1.876)--(-7.698,1.87)--cycle;
\draw(-7.661,1.871)--(-7.641,1.88);
\draw(-7.683,1.876)--(-7.698,1.87);
\filldraw[fill opacity=0.8,fill=gray!20,draw=none](-7.588,1.819)--(-7.582,1.822)--(-7.617,1.855)--(-7.631,1.849)--cycle;
\draw(-7.588,1.819)--(-7.582,1.822);
\draw(-7.617,1.855)--(-7.631,1.849);
\filldraw[fill opacity=0.8,fill=gray!20,draw=none](-7.553,1.779)--(-7.535,1.733)--(-7.525,1.761)--cycle;
\filldraw[fill opacity=0.8,fill=gray!20,draw=none](-7.577,1.791)--(-7.657,1.763)--(-7.639,1.715)--(-7.586,1.733)--cycle;
\draw(-7.577,1.791)--(-7.657,1.763)--(-7.639,1.715)--(-7.586,1.733);
\filldraw[fill opacity=0.8,fill=gray!20,draw=none](-7.641,1.845)--(-7.673,1.828)--(-7.673,1.685)--(-7.616,1.704)--(-7.616,1.828)--cycle;
\draw(-7.673,1.828)--(-7.673,1.685);
\draw(-7.616,1.704)--(-7.616,1.828);
\filldraw[fill opacity=0.8,fill=gray!20,draw=none](-7.641,1.844)--(-7.616,1.828)--(-7.592,1.818)--(-7.588,1.819)--(-7.631,1.849)--(-7.641,1.845)--cycle;
\draw(-7.592,1.818)--(-7.588,1.819);
\draw(-7.631,1.849)--(-7.641,1.845);
\filldraw[fill opacity=0.8,fill=gray!20,draw=none](-7.56,1.776)--(-7.571,1.793)--(-7.577,1.791)--(-7.586,1.733)--(-7.555,1.745)--cycle;
\draw(-7.571,1.793)--(-7.577,1.791);
\draw(-7.586,1.733)--(-7.555,1.745);
\filldraw[fill opacity=0.8,fill=gray!20,draw=none](-7.575,1.799)--(-7.592,1.818)--(-7.616,1.828)--(-7.616,1.76)--(-7.586,1.733)--cycle;
\draw(-7.616,1.828)--(-7.616,1.76);
\filldraw[fill opacity=0.8,fill=gray!20,draw=none](-7.573,1.797)--(-7.577,1.791)--(-7.571,1.793)--cycle;
\draw(-7.577,1.791)--(-7.571,1.793);
\filldraw[fill opacity=0.8,fill=gray!20,draw=none](-7.57,1.79)--(-7.571,1.793)--(-7.588,1.819)--(-7.605,1.812)--cycle;
\draw(-7.588,1.819)--(-7.605,1.812);
\filldraw[fill opacity=0.8,fill=gray!20,draw=none](-7.592,1.818)--(-7.574,1.809)--(-7.57,1.832)--(-7.579,1.867)--(-7.616,1.857)--(-7.616,1.843)--cycle;
\draw(-7.616,1.857)--(-7.616,1.843);
\filldraw[fill opacity=0.8,fill=gray!20,draw=none](-7.592,1.818)--(-7.616,1.843)--(-7.616,1.828)--cycle;
\draw(-7.616,1.843)--(-7.616,1.828);
\filldraw[fill opacity=0.8,fill=gray!20,draw=none](-7.641,1.844)--(-7.641,1.819)--(-7.605,1.812)--(-7.597,1.816)--cycle;
\draw(-7.605,1.812)--(-7.597,1.816);
\filldraw[fill opacity=0.8,fill=gray!20,draw=none](-7.641,1.845)--(-7.616,1.828)--(-7.616,1.839)--cycle;
\draw(-7.616,1.828)--(-7.616,1.839);
\filldraw[fill opacity=0.8,fill=gray!20,draw=none](-7.641,1.845)--(-7.616,1.839)--(-7.616,1.857)--cycle;
\draw(-7.616,1.839)--(-7.616,1.857);
\filldraw[fill opacity=0.8,fill=gray!20,draw=none](-7.641,1.819)--(-7.641,1.814)--(-7.64,1.809)--(-7.624,1.804)--(-7.605,1.812)--cycle;
\draw(-7.624,1.804)--(-7.605,1.812);
\filldraw[fill opacity=0.8,fill=gray!20](-7.595,1.848)--(-7.759,1.919)--(-7.801,1.884)--(-7.637,1.812)--cycle;
\filldraw[fill opacity=0.8,fill=gray!20,draw=none](-3.302,2.002)--(-3.286,1.942)--(-3.296,2.019)--cycle;
\draw(-3.286,1.942)--(-3.296,2.019);
\filldraw[fill opacity=0.8,fill=gray!20,draw=none](-7.549,1.866)--(-7.692,1.929)--(-7.722,1.921)--(-7.737,1.91)--(-7.616,1.857)--cycle;
\draw(-7.549,1.866)--(-7.692,1.929);
\draw(-7.737,1.91)--(-7.616,1.857);
\filldraw[fill opacity=0.8,fill=gray!20,draw=none](-7.722,1.921)--(-7.747,1.914)--(-7.737,1.91)--cycle;
\draw(-7.747,1.914)--(-7.737,1.91);
\filldraw[fill opacity=0.8,fill=gray!20,draw=none](-7.692,1.964)--(-7.692,1.929)--(-7.737,1.91)--(-7.737,1.957)--cycle;
\draw(-7.737,1.91)--(-7.737,1.957)--(-7.692,1.964)--(-7.692,1.929);
\filldraw[fill opacity=0.8,fill=gray!20,draw=none](-7.737,1.957)--(-7.737,1.91)--(-7.766,1.868)--(-7.766,1.946)--cycle;
\draw(-7.766,1.868)--(-7.766,1.946)--(-7.737,1.957)--(-7.737,1.91);
\filldraw[fill opacity=0.8,fill=gray!20,draw=none](-7.533,1.895)--(-7.568,1.9)--(-7.568,1.9)--(-7.53,1.914)--cycle;
\draw(-7.568,1.9)--(-7.53,1.914);
\filldraw[fill opacity=0.8,fill=gray!20,draw=none](-7.568,1.9)--(-7.569,1.9)--(-7.591,1.901)--(-7.598,1.906)--(-7.566,1.918)--cycle;
\draw(-7.568,1.9)--(-7.569,1.9);
\draw(-7.598,1.906)--(-7.566,1.918);
\filldraw[fill opacity=0.8,fill=gray!20,draw=none](-7.49,1.946)--(-7.495,1.926)--(-7.496,1.925)--(-7.496,1.95)--cycle;
\draw(-7.496,1.925)--(-7.496,1.95)--(-7.49,1.946);
\filldraw[fill opacity=0.8,fill=gray!20,draw=none](-7.495,1.926)--(-7.49,1.946)--(-7.486,1.942)--cycle;
\draw(-7.49,1.946)--(-7.486,1.942);
\filldraw[fill opacity=0.8,fill=gray!20,draw=none](-7.486,1.942)--(-7.495,1.926)--(-7.568,1.9)--(-7.566,1.918)--(-7.503,1.94)--cycle;
\draw(-7.495,1.926)--(-7.568,1.9);
\draw(-7.566,1.918)--(-7.503,1.94);
\filldraw[fill opacity=0.8,fill=gray!20,draw=none](-7.495,1.926)--(-7.496,1.924)--(-7.496,1.925)--cycle;
\draw(-7.496,1.924)--(-7.496,1.925);
\filldraw[fill opacity=0.8,fill=gray!20,draw=none](-7.747,1.86)--(-7.801,1.884)--(-7.807,1.874)--(-7.773,1.81)--(-7.77,1.808)--cycle;
\draw(-7.747,1.86)--(-7.801,1.884)--(-7.807,1.874);
\draw(-7.773,1.81)--(-7.77,1.808);
\filldraw[fill opacity=0.8,fill=gray!20,draw=none](-7.766,1.946)--(-7.766,1.868)--(-7.773,1.835)--(-7.773,1.933)--cycle;
\draw(-7.773,1.835)--(-7.773,1.933)--(-7.766,1.946)--(-7.766,1.868);
\filldraw[fill opacity=0.8,fill=gray!20,draw=none](-7.591,1.901)--(-7.611,1.902)--(-7.598,1.906)--cycle;
\draw(-7.611,1.902)--(-7.598,1.906);
\filldraw[fill opacity=0.8,fill=gray!20,draw=none](-7.598,1.906)--(-7.611,1.902)--(-7.615,1.896)--(-7.589,1.905)--cycle;
\draw(-7.598,1.906)--(-7.611,1.902);
\draw(-7.615,1.896)--(-7.589,1.905);
\filldraw[fill opacity=0.8,fill=gray!20,draw=none](-7.503,1.94)--(-7.598,1.906)--(-7.589,1.905)--(-7.522,1.929)--cycle;
\draw(-7.503,1.94)--(-7.598,1.906);
\draw(-7.589,1.905)--(-7.522,1.929);
\filldraw[fill opacity=0.8,fill=gray!20,draw=none](-7.568,1.9)--(-7.569,1.9)--(-7.568,1.9)--cycle;
\draw(-7.569,1.9)--(-7.568,1.9);
\filldraw[fill opacity=0.8,fill=gray!20,draw=none](-7.522,1.929)--(-7.615,1.896)--(-7.582,1.883)--(-7.485,1.917)--cycle;
\draw(-7.522,1.929)--(-7.615,1.896);
\draw(-7.582,1.883)--(-7.485,1.917);
\filldraw[fill opacity=0.8,fill=gray!20](-7.561,1.907)--(-7.516,1.915)--(-7.488,1.926)--(-7.481,1.938)--(-7.496,1.95)--(-7.531,1.96)--(-7.58,1.966)--(-7.637,1.968)--(-7.692,1.964)--(-7.737,1.957)--(-7.766,1.946)--(-7.773,1.933)--(-7.758,1.921)--(-7.723,1.912)--(-7.673,1.905)--(-7.616,1.904)--cycle;
\filldraw[fill opacity=0.8,fill=gray!20,draw=none](-7.478,1.146)--(-7.458,1.121)--(-7.514,1.153)--cycle;
\filldraw[fill opacity=0.8,fill=gray!20,draw=none](-7.458,1.142)--(-7.458,1.121)--(-7.478,1.146)--cycle;
\draw(-7.458,1.142)--(-7.458,1.121);
\filldraw[fill opacity=0.8,fill=gray!20,draw=none](-7.41,1.131)--(-7.421,1.109)--(-7.422,1.132)--cycle;
\filldraw[fill opacity=0.8,fill=gray!20,draw=none](-7.444,1.115)--(-7.454,1.121)--(-7.45,1.123)--cycle;
\draw(-7.454,1.121)--(-7.45,1.123);
\filldraw[fill opacity=0.8,fill=gray!20,draw=none](-7.455,1.136)--(-7.45,1.123)--(-7.454,1.121)--(-7.481,1.138)--cycle;
\draw(-7.45,1.123)--(-7.454,1.121);
\filldraw[fill opacity=0.8,fill=gray!20,draw=none](-7.432,1.134)--(-7.45,1.123)--(-7.455,1.136)--cycle;
\draw(-7.432,1.134)--(-7.45,1.123);
\filldraw[fill opacity=0.8,fill=gray!20,draw=none](-7.375,1.101)--(-7.393,1.1)--(-7.365,1.08)--cycle;
\filldraw[fill opacity=0.8,fill=gray!20,draw=none](-7.409,1.13)--(-7.41,1.131)--(-7.408,1.135)--cycle;
\filldraw[fill opacity=0.8,fill=gray!20,draw=none](-7.429,1.135)--(-7.447,1.197)--(-7.408,1.192)--(-7.408,1.13)--cycle;
\draw(-7.447,1.197)--(-7.408,1.192)--(-7.408,1.13);
\filldraw[fill opacity=0.8,fill=gray!20,draw=none](-7.408,1.135)--(-7.404,1.151)--(-7.399,1.155)--cycle;
\draw(-7.404,1.151)--(-7.399,1.155);
\filldraw[fill opacity=0.8,fill=gray!20,draw=none](-7.395,1.188)--(-7.404,1.151)--(-7.408,1.144)--(-7.408,1.192)--cycle;
\draw(-7.408,1.144)--(-7.408,1.192)--(-7.395,1.188);
\filldraw[fill opacity=0.8,fill=gray!20,draw=none](-7.404,1.151)--(-7.395,1.188)--(-7.386,1.186)--cycle;
\draw(-7.395,1.188)--(-7.386,1.186);
\filldraw[fill opacity=0.8,fill=gray!20,draw=none](-7.371,1.187)--(-7.399,1.155)--(-7.423,1.14)--(-7.416,1.182)--(-7.413,1.183)--cycle;
\draw(-7.399,1.155)--(-7.423,1.14);
\draw(-7.416,1.182)--(-7.413,1.183);
\filldraw[fill opacity=0.8,fill=gray!20,draw=none](-7.404,1.151)--(-7.386,1.186)--(-7.373,1.182)--(-7.373,1.118)--(-7.408,1.13)--(-7.408,1.135)--cycle;
\draw(-7.386,1.186)--(-7.373,1.182)--(-7.373,1.118);
\draw(-7.408,1.13)--(-7.408,1.135);
\filldraw[fill opacity=0.8,fill=gray!20,draw=none](-7.393,1.147)--(-7.393,1.139)--(-7.365,1.08)--(-7.365,1.158)--cycle;
\draw(-7.365,1.08)--(-7.365,1.158)--(-7.393,1.147)--(-7.393,1.139);
\filldraw[fill opacity=0.8,fill=gray!20](-7.329,1.065)--(-7.493,1.136)--(-7.536,1.162)--(-7.372,1.091)--cycle;
\filldraw[fill opacity=0.8,fill=gray!20,draw=none](-2.944,2.136)--(-2.884,1.668)--(-2.8,1.696)--(-2.854,2.122)--cycle;
\draw(-2.944,2.136)--(-2.884,1.668);
\draw(-2.8,1.696)--(-2.854,2.122);
\filldraw[fill opacity=0.8,fill=gray!20,draw=none](-7.782,1.814)--(-7.83,1.834)--(-7.839,1.779)--(-7.782,1.754)--cycle;
\draw(-7.782,1.814)--(-7.83,1.834)--(-7.839,1.779)--(-7.782,1.754);
\filldraw[fill opacity=0.8,fill=gray!20](-3.578,.906)--(-3.568,.853)--(-3.539,.81)--(-3.497,.783)--(-3.447,.778)--(-3.397,.794)--(-3.354,.83)--(-3.326,.879)--(-3.316,.935)--(-3.326,.989)--(-3.354,1.032)--(-3.397,1.058)--(-3.447,1.064)--(-3.497,1.047)--(-3.539,1.012)--(-3.568,.962)--cycle;
\filldraw[fill opacity=0.8,fill=gray!20,draw=none](-7.524,1.157)--(-7.536,1.162)--(-7.586,1.168)--(-7.569,1.161)--cycle;
\draw(-7.524,1.157)--(-7.536,1.162)--(-7.586,1.168)--(-7.569,1.161);
\filldraw[fill opacity=0.8,fill=gray!20,draw=none](-7.514,1.153)--(-7.524,1.157)--(-7.569,1.161)--cycle;
\draw(-7.514,1.153)--(-7.524,1.157);
\filldraw[fill opacity=0.8,fill=gray!20,draw=none](-7.514,1.153)--(-7.514,.889)--(-7.569,.875)--(-7.569,1.186)--cycle;
\draw(-7.514,1.153)--(-7.514,.889);
\draw(-7.569,.875)--(-7.569,1.186);
\filldraw[fill opacity=0.8,fill=gray!20,draw=none](-7.458,1.121)--(-7.458,.919)--(-7.514,.889)--(-7.514,1.153)--cycle;
\draw(-7.458,1.121)--(-7.458,.919);
\draw(-7.514,.889)--(-7.514,1.153);
\filldraw[fill opacity=0.8,fill=gray!20,draw=none](-3.356,2.044)--(-3.363,2.044)--(-3.353,2.004)--(-3.346,2.021)--cycle;
\filldraw[fill opacity=0.8,fill=gray!20,draw=none](-7.807,1.874)--(-7.83,1.834)--(-7.773,1.81)--cycle;
\draw(-7.807,1.874)--(-7.83,1.834)--(-7.773,1.81);
\filldraw[fill opacity=0.8,fill=gray!20,draw=none](-7.594,.865)--(-7.569,.849)--(-7.569,.839)--(-7.614,.832)--(-7.614,.856)--cycle;
\draw(-7.569,.849)--(-7.569,.839)--(-7.614,.832)--(-7.614,.856);
\filldraw[fill opacity=0.8,fill=gray!20](-3.177,3.078)--(-3.175,3.134)--(-3.071,3.127)--(-3.083,3.071)--cycle;
\filldraw[fill opacity=0.8,fill=gray!20](-3.175,3.134)--(-3.175,3.191)--(-3.067,3.184)--(-3.071,3.127)--cycle;
\filldraw[fill opacity=0.8,fill=gray!20,draw=none](-3.478,1.077)--(-3.455,1.078)--(-3.461,1.09)--cycle;
\draw(-3.455,1.078)--(-3.461,1.09);
\filldraw[fill opacity=0.8,fill=gray!20,draw=none](-3.296,2.019)--(-3.286,1.942)--(-3.2,1.765)--(-3.221,1.928)--cycle;
\draw(-3.296,2.019)--(-3.286,1.942);
\draw(-3.2,1.765)--(-3.221,1.928);
\filldraw[fill opacity=0.8,fill=gray!20,draw=none](-3.305,2.042)--(-3.296,2.019)--(-3.299,2.042)--cycle;
\draw(-3.296,2.019)--(-3.299,2.042);
\filldraw[fill opacity=0.8,fill=gray!20,draw=none](-7.766,1.666)--(-7.766,1.589)--(-7.773,1.576)--(-7.773,1.675)--cycle;
\draw(-7.766,1.666)--(-7.766,1.589)--(-7.773,1.576)--(-7.773,1.675);
\filldraw[fill opacity=0.8,fill=gray!20,draw=none](-7.569,1.161)--(-7.569,.875)--(-7.614,.878)--(-7.614,1.142)--cycle;
\draw(-7.569,1.161)--(-7.569,.875);
\draw(-7.614,.878)--(-7.614,1.142);
\filldraw[fill opacity=0.8,fill=gray!20,draw=none](-7.599,1.153)--(-7.581,1.166)--(-7.586,1.168)--(-7.636,1.151)--(-7.624,1.147)--cycle;
\draw(-7.581,1.166)--(-7.586,1.168)--(-7.636,1.151)--(-7.624,1.147);
\filldraw[fill opacity=0.8,fill=gray!20,draw=none](-7.614,.995)--(-7.614,.878)--(-7.643,.898)--(-7.643,.978)--cycle;
\draw(-7.614,.995)--(-7.614,.878);
\draw(-7.643,.898)--(-7.643,.978);
\filldraw[fill opacity=0.8,fill=gray!20,draw=none](-7.61,.884)--(-7.623,.896)--(-7.581,.91)--(-7.546,.877)--(-7.579,.866)--cycle;
\draw(-7.623,.896)--(-7.581,.91)--(-7.546,.877)--(-7.579,.866);
\filldraw[fill opacity=0.8,fill=gray!20,draw=none](-7.579,.866)--(-7.546,.877)--(-7.504,.861)--(-7.543,.848)--cycle;
\draw(-7.579,.866)--(-7.546,.877)--(-7.504,.861)--(-7.543,.848);
\filldraw[fill opacity=0.8,fill=gray!20,draw=none](-7.543,.848)--(-7.504,.861)--(-7.462,.864)--(-7.527,.843)--cycle;
\draw(-7.543,.848)--(-7.504,.861)--(-7.462,.864)--(-7.527,.843);
\filldraw[fill opacity=0.8,fill=gray!20,draw=none](-7.431,1.022)--(-7.426,.944)--(-7.458,.919)--(-7.458,1.042)--cycle;
\draw(-7.458,.919)--(-7.458,1.042);
\filldraw[fill opacity=0.8,fill=gray!20,draw=none](-7.431,1.022)--(-7.408,1.005)--(-7.408,.959)--(-7.426,.944)--cycle;
\draw(-7.408,1.005)--(-7.408,.959);
\filldraw[fill opacity=0.8,fill=gray!20,draw=none](-7.436,1.097)--(-7.434,1.08)--(-7.454,1.039)--(-7.458,1.042)--(-7.458,1.121)--cycle;
\draw(-7.458,1.042)--(-7.458,1.121);
\filldraw[fill opacity=0.8,fill=gray!20,draw=none](-7.434,1.08)--(-7.431,1.022)--(-7.454,1.039)--cycle;
\filldraw[fill opacity=0.8,fill=gray!20,draw=none](-7.415,1.071)--(-7.406,1.037)--(-7.479,.992)--(-7.507,1.035)--(-7.426,1.085)--cycle;
\draw(-7.406,1.037)--(-7.479,.992)--(-7.507,1.035)--(-7.426,1.085);
\filldraw[fill opacity=0.8,fill=gray!20,draw=none](-7.395,1.013)--(-7.396,.985)--(-7.4,.983)--(-7.417,1.03)--(-7.406,1.037)--cycle;
\draw(-7.396,.985)--(-7.4,.983);
\draw(-7.417,1.03)--(-7.406,1.037);
\filldraw[fill opacity=0.8,fill=gray!20,draw=none](-7.394,.977)--(-7.408,.959)--(-7.408,.99)--cycle;
\draw(-7.408,.959)--(-7.408,.99);
\filldraw[fill opacity=0.8,fill=gray!20,draw=none](-7.385,.949)--(-7.387,.944)--(-7.438,.912)--(-7.455,.949)--(-7.396,.985)--cycle;
\draw(-7.387,.944)--(-7.438,.912)--(-7.455,.949)--(-7.396,.985);
\filldraw[fill opacity=0.8,fill=gray!20,draw=none](-7.373,1.004)--(-7.394,.977)--(-7.408,.99)--(-7.408,1.068)--cycle;
\draw(-7.408,.99)--(-7.408,1.068);
\filldraw[fill opacity=0.8,fill=gray!20,draw=none](-7.4,.983)--(-7.455,.949)--(-7.479,.992)--(-7.417,1.03)--cycle;
\draw(-7.4,.983)--(-7.455,.949)--(-7.479,.992)--(-7.417,1.03);
\filldraw[fill opacity=0.8,fill=gray!20,draw=none](-7.44,.879)--(-7.462,.864)--(-7.504,.861)--(-7.546,.877)--(-7.581,.91)--(-7.605,.955)--(-7.613,1.005)--(-7.605,1.053)--(-7.581,1.092)--(-7.568,1.1)--cycle;
\draw(-7.44,.879)--(-7.462,.864)--(-7.504,.861)--(-7.546,.877)--(-7.581,.91)--(-7.605,.955)--(-7.613,1.005)--(-7.605,1.053)--(-7.581,1.092)--(-7.568,1.1);
\filldraw[fill opacity=0.8,fill=gray!20,draw=none](-7.494,.865)--(-7.528,.845)--(-7.527,.843)--(-7.462,.864)--(-7.44,.879)--(-7.449,.88)--(-7.489,.867)--cycle;
\draw(-7.527,.843)--(-7.462,.864)--(-7.44,.879);
\draw(-7.449,.88)--(-7.489,.867);
\filldraw[fill opacity=0.8,fill=gray!20,draw=none](-7.489,.867)--(-7.449,.88)--(-7.467,.89)--cycle;
\draw(-7.489,.867)--(-7.449,.88);
\filldraw[fill opacity=0.8,fill=gray!20,draw=none](-7.467,.89)--(-7.494,.862)--(-7.494,.825)--(-7.438,.818)--(-7.438,.849)--cycle;
\draw(-7.494,.862)--(-7.494,.825);
\draw(-7.438,.818)--(-7.438,.849);
\filldraw[fill opacity=0.8,fill=gray!20,draw=none](-7.55,1.014)--(-7.55,.858)--(-7.528,.845)--(-7.494,.865)--(-7.494,.927)--cycle;
\draw(-7.55,1.014)--(-7.55,.858);
\draw(-7.494,.865)--(-7.494,.927);
\filldraw[fill opacity=0.8,fill=gray!20,draw=none](-7.528,.845)--(-7.494,.825)--(-7.494,.865)--cycle;
\draw(-7.494,.825)--(-7.494,.865);
\filldraw[fill opacity=0.8,fill=gray!20,draw=none](-7.494,.865)--(-7.529,.847)--(-7.528,.845)--cycle;
\filldraw[fill opacity=0.8,fill=gray!20,draw=none](-7.467,.89)--(-7.494,.927)--(-7.494,.862)--cycle;
\draw(-7.494,.927)--(-7.494,.862);
\filldraw[fill opacity=0.8,fill=gray!20,draw=none](-7.531,.849)--(-7.529,.847)--(-7.489,.867)--(-7.527,.855)--cycle;
\draw(-7.489,.867)--(-7.527,.855);
\filldraw[fill opacity=0.8,fill=gray!20](-7.515,.842)--(-7.678,.914)--(-7.636,.888)--(-7.472,.816)--cycle;
\filldraw[fill opacity=0.8,fill=gray!20,draw=none](-2.863,2.189)--(-2.854,2.122)--(-2.794,2.09)--(-2.809,2.208)--cycle;
\draw(-2.863,2.189)--(-2.854,2.122);
\draw(-2.794,2.09)--(-2.809,2.208);
\filldraw[fill opacity=0.8,fill=gray!20,draw=none](-2.817,2.145)--(-2.802,2.027)--(-2.733,2.022)--(-2.744,2.107)--cycle;
\draw(-2.817,2.145)--(-2.802,2.027);
\draw(-2.733,2.022)--(-2.744,2.107);
\filldraw[fill opacity=0.8,fill=gray!20,draw=none](-3.361,2.339)--(-3.362,2.34)--(-3.36,2.325)--cycle;
\draw(-3.361,2.339)--(-3.362,2.34)--(-3.36,2.325);
\filldraw[fill opacity=0.8,fill=gray!20](-3.175,3.191)--(-3.175,3.245)--(-3.071,3.238)--(-3.067,3.184)--cycle;
\filldraw[fill opacity=0.8,fill=gray!20,draw=none](-3.512,1.124)--(-3.489,1.077)--(-3.478,1.077)--(-3.461,1.09)--(-3.461,1.091)--cycle;
\draw(-3.512,1.124)--(-3.489,1.077);
\draw(-3.461,1.09)--(-3.461,1.091);
\filldraw[fill opacity=0.8,fill=gray!20,draw=none](-3.706,1.539)--(-3.512,1.124)--(-3.461,1.091)--(-3.561,1.304)--cycle;
\draw(-3.706,1.539)--(-3.512,1.124);
\draw(-3.461,1.091)--(-3.561,1.304);
\filldraw[fill opacity=0.8,fill=gray!20,draw=none](-3.353,2.004)--(-3.336,1.945)--(-3.346,2.021)--cycle;
\draw(-3.336,1.945)--(-3.346,2.021);
\filldraw[fill opacity=0.8,fill=gray!20](-3.179,3.026)--(-3.177,3.078)--(-3.083,3.071)--(-3.103,3.02)--cycle;
\filldraw[fill opacity=0.8,fill=gray!20,draw=none](-3.641,.963)--(-3.634,.948)--(-3.61,.981)--cycle;
\draw(-3.641,.963)--(-3.634,.948);
\filldraw[fill opacity=0.8,fill=gray!20,draw=none](-3.487,2.889)--(-3.419,2.354)--(-3.412,2.345)--(-3.45,2.831)--(-3.453,2.859)--cycle;
\draw(-3.45,2.831)--(-3.453,2.859)--(-3.487,2.889)--(-3.419,2.354);
\filldraw[fill opacity=0.8,fill=gray!20,draw=none](-7.766,1.666)--(-7.773,1.675)--(-7.773,1.7)--cycle;
\draw(-7.773,1.675)--(-7.773,1.7);
\filldraw[fill opacity=0.8,fill=gray!20,draw=none](-2.851,2.942)--(-2.818,2.683)--(-2.78,2.542)--(-2.835,2.979)--cycle;
\draw(-2.78,2.542)--(-2.835,2.979)--(-2.851,2.942)--(-2.818,2.683);
\filldraw[fill opacity=0.8,fill=gray!20,draw=none](-3.644,.993)--(-3.644,.969)--(-3.641,.963)--(-3.61,.981)--(-3.607,.986)--(-3.614,1.001)--cycle;
\draw(-3.644,.969)--(-3.641,.963);
\draw(-3.607,.986)--(-3.614,1.001);
\filldraw[fill opacity=0.8,fill=gray!20,draw=none](-3.268,2.042)--(-3.296,2.019)--(-3.221,1.928)--(-3.235,2.042)--cycle;
\draw(-3.221,1.928)--(-3.235,2.042);
\filldraw[fill opacity=0.8,fill=gray!20,draw=none](-3.299,2.042)--(-3.296,2.019)--(-3.268,2.042)--cycle;
\draw(-3.299,2.042)--(-3.296,2.019);
\filldraw[fill opacity=0.8,fill=gray!20,draw=none](-2.95,2.185)--(-2.944,2.136)--(-2.854,2.122)--(-2.863,2.189)--cycle;
\draw(-2.95,2.185)--(-2.944,2.136);
\draw(-2.854,2.122)--(-2.863,2.189);
\filldraw[fill opacity=0.8,fill=gray!20,draw=none](-2.924,2.161)--(-2.859,1.657)--(-2.759,1.691)--(-2.817,2.145)--cycle;
\draw(-2.924,2.161)--(-2.859,1.657)--(-2.759,1.691)--(-2.817,2.145);
\filldraw[fill opacity=0.8,fill=gray!20,draw=none](-3.654,.991)--(-3.644,.969)--(-3.644,.993)--cycle;
\draw(-3.654,.991)--(-3.644,.969);
\filldraw[fill opacity=0.8,fill=gray!20,draw=none](-7.594,.865)--(-7.614,.856)--(-7.614,.878)--cycle;
\draw(-7.614,.856)--(-7.614,.878);
\filldraw[fill opacity=0.8,fill=gray!20](-3.175,3.245)--(-3.177,3.292)--(-3.083,3.285)--(-3.071,3.238)--cycle;
\filldraw[fill opacity=0.8,fill=gray!20,draw=none](-2.82,2.168)--(-2.817,2.145)--(-2.744,2.107)--(-2.755,2.191)--cycle;
\draw(-2.82,2.168)--(-2.817,2.145);
\draw(-2.744,2.107)--(-2.755,2.191);
\filldraw[fill opacity=0.8,fill=gray!20,draw=none](-3.346,2.021)--(-3.336,1.945)--(-3.233,1.732)--(-3.256,1.912)--cycle;
\draw(-3.346,2.021)--(-3.336,1.945);
\draw(-3.233,1.732)--(-3.256,1.912);
\filldraw[fill opacity=0.8,fill=gray!20,draw=none](-3.356,2.044)--(-3.346,2.021)--(-3.349,2.044)--cycle;
\draw(-3.346,2.021)--(-3.349,2.044);
\filldraw[fill opacity=0.8,fill=gray!20,draw=none](-7.771,1.841)--(-7.766,1.868)--(-7.773,1.81)--cycle;
\filldraw[fill opacity=0.8,fill=gray!20,draw=none](-7.771,1.802)--(-7.773,1.81)--(-7.773,1.783)--cycle;
\draw(-7.773,1.81)--(-7.773,1.783);
\filldraw[fill opacity=0.8,fill=gray!20,draw=none](-7.771,1.802)--(-7.773,1.81)--(-7.782,1.814)--(-7.782,1.754)--cycle;
\draw(-7.773,1.81)--(-7.782,1.814);
\filldraw[fill opacity=0.8,fill=gray!20,draw=none](-8.098,1.566)--(-8.131,1.565)--(-8.134,1.621)--(-8.079,1.624)--cycle;
\draw(-8.098,1.566)--(-8.131,1.565)--(-8.134,1.621)--(-8.079,1.624);
\filldraw[fill opacity=0.8,fill=gray!20,draw=none](-8.071,1.672)--(-8.079,1.624)--(-8.109,1.623)--cycle;
\draw(-8.079,1.624)--(-8.109,1.623);
\filldraw[fill opacity=0.8,fill=gray!20](-8.213,1.549)--(-8.219,1.605)--(-8.134,1.621)--(-8.131,1.565)--cycle;
\filldraw[fill opacity=0.8,fill=gray!20,draw=none](-8.071,1.672)--(-8.109,1.623)--(-8.134,1.621)--(-8.131,1.675)--(-8.069,1.678)--cycle;
\draw(-8.109,1.623)--(-8.134,1.621)--(-8.131,1.675)--(-8.069,1.678);
\filldraw[fill opacity=0.8,fill=gray!20](-8.219,1.605)--(-8.213,1.66)--(-8.131,1.675)--(-8.134,1.621)--cycle;
\filldraw[fill opacity=0.8,fill=gray!20](-8.145,1.64)--(-7.736,1.818)--(-7.745,1.77)--(-8.154,1.591)--cycle;
\filldraw[fill opacity=0.8,fill=gray!20,draw=none](-7.645,.995)--(-7.643,.954)--(-7.643,.898)--(-7.649,.927)--cycle;
\draw(-7.643,.954)--(-7.643,.898);
\filldraw[fill opacity=0.8,fill=gray!20,draw=none](-7.65,.931)--(-7.648,.941)--(-7.605,.955)--(-7.581,.91)--(-7.623,.896)--cycle;
\draw(-7.648,.941)--(-7.605,.955)--(-7.581,.91)--(-7.623,.896);
\filldraw[fill opacity=0.8,fill=gray!20,draw=none](-7.614,1.025)--(-7.614,.995)--(-7.643,.978)--(-7.643,1.02)--cycle;
\draw(-7.614,1.025)--(-7.614,.995);
\draw(-7.643,.978)--(-7.643,1.02);
\filldraw[fill opacity=0.8,fill=gray!20,draw=none](-7.614,1.077)--(-7.614,1.025)--(-7.643,1.02)--(-7.643,1.045)--cycle;
\draw(-7.614,1.077)--(-7.614,1.025);
\draw(-7.643,1.02)--(-7.643,1.045);
\filldraw[fill opacity=0.8,fill=gray!20,draw=none](-7.645,.995)--(-7.644,1.024)--(-7.643,1.02)--(-7.643,.954)--cycle;
\draw(-7.643,1.02)--(-7.643,.954);
\filldraw[fill opacity=0.8,fill=gray!20,draw=none](-7.644,1.024)--(-7.643,1.045)--(-7.643,1.02)--cycle;
\draw(-7.643,1.045)--(-7.643,1.02);
\filldraw[fill opacity=0.8,fill=gray!20,draw=none](-7.614,1.142)--(-7.614,1.077)--(-7.643,1.045)--(-7.643,1.1)--cycle;
\draw(-7.614,1.142)--(-7.614,1.077);
\draw(-7.643,1.045)--(-7.643,1.1);
\filldraw[fill opacity=0.8,fill=gray!20,draw=none](-7.65,1.045)--(-7.643,1.1)--(-7.643,1.045)--(-7.644,1.024)--cycle;
\draw(-7.643,1.1)--(-7.643,1.045);
\filldraw[fill opacity=0.8,fill=gray!20,draw=none](-7.652,1.038)--(-7.648,1.039)--(-7.645,.995)--(-7.656,.992)--cycle;
\draw(-7.652,1.038)--(-7.648,1.039);
\draw(-7.645,.995)--(-7.656,.992);
\filldraw[fill opacity=0.8,fill=gray!20,draw=none](-7.666,.988)--(-7.645,.995)--(-7.648,.941)--(-7.654,.939)--cycle;
\draw(-7.666,.988)--(-7.645,.995);
\draw(-7.648,.941)--(-7.654,.939);
\filldraw[fill opacity=0.8,fill=gray!20,draw=none](-7.65,.931)--(-7.651,.933)--(-7.654,.939)--(-7.648,.941)--cycle;
\draw(-7.654,.939)--(-7.648,.941);
\filldraw[fill opacity=0.8,fill=gray!20,draw=none](-7.651,1.043)--(-7.65,1.045)--(-7.648,1.039)--(-7.652,1.038)--cycle;
\draw(-7.648,1.039)--(-7.652,1.038);
\filldraw[fill opacity=0.8,fill=gray!20,draw=none](-7.65,1.045)--(-7.648,1.039)--(-7.645,.995)--(-7.648,.941)--(-7.65,.931)--(-7.65,.932)--(-7.65,1.042)--cycle;
\draw(-7.65,.932)--(-7.65,1.042);
\filldraw[fill opacity=0.8,fill=gray!20,draw=none](-7.648,.941)--(-7.649,.927)--(-7.65,.931)--cycle;
\filldraw[fill opacity=0.8,fill=gray!20,draw=none](-7.568,1.045)--(-7.6,1.02)--(-7.6,.91)--(-7.55,.858)--(-7.55,1.014)--cycle;
\draw(-7.6,1.02)--(-7.6,.91);
\draw(-7.55,.858)--(-7.55,1.014);
\filldraw[fill opacity=0.8,fill=gray!20,draw=none](-7.528,.863)--(-7.527,.855)--(-7.489,.867)--(-7.467,.89)--(-7.484,.899)--(-7.491,.897)--cycle;
\draw(-7.527,.855)--(-7.489,.867);
\draw(-7.484,.899)--(-7.491,.897);
\filldraw[fill opacity=0.8,fill=gray!20](-7.543,.886)--(-7.707,.957)--(-7.678,.914)--(-7.515,.842)--cycle;
\filldraw[fill opacity=0.8,fill=gray!20,draw=none](-3.349,2.326)--(-3.349,2.329)--(-3.355,2.334)--cycle;
\draw(-3.349,2.329)--(-3.355,2.334);
\filldraw[fill opacity=0.8,fill=gray!20,draw=none](-7.614,.878)--(-7.614,.832)--(-7.643,.821)--(-7.643,.877)--cycle;
\draw(-7.614,.878)--(-7.614,.832)--(-7.643,.821)--(-7.643,.877);
\filldraw[fill opacity=0.8,fill=gray!20](-3.071,3.127)--(-3.067,3.184)--(-2.992,3.165)--(-2.998,3.109)--cycle;
\filldraw[fill opacity=0.8,fill=gray!20](-3.083,3.071)--(-3.071,3.127)--(-2.998,3.109)--(-3.018,3.055)--cycle;
\filldraw[fill opacity=0.8,fill=gray!20,draw=none](-4.415,2.727)--(-4.407,2.698)--(-4.415,2.705)--cycle;
\draw(-4.407,2.698)--(-4.415,2.705);
\filldraw[fill opacity=0.8,fill=gray!20,draw=none](-4.401,2.691)--(-4.404,2.694)--(-4.41,2.7)--(-4.406,2.698)--cycle;
\draw(-4.41,2.7)--(-4.406,2.698);
\filldraw[fill opacity=0.8,fill=gray!20,draw=none](-4.404,2.694)--(-4.405,2.695)--(-4.412,2.702)--(-4.41,2.7)--cycle;
\draw(-4.412,2.702)--(-4.41,2.7);
\filldraw[fill opacity=0.8,fill=gray!20,draw=none](-4.405,2.695)--(-4.415,2.705)--(-4.412,2.702)--cycle;
\draw(-4.415,2.705)--(-4.412,2.702);
\filldraw[fill opacity=0.8,fill=gray!20](-4.434,2.695)--(-4.431,2.728)--(-4.377,2.724)--(-4.407,2.693)--cycle;
\filldraw[fill opacity=0.8,fill=gray!20,draw=none](-4.438,2.728)--(-4.431,2.728)--(-4.432,2.722)--cycle;
\draw(-4.438,2.728)--(-4.431,2.728)--(-4.432,2.722);
\filldraw[fill opacity=0.8,fill=gray!20,draw=none](-4.474,2.797)--(-4.487,2.789)--(-4.487,2.791)--(-4.482,2.812)--cycle;
\draw(-4.474,2.797)--(-4.487,2.789);
\filldraw[fill opacity=0.8,fill=gray!20,draw=none](-4.467,2.777)--(-4.418,2.736)--(-4.415,2.727)--(-4.415,2.705)--(-4.452,2.736)--cycle;
\draw(-4.467,2.777)--(-4.418,2.736);
\draw(-4.415,2.705)--(-4.452,2.736);
\filldraw[fill opacity=0.8,fill=gray!20,draw=none](-4.464,2.858)--(-4.475,2.8)--(-4.482,2.812)--(-4.471,2.857)--cycle;
\filldraw[fill opacity=0.8,fill=gray!20,draw=none](-4.462,2.864)--(-4.479,2.807)--(-4.481,2.806)--(-4.485,2.814)--(-4.47,2.859)--(-4.466,2.864)--cycle;
\draw(-4.479,2.807)--(-4.481,2.806);
\filldraw[fill opacity=0.8,fill=gray!20,draw=none](-4.426,2.863)--(-4.47,2.859)--(-4.466,2.864)--(-4.441,2.873)--cycle;
\draw(-4.466,2.864)--(-4.441,2.873);
\filldraw[fill opacity=0.8,fill=gray!20,draw=none](-4.463,2.861)--(-4.47,2.83)--(-4.472,2.832)--cycle;
\draw(-4.47,2.83)--(-4.472,2.832);
\filldraw[fill opacity=0.8,fill=gray!20,draw=none](-4.467,2.819)--(-4.47,2.83)--(-4.466,2.826)--cycle;
\draw(-4.47,2.83)--(-4.466,2.826);
\filldraw[fill opacity=0.8,fill=gray!20,draw=none](-4.464,2.858)--(-4.444,2.861)--(-4.43,2.824)--(-4.474,2.797)--(-4.475,2.8)--cycle;
\draw(-4.43,2.824)--(-4.474,2.797);
\filldraw[fill opacity=0.8,fill=gray!20,draw=none](-4.467,2.777)--(-4.474,2.797)--(-4.464,2.803)--cycle;
\draw(-4.474,2.797)--(-4.464,2.803);
\filldraw[fill opacity=0.8,fill=gray!20,draw=none](-4.469,2.811)--(-4.472,2.832)--(-4.47,2.83)--(-4.467,2.819)--cycle;
\draw(-4.472,2.832)--(-4.47,2.83);
\filldraw[fill opacity=0.8,fill=gray!20,draw=none](-4.466,2.787)--(-4.469,2.811)--(-4.467,2.819)--(-4.464,2.803)--cycle;
\filldraw[fill opacity=0.8,fill=gray!20,draw=none](-4.472,2.832)--(-4.462,2.864)--(-4.446,2.864)--(-4.434,2.823)--(-4.469,2.811)--cycle;
\draw(-4.434,2.823)--(-4.469,2.811);
\filldraw[fill opacity=0.8,fill=gray!20,draw=none](-4.464,2.858)--(-4.471,2.857)--(-4.463,2.862)--cycle;
\draw(-4.471,2.857)--(-4.463,2.862);
\filldraw[fill opacity=0.8,fill=gray!20,draw=none](-4.463,2.862)--(-4.471,2.857)--(-4.471,2.86)--(-4.466,2.876)--cycle;
\draw(-4.463,2.862)--(-4.471,2.857);
\filldraw[fill opacity=0.8,fill=gray!20,draw=none](-4.497,2.916)--(-4.461,2.869)--(-4.472,2.832)--(-4.522,2.873)--cycle;
\draw(-4.472,2.832)--(-4.522,2.873)--(-4.497,2.916);
\filldraw[fill opacity=0.8,fill=gray!20,draw=none](-4.45,2.87)--(-4.463,2.862)--(-4.466,2.876)--(-4.459,2.896)--cycle;
\draw(-4.45,2.87)--(-4.463,2.862);
\filldraw[fill opacity=0.8,fill=gray!20,draw=none](-4.444,2.861)--(-4.464,2.858)--(-4.463,2.862)--(-4.448,2.871)--cycle;
\draw(-4.463,2.862)--(-4.448,2.871);
\filldraw[fill opacity=0.8,fill=gray!20,draw=none](-4.497,2.916)--(-4.446,2.89)--(-4.449,2.873)--(-4.488,2.865)--cycle;
\draw(-4.449,2.873)--(-4.488,2.865)--(-4.497,2.916);
\filldraw[fill opacity=0.8,fill=gray!20,draw=none](-4.451,2.873)--(-4.448,2.873)--(-4.447,2.87)--cycle;
\draw(-4.451,2.873)--(-4.448,2.873);
\filldraw[fill opacity=0.8,fill=gray!20,draw=none](-4.467,2.777)--(-4.469,2.752)--(-4.49,2.773)--(-4.487,2.789)--(-4.474,2.797)--cycle;
\draw(-4.487,2.789)--(-4.474,2.797);
\filldraw[fill opacity=0.8,fill=gray!20,draw=none](-4.472,2.832)--(-4.469,2.811)--(-4.479,2.807)--cycle;
\draw(-4.469,2.811)--(-4.479,2.807);
\filldraw[fill opacity=0.8,fill=gray!20,draw=none](-4.481,2.806)--(-4.489,2.804)--(-4.485,2.814)--cycle;
\draw(-4.481,2.806)--(-4.489,2.804);
\filldraw[fill opacity=0.8,fill=gray!20,draw=none](-4.47,2.769)--(-4.483,2.764)--(-4.497,2.778)--(-4.491,2.8)--(-4.489,2.804)--(-4.481,2.806)--cycle;
\draw(-4.47,2.769)--(-4.483,2.764);
\draw(-4.489,2.804)--(-4.481,2.806);
\filldraw[fill opacity=0.8,fill=gray!20,draw=none](-4.466,2.787)--(-4.464,2.803)--(-4.457,2.769)--(-4.464,2.775)--cycle;
\draw(-4.457,2.769)--(-4.464,2.775);
\filldraw[fill opacity=0.8,fill=gray!20,draw=none](-4.469,2.811)--(-4.464,2.775)--(-4.474,2.784)--cycle;
\draw(-4.464,2.775)--(-4.474,2.784);
\filldraw[fill opacity=0.8,fill=gray!20,draw=none](-4.425,2.818)--(-4.466,2.787)--(-4.464,2.803)--(-4.428,2.825)--cycle;
\draw(-4.464,2.803)--(-4.428,2.825);
\filldraw[fill opacity=0.8,fill=gray!20,draw=none](-4.43,2.817)--(-4.479,2.799)--(-4.481,2.806)--(-4.432,2.824)--cycle;
\draw(-4.481,2.806)--(-4.432,2.824);
\filldraw[fill opacity=0.8,fill=gray!20,draw=none](-4.426,2.845)--(-4.429,2.827)--(-4.433,2.826)--cycle;
\draw(-4.429,2.827)--(-4.433,2.826);
\filldraw[fill opacity=0.8,fill=gray!20,draw=none](-4.418,2.89)--(-4.45,2.87)--(-4.459,2.896)--(-4.458,2.899)--cycle;
\draw(-4.418,2.89)--(-4.45,2.87);
\filldraw[fill opacity=0.8,fill=gray!20,draw=none](-4.419,2.864)--(-4.426,2.863)--(-4.441,2.873)--(-4.413,2.883)--cycle;
\draw(-4.441,2.873)--(-4.413,2.883);
\filldraw[fill opacity=0.8,fill=gray!20,draw=none](-4.428,2.825)--(-4.431,2.827)--(-4.429,2.827)--cycle;
\draw(-4.431,2.827)--(-4.429,2.827);
\filldraw[fill opacity=0.8,fill=gray!20,draw=none](-4.444,2.861)--(-4.426,2.863)--(-4.419,2.859)--(-4.419,2.831)--(-4.43,2.824)--cycle;
\draw(-4.419,2.831)--(-4.43,2.824);
\filldraw[fill opacity=0.8,fill=gray!20,draw=none](-4.419,2.859)--(-4.418,2.91)--(-4.381,2.906)--(-4.416,2.857)--cycle;
\draw(-4.381,2.906)--(-4.416,2.857)--(-4.419,2.859);
\filldraw[fill opacity=0.8,fill=gray!20,draw=none](-4.425,2.818)--(-4.418,2.804)--(-4.46,2.758)--(-4.467,2.777)--(-4.466,2.787)--cycle;
\filldraw[fill opacity=0.8,fill=gray!20,draw=none](-4.467,2.777)--(-4.46,2.758)--(-4.469,2.752)--cycle;
\draw(-4.46,2.758)--(-4.469,2.752);
\filldraw[fill opacity=0.8,fill=gray!20,draw=none](-4.46,2.743)--(-4.469,2.752)--(-4.46,2.758)--cycle;
\draw(-4.469,2.752)--(-4.46,2.758);
\filldraw[fill opacity=0.8,fill=gray!20,draw=none](-4.472,2.753)--(-4.483,2.764)--(-4.47,2.769)--cycle;
\draw(-4.483,2.764)--(-4.47,2.769);
\filldraw[fill opacity=0.8,fill=gray!20,draw=none](-4.515,2.817)--(-4.494,2.787)--(-4.501,2.778)--(-4.518,2.793)--cycle;
\draw(-4.494,2.787)--(-4.501,2.778);
\filldraw[fill opacity=0.8,fill=gray!20,draw=none](-4.503,2.817)--(-4.485,2.843)--(-4.472,2.832)--(-4.471,2.815)--(-4.488,2.795)--cycle;
\draw(-4.485,2.843)--(-4.472,2.832);
\filldraw[fill opacity=0.8,fill=gray!20,draw=none](-4.472,2.832)--(-4.469,2.811)--(-4.471,2.802)--cycle;
\filldraw[fill opacity=0.8,fill=gray!20,draw=none](-4.489,2.786)--(-4.488,2.795)--(-4.467,2.777)--(-4.459,2.754)--cycle;
\draw(-4.488,2.795)--(-4.467,2.777);
\filldraw[fill opacity=0.8,fill=gray!20,draw=none](-4.461,2.805)--(-4.468,2.77)--(-4.47,2.769)--(-4.479,2.799)--cycle;
\draw(-4.468,2.77)--(-4.47,2.769);
\filldraw[fill opacity=0.8,fill=gray!20,draw=none](-4.471,2.815)--(-4.471,2.802)--(-4.474,2.784)--(-4.488,2.795)--cycle;
\draw(-4.474,2.784)--(-4.488,2.795);
\filldraw[fill opacity=0.8,fill=gray!20,draw=none](-4.445,2.862)--(-4.432,2.83)--(-4.433,2.826)--(-4.435,2.826)--cycle;
\draw(-4.433,2.826)--(-4.435,2.826);
\filldraw[fill opacity=0.8,fill=gray!20,draw=none](-4.435,2.826)--(-4.432,2.824)--(-4.434,2.823)--cycle;
\draw(-4.432,2.824)--(-4.434,2.823);
\filldraw[fill opacity=0.8,fill=gray!20,draw=none](-4.513,2.831)--(-4.511,2.85)--(-4.505,2.857)--(-4.487,2.84)--(-4.503,2.817)--cycle;
\filldraw[fill opacity=0.8,fill=gray!20,draw=none](-4.426,2.796)--(-4.418,2.804)--(-4.412,2.791)--cycle;
\filldraw[fill opacity=0.8,fill=gray!20,draw=none](-4.426,2.796)--(-4.412,2.791)--(-4.411,2.789)--(-4.46,2.758)--cycle;
\draw(-4.411,2.789)--(-4.46,2.758);
\filldraw[fill opacity=0.8,fill=gray!20,draw=none](-4.461,2.804)--(-4.461,2.805)--(-4.43,2.817)--(-4.421,2.789)--cycle;
\filldraw[fill opacity=0.8,fill=gray!20,draw=none](-4.461,2.804)--(-4.421,2.789)--(-4.42,2.787)--(-4.468,2.77)--cycle;
\draw(-4.42,2.787)--(-4.468,2.77);
\filldraw[fill opacity=0.8,fill=gray!20,draw=none](-4.433,2.79)--(-4.472,2.818)--(-4.438,2.825)--(-4.43,2.817)--(-4.429,2.814)--cycle;
\draw(-4.472,2.818)--(-4.438,2.825);
\filldraw[fill opacity=0.8,fill=gray!20,draw=none](-4.254,3.012)--(-4.152,3.032)--(-4.189,2.978)--(-4.297,2.956)--cycle;
\draw(-4.254,3.012)--(-4.152,3.032);
\draw(-4.189,2.978)--(-4.297,2.956);
\filldraw[fill opacity=0.8,fill=gray!20,draw=none](-4.298,3.013)--(-4.213,3.02)--(-4.315,2.999)--cycle;
\draw(-4.213,3.02)--(-4.315,2.999);
\filldraw[fill opacity=0.8,fill=gray!20,draw=none](-4.249,3.017)--(-4.298,3.013)--(-4.26,3.044)--(-4.214,3.053)--cycle;
\draw(-4.26,3.044)--(-4.214,3.053);
\filldraw[fill opacity=0.8,fill=gray!20,draw=none](-4.298,3.013)--(-4.327,3.011)--(-4.286,3.039)--(-4.26,3.044)--cycle;
\draw(-4.286,3.039)--(-4.26,3.044);
\filldraw[fill opacity=0.8,fill=gray!20,draw=none](-4.361,3.021)--(-4.352,3.017)--(-4.356,3.015)--cycle;
\filldraw[fill opacity=0.8,fill=gray!20,draw=none](-4.339,3.021)--(-4.342,3.02)--(-4.359,3.025)--(-4.351,3.035)--cycle;
\draw(-4.359,3.025)--(-4.351,3.035);
\filldraw[fill opacity=0.8,fill=gray!20,draw=none](-4.333,3.014)--(-4.339,3.021)--(-4.286,3.039)--cycle;
\filldraw[fill opacity=0.8,fill=gray!20,draw=none](-4.342,3.02)--(-4.368,3.012)--(-4.359,3.025)--cycle;
\draw(-4.368,3.012)--(-4.359,3.025);
\filldraw[fill opacity=0.8,fill=gray!20,draw=none](-4.374,3.035)--(-4.359,3.025)--(-4.361,3.021)--cycle;
\draw(-4.359,3.025)--(-4.361,3.021);
\filldraw[fill opacity=0.8,fill=gray!20,draw=none](-4.384,3.042)--(-4.341,3.05)--(-4.359,3.025)--cycle;
\draw(-4.341,3.05)--(-4.359,3.025);
\filldraw[fill opacity=0.8,fill=gray!20,draw=none](-4.352,3.017)--(-4.361,3.021)--(-4.374,3.035)--(-4.333,3.032)--(-4.332,3.028)--cycle;
\draw(-4.374,3.035)--(-4.333,3.032)--(-4.332,3.028);
\filldraw[fill opacity=0.8,fill=gray!20,draw=none](-4.333,3.014)--(-4.338,3.011)--(-4.352,3.017)--(-4.339,3.021)--cycle;
\filldraw[fill opacity=0.8,fill=gray!20,draw=none](-4.338,3.011)--(-4.352,3.017)--(-4.332,3.028)--(-4.332,3.026)--cycle;
\draw(-4.332,3.028)--(-4.332,3.026);
\filldraw[fill opacity=0.8,fill=gray!20,draw=none](-4.332,3.026)--(-4.333,3.032)--(-4.331,3.031)--cycle;
\draw(-4.332,3.026)--(-4.333,3.032)--(-4.331,3.031);
\filldraw[fill opacity=0.8,fill=gray!20,draw=none](-4.333,3.014)--(-4.34,3.01)--(-4.46,3)--(-4.461,3.003)--(-4.331,3.03)--cycle;
\draw(-4.461,3.003)--(-4.331,3.03);
\filldraw[fill opacity=0.8,fill=gray!20,draw=none](-4.448,2.871)--(-4.447,2.87)--(-4.445,2.862)--cycle;
\filldraw[fill opacity=0.8,fill=gray!20,draw=none](-4.426,2.863)--(-4.444,2.861)--(-4.448,2.871)--(-4.443,2.874)--cycle;
\draw(-4.448,2.871)--(-4.443,2.874);
\filldraw[fill opacity=0.8,fill=gray!20,draw=none](-4.47,2.869)--(-4.451,2.873)--(-4.448,2.871)--(-4.445,2.862)--(-4.435,2.826)--(-4.438,2.825)--cycle;
\draw(-4.47,2.869)--(-4.451,2.873);
\draw(-4.435,2.826)--(-4.438,2.825);
\filldraw[fill opacity=0.8,fill=gray!20,draw=none](-4.429,2.865)--(-4.418,2.865)--(-4.42,2.859)--cycle;
\filldraw[fill opacity=0.8,fill=gray!20,draw=none](-4.419,2.864)--(-4.426,2.863)--(-4.443,2.874)--(-4.418,2.89)--cycle;
\draw(-4.443,2.874)--(-4.418,2.89);
\filldraw[fill opacity=0.8,fill=gray!20,draw=none](-4.425,2.818)--(-4.435,2.826)--(-4.431,2.827)--(-4.428,2.825)--cycle;
\draw(-4.435,2.826)--(-4.431,2.827);
\filldraw[fill opacity=0.8,fill=gray!20,draw=none](-4.435,2.826)--(-4.446,2.864)--(-4.429,2.865)--(-4.42,2.859)--(-4.431,2.824)--(-4.432,2.824)--cycle;
\draw(-4.431,2.824)--(-4.432,2.824);
\filldraw[fill opacity=0.8,fill=gray!20,draw=none](-4.426,2.863)--(-4.419,2.864)--(-4.419,2.859)--cycle;
\filldraw[fill opacity=0.8,fill=gray!20,draw=none](-4.419,2.859)--(-4.458,2.886)--(-4.439,2.912)--(-4.418,2.91)--cycle;
\draw(-4.419,2.859)--(-4.458,2.886)--(-4.439,2.912);
\filldraw[fill opacity=0.8,fill=gray!20,draw=none](-4.382,2.835)--(-4.402,2.876)--(-4.381,2.906)--(-4.331,2.896)--(-4.377,2.832)--cycle;
\draw(-4.402,2.876)--(-4.381,2.906);
\draw(-4.331,2.896)--(-4.377,2.832)--(-4.382,2.835);
\filldraw[fill opacity=0.8,fill=gray!20,draw=none](-4.438,2.932)--(-4.413,2.893)--(-4.418,2.89)--(-4.458,2.899)--(-4.447,2.932)--cycle;
\draw(-4.438,2.932)--(-4.413,2.893)--(-4.418,2.89);
\filldraw[fill opacity=0.8,fill=gray!20,draw=none](-4.327,3.011)--(-4.34,3.01)--(-4.286,3.039)--cycle;
\filldraw[fill opacity=0.8,fill=gray!20,draw=none](-4.374,3.035)--(-4.384,3.042)--(-4.341,3.047)--(-4.333,3.032)--cycle;
\draw(-4.341,3.047)--(-4.333,3.032)--(-4.374,3.035);
\filldraw[fill opacity=0.8,fill=gray!20,draw=none](-4.331,3.031)--(-4.333,3.032)--(-4.341,3.047)--(-4.335,3.047)--cycle;
\draw(-4.331,3.031)--(-4.333,3.032)--(-4.341,3.047);
\filldraw[fill opacity=0.8,fill=gray!20,draw=none](-4.331,3.03)--(-4.331,3.031)--(-4.33,3.031)--cycle;
\draw(-4.331,3.031)--(-4.33,3.031);
\filldraw[fill opacity=0.8,fill=gray!20,draw=none](-4.333,3.014)--(-4.331,3.03)--(-4.286,3.039)--cycle;
\draw(-4.331,3.03)--(-4.286,3.039);
\filldraw[fill opacity=0.8,fill=gray!20,draw=none](-4.33,3.031)--(-4.331,3.031)--(-4.335,3.047)--(-4.32,3.048)--cycle;
\draw(-4.33,3.031)--(-4.331,3.031);
\filldraw[fill opacity=0.8,fill=gray!20,draw=none](-4.399,2.857)--(-4.412,2.854)--(-4.416,2.857)--(-4.403,2.874)--cycle;
\draw(-4.412,2.854)--(-4.416,2.857)--(-4.403,2.874);
\filldraw[fill opacity=0.8,fill=gray!20,draw=none](-4.402,2.866)--(-4.419,2.864)--(-4.413,2.883)--(-4.407,2.885)--cycle;
\draw(-4.413,2.883)--(-4.407,2.885)--(-4.402,2.866);
\filldraw[fill opacity=0.8,fill=gray!20,draw=none](-4.407,2.882)--(-4.402,2.866)--(-4.419,2.864)--(-4.418,2.89)--(-4.413,2.893)--cycle;
\draw(-4.418,2.89)--(-4.413,2.893)--(-4.407,2.882);
\filldraw[fill opacity=0.8,fill=gray!20,draw=none](-4.472,2.931)--(-4.457,2.926)--(-4.467,2.919)--(-4.47,2.921)--cycle;
\draw(-4.457,2.926)--(-4.467,2.919);
\filldraw[fill opacity=0.8,fill=gray!20,draw=none](-4.399,2.857)--(-4.394,2.858)--(-4.389,2.85)--(-4.396,2.845)--cycle;
\draw(-4.394,2.858)--(-4.389,2.85)--(-4.396,2.845);
\filldraw[fill opacity=0.8,fill=gray!20,draw=none](-4.402,2.866)--(-4.399,2.857)--(-4.412,2.854)--(-4.419,2.859)--(-4.419,2.864)--cycle;
\filldraw[fill opacity=0.8,fill=gray!20,draw=none](-4.412,2.854)--(-4.399,2.857)--(-4.396,2.845)--(-4.397,2.844)--cycle;
\draw(-4.396,2.845)--(-4.397,2.844);
\filldraw[fill opacity=0.8,fill=gray!20,draw=none](-4.399,2.857)--(-4.398,2.853)--(-4.406,2.85)--(-4.412,2.854)--cycle;
\draw(-4.406,2.85)--(-4.412,2.854);
\filldraw[fill opacity=0.8,fill=gray!20,draw=none](-4.395,2.843)--(-4.397,2.844)--(-4.396,2.845)--cycle;
\draw(-4.397,2.844)--(-4.396,2.845);
\filldraw[fill opacity=0.8,fill=gray!20,draw=none](-4.42,2.859)--(-4.418,2.865)--(-4.402,2.866)--(-4.396,2.844)--cycle;
\draw(-4.402,2.866)--(-4.396,2.844);
\filldraw[fill opacity=0.8,fill=gray!20,draw=none](-4.398,2.853)--(-4.396,2.844)--(-4.406,2.85)--cycle;
\draw(-4.396,2.844)--(-4.406,2.85);
\filldraw[fill opacity=0.8,fill=gray!20,draw=none](-4.438,2.89)--(-4.497,2.916)--(-4.458,2.886)--(-4.42,2.859)--(-4.406,2.85)--(-4.396,2.844)--cycle;
\draw(-4.497,2.916)--(-4.458,2.886)--(-4.42,2.859);
\draw(-4.406,2.85)--(-4.396,2.844);
\filldraw[fill opacity=0.8,fill=gray!20,draw=none](-4.475,2.826)--(-4.488,2.865)--(-4.47,2.869)--(-4.438,2.825)--(-4.466,2.82)--cycle;
\draw(-4.475,2.826)--(-4.488,2.865)--(-4.47,2.869);
\draw(-4.438,2.825)--(-4.466,2.82);
\filldraw[fill opacity=0.8,fill=gray!20,draw=none](-4.412,2.854)--(-4.419,2.852)--(-4.419,2.859)--cycle;
\filldraw[fill opacity=0.8,fill=gray!20,draw=none](-4.412,2.854)--(-4.397,2.844)--(-4.419,2.831)--(-4.419,2.852)--cycle;
\draw(-4.397,2.844)--(-4.419,2.831);
\filldraw[fill opacity=0.8,fill=gray!20,draw=none](-4.425,2.844)--(-4.42,2.859)--(-4.406,2.85)--cycle;
\filldraw[fill opacity=0.8,fill=gray!20,draw=none](-4.406,2.804)--(-4.425,2.818)--(-4.428,2.825)--(-4.404,2.809)--cycle;
\filldraw[fill opacity=0.8,fill=gray!20,draw=none](-4.395,2.843)--(-4.395,2.841)--(-4.425,2.818)--(-4.428,2.825)--(-4.397,2.844)--cycle;
\draw(-4.428,2.825)--(-4.397,2.844);
\filldraw[fill opacity=0.8,fill=gray!20,draw=none](-4.425,2.844)--(-4.406,2.85)--(-4.396,2.844)--(-4.394,2.837)--(-4.431,2.824)--cycle;
\draw(-4.396,2.844)--(-4.394,2.837)--(-4.431,2.824);
\filldraw[fill opacity=0.8,fill=gray!20,draw=none](-4.438,2.825)--(-4.435,2.826)--(-4.425,2.818)--(-4.418,2.804)--cycle;
\draw(-4.438,2.825)--(-4.435,2.826);
\filldraw[fill opacity=0.8,fill=gray!20,draw=none](-4.425,2.818)--(-4.406,2.804)--(-4.41,2.791)--(-4.412,2.791)--cycle;
\draw(-4.41,2.791)--(-4.412,2.791);
\filldraw[fill opacity=0.8,fill=gray!20,draw=none](-4.392,2.828)--(-4.404,2.834)--(-4.394,2.837)--cycle;
\draw(-4.404,2.834)--(-4.394,2.837)--(-4.392,2.828);
\filldraw[fill opacity=0.8,fill=gray!20,draw=none](-4.425,2.818)--(-4.395,2.841)--(-4.393,2.832)--(-4.418,2.804)--cycle;
\filldraw[fill opacity=0.8,fill=gray!20,draw=none](-4.395,2.829)--(-4.43,2.817)--(-4.432,2.824)--(-4.404,2.834)--cycle;
\draw(-4.432,2.824)--(-4.404,2.834);
\filldraw[fill opacity=0.8,fill=gray!20,draw=none](-4.429,2.814)--(-4.43,2.817)--(-4.429,2.816)--cycle;
\filldraw[fill opacity=0.8,fill=gray!20,draw=none](-4.429,2.814)--(-4.429,2.816)--(-4.418,2.804)--(-4.412,2.791)--(-4.421,2.789)--cycle;
\draw(-4.412,2.791)--(-4.421,2.789);
\filldraw[fill opacity=0.8,fill=gray!20,draw=none](-4.406,2.804)--(-4.404,2.809)--(-4.385,2.796)--(-4.394,2.794)--cycle;
\draw(-4.385,2.796)--(-4.394,2.794);
\filldraw[fill opacity=0.8,fill=gray!20,draw=none](-4.406,2.804)--(-4.394,2.794)--(-4.41,2.791)--cycle;
\draw(-4.394,2.794)--(-4.41,2.791);
\filldraw[fill opacity=0.8,fill=gray!20,draw=none](-4.421,2.789)--(-4.418,2.788)--(-4.42,2.787)--cycle;
\draw(-4.418,2.788)--(-4.42,2.787);
\filldraw[fill opacity=0.8,fill=gray!20,draw=none](-4.418,2.788)--(-4.429,2.787)--(-4.418,2.789)--cycle;
\draw(-4.429,2.787)--(-4.418,2.789);
\filldraw[fill opacity=0.8,fill=gray!20,draw=none](-4.421,2.789)--(-4.43,2.817)--(-4.419,2.821)--(-4.418,2.788)--cycle;
\filldraw[fill opacity=0.8,fill=gray!20,draw=none](-4.433,2.79)--(-4.429,2.814)--(-4.421,2.789)--(-4.429,2.787)--cycle;
\draw(-4.421,2.789)--(-4.429,2.787);
\filldraw[fill opacity=0.8,fill=gray!20,draw=none](-4.411,2.812)--(-4.393,2.832)--(-4.387,2.803)--(-4.406,2.791)--cycle;
\draw(-4.387,2.803)--(-4.406,2.791);
\filldraw[fill opacity=0.8,fill=gray!20,draw=none](-4.408,2.788)--(-4.41,2.791)--(-4.394,2.794)--(-4.37,2.79)--cycle;
\draw(-4.41,2.791)--(-4.394,2.794);
\filldraw[fill opacity=0.8,fill=gray!20,draw=none](-4.412,2.791)--(-4.418,2.804)--(-4.411,2.812)--(-4.406,2.791)--(-4.409,2.79)--cycle;
\draw(-4.406,2.791)--(-4.409,2.79);
\filldraw[fill opacity=0.8,fill=gray!20,draw=none](-4.403,2.693)--(-4.407,2.693)--(-4.405,2.695)--cycle;
\draw(-4.403,2.693)--(-4.407,2.693)--(-4.405,2.695);
\filldraw[fill opacity=0.8,fill=gray!20,draw=none](-4.413,2.69)--(-4.407,2.693)--(-4.403,2.693)--(-4.397,2.686)--cycle;
\draw(-4.413,2.69)--(-4.407,2.693)--(-4.403,2.693);
\filldraw[fill opacity=0.8,fill=gray!20,draw=none](-4.452,2.734)--(-4.46,2.729)--(-4.46,2.743)--cycle;
\draw(-4.452,2.734)--(-4.46,2.729);
\filldraw[fill opacity=0.8,fill=gray!20,draw=none](-4.444,2.725)--(-4.452,2.735)--(-4.453,2.737)--(-4.415,2.705)--(-4.405,2.695)--(-4.403,2.693)--(-4.397,2.686)--(-4.442,2.723)--cycle;
\draw(-4.453,2.737)--(-4.415,2.705);
\draw(-4.397,2.686)--(-4.442,2.723);
\filldraw[fill opacity=0.8,fill=gray!20,draw=none](-4.518,2.87)--(-4.518,2.87)--(-4.485,2.843)--(-4.487,2.84)--cycle;
\draw(-4.518,2.87)--(-4.485,2.843);
\filldraw[fill opacity=0.8,fill=gray!20,draw=none](-4.418,2.812)--(-4.419,2.821)--(-4.395,2.829)--(-4.392,2.828)--(-4.389,2.805)--cycle;
\draw(-4.392,2.828)--(-4.389,2.805);
\filldraw[fill opacity=0.8,fill=gray!20,draw=none](-4.511,2.85)--(-4.51,2.862)--(-4.505,2.857)--cycle;
\filldraw[fill opacity=0.8,fill=gray!20,draw=none](-4.503,2.817)--(-4.488,2.795)--(-4.507,2.811)--cycle;
\draw(-4.488,2.795)--(-4.507,2.811);
\filldraw[fill opacity=0.8,fill=gray!20,draw=none](-4.489,2.786)--(-4.528,2.828)--(-4.488,2.795)--cycle;
\draw(-4.528,2.828)--(-4.488,2.795);
\filldraw[fill opacity=0.8,fill=gray!20,draw=none](-4.513,2.831)--(-4.503,2.817)--(-4.507,2.811)--(-4.515,2.817)--cycle;
\draw(-4.507,2.811)--(-4.515,2.817);
\filldraw[fill opacity=0.8,fill=gray!20,draw=none](-4.478,2.763)--(-4.507,2.806)--(-4.48,2.777)--cycle;
\filldraw[fill opacity=0.8,fill=gray!20,draw=none](-4.471,2.822)--(-4.494,2.787)--(-4.525,2.833)--(-4.511,2.854)--cycle;
\draw(-4.525,2.833)--(-4.511,2.854)--(-4.471,2.822)--(-4.494,2.787);
\filldraw[fill opacity=0.8,fill=gray!20,draw=none](-4.469,2.822)--(-4.474,2.817)--(-4.471,2.822)--cycle;
\draw(-4.474,2.817)--(-4.471,2.822)--(-4.469,2.822);
\filldraw[fill opacity=0.8,fill=gray!20,draw=none](-4.472,2.931)--(-4.47,2.921)--(-4.48,2.927)--cycle;
\filldraw[fill opacity=0.8,fill=gray!20,draw=none](-4.457,2.926)--(-4.472,2.931)--(-4.464,2.937)--(-4.459,2.932)--cycle;
\filldraw[fill opacity=0.8,fill=gray!20,draw=none](-4.464,2.953)--(-4.459,2.932)--(-4.474,2.947)--(-4.47,2.954)--cycle;
\draw(-4.474,2.947)--(-4.47,2.954);
\filldraw[fill opacity=0.8,fill=gray!20,draw=none](-4.46,2.939)--(-4.464,2.937)--(-4.465,2.963)--(-4.453,2.957)--cycle;
\filldraw[fill opacity=0.8,fill=gray!20,draw=none](-4.472,2.931)--(-4.483,2.935)--(-4.474,2.947)--(-4.464,2.937)--cycle;
\draw(-4.483,2.935)--(-4.474,2.947);
\filldraw[fill opacity=0.8,fill=gray!20,draw=none](-4.464,2.937)--(-4.465,2.936)--(-4.469,2.962)--(-4.468,2.964)--(-4.465,2.963)--cycle;
\filldraw[fill opacity=0.8,fill=gray!20,draw=none](-4.455,2.955)--(-4.445,2.942)--(-4.456,2.948)--(-4.463,2.954)--cycle;
\draw(-4.456,2.948)--(-4.463,2.954);
\filldraw[fill opacity=0.8,fill=gray!20,draw=none](-4.469,2.962)--(-4.47,2.965)--(-4.468,2.964)--cycle;
\filldraw[fill opacity=0.8,fill=gray!20,draw=none](-4.452,2.931)--(-4.458,2.93)--(-4.485,2.95)--(-4.474,2.964)--(-4.467,2.966)--cycle;
\draw(-4.474,2.964)--(-4.467,2.966);
\filldraw[fill opacity=0.8,fill=gray!20,draw=none](-4.49,2.933)--(-4.48,2.927)--(-4.502,2.918)--cycle;
\filldraw[fill opacity=0.8,fill=gray!20,draw=none](-4.483,2.935)--(-4.472,2.931)--(-4.472,2.931)--(-4.48,2.927)--cycle;
\filldraw[fill opacity=0.8,fill=gray!20,draw=none](-4.438,2.89)--(-4.497,2.916)--(-4.488,2.865)--(-4.475,2.826)--cycle;
\draw(-4.497,2.916)--(-4.488,2.865)--(-4.475,2.826);
\filldraw[fill opacity=0.8,fill=gray!20,draw=none](-4.483,2.935)--(-4.48,2.927)--(-4.487,2.931)--cycle;
\filldraw[fill opacity=0.8,fill=gray!20,draw=none](-4.472,2.931)--(-4.472,2.931)--(-4.497,2.916)--cycle;
\filldraw[fill opacity=0.8,fill=gray!20,draw=none](-4.483,2.935)--(-4.487,2.931)--(-4.512,2.945)--cycle;
\filldraw[fill opacity=0.8,fill=gray!20,draw=none](-4.483,2.935)--(-4.512,2.945)--(-4.49,2.958)--cycle;
\draw(-4.512,2.945)--(-4.49,2.958);
\filldraw[fill opacity=0.8,fill=gray!20,draw=none](-4.51,2.96)--(-4.501,2.961)--(-4.49,2.958)--(-4.508,2.947)--cycle;
\draw(-4.49,2.958)--(-4.508,2.947);
\filldraw[fill opacity=0.8,fill=gray!20,draw=none](-4.503,2.975)--(-4.477,2.946)--(-4.477,2.944)--(-4.497,2.916)--cycle;
\draw(-4.477,2.944)--(-4.497,2.916);
\filldraw[fill opacity=0.8,fill=gray!20,draw=none](-4.503,2.975)--(-4.488,2.996)--(-4.477,2.946)--cycle;
\draw(-4.503,2.975)--(-4.488,2.996);
\filldraw[fill opacity=0.8,fill=gray!20,draw=none](-4.477,2.95)--(-4.472,2.931)--(-4.483,2.935)--(-4.49,2.958)--cycle;
\filldraw[fill opacity=0.8,fill=gray!20,draw=none](-4.472,2.931)--(-4.497,2.916)--(-4.483,2.935)--cycle;
\draw(-4.497,2.916)--(-4.483,2.935);
\filldraw[fill opacity=0.8,fill=gray!20,draw=none](-4.475,2.942)--(-4.47,2.921)--(-4.497,2.916)--cycle;
\draw(-4.47,2.921)--(-4.497,2.916);
\filldraw[fill opacity=0.8,fill=gray!20,draw=none](-4.457,2.926)--(-4.454,2.913)--(-4.497,2.916)--cycle;
\filldraw[fill opacity=0.8,fill=gray!20,draw=none](-4.472,2.931)--(-4.457,2.926)--(-4.497,2.916)--cycle;
\filldraw[fill opacity=0.8,fill=gray!20,draw=none](-4.443,2.939)--(-4.439,2.934)--(-4.452,2.931)--cycle;
\filldraw[fill opacity=0.8,fill=gray!20,draw=none](-4.472,2.931)--(-4.477,2.95)--(-4.47,2.962)--(-4.465,2.936)--cycle;
\filldraw[fill opacity=0.8,fill=gray!20,draw=none](-4.516,2.856)--(-4.511,2.854)--(-4.52,2.84)--cycle;
\draw(-4.511,2.854)--(-4.52,2.84);
\filldraw[fill opacity=0.8,fill=gray!20,draw=none](-4.475,2.826)--(-4.466,2.82)--(-4.472,2.818)--cycle;
\draw(-4.466,2.82)--(-4.472,2.818)--(-4.475,2.826);
\filldraw[fill opacity=0.8,fill=gray!20,draw=none](-4.478,2.763)--(-4.48,2.777)--(-4.459,2.754)--(-4.452,2.736)--(-4.469,2.75)--cycle;
\draw(-4.452,2.736)--(-4.469,2.75);
\filldraw[fill opacity=0.8,fill=gray!20,draw=none](-4.458,2.74)--(-4.46,2.743)--(-4.46,2.758)--(-4.44,2.77)--cycle;
\draw(-4.46,2.758)--(-4.44,2.77);
\filldraw[fill opacity=0.8,fill=gray!20,draw=none](-4.469,2.822)--(-4.425,2.801)--(-4.435,2.787)--(-4.493,2.784)--(-4.494,2.787)--(-4.474,2.817)--cycle;
\draw(-4.469,2.822)--(-4.425,2.801)--(-4.435,2.787);
\draw(-4.494,2.787)--(-4.474,2.817);
\filldraw[fill opacity=0.8,fill=gray!20,draw=none](-4.472,2.931)--(-4.456,2.937)--(-4.447,2.932)--(-4.457,2.926)--cycle;
\draw(-4.447,2.932)--(-4.457,2.926);
\filldraw[fill opacity=0.8,fill=gray!20,draw=none](-4.52,2.841)--(-4.524,2.846)--(-4.518,2.87)--(-4.51,2.862)--(-4.511,2.85)--cycle;
\filldraw[fill opacity=0.8,fill=gray!20,draw=none](-4.438,2.89)--(-4.516,2.859)--(-4.514,2.868)--(-4.497,2.916)--cycle;
\draw(-4.516,2.859)--(-4.514,2.868)--(-4.497,2.916);
\filldraw[fill opacity=0.8,fill=gray!20,draw=none](-4.514,2.868)--(-4.518,2.87)--(-4.525,2.877)--(-4.497,2.916)--cycle;
\draw(-4.497,2.916)--(-4.514,2.868)--(-4.518,2.87);
\filldraw[fill opacity=0.8,fill=gray!20,draw=none](-4.452,2.735)--(-4.457,2.74)--(-4.453,2.737)--cycle;
\draw(-4.457,2.74)--(-4.453,2.737);
\filldraw[fill opacity=0.8,fill=gray!20,draw=none](-4.408,2.788)--(-4.433,2.746)--(-4.452,2.734)--(-4.458,2.74)--(-4.44,2.77)--(-4.409,2.79)--cycle;
\draw(-4.433,2.746)--(-4.452,2.734);
\draw(-4.44,2.77)--(-4.409,2.79);
\filldraw[fill opacity=0.8,fill=gray!20,draw=none](-4.412,2.791)--(-4.409,2.79)--(-4.411,2.789)--cycle;
\draw(-4.409,2.79)--(-4.411,2.789);
\filldraw[fill opacity=0.8,fill=gray!20,draw=none](-4.408,2.788)--(-4.418,2.788)--(-4.418,2.789)--(-4.41,2.791)--cycle;
\draw(-4.418,2.789)--(-4.41,2.791);
\filldraw[fill opacity=0.8,fill=gray!20,draw=none](-4.398,2.771)--(-4.401,2.766)--(-4.433,2.746)--(-4.408,2.788)--cycle;
\draw(-4.401,2.766)--(-4.433,2.746);
\filldraw[fill opacity=0.8,fill=gray!20,draw=none](-4.398,2.771)--(-4.402,2.788)--(-4.373,2.79)--(-4.389,2.775)--cycle;
\filldraw[fill opacity=0.8,fill=gray!20,draw=none](-4.398,2.771)--(-4.406,2.785)--(-4.406,2.788)--(-4.402,2.788)--cycle;
\filldraw[fill opacity=0.8,fill=gray!20,draw=none](-4.406,2.785)--(-4.408,2.788)--(-4.406,2.788)--cycle;
\filldraw[fill opacity=0.8,fill=gray!20,draw=none](-4.385,2.797)--(-4.386,2.795)--(-4.404,2.793)--(-4.391,2.801)--cycle;
\draw(-4.404,2.793)--(-4.391,2.801);
\filldraw[fill opacity=0.8,fill=gray!20,draw=none](-4.385,2.797)--(-4.391,2.801)--(-4.379,2.808)--cycle;
\draw(-4.391,2.801)--(-4.379,2.808);
\filldraw[fill opacity=0.8,fill=gray!20,draw=none](-4.418,2.798)--(-4.418,2.812)--(-4.389,2.805)--(-4.387,2.798)--(-4.397,2.795)--cycle;
\draw(-4.389,2.805)--(-4.387,2.798)--(-4.397,2.795);
\filldraw[fill opacity=0.8,fill=gray!20,draw=none](-4.505,2.849)--(-4.511,2.854)--(-4.517,2.861)--cycle;
\draw(-4.505,2.849)--(-4.511,2.854)--(-4.517,2.861);
\filldraw[fill opacity=0.8,fill=gray!20,draw=none](-4.515,2.861)--(-4.532,2.856)--(-4.534,2.858)--(-4.527,2.875)--(-4.514,2.868)--cycle;
\draw(-4.527,2.875)--(-4.514,2.868)--(-4.515,2.861);
\filldraw[fill opacity=0.8,fill=gray!20,draw=none](-4.387,2.795)--(-4.397,2.795)--(-4.387,2.798)--cycle;
\draw(-4.397,2.795)--(-4.387,2.798)--(-4.387,2.795);
\filldraw[fill opacity=0.8,fill=gray!20,draw=none](-4.438,2.89)--(-4.388,2.795)--(-4.425,2.801)--(-4.471,2.822)--(-4.475,2.826)--cycle;
\draw(-4.388,2.795)--(-4.425,2.801)--(-4.471,2.822)--(-4.475,2.826);
\filldraw[fill opacity=0.8,fill=gray!20](-4.438,2.676)--(-4.463,2.694)--(-4.434,2.695)--(-4.438,2.676)--cycle;
\filldraw[fill opacity=0.8,fill=gray!20,draw=none](-4.418,2.798)--(-4.397,2.795)--(-4.418,2.788)--cycle;
\draw(-4.397,2.795)--(-4.418,2.788);
\filldraw[fill opacity=0.8,fill=gray!20,draw=none](-4.414,2.779)--(-4.452,2.775)--(-4.418,2.788)--cycle;
\draw(-4.452,2.775)--(-4.418,2.788);
\filldraw[fill opacity=0.8,fill=gray!20,draw=none](-4.408,2.788)--(-4.409,2.79)--(-4.406,2.791)--cycle;
\draw(-4.409,2.79)--(-4.406,2.791);
\filldraw[fill opacity=0.8,fill=gray!20,draw=none](-4.386,2.795)--(-4.389,2.788)--(-4.406,2.785)--(-4.408,2.788)--(-4.406,2.791)--(-4.404,2.793)--cycle;
\draw(-4.406,2.791)--(-4.404,2.793);
\filldraw[fill opacity=0.8,fill=gray!20,draw=none](-4.406,2.785)--(-4.408,2.775)--(-4.414,2.778)--(-4.418,2.788)--(-4.408,2.788)--cycle;
\filldraw[fill opacity=0.8,fill=gray!20,draw=none](-4.433,2.79)--(-4.429,2.787)--(-4.434,2.786)--cycle;
\draw(-4.429,2.787)--(-4.434,2.786);
\filldraw[fill opacity=0.8,fill=gray!20,draw=none](-4.414,2.778)--(-4.434,2.786)--(-4.429,2.787)--(-4.418,2.788)--cycle;
\draw(-4.434,2.786)--(-4.429,2.787);
\filldraw[fill opacity=0.8,fill=gray!20,draw=none](-4.389,2.775)--(-4.373,2.79)--(-4.37,2.79)--(-4.36,2.788)--cycle;
\filldraw[fill opacity=0.8,fill=gray!20,draw=none](-4.389,2.775)--(-4.36,2.788)--(-4.33,2.783)--(-4.395,2.77)--cycle;
\draw(-4.33,2.783)--(-4.395,2.77);
\filldraw[fill opacity=0.8,fill=gray!20,draw=none](-4.406,2.785)--(-4.398,2.771)--(-4.408,2.775)--cycle;
\filldraw[fill opacity=0.8,fill=gray!20,draw=none](-4.389,2.788)--(-4.398,2.771)--(-4.406,2.785)--cycle;
\filldraw[fill opacity=0.8,fill=gray!20,draw=none](-4.388,2.795)--(-4.387,2.787)--(-4.393,2.777)--(-4.412,2.779)--cycle;
\draw(-4.387,2.787)--(-4.393,2.777);
\filldraw[fill opacity=0.8,fill=gray!20,draw=none](-4.388,2.795)--(-4.382,2.794)--(-4.387,2.787)--cycle;
\draw(-4.388,2.795)--(-4.382,2.794)--(-4.387,2.787);
\filldraw[fill opacity=0.8,fill=gray!20,draw=none](-4.445,2.787)--(-4.458,2.794)--(-4.472,2.818)--(-4.438,2.793)--cycle;
\draw(-4.458,2.794)--(-4.472,2.818);
\filldraw[fill opacity=0.8,fill=gray!20,draw=none](-4.438,2.89)--(-4.475,2.826)--(-4.472,2.818)--(-4.458,2.794)--cycle;
\draw(-4.475,2.826)--(-4.472,2.818)--(-4.458,2.794);
\filldraw[fill opacity=0.8,fill=gray!20](-4.438,2.676)--(-4.434,2.695)--(-4.407,2.693)--(-4.438,2.676)--cycle;
\filldraw[fill opacity=0.8,fill=gray!20,draw=none](-4.438,2.89)--(-4.475,2.826)--(-4.505,2.849)--(-4.516,2.859)--cycle;
\draw(-4.475,2.826)--(-4.505,2.849);
\filldraw[fill opacity=0.8,fill=gray!20,draw=none](-4.518,2.87)--(-4.527,2.875)--(-4.525,2.877)--cycle;
\draw(-4.518,2.87)--(-4.527,2.875);
\filldraw[fill opacity=0.8,fill=gray!20,draw=none](-4.522,2.859)--(-4.517,2.861)--(-4.511,2.854)--cycle;
\draw(-4.517,2.861)--(-4.511,2.854);
\filldraw[fill opacity=0.8,fill=gray!20,draw=none](-4.52,2.841)--(-4.511,2.85)--(-4.513,2.831)--cycle;
\filldraw[fill opacity=0.8,fill=gray!20,draw=none](-4.47,2.767)--(-4.47,2.769)--(-4.468,2.77)--cycle;
\draw(-4.47,2.769)--(-4.468,2.77);
\filldraw[fill opacity=0.8,fill=gray!20,draw=none](-4.472,2.931)--(-4.472,2.931)--(-4.477,2.95)--(-4.456,2.937)--cycle;
\filldraw[fill opacity=0.8,fill=gray!20,draw=none](-4.518,2.87)--(-4.522,2.873)--(-4.518,2.87)--cycle;
\draw(-4.522,2.873)--(-4.518,2.87);
\filldraw[fill opacity=0.8,fill=gray!20,draw=none](-4.532,2.856)--(-4.535,2.855)--(-4.534,2.858)--cycle;
\filldraw[fill opacity=0.8,fill=gray!20,draw=none](-4.534,2.841)--(-4.536,2.866)--(-4.526,2.861)--cycle;
\filldraw[fill opacity=0.8,fill=gray!20,draw=none](-4.489,2.772)--(-4.489,2.779)--(-4.469,2.75)--(-4.481,2.76)--cycle;
\draw(-4.469,2.75)--(-4.481,2.76);
\filldraw[fill opacity=0.8,fill=gray!20,draw=none](-4.471,2.751)--(-4.472,2.753)--(-4.47,2.767)--(-4.468,2.77)--(-4.455,2.774)--cycle;
\draw(-4.468,2.77)--(-4.455,2.774);
\filldraw[fill opacity=0.8,fill=gray!20,draw=none](-4.452,2.735)--(-4.444,2.725)--(-4.451,2.732)--cycle;
\filldraw[fill opacity=0.8,fill=gray!20,draw=none](-4.452,2.734)--(-4.451,2.732)--(-4.462,2.742)--(-4.466,2.747)--(-4.466,2.747)--(-4.46,2.743)--cycle;
\draw(-4.466,2.747)--(-4.46,2.743);
\filldraw[fill opacity=0.8,fill=gray!20,draw=none](-4.466,2.747)--(-4.468,2.749)--(-4.466,2.747)--cycle;
\draw(-4.468,2.749)--(-4.466,2.747);
\filldraw[fill opacity=0.8,fill=gray!20,draw=none](-4.414,2.779)--(-4.414,2.778)--(-4.462,2.747)--(-4.466,2.746)--(-4.471,2.751)--(-4.455,2.774)--(-4.452,2.775)--cycle;
\draw(-4.462,2.747)--(-4.466,2.746);
\draw(-4.455,2.774)--(-4.452,2.775);
\filldraw[fill opacity=0.8,fill=gray!20,draw=none](-4.406,2.78)--(-4.414,2.779)--(-4.418,2.788)--(-4.4,2.794)--cycle;
\draw(-4.418,2.788)--(-4.4,2.794);
\filldraw[fill opacity=0.8,fill=gray!20,draw=none](-4.368,2.797)--(-4.394,2.772)--(-4.393,2.778)--(-4.382,2.794)--cycle;
\draw(-4.393,2.778)--(-4.382,2.794)--(-4.368,2.797);
\filldraw[fill opacity=0.8,fill=gray!20,draw=none](-4.398,2.771)--(-4.389,2.775)--(-4.395,2.77)--cycle;
\filldraw[fill opacity=0.8,fill=gray!20,draw=none](-4.452,2.735)--(-4.452,2.734)--(-4.46,2.743)--(-4.457,2.74)--cycle;
\draw(-4.46,2.743)--(-4.457,2.74);
\filldraw[fill opacity=0.8,fill=gray!20,draw=none](-4.387,2.795)--(-4.404,2.784)--(-4.4,2.794)--(-4.397,2.795)--cycle;
\draw(-4.4,2.794)--(-4.397,2.795);
\filldraw[fill opacity=0.8,fill=gray!20,draw=none](-4.388,2.795)--(-4.404,2.784)--(-4.428,2.797)--(-4.425,2.801)--cycle;
\draw(-4.428,2.797)--(-4.425,2.801)--(-4.388,2.795);
\filldraw[fill opacity=0.8,fill=gray!20,draw=none](-4.404,2.784)--(-4.412,2.779)--(-4.419,2.78)--(-4.435,2.787)--(-4.428,2.797)--cycle;
\draw(-4.435,2.787)--(-4.428,2.797);
\filldraw[fill opacity=0.8,fill=gray!20,draw=none](-4.485,2.764)--(-4.501,2.778)--(-4.494,2.787)--cycle;
\draw(-4.501,2.778)--(-4.494,2.787);
\filldraw[fill opacity=0.8,fill=gray!20,draw=none](-4.489,2.772)--(-4.53,2.83)--(-4.528,2.828)--(-4.507,2.806)--(-4.489,2.779)--cycle;
\draw(-4.53,2.83)--(-4.528,2.828);
\filldraw[fill opacity=0.8,fill=gray!20,draw=none](-4.466,2.747)--(-4.466,2.746)--(-4.472,2.753)--(-4.468,2.749)--cycle;
\draw(-4.472,2.753)--(-4.468,2.749);
\filldraw[fill opacity=0.8,fill=gray!20,draw=none](-4.52,2.841)--(-4.526,2.834)--(-4.531,2.832)--(-4.536,2.838)--(-4.527,2.848)--cycle;
\draw(-4.536,2.838)--(-4.527,2.848)--(-4.52,2.841);
\filldraw[fill opacity=0.8,fill=gray!20,draw=none](-4.52,2.841)--(-4.527,2.848)--(-4.534,2.861)--cycle;
\draw(-4.52,2.841)--(-4.527,2.848)--(-4.534,2.861);
\filldraw[fill opacity=0.8,fill=gray!20,draw=none](-4.532,2.856)--(-4.541,2.835)--(-4.543,2.836)--(-4.535,2.855)--cycle;
\draw(-4.541,2.835)--(-4.543,2.836);
\filldraw[fill opacity=0.8,fill=gray!20,draw=none](-4.479,2.785)--(-4.435,2.787)--(-4.461,2.748)--cycle;
\draw(-4.435,2.787)--(-4.461,2.748);
\filldraw[fill opacity=0.8,fill=gray!20,draw=none](-4.445,2.787)--(-4.438,2.793)--(-4.433,2.79)--(-4.434,2.786)--(-4.44,2.785)--cycle;
\draw(-4.434,2.786)--(-4.44,2.785);
\filldraw[fill opacity=0.8,fill=gray!20,draw=none](-4.419,2.78)--(-4.438,2.782)--(-4.435,2.787)--cycle;
\draw(-4.438,2.782)--(-4.435,2.787);
\filldraw[fill opacity=0.8,fill=gray!20,draw=none](-4.412,2.779)--(-4.414,2.778)--(-4.414,2.779)--cycle;
\filldraw[fill opacity=0.8,fill=gray!20,draw=none](-4.412,2.779)--(-4.461,2.747)--(-4.438,2.782)--cycle;
\draw(-4.461,2.747)--(-4.438,2.782);
\filldraw[fill opacity=0.8,fill=gray!20,draw=none](-4.466,2.747)--(-4.462,2.742)--(-4.465,2.744)--(-4.466,2.746)--cycle;
\filldraw[fill opacity=0.8,fill=gray!20,draw=none](-4.451,2.731)--(-4.461,2.728)--(-4.46,2.729)--(-4.452,2.734)--cycle;
\draw(-4.46,2.729)--(-4.452,2.734);
\filldraw[fill opacity=0.8,fill=gray!20,draw=none](-4.451,2.731)--(-4.452,2.734)--(-4.401,2.766)--cycle;
\draw(-4.452,2.734)--(-4.401,2.766);
\filldraw[fill opacity=0.8,fill=gray!20,draw=none](-4.412,2.779)--(-4.393,2.777)--(-4.397,2.771)--(-4.461,2.747)--cycle;
\draw(-4.393,2.777)--(-4.397,2.771);
\filldraw[fill opacity=0.8,fill=gray!20,draw=none](-4.436,2.764)--(-4.414,2.778)--(-4.411,2.77)--(-4.412,2.766)--cycle;
\filldraw[fill opacity=0.8,fill=gray!20,draw=none](-4.451,2.783)--(-4.434,2.786)--(-4.414,2.778)--(-4.411,2.77)--cycle;
\draw(-4.451,2.783)--(-4.434,2.786);
\filldraw[fill opacity=0.8,fill=gray!20,draw=none](-4.387,2.788)--(-4.389,2.788)--(-4.387,2.792)--cycle;
\filldraw[fill opacity=0.8,fill=gray!20,draw=none](-4.397,2.771)--(-4.395,2.77)--(-4.397,2.769)--cycle;
\draw(-4.395,2.77)--(-4.397,2.769);
\filldraw[fill opacity=0.8,fill=gray!20,draw=none](-4.394,2.772)--(-4.397,2.769)--(-4.397,2.771)--(-4.393,2.778)--cycle;
\draw(-4.397,2.771)--(-4.393,2.778);
\filldraw[fill opacity=0.8,fill=gray!20,draw=none](-4.394,2.77)--(-4.397,2.769)--(-4.395,2.77)--cycle;
\draw(-4.397,2.769)--(-4.395,2.77);
\filldraw[fill opacity=0.8,fill=gray!20,draw=none](-4.387,2.779)--(-4.387,2.774)--(-4.396,2.768)--(-4.398,2.771)--cycle;
\draw(-4.387,2.774)--(-4.396,2.768);
\filldraw[fill opacity=0.8,fill=gray!20,draw=none](-4.387,2.788)--(-4.387,2.779)--(-4.394,2.771)--(-4.394,2.779)--(-4.389,2.788)--cycle;
\filldraw[fill opacity=0.8,fill=gray!20,draw=none](-4.387,2.795)--(-4.388,2.781)--(-4.406,2.78)--(-4.404,2.784)--cycle;
\draw(-4.387,2.795)--(-4.388,2.781);
\filldraw[fill opacity=0.8,fill=gray!20,draw=none](-4.438,2.89)--(-4.458,2.794)--(-4.464,2.797)--(-4.491,2.812)--(-4.518,2.839)--cycle;
\draw(-4.464,2.797)--(-4.491,2.812)--(-4.518,2.839);
\filldraw[fill opacity=0.8,fill=gray!20,draw=none](-4.532,2.856)--(-4.515,2.861)--(-4.518,2.839)--cycle;
\draw(-4.515,2.861)--(-4.518,2.839);
\filldraw[fill opacity=0.8,fill=gray!20,draw=none](-4.527,2.857)--(-4.526,2.861)--(-4.522,2.859)--cycle;
\filldraw[fill opacity=0.8,fill=gray!20,draw=none](-4.543,2.835)--(-4.547,2.845)--(-4.541,2.874)--(-4.527,2.848)--(-4.539,2.835)--cycle;
\draw(-4.541,2.874)--(-4.527,2.848)--(-4.539,2.835);
\filldraw[fill opacity=0.8,fill=gray!20,draw=none](-4.526,2.834)--(-4.52,2.841)--(-4.513,2.831)--(-4.515,2.817)--(-4.528,2.828)--cycle;
\draw(-4.515,2.817)--(-4.528,2.828);
\filldraw[fill opacity=0.8,fill=gray!20,draw=none](-4.519,2.836)--(-4.516,2.837)--(-4.491,2.812)--(-4.497,2.804)--(-4.519,2.821)--cycle;
\draw(-4.516,2.837)--(-4.491,2.812)--(-4.497,2.804);
\filldraw[fill opacity=0.8,fill=gray!20,draw=none](-4.52,2.816)--(-4.505,2.795)--(-4.512,2.787)--(-4.521,2.795)--cycle;
\draw(-4.505,2.795)--(-4.512,2.787);
\filldraw[fill opacity=0.8,fill=gray!20,draw=none](-4.5,2.776)--(-4.512,2.787)--(-4.505,2.795)--cycle;
\draw(-4.512,2.787)--(-4.505,2.795);
\filldraw[fill opacity=0.8,fill=gray!20,draw=none](-4.527,2.827)--(-4.481,2.76)--(-4.523,2.795)--cycle;
\draw(-4.481,2.76)--(-4.523,2.795);
\filldraw[fill opacity=0.8,fill=gray!20,draw=none](-4.479,2.785)--(-4.461,2.748)--(-4.461,2.747)--(-4.466,2.746)--(-4.485,2.764)--(-4.493,2.784)--cycle;
\draw(-4.461,2.748)--(-4.461,2.747);
\filldraw[fill opacity=0.8,fill=gray!20,draw=none](-4.466,2.746)--(-4.466,2.745)--(-4.485,2.764)--(-4.472,2.753)--cycle;
\draw(-4.485,2.764)--(-4.472,2.753);
\filldraw[fill opacity=0.8,fill=gray!20,draw=none](-4.471,2.751)--(-4.473,2.748)--(-4.472,2.753)--cycle;
\filldraw[fill opacity=0.8,fill=gray!20,draw=none](-4.466,2.746)--(-4.474,2.743)--(-4.473,2.748)--(-4.471,2.751)--cycle;
\draw(-4.466,2.746)--(-4.474,2.743);
\filldraw[fill opacity=0.8,fill=gray!20,draw=none](-4.466,2.746)--(-4.477,2.741)--(-4.485,2.764)--cycle;
\filldraw[fill opacity=0.8,fill=gray!20,draw=none](-4.466,2.746)--(-4.465,2.744)--(-4.466,2.745)--cycle;
\filldraw[fill opacity=0.8,fill=gray!20,draw=none](-4.466,2.746)--(-4.461,2.747)--(-4.465,2.744)--cycle;
\filldraw[fill opacity=0.8,fill=gray!20,draw=none](-4.477,2.741)--(-4.474,2.743)--(-4.462,2.747)--cycle;
\draw(-4.474,2.743)--(-4.462,2.747);
\filldraw[fill opacity=0.8,fill=gray!20,draw=none](-4.472,2.738)--(-4.461,2.747)--(-4.469,2.737)--cycle;
\draw(-4.461,2.747)--(-4.469,2.737);
\filldraw[fill opacity=0.8,fill=gray!20,draw=none](-4.397,2.771)--(-4.401,2.766)--(-4.451,2.731)--(-4.464,2.744)--(-4.461,2.747)--cycle;
\draw(-4.397,2.771)--(-4.401,2.766);
\draw(-4.464,2.744)--(-4.461,2.747);
\filldraw[fill opacity=0.8,fill=gray!20,draw=none](-4.466,2.746)--(-4.465,2.744)--(-4.472,2.738)--(-4.476,2.74)--(-4.477,2.741)--cycle;
\filldraw[fill opacity=0.8,fill=gray!20,draw=none](-4.451,2.731)--(-4.469,2.737)--(-4.464,2.744)--cycle;
\draw(-4.469,2.737)--(-4.464,2.744);
\filldraw[fill opacity=0.8,fill=gray!20,draw=none](-4.466,2.745)--(-4.462,2.742)--(-4.456,2.735)--(-4.465,2.743)--cycle;
\draw(-4.456,2.735)--(-4.465,2.743);
\filldraw[fill opacity=0.8,fill=gray!20,draw=none](-4.465,2.743)--(-4.477,2.741)--(-4.477,2.741)--(-4.462,2.747)--(-4.414,2.764)--cycle;
\draw(-4.462,2.747)--(-4.414,2.764);
\filldraw[fill opacity=0.8,fill=gray!20,draw=none](-4.489,2.767)--(-4.485,2.764)--(-4.466,2.745)--(-4.465,2.743)--cycle;
\draw(-4.489,2.767)--(-4.485,2.764);
\filldraw[fill opacity=0.8,fill=gray!20,draw=none](-4.447,2.787)--(-4.476,2.753)--(-4.5,2.776)--(-4.505,2.795)--(-4.491,2.812)--cycle;
\draw(-4.505,2.795)--(-4.491,2.812)--(-4.447,2.787)--(-4.476,2.753);
\filldraw[fill opacity=0.8,fill=gray!20,draw=none](-4.518,2.839)--(-4.516,2.837)--(-4.519,2.836)--cycle;
\draw(-4.518,2.839)--(-4.516,2.837);
\filldraw[fill opacity=0.8,fill=gray!20,draw=none](-4.445,2.787)--(-4.447,2.787)--(-4.464,2.797)--cycle;
\draw(-4.445,2.787)--(-4.447,2.787)--(-4.464,2.797);
\filldraw[fill opacity=0.8,fill=gray!20,draw=none](-4.445,2.787)--(-4.44,2.785)--(-4.451,2.783)--cycle;
\draw(-4.44,2.785)--(-4.451,2.783);
\filldraw[fill opacity=0.8,fill=gray!20,draw=none](-4.409,2.769)--(-4.411,2.77)--(-4.414,2.778)--(-4.408,2.775)--cycle;
\filldraw[fill opacity=0.8,fill=gray!20,draw=none](-4.412,2.779)--(-4.406,2.78)--(-4.411,2.77)--(-4.414,2.778)--cycle;
\filldraw[fill opacity=0.8,fill=gray!20,draw=none](-4.396,2.768)--(-4.401,2.766)--(-4.398,2.771)--cycle;
\draw(-4.396,2.768)--(-4.401,2.766);
\filldraw[fill opacity=0.8,fill=gray!20,draw=none](-4.397,2.769)--(-4.398,2.769)--(-4.397,2.771)--cycle;
\draw(-4.398,2.769)--(-4.397,2.771);
\filldraw[fill opacity=0.8,fill=gray!20,draw=none](-4.409,2.769)--(-4.408,2.775)--(-4.397,2.771)--(-4.397,2.769)--(-4.405,2.768)--cycle;
\draw(-4.397,2.769)--(-4.405,2.768);
\filldraw[fill opacity=0.8,fill=gray!20,draw=none](-4.401,2.779)--(-4.409,2.769)--(-4.45,2.783)--(-4.447,2.787)--cycle;
\draw(-4.45,2.783)--(-4.447,2.787)--(-4.401,2.779)--(-4.409,2.769);
\filldraw[fill opacity=0.8,fill=gray!20,draw=none](-4.458,2.794)--(-4.445,2.787)--(-4.451,2.783)--cycle;
\draw(-4.451,2.783)--(-4.458,2.794);
\filldraw[fill opacity=0.8,fill=gray!20,draw=none](-4.438,2.89)--(-4.388,2.781)--(-4.406,2.78)--(-4.445,2.787)--(-4.458,2.794)--cycle;
\draw(-4.406,2.78)--(-4.445,2.787);
\filldraw[fill opacity=0.8,fill=gray!20,draw=none](-4.436,2.764)--(-4.412,2.766)--(-4.414,2.764)--(-4.462,2.747)--cycle;
\draw(-4.414,2.764)--(-4.462,2.747);
\filldraw[fill opacity=0.8,fill=gray!20,draw=none](-4.409,2.769)--(-4.411,2.767)--(-4.473,2.756)--(-4.45,2.783)--cycle;
\draw(-4.409,2.769)--(-4.411,2.767);
\draw(-4.473,2.756)--(-4.45,2.783);
\filldraw[fill opacity=0.8,fill=gray!20,draw=none](-4.394,2.771)--(-4.396,2.768)--(-4.398,2.771)--(-4.394,2.779)--cycle;
\filldraw[fill opacity=0.8,fill=gray!20,draw=none](-4.41,2.767)--(-4.411,2.767)--(-4.409,2.769)--cycle;
\draw(-4.411,2.767)--(-4.409,2.769);
\filldraw[fill opacity=0.8,fill=gray!20,draw=none](-4.412,2.766)--(-4.411,2.77)--(-4.41,2.767)--cycle;
\filldraw[fill opacity=0.8,fill=gray!20,draw=none](-4.451,2.783)--(-4.411,2.77)--(-4.41,2.767)--(-4.428,2.763)--cycle;
\draw(-4.41,2.767)--(-4.428,2.763)--(-4.451,2.783);
\filldraw[fill opacity=0.8,fill=gray!20,draw=none](-4.388,2.781)--(-4.401,2.779)--(-4.406,2.78)--cycle;
\draw(-4.388,2.781)--(-4.401,2.779)--(-4.406,2.78);
\filldraw[fill opacity=0.8,fill=gray!20,draw=none](-4.409,2.769)--(-4.405,2.768)--(-4.41,2.767)--cycle;
\draw(-4.405,2.768)--(-4.41,2.767);
\filldraw[fill opacity=0.8,fill=gray!20,draw=none](-4.388,2.781)--(-4.41,2.767)--(-4.409,2.769)--(-4.401,2.779)--cycle;
\draw(-4.409,2.769)--(-4.401,2.779)--(-4.388,2.781);
\filldraw[fill opacity=0.8,fill=gray!20,draw=none](-4.411,2.77)--(-4.409,2.769)--(-4.41,2.767)--cycle;
\filldraw[fill opacity=0.8,fill=gray!20,draw=none](-4.406,2.78)--(-4.388,2.781)--(-4.408,2.767)--(-4.41,2.767)--(-4.411,2.77)--cycle;
\filldraw[fill opacity=0.8,fill=gray!20,draw=none](-4.377,2.771)--(-4.394,2.77)--(-4.395,2.77)--(-4.379,2.773)--cycle;
\draw(-4.395,2.77)--(-4.379,2.773);
\filldraw[fill opacity=0.8,fill=gray!20,draw=none](-4.394,2.77)--(-4.397,2.769)--(-4.387,2.779)--(-4.387,2.774)--cycle;
\filldraw[fill opacity=0.8,fill=gray!20,draw=none](-4.408,2.767)--(-4.388,2.781)--(-4.388,2.773)--(-4.405,2.767)--cycle;
\draw(-4.388,2.781)--(-4.388,2.773)--(-4.405,2.767);
\filldraw[fill opacity=0.8,fill=gray!20,draw=none](-4.397,2.769)--(-4.401,2.766)--(-4.398,2.769)--cycle;
\draw(-4.401,2.766)--(-4.398,2.769);
\filldraw[fill opacity=0.8,fill=gray!20,draw=none](-4.397,2.769)--(-4.394,2.77)--(-4.401,2.766)--cycle;
\filldraw[fill opacity=0.8,fill=gray!20,draw=none](-4.394,2.77)--(-4.387,2.77)--(-4.388,2.76)--(-4.413,2.747)--(-4.401,2.766)--cycle;
\draw(-4.413,2.747)--(-4.401,2.766);
\filldraw[fill opacity=0.8,fill=gray!20,draw=none](-4.401,2.766)--(-4.429,2.724)--(-4.451,2.731)--cycle;
\draw(-4.401,2.766)--(-4.429,2.724);
\filldraw[fill opacity=0.8,fill=gray!20,draw=none](-4.421,2.748)--(-4.442,2.737)--(-4.401,2.766)--(-4.394,2.77)--cycle;
\draw(-4.401,2.766)--(-4.394,2.77);
\filldraw[fill opacity=0.8,fill=gray!20,draw=none](-4.394,2.77)--(-4.397,2.769)--(-4.394,2.77)--cycle;
\filldraw[fill opacity=0.8,fill=gray!20,draw=none](-4.392,2.768)--(-4.414,2.752)--(-4.421,2.748)--(-4.394,2.77)--cycle;
\filldraw[fill opacity=0.8,fill=gray!20,draw=none](-4.405,2.767)--(-4.41,2.767)--(-4.388,2.781)--(-4.389,2.772)--cycle;
\filldraw[fill opacity=0.8,fill=gray!20,draw=none](-4.392,2.768)--(-4.405,2.767)--(-4.405,2.768)--(-4.397,2.769)--(-4.394,2.77)--cycle;
\draw(-4.405,2.768)--(-4.397,2.769);
\filldraw[fill opacity=0.8,fill=gray!20,draw=none](-4.405,2.767)--(-4.41,2.767)--(-4.405,2.768)--cycle;
\draw(-4.41,2.767)--(-4.405,2.768);
\filldraw[fill opacity=0.8,fill=gray!20,draw=none](-4.41,2.767)--(-4.414,2.764)--(-4.411,2.767)--cycle;
\draw(-4.414,2.764)--(-4.411,2.767);
\filldraw[fill opacity=0.8,fill=gray!20,draw=none](-4.421,2.748)--(-4.437,2.736)--(-4.441,2.734)--(-4.451,2.731)--(-4.442,2.737)--cycle;
\draw(-4.437,2.736)--(-4.441,2.734);
\filldraw[fill opacity=0.8,fill=gray!20,draw=none](-4.411,2.767)--(-4.428,2.746)--(-4.456,2.74)--(-4.48,2.748)--(-4.473,2.756)--cycle;
\draw(-4.411,2.767)--(-4.428,2.746);
\draw(-4.48,2.748)--(-4.473,2.756);
\filldraw[fill opacity=0.8,fill=gray!20,draw=none](-4.412,2.766)--(-4.41,2.767)--(-4.409,2.766)--(-4.414,2.764)--cycle;
\draw(-4.409,2.766)--(-4.414,2.764);
\filldraw[fill opacity=0.8,fill=gray!20,draw=none](-4.408,2.767)--(-4.409,2.766)--(-4.41,2.767)--cycle;
\filldraw[fill opacity=0.8,fill=gray!20,draw=none](-4.41,2.767)--(-4.405,2.767)--(-4.414,2.764)--cycle;
\filldraw[fill opacity=0.8,fill=gray!20,draw=none](-4.414,2.752)--(-4.437,2.736)--(-4.421,2.748)--cycle;
\filldraw[fill opacity=0.8,fill=gray!20,draw=none](-4.405,2.767)--(-4.392,2.768)--(-4.413,2.753)--(-4.429,2.746)--(-4.414,2.764)--cycle;
\draw(-4.429,2.746)--(-4.414,2.764);
\filldraw[fill opacity=0.8,fill=gray!20,draw=none](-4.405,2.767)--(-4.404,2.763)--(-4.456,2.745)--(-4.465,2.743)--(-4.414,2.764)--cycle;
\draw(-4.404,2.763)--(-4.456,2.745);
\filldraw[fill opacity=0.8,fill=gray!20,draw=none](-4.405,2.767)--(-4.414,2.764)--(-4.405,2.767)--cycle;
\draw(-4.414,2.764)--(-4.405,2.767);
\filldraw[fill opacity=0.8,fill=gray!20,draw=none](-4.408,2.767)--(-4.405,2.767)--(-4.409,2.766)--cycle;
\draw(-4.405,2.767)--(-4.409,2.766);
\filldraw[fill opacity=0.8,fill=gray!20,draw=none](-4.428,2.763)--(-4.41,2.767)--(-4.405,2.767)--(-4.406,2.763)--cycle;
\draw(-4.406,2.763)--(-4.428,2.763)--(-4.41,2.767);
\filldraw[fill opacity=0.8,fill=gray!20,draw=none](-4.377,2.771)--(-4.394,2.77)--(-4.394,2.77)--(-4.377,2.771)--cycle;
\filldraw[fill opacity=0.8,fill=gray!20,draw=none](-4.392,2.768)--(-4.394,2.77)--(-4.387,2.77)--(-4.39,2.768)--cycle;
\filldraw[fill opacity=0.8,fill=gray!20,draw=none](-4.394,2.77)--(-4.387,2.774)--(-4.387,2.77)--cycle;
\filldraw[fill opacity=0.8,fill=gray!20,draw=none](-4.387,2.774)--(-4.39,2.769)--(-4.392,2.768)--(-4.394,2.77)--cycle;
\filldraw[fill opacity=0.8,fill=gray!20,draw=none](-4.392,2.768)--(-4.405,2.767)--(-4.389,2.772)--(-4.39,2.769)--cycle;
\filldraw[fill opacity=0.8,fill=gray!20,draw=none](-4.405,2.767)--(-4.392,2.768)--(-4.391,2.768)--(-4.395,2.765)--(-4.406,2.763)--cycle;
\draw(-4.395,2.765)--(-4.406,2.763);
\filldraw[fill opacity=0.8,fill=gray!20,draw=none](-4.397,2.766)--(-4.404,2.763)--(-4.405,2.767)--(-4.388,2.773)--cycle;
\draw(-4.405,2.767)--(-4.388,2.773)--(-4.397,2.766);
\filldraw[fill opacity=0.8,fill=gray!20,draw=none](-4.438,2.89)--(-4.458,2.794)--(-4.451,2.783)--(-4.428,2.763)--(-4.406,2.763)--(-4.401,2.769)--cycle;
\draw(-4.458,2.794)--(-4.451,2.783)--(-4.428,2.763)--(-4.406,2.763)--(-4.401,2.769);
\filldraw[fill opacity=0.8,fill=gray!20,draw=none](-4.497,2.804)--(-4.505,2.795)--(-4.527,2.827)--cycle;
\draw(-4.497,2.804)--(-4.505,2.795);
\filldraw[fill opacity=0.8,fill=gray!20,draw=none](-4.438,2.89)--(-4.426,2.799)--(-4.44,2.786)--(-4.468,2.771)--(-4.495,2.772)--(-4.514,2.79)--(-4.519,2.821)--(-4.518,2.839)--cycle;
\draw(-4.44,2.786)--(-4.468,2.771)--(-4.495,2.772)--(-4.514,2.79)--(-4.519,2.821);
\filldraw[fill opacity=0.8,fill=gray!20,draw=none](-4.53,2.83)--(-4.533,2.83)--(-4.539,2.835)--(-4.536,2.838)--cycle;
\draw(-4.539,2.835)--(-4.536,2.838);
\filldraw[fill opacity=0.8,fill=gray!20,draw=none](-4.532,2.856)--(-4.524,2.846)--(-4.526,2.834)--(-4.534,2.831)--(-4.541,2.835)--cycle;
\draw(-4.534,2.831)--(-4.541,2.835);
\filldraw[fill opacity=0.8,fill=gray!20,draw=none](-4.526,2.834)--(-4.528,2.828)--(-4.534,2.831)--cycle;
\draw(-4.528,2.828)--(-4.534,2.831);
\filldraw[fill opacity=0.8,fill=gray!20,draw=none](-4.526,2.834)--(-4.53,2.83)--(-4.531,2.832)--cycle;
\filldraw[fill opacity=0.8,fill=gray!20,draw=none](-4.522,2.873)--(-4.518,2.87)--(-4.526,2.834)--(-4.53,2.83)--(-4.534,2.834)--cycle;
\draw(-4.53,2.83)--(-4.534,2.834)--(-4.522,2.873);
\filldraw[fill opacity=0.8,fill=gray!20,draw=none](-4.438,2.89)--(-4.518,2.839)--(-4.516,2.859)--cycle;
\draw(-4.518,2.839)--(-4.516,2.859);
\filldraw[fill opacity=0.8,fill=gray!20,draw=none](-4.52,2.841)--(-4.518,2.839)--(-4.519,2.836)--(-4.526,2.834)--cycle;
\draw(-4.52,2.841)--(-4.518,2.839);
\filldraw[fill opacity=0.8,fill=gray!20,draw=none](-4.438,2.89)--(-4.518,2.839)--(-4.52,2.841)--(-4.528,2.853)--cycle;
\draw(-4.518,2.839)--(-4.52,2.841);
\filldraw[fill opacity=0.8,fill=gray!20,draw=none](-4.52,2.841)--(-4.526,2.834)--(-4.524,2.846)--cycle;
\filldraw[fill opacity=0.8,fill=gray!20,draw=none](-4.526,2.834)--(-4.524,2.846)--(-4.518,2.839)--cycle;
\filldraw[fill opacity=0.8,fill=gray!20,draw=none](-4.526,2.834)--(-4.518,2.839)--(-4.52,2.824)--(-4.528,2.828)--cycle;
\draw(-4.518,2.839)--(-4.52,2.824)--(-4.528,2.828);
\filldraw[fill opacity=0.8,fill=gray!20,draw=none](-4.519,2.821)--(-4.52,2.824)--(-4.518,2.839)--cycle;
\draw(-4.519,2.821)--(-4.52,2.824)--(-4.518,2.839);
\filldraw[fill opacity=0.8,fill=gray!20,draw=none](-4.514,2.79)--(-4.523,2.795)--(-4.528,2.828)--(-4.52,2.824)--cycle;
\draw(-4.528,2.828)--(-4.52,2.824)--(-4.514,2.79)--(-4.523,2.795);
\filldraw[fill opacity=0.8,fill=gray!20,draw=none](-4.526,2.834)--(-4.528,2.828)--(-4.53,2.83)--cycle;
\draw(-4.528,2.828)--(-4.53,2.83);
\filldraw[fill opacity=0.8,fill=gray!20,draw=none](-4.526,2.834)--(-4.519,2.836)--(-4.519,2.821)--(-4.527,2.827)--(-4.53,2.83)--cycle;
\filldraw[fill opacity=0.8,fill=gray!20,draw=none](-4.533,2.83)--(-4.542,2.831)--(-4.539,2.835)--cycle;
\draw(-4.542,2.831)--(-4.539,2.835);
\filldraw[fill opacity=0.8,fill=gray!20,draw=none](-4.543,2.835)--(-4.539,2.835)--(-4.542,2.831)--cycle;
\draw(-4.539,2.835)--(-4.542,2.831);
\filldraw[fill opacity=0.8,fill=gray!20,draw=none](-4.543,2.832)--(-4.543,2.836)--(-4.528,2.828)--cycle;
\draw(-4.543,2.836)--(-4.528,2.828);
\filldraw[fill opacity=0.8,fill=gray!20,draw=none](-4.523,2.795)--(-4.533,2.804)--(-4.535,2.83)--(-4.528,2.828)--cycle;
\filldraw[fill opacity=0.8,fill=gray!20,draw=none](-4.531,2.829)--(-4.53,2.83)--(-4.527,2.827)--cycle;
\filldraw[fill opacity=0.8,fill=gray!20,draw=none](-4.533,2.83)--(-4.53,2.83)--(-4.531,2.829)--cycle;
\filldraw[fill opacity=0.8,fill=gray!20,draw=none](-4.534,2.834)--(-4.53,2.83)--(-4.527,2.827)--(-4.523,2.795)--(-4.532,2.802)--cycle;
\draw(-4.523,2.795)--(-4.532,2.802)--(-4.534,2.834)--(-4.53,2.83);
\filldraw[fill opacity=0.8,fill=gray!20,draw=none](-4.476,2.753)--(-4.48,2.748)--(-4.495,2.756)--(-4.5,2.776)--cycle;
\draw(-4.476,2.753)--(-4.48,2.748);
\filldraw[fill opacity=0.8,fill=gray!20,draw=none](-4.532,2.802)--(-4.5,2.776)--(-4.48,2.757)--(-4.506,2.777)--(-4.515,2.784)--cycle;
\draw(-4.506,2.777)--(-4.515,2.784)--(-4.532,2.802)--(-4.5,2.776);
\filldraw[fill opacity=0.8,fill=gray!20,draw=none](-4.495,2.772)--(-4.506,2.777)--(-4.523,2.795)--(-4.514,2.79)--cycle;
\draw(-4.523,2.795)--(-4.514,2.79)--(-4.495,2.772)--(-4.506,2.777);
\filldraw[fill opacity=0.8,fill=gray!20,draw=none](-4.438,2.89)--(-4.528,2.853)--(-4.534,2.834)--(-4.532,2.802)--(-4.515,2.784)--(-4.487,2.783)--(-4.451,2.797)--(-4.433,2.811)--cycle;
\draw(-4.528,2.853)--(-4.534,2.834)--(-4.532,2.802)--(-4.515,2.784)--(-4.487,2.783)--(-4.451,2.797)--(-4.433,2.811);
\filldraw[fill opacity=0.8,fill=gray!20,draw=none](-4.438,2.676)--(-4.413,2.69)--(-4.397,2.686)--(-4.438,2.676)--cycle;
\draw(-4.397,2.686)--(-4.438,2.676)--(-4.438,2.676)--(-4.413,2.69);
\filldraw[fill opacity=0.8,fill=gray!20](-4.377,2.724)--(-4.352,2.767)--(-4.299,2.754)--(-4.34,2.715)--cycle;
\filldraw[fill opacity=0.8,fill=gray!20,draw=none](-4.403,2.693)--(-4.405,2.695)--(-4.377,2.724)--(-4.34,2.715)--(-4.387,2.689)--cycle;
\draw(-4.405,2.695)--(-4.377,2.724)--(-4.34,2.715)--(-4.387,2.689)--(-4.403,2.693);
\filldraw[fill opacity=0.8,fill=gray!20,draw=none](-4.403,2.693)--(-4.405,2.695)--(-4.404,2.694)--cycle;
\filldraw[fill opacity=0.8,fill=gray!20,draw=none](-4.4,2.689)--(-4.403,2.693)--(-4.404,2.694)--(-4.401,2.691)--cycle;
\filldraw[fill opacity=0.8,fill=gray!20,draw=none](-4.382,2.684)--(-4.387,2.689)--(-4.34,2.715)--(-4.328,2.703)--(-4.379,2.683)--cycle;
\draw(-4.382,2.684)--(-4.387,2.689)--(-4.34,2.715)--(-4.328,2.703)--(-4.379,2.683);
\filldraw[fill opacity=0.8,fill=gray!20,draw=none](-4.403,2.693)--(-4.387,2.689)--(-4.397,2.686)--cycle;
\draw(-4.403,2.693)--(-4.387,2.689)--(-4.397,2.686);
\filldraw[fill opacity=0.8,fill=gray!20,draw=none](-4.403,2.693)--(-4.395,2.684)--(-4.397,2.686)--cycle;
\draw(-4.395,2.684)--(-4.397,2.686);
\filldraw[fill opacity=0.8,fill=gray!20,draw=none](-4.4,2.689)--(-4.401,2.691)--(-4.391,2.681)--(-4.395,2.684)--cycle;
\draw(-4.391,2.681)--(-4.395,2.684);
\filldraw[fill opacity=0.8,fill=gray!20,draw=none](-4.438,2.676)--(-4.397,2.686)--(-4.393,2.681)--(-4.438,2.676)--cycle;
\draw(-4.393,2.681)--(-4.438,2.676)--(-4.438,2.676)--(-4.397,2.686);
\filldraw[fill opacity=0.8,fill=gray!20,draw=none](-4.397,2.686)--(-4.387,2.689)--(-4.382,2.684)--cycle;
\draw(-4.397,2.686)--(-4.387,2.689)--(-4.382,2.684);
\filldraw[fill opacity=0.8,fill=gray!20,draw=none](-4.5,2.776)--(-4.489,2.767)--(-4.476,2.753)--cycle;
\draw(-4.5,2.776)--(-4.489,2.767);
\filldraw[fill opacity=0.8,fill=gray!20,draw=none](-4.38,2.683)--(-4.391,2.681)--(-4.347,2.666)--(-4.364,2.681)--cycle;
\draw(-4.347,2.666)--(-4.364,2.681);
\filldraw[fill opacity=0.8,fill=gray!20,draw=none](-4.38,2.683)--(-4.364,2.681)--(-4.37,2.685)--cycle;
\draw(-4.364,2.681)--(-4.37,2.685);
\filldraw[fill opacity=0.8,fill=gray!20,draw=none](-4.384,2.681)--(-4.381,2.682)--(-4.328,2.703)--(-4.341,2.695)--cycle;
\draw(-4.384,2.681)--(-4.381,2.682)--(-4.328,2.703)--(-4.341,2.695);
\filldraw[fill opacity=0.8,fill=gray!20,draw=none](-4.422,2.706)--(-4.397,2.686)--(-4.38,2.683)--(-4.37,2.685)--(-4.407,2.716)--cycle;
\draw(-4.422,2.706)--(-4.397,2.686);
\draw(-4.37,2.685)--(-4.407,2.716);
\filldraw[fill opacity=0.8,fill=gray!20,draw=none](-4.388,2.76)--(-4.388,2.759)--(-4.422,2.723)--(-4.429,2.724)--(-4.413,2.747)--cycle;
\draw(-4.429,2.724)--(-4.413,2.747);
\filldraw[fill opacity=0.8,fill=gray!20,draw=none](-4.451,2.731)--(-4.471,2.717)--(-4.461,2.728)--cycle;
\filldraw[fill opacity=0.8,fill=gray!20,draw=none](-4.388,2.759)--(-4.388,2.739)--(-4.395,2.73)--(-4.415,2.723)--(-4.422,2.723)--cycle;
\draw(-4.388,2.739)--(-4.395,2.73);
\filldraw[fill opacity=0.8,fill=gray!20,draw=none](-4.422,2.723)--(-4.434,2.716)--(-4.422,2.706)--(-4.407,2.716)--(-4.415,2.723)--cycle;
\draw(-4.434,2.716)--(-4.422,2.706);
\draw(-4.407,2.716)--(-4.415,2.723);
\filldraw[fill opacity=0.8,fill=gray!20,draw=none](-4.438,2.676)--(-4.472,2.677)--(-4.471,2.68)--(-4.438,2.676)--cycle;
\draw(-4.471,2.68)--(-4.438,2.676)--(-4.438,2.676)--(-4.472,2.677);
\filldraw[fill opacity=0.8,fill=gray!20](-4.438,2.676)--(-4.381,2.682)--(-4.39,2.676)--(-4.438,2.676)--cycle;
\filldraw[fill opacity=0.8,fill=gray!20,draw=none](-4.438,2.676)--(-4.462,2.673)--(-4.476,2.677)--(-4.438,2.676)--cycle;
\draw(-4.476,2.677)--(-4.438,2.676)--(-4.438,2.676)--(-4.462,2.673);
\filldraw[fill opacity=0.8,fill=gray!20](-4.438,2.676)--(-4.39,2.676)--(-4.412,2.672)--(-4.438,2.676)--cycle;
\filldraw[fill opacity=0.8,fill=gray!20](-4.438,2.676)--(-4.412,2.672)--(-4.441,2.671)--(-4.438,2.676)--cycle;
\filldraw[fill opacity=0.8,fill=gray!20,draw=none](-4.438,2.676)--(-4.441,2.671)--(-4.448,2.671)--(-4.462,2.673)--(-4.438,2.676)--cycle;
\draw(-4.462,2.673)--(-4.438,2.676)--(-4.438,2.676)--(-4.441,2.671)--(-4.448,2.671);
\filldraw[fill opacity=0.8,fill=gray!20,draw=none](-4.462,2.742)--(-4.451,2.732)--(-4.451,2.731)--(-4.456,2.735)--cycle;
\draw(-4.451,2.731)--(-4.456,2.735);
\filldraw[fill opacity=0.8,fill=gray!20,draw=none](-4.472,2.738)--(-4.475,2.735)--(-4.476,2.74)--cycle;
\filldraw[fill opacity=0.8,fill=gray!20,draw=none](-4.465,2.743)--(-4.49,2.733)--(-4.477,2.741)--cycle;
\filldraw[fill opacity=0.8,fill=gray!20,draw=none](-4.472,2.738)--(-4.469,2.737)--(-4.473,2.73)--(-4.475,2.735)--cycle;
\draw(-4.469,2.737)--(-4.473,2.73);
\filldraw[fill opacity=0.8,fill=gray!20,draw=none](-4.451,2.732)--(-4.442,2.723)--(-4.451,2.731)--cycle;
\draw(-4.442,2.723)--(-4.451,2.731);
\filldraw[fill opacity=0.8,fill=gray!20,draw=none](-4.441,2.734)--(-4.451,2.731)--(-4.442,2.723)--cycle;
\draw(-4.451,2.731)--(-4.442,2.723);
\filldraw[fill opacity=0.8,fill=gray!20,draw=none](-4.468,2.719)--(-4.471,2.717)--(-4.488,2.704)--(-4.489,2.705)--cycle;
\filldraw[fill opacity=0.8,fill=gray!20,draw=none](-4.441,2.728)--(-4.463,2.717)--(-4.482,2.717)--(-4.469,2.737)--cycle;
\draw(-4.482,2.717)--(-4.469,2.737);
\filldraw[fill opacity=0.8,fill=gray!20,draw=none](-4.44,2.734)--(-4.441,2.734)--(-4.442,2.723)--(-4.434,2.716)--(-4.418,2.725)--(-4.427,2.733)--cycle;
\draw(-4.442,2.723)--(-4.434,2.716);
\draw(-4.418,2.725)--(-4.427,2.733);
\filldraw[fill opacity=0.8,fill=gray!20,draw=none](-4.48,2.757)--(-4.476,2.753)--(-4.465,2.743)--(-4.506,2.777)--cycle;
\draw(-4.465,2.743)--(-4.506,2.777);
\filldraw[fill opacity=0.8,fill=gray!20,draw=none](-4.451,2.731)--(-4.441,2.734)--(-4.457,2.724)--(-4.471,2.717)--cycle;
\draw(-4.441,2.734)--(-4.457,2.724);
\filldraw[fill opacity=0.8,fill=gray!20,draw=none](-4.428,2.746)--(-4.437,2.736)--(-4.441,2.735)--(-4.456,2.74)--cycle;
\draw(-4.428,2.746)--(-4.437,2.736);
\filldraw[fill opacity=0.8,fill=gray!20,draw=none](-4.456,2.754)--(-4.477,2.753)--(-4.456,2.735)--cycle;
\draw(-4.477,2.753)--(-4.456,2.735);
\filldraw[fill opacity=0.8,fill=gray!20,draw=none](-4.456,2.74)--(-4.487,2.733)--(-4.493,2.733)--(-4.48,2.748)--cycle;
\draw(-4.493,2.733)--(-4.48,2.748);
\filldraw[fill opacity=0.8,fill=gray!20,draw=none](-4.441,2.734)--(-4.455,2.735)--(-4.455,2.734)--(-4.451,2.731)--(-4.441,2.734)--cycle;
\draw(-4.455,2.734)--(-4.451,2.731);
\filldraw[fill opacity=0.8,fill=gray!20,draw=none](-4.463,2.717)--(-4.488,2.704)--(-4.488,2.704)--(-4.489,2.706)--(-4.482,2.717)--cycle;
\draw(-4.489,2.706)--(-4.482,2.717);
\filldraw[fill opacity=0.8,fill=gray!20,draw=none](-4.456,2.735)--(-4.455,2.734)--(-4.455,2.735)--cycle;
\draw(-4.456,2.735)--(-4.455,2.734);
\filldraw[fill opacity=0.8,fill=gray!20,draw=none](-4.48,2.748)--(-4.49,2.736)--(-4.495,2.756)--cycle;
\draw(-4.48,2.748)--(-4.49,2.736);
\filldraw[fill opacity=0.8,fill=gray!20,draw=none](-4.456,2.745)--(-4.485,2.734)--(-4.49,2.733)--(-4.465,2.743)--cycle;
\draw(-4.456,2.745)--(-4.485,2.734);
\filldraw[fill opacity=0.8,fill=gray!20,draw=none](-4.403,2.77)--(-4.412,2.777)--(-4.441,2.767)--(-4.455,2.756)--(-4.456,2.745)--(-4.413,2.76)--cycle;
\draw(-4.403,2.77)--(-4.412,2.777)--(-4.441,2.767);
\draw(-4.456,2.745)--(-4.413,2.76);
\filldraw[fill opacity=0.8,fill=gray!20,draw=none](-4.456,2.754)--(-4.456,2.735)--(-4.455,2.735)--(-4.452,2.754)--cycle;
\filldraw[fill opacity=0.8,fill=gray!20,draw=none](-4.413,2.753)--(-4.432,2.739)--(-4.437,2.736)--(-4.429,2.746)--cycle;
\draw(-4.437,2.736)--(-4.429,2.746);
\filldraw[fill opacity=0.8,fill=gray!20,draw=none](-4.455,2.735)--(-4.441,2.734)--(-4.44,2.744)--(-4.452,2.754)--cycle;
\draw(-4.44,2.744)--(-4.452,2.754);
\filldraw[fill opacity=0.8,fill=gray!20,draw=none](-4.457,2.724)--(-4.474,2.713)--(-4.471,2.717)--cycle;
\draw(-4.457,2.724)--(-4.474,2.713);
\filldraw[fill opacity=0.8,fill=gray!20,draw=none](-4.441,2.728)--(-4.429,2.724)--(-4.433,2.717)--(-4.463,2.717)--cycle;
\draw(-4.429,2.724)--(-4.433,2.717);
\filldraw[fill opacity=0.8,fill=gray!20,draw=none](-4.463,2.717)--(-4.433,2.717)--(-4.449,2.693)--(-4.46,2.691)--(-4.488,2.704)--cycle;
\draw(-4.433,2.717)--(-4.449,2.693);
\filldraw[fill opacity=0.8,fill=gray!20,draw=none](-4.38,2.683)--(-4.397,2.686)--(-4.391,2.681)--cycle;
\draw(-4.397,2.686)--(-4.391,2.681);
\filldraw[fill opacity=0.8,fill=gray!20,draw=none](-4.397,2.686)--(-4.382,2.684)--(-4.381,2.682)--(-4.393,2.681)--cycle;
\draw(-4.382,2.684)--(-4.381,2.682)--(-4.393,2.681);
\filldraw[fill opacity=0.8,fill=gray!20,draw=none](-4.382,2.684)--(-4.379,2.683)--(-4.381,2.682)--cycle;
\draw(-4.379,2.683)--(-4.381,2.682)--(-4.382,2.684);
\filldraw[fill opacity=0.8,fill=gray!20,draw=none](-4.384,2.681)--(-4.38,2.682)--(-4.386,2.678)--(-4.39,2.676)--cycle;
\draw(-4.386,2.678)--(-4.39,2.676)--(-4.384,2.681);
\filldraw[fill opacity=0.8,fill=gray!20,draw=none](-4.39,2.676)--(-4.386,2.678)--(-4.398,2.679)--(-4.412,2.672)--cycle;
\draw(-4.398,2.679)--(-4.412,2.672)--(-4.39,2.676)--(-4.386,2.678);
\filldraw[fill opacity=0.8,fill=gray!20,draw=none](-4.413,2.753)--(-4.392,2.768)--(-4.39,2.768)--(-4.391,2.762)--cycle;
\filldraw[fill opacity=0.8,fill=gray!20,draw=none](-4.414,2.752)--(-4.392,2.768)--(-4.391,2.768)--(-4.393,2.764)--cycle;
\filldraw[fill opacity=0.8,fill=gray!20,draw=none](-4.471,2.717)--(-4.474,2.713)--(-4.488,2.705)--(-4.488,2.704)--cycle;
\draw(-4.474,2.713)--(-4.488,2.705);
\filldraw[fill opacity=0.8,fill=gray!20,draw=none](-4.432,2.739)--(-4.413,2.753)--(-4.391,2.762)--(-4.391,2.761)--cycle;
\filldraw[fill opacity=0.8,fill=gray!20,draw=none](-4.414,2.752)--(-4.393,2.764)--(-4.394,2.763)--(-4.437,2.736)--cycle;
\draw(-4.394,2.763)--(-4.437,2.736);
\filldraw[fill opacity=0.8,fill=gray!20,draw=none](-4.456,2.74)--(-4.454,2.739)--(-4.48,2.732)--(-4.487,2.733)--cycle;
\filldraw[fill opacity=0.8,fill=gray!20,draw=none](-4.479,2.719)--(-4.488,2.705)--(-4.437,2.736)--cycle;
\draw(-4.488,2.705)--(-4.437,2.736);
\filldraw[fill opacity=0.8,fill=gray!20,draw=none](-4.415,2.723)--(-4.433,2.717)--(-4.429,2.724)--cycle;
\draw(-4.433,2.717)--(-4.429,2.724);
\filldraw[fill opacity=0.8,fill=gray!20,draw=none](-4.454,2.739)--(-4.439,2.734)--(-4.445,2.73)--(-4.48,2.732)--cycle;
\filldraw[fill opacity=0.8,fill=gray!20,draw=none](-4.515,2.784)--(-4.506,2.777)--(-4.478,2.776)--(-4.487,2.783)--cycle;
\draw(-4.478,2.776)--(-4.487,2.783)--(-4.515,2.784)--(-4.506,2.777);
\filldraw[fill opacity=0.8,fill=gray!20,draw=none](-4.468,2.771)--(-4.478,2.776)--(-4.505,2.786)--(-4.528,2.788)--(-4.495,2.772)--cycle;
\draw(-4.528,2.788)--(-4.495,2.772)--(-4.468,2.771)--(-4.478,2.776);
\filldraw[fill opacity=0.8,fill=gray!20,draw=none](-4.506,2.777)--(-4.477,2.753)--(-4.456,2.754)--(-4.455,2.756)--(-4.478,2.776)--cycle;
\draw(-4.506,2.777)--(-4.477,2.753);
\draw(-4.455,2.756)--(-4.478,2.776);
\filldraw[fill opacity=0.8,fill=gray!20,draw=none](-4.489,2.733)--(-4.49,2.733)--(-4.494,2.731)--cycle;
\filldraw[fill opacity=0.8,fill=gray!20,draw=none](-4.487,2.733)--(-4.494,2.731)--(-4.493,2.733)--cycle;
\draw(-4.494,2.731)--(-4.493,2.733);
\filldraw[fill opacity=0.8,fill=gray!20,draw=none](-4.385,2.841)--(-4.387,2.838)--(-4.395,2.843)--(-4.396,2.845)--(-4.389,2.85)--cycle;
\draw(-4.396,2.845)--(-4.389,2.85)--(-4.385,2.841);
\filldraw[fill opacity=0.8,fill=gray!20,draw=none](-4.385,2.841)--(-4.387,2.838)--(-4.396,2.844)--(-4.403,2.874)--(-4.402,2.876)--cycle;
\draw(-4.387,2.838)--(-4.396,2.844);
\draw(-4.403,2.874)--(-4.402,2.876);
\filldraw[fill opacity=0.8,fill=gray!20,draw=none](-4.367,2.772)--(-4.377,2.771)--(-4.379,2.773)--(-4.37,2.775)--cycle;
\draw(-4.379,2.773)--(-4.37,2.775);
\filldraw[fill opacity=0.8,fill=gray!20,draw=none](-4.432,2.739)--(-4.391,2.761)--(-4.393,2.749)--(-4.402,2.739)--(-4.427,2.733)--(-4.439,2.734)--cycle;
\draw(-4.393,2.749)--(-4.402,2.739);
\filldraw[fill opacity=0.8,fill=gray!20,draw=none](-4.432,2.739)--(-4.439,2.734)--(-4.437,2.736)--cycle;
\draw(-4.439,2.734)--(-4.437,2.736);
\filldraw[fill opacity=0.8,fill=gray!20,draw=none](-4.44,2.786)--(-4.446,2.791)--(-4.478,2.776)--(-4.468,2.771)--cycle;
\draw(-4.478,2.776)--(-4.468,2.771)--(-4.44,2.786);
\filldraw[fill opacity=0.8,fill=gray!20,draw=none](-4.478,2.776)--(-4.44,2.744)--(-4.437,2.747)--(-4.446,2.791)--cycle;
\draw(-4.478,2.776)--(-4.44,2.744);
\filldraw[fill opacity=0.8,fill=gray!20,draw=none](-4.441,2.767)--(-4.484,2.752)--(-4.501,2.73)--(-4.501,2.729)--(-4.485,2.734)--cycle;
\draw(-4.441,2.767)--(-4.484,2.752);
\draw(-4.501,2.729)--(-4.485,2.734);
\filldraw[fill opacity=0.8,fill=gray!20,draw=none](-4.487,2.733)--(-4.452,2.73)--(-4.479,2.719)--(-4.497,2.728)--(-4.494,2.731)--cycle;
\draw(-4.497,2.728)--(-4.494,2.731);
\filldraw[fill opacity=0.8,fill=gray!20,draw=none](-4.407,2.716)--(-4.379,2.692)--(-4.376,2.711)--(-4.39,2.734)--cycle;
\draw(-4.407,2.716)--(-4.379,2.692);
\filldraw[fill opacity=0.8,fill=gray!20,draw=none](-4.39,2.734)--(-4.395,2.73)--(-4.388,2.739)--cycle;
\draw(-4.395,2.73)--(-4.388,2.739);
\filldraw[fill opacity=0.8,fill=gray!20,draw=none](-4.418,2.725)--(-4.407,2.716)--(-4.39,2.734)--(-4.396,2.744)--cycle;
\draw(-4.418,2.725)--(-4.407,2.716);
\filldraw[fill opacity=0.8,fill=gray!20,draw=none](-4.395,2.73)--(-4.399,2.722)--(-4.449,2.693)--(-4.433,2.717)--cycle;
\draw(-4.395,2.73)--(-4.399,2.722);
\draw(-4.449,2.693)--(-4.433,2.717);
\filldraw[fill opacity=0.8,fill=gray!20,draw=none](-4.455,2.756)--(-4.485,2.734)--(-4.456,2.745)--cycle;
\draw(-4.485,2.734)--(-4.456,2.745);
\filldraw[fill opacity=0.8,fill=gray!20,draw=none](-4.441,2.734)--(-4.441,2.734)--(-4.44,2.734)--cycle;
\filldraw[fill opacity=0.8,fill=gray!20,draw=none](-4.401,2.769)--(-4.406,2.763)--(-4.395,2.765)--cycle;
\draw(-4.401,2.769)--(-4.406,2.763)--(-4.395,2.765);
\filldraw[fill opacity=0.8,fill=gray!20,draw=none](-4.403,2.77)--(-4.413,2.76)--(-4.397,2.766)--cycle;
\draw(-4.413,2.76)--(-4.397,2.766)--(-4.403,2.77);
\filldraw[fill opacity=0.8,fill=gray!20,draw=none](-4.397,2.766)--(-4.397,2.766)--(-4.404,2.763)--cycle;
\draw(-4.397,2.766)--(-4.397,2.766)--(-4.404,2.763);
\filldraw[fill opacity=0.8,fill=gray!20,draw=none](-4.408,2.771)--(-4.467,2.724)--(-4.437,2.736)--(-4.394,2.763)--cycle;
\draw(-4.437,2.736)--(-4.394,2.763);
\filldraw[fill opacity=0.8,fill=gray!20,draw=none](-4.437,2.736)--(-4.439,2.734)--(-4.441,2.735)--cycle;
\draw(-4.437,2.736)--(-4.439,2.734);
\filldraw[fill opacity=0.8,fill=gray!20,draw=none](-4.441,2.734)--(-4.44,2.734)--(-4.431,2.737)--(-4.44,2.744)--cycle;
\draw(-4.431,2.737)--(-4.44,2.744);
\filldraw[fill opacity=0.8,fill=gray!20,draw=none](-4.44,2.734)--(-4.427,2.733)--(-4.431,2.737)--cycle;
\draw(-4.427,2.733)--(-4.431,2.737);
\filldraw[fill opacity=0.8,fill=gray!20,draw=none](-4.439,2.734)--(-4.443,2.729)--(-4.445,2.73)--cycle;
\draw(-4.439,2.734)--(-4.443,2.729);
\filldraw[fill opacity=0.8,fill=gray!20,draw=none](-4.427,2.733)--(-4.443,2.729)--(-4.439,2.734)--cycle;
\draw(-4.443,2.729)--(-4.439,2.734);
\filldraw[fill opacity=0.8,fill=gray!20,draw=none](-4.452,2.73)--(-4.443,2.729)--(-4.457,2.713)--(-4.466,2.712)--(-4.479,2.719)--cycle;
\draw(-4.443,2.729)--(-4.457,2.713);
\filldraw[fill opacity=0.8,fill=gray!20,draw=none](-4.412,2.672)--(-4.398,2.679)--(-4.399,2.68)--(-4.403,2.68)--(-4.442,2.672)--(-4.441,2.671)--cycle;
\draw(-4.442,2.672)--(-4.441,2.671)--(-4.412,2.672)--(-4.398,2.679);
\filldraw[fill opacity=0.8,fill=gray!20,draw=none](-4.396,2.744)--(-4.402,2.739)--(-4.393,2.749)--cycle;
\draw(-4.402,2.739)--(-4.393,2.749);
\filldraw[fill opacity=0.8,fill=gray!20,draw=none](-4.437,2.748)--(-4.408,2.771)--(-4.428,2.758)--cycle;
\draw(-4.408,2.771)--(-4.428,2.758);
\filldraw[fill opacity=0.8,fill=gray!20,draw=none](-4.428,2.758)--(-4.44,2.744)--(-4.424,2.73)--cycle;
\draw(-4.44,2.744)--(-4.424,2.73);
\filldraw[fill opacity=0.8,fill=gray!20,draw=none](-4.424,2.734)--(-4.45,2.721)--(-4.443,2.729)--cycle;
\draw(-4.45,2.721)--(-4.443,2.729);
\filldraw[fill opacity=0.8,fill=gray!20,draw=none](-4.445,2.672)--(-4.448,2.671)--(-4.441,2.671)--(-4.442,2.672)--cycle;
\draw(-4.448,2.671)--(-4.441,2.671)--(-4.442,2.672);
\filldraw[fill opacity=0.8,fill=gray!20,draw=none](-4.487,2.783)--(-4.478,2.776)--(-4.446,2.791)--(-4.447,2.794)--(-4.451,2.797)--cycle;
\draw(-4.447,2.794)--(-4.451,2.797)--(-4.487,2.783)--(-4.478,2.776);
\filldraw[fill opacity=0.8,fill=gray!20,draw=none](-4.446,2.791)--(-4.449,2.793)--(-4.464,2.798)--(-4.498,2.786)--(-4.478,2.776)--cycle;
\draw(-4.498,2.786)--(-4.478,2.776);
\filldraw[fill opacity=0.8,fill=gray!20,draw=none](-4.449,2.793)--(-4.465,2.788)--(-4.484,2.755)--(-4.484,2.752)--(-4.425,2.773)--cycle;
\draw(-4.484,2.752)--(-4.425,2.773);
\filldraw[fill opacity=0.8,fill=gray!20,draw=none](-4.425,2.797)--(-4.449,2.793)--(-4.425,2.773)--(-4.412,2.777)--cycle;
\draw(-4.425,2.773)--(-4.412,2.777)--(-4.425,2.797);
\filldraw[fill opacity=0.8,fill=gray!20,draw=none](-4.44,2.786)--(-4.437,2.788)--(-4.444,2.792)--(-4.446,2.791)--cycle;
\draw(-4.44,2.786)--(-4.437,2.788)--(-4.444,2.792);
\filldraw[fill opacity=0.8,fill=gray!20,draw=none](-4.446,2.791)--(-4.437,2.747)--(-4.428,2.758)--(-4.43,2.773)--(-4.444,2.792)--cycle;
\filldraw[fill opacity=0.8,fill=gray!20,draw=none](-4.437,2.748)--(-4.428,2.758)--(-4.456,2.741)--(-4.473,2.729)--(-4.479,2.719)--(-4.467,2.724)--cycle;
\draw(-4.428,2.758)--(-4.456,2.741);
\filldraw[fill opacity=0.8,fill=gray!20,draw=none](-4.46,2.691)--(-4.449,2.693)--(-4.453,2.687)--cycle;
\draw(-4.449,2.693)--(-4.453,2.687);
\filldraw[fill opacity=0.8,fill=gray!20,draw=none](-4.399,2.722)--(-4.414,2.7)--(-4.452,2.687)--(-4.453,2.687)--(-4.449,2.693)--cycle;
\draw(-4.399,2.722)--(-4.414,2.7);
\draw(-4.453,2.687)--(-4.449,2.693);
\filldraw[fill opacity=0.8,fill=gray!20,draw=none](-4.456,2.754)--(-4.452,2.754)--(-4.455,2.756)--cycle;
\draw(-4.452,2.754)--(-4.455,2.756);
\filldraw[fill opacity=0.8,fill=gray!20,draw=none](-4.419,2.787)--(-4.456,2.741)--(-4.408,2.771)--cycle;
\draw(-4.456,2.741)--(-4.408,2.771);
\filldraw[fill opacity=0.8,fill=gray!20,draw=none](-4.424,2.764)--(-4.428,2.758)--(-4.424,2.73)--(-4.418,2.725)--(-4.404,2.737)--cycle;
\draw(-4.424,2.73)--(-4.418,2.725);
\filldraw[fill opacity=0.8,fill=gray!20,draw=none](-4.424,2.734)--(-4.405,2.738)--(-4.453,2.717)--(-4.45,2.721)--cycle;
\draw(-4.453,2.717)--(-4.45,2.721);
\filldraw[fill opacity=0.8,fill=gray!20,draw=none](-4.424,2.764)--(-4.404,2.737)--(-4.396,2.744)--(-4.405,2.759)--(-4.418,2.77)--cycle;
\draw(-4.405,2.759)--(-4.418,2.77);
\filldraw[fill opacity=0.8,fill=gray!20,draw=none](-4.405,2.738)--(-4.402,2.739)--(-4.405,2.734)--(-4.457,2.713)--(-4.453,2.717)--cycle;
\draw(-4.402,2.739)--(-4.405,2.734);
\draw(-4.457,2.713)--(-4.453,2.717);
\filldraw[fill opacity=0.8,fill=gray!20,draw=none](-4.466,2.712)--(-4.457,2.713)--(-4.459,2.71)--(-4.463,2.711)--cycle;
\draw(-4.457,2.713)--(-4.459,2.71);
\filldraw[fill opacity=0.8,fill=gray!20,draw=none](-4.422,2.723)--(-4.415,2.723)--(-4.418,2.725)--cycle;
\draw(-4.415,2.723)--(-4.418,2.725);
\filldraw[fill opacity=0.8,fill=gray!20,draw=none](-4.405,2.734)--(-4.418,2.72)--(-4.421,2.718)--(-4.459,2.71)--(-4.457,2.713)--cycle;
\draw(-4.405,2.734)--(-4.418,2.72);
\draw(-4.459,2.71)--(-4.457,2.713);
\filldraw[fill opacity=0.8,fill=gray!20,draw=none](-4.391,2.732)--(-4.394,2.724)--(-4.413,2.701)--(-4.414,2.7)--(-4.399,2.722)--cycle;
\draw(-4.414,2.7)--(-4.399,2.722);
\filldraw[fill opacity=0.8,fill=gray!20,draw=none](-4.456,2.741)--(-4.471,2.732)--(-4.472,2.73)--(-4.473,2.729)--cycle;
\draw(-4.456,2.741)--(-4.471,2.732);
\filldraw[fill opacity=0.8,fill=gray!20,draw=none](-4.44,2.785)--(-4.43,2.773)--(-4.432,2.781)--(-4.438,2.786)--cycle;
\draw(-4.432,2.781)--(-4.438,2.786);
\filldraw[fill opacity=0.8,fill=gray!20,draw=none](-4.43,2.773)--(-4.432,2.781)--(-4.467,2.741)--(-4.471,2.732)--(-4.456,2.741)--cycle;
\draw(-4.471,2.732)--(-4.456,2.741);
\filldraw[fill opacity=0.8,fill=gray!20,draw=none](-4.43,2.773)--(-4.419,2.787)--(-4.422,2.791)--(-4.432,2.781)--cycle;
\filldraw[fill opacity=0.8,fill=gray!20,draw=none](-4.426,2.799)--(-4.43,2.805)--(-4.44,2.796)--cycle;
\filldraw[fill opacity=0.8,fill=gray!20,draw=none](-4.425,2.797)--(-4.427,2.8)--(-4.449,2.793)--cycle;
\draw(-4.425,2.797)--(-4.427,2.8);
\filldraw[fill opacity=0.8,fill=gray!20,draw=none](-4.426,2.799)--(-4.437,2.788)--(-4.44,2.786)--cycle;
\draw(-4.426,2.799)--(-4.437,2.788)--(-4.44,2.786);
\filldraw[fill opacity=0.8,fill=gray!20,draw=none](-4.426,2.799)--(-4.44,2.796)--(-4.444,2.792)--(-4.437,2.788)--cycle;
\draw(-4.444,2.792)--(-4.437,2.788)--(-4.426,2.799);
\filldraw[fill opacity=0.8,fill=gray!20,draw=none](-4.427,2.8)--(-4.44,2.796)--(-4.444,2.792)--(-4.405,2.759)--cycle;
\draw(-4.444,2.792)--(-4.405,2.759);
\filldraw[fill opacity=0.8,fill=gray!20,draw=none](-4.44,2.785)--(-4.438,2.786)--(-4.444,2.792)--cycle;
\draw(-4.438,2.786)--(-4.444,2.792);
\filldraw[fill opacity=0.8,fill=gray!20,draw=none](-4.422,2.791)--(-4.426,2.797)--(-4.433,2.793)--(-4.453,2.775)--(-4.467,2.741)--cycle;
\draw(-4.426,2.797)--(-4.433,2.793);
\filldraw[fill opacity=0.8,fill=gray!20,draw=none](-4.478,2.776)--(-4.498,2.786)--(-4.505,2.786)--cycle;
\draw(-4.478,2.776)--(-4.498,2.786);
\filldraw[fill opacity=0.8,fill=gray!20,draw=none](-4.433,2.793)--(-4.45,2.782)--(-4.453,2.775)--cycle;
\draw(-4.433,2.793)--(-4.45,2.782);
\filldraw[fill opacity=0.8,fill=gray!20,draw=none](-4.446,2.791)--(-4.444,2.792)--(-4.447,2.794)--cycle;
\draw(-4.444,2.792)--(-4.447,2.794);
\filldraw[fill opacity=0.8,fill=gray!20,draw=none](-4.444,2.792)--(-4.457,2.798)--(-4.461,2.799)--(-4.464,2.798)--cycle;
\draw(-4.444,2.792)--(-4.457,2.798);
\filldraw[fill opacity=0.8,fill=gray!20,draw=none](-4.449,2.793)--(-4.454,2.798)--(-4.46,2.796)--(-4.465,2.788)--cycle;
\draw(-4.454,2.798)--(-4.46,2.796);
\filldraw[fill opacity=0.8,fill=gray!20,draw=none](-4.461,2.799)--(-4.46,2.796)--(-4.454,2.798)--cycle;
\draw(-4.46,2.796)--(-4.454,2.798);
\filldraw[fill opacity=0.8,fill=gray!20,draw=none](-4.43,2.805)--(-4.433,2.811)--(-4.451,2.797)--(-4.444,2.792)--cycle;
\draw(-4.433,2.811)--(-4.451,2.797)--(-4.444,2.792);
\filldraw[fill opacity=0.8,fill=gray!20,draw=none](-4.44,2.796)--(-4.43,2.805)--(-4.434,2.813)--(-4.457,2.798)--(-4.451,2.795)--cycle;
\draw(-4.457,2.798)--(-4.451,2.795);
\filldraw[fill opacity=0.8,fill=gray!20,draw=none](-4.443,2.831)--(-4.464,2.809)--(-4.461,2.799)--(-4.454,2.798)--(-4.431,2.806)--cycle;
\draw(-4.454,2.798)--(-4.431,2.806)--(-4.443,2.831);
\filldraw[fill opacity=0.8,fill=gray!20,draw=none](-4.457,2.798)--(-4.46,2.799)--(-4.461,2.799)--cycle;
\draw(-4.457,2.798)--(-4.46,2.799);
\filldraw[fill opacity=0.8,fill=gray!20,draw=none](-4.513,2.817)--(-3.654,.991)--(-3.644,.993)--(-3.644,1.065)--(-4.471,2.824)--cycle;
\draw(-3.644,1.065)--(-4.471,2.824)--(-4.513,2.817)--(-3.654,.991);
\filldraw[fill opacity=0.8,fill=gray!20,draw=none](-3.478,1.077)--(-3.489,1.077)--(-3.487,1.071)--cycle;
\draw(-3.489,1.077)--(-3.487,1.071);
\filldraw[fill opacity=0.8,fill=gray!20,draw=none](-7.485,1.139)--(-7.557,1.094)--(-7.571,1.102)--(-7.513,1.138)--cycle;
\draw(-7.485,1.139)--(-7.557,1.094)--(-7.571,1.102)--(-7.513,1.138);
\filldraw[fill opacity=0.8,fill=gray!20,draw=none](-7.572,1.1)--(-7.571,1.102)--(-7.567,1.1)--cycle;
\draw(-7.572,1.1)--(-7.571,1.102)--(-7.567,1.1);
\filldraw[fill opacity=0.8,fill=gray!20,draw=none](-7.552,1.112)--(-7.546,1.114)--(-7.573,1.097)--(-7.587,1.09)--(-7.594,1.087)--cycle;
\draw(-7.552,1.112)--(-7.546,1.114)--(-7.573,1.097);
\draw(-7.587,1.09)--(-7.594,1.087);
\filldraw[fill opacity=0.8,fill=gray!20,draw=none](-7.529,.986)--(-7.553,1.03)--(-7.569,1.067)--(-7.576,1.092)--(-7.572,1.1)--(-7.567,1.1)--(-7.557,1.094)--(-7.534,1.071)--(-7.507,1.035)--(-7.479,.992)--(-7.455,.949)--(-7.438,.912)--(-7.432,.886)--(-7.436,.877)--(-7.451,.884)--(-7.474,.908)--(-7.501,.944)--cycle;
\draw(-7.567,1.1)--(-7.557,1.094)--(-7.534,1.071)--(-7.507,1.035)--(-7.479,.992)--(-7.455,.949)--(-7.438,.912)--(-7.432,.886)--(-7.436,.877)--(-7.451,.884)--(-7.474,.908)--(-7.501,.944)--(-7.529,.986)--(-7.553,1.03)--(-7.569,1.067)--(-7.576,1.092)--(-7.572,1.1);
\filldraw[fill opacity=0.8,fill=gray!20,draw=none](-2.884,2.352)--(-2.863,2.189)--(-2.809,2.208)--(-2.832,2.384)--cycle;
\draw(-2.809,2.208)--(-2.832,2.384)--(-2.884,2.352)--(-2.863,2.189);
\filldraw[fill opacity=0.8,fill=gray!20,draw=none](-3.412,2.345)--(-3.4,2.328)--(-3.395,2.405)--(-3.45,2.831)--cycle;
\draw(-3.395,2.405)--(-3.45,2.831);
\filldraw[fill opacity=0.8,fill=gray!20](-3.103,3.02)--(-3.083,3.071)--(-3.018,3.055)--(-3.049,3.007)--cycle;
\filldraw[fill opacity=0.8,fill=gray!20](-3.067,3.184)--(-3.071,3.238)--(-2.998,3.22)--(-2.992,3.165)--cycle;
\filldraw[fill opacity=0.8,fill=gray!20,draw=none](-3.233,1.732)--(-3.22,1.627)--(-3.154,1.624)--cycle;
\draw(-3.233,1.732)--(-3.22,1.627)--(-3.154,1.624);
\filldraw[fill opacity=0.8,fill=gray!20,draw=none](-3.644,.993)--(-3.614,1.001)--(-3.644,1.065)--cycle;
\draw(-3.614,1.001)--(-3.644,1.065);
\filldraw[fill opacity=0.8,fill=gray!20,draw=none](-7.434,1.08)--(-7.436,1.097)--(-7.429,1.091)--cycle;
\filldraw[fill opacity=0.8,fill=gray!20,draw=none](-7.444,1.115)--(-7.426,1.085)--(-7.507,1.035)--(-7.534,1.071)--(-7.454,1.121)--cycle;
\draw(-7.426,1.085)--(-7.507,1.035)--(-7.534,1.071)--(-7.454,1.121);
\filldraw[fill opacity=0.8,fill=gray!20,draw=none](-7.614,.878)--(-7.643,.877)--(-7.643,.898)--cycle;
\draw(-7.643,.877)--(-7.643,.898);
\filldraw[fill opacity=0.8,fill=gray!20,draw=none](-7.431,1.022)--(-7.434,1.08)--(-7.429,1.091)--(-7.408,1.068)--(-7.408,1.005)--cycle;
\draw(-7.408,1.068)--(-7.408,1.005);
\filldraw[fill opacity=0.8,fill=gray!20,draw=none](-3.07,2.304)--(-2.986,1.648)--(-2.884,1.668)--(-2.968,2.324)--cycle;
\draw(-2.884,1.668)--(-2.968,2.324)--(-3.07,2.304)--(-2.986,1.648);
\filldraw[fill opacity=0.8,fill=gray!20](-3.605,2.038)--(-3.768,2.11)--(-3.726,2.083)--(-3.562,2.012)--cycle;
\filldraw[fill opacity=0.8,fill=gray!20,draw=none](-7.599,1.153)--(-7.569,1.161)--(-7.581,1.166)--cycle;
\draw(-7.569,1.161)--(-7.581,1.166);
\filldraw[fill opacity=0.8,fill=gray!20,draw=none](-3.5,1.078)--(-3.521,1.058)--(-3.487,1.071)--(-3.489,1.077)--cycle;
\draw(-3.487,1.071)--(-3.489,1.077);
\filldraw[fill opacity=0.8,fill=gray!20,draw=none](-3.319,2.044)--(-3.346,2.021)--(-3.256,1.912)--(-3.273,2.044)--cycle;
\draw(-3.256,1.912)--(-3.273,2.044);
\filldraw[fill opacity=0.8,fill=gray!20,draw=none](-3.349,2.044)--(-3.346,2.021)--(-3.319,2.044)--cycle;
\draw(-3.349,2.044)--(-3.346,2.021);
\filldraw[fill opacity=0.8,fill=gray!20,draw=none](-3.349,2.329)--(-3.349,2.326)--(-3.332,2.303)--(-3.334,2.315)--cycle;
\draw(-3.332,2.303)--(-3.334,2.315)--(-3.349,2.329);
\filldraw[fill opacity=0.8,fill=gray!20,draw=none](-3.582,1.169)--(-3.54,1.081)--(-3.489,1.077)--(-3.512,1.124)--cycle;
\draw(-3.582,1.169)--(-3.54,1.081);
\draw(-3.489,1.077)--(-3.512,1.124);
\filldraw[fill opacity=0.8,fill=gray!20,draw=none](-2.898,2.779)--(-2.82,2.168)--(-2.755,2.191)--(-2.818,2.683)--cycle;
\draw(-2.898,2.779)--(-2.82,2.168);
\draw(-2.755,2.191)--(-2.818,2.683);
\filldraw[fill opacity=0.8,fill=gray!20,draw=none](-3.61,.981)--(-3.606,.984)--(-3.607,.986)--cycle;
\draw(-3.606,.984)--(-3.607,.986);
\filldraw[fill opacity=0.8,fill=gray!20,draw=none](-7.481,1.138)--(-7.454,1.121)--(-7.534,1.071)--(-7.557,1.094)--(-7.485,1.139)--cycle;
\draw(-7.454,1.121)--(-7.534,1.071)--(-7.557,1.094)--(-7.485,1.139);
\filldraw[fill opacity=0.8,fill=gray!20,draw=none](-7.367,.927)--(-7.432,.886)--(-7.438,.912)--(-7.387,.944)--cycle;
\draw(-7.367,.927)--(-7.432,.886)--(-7.438,.912)--(-7.387,.944);
\filldraw[fill opacity=0.8,fill=gray!20,draw=none](-7.517,1.133)--(-7.513,1.136)--(-7.513,1.138)--(-7.554,1.113)--cycle;
\draw(-7.513,1.138)--(-7.554,1.113);
\filldraw[fill opacity=0.8,fill=gray!20,draw=none](-7.358,1.046)--(-7.358,.937)--(-7.373,1.004)--cycle;
\draw(-7.358,1.046)--(-7.358,.937);
\filldraw[fill opacity=0.8,fill=gray!20,draw=none](-7.359,.931)--(-7.364,.921)--(-7.436,.877)--(-7.432,.886)--(-7.367,.927)--cycle;
\draw(-7.364,.921)--(-7.436,.877)--(-7.432,.886)--(-7.367,.927);
\filldraw[fill opacity=0.8,fill=gray!20,draw=none](-7.393,1.1)--(-7.393,.932)--(-7.387,.924)--(-7.365,.923)--(-7.365,1.08)--cycle;
\draw(-7.393,1.1)--(-7.393,.932);
\draw(-7.365,.923)--(-7.365,1.08);
\filldraw[fill opacity=0.8,fill=gray!20,draw=none](-7.403,.925)--(-7.425,.89)--(-7.435,.882)--(-7.44,.879)--(-7.568,1.1)--(-7.546,1.114)--(-7.504,1.118)--(-7.462,1.102)--(-7.427,1.069)--(-7.403,1.024)--(-7.395,.973)--cycle;
\draw(-7.435,.882)--(-7.44,.879);
\draw(-7.568,1.1)--(-7.546,1.114)--(-7.504,1.118)--(-7.462,1.102)--(-7.427,1.069)--(-7.403,1.024)--(-7.395,.973)--(-7.403,.925)--(-7.425,.89);
\filldraw[fill opacity=0.8,fill=gray!20,draw=none](-7.519,1.098)--(-7.501,1.089)--(-7.462,1.102)--(-7.504,1.118)--(-7.53,1.11)--cycle;
\draw(-7.501,1.089)--(-7.462,1.102)--(-7.504,1.118)--(-7.53,1.11);
\filldraw[fill opacity=0.8,fill=gray!20,draw=none](-7.645,.995)--(-7.613,1.005)--(-7.605,.955)--(-7.648,.941)--cycle;
\draw(-7.645,.995)--(-7.613,1.005)--(-7.605,.955)--(-7.648,.941);
\filldraw[fill opacity=0.8,fill=gray!20,draw=none](-7.501,1.089)--(-7.519,1.098)--(-7.569,1.067)--(-7.553,1.03)--(-7.489,1.07)--cycle;
\draw(-7.519,1.098)--(-7.569,1.067)--(-7.553,1.03)--(-7.489,1.07);
\filldraw[fill opacity=0.8,fill=gray!20,draw=none](-7.472,1.054)--(-7.427,1.069)--(-7.462,1.102)--(-7.508,1.087)--cycle;
\draw(-7.472,1.054)--(-7.427,1.069)--(-7.462,1.102)--(-7.508,1.087);
\filldraw[fill opacity=0.8,fill=gray!20,draw=none](-7.458,1.112)--(-7.496,1.112)--(-7.51,1.104)--(-7.501,1.089)--(-7.476,1.078)--(-7.432,1.105)--cycle;
\draw(-7.496,1.112)--(-7.51,1.104);
\draw(-7.476,1.078)--(-7.432,1.105);
\filldraw[fill opacity=0.8,fill=gray!20,draw=none](-7.678,.994)--(-7.717,1.011)--(-7.707,.957)--(-7.651,.933)--cycle;
\draw(-7.678,.994)--(-7.717,1.011)--(-7.707,.957)--(-7.651,.933);
\filldraw[fill opacity=0.8,fill=gray!20,draw=none](-7.648,1.039)--(-7.644,1.024)--(-7.645,.995)--cycle;
\filldraw[fill opacity=0.8,fill=gray!20,draw=none](-7.648,1.039)--(-7.605,1.053)--(-7.613,1.005)--(-7.645,.995)--cycle;
\draw(-7.648,1.039)--(-7.605,1.053)--(-7.613,1.005)--(-7.645,.995);
\filldraw[fill opacity=0.8,fill=gray!20,draw=none](-7.65,1.045)--(-7.623,1.078)--(-7.581,1.092)--(-7.605,1.053)--(-7.648,1.039)--cycle;
\draw(-7.623,1.078)--(-7.581,1.092)--(-7.605,1.053)--(-7.648,1.039);
\filldraw[fill opacity=0.8,fill=gray!20,draw=none](-7.654,1.037)--(-7.652,1.038)--(-7.656,.992)--(-7.666,.988)--cycle;
\draw(-7.654,1.037)--(-7.652,1.038);
\draw(-7.656,.992)--(-7.666,.988);
\filldraw[fill opacity=0.8,fill=gray!20,draw=none](-7.461,1.028)--(-7.447,1.009)--(-7.403,1.024)--(-7.427,1.069)--(-7.472,1.054)--cycle;
\draw(-7.447,1.009)--(-7.403,1.024)--(-7.427,1.069)--(-7.472,1.054);
\filldraw[fill opacity=0.8,fill=gray!20,draw=none](-7.42,1.102)--(-7.438,1.048)--(-7.438,1.029)--(-7.393,1.025)--(-7.393,1.1)--cycle;
\draw(-7.438,1.048)--(-7.438,1.029);
\draw(-7.393,1.025)--(-7.393,1.1);
\filldraw[fill opacity=0.8,fill=gray!20,draw=none](-7.438,1.048)--(-7.42,1.102)--(-7.432,1.105)--(-7.48,1.075)--(-7.461,1.028)--(-7.444,1.039)--cycle;
\draw(-7.432,1.105)--(-7.48,1.075);
\draw(-7.461,1.028)--(-7.444,1.039);
\filldraw[fill opacity=0.8,fill=gray!20,draw=none](-7.438,1.048)--(-7.405,1.099)--(-7.42,1.102)--cycle;
\filldraw[fill opacity=0.8,fill=gray!20,draw=none](-7.42,1.102)--(-7.438,1.104)--(-7.438,1.048)--cycle;
\draw(-7.438,1.104)--(-7.438,1.048);
\filldraw[fill opacity=0.8,fill=gray!20,draw=none](-7.643,.898)--(-7.643,.821)--(-7.65,.808)--(-7.65,.907)--cycle;
\draw(-7.643,.898)--(-7.643,.821)--(-7.65,.808)--(-7.65,.907);
\filldraw[fill opacity=0.8,fill=gray!20,draw=none](-7.65,.931)--(-7.643,.898)--(-7.65,.907)--(-7.65,.928)--cycle;
\draw(-7.65,.907)--(-7.65,.928);
\filldraw[fill opacity=0.8,fill=gray!20,draw=none](-7.438,.981)--(-7.428,.962)--(-7.395,.973)--(-7.403,1.024)--(-7.447,1.009)--cycle;
\draw(-7.428,.962)--(-7.395,.973)--(-7.403,1.024)--(-7.447,1.009);
\filldraw[fill opacity=0.8,fill=gray!20,draw=none](-7.447,1.008)--(-7.438,.981)--(-7.438,1.104)--cycle;
\draw(-7.438,.981)--(-7.438,1.104);
\filldraw[fill opacity=0.8,fill=gray!20,draw=none](-7.396,1.097)--(-7.405,1.099)--(-7.438,1.048)--(-7.441,1.041)--(-7.412,1.059)--cycle;
\draw(-7.441,1.041)--(-7.412,1.059);
\filldraw[fill opacity=0.8,fill=gray!20,draw=none](-7.376,1.092)--(-7.396,1.097)--(-7.412,1.059)--(-7.376,1.081)--cycle;
\draw(-7.412,1.059)--(-7.376,1.081);
\filldraw[fill opacity=0.8,fill=gray!20,draw=none](-7.372,1.091)--(-7.514,1.153)--(-7.569,1.161)--(-7.422,1.096)--cycle;
\draw(-7.569,1.161)--(-7.422,1.096)--(-7.372,1.091)--(-7.514,1.153);
\filldraw[fill opacity=0.8,fill=gray!20,draw=none](-7.514,1.2)--(-7.514,1.153)--(-7.569,1.186)--(-7.569,1.196)--cycle;
\draw(-7.569,1.186)--(-7.569,1.196)--(-7.514,1.2)--(-7.514,1.153);
\filldraw[fill opacity=0.8,fill=gray!20,draw=none](-7.478,1.146)--(-7.514,1.153)--(-7.514,1.191)--cycle;
\draw(-7.514,1.153)--(-7.514,1.191);
\filldraw[fill opacity=0.8,fill=gray!20,draw=none](-3.119,1.849)--(-3.092,1.639)--(-2.986,1.648)--(-3.005,1.795)--cycle;
\draw(-3.119,1.849)--(-3.092,1.639);
\draw(-2.986,1.648)--(-3.005,1.795);
\filldraw[fill opacity=0.8,fill=gray!20,draw=none](-3.598,1.008)--(-3.607,.986)--(-3.606,.984)--(-3.568,1.021)--(-3.569,1.022)--cycle;
\draw(-3.607,.986)--(-3.606,.984);
\draw(-3.568,1.021)--(-3.569,1.022);
\filldraw[fill opacity=0.8,fill=gray!20,draw=none](-7.458,1.198)--(-7.458,1.142)--(-7.478,1.146)--(-7.514,1.191)--(-7.514,1.2)--cycle;
\draw(-7.514,1.191)--(-7.514,1.2)--(-7.458,1.198)--(-7.458,1.142);
\filldraw[fill opacity=0.8,fill=gray!20,draw=none](-7.771,1.705)--(-7.773,1.7)--(-7.77,1.699)--cycle;
\draw(-7.773,1.7)--(-7.77,1.699);
\filldraw[fill opacity=0.8,fill=gray!20,draw=none](-7.764,1.672)--(-7.77,1.701)--(-7.741,1.662)--(-7.745,1.66)--cycle;
\draw(-7.741,1.662)--(-7.745,1.66);
\filldraw[fill opacity=0.8,fill=gray!20,draw=none](-7.771,1.759)--(-7.773,1.726)--(-7.773,1.783)--cycle;
\draw(-7.773,1.726)--(-7.773,1.783);
\filldraw[fill opacity=0.8,fill=gray!20,draw=none](-7.758,1.743)--(-7.771,1.705)--(-7.77,1.699)--(-7.723,1.678)--cycle;
\draw(-7.77,1.699)--(-7.723,1.678);
\filldraw[fill opacity=0.8,fill=gray!20,draw=none](-7.768,1.698)--(-7.77,1.701)--(-7.771,1.704)--cycle;
\filldraw[fill opacity=0.8,fill=gray!20,draw=none](-7.735,1.621)--(-7.758,1.648)--(-7.76,1.654)--(-7.745,1.66)--(-7.728,1.65)--(-7.706,1.628)--(-7.724,1.62)--cycle;
\draw(-7.76,1.654)--(-7.745,1.66);
\draw(-7.706,1.628)--(-7.724,1.62);
\filldraw[fill opacity=0.8,fill=gray!20,draw=none](-7.764,1.672)--(-7.745,1.66)--(-7.76,1.654)--cycle;
\draw(-7.745,1.66)--(-7.76,1.654);
\filldraw[fill opacity=0.8,fill=gray!20,draw=none](-7.773,1.7)--(-7.773,1.576)--(-7.758,1.564)--(-7.758,1.704)--cycle;
\draw(-7.773,1.7)--(-7.773,1.576)--(-7.758,1.564)--(-7.758,1.704);
\filldraw[fill opacity=0.8,fill=gray!20](-3.177,3.292)--(-3.179,3.329)--(-3.103,3.323)--(-3.083,3.285)--cycle;
\filldraw[fill opacity=0.8,fill=gray!20](-3.273,3.288)--(-3.258,3.325)--(-3.179,3.329)--(-3.177,3.292)--cycle;
\filldraw[fill opacity=0.8,fill=gray!20,draw=none](-3.598,1.008)--(-3.614,1.001)--(-3.607,.986)--cycle;
\draw(-3.614,1.001)--(-3.607,.986);
\filldraw[fill opacity=0.8,fill=gray!20,draw=none](-7.447,1.197)--(-7.429,1.135)--(-7.458,1.142)--(-7.458,1.198)--cycle;
\draw(-7.458,1.142)--(-7.458,1.198)--(-7.447,1.197);
\filldraw[fill opacity=0.8,fill=gray!20](-3.128,2.978)--(-3.103,3.02)--(-3.049,3.007)--(-3.09,2.969)--cycle;
\filldraw[fill opacity=0.8,fill=gray!20](-3.071,3.238)--(-3.083,3.285)--(-3.018,3.269)--(-2.998,3.22)--cycle;
\filldraw[fill opacity=0.8,fill=gray!20,draw=none](-3.5,1.078)--(-3.54,1.081)--(-3.528,1.055)--(-3.521,1.058)--cycle;
\draw(-3.54,1.081)--(-3.528,1.055);
\filldraw[fill opacity=0.8,fill=gray!20,draw=none](-7.573,1.097)--(-7.581,1.092)--(-7.587,1.09)--cycle;
\draw(-7.573,1.097)--(-7.581,1.092)--(-7.587,1.09);
\filldraw[fill opacity=0.8,fill=gray!20,draw=none](-7.552,1.114)--(-7.554,1.113)--(-7.571,1.102)--(-7.576,1.092)--(-7.55,1.108)--cycle;
\draw(-7.554,1.113)--(-7.571,1.102)--(-7.576,1.092)--(-7.55,1.108);
\filldraw[fill opacity=0.8,fill=gray!20,draw=none](-7.61,1.086)--(-7.579,1.104)--(-7.552,1.112)--(-7.594,1.087)--(-7.623,1.078)--cycle;
\draw(-7.579,1.104)--(-7.552,1.112);
\draw(-7.594,1.087)--(-7.623,1.078);
\filldraw[fill opacity=0.8,fill=gray!20,draw=none](-7.53,1.11)--(-7.504,1.118)--(-7.546,1.114)--(-7.566,1.108)--cycle;
\draw(-7.53,1.11)--(-7.504,1.118)--(-7.546,1.114)--(-7.566,1.108);
\filldraw[fill opacity=0.8,fill=gray!20,draw=none](-7.517,1.133)--(-7.552,1.114)--(-7.55,1.108)--(-7.528,1.122)--cycle;
\draw(-7.55,1.108)--(-7.528,1.122);
\filldraw[fill opacity=0.8,fill=gray!20,draw=none](-7.528,1.122)--(-7.576,1.092)--(-7.569,1.067)--(-7.496,1.112)--cycle;
\draw(-7.528,1.122)--(-7.576,1.092)--(-7.569,1.067)--(-7.496,1.112);
\filldraw[fill opacity=0.8,fill=gray!20,draw=none](-7.543,1.105)--(-7.53,1.11)--(-7.566,1.108)--(-7.579,1.104)--cycle;
\draw(-7.543,1.105)--(-7.53,1.11);
\draw(-7.566,1.108)--(-7.579,1.104);
\filldraw[fill opacity=0.8,fill=gray!20,draw=none](-7.527,1.081)--(-7.509,1.087)--(-7.53,1.11)--(-7.543,1.105)--cycle;
\draw(-7.527,1.081)--(-7.509,1.087);
\draw(-7.53,1.11)--(-7.543,1.105);
\filldraw[fill opacity=0.8,fill=gray!20,draw=none](-7.519,1.098)--(-7.509,1.087)--(-7.501,1.089)--cycle;
\draw(-7.509,1.087)--(-7.501,1.089);
\filldraw[fill opacity=0.8,fill=gray!20,draw=none](-7.51,1.104)--(-7.519,1.098)--(-7.501,1.089)--cycle;
\draw(-7.51,1.104)--(-7.519,1.098);
\filldraw[fill opacity=0.8,fill=gray!20,draw=none](-7.501,1.089)--(-7.489,1.07)--(-7.476,1.078)--cycle;
\draw(-7.489,1.07)--(-7.476,1.078);
\filldraw[fill opacity=0.8,fill=gray!20,draw=none](-7.48,1.075)--(-7.553,1.03)--(-7.529,.986)--(-7.461,1.028)--cycle;
\draw(-7.48,1.075)--(-7.553,1.03)--(-7.529,.986)--(-7.461,1.028);
\filldraw[fill opacity=0.8,fill=gray!20,draw=none](-7.528,1.071)--(-7.494,1.05)--(-7.489,1.049)--(-7.472,1.054)--(-7.508,1.087)--(-7.527,1.081)--cycle;
\draw(-7.489,1.049)--(-7.472,1.054);
\draw(-7.508,1.087)--(-7.527,1.081);
\filldraw[fill opacity=0.8,fill=gray!20,draw=none](-7.365,1.08)--(-7.365,.894)--(-7.364,.886)--(-7.358,.937)--(-7.358,1.046)--cycle;
\draw(-7.365,1.08)--(-7.365,.894);
\draw(-7.358,.937)--(-7.358,1.046);
\filldraw[fill opacity=0.8,fill=gray!20,draw=none](-7.447,1.008)--(-7.461,1.028)--(-7.529,.986)--(-7.501,.944)--(-7.45,.976)--cycle;
\draw(-7.461,1.028)--(-7.529,.986)--(-7.501,.944)--(-7.45,.976);
\filldraw[fill opacity=0.8,fill=gray!20,draw=none](-7.457,1.019)--(-7.472,1.054)--(-7.489,1.049)--cycle;
\draw(-7.472,1.054)--(-7.489,1.049);
\filldraw[fill opacity=0.8,fill=gray!20,draw=none](-7.447,1.008)--(-7.452,1.034)--(-7.461,1.028)--cycle;
\draw(-7.452,1.034)--(-7.461,1.028);
\filldraw[fill opacity=0.8,fill=gray!20,draw=none](-7.461,1.028)--(-7.457,1.019)--(-7.447,1.009)--cycle;
\filldraw[fill opacity=0.8,fill=gray!20,draw=none](-7.447,1.008)--(-7.473,1.095)--(-7.494,1.089)--(-7.494,1.052)--cycle;
\draw(-7.494,1.089)--(-7.494,1.052);
\filldraw[fill opacity=0.8,fill=gray!20,draw=none](-7.528,1.071)--(-7.55,1.06)--(-7.55,1.014)--(-7.521,.97)--(-7.494,.992)--(-7.494,1.05)--cycle;
\draw(-7.55,1.06)--(-7.55,1.014);
\draw(-7.494,.992)--(-7.494,1.05);
\filldraw[fill opacity=0.8,fill=gray!20,draw=none](-7.447,1.008)--(-7.494,1.052)--(-7.494,.992)--(-7.45,.976)--cycle;
\draw(-7.494,1.052)--(-7.494,.992);
\filldraw[fill opacity=0.8,fill=gray!20,draw=none](-7.529,1.064)--(-7.531,1.051)--(-7.527,1.036)--(-7.489,1.049)--cycle;
\draw(-7.527,1.036)--(-7.489,1.049);
\filldraw[fill opacity=0.8,fill=gray!20,draw=none](-7.528,1.071)--(-7.529,1.064)--(-7.494,1.05)--cycle;
\filldraw[fill opacity=0.8,fill=gray!20,draw=none](-7.513,1.079)--(-7.528,1.071)--(-7.494,1.05)--(-7.494,1.052)--cycle;
\draw(-7.494,1.05)--(-7.494,1.052);
\filldraw[fill opacity=0.8,fill=gray!20,draw=none](-7.513,1.079)--(-7.494,1.052)--(-7.494,1.089)--cycle;
\draw(-7.494,1.052)--(-7.494,1.089);
\filldraw[fill opacity=0.8,fill=gray!20](-7.472,1.08)--(-7.636,1.151)--(-7.678,1.116)--(-7.515,1.044)--cycle;
\filldraw[fill opacity=0.8,fill=gray!20,draw=none](-7.513,1.136)--(-7.569,1.161)--(-7.599,1.153)--(-7.614,1.142)--(-7.54,1.11)--cycle;
\draw(-7.513,1.136)--(-7.569,1.161);
\draw(-7.614,1.142)--(-7.54,1.11);
\filldraw[fill opacity=0.8,fill=gray!20,draw=none](-7.599,1.153)--(-7.624,1.147)--(-7.614,1.142)--cycle;
\draw(-7.624,1.147)--(-7.614,1.142);
\filldraw[fill opacity=0.8,fill=gray!20,draw=none](-7.569,1.196)--(-7.569,1.161)--(-7.614,1.142)--(-7.614,1.189)--cycle;
\draw(-7.614,1.142)--(-7.614,1.189)--(-7.569,1.196)--(-7.569,1.161);
\filldraw[fill opacity=0.8,fill=gray!20,draw=none](-7.404,1.151)--(-7.408,1.135)--(-7.408,1.144)--cycle;
\draw(-7.408,1.135)--(-7.408,1.144);
\filldraw[fill opacity=0.8,fill=gray!20,draw=none](-7.408,1.135)--(-7.41,1.131)--(-7.422,1.132)--(-7.423,1.14)--(-7.404,1.151)--cycle;
\draw(-7.423,1.14)--(-7.404,1.151);
\filldraw[fill opacity=0.8,fill=gray!20,draw=none](-7.423,1.14)--(-7.432,1.134)--(-7.455,1.136)--(-7.461,1.154)--(-7.416,1.182)--cycle;
\draw(-7.423,1.14)--(-7.432,1.134);
\draw(-7.461,1.154)--(-7.416,1.182);
\filldraw[fill opacity=0.8,fill=gray!20,draw=none](-7.614,1.189)--(-7.614,1.142)--(-7.643,1.1)--(-7.643,1.178)--cycle;
\draw(-7.643,1.1)--(-7.643,1.178)--(-7.614,1.189)--(-7.614,1.142);
\filldraw[fill opacity=0.8,fill=gray!20,draw=none](-7.455,1.136)--(-7.481,1.138)--(-7.483,1.14)--(-7.461,1.154)--cycle;
\draw(-7.483,1.14)--(-7.461,1.154);
\filldraw[fill opacity=0.8,fill=gray!20,draw=none](-7.481,1.138)--(-7.485,1.139)--(-7.483,1.14)--cycle;
\draw(-7.485,1.139)--(-7.483,1.14);
\filldraw[fill opacity=0.8,fill=gray!20,draw=none](-7.483,1.14)--(-7.485,1.139)--(-7.513,1.138)--(-7.494,1.15)--cycle;
\draw(-7.483,1.14)--(-7.485,1.139);
\draw(-7.513,1.138)--(-7.494,1.15);
\filldraw[fill opacity=0.8,fill=gray!20,draw=none](-7.461,1.154)--(-7.483,1.14)--(-7.494,1.15)--(-7.459,1.171)--cycle;
\draw(-7.461,1.154)--(-7.483,1.14);
\draw(-7.494,1.15)--(-7.459,1.171);
\filldraw[fill opacity=0.8,fill=gray!20,draw=none](-7.426,1.181)--(-7.413,1.183)--(-7.461,1.154)--(-7.461,1.159)--cycle;
\draw(-7.413,1.183)--(-7.461,1.154);
\filldraw[fill opacity=0.8,fill=gray!20,draw=none](-7.426,1.181)--(-7.461,1.159)--(-7.459,1.171)--(-7.456,1.174)--cycle;
\draw(-7.459,1.171)--(-7.456,1.174);
\filldraw[fill opacity=0.8,fill=gray!20,draw=none](-7.615,1.088)--(-7.678,1.116)--(-7.684,1.106)--(-7.651,1.043)--cycle;
\draw(-7.615,1.088)--(-7.678,1.116)--(-7.684,1.106);
\filldraw[fill opacity=0.8,fill=gray!20,draw=none](-7.643,1.178)--(-7.643,1.1)--(-7.65,1.093)--(-7.65,1.166)--cycle;
\draw(-7.65,1.093)--(-7.65,1.166)--(-7.643,1.178)--(-7.643,1.1);
\filldraw[fill opacity=0.8,fill=gray!20,draw=none](-7.509,1.141)--(-7.513,1.138)--(-7.513,1.136)--cycle;
\draw(-7.509,1.141)--(-7.513,1.138);
\filldraw[fill opacity=0.8,fill=gray!20,draw=none](-7.458,1.112)--(-7.513,1.136)--(-7.538,1.112)--cycle;
\draw(-7.458,1.112)--(-7.513,1.136);
\filldraw[fill opacity=0.8,fill=gray!20,draw=none](-7.494,1.15)--(-7.509,1.141)--(-7.513,1.136)--(-7.513,1.131)--(-7.485,1.148)--cycle;
\draw(-7.494,1.15)--(-7.509,1.141);
\draw(-7.513,1.131)--(-7.485,1.148);
\filldraw[fill opacity=0.8,fill=gray!20,draw=none](-7.456,1.174)--(-7.494,1.15)--(-7.485,1.148)--(-7.462,1.163)--cycle;
\draw(-7.456,1.174)--(-7.494,1.15);
\draw(-7.485,1.148)--(-7.462,1.163);
\filldraw[fill opacity=0.8,fill=gray!20,draw=none](-7.462,1.163)--(-7.528,1.122)--(-7.496,1.112)--(-7.433,1.152)--cycle;
\draw(-7.462,1.163)--(-7.528,1.122);
\draw(-7.496,1.112)--(-7.433,1.152);
\filldraw[fill opacity=0.8,fill=gray!20,draw=none](-7.422,1.132)--(-7.432,1.134)--(-7.423,1.14)--cycle;
\draw(-7.432,1.134)--(-7.423,1.14);
\filldraw[fill opacity=0.8,fill=gray!20,draw=none](-7.375,1.101)--(-7.393,1.139)--(-7.393,1.1)--cycle;
\draw(-7.393,1.139)--(-7.393,1.1);
\filldraw[fill opacity=0.8,fill=gray!20,draw=none](-7.397,1.146)--(-7.433,1.152)--(-7.485,1.119)--(-7.432,1.105)--(-7.415,1.115)--cycle;
\draw(-7.433,1.152)--(-7.485,1.119);
\draw(-7.432,1.105)--(-7.415,1.115);
\filldraw[fill opacity=0.8,fill=gray!20](-7.438,1.139)--(-7.393,1.147)--(-7.365,1.158)--(-7.358,1.17)--(-7.373,1.182)--(-7.408,1.192)--(-7.458,1.198)--(-7.514,1.2)--(-7.569,1.196)--(-7.614,1.189)--(-7.643,1.178)--(-7.65,1.166)--(-7.635,1.154)--(-7.6,1.144)--(-7.55,1.138)--(-7.494,1.136)--cycle;
\filldraw[fill opacity=0.8,fill=gray!20,draw=none](-3.598,1.008)--(-3.585,1.041)--(-3.63,1.035)--(-3.614,1.001)--cycle;
\draw(-3.63,1.035)--(-3.614,1.001);
\filldraw[fill opacity=0.8,fill=gray!20,draw=none](-7.771,1.803)--(-7.773,1.783)--(-7.773,1.7)--(-7.758,1.704)--(-7.758,1.743)--cycle;
\draw(-7.773,1.783)--(-7.773,1.7);
\draw(-7.758,1.704)--(-7.758,1.743);
\filldraw[fill opacity=0.8,fill=gray!20,draw=none](-7.771,1.705)--(-7.782,1.754)--(-7.782,1.704)--(-7.773,1.7)--cycle;
\draw(-7.782,1.704)--(-7.773,1.7);
\filldraw[fill opacity=0.8,fill=gray!20,draw=none](-3.585,1.041)--(-3.598,1.008)--(-3.569,1.022)--(-3.578,1.042)--cycle;
\draw(-3.569,1.022)--(-3.578,1.042);
\filldraw[fill opacity=0.8,fill=gray!20](-3.185,2.949)--(-3.182,2.982)--(-3.128,2.978)--(-3.157,2.947)--cycle;
\filldraw[fill opacity=0.8,fill=gray!20](-3.214,2.948)--(-3.237,2.979)--(-3.182,2.982)--(-3.185,2.949)--cycle;
\filldraw[fill opacity=0.8,fill=gray!20,draw=none](-3.256,1.912)--(-3.233,1.732)--(-3.154,1.624)--(-3.109,1.622)--(-3.134,1.817)--cycle;
\draw(-3.256,1.912)--(-3.233,1.732);
\draw(-3.154,1.624)--(-3.109,1.622)--(-3.134,1.817);
\filldraw[fill opacity=0.8,fill=gray!20,draw=none](-7.643,1.1)--(-7.65,1.067)--(-7.65,1.093)--cycle;
\draw(-7.65,1.067)--(-7.65,1.093);
\filldraw[fill opacity=0.8,fill=gray!20,draw=none](-7.65,1.042)--(-7.65,.932)--(-7.635,.975)--cycle;
\draw(-7.65,1.042)--(-7.65,.932);
\filldraw[fill opacity=0.8,fill=gray!20,draw=none](-7.651,1.042)--(-7.651,1.043)--(-7.652,1.038)--(-7.654,1.037)--cycle;
\draw(-7.652,1.038)--(-7.654,1.037);
\filldraw[fill opacity=0.8,fill=gray!20,draw=none](-7.65,1.042)--(-7.707,1.066)--(-7.717,1.011)--(-7.635,.975)--cycle;
\draw(-7.65,1.042)--(-7.707,1.066)--(-7.717,1.011)--(-7.635,.975);
\filldraw[fill opacity=0.8,fill=gray!20,draw=none](-3.245,2.062)--(-3.268,2.042)--(-3.252,2.042)--cycle;
\filldraw[fill opacity=0.8,fill=gray!20,draw=none](-3.521,1.058)--(-3.528,1.055)--(-3.527,1.052)--cycle;
\draw(-3.528,1.055)--(-3.527,1.052);
\filldraw[fill opacity=0.8,fill=gray!20,draw=none](-3.888,1.821)--(-3.582,1.169)--(-3.512,1.124)--(-3.706,1.539)--cycle;
\draw(-3.888,1.821)--(-3.582,1.169);
\draw(-3.512,1.124)--(-3.706,1.539);
\filldraw[fill opacity=0.8,fill=gray!20,draw=none](-7.433,1.948)--(-7.503,1.94)--(-7.325,2.003)--cycle;
\draw(-7.503,1.94)--(-7.325,2.003);
\filldraw[fill opacity=0.8,fill=gray!20,draw=none](-3.731,2.124)--(-3.797,2.153)--(-3.768,2.11)--(-3.693,2.077)--cycle;
\draw(-3.731,2.124)--(-3.797,2.153)--(-3.768,2.11)--(-3.693,2.077);
\filldraw[fill opacity=0.8,fill=gray!20,draw=none](-7.684,1.106)--(-7.707,1.066)--(-7.65,1.042)--cycle;
\draw(-7.684,1.106)--(-7.707,1.066)--(-7.65,1.042);
\filldraw[fill opacity=0.8,fill=gray!20,draw=none](-3.549,1.049)--(-3.566,1.023)--(-3.527,1.052)--(-3.528,1.055)--cycle;
\draw(-3.527,1.052)--(-3.528,1.055);
\filldraw[fill opacity=0.8,fill=gray!20,draw=none](-3.6,1.089)--(-3.578,1.042)--(-3.528,1.055)--(-3.54,1.081)--cycle;
\draw(-3.6,1.089)--(-3.578,1.042);
\draw(-3.528,1.055)--(-3.54,1.081);
\filldraw[fill opacity=0.8,fill=gray!20](-3.347,3.274)--(-3.318,3.314)--(-3.258,3.325)--(-3.273,3.288)--cycle;
\filldraw[fill opacity=0.8,fill=gray!20,draw=none](-3.4,2.328)--(-3.383,2.306)--(-3.395,2.405)--cycle;
\draw(-3.383,2.306)--(-3.395,2.405);
\filldraw[fill opacity=0.8,fill=gray!20,draw=none](-3.566,1.023)--(-3.569,1.022)--(-3.568,1.021)--cycle;
\draw(-3.569,1.022)--(-3.568,1.021);
\filldraw[fill opacity=0.8,fill=gray!20,draw=none](-3.549,1.049)--(-3.578,1.042)--(-3.569,1.022)--(-3.566,1.023)--cycle;
\draw(-3.578,1.042)--(-3.569,1.022);
\filldraw[fill opacity=0.8,fill=gray!20,draw=none](-4.445,2.942)--(-4.444,2.941)--(-4.447,2.932)--(-4.456,2.937)--cycle;
\draw(-4.445,2.942)--(-4.444,2.941);
\filldraw[fill opacity=0.8,fill=gray!20,draw=none](-4.42,2.859)--(-4.416,2.857)--(-4.406,2.85)--cycle;
\draw(-4.42,2.859)--(-4.416,2.857)--(-4.406,2.85);
\filldraw[fill opacity=0.8,fill=gray!20,draw=none](-4.438,2.89)--(-4.387,2.795)--(-4.387,2.798)--(-4.394,2.837)--(-4.406,2.882)--cycle;
\draw(-4.387,2.795)--(-4.387,2.798)--(-4.394,2.837)--(-4.406,2.882);
\filldraw[fill opacity=0.8,fill=gray!20,draw=none](-4.393,2.832)--(-4.387,2.838)--(-4.383,2.836)--(-4.372,2.812)--(-4.387,2.803)--cycle;
\draw(-4.383,2.836)--(-4.372,2.812)--(-4.387,2.803);
\filldraw[fill opacity=0.8,fill=gray!20,draw=none](-4.395,2.843)--(-4.387,2.838)--(-4.393,2.832)--cycle;
\filldraw[fill opacity=0.8,fill=gray!20,draw=none](-4.385,2.841)--(-4.382,2.835)--(-4.387,2.838)--cycle;
\draw(-4.382,2.835)--(-4.387,2.838);
\filldraw[fill opacity=0.8,fill=gray!20,draw=none](-4.387,2.838)--(-4.385,2.841)--(-4.383,2.836)--cycle;
\draw(-4.385,2.841)--(-4.383,2.836);
\filldraw[fill opacity=0.8,fill=gray!20,draw=none](-4.438,2.89)--(-4.396,2.844)--(-4.382,2.835)--cycle;
\draw(-4.396,2.844)--(-4.382,2.835);
\filldraw[fill opacity=0.8,fill=gray!20,draw=none](-4.368,2.797)--(-4.386,2.795)--(-4.379,2.808)--(-4.372,2.812)--cycle;
\draw(-4.379,2.808)--(-4.372,2.812)--(-4.368,2.797);
\filldraw[fill opacity=0.8,fill=gray!20,draw=none](-4.438,2.89)--(-4.426,2.799)--(-4.412,2.777)--(-4.397,2.766)--(-4.388,2.773)--(-4.387,2.795)--cycle;
\draw(-4.426,2.799)--(-4.412,2.777)--(-4.397,2.766)--(-4.388,2.773)--(-4.387,2.795);
\filldraw[fill opacity=0.8,fill=gray!20,draw=none](-4.438,2.89)--(-4.368,2.797)--(-4.382,2.794)--(-4.388,2.795)--cycle;
\draw(-4.368,2.797)--(-4.382,2.794)--(-4.388,2.795);
\filldraw[fill opacity=0.8,fill=gray!20,draw=none](-4.368,2.797)--(-4.366,2.787)--(-4.387,2.774)--(-4.387,2.792)--(-4.386,2.795)--cycle;
\draw(-4.368,2.797)--(-4.366,2.787)--(-4.387,2.774);
\filldraw[fill opacity=0.8,fill=gray!20,draw=none](-4.366,2.786)--(-4.394,2.77)--(-4.366,2.787)--cycle;
\draw(-4.394,2.77)--(-4.366,2.787)--(-4.366,2.786);
\filldraw[fill opacity=0.8,fill=gray!20,draw=none](-4.438,2.89)--(-4.407,2.882)--(-4.413,2.893)--(-4.438,2.932)--(-4.444,2.941)--(-4.445,2.942)--cycle;
\draw(-4.407,2.882)--(-4.413,2.893)--(-4.438,2.932);
\draw(-4.444,2.941)--(-4.445,2.942);
\filldraw[fill opacity=0.8,fill=gray!20,draw=none](-4.443,2.939)--(-4.433,2.949)--(-4.427,2.936)--(-4.439,2.934)--cycle;
\draw(-4.433,2.949)--(-4.427,2.936);
\filldraw[fill opacity=0.8,fill=gray!20,draw=none](-4.438,2.89)--(-4.406,2.882)--(-4.407,2.885)--(-4.424,2.929)--(-4.432,2.947)--cycle;
\draw(-4.406,2.882)--(-4.407,2.885)--(-4.424,2.929);
\filldraw[fill opacity=0.8,fill=gray!20,draw=none](-4.407,2.882)--(-4.398,2.866)--(-4.402,2.866)--cycle;
\draw(-4.407,2.882)--(-4.398,2.866);
\filldraw[fill opacity=0.8,fill=gray!20,draw=none](-4.402,2.866)--(-4.398,2.866)--(-4.394,2.858)--(-4.399,2.857)--cycle;
\draw(-4.398,2.866)--(-4.394,2.858);
\filldraw[fill opacity=0.8,fill=gray!20,draw=none](-4.438,2.89)--(-4.383,2.836)--(-4.387,2.846)--(-4.398,2.866)--(-4.407,2.882)--cycle;
\draw(-4.383,2.836)--(-4.387,2.846);
\draw(-4.398,2.866)--(-4.407,2.882);
\filldraw[fill opacity=0.8,fill=gray!20,draw=none](-4.439,2.934)--(-4.441,2.936)--(-4.444,2.941)--cycle;
\draw(-4.439,2.934)--(-4.441,2.936)--(-4.444,2.941);
\filldraw[fill opacity=0.8,fill=gray!20,draw=none](-4.367,2.771)--(-4.377,2.771)--(-4.377,2.771)--(-4.367,2.772)--cycle;
\filldraw[fill opacity=0.8,fill=gray!20,draw=none](-4.367,2.771)--(-4.367,2.772)--(-4.358,2.773)--(-4.365,2.771)--cycle;
\draw(-4.358,2.773)--(-4.365,2.771);
\filldraw[fill opacity=0.8,fill=gray!20,draw=none](-4.388,2.76)--(-4.387,2.779)--(-4.368,2.797)--(-4.364,2.775)--(-4.368,2.77)--cycle;
\draw(-4.364,2.775)--(-4.368,2.77);
\filldraw[fill opacity=0.8,fill=gray!20,draw=none](-4.368,2.797)--(-4.347,2.801)--(-4.364,2.775)--cycle;
\draw(-4.368,2.797)--(-4.347,2.801)--(-4.364,2.775);
\filldraw[fill opacity=0.8,fill=gray!20,draw=none](-4.39,2.769)--(-4.387,2.774)--(-4.366,2.786)--cycle;
\filldraw[fill opacity=0.8,fill=gray!20,draw=none](-4.39,2.769)--(-4.388,2.781)--(-4.367,2.786)--cycle;
\draw(-4.388,2.781)--(-4.367,2.786);
\filldraw[fill opacity=0.8,fill=gray!20,draw=none](-4.367,2.683)--(-4.295,2.623)--(-4.237,2.619)--(-4.325,2.692)--cycle;
\draw(-4.367,2.683)--(-4.295,2.623);
\draw(-4.237,2.619)--(-4.325,2.692);
\filldraw[fill opacity=0.8,fill=gray!20,draw=none](-4.438,2.89)--(-4.438,2.89)--(-4.366,2.786)--(-4.388,2.781)--cycle;
\draw(-4.366,2.786)--(-4.388,2.781);
\filldraw[fill opacity=0.8,fill=gray!20,draw=none](-4.376,2.769)--(-4.39,2.768)--(-4.387,2.77)--(-4.377,2.771)--cycle;
\filldraw[fill opacity=0.8,fill=gray!20,draw=none](-4.376,2.769)--(-4.377,2.771)--(-4.367,2.771)--(-4.366,2.771)--(-4.373,2.77)--cycle;
\draw(-4.366,2.771)--(-4.373,2.77);
\filldraw[fill opacity=0.8,fill=gray!20,draw=none](-4.39,2.768)--(-4.39,2.769)--(-4.367,2.786)--(-4.361,2.787)--(-4.376,2.769)--cycle;
\draw(-4.367,2.786)--(-4.361,2.787)--(-4.376,2.769);
\filldraw[fill opacity=0.8,fill=gray!20,draw=none](-4.388,2.76)--(-4.368,2.77)--(-4.388,2.739)--cycle;
\draw(-4.368,2.77)--(-4.388,2.739);
\filldraw[fill opacity=0.8,fill=gray!20,draw=none](-4.391,2.762)--(-4.39,2.768)--(-4.376,2.769)--(-4.377,2.768)--cycle;
\draw(-4.376,2.769)--(-4.377,2.768);
\filldraw[fill opacity=0.8,fill=gray!20,draw=none](-4.391,2.768)--(-4.39,2.768)--(-4.376,2.769)--(-4.376,2.769)--(-4.389,2.766)--cycle;
\draw(-4.376,2.769)--(-4.389,2.766);
\filldraw[fill opacity=0.8,fill=gray!20,draw=none](-4.389,2.766)--(-4.391,2.768)--(-4.39,2.769)--(-4.366,2.786)--(-4.37,2.778)--(-4.385,2.769)--cycle;
\draw(-4.366,2.786)--(-4.37,2.778)--(-4.385,2.769);
\filldraw[fill opacity=0.8,fill=gray!20,draw=none](-4.379,2.692)--(-4.367,2.683)--(-4.36,2.684)--(-4.376,2.711)--cycle;
\draw(-4.379,2.692)--(-4.367,2.683);
\filldraw[fill opacity=0.8,fill=gray!20,draw=none](-4.386,2.727)--(-4.36,2.684)--(-4.352,2.686)--(-4.334,2.7)--(-4.371,2.731)--cycle;
\draw(-4.334,2.7)--(-4.371,2.731);
\filldraw[fill opacity=0.8,fill=gray!20,draw=none](-4.38,2.682)--(-4.341,2.695)--(-4.343,2.694)--(-4.369,2.684)--(-4.386,2.678)--cycle;
\draw(-4.341,2.695)--(-4.343,2.694);
\draw(-4.369,2.684)--(-4.386,2.678);
\filldraw[fill opacity=0.8,fill=gray!20,draw=none](-4.387,2.846)--(-4.389,2.85)--(-4.398,2.866)--cycle;
\draw(-4.387,2.846)--(-4.389,2.85)--(-4.398,2.866);
\filldraw[fill opacity=0.8,fill=gray!20,draw=none](-4.37,2.829)--(-4.377,2.832)--(-4.331,2.896)--(-4.33,2.896)--cycle;
\draw(-4.37,2.829)--(-4.377,2.832)--(-4.331,2.896);
\filldraw[fill opacity=0.8,fill=gray!20,draw=none](-4.438,2.89)--(-4.382,2.835)--(-4.377,2.832)--(-4.348,2.817)--(-4.332,2.813)--(-4.332,2.814)--cycle;
\draw(-4.382,2.835)--(-4.377,2.832)--(-4.348,2.817)--(-4.332,2.813)--(-4.332,2.814);
\filldraw[fill opacity=0.8,fill=gray!20,draw=none](-4.333,2.814)--(-4.331,2.815)--(-4.332,2.813)--(-4.355,2.789)--(-4.347,2.801)--cycle;
\draw(-4.355,2.789)--(-4.347,2.801)--(-4.333,2.814);
\filldraw[fill opacity=0.8,fill=gray!20,draw=none](-4.392,2.768)--(-4.39,2.769)--(-4.391,2.768)--cycle;
\filldraw[fill opacity=0.8,fill=gray!20,draw=none](-4.392,2.768)--(-4.39,2.769)--(-4.39,2.768)--cycle;
\filldraw[fill opacity=0.8,fill=gray!20,draw=none](-4.392,2.768)--(-4.39,2.768)--(-4.391,2.768)--cycle;
\filldraw[fill opacity=0.8,fill=gray!20,draw=none](-4.393,2.764)--(-4.391,2.768)--(-4.389,2.766)--cycle;
\filldraw[fill opacity=0.8,fill=gray!20,draw=none](-4.391,2.768)--(-4.389,2.766)--(-4.395,2.765)--cycle;
\draw(-4.389,2.766)--(-4.395,2.765);
\filldraw[fill opacity=0.8,fill=gray!20](-4.34,2.715)--(-4.299,2.754)--(-4.283,2.737)--(-4.328,2.703)--cycle;
\filldraw[fill opacity=0.8,fill=gray!20,draw=none](-4.386,2.678)--(-4.346,2.691)--(-4.38,2.685)--(-4.394,2.681)--(-4.398,2.679)--cycle;
\draw(-4.386,2.678)--(-4.346,2.691)--(-4.38,2.685);
\draw(-4.394,2.681)--(-4.398,2.679);
\filldraw[fill opacity=0.8,fill=gray!20,draw=none](-4.366,2.771)--(-4.363,2.772)--(-4.365,2.773)--cycle;
\draw(-4.366,2.771)--(-4.363,2.772);
\filldraw[fill opacity=0.8,fill=gray!20,draw=none](-4.365,2.773)--(-4.368,2.758)--(-4.383,2.741)--(-4.39,2.734)--(-4.388,2.739)--(-4.368,2.77)--cycle;
\draw(-4.388,2.739)--(-4.368,2.77);
\filldraw[fill opacity=0.8,fill=gray!20,draw=none](-4.391,2.762)--(-4.377,2.768)--(-4.393,2.749)--cycle;
\draw(-4.377,2.768)--(-4.393,2.749);
\filldraw[fill opacity=0.8,fill=gray!20,draw=none](-4.376,2.769)--(-4.366,2.771)--(-4.365,2.773)--(-4.369,2.775)--cycle;
\draw(-4.376,2.769)--(-4.366,2.771);
\filldraw[fill opacity=0.8,fill=gray!20,draw=none](-4.376,2.769)--(-4.373,2.77)--(-4.376,2.769)--cycle;
\draw(-4.373,2.77)--(-4.376,2.769);
\filldraw[fill opacity=0.8,fill=gray!20,draw=none](-4.369,2.775)--(-4.37,2.771)--(-4.393,2.749)--(-4.377,2.768)--cycle;
\draw(-4.393,2.749)--(-4.377,2.768);
\filldraw[fill opacity=0.8,fill=gray!20,draw=none](-4.366,2.786)--(-4.366,2.786)--(-4.368,2.797)--(-4.372,2.812)--(-4.375,2.819)--(-4.383,2.836)--(-4.438,2.89)--cycle;
\draw(-4.366,2.786)--(-4.366,2.786);
\draw(-4.368,2.797)--(-4.372,2.812);
\draw(-4.375,2.819)--(-4.383,2.836);
\filldraw[fill opacity=0.8,fill=gray!20,draw=none](-4.372,2.812)--(-4.372,2.812)--(-4.375,2.819)--cycle;
\draw(-4.372,2.812)--(-4.372,2.812)--(-4.375,2.819);
\filldraw[fill opacity=0.8,fill=gray!20,draw=none](-4.366,2.786)--(-4.366,2.787)--(-4.368,2.797)--cycle;
\draw(-4.366,2.786)--(-4.366,2.787)--(-4.368,2.797);
\filldraw[fill opacity=0.8,fill=gray!20,draw=none](-4.325,2.828)--(-4.33,2.817)--(-4.311,2.844)--cycle;
\filldraw[fill opacity=0.8,fill=gray!20,draw=none](-4.322,2.845)--(-4.324,2.834)--(-4.327,2.82)--(-4.347,2.801)--(-4.368,2.797)--(-4.438,2.89)--cycle;
\draw(-4.322,2.845)--(-4.324,2.834);
\draw(-4.327,2.82)--(-4.347,2.801)--(-4.368,2.797);
\filldraw[fill opacity=0.8,fill=gray!20,draw=none](-4.337,2.808)--(-4.359,2.778)--(-4.365,2.773)--(-4.364,2.775)--(-4.355,2.789)--cycle;
\draw(-4.364,2.775)--(-4.355,2.789);
\filldraw[fill opacity=0.8,fill=gray!20,draw=none](-4.34,2.805)--(-4.339,2.805)--(-4.356,2.785)--(-4.377,2.768)--(-4.361,2.787)--cycle;
\draw(-4.377,2.768)--(-4.361,2.787)--(-4.34,2.805);
\filldraw[fill opacity=0.8,fill=gray!20,draw=none](-4.438,2.89)--(-4.433,2.81)--(-4.426,2.799)--cycle;
\filldraw[fill opacity=0.8,fill=gray!20,draw=none](-4.433,2.81)--(-4.438,2.89)--(-4.444,2.833)--(-4.443,2.831)--(-4.434,2.812)--cycle;
\draw(-4.443,2.831)--(-4.434,2.812);
\filldraw[fill opacity=0.8,fill=gray!20,draw=none](-4.322,2.845)--(-4.325,2.832)--(-4.329,2.822)--(-4.332,2.816)--(-4.438,2.89)--cycle;
\draw(-4.325,2.832)--(-4.329,2.822);
\filldraw[fill opacity=0.8,fill=gray!20,draw=none](-4.438,2.89)--(-4.332,2.814)--(-4.332,2.816)--cycle;
\draw(-4.332,2.814)--(-4.332,2.816);
\filldraw[fill opacity=0.8,fill=gray!20,draw=none](-4.332,2.816)--(-4.338,2.806)--(-4.361,2.787)--(-4.366,2.786)--(-4.438,2.89)--cycle;
\draw(-4.338,2.806)--(-4.361,2.787)--(-4.366,2.786);
\filldraw[fill opacity=0.8,fill=gray!20,draw=none](-4.387,2.787)--(-4.384,2.785)--(-4.37,2.778)--(-4.366,2.786)--(-4.438,2.89)--cycle;
\draw(-4.384,2.785)--(-4.37,2.778)--(-4.366,2.786);
\filldraw[fill opacity=0.8,fill=gray!20,draw=none](-4.379,2.864)--(-4.438,2.89)--(-4.332,2.816)--(-4.332,2.818)--(-4.334,2.822)--(-4.349,2.838)--cycle;
\draw(-4.332,2.816)--(-4.332,2.818);
\draw(-4.334,2.822)--(-4.349,2.838)--(-4.379,2.864);
\filldraw[fill opacity=0.8,fill=gray!20,draw=none](-4.379,2.864)--(-4.438,2.89)--(-4.401,2.769)--(-4.389,2.782)--(-4.38,2.818)--cycle;
\draw(-4.401,2.769)--(-4.389,2.782)--(-4.38,2.818)--(-4.379,2.864);
\filldraw[fill opacity=0.8,fill=gray!20,draw=none](-4.438,2.89)--(-4.412,2.814)--(-4.426,2.799)--cycle;
\draw(-4.412,2.814)--(-4.426,2.799);
\filldraw[fill opacity=0.8,fill=gray!20,draw=none](-4.418,2.822)--(-4.438,2.89)--(-4.433,2.811)--(-4.424,2.817)--cycle;
\draw(-4.433,2.811)--(-4.424,2.817);
\filldraw[fill opacity=0.8,fill=gray!20,draw=none](-4.43,2.838)--(-4.413,2.815)--(-4.404,2.805)--(-4.387,2.787)--(-4.438,2.89)--cycle;
\draw(-4.43,2.838)--(-4.413,2.815);
\draw(-4.404,2.805)--(-4.387,2.787);
\filldraw[fill opacity=0.8,fill=gray!20,draw=none](-4.379,2.864)--(-4.405,2.821)--(-4.412,2.814)--(-4.438,2.89)--cycle;
\draw(-4.379,2.864)--(-4.405,2.821)--(-4.412,2.814);
\filldraw[fill opacity=0.8,fill=gray!20,draw=none](-4.379,2.864)--(-4.438,2.89)--(-4.418,2.822)--(-4.413,2.826)--cycle;
\draw(-4.418,2.822)--(-4.413,2.826)--(-4.379,2.864);
\filldraw[fill opacity=0.8,fill=gray!20,draw=none](-4.343,2.694)--(-4.346,2.691)--(-4.369,2.684)--cycle;
\draw(-4.343,2.694)--(-4.346,2.691)--(-4.369,2.684);
\filldraw[fill opacity=0.8,fill=gray!20,draw=none](-4.393,2.764)--(-4.389,2.766)--(-4.388,2.766)--(-4.394,2.763)--cycle;
\draw(-4.388,2.766)--(-4.394,2.763);
\filldraw[fill opacity=0.8,fill=gray!20,draw=none](-4.401,2.769)--(-4.395,2.765)--(-4.384,2.767)--(-4.389,2.782)--cycle;
\draw(-4.395,2.765)--(-4.384,2.767);
\draw(-4.389,2.782)--(-4.401,2.769);
\filldraw[fill opacity=0.8,fill=gray!20,draw=none](-4.384,2.785)--(-4.391,2.781)--(-4.408,2.771)--(-4.394,2.763)--(-4.37,2.778)--cycle;
\draw(-4.391,2.781)--(-4.408,2.771);
\draw(-4.394,2.763)--(-4.37,2.778)--(-4.384,2.785);
\filldraw[fill opacity=0.8,fill=gray!20,draw=none](-4.384,2.767)--(-4.376,2.769)--(-4.369,2.775)--(-4.383,2.783)--(-4.389,2.782)--cycle;
\draw(-4.384,2.767)--(-4.376,2.769);
\draw(-4.383,2.783)--(-4.389,2.782);
\filldraw[fill opacity=0.8,fill=gray!20,draw=none](-4.327,2.706)--(-4.367,2.757)--(-4.374,2.733)--(-4.334,2.7)--cycle;
\draw(-4.374,2.733)--(-4.334,2.7);
\filldraw[fill opacity=0.8,fill=gray!20,draw=none](-4.358,2.746)--(-4.327,2.706)--(-4.301,2.725)--(-4.298,2.729)--(-4.334,2.76)--cycle;
\draw(-4.298,2.729)--(-4.334,2.76);
\filldraw[fill opacity=0.8,fill=gray!20,draw=none](-4.357,2.698)--(-4.38,2.685)--(-4.346,2.691)--(-4.311,2.718)--cycle;
\draw(-4.38,2.685)--(-4.346,2.691)--(-4.311,2.718);
\filldraw[fill opacity=0.8,fill=gray!20,draw=none](-4.37,2.771)--(-4.373,2.764)--(-4.391,2.747)--(-4.396,2.744)--(-4.393,2.749)--cycle;
\filldraw[fill opacity=0.8,fill=gray!20,draw=none](-4.39,2.734)--(-4.391,2.732)--(-4.399,2.722)--(-4.395,2.73)--cycle;
\draw(-4.399,2.722)--(-4.395,2.73);
\filldraw[fill opacity=0.8,fill=gray!20,draw=none](-4.388,2.731)--(-4.386,2.727)--(-4.371,2.731)--(-4.377,2.735)--cycle;
\draw(-4.371,2.731)--(-4.377,2.735);
\filldraw[fill opacity=0.8,fill=gray!20,draw=none](-4.39,2.734)--(-4.388,2.731)--(-4.377,2.735)--(-4.383,2.741)--cycle;
\draw(-4.377,2.735)--(-4.383,2.741);
\filldraw[fill opacity=0.8,fill=gray!20,draw=none](-4.396,2.744)--(-4.39,2.734)--(-4.383,2.741)--(-4.391,2.747)--cycle;
\draw(-4.383,2.741)--(-4.391,2.747);
\filldraw[fill opacity=0.8,fill=gray!20,draw=none](-4.383,2.741)--(-4.391,2.732)--(-4.39,2.734)--cycle;
\filldraw[fill opacity=0.8,fill=gray!20,draw=none](-4.398,2.679)--(-4.394,2.681)--(-4.399,2.68)--cycle;
\draw(-4.398,2.679)--(-4.394,2.681);
\filldraw[fill opacity=0.8,fill=gray!20,draw=none](-4.396,2.744)--(-4.396,2.743)--(-4.405,2.734)--(-4.402,2.739)--cycle;
\draw(-4.405,2.734)--(-4.402,2.739);
\filldraw[fill opacity=0.8,fill=gray!20,draw=none](-4.367,2.757)--(-4.368,2.758)--(-4.383,2.741)--(-4.374,2.733)--cycle;
\draw(-4.383,2.741)--(-4.374,2.733);
\filldraw[fill opacity=0.8,fill=gray!20,draw=none](-4.365,2.773)--(-4.368,2.77)--(-4.364,2.775)--cycle;
\draw(-4.368,2.77)--(-4.364,2.775);
\filldraw[fill opacity=0.8,fill=gray!20,draw=none](-4.391,2.732)--(-4.367,2.76)--(-4.379,2.741)--(-4.394,2.724)--cycle;
\draw(-4.367,2.76)--(-4.379,2.741);
\filldraw[fill opacity=0.8,fill=gray!20,draw=none](-4.377,2.749)--(-4.371,2.762)--(-4.374,2.765)--(-4.387,2.744)--(-4.383,2.741)--cycle;
\draw(-4.387,2.744)--(-4.383,2.741);
\filldraw[fill opacity=0.8,fill=gray!20,draw=none](-4.389,2.766)--(-4.385,2.769)--(-4.388,2.766)--cycle;
\draw(-4.385,2.769)--(-4.388,2.766);
\filldraw[fill opacity=0.8,fill=gray!20,draw=none](-4.389,2.782)--(-4.355,2.789)--(-4.348,2.8)--(-4.342,2.817)--(-4.341,2.824)--(-4.342,2.825)--(-4.38,2.818)--cycle;
\draw(-4.342,2.825)--(-4.38,2.818)--(-4.389,2.782)--(-4.355,2.789);
\filldraw[fill opacity=0.8,fill=gray!20,draw=none](-4.408,2.808)--(-4.422,2.791)--(-4.408,2.771)--(-4.385,2.785)--cycle;
\draw(-4.408,2.771)--(-4.385,2.785)--(-4.408,2.808);
\filldraw[fill opacity=0.8,fill=gray!20,draw=none](-4.384,2.785)--(-4.385,2.785)--(-4.391,2.781)--cycle;
\draw(-4.384,2.785)--(-4.385,2.785)--(-4.391,2.781);
\filldraw[fill opacity=0.8,fill=gray!20,draw=none](-4.374,2.765)--(-4.405,2.805)--(-4.418,2.784)--(-4.405,2.759)--(-4.387,2.744)--cycle;
\draw(-4.405,2.759)--(-4.387,2.744);
\filldraw[fill opacity=0.8,fill=gray!20,draw=none](-4.347,2.79)--(-4.365,2.773)--(-4.363,2.772)--(-4.34,2.776)--cycle;
\draw(-4.363,2.772)--(-4.34,2.776);
\filldraw[fill opacity=0.8,fill=gray!20,draw=none](-4.367,2.771)--(-4.365,2.771)--(-4.366,2.771)--cycle;
\draw(-4.365,2.771)--(-4.366,2.771);
\filldraw[fill opacity=0.8,fill=gray!20,draw=none](-4.359,2.778)--(-4.366,2.769)--(-4.365,2.773)--cycle;
\filldraw[fill opacity=0.8,fill=gray!20,draw=none](-4.34,2.805)--(-4.333,2.811)--(-4.338,2.805)--cycle;
\draw(-4.34,2.805)--(-4.333,2.811)--(-4.338,2.805);
\filldraw[fill opacity=0.8,fill=gray!20,draw=none](-4.332,2.813)--(-4.333,2.811)--(-4.338,2.806)--cycle;
\draw(-4.332,2.813)--(-4.333,2.811)--(-4.338,2.806);
\filldraw[fill opacity=0.8,fill=gray!20,draw=none](-4.337,2.808)--(-4.332,2.813)--(-4.367,2.762)--(-4.366,2.769)--cycle;
\filldraw[fill opacity=0.8,fill=gray!20,draw=none](-4.339,2.805)--(-4.338,2.805)--(-4.352,2.789)--(-4.356,2.785)--cycle;
\draw(-4.338,2.805)--(-4.352,2.789);
\filldraw[fill opacity=0.8,fill=gray!20,draw=none](-4.365,2.773)--(-4.351,2.786)--(-4.352,2.79)--(-4.383,2.783)--cycle;
\draw(-4.352,2.79)--(-4.383,2.783);
\filldraw[fill opacity=0.8,fill=gray!20,draw=none](-4.369,2.775)--(-4.352,2.789)--(-4.353,2.787)--(-4.37,2.771)--cycle;
\draw(-4.352,2.789)--(-4.353,2.787);
\filldraw[fill opacity=0.8,fill=gray!20,draw=none](-4.37,2.771)--(-4.353,2.787)--(-4.369,2.767)--(-4.373,2.764)--cycle;
\draw(-4.353,2.787)--(-4.369,2.767);
\filldraw[fill opacity=0.8,fill=gray!20,draw=none](-4.387,2.787)--(-4.385,2.785)--(-4.384,2.785)--cycle;
\draw(-4.387,2.787)--(-4.385,2.785)--(-4.384,2.785);
\filldraw[fill opacity=0.8,fill=gray!20,draw=none](-4.392,2.789)--(-4.371,2.762)--(-4.363,2.784)--(-4.376,2.795)--cycle;
\draw(-4.363,2.784)--(-4.376,2.795);
\filldraw[fill opacity=0.8,fill=gray!20,draw=none](-4.369,2.767)--(-4.379,2.756)--(-4.383,2.752)--(-4.418,2.72)--(-4.405,2.734)--cycle;
\draw(-4.369,2.767)--(-4.379,2.756);
\draw(-4.418,2.72)--(-4.405,2.734);
\filldraw[fill opacity=0.8,fill=gray!20,draw=none](-4.396,2.744)--(-4.391,2.747)--(-4.405,2.759)--cycle;
\draw(-4.391,2.747)--(-4.405,2.759);
\filldraw[fill opacity=0.8,fill=gray!20,draw=none](-4.391,2.747)--(-4.396,2.743)--(-4.396,2.744)--cycle;
\filldraw[fill opacity=0.8,fill=gray!20,draw=none](-4.428,2.758)--(-4.418,2.77)--(-4.432,2.781)--cycle;
\draw(-4.418,2.77)--(-4.432,2.781);
\filldraw[fill opacity=0.8,fill=gray!20,draw=none](-4.332,2.815)--(-4.332,2.814)--(-4.33,2.817)--cycle;
\draw(-4.332,2.815)--(-4.332,2.814);
\filldraw[fill opacity=0.8,fill=gray!20,draw=none](-4.333,2.814)--(-4.325,2.822)--(-4.33,2.815)--cycle;
\draw(-4.333,2.814)--(-4.325,2.822)--(-4.33,2.815);
\filldraw[fill opacity=0.8,fill=gray!20,draw=none](-4.324,2.834)--(-4.325,2.822)--(-4.327,2.82)--cycle;
\draw(-4.324,2.834)--(-4.325,2.822)--(-4.327,2.82);
\filldraw[fill opacity=0.8,fill=gray!20,draw=none](-4.333,2.819)--(-4.332,2.815)--(-4.33,2.817)--(-4.325,2.828)--cycle;
\draw(-4.333,2.819)--(-4.332,2.815);
\filldraw[fill opacity=0.8,fill=gray!20,draw=none](-4.329,2.822)--(-4.332,2.813)--(-4.338,2.806)--cycle;
\draw(-4.329,2.822)--(-4.332,2.813);
\filldraw[fill opacity=0.8,fill=gray!20,draw=none](-4.347,2.79)--(-4.34,2.776)--(-4.312,2.782)--(-4.299,2.787)--(-4.306,2.799)--(-4.346,2.791)--cycle;
\draw(-4.34,2.776)--(-4.312,2.782);
\draw(-4.306,2.799)--(-4.346,2.791);
\filldraw[fill opacity=0.8,fill=gray!20,draw=none](-4.346,2.791)--(-4.268,2.807)--(-4.313,2.83)--cycle;
\draw(-4.346,2.791)--(-4.268,2.807);
\filldraw[fill opacity=0.8,fill=gray!20,draw=none](-4.332,2.813)--(-4.338,2.805)--(-4.333,2.811)--cycle;
\draw(-4.338,2.805)--(-4.333,2.811)--(-4.332,2.813);
\filldraw[fill opacity=0.8,fill=gray!20,draw=none](-4.334,2.904)--(-4.326,2.886)--(-4.321,2.851)--(-4.322,2.845)--(-4.438,2.89)--cycle;
\draw(-4.326,2.886)--(-4.321,2.851);
\filldraw[fill opacity=0.8,fill=gray!20,draw=none](-4.331,2.815)--(-4.33,2.815)--(-4.332,2.813)--cycle;
\filldraw[fill opacity=0.8,fill=gray!20,draw=none](-4.332,2.813)--(-4.329,2.822)--(-4.348,2.8)--(-4.353,2.787)--(-4.338,2.805)--cycle;
\draw(-4.332,2.813)--(-4.329,2.822);
\draw(-4.353,2.787)--(-4.338,2.805);
\filldraw[fill opacity=0.8,fill=gray!20,draw=none](-4.348,2.8)--(-4.355,2.789)--(-4.352,2.79)--cycle;
\draw(-4.355,2.789)--(-4.352,2.79);
\filldraw[fill opacity=0.8,fill=gray!20,draw=none](-4.348,2.8)--(-4.365,2.781)--(-4.374,2.762)--(-4.353,2.787)--cycle;
\draw(-4.374,2.762)--(-4.353,2.787);
\filldraw[fill opacity=0.8,fill=gray!20,draw=none](-4.34,2.896)--(-4.339,2.895)--(-4.334,2.886)--(-4.326,2.866)--(-4.322,2.85)--(-4.322,2.845)--(-4.438,2.89)--cycle;
\draw(-4.34,2.896)--(-4.339,2.895);
\draw(-4.334,2.886)--(-4.326,2.866);
\draw(-4.322,2.85)--(-4.322,2.845);
\filldraw[fill opacity=0.8,fill=gray!20,draw=none](-4.334,2.821)--(-4.326,2.845)--(-4.331,2.849)--(-4.341,2.85)--(-4.349,2.838)--cycle;
\draw(-4.341,2.85)--(-4.349,2.838)--(-4.334,2.821);
\filldraw[fill opacity=0.8,fill=gray!20,draw=none](-4.342,2.807)--(-4.331,2.828)--(-4.345,2.817)--(-4.349,2.812)--(-4.365,2.781)--cycle;
\filldraw[fill opacity=0.8,fill=gray!20,draw=none](-4.322,2.85)--(-4.337,2.83)--(-4.349,2.812)--(-4.373,2.75)--(-4.325,2.822)--cycle;
\draw(-4.337,2.83)--(-4.349,2.812);
\draw(-4.373,2.75)--(-4.325,2.822)--(-4.322,2.85);
\filldraw[fill opacity=0.8,fill=gray!20,draw=none](-4.342,2.817)--(-4.339,2.826)--(-4.341,2.826)--cycle;
\draw(-4.339,2.826)--(-4.341,2.826);
\filldraw[fill opacity=0.8,fill=gray!20,draw=none](-4.357,2.792)--(-4.36,2.788)--(-4.379,2.744)--(-4.379,2.741)--(-4.373,2.75)--cycle;
\draw(-4.379,2.741)--(-4.373,2.75);
\filldraw[fill opacity=0.8,fill=gray!20,draw=none](-4.419,2.815)--(-4.43,2.805)--(-4.426,2.799)--(-4.411,2.815)--cycle;
\draw(-4.426,2.799)--(-4.411,2.815);
\filldraw[fill opacity=0.8,fill=gray!20,draw=none](-4.405,2.805)--(-4.409,2.811)--(-4.419,2.815)--(-4.43,2.805)--(-4.418,2.784)--cycle;
\filldraw[fill opacity=0.8,fill=gray!20,draw=none](-4.446,2.791)--(-4.444,2.792)--(-4.449,2.793)--cycle;
\filldraw[fill opacity=0.8,fill=gray!20,draw=none](-4.408,2.808)--(-4.426,2.797)--(-4.422,2.791)--cycle;
\draw(-4.408,2.808)--(-4.426,2.797);
\filldraw[fill opacity=0.8,fill=gray!20,draw=none](-4.44,2.796)--(-4.451,2.795)--(-4.444,2.792)--cycle;
\draw(-4.451,2.795)--(-4.444,2.792);
\filldraw[fill opacity=0.8,fill=gray!20,draw=none](-4.427,2.8)--(-4.43,2.805)--(-4.44,2.796)--cycle;
\filldraw[fill opacity=0.8,fill=gray!20,draw=none](-4.449,2.793)--(-4.427,2.8)--(-4.431,2.806)--(-4.454,2.798)--cycle;
\draw(-4.427,2.8)--(-4.431,2.806)--(-4.454,2.798);
\filldraw[fill opacity=0.8,fill=gray!20,draw=none](-4.433,2.81)--(-4.433,2.81)--(-4.431,2.806)--(-4.426,2.799)--cycle;
\draw(-4.433,2.81)--(-4.431,2.806)--(-4.426,2.799);
\filldraw[fill opacity=0.8,fill=gray!20,draw=none](-4.434,2.813)--(-4.439,2.813)--(-4.46,2.799)--(-4.457,2.798)--cycle;
\draw(-4.46,2.799)--(-4.457,2.798);
\filldraw[fill opacity=0.8,fill=gray!20,draw=none](-4.419,2.815)--(-4.424,2.817)--(-4.433,2.811)--(-4.43,2.805)--cycle;
\draw(-4.424,2.817)--(-4.433,2.811);
\filldraw[fill opacity=0.8,fill=gray!20,draw=none](-4.413,2.815)--(-4.43,2.838)--(-4.432,2.838)--(-4.44,2.813)--cycle;
\draw(-4.413,2.815)--(-4.43,2.838);
\filldraw[fill opacity=0.8,fill=gray!20,draw=none](-4.419,2.815)--(-4.411,2.815)--(-4.405,2.821)--(-4.409,2.823)--cycle;
\draw(-4.411,2.815)--(-4.405,2.821)--(-4.409,2.823);
\filldraw[fill opacity=0.8,fill=gray!20,draw=none](-4.415,2.818)--(-4.409,2.811)--(-4.376,2.795)--(-4.409,2.823)--cycle;
\draw(-4.376,2.795)--(-4.409,2.823);
\filldraw[fill opacity=0.8,fill=gray!20,draw=none](-4.409,2.811)--(-4.418,2.822)--(-4.424,2.817)--cycle;
\draw(-4.418,2.822)--(-4.424,2.817);
\filldraw[fill opacity=0.8,fill=gray!20,draw=none](-4.418,2.822)--(-4.415,2.818)--(-4.409,2.823)--(-4.413,2.826)--cycle;
\draw(-4.409,2.823)--(-4.413,2.826)--(-4.418,2.822);
\filldraw[fill opacity=0.8,fill=gray!20,draw=none](-4.418,2.822)--(-4.424,2.817)--(-4.418,2.822)--cycle;
\draw(-4.424,2.817)--(-4.418,2.822);
\filldraw[fill opacity=0.8,fill=gray!20,draw=none](-4.43,2.805)--(-4.409,2.823)--(-4.418,2.827)--(-4.436,2.815)--cycle;
\draw(-4.409,2.823)--(-4.418,2.827);
\filldraw[fill opacity=0.8,fill=gray!20,draw=none](-4.433,2.81)--(-4.434,2.812)--(-4.433,2.81)--cycle;
\draw(-4.434,2.812)--(-4.433,2.81);
\filldraw[fill opacity=0.8,fill=gray!20,draw=none](-4.413,2.815)--(-4.44,2.813)--(-4.45,2.782)--(-4.408,2.808)--cycle;
\draw(-4.45,2.782)--(-4.408,2.808)--(-4.413,2.815);
\filldraw[fill opacity=0.8,fill=gray!20,draw=none](-4.434,2.813)--(-4.436,2.815)--(-4.439,2.813)--cycle;
\filldraw[fill opacity=0.8,fill=gray!20,draw=none](-4.471,2.824)--(-3.63,1.035)--(-3.578,1.042)--(-4.424,2.84)--cycle;
\draw(-3.578,1.042)--(-4.424,2.84)--(-4.471,2.824)--(-3.63,1.035);
\filldraw[fill opacity=0.8,fill=gray!20](-3.236,2.943)--(-3.28,2.971)--(-3.237,2.979)--(-3.214,2.948)--cycle;
\filldraw[fill opacity=0.8,fill=gray!20,draw=none](-3.23,2.041)--(-3.232,2.018)--(-3.221,1.928)--(-3.119,1.849)--(-3.143,2.039)--cycle;
\draw(-3.232,2.018)--(-3.221,1.928);
\draw(-3.119,1.849)--(-3.143,2.039);
\filldraw[fill opacity=0.8,fill=gray!20,draw=none](-7.728,1.65)--(-7.745,1.66)--(-7.741,1.662)--cycle;
\draw(-7.745,1.66)--(-7.741,1.662);
\filldraw[fill opacity=0.8,fill=gray!20](-3.157,2.947)--(-3.128,2.978)--(-3.09,2.969)--(-3.137,2.942)--cycle;
\filldraw[fill opacity=0.8,fill=gray!20](-3.083,3.285)--(-3.103,3.323)--(-3.049,3.31)--(-3.018,3.269)--cycle;
\filldraw[fill opacity=0.8,fill=gray!20,draw=none](-7.724,1.62)--(-7.698,1.632)--(-7.661,1.614)--(-7.668,1.611)--cycle;
\draw(-7.724,1.62)--(-7.698,1.632);
\draw(-7.661,1.614)--(-7.668,1.611);
\filldraw[fill opacity=0.8,fill=gray!20,draw=none](-7.728,1.65)--(-7.698,1.632)--(-7.706,1.628)--cycle;
\draw(-7.698,1.632)--(-7.706,1.628);
\filldraw[fill opacity=0.8,fill=gray!20,draw=none](-7.782,1.754)--(-7.775,1.757)--(-7.771,1.704)--cycle;
\draw(-7.782,1.754)--(-7.775,1.757);
\filldraw[fill opacity=0.8,fill=gray!20,draw=none](-4.328,2.703)--(-4.283,2.737)--(-4.307,2.721)--(-4.311,2.718)--(-4.346,2.691)--cycle;
\draw(-4.311,2.718)--(-4.346,2.691)--(-4.328,2.703)--(-4.283,2.737)--(-4.307,2.721);
\filldraw[fill opacity=0.8,fill=gray!20,draw=none](-4.327,2.706)--(-4.299,2.67)--(-4.143,2.54)--(-4.246,2.687)--(-4.296,2.728)--cycle;
\draw(-4.299,2.67)--(-4.143,2.54);
\draw(-4.246,2.687)--(-4.296,2.728);
\filldraw[fill opacity=0.8,fill=gray!20,draw=none](-4.352,2.686)--(-4.325,2.692)--(-4.334,2.7)--cycle;
\draw(-4.325,2.692)--(-4.334,2.7);
\filldraw[fill opacity=0.8,fill=gray!20,draw=none](-4.327,2.706)--(-4.334,2.7)--(-4.299,2.67)--cycle;
\draw(-4.334,2.7)--(-4.299,2.67);
\filldraw[fill opacity=0.8,fill=gray!20,draw=none](-4.332,2.813)--(-4.33,2.815)--(-4.33,2.815)--(-4.367,2.76)--(-4.368,2.758)--(-4.367,2.762)--cycle;
\draw(-4.33,2.815)--(-4.367,2.76);
\filldraw[fill opacity=0.8,fill=gray!20,draw=none](-4.351,2.786)--(-4.346,2.791)--(-4.352,2.79)--cycle;
\draw(-4.346,2.791)--(-4.352,2.79);
\filldraw[fill opacity=0.8,fill=gray!20,draw=none](-4.342,2.807)--(-4.329,2.822)--(-4.325,2.832)--(-4.331,2.828)--cycle;
\draw(-4.329,2.822)--(-4.325,2.832);
\filldraw[fill opacity=0.8,fill=gray!20,draw=none](-4.322,2.845)--(-4.321,2.845)--(-4.325,2.832)--cycle;
\draw(-4.321,2.845)--(-4.325,2.832);
\filldraw[fill opacity=0.8,fill=gray!20,draw=none](-4.333,2.819)--(-4.325,2.828)--(-4.319,2.839)--(-4.333,2.82)--cycle;
\draw(-4.319,2.839)--(-4.333,2.82)--(-4.333,2.819);
\filldraw[fill opacity=0.8,fill=gray!20,draw=none](-4.332,2.818)--(-4.333,2.82)--(-4.334,2.822)--cycle;
\draw(-4.332,2.818)--(-4.333,2.82)--(-4.334,2.822);
\filldraw[fill opacity=0.8,fill=gray!20,draw=none](-4.331,2.828)--(-4.325,2.832)--(-4.321,2.847)--(-4.323,2.843)--cycle;
\draw(-4.325,2.832)--(-4.321,2.847)--(-4.323,2.843);
\filldraw[fill opacity=0.8,fill=gray!20,draw=none](-4.321,2.851)--(-4.321,2.847)--(-4.321,2.845)--(-4.322,2.845)--cycle;
\draw(-4.321,2.851)--(-4.321,2.847)--(-4.321,2.845);
\filldraw[fill opacity=0.8,fill=gray!20,draw=none](-4.337,2.818)--(-4.344,2.806)--(-4.347,2.791)--(-4.346,2.791)--(-4.331,2.808)--cycle;
\draw(-4.347,2.791)--(-4.346,2.791);
\filldraw[fill opacity=0.8,fill=gray!20,draw=none](-4.344,2.806)--(-4.348,2.8)--(-4.352,2.79)--(-4.347,2.791)--cycle;
\draw(-4.352,2.79)--(-4.347,2.791);
\filldraw[fill opacity=0.8,fill=gray!20,draw=none](-4.326,2.845)--(-4.334,2.821)--(-4.333,2.82)--(-4.319,2.84)--cycle;
\draw(-4.334,2.821)--(-4.333,2.82)--(-4.319,2.84);
\filldraw[fill opacity=0.8,fill=gray!20,draw=none](-4.343,2.818)--(-4.331,2.828)--(-4.323,2.843)--(-4.341,2.822)--cycle;
\draw(-4.323,2.843)--(-4.341,2.822);
\filldraw[fill opacity=0.8,fill=gray!20,draw=none](-4.348,2.8)--(-4.331,2.828)--(-4.339,2.826)--cycle;
\draw(-4.331,2.828)--(-4.339,2.826);
\filldraw[fill opacity=0.8,fill=gray!20,draw=none](-4.324,2.875)--(-4.338,2.826)--(-4.321,2.847)--cycle;
\draw(-4.338,2.826)--(-4.321,2.847)--(-4.324,2.875);
\filldraw[fill opacity=0.8,fill=gray!20,draw=none](-4.326,2.866)--(-4.321,2.853)--(-4.322,2.85)--cycle;
\draw(-4.326,2.866)--(-4.321,2.853)--(-4.322,2.85);
\filldraw[fill opacity=0.8,fill=gray!20,draw=none](-4.334,2.886)--(-4.339,2.884)--(-4.343,2.864)--(-4.341,2.85)--(-4.324,2.849)--(-4.321,2.853)--cycle;
\draw(-4.324,2.849)--(-4.321,2.853)--(-4.334,2.886);
\filldraw[fill opacity=0.8,fill=gray!20,draw=none](-4.343,2.818)--(-4.341,2.822)--(-4.347,2.815)--cycle;
\draw(-4.341,2.822)--(-4.347,2.815);
\filldraw[fill opacity=0.8,fill=gray!20,draw=none](-4.322,2.85)--(-4.321,2.853)--(-4.337,2.83)--cycle;
\draw(-4.322,2.85)--(-4.321,2.853)--(-4.337,2.83);
\filldraw[fill opacity=0.8,fill=gray!20,draw=none](-4.338,2.854)--(-4.331,2.849)--(-4.324,2.875)--(-4.326,2.886)--(-4.336,2.876)--(-4.338,2.865)--cycle;
\draw(-4.324,2.875)--(-4.326,2.886);
\filldraw[fill opacity=0.8,fill=gray!20,draw=none](-4.326,2.845)--(-4.318,2.866)--(-4.33,2.865)--(-4.338,2.854)--cycle;
\draw(-4.33,2.865)--(-4.338,2.854);
\filldraw[fill opacity=0.8,fill=gray!20,draw=none](-4.331,2.849)--(-4.338,2.828)--(-4.324,2.849)--cycle;
\draw(-4.338,2.828)--(-4.324,2.849);
\filldraw[fill opacity=0.8,fill=gray!20,draw=none](-4.331,2.849)--(-4.341,2.85)--(-4.338,2.828)--cycle;
\filldraw[fill opacity=0.8,fill=gray!20,draw=none](-4.331,2.849)--(-4.338,2.854)--(-4.341,2.85)--cycle;
\draw(-4.338,2.854)--(-4.341,2.85);
\filldraw[fill opacity=0.8,fill=gray!20,draw=none](-4.331,2.849)--(-4.338,2.854)--(-4.338,2.826)--cycle;
\filldraw[fill opacity=0.8,fill=gray!20,draw=none](-4.349,2.838)--(-4.341,2.85)--(-4.343,2.865)--(-4.379,2.864)--cycle;
\draw(-4.379,2.864)--(-4.349,2.838)--(-4.341,2.85);
\filldraw[fill opacity=0.8,fill=gray!20,draw=none](-4.341,2.85)--(-4.345,2.85)--(-4.346,2.847)--(-4.342,2.823)--(-4.338,2.828)--cycle;
\draw(-4.342,2.823)--(-4.338,2.828);
\filldraw[fill opacity=0.8,fill=gray!20,draw=none](-4.341,2.824)--(-4.341,2.826)--(-4.342,2.825)--cycle;
\draw(-4.341,2.826)--(-4.342,2.825);
\filldraw[fill opacity=0.8,fill=gray!20,draw=none](-4.38,2.818)--(-4.342,2.825)--(-4.346,2.847)--(-4.379,2.864)--cycle;
\draw(-4.379,2.864)--(-4.38,2.818)--(-4.342,2.825);
\filldraw[fill opacity=0.8,fill=gray!20,draw=none](-4.346,2.847)--(-4.35,2.825)--(-4.349,2.812)--(-4.342,2.823)--cycle;
\draw(-4.349,2.812)--(-4.342,2.823);
\filldraw[fill opacity=0.8,fill=gray!20,draw=none](-4.342,2.825)--(-4.341,2.826)--(-4.342,2.845)--(-4.346,2.847)--cycle;
\draw(-4.342,2.825)--(-4.341,2.826);
\filldraw[fill opacity=0.8,fill=gray!20,draw=none](-4.342,2.845)--(-4.347,2.822)--(-4.347,2.815)--(-4.341,2.822)--cycle;
\draw(-4.347,2.815)--(-4.341,2.822);
\filldraw[fill opacity=0.8,fill=gray!20,draw=none](-4.345,2.817)--(-4.347,2.815)--(-4.349,2.812)--cycle;
\filldraw[fill opacity=0.8,fill=gray!20,draw=none](-4.347,2.816)--(-4.347,2.793)--(-4.345,2.769)--(-4.298,2.729)--(-4.287,2.742)--(-4.345,2.819)--cycle;
\draw(-4.345,2.769)--(-4.298,2.729);
\filldraw[fill opacity=0.8,fill=gray!20,draw=none](-4.368,2.758)--(-4.358,2.746)--(-4.334,2.76)--(-4.353,2.776)--cycle;
\draw(-4.334,2.76)--(-4.353,2.776);
\filldraw[fill opacity=0.8,fill=gray!20,draw=none](-4.377,2.749)--(-4.368,2.758)--(-4.371,2.762)--cycle;
\filldraw[fill opacity=0.8,fill=gray!20,draw=none](-4.371,2.762)--(-4.368,2.758)--(-4.367,2.76)--(-4.361,2.782)--(-4.363,2.784)--cycle;
\draw(-4.361,2.782)--(-4.363,2.784);
\filldraw[fill opacity=0.8,fill=gray!20,draw=none](-4.367,2.76)--(-4.353,2.776)--(-4.361,2.782)--cycle;
\draw(-4.353,2.776)--(-4.361,2.782);
\filldraw[fill opacity=0.8,fill=gray!20,draw=none](-4.347,2.793)--(-4.347,2.771)--(-4.345,2.769)--cycle;
\draw(-4.347,2.771)--(-4.345,2.769);
\filldraw[fill opacity=0.8,fill=gray!20,draw=none](-4.361,2.782)--(-4.347,2.771)--(-4.347,2.816)--cycle;
\draw(-4.361,2.782)--(-4.347,2.771);
\filldraw[fill opacity=0.8,fill=gray!20,draw=none](-4.413,2.826)--(-4.361,2.782)--(-4.349,2.812)--(-4.35,2.826)--(-4.379,2.864)--cycle;
\draw(-4.379,2.864)--(-4.413,2.826)--(-4.361,2.782);
\filldraw[fill opacity=0.8,fill=gray!20,draw=none](-4.349,2.812)--(-4.352,2.808)--(-4.379,2.756)--(-4.374,2.762)--cycle;
\draw(-4.379,2.756)--(-4.374,2.762);
\filldraw[fill opacity=0.8,fill=gray!20,draw=none](-4.357,2.792)--(-4.349,2.812)--(-4.353,2.806)--(-4.36,2.788)--cycle;
\draw(-4.349,2.812)--(-4.353,2.806);
\filldraw[fill opacity=0.8,fill=gray!20,draw=none](-4.409,2.811)--(-4.392,2.789)--(-4.376,2.795)--cycle;
\filldraw[fill opacity=0.8,fill=gray!20,draw=none](-4.35,2.825)--(-4.353,2.809)--(-4.353,2.806)--(-4.349,2.812)--cycle;
\draw(-4.353,2.806)--(-4.349,2.812);
\filldraw[fill opacity=0.8,fill=gray!20,draw=none](-4.413,2.815)--(-4.408,2.808)--(-4.404,2.805)--cycle;
\draw(-4.413,2.815)--(-4.408,2.808)--(-4.404,2.805);
\filldraw[fill opacity=0.8,fill=gray!20,draw=none](-4.379,2.864)--(-4.418,2.827)--(-4.405,2.821)--cycle;
\draw(-4.418,2.827)--(-4.405,2.821)--(-4.379,2.864);
\filldraw[fill opacity=0.8,fill=gray!20,draw=none](-4.424,2.84)--(-3.6,1.089)--(-3.54,1.081)--(-4.379,2.864)--cycle;
\draw(-3.54,1.081)--(-4.379,2.864)--(-4.424,2.84)--(-3.6,1.089);
\filldraw[fill opacity=0.8,fill=gray!20,draw=none](-7.488,1.668)--(-7.485,1.672)--(-7.478,1.673)--cycle;
\filldraw[fill opacity=0.8,fill=gray!20,draw=none](-7.77,1.701)--(-7.772,1.704)--(-7.771,1.704)--cycle;
\draw(-7.772,1.704)--(-7.771,1.704);
\filldraw[fill opacity=0.8,fill=gray!20,draw=none](-7.758,1.743)--(-7.782,1.754)--(-7.771,1.705)--cycle;
\draw(-7.758,1.743)--(-7.782,1.754);
\filldraw[fill opacity=0.8,fill=gray!20,draw=none](-7.797,1.747)--(-7.782,1.754)--(-7.771,1.704)--(-7.772,1.704)--cycle;
\draw(-7.797,1.747)--(-7.782,1.754);
\draw(-7.771,1.704)--(-7.772,1.704);
\filldraw[fill opacity=0.8,fill=gray!20,draw=none](-3.392,3.096)--(-3.409,3.136)--(-3.395,3.139)--cycle;
\draw(-3.409,3.136)--(-3.395,3.139);
\filldraw[fill opacity=0.8,fill=gray!20,draw=none](-3.399,3.152)--(-3.395,3.139)--(-3.4,3.138)--cycle;
\draw(-3.395,3.139)--(-3.4,3.138);
\filldraw[fill opacity=0.8,fill=gray!20,draw=none](-4.152,3.032)--(-3.408,3.183)--(-3.399,3.152)--(-3.4,3.138)--(-4.189,2.978)--cycle;
\draw(-4.152,3.032)--(-3.408,3.183);
\draw(-3.4,3.138)--(-4.189,2.978);
\filldraw[fill opacity=0.8,fill=gray!20](-3.4,3.091)--(-3.407,3.146)--(-3.371,3.17)--(-3.365,3.114)--cycle;
\filldraw[fill opacity=0.8,fill=gray!20,draw=none](-3.4,3.086)--(-3.389,3.088)--(-3.381,3.046)--cycle;
\draw(-3.4,3.086)--(-3.389,3.088);
\filldraw[fill opacity=0.8,fill=gray!20,draw=none](-3.392,3.096)--(-3.389,3.088)--(-3.392,3.088)--cycle;
\draw(-3.389,3.088)--(-3.392,3.088);
\filldraw[fill opacity=0.8,fill=gray!20](-3.378,3.039)--(-3.4,3.091)--(-3.365,3.114)--(-3.347,3.059)--cycle;
\filldraw[fill opacity=0.8,fill=gray!20](-3.343,2.994)--(-3.378,3.039)--(-3.347,3.059)--(-3.318,3.011)--cycle;
\filldraw[fill opacity=0.8,fill=gray!20,draw=none](-7.764,1.672)--(-7.797,1.693)--(-7.772,1.704)--(-7.77,1.701)--cycle;
\draw(-7.797,1.693)--(-7.772,1.704);
\filldraw[fill opacity=0.8,fill=gray!20,draw=none](-3.245,2.062)--(-3.252,2.042)--(-3.235,2.042)--(-3.238,2.067)--cycle;
\draw(-3.235,2.042)--(-3.238,2.067);
\filldraw[fill opacity=0.8,fill=gray!20,draw=none](-8.071,1.574)--(-8.069,1.567)--(-8.098,1.566)--(-8.088,1.596)--cycle;
\draw(-8.069,1.567)--(-8.098,1.566);
\filldraw[fill opacity=0.8,fill=gray!20,draw=none](-8.069,1.545)--(-8.103,1.566)--(-8.069,1.567)--cycle;
\draw(-8.103,1.566)--(-8.069,1.567);
\filldraw[fill opacity=0.8,fill=gray!20,draw=none](-8.071,1.574)--(-8.088,1.596)--(-8.079,1.624)--cycle;
\filldraw[fill opacity=0.8,fill=gray!20,draw=none](-8.069,1.545)--(-8.068,1.511)--(-8.121,1.509)--(-8.131,1.565)--(-8.103,1.566)--cycle;
\draw(-8.068,1.511)--(-8.121,1.509)--(-8.131,1.565)--(-8.103,1.566);
\filldraw[fill opacity=0.8,fill=gray!20](-8.195,1.494)--(-8.213,1.549)--(-8.131,1.565)--(-8.121,1.509)--cycle;
\filldraw[fill opacity=0.8,fill=gray!20,draw=none](-8.154,1.591)--(-7.797,1.747)--(-7.772,1.704)--(-8.145,1.541)--cycle;
\draw(-7.772,1.704)--(-8.145,1.541)--(-8.154,1.591)--(-7.797,1.747);
\filldraw[fill opacity=0.8,fill=gray!20,draw=none](-2.968,2.324)--(-2.95,2.185)--(-2.863,2.189)--(-2.884,2.352)--cycle;
\draw(-2.863,2.189)--(-2.884,2.352)--(-2.968,2.324)--(-2.95,2.185);
\filldraw[fill opacity=0.8,fill=gray!20](-3.137,2.846)--(-2.982,1.633)--(-2.859,1.657)--(-3.014,2.87)--cycle;
\filldraw[fill opacity=0.8,fill=gray!20,draw=none](-5.584,2.227)--(-7.252,1.193)--(-7.342,1.19)--(-5.683,2.219)--cycle;
\draw(-5.584,2.227)--(-7.252,1.193);
\draw(-7.342,1.19)--(-5.683,2.219);
\filldraw[fill opacity=0.8,fill=gray!20,draw=none](-3.295,2.064)--(-3.319,2.044)--(-3.302,2.044)--cycle;
\filldraw[fill opacity=0.8,fill=gray!20,draw=none](-7.549,1.692)--(-7.547,1.683)--(-7.518,1.696)--(-7.509,1.744)--(-7.535,1.733)--cycle;
\draw(-7.547,1.683)--(-7.518,1.696)--(-7.509,1.744)--(-7.535,1.733);
\filldraw[fill opacity=0.8,fill=gray!20,draw=none](-7.516,1.716)--(-7.516,1.674)--(-7.488,1.664)--(-7.488,1.732)--cycle;
\draw(-7.516,1.716)--(-7.516,1.674);
\draw(-7.488,1.664)--(-7.488,1.732);
\filldraw[fill opacity=0.8,fill=gray!20,draw=none](-7.488,1.668)--(-7.478,1.673)--(-7.452,1.68)--(-7.489,1.667)--cycle;
\draw(-7.452,1.68)--(-7.489,1.667);
\filldraw[fill opacity=0.8,fill=gray!20,draw=none](-7.488,1.664)--(-7.488,1.646)--(-7.486,1.66)--cycle;
\draw(-7.488,1.664)--(-7.488,1.646);
\filldraw[fill opacity=0.8,fill=gray!20,draw=none](-3.392,3.096)--(-3.395,3.139)--(-3.247,3.169)--(-3.239,3.119)--(-3.389,3.088)--cycle;
\draw(-3.395,3.139)--(-3.247,3.169)--(-3.239,3.119)--(-3.389,3.088);
\filldraw[fill opacity=0.8,fill=gray!20,draw=none](-3.377,3.038)--(-3.378,3.04)--(-3.38,3.043)--(-3.378,3.039)--cycle;
\draw(-3.38,3.043)--(-3.378,3.039)--(-3.377,3.038);
\filldraw[fill opacity=0.8,fill=gray!20,draw=none](-3.378,3.04)--(-3.381,3.046)--(-3.38,3.043)--cycle;
\draw(-3.381,3.046)--(-3.38,3.043);
\filldraw[fill opacity=0.8,fill=gray!20,draw=none](-3.389,3.088)--(-3.239,3.119)--(-3.223,3.072)--(-3.38,3.04)--cycle;
\draw(-3.389,3.088)--(-3.239,3.119)--(-3.223,3.072)--(-3.38,3.04);
\filldraw[fill opacity=0.8,fill=gray!20](-3.13,3.071)--(-3.139,3.036)--(-3.156,3.016)--(-3.178,3.017)--(-3.201,3.036)--(-3.223,3.072)--(-3.239,3.119)--(-3.247,3.169)--(-3.246,3.216)--(-3.237,3.252)--(-3.22,3.271)--(-3.198,3.271)--(-3.175,3.251)--(-3.154,3.215)--(-3.138,3.168)--(-3.129,3.118)--cycle;
\filldraw[fill opacity=0.8,fill=gray!20,draw=none](-7.585,1.633)--(-7.557,1.651)--(-7.616,1.625)--cycle;
\draw(-7.557,1.651)--(-7.616,1.625);
\filldraw[fill opacity=0.8,fill=gray!20,draw=none](-3.399,3.152)--(-3.408,3.183)--(-3.396,3.186)--cycle;
\draw(-3.408,3.183)--(-3.396,3.186);
\filldraw[fill opacity=0.8,fill=gray!20,draw=none](-4.128,3.071)--(-3.402,3.218)--(-3.396,3.186)--(-4.152,3.032)--cycle;
\draw(-4.128,3.071)--(-3.402,3.218);
\draw(-3.396,3.186)--(-4.152,3.032);
\filldraw[fill opacity=0.8,fill=gray!20](-3.407,3.146)--(-3.4,3.202)--(-3.365,3.225)--(-3.371,3.17)--cycle;
\filldraw[fill opacity=0.8,fill=gray!20,draw=none](-7.733,1.854)--(-7.747,1.86)--(-7.77,1.808)--(-7.768,1.808)--cycle;
\draw(-7.733,1.854)--(-7.747,1.86);
\draw(-7.77,1.808)--(-7.768,1.808);
\filldraw[fill opacity=0.8,fill=gray!20](-3.298,2.959)--(-3.343,2.994)--(-3.318,3.011)--(-3.28,2.971)--cycle;
\filldraw[fill opacity=0.8,fill=gray!20](-3.258,3.325)--(-3.237,3.351)--(-3.182,3.353)--(-3.179,3.329)--cycle;
\filldraw[fill opacity=0.8,fill=gray!20](-3.179,3.329)--(-3.182,3.353)--(-3.128,3.349)--(-3.103,3.323)--cycle;
\filldraw[fill opacity=0.8,fill=gray!20,draw=none](-7.758,1.743)--(-7.758,1.704)--(-7.723,1.678)--(-7.723,1.788)--cycle;
\draw(-7.758,1.743)--(-7.758,1.704);
\draw(-7.723,1.678)--(-7.723,1.788);
\filldraw[fill opacity=0.8,fill=gray!20,draw=none](-7.666,1.763)--(-7.77,1.808)--(-7.782,1.754)--(-7.676,1.707)--cycle;
\draw(-7.782,1.754)--(-7.676,1.707)--(-7.666,1.763)--(-7.77,1.808);
\filldraw[fill opacity=0.8,fill=gray!20,draw=none](-3.399,3.152)--(-3.396,3.186)--(-3.246,3.216)--(-3.247,3.169)--(-3.395,3.139)--cycle;
\draw(-3.396,3.186)--(-3.246,3.216)--(-3.247,3.169)--(-3.395,3.139);
\filldraw[fill opacity=0.8,fill=gray!20,draw=none](-3.323,2.312)--(-3.334,2.315)--(-3.332,2.303)--(-3.299,2.283)--cycle;
\draw(-3.323,2.312)--(-3.334,2.315)--(-3.332,2.303);
\filldraw[fill opacity=0.8,fill=gray!20,draw=none](-3.143,2.039)--(-3.119,1.849)--(-3.005,1.795)--(-3.036,2.036)--cycle;
\draw(-3.143,2.039)--(-3.119,1.849);
\draw(-3.005,1.795)--(-3.036,2.036);
\filldraw[fill opacity=0.8,fill=gray!20,draw=none](-3.227,2.072)--(-3.23,2.041)--(-3.143,2.039)--(-3.151,2.106)--cycle;
\draw(-3.143,2.039)--(-3.151,2.106);
\filldraw[fill opacity=0.8,fill=gray!20,draw=none](-3.151,2.106)--(-3.143,2.039)--(-3.036,2.036)--(-3.048,2.13)--cycle;
\draw(-3.151,2.106)--(-3.143,2.039);
\draw(-3.036,2.036)--(-3.048,2.13);
\filldraw[fill opacity=0.8,fill=gray!20,draw=none](-3.162,2.041)--(-3.109,1.622)--(-2.982,1.633)--(-3.034,2.037)--cycle;
\draw(-3.162,2.041)--(-3.109,1.622)--(-2.982,1.633)--(-3.034,2.037);
\filldraw[fill opacity=0.8,fill=gray!20,draw=none](-7.771,1.841)--(-7.773,1.81)--(-7.773,1.835)--cycle;
\draw(-7.773,1.81)--(-7.773,1.835);
\filldraw[fill opacity=0.8,fill=gray!20,draw=none](-4.287,2.742)--(-4.302,2.724)--(-4.283,2.737)--(-4.253,2.774)--(-4.277,2.753)--cycle;
\draw(-4.302,2.724)--(-4.283,2.737)--(-4.253,2.774);
\filldraw[fill opacity=0.8,fill=gray!20,draw=none](-4.326,2.845)--(-4.319,2.84)--(-4.298,2.869)--(-4.318,2.866)--cycle;
\draw(-4.319,2.84)--(-4.298,2.869);
\filldraw[fill opacity=0.8,fill=gray!20,draw=none](-4.337,2.818)--(-4.331,2.808)--(-4.313,2.83)--(-4.314,2.831)--(-4.331,2.828)--cycle;
\draw(-4.314,2.831)--(-4.331,2.828);
\filldraw[fill opacity=0.8,fill=gray!20,draw=none](-4.339,2.895)--(-4.34,2.896)--(-4.347,2.89)--(-4.339,2.884)--cycle;
\draw(-4.339,2.895)--(-4.34,2.896);
\filldraw[fill opacity=0.8,fill=gray!20,draw=none](-4.343,2.892)--(-3.706,1.539)--(-3.561,1.304)--(-4.321,2.92)--cycle;
\draw(-3.561,1.304)--(-4.321,2.92)--(-4.343,2.892)--(-3.706,1.539);
\filldraw[fill opacity=0.8,fill=gray!20,draw=none](-7.561,1.7)--(-7.561,1.673)--(-7.553,1.681)--(-7.549,1.692)--cycle;
\draw(-7.561,1.7)--(-7.561,1.673);
\filldraw[fill opacity=0.8,fill=gray!20,draw=none](-7.616,1.76)--(-7.616,1.661)--(-7.561,1.673)--(-7.561,1.711)--cycle;
\draw(-7.616,1.76)--(-7.616,1.661);
\draw(-7.561,1.673)--(-7.561,1.711);
\filldraw[fill opacity=0.8,fill=gray!20,draw=none](-7.516,1.786)--(-7.516,1.716)--(-7.488,1.732)--(-7.488,1.812)--cycle;
\draw(-7.516,1.786)--(-7.516,1.716);
\draw(-7.488,1.732)--(-7.488,1.812);
\filldraw[fill opacity=0.8,fill=gray!20,draw=none](-7.549,1.692)--(-7.544,1.689)--(-7.516,1.716)--(-7.516,1.786)--cycle;
\draw(-7.516,1.716)--(-7.516,1.786);
\filldraw[fill opacity=0.8,fill=gray!20,draw=none](-7.548,1.727)--(-7.555,1.745)--(-7.573,1.738)--(-7.555,1.696)--(-7.55,1.698)--cycle;
\draw(-7.555,1.745)--(-7.573,1.738);
\draw(-7.555,1.696)--(-7.55,1.698);
\filldraw[fill opacity=0.8,fill=gray!20,draw=none](-7.561,1.711)--(-7.573,1.738)--(-7.586,1.733)--cycle;
\draw(-7.573,1.738)--(-7.586,1.733);
\filldraw[fill opacity=0.8,fill=gray!20,draw=none](-7.676,1.707)--(-7.666,1.654)--(-7.64,1.613)--(-7.624,1.622)--(-7.582,1.668)--(-7.567,1.719)--(-7.576,1.747)--(-7.66,1.774)--(-7.666,1.763)--cycle;
\draw(-7.66,1.774)--(-7.666,1.763)--(-7.676,1.707)--(-7.666,1.654)--(-7.64,1.613);
\filldraw[fill opacity=0.8,fill=gray!20,draw=none](-7.636,1.617)--(-7.543,1.657)--(-7.518,1.696)--(-7.636,1.644)--cycle;
\draw(-7.636,1.617)--(-7.543,1.657)--(-7.518,1.696)--(-7.636,1.644);
\filldraw[fill opacity=0.8,fill=gray!20,draw=none](-6.606,1.668)--(-6.955,1.43)--(-7.342,1.19)--(-7.413,1.183)--(-7.078,1.391)--cycle;
\draw(-6.955,1.43)--(-7.342,1.19);
\draw(-7.413,1.183)--(-7.078,1.391);
\filldraw[fill opacity=0.8,fill=gray!20,draw=none](-6.36,1.837)--(-7.078,1.391)--(-7.426,1.181)--(-7.456,1.174)--(-5.565,2.346)--cycle;
\draw(-6.36,1.837)--(-7.078,1.391);
\draw(-7.456,1.174)--(-5.565,2.346);
\filldraw[fill opacity=0.8,fill=gray!20,draw=none](-4.615,2.971)--(-4.528,3.008)--(-4.533,2.987)--cycle;
\draw(-4.528,3.008)--(-4.533,2.987)--(-4.615,2.971);
\filldraw[fill opacity=0.8,fill=gray!20,draw=none](-4.438,2.89)--(-4.516,2.859)--(-4.517,2.861)--(-4.523,2.868)--cycle;
\draw(-4.517,2.861)--(-4.523,2.868);
\filldraw[fill opacity=0.8,fill=gray!20,draw=none](-4.522,2.859)--(-4.526,2.861)--(-4.523,2.868)--(-4.517,2.861)--cycle;
\draw(-4.523,2.868)--(-4.517,2.861);
\filldraw[fill opacity=0.8,fill=gray!20,draw=none](-4.438,2.89)--(-4.497,2.916)--(-4.522,2.873)--(-4.528,2.853)--cycle;
\draw(-4.497,2.916)--(-4.522,2.873)--(-4.528,2.853);
\filldraw[fill opacity=0.8,fill=gray!20,draw=none](-4.438,2.89)--(-4.523,2.868)--(-4.536,2.884)--cycle;
\draw(-4.523,2.868)--(-4.536,2.884);
\filldraw[fill opacity=0.8,fill=gray!20,draw=none](-4.526,2.861)--(-4.536,2.866)--(-4.536,2.883)--(-4.536,2.884)--(-4.523,2.868)--cycle;
\draw(-4.536,2.884)--(-4.523,2.868);
\filldraw[fill opacity=0.8,fill=gray!20,draw=none](-4.438,2.89)--(-4.528,2.853)--(-4.534,2.861)--(-4.541,2.874)--(-4.542,2.876)--cycle;
\draw(-4.534,2.861)--(-4.541,2.874);
\filldraw[fill opacity=0.8,fill=gray!20,draw=none](-4.542,2.876)--(-4.541,2.874)--(-4.542,2.876)--cycle;
\draw(-4.541,2.874)--(-4.542,2.876);
\filldraw[fill opacity=0.8,fill=gray!20,draw=none](-4.334,2.904)--(-4.321,2.92)--(-4.318,2.943)--(-4.332,2.958)--(-4.363,2.963)--(-4.405,2.956)--(-4.452,2.94)--(-4.497,2.916)--(-4.533,2.888)--(-4.542,2.876)--cycle;
\draw(-4.334,2.904)--(-4.321,2.92)--(-4.318,2.943)--(-4.332,2.958)--(-4.363,2.963)--(-4.405,2.956)--(-4.452,2.94)--(-4.497,2.916)--(-4.533,2.888)--(-4.542,2.876);
\filldraw[fill opacity=0.8,fill=gray!20,draw=none](-4.457,2.958)--(-4.455,2.955)--(-4.463,2.954)--cycle;
\draw(-4.463,2.954)--(-4.457,2.958);
\filldraw[fill opacity=0.8,fill=gray!20,draw=none](-4.457,2.958)--(-4.457,2.958)--(-4.463,2.954)--(-4.469,2.947)--cycle;
\draw(-4.457,2.958)--(-4.463,2.954)--(-4.469,2.947);
\filldraw[fill opacity=0.8,fill=gray!20,draw=none](-4.503,2.978)--(-4.502,2.988)--(-4.496,2.985)--(-4.502,2.976)--cycle;
\draw(-4.496,2.985)--(-4.502,2.976);
\filldraw[fill opacity=0.8,fill=gray!20,draw=none](-4.503,2.978)--(-4.474,2.981)--(-4.474,2.964)--(-4.481,2.961)--(-4.502,2.976)--cycle;
\draw(-4.474,2.964)--(-4.481,2.961);
\filldraw[fill opacity=0.8,fill=gray!20,draw=none](-4.48,2.964)--(-4.473,2.957)--(-4.477,2.95)--cycle;
\filldraw[fill opacity=0.8,fill=gray!20,draw=none](-4.477,2.95)--(-4.49,2.958)--(-4.479,2.965)--cycle;
\draw(-4.49,2.958)--(-4.479,2.965);
\filldraw[fill opacity=0.8,fill=gray!20,draw=none](-4.632,2.96)--(-4.615,2.971)--(-4.618,2.944)--cycle;
\draw(-4.632,2.96)--(-4.615,2.971)--(-4.618,2.944);
\filldraw[fill opacity=0.8,fill=gray!20,draw=none](-4.628,2.944)--(-4.64,2.94)--(-4.678,2.944)--(-4.632,2.96)--cycle;
\draw(-4.628,2.944)--(-4.64,2.94);
\draw(-4.678,2.944)--(-4.632,2.96);
\filldraw[fill opacity=0.8,fill=gray!20,draw=none](-4.602,2.968)--(-4.573,2.971)--(-4.593,2.957)--(-4.628,2.944)--(-4.632,2.96)--(-4.614,2.966)--cycle;
\draw(-4.593,2.957)--(-4.628,2.944);
\draw(-4.632,2.96)--(-4.614,2.966);
\filldraw[fill opacity=0.8,fill=gray!20,draw=none](-4.639,2.922)--(-4.636,2.957)--(-4.632,2.96)--(-4.618,2.944)--(-4.618,2.943)--cycle;
\draw(-4.636,2.957)--(-4.632,2.96);
\draw(-4.618,2.944)--(-4.618,2.943);
\filldraw[fill opacity=0.8,fill=gray!20,draw=none](-4.457,2.958)--(-4.445,2.942)--(-4.456,2.937)--(-4.468,2.945)--(-4.47,2.947)--cycle;
\draw(-4.457,2.958)--(-4.445,2.942);
\filldraw[fill opacity=0.8,fill=gray!20,draw=none](-4.497,2.916)--(-4.546,2.94)--(-4.501,2.958)--cycle;
\draw(-4.497,2.916)--(-4.546,2.94);
\filldraw[fill opacity=0.8,fill=gray!20,draw=none](-4.479,2.965)--(-4.49,2.958)--(-4.501,2.961)--cycle;
\draw(-4.479,2.965)--(-4.49,2.958);
\filldraw[fill opacity=0.8,fill=gray!20,draw=none](-4.483,2.964)--(-4.497,2.916)--(-4.501,2.958)--(-4.487,2.963)--cycle;
\filldraw[fill opacity=0.8,fill=gray!20,draw=none](-4.543,2.975)--(-4.593,2.957)--(-4.588,2.961)--cycle;
\draw(-4.543,2.975)--(-4.593,2.957);
\filldraw[fill opacity=0.8,fill=gray!20,draw=none](-4.546,2.94)--(-4.55,2.943)--(-4.524,2.982)--(-4.523,2.982)--(-4.508,2.978)--(-4.482,2.965)--cycle;
\draw(-4.546,2.94)--(-4.55,2.943);
\draw(-4.508,2.978)--(-4.482,2.965);
\filldraw[fill opacity=0.8,fill=gray!20,draw=none](-4.496,2.962)--(-4.487,2.964)--(-4.48,2.964)--(-4.497,2.916)--cycle;
\draw(-4.497,2.916)--(-4.496,2.962)--(-4.487,2.964);
\filldraw[fill opacity=0.8,fill=gray!20,draw=none](-4.48,2.964)--(-4.477,2.95)--(-4.497,2.916)--cycle;
\filldraw[fill opacity=0.8,fill=gray!20,draw=none](-4.477,2.95)--(-4.475,2.942)--(-4.497,2.916)--cycle;
\filldraw[fill opacity=0.8,fill=gray!20,draw=none](-4.48,2.964)--(-4.47,2.963)--(-4.47,2.962)--(-4.473,2.957)--cycle;
\filldraw[fill opacity=0.8,fill=gray!20,draw=none](-4.474,2.972)--(-4.47,2.975)--(-4.464,2.969)--(-4.474,2.967)--cycle;
\draw(-4.464,2.969)--(-4.474,2.967);
\filldraw[fill opacity=0.8,fill=gray!20,draw=none](-4.641,2.954)--(-4.64,2.955)--(-4.615,2.971)--cycle;
\draw(-4.615,2.971)--(-4.641,2.954);
\filldraw[fill opacity=0.8,fill=gray!20,draw=none](-4.495,3.012)--(-4.467,3.025)--(-4.489,2.995)--cycle;
\draw(-4.467,3.025)--(-4.489,2.995);
\filldraw[fill opacity=0.8,fill=gray!20,draw=none](-4.502,2.988)--(-4.5,3.01)--(-4.495,3.012)--(-4.489,2.995)--(-4.496,2.985)--cycle;
\draw(-4.489,2.995)--(-4.496,2.985);
\filldraw[fill opacity=0.8,fill=gray!20,draw=none](-4.462,3.002)--(-4.474,2.98)--(-4.497,2.979)--(-4.502,2.983)--(-4.5,3.01)--(-4.485,3.015)--(-4.463,3.004)--cycle;
\draw(-4.485,3.015)--(-4.463,3.004);
\filldraw[fill opacity=0.8,fill=gray!20,draw=none](-4.503,2.978)--(-4.502,2.976)--(-4.505,2.978)--cycle;
\filldraw[fill opacity=0.8,fill=gray!20,draw=none](-4.503,2.978)--(-4.502,2.976)--(-4.503,2.975)--cycle;
\draw(-4.502,2.976)--(-4.503,2.975);
\filldraw[fill opacity=0.8,fill=gray!20,draw=none](-4.482,2.965)--(-4.522,2.985)--(-4.515,2.995)--cycle;
\draw(-4.482,2.965)--(-4.522,2.985);
\filldraw[fill opacity=0.8,fill=gray!20,draw=none](-4.502,2.983)--(-4.515,2.995)--(-4.508,3.007)--(-4.5,3.01)--cycle;
\filldraw[fill opacity=0.8,fill=gray!20,draw=none](-4.503,2.978)--(-4.504,2.989)--(-4.502,2.988)--cycle;
\filldraw[fill opacity=0.8,fill=gray!20,draw=none](-4.502,2.988)--(-4.504,2.989)--(-4.506,3.007)--(-4.5,3.01)--cycle;
\filldraw[fill opacity=0.8,fill=gray!20,draw=none](-4.474,2.981)--(-4.505,2.978)--(-4.515,2.985)--(-4.474,2.999)--cycle;
\draw(-4.515,2.985)--(-4.474,2.999);
\filldraw[fill opacity=0.8,fill=gray!20,draw=none](-4.523,2.982)--(-4.553,2.99)--(-4.542,2.995)--(-4.518,2.983)--cycle;
\draw(-4.542,2.995)--(-4.518,2.983);
\filldraw[fill opacity=0.8,fill=gray!20,draw=none](-4.602,2.968)--(-4.605,2.97)--(-4.553,2.99)--(-4.523,2.982)--cycle;
\draw(-4.602,2.968)--(-4.605,2.97);
\filldraw[fill opacity=0.8,fill=gray!20,draw=none](-4.522,2.985)--(-4.542,2.995)--(-4.508,3.007)--cycle;
\draw(-4.522,2.985)--(-4.542,2.995);
\filldraw[fill opacity=0.8,fill=gray!20,draw=none](-4.519,2.994)--(-4.524,2.982)--(-4.543,2.975)--(-4.588,2.961)--(-4.541,2.992)--(-4.527,2.997)--cycle;
\draw(-4.524,2.982)--(-4.543,2.975);
\draw(-4.541,2.992)--(-4.527,2.997);
\filldraw[fill opacity=0.8,fill=gray!20,draw=none](-4.501,2.961)--(-4.542,2.955)--(-4.534,2.968)--(-4.532,2.97)--cycle;
\draw(-4.534,2.968)--(-4.532,2.97);
\filldraw[fill opacity=0.8,fill=gray!20,draw=none](-4.527,2.942)--(-4.503,2.975)--(-4.497,2.916)--cycle;
\draw(-4.497,2.916)--(-4.527,2.942)--(-4.503,2.975);
\filldraw[fill opacity=0.8,fill=gray!20,draw=none](-4.468,2.945)--(-4.477,2.95)--(-4.479,2.965)--cycle;
\filldraw[fill opacity=0.8,fill=gray!20,draw=none](-4.47,2.965)--(-4.474,2.967)--(-4.47,2.968)--cycle;
\draw(-4.474,2.967)--(-4.47,2.968);
\filldraw[fill opacity=0.8,fill=gray!20,draw=none](-4.474,2.954)--(-4.497,2.916)--(-4.482,2.965)--(-4.479,2.963)--cycle;
\draw(-4.474,2.954)--(-4.497,2.916);
\draw(-4.482,2.965)--(-4.479,2.963);
\filldraw[fill opacity=0.8,fill=gray!20,draw=none](-4.523,2.982)--(-4.518,2.983)--(-4.508,2.978)--cycle;
\draw(-4.518,2.983)--(-4.508,2.978);
\filldraw[fill opacity=0.8,fill=gray!20,draw=none](-4.503,2.977)--(-4.524,2.982)--(-4.515,2.985)--cycle;
\draw(-4.524,2.982)--(-4.515,2.985);
\filldraw[fill opacity=0.8,fill=gray!20,draw=none](-4.485,3.024)--(-4.528,3.008)--(-4.523,3.034)--(-4.477,3.036)--cycle;
\draw(-4.528,3.008)--(-4.523,3.034)--(-4.477,3.036);
\filldraw[fill opacity=0.8,fill=gray!20,draw=none](-4.477,3.036)--(-4.523,3.034)--(-4.507,3.072)--(-4.455,3.074)--cycle;
\draw(-4.477,3.036)--(-4.523,3.034)--(-4.507,3.072)--(-4.455,3.074);
\filldraw[fill opacity=0.8,fill=gray!20,draw=none](-4.436,3.024)--(-4.32,3.048)--(-4.33,3.031)--(-4.331,3.03)--(-4.429,3.01)--cycle;
\draw(-4.436,3.024)--(-4.32,3.048);
\draw(-4.331,3.03)--(-4.429,3.01);
\filldraw[fill opacity=0.8,fill=gray!20,draw=none](-4.455,3.074)--(-4.507,3.072)--(-4.506,3.073)--(-4.471,3.098)--(-4.443,3.099)--cycle;
\draw(-4.455,3.074)--(-4.507,3.072)--(-4.506,3.073);
\draw(-4.471,3.098)--(-4.443,3.099);
\filldraw[fill opacity=0.8,fill=gray!20,draw=none](-4.443,3.1)--(-4.44,3.099)--(-4.443,3.099)--cycle;
\draw(-4.44,3.099)--(-4.443,3.099);
\filldraw[fill opacity=0.8,fill=gray!20,draw=none](-4.602,2.968)--(-4.567,2.974)--(-4.573,2.971)--cycle;
\filldraw[fill opacity=0.8,fill=gray!20,draw=none](-4.536,2.964)--(-4.565,2.95)--(-4.602,2.968)--(-4.524,2.982)--cycle;
\draw(-4.565,2.95)--(-4.602,2.968);
\filldraw[fill opacity=0.8,fill=gray!20,draw=none](-4.515,2.993)--(-4.516,2.998)--(-4.513,3.002)--(-4.506,3.007)--(-4.504,2.989)--cycle;
\draw(-4.516,2.998)--(-4.513,3.002);
\filldraw[fill opacity=0.8,fill=gray!20,draw=none](-4.471,3.024)--(-4.495,3.012)--(-4.499,3.022)--(-4.487,3.039)--cycle;
\draw(-4.499,3.022)--(-4.487,3.039);
\filldraw[fill opacity=0.8,fill=gray!20,draw=none](-4.461,3.003)--(-4.489,3.017)--(-4.47,3.02)--cycle;
\draw(-4.461,3.003)--(-4.489,3.017);
\filldraw[fill opacity=0.8,fill=gray!20,draw=none](-4.519,2.994)--(-4.516,2.998)--(-4.515,2.993)--cycle;
\draw(-4.519,2.994)--(-4.516,2.998);
\filldraw[fill opacity=0.8,fill=gray!20,draw=none](-4.485,2.998)--(-4.49,2.994)--(-4.524,2.982)--(-4.518,2.996)--cycle;
\draw(-4.49,2.994)--(-4.524,2.982);
\filldraw[fill opacity=0.8,fill=gray!20,draw=none](-4.515,2.993)--(-4.504,2.989)--(-4.503,2.978)--(-4.503,2.975)--(-4.511,2.964)--cycle;
\draw(-4.503,2.975)--(-4.511,2.964);
\filldraw[fill opacity=0.8,fill=gray!20,draw=none](-4.474,2.972)--(-4.474,2.981)--(-4.463,3)--(-4.441,3.002)--cycle;
\filldraw[fill opacity=0.8,fill=gray!20,draw=none](-4.45,2.981)--(-4.466,2.967)--(-4.474,2.964)--(-4.474,2.981)--cycle;
\draw(-4.466,2.967)--(-4.474,2.964);
\filldraw[fill opacity=0.8,fill=gray!20,draw=none](-4.481,2.964)--(-4.482,2.965)--(-4.474,2.967)--(-4.47,2.965)--(-4.47,2.963)--cycle;
\draw(-4.482,2.965)--(-4.474,2.967);
\filldraw[fill opacity=0.8,fill=gray!20,draw=none](-4.45,2.981)--(-4.445,2.974)--(-4.466,2.967)--cycle;
\draw(-4.45,2.981)--(-4.445,2.974)--(-4.466,2.967);
\filldraw[fill opacity=0.8,fill=gray!20,draw=none](-4.438,2.89)--(-4.457,2.958)--(-4.469,2.947)--(-4.497,2.916)--cycle;
\draw(-4.469,2.947)--(-4.497,2.916);
\filldraw[fill opacity=0.8,fill=gray!20,draw=none](-4.474,2.954)--(-4.47,2.959)--(-4.464,2.966)--cycle;
\draw(-4.474,2.954)--(-4.47,2.959)--(-4.464,2.966);
\filldraw[fill opacity=0.8,fill=gray!20,draw=none](-4.474,2.954)--(-4.479,2.963)--(-4.47,2.959)--cycle;
\draw(-4.479,2.963)--(-4.47,2.959)--(-4.474,2.954);
\filldraw[fill opacity=0.8,fill=gray!20,draw=none](-4.474,2.98)--(-4.482,2.965)--(-4.497,2.979)--cycle;
\filldraw[fill opacity=0.8,fill=gray!20,draw=none](-4.511,2.966)--(-4.487,2.965)--(-4.487,2.963)--(-4.501,2.961)--(-4.511,2.964)--cycle;
\filldraw[fill opacity=0.8,fill=gray!20,draw=none](-4.485,2.989)--(-4.487,2.965)--(-4.518,2.966)--(-4.532,2.97)--(-4.491,2.995)--cycle;
\draw(-4.532,2.97)--(-4.491,2.995)--(-4.485,2.989);
\filldraw[fill opacity=0.8,fill=gray!20,draw=none](-4.46,3)--(-4.463,3)--(-4.461,3.003)--cycle;
\filldraw[fill opacity=0.8,fill=gray!20,draw=none](-4.45,2.981)--(-4.463,3.004)--(-4.439,2.992)--cycle;
\draw(-4.463,3.004)--(-4.439,2.992)--(-4.45,2.981);
\filldraw[fill opacity=0.8,fill=gray!20,draw=none](-4.463,3)--(-4.474,2.999)--(-4.474,3.001)--(-4.461,3.003)--cycle;
\draw(-4.474,3.001)--(-4.461,3.003);
\filldraw[fill opacity=0.8,fill=gray!20,draw=none](-4.475,3.011)--(-4.463,3.003)--(-4.487,2.998)--cycle;
\draw(-4.463,3.003)--(-4.487,2.998)--(-4.475,3.011);
\filldraw[fill opacity=0.8,fill=gray!20,draw=none](-4.487,2.998)--(-4.474,3.001)--(-4.474,2.999)--cycle;
\draw(-4.487,2.998)--(-4.474,3.001);
\filldraw[fill opacity=0.8,fill=gray!20,draw=none](-4.473,3.01)--(-4.464,3.003)--(-4.49,2.994)--cycle;
\draw(-4.473,3.01)--(-4.464,3.003)--(-4.49,2.994);
\filldraw[fill opacity=0.8,fill=gray!20,draw=none](-4.487,2.998)--(-4.474,2.999)--(-4.474,2.967)--(-4.496,2.962)--cycle;
\draw(-4.474,2.967)--(-4.496,2.962)--(-4.487,2.998);
\filldraw[fill opacity=0.8,fill=gray!20,draw=none](-4.463,2.965)--(-4.479,2.965)--(-4.468,2.972)--cycle;
\draw(-4.479,2.965)--(-4.468,2.972)--(-4.463,2.965);
\filldraw[fill opacity=0.8,fill=gray!20,draw=none](-4.463,2.965)--(-4.468,2.972)--(-4.472,2.976)--cycle;
\draw(-4.463,2.965)--(-4.468,2.972)--(-4.472,2.976);
\filldraw[fill opacity=0.8,fill=gray!20,draw=none](-4.472,2.976)--(-4.468,2.972)--(-4.479,2.965)--cycle;
\draw(-4.472,2.976)--(-4.468,2.972)--(-4.479,2.965);
\filldraw[fill opacity=0.8,fill=gray!20,draw=none](-4.474,2.98)--(-4.45,2.981)--(-4.47,2.959)--(-4.482,2.965)--cycle;
\draw(-4.45,2.981)--(-4.47,2.959)--(-4.482,2.965);
\filldraw[fill opacity=0.8,fill=gray!20,draw=none](-4.463,2.965)--(-4.457,2.958)--(-4.47,2.947)--(-4.479,2.965)--cycle;
\draw(-4.463,2.965)--(-4.457,2.958);
\filldraw[fill opacity=0.8,fill=gray!20,draw=none](-4.438,2.89)--(-4.497,2.916)--(-4.474,2.954)--(-4.464,2.966)--cycle;
\draw(-4.497,2.916)--(-4.474,2.954);
\filldraw[fill opacity=0.8,fill=gray!20,draw=none](-4.485,2.989)--(-4.472,2.976)--(-4.479,2.965)--(-4.487,2.965)--cycle;
\draw(-4.485,2.989)--(-4.472,2.976);
\filldraw[fill opacity=0.8,fill=gray!20,draw=none](-4.438,2.89)--(-4.457,2.958)--(-4.463,2.965)--(-4.472,2.976)--(-4.488,2.993)--cycle;
\draw(-4.457,2.958)--(-4.463,2.965);
\draw(-4.472,2.976)--(-4.488,2.993);
\filldraw[fill opacity=0.8,fill=gray!20,draw=none](-4.438,2.89)--(-4.445,2.942)--(-4.457,2.958)--cycle;
\draw(-4.445,2.942)--(-4.457,2.958);
\filldraw[fill opacity=0.8,fill=gray!20,draw=none](-4.432,2.947)--(-4.432,2.947)--(-4.433,2.949)--cycle;
\draw(-4.432,2.947)--(-4.433,2.949);
\filldraw[fill opacity=0.8,fill=gray!20,draw=none](-4.452,2.951)--(-4.455,2.955)--(-4.439,2.962)--(-4.437,2.958)--cycle;
\filldraw[fill opacity=0.8,fill=gray!20,draw=none](-4.455,2.955)--(-4.457,2.958)--(-4.443,2.969)--(-4.439,2.962)--cycle;
\draw(-4.457,2.958)--(-4.443,2.969);
\filldraw[fill opacity=0.8,fill=gray!20,draw=none](-4.443,2.969)--(-4.457,2.958)--(-4.438,2.89)--cycle;
\draw(-4.443,2.969)--(-4.457,2.958);
\filldraw[fill opacity=0.8,fill=gray!20,draw=none](-4.438,2.89)--(-4.464,2.966)--(-4.45,2.981)--cycle;
\draw(-4.464,2.966)--(-4.45,2.981);
\filldraw[fill opacity=0.8,fill=gray!20,draw=none](-4.483,2.964)--(-4.487,2.963)--(-4.482,2.965)--cycle;
\filldraw[fill opacity=0.8,fill=gray!20,draw=none](-4.487,2.965)--(-4.479,2.965)--(-4.487,2.963)--cycle;
\filldraw[fill opacity=0.8,fill=gray!20,draw=none](-4.481,2.964)--(-4.487,2.964)--(-4.482,2.965)--cycle;
\draw(-4.487,2.964)--(-4.482,2.965);
\filldraw[fill opacity=0.8,fill=gray!20,draw=none](-4.462,3.002)--(-4.45,2.981)--(-4.474,2.98)--cycle;
\filldraw[fill opacity=0.8,fill=gray!20,draw=none](-4.474,2.981)--(-4.474,2.999)--(-4.463,3)--cycle;
\filldraw[fill opacity=0.8,fill=gray!20,draw=none](-4.45,2.981)--(-4.474,2.981)--(-4.474,2.999)--(-4.464,3.003)--cycle;
\draw(-4.474,2.999)--(-4.464,3.003)--(-4.45,2.981);
\filldraw[fill opacity=0.8,fill=gray!20,draw=none](-4.473,3.01)--(-4.485,2.998)--(-4.485,3.012)--(-4.479,3.014)--cycle;
\draw(-4.485,3.012)--(-4.479,3.014)--(-4.473,3.01);
\filldraw[fill opacity=0.8,fill=gray!20,draw=none](-4.485,2.998)--(-4.518,2.996)--(-4.517,2.998)--(-4.502,3.006)--(-4.485,3.012)--cycle;
\draw(-4.502,3.006)--(-4.485,3.012);
\filldraw[fill opacity=0.8,fill=gray!20,draw=none](-4.438,2.89)--(-4.475,3.011)--(-4.487,2.998)--(-4.496,2.962)--(-4.497,2.916)--cycle;
\draw(-4.475,3.011)--(-4.487,2.998)--(-4.496,2.962)--(-4.497,2.916);
\filldraw[fill opacity=0.8,fill=gray!20,draw=none](-4.511,2.966)--(-4.511,2.964)--(-4.518,2.966)--cycle;
\filldraw[fill opacity=0.8,fill=gray!20,draw=none](-4.513,2.981)--(-4.534,2.968)--(-4.529,2.98)--cycle;
\draw(-4.513,2.981)--(-4.534,2.968);
\filldraw[fill opacity=0.8,fill=gray!20,draw=none](-4.521,2.991)--(-4.519,2.994)--(-4.515,2.993)--(-4.511,2.966)--(-4.515,2.959)--cycle;
\draw(-4.521,2.991)--(-4.519,2.994);
\filldraw[fill opacity=0.8,fill=gray!20,draw=none](-4.491,2.995)--(-4.491,2.995)--(-4.513,2.981)--(-4.519,2.981)--cycle;
\draw(-4.491,2.995)--(-4.491,2.995)--(-4.513,2.981);
\filldraw[fill opacity=0.8,fill=gray!20,draw=none](-4.592,2.94)--(-4.581,2.958)--(-4.565,2.95)--cycle;
\draw(-4.581,2.958)--(-4.565,2.95);
\filldraw[fill opacity=0.8,fill=gray!20,draw=none](-4.529,2.98)--(-4.521,2.991)--(-4.519,2.981)--cycle;
\draw(-4.529,2.98)--(-4.521,2.991);
\filldraw[fill opacity=0.8,fill=gray!20,draw=none](-4.519,2.994)--(-4.527,2.997)--(-4.516,3.001)--cycle;
\draw(-4.527,2.997)--(-4.516,3.001);
\filldraw[fill opacity=0.8,fill=gray!20,draw=none](-4.524,2.991)--(-4.513,3.002)--(-4.529,2.98)--cycle;
\draw(-4.513,3.002)--(-4.529,2.98);
\filldraw[fill opacity=0.8,fill=gray!20,draw=none](-4.491,2.995)--(-4.519,2.981)--(-4.529,2.98)--(-4.524,2.991)--(-4.505,3.003)--cycle;
\draw(-4.524,2.991)--(-4.505,3.003)--(-4.491,2.995);
\filldraw[fill opacity=0.8,fill=gray!20,draw=none](-4.438,2.89)--(-4.488,2.993)--(-4.491,2.995)--(-4.505,3.003)--(-4.509,2.994)--cycle;
\draw(-4.491,2.995)--(-4.505,3.003)--(-4.509,2.994);
\filldraw[fill opacity=0.8,fill=gray!20,draw=none](-4.618,2.943)--(-4.618,2.944)--(-4.617,2.944)--cycle;
\draw(-4.618,2.943)--(-4.618,2.944);
\filldraw[fill opacity=0.8,fill=gray!20,draw=none](-4.619,2.934)--(-4.621,2.947)--(-4.593,2.957)--cycle;
\draw(-4.621,2.947)--(-4.593,2.957);
\filldraw[fill opacity=0.8,fill=gray!20,draw=none](-4.562,2.951)--(-4.613,2.919)--(-4.614,2.932)--cycle;
\draw(-4.562,2.951)--(-4.613,2.919);
\filldraw[fill opacity=0.8,fill=gray!20,draw=none](-4.595,2.95)--(-4.606,2.968)--(-4.602,2.968)--(-4.583,2.959)--cycle;
\draw(-4.602,2.968)--(-4.583,2.959);
\filldraw[fill opacity=0.8,fill=gray!20,draw=none](-4.61,2.968)--(-4.605,2.97)--(-4.602,2.968)--cycle;
\draw(-4.605,2.97)--(-4.602,2.968);
\filldraw[fill opacity=0.8,fill=gray!20,draw=none](-4.567,2.974)--(-4.614,2.966)--(-4.541,2.992)--cycle;
\draw(-4.614,2.966)--(-4.541,2.992);
\filldraw[fill opacity=0.8,fill=gray!20,draw=none](-4.608,2.915)--(-4.629,2.926)--(-4.583,2.959)--(-4.581,2.958)--cycle;
\draw(-4.608,2.915)--(-4.629,2.926);
\draw(-4.583,2.959)--(-4.581,2.958);
\filldraw[fill opacity=0.8,fill=gray!20,draw=none](-4.529,2.98)--(-4.534,2.968)--(-4.562,2.951)--(-4.614,2.932)--(-4.614,2.935)--(-4.542,2.98)--cycle;
\draw(-4.534,2.968)--(-4.562,2.951);
\draw(-4.614,2.935)--(-4.542,2.98);
\filldraw[fill opacity=0.8,fill=gray!20,draw=none](-4.606,2.89)--(-4.654,2.9)--(-4.654,2.906)--(-4.629,2.926)--(-4.594,2.909)--cycle;
\draw(-4.629,2.926)--(-4.594,2.909);
\filldraw[fill opacity=0.8,fill=gray!20,draw=none](-4.614,2.932)--(-4.626,2.928)--(-4.614,2.935)--cycle;
\draw(-4.626,2.928)--(-4.614,2.935);
\filldraw[fill opacity=0.8,fill=gray!20,draw=none](-4.554,2.971)--(-4.565,2.966)--(-4.663,2.905)--(-4.651,2.906)--(-4.545,2.971)--cycle;
\draw(-4.565,2.966)--(-4.663,2.905);
\draw(-4.651,2.906)--(-4.545,2.971);
\filldraw[fill opacity=0.8,fill=gray!20,draw=none](-4.636,2.955)--(-4.639,2.922)--(-4.655,2.905)--(-4.654,2.916)--cycle;
\draw(-4.655,2.905)--(-4.654,2.916);
\filldraw[fill opacity=0.8,fill=gray!20,draw=none](-4.595,2.95)--(-4.629,2.926)--(-4.645,2.934)--(-4.619,2.964)--(-4.61,2.968)--(-4.606,2.968)--cycle;
\draw(-4.629,2.926)--(-4.645,2.934);
\filldraw[fill opacity=0.8,fill=gray!20,draw=none](-4.524,2.991)--(-4.521,2.997)--(-4.511,3.005)--(-4.513,3.002)--cycle;
\draw(-4.511,3.005)--(-4.513,3.002);
\filldraw[fill opacity=0.8,fill=gray!20,draw=none](-4.654,2.906)--(-4.654,2.939)--(-4.629,2.926)--cycle;
\draw(-4.654,2.939)--(-4.629,2.926);
\filldraw[fill opacity=0.8,fill=gray!20,draw=none](-4.636,2.955)--(-4.646,2.934)--(-4.641,2.954)--(-4.636,2.957)--cycle;
\draw(-4.641,2.954)--(-4.636,2.957);
\filldraw[fill opacity=0.8,fill=gray!20,draw=none](-4.536,2.964)--(-4.55,2.943)--(-4.565,2.95)--cycle;
\draw(-4.55,2.943)--(-4.565,2.95);
\filldraw[fill opacity=0.8,fill=gray!20,draw=none](-4.542,2.955)--(-4.56,2.953)--(-4.534,2.968)--cycle;
\draw(-4.56,2.953)--(-4.534,2.968);
\filldraw[fill opacity=0.8,fill=gray!20,draw=none](-4.529,2.98)--(-4.542,2.98)--(-4.524,2.991)--cycle;
\draw(-4.542,2.98)--(-4.524,2.991);
\filldraw[fill opacity=0.8,fill=gray!20,draw=none](-4.645,2.934)--(-4.646,2.934)--(-4.641,2.955)--(-4.619,2.964)--cycle;
\draw(-4.645,2.934)--(-4.646,2.934);
\filldraw[fill opacity=0.8,fill=gray!20,draw=none](-4.517,2.998)--(-4.516,3.001)--(-4.502,3.006)--cycle;
\draw(-4.516,3.001)--(-4.502,3.006);
\filldraw[fill opacity=0.8,fill=gray!20,draw=none](-4.641,2.955)--(-4.651,2.937)--(-4.666,2.944)--cycle;
\draw(-4.651,2.937)--(-4.666,2.944);
\filldraw[fill opacity=0.8,fill=gray!20,draw=none](-4.641,2.954)--(-4.65,2.948)--(-4.649,2.95)--(-4.64,2.955)--cycle;
\draw(-4.641,2.954)--(-4.65,2.948);
\filldraw[fill opacity=0.8,fill=gray!20,draw=none](-4.641,2.955)--(-4.646,2.934)--(-4.651,2.937)--cycle;
\draw(-4.646,2.934)--(-4.651,2.937);
\filldraw[fill opacity=0.8,fill=gray!20,draw=none](-4.646,2.934)--(-4.654,2.916)--(-4.65,2.948)--(-4.641,2.954)--cycle;
\draw(-4.654,2.916)--(-4.65,2.948)--(-4.641,2.954);
\filldraw[fill opacity=0.8,fill=gray!20,draw=none](-4.533,2.982)--(-4.527,2.988)--(-4.527,2.989)--(-4.556,2.971)--(-4.554,2.971)--cycle;
\draw(-4.527,2.989)--(-4.556,2.971);
\filldraw[fill opacity=0.8,fill=gray!20,draw=none](-4.533,2.982)--(-4.532,2.984)--(-4.524,2.996)--(-4.521,2.997)--(-4.524,2.991)--cycle;
\draw(-4.532,2.984)--(-4.524,2.996);
\filldraw[fill opacity=0.8,fill=gray!20,draw=none](-4.519,2.989)--(-4.509,2.994)--(-4.505,3.003)--(-4.518,2.995)--cycle;
\draw(-4.509,2.994)--(-4.505,3.003)--(-4.518,2.995);
\filldraw[fill opacity=0.8,fill=gray!20,draw=none](-4.543,2.96)--(-4.529,2.98)--(-4.519,2.981)--(-4.515,2.959)--(-4.527,2.942)--cycle;
\draw(-4.515,2.959)--(-4.527,2.942)--(-4.543,2.96)--(-4.529,2.98);
\filldraw[fill opacity=0.8,fill=gray!20,draw=none](-4.543,2.964)--(-4.538,2.973)--(-4.525,2.989)--(-4.529,2.98)--(-4.543,2.96)--cycle;
\draw(-4.529,2.98)--(-4.543,2.96)--(-4.543,2.964);
\filldraw[fill opacity=0.8,fill=gray!20,draw=none](-4.538,2.973)--(-4.527,2.988)--(-4.524,2.991)--(-4.525,2.989)--cycle;
\filldraw[fill opacity=0.8,fill=gray!20,draw=none](-4.527,2.988)--(-4.524,2.991)--(-4.527,2.989)--cycle;
\draw(-4.524,2.991)--(-4.527,2.989);
\filldraw[fill opacity=0.8,fill=gray!20,draw=none](-4.527,2.988)--(-4.526,2.986)--(-4.519,2.989)--(-4.518,2.995)--(-4.524,2.991)--cycle;
\draw(-4.518,2.995)--(-4.524,2.991);
\filldraw[fill opacity=0.8,fill=gray!20,draw=none](-4.511,2.966)--(-4.511,2.964)--(-4.515,2.959)--cycle;
\draw(-4.511,2.964)--(-4.515,2.959);
\filldraw[fill opacity=0.8,fill=gray!20,draw=none](-4.438,2.891)--(-4.438,2.89)--(-4.545,2.965)--(-4.543,2.969)--(-4.519,2.989)--(-4.509,2.994)--cycle;
\draw(-4.545,2.965)--(-4.543,2.969)--(-4.519,2.989);
\filldraw[fill opacity=0.8,fill=gray!20,draw=none](-4.581,2.902)--(-4.605,2.914)--(-4.606,2.917)--(-4.592,2.94)--(-4.565,2.95)--(-4.55,2.943)--cycle;
\draw(-4.581,2.902)--(-4.605,2.914);
\draw(-4.565,2.95)--(-4.55,2.943);
\filldraw[fill opacity=0.8,fill=gray!20,draw=none](-4.538,2.973)--(-4.533,2.982)--(-4.527,2.988)--cycle;
\filldraw[fill opacity=0.8,fill=gray!20,draw=none](-4.533,2.982)--(-4.526,2.986)--(-4.527,2.988)--cycle;
\filldraw[fill opacity=0.8,fill=gray!20,draw=none](-4.538,2.974)--(-4.543,2.964)--(-4.544,2.967)--cycle;
\draw(-4.543,2.964)--(-4.544,2.967);
\filldraw[fill opacity=0.8,fill=gray!20,draw=none](-4.544,2.967)--(-4.548,2.96)--(-4.548,2.964)--(-4.543,2.969)--cycle;
\draw(-4.548,2.964)--(-4.543,2.969)--(-4.544,2.967);
\filldraw[fill opacity=0.8,fill=gray!20,draw=none](-4.536,2.975)--(-4.543,2.969)--(-4.545,2.966)--(-4.544,2.967)--(-4.533,2.975)--cycle;
\draw(-4.536,2.975)--(-4.543,2.969)--(-4.545,2.966);
\filldraw[fill opacity=0.8,fill=gray!20,draw=none](-4.538,2.974)--(-4.544,2.967)--(-4.542,2.97)--(-4.532,2.984)--cycle;
\draw(-4.542,2.97)--(-4.532,2.984);
\filldraw[fill opacity=0.8,fill=gray!20,draw=none](-4.519,2.989)--(-4.515,2.993)--(-4.509,2.994)--cycle;
\draw(-4.519,2.989)--(-4.515,2.993)--(-4.509,2.994);
\filldraw[fill opacity=0.8,fill=gray!20,draw=none](-4.519,2.989)--(-4.536,2.975)--(-4.533,2.975)--(-4.52,2.987)--cycle;
\draw(-4.519,2.989)--(-4.536,2.975);
\filldraw[fill opacity=0.8,fill=gray!20,draw=none](-4.519,2.989)--(-4.533,2.982)--(-4.534,2.981)--(-4.52,2.987)--cycle;
\filldraw[fill opacity=0.8,fill=gray!20,draw=none](-4.526,2.985)--(-4.521,2.991)--(-4.522,2.996)--(-4.524,2.996)--(-4.534,2.981)--cycle;
\draw(-4.524,2.996)--(-4.534,2.981);
\filldraw[fill opacity=0.8,fill=gray!20,draw=none](-4.521,2.991)--(-4.516,2.997)--(-4.522,2.996)--cycle;
\filldraw[fill opacity=0.8,fill=gray!20,draw=none](-4.479,3.014)--(-7.503,1.94)--(-7.522,1.929)--(-4.487,3.007)--cycle;
\draw(-7.522,1.929)--(-4.487,3.007)--(-4.479,3.014)--(-7.503,1.94);
\filldraw[fill opacity=0.8,fill=gray!20,draw=none](-7.517,1.674)--(-7.558,1.659)--(-7.555,1.644)--(-7.493,1.666)--cycle;
\draw(-7.517,1.674)--(-7.558,1.659);
\draw(-7.555,1.644)--(-7.493,1.666);
\filldraw[fill opacity=0.8,fill=gray!20,draw=none](-7.771,1.802)--(-7.77,1.808)--(-7.773,1.81)--cycle;
\draw(-7.77,1.808)--(-7.773,1.81);
\filldraw[fill opacity=0.8,fill=gray!20,draw=none](-7.771,1.802)--(-7.761,1.924)--(-7.773,1.933)--(-7.773,1.81)--cycle;
\draw(-7.761,1.924)--(-7.773,1.933)--(-7.773,1.81);
\filldraw[fill opacity=0.8,fill=gray!20,draw=none](-7.516,1.674)--(-7.516,1.604)--(-7.488,1.646)--(-7.488,1.664)--cycle;
\draw(-7.516,1.674)--(-7.516,1.604);
\draw(-7.488,1.646)--(-7.488,1.664);
\filldraw[fill opacity=0.8,fill=gray!20,draw=none](-3.23,2.041)--(-3.235,2.042)--(-3.232,2.018)--cycle;
\draw(-3.235,2.042)--(-3.232,2.018);
\filldraw[fill opacity=0.8,fill=gray!20,draw=none](-7.723,1.611)--(-7.732,1.617)--(-7.724,1.62)--(-7.668,1.611)--(-7.691,1.601)--cycle;
\draw(-7.732,1.617)--(-7.724,1.62);
\draw(-7.668,1.611)--(-7.691,1.601);
\filldraw[fill opacity=0.8,fill=gray!20,draw=none](-7.735,1.621)--(-7.724,1.62)--(-7.732,1.617)--cycle;
\draw(-7.724,1.62)--(-7.732,1.617);
\filldraw[fill opacity=0.8,fill=gray!20,draw=none](-7.758,1.704)--(-7.758,1.564)--(-7.723,1.554)--(-7.723,1.678)--cycle;
\draw(-7.758,1.704)--(-7.758,1.564)--(-7.723,1.554)--(-7.723,1.678);
\filldraw[fill opacity=0.8,fill=gray!20,draw=none](-3.397,2.306)--(-3.383,2.306)--(-3.371,2.305)--cycle;
\draw(-3.383,2.306)--(-3.371,2.305);
\filldraw[fill opacity=0.8,fill=gray!20,draw=none](-7.964,1.62)--(-7.797,1.693)--(-7.764,1.672)--(-7.76,1.654)--(-7.828,1.624)--cycle;
\draw(-7.964,1.62)--(-7.797,1.693);
\draw(-7.76,1.654)--(-7.828,1.624);
\filldraw[fill opacity=0.8,fill=gray!20](-3.049,3.007)--(-3.018,3.055)--(-2.998,3.034)--(-3.033,2.99)--cycle;
\filldraw[fill opacity=0.8,fill=gray!20](-3.018,3.055)--(-2.998,3.109)--(-2.976,3.085)--(-2.998,3.034)--cycle;
\filldraw[fill opacity=0.8,fill=gray!20,draw=none](-3.245,2.062)--(-3.238,2.067)--(-3.24,2.077)--cycle;
\draw(-3.238,2.067)--(-3.24,2.077);
\filldraw[fill opacity=0.8,fill=gray!20,draw=none](-7.771,1.806)--(-7.771,1.803)--(-7.77,1.805)--cycle;
\filldraw[fill opacity=0.8,fill=gray!20,draw=none](-7.761,1.924)--(-7.771,1.806)--(-7.77,1.805)--(-7.758,1.842)--(-7.758,1.921)--cycle;
\draw(-7.758,1.842)--(-7.758,1.921)--(-7.761,1.924);
\filldraw[fill opacity=0.8,fill=gray!20,draw=none](-7.754,1.834)--(-7.771,1.803)--(-7.797,1.791)--cycle;
\draw(-7.771,1.803)--(-7.797,1.791);
\filldraw[fill opacity=0.8,fill=gray!20,draw=none](-7.77,1.805)--(-7.771,1.803)--(-7.758,1.743)--(-7.758,1.781)--cycle;
\draw(-7.758,1.743)--(-7.758,1.781);
\filldraw[fill opacity=0.8,fill=gray!20,draw=none](-3.338,2.989)--(-3.336,2.987)--(-3.351,3.006)--(-3.364,3.021)--(-3.343,2.994)--cycle;
\draw(-3.364,3.021)--(-3.343,2.994)--(-3.338,2.989);
\filldraw[fill opacity=0.8,fill=gray!20,draw=none](-3.351,3.006)--(-3.377,3.038)--(-3.378,3.039)--(-3.364,3.021)--cycle;
\draw(-3.377,3.038)--(-3.378,3.039)--(-3.364,3.021);
\filldraw[fill opacity=0.8,fill=gray!20,draw=none](-3.378,3.04)--(-3.377,3.038)--(-3.377,3.038)--cycle;
\draw(-3.377,3.038)--(-3.377,3.038);
\filldraw[fill opacity=0.8,fill=gray!20,draw=none](-3.38,3.04)--(-3.223,3.072)--(-3.201,3.036)--(-3.383,2.999)--cycle;
\draw(-3.38,3.04)--(-3.223,3.072)--(-3.201,3.036)--(-3.383,2.999);
\filldraw[fill opacity=0.8,fill=gray!20](-2.998,3.109)--(-2.992,3.165)--(-2.969,3.141)--(-2.976,3.085)--cycle;
\filldraw[fill opacity=0.8,fill=gray!20,draw=none](-7.488,1.668)--(-7.489,1.667)--(-7.493,1.666)--cycle;
\draw(-7.489,1.667)--(-7.493,1.666);
\filldraw[fill opacity=0.8,fill=gray!20,draw=none](-3.23,2.041)--(-3.227,2.072)--(-3.238,2.067)--(-3.235,2.042)--cycle;
\draw(-3.238,2.067)--(-3.235,2.042);
\filldraw[fill opacity=0.8,fill=gray!20,draw=none](-7.768,1.808)--(-7.754,1.834)--(-7.745,1.842)--(-7.741,1.844)--cycle;
\draw(-7.745,1.842)--(-7.741,1.844);
\filldraw[fill opacity=0.8,fill=gray!20,draw=none](-7.643,1.802)--(-7.64,1.809)--(-7.641,1.814)--(-7.728,1.852)--(-7.741,1.844)--(-7.768,1.808)--(-7.723,1.788)--cycle;
\draw(-7.643,1.802)--(-7.64,1.809);
\draw(-7.641,1.814)--(-7.728,1.852);
\draw(-7.768,1.808)--(-7.723,1.788);
\filldraw[fill opacity=0.8,fill=gray!20,draw=none](-7.728,1.852)--(-7.733,1.854)--(-7.741,1.844)--cycle;
\draw(-7.728,1.852)--(-7.733,1.854);
\filldraw[fill opacity=0.8,fill=gray!20,draw=none](-7.758,1.837)--(-7.732,1.854)--(-7.698,1.87)--(-7.745,1.842)--(-7.76,1.835)--cycle;
\draw(-7.732,1.854)--(-7.698,1.87);
\draw(-7.745,1.842)--(-7.76,1.835);
\filldraw[fill opacity=0.8,fill=gray!20,draw=none](-8.066,1.678)--(-8.038,1.714)--(-7.745,1.842)--(-7.797,1.791)--(-8.07,1.672)--cycle;
\draw(-8.038,1.714)--(-7.745,1.842);
\draw(-7.797,1.791)--(-8.07,1.672);
\filldraw[fill opacity=0.8,fill=gray!20,draw=none](-7.77,1.805)--(-7.758,1.781)--(-7.758,1.842)--cycle;
\draw(-7.758,1.781)--(-7.758,1.842);
\filldraw[fill opacity=0.8,fill=gray!20](-3.09,2.969)--(-3.049,3.007)--(-3.033,2.99)--(-3.079,2.957)--cycle;
\filldraw[fill opacity=0.8,fill=gray!20](-3.188,2.929)--(-3.214,2.948)--(-3.185,2.949)--(-3.188,2.929)--cycle;
\filldraw[fill opacity=0.8,fill=gray!20](-3.188,2.929)--(-3.185,2.949)--(-3.157,2.947)--(-3.188,2.929)--cycle;
\filldraw[fill opacity=0.8,fill=gray!20](-3.4,3.202)--(-3.378,3.253)--(-3.347,3.274)--(-3.365,3.225)--cycle;
\filldraw[fill opacity=0.8,fill=gray!20](-3.318,3.314)--(-3.28,3.342)--(-3.237,3.351)--(-3.258,3.325)--cycle;
\filldraw[fill opacity=0.8,fill=gray!20](-3.245,2.937)--(-3.298,2.959)--(-3.28,2.971)--(-3.236,2.943)--cycle;
\filldraw[fill opacity=0.8,fill=gray!20,draw=none](-3.295,2.064)--(-3.302,2.044)--(-3.273,2.044)--(-3.277,2.079)--cycle;
\draw(-3.273,2.044)--(-3.277,2.079);
\filldraw[fill opacity=0.8,fill=gray!20,draw=none](-3.323,2.312)--(-3.299,2.283)--(-3.263,2.261)--(-3.268,2.299)--cycle;
\draw(-3.263,2.261)--(-3.268,2.299)--(-3.323,2.312);
\filldraw[fill opacity=0.8,fill=gray!20,draw=none](-7.507,1.749)--(-7.489,1.72)--(-7.436,1.703)--(-7.419,1.705)--(-7.414,1.736)--(-7.424,1.789)--(-7.443,1.819)--cycle;
\draw(-7.419,1.705)--(-7.414,1.736)--(-7.424,1.789)--(-7.443,1.819);
\filldraw[fill opacity=0.8,fill=gray!20,draw=none](-7.44,1.687)--(-7.411,1.712)--(-7.517,1.674)--(-7.493,1.666)--(-7.452,1.68)--cycle;
\draw(-7.411,1.712)--(-7.517,1.674);
\draw(-7.493,1.666)--(-7.452,1.68);
\filldraw[fill opacity=0.8,fill=gray!20,draw=none](-7.758,1.648)--(-7.761,1.653)--(-7.76,1.654)--cycle;
\draw(-7.761,1.653)--(-7.76,1.654);
\filldraw[fill opacity=0.8,fill=gray!20,draw=none](-3.273,2.044)--(-3.256,1.912)--(-3.134,1.817)--(-3.162,2.041)--cycle;
\draw(-3.273,2.044)--(-3.256,1.912);
\draw(-3.134,1.817)--(-3.162,2.041);
\filldraw[fill opacity=0.8,fill=gray!20](-3.188,2.929)--(-3.236,2.943)--(-3.214,2.948)--(-3.188,2.929)--cycle;
\filldraw[fill opacity=0.8,fill=gray!20](-2.992,3.165)--(-2.998,3.22)--(-2.976,3.196)--(-2.969,3.141)--cycle;
\filldraw[fill opacity=0.8,fill=gray!20,draw=none](-3.225,2.098)--(-3.24,2.077)--(-3.238,2.067)--(-3.227,2.072)--cycle;
\draw(-3.24,2.077)--(-3.238,2.067);
\filldraw[fill opacity=0.8,fill=gray!20,draw=none](-7.668,1.603)--(-7.673,1.602)--(-7.673,1.564)--(-7.616,1.572)--(-7.616,1.593)--cycle;
\draw(-7.673,1.602)--(-7.673,1.564);
\draw(-7.616,1.572)--(-7.616,1.593);
\filldraw[fill opacity=0.8,fill=gray!20,draw=none](-7.641,1.607)--(-7.668,1.603)--(-7.653,1.6)--cycle;
\filldraw[fill opacity=0.8,fill=gray!20,draw=none](-7.668,1.611)--(-7.661,1.614)--(-7.641,1.607)--(-7.668,1.596)--cycle;
\draw(-7.668,1.611)--(-7.661,1.614);
\draw(-7.641,1.607)--(-7.668,1.596);
\filldraw[fill opacity=0.8,fill=gray!20,draw=none](-3.389,3.221)--(-3.394,3.216)--(-3.399,3.204)--cycle;
\draw(-3.394,3.216)--(-3.399,3.204);
\filldraw[fill opacity=0.8,fill=gray!20,draw=none](-3.389,3.221)--(-3.377,3.241)--(-3.383,3.242)--(-3.394,3.216)--cycle;
\draw(-3.383,3.242)--(-3.394,3.216);
\filldraw[fill opacity=0.8,fill=gray!20,draw=none](-3.399,3.202)--(-3.395,3.219)--(-3.237,3.252)--(-3.246,3.216)--(-3.396,3.186)--cycle;
\draw(-3.395,3.219)--(-3.237,3.252)--(-3.246,3.216)--(-3.396,3.186);
\filldraw[fill opacity=0.8,fill=gray!20,draw=none](-7.758,1.648)--(-7.746,1.621)--(-7.828,1.624)--(-7.761,1.653)--cycle;
\draw(-7.828,1.624)--(-7.761,1.653);
\filldraw[fill opacity=0.8,fill=gray!20,draw=none](-7.728,1.852)--(-7.741,1.844)--(-7.745,1.842)--cycle;
\draw(-7.741,1.844)--(-7.745,1.842);
\filldraw[fill opacity=0.8,fill=gray!20](-3.188,2.929)--(-3.157,2.947)--(-3.137,2.942)--(-3.188,2.929)--cycle;
\filldraw[fill opacity=0.8,fill=gray!20](-3.103,3.323)--(-3.128,3.349)--(-3.09,3.34)--(-3.049,3.31)--cycle;
\filldraw[fill opacity=0.8,fill=gray!20,draw=none](-3.384,2.306)--(-3.391,2.306)--(-3.383,2.306)--cycle;
\draw(-3.391,2.306)--(-3.383,2.306);
\filldraw[fill opacity=0.8,fill=gray!20,draw=none](-7.735,1.621)--(-7.746,1.621)--(-7.758,1.648)--cycle;
\filldraw[fill opacity=0.8,fill=gray!20](-3.137,2.942)--(-3.09,2.969)--(-3.079,2.957)--(-3.131,2.936)--cycle;
\filldraw[fill opacity=0.8,fill=gray!20,draw=none](-3.386,2.329)--(-3.383,2.306)--(-3.35,2.285)--cycle;
\draw(-3.386,2.329)--(-3.383,2.306);
\filldraw[fill opacity=0.8,fill=gray!20,draw=none](-3.212,2.117)--(-3.225,2.098)--(-3.227,2.072)--(-3.201,2.084)--cycle;
\filldraw[fill opacity=0.8,fill=gray!20,draw=none](-7.645,.905)--(-7.651,.932)--(-7.623,.896)--(-7.628,.895)--cycle;
\draw(-7.623,.896)--(-7.628,.895);
\filldraw[fill opacity=0.8,fill=gray!20,draw=none](-7.625,.857)--(-7.635,.87)--(-7.642,.89)--(-7.628,.895)--(-7.61,.884)--(-7.588,.863)--(-7.61,.856)--cycle;
\draw(-7.642,.89)--(-7.628,.895);
\draw(-7.588,.863)--(-7.61,.856);
\filldraw[fill opacity=0.8,fill=gray!20,draw=none](-7.645,.905)--(-7.628,.895)--(-7.642,.89)--cycle;
\draw(-7.628,.895)--(-7.642,.89);
\filldraw[fill opacity=0.8,fill=gray!20,draw=none](-7.635,.975)--(-7.65,.932)--(-7.6,.91)--cycle;
\draw(-7.65,.932)--(-7.6,.91);
\filldraw[fill opacity=0.8,fill=gray!20,draw=none](-7.65,.932)--(-7.65,.808)--(-7.635,.796)--(-7.635,.937)--cycle;
\draw(-7.65,.932)--(-7.65,.808)--(-7.635,.796)--(-7.635,.937);
\filldraw[fill opacity=0.8,fill=gray!20,draw=none](-7.426,1.181)--(-7.078,1.391)--(-7.413,1.183)--cycle;
\draw(-7.078,1.391)--(-7.413,1.183);
\filldraw[fill opacity=0.8,fill=gray!20,draw=none](-3.395,2.405)--(-3.386,2.329)--(-3.35,2.285)--(-3.3,2.255)--(-3.34,2.567)--cycle;
\draw(-3.395,2.405)--(-3.386,2.329);
\draw(-3.3,2.255)--(-3.34,2.567);
\filldraw[fill opacity=0.8,fill=gray!20,draw=none](-7.691,1.601)--(-7.723,1.597)--(-7.723,1.554)--(-7.673,1.548)--(-7.673,1.594)--cycle;
\draw(-7.723,1.597)--(-7.723,1.554)--(-7.673,1.548)--(-7.673,1.594);
\filldraw[fill opacity=0.8,fill=gray!20,draw=none](-7.691,1.601)--(-7.673,1.594)--(-7.673,1.602)--cycle;
\draw(-7.673,1.594)--(-7.673,1.602);
\filldraw[fill opacity=0.8,fill=gray!20,draw=none](-7.723,1.587)--(-7.668,1.611)--(-7.668,1.596)--(-7.703,1.58)--cycle;
\draw(-7.723,1.587)--(-7.668,1.611);
\draw(-7.668,1.596)--(-7.703,1.58);
\filldraw[fill opacity=0.8,fill=gray!20](-2.998,3.22)--(-3.018,3.269)--(-2.998,3.248)--(-2.976,3.196)--cycle;
\filldraw[fill opacity=0.8,fill=gray!20,draw=none](-3.283,2.1)--(-3.295,2.064)--(-3.283,2.074)--cycle;
\filldraw[fill opacity=0.8,fill=gray!20,draw=none](-7.521,1.557)--(-7.516,1.558)--(-7.516,1.566)--cycle;
\draw(-7.521,1.557)--(-7.516,1.558)--(-7.516,1.566);
\filldraw[fill opacity=0.8,fill=gray!20](-3.378,3.253)--(-3.343,3.297)--(-3.318,3.314)--(-3.347,3.274)--cycle;
\filldraw[fill opacity=0.8,fill=gray!20,draw=none](-3.281,2.272)--(-3.299,2.283)--(-3.268,2.246)--cycle;
\filldraw[fill opacity=0.8,fill=gray!20,draw=none](-7.732,1.854)--(-7.758,1.837)--(-7.758,1.743)--(-7.723,1.788)--(-7.723,1.855)--cycle;
\draw(-7.758,1.837)--(-7.758,1.743);
\draw(-7.723,1.788)--(-7.723,1.855);
\filldraw[fill opacity=0.8,fill=gray!20,draw=none](-7.732,1.854)--(-7.751,1.854)--(-7.758,1.842)--(-7.758,1.837)--cycle;
\draw(-7.758,1.842)--(-7.758,1.837);
\filldraw[fill opacity=0.8,fill=gray!20,draw=none](-7.758,1.837)--(-7.743,1.85)--(-7.732,1.854)--cycle;
\draw(-7.743,1.85)--(-7.732,1.854);
\filldraw[fill opacity=0.8,fill=gray!20,draw=none](-7.691,1.858)--(-7.723,1.855)--(-7.723,1.813)--(-7.673,1.828)--(-7.673,1.851)--cycle;
\draw(-7.723,1.855)--(-7.723,1.813);
\draw(-7.673,1.828)--(-7.673,1.851);
\filldraw[fill opacity=0.8,fill=gray!20,draw=none](-7.723,1.844)--(-7.661,1.871)--(-7.698,1.87)--(-7.743,1.85)--cycle;
\draw(-7.723,1.844)--(-7.661,1.871);
\draw(-7.698,1.87)--(-7.743,1.85);
\filldraw[fill opacity=0.8,fill=gray!20](-3.188,2.929)--(-3.245,2.937)--(-3.236,2.943)--(-3.188,2.929)--cycle;
\filldraw[fill opacity=0.8,fill=gray!20,draw=none](-3.212,2.117)--(-3.221,2.145)--(-3.225,2.098)--cycle;
\filldraw[fill opacity=0.8,fill=gray!20,draw=none](-3.294,2.956)--(-3.312,2.966)--(-3.327,2.977)--(-3.338,2.989)--(-3.339,2.991)--(-3.298,2.959)--cycle;
\draw(-3.312,2.966)--(-3.327,2.977)--(-3.338,2.989);
\draw(-3.339,2.991)--(-3.298,2.959)--(-3.294,2.956);
\filldraw[fill opacity=0.8,fill=gray!20,draw=none](-3.338,2.989)--(-3.343,2.994)--(-3.339,2.991)--cycle;
\draw(-3.338,2.989)--(-3.343,2.994)--(-3.339,2.991);
\filldraw[fill opacity=0.8,fill=gray!20,draw=none](-3.338,2.989)--(-3.331,2.981)--(-3.336,2.987)--cycle;
\draw(-3.338,2.989)--(-3.331,2.981);
\filldraw[fill opacity=0.8,fill=gray!20,draw=none](-3.351,3.006)--(-3.336,2.987)--(-3.333,2.985)--cycle;
\filldraw[fill opacity=0.8,fill=gray!20,draw=none](-3.336,2.987)--(-3.331,2.981)--(-3.327,2.977)--(-3.333,2.985)--cycle;
\draw(-3.331,2.981)--(-3.327,2.977)--(-3.333,2.985);
\filldraw[fill opacity=0.8,fill=gray!20,draw=none](-3.372,3.002)--(-3.201,3.036)--(-3.178,3.017)--(-3.382,2.975)--cycle;
\draw(-3.372,3.002)--(-3.201,3.036)--(-3.178,3.017)--(-3.382,2.975);
\filldraw[fill opacity=0.8,fill=gray!20,draw=none](-7.65,1.045)--(-7.649,1.074)--(-7.643,1.1)--(-7.65,1.045)--cycle;
\filldraw[fill opacity=0.8,fill=gray!20,draw=none](-7.635,.975)--(-7.678,.994)--(-7.651,.933)--(-7.65,.932)--cycle;
\draw(-7.635,.975)--(-7.678,.994);
\draw(-7.651,.933)--(-7.65,.932);
\filldraw[fill opacity=0.8,fill=gray!20,draw=none](-3.587,2.039)--(-3.601,2.045)--(-3.585,2.054)--(-3.531,2.053)--(-3.524,2.05)--cycle;
\draw(-3.587,2.039)--(-3.601,2.045);
\draw(-3.531,2.053)--(-3.524,2.05);
\filldraw[fill opacity=0.8,fill=gray!20](-3.188,2.929)--(-3.137,2.942)--(-3.131,2.936)--(-3.188,2.929)--cycle;
\filldraw[fill opacity=0.8,fill=gray!20,draw=none](-7.359,.931)--(-7.352,.934)--(-7.327,.944)--(-7.364,.921)--cycle;
\draw(-7.327,.944)--(-7.364,.921);
\filldraw[fill opacity=0.8,fill=gray!20,draw=none](-7.676,1.707)--(-7.758,1.743)--(-7.723,1.678)--(-7.666,1.654)--cycle;
\draw(-7.723,1.678)--(-7.666,1.654)--(-7.676,1.707)--(-7.758,1.743);
\filldraw[fill opacity=0.8,fill=gray!20,draw=none](-7.52,1.675)--(-7.528,1.67)--(-7.518,1.674)--cycle;
\draw(-7.528,1.67)--(-7.518,1.674);
\filldraw[fill opacity=0.8,fill=gray!20,draw=none](-7.523,1.674)--(-7.561,1.653)--(-7.561,1.586)--(-7.516,1.604)--(-7.516,1.674)--cycle;
\draw(-7.561,1.653)--(-7.561,1.586);
\draw(-7.516,1.604)--(-7.516,1.674);
\filldraw[fill opacity=0.8,fill=gray!20,draw=none](-5.563,2.355)--(-5.835,2.255)--(-6.862,1.89)--(-6.084,2.171)--(-5.565,2.355)--cycle;
\draw(-5.835,2.255)--(-6.862,1.89);
\draw(-6.084,2.171)--(-5.565,2.355);
\filldraw[fill opacity=0.8,fill=gray!20,draw=none](-5.565,2.355)--(-6.084,2.171)--(-5.509,2.394)--cycle;
\draw(-5.565,2.355)--(-6.084,2.171);
\filldraw[fill opacity=0.8,fill=gray!20,draw=none](-6.081,2.184)--(-7.383,1.722)--(-7.409,1.706)--(-7.364,1.712)--(-5.835,2.255)--cycle;
\draw(-6.081,2.184)--(-7.383,1.722);
\draw(-7.364,1.712)--(-5.835,2.255);
\filldraw[fill opacity=0.8,fill=gray!20,draw=none](-7.746,1.621)--(-7.735,1.621)--(-7.732,1.617)--(-7.743,1.612)--cycle;
\draw(-7.732,1.617)--(-7.743,1.612);
\filldraw[fill opacity=0.8,fill=gray!20,draw=none](-3.196,2.14)--(-3.212,2.117)--(-3.201,2.084)--(-3.166,2.1)--cycle;
\filldraw[fill opacity=0.8,fill=gray!20](-3.182,3.353)--(-3.185,3.363)--(-3.157,3.361)--(-3.128,3.349)--cycle;
\filldraw[fill opacity=0.8,fill=gray!20](-3.237,3.351)--(-3.214,3.362)--(-3.185,3.363)--(-3.182,3.353)--cycle;
\filldraw[fill opacity=0.8,fill=gray!20,draw=none](-2.924,2.163)--(-2.924,2.161)--(-2.817,2.145)--(-2.82,2.168)--cycle;
\draw(-2.924,2.163)--(-2.924,2.161);
\draw(-2.817,2.145)--(-2.82,2.168);
\filldraw[fill opacity=0.8,fill=gray!20,draw=none](-7.61,.884)--(-7.628,.895)--(-7.623,.896)--cycle;
\draw(-7.628,.895)--(-7.623,.896);
\filldraw[fill opacity=0.8,fill=gray!20,draw=none](-7.723,1.611)--(-7.723,1.597)--(-7.691,1.601)--cycle;
\draw(-7.723,1.611)--(-7.723,1.597);
\filldraw[fill opacity=0.8,fill=gray!20,draw=none](-7.743,1.612)--(-7.732,1.617)--(-7.698,1.597)--(-7.723,1.587)--cycle;
\draw(-7.743,1.612)--(-7.732,1.617);
\draw(-7.698,1.597)--(-7.723,1.587);
\filldraw[fill opacity=0.8,fill=gray!20,draw=none](-6.606,1.668)--(-5.939,2.06)--(-6.955,1.43)--cycle;
\draw(-5.939,2.06)--(-6.955,1.43);
\filldraw[fill opacity=0.8,fill=gray!20,draw=none](-4.438,2.89)--(-4.543,2.966)--(-4.533,2.975)--(-4.524,2.98)--(-4.507,2.983)--cycle;
\draw(-4.543,2.966)--(-4.533,2.975);
\draw(-4.524,2.98)--(-4.507,2.983);
\filldraw[fill opacity=0.8,fill=gray!20,draw=none](-4.438,2.89)--(-4.493,2.944)--(-4.506,2.952)--(-4.528,2.963)--(-4.54,2.966)--(-4.543,2.966)--(-4.543,2.96)--(-4.527,2.942)--(-4.497,2.916)--cycle;
\draw(-4.506,2.952)--(-4.528,2.963)--(-4.54,2.966);
\draw(-4.543,2.966)--(-4.543,2.96)--(-4.527,2.942)--(-4.497,2.916);
\filldraw[fill opacity=0.8,fill=gray!20,draw=none](-4.535,2.902)--(-4.545,2.94)--(-4.497,2.916)--cycle;
\draw(-4.545,2.94)--(-4.497,2.916);
\filldraw[fill opacity=0.8,fill=gray!20,draw=none](-4.438,2.89)--(-4.536,2.884)--(-4.537,2.885)--(-4.542,2.894)--(-4.554,2.927)--(-4.553,2.935)--cycle;
\draw(-4.536,2.884)--(-4.537,2.885);
\draw(-4.542,2.894)--(-4.554,2.927)--(-4.553,2.935);
\filldraw[fill opacity=0.8,fill=gray!20,draw=none](-4.536,2.883)--(-4.537,2.885)--(-4.536,2.884)--cycle;
\draw(-4.537,2.885)--(-4.536,2.884);
\filldraw[fill opacity=0.8,fill=gray!20,draw=none](-4.537,2.885)--(-4.541,2.89)--(-4.542,2.894)--cycle;
\draw(-4.537,2.885)--(-4.541,2.89)--(-4.542,2.894);
\filldraw[fill opacity=0.8,fill=gray!20,draw=none](-4.605,2.914)--(-4.608,2.915)--(-4.606,2.917)--cycle;
\draw(-4.605,2.914)--(-4.608,2.915);
\filldraw[fill opacity=0.8,fill=gray!20,draw=none](-4.517,2.723)--(-4.516,2.729)--(-4.511,2.721)--(-4.516,2.72)--cycle;
\draw(-4.511,2.721)--(-4.516,2.72);
\filldraw[fill opacity=0.8,fill=gray!20,draw=none](-4.438,2.891)--(-4.438,2.89)--(-4.542,2.876)--(-4.55,2.894)--(-4.555,2.929)--(-4.55,2.948)--(-4.545,2.965)--cycle;
\draw(-4.55,2.894)--(-4.555,2.929);
\draw(-4.55,2.948)--(-4.545,2.965);
\filldraw[fill opacity=0.8,fill=gray!20,draw=none](-4.488,2.993)--(-4.491,2.995)--(-4.491,2.995)--cycle;
\draw(-4.488,2.993)--(-4.491,2.995)--(-4.491,2.995);
\filldraw[fill opacity=0.8,fill=gray!20,draw=none](-4.555,2.929)--(-4.555,2.933)--(-4.55,2.948)--cycle;
\draw(-4.555,2.929)--(-4.555,2.933)--(-4.55,2.948);
\filldraw[fill opacity=0.8,fill=gray!20,draw=none](-4.55,2.948)--(-4.555,2.935)--(-4.552,2.946)--cycle;
\draw(-4.55,2.948)--(-4.555,2.935);
\filldraw[fill opacity=0.8,fill=gray!20,draw=none](-4.544,2.967)--(-4.55,2.948)--(-4.552,2.946)--(-4.548,2.96)--cycle;
\draw(-4.544,2.967)--(-4.55,2.948);
\filldraw[fill opacity=0.8,fill=gray!20,draw=none](-4.438,2.89)--(-4.553,2.935)--(-4.55,2.958)--(-4.543,2.966)--cycle;
\draw(-4.553,2.935)--(-4.55,2.958)--(-4.543,2.966);
\filldraw[fill opacity=0.8,fill=gray!20,draw=none](-4.516,2.709)--(-4.522,2.72)--(-4.518,2.722)--cycle;
\draw(-4.522,2.72)--(-4.518,2.722);
\filldraw[fill opacity=0.8,fill=gray!20,draw=none](-4.536,2.761)--(-4.525,2.743)--(-4.517,2.723)--(-4.518,2.72)--(-4.53,2.718)--(-4.548,2.737)--cycle;
\draw(-4.518,2.72)--(-4.53,2.718)--(-4.548,2.737);
\filldraw[fill opacity=0.8,fill=gray!20,draw=none](-4.517,2.723)--(-4.525,2.743)--(-4.516,2.729)--cycle;
\filldraw[fill opacity=0.8,fill=gray!20,draw=none](-4.525,2.743)--(-4.518,2.733)--(-4.52,2.729)--cycle;
\filldraw[fill opacity=0.8,fill=gray!20,draw=none](-4.516,2.737)--(-4.518,2.733)--(-4.522,2.739)--cycle;
\filldraw[fill opacity=0.8,fill=gray!20,draw=none](-4.538,2.722)--(-4.547,2.717)--(-4.548,2.737)--(-4.543,2.732)--cycle;
\draw(-4.548,2.737)--(-4.543,2.732);
\filldraw[fill opacity=0.8,fill=gray!20,draw=none](-4.533,2.715)--(-4.543,2.732)--(-4.53,2.718)--cycle;
\draw(-4.543,2.732)--(-4.53,2.718)--(-4.533,2.715);
\filldraw[fill opacity=0.8,fill=gray!20,draw=none](-4.538,2.722)--(-4.533,2.715)--(-4.546,2.707)--(-4.547,2.717)--cycle;
\draw(-4.533,2.715)--(-4.546,2.707);
\filldraw[fill opacity=0.8,fill=gray!20,draw=none](-4.516,2.738)--(-4.526,2.724)--(-4.579,2.705)--(-4.527,2.749)--(-4.525,2.749)--cycle;
\draw(-4.526,2.724)--(-4.579,2.705);
\draw(-4.527,2.749)--(-4.525,2.749);
\filldraw[fill opacity=0.8,fill=gray!20,draw=none](-4.548,2.731)--(-4.553,2.727)--(-4.548,2.737)--cycle;
\filldraw[fill opacity=0.8,fill=gray!20,draw=none](-4.547,2.717)--(-4.546,2.707)--(-4.547,2.706)--(-4.555,2.712)--cycle;
\draw(-4.546,2.707)--(-4.547,2.706)--(-4.555,2.712);
\filldraw[fill opacity=0.8,fill=gray!20,draw=none](-4.555,2.71)--(-4.635,2.681)--(-4.579,2.705)--(-4.556,2.714)--cycle;
\draw(-4.555,2.71)--(-4.635,2.681);
\draw(-4.579,2.705)--(-4.556,2.714);
\filldraw[fill opacity=0.8,fill=gray!20,draw=none](-4.578,2.655)--(-4.86,2.481)--(-4.695,2.613)--(-4.526,2.717)--cycle;
\draw(-4.578,2.655)--(-4.86,2.481);
\draw(-4.695,2.613)--(-4.526,2.717);
\filldraw[fill opacity=0.8,fill=gray!20,draw=none](-4.516,2.709)--(-4.515,2.699)--(-4.517,2.693)--(-4.578,2.655)--(-4.526,2.717)--(-4.522,2.72)--cycle;
\draw(-4.517,2.693)--(-4.578,2.655);
\draw(-4.526,2.717)--(-4.522,2.72);
\filldraw[fill opacity=0.8,fill=gray!20,draw=none](-4.546,2.705)--(-4.557,2.711)--(-4.591,2.738)--(-4.593,2.741)--(-4.547,2.706)--cycle;
\draw(-4.591,2.738)--(-4.593,2.741)--(-4.547,2.706)--(-4.546,2.705);
\filldraw[fill opacity=0.8,fill=gray!20,draw=none](-4.555,2.71)--(-4.556,2.714)--(-4.526,2.724)--cycle;
\draw(-4.556,2.714)--(-4.526,2.724);
\filldraw[fill opacity=0.8,fill=gray!20,draw=none](-4.547,2.717)--(-4.555,2.712)--(-4.593,2.741)--(-4.567,2.757)--(-4.548,2.737)--cycle;
\draw(-4.555,2.712)--(-4.593,2.741)--(-4.567,2.757)--(-4.548,2.737);
\filldraw[fill opacity=0.8,fill=gray!20,draw=none](-4.554,2.818)--(-4.581,2.83)--(-4.569,2.848)--cycle;
\draw(-4.581,2.83)--(-4.569,2.848);
\filldraw[fill opacity=0.8,fill=gray!20,draw=none](-4.555,2.929)--(-4.585,2.889)--(-4.587,2.896)--(-4.555,2.933)--cycle;
\draw(-4.587,2.896)--(-4.555,2.933)--(-4.555,2.929);
\filldraw[fill opacity=0.8,fill=gray!20,draw=none](-4.552,2.946)--(-4.568,2.931)--(-4.55,2.958)--cycle;
\draw(-4.568,2.931)--(-4.55,2.958)--(-4.552,2.946);
\filldraw[fill opacity=0.8,fill=gray!20,draw=none](-4.554,2.971)--(-4.556,2.971)--(-4.565,2.966)--cycle;
\draw(-4.556,2.971)--(-4.565,2.966);
\filldraw[fill opacity=0.8,fill=gray!20,draw=none](-4.617,2.873)--(-4.653,2.891)--(-4.654,2.9)--(-4.606,2.89)--cycle;
\draw(-4.617,2.873)--(-4.653,2.891);
\filldraw[fill opacity=0.8,fill=gray!20,draw=none](-4.656,2.905)--(-4.663,2.905)--(-4.681,2.894)--(-4.683,2.886)--(-4.663,2.898)--cycle;
\draw(-4.663,2.905)--(-4.681,2.894);
\draw(-4.683,2.886)--(-4.663,2.898);
\filldraw[fill opacity=0.8,fill=gray!20,draw=none](-4.622,2.864)--(-4.647,2.872)--(-4.654,2.891)--(-4.617,2.873)--cycle;
\draw(-4.654,2.891)--(-4.617,2.873);
\filldraw[fill opacity=0.8,fill=gray!20,draw=none](-4.654,2.915)--(-4.661,2.901)--(-4.68,2.905)--(-4.677,2.933)--(-4.67,2.942)--(-4.666,2.944)--(-4.654,2.939)--cycle;
\draw(-4.666,2.944)--(-4.654,2.939);
\filldraw[fill opacity=0.8,fill=gray!20,draw=none](-4.533,2.982)--(-4.554,2.971)--(-4.545,2.971)--(-4.542,2.973)--cycle;
\draw(-4.545,2.971)--(-4.542,2.973);
\filldraw[fill opacity=0.8,fill=gray!20,draw=none](-4.653,2.889)--(-4.642,2.877)--(-4.643,2.881)--(-4.654,2.916)--(-4.656,2.904)--cycle;
\draw(-4.653,2.889)--(-4.642,2.877);
\draw(-4.654,2.916)--(-4.656,2.904);
\filldraw[fill opacity=0.8,fill=gray!20,draw=none](-4.653,2.889)--(-4.656,2.904)--(-4.657,2.893)--cycle;
\draw(-4.656,2.904)--(-4.657,2.893)--(-4.653,2.889);
\filldraw[fill opacity=0.8,fill=gray!20,draw=none](-4.654,2.9)--(-4.66,2.901)--(-4.654,2.906)--cycle;
\filldraw[fill opacity=0.8,fill=gray!20,draw=none](-4.652,2.897)--(-4.644,2.91)--(-4.655,2.903)--cycle;
\draw(-4.644,2.91)--(-4.655,2.903);
\filldraw[fill opacity=0.8,fill=gray!20,draw=none](-4.558,2.963)--(-4.644,2.91)--(-4.649,2.901)--(-4.643,2.881)--(-4.589,2.915)--cycle;
\draw(-4.558,2.963)--(-4.644,2.91);
\draw(-4.643,2.881)--(-4.589,2.915);
\filldraw[fill opacity=0.8,fill=gray!20,draw=none](-4.643,2.881)--(-4.646,2.94)--(-4.65,2.948)--(-4.654,2.916)--cycle;
\draw(-4.65,2.948)--(-4.654,2.916);
\filldraw[fill opacity=0.8,fill=gray!20,draw=none](-4.548,2.96)--(-4.567,2.934)--(-4.588,2.915)--(-4.548,2.964)--cycle;
\draw(-4.588,2.915)--(-4.548,2.964);
\filldraw[fill opacity=0.8,fill=gray!20,draw=none](-4.593,2.855)--(-4.622,2.864)--(-4.617,2.873)--(-4.589,2.859)--cycle;
\draw(-4.617,2.873)--(-4.589,2.859);
\filldraw[fill opacity=0.8,fill=gray!20,draw=none](-4.647,2.872)--(-4.593,2.855)--(-4.612,2.84)--(-4.642,2.855)--cycle;
\draw(-4.612,2.84)--(-4.642,2.855);
\filldraw[fill opacity=0.8,fill=gray!20,draw=none](-4.554,2.818)--(-4.546,2.801)--(-4.589,2.737)--(-4.618,2.775)--(-4.581,2.83)--cycle;
\draw(-4.546,2.801)--(-4.589,2.737);
\draw(-4.618,2.775)--(-4.581,2.83);
\filldraw[fill opacity=0.8,fill=gray!20,draw=none](-4.567,2.934)--(-4.552,2.946)--(-4.555,2.935)--(-4.555,2.933)--(-4.62,2.856)--cycle;
\draw(-4.555,2.935)--(-4.555,2.933)--(-4.62,2.856);
\filldraw[fill opacity=0.8,fill=gray!20,draw=none](-4.552,2.906)--(-4.559,2.88)--(-4.609,2.845)--(-4.556,2.925)--cycle;
\draw(-4.609,2.845)--(-4.556,2.925);
\filldraw[fill opacity=0.8,fill=gray!20,draw=none](-4.622,2.839)--(-4.63,2.841)--(-4.63,2.842)--(-4.619,2.843)--(-4.612,2.84)--cycle;
\draw(-4.619,2.843)--(-4.612,2.84);
\filldraw[fill opacity=0.8,fill=gray!20,draw=none](-4.63,2.842)--(-4.63,2.849)--(-4.619,2.843)--cycle;
\draw(-4.63,2.849)--(-4.619,2.843);
\filldraw[fill opacity=0.8,fill=gray!20,draw=none](-4.585,2.889)--(-4.622,2.839)--(-4.63,2.841)--(-4.62,2.856)--(-4.587,2.896)--cycle;
\draw(-4.62,2.856)--(-4.587,2.896);
\filldraw[fill opacity=0.8,fill=gray!20,draw=none](-4.622,2.839)--(-4.627,2.833)--(-4.632,2.838)--(-4.632,2.841)--cycle;
\filldraw[fill opacity=0.8,fill=gray!20,draw=none](-4.63,2.841)--(-4.622,2.839)--(-4.629,2.839)--(-4.63,2.839)--cycle;
\filldraw[fill opacity=0.8,fill=gray!20,draw=none](-4.552,2.946)--(-4.554,2.93)--(-4.622,2.839)--(-4.629,2.839)--(-4.627,2.844)--(-4.568,2.931)--cycle;
\draw(-4.552,2.946)--(-4.554,2.93);
\draw(-4.627,2.844)--(-4.568,2.931);
\filldraw[fill opacity=0.8,fill=gray!20,draw=none](-4.548,2.96)--(-4.552,2.946)--(-4.567,2.934)--cycle;
\filldraw[fill opacity=0.8,fill=gray!20,draw=none](-4.533,2.975)--(-4.548,2.96)--(-4.568,2.931)--(-4.578,2.917)--cycle;
\draw(-4.533,2.975)--(-4.548,2.96);
\draw(-4.568,2.931)--(-4.578,2.917);
\filldraw[fill opacity=0.8,fill=gray!20,draw=none](-4.548,2.96)--(-4.55,2.958)--(-4.568,2.931)--cycle;
\draw(-4.548,2.96)--(-4.55,2.958)--(-4.568,2.931);
\filldraw[fill opacity=0.8,fill=gray!20,draw=none](-4.54,2.966)--(-4.544,2.968)--(-4.543,2.966)--cycle;
\draw(-4.54,2.966)--(-4.544,2.968)--(-4.543,2.966);
\filldraw[fill opacity=0.8,fill=gray!20,draw=none](-4.542,2.967)--(-4.545,2.966)--(-4.553,2.957)--cycle;
\draw(-4.545,2.966)--(-4.553,2.957);
\filldraw[fill opacity=0.8,fill=gray!20,draw=none](-4.438,2.89)--(-4.509,2.994)--(-4.488,2.999)--cycle;
\draw(-4.509,2.994)--(-4.488,2.999);
\filldraw[fill opacity=0.8,fill=gray!20,draw=none](-4.438,2.89)--(-4.509,2.994)--(-4.509,2.992)--(-4.507,2.983)--(-4.504,2.971)--(-4.501,2.961)--(-4.493,2.944)--cycle;
\draw(-4.507,2.983)--(-4.504,2.971);
\draw(-4.501,2.961)--(-4.493,2.944);
\filldraw[fill opacity=0.8,fill=gray!20,draw=none](-4.438,2.89)--(-4.507,2.983)--(-4.493,2.985)--(-4.488,2.985)--cycle;
\filldraw[fill opacity=0.8,fill=gray!20,draw=none](-4.68,2.905)--(-4.681,2.905)--(-4.694,2.911)--(-4.677,2.933)--cycle;
\draw(-4.681,2.905)--(-4.694,2.911);
\filldraw[fill opacity=0.8,fill=gray!20,draw=none](-4.544,2.967)--(-4.544,2.968)--(-4.542,2.97)--cycle;
\draw(-4.544,2.967)--(-4.544,2.968)--(-4.542,2.97);
\filldraw[fill opacity=0.8,fill=gray!20,draw=none](-4.534,2.975)--(-4.544,2.967)--(-4.542,2.967)--cycle;
\filldraw[fill opacity=0.8,fill=gray!20,draw=none](-4.528,2.963)--(-4.524,2.98)--(-4.542,2.97)--(-4.544,2.968)--cycle;
\draw(-4.542,2.97)--(-4.544,2.968)--(-4.528,2.963);
\filldraw[fill opacity=0.8,fill=gray!20,draw=none](-4.539,2.968)--(-4.542,2.967)--(-4.553,2.957)--(-4.585,2.919)--cycle;
\draw(-4.553,2.957)--(-4.585,2.919);
\filldraw[fill opacity=0.8,fill=gray!20,draw=none](-4.686,2.891)--(-5.526,2.37)--(-5.464,2.403)--(-5.412,2.434)--(-4.839,2.789)--cycle;
\draw(-4.686,2.891)--(-5.526,2.37);
\draw(-5.412,2.434)--(-4.839,2.789);
\filldraw[fill opacity=0.8,fill=gray!20,draw=none](-4.681,2.894)--(-4.686,2.891)--(-4.839,2.789)--(-4.683,2.886)--cycle;
\draw(-4.681,2.894)--(-4.686,2.891);
\draw(-4.839,2.789)--(-4.683,2.886);
\filldraw[fill opacity=0.8,fill=gray!20,draw=none](-4.524,2.98)--(-4.523,2.986)--(-4.534,2.981)--(-4.542,2.97)--cycle;
\draw(-4.534,2.981)--(-4.542,2.97);
\filldraw[fill opacity=0.8,fill=gray!20,draw=none](-4.534,2.981)--(-4.542,2.973)--(-4.52,2.987)--cycle;
\draw(-4.542,2.973)--(-4.52,2.987);
\filldraw[fill opacity=0.8,fill=gray!20,draw=none](-4.63,2.842)--(-4.632,2.841)--(-4.642,2.855)--(-4.63,2.849)--cycle;
\draw(-4.642,2.855)--(-4.63,2.849);
\filldraw[fill opacity=0.8,fill=gray!20,draw=none](-4.567,2.934)--(-4.62,2.856)--(-4.632,2.842)--(-4.632,2.843)--(-4.588,2.915)--cycle;
\draw(-4.62,2.856)--(-4.632,2.842);
\filldraw[fill opacity=0.8,fill=gray!20,draw=none](-4.589,2.915)--(-4.643,2.881)--(-4.639,2.865)--cycle;
\draw(-4.589,2.915)--(-4.643,2.881);
\filldraw[fill opacity=0.8,fill=gray!20,draw=none](-4.642,2.877)--(-4.637,2.872)--(-4.637,2.893)--(-4.641,2.927)--(-4.646,2.94)--cycle;
\draw(-4.642,2.877)--(-4.637,2.872);
\filldraw[fill opacity=0.8,fill=gray!20,draw=none](-4.533,2.975)--(-4.578,2.917)--(-4.627,2.844)--(-4.56,2.932)--(-4.531,2.975)--cycle;
\draw(-4.578,2.917)--(-4.627,2.844);
\draw(-4.56,2.932)--(-4.531,2.975);
\filldraw[fill opacity=0.8,fill=gray!20,draw=none](-4.533,2.975)--(-4.528,2.979)--(-4.524,2.98)--cycle;
\draw(-4.528,2.979)--(-4.524,2.98);
\filldraw[fill opacity=0.8,fill=gray!20,draw=none](-4.528,2.979)--(-4.534,2.975)--(-4.542,2.967)--(-4.539,2.968)--cycle;
\filldraw[fill opacity=0.8,fill=gray!20,draw=none](-4.639,2.865)--(-4.642,2.877)--(-4.657,2.893)--(-4.653,2.862)--cycle;
\draw(-4.642,2.877)--(-4.657,2.893)--(-4.653,2.862);
\filldraw[fill opacity=0.8,fill=gray!20,draw=none](-4.653,2.891)--(-4.663,2.896)--(-4.662,2.899)--(-4.66,2.901)--(-4.654,2.9)--cycle;
\draw(-4.653,2.891)--(-4.663,2.896);
\filldraw[fill opacity=0.8,fill=gray!20,draw=none](-4.647,2.872)--(-4.664,2.877)--(-4.663,2.896)--(-4.654,2.891)--cycle;
\draw(-4.663,2.896)--(-4.654,2.891);
\filldraw[fill opacity=0.8,fill=gray!20,draw=none](-4.652,2.897)--(-4.655,2.903)--(-4.73,2.857)--(-4.69,2.852)--(-4.673,2.863)--cycle;
\draw(-4.655,2.903)--(-4.73,2.857);
\draw(-4.69,2.852)--(-4.673,2.863);
\filldraw[fill opacity=0.8,fill=gray!20,draw=none](-4.673,2.888)--(-4.684,2.877)--(-4.694,2.909)--(-4.694,2.911)--(-4.68,2.904)--cycle;
\draw(-4.694,2.911)--(-4.68,2.904);
\filldraw[fill opacity=0.8,fill=gray!20,draw=none](-4.509,2.994)--(-4.519,2.989)--(-4.52,2.987)--cycle;
\filldraw[fill opacity=0.8,fill=gray!20,draw=none](-4.519,2.989)--(-4.52,2.987)--(-4.515,2.993)--cycle;
\draw(-4.52,2.987)--(-4.515,2.993)--(-4.519,2.989);
\filldraw[fill opacity=0.8,fill=gray!20,draw=none](-4.533,2.975)--(-4.529,2.979)--(-4.528,2.979)--cycle;
\draw(-4.533,2.975)--(-4.529,2.979)--(-4.528,2.979);
\filldraw[fill opacity=0.8,fill=gray!20,draw=none](-4.526,2.985)--(-4.519,2.988)--(-4.515,2.997)--(-4.516,2.997)--cycle;
\filldraw[fill opacity=0.8,fill=gray!20,draw=none](-4.524,2.98)--(-4.521,2.982)--(-4.519,2.988)--(-4.523,2.986)--cycle;
\filldraw[fill opacity=0.8,fill=gray!20,draw=none](-4.533,2.975)--(-4.525,2.981)--(-4.52,2.987)--cycle;
\filldraw[fill opacity=0.8,fill=gray!20,draw=none](-4.533,2.975)--(-4.531,2.975)--(-4.529,2.979)--cycle;
\draw(-4.531,2.975)--(-4.529,2.979)--(-4.533,2.975);
\filldraw[fill opacity=0.8,fill=gray!20,draw=none](-4.627,2.833)--(-4.554,2.93)--(-4.554,2.927)--(-4.622,2.825)--cycle;
\draw(-4.554,2.93)--(-4.554,2.927)--(-4.622,2.825);
\filldraw[fill opacity=0.8,fill=gray!20,draw=none](-4.509,2.992)--(-4.509,2.989)--(-4.507,2.983)--cycle;
\draw(-4.509,2.989)--(-4.507,2.983);
\filldraw[fill opacity=0.8,fill=gray!20,draw=none](-4.5,2.99)--(-4.515,2.993)--(-4.52,2.987)--cycle;
\draw(-4.515,2.993)--(-4.52,2.987);
\filldraw[fill opacity=0.8,fill=gray!20,draw=none](-4.5,2.99)--(-4.488,2.999)--(-4.515,2.993)--cycle;
\draw(-4.488,2.999)--(-4.515,2.993);
\filldraw[fill opacity=0.8,fill=gray!20,draw=none](-4.509,2.994)--(-4.52,2.987)--(-4.51,2.993)--cycle;
\draw(-4.52,2.987)--(-4.51,2.993)--(-4.509,2.994);
\filldraw[fill opacity=0.8,fill=gray!20,draw=none](-4.527,2.964)--(-4.507,2.983)--(-4.529,2.979)--(-4.545,2.955)--cycle;
\draw(-4.507,2.983)--(-4.529,2.979)--(-4.545,2.955);
\filldraw[fill opacity=0.8,fill=gray!20,draw=none](-4.528,2.979)--(-4.539,2.968)--(-4.538,2.968)--(-4.525,2.981)--cycle;
\filldraw[fill opacity=0.8,fill=gray!20,draw=none](-4.528,2.963)--(-4.521,2.982)--(-4.524,2.98)--cycle;
\filldraw[fill opacity=0.8,fill=gray!20,draw=none](-4.527,2.964)--(-4.545,2.955)--(-4.56,2.932)--cycle;
\draw(-4.545,2.955)--(-4.56,2.932);
\filldraw[fill opacity=0.8,fill=gray!20,draw=none](-4.504,2.971)--(-4.509,2.989)--(-4.52,2.987)--(-4.525,2.983)--(-4.536,2.967)--(-4.529,2.964)--cycle;
\draw(-4.504,2.971)--(-4.509,2.989);
\draw(-4.52,2.987)--(-4.525,2.983);
\filldraw[fill opacity=0.8,fill=gray!20,draw=none](-4.519,2.987)--(-4.528,2.963)--(-4.516,2.979)--cycle;
\draw(-4.528,2.963)--(-4.516,2.979);
\filldraw[fill opacity=0.8,fill=gray!20,draw=none](-4.536,2.967)--(-4.56,2.932)--(-4.518,2.959)--cycle;
\draw(-4.56,2.932)--(-4.518,2.959);
\filldraw[fill opacity=0.8,fill=gray!20,draw=none](-4.516,2.979)--(-4.5,2.99)--(-4.52,2.987)--(-4.528,2.977)--cycle;
\draw(-4.52,2.987)--(-4.528,2.977);
\filldraw[fill opacity=0.8,fill=gray!20,draw=none](-4.509,2.989)--(-4.51,2.993)--(-4.52,2.987)--cycle;
\draw(-4.509,2.989)--(-4.51,2.993)--(-4.52,2.987);
\filldraw[fill opacity=0.8,fill=gray!20,draw=none](-4.519,2.987)--(-4.516,2.979)--(-4.515,2.997)--cycle;
\filldraw[fill opacity=0.8,fill=gray!20,draw=none](-4.539,2.968)--(-4.585,2.919)--(-4.588,2.915)--(-4.538,2.968)--cycle;
\draw(-4.585,2.919)--(-4.588,2.915);
\filldraw[fill opacity=0.8,fill=gray!20,draw=none](-4.536,2.967)--(-4.545,2.971)--(-4.558,2.963)--(-4.589,2.915)--(-4.56,2.932)--cycle;
\draw(-4.545,2.971)--(-4.558,2.963);
\draw(-4.589,2.915)--(-4.56,2.932);
\filldraw[fill opacity=0.8,fill=gray!20,draw=none](-4.536,2.967)--(-4.525,2.983)--(-4.545,2.971)--cycle;
\draw(-4.525,2.983)--(-4.545,2.971);
\filldraw[fill opacity=0.8,fill=gray!20,draw=none](-4.509,2.994)--(-4.51,2.993)--(-4.509,2.989)--cycle;
\draw(-4.509,2.994)--(-4.51,2.993)--(-4.509,2.989);
\filldraw[fill opacity=0.8,fill=gray!20,draw=none](-4.588,2.915)--(-4.537,2.966)--(-4.52,2.987)--cycle;
\draw(-4.537,2.966)--(-4.52,2.987);
\filldraw[fill opacity=0.8,fill=gray!20,draw=none](-4.516,2.979)--(-4.528,2.977)--(-4.539,2.964)--cycle;
\draw(-4.528,2.977)--(-4.539,2.964);
\filldraw[fill opacity=0.8,fill=gray!20,draw=none](-4.5,2.985)--(-4.507,2.983)--(-4.527,2.964)--(-4.5,2.978)--cycle;
\draw(-4.5,2.985)--(-4.507,2.983);
\filldraw[fill opacity=0.8,fill=gray!20,draw=none](-4.507,2.983)--(-4.5,2.985)--(-4.493,2.985)--cycle;
\draw(-4.507,2.983)--(-4.5,2.985);
\filldraw[fill opacity=0.8,fill=gray!20,draw=none](-4.506,2.952)--(-4.5,2.961)--(-4.498,2.978)--(-4.516,2.979)--(-4.528,2.963)--cycle;
\draw(-4.516,2.979)--(-4.528,2.963)--(-4.506,2.952);
\filldraw[fill opacity=0.8,fill=gray!20,draw=none](-4.641,2.828)--(-4.633,2.837)--(-4.632,2.843)--(-4.639,2.865)--(-4.653,2.862)--(-4.65,2.837)--cycle;
\draw(-4.653,2.862)--(-4.65,2.837)--(-4.641,2.828);
\filldraw[fill opacity=0.8,fill=gray!20,draw=none](-4.63,2.841)--(-4.632,2.841)--(-4.632,2.842)--(-4.62,2.856)--cycle;
\draw(-4.632,2.842)--(-4.62,2.856);
\filldraw[fill opacity=0.8,fill=gray!20,draw=none](-4.589,2.915)--(-4.639,2.865)--(-4.631,2.841)--(-4.627,2.844)--cycle;
\draw(-4.631,2.841)--(-4.627,2.844);
\filldraw[fill opacity=0.8,fill=gray!20,draw=none](-4.632,2.843)--(-4.632,2.844)--(-4.608,2.887)--(-4.588,2.915)--cycle;
\filldraw[fill opacity=0.8,fill=gray!20,draw=none](-4.61,2.885)--(-4.601,2.9)--(-4.588,2.915)--cycle;
\draw(-4.601,2.9)--(-4.588,2.915);
\filldraw[fill opacity=0.8,fill=gray!20,draw=none](-4.555,2.929)--(-4.552,2.906)--(-4.559,2.88)--(-4.574,2.862)--(-4.585,2.889)--cycle;
\draw(-4.555,2.929)--(-4.552,2.906);
\draw(-4.559,2.88)--(-4.574,2.862);
\filldraw[fill opacity=0.8,fill=gray!20,draw=none](-4.618,2.775)--(-4.633,2.82)--(-4.65,2.837)--(-4.628,2.785)--cycle;
\draw(-4.633,2.82)--(-4.65,2.837)--(-4.628,2.785)--(-4.618,2.775);
\filldraw[fill opacity=0.8,fill=gray!20,draw=none](-4.61,2.839)--(-4.569,2.848)--(-4.572,2.843)--(-4.584,2.831)--cycle;
\draw(-4.569,2.848)--(-4.572,2.843);
\filldraw[fill opacity=0.8,fill=gray!20,draw=none](-4.593,2.855)--(-4.569,2.848)--(-4.61,2.839)--(-4.612,2.84)--cycle;
\draw(-4.61,2.839)--(-4.612,2.84);
\filldraw[fill opacity=0.8,fill=gray!20,draw=none](-4.577,2.867)--(-4.569,2.848)--(-4.61,2.839)--(-4.613,2.84)--(-4.609,2.845)--cycle;
\draw(-4.613,2.84)--(-4.609,2.845);
\filldraw[fill opacity=0.8,fill=gray!20,draw=none](-4.584,2.831)--(-4.572,2.843)--(-4.581,2.83)--cycle;
\draw(-4.572,2.843)--(-4.581,2.83);
\filldraw[fill opacity=0.8,fill=gray!20,draw=none](-4.569,2.848)--(-4.562,2.835)--(-4.605,2.836)--(-4.61,2.839)--cycle;
\draw(-4.605,2.836)--(-4.61,2.839);
\filldraw[fill opacity=0.8,fill=gray!20,draw=none](-4.603,2.81)--(-4.584,2.831)--(-4.581,2.83)--(-4.597,2.806)--cycle;
\draw(-4.581,2.83)--(-4.597,2.806);
\filldraw[fill opacity=0.8,fill=gray!20,draw=none](-4.61,2.839)--(-4.584,2.831)--(-4.603,2.81)--(-4.622,2.825)--(-4.614,2.838)--cycle;
\draw(-4.622,2.825)--(-4.614,2.838);
\filldraw[fill opacity=0.8,fill=gray!20,draw=none](-4.554,2.819)--(-4.554,2.816)--(-4.598,2.833)--(-4.574,2.862)--cycle;
\draw(-4.598,2.833)--(-4.574,2.862);
\filldraw[fill opacity=0.8,fill=gray!20,draw=none](-4.554,2.816)--(-4.571,2.796)--(-4.61,2.819)--(-4.598,2.833)--cycle;
\draw(-4.554,2.816)--(-4.571,2.796);
\draw(-4.61,2.819)--(-4.598,2.833);
\filldraw[fill opacity=0.8,fill=gray!20,draw=none](-4.536,2.761)--(-4.548,2.737)--(-4.567,2.757)--(-4.537,2.763)--cycle;
\draw(-4.548,2.737)--(-4.567,2.757)--(-4.537,2.763);
\filldraw[fill opacity=0.8,fill=gray!20,draw=none](-4.526,2.75)--(-4.527,2.749)--(-4.546,2.742)--(-4.551,2.76)--(-4.549,2.777)--(-4.546,2.779)--cycle;
\draw(-4.527,2.749)--(-4.546,2.742);
\filldraw[fill opacity=0.8,fill=gray!20,draw=none](-4.635,2.681)--(-4.895,2.589)--(-4.768,2.638)--(-4.579,2.705)--cycle;
\draw(-4.635,2.681)--(-4.895,2.589);
\draw(-4.768,2.638)--(-4.579,2.705);
\filldraw[fill opacity=0.8,fill=gray!20,draw=none](-4.557,2.711)--(-4.559,2.712)--(-4.585,2.732)--(-4.591,2.738)--cycle;
\draw(-4.585,2.732)--(-4.591,2.738);
\filldraw[fill opacity=0.8,fill=gray!20,draw=none](-4.541,2.792)--(-4.563,2.716)--(-4.585,2.732)--(-4.589,2.737)--(-4.546,2.801)--cycle;
\draw(-4.589,2.737)--(-4.546,2.801);
\filldraw[fill opacity=0.8,fill=gray!20,draw=none](-4.548,2.731)--(-4.548,2.737)--(-4.546,2.742)--(-4.527,2.749)--cycle;
\draw(-4.546,2.742)--(-4.527,2.749);
\filldraw[fill opacity=0.8,fill=gray!20,draw=none](-4.525,2.749)--(-4.527,2.749)--(-4.526,2.75)--cycle;
\draw(-4.525,2.749)--(-4.527,2.749);
\filldraw[fill opacity=0.8,fill=gray!20,draw=none](-4.513,2.725)--(-4.517,2.723)--(-4.52,2.729)--(-4.518,2.733)--cycle;
\draw(-4.513,2.725)--(-4.517,2.723);
\filldraw[fill opacity=0.8,fill=gray!20,draw=none](-4.508,2.73)--(-4.514,2.726)--(-4.544,2.714)--(-4.555,2.71)--(-4.526,2.724)--(-4.509,2.73)--cycle;
\draw(-4.544,2.714)--(-4.555,2.71);
\draw(-4.526,2.724)--(-4.509,2.73);
\filldraw[fill opacity=0.8,fill=gray!20,draw=none](-4.518,2.751)--(-4.543,2.714)--(-4.564,2.712)--(-4.541,2.792)--cycle;
\draw(-4.518,2.751)--(-4.543,2.714);
\filldraw[fill opacity=0.8,fill=gray!20,draw=none](-4.529,2.775)--(-4.528,2.768)--(-4.566,2.722)--(-4.579,2.727)--(-4.598,2.751)--(-4.554,2.816)--(-4.553,2.818)--cycle;
\draw(-4.528,2.768)--(-4.566,2.722);
\draw(-4.554,2.816)--(-4.553,2.818);
\filldraw[fill opacity=0.8,fill=gray!20,draw=none](-4.598,2.751)--(-4.603,2.758)--(-4.554,2.816)--cycle;
\draw(-4.603,2.758)--(-4.554,2.816);
\filldraw[fill opacity=0.8,fill=gray!20,draw=none](-4.603,2.81)--(-4.651,2.76)--(-4.648,2.788)--(-4.622,2.825)--cycle;
\draw(-4.648,2.788)--(-4.622,2.825);
\filldraw[fill opacity=0.8,fill=gray!20,draw=none](-4.596,2.836)--(-4.601,2.834)--(-4.605,2.836)--cycle;
\draw(-4.601,2.834)--(-4.605,2.836);
\filldraw[fill opacity=0.8,fill=gray!20,draw=none](-4.61,2.839)--(-4.614,2.838)--(-4.613,2.84)--cycle;
\draw(-4.614,2.838)--(-4.613,2.84);
\filldraw[fill opacity=0.8,fill=gray!20,draw=none](-4.622,2.839)--(-4.612,2.84)--(-4.601,2.834)--cycle;
\draw(-4.612,2.84)--(-4.601,2.834);
\filldraw[fill opacity=0.8,fill=gray!20,draw=none](-4.622,2.839)--(-4.585,2.889)--(-4.574,2.862)--(-4.596,2.836)--(-4.601,2.834)--cycle;
\draw(-4.574,2.862)--(-4.596,2.836);
\filldraw[fill opacity=0.8,fill=gray!20,draw=none](-4.56,2.932)--(-4.589,2.915)--(-4.591,2.91)--cycle;
\draw(-4.56,2.932)--(-4.589,2.915);
\filldraw[fill opacity=0.8,fill=gray!20,draw=none](-4.588,2.915)--(-4.601,2.9)--(-4.539,2.964)--(-4.537,2.966)--cycle;
\draw(-4.588,2.915)--(-4.601,2.9);
\draw(-4.539,2.964)--(-4.537,2.966);
\filldraw[fill opacity=0.8,fill=gray!20,draw=none](-4.488,2.985)--(-4.5,2.978)--(-4.487,2.975)--cycle;
\filldraw[fill opacity=0.8,fill=gray!20,draw=none](-4.487,3.007)--(-7.522,1.929)--(-7.485,1.917)--(-4.488,2.982)--cycle;
\draw(-7.485,1.917)--(-4.488,2.982)--(-4.487,3.007)--(-7.522,1.929);
\filldraw[fill opacity=0.8,fill=gray!20,draw=none](-3.196,2.14)--(-3.219,2.171)--(-3.221,2.145)--(-3.212,2.117)--cycle;
\filldraw[fill opacity=0.8,fill=gray!20,draw=none](-7.52,1.675)--(-7.544,1.689)--(-7.558,1.659)--(-7.528,1.67)--cycle;
\draw(-7.558,1.659)--(-7.528,1.67);
\filldraw[fill opacity=0.8,fill=gray!20](-3.018,3.269)--(-3.049,3.31)--(-3.033,3.293)--(-2.998,3.248)--cycle;
\filldraw[fill opacity=0.8,fill=gray!20,draw=none](-3.45,2.858)--(-3.45,2.831)--(-3.395,2.405)--(-3.34,2.567)--(-3.375,2.84)--cycle;
\draw(-3.45,2.831)--(-3.395,2.405);
\draw(-3.34,2.567)--(-3.375,2.84)--(-3.45,2.858);
\filldraw[fill opacity=0.8,fill=gray!20,draw=none](-7.683,.983)--(-7.666,.988)--(-7.654,.939)--(-7.656,.938)--cycle;
\draw(-7.683,.983)--(-7.666,.988);
\draw(-7.654,.939)--(-7.656,.938);
\filldraw[fill opacity=0.8,fill=gray!20,draw=none](-7.651,.933)--(-7.656,.938)--(-7.654,.939)--cycle;
\draw(-7.656,.938)--(-7.654,.939);
\filldraw[fill opacity=0.8,fill=gray!20,draw=none](-7.61,.856)--(-7.579,.866)--(-7.543,.848)--(-7.551,.845)--cycle;
\draw(-7.61,.856)--(-7.579,.866);
\draw(-7.543,.848)--(-7.551,.845);
\filldraw[fill opacity=0.8,fill=gray!20,draw=none](-7.61,.884)--(-7.579,.866)--(-7.588,.863)--cycle;
\draw(-7.579,.866)--(-7.588,.863);
\filldraw[fill opacity=0.8,fill=gray!20,draw=none](-3.014,2.87)--(-2.924,2.163)--(-2.82,2.168)--(-2.914,2.904)--cycle;
\draw(-2.82,2.168)--(-2.914,2.904)--(-3.014,2.87)--(-2.924,2.163);
\filldraw[fill opacity=0.8,fill=gray!20,draw=none](-3.281,2.272)--(-3.268,2.246)--(-3.26,2.236)--(-3.263,2.261)--cycle;
\draw(-3.26,2.236)--(-3.263,2.261);
\filldraw[fill opacity=0.8,fill=gray!20,draw=none](-3.216,2.202)--(-3.219,2.171)--(-3.166,2.1)--(-3.151,2.106)--(-3.158,2.157)--cycle;
\draw(-3.151,2.106)--(-3.158,2.157);
\filldraw[fill opacity=0.8,fill=gray!20,draw=none](-3.277,2.079)--(-3.273,2.044)--(-3.162,2.041)--(-3.173,2.126)--cycle;
\draw(-3.277,2.079)--(-3.273,2.044);
\draw(-3.162,2.041)--(-3.173,2.126);
\filldraw[fill opacity=0.8,fill=gray!20,draw=none](-7.65,.931)--(-7.65,.928)--(-7.65,.932)--cycle;
\draw(-7.65,.928)--(-7.65,.932);
\filldraw[fill opacity=0.8,fill=gray!20,draw=none](-3.37,3.24)--(-3.377,3.241)--(-3.389,3.221)--cycle;
\filldraw[fill opacity=0.8,fill=gray!20,draw=none](-3.401,3.234)--(-3.22,3.271)--(-3.237,3.252)--(-3.395,3.219)--cycle;
\draw(-3.401,3.234)--(-3.22,3.271)--(-3.237,3.252)--(-3.395,3.219);
\filldraw[fill opacity=0.8,fill=gray!20,draw=none](-3.283,2.1)--(-3.283,2.074)--(-3.277,2.079)--(-3.281,2.106)--cycle;
\draw(-3.277,2.079)--(-3.281,2.106);
\filldraw[fill opacity=0.8,fill=gray!20,draw=none](-7.645,.905)--(-7.684,.929)--(-7.656,.938)--(-7.651,.932)--cycle;
\draw(-7.684,.929)--(-7.656,.938);
\filldraw[fill opacity=0.8,fill=gray!20,draw=none](-7.98,.834)--(-7.979,.827)--(-8.011,.826)--(-7.999,.856)--cycle;
\draw(-7.979,.827)--(-8.011,.826);
\filldraw[fill opacity=0.8,fill=gray!20,draw=none](-7.98,.805)--(-8.017,.826)--(-7.979,.827)--cycle;
\draw(-8.017,.826)--(-7.979,.827);
\filldraw[fill opacity=0.8,fill=gray!20,draw=none](-7.98,.834)--(-7.999,.856)--(-7.989,.884)--cycle;
\filldraw[fill opacity=0.8,fill=gray!20,draw=none](-7.98,.805)--(-7.981,.771)--(-8.06,.767)--(-8.069,.823)--(-8.017,.826)--cycle;
\draw(-7.981,.771)--(-8.06,.767)--(-8.069,.823)--(-8.017,.826);
\filldraw[fill opacity=0.8,fill=gray!20,draw=none](-8.011,.826)--(-8.069,.823)--(-8.073,.88)--(-7.989,.884)--cycle;
\draw(-8.011,.826)--(-8.069,.823)--(-8.073,.88)--(-7.989,.884);
\filldraw[fill opacity=0.8,fill=gray!20](-8.133,.753)--(-8.151,.808)--(-8.069,.823)--(-8.06,.767)--cycle;
\filldraw[fill opacity=0.8,fill=gray!20](-8.151,.808)--(-8.158,.864)--(-8.073,.88)--(-8.069,.823)--cycle;
\filldraw[fill opacity=0.8,fill=gray!20,draw=none](-8.084,.854)--(-7.683,.983)--(-7.656,.938)--(-8.075,.803)--cycle;
\draw(-7.656,.938)--(-8.075,.803)--(-8.084,.854)--(-7.683,.983);
\filldraw[fill opacity=0.8,fill=gray!20,draw=none](-7.758,1.921)--(-7.758,1.842)--(-7.723,1.907)--(-7.723,1.912)--cycle;
\draw(-7.723,1.907)--(-7.723,1.912)--(-7.758,1.921)--(-7.758,1.842);
\filldraw[fill opacity=0.8,fill=gray!20,draw=none](-8.069,1.697)--(-8.103,1.677)--(-8.131,1.675)--(-8.123,1.715)--(-8.068,1.725)--cycle;
\draw(-8.103,1.677)--(-8.131,1.675)--(-8.123,1.715);
\filldraw[fill opacity=0.8,fill=gray!20,draw=none](-8.044,1.712)--(-8.069,1.697)--(-8.068,1.725)--(-8.034,1.727)--cycle;
\draw(-8.068,1.725)--(-8.034,1.727);
\filldraw[fill opacity=0.8,fill=gray!20,draw=none](-8.044,1.712)--(-8.034,1.727)--(-8.024,1.727)--(-8.024,1.723)--cycle;
\draw(-8.034,1.727)--(-8.024,1.727)--(-8.024,1.723);
\filldraw[fill opacity=0.8,fill=gray!20,draw=none](-8.024,1.723)--(-8.024,1.727)--(-8.018,1.727)--cycle;
\draw(-8.024,1.723)--(-8.024,1.727)--(-8.018,1.727);
\filldraw[fill opacity=0.8,fill=gray!20,draw=none](-8.005,1.735)--(-8.018,1.727)--(-8.024,1.727)--(-8.025,1.733)--cycle;
\draw(-8.018,1.727)--(-8.024,1.727)--(-8.025,1.733);
\filldraw[fill opacity=0.8,fill=gray!20,draw=none](-8.081,1.702)--(-7.743,1.85)--(-7.76,1.835)--(-8.119,1.679)--cycle;
\draw(-7.76,1.835)--(-8.119,1.679)--(-8.081,1.702)--(-7.743,1.85);
\filldraw[fill opacity=0.8,fill=gray!20,draw=none](-3.221,2.145)--(-3.219,2.171)--(-3.238,2.198)--cycle;
\filldraw[fill opacity=0.8,fill=gray!20,draw=none](-3.694,2.333)--(-3.71,2.325)--(-3.715,2.33)--cycle;
\draw(-3.71,2.325)--(-3.715,2.33);
\filldraw[fill opacity=0.8,fill=gray!20,draw=none](-3.721,2.334)--(-3.71,2.325)--(-3.745,2.301)--cycle;
\draw(-3.721,2.334)--(-3.71,2.325);
\filldraw[fill opacity=0.8,fill=gray!20,draw=none](-7.65,.932)--(-7.635,.937)--(-7.635,.975)--cycle;
\draw(-7.635,.937)--(-7.635,.975);
\filldraw[fill opacity=0.8,fill=gray!20](-3.28,3.342)--(-3.236,3.357)--(-3.214,3.362)--(-3.237,3.351)--cycle;
\filldraw[fill opacity=0.8,fill=gray!20,draw=none](-8.018,1.512)--(-8.024,1.513)--(-8.024,1.518)--cycle;
\draw(-8.018,1.512)--(-8.024,1.513)--(-8.024,1.518);
\filldraw[fill opacity=0.8,fill=gray!20,draw=none](-8.034,1.512)--(-8.044,1.53)--(-8.024,1.518)--(-8.024,1.513)--cycle;
\draw(-8.024,1.518)--(-8.024,1.513)--(-8.034,1.512);
\filldraw[fill opacity=0.8,fill=gray!20,draw=none](-8.005,1.498)--(-8.025,1.504)--(-8.024,1.513)--(-8.018,1.512)--cycle;
\draw(-8.025,1.504)--(-8.024,1.513)--(-8.018,1.512);
\filldraw[fill opacity=0.8,fill=gray!20,draw=none](-8.069,1.511)--(-8.024,1.513)--(-8.025,1.504)--cycle;
\draw(-8.069,1.511)--(-8.024,1.513)--(-8.025,1.504);
\filldraw[fill opacity=0.8,fill=gray!20,draw=none](-8.034,1.512)--(-8.068,1.511)--(-8.069,1.545)--(-8.044,1.53)--cycle;
\draw(-8.034,1.512)--(-8.068,1.511);
\filldraw[fill opacity=0.8,fill=gray!20,draw=none](-8.031,1.505)--(-8.025,1.504)--(-8.025,1.49)--cycle;
\draw(-8.025,1.504)--(-8.025,1.49);
\filldraw[fill opacity=0.8,fill=gray!20,draw=none](-8.005,1.498)--(-7.988,1.479)--(-8.026,1.473)--(-8.025,1.504)--cycle;
\draw(-8.026,1.473)--(-8.025,1.504);
\filldraw[fill opacity=0.8,fill=gray!20,draw=none](-8.105,1.457)--(-8.121,1.509)--(-8.069,1.511)--(-8.031,1.505)--(-8.025,1.49)--(-8.027,1.461)--cycle;
\draw(-8.025,1.49)--(-8.027,1.461)--(-8.105,1.457)--(-8.121,1.509)--(-8.069,1.511);
\filldraw[fill opacity=0.8,fill=gray!20](-8.165,1.446)--(-8.195,1.494)--(-8.121,1.509)--(-8.105,1.457)--cycle;
\filldraw[fill opacity=0.8,fill=gray!20,draw=none](-8.119,1.497)--(-7.828,1.624)--(-7.746,1.621)--(-7.743,1.612)--(-8.081,1.465)--cycle;
\draw(-7.743,1.612)--(-8.081,1.465)--(-8.119,1.497)--(-7.828,1.624);
\filldraw[fill opacity=0.8,fill=gray!20,draw=none](-3.745,2.301)--(-3.751,2.292)--(-3.754,2.295)--cycle;
\draw(-3.751,2.292)--(-3.754,2.295);
\filldraw[fill opacity=0.8,fill=gray!20,draw=none](-3.268,2.246)--(-3.259,2.226)--(-3.26,2.236)--cycle;
\draw(-3.259,2.226)--(-3.26,2.236);
\filldraw[fill opacity=0.8,fill=gray!20,draw=none](-7.523,1.674)--(-7.516,1.674)--(-7.516,1.677)--cycle;
\draw(-7.516,1.674)--(-7.516,1.677);
\filldraw[fill opacity=0.8,fill=gray!20,draw=none](-7.549,1.692)--(-7.547,1.683)--(-7.544,1.689)--cycle;
\filldraw[fill opacity=0.8,fill=gray!20,draw=none](-7.553,1.681)--(-7.547,1.683)--(-7.549,1.692)--cycle;
\draw(-7.553,1.681)--(-7.547,1.683);
\filldraw[fill opacity=0.8,fill=gray!20,draw=none](-7.561,1.673)--(-7.523,1.674)--(-7.516,1.677)--(-7.516,1.716)--cycle;
\draw(-7.516,1.677)--(-7.516,1.716);
\filldraw[fill opacity=0.8,fill=gray!20,draw=none](-7.723,1.616)--(-7.723,1.611)--(-7.691,1.601)--(-7.673,1.602)--(-7.673,1.604)--cycle;
\draw(-7.723,1.616)--(-7.723,1.611);
\draw(-7.673,1.602)--(-7.673,1.604);
\filldraw[fill opacity=0.8,fill=gray!20,draw=none](-7.723,1.611)--(-7.691,1.601)--(-7.698,1.597)--cycle;
\draw(-7.691,1.601)--(-7.698,1.597);
\filldraw[fill opacity=0.8,fill=gray!20,draw=none](-3.332,2.274)--(-3.35,2.285)--(-3.319,2.248)--cycle;
\filldraw[fill opacity=0.8,fill=gray!20,draw=none](-7.496,1.69)--(-7.52,1.675)--(-7.518,1.674)--(-7.495,1.682)--cycle;
\draw(-7.518,1.674)--(-7.495,1.682);
\filldraw[fill opacity=0.8,fill=gray!20,draw=none](-7.723,1.653)--(-7.723,1.616)--(-7.673,1.604)--(-7.673,1.626)--cycle;
\draw(-7.723,1.653)--(-7.723,1.616);
\draw(-7.673,1.604)--(-7.673,1.626);
\filldraw[fill opacity=0.8,fill=gray!20](-3.188,2.929)--(-3.239,2.931)--(-3.245,2.937)--(-3.188,2.929)--cycle;
\filldraw[fill opacity=0.8,fill=gray!20](-3.239,2.931)--(-3.286,2.947)--(-3.298,2.959)--(-3.245,2.937)--cycle;
\filldraw[fill opacity=0.8,fill=gray!20](-3.343,3.297)--(-3.298,3.331)--(-3.28,3.342)--(-3.318,3.314)--cycle;
\filldraw[fill opacity=0.8,fill=gray!20,draw=none](-3.682,2.335)--(-3.725,2.398)--(-3.655,2.34)--cycle;
\draw(-3.725,2.398)--(-3.655,2.34);
\filldraw[fill opacity=0.8,fill=gray!20,draw=none](-7.52,1.675)--(-7.496,1.69)--(-7.499,1.716)--(-7.538,1.702)--(-7.544,1.689)--cycle;
\draw(-7.499,1.716)--(-7.538,1.702);
\filldraw[fill opacity=0.8,fill=gray!20,draw=none](-3.247,2.226)--(-3.26,2.236)--(-3.259,2.226)--(-3.238,2.198)--cycle;
\draw(-3.26,2.236)--(-3.259,2.226);
\filldraw[fill opacity=0.8,fill=gray!20,draw=none](-7.98,.932)--(-7.989,.884)--(-7.999,.909)--cycle;
\filldraw[fill opacity=0.8,fill=gray!20,draw=none](-7.999,.909)--(-7.989,.884)--(-8.022,.883)--cycle;
\draw(-7.989,.884)--(-8.022,.883);
\filldraw[fill opacity=0.8,fill=gray!20,draw=none](-7.999,.909)--(-8.022,.883)--(-8.073,.88)--(-8.069,.934)--(-8.011,.937)--cycle;
\draw(-8.022,.883)--(-8.073,.88)--(-8.069,.934)--(-8.011,.937);
\filldraw[fill opacity=0.8,fill=gray!20,draw=none](-7.98,.932)--(-7.999,.909)--(-8.011,.937)--(-7.979,.938)--cycle;
\draw(-8.011,.937)--(-7.979,.938);
\filldraw[fill opacity=0.8,fill=gray!20](-8.158,.864)--(-8.151,.918)--(-8.069,.934)--(-8.073,.88)--cycle;
\filldraw[fill opacity=0.8,fill=gray!20,draw=none](-8.075,.902)--(-7.667,1.033)--(-7.666,.988)--(-8.084,.854)--cycle;
\draw(-7.666,.988)--(-8.084,.854)--(-8.075,.902)--(-7.667,1.033);
\filldraw[fill opacity=0.8,fill=gray!20,draw=none](-7.667,1.033)--(-7.654,1.037)--(-7.666,.988)--cycle;
\draw(-7.667,1.033)--(-7.654,1.037);
\filldraw[fill opacity=0.8,fill=gray!20,draw=none](-3.271,2.12)--(-3.281,2.106)--(-3.277,2.079)--(-3.26,2.087)--cycle;
\draw(-3.281,2.106)--(-3.277,2.079);
\filldraw[fill opacity=0.8,fill=gray!20](-3.188,2.929)--(-3.192,2.924)--(-3.219,2.926)--(-3.188,2.929)--cycle;
\filldraw[fill opacity=0.8,fill=gray!20](-3.188,2.929)--(-3.163,2.925)--(-3.192,2.924)--(-3.188,2.929)--cycle;
\filldraw[fill opacity=0.8,fill=gray!20](-3.188,2.929)--(-3.131,2.936)--(-3.141,2.93)--(-3.188,2.929)--cycle;
\filldraw[fill opacity=0.8,fill=gray!20](-3.188,2.929)--(-3.141,2.93)--(-3.163,2.925)--(-3.188,2.929)--cycle;
\filldraw[fill opacity=0.8,fill=gray!20](-3.188,2.929)--(-3.219,2.926)--(-3.239,2.931)--(-3.188,2.929)--cycle;
\filldraw[fill opacity=0.8,fill=gray!20](-3.128,3.349)--(-3.157,3.361)--(-3.137,3.356)--(-3.09,3.34)--cycle;
\filldraw[fill opacity=0.8,fill=gray!20,draw=none](-7.561,1.673)--(-7.561,1.653)--(-7.523,1.674)--cycle;
\draw(-7.561,1.673)--(-7.561,1.653);
\filldraw[fill opacity=0.8,fill=gray!20,draw=none](-3.624,2.341)--(-3.655,2.34)--(-3.667,2.349)--cycle;
\draw(-3.655,2.34)--(-3.667,2.349);
\filldraw[fill opacity=0.8,fill=gray!20,draw=none](-3.216,2.202)--(-3.247,2.226)--(-3.238,2.198)--(-3.219,2.171)--cycle;
\filldraw[fill opacity=0.8,fill=gray!20,draw=none](-3.258,2.259)--(-3.263,2.261)--(-3.26,2.236)--(-3.247,2.226)--cycle;
\draw(-3.263,2.261)--(-3.26,2.236);
\filldraw[fill opacity=0.8,fill=gray!20,draw=none](-7.419,1.705)--(-7.409,1.706)--(-7.383,1.722)--(-7.411,1.712)--cycle;
\draw(-7.383,1.722)--(-7.411,1.712);
\filldraw[fill opacity=0.8,fill=gray!20,draw=none](-7.38,1.758)--(-7.389,1.755)--(-7.496,1.69)--(-7.495,1.682)--(-7.411,1.712)--cycle;
\draw(-7.38,1.758)--(-7.389,1.755);
\draw(-7.495,1.682)--(-7.411,1.712);
\filldraw[fill opacity=0.8,fill=gray!20,draw=none](-3.283,2.1)--(-3.281,2.106)--(-3.283,2.121)--cycle;
\draw(-3.281,2.106)--(-3.283,2.121);
\filldraw[fill opacity=0.8,fill=gray!20,draw=none](-3.529,2.312)--(-3.527,2.312)--(-3.505,2.311)--cycle;
\draw(-3.527,2.312)--(-3.505,2.311);
\filldraw[fill opacity=0.8,fill=gray!20,draw=none](-3.487,2.291)--(-3.566,2.332)--(-3.535,2.318)--cycle;
\draw(-3.566,2.332)--(-3.535,2.318);
\filldraw[fill opacity=0.8,fill=gray!20,draw=none](-3.611,2.338)--(-3.611,2.338)--(-3.597,2.336)--(-3.595,2.334)--cycle;
\draw(-3.597,2.336)--(-3.595,2.334);
\filldraw[fill opacity=0.8,fill=gray!20,draw=none](-3.557,2.324)--(-3.543,2.313)--(-3.595,2.334)--(-3.597,2.336)--cycle;
\draw(-3.557,2.324)--(-3.543,2.313);
\draw(-3.595,2.334)--(-3.597,2.336);
\filldraw[fill opacity=0.8,fill=gray!20,draw=none](-3.552,2.316)--(-3.525,2.293)--(-3.543,2.313)--(-3.557,2.324)--cycle;
\draw(-3.552,2.316)--(-3.525,2.293);
\draw(-3.543,2.313)--(-3.557,2.324);
\filldraw[fill opacity=0.8,fill=gray!20,draw=none](-3.535,2.318)--(-3.566,2.332)--(-3.624,2.341)--(-3.587,2.325)--cycle;
\draw(-3.535,2.318)--(-3.566,2.332);
\draw(-3.624,2.341)--(-3.587,2.325);
\filldraw[fill opacity=0.8,fill=gray!20,draw=none](-7.576,1.549)--(-7.561,1.55)--(-7.561,1.56)--cycle;
\draw(-7.576,1.549)--(-7.561,1.55)--(-7.561,1.56);
\filldraw[fill opacity=0.8,fill=gray!20,draw=none](-3.662,2.138)--(-3.607,2.092)--(-3.569,2.121)--(-3.618,2.162)--cycle;
\draw(-3.662,2.138)--(-3.607,2.092)--(-3.569,2.121)--(-3.618,2.162);
\filldraw[fill opacity=0.8,fill=gray!20,draw=none](-3.66,2.203)--(-3.661,2.198)--(-3.569,2.121)--(-3.535,2.159)--(-3.633,2.241)--cycle;
\draw(-3.661,2.198)--(-3.569,2.121)--(-3.535,2.159)--(-3.633,2.241);
\filldraw[fill opacity=0.8,fill=gray!20,draw=none](-3.683,2.156)--(-3.662,2.138)--(-3.618,2.162)--(-3.661,2.198)--cycle;
\draw(-3.683,2.156)--(-3.662,2.138);
\draw(-3.618,2.162)--(-3.661,2.198);
\filldraw[fill opacity=0.8,fill=gray!20,draw=none](-3.685,2.153)--(-3.687,2.148)--(-3.662,2.138)--(-3.683,2.156)--cycle;
\draw(-3.662,2.138)--(-3.683,2.156);
\filldraw[fill opacity=0.8,fill=gray!20,draw=none](-3.633,2.191)--(-3.771,2.251)--(-3.766,2.189)--(-3.643,2.135)--cycle;
\draw(-3.766,2.189)--(-3.643,2.135)--(-3.633,2.191)--(-3.771,2.251);
\filldraw[fill opacity=0.8,fill=gray!20,draw=none](-7.496,1.69)--(-7.389,1.755)--(-7.499,1.716)--cycle;
\draw(-7.389,1.755)--(-7.499,1.716);
\filldraw[fill opacity=0.8,fill=gray!20,draw=none](-7.549,1.692)--(-7.553,1.681)--(-7.544,1.689)--cycle;
\filldraw[fill opacity=0.8,fill=gray!20,draw=none](-7.592,1.636)--(-7.609,1.616)--(-7.561,1.634)--(-7.561,1.653)--cycle;
\draw(-7.561,1.634)--(-7.561,1.653);
\filldraw[fill opacity=0.8,fill=gray!20,draw=none](-7.549,1.692)--(-7.55,1.698)--(-7.555,1.696)--cycle;
\draw(-7.55,1.698)--(-7.555,1.696);
\filldraw[fill opacity=0.8,fill=gray!20,draw=none](-7.582,1.668)--(-7.553,1.681)--(-7.549,1.692)--(-7.556,1.724)--(-7.567,1.719)--cycle;
\draw(-7.582,1.668)--(-7.553,1.681);
\draw(-7.556,1.724)--(-7.567,1.719);
\filldraw[fill opacity=0.8,fill=gray!20,draw=none](-7.592,1.636)--(-7.561,1.653)--(-7.561,1.673)--cycle;
\draw(-7.561,1.653)--(-7.561,1.673);
\filldraw[fill opacity=0.8,fill=gray!20,draw=none](-7.592,1.636)--(-7.616,1.622)--(-7.616,1.613)--(-7.609,1.616)--cycle;
\draw(-7.616,1.622)--(-7.616,1.613);
\filldraw[fill opacity=0.8,fill=gray!20,draw=none](-7.436,1.703)--(-7.564,1.687)--(-7.636,1.609)--(-7.595,1.584)--(-7.545,1.579)--(-7.495,1.595)--(-7.452,1.631)--(-7.424,1.68)--(-7.421,1.698)--cycle;
\draw(-7.636,1.609)--(-7.595,1.584)--(-7.545,1.579)--(-7.495,1.595)--(-7.452,1.631)--(-7.424,1.68)--(-7.421,1.698);
\filldraw[fill opacity=0.8,fill=gray!20,draw=none](-7.98,.805)--(-7.979,.827)--(-7.975,.828)--(-7.962,.807)--(-7.963,.796)--cycle;
\draw(-7.979,.827)--(-7.975,.828);
\draw(-7.962,.807)--(-7.963,.796);
\filldraw[fill opacity=0.8,fill=gray!20,draw=none](-7.975,.828)--(-7.979,.827)--(-7.98,.834)--cycle;
\draw(-7.975,.828)--(-7.979,.827);
\filldraw[fill opacity=0.8,fill=gray!20,draw=none](-7.955,.789)--(-7.963,.796)--(-7.962,.807)--cycle;
\draw(-7.963,.796)--(-7.962,.807);
\filldraw[fill opacity=0.8,fill=gray!20,draw=none](-7.955,.789)--(-7.948,.771)--(-7.963,.772)--(-7.963,.796)--cycle;
\draw(-7.948,.771)--(-7.963,.772)--(-7.963,.796);
\filldraw[fill opacity=0.8,fill=gray!20,draw=none](-7.981,.771)--(-7.98,.805)--(-7.963,.796)--(-7.963,.772)--cycle;
\draw(-7.963,.796)--(-7.963,.772)--(-7.981,.771);
\filldraw[fill opacity=0.8,fill=gray!20,draw=none](-8.044,.716)--(-8.06,.767)--(-7.982,.771)--(-7.963,.768)--(-7.965,.72)--cycle;
\draw(-7.963,.768)--(-7.965,.72)--(-8.044,.716)--(-8.06,.767)--(-7.982,.771);
\filldraw[fill opacity=0.8,fill=gray!20,draw=none](-8.075,.803)--(-7.684,.929)--(-7.645,.905)--(-7.642,.89)--(-8.052,.758)--cycle;
\draw(-7.642,.89)--(-8.052,.758)--(-8.075,.803)--(-7.684,.929);
\filldraw[fill opacity=0.8,fill=gray!20](-3.131,2.936)--(-3.079,2.957)--(-3.097,2.945)--(-3.141,2.93)--cycle;
\filldraw[fill opacity=0.8,fill=gray!20,draw=none](-7.723,1.678)--(-7.723,1.653)--(-7.673,1.626)--cycle;
\draw(-7.723,1.678)--(-7.723,1.653);
\filldraw[fill opacity=0.8,fill=gray!20,draw=none](-7.988,1.479)--(-8.005,1.498)--(-7.96,1.483)--cycle;
\filldraw[fill opacity=0.8,fill=gray!20,draw=none](-7.988,1.479)--(-7.968,1.457)--(-8.027,1.461)--(-8.026,1.473)--cycle;
\draw(-7.968,1.457)--(-8.027,1.461)--(-8.026,1.473);
\filldraw[fill opacity=0.8,fill=gray!20,draw=none](-8.085,1.414)--(-8.105,1.456)--(-8.098,1.458)--(-8.027,1.461)--(-8.029,1.417)--cycle;
\draw(-8.098,1.458)--(-8.027,1.461)--(-8.029,1.417)--(-8.085,1.414)--(-8.105,1.456);
\filldraw[fill opacity=0.8,fill=gray!20,draw=none](-8.029,1.422)--(-8.027,1.461)--(-7.968,1.457)--cycle;
\draw(-8.029,1.422)--(-8.027,1.461)--(-7.968,1.457);
\filldraw[fill opacity=0.8,fill=gray!20,draw=none](-7.968,1.457)--(-7.988,1.479)--(-7.96,1.483)--(-7.942,1.477)--(-7.945,1.47)--cycle;
\draw(-7.942,1.477)--(-7.945,1.47);
\filldraw[fill opacity=0.8,fill=gray!20,draw=none](-8.081,1.465)--(-7.743,1.612)--(-7.723,1.587)--(-8.036,1.45)--cycle;
\draw(-7.723,1.587)--(-8.036,1.45)--(-8.081,1.465)--(-7.743,1.612);
\filldraw[fill opacity=0.8,fill=gray!20,draw=none](-3.258,2.259)--(-3.247,2.226)--(-3.158,2.157)--(-3.166,2.224)--cycle;
\draw(-3.158,2.157)--(-3.166,2.224);
\filldraw[fill opacity=0.8,fill=gray!20,draw=none](-7.441,.879)--(-7.44,.879)--(-7.435,.882)--cycle;
\draw(-7.44,.879)--(-7.435,.882);
\filldraw[fill opacity=0.8,fill=gray!20,draw=none](-7.436,.893)--(-7.425,.89)--(-7.403,.925)--(-7.424,.918)--cycle;
\draw(-7.425,.89)--(-7.403,.925)--(-7.424,.918);
\filldraw[fill opacity=0.8,fill=gray!20,draw=none](-7.37,.934)--(-7.451,.884)--(-7.436,.877)--(-7.364,.921)--cycle;
\draw(-7.37,.934)--(-7.451,.884)--(-7.436,.877)--(-7.364,.921);
\filldraw[fill opacity=0.8,fill=gray!20,draw=none](-7.436,1.703)--(-7.421,1.698)--(-7.419,1.705)--cycle;
\draw(-7.421,1.698)--(-7.419,1.705);
\filldraw[fill opacity=0.8,fill=gray!20,draw=none](-7.516,1.89)--(-7.508,1.909)--(-7.558,1.891)--(-7.527,1.875)--cycle;
\draw(-7.508,1.909)--(-7.558,1.891);
\filldraw[fill opacity=0.8,fill=gray!20,draw=none](-7.498,1.869)--(-7.516,1.907)--(-7.516,1.868)--cycle;
\draw(-7.516,1.907)--(-7.516,1.868);
\filldraw[fill opacity=0.8,fill=gray!20,draw=none](-3.271,2.12)--(-3.28,2.148)--(-3.283,2.121)--(-3.281,2.106)--cycle;
\draw(-3.283,2.121)--(-3.281,2.106);
\filldraw[fill opacity=0.8,fill=gray!20,draw=none](-7.513,1.136)--(-7.528,1.122)--(-7.513,1.131)--cycle;
\draw(-7.528,1.122)--(-7.513,1.131);
\filldraw[fill opacity=0.8,fill=gray!20,draw=none](-4.854,2.779)--(-5.412,2.434)--(-5.242,2.518)--(-5.144,2.57)--(-4.85,2.753)--cycle;
\draw(-4.854,2.779)--(-5.412,2.434);
\draw(-5.144,2.57)--(-4.85,2.753);
\filldraw[fill opacity=0.8,fill=gray!20,draw=none](-5.079,2.576)--(-6.565,2.048)--(-7.393,1.739)--(-7.411,1.712)--(-5.004,2.567)--cycle;
\draw(-5.079,2.576)--(-6.565,2.048);
\draw(-7.411,1.712)--(-5.004,2.567);
\filldraw[fill opacity=0.8,fill=gray!20,draw=none](-5.5,2.369)--(-5.697,2.247)--(-5.901,2.13)--(-5.656,2.287)--cycle;
\draw(-5.5,2.369)--(-5.697,2.247);
\filldraw[fill opacity=0.8,fill=gray!20,draw=none](-5.527,2.364)--(-5.472,2.387)--(-5.5,2.369)--(-5.527,2.355)--cycle;
\draw(-5.472,2.387)--(-5.5,2.369);
\filldraw[fill opacity=0.8,fill=gray!20,draw=none](-5.42,2.406)--(-5.5,2.369)--(-5.472,2.387)--cycle;
\draw(-5.5,2.369)--(-5.472,2.387);
\filldraw[fill opacity=0.8,fill=gray!20,draw=none](-4.895,2.589)--(-5.527,2.364)--(-5.472,2.387)--(-5.42,2.406)--(-4.768,2.638)--cycle;
\draw(-4.895,2.589)--(-5.527,2.364);
\draw(-5.42,2.406)--(-4.768,2.638);
\filldraw[fill opacity=0.8,fill=gray!20,draw=none](-5.527,2.364)--(-5.527,2.355)--(-5.656,2.287)--(-5.565,2.346)--(-5.551,2.354)--cycle;
\draw(-5.565,2.346)--(-5.551,2.354);
\filldraw[fill opacity=0.8,fill=gray!20,draw=none](-5.004,2.567)--(-5.464,2.403)--(-5.555,2.354)--(-4.895,2.589)--cycle;
\draw(-5.004,2.567)--(-5.464,2.403);
\draw(-5.555,2.354)--(-4.895,2.589);
\filldraw[fill opacity=0.8,fill=gray!20,draw=none](-5.527,2.364)--(-5.551,2.354)--(-5.526,2.37)--cycle;
\draw(-5.551,2.354)--(-5.526,2.37);
\filldraw[fill opacity=0.8,fill=gray!20,draw=none](-5.59,2.33)--(-7.456,1.174)--(-7.462,1.163)--(-5.412,2.434)--cycle;
\draw(-5.59,2.33)--(-7.456,1.174);
\draw(-7.462,1.163)--(-5.412,2.434);
\filldraw[fill opacity=0.8,fill=gray!20,draw=none](-3.268,2.299)--(-3.263,2.261)--(-3.166,2.224)--(-3.176,2.295)--cycle;
\draw(-3.166,2.224)--(-3.176,2.295)--(-3.268,2.299)--(-3.263,2.261);
\filldraw[fill opacity=0.8,fill=gray!20,draw=none](-3.529,2.312)--(-3.54,2.312)--(-3.543,2.313)--(-3.527,2.312)--cycle;
\draw(-3.543,2.313)--(-3.527,2.312);
\filldraw[fill opacity=0.8,fill=gray!20,draw=none](-7.651,1.042)--(-7.654,1.037)--(-7.656,1.037)--cycle;
\draw(-7.654,1.037)--(-7.656,1.037);
\filldraw[fill opacity=0.8,fill=gray!20,draw=none](-7.673,1.851)--(-7.673,1.828)--(-7.641,1.845)--cycle;
\draw(-7.673,1.851)--(-7.673,1.828);
\filldraw[fill opacity=0.8,fill=gray!20,draw=none](-7.694,1.822)--(-7.641,1.845)--(-7.661,1.871)--(-7.723,1.844)--cycle;
\draw(-7.694,1.822)--(-7.641,1.845);
\draw(-7.661,1.871)--(-7.723,1.844);
\filldraw[fill opacity=0.8,fill=gray!20,draw=none](-7.673,1.564)--(-7.673,1.548)--(-7.616,1.547)--(-7.616,1.572)--cycle;
\draw(-7.673,1.564)--(-7.673,1.548)--(-7.616,1.547)--(-7.616,1.572);
\filldraw[fill opacity=0.8,fill=gray!20,draw=none](-7.365,.894)--(-7.365,.878)--(-7.364,.886)--cycle;
\draw(-7.365,.894)--(-7.365,.878);
\filldraw[fill opacity=0.8,fill=gray!20,draw=none](-3.516,2.053)--(-3.524,2.05)--(-3.531,2.053)--cycle;
\draw(-3.524,2.05)--(-3.531,2.053);
\filldraw[fill opacity=0.8,fill=gray!20,draw=none](-3.755,2.153)--(-3.688,2.097)--(-3.671,2.079)--(-3.713,2.115)--cycle;
\draw(-3.755,2.153)--(-3.688,2.097)--(-3.671,2.079)--(-3.713,2.115);
\filldraw[fill opacity=0.8,fill=gray!20,draw=none](-7.606,.833)--(-7.621,.852)--(-7.635,.86)--(-7.635,.832)--cycle;
\draw(-7.635,.86)--(-7.635,.832);
\filldraw[fill opacity=0.8,fill=gray!20,draw=none](-7.621,.852)--(-7.635,.87)--(-7.635,.86)--cycle;
\draw(-7.635,.87)--(-7.635,.86);
\filldraw[fill opacity=0.8,fill=gray!20,draw=none](-7.63,.857)--(-7.724,.864)--(-7.642,.89)--cycle;
\draw(-7.724,.864)--(-7.642,.89);
\filldraw[fill opacity=0.8,fill=gray!20](-3.079,2.957)--(-3.033,2.99)--(-3.059,2.973)--(-3.097,2.945)--cycle;
\filldraw[fill opacity=0.8,fill=gray!20,draw=none](-3.255,2.143)--(-3.271,2.12)--(-3.26,2.087)--(-3.225,2.102)--cycle;
\filldraw[fill opacity=0.8,fill=gray!20,draw=none](-7.751,1.854)--(-7.723,1.855)--(-7.723,1.907)--cycle;
\draw(-7.723,1.855)--(-7.723,1.907);
\filldraw[fill opacity=0.8,fill=gray!20,draw=none](-7.649,1.074)--(-7.65,1.045)--(-7.65,1.046)--(-7.65,1.067)--cycle;
\draw(-7.65,1.046)--(-7.65,1.067);
\filldraw[fill opacity=0.8,fill=gray!20,draw=none](-7.65,1.064)--(-7.65,1.046)--(-7.642,1.072)--cycle;
\draw(-7.65,1.064)--(-7.65,1.046);
\filldraw[fill opacity=0.8,fill=gray!20,draw=none](-7.637,1.069)--(-7.651,1.042)--(-7.656,1.037)--(-7.684,1.028)--cycle;
\draw(-7.656,1.037)--(-7.684,1.028);
\filldraw[fill opacity=0.8,fill=gray!20](-3.049,3.31)--(-3.09,3.34)--(-3.079,3.328)--(-3.033,3.293)--cycle;
\filldraw[fill opacity=0.8,fill=gray!20,draw=none](-3.378,3.04)--(-3.377,3.038)--(-3.358,3.018)--(-3.378,3.067)--(-3.4,3.091)--(-3.381,3.046)--cycle;
\draw(-3.377,3.038)--(-3.358,3.018)--(-3.378,3.067)--(-3.4,3.091)--(-3.381,3.046);
\filldraw[fill opacity=0.8,fill=gray!20,draw=none](-3.313,2.973)--(-3.311,2.976)--(-3.333,2.985)--(-3.327,2.977)--cycle;
\draw(-3.333,2.985)--(-3.327,2.977)--(-3.313,2.973);
\filldraw[fill opacity=0.8,fill=gray!20,draw=none](-3.311,2.976)--(-3.306,2.986)--(-3.333,2.985)--cycle;
\filldraw[fill opacity=0.8,fill=gray!20,draw=none](-4.16,2.817)--(-3.178,3.017)--(-3.156,3.016)--(-4.07,2.831)--cycle;
\draw(-4.16,2.817)--(-3.178,3.017)--(-3.156,3.016)--(-4.07,2.831);
\filldraw[fill opacity=0.8,fill=gray!20,draw=none](-7.651,1.042)--(-7.637,1.069)--(-7.628,1.076)--(-7.623,1.078)--cycle;
\draw(-7.628,1.076)--(-7.623,1.078);
\filldraw[fill opacity=0.8,fill=gray!20,draw=none](-7.453,1.007)--(-7.447,1.009)--(-7.489,1.049)--(-7.526,1.037)--cycle;
\draw(-7.453,1.007)--(-7.447,1.009);
\draw(-7.489,1.049)--(-7.526,1.037);
\filldraw[fill opacity=0.8,fill=gray!20,draw=none](-7.568,1.045)--(-7.55,1.014)--(-7.55,1.06)--cycle;
\draw(-7.55,1.014)--(-7.55,1.06);
\filldraw[fill opacity=0.8,fill=gray!20,draw=none](-7.528,1.021)--(-7.491,.995)--(-7.453,1.007)--(-7.526,1.037)--(-7.527,1.036)--cycle;
\draw(-7.491,.995)--(-7.453,1.007);
\draw(-7.526,1.037)--(-7.527,1.036);
\filldraw[fill opacity=0.8,fill=gray!20,draw=none](-7.515,1.044)--(-7.615,1.088)--(-7.651,1.043)--(-7.65,1.042)--(-7.543,.995)--cycle;
\draw(-7.65,1.042)--(-7.543,.995)--(-7.515,1.044)--(-7.615,1.088);
\filldraw[fill opacity=0.8,fill=gray!20,draw=none](-7.549,1.692)--(-7.535,1.733)--(-7.556,1.724)--cycle;
\draw(-7.535,1.733)--(-7.556,1.724);
\filldraw[fill opacity=0.8,fill=gray!20,draw=none](-7.549,1.692)--(-7.544,1.689)--(-7.538,1.702)--(-7.55,1.698)--cycle;
\draw(-7.538,1.702)--(-7.55,1.698);
\filldraw[fill opacity=0.8,fill=gray!20,draw=none](-7.548,1.727)--(-7.535,1.733)--(-7.553,1.779)--(-7.566,1.773)--cycle;
\draw(-7.548,1.727)--(-7.535,1.733);
\draw(-7.553,1.779)--(-7.566,1.773);
\filldraw[fill opacity=0.8,fill=gray!20,draw=none](-7.536,1.706)--(-7.543,1.749)--(-7.546,1.748)--(-7.55,1.698)--(-7.538,1.702)--cycle;
\draw(-7.543,1.749)--(-7.546,1.748);
\draw(-7.55,1.698)--(-7.538,1.702);
\filldraw[fill opacity=0.8,fill=gray!20,draw=none](-7.536,1.706)--(-7.538,1.702)--(-7.535,1.703)--cycle;
\draw(-7.538,1.702)--(-7.535,1.703);
\filldraw[fill opacity=0.8,fill=gray!20,draw=none](-7.507,1.749)--(-7.536,1.706)--(-7.535,1.703)--(-7.489,1.72)--cycle;
\draw(-7.535,1.703)--(-7.489,1.72);
\filldraw[fill opacity=0.8,fill=gray!20,draw=none](-7.536,1.706)--(-7.507,1.749)--(-7.514,1.759)--(-7.543,1.749)--cycle;
\draw(-7.514,1.759)--(-7.543,1.749);
\filldraw[fill opacity=0.8,fill=gray!20,draw=none](-7.524,1.731)--(-7.564,1.687)--(-7.436,1.703)--cycle;
\filldraw[fill opacity=0.8,fill=gray!20,draw=none](-7.65,1.045)--(-7.65,1.045)--(-7.65,1.042)--cycle;
\filldraw[fill opacity=0.8,fill=gray!20,draw=none](-3.332,2.274)--(-3.319,2.248)--(-3.295,2.22)--(-3.3,2.255)--cycle;
\draw(-3.295,2.22)--(-3.3,2.255);
\filldraw[fill opacity=0.8,fill=gray!20,draw=none](-4.205,2.873)--(-3.38,3.04)--(-3.383,2.999)--(-4.169,2.84)--cycle;
\draw(-4.205,2.873)--(-3.38,3.04);
\draw(-3.383,2.999)--(-4.169,2.84);
\filldraw[fill opacity=0.8,fill=gray!20,draw=none](-3.294,2.956)--(-3.286,2.947)--(-3.312,2.966)--cycle;
\draw(-3.294,2.956)--(-3.286,2.947)--(-3.312,2.966);
\filldraw[fill opacity=0.8,fill=gray!20,draw=none](-7.527,1.875)--(-7.532,1.869)--(-7.516,1.868)--cycle;
\filldraw[fill opacity=0.8,fill=gray!20,draw=none](-7.523,1.873)--(-7.558,1.891)--(-7.582,1.883)--(-7.526,1.868)--cycle;
\draw(-7.558,1.891)--(-7.582,1.883);
\filldraw[fill opacity=0.8,fill=gray!20,draw=none](-3.611,2.338)--(-3.614,2.35)--(-3.597,2.336)--cycle;
\draw(-3.614,2.35)--(-3.597,2.336);
\filldraw[fill opacity=0.8,fill=gray!20,draw=none](-7.65,1.045)--(-7.65,1.042)--(-7.65,1.046)--cycle;
\draw(-7.65,1.042)--(-7.65,1.046);
\filldraw[fill opacity=0.8,fill=gray!20,draw=none](-3.351,3.006)--(-3.333,2.985)--(-3.358,3.018)--(-3.377,3.038)--cycle;
\draw(-3.333,2.985)--(-3.358,3.018)--(-3.377,3.038);
\filldraw[fill opacity=0.8,fill=gray!20,draw=none](-7.635,.975)--(-7.635,.937)--(-7.6,.91)--(-7.6,1.02)--cycle;
\draw(-7.635,.975)--(-7.635,.937);
\draw(-7.6,.91)--(-7.6,1.02);
\filldraw[fill opacity=0.8,fill=gray!20,draw=none](-7.543,.995)--(-7.65,1.042)--(-7.635,.975)--(-7.553,.939)--cycle;
\draw(-7.635,.975)--(-7.553,.939)--(-7.543,.995)--(-7.65,1.042);
\filldraw[fill opacity=0.8,fill=gray!20,draw=none](-7.634,1.075)--(-7.628,1.076)--(-7.637,1.069)--cycle;
\draw(-7.634,1.075)--(-7.628,1.076);
\filldraw[fill opacity=0.8,fill=gray!20,draw=none](-7.642,1.072)--(-7.634,1.075)--(-7.637,1.069)--(-7.645,1.061)--cycle;
\draw(-7.642,1.072)--(-7.634,1.075);
\filldraw[fill opacity=0.8,fill=gray!20,draw=none](-7.635,1.08)--(-7.621,1.09)--(-7.608,1.094)--(-7.634,1.075)--(-7.642,1.072)--cycle;
\draw(-7.621,1.09)--(-7.608,1.094);
\draw(-7.634,1.075)--(-7.642,1.072);
\filldraw[fill opacity=0.8,fill=gray!20,draw=none](-7.65,1.046)--(-7.65,1.042)--(-7.635,.975)--(-7.635,1.097)--cycle;
\draw(-7.65,1.046)--(-7.65,1.042);
\draw(-7.635,.975)--(-7.635,1.097);
\filldraw[fill opacity=0.8,fill=gray!20,draw=none](-7.616,1.593)--(-7.616,1.572)--(-7.591,1.578)--cycle;
\draw(-7.616,1.593)--(-7.616,1.572);
\filldraw[fill opacity=0.8,fill=gray!20,draw=none](-7.431,.957)--(-7.431,.934)--(-7.424,.918)--(-7.403,.925)--(-7.395,.973)--(-7.428,.962)--cycle;
\draw(-7.424,.918)--(-7.403,.925)--(-7.395,.973)--(-7.428,.962);
\filldraw[fill opacity=0.8,fill=gray!20,draw=none](-7.393,.932)--(-7.414,.938)--(-7.424,.918)--(-7.417,.905)--(-7.387,.924)--cycle;
\draw(-7.417,.905)--(-7.387,.924);
\filldraw[fill opacity=0.8,fill=gray!20,draw=none](-7.393,.924)--(-7.393,.836)--(-7.365,.878)--(-7.365,.923)--cycle;
\draw(-7.393,.924)--(-7.393,.836);
\draw(-7.365,.878)--(-7.365,.923);
\filldraw[fill opacity=0.8,fill=gray!20,draw=none](-3.255,2.143)--(-3.278,2.174)--(-3.28,2.148)--(-3.271,2.12)--cycle;
\filldraw[fill opacity=0.8,fill=gray!20,draw=none](-4.055,2.74)--(-3.557,2.324)--(-3.597,2.336)--(-4.161,2.807)--cycle;
\draw(-4.055,2.74)--(-3.557,2.324);
\draw(-3.597,2.336)--(-4.161,2.807);
\filldraw[fill opacity=0.8,fill=gray!20,draw=none](-7.499,1.922)--(-7.516,1.915)--(-7.516,1.907)--(-7.514,1.902)--cycle;
\draw(-7.499,1.922)--(-7.516,1.915)--(-7.516,1.907);
\filldraw[fill opacity=0.8,fill=gray!20,draw=none](-3.166,2.224)--(-3.151,2.106)--(-3.048,2.13)--(-3.056,2.197)--cycle;
\draw(-3.166,2.224)--(-3.151,2.106);
\draw(-3.048,2.13)--(-3.056,2.197);
\filldraw[fill opacity=0.8,fill=gray!20,draw=none](-7.606,.833)--(-7.635,.832)--(-7.635,.796)--(-7.6,.787)--(-7.6,.825)--cycle;
\draw(-7.635,.832)--(-7.635,.796)--(-7.6,.787)--(-7.6,.825);
\filldraw[fill opacity=0.8,fill=gray!20,draw=none](-7.425,.89)--(-7.427,.887)--(-7.435,.882)--cycle;
\draw(-7.425,.89)--(-7.427,.887)--(-7.435,.882);
\filldraw[fill opacity=0.8,fill=gray!20,draw=none](-7.441,.879)--(-7.435,.882)--(-7.427,.887)--(-7.449,.88)--cycle;
\draw(-7.435,.882)--(-7.427,.887)--(-7.449,.88);
\filldraw[fill opacity=0.8,fill=gray!20,draw=none](-7.301,.972)--(-7.297,.98)--(-7.37,.934)--(-7.364,.921)--(-7.327,.944)--cycle;
\draw(-7.297,.98)--(-7.37,.934);
\draw(-7.364,.921)--(-7.327,.944);
\filldraw[fill opacity=0.8,fill=gray!20,draw=none](-3.302,2.228)--(-3.319,2.248)--(-3.296,2.2)--cycle;
\filldraw[fill opacity=0.8,fill=gray!20,draw=none](-2.914,2.904)--(-2.898,2.779)--(-2.818,2.683)--(-2.851,2.942)--cycle;
\draw(-2.818,2.683)--(-2.851,2.942)--(-2.914,2.904)--(-2.898,2.779);
\filldraw[fill opacity=0.8,fill=gray!20,draw=none](-7.65,1.093)--(-7.65,1.064)--(-7.642,1.072)--(-7.637,1.09)--cycle;
\draw(-7.65,1.093)--(-7.65,1.064);
\filldraw[fill opacity=0.8,fill=gray!20](-3.378,3.067)--(-3.385,3.122)--(-3.407,3.146)--(-3.4,3.091)--cycle;
\filldraw[fill opacity=0.8,fill=gray!20](-3.033,2.99)--(-2.998,3.034)--(-3.029,3.014)--(-3.059,2.973)--cycle;
\filldraw[fill opacity=0.8,fill=gray!20,draw=none](-3.512,2.006)--(-3.587,2.039)--(-3.524,2.05)--(-3.462,2.023)--cycle;
\draw(-3.524,2.05)--(-3.462,2.023)--(-3.512,2.006)--(-3.587,2.039);
\filldraw[fill opacity=0.8,fill=gray!20,draw=none](-7.668,1.603)--(-7.673,1.604)--(-7.673,1.602)--cycle;
\draw(-7.673,1.604)--(-7.673,1.602);
\filldraw[fill opacity=0.8,fill=gray!20,draw=none](-3.28,2.148)--(-3.278,2.174)--(-3.292,2.193)--(-3.29,2.178)--cycle;
\draw(-3.292,2.193)--(-3.29,2.178);
\filldraw[fill opacity=0.8,fill=gray!20,draw=none](-7.556,.838)--(-7.585,.835)--(-7.55,.823)--(-7.55,.836)--cycle;
\draw(-7.55,.823)--(-7.55,.836);
\filldraw[fill opacity=0.8,fill=gray!20,draw=none](-7.556,.838)--(-7.55,.836)--(-7.55,.838)--cycle;
\draw(-7.55,.836)--(-7.55,.838);
\filldraw[fill opacity=0.8,fill=gray!20,draw=none](-7.538,.84)--(-7.55,.838)--(-7.55,.836)--(-7.546,.835)--cycle;
\draw(-7.55,.838)--(-7.55,.836);
\filldraw[fill opacity=0.8,fill=gray!20,draw=none](-7.551,.845)--(-7.543,.848)--(-7.527,.843)--(-7.558,.833)--cycle;
\draw(-7.551,.845)--(-7.543,.848);
\draw(-7.527,.843)--(-7.558,.833);
\filldraw[fill opacity=0.8,fill=gray!20,draw=none](-7.641,1.607)--(-7.673,1.626)--(-7.673,1.604)--(-7.668,1.603)--cycle;
\draw(-7.673,1.626)--(-7.673,1.604);
\filldraw[fill opacity=0.8,fill=gray!20,draw=none](-4.011,2.397)--(-3.766,2.192)--(-3.753,2.151)--(-4.013,2.369)--cycle;
\draw(-4.011,2.397)--(-3.766,2.192);
\draw(-3.753,2.151)--(-4.013,2.369);
\filldraw[fill opacity=0.8,fill=gray!20,draw=none](-7.61,1.086)--(-7.623,1.078)--(-7.628,1.076)--cycle;
\draw(-7.623,1.078)--(-7.628,1.076);
\filldraw[fill opacity=0.8,fill=gray!20,draw=none](-7.641,1.607)--(-7.653,1.6)--(-7.616,1.593)--cycle;
\filldraw[fill opacity=0.8,fill=gray!20,draw=none](-3.428,2.052)--(-3.462,2.023)--(-3.524,2.05)--(-3.516,2.053)--cycle;
\draw(-3.428,2.052)--(-3.462,2.023)--(-3.524,2.05);
\filldraw[fill opacity=0.8,fill=gray!20,draw=none](-7.414,.938)--(-7.422,.94)--(-7.474,.908)--(-7.451,.884)--(-7.436,.893)--cycle;
\draw(-7.422,.94)--(-7.474,.908)--(-7.451,.884)--(-7.436,.893);
\filldraw[fill opacity=0.8,fill=gray!20,draw=none](-7.447,1.008)--(-7.45,.976)--(-7.438,.971)--(-7.438,.981)--cycle;
\draw(-7.438,.971)--(-7.438,.981);
\filldraw[fill opacity=0.8,fill=gray!20,draw=none](-7.431,.957)--(-7.443,.98)--(-7.501,.944)--(-7.474,.908)--(-7.431,.934)--cycle;
\draw(-7.443,.98)--(-7.501,.944)--(-7.474,.908)--(-7.431,.934);
\filldraw[fill opacity=0.8,fill=gray!20,draw=none](-7.494,.931)--(-7.494,.927)--(-7.47,.894)--(-7.458,.9)--(-7.447,.911)--(-7.438,.94)--(-7.438,.945)--cycle;
\draw(-7.494,.931)--(-7.494,.927);
\draw(-7.438,.94)--(-7.438,.945);
\filldraw[fill opacity=0.8,fill=gray!20,draw=none](-7.458,.9)--(-7.47,.894)--(-7.467,.89)--cycle;
\filldraw[fill opacity=0.8,fill=gray!20,draw=none](-7.449,.88)--(-7.427,.887)--(-7.425,.89)--(-7.471,.903)--(-7.484,.899)--cycle;
\draw(-7.449,.88)--(-7.427,.887)--(-7.425,.89);
\draw(-7.471,.903)--(-7.484,.899);
\filldraw[fill opacity=0.8,fill=gray!20](-3.219,2.926)--(-3.249,2.938)--(-3.286,2.947)--(-3.239,2.931)--cycle;
\filldraw[fill opacity=0.8,fill=gray!20,draw=none](-7.65,1.604)--(-7.641,1.607)--(-7.641,1.612)--cycle;
\draw(-7.65,1.604)--(-7.641,1.607);
\filldraw[fill opacity=0.8,fill=gray!20,draw=none](-3.713,2.116)--(-3.713,2.115)--(-3.671,2.079)--(-3.643,2.078)--(-3.695,2.121)--cycle;
\draw(-3.713,2.115)--(-3.671,2.079)--(-3.643,2.078)--(-3.695,2.121);
\filldraw[fill opacity=0.8,fill=gray!20,draw=none](-3.687,2.148)--(-3.695,2.121)--(-3.643,2.078)--(-3.607,2.092)--(-3.662,2.138)--cycle;
\draw(-3.695,2.121)--(-3.643,2.078)--(-3.607,2.092)--(-3.662,2.138);
\filldraw[fill opacity=0.8,fill=gray!20,draw=none](-3.643,2.135)--(-3.765,2.188)--(-3.74,2.141)--(-3.717,2.118)--(-3.633,2.081)--cycle;
\draw(-3.717,2.118)--(-3.633,2.081)--(-3.643,2.135)--(-3.765,2.188);
\filldraw[fill opacity=0.8,fill=gray!20,draw=none](-3.296,2.2)--(-3.29,2.178)--(-3.292,2.193)--cycle;
\draw(-3.29,2.178)--(-3.292,2.193);
\filldraw[fill opacity=0.8,fill=gray!20,draw=none](-7.65,1.166)--(-7.65,1.093)--(-7.637,1.09)--(-7.635,1.097)--(-7.635,1.154)--cycle;
\draw(-7.635,1.097)--(-7.635,1.154)--(-7.65,1.166)--(-7.65,1.093);
\filldraw[fill opacity=0.8,fill=gray!20,draw=none](-3.498,2.053)--(-3.516,2.053)--(-3.465,2.071)--cycle;
\filldraw[fill opacity=0.8,fill=gray!20](-3.298,3.331)--(-3.245,3.351)--(-3.236,3.357)--(-3.28,3.342)--cycle;
\filldraw[fill opacity=0.8,fill=gray!20,draw=none](-3.302,2.228)--(-3.296,2.2)--(-3.292,2.193)--(-3.295,2.22)--cycle;
\draw(-3.292,2.193)--(-3.295,2.22);
\filldraw[fill opacity=0.8,fill=gray!20](-3.141,2.93)--(-3.097,2.945)--(-3.139,2.937)--(-3.163,2.925)--cycle;
\filldraw[fill opacity=0.8,fill=gray!20,draw=none](-3.275,2.204)--(-3.295,2.22)--(-3.292,2.193)--(-3.278,2.174)--cycle;
\draw(-3.295,2.22)--(-3.292,2.193);
\filldraw[fill opacity=0.8,fill=gray!20,draw=none](-7.608,1.094)--(-7.588,1.1)--(-7.61,1.086)--(-7.628,1.076)--(-7.634,1.075)--cycle;
\draw(-7.608,1.094)--(-7.588,1.1);
\draw(-7.628,1.076)--(-7.634,1.075);
\filldraw[fill opacity=0.8,fill=gray!20,draw=none](-7.508,1.909)--(-7.514,1.902)--(-7.512,1.899)--cycle;
\filldraw[fill opacity=0.8,fill=gray!20,draw=none](-7.527,1.875)--(-7.516,1.868)--(-7.516,1.89)--cycle;
\draw(-7.516,1.868)--(-7.516,1.89);
\filldraw[fill opacity=0.8,fill=gray!20,draw=none](-4.656,2.905)--(-4.663,2.898)--(-4.651,2.906)--cycle;
\draw(-4.663,2.898)--(-4.651,2.906);
\filldraw[fill opacity=0.8,fill=gray!20,draw=none](-4.73,2.857)--(-4.745,2.848)--(-4.739,2.821)--(-4.69,2.852)--cycle;
\draw(-4.73,2.857)--(-4.745,2.848);
\draw(-4.739,2.821)--(-4.69,2.852);
\filldraw[fill opacity=0.8,fill=gray!20,draw=none](-4.654,2.915)--(-4.654,2.906)--(-4.66,2.901)--(-4.661,2.901)--cycle;
\filldraw[fill opacity=0.8,fill=gray!20,draw=none](-4.438,2.89)--(-4.493,2.944)--(-4.491,2.939)--(-4.484,2.926)--(-4.478,2.914)--(-4.469,2.898)--cycle;
\draw(-4.493,2.944)--(-4.491,2.939);
\draw(-4.478,2.914)--(-4.469,2.898);
\filldraw[fill opacity=0.8,fill=gray!20,draw=none](-4.438,2.89)--(-4.48,2.936)--(-4.489,2.942)--(-4.493,2.944)--cycle;
\draw(-4.48,2.936)--(-4.489,2.942);
\filldraw[fill opacity=0.8,fill=gray!20,draw=none](-4.665,2.897)--(-4.681,2.905)--(-4.66,2.901)--cycle;
\draw(-4.665,2.897)--(-4.681,2.905);
\filldraw[fill opacity=0.8,fill=gray!20,draw=none](-4.663,2.896)--(-4.665,2.897)--(-4.662,2.899)--cycle;
\draw(-4.663,2.896)--(-4.665,2.897);
\filldraw[fill opacity=0.8,fill=gray!20,draw=none](-4.664,2.877)--(-4.673,2.863)--(-4.665,2.868)--cycle;
\draw(-4.673,2.863)--(-4.665,2.868);
\filldraw[fill opacity=0.8,fill=gray!20,draw=none](-4.665,2.868)--(-4.673,2.888)--(-4.665,2.897)--(-4.663,2.896)--cycle;
\draw(-4.665,2.897)--(-4.663,2.896);
\filldraw[fill opacity=0.8,fill=gray!20,draw=none](-4.673,2.888)--(-4.68,2.904)--(-4.665,2.897)--cycle;
\draw(-4.68,2.904)--(-4.665,2.897);
\filldraw[fill opacity=0.8,fill=gray!20,draw=none](-4.649,2.901)--(-4.652,2.897)--(-4.643,2.881)--(-4.643,2.881)--cycle;
\draw(-4.643,2.881)--(-4.643,2.881);
\filldraw[fill opacity=0.8,fill=gray!20,draw=none](-4.647,2.872)--(-4.642,2.855)--(-4.664,2.866)--(-4.665,2.868)--(-4.664,2.877)--cycle;
\draw(-4.642,2.855)--(-4.664,2.866);
\filldraw[fill opacity=0.8,fill=gray!20,draw=none](-4.652,2.897)--(-4.664,2.877)--(-4.665,2.868)--(-4.643,2.881)--cycle;
\draw(-4.665,2.868)--(-4.643,2.881);
\filldraw[fill opacity=0.8,fill=gray!20,draw=none](-4.632,2.841)--(-4.633,2.842)--(-4.649,2.854)--(-4.649,2.858)--(-4.642,2.855)--cycle;
\draw(-4.649,2.858)--(-4.642,2.855);
\filldraw[fill opacity=0.8,fill=gray!20,draw=none](-4.649,2.854)--(-4.664,2.866)--(-4.649,2.858)--cycle;
\draw(-4.664,2.866)--(-4.649,2.858);
\filldraw[fill opacity=0.8,fill=gray!20,draw=none](-4.649,2.854)--(-4.648,2.856)--(-4.657,2.873)--(-4.666,2.867)--cycle;
\draw(-4.657,2.873)--(-4.666,2.867);
\filldraw[fill opacity=0.8,fill=gray!20,draw=none](-4.665,2.868)--(-4.665,2.866)--(-4.683,2.875)--(-4.684,2.877)--(-4.673,2.888)--cycle;
\draw(-4.665,2.866)--(-4.683,2.875);
\filldraw[fill opacity=0.8,fill=gray!20,draw=none](-4.658,2.845)--(-4.649,2.854)--(-4.666,2.867)--(-4.69,2.852)--cycle;
\draw(-4.666,2.867)--(-4.69,2.852);
\filldraw[fill opacity=0.8,fill=gray!20,draw=none](-4.643,2.847)--(-4.637,2.861)--(-4.643,2.881)--(-4.657,2.873)--cycle;
\draw(-4.643,2.881)--(-4.657,2.873);
\filldraw[fill opacity=0.8,fill=gray!20,draw=none](-4.649,2.854)--(-4.658,2.845)--(-4.645,2.842)--(-4.643,2.847)--(-4.645,2.851)--cycle;
\filldraw[fill opacity=0.8,fill=gray!20,draw=none](-4.649,2.854)--(-4.65,2.853)--(-4.683,2.875)--(-4.664,2.866)--cycle;
\draw(-4.683,2.875)--(-4.664,2.866);
\filldraw[fill opacity=0.8,fill=gray!20,draw=none](-4.504,2.969)--(-4.504,2.971)--(-4.529,2.964)--(-4.518,2.959)--(-4.508,2.965)--cycle;
\draw(-4.504,2.969)--(-4.504,2.971);
\draw(-4.518,2.959)--(-4.508,2.965);
\filldraw[fill opacity=0.8,fill=gray!20,draw=none](-4.527,2.964)--(-4.56,2.932)--(-4.512,2.967)--(-4.5,2.978)--cycle;
\filldraw[fill opacity=0.8,fill=gray!20,draw=none](-7.398,1.146)--(-7.405,1.145)--(-7.416,1.115)--(-7.393,1.1)--(-7.393,1.139)--cycle;
\draw(-7.398,1.146)--(-7.405,1.145);
\draw(-7.393,1.1)--(-7.393,1.139);
\filldraw[fill opacity=0.8,fill=gray!20,draw=none](-4.438,2.89)--(-4.542,2.876)--(-4.554,2.86)--(-4.558,2.837)--(-4.544,2.822)--(-4.513,2.817)--(-4.471,2.824)--(-4.444,2.833)--cycle;
\draw(-4.542,2.876)--(-4.554,2.86)--(-4.558,2.837)--(-4.544,2.822)--(-4.513,2.817)--(-4.471,2.824)--(-4.444,2.833);
\filldraw[fill opacity=0.8,fill=gray!20,draw=none](-4.601,2.834)--(-4.596,2.836)--(-4.598,2.833)--cycle;
\draw(-4.596,2.836)--(-4.598,2.833);
\filldraw[fill opacity=0.8,fill=gray!20,draw=none](-4.596,2.836)--(-4.562,2.835)--(-4.561,2.834)--(-4.592,2.83)--(-4.601,2.834)--cycle;
\draw(-4.592,2.83)--(-4.601,2.834);
\filldraw[fill opacity=0.8,fill=gray!20,draw=none](-4.65,2.853)--(-4.65,2.852)--(-4.68,2.872)--(-4.683,2.875)--cycle;
\filldraw[fill opacity=0.8,fill=gray!20,draw=none](-4.501,2.961)--(-4.508,2.965)--(-4.56,2.932)--(-4.591,2.91)--(-4.627,2.844)--(-4.537,2.9)--cycle;
\draw(-4.508,2.965)--(-4.56,2.932);
\draw(-4.627,2.844)--(-4.537,2.9);
\filldraw[fill opacity=0.8,fill=gray!20,draw=none](-4.56,2.932)--(-4.576,2.909)--(-4.512,2.967)--cycle;
\draw(-4.56,2.932)--(-4.576,2.909);
\filldraw[fill opacity=0.8,fill=gray!20,draw=none](-4.627,2.844)--(-4.576,2.909)--(-4.56,2.932)--cycle;
\draw(-4.576,2.909)--(-4.56,2.932);
\filldraw[fill opacity=0.8,fill=gray!20,draw=none](-4.637,2.872)--(-4.637,2.888)--(-4.637,2.893)--cycle;
\filldraw[fill opacity=0.8,fill=gray!20,draw=none](-4.504,2.971)--(-4.504,2.969)--(-4.501,2.961)--cycle;
\draw(-4.504,2.971)--(-4.504,2.969);
\filldraw[fill opacity=0.8,fill=gray!20,draw=none](-4.491,2.939)--(-4.501,2.961)--(-4.537,2.9)--(-4.52,2.91)--cycle;
\draw(-4.491,2.939)--(-4.501,2.961);
\draw(-4.537,2.9)--(-4.52,2.91);
\filldraw[fill opacity=0.8,fill=gray!20,draw=none](-4.506,2.952)--(-4.501,2.949)--(-4.5,2.961)--cycle;
\draw(-4.506,2.952)--(-4.501,2.949);
\filldraw[fill opacity=0.8,fill=gray!20,draw=none](-4.489,2.942)--(-4.495,2.946)--(-4.501,2.949)--(-4.506,2.952)--cycle;
\draw(-4.489,2.942)--(-4.495,2.946);
\draw(-4.501,2.949)--(-4.506,2.952);
\filldraw[fill opacity=0.8,fill=gray!20,draw=none](-4.483,2.95)--(-4.495,2.946)--(-4.48,2.936)--cycle;
\draw(-4.495,2.946)--(-4.48,2.936);
\filldraw[fill opacity=0.8,fill=gray!20,draw=none](-4.488,2.982)--(-7.508,1.909)--(-7.523,1.873)--(-7.51,1.867)--(-4.482,2.943)--cycle;
\draw(-7.51,1.867)--(-4.482,2.943)--(-4.488,2.982)--(-7.508,1.909);
\filldraw[fill opacity=0.8,fill=gray!20,draw=none](-7.621,.852)--(-7.606,.833)--(-7.6,.833)--(-7.6,.84)--cycle;
\draw(-7.6,.833)--(-7.6,.84);
\filldraw[fill opacity=0.8,fill=gray!20,draw=none](-7.585,.835)--(-7.6,.833)--(-7.6,.787)--(-7.55,.78)--(-7.55,.823)--cycle;
\draw(-7.6,.833)--(-7.6,.787)--(-7.55,.78)--(-7.55,.823);
\filldraw[fill opacity=0.8,fill=gray!20,draw=none](-7.606,.833)--(-7.6,.825)--(-7.6,.833)--cycle;
\draw(-7.6,.825)--(-7.6,.833);
\filldraw[fill opacity=0.8,fill=gray!20,draw=none](-7.585,.835)--(-7.6,.84)--(-7.6,.833)--cycle;
\draw(-7.6,.84)--(-7.6,.833);
\filldraw[fill opacity=0.8,fill=gray!20,draw=none](-7.627,.85)--(-7.61,.856)--(-7.551,.845)--(-7.61,.827)--cycle;
\draw(-7.627,.85)--(-7.61,.856);
\draw(-7.551,.845)--(-7.61,.827);
\filldraw[fill opacity=0.8,fill=gray!20,draw=none](-7.63,.857)--(-7.61,.856)--(-7.627,.85)--cycle;
\draw(-7.61,.856)--(-7.627,.85);
\filldraw[fill opacity=0.8,fill=gray!20,draw=none](-3.275,2.204)--(-3.278,2.174)--(-3.225,2.102)--(-3.173,2.126)--cycle;
\filldraw[fill opacity=0.8,fill=gray!20,draw=none](-3.378,3.178)--(-3.358,3.232)--(-3.365,3.24)--(-3.37,3.24)--(-3.389,3.221)--(-3.399,3.204)--(-3.4,3.202)--cycle;
\draw(-3.399,3.204)--(-3.4,3.202)--(-3.378,3.178)--(-3.358,3.232)--(-3.365,3.24);
\filldraw[fill opacity=0.8,fill=gray!20,draw=none](-4.122,3.088)--(-3.401,3.234)--(-3.395,3.219)--(-4.128,3.071)--cycle;
\draw(-4.122,3.088)--(-3.401,3.234);
\draw(-3.395,3.219)--(-4.128,3.071);
\filldraw[fill opacity=0.8,fill=gray!20,draw=none](-3.365,3.24)--(-3.358,3.232)--(-3.354,3.239)--cycle;
\draw(-3.365,3.24)--(-3.358,3.232)--(-3.354,3.239);
\filldraw[fill opacity=0.8,fill=gray!20,draw=none](-3.42,3.226)--(-3.198,3.271)--(-3.22,3.271)--(-3.401,3.234)--cycle;
\draw(-3.42,3.226)--(-3.198,3.271)--(-3.22,3.271)--(-3.401,3.234);
\filldraw[fill opacity=0.8,fill=gray!20](-2.998,3.034)--(-2.976,3.085)--(-3.011,3.063)--(-3.029,3.014)--cycle;
\filldraw[fill opacity=0.8,fill=gray!20](-3.385,3.122)--(-3.378,3.178)--(-3.4,3.202)--(-3.407,3.146)--cycle;
\filldraw[fill opacity=0.8,fill=gray!20,draw=none](-7.61,.827)--(-7.551,.845)--(-7.558,.833)--(-7.592,.822)--cycle;
\draw(-7.61,.827)--(-7.551,.845);
\draw(-7.558,.833)--(-7.592,.822);
\filldraw[fill opacity=0.8,fill=gray!20,draw=none](-5.443,2.367)--(-5.939,2.06)--(-7.078,1.391)--(-5.528,2.352)--cycle;
\draw(-5.443,2.367)--(-5.939,2.06);
\draw(-7.078,1.391)--(-5.528,2.352);
\filldraw[fill opacity=0.8,fill=gray!20,draw=none](-7.713,1.816)--(-7.723,1.788)--(-7.673,1.828)--cycle;
\filldraw[fill opacity=0.8,fill=gray!20,draw=none](-7.713,1.816)--(-7.723,1.813)--(-7.723,1.788)--cycle;
\draw(-7.723,1.813)--(-7.723,1.788);
\filldraw[fill opacity=0.8,fill=gray!20,draw=none](-7.643,1.802)--(-7.723,1.788)--(-7.666,1.763)--cycle;
\draw(-7.723,1.788)--(-7.666,1.763)--(-7.643,1.802);
\filldraw[fill opacity=0.8,fill=gray!20,draw=none](-7.635,.937)--(-7.635,.87)--(-7.621,.852)--(-7.6,.84)--(-7.6,.91)--cycle;
\draw(-7.635,.937)--(-7.635,.87);
\draw(-7.6,.84)--(-7.6,.91);
\filldraw[fill opacity=0.8,fill=gray!20,draw=none](-7.625,.857)--(-7.63,.857)--(-7.635,.87)--cycle;
\filldraw[fill opacity=0.8,fill=gray!20,draw=none](-7.588,1.1)--(-7.579,1.104)--(-7.61,1.086)--cycle;
\draw(-7.588,1.1)--(-7.579,1.104);
\filldraw[fill opacity=0.8,fill=gray!20](-3.185,3.363)--(-3.188,3.358)--(-3.188,3.358)--(-3.157,3.361)--cycle;
\filldraw[fill opacity=0.8,fill=gray!20](-3.214,3.362)--(-3.188,3.358)--(-3.188,3.358)--(-3.185,3.363)--cycle;
\filldraw[fill opacity=0.8,fill=gray!20](-2.373,5.712)--(-2.372,5.759)--(-2.286,5.753)--(-2.296,5.707)--cycle;
\filldraw[fill opacity=0.8,fill=gray!20](-2.372,5.759)--(-2.372,5.807)--(-2.282,5.8)--(-2.286,5.753)--cycle;
\filldraw[fill opacity=0.8,fill=gray!20,draw=none](-7.98,.932)--(-7.979,.938)--(-7.975,.939)--cycle;
\draw(-7.979,.938)--(-7.975,.939);
\filldraw[fill opacity=0.8,fill=gray!20,draw=none](-7.975,.939)--(-8.017,.937)--(-7.963,.966)--(-7.962,.957)--cycle;
\draw(-7.975,.939)--(-8.017,.937);
\draw(-7.963,.966)--(-7.962,.957);
\filldraw[fill opacity=0.8,fill=gray!20,draw=none](-8.017,.937)--(-8.069,.934)--(-8.062,.971)--(-7.982,.985)--(-7.963,.986)--(-7.963,.966)--cycle;
\draw(-8.017,.937)--(-8.069,.934)--(-8.062,.971);
\draw(-7.982,.985)--(-7.963,.986)--(-7.963,.966);
\filldraw[fill opacity=0.8,fill=gray!20,draw=none](-7.955,.971)--(-7.962,.957)--(-7.963,.966)--cycle;
\draw(-7.962,.957)--(-7.963,.966);
\filldraw[fill opacity=0.8,fill=gray!20,draw=none](-7.955,.971)--(-7.963,.966)--(-7.963,.986)--(-7.948,.985)--cycle;
\draw(-7.963,.966)--(-7.963,.986)--(-7.948,.985);
\filldraw[fill opacity=0.8,fill=gray!20,draw=none](-8.052,.94)--(-7.642,1.072)--(-7.645,1.061)--(-7.684,1.028)--(-8.075,.902)--cycle;
\draw(-7.684,1.028)--(-8.075,.902)--(-8.052,.94)--(-7.642,1.072);
\filldraw[fill opacity=0.8,fill=gray!20,draw=none](-3.272,2.244)--(-3.3,2.255)--(-3.295,2.22)--(-3.275,2.204)--cycle;
\draw(-3.3,2.255)--(-3.295,2.22);
\filldraw[fill opacity=0.8,fill=gray!20](-3.09,3.34)--(-3.137,3.356)--(-3.131,3.35)--(-3.079,3.328)--cycle;
\filldraw[fill opacity=0.8,fill=gray!20,draw=none](-7.621,1.09)--(-7.635,1.08)--(-7.635,1.013)--(-7.6,1.02)--(-7.6,1.09)--cycle;
\draw(-7.635,1.08)--(-7.635,1.013);
\draw(-7.6,1.02)--(-7.6,1.09);
\filldraw[fill opacity=0.8,fill=gray!20,draw=none](-7.621,1.09)--(-7.635,1.09)--(-7.635,1.08)--cycle;
\draw(-7.635,1.09)--(-7.635,1.08);
\filldraw[fill opacity=0.8,fill=gray!20,draw=none](-7.635,1.08)--(-7.627,1.088)--(-7.621,1.09)--cycle;
\draw(-7.627,1.088)--(-7.621,1.09);
\filldraw[fill opacity=0.8,fill=gray!20,draw=none](-7.573,1.053)--(-7.595,1.091)--(-7.6,1.09)--(-7.6,1.045)--cycle;
\draw(-7.6,1.09)--(-7.6,1.045);
\filldraw[fill opacity=0.8,fill=gray!20,draw=none](-7.585,1.092)--(-7.595,1.091)--(-7.573,1.053)--(-7.55,1.06)--(-7.55,1.08)--cycle;
\draw(-7.55,1.06)--(-7.55,1.08);
\filldraw[fill opacity=0.8,fill=gray!20,draw=none](-7.61,1.084)--(-7.543,1.105)--(-7.579,1.104)--(-7.627,1.088)--cycle;
\draw(-7.61,1.084)--(-7.543,1.105);
\draw(-7.579,1.104)--(-7.627,1.088);
\filldraw[fill opacity=0.8,fill=gray!20,draw=none](-7.521,1.914)--(-7.561,1.907)--(-7.561,1.897)--(-7.527,1.875)--(-7.516,1.89)--(-7.516,1.907)--cycle;
\draw(-7.521,1.914)--(-7.561,1.907)--(-7.561,1.897);
\draw(-7.516,1.89)--(-7.516,1.907);
\filldraw[fill opacity=0.8,fill=gray!20,draw=none](-7.516,1.89)--(-7.527,1.875)--(-7.523,1.873)--cycle;
\filldraw[fill opacity=0.8,fill=gray!20](-2.372,5.807)--(-2.372,5.852)--(-2.286,5.846)--(-2.282,5.8)--cycle;
\filldraw[fill opacity=0.8,fill=gray!20,draw=none](-7.928,1.488)--(-7.952,1.487)--(-7.723,1.587)--(-7.703,1.58)--(-7.909,1.491)--cycle;
\draw(-7.952,1.487)--(-7.723,1.587);
\draw(-7.703,1.58)--(-7.909,1.491);
\filldraw[fill opacity=0.8,fill=gray!20,draw=none](-7.668,1.596)--(-7.65,1.604)--(-7.641,1.612)--(-7.641,1.614)--(-7.724,1.578)--cycle;
\draw(-7.668,1.596)--(-7.65,1.604);
\draw(-7.641,1.614)--(-7.724,1.578);
\filldraw[fill opacity=0.8,fill=gray!20](-3.236,3.357)--(-3.188,3.358)--(-3.188,3.358)--(-3.214,3.362)--cycle;
\filldraw[fill opacity=0.8,fill=gray!20,draw=none](-7.56,1.776)--(-7.553,1.779)--(-7.569,1.789)--cycle;
\draw(-7.56,1.776)--(-7.553,1.779);
\filldraw[fill opacity=0.8,fill=gray!20,draw=none](-7.548,1.727)--(-7.546,1.748)--(-7.555,1.745)--cycle;
\draw(-7.546,1.748)--(-7.555,1.745);
\filldraw[fill opacity=0.8,fill=gray!20,draw=none](-7.567,1.719)--(-7.548,1.727)--(-7.566,1.773)--(-7.582,1.767)--cycle;
\draw(-7.567,1.719)--(-7.548,1.727);
\draw(-7.566,1.773)--(-7.582,1.767);
\filldraw[fill opacity=0.8,fill=gray!20,draw=none](-7.573,1.797)--(-7.57,1.79)--(-7.561,1.785)--cycle;
\filldraw[fill opacity=0.8,fill=gray!20,draw=none](-7.532,1.807)--(-7.564,1.796)--(-7.555,1.745)--(-7.546,1.748)--cycle;
\draw(-7.532,1.807)--(-7.564,1.796);
\draw(-7.555,1.745)--(-7.546,1.748);
\filldraw[fill opacity=0.8,fill=gray!20,draw=none](-7.543,1.761)--(-7.546,1.748)--(-7.543,1.749)--cycle;
\draw(-7.546,1.748)--(-7.543,1.749);
\filldraw[fill opacity=0.8,fill=gray!20,draw=none](-7.507,1.816)--(-7.532,1.807)--(-7.543,1.761)--(-7.543,1.749)--(-7.514,1.759)--cycle;
\draw(-7.507,1.816)--(-7.532,1.807);
\draw(-7.543,1.749)--(-7.514,1.759);
\filldraw[fill opacity=0.8,fill=gray!20,draw=none](-7.561,1.811)--(-7.573,1.797)--(-7.571,1.793)--(-7.568,1.794)--cycle;
\draw(-7.571,1.793)--(-7.568,1.794);
\filldraw[fill opacity=0.8,fill=gray!20,draw=none](-7.575,1.799)--(-7.571,1.793)--(-7.573,1.797)--cycle;
\filldraw[fill opacity=0.8,fill=gray!20,draw=none](-7.582,1.767)--(-7.56,1.776)--(-7.569,1.789)--(-7.605,1.812)--(-7.624,1.804)--cycle;
\draw(-7.582,1.767)--(-7.56,1.776);
\draw(-7.605,1.812)--(-7.624,1.804);
\filldraw[fill opacity=0.8,fill=gray!20,draw=none](-7.56,1.776)--(-7.564,1.796)--(-7.571,1.793)--cycle;
\draw(-7.564,1.796)--(-7.571,1.793);
\filldraw[fill opacity=0.8,fill=gray!20,draw=none](-7.563,1.806)--(-7.568,1.794)--(-7.564,1.796)--cycle;
\draw(-7.568,1.794)--(-7.564,1.796);
\filldraw[fill opacity=0.8,fill=gray!20,draw=none](-7.563,1.804)--(-7.564,1.796)--(-7.557,1.798)--cycle;
\draw(-7.564,1.796)--(-7.557,1.798);
\filldraw[fill opacity=0.8,fill=gray!20,draw=none](-7.575,1.799)--(-7.574,1.809)--(-7.592,1.818)--cycle;
\filldraw[fill opacity=0.8,fill=gray!20,draw=none](-7.616,1.828)--(-7.597,1.816)--(-7.592,1.818)--cycle;
\draw(-7.597,1.816)--(-7.592,1.818);
\filldraw[fill opacity=0.8,fill=gray!20,draw=none](-7.524,1.731)--(-7.443,1.819)--(-7.452,1.833)--(-7.495,1.859)--(-7.545,1.864)--(-7.595,1.848)--(-7.637,1.812)--(-7.66,1.774)--cycle;
\draw(-7.443,1.819)--(-7.452,1.833)--(-7.495,1.859)--(-7.545,1.864)--(-7.595,1.848)--(-7.637,1.812)--(-7.66,1.774);
\filldraw[fill opacity=0.8,fill=gray!20,draw=none](-6.356,2.17)--(-7.497,1.765)--(-7.507,1.749)--(-7.489,1.72)--(-6.721,1.992)--cycle;
\draw(-6.356,2.17)--(-7.497,1.765);
\draw(-7.489,1.72)--(-6.721,1.992);
\filldraw[fill opacity=0.8,fill=gray!20,draw=none](-7.507,1.749)--(-7.524,1.731)--(-7.489,1.72)--cycle;
\filldraw[fill opacity=0.8,fill=gray!20,draw=none](-3.399,3.202)--(-3.402,3.218)--(-3.395,3.219)--cycle;
\draw(-3.402,3.218)--(-3.395,3.219);
\filldraw[fill opacity=0.8,fill=gray!20,draw=none](-3.272,2.244)--(-3.275,2.204)--(-3.173,2.126)--(-3.184,2.21)--cycle;
\draw(-3.173,2.126)--(-3.184,2.21);
\filldraw[fill opacity=0.8,fill=gray!20,draw=none](-7.521,1.914)--(-7.516,1.907)--(-7.516,1.915)--cycle;
\draw(-7.516,1.907)--(-7.516,1.915)--(-7.521,1.914);
\filldraw[fill opacity=0.8,fill=gray!20](-2.375,5.669)--(-2.373,5.712)--(-2.296,5.707)--(-2.312,5.664)--cycle;
\filldraw[fill opacity=0.8,fill=gray!20](-2.976,3.085)--(-2.969,3.141)--(-3.005,3.117)--(-3.011,3.063)--cycle;
\filldraw[fill opacity=0.8,fill=gray!20](-3.192,2.924)--(-3.195,2.934)--(-3.249,2.938)--(-3.219,2.926)--cycle;
\filldraw[fill opacity=0.8,fill=gray!20](-3.157,3.361)--(-3.188,3.358)--(-3.188,3.358)--(-3.137,3.356)--cycle;
\filldraw[fill opacity=0.8,fill=gray!20,draw=none](-7.527,1.875)--(-7.561,1.897)--(-7.561,1.871)--(-7.532,1.869)--cycle;
\draw(-7.561,1.897)--(-7.561,1.871);
\filldraw[fill opacity=0.8,fill=gray!20,draw=none](-7.398,.789)--(-7.393,.79)--(-7.393,.798)--cycle;
\draw(-7.398,.789)--(-7.393,.79)--(-7.393,.798);
\filldraw[fill opacity=0.8,fill=gray!20,draw=none](-6.565,2.048)--(-7.38,1.758)--(-7.393,1.739)--cycle;
\draw(-6.565,2.048)--(-7.38,1.758);
\filldraw[fill opacity=0.8,fill=gray!20,draw=none](-7.723,1.869)--(-7.723,1.855)--(-7.691,1.858)--cycle;
\draw(-7.723,1.869)--(-7.723,1.855);
\filldraw[fill opacity=0.8,fill=gray!20,draw=none](-3.34,2.567)--(-3.3,2.255)--(-3.184,2.21)--(-3.247,2.702)--cycle;
\draw(-3.34,2.567)--(-3.3,2.255);
\draw(-3.184,2.21)--(-3.247,2.702);
\filldraw[fill opacity=0.8,fill=gray!20,draw=none](-7.935,.77)--(-7.948,.771)--(-7.955,.789)--cycle;
\draw(-7.935,.77)--(-7.948,.771);
\filldraw[fill opacity=0.8,fill=gray!20,draw=none](-7.982,.771)--(-7.963,.772)--(-7.963,.768)--cycle;
\draw(-7.982,.771)--(-7.963,.772)--(-7.963,.768);
\filldraw[fill opacity=0.8,fill=gray!20,draw=none](-7.963,.768)--(-7.963,.772)--(-7.935,.77)--(-7.922,.755)--cycle;
\draw(-7.963,.768)--(-7.963,.772)--(-7.935,.77);
\filldraw[fill opacity=0.8,fill=gray!20,draw=none](-7.965,.728)--(-7.963,.768)--(-7.922,.755)--(-7.906,.736)--cycle;
\draw(-7.965,.728)--(-7.963,.768);
\filldraw[fill opacity=0.8,fill=gray!20,draw=none](-8.052,.758)--(-7.724,.864)--(-7.63,.857)--(-7.627,.85)--(-8.016,.725)--cycle;
\draw(-7.627,.85)--(-8.016,.725)--(-8.052,.758)--(-7.724,.864);
\filldraw[fill opacity=0.8,fill=gray!20](-3.163,2.925)--(-3.139,2.937)--(-3.195,2.934)--(-3.192,2.924)--cycle;
\filldraw[fill opacity=0.8,fill=gray!20,draw=none](-3.428,2.052)--(-3.498,2.053)--(-3.465,2.071)--(-3.456,2.074)--(-3.42,2.058)--cycle;
\draw(-3.456,2.074)--(-3.42,2.058)--(-3.428,2.052);
\filldraw[fill opacity=0.8,fill=gray!20,draw=none](-3.633,2.081)--(-3.731,2.124)--(-3.693,2.077)--(-3.605,2.038)--cycle;
\draw(-3.693,2.077)--(-3.605,2.038)--(-3.633,2.081)--(-3.731,2.124);
\filldraw[fill opacity=0.8,fill=gray!20,draw=none](-4.169,2.84)--(-3.372,3.002)--(-3.382,2.975)--(-4.15,2.819)--cycle;
\draw(-4.169,2.84)--(-3.372,3.002);
\draw(-3.382,2.975)--(-4.15,2.819);
\filldraw[fill opacity=0.8,fill=gray!20,draw=none](-7.523,1.873)--(-7.526,1.868)--(-7.516,1.865)--(-7.51,1.867)--cycle;
\draw(-7.516,1.865)--(-7.51,1.867);
\filldraw[fill opacity=0.8,fill=gray!20,draw=none](-7.955,.971)--(-7.948,.985)--(-7.935,.984)--cycle;
\draw(-7.948,.985)--(-7.935,.984);
\filldraw[fill opacity=0.8,fill=gray!20,draw=none](-7.922,.993)--(-7.935,.984)--(-7.948,.985)--(-7.947,.99)--cycle;
\draw(-7.935,.984)--(-7.948,.985);
\filldraw[fill opacity=0.8,fill=gray!20,draw=none](-8.016,.962)--(-7.627,1.088)--(-7.642,1.072)--(-8.052,.94)--cycle;
\draw(-7.642,1.072)--(-8.052,.94)--(-8.016,.962)--(-7.627,1.088);
\filldraw[fill opacity=0.8,fill=gray!20,draw=none](-3.42,2.26)--(-3.468,2.282)--(-3.487,2.291)--(-3.535,2.318)--(-3.462,2.287)--cycle;
\draw(-3.535,2.318)--(-3.462,2.287)--(-3.42,2.26)--(-3.468,2.282);
\filldraw[fill opacity=0.8,fill=gray!20,draw=none](-3.487,2.291)--(-3.468,2.282)--(-3.476,2.285)--cycle;
\draw(-3.468,2.282)--(-3.476,2.285);
\filldraw[fill opacity=0.8,fill=gray!20,draw=none](-3.533,2.318)--(-3.535,2.318)--(-3.587,2.325)--(-3.564,2.315)--cycle;
\draw(-3.533,2.318)--(-3.535,2.318);
\draw(-3.587,2.325)--(-3.564,2.315);
\filldraw[fill opacity=0.8,fill=gray!20,draw=none](-4.149,2.815)--(-3.552,2.316)--(-3.557,2.324)--(-4.15,2.819)--cycle;
\draw(-4.149,2.815)--(-3.552,2.316);
\draw(-3.557,2.324)--(-4.15,2.819);
\filldraw[fill opacity=0.8,fill=gray!20,draw=none](-3.176,2.295)--(-3.166,2.224)--(-3.056,2.197)--(-3.07,2.304)--cycle;
\draw(-3.056,2.197)--(-3.07,2.304)--(-3.176,2.295)--(-3.166,2.224);
\filldraw[fill opacity=0.8,fill=gray!20,draw=none](-7.723,1.912)--(-7.723,1.869)--(-7.691,1.858)--(-7.673,1.859)--(-7.673,1.905)--cycle;
\draw(-7.673,1.859)--(-7.673,1.905)--(-7.723,1.912)--(-7.723,1.869);
\filldraw[fill opacity=0.8,fill=gray!20](-3.249,2.938)--(-3.274,2.964)--(-3.327,2.977)--(-3.286,2.947)--cycle;
\filldraw[fill opacity=0.8,fill=gray!20,draw=none](-3.173,2.126)--(-3.162,2.041)--(-3.034,2.037)--(-3.049,2.155)--cycle;
\draw(-3.173,2.126)--(-3.162,2.041);
\draw(-3.034,2.037)--(-3.049,2.155);
\filldraw[fill opacity=0.8,fill=gray!20,draw=none](-5.174,2.296)--(-7.215,1.031)--(-7.249,1.003)--(-7.197,1.025)--(-5.028,2.37)--cycle;
\draw(-5.174,2.296)--(-7.215,1.031);
\draw(-7.197,1.025)--(-5.028,2.37);
\filldraw[fill opacity=0.8,fill=gray!20,draw=none](-3.465,2.071)--(-3.458,2.075)--(-3.456,2.074)--cycle;
\draw(-3.458,2.075)--(-3.456,2.074);
\filldraw[fill opacity=0.8,fill=gray!20,draw=none](-3.377,3.241)--(-3.373,3.248)--(-3.378,3.253)--(-3.383,3.242)--cycle;
\draw(-3.373,3.248)--(-3.378,3.253)--(-3.383,3.242);
\filldraw[fill opacity=0.8,fill=gray!20,draw=none](-3.426,2.107)--(-3.444,2.081)--(-3.456,2.074)--(-3.458,2.075)--cycle;
\draw(-3.456,2.074)--(-3.458,2.075);
\filldraw[fill opacity=0.8,fill=gray!20,draw=none](-7.424,.918)--(-7.436,.893)--(-7.417,.905)--cycle;
\draw(-7.436,.893)--(-7.417,.905);
\filldraw[fill opacity=0.8,fill=gray!20,draw=none](-7.436,.893)--(-7.424,.918)--(-7.471,.903)--cycle;
\draw(-7.424,.918)--(-7.471,.903);
\filldraw[fill opacity=0.8,fill=gray!20,draw=none](-8.023,1.68)--(-8.024,1.723)--(-8.018,1.727)--(-7.931,1.72)--(-7.919,1.673)--cycle;
\draw(-8.018,1.727)--(-7.931,1.72)--(-7.919,1.673)--(-8.023,1.68)--(-8.024,1.723);
\filldraw[fill opacity=0.8,fill=gray!20,draw=none](-8.066,1.678)--(-8.07,1.672)--(-8.071,1.672)--cycle;
\draw(-8.07,1.672)--(-8.071,1.672);
\filldraw[fill opacity=0.8,fill=gray!20,draw=none](-8.079,1.624)--(-8.069,1.678)--(-8.023,1.68)--(-8.023,1.626)--cycle;
\draw(-8.069,1.678)--(-8.023,1.68)--(-8.023,1.626)--(-8.079,1.624);
\filldraw[fill opacity=0.8,fill=gray!20,draw=none](-8.066,1.678)--(-8.069,1.678)--(-8.069,1.697)--(-8.044,1.712)--cycle;
\draw(-8.066,1.678)--(-8.069,1.678);
\filldraw[fill opacity=0.8,fill=gray!20,draw=none](-8.069,1.697)--(-8.069,1.678)--(-8.103,1.677)--cycle;
\draw(-8.069,1.678)--(-8.103,1.677);
\filldraw[fill opacity=0.8,fill=gray!20,draw=none](-8.119,1.679)--(-8.038,1.714)--(-8.071,1.672)--(-8.145,1.64)--cycle;
\draw(-8.071,1.672)--(-8.145,1.64)--(-8.119,1.679)--(-8.038,1.714);
\filldraw[fill opacity=0.8,fill=gray!20,draw=none](-8.066,1.678)--(-8.044,1.712)--(-8.024,1.723)--(-8.023,1.68)--cycle;
\draw(-8.024,1.723)--(-8.023,1.68)--(-8.066,1.678);
\filldraw[fill opacity=0.8,fill=gray!20,draw=none](-8.018,1.727)--(-8.005,1.735)--(-7.96,1.74)--(-7.939,1.735)--(-7.931,1.72)--cycle;
\draw(-7.939,1.735)--(-7.931,1.72)--(-8.018,1.727);
\filldraw[fill opacity=0.8,fill=gray!20,draw=none](-8.036,1.707)--(-7.723,1.844)--(-7.743,1.85)--(-8.081,1.702)--cycle;
\draw(-7.743,1.85)--(-8.081,1.702)--(-8.036,1.707)--(-7.723,1.844);
\filldraw[fill opacity=0.8,fill=gray!20,draw=none](-5.697,2.247)--(-6.36,1.837)--(-5.901,2.13)--cycle;
\draw(-5.697,2.247)--(-6.36,1.837);
\filldraw[fill opacity=0.8,fill=gray!20,draw=none](-4.645,2.842)--(-4.69,2.852)--(-5.144,2.57)--(-4.89,2.692)--(-4.747,2.769)--(-4.649,2.83)--cycle;
\draw(-4.69,2.852)--(-5.144,2.57);
\draw(-4.747,2.769)--(-4.649,2.83);
\filldraw[fill opacity=0.8,fill=gray!20,draw=none](-4.877,2.696)--(-4.976,2.661)--(-5.242,2.518)--(-4.768,2.686)--cycle;
\draw(-4.877,2.696)--(-4.976,2.661);
\draw(-5.242,2.518)--(-4.768,2.686);
\filldraw[fill opacity=0.8,fill=gray!20,draw=none](-5.412,2.434)--(-7.462,1.163)--(-7.433,1.152)--(-5.178,2.549)--cycle;
\draw(-5.412,2.434)--(-7.462,1.163);
\draw(-7.433,1.152)--(-5.178,2.549);
\filldraw[fill opacity=0.8,fill=gray!20,draw=none](-7.906,.736)--(-7.922,.755)--(-7.879,.741)--(-7.879,.74)--cycle;
\draw(-7.879,.741)--(-7.879,.74);
\filldraw[fill opacity=0.8,fill=gray!20,draw=none](-7.965,.72)--(-7.965,.728)--(-7.906,.736)--(-7.888,.715)--(-7.889,.714)--cycle;
\draw(-7.888,.715)--(-7.889,.714)--(-7.965,.72)--(-7.965,.728);
\filldraw[fill opacity=0.8,fill=gray!20,draw=none](-8.024,.673)--(-8.042,.713)--(-8.02,.717)--(-7.965,.72)--(-7.968,.676)--cycle;
\draw(-8.02,.717)--(-7.965,.72)--(-7.968,.676)--(-8.024,.673)--(-8.042,.713);
\filldraw[fill opacity=0.8,fill=gray!20,draw=none](-7.964,.676)--(-7.968,.676)--(-7.965,.72)--(-7.889,.714)--(-7.89,.713)--cycle;
\draw(-7.964,.676)--(-7.968,.676)--(-7.965,.72)--(-7.889,.714)--(-7.89,.713);
\filldraw[fill opacity=0.8,fill=gray!20,draw=none](-7.906,.736)--(-7.879,.74)--(-7.888,.715)--cycle;
\draw(-7.879,.74)--(-7.888,.715);
\filldraw[fill opacity=0.8,fill=gray!20,draw=none](-7.879,.74)--(-7.879,.741)--(-7.878,.74)--cycle;
\draw(-7.879,.74)--(-7.879,.741);
\filldraw[fill opacity=0.8,fill=gray!20,draw=none](-8.016,.725)--(-7.627,.85)--(-7.61,.827)--(-7.974,.709)--cycle;
\draw(-7.61,.827)--(-7.974,.709)--(-8.016,.725)--(-7.627,.85);
\filldraw[fill opacity=0.8,fill=gray!20,draw=none](-7.635,1.013)--(-7.635,.975)--(-7.6,1.02)--cycle;
\draw(-7.635,1.013)--(-7.635,.975);
\filldraw[fill opacity=0.8,fill=gray!20,draw=none](-7.553,.939)--(-7.635,.975)--(-7.6,.91)--(-7.543,.886)--cycle;
\draw(-7.6,.91)--(-7.543,.886)--(-7.553,.939)--(-7.635,.975);
\filldraw[fill opacity=0.8,fill=gray!20,draw=none](-7.438,.909)--(-7.438,.818)--(-7.393,.836)--(-7.393,.924)--cycle;
\draw(-7.438,.909)--(-7.438,.818);
\draw(-7.393,.836)--(-7.393,.924);
\filldraw[fill opacity=0.8,fill=gray!20](-2.462,5.848)--(-2.454,5.887)--(-2.373,5.891)--(-2.372,5.852)--cycle;
\filldraw[fill opacity=0.8,fill=gray!20](-2.372,5.852)--(-2.373,5.891)--(-2.296,5.885)--(-2.286,5.846)--cycle;
\filldraw[fill opacity=0.8,fill=gray!20](-2.969,3.141)--(-2.976,3.196)--(-3.011,3.174)--(-3.005,3.117)--cycle;
\filldraw[fill opacity=0.8,fill=gray!20,draw=none](-3.368,3.242)--(-3.336,3.266)--(-3.327,3.28)--(-3.343,3.297)--(-3.378,3.253)--cycle;
\draw(-3.336,3.266)--(-3.327,3.28)--(-3.343,3.297)--(-3.378,3.253)--(-3.368,3.242);
\filldraw[fill opacity=0.8,fill=gray!20,draw=none](-3.306,2.986)--(-3.296,3.003)--(-3.358,3.018)--(-3.333,2.985)--cycle;
\draw(-3.296,3.003)--(-3.358,3.018)--(-3.333,2.985);
\filldraw[fill opacity=0.8,fill=gray!20,draw=none](-3.296,3.003)--(-3.295,3.004)--(-3.295,3.008)--(-3.305,3.049)--(-3.378,3.067)--(-3.358,3.018)--cycle;
\draw(-3.295,3.008)--(-3.305,3.049)--(-3.378,3.067)--(-3.358,3.018)--(-3.296,3.003);
\filldraw[fill opacity=0.8,fill=gray!20,draw=none](-4.07,2.831)--(-3.156,3.016)--(-3.139,3.036)--(-4.002,2.861)--cycle;
\draw(-4.07,2.831)--(-3.156,3.016)--(-3.139,3.036)--(-4.002,2.861);
\filldraw[fill opacity=0.8,fill=gray!20,draw=none](-3.45,2.072)--(-3.456,2.074)--(-3.444,2.081)--cycle;
\draw(-3.45,2.072)--(-3.456,2.074);
\filldraw[fill opacity=0.8,fill=gray!20,draw=none](-3.751,2.149)--(-3.717,2.118)--(-3.715,2.116)--(-3.732,2.13)--cycle;
\draw(-3.715,2.116)--(-3.732,2.13);
\filldraw[fill opacity=0.8,fill=gray!20,draw=none](-7.507,1.749)--(-7.497,1.765)--(-7.514,1.759)--cycle;
\draw(-7.497,1.765)--(-7.514,1.759);
\filldraw[fill opacity=0.8,fill=gray!20,draw=none](-7.501,1.862)--(-7.51,1.867)--(-7.516,1.865)--cycle;
\draw(-7.51,1.867)--(-7.516,1.865);
\filldraw[fill opacity=0.8,fill=gray!20](-3.245,3.351)--(-3.188,3.358)--(-3.188,3.358)--(-3.236,3.357)--cycle;
\filldraw[fill opacity=0.8,fill=gray!20,draw=none](-2.474,5.663)--(-2.473,5.665)--(-2.47,5.66)--cycle;
\draw(-2.474,5.663)--(-2.473,5.665);
\filldraw[fill opacity=0.8,fill=gray!20,draw=none](-2.494,5.669)--(-2.508,5.685)--(-2.473,5.665)--(-2.474,5.663)--cycle;
\draw(-2.473,5.665)--(-2.474,5.663);
\filldraw[fill opacity=0.8,fill=gray!20,draw=none](-2.494,5.669)--(-2.474,5.663)--(-2.481,5.654)--cycle;
\draw(-2.474,5.663)--(-2.481,5.654);
\filldraw[fill opacity=0.8,fill=gray!20,draw=none](-2.484,5.649)--(-2.474,5.663)--(-2.47,5.66)--(-2.457,5.64)--cycle;
\draw(-2.484,5.649)--(-2.474,5.663);
\filldraw[fill opacity=0.8,fill=gray!20,draw=none](-2.494,5.669)--(-2.524,5.678)--(-2.516,5.69)--(-2.508,5.685)--cycle;
\draw(-2.524,5.678)--(-2.516,5.69);
\filldraw[fill opacity=0.8,fill=gray!20](-2.491,5.656)--(-2.515,5.697)--(-2.454,5.709)--(-2.441,5.666)--cycle;
\filldraw[fill opacity=0.8,fill=gray!20,draw=none](-4.802,2.812)--(-4.854,2.779)--(-4.85,2.753)--(-4.809,2.778)--cycle;
\draw(-4.802,2.812)--(-4.854,2.779);
\draw(-4.85,2.753)--(-4.809,2.778);
\filldraw[fill opacity=0.8,fill=gray!20,draw=none](-4.745,2.848)--(-4.802,2.812)--(-4.809,2.778)--(-4.739,2.821)--cycle;
\draw(-4.745,2.848)--(-4.802,2.812);
\draw(-4.809,2.778)--(-4.739,2.821);
\filldraw[fill opacity=0.8,fill=gray!20,draw=none](-4.664,2.866)--(-4.665,2.866)--(-4.665,2.868)--cycle;
\draw(-4.664,2.866)--(-4.665,2.866);
\filldraw[fill opacity=0.8,fill=gray!20,draw=none](-4.639,2.865)--(-4.636,2.866)--(-4.637,2.872)--(-4.642,2.877)--cycle;
\draw(-4.637,2.872)--(-4.642,2.877);
\filldraw[fill opacity=0.8,fill=gray!20,draw=none](-4.632,2.838)--(-4.641,2.828)--(-4.632,2.818)--(-4.631,2.827)--cycle;
\draw(-4.641,2.828)--(-4.632,2.818);
\filldraw[fill opacity=0.8,fill=gray!20,draw=none](-4.726,2.778)--(-4.64,2.816)--(-4.634,2.839)--(-4.715,2.789)--cycle;
\draw(-4.634,2.839)--(-4.715,2.789);
\filldraw[fill opacity=0.8,fill=gray!20,draw=none](-4.63,2.841)--(-4.632,2.841)--(-4.63,2.842)--cycle;
\filldraw[fill opacity=0.8,fill=gray!20,draw=none](-4.632,2.841)--(-4.63,2.841)--(-4.63,2.839)--(-4.631,2.84)--cycle;
\filldraw[fill opacity=0.8,fill=gray!20,draw=none](-4.639,2.865)--(-4.632,2.843)--(-4.632,2.846)--(-4.636,2.866)--cycle;
\filldraw[fill opacity=0.8,fill=gray!20,draw=none](-4.649,2.854)--(-4.645,2.851)--(-4.648,2.856)--cycle;
\filldraw[fill opacity=0.8,fill=gray!20,draw=none](-4.633,2.837)--(-4.631,2.839)--(-4.631,2.841)--(-4.632,2.843)--cycle;
\filldraw[fill opacity=0.8,fill=gray!20,draw=none](-4.631,2.827)--(-4.633,2.825)--(-4.636,2.833)--(-4.634,2.84)--(-4.632,2.842)--cycle;
\draw(-4.634,2.84)--(-4.632,2.842);
\filldraw[fill opacity=0.8,fill=gray!20,draw=none](-4.632,2.841)--(-4.631,2.84)--(-4.632,2.841)--(-4.633,2.842)--cycle;
\filldraw[fill opacity=0.8,fill=gray!20,draw=none](-4.632,2.843)--(-4.637,2.861)--(-4.643,2.847)--(-4.637,2.837)--(-4.634,2.839)--cycle;
\draw(-4.637,2.837)--(-4.634,2.839);
\filldraw[fill opacity=0.8,fill=gray!20,draw=none](-4.649,2.854)--(-4.633,2.841)--(-4.633,2.841)--(-4.65,2.853)--cycle;
\filldraw[fill opacity=0.8,fill=gray!20,draw=none](-4.627,2.833)--(-4.631,2.827)--(-4.632,2.838)--cycle;
\filldraw[fill opacity=0.8,fill=gray!20,draw=none](-4.632,2.838)--(-4.631,2.827)--(-4.63,2.836)--(-4.631,2.839)--cycle;
\filldraw[fill opacity=0.8,fill=gray!20,draw=none](-4.643,2.847)--(-4.649,2.83)--(-4.637,2.837)--cycle;
\draw(-4.649,2.83)--(-4.637,2.837);
\filldraw[fill opacity=0.8,fill=gray!20,draw=none](-4.65,2.853)--(-4.634,2.842)--(-4.636,2.842)--(-4.65,2.852)--cycle;
\filldraw[fill opacity=0.8,fill=gray!20,draw=none](-4.632,2.842)--(-4.633,2.841)--(-4.633,2.841)--(-4.632,2.843)--cycle;
\draw(-4.632,2.842)--(-4.633,2.841);
\filldraw[fill opacity=0.8,fill=gray!20,draw=none](-4.633,2.841)--(-4.634,2.84)--(-4.633,2.841)--cycle;
\draw(-4.633,2.841)--(-4.634,2.84);
\filldraw[fill opacity=0.8,fill=gray!20,draw=none](-4.632,2.843)--(-4.633,2.841)--(-4.633,2.841)--(-4.632,2.844)--cycle;
\filldraw[fill opacity=0.8,fill=gray!20,draw=none](-4.632,2.844)--(-4.632,2.843)--(-4.631,2.841)--cycle;
\filldraw[fill opacity=0.8,fill=gray!20,draw=none](-4.633,2.841)--(-4.632,2.841)--(-4.633,2.841)--cycle;
\filldraw[fill opacity=0.8,fill=gray!20,draw=none](-4.632,2.843)--(-4.634,2.839)--(-4.631,2.841)--cycle;
\draw(-4.634,2.839)--(-4.631,2.841);
\filldraw[fill opacity=0.8,fill=gray!20,draw=none](-4.633,2.841)--(-4.634,2.84)--(-4.633,2.841)--cycle;
\filldraw[fill opacity=0.8,fill=gray!20,draw=none](-4.634,2.842)--(-4.633,2.841)--(-4.634,2.841)--(-4.636,2.842)--cycle;
\filldraw[fill opacity=0.8,fill=gray!20,draw=none](-4.64,2.816)--(-4.638,2.828)--(-4.634,2.84)--cycle;
\filldraw[fill opacity=0.8,fill=gray!20,draw=none](-4.638,2.828)--(-4.638,2.836)--(-4.634,2.84)--cycle;
\draw(-4.638,2.836)--(-4.634,2.84);
\filldraw[fill opacity=0.8,fill=gray!20,draw=none](-4.634,2.84)--(-4.638,2.836)--(-4.634,2.841)--cycle;
\draw(-4.634,2.84)--(-4.638,2.836);
\filldraw[fill opacity=0.8,fill=gray!20,draw=none](-4.633,2.841)--(-4.634,2.84)--(-4.634,2.841)--(-4.632,2.846)--cycle;
\filldraw[fill opacity=0.8,fill=gray!20,draw=none](-4.633,2.841)--(-4.632,2.846)--(-4.61,2.885)--(-4.608,2.887)--cycle;
\filldraw[fill opacity=0.8,fill=gray!20,draw=none](-4.633,2.841)--(-4.632,2.841)--(-4.63,2.839)--(-4.634,2.841)--cycle;
\draw(-4.63,2.839)--(-4.634,2.841);
\filldraw[fill opacity=0.8,fill=gray!20,draw=none](-4.618,2.826)--(-4.601,2.834)--(-4.598,2.833)--(-4.61,2.819)--cycle;
\draw(-4.598,2.833)--(-4.61,2.819);
\filldraw[fill opacity=0.8,fill=gray!20,draw=none](-4.632,2.818)--(-4.631,2.827)--(-4.627,2.833)--(-4.622,2.825)--(-4.631,2.813)--cycle;
\draw(-4.622,2.825)--(-4.631,2.813);
\filldraw[fill opacity=0.8,fill=gray!20,draw=none](-4.632,2.818)--(-4.633,2.825)--(-4.631,2.827)--cycle;
\filldraw[fill opacity=0.8,fill=gray!20,draw=none](-4.631,2.827)--(-4.622,2.839)--(-4.601,2.834)--(-4.631,2.821)--cycle;
\filldraw[fill opacity=0.8,fill=gray!20,draw=none](-4.63,2.817)--(-4.631,2.821)--(-4.618,2.826)--(-4.61,2.819)--(-4.625,2.801)--cycle;
\draw(-4.61,2.819)--(-4.625,2.801);
\filldraw[fill opacity=0.8,fill=gray!20,draw=none](-4.633,2.825)--(-4.639,2.817)--(-4.636,2.833)--cycle;
\filldraw[fill opacity=0.8,fill=gray!20,draw=none](-4.636,2.833)--(-4.634,2.829)--(-4.63,2.836)--(-4.631,2.839)--(-4.633,2.84)--(-4.634,2.839)--cycle;
\draw(-4.633,2.84)--(-4.634,2.839);
\filldraw[fill opacity=0.8,fill=gray!20,draw=none](-4.622,2.839)--(-4.627,2.833)--(-4.629,2.839)--cycle;
\filldraw[fill opacity=0.8,fill=gray!20,draw=none](-4.56,2.804)--(-4.629,2.839)--(-4.622,2.839)--(-4.601,2.834)--(-4.543,2.805)--cycle;
\draw(-4.56,2.804)--(-4.629,2.839);
\draw(-4.601,2.834)--(-4.543,2.805);
\filldraw[fill opacity=0.8,fill=gray!20,draw=none](-4.469,2.898)--(-4.467,2.891)--(-4.457,2.864)--(-4.451,2.848)--(-4.444,2.833)--(-4.438,2.89)--cycle;
\draw(-4.467,2.891)--(-4.457,2.864);
\filldraw[fill opacity=0.8,fill=gray!20,draw=none](-4.629,2.839)--(-4.604,2.831)--(-4.465,2.891)--(-4.478,2.914)--(-4.52,2.91)--(-4.632,2.841)--cycle;
\draw(-4.465,2.891)--(-4.478,2.914);
\draw(-4.52,2.91)--(-4.632,2.841);
\filldraw[fill opacity=0.8,fill=gray!20,draw=none](-4.632,2.844)--(-4.631,2.841)--(-4.631,2.839)--(-4.631,2.839)--(-4.632,2.846)--cycle;
\draw(-4.631,2.839)--(-4.632,2.846);
\filldraw[fill opacity=0.8,fill=gray!20,draw=none](-4.629,2.839)--(-4.632,2.841)--(-4.633,2.84)--cycle;
\draw(-4.632,2.841)--(-4.633,2.84);
\filldraw[fill opacity=0.8,fill=gray!20,draw=none](-4.629,2.839)--(-4.63,2.839)--(-4.632,2.841)--cycle;
\draw(-4.629,2.839)--(-4.63,2.839);
\filldraw[fill opacity=0.8,fill=gray!20,draw=none](-4.631,2.839)--(-4.631,2.839)--(-4.631,2.841)--cycle;
\filldraw[fill opacity=0.8,fill=gray!20,draw=none](-4.63,2.836)--(-4.629,2.839)--(-4.631,2.839)--cycle;
\filldraw[fill opacity=0.8,fill=gray!20,draw=none](-4.631,2.839)--(-4.63,2.836)--(-4.631,2.839)--(-4.631,2.839)--cycle;
\draw(-4.63,2.836)--(-4.631,2.839);
\filldraw[fill opacity=0.8,fill=gray!20,draw=none](-4.626,2.786)--(-4.603,2.81)--(-4.597,2.806)--(-4.618,2.775)--cycle;
\draw(-4.597,2.806)--(-4.618,2.775);
\filldraw[fill opacity=0.8,fill=gray!20,draw=none](-4.618,2.775)--(-4.612,2.768)--(-4.633,2.819)--(-4.633,2.82)--cycle;
\draw(-4.618,2.775)--(-4.612,2.768);
\draw(-4.633,2.819)--(-4.633,2.82);
\filldraw[fill opacity=0.8,fill=gray!20,draw=none](-4.633,2.825)--(-4.631,2.813)--(-4.657,2.774)--(-4.64,2.816)--cycle;
\draw(-4.631,2.813)--(-4.657,2.774);
\filldraw[fill opacity=0.8,fill=gray!20,draw=none](-4.571,2.796)--(-4.603,2.758)--(-4.631,2.794)--(-4.61,2.819)--cycle;
\draw(-4.571,2.796)--(-4.603,2.758);
\draw(-4.631,2.794)--(-4.61,2.819);
\filldraw[fill opacity=0.8,fill=gray!20,draw=none](-4.612,2.768)--(-4.608,2.764)--(-4.628,2.814)--(-4.633,2.819)--cycle;
\draw(-4.612,2.768)--(-4.608,2.764)--(-4.628,2.814)--(-4.633,2.819);
\filldraw[fill opacity=0.8,fill=gray!20,draw=none](-4.632,2.821)--(-4.625,2.801)--(-4.644,2.779)--(-4.64,2.816)--(-4.639,2.817)--cycle;
\draw(-4.625,2.801)--(-4.644,2.779);
\filldraw[fill opacity=0.8,fill=gray!20,draw=none](-4.627,2.833)--(-4.634,2.824)--(-4.63,2.836)--(-4.629,2.839)--cycle;
\filldraw[fill opacity=0.8,fill=gray!20,draw=none](-4.632,2.818)--(-4.628,2.814)--(-4.63,2.836)--cycle;
\draw(-4.632,2.818)--(-4.628,2.814)--(-4.63,2.836);
\filldraw[fill opacity=0.8,fill=gray!20,draw=none](-4.63,2.817)--(-4.632,2.821)--(-4.631,2.821)--cycle;
\filldraw[fill opacity=0.8,fill=gray!20,draw=none](-4.631,2.827)--(-4.631,2.821)--(-4.632,2.821)--(-4.633,2.825)--cycle;
\filldraw[fill opacity=0.8,fill=gray!20,draw=none](-4.633,2.825)--(-4.632,2.821)--(-4.639,2.817)--cycle;
\filldraw[fill opacity=0.8,fill=gray!20,draw=none](-4.634,2.824)--(-4.638,2.818)--(-4.639,2.817)--(-4.63,2.836)--cycle;
\filldraw[fill opacity=0.8,fill=gray!20,draw=none](-4.638,2.818)--(-4.64,2.816)--(-4.639,2.817)--cycle;
\filldraw[fill opacity=0.8,fill=gray!20,draw=none](-4.636,2.833)--(-4.64,2.816)--(-4.629,2.82)--cycle;
\filldraw[fill opacity=0.8,fill=gray!20,draw=none](-4.629,2.839)--(-4.629,2.839)--(-4.627,2.844)--cycle;
\filldraw[fill opacity=0.8,fill=gray!20,draw=none](-4.629,2.839)--(-4.629,2.839)--(-4.629,2.84)--(-4.627,2.844)--cycle;
\draw(-4.629,2.84)--(-4.627,2.844);
\filldraw[fill opacity=0.8,fill=gray!20,draw=none](-4.627,2.844)--(-4.629,2.84)--(-4.628,2.841)--(-4.576,2.909)--cycle;
\draw(-4.627,2.844)--(-4.629,2.84);
\filldraw[fill opacity=0.8,fill=gray!20,draw=none](-4.604,2.831)--(-4.626,2.839)--(-4.631,2.839)--(-4.629,2.839)--cycle;
\draw(-4.631,2.839)--(-4.629,2.839);
\filldraw[fill opacity=0.8,fill=gray!20,draw=none](-4.554,2.819)--(-4.553,2.818)--(-4.554,2.816)--cycle;
\draw(-4.553,2.818)--(-4.554,2.816);
\filldraw[fill opacity=0.8,fill=gray!20,draw=none](-4.543,2.805)--(-4.592,2.83)--(-4.561,2.834)--cycle;
\draw(-4.543,2.805)--(-4.592,2.83);
\filldraw[fill opacity=0.8,fill=gray!20,draw=none](-4.528,2.788)--(-4.56,2.804)--(-4.543,2.805)--cycle;
\draw(-4.528,2.788)--(-4.56,2.804);
\filldraw[fill opacity=0.8,fill=gray!20,draw=none](-4.484,2.755)--(-4.533,2.8)--(-4.501,2.73)--(-4.484,2.752)--cycle;
\draw(-4.533,2.8)--(-4.501,2.73);
\filldraw[fill opacity=0.8,fill=gray!20,draw=none](-4.544,2.822)--(-4.533,2.8)--(-4.504,2.772)--(-4.498,2.786)--(-4.513,2.817)--cycle;
\draw(-4.498,2.786)--(-4.513,2.817)--(-4.544,2.822)--(-4.533,2.8);
\filldraw[fill opacity=0.8,fill=gray!20,draw=none](-4.526,2.799)--(-4.556,2.814)--(-4.604,2.831)--(-4.629,2.839)--(-4.528,2.788)--cycle;
\draw(-4.526,2.799)--(-4.556,2.814);
\draw(-4.629,2.839)--(-4.528,2.788);
\filldraw[fill opacity=0.8,fill=gray!20,draw=none](-4.629,2.839)--(-4.631,2.839)--(-4.63,2.836)--cycle;
\draw(-4.631,2.839)--(-4.63,2.836);
\filldraw[fill opacity=0.8,fill=gray!20,draw=none](-4.631,2.839)--(-4.631,2.839)--(-4.631,2.839)--cycle;
\draw(-4.631,2.839)--(-4.631,2.839);
\filldraw[fill opacity=0.8,fill=gray!20,draw=none](-4.628,2.841)--(-4.632,2.846)--(-4.631,2.839)--cycle;
\draw(-4.632,2.846)--(-4.631,2.839);
\filldraw[fill opacity=0.8,fill=gray!20,draw=none](-4.626,2.839)--(-4.627,2.839)--(-4.628,2.841)--(-4.631,2.839)--cycle;
\filldraw[fill opacity=0.8,fill=gray!20,draw=none](-4.629,2.84)--(-4.634,2.841)--(-4.631,2.839)--cycle;
\draw(-4.634,2.841)--(-4.631,2.839);
\filldraw[fill opacity=0.8,fill=gray!20,draw=none](-4.626,2.839)--(-4.627,2.839)--(-4.629,2.84)--(-4.631,2.839)--cycle;
\filldraw[fill opacity=0.8,fill=gray!20,draw=none](-4.629,2.839)--(-4.634,2.831)--(-4.631,2.838)--(-4.63,2.839)--cycle;
\draw(-4.631,2.838)--(-4.63,2.839);
\filldraw[fill opacity=0.8,fill=gray!20,draw=none](-4.629,2.839)--(-4.63,2.839)--(-4.629,2.84)--cycle;
\draw(-4.63,2.839)--(-4.629,2.84);
\filldraw[fill opacity=0.8,fill=gray!20,draw=none](-4.629,2.84)--(-4.63,2.839)--(-4.629,2.84)--(-4.628,2.841)--cycle;
\draw(-4.629,2.84)--(-4.63,2.839);
\filldraw[fill opacity=0.8,fill=gray!20,draw=none](-4.63,2.839)--(-4.631,2.838)--(-4.629,2.84)--cycle;
\draw(-4.63,2.839)--(-4.631,2.838);
\filldraw[fill opacity=0.8,fill=gray!20,draw=none](-4.488,2.934)--(-4.491,2.939)--(-4.52,2.91)--cycle;
\draw(-4.488,2.934)--(-4.491,2.939);
\filldraw[fill opacity=0.8,fill=gray!20,draw=none](-4.491,2.939)--(-4.488,2.934)--(-4.484,2.926)--cycle;
\draw(-4.491,2.939)--(-4.488,2.934);
\filldraw[fill opacity=0.8,fill=gray!20,draw=none](-4.482,2.943)--(-7.51,1.867)--(-7.453,1.835)--(-4.469,2.896)--cycle;
\draw(-7.453,1.835)--(-4.469,2.896)--(-4.482,2.943)--(-7.51,1.867);
\filldraw[fill opacity=0.8,fill=gray!20](-3.097,2.945)--(-3.059,2.973)--(-3.119,2.962)--(-3.139,2.937)--cycle;
\filldraw[fill opacity=0.8,fill=gray!20,draw=none](-7.582,1.668)--(-7.524,1.731)--(-7.56,1.742)--cycle;
\filldraw[fill opacity=0.8,fill=gray!20](-2.424,5.63)--(-2.441,5.666)--(-2.375,5.669)--(-2.377,5.632)--cycle;
\filldraw[fill opacity=0.8,fill=gray!20](-2.296,5.707)--(-2.286,5.753)--(-2.225,5.738)--(-2.241,5.693)--cycle;
\filldraw[fill opacity=0.8,fill=gray!20](-2.286,5.753)--(-2.282,5.8)--(-2.219,5.785)--(-2.225,5.738)--cycle;
\filldraw[fill opacity=0.8,fill=gray!20](-2.377,5.632)--(-2.375,5.669)--(-2.312,5.664)--(-2.333,5.629)--cycle;
\filldraw[fill opacity=0.8,fill=gray!20,draw=none](-5.487,2.534)--(-7.507,1.816)--(-7.514,1.759)--(-5.319,2.539)--cycle;
\draw(-5.487,2.534)--(-7.507,1.816);
\draw(-7.514,1.759)--(-5.319,2.539);
\filldraw[fill opacity=0.8,fill=gray!20,draw=none](-4.174,2.806)--(-3.541,2.278)--(-3.525,2.293)--(-4.149,2.815)--cycle;
\draw(-4.174,2.806)--(-3.541,2.278);
\draw(-3.525,2.293)--(-4.149,2.815);
\filldraw[fill opacity=0.8,fill=gray!20,draw=none](-3.462,2.287)--(-3.533,2.318)--(-3.564,2.315)--(-3.512,2.292)--cycle;
\draw(-3.564,2.315)--(-3.512,2.292)--(-3.462,2.287)--(-3.533,2.318);
\filldraw[fill opacity=0.8,fill=gray!20,draw=none](-7.414,.938)--(-7.393,.932)--(-7.393,.954)--cycle;
\draw(-7.393,.932)--(-7.393,.954);
\filldraw[fill opacity=0.8,fill=gray!20,draw=none](-7.431,.957)--(-7.437,.944)--(-7.431,.934)--cycle;
\filldraw[fill opacity=0.8,fill=gray!20,draw=none](-7.398,1.026)--(-7.431,.957)--(-7.422,.94)--(-7.414,.938)--(-7.393,.954)--(-7.393,1.025)--cycle;
\draw(-7.393,.954)--(-7.393,1.025);
\filldraw[fill opacity=0.8,fill=gray!20,draw=none](-7.368,1.089)--(-7.375,1.092)--(-7.376,1.092)--(-7.376,1.081)--(-7.37,1.085)--cycle;
\draw(-7.376,1.081)--(-7.37,1.085);
\filldraw[fill opacity=0.8,fill=gray!20,draw=none](-7.373,1.023)--(-7.341,.99)--(-7.295,.99)--(-7.301,1.021)--(-7.329,1.065)--(-7.37,1.089)--cycle;
\draw(-7.295,.99)--(-7.301,1.021)--(-7.329,1.065)--(-7.37,1.089);
\filldraw[fill opacity=0.8,fill=gray!20,draw=none](-7.29,1.021)--(-7.406,.949)--(-7.387,.924)--(-7.297,.98)--cycle;
\draw(-7.29,1.021)--(-7.406,.949);
\draw(-7.387,.924)--(-7.297,.98);
\filldraw[fill opacity=0.8,fill=gray!20,draw=none](-7.393,.932)--(-7.393,.924)--(-7.387,.924)--cycle;
\draw(-7.393,.932)--(-7.393,.924);
\filldraw[fill opacity=0.8,fill=gray!20,draw=none](-3.466,2.28)--(-3.444,2.254)--(-3.476,2.285)--(-3.468,2.282)--cycle;
\draw(-3.476,2.285)--(-3.468,2.282);
\filldraw[fill opacity=0.8,fill=gray!20](-3.137,3.356)--(-3.188,3.358)--(-3.188,3.358)--(-3.131,3.35)--cycle;
\filldraw[fill opacity=0.8,fill=gray!20,draw=none](-7.691,1.858)--(-7.673,1.851)--(-7.673,1.859)--cycle;
\draw(-7.673,1.851)--(-7.673,1.859);
\filldraw[fill opacity=0.8,fill=gray!20,draw=none](-7.443,.988)--(-7.438,.981)--(-7.447,1.009)--cycle;
\filldraw[fill opacity=0.8,fill=gray!20,draw=none](-7.443,.988)--(-7.441,1.041)--(-7.452,1.034)--cycle;
\draw(-7.441,1.041)--(-7.452,1.034);
\filldraw[fill opacity=0.8,fill=gray!20,draw=none](-7.473,1.095)--(-7.447,1.008)--(-7.438,1.104)--cycle;
\filldraw[fill opacity=0.8,fill=gray!20,draw=none](-7.422,1.096)--(-7.458,1.112)--(-7.538,1.112)--(-7.54,1.11)--(-7.472,1.08)--cycle;
\draw(-7.54,1.11)--(-7.472,1.08)--(-7.422,1.096)--(-7.458,1.112);
\filldraw[fill opacity=0.8,fill=gray!20](-3.327,3.28)--(-3.286,3.318)--(-3.298,3.331)--(-3.343,3.297)--cycle;
\filldraw[fill opacity=0.8,fill=gray!20](-2.976,3.196)--(-2.998,3.248)--(-3.029,3.228)--(-3.011,3.174)--cycle;
\filldraw[fill opacity=0.8,fill=gray!20](-2.282,5.8)--(-2.286,5.846)--(-2.225,5.831)--(-2.219,5.785)--cycle;
\filldraw[fill opacity=0.8,fill=gray!20](-2.312,5.664)--(-2.296,5.707)--(-2.241,5.693)--(-2.267,5.654)--cycle;
\filldraw[fill opacity=0.8,fill=gray!20,draw=none](-3.42,2.058)--(-3.45,2.072)--(-3.444,2.081)--(-3.394,2.109)--(-3.391,2.108)--cycle;
\draw(-3.394,2.109)--(-3.391,2.108)--(-3.42,2.058)--(-3.45,2.072);
\filldraw[fill opacity=0.8,fill=gray!20,draw=none](-3.444,2.081)--(-3.426,2.107)--(-3.415,2.118)--(-3.394,2.109)--cycle;
\draw(-3.415,2.118)--(-3.394,2.109);
\filldraw[fill opacity=0.8,fill=gray!20,draw=none](-7.486,1.854)--(-7.507,1.816)--(-7.453,1.835)--cycle;
\draw(-7.507,1.816)--(-7.453,1.835);
\filldraw[fill opacity=0.8,fill=gray!20,draw=none](-7.591,1.876)--(-7.616,1.857)--(-7.561,1.871)--cycle;
\filldraw[fill opacity=0.8,fill=gray!20,draw=none](-7.576,1.906)--(-7.616,1.904)--(-7.616,1.879)--(-7.561,1.871)--(-7.561,1.897)--cycle;
\draw(-7.576,1.906)--(-7.616,1.904)--(-7.616,1.879);
\draw(-7.561,1.871)--(-7.561,1.897);
\filldraw[fill opacity=0.8,fill=gray!20,draw=none](-7.585,.835)--(-7.556,.838)--(-7.6,.849)--(-7.6,.84)--cycle;
\draw(-7.6,.849)--(-7.6,.84);
\filldraw[fill opacity=0.8,fill=gray!20,draw=none](-3.426,2.107)--(-3.417,2.119)--(-3.415,2.118)--cycle;
\draw(-3.417,2.119)--(-3.415,2.118);
\filldraw[fill opacity=0.8,fill=gray!20,draw=none](-7.6,.885)--(-7.6,.849)--(-7.556,.838)--(-7.55,.838)--(-7.55,.858)--cycle;
\draw(-7.6,.885)--(-7.6,.849);
\draw(-7.55,.838)--(-7.55,.858);
\filldraw[fill opacity=0.8,fill=gray!20,draw=none](-3.37,3.24)--(-3.368,3.242)--(-3.373,3.248)--(-3.377,3.241)--cycle;
\draw(-3.368,3.242)--(-3.373,3.248);
\filldraw[fill opacity=0.8,fill=gray!20,draw=none](-7.249,1.003)--(-7.215,1.031)--(-7.279,.991)--cycle;
\draw(-7.215,1.031)--(-7.279,.991);
\filldraw[fill opacity=0.8,fill=gray!20](-2.53,5.835)--(-2.515,5.875)--(-2.454,5.887)--(-2.462,5.848)--cycle;
\filldraw[fill opacity=0.8,fill=gray!20,draw=none](-3.411,2.116)--(-3.417,2.119)--(-3.408,2.154)--cycle;
\draw(-3.411,2.116)--(-3.417,2.119);
\filldraw[fill opacity=0.8,fill=gray!20,draw=none](-7.673,1.859)--(-7.673,1.851)--(-7.641,1.845)--(-7.616,1.857)--(-7.616,1.868)--cycle;
\draw(-7.673,1.859)--(-7.673,1.851);
\draw(-7.616,1.857)--(-7.616,1.868);
\filldraw[fill opacity=0.8,fill=gray!20,draw=none](-7.447,.911)--(-7.424,.918)--(-7.437,.945)--cycle;
\draw(-7.447,.911)--(-7.424,.918);
\filldraw[fill opacity=0.8,fill=gray!20,draw=none](-7.414,.938)--(-7.438,.92)--(-7.438,.909)--(-7.393,.924)--(-7.393,.932)--cycle;
\draw(-7.438,.92)--(-7.438,.909);
\draw(-7.393,.924)--(-7.393,.932);
\filldraw[fill opacity=0.8,fill=gray!20,draw=none](-3.184,2.21)--(-3.173,2.126)--(-3.049,2.155)--(-3.052,2.178)--cycle;
\draw(-3.184,2.21)--(-3.173,2.126);
\draw(-3.049,2.155)--(-3.052,2.178);
\filldraw[fill opacity=0.8,fill=gray!20,draw=none](-7.928,1.488)--(-7.939,1.486)--(-7.955,1.485)--(-7.952,1.487)--cycle;
\draw(-7.955,1.485)--(-7.952,1.487);
\filldraw[fill opacity=0.8,fill=gray!20,draw=none](-7.878,1.511)--(-7.899,1.498)--(-7.931,1.506)--(-7.919,1.562)--(-7.853,1.546)--cycle;
\draw(-7.899,1.498)--(-7.931,1.506)--(-7.919,1.562)--(-7.853,1.546);
\filldraw[fill opacity=0.8,fill=gray!20,draw=none](-7.96,1.483)--(-7.939,1.486)--(-7.942,1.477)--cycle;
\draw(-7.939,1.486)--(-7.942,1.477);
\filldraw[fill opacity=0.8,fill=gray!20,draw=none](-7.916,1.488)--(-7.938,1.474)--(-7.942,1.477)--(-7.939,1.486)--cycle;
\draw(-7.942,1.477)--(-7.939,1.486);
\filldraw[fill opacity=0.8,fill=gray!20,draw=none](-7.939,1.486)--(-7.909,1.491)--(-7.916,1.488)--cycle;
\draw(-7.909,1.491)--(-7.916,1.488);
\filldraw[fill opacity=0.8,fill=gray!20,draw=none](-7.916,1.488)--(-7.939,1.486)--(-7.931,1.506)--(-7.899,1.498)--cycle;
\draw(-7.939,1.486)--(-7.931,1.506)--(-7.899,1.498);
\filldraw[fill opacity=0.8,fill=gray!20,draw=none](-8.036,1.45)--(-7.96,1.483)--(-7.939,1.486)--(-7.916,1.488)--(-7.991,1.455)--cycle;
\draw(-7.916,1.488)--(-7.991,1.455)--(-8.036,1.45)--(-7.96,1.483);
\filldraw[fill opacity=0.8,fill=gray!20,draw=none](-7.991,1.455)--(-7.668,1.596)--(-7.724,1.578)--(-7.953,1.479)--cycle;
\draw(-7.724,1.578)--(-7.953,1.479)--(-7.991,1.455)--(-7.668,1.596);
\filldraw[fill opacity=0.8,fill=gray!20,draw=none](-7.641,1.809)--(-7.641,1.845)--(-7.688,1.825)--cycle;
\draw(-7.641,1.845)--(-7.688,1.825);
\filldraw[fill opacity=0.8,fill=gray!20,draw=none](-2.457,5.64)--(-2.47,5.66)--(-2.433,5.633)--(-2.434,5.631)--cycle;
\draw(-2.433,5.633)--(-2.434,5.631);
\filldraw[fill opacity=0.8,fill=gray!20,draw=none](-4.341,3.05)--(-2.484,5.649)--(-2.457,5.64)--(-2.443,5.618)--(-4.269,3.062)--cycle;
\draw(-4.341,3.05)--(-2.484,5.649);
\draw(-2.443,5.618)--(-4.269,3.062);
\filldraw[fill opacity=0.8,fill=gray!20,draw=none](-2.457,5.64)--(-2.434,5.631)--(-2.443,5.618)--cycle;
\draw(-2.434,5.631)--(-2.443,5.618);
\filldraw[fill opacity=0.8,fill=gray!20,draw=none](-2.434,5.631)--(-2.433,5.633)--(-2.43,5.629)--cycle;
\draw(-2.434,5.631)--(-2.433,5.633);
\filldraw[fill opacity=0.8,fill=gray!20,draw=none](-4.269,3.062)--(-2.434,5.631)--(-2.43,5.629)--(-2.406,5.592)--(-4.207,3.071)--cycle;
\draw(-4.269,3.062)--(-2.434,5.631);
\draw(-2.406,5.592)--(-4.207,3.071);
\filldraw[fill opacity=0.8,fill=gray!20](-2.459,5.623)--(-2.491,5.656)--(-2.441,5.666)--(-2.424,5.63)--cycle;
\filldraw[fill opacity=0.8,fill=gray!20,draw=none](-7.673,1.89)--(-7.673,1.859)--(-7.616,1.868)--(-7.616,1.879)--cycle;
\draw(-7.673,1.89)--(-7.673,1.859);
\draw(-7.616,1.868)--(-7.616,1.879);
\filldraw[fill opacity=0.8,fill=gray!20,draw=none](-7.576,1.906)--(-7.561,1.897)--(-7.561,1.907)--cycle;
\draw(-7.561,1.897)--(-7.561,1.907)--(-7.576,1.906);
\filldraw[fill opacity=0.8,fill=gray!20,draw=none](-7.393,.932)--(-7.406,.949)--(-7.41,.947)--(-7.414,.938)--cycle;
\draw(-7.406,.949)--(-7.41,.947);
\filldraw[fill opacity=0.8,fill=gray!20,draw=none](-7.538,1.077)--(-7.55,1.08)--(-7.55,1.06)--(-7.536,1.076)--cycle;
\draw(-7.55,1.08)--(-7.55,1.06);
\filldraw[fill opacity=0.8,fill=gray!20,draw=none](-7.962,.939)--(-7.962,.957)--(-7.955,.971)--(-7.935,.984)--(-7.87,.979)--(-7.858,.932)--cycle;
\draw(-7.935,.984)--(-7.87,.979)--(-7.858,.932)--(-7.962,.939)--(-7.962,.957);
\filldraw[fill opacity=0.8,fill=gray!20,draw=none](-7.858,.932)--(-7.87,.979)--(-7.821,.967)--(-7.801,.948)--(-7.795,.941)--(-7.785,.914)--cycle;
\draw(-7.795,.941)--(-7.785,.914)--(-7.858,.932)--(-7.87,.979)--(-7.821,.967);
\filldraw[fill opacity=0.8,fill=gray!20,draw=none](-7.935,.984)--(-7.922,.993)--(-7.879,.997)--(-7.87,.979)--cycle;
\draw(-7.879,.997)--(-7.87,.979)--(-7.935,.984);
\filldraw[fill opacity=0.8,fill=gray!20,draw=none](-7.821,.967)--(-7.87,.979)--(-7.879,.997)--(-7.878,.997)--(-7.839,.98)--cycle;
\draw(-7.821,.967)--(-7.87,.979)--(-7.879,.997);
\filldraw[fill opacity=0.8,fill=gray!20,draw=none](-7.933,.95)--(-7.527,1.081)--(-7.543,1.105)--(-7.974,.966)--cycle;
\draw(-7.543,1.105)--(-7.974,.966)--(-7.933,.95)--(-7.527,1.081);
\filldraw[fill opacity=0.8,fill=gray!20,draw=none](-3.305,3.16)--(-3.304,3.163)--(-3.306,3.184)--(-3.325,3.221)--(-3.334,3.226)--(-3.358,3.232)--(-3.378,3.178)--cycle;
\draw(-3.334,3.226)--(-3.358,3.232)--(-3.378,3.178)--(-3.305,3.16)--(-3.304,3.163);
\filldraw[fill opacity=0.8,fill=gray!20,draw=none](-3.334,3.226)--(-3.354,3.239)--(-3.358,3.232)--cycle;
\draw(-3.354,3.239)--(-3.358,3.232)--(-3.334,3.226);
\filldraw[fill opacity=0.8,fill=gray!20,draw=none](-4.249,3.017)--(-4.214,3.053)--(-4.128,3.071)--(-4.152,3.032)--(-4.213,3.02)--cycle;
\draw(-4.214,3.053)--(-4.128,3.071);
\draw(-4.152,3.032)--(-4.213,3.02);
\filldraw[fill opacity=0.8,fill=gray!20,draw=none](-4.182,3.076)--(-4.122,3.088)--(-4.128,3.071)--(-4.214,3.053)--cycle;
\draw(-4.182,3.076)--(-4.122,3.088);
\draw(-4.128,3.071)--(-4.214,3.053);
\filldraw[fill opacity=0.8,fill=gray!20,draw=none](-4.207,3.071)--(-4.182,3.076)--(-4.214,3.053)--(-4.26,3.044)--cycle;
\draw(-4.207,3.071)--(-4.182,3.076);
\draw(-4.214,3.053)--(-4.26,3.044);
\filldraw[fill opacity=0.8,fill=gray!20,draw=none](-4.207,3.071)--(-4.26,3.044)--(-4.286,3.039)--cycle;
\draw(-4.26,3.044)--(-4.286,3.039);
\filldraw[fill opacity=0.8,fill=gray!20,draw=none](-4.286,3.039)--(-4.269,3.062)--(-4.207,3.071)--cycle;
\draw(-4.286,3.039)--(-4.269,3.062);
\filldraw[fill opacity=0.8,fill=gray!20,draw=none](-4.33,3.031)--(-4.312,3.049)--(-4.207,3.071)--(-4.286,3.039)--(-4.33,3.03)--cycle;
\draw(-4.312,3.049)--(-4.207,3.071);
\draw(-4.286,3.039)--(-4.33,3.03);
\filldraw[fill opacity=0.8,fill=gray!20,draw=none](-4.3,2.956)--(-4.264,2.963)--(-4.245,2.915)--(-4.251,2.913)--cycle;
\draw(-4.3,2.956)--(-4.264,2.963);
\draw(-4.245,2.915)--(-4.251,2.913);
\filldraw[fill opacity=0.8,fill=gray!20,draw=none](-4.294,2.978)--(-4.302,3.016)--(-4.286,3.039)--(-4.207,3.071)--(-4.282,2.966)--cycle;
\draw(-4.302,3.016)--(-4.286,3.039);
\draw(-4.207,3.071)--(-4.282,2.966);
\filldraw[fill opacity=0.8,fill=gray!20,draw=none](-4.337,2.985)--(-4.321,2.984)--(-4.32,2.97)--cycle;
\draw(-4.337,2.985)--(-4.321,2.984)--(-4.32,2.97);
\filldraw[fill opacity=0.8,fill=gray!20,draw=none](-4.333,3.014)--(-4.286,3.039)--(-4.317,2.995)--cycle;
\draw(-4.286,3.039)--(-4.317,2.995);
\filldraw[fill opacity=0.8,fill=gray!20,draw=none](-4.29,2.968)--(-4.277,2.929)--(-4.32,2.97)--(-4.32,2.975)--cycle;
\draw(-4.32,2.97)--(-4.32,2.975);
\filldraw[fill opacity=0.8,fill=gray!20,draw=none](-4.29,2.968)--(-4.32,2.975)--(-4.321,2.984)--(-4.294,2.977)--cycle;
\draw(-4.32,2.975)--(-4.321,2.984)--(-4.294,2.977);
\filldraw[fill opacity=0.8,fill=gray!20,draw=none](-4.293,2.977)--(-4.321,2.984)--(-4.325,3)--cycle;
\draw(-4.293,2.977)--(-4.321,2.984)--(-4.325,3);
\filldraw[fill opacity=0.8,fill=gray!20,draw=none](-4.33,2.977)--(-4.302,3.016)--(-4.289,2.956)--(-4.296,2.946)--cycle;
\draw(-4.33,2.977)--(-4.302,3.016);
\draw(-4.289,2.956)--(-4.296,2.946);
\filldraw[fill opacity=0.8,fill=gray!20,draw=none](-4.258,2.96)--(-4.282,2.966)--(-4.207,3.071)--(-4.164,3.075)--(-4.235,2.975)--cycle;
\draw(-4.282,2.966)--(-4.207,3.071);
\draw(-4.164,3.075)--(-4.235,2.975);
\filldraw[fill opacity=0.8,fill=gray!20,draw=none](-4.293,2.977)--(-4.325,3)--(-4.332,3.026)--(-4.331,3.03)--(-4.33,3.031)--(-4.268,3.016)--(-4.248,2.966)--cycle;
\draw(-4.325,3)--(-4.332,3.026);
\draw(-4.33,3.031)--(-4.268,3.016)--(-4.248,2.966)--(-4.293,2.977);
\filldraw[fill opacity=0.8,fill=gray!20,draw=none](-4.338,3.01)--(-4.332,3.026)--(-4.329,3.015)--cycle;
\draw(-4.332,3.026)--(-4.329,3.015);
\filldraw[fill opacity=0.8,fill=gray!20,draw=none](-4.33,3.031)--(-4.33,3.03)--(-4.331,3.03)--cycle;
\draw(-4.33,3.03)--(-4.331,3.03);
\filldraw[fill opacity=0.8,fill=gray!20,draw=none](-4.32,3.048)--(-4.312,3.049)--(-4.33,3.031)--cycle;
\draw(-4.32,3.048)--(-4.312,3.049);
\filldraw[fill opacity=0.8,fill=gray!20,draw=none](-4.33,3.031)--(-4.32,3.048)--(-4.293,3.049)--(-4.268,3.016)--cycle;
\draw(-4.293,3.049)--(-4.268,3.016)--(-4.33,3.031);
\filldraw[fill opacity=0.8,fill=gray!20,draw=none](-4.235,2.975)--(-4.164,3.075)--(-4.162,3.051)--(-4.169,3.041)--cycle;
\draw(-4.235,2.975)--(-4.164,3.075);
\draw(-4.162,3.051)--(-4.169,3.041);
\filldraw[fill opacity=0.8,fill=gray!20,draw=none](-4.469,2.898)--(-4.438,2.89)--(-4.432,2.947)--(-4.433,2.949)--(-4.445,2.974)--(-4.464,3.003)--(-4.479,3.014)--(-4.486,3.008)--(-4.488,2.999)--(-4.488,2.985)--(-4.487,2.975)--(-4.483,2.95)--(-4.478,2.928)--(-4.474,2.914)--cycle;
\draw(-4.433,2.949)--(-4.445,2.974)--(-4.464,3.003)--(-4.479,3.014)--(-4.486,3.008);
\draw(-4.488,2.999)--(-4.488,2.985);
\draw(-4.487,2.975)--(-4.483,2.95);
\draw(-4.478,2.928)--(-4.474,2.914);
\filldraw[fill opacity=0.8,fill=gray!20,draw=none](-4.418,2.843)--(-4.379,2.864)--(-4.347,2.889)--(-4.339,2.897)--(-4.334,2.904)--(-4.438,2.89)--(-4.444,2.833)--(-4.432,2.838)--cycle;
\draw(-4.418,2.843)--(-4.379,2.864)--(-4.347,2.889);
\draw(-4.339,2.897)--(-4.334,2.904);
\draw(-4.444,2.833)--(-4.432,2.838);
\filldraw[fill opacity=0.8,fill=gray!20,draw=none](-4.348,2.928)--(-4.342,2.947)--(-4.344,2.978)--(-4.361,2.996)--(-4.389,2.997)--(-4.425,2.983)--(-4.443,2.969)--(-4.438,2.89)--cycle;
\draw(-4.348,2.928)--(-4.342,2.947)--(-4.344,2.978)--(-4.361,2.996)--(-4.389,2.997)--(-4.425,2.983)--(-4.443,2.969);
\filldraw[fill opacity=0.8,fill=gray!20,draw=none](-4.438,2.89)--(-4.45,2.981)--(-4.439,2.992)--(-4.407,3.009)--(-4.38,3.009)--(-4.362,2.99)--(-4.356,2.959)--(-4.358,2.941)--cycle;
\draw(-4.45,2.981)--(-4.439,2.992)--(-4.407,3.009)--(-4.38,3.009)--(-4.362,2.99)--(-4.356,2.959);
\filldraw[fill opacity=0.8,fill=gray!20,draw=none](-4.436,3.024)--(-4.429,3.01)--(-4.461,3.003)--cycle;
\draw(-4.429,3.01)--(-4.461,3.003);
\filldraw[fill opacity=0.8,fill=gray!20,draw=none](-4.439,2.992)--(-4.461,3.003)--(-4.47,3.02)--(-4.436,3.024)--(-4.407,3.009)--cycle;
\draw(-4.436,3.024)--(-4.407,3.009)--(-4.439,2.992)--(-4.461,3.003);
\filldraw[fill opacity=0.8,fill=gray!20,draw=none](-4.475,3.011)--(-4.47,3.017)--(-4.436,3.024)--(-4.461,3.003)--(-4.463,3.003)--cycle;
\draw(-4.475,3.011)--(-4.47,3.017)--(-4.436,3.024);
\draw(-4.461,3.003)--(-4.463,3.003);
\filldraw[fill opacity=0.8,fill=gray!20,draw=none](-4.249,2.996)--(-4.268,3.016)--(-4.293,3.049)--(-4.275,3.028)--cycle;
\draw(-4.249,2.996)--(-4.268,3.016)--(-4.293,3.049);
\filldraw[fill opacity=0.8,fill=gray!20,draw=none](-4.407,3.009)--(-4.436,3.024)--(-4.412,3.024)--(-4.38,3.009)--cycle;
\draw(-4.412,3.024)--(-4.38,3.009)--(-4.407,3.009)--(-4.436,3.024);
\filldraw[fill opacity=0.8,fill=gray!20,draw=none](-4.38,3.009)--(-4.412,3.024)--(-4.391,3.004)--(-4.362,2.99)--cycle;
\draw(-4.391,3.004)--(-4.362,2.99)--(-4.38,3.009)--(-4.412,3.024);
\filldraw[fill opacity=0.8,fill=gray!20](-4.425,2.998)--(-3.175,3.251)--(-3.198,3.271)--(-4.448,3.017)--cycle;
\filldraw[fill opacity=0.8,fill=gray!20,draw=none](-7.635,1.645)--(-7.582,1.668)--(-7.567,1.719)--(-7.635,1.689)--cycle;
\draw(-7.635,1.645)--(-7.582,1.668);
\draw(-7.567,1.719)--(-7.635,1.689);
\filldraw[fill opacity=0.8,fill=gray!20](-2.998,3.248)--(-3.033,3.293)--(-3.059,3.276)--(-3.029,3.228)--cycle;
\filldraw[fill opacity=0.8,fill=gray!20](-3.286,3.318)--(-3.239,3.345)--(-3.245,3.351)--(-3.298,3.331)--cycle;
\filldraw[fill opacity=0.8,fill=gray!20,draw=none](-7.454,.781)--(-7.438,.782)--(-7.438,.792)--cycle;
\draw(-7.454,.781)--(-7.438,.782)--(-7.438,.792);
\filldraw[fill opacity=0.8,fill=gray!20,draw=none](-7.458,.9)--(-7.467,.89)--(-7.438,.849)--(-7.438,.909)--cycle;
\draw(-7.438,.849)--(-7.438,.909);
\filldraw[fill opacity=0.8,fill=gray!20,draw=none](-7.422,.94)--(-7.419,.935)--(-7.414,.938)--cycle;
\filldraw[fill opacity=0.8,fill=gray!20,draw=none](-7.414,.938)--(-7.41,.947)--(-7.422,.94)--cycle;
\draw(-7.41,.947)--(-7.422,.94);
\filldraw[fill opacity=0.8,fill=gray!20,draw=none](-7.431,.957)--(-7.398,1.026)--(-7.438,1.029)--(-7.438,.971)--cycle;
\draw(-7.438,1.029)--(-7.438,.971);
\filldraw[fill opacity=0.8,fill=gray!20,draw=none](-7.431,.957)--(-7.428,.962)--(-7.43,.962)--cycle;
\draw(-7.428,.962)--(-7.43,.962);
\filldraw[fill opacity=0.8,fill=gray!20,draw=none](-7.43,.962)--(-7.431,.934)--(-7.419,.941)--cycle;
\draw(-7.431,.934)--(-7.419,.941);
\filldraw[fill opacity=0.8,fill=gray!20,draw=none](-7.422,.94)--(-7.437,.944)--(-7.438,.94)--(-7.438,.92)--(-7.419,.935)--cycle;
\draw(-7.438,.94)--(-7.438,.92);
\filldraw[fill opacity=0.8,fill=gray!20,draw=none](-7.431,.957)--(-7.437,.944)--(-7.422,.94)--cycle;
\filldraw[fill opacity=0.8,fill=gray!20,draw=none](-7.443,.988)--(-7.447,1.008)--(-7.45,.976)--(-7.443,.98)--cycle;
\draw(-7.45,.976)--(-7.443,.98);
\filldraw[fill opacity=0.8,fill=gray!20,draw=none](-7.441,.981)--(-7.43,.962)--(-7.428,.962)--(-7.443,.988)--cycle;
\draw(-7.43,.962)--(-7.428,.962);
\filldraw[fill opacity=0.8,fill=gray!20,draw=none](-7.373,1.023)--(-7.429,.988)--(-7.43,.962)--(-7.419,.941)--(-7.341,.99)--cycle;
\draw(-7.373,1.023)--(-7.429,.988);
\draw(-7.419,.941)--(-7.341,.99);
\filldraw[fill opacity=0.8,fill=gray!20,draw=none](-7.458,.9)--(-7.438,.909)--(-7.438,.92)--cycle;
\draw(-7.438,.909)--(-7.438,.92);
\filldraw[fill opacity=0.8,fill=gray!20,draw=none](-7.447,.911)--(-7.438,.92)--(-7.438,.94)--cycle;
\draw(-7.438,.92)--(-7.438,.94);
\filldraw[fill opacity=0.8,fill=gray!20,draw=none](-7.431,.957)--(-7.438,.971)--(-7.438,.94)--cycle;
\draw(-7.438,.971)--(-7.438,.94);
\filldraw[fill opacity=0.8,fill=gray!20,draw=none](-7.494,.992)--(-7.494,.931)--(-7.438,.945)--(-7.438,.971)--cycle;
\draw(-7.494,.992)--(-7.494,.931);
\draw(-7.438,.945)--(-7.438,.971);
\filldraw[fill opacity=0.8,fill=gray!20,draw=none](-7.491,.897)--(-7.447,.911)--(-7.437,.945)--(-7.443,.958)--(-7.473,.948)--cycle;
\draw(-7.491,.897)--(-7.447,.911);
\draw(-7.443,.958)--(-7.473,.948);
\filldraw[fill opacity=0.8,fill=gray!20,draw=none](-7.431,.957)--(-7.429,.988)--(-7.443,.98)--cycle;
\draw(-7.429,.988)--(-7.443,.98);
\filldraw[fill opacity=0.8,fill=gray!20,draw=none](-7.437,.944)--(-7.431,.957)--(-7.43,.962)--(-7.443,.958)--cycle;
\draw(-7.43,.962)--(-7.443,.958);
\filldraw[fill opacity=0.8,fill=gray!20,draw=none](-7.437,.96)--(-7.43,.962)--(-7.441,.981)--cycle;
\draw(-7.437,.96)--(-7.43,.962);
\filldraw[fill opacity=0.8,fill=gray!20,draw=none](-7.412,1.059)--(-7.441,1.041)--(-7.443,.98)--(-7.429,.988)--cycle;
\draw(-7.412,1.059)--(-7.441,1.041);
\draw(-7.443,.98)--(-7.429,.988);
\filldraw[fill opacity=0.8,fill=gray!20,draw=none](-7.473,.948)--(-7.437,.96)--(-7.441,.981)--(-7.455,1.007)--(-7.491,.995)--cycle;
\draw(-7.473,.948)--(-7.437,.96);
\draw(-7.455,1.007)--(-7.491,.995);
\filldraw[fill opacity=0.8,fill=gray!20,draw=none](-7.521,.97)--(-7.494,.927)--(-7.494,.992)--cycle;
\draw(-7.494,.927)--(-7.494,.992);
\filldraw[fill opacity=0.8,fill=gray!20,draw=none](-7.441,.981)--(-7.443,.988)--(-7.453,1.007)--(-7.455,1.007)--cycle;
\draw(-7.453,1.007)--(-7.455,1.007);
\filldraw[fill opacity=0.8,fill=gray!20,draw=none](-7.544,.987)--(-7.553,.939)--(-7.543,.886)--(-7.515,.842)--(-7.472,.816)--(-7.422,.811)--(-7.372,.827)--(-7.329,.863)--(-7.301,.912)--(-7.291,.968)--(-7.295,.99)--cycle;
\draw(-7.544,.987)--(-7.553,.939)--(-7.543,.886)--(-7.515,.842)--(-7.472,.816)--(-7.422,.811)--(-7.372,.827)--(-7.329,.863)--(-7.301,.912)--(-7.291,.968)--(-7.295,.99);
\filldraw[fill opacity=0.8,fill=gray!20,draw=none](-3.466,2.28)--(-3.468,2.282)--(-3.467,2.281)--cycle;
\draw(-3.468,2.282)--(-3.467,2.281);
\filldraw[fill opacity=0.8,fill=gray!20,draw=none](-3.717,2.118)--(-3.713,2.115)--(-3.715,2.116)--cycle;
\draw(-3.713,2.115)--(-3.715,2.116);
\filldraw[fill opacity=0.8,fill=gray!20,draw=none](-7.591,1.876)--(-7.616,1.879)--(-7.616,1.857)--cycle;
\draw(-7.616,1.879)--(-7.616,1.857);
\filldraw[fill opacity=0.8,fill=gray!20,draw=none](-7.635,1.154)--(-7.635,1.09)--(-7.6,1.09)--(-7.6,1.144)--cycle;
\draw(-7.6,1.09)--(-7.6,1.144)--(-7.635,1.154)--(-7.635,1.09);
\filldraw[fill opacity=0.8,fill=gray!20,draw=none](-7.6,.91)--(-7.6,.885)--(-7.55,.858)--cycle;
\draw(-7.6,.91)--(-7.6,.885);
\filldraw[fill opacity=0.8,fill=gray!20,draw=none](-8.066,1.567)--(-8.069,1.567)--(-8.071,1.574)--cycle;
\draw(-8.066,1.567)--(-8.069,1.567);
\filldraw[fill opacity=0.8,fill=gray!20,draw=none](-8.044,1.53)--(-8.069,1.545)--(-8.069,1.567)--(-8.066,1.567)--cycle;
\draw(-8.069,1.567)--(-8.066,1.567);
\filldraw[fill opacity=0.8,fill=gray!20,draw=none](-8.145,1.541)--(-7.964,1.62)--(-7.828,1.624)--(-8.119,1.497)--cycle;
\draw(-7.828,1.624)--(-8.119,1.497)--(-8.145,1.541)--(-7.964,1.62);
\filldraw[fill opacity=0.8,fill=gray!20,draw=none](-7.567,1.719)--(-7.56,1.742)--(-7.576,1.747)--cycle;
\filldraw[fill opacity=0.8,fill=gray!20,draw=none](-7.635,1.689)--(-7.567,1.719)--(-7.582,1.767)--(-7.635,1.743)--cycle;
\draw(-7.635,1.689)--(-7.567,1.719);
\draw(-7.582,1.767)--(-7.635,1.743);
\filldraw[fill opacity=0.8,fill=gray!20,draw=none](-3.405,2.239)--(-3.466,2.28)--(-3.467,2.281)--(-3.42,2.26)--cycle;
\draw(-3.467,2.281)--(-3.42,2.26)--(-3.405,2.239);
\filldraw[fill opacity=0.8,fill=gray!20,draw=none](-3.466,2.28)--(-3.405,2.239)--(-3.391,2.217)--(-3.419,2.229)--(-3.444,2.254)--cycle;
\draw(-3.405,2.239)--(-3.391,2.217)--(-3.419,2.229);
\filldraw[fill opacity=0.8,fill=gray!20,draw=none](-3.444,2.254)--(-3.419,2.229)--(-3.426,2.232)--cycle;
\draw(-3.419,2.229)--(-3.426,2.232);
\filldraw[fill opacity=0.8,fill=gray!20,draw=none](-3.713,2.116)--(-3.715,2.116)--(-3.713,2.115)--cycle;
\draw(-3.715,2.116)--(-3.713,2.115);
\filldraw[fill opacity=0.8,fill=gray!20,draw=none](-3.414,2.205)--(-3.426,2.232)--(-3.421,2.23)--cycle;
\draw(-3.426,2.232)--(-3.421,2.23);
\filldraw[fill opacity=0.8,fill=gray!20](-2.454,5.887)--(-2.441,5.919)--(-2.375,5.922)--(-2.373,5.891)--cycle;
\filldraw[fill opacity=0.8,fill=gray!20](-2.373,5.891)--(-2.375,5.922)--(-2.312,5.917)--(-2.296,5.885)--cycle;
\filldraw[fill opacity=0.8,fill=gray!20,draw=none](-7.673,1.905)--(-7.673,1.89)--(-7.616,1.879)--(-7.616,1.904)--cycle;
\draw(-7.616,1.879)--(-7.616,1.904)--(-7.673,1.905)--(-7.673,1.89);
\filldraw[fill opacity=0.8,fill=gray!20,draw=none](-7.59,1.85)--(-7.545,1.864)--(-7.549,1.866)--(-7.616,1.857)--(-7.599,1.85)--cycle;
\draw(-7.59,1.85)--(-7.545,1.864)--(-7.549,1.866);
\draw(-7.616,1.857)--(-7.599,1.85);
\filldraw[fill opacity=0.8,fill=gray!20,draw=none](-3.905,2.275)--(-3.715,2.116)--(-3.713,2.116)--(-3.71,2.133)--(-3.763,2.178)--cycle;
\draw(-3.905,2.275)--(-3.715,2.116);
\draw(-3.71,2.133)--(-3.763,2.178);
\filldraw[fill opacity=0.8,fill=gray!20,draw=none](-7.106,1.136)--(-7.257,1.042)--(-7.291,.983)--(-7.215,1.031)--cycle;
\draw(-7.106,1.136)--(-7.257,1.042);
\draw(-7.291,.983)--(-7.215,1.031);
\filldraw[fill opacity=0.8,fill=gray!20](-2.333,5.629)--(-2.312,5.664)--(-2.267,5.654)--(-2.301,5.621)--cycle;
\filldraw[fill opacity=0.8,fill=gray!20](-2.286,5.846)--(-2.296,5.885)--(-2.241,5.872)--(-2.225,5.831)--cycle;
\filldraw[fill opacity=0.8,fill=gray!20,draw=none](-3.414,2.205)--(-3.4,2.172)--(-3.406,2.174)--cycle;
\draw(-3.4,2.172)--(-3.406,2.174);
\filldraw[fill opacity=0.8,fill=gray!20,draw=none](-3.408,2.154)--(-3.406,2.174)--(-3.402,2.173)--cycle;
\draw(-3.406,2.174)--(-3.402,2.173);
\filldraw[fill opacity=0.8,fill=gray!20,draw=none](-3.393,2.108)--(-3.411,2.116)--(-3.408,2.154)--(-3.402,2.173)--(-3.4,2.172)--cycle;
\draw(-3.393,2.108)--(-3.411,2.116);
\draw(-3.402,2.173)--(-3.4,2.172);
\filldraw[fill opacity=0.8,fill=gray!20,draw=none](-7.538,.84)--(-7.528,.845)--(-7.55,.858)--(-7.55,.838)--cycle;
\draw(-7.55,.858)--(-7.55,.838);
\filldraw[fill opacity=0.8,fill=gray!20,draw=none](-7.537,.84)--(-7.527,.843)--(-7.531,.849)--cycle;
\draw(-7.537,.84)--(-7.527,.843);
\filldraw[fill opacity=0.8,fill=gray!20,draw=none](-7.641,1.612)--(-7.636,1.617)--(-7.641,1.614)--cycle;
\draw(-7.636,1.617)--(-7.641,1.614);
\filldraw[fill opacity=0.8,fill=gray!20](-3.033,3.293)--(-3.079,3.328)--(-3.097,3.316)--(-3.059,3.276)--cycle;
\filldraw[fill opacity=0.8,fill=gray!20](-3.239,3.345)--(-3.188,3.358)--(-3.188,3.358)--(-3.245,3.351)--cycle;
\filldraw[fill opacity=0.8,fill=gray!20,draw=none](-7.55,.796)--(-7.55,.78)--(-7.494,.779)--(-7.494,.804)--cycle;
\draw(-7.55,.796)--(-7.55,.78)--(-7.494,.779)--(-7.494,.804);
\filldraw[fill opacity=0.8,fill=gray!20,draw=none](-2.47,5.66)--(-2.473,5.665)--(-2.403,5.763)--(-2.361,5.734)--(-2.433,5.633)--cycle;
\draw(-2.473,5.665)--(-2.403,5.763)--(-2.361,5.734)--(-2.433,5.633);
\filldraw[fill opacity=0.8,fill=gray!20,draw=none](-2.514,5.692)--(-2.517,5.696)--(-2.515,5.697)--(-2.511,5.69)--cycle;
\draw(-2.517,5.696)--(-2.515,5.697)--(-2.511,5.69);
\filldraw[fill opacity=0.8,fill=gray!20,draw=none](-2.514,5.692)--(-2.511,5.69)--(-2.508,5.685)--cycle;
\draw(-2.511,5.69)--(-2.508,5.685);
\filldraw[fill opacity=0.8,fill=gray!20,draw=none](-2.516,5.69)--(-2.442,5.793)--(-2.403,5.763)--(-2.473,5.665)--cycle;
\draw(-2.516,5.69)--(-2.442,5.793)--(-2.403,5.763)--(-2.473,5.665);
\filldraw[fill opacity=0.8,fill=gray!20,draw=none](-2.468,5.885)--(-2.5,5.896)--(-2.444,5.912)--(-2.454,5.887)--cycle;
\draw(-2.444,5.912)--(-2.454,5.887)--(-2.468,5.885);
\filldraw[fill opacity=0.8,fill=gray!20,draw=none](-2.409,5.906)--(-2.393,5.804)--(-2.442,5.793)--cycle;
\draw(-2.409,5.906)--(-2.393,5.804)--(-2.442,5.793);
\filldraw[fill opacity=0.8,fill=gray!20,draw=none](-2.442,5.793)--(-2.393,5.804)--(-2.342,5.81)--(-2.298,5.809)--(-2.267,5.802)--(-2.253,5.789)--(-2.259,5.773)--(-2.284,5.757)--(-2.324,5.741)--cycle;
\draw(-2.442,5.793)--(-2.393,5.804)--(-2.342,5.81)--(-2.298,5.809)--(-2.267,5.802)--(-2.253,5.789)--(-2.259,5.773)--(-2.284,5.757)--(-2.324,5.741);
\filldraw[fill opacity=0.8,fill=gray!20,draw=none](-2.468,5.885)--(-2.515,5.875)--(-2.5,5.896)--cycle;
\draw(-2.468,5.885)--(-2.515,5.875)--(-2.5,5.896);
\filldraw[fill opacity=0.8,fill=gray!20,draw=none](-2.465,5.935)--(-2.442,5.793)--(-2.485,5.8)--(-2.501,5.896)--cycle;
\draw(-2.465,5.935)--(-2.442,5.793);
\draw(-2.485,5.8)--(-2.501,5.896);
\filldraw[fill opacity=0.8,fill=gray!20,draw=none](-2.46,5.907)--(-2.444,5.918)--(-2.441,5.919)--(-2.444,5.912)--cycle;
\draw(-2.444,5.918)--(-2.441,5.919)--(-2.444,5.912);
\filldraw[fill opacity=0.8,fill=gray!20,draw=none](-2.411,5.92)--(-2.376,5.928)--(-2.375,5.922)--cycle;
\draw(-2.376,5.928)--(-2.375,5.922)--(-2.411,5.92);
\filldraw[fill opacity=0.8,fill=gray!20,draw=none](-2.335,5.919)--(-2.375,5.922)--(-2.376,5.925)--(-2.322,5.928)--(-2.321,5.926)--cycle;
\draw(-2.335,5.919)--(-2.375,5.922)--(-2.376,5.925);
\draw(-2.322,5.928)--(-2.321,5.926);
\filldraw[fill opacity=0.8,fill=gray!20,draw=none](-2.335,5.919)--(-2.321,5.926)--(-2.312,5.917)--cycle;
\draw(-2.321,5.926)--(-2.312,5.917)--(-2.335,5.919);
\filldraw[fill opacity=0.8,fill=gray!20,draw=none](-2.294,5.913)--(-2.312,5.917)--(-2.321,5.926)--(-2.286,5.92)--cycle;
\draw(-2.294,5.913)--(-2.312,5.917)--(-2.321,5.926);
\filldraw[fill opacity=0.8,fill=gray!20,draw=none](-2.317,5.926)--(-2.298,5.809)--(-2.342,5.81)--(-2.364,5.946)--cycle;
\draw(-2.317,5.926)--(-2.298,5.809)--(-2.342,5.81)--(-2.364,5.946);
\filldraw[fill opacity=0.8,fill=gray!20,draw=none](-2.361,5.925)--(-2.376,5.925)--(-2.376,5.928)--cycle;
\draw(-2.376,5.925)--(-2.376,5.928);
\filldraw[fill opacity=0.8,fill=gray!20,draw=none](-2.388,5.93)--(-2.361,5.925)--(-2.342,5.81)--(-2.393,5.804)--(-2.412,5.92)--cycle;
\draw(-2.361,5.925)--(-2.342,5.81)--(-2.393,5.804)--(-2.412,5.92);
\filldraw[fill opacity=0.8,fill=gray!20](-2.812,7.832)--(-2.815,7.889)--(-2.704,7.894)--(-2.704,7.837)--cycle;
\filldraw[fill opacity=0.8,fill=gray!20,draw=none](-2.646,7.699)--(-2.364,5.941)--(-2.412,5.92)--(-2.694,7.682)--cycle;
\draw(-2.646,7.699)--(-2.364,5.941);
\draw(-2.412,5.92)--(-2.694,7.682);
\filldraw[fill opacity=0.8,fill=gray!20,draw=none](-2.411,5.92)--(-2.441,5.919)--(-2.44,5.92)--cycle;
\draw(-2.411,5.92)--(-2.441,5.919)--(-2.44,5.92);
\filldraw[fill opacity=0.8,fill=gray!20,draw=none](-2.444,5.918)--(-2.44,5.92)--(-2.412,5.92)--(-2.409,5.906)--(-2.442,5.793)--(-2.46,5.907)--cycle;
\draw(-2.412,5.92)--(-2.409,5.906);
\draw(-2.442,5.793)--(-2.46,5.907);
\filldraw[fill opacity=0.8,fill=gray!20](-2.294,5.715)--(-2.278,5.697)--(-2.277,5.69)--(-2.293,5.694)--(-2.322,5.709)--(-2.361,5.734)--(-2.403,5.763)--(-2.442,5.793)--(-2.472,5.819)--(-2.488,5.837)--(-2.489,5.845)--(-2.473,5.84)--(-2.444,5.825)--(-2.405,5.801)--(-2.363,5.771)--(-2.324,5.741)--cycle;
\filldraw[fill opacity=0.8,fill=gray!20,draw=none](-3.713,2.116)--(-3.695,2.121)--(-3.71,2.133)--cycle;
\draw(-3.695,2.121)--(-3.71,2.133);
\filldraw[fill opacity=0.8,fill=gray!20,draw=none](-3.384,2.176)--(-3.381,2.164)--(-3.4,2.172)--(-3.414,2.205)--(-3.421,2.23)--(-3.419,2.229)--cycle;
\draw(-3.384,2.176)--(-3.381,2.164)--(-3.4,2.172);
\draw(-3.421,2.23)--(-3.419,2.229);
\filldraw[fill opacity=0.8,fill=gray!20,draw=none](-7.64,1.613)--(-7.624,1.622)--(-7.636,1.617)--cycle;
\draw(-7.624,1.622)--(-7.636,1.617);
\filldraw[fill opacity=0.8,fill=gray!20,draw=none](-7.64,1.613)--(-7.637,1.61)--(-7.636,1.609)--(-7.624,1.622)--cycle;
\draw(-7.64,1.613)--(-7.637,1.61)--(-7.636,1.609);
\filldraw[fill opacity=0.8,fill=gray!20,draw=none](-4.296,2.728)--(-4.246,2.687)--(-4.284,2.737)--cycle;
\draw(-4.296,2.728)--(-4.246,2.687);
\filldraw[fill opacity=0.8,fill=gray!20,draw=none](-4.301,2.725)--(-4.296,2.728)--(-4.298,2.729)--cycle;
\draw(-4.296,2.728)--(-4.298,2.729);
\filldraw[fill opacity=0.8,fill=gray!20,draw=none](-4.298,2.729)--(-4.296,2.728)--(-4.284,2.737)--(-4.287,2.742)--cycle;
\draw(-4.298,2.729)--(-4.296,2.728);
\filldraw[fill opacity=0.8,fill=gray!20,draw=none](-4.287,2.742)--(-4.307,2.721)--(-4.302,2.724)--cycle;
\draw(-4.307,2.721)--(-4.302,2.724);
\filldraw[fill opacity=0.8,fill=gray!20,draw=none](-4.341,2.85)--(-4.338,2.854)--(-4.338,2.865)--(-4.343,2.865)--cycle;
\draw(-4.341,2.85)--(-4.338,2.854);
\filldraw[fill opacity=0.8,fill=gray!20,draw=none](-4.338,2.854)--(-4.34,2.855)--(-4.342,2.845)--(-4.341,2.822)--(-4.338,2.826)--cycle;
\draw(-4.341,2.822)--(-4.338,2.826);
\filldraw[fill opacity=0.8,fill=gray!20,draw=none](-4.341,2.826)--(-4.314,2.831)--(-4.342,2.845)--cycle;
\draw(-4.341,2.826)--(-4.314,2.831);
\filldraw[fill opacity=0.8,fill=gray!20,draw=none](-4.338,2.854)--(-4.33,2.865)--(-4.338,2.865)--cycle;
\draw(-4.338,2.854)--(-4.33,2.865);
\filldraw[fill opacity=0.8,fill=gray!20,draw=none](-4.338,2.854)--(-4.338,2.865)--(-4.34,2.855)--cycle;
\filldraw[fill opacity=0.8,fill=gray!20,draw=none](-4.343,2.864)--(-4.345,2.85)--(-4.341,2.85)--cycle;
\filldraw[fill opacity=0.8,fill=gray!20,draw=none](-4.349,2.812)--(-4.347,2.815)--(-4.349,2.813)--(-4.352,2.808)--cycle;
\draw(-4.347,2.815)--(-4.349,2.813);
\filldraw[fill opacity=0.8,fill=gray!20,draw=none](-4.349,2.812)--(-4.347,2.816)--(-4.347,2.822)--(-4.35,2.826)--cycle;
\filldraw[fill opacity=0.8,fill=gray!20,draw=none](-4.347,2.822)--(-4.349,2.813)--(-4.347,2.815)--cycle;
\draw(-4.349,2.813)--(-4.347,2.815);
\filldraw[fill opacity=0.8,fill=gray!20,draw=none](-4.347,2.816)--(-4.345,2.819)--(-4.346,2.82)--cycle;
\filldraw[fill opacity=0.8,fill=gray!20,draw=none](-4.347,2.816)--(-4.346,2.82)--(-4.347,2.822)--cycle;
\filldraw[fill opacity=0.8,fill=gray!20,draw=none](-4.379,2.864)--(-3.888,1.821)--(-3.706,1.539)--(-4.343,2.892)--cycle;
\draw(-3.706,1.539)--(-4.343,2.892)--(-4.379,2.864)--(-3.888,1.821);
\filldraw[fill opacity=0.8,fill=gray!20,draw=none](-7.287,.99)--(-7.257,1.042)--(-7.29,1.021)--(-7.295,.99)--cycle;
\draw(-7.257,1.042)--(-7.29,1.021);
\filldraw[fill opacity=0.8,fill=gray!20,draw=none](-7.458,1.112)--(-7.485,1.119)--(-7.496,1.112)--cycle;
\draw(-7.485,1.119)--(-7.496,1.112);
\filldraw[fill opacity=0.8,fill=gray!20,draw=none](-7.55,.836)--(-7.55,.796)--(-7.494,.804)--(-7.494,.825)--cycle;
\draw(-7.55,.836)--(-7.55,.796);
\draw(-7.494,.804)--(-7.494,.825);
\filldraw[fill opacity=0.8,fill=gray!20](-3.192,3.338)--(-3.188,3.358)--(-3.188,3.358)--(-3.219,3.34)--cycle;
\filldraw[fill opacity=0.8,fill=gray!20](-3.141,3.344)--(-3.188,3.358)--(-3.188,3.358)--(-3.163,3.339)--cycle;
\filldraw[fill opacity=0.8,fill=gray!20](-3.163,3.339)--(-3.188,3.358)--(-3.188,3.358)--(-3.192,3.338)--cycle;
\filldraw[fill opacity=0.8,fill=gray!20](-3.219,3.34)--(-3.188,3.358)--(-3.188,3.358)--(-3.239,3.345)--cycle;
\filldraw[fill opacity=0.8,fill=gray!20](-3.131,3.35)--(-3.188,3.358)--(-3.188,3.358)--(-3.141,3.344)--cycle;
\filldraw[fill opacity=0.8,fill=gray!20](-3.079,3.328)--(-3.131,3.35)--(-3.141,3.344)--(-3.097,3.316)--cycle;
\filldraw[fill opacity=0.8,fill=gray!20](-3.195,2.934)--(-3.197,2.958)--(-3.274,2.964)--(-3.249,2.938)--cycle;
\filldraw[fill opacity=0.8,fill=gray!20,draw=none](-2.387,5.605)--(-2.395,5.608)--(-2.408,5.631)--(-2.377,5.632)--(-2.38,5.605)--cycle;
\draw(-2.408,5.631)--(-2.377,5.632)--(-2.38,5.605)--(-2.387,5.605);
\filldraw[fill opacity=0.8,fill=gray!20,draw=none](-2.395,5.608)--(-2.415,5.618)--(-2.424,5.63)--(-2.408,5.631)--cycle;
\draw(-2.415,5.618)--(-2.424,5.63)--(-2.408,5.631);
\filldraw[fill opacity=0.8,fill=gray!20,draw=none](-2.43,5.629)--(-2.424,5.63)--(-2.415,5.618)--cycle;
\draw(-2.43,5.629)--(-2.424,5.63)--(-2.415,5.618);
\filldraw[fill opacity=0.8,fill=gray!20,draw=none](-2.433,5.633)--(-2.361,5.734)--(-2.322,5.709)--(-2.406,5.592)--cycle;
\draw(-2.433,5.633)--(-2.361,5.734)--(-2.322,5.709)--(-2.406,5.592);
\filldraw[fill opacity=0.8,fill=gray!20,draw=none](-3.313,2.973)--(-3.289,2.968)--(-3.311,2.976)--cycle;
\draw(-3.313,2.973)--(-3.289,2.968);
\filldraw[fill opacity=0.8,fill=gray!20](-3.059,2.973)--(-3.029,3.014)--(-3.103,2.999)--(-3.119,2.962)--cycle;
\filldraw[fill opacity=0.8,fill=gray!20,draw=none](-7.494,.825)--(-7.494,.804)--(-7.468,.81)--cycle;
\draw(-7.494,.825)--(-7.494,.804);
\filldraw[fill opacity=0.8,fill=gray!20](-2.38,5.605)--(-2.377,5.632)--(-2.333,5.629)--(-2.357,5.603)--cycle;
\filldraw[fill opacity=0.8,fill=gray!20,draw=none](-3.384,2.176)--(-3.419,2.229)--(-3.391,2.217)--cycle;
\draw(-3.419,2.229)--(-3.391,2.217)--(-3.384,2.176);
\filldraw[fill opacity=0.8,fill=gray!20,draw=none](-3.391,2.108)--(-3.393,2.108)--(-3.4,2.172)--(-3.381,2.164)--cycle;
\draw(-3.4,2.172)--(-3.381,2.164)--(-3.391,2.108)--(-3.393,2.108);
\filldraw[fill opacity=0.8,fill=gray!20](-3.139,2.937)--(-3.119,2.962)--(-3.197,2.958)--(-3.195,2.934)--cycle;
\filldraw[fill opacity=0.8,fill=gray!20,draw=none](-3.264,2.835)--(-3.184,2.21)--(-3.052,2.178)--(-3.137,2.846)--cycle;
\draw(-3.052,2.178)--(-3.137,2.846)--(-3.264,2.835)--(-3.184,2.21);
\filldraw[fill opacity=0.8,fill=gray!20,draw=none](-5.276,2.27)--(-7.106,1.136)--(-7.215,1.031)--(-5.174,2.296)--cycle;
\draw(-5.276,2.27)--(-7.106,1.136);
\draw(-7.215,1.031)--(-5.174,2.296);
\filldraw[fill opacity=0.8,fill=gray!20,draw=none](-7.523,.842)--(-7.538,.84)--(-7.546,.835)--(-7.494,.825)--cycle;
\filldraw[fill opacity=0.8,fill=gray!20,draw=none](-7.878,.74)--(-7.61,.827)--(-7.592,.822)--(-7.839,.743)--cycle;
\draw(-7.878,.74)--(-7.61,.827);
\draw(-7.592,.822)--(-7.839,.743);
\filldraw[fill opacity=0.8,fill=gray!20,draw=none](-7.558,.833)--(-7.537,.84)--(-7.531,.849)--(-7.533,.853)--(-7.628,.822)--cycle;
\draw(-7.558,.833)--(-7.537,.84);
\draw(-7.533,.853)--(-7.628,.822);
\filldraw[fill opacity=0.8,fill=gray!20,draw=none](-7.636,1.743)--(-7.582,1.767)--(-7.624,1.804)--(-7.636,1.798)--cycle;
\draw(-7.636,1.743)--(-7.582,1.767);
\draw(-7.624,1.804)--(-7.636,1.798);
\filldraw[fill opacity=0.8,fill=gray!20,draw=none](-3.365,3.24)--(-3.368,3.242)--(-3.37,3.24)--cycle;
\draw(-3.365,3.24)--(-3.368,3.242);
\filldraw[fill opacity=0.8,fill=gray!20,draw=none](-3.311,2.976)--(-3.289,2.968)--(-3.274,2.964)--(-3.285,2.987)--(-3.306,2.986)--cycle;
\draw(-3.289,2.968)--(-3.274,2.964)--(-3.285,2.987);
\filldraw[fill opacity=0.8,fill=gray!20,draw=none](-3.28,3.041)--(-3.282,3.048)--(-3.305,3.049)--(-3.295,3.008)--cycle;
\draw(-3.282,3.048)--(-3.305,3.049)--(-3.295,3.008);
\filldraw[fill opacity=0.8,fill=gray!20,draw=none](-4.002,2.861)--(-3.139,3.036)--(-3.13,3.071)--(-3.966,2.902)--cycle;
\draw(-4.002,2.861)--(-3.139,3.036)--(-3.13,3.071)--(-3.966,2.902);
\filldraw[fill opacity=0.8,fill=gray!20,draw=none](-7.538,.84)--(-7.523,.842)--(-7.528,.845)--cycle;
\filldraw[fill opacity=0.8,fill=gray!20,draw=none](-2.518,5.695)--(-2.546,5.732)--(-2.53,5.742)--(-2.515,5.697)--cycle;
\draw(-2.546,5.732)--(-2.53,5.742)--(-2.515,5.697)--(-2.518,5.695);
\filldraw[fill opacity=0.8,fill=gray!20,draw=none](-2.518,5.695)--(-2.517,5.696)--(-2.514,5.692)--cycle;
\draw(-2.518,5.695)--(-2.517,5.696);
\filldraw[fill opacity=0.8,fill=gray!20,draw=none](-2.564,5.689)--(-2.472,5.819)--(-2.442,5.793)--(-2.516,5.69)--cycle;
\draw(-2.564,5.689)--(-2.472,5.819)--(-2.442,5.793)--(-2.516,5.69);
\filldraw[fill opacity=0.8,fill=gray!20,draw=none](-2.373,5.73)--(-2.423,5.725)--(-2.468,5.725)--(-2.499,5.733)--(-2.513,5.745)--(-2.507,5.761)--(-2.482,5.778)--(-2.442,5.793)--(-2.324,5.741)--cycle;
\draw(-2.324,5.741)--(-2.373,5.73)--(-2.423,5.725)--(-2.468,5.725)--(-2.499,5.733)--(-2.513,5.745)--(-2.507,5.761)--(-2.482,5.778)--(-2.442,5.793);
\filldraw[fill opacity=0.8,fill=gray!20,draw=none](-2.442,5.793)--(-2.482,5.778)--(-2.485,5.8)--cycle;
\draw(-2.442,5.793)--(-2.482,5.778)--(-2.485,5.8);
\filldraw[fill opacity=0.8,fill=gray!20,draw=none](-7.397,1.146)--(-7.397,1.145)--(-7.393,1.139)--(-7.393,1.147)--cycle;
\draw(-7.393,1.139)--(-7.393,1.147)--(-7.397,1.146);
\filldraw[fill opacity=0.8,fill=gray!20,draw=none](-4.676,2.719)--(-5.079,2.576)--(-5.004,2.567)--(-4.61,2.707)--cycle;
\draw(-4.676,2.719)--(-5.079,2.576);
\draw(-5.004,2.567)--(-4.61,2.707);
\filldraw[fill opacity=0.8,fill=gray!20,draw=none](-5.178,2.549)--(-7.433,1.152)--(-7.371,1.143)--(-4.925,2.659)--cycle;
\draw(-5.178,2.549)--(-7.433,1.152);
\draw(-7.371,1.143)--(-4.925,2.659);
\filldraw[fill opacity=0.8,fill=gray!20,draw=none](-3.368,3.242)--(-3.365,3.24)--(-3.354,3.239)--(-3.336,3.266)--cycle;
\draw(-3.368,3.242)--(-3.365,3.24);
\draw(-3.354,3.239)--(-3.336,3.266);
\filldraw[fill opacity=0.8,fill=gray!20,draw=none](-3.45,2.858)--(-3.453,2.859)--(-3.45,2.831)--cycle;
\draw(-3.45,2.858)--(-3.453,2.859)--(-3.45,2.831);
\filldraw[fill opacity=0.8,fill=gray!20,draw=none](-7.974,.966)--(-7.61,1.084)--(-7.627,1.088)--(-8.016,.962)--cycle;
\draw(-7.627,1.088)--(-8.016,.962)--(-7.974,.966)--(-7.61,1.084);
\filldraw[fill opacity=0.8,fill=gray!20](-2.357,5.603)--(-2.333,5.629)--(-2.301,5.621)--(-2.341,5.599)--cycle;
\filldraw[fill opacity=0.8,fill=gray!20](-2.296,5.885)--(-2.312,5.917)--(-2.267,5.906)--(-2.241,5.872)--cycle;
\filldraw[fill opacity=0.8,fill=gray!20,draw=none](-2.422,5.6)--(-2.459,5.623)--(-2.43,5.629)--(-2.415,5.618)--(-2.404,5.604)--cycle;
\draw(-2.415,5.618)--(-2.404,5.604)--(-2.422,5.6)--(-2.459,5.623)--(-2.43,5.629);
\filldraw[fill opacity=0.8,fill=gray!20,draw=none](-7.568,1.045)--(-7.573,1.053)--(-7.59,1.048)--(-7.6,1.02)--cycle;
\filldraw[fill opacity=0.8,fill=gray!20,draw=none](-7.59,1.048)--(-7.6,1.045)--(-7.6,1.02)--cycle;
\draw(-7.6,1.045)--(-7.6,1.02);
\filldraw[fill opacity=0.8,fill=gray!20,draw=none](-2.562,5.723)--(-2.555,5.73)--(-2.551,5.729)--(-2.542,5.72)--(-2.562,5.693)--cycle;
\draw(-2.542,5.72)--(-2.562,5.693);
\filldraw[fill opacity=0.8,fill=gray!20,draw=none](-2.551,5.729)--(-2.555,5.73)--(-2.553,5.732)--cycle;
\filldraw[fill opacity=0.8,fill=gray!20,draw=none](-2.553,5.732)--(-2.562,5.723)--(-2.562,5.734)--(-2.559,5.737)--cycle;
\draw(-2.562,5.734)--(-2.559,5.737);
\filldraw[fill opacity=0.8,fill=gray!20,draw=none](-2.561,5.736)--(-2.559,5.737)--(-2.562,5.734)--cycle;
\draw(-2.559,5.737)--(-2.562,5.734);
\filldraw[fill opacity=0.8,fill=gray!20](-2.559,5.723)--(-2.566,5.77)--(-2.536,5.789)--(-2.53,5.742)--cycle;
\filldraw[fill opacity=0.8,fill=gray!20,draw=none](-7.641,1.614)--(-7.636,1.617)--(-7.636,1.644)--(-7.661,1.634)--cycle;
\draw(-7.641,1.614)--(-7.636,1.617);
\draw(-7.636,1.644)--(-7.661,1.634);
\filldraw[fill opacity=0.8,fill=gray!20,draw=none](-7.42,1.102)--(-7.416,1.115)--(-7.432,1.105)--cycle;
\draw(-7.416,1.115)--(-7.432,1.105);
\filldraw[fill opacity=0.8,fill=gray!20,draw=none](-7.438,1.129)--(-7.438,1.104)--(-7.393,1.1)--cycle;
\draw(-7.438,1.129)--(-7.438,1.104);
\filldraw[fill opacity=0.8,fill=gray!20,draw=none](-2.512,5.643)--(-2.541,5.68)--(-2.518,5.695)--(-2.514,5.692)--(-2.508,5.685)--(-2.491,5.656)--cycle;
\draw(-2.508,5.685)--(-2.491,5.656)--(-2.512,5.643)--(-2.541,5.68)--(-2.518,5.695);
\filldraw[fill opacity=0.8,fill=gray!20](-3.249,3.309)--(-3.219,3.34)--(-3.239,3.345)--(-3.286,3.318)--cycle;
\filldraw[fill opacity=0.8,fill=gray!20,draw=none](-4.448,3.017)--(-3.42,3.226)--(-3.401,3.234)--(-4.47,3.017)--cycle;
\draw(-3.401,3.234)--(-4.47,3.017)--(-4.448,3.017)--(-3.42,3.226);
\filldraw[fill opacity=0.8,fill=gray!20,draw=none](-2.395,5.608)--(-2.387,5.605)--(-2.392,5.604)--cycle;
\draw(-2.387,5.605)--(-2.392,5.604);
\filldraw[fill opacity=0.8,fill=gray!20,draw=none](-2.392,5.604)--(-2.38,5.605)--(-2.381,5.599)--cycle;
\draw(-2.392,5.604)--(-2.38,5.605)--(-2.381,5.599);
\filldraw[fill opacity=0.8,fill=gray!20,draw=none](-2.381,5.599)--(-2.38,5.605)--(-2.357,5.603)--(-2.372,5.595)--cycle;
\draw(-2.381,5.599)--(-2.38,5.605)--(-2.357,5.603)--(-2.372,5.595);
\filldraw[fill opacity=0.8,fill=gray!20,draw=none](-2.367,5.597)--(-2.357,5.603)--(-2.341,5.599)--(-2.365,5.593)--cycle;
\draw(-2.367,5.597)--(-2.357,5.603)--(-2.341,5.599)--(-2.365,5.593);
\filldraw[fill opacity=0.8,fill=gray!20,draw=none](-2.372,5.595)--(-2.367,5.597)--(-2.365,5.593)--cycle;
\draw(-2.372,5.595)--(-2.367,5.597);
\filldraw[fill opacity=0.8,fill=gray!20,draw=none](-3.764,3.691)--(-2.322,5.709)--(-2.293,5.694)--(-3.764,3.635)--cycle;
\draw(-3.764,3.691)--(-2.322,5.709)--(-2.293,5.694)--(-3.764,3.635);
\filldraw[fill opacity=0.8,fill=gray!20,draw=none](-7.59,1.85)--(-7.599,1.85)--(-7.595,1.848)--cycle;
\draw(-7.599,1.85)--(-7.595,1.848)--(-7.59,1.85);
\filldraw[fill opacity=0.8,fill=gray!20,draw=none](-5.318,2.297)--(-7.373,1.023)--(-7.341,.99)--(-5.276,2.27)--cycle;
\draw(-5.318,2.297)--(-7.373,1.023);
\draw(-7.341,.99)--(-5.276,2.27);
\filldraw[fill opacity=0.8,fill=gray!20,draw=none](-7.443,.988)--(-7.447,1.009)--(-7.453,1.007)--cycle;
\draw(-7.447,1.009)--(-7.453,1.007);
\filldraw[fill opacity=0.8,fill=gray!20,draw=none](-7.376,1.081)--(-7.412,1.059)--(-7.429,.988)--(-7.377,1.021)--cycle;
\draw(-7.376,1.081)--(-7.412,1.059);
\draw(-7.429,.988)--(-7.377,1.021);
\filldraw[fill opacity=0.8,fill=gray!20,draw=none](-7.438,1.048)--(-7.444,1.039)--(-7.441,1.041)--cycle;
\draw(-7.444,1.039)--(-7.441,1.041);
\filldraw[fill opacity=0.8,fill=gray!20,draw=none](-7.37,1.085)--(-7.376,1.081)--(-7.377,1.021)--(-7.373,1.023)--cycle;
\draw(-7.37,1.085)--(-7.376,1.081);
\draw(-7.377,1.021)--(-7.373,1.023);
\filldraw[fill opacity=0.8,fill=gray!20,draw=none](-7.544,.987)--(-7.341,.99)--(-7.438,1.091)--(-7.472,1.08)--(-7.515,1.044)--(-7.543,.995)--cycle;
\draw(-7.438,1.091)--(-7.472,1.08)--(-7.515,1.044)--(-7.543,.995)--(-7.544,.987);
\filldraw[fill opacity=0.8,fill=gray!20](-3.309,3.104)--(-3.305,3.16)--(-3.378,3.178)--(-3.385,3.122)--cycle;
\filldraw[fill opacity=0.8,fill=gray!20,draw=none](-4.245,2.895)--(-4.239,2.894)--(-4.15,2.819)--(-4.194,2.835)--(-4.256,2.886)--cycle;
\draw(-4.239,2.894)--(-4.15,2.819);
\draw(-4.194,2.835)--(-4.256,2.886);
\filldraw[fill opacity=0.8,fill=gray!20,draw=none](-4.234,2.802)--(-4.16,2.817)--(-4.089,2.828)--(-4.105,2.824)--(-4.184,2.808)--cycle;
\draw(-4.234,2.802)--(-4.16,2.817);
\draw(-4.105,2.824)--(-4.184,2.808);
\filldraw[fill opacity=0.8,fill=gray!20,draw=none](-4.343,3.001)--(-4.338,3.01)--(-4.337,3.011)--(-4.327,3.006)--(-4.325,3)--cycle;
\draw(-4.327,3.006)--(-4.325,3);
\filldraw[fill opacity=0.8,fill=gray!20,draw=none](-4.357,3.001)--(-4.333,3.014)--(-4.317,2.995)--(-4.33,2.977)--cycle;
\draw(-4.317,2.995)--(-4.33,2.977);
\filldraw[fill opacity=0.8,fill=gray!20,draw=none](-4.337,3.011)--(-4.329,3.015)--(-4.327,3.006)--cycle;
\draw(-4.329,3.015)--(-4.327,3.006);
\filldraw[fill opacity=0.8,fill=gray!20,draw=none](-4.23,2.82)--(-4.16,2.817)--(-4.234,2.802)--cycle;
\draw(-4.16,2.817)--(-4.234,2.802);
\filldraw[fill opacity=0.8,fill=gray!20,draw=none](-4.266,2.9)--(-4.245,2.895)--(-4.256,2.886)--cycle;
\filldraw[fill opacity=0.8,fill=gray!20,draw=none](-4.245,2.895)--(-4.266,2.9)--(-4.281,2.918)--(-4.257,2.909)--(-4.243,2.897)--cycle;
\draw(-4.257,2.909)--(-4.243,2.897);
\filldraw[fill opacity=0.8,fill=gray!20,draw=none](-4.245,2.895)--(-4.243,2.897)--(-4.239,2.894)--cycle;
\draw(-4.243,2.897)--(-4.239,2.894);
\filldraw[fill opacity=0.8,fill=gray!20,draw=none](-4.256,2.857)--(-4.269,2.903)--(-4.243,2.897)--(-4.248,2.855)--cycle;
\draw(-4.243,2.897)--(-4.248,2.855)--(-4.256,2.857);
\filldraw[fill opacity=0.8,fill=gray!20,draw=none](-4.247,2.958)--(-4.235,2.975)--(-4.169,3.041)--(-4.216,2.976)--cycle;
\draw(-4.247,2.958)--(-4.235,2.975);
\draw(-4.169,3.041)--(-4.216,2.976);
\filldraw[fill opacity=0.8,fill=gray!20,draw=none](-4.25,2.966)--(-4.235,2.975)--(-4.247,2.958)--cycle;
\draw(-4.235,2.975)--(-4.247,2.958);
\filldraw[fill opacity=0.8,fill=gray!20,draw=none](-4.251,2.913)--(-4.245,2.915)--(-4.243,2.906)--cycle;
\draw(-4.251,2.913)--(-4.245,2.915);
\filldraw[fill opacity=0.8,fill=gray!20,draw=none](-4.258,2.96)--(-4.25,2.966)--(-4.247,2.958)--cycle;
\filldraw[fill opacity=0.8,fill=gray!20,draw=none](-4.263,2.917)--(-4.277,2.929)--(-4.294,2.977)--(-4.248,2.966)--(-4.242,2.912)--cycle;
\draw(-4.294,2.977)--(-4.248,2.966)--(-4.242,2.912)--(-4.263,2.917);
\filldraw[fill opacity=0.8,fill=gray!20,draw=none](-4.273,2.922)--(-4.257,2.909)--(-4.284,2.92)--cycle;
\draw(-4.273,2.922)--(-4.257,2.909);
\filldraw[fill opacity=0.8,fill=gray!20,draw=none](-4.233,2.885)--(-4.149,2.815)--(-4.15,2.819)--(-4.243,2.897)--cycle;
\draw(-4.233,2.885)--(-4.149,2.815);
\draw(-4.15,2.819)--(-4.243,2.897);
\filldraw[fill opacity=0.8,fill=gray!20,draw=none](-4.247,2.958)--(-4.23,2.968)--(-4.238,2.948)--cycle;
\filldraw[fill opacity=0.8,fill=gray!20,draw=none](-4.24,2.958)--(-4.248,2.966)--(-4.268,3.016)--(-4.264,3.012)--cycle;
\draw(-4.24,2.958)--(-4.248,2.966)--(-4.268,3.016)--(-4.264,3.012);
\filldraw[fill opacity=0.8,fill=gray!20,draw=none](-4.287,2.942)--(-4.258,2.96)--(-4.247,2.958)--(-4.283,2.907)--cycle;
\draw(-4.247,2.958)--(-4.283,2.907);
\filldraw[fill opacity=0.8,fill=gray!20,draw=none](-4.302,2.946)--(-4.273,2.922)--(-4.285,2.925)--cycle;
\draw(-4.302,2.946)--(-4.273,2.922);
\filldraw[fill opacity=0.8,fill=gray!20,draw=none](-4.269,2.918)--(-4.283,2.907)--(-4.273,2.922)--cycle;
\draw(-4.283,2.907)--(-4.273,2.922);
\filldraw[fill opacity=0.8,fill=gray!20,draw=none](-4.263,2.917)--(-4.242,2.912)--(-4.243,2.897)--cycle;
\draw(-4.263,2.917)--(-4.242,2.912)--(-4.243,2.897);
\filldraw[fill opacity=0.8,fill=gray!20,draw=none](-4.227,2.851)--(-4.227,2.832)--(-4.248,2.855)--(-4.242,2.912)--(-4.237,2.906)--cycle;
\draw(-4.227,2.832)--(-4.248,2.855)--(-4.242,2.912)--(-4.237,2.906);
\filldraw[fill opacity=0.8,fill=gray!20,draw=none](-4.249,2.899)--(-4.233,2.885)--(-4.243,2.897)--(-4.273,2.922)--cycle;
\draw(-4.249,2.899)--(-4.233,2.885);
\draw(-4.243,2.897)--(-4.273,2.922);
\filldraw[fill opacity=0.8,fill=gray!20,draw=none](-4.285,2.891)--(-4.283,2.907)--(-4.269,2.918)--(-4.263,2.91)--(-4.274,2.895)--cycle;
\draw(-4.263,2.91)--(-4.274,2.895);
\filldraw[fill opacity=0.8,fill=gray!20,draw=none](-4.344,2.978)--(-4.263,2.91)--(-4.273,2.922)--(-4.361,2.996)--cycle;
\draw(-4.273,2.922)--(-4.361,2.996)--(-4.344,2.978)--(-4.263,2.91);
\filldraw[fill opacity=0.8,fill=gray!20,draw=none](-4.238,2.948)--(-4.23,2.968)--(-4.227,2.969)--(-4.223,2.965)--(-4.237,2.946)--cycle;
\draw(-4.223,2.965)--(-4.237,2.946);
\filldraw[fill opacity=0.8,fill=gray!20,draw=none](-4.227,2.969)--(-4.216,2.976)--(-4.223,2.965)--cycle;
\draw(-4.216,2.976)--(-4.223,2.965);
\filldraw[fill opacity=0.8,fill=gray!20,draw=none](-4.235,2.952)--(-4.235,2.949)--(-4.162,3.051)--(-4.202,3.002)--(-4.219,2.979)--cycle;
\draw(-4.235,2.949)--(-4.162,3.051);
\draw(-4.202,3.002)--(-4.219,2.979);
\filldraw[fill opacity=0.8,fill=gray!20,draw=none](-4.234,2.958)--(-4.235,2.952)--(-4.24,2.958)--(-4.264,3.012)--(-4.249,2.996)--(-4.248,2.993)--(-4.245,2.988)--(-4.236,2.966)--cycle;
\draw(-4.235,2.952)--(-4.24,2.958);
\draw(-4.264,3.012)--(-4.249,2.996);
\draw(-4.245,2.988)--(-4.236,2.966);
\filldraw[fill opacity=0.8,fill=gray!20,draw=none](-4.285,2.891)--(-4.274,2.895)--(-4.286,2.877)--cycle;
\draw(-4.274,2.895)--(-4.286,2.877);
\filldraw[fill opacity=0.8,fill=gray!20,draw=none](-4.342,2.947)--(-4.279,2.895)--(-4.263,2.91)--(-4.344,2.978)--cycle;
\draw(-4.263,2.91)--(-4.344,2.978)--(-4.342,2.947)--(-4.279,2.895);
\filldraw[fill opacity=0.8,fill=gray!20,draw=none](-4.352,2.934)--(-4.342,2.919)--(-4.335,2.906)--(-4.334,2.904)--(-4.438,2.89)--(-4.418,2.987)--(-4.412,2.983)--(-4.385,2.968)--(-4.359,2.943)--cycle;
\draw(-4.342,2.919)--(-4.335,2.906);
\draw(-4.412,2.983)--(-4.385,2.968)--(-4.359,2.943);
\filldraw[fill opacity=0.8,fill=gray!20,draw=none](-4.362,2.99)--(-4.391,3.004)--(-4.378,2.967)--(-4.356,2.956)--cycle;
\draw(-4.378,2.967)--(-4.356,2.956)--(-4.362,2.99)--(-4.391,3.004);
\filldraw[fill opacity=0.8,fill=gray!20,draw=none](-4.418,2.987)--(-4.403,2.962)--(-4.39,2.962)--(-4.385,2.968)--cycle;
\draw(-4.39,2.962)--(-4.385,2.968)--(-4.418,2.987);
\filldraw[fill opacity=0.8,fill=gray!20,draw=none](-4.353,2.912)--(-4.35,2.919)--(-4.348,2.928)--(-4.438,2.89)--cycle;
\draw(-4.35,2.919)--(-4.348,2.928);
\filldraw[fill opacity=0.8,fill=gray!20,draw=none](-4.438,2.89)--(-4.358,2.941)--(-4.36,2.921)--cycle;
\draw(-4.358,2.941)--(-4.36,2.921);
\filldraw[fill opacity=0.8,fill=gray!20,draw=none](-4.357,2.941)--(-4.385,2.968)--(-4.39,2.962)--(-4.375,2.938)--cycle;
\draw(-4.357,2.941)--(-4.385,2.968)--(-4.39,2.962);
\filldraw[fill opacity=0.8,fill=gray!20](-4.403,2.962)--(-3.154,3.215)--(-3.175,3.251)--(-4.425,2.998)--cycle;
\filldraw[fill opacity=0.8,fill=gray!20,draw=none](-2.395,5.608)--(-2.392,5.604)--(-2.404,5.604)--(-2.415,5.618)--cycle;
\draw(-2.392,5.604)--(-2.404,5.604)--(-2.415,5.618);
\filldraw[fill opacity=0.8,fill=gray!20](-2.566,5.77)--(-2.559,5.816)--(-2.53,5.835)--(-2.536,5.789)--cycle;
\filldraw[fill opacity=0.8,fill=gray!20](-2.474,5.614)--(-2.512,5.643)--(-2.491,5.656)--(-2.459,5.623)--cycle;
\filldraw[fill opacity=0.8,fill=gray!20,draw=none](-4.242,2.798)--(-4.234,2.802)--(-4.184,2.808)--(-4.229,2.799)--cycle;
\draw(-4.184,2.808)--(-4.229,2.799);
\filldraw[fill opacity=0.8,fill=gray!20,draw=none](-4.256,2.857)--(-4.248,2.855)--(-4.253,2.842)--cycle;
\draw(-4.256,2.857)--(-4.248,2.855)--(-4.253,2.842);
\filldraw[fill opacity=0.8,fill=gray!20,draw=none](-4.168,2.818)--(-4.184,2.808)--(-4.149,2.815)--cycle;
\draw(-4.184,2.808)--(-4.149,2.815);
\filldraw[fill opacity=0.8,fill=gray!20,draw=none](-4.253,2.822)--(-4.253,2.842)--(-4.248,2.855)--(-4.232,2.838)--cycle;
\draw(-4.253,2.842)--(-4.248,2.855)--(-4.232,2.838);
\filldraw[fill opacity=0.8,fill=gray!20,draw=none](-4.227,2.851)--(-4.193,2.822)--(-4.149,2.815)--(-4.233,2.885)--cycle;
\draw(-4.227,2.851)--(-4.193,2.822);
\draw(-4.149,2.815)--(-4.233,2.885);
\filldraw[fill opacity=0.8,fill=gray!20,draw=none](-4.229,2.799)--(-4.184,2.808)--(-4.161,2.823)--(-4.193,2.822)--(-4.214,2.818)--cycle;
\draw(-4.229,2.799)--(-4.184,2.808);
\draw(-4.193,2.822)--(-4.214,2.818);
\filldraw[fill opacity=0.8,fill=gray!20,draw=none](-4.242,2.798)--(-4.229,2.799)--(-4.248,2.795)--cycle;
\draw(-4.229,2.799)--(-4.248,2.795);
\filldraw[fill opacity=0.8,fill=gray!20,draw=none](-4.237,2.813)--(-4.248,2.801)--(-4.252,2.804)--(-4.253,2.822)--(-4.232,2.838)--(-4.227,2.832)--(-4.227,2.829)--cycle;
\draw(-4.232,2.838)--(-4.227,2.832);
\filldraw[fill opacity=0.8,fill=gray!20,draw=none](-4.248,2.801)--(-4.248,2.795)--(-4.229,2.799)--(-4.214,2.818)--(-4.237,2.813)--cycle;
\draw(-4.248,2.795)--(-4.229,2.799);
\draw(-4.214,2.818)--(-4.237,2.813);
\filldraw[fill opacity=0.8,fill=gray!20,draw=none](-4.245,2.799)--(-4.248,2.801)--(-4.237,2.813)--cycle;
\filldraw[fill opacity=0.8,fill=gray!20,draw=none](-4.248,2.801)--(-4.237,2.813)--(-4.249,2.811)--cycle;
\draw(-4.237,2.813)--(-4.249,2.811);
\filldraw[fill opacity=0.8,fill=gray!20,draw=none](-4.213,2.826)--(-4.214,2.818)--(-4.193,2.822)--cycle;
\draw(-4.214,2.818)--(-4.193,2.822);
\filldraw[fill opacity=0.8,fill=gray!20,draw=none](-4.235,2.952)--(-4.237,2.946)--(-4.235,2.949)--cycle;
\draw(-4.237,2.946)--(-4.235,2.949);
\filldraw[fill opacity=0.8,fill=gray!20,draw=none](-4.243,2.927)--(-4.248,2.966)--(-4.235,2.952)--cycle;
\draw(-4.243,2.927)--(-4.248,2.966)--(-4.235,2.952);
\filldraw[fill opacity=0.8,fill=gray!20,draw=none](-4.263,2.91)--(-4.26,2.909)--(-4.273,2.922)--cycle;
\filldraw[fill opacity=0.8,fill=gray!20,draw=none](-4.269,2.918)--(-4.244,2.936)--(-4.263,2.91)--cycle;
\draw(-4.244,2.936)--(-4.263,2.91);
\filldraw[fill opacity=0.8,fill=gray!20,draw=none](-4.286,2.877)--(-4.244,2.936)--(-4.243,2.937)--(-4.279,2.895)--(-4.285,2.887)--cycle;
\draw(-4.286,2.877)--(-4.244,2.936);
\draw(-4.279,2.895)--(-4.285,2.887);
\filldraw[fill opacity=0.8,fill=gray!20,draw=none](-4.227,2.829)--(-4.257,2.836)--(-4.249,2.811)--(-4.231,2.814)--cycle;
\draw(-4.249,2.811)--(-4.231,2.814);
\filldraw[fill opacity=0.8,fill=gray!20,draw=none](-4.245,2.799)--(-4.237,2.813)--(-4.232,2.818)--(-4.241,2.796)--cycle;
\draw(-4.232,2.818)--(-4.241,2.796);
\filldraw[fill opacity=0.8,fill=gray!20,draw=none](-4.227,2.829)--(-4.231,2.814)--(-4.214,2.818)--(-4.213,2.826)--cycle;
\draw(-4.231,2.814)--(-4.214,2.818);
\filldraw[fill opacity=0.8,fill=gray!20,draw=none](-4.286,2.877)--(-4.285,2.887)--(-4.298,2.869)--cycle;
\draw(-4.285,2.887)--(-4.298,2.869);
\filldraw[fill opacity=0.8,fill=gray!20,draw=none](-4.237,2.813)--(-4.227,2.829)--(-4.232,2.818)--cycle;
\draw(-4.227,2.829)--(-4.232,2.818);
\filldraw[fill opacity=0.8,fill=gray!20,draw=none](-4.241,2.812)--(-4.243,2.808)--(-4.241,2.796)--(-4.227,2.829)--cycle;
\draw(-4.241,2.796)--(-4.227,2.829);
\filldraw[fill opacity=0.8,fill=gray!20,draw=none](-4.268,2.807)--(-4.249,2.811)--(-4.257,2.836)--(-4.273,2.839)--(-4.292,2.836)--cycle;
\draw(-4.268,2.807)--(-4.249,2.811);
\draw(-4.273,2.839)--(-4.292,2.836);
\filldraw[fill opacity=0.8,fill=gray!20,draw=none](-4.359,2.919)--(-4.34,2.896)--(-4.438,2.89)--(-4.401,2.954)--(-4.395,2.95)--(-4.371,2.932)--cycle;
\draw(-4.359,2.919)--(-4.34,2.896);
\draw(-4.395,2.95)--(-4.371,2.932);
\filldraw[fill opacity=0.8,fill=gray!20,draw=none](-4.311,2.882)--(-4.298,2.869)--(-4.279,2.895)--cycle;
\draw(-4.298,2.869)--(-4.279,2.895);
\filldraw[fill opacity=0.8,fill=gray!20,draw=none](-4.298,2.869)--(-4.311,2.882)--(-4.32,2.879)--(-4.33,2.865)--cycle;
\draw(-4.32,2.879)--(-4.33,2.865);
\filldraw[fill opacity=0.8,fill=gray!20,draw=none](-4.335,2.906)--(-4.326,2.889)--(-4.326,2.886)--cycle;
\draw(-4.335,2.906)--(-4.326,2.889)--(-4.326,2.886);
\filldraw[fill opacity=0.8,fill=gray!20,draw=none](-4.336,2.876)--(-4.326,2.886)--(-4.326,2.889)--(-4.335,2.879)--cycle;
\draw(-4.326,2.886)--(-4.326,2.889)--(-4.335,2.879);
\filldraw[fill opacity=0.8,fill=gray!20,draw=none](-4.328,2.893)--(-4.334,2.888)--(-4.331,2.883)--(-4.326,2.889)--cycle;
\draw(-4.331,2.883)--(-4.326,2.889)--(-4.328,2.893);
\filldraw[fill opacity=0.8,fill=gray!20,draw=none](-4.347,2.889)--(-4.343,2.892)--(-4.339,2.897)--cycle;
\draw(-4.347,2.889)--(-4.343,2.892)--(-4.339,2.897);
\filldraw[fill opacity=0.8,fill=gray!20,draw=none](-4.334,2.888)--(-4.328,2.893)--(-4.342,2.919)--(-4.349,2.909)--cycle;
\draw(-4.328,2.893)--(-4.342,2.919);
\filldraw[fill opacity=0.8,fill=gray!20,draw=none](-4.34,2.896)--(-4.353,2.912)--(-4.359,2.899)--(-4.355,2.895)--(-4.347,2.89)--cycle;
\draw(-4.34,2.896)--(-4.353,2.912);
\filldraw[fill opacity=0.8,fill=gray!20,draw=none](-4.355,2.895)--(-4.342,2.879)--(-4.339,2.884)--cycle;
\draw(-4.342,2.879)--(-4.339,2.884);
\filldraw[fill opacity=0.8,fill=gray!20,draw=none](-4.338,2.885)--(-4.34,2.896)--(-4.349,2.909)--(-4.352,2.905)--(-4.341,2.883)--cycle;
\filldraw[fill opacity=0.8,fill=gray!20,draw=none](-4.354,2.907)--(-3.55,2.235)--(-3.516,2.257)--(-4.342,2.947)--cycle;
\draw(-3.516,2.257)--(-4.342,2.947)--(-4.354,2.907)--(-3.55,2.235);
\filldraw[fill opacity=0.8,fill=gray!20,draw=none](-3.55,2.235)--(-3.52,2.21)--(-3.516,2.257)--cycle;
\draw(-3.55,2.235)--(-3.52,2.21);
\filldraw[fill opacity=0.8,fill=gray!20](-3.097,3.316)--(-3.141,3.344)--(-3.163,3.339)--(-3.139,3.308)--cycle;
\filldraw[fill opacity=0.8,fill=gray!20,draw=none](-2.551,5.729)--(-2.54,5.724)--(-2.542,5.72)--cycle;
\draw(-2.54,5.724)--(-2.542,5.72);
\filldraw[fill opacity=0.8,fill=gray!20,draw=none](-2.518,5.695)--(-2.541,5.68)--(-2.559,5.723)--(-2.546,5.732)--cycle;
\draw(-2.518,5.695)--(-2.541,5.68)--(-2.559,5.723)--(-2.546,5.732);
\filldraw[fill opacity=0.8,fill=gray!20](-3.029,3.014)--(-3.011,3.063)--(-3.093,3.047)--(-3.103,2.999)--cycle;
\filldraw[fill opacity=0.8,fill=gray!20,draw=none](-7.724,1.578)--(-7.641,1.614)--(-7.661,1.634)--(-7.828,1.561)--cycle;
\draw(-7.724,1.578)--(-7.641,1.614);
\draw(-7.661,1.634)--(-7.828,1.561);
\filldraw[fill opacity=0.8,fill=gray!20](-3.305,3.049)--(-3.309,3.104)--(-3.385,3.122)--(-3.378,3.067)--cycle;
\filldraw[fill opacity=0.8,fill=gray!20,draw=none](-3.636,2.243)--(-3.579,2.196)--(-3.52,2.21)--(-3.601,2.278)--cycle;
\draw(-3.636,2.243)--(-3.579,2.196);
\draw(-3.52,2.21)--(-3.601,2.278);
\filldraw[fill opacity=0.8,fill=gray!20,draw=none](-3.579,2.196)--(-3.535,2.159)--(-3.52,2.21)--cycle;
\draw(-3.579,2.196)--(-3.535,2.159);
\filldraw[fill opacity=0.8,fill=gray!20](-3.643,2.135)--(-3.633,2.081)--(-3.605,2.038)--(-3.562,2.012)--(-3.512,2.006)--(-3.462,2.023)--(-3.42,2.058)--(-3.391,2.108)--(-3.381,2.164)--(-3.391,2.217)--(-3.42,2.26)--(-3.462,2.287)--(-3.512,2.292)--(-3.562,2.276)--(-3.605,2.24)--(-3.633,2.191)--cycle;
\filldraw[fill opacity=0.8,fill=gray!20,draw=none](-2.294,5.913)--(-2.286,5.92)--(-2.267,5.906)--cycle;
\draw(-2.286,5.92)--(-2.267,5.906)--(-2.294,5.913);
\filldraw[fill opacity=0.8,fill=gray!20,draw=none](-2.295,5.976)--(-2.267,5.802)--(-2.298,5.809)--(-2.319,5.941)--cycle;
\draw(-2.295,5.976)--(-2.267,5.802)--(-2.298,5.809)--(-2.319,5.941);
\filldraw[fill opacity=0.8,fill=gray!20,draw=none](-3.306,2.986)--(-3.285,2.987)--(-3.293,3.002)--(-3.296,3.003)--cycle;
\draw(-3.285,2.987)--(-3.293,3.002)--(-3.296,3.003);
\filldraw[fill opacity=0.8,fill=gray!20,draw=none](-2.44,5.92)--(-2.437,5.921)--(-2.412,5.92)--cycle;
\filldraw[fill opacity=0.8,fill=gray!20,draw=none](-2.421,5.98)--(-2.412,5.92)--(-2.463,5.922)--(-2.463,5.927)--cycle;
\draw(-2.421,5.98)--(-2.412,5.92);
\draw(-2.463,5.922)--(-2.463,5.927);
\filldraw[fill opacity=0.8,fill=gray!20,draw=none](-2.388,5.93)--(-2.376,5.936)--(-2.362,5.931)--(-2.361,5.925)--cycle;
\draw(-2.362,5.931)--(-2.361,5.925);
\filldraw[fill opacity=0.8,fill=gray!20,draw=none](-2.44,5.92)--(-2.442,5.92)--(-2.44,5.921)--(-2.437,5.921)--cycle;
\filldraw[fill opacity=0.8,fill=gray!20,draw=none](-2.411,5.92)--(-2.44,5.92)--(-2.424,5.94)--(-2.377,5.942)--(-2.376,5.928)--cycle;
\draw(-2.44,5.92)--(-2.424,5.94)--(-2.377,5.942)--(-2.376,5.928);
\filldraw[fill opacity=0.8,fill=gray!20,draw=none](-7.595,1.091)--(-7.6,1.098)--(-7.6,1.09)--cycle;
\draw(-7.6,1.098)--(-7.6,1.09);
\filldraw[fill opacity=0.8,fill=gray!20,draw=none](-7.397,1.146)--(-7.415,1.115)--(-7.371,1.143)--cycle;
\draw(-7.415,1.115)--(-7.371,1.143);
\filldraw[fill opacity=0.8,fill=gray!20,draw=none](-7.397,1.146)--(-7.398,1.146)--(-7.397,1.145)--cycle;
\draw(-7.397,1.146)--(-7.398,1.146);
\filldraw[fill opacity=0.8,fill=gray!20,draw=none](-3.558,2.312)--(-3.564,2.315)--(-3.572,2.303)--cycle;
\draw(-3.558,2.312)--(-3.564,2.315);
\filldraw[fill opacity=0.8,fill=gray!20,draw=none](-7.853,1.656)--(-7.919,1.673)--(-7.931,1.72)--(-7.899,1.712)--(-7.878,1.693)--cycle;
\draw(-7.853,1.656)--(-7.919,1.673)--(-7.931,1.72)--(-7.899,1.712);
\filldraw[fill opacity=0.8,fill=gray!20,draw=none](-7.916,1.725)--(-7.899,1.712)--(-7.931,1.72)--(-7.939,1.735)--cycle;
\draw(-7.899,1.712)--(-7.931,1.72)--(-7.939,1.735);
\filldraw[fill opacity=0.8,fill=gray!20,draw=none](-7.991,1.692)--(-7.694,1.822)--(-7.723,1.844)--(-8.036,1.707)--cycle;
\draw(-7.723,1.844)--(-8.036,1.707)--(-7.991,1.692)--(-7.694,1.822);
\filldraw[fill opacity=0.8,fill=gray!20](-2.241,5.693)--(-2.225,5.738)--(-2.207,5.719)--(-2.225,5.676)--cycle;
\filldraw[fill opacity=0.8,fill=gray!20](-2.267,5.654)--(-2.241,5.693)--(-2.225,5.676)--(-2.254,5.639)--cycle;
\filldraw[fill opacity=0.8,fill=gray!20,draw=none](-7.6,1.105)--(-7.6,1.098)--(-7.599,1.097)--(-7.585,1.092)--(-7.55,1.095)--(-7.55,1.136)--cycle;
\draw(-7.6,1.105)--(-7.6,1.098);
\draw(-7.55,1.095)--(-7.55,1.136);
\filldraw[fill opacity=0.8,fill=gray!20,draw=none](-7.599,1.097)--(-7.595,1.091)--(-7.585,1.092)--cycle;
\filldraw[fill opacity=0.8,fill=gray!20,draw=none](-7.64,1.809)--(-7.636,1.798)--(-7.624,1.804)--cycle;
\draw(-7.636,1.798)--(-7.624,1.804);
\filldraw[fill opacity=0.8,fill=gray!20](-2.225,5.738)--(-2.219,5.785)--(-2.2,5.765)--(-2.207,5.719)--cycle;
\filldraw[fill opacity=0.8,fill=gray!20,draw=none](-7.801,.766)--(-7.821,.753)--(-7.87,.765)--(-7.858,.821)--(-7.785,.803)--(-7.795,.773)--cycle;
\draw(-7.821,.753)--(-7.87,.765)--(-7.858,.821)--(-7.785,.803)--(-7.795,.773);
\filldraw[fill opacity=0.8,fill=gray!20,draw=none](-7.861,.73)--(-7.879,.741)--(-7.87,.765)--(-7.821,.753)--cycle;
\draw(-7.879,.741)--(-7.87,.765)--(-7.821,.753);
\filldraw[fill opacity=0.8,fill=gray!20,draw=none](-7.888,.714)--(-7.889,.714)--(-7.879,.74)--(-7.878,.74)--(-7.861,.73)--cycle;
\draw(-7.888,.714)--(-7.889,.714)--(-7.879,.74);
\filldraw[fill opacity=0.8,fill=gray!20,draw=none](-7.974,.709)--(-7.878,.74)--(-7.839,.743)--(-7.933,.713)--cycle;
\draw(-7.839,.743)--(-7.933,.713)--(-7.974,.709)--(-7.878,.74);
\filldraw[fill opacity=0.8,fill=gray!20,draw=none](-7.933,.713)--(-7.558,.833)--(-7.628,.822)--(-7.897,.735)--cycle;
\draw(-7.628,.822)--(-7.897,.735)--(-7.933,.713)--(-7.558,.833);
\filldraw[fill opacity=0.8,fill=gray!20,draw=none](-7.592,1.143)--(-7.6,1.137)--(-7.6,1.105)--(-7.55,1.136)--(-7.55,1.138)--cycle;
\draw(-7.6,1.137)--(-7.6,1.105);
\draw(-7.55,1.136)--(-7.55,1.138)--(-7.592,1.143);
\filldraw[fill opacity=0.8,fill=gray!20,draw=none](-2.543,5.826)--(-2.539,5.86)--(-2.517,5.874)--(-2.516,5.874)--(-2.53,5.835)--cycle;
\draw(-2.539,5.86)--(-2.517,5.874);
\draw(-2.516,5.874)--(-2.53,5.835)--(-2.543,5.826);
\filldraw[fill opacity=0.8,fill=gray!20,draw=none](-2.505,5.78)--(-2.513,5.802)--(-2.485,5.8)--cycle;
\filldraw[fill opacity=0.8,fill=gray!20,draw=none](-2.505,5.78)--(-2.485,5.8)--(-2.503,5.775)--cycle;
\draw(-2.485,5.8)--(-2.503,5.775);
\filldraw[fill opacity=0.8,fill=gray!20,draw=none](-2.551,5.729)--(-2.553,5.732)--(-2.505,5.78)--(-2.503,5.775)--(-2.54,5.724)--cycle;
\draw(-2.503,5.775)--(-2.54,5.724);
\filldraw[fill opacity=0.8,fill=gray!20,draw=none](-2.505,5.78)--(-2.553,5.732)--(-2.559,5.737)--(-2.513,5.802)--cycle;
\draw(-2.559,5.737)--(-2.513,5.802);
\filldraw[fill opacity=0.8,fill=gray!20,draw=none](-2.517,5.874)--(-2.515,5.875)--(-2.516,5.874)--cycle;
\draw(-2.517,5.874)--(-2.515,5.875)--(-2.516,5.874);
\filldraw[fill opacity=0.8,fill=gray!20,draw=none](-2.517,5.874)--(-2.525,5.874)--(-2.5,5.896)--(-2.515,5.875)--cycle;
\draw(-2.5,5.896)--(-2.515,5.875)--(-2.517,5.874);
\filldraw[fill opacity=0.8,fill=gray!20,draw=none](-2.517,5.874)--(-2.539,5.86)--(-2.538,5.862)--(-2.525,5.874)--cycle;
\draw(-2.517,5.874)--(-2.539,5.86);
\filldraw[fill opacity=0.8,fill=gray!20,draw=none](-2.485,5.8)--(-2.482,5.778)--(-2.507,5.761)--(-2.533,5.926)--cycle;
\draw(-2.485,5.8)--(-2.482,5.778)--(-2.507,5.761)--(-2.533,5.926);
\filldraw[fill opacity=0.8,fill=gray!20,draw=none](-7.64,1.809)--(-7.641,1.814)--(-7.641,1.809)--cycle;
\filldraw[fill opacity=0.8,fill=gray!20,draw=none](-7.64,1.809)--(-7.637,1.812)--(-7.641,1.814)--cycle;
\draw(-7.64,1.809)--(-7.637,1.812)--(-7.641,1.814);
\filldraw[fill opacity=0.8,fill=gray!20,draw=none](-3.687,2.148)--(-3.763,2.178)--(-3.695,2.121)--cycle;
\draw(-3.763,2.178)--(-3.695,2.121);
\filldraw[fill opacity=0.8,fill=gray!20](-2.301,5.621)--(-2.267,5.654)--(-2.254,5.639)--(-2.292,5.611)--cycle;
\filldraw[fill opacity=0.8,fill=gray!20,draw=none](-2.444,5.918)--(-2.44,5.92)--(-2.441,5.919)--cycle;
\draw(-2.44,5.92)--(-2.441,5.919)--(-2.444,5.918);
\filldraw[fill opacity=0.8,fill=gray!20](-3.274,3.267)--(-3.249,3.309)--(-3.286,3.318)--(-3.327,3.28)--cycle;
\filldraw[fill opacity=0.8,fill=gray!20,draw=none](-7.405,1.145)--(-7.438,1.139)--(-7.438,1.129)--(-7.416,1.115)--cycle;
\draw(-7.405,1.145)--(-7.438,1.139)--(-7.438,1.129);
\filldraw[fill opacity=0.8,fill=gray!20](-2.43,5.595)--(-2.474,5.614)--(-2.459,5.623)--(-2.422,5.6)--cycle;
\filldraw[fill opacity=0.8,fill=gray!20,draw=none](-2.376,5.936)--(-2.364,5.941)--(-2.362,5.931)--cycle;
\draw(-2.364,5.941)--(-2.362,5.931);
\filldraw[fill opacity=0.8,fill=gray!20,draw=none](-2.665,7.778)--(-2.704,7.837)--(-2.6,7.83)--(-2.612,7.774)--cycle;
\draw(-2.704,7.837)--(-2.6,7.83)--(-2.612,7.774)--(-2.665,7.778);
\filldraw[fill opacity=0.8,fill=gray!20,draw=none](-2.574,7.525)--(-2.317,5.926)--(-2.364,5.946)--(-2.625,7.568)--cycle;
\draw(-2.574,7.525)--(-2.317,5.926);
\draw(-2.364,5.946)--(-2.625,7.568);
\filldraw[fill opacity=0.8,fill=gray!20,draw=none](-2.361,5.925)--(-2.376,5.928)--(-2.377,5.942)--(-2.333,5.938)--(-2.322,5.928)--cycle;
\draw(-2.376,5.928)--(-2.377,5.942)--(-2.333,5.938)--(-2.322,5.928);
\filldraw[fill opacity=0.8,fill=gray!20,draw=none](-7.573,1.053)--(-7.568,1.045)--(-7.55,1.06)--cycle;
\filldraw[fill opacity=0.8,fill=gray!20,draw=none](-2.44,5.921)--(-2.444,5.918)--(-2.461,5.911)--(-2.463,5.922)--cycle;
\draw(-2.461,5.911)--(-2.463,5.922);
\filldraw[fill opacity=0.8,fill=gray!20](-2.802,7.776)--(-2.812,7.832)--(-2.704,7.837)--(-2.705,7.781)--cycle;
\filldraw[fill opacity=0.8,fill=gray!20,draw=none](-2.727,7.707)--(-2.699,7.711)--(-2.421,5.98)--(-2.463,5.927)--(-2.748,7.697)--cycle;
\draw(-2.699,7.711)--(-2.421,5.98);
\draw(-2.463,5.927)--(-2.748,7.697);
\filldraw[fill opacity=0.8,fill=gray!20,draw=none](-2.754,7.699)--(-2.748,7.7)--(-2.465,5.935)--(-2.501,5.896)--(-2.786,7.669)--cycle;
\draw(-2.748,7.7)--(-2.465,5.935);
\draw(-2.501,5.896)--(-2.786,7.669);
\filldraw[fill opacity=0.8,fill=gray!20,draw=none](-2.444,5.918)--(-2.491,5.909)--(-2.459,5.933)--(-2.424,5.94)--(-2.44,5.92)--cycle;
\draw(-2.444,5.918)--(-2.491,5.909)--(-2.459,5.933)--(-2.424,5.94)--(-2.44,5.92);
\filldraw[fill opacity=0.8,fill=gray!20,draw=none](-2.444,5.918)--(-2.442,5.92)--(-2.44,5.92)--cycle;
\filldraw[fill opacity=0.8,fill=gray!20,draw=none](-4.688,2.679)--(-5.004,2.567)--(-4.895,2.589)--(-4.635,2.681)--cycle;
\draw(-4.688,2.679)--(-5.004,2.567);
\draw(-4.895,2.589)--(-4.635,2.681);
\filldraw[fill opacity=0.8,fill=gray!20,draw=none](-4.59,2.736)--(-4.6,2.721)--(-4.643,2.737)--(-4.626,2.762)--cycle;
\draw(-4.59,2.736)--(-4.6,2.721);
\draw(-4.643,2.737)--(-4.626,2.762);
\filldraw[fill opacity=0.8,fill=gray!20,draw=none](-4.679,2.767)--(-4.877,2.696)--(-4.768,2.686)--(-4.649,2.728)--cycle;
\draw(-4.679,2.767)--(-4.877,2.696);
\draw(-4.768,2.686)--(-4.649,2.728);
\filldraw[fill opacity=0.8,fill=gray!20,draw=none](-4.637,2.612)--(-5.028,2.37)--(-4.86,2.481)--(-4.578,2.655)--cycle;
\draw(-4.637,2.612)--(-5.028,2.37);
\draw(-4.86,2.481)--(-4.578,2.655);
\filldraw[fill opacity=0.8,fill=gray!20,draw=none](-4.825,2.727)--(-4.88,2.697)--(-4.877,2.696)--(-4.808,2.721)--cycle;
\draw(-4.877,2.696)--(-4.808,2.721);
\filldraw[fill opacity=0.8,fill=gray!20,draw=none](-4.501,2.707)--(-4.507,2.703)--(-4.516,2.709)--(-4.518,2.72)--(-4.511,2.721)--cycle;
\draw(-4.507,2.703)--(-4.516,2.709);
\draw(-4.518,2.72)--(-4.511,2.721);
\filldraw[fill opacity=0.8,fill=gray!20,draw=none](-4.517,2.723)--(-4.516,2.72)--(-4.518,2.72)--cycle;
\draw(-4.516,2.72)--(-4.518,2.72);
\filldraw[fill opacity=0.8,fill=gray!20,draw=none](-4.513,2.703)--(-4.515,2.699)--(-4.516,2.709)--cycle;
\filldraw[fill opacity=0.8,fill=gray!20,draw=none](-4.51,2.698)--(-4.516,2.709)--(-4.506,2.703)--cycle;
\draw(-4.516,2.709)--(-4.506,2.703);
\filldraw[fill opacity=0.8,fill=gray!20,draw=none](-4.506,2.717)--(-4.513,2.703)--(-4.516,2.709)--(-4.518,2.722)--(-4.513,2.725)--cycle;
\draw(-4.518,2.722)--(-4.513,2.725);
\filldraw[fill opacity=0.8,fill=gray!20,draw=none](-4.516,2.709)--(-4.53,2.718)--(-4.518,2.72)--cycle;
\draw(-4.516,2.709)--(-4.53,2.718)--(-4.518,2.72);
\filldraw[fill opacity=0.8,fill=gray!20,draw=none](-4.492,2.694)--(-4.507,2.703)--(-4.501,2.707)--cycle;
\draw(-4.492,2.694)--(-4.507,2.703);
\filldraw[fill opacity=0.8,fill=gray!20,draw=none](-4.493,2.685)--(-4.495,2.684)--(-4.505,2.688)--(-4.51,2.698)--(-4.506,2.703)--(-4.489,2.692)--cycle;
\draw(-4.493,2.685)--(-4.495,2.684)--(-4.505,2.688);
\draw(-4.506,2.703)--(-4.489,2.692);
\filldraw[fill opacity=0.8,fill=gray!20,draw=none](-4.497,2.707)--(-4.497,2.706)--(-4.517,2.693)--(-4.506,2.717)--cycle;
\draw(-4.497,2.706)--(-4.517,2.693);
\filldraw[fill opacity=0.8,fill=gray!20,draw=none](-4.51,2.698)--(-4.505,2.688)--(-4.514,2.692)--cycle;
\draw(-4.505,2.688)--(-4.514,2.692);
\filldraw[fill opacity=0.8,fill=gray!20,draw=none](-4.514,2.688)--(-4.517,2.686)--(-4.525,2.687)--(-4.517,2.693)--(-4.514,2.695)--cycle;
\draw(-4.514,2.688)--(-4.517,2.686);
\draw(-4.517,2.693)--(-4.514,2.695);
\filldraw[fill opacity=0.8,fill=gray!20,draw=none](-4.545,2.669)--(-4.637,2.612)--(-4.578,2.655)--(-4.517,2.693)--cycle;
\draw(-4.545,2.669)--(-4.637,2.612);
\draw(-4.578,2.655)--(-4.517,2.693);
\filldraw[fill opacity=0.8,fill=gray!20,draw=none](-4.519,2.689)--(-4.545,2.704)--(-4.547,2.706)--(-4.5,2.686)--cycle;
\draw(-4.545,2.704)--(-4.547,2.706)--(-4.5,2.686);
\filldraw[fill opacity=0.8,fill=gray!20,draw=none](-4.506,2.694)--(-4.514,2.688)--(-4.514,2.695)--(-4.497,2.706)--cycle;
\draw(-4.506,2.694)--(-4.514,2.688);
\draw(-4.514,2.695)--(-4.497,2.706);
\filldraw[fill opacity=0.8,fill=gray!20,draw=none](-4.51,2.698)--(-4.514,2.692)--(-4.547,2.706)--(-4.53,2.718)--(-4.516,2.709)--cycle;
\draw(-4.514,2.692)--(-4.547,2.706)--(-4.53,2.718)--(-4.516,2.709);
\filldraw[fill opacity=0.8,fill=gray!20,draw=none](-4.509,2.73)--(-4.526,2.724)--(-4.516,2.738)--cycle;
\draw(-4.509,2.73)--(-4.526,2.724);
\filldraw[fill opacity=0.8,fill=gray!20,draw=none](-4.591,2.738)--(-4.577,2.724)--(-4.623,2.78)--(-4.628,2.785)--(-4.601,2.751)--cycle;
\draw(-4.591,2.738)--(-4.577,2.724);
\draw(-4.623,2.78)--(-4.628,2.785)--(-4.601,2.751);
\filldraw[fill opacity=0.8,fill=gray!20,draw=none](-4.563,2.716)--(-4.564,2.712)--(-4.569,2.711)--(-4.585,2.732)--cycle;
\filldraw[fill opacity=0.8,fill=gray!20,draw=none](-4.559,2.712)--(-4.572,2.72)--(-4.576,2.723)--(-4.577,2.724)--(-4.585,2.732)--cycle;
\draw(-4.577,2.724)--(-4.585,2.732);
\filldraw[fill opacity=0.8,fill=gray!20,draw=none](-4.577,2.724)--(-4.577,2.724)--(-4.579,2.727)--(-4.613,2.77)--(-4.623,2.78)--cycle;
\draw(-4.577,2.724)--(-4.577,2.724);
\draw(-4.613,2.77)--(-4.623,2.78);
\filldraw[fill opacity=0.8,fill=gray!20,draw=none](-4.591,2.738)--(-4.601,2.751)--(-4.593,2.741)--cycle;
\draw(-4.601,2.751)--(-4.593,2.741)--(-4.591,2.738);
\filldraw[fill opacity=0.8,fill=gray!20,draw=none](-4.589,2.737)--(-4.59,2.736)--(-4.626,2.762)--(-4.618,2.775)--cycle;
\draw(-4.589,2.737)--(-4.59,2.736);
\draw(-4.626,2.762)--(-4.618,2.775);
\filldraw[fill opacity=0.8,fill=gray!20,draw=none](-4.626,2.786)--(-4.618,2.775)--(-4.637,2.747)--(-4.652,2.75)--(-4.651,2.76)--cycle;
\draw(-4.618,2.775)--(-4.637,2.747);
\filldraw[fill opacity=0.8,fill=gray!20,draw=none](-4.603,2.758)--(-4.611,2.749)--(-4.636,2.767)--(-4.644,2.779)--(-4.631,2.794)--cycle;
\draw(-4.603,2.758)--(-4.611,2.749);
\draw(-4.644,2.779)--(-4.631,2.794);
\filldraw[fill opacity=0.8,fill=gray!20,draw=none](-4.726,2.778)--(-4.715,2.789)--(-4.747,2.769)--cycle;
\draw(-4.715,2.789)--(-4.747,2.769);
\filldraw[fill opacity=0.8,fill=gray!20,draw=none](-4.651,2.76)--(-4.654,2.756)--(-4.657,2.774)--(-4.648,2.788)--cycle;
\draw(-4.657,2.774)--(-4.648,2.788);
\filldraw[fill opacity=0.8,fill=gray!20,draw=none](-4.684,2.803)--(-4.825,2.727)--(-4.808,2.721)--(-4.65,2.777)--cycle;
\draw(-4.808,2.721)--(-4.65,2.777);
\filldraw[fill opacity=0.8,fill=gray!20,draw=none](-4.6,2.721)--(-4.609,2.707)--(-4.649,2.728)--(-4.643,2.737)--cycle;
\draw(-4.6,2.721)--(-4.609,2.707);
\draw(-4.649,2.728)--(-4.643,2.737);
\filldraw[fill opacity=0.8,fill=gray!20,draw=none](-4.598,2.751)--(-4.609,2.736)--(-4.619,2.739)--(-4.603,2.758)--cycle;
\draw(-4.619,2.739)--(-4.603,2.758);
\filldraw[fill opacity=0.8,fill=gray!20,draw=none](-4.637,2.747)--(-4.645,2.735)--(-4.651,2.739)--(-4.653,2.75)--cycle;
\draw(-4.637,2.747)--(-4.645,2.735);
\filldraw[fill opacity=0.8,fill=gray!20,draw=none](-4.611,2.749)--(-4.619,2.739)--(-4.636,2.767)--cycle;
\draw(-4.611,2.749)--(-4.619,2.739);
\filldraw[fill opacity=0.8,fill=gray!20,draw=none](-4.645,2.735)--(-4.649,2.728)--(-4.651,2.739)--cycle;
\draw(-4.645,2.735)--(-4.649,2.728);
\filldraw[fill opacity=0.8,fill=gray!20,draw=none](-4.651,2.76)--(-4.652,2.75)--(-4.653,2.75)--(-4.654,2.756)--cycle;
\filldraw[fill opacity=0.8,fill=gray!20,draw=none](-4.65,2.777)--(-4.679,2.767)--(-4.649,2.728)--(-4.606,2.744)--cycle;
\draw(-4.65,2.777)--(-4.679,2.767);
\draw(-4.649,2.728)--(-4.606,2.744);
\filldraw[fill opacity=0.8,fill=gray!20,draw=none](-4.649,2.728)--(-4.659,2.744)--(-4.657,2.774)--cycle;
\filldraw[fill opacity=0.8,fill=gray!20,draw=none](-4.579,2.727)--(-4.608,2.764)--(-4.613,2.77)--cycle;
\draw(-4.579,2.727)--(-4.608,2.764)--(-4.613,2.77);
\filldraw[fill opacity=0.8,fill=gray!20,draw=none](-4.569,2.711)--(-4.609,2.707)--(-4.589,2.737)--cycle;
\draw(-4.609,2.707)--(-4.589,2.737);
\filldraw[fill opacity=0.8,fill=gray!20,draw=none](-4.579,2.727)--(-4.609,2.736)--(-4.598,2.751)--cycle;
\filldraw[fill opacity=0.8,fill=gray!20,draw=none](-4.624,2.786)--(-4.644,2.779)--(-4.643,2.772)--(-4.606,2.744)--(-4.595,2.748)--cycle;
\draw(-4.624,2.786)--(-4.644,2.779);
\draw(-4.606,2.744)--(-4.595,2.748);
\filldraw[fill opacity=0.8,fill=gray!20,draw=none](-4.644,2.779)--(-4.65,2.777)--(-4.643,2.772)--cycle;
\draw(-4.644,2.779)--(-4.65,2.777);
\filldraw[fill opacity=0.8,fill=gray!20,draw=none](-4.619,2.739)--(-4.646,2.759)--(-4.65,2.772)--(-4.644,2.779)--cycle;
\draw(-4.65,2.772)--(-4.644,2.779);
\filldraw[fill opacity=0.8,fill=gray!20,draw=none](-4.596,2.761)--(-4.605,2.793)--(-4.619,2.812)--(-4.628,2.814)--(-4.608,2.764)--cycle;
\draw(-4.619,2.812)--(-4.628,2.814)--(-4.608,2.764)--(-4.596,2.761);
\filldraw[fill opacity=0.8,fill=gray!20,draw=none](-4.574,2.804)--(-4.605,2.793)--(-4.596,2.761)--(-4.587,2.75)--(-4.553,2.762)--cycle;
\draw(-4.574,2.804)--(-4.605,2.793);
\draw(-4.587,2.75)--(-4.553,2.762);
\filldraw[fill opacity=0.8,fill=gray!20,draw=none](-4.659,2.744)--(-4.665,2.754)--(-4.664,2.762)--(-4.657,2.774)--cycle;
\draw(-4.664,2.762)--(-4.657,2.774);
\filldraw[fill opacity=0.8,fill=gray!20,draw=none](-4.655,2.819)--(-4.684,2.803)--(-4.65,2.777)--(-4.644,2.779)--cycle;
\draw(-4.65,2.777)--(-4.644,2.779);
\filldraw[fill opacity=0.8,fill=gray!20,draw=none](-4.652,2.821)--(-4.655,2.819)--(-4.644,2.779)--(-4.624,2.786)--cycle;
\draw(-4.644,2.779)--(-4.624,2.786);
\filldraw[fill opacity=0.8,fill=gray!20,draw=none](-4.619,2.812)--(-4.622,2.823)--(-4.63,2.836)--(-4.628,2.814)--cycle;
\draw(-4.63,2.836)--(-4.628,2.814)--(-4.619,2.812);
\filldraw[fill opacity=0.8,fill=gray!20,draw=none](-4.629,2.839)--(-4.64,2.816)--(-4.634,2.831)--(-4.629,2.839)--cycle;
\filldraw[fill opacity=0.8,fill=gray!20,draw=none](-4.62,2.817)--(-4.649,2.818)--(-4.624,2.786)--(-4.613,2.79)--cycle;
\draw(-4.624,2.786)--(-4.613,2.79);
\filldraw[fill opacity=0.8,fill=gray!20,draw=none](-4.657,2.774)--(-4.651,2.791)--(-4.64,2.816)--cycle;
\filldraw[fill opacity=0.8,fill=gray!20,draw=none](-4.657,2.774)--(-4.664,2.762)--(-4.651,2.791)--cycle;
\draw(-4.657,2.774)--(-4.664,2.762);
\filldraw[fill opacity=0.8,fill=gray!20,draw=none](-4.64,2.816)--(-4.644,2.779)--(-4.65,2.772)--cycle;
\draw(-4.644,2.779)--(-4.65,2.772);
\filldraw[fill opacity=0.8,fill=gray!20,draw=none](-4.63,2.836)--(-4.622,2.823)--(-4.613,2.827)--(-4.629,2.839)--cycle;
\filldraw[fill opacity=0.8,fill=gray!20,draw=none](-4.613,2.827)--(-4.604,2.831)--(-4.629,2.839)--cycle;
\filldraw[fill opacity=0.8,fill=gray!20,draw=none](-4.63,2.836)--(-4.627,2.821)--(-4.622,2.823)--cycle;
\filldraw[fill opacity=0.8,fill=gray!20,draw=none](-4.634,2.829)--(-4.629,2.82)--(-4.627,2.821)--(-4.63,2.836)--cycle;
\filldraw[fill opacity=0.8,fill=gray!20,draw=none](-4.622,2.823)--(-4.626,2.839)--(-4.629,2.839)--(-4.63,2.836)--cycle;
\filldraw[fill opacity=0.8,fill=gray!20,draw=none](-4.555,2.814)--(-4.62,2.817)--(-4.613,2.79)--(-4.552,2.811)--cycle;
\draw(-4.613,2.79)--(-4.552,2.811);
\filldraw[fill opacity=0.8,fill=gray!20,draw=none](-4.605,2.793)--(-4.581,2.758)--(-4.576,2.757)--(-4.603,2.8)--(-4.61,2.807)--cycle;
\draw(-4.581,2.758)--(-4.576,2.757);
\filldraw[fill opacity=0.8,fill=gray!20,draw=none](-4.605,2.793)--(-4.61,2.807)--(-4.614,2.81)--(-4.619,2.812)--cycle;
\draw(-4.614,2.81)--(-4.619,2.812);
\filldraw[fill opacity=0.8,fill=gray!20,draw=none](-4.62,2.817)--(-4.555,2.814)--(-4.622,2.837)--(-4.625,2.835)--cycle;
\filldraw[fill opacity=0.8,fill=gray!20,draw=none](-4.555,2.814)--(-4.596,2.85)--(-4.602,2.848)--(-4.622,2.837)--cycle;
\draw(-4.596,2.85)--(-4.602,2.848);
\filldraw[fill opacity=0.8,fill=gray!20,draw=none](-4.556,2.814)--(-4.565,2.819)--(-4.57,2.821)--(-4.604,2.831)--cycle;
\draw(-4.556,2.814)--(-4.565,2.819);
\filldraw[fill opacity=0.8,fill=gray!20,draw=none](-4.534,2.872)--(-4.596,2.85)--(-4.555,2.814)--(-4.545,2.814)--(-4.501,2.83)--cycle;
\draw(-4.534,2.872)--(-4.596,2.85);
\draw(-4.545,2.814)--(-4.501,2.83);
\filldraw[fill opacity=0.8,fill=gray!20,draw=none](-4.815,2.727)--(-7.416,1.115)--(-7.366,1.087)--(-4.564,2.825)--cycle;
\draw(-4.815,2.727)--(-7.416,1.115);
\draw(-7.366,1.087)--(-4.564,2.825);
\filldraw[fill opacity=0.8,fill=gray!20,draw=none](-7.42,1.102)--(-7.375,1.092)--(-7.416,1.115)--cycle;
\filldraw[fill opacity=0.8,fill=gray!20,draw=none](-2.504,5.913)--(-2.485,5.8)--(-2.533,5.926)--(-2.535,5.938)--cycle;
\draw(-2.504,5.913)--(-2.485,5.8);
\draw(-2.533,5.926)--(-2.535,5.938);
\filldraw[fill opacity=0.8,fill=gray!20,draw=none](-2.513,5.802)--(-2.488,5.837)--(-2.472,5.819)--(-2.485,5.8)--cycle;
\draw(-2.513,5.802)--(-2.488,5.837)--(-2.472,5.819)--(-2.485,5.8);
\filldraw[fill opacity=0.8,fill=gray!20,draw=none](-2.383,5.589)--(-2.404,5.604)--(-2.392,5.604)--(-2.381,5.599)--(-2.383,5.589)--cycle;
\draw(-2.381,5.599)--(-2.383,5.589)--(-2.383,5.589)--(-2.404,5.604)--(-2.392,5.604);
\filldraw[fill opacity=0.8,fill=gray!20,draw=none](-2.444,5.918)--(-2.46,5.907)--(-2.461,5.911)--cycle;
\draw(-2.46,5.907)--(-2.461,5.911);
\filldraw[fill opacity=0.8,fill=gray!20,draw=none](-2.46,5.907)--(-2.5,5.896)--(-2.491,5.909)--(-2.444,5.918)--cycle;
\draw(-2.5,5.896)--(-2.491,5.909)--(-2.444,5.918);
\filldraw[fill opacity=0.8,fill=gray!20](-3.195,3.305)--(-3.192,3.338)--(-3.219,3.34)--(-3.249,3.309)--cycle;
\filldraw[fill opacity=0.8,fill=gray!20,draw=none](-3.197,2.958)--(-3.2,2.995)--(-3.21,2.996)--(-3.285,2.987)--(-3.274,2.964)--cycle;
\draw(-3.285,2.987)--(-3.274,2.964)--(-3.197,2.958)--(-3.2,2.995)--(-3.21,2.996);
\filldraw[fill opacity=0.8,fill=gray!20,draw=none](-7.585,1.092)--(-7.55,1.08)--(-7.55,1.095)--cycle;
\draw(-7.55,1.08)--(-7.55,1.095);
\filldraw[fill opacity=0.8,fill=gray!20,draw=none](-3.375,2.84)--(-3.34,2.567)--(-3.247,2.702)--(-3.264,2.835)--cycle;
\draw(-3.247,2.702)--(-3.264,2.835)--(-3.375,2.84)--(-3.34,2.567);
\filldraw[fill opacity=0.8,fill=gray!20,draw=none](-3.282,3.048)--(-3.277,3.088)--(-3.284,3.102)--(-3.309,3.104)--(-3.305,3.049)--cycle;
\draw(-3.284,3.102)--(-3.309,3.104)--(-3.305,3.049)--(-3.282,3.048);
\filldraw[fill opacity=0.8,fill=gray!20,draw=none](-3.966,2.902)--(-3.13,3.071)--(-3.129,3.118)--(-3.968,2.948)--cycle;
\draw(-3.966,2.902)--(-3.13,3.071)--(-3.129,3.118)--(-3.968,2.948);
\filldraw[fill opacity=0.8,fill=gray!20,draw=none](-7.592,1.143)--(-7.6,1.144)--(-7.6,1.137)--cycle;
\draw(-7.592,1.143)--(-7.6,1.144)--(-7.6,1.137);
\filldraw[fill opacity=0.8,fill=gray!20](-2.383,5.589)--(-2.422,5.6)--(-2.404,5.604)--(-2.383,5.589)--cycle;
\filldraw[fill opacity=0.8,fill=gray!20](-2.219,5.785)--(-2.225,5.831)--(-2.207,5.811)--(-2.2,5.765)--cycle;
\filldraw[fill opacity=0.8,fill=gray!20,draw=none](-7.64,1.809)--(-7.641,1.809)--(-7.641,1.796)--(-7.636,1.798)--cycle;
\draw(-7.641,1.796)--(-7.636,1.798);
\filldraw[fill opacity=0.8,fill=gray!20,draw=none](-7.536,1.076)--(-7.55,1.06)--(-7.528,1.071)--cycle;
\filldraw[fill opacity=0.8,fill=gray!20,draw=none](-7.531,1.051)--(-7.527,1.081)--(-7.537,1.078)--cycle;
\draw(-7.527,1.081)--(-7.537,1.078);
\filldraw[fill opacity=0.8,fill=gray!20](-3.139,3.308)--(-3.163,3.339)--(-3.192,3.338)--(-3.195,3.305)--cycle;
\filldraw[fill opacity=0.8,fill=gray!20,draw=none](-7.667,1.785)--(-7.641,1.796)--(-7.641,1.809)--(-7.688,1.825)--(-7.694,1.822)--cycle;
\draw(-7.667,1.785)--(-7.641,1.796);
\draw(-7.688,1.825)--(-7.694,1.822);
\filldraw[fill opacity=0.8,fill=gray!20,draw=none](-3.529,2.287)--(-3.547,2.307)--(-3.558,2.312)--(-3.572,2.303)--(-3.584,2.285)--(-3.562,2.276)--cycle;
\draw(-3.547,2.307)--(-3.558,2.312);
\draw(-3.584,2.285)--(-3.562,2.276)--(-3.529,2.287);
\filldraw[fill opacity=0.8,fill=gray!20,draw=none](-7.366,1.087)--(-7.37,1.085)--(-7.373,1.023)--(-7.316,1.059)--cycle;
\draw(-7.366,1.087)--(-7.37,1.085);
\draw(-7.373,1.023)--(-7.316,1.059);
\filldraw[fill opacity=0.8,fill=gray!20,draw=none](-7.373,1.023)--(-7.37,1.089)--(-7.372,1.091)--(-7.422,1.096)--(-7.438,1.091)--cycle;
\draw(-7.37,1.089)--(-7.372,1.091)--(-7.422,1.096)--(-7.438,1.091);
\filldraw[fill opacity=0.8,fill=gray!20,draw=none](-7.531,.849)--(-7.527,.855)--(-7.533,.853)--cycle;
\draw(-7.527,.855)--(-7.533,.853);
\filldraw[fill opacity=0.8,fill=gray!20](-2.341,5.599)--(-2.301,5.621)--(-2.292,5.611)--(-2.336,5.594)--cycle;
\filldraw[fill opacity=0.8,fill=gray!20](-3.059,3.276)--(-3.097,3.316)--(-3.139,3.308)--(-3.119,3.265)--cycle;
\filldraw[fill opacity=0.8,fill=gray!20,draw=none](-3.282,3.246)--(-3.274,3.267)--(-3.327,3.28)--(-3.354,3.239)--cycle;
\draw(-3.282,3.246)--(-3.274,3.267)--(-3.327,3.28)--(-3.354,3.239);
\filldraw[fill opacity=0.8,fill=gray!20](-3.011,3.063)--(-3.005,3.117)--(-3.09,3.101)--(-3.093,3.047)--cycle;
\filldraw[fill opacity=0.8,fill=gray!20](-3.119,2.962)--(-3.103,2.999)--(-3.2,2.995)--(-3.197,2.958)--cycle;
\filldraw[fill opacity=0.8,fill=gray!20,draw=none](-3.284,3.102)--(-3.287,3.138)--(-3.304,3.16)--(-3.305,3.16)--(-3.309,3.104)--cycle;
\draw(-3.304,3.16)--(-3.305,3.16)--(-3.309,3.104)--(-3.284,3.102);
\filldraw[fill opacity=0.8,fill=gray!20,draw=none](-3.304,3.16)--(-3.304,3.163)--(-3.305,3.16)--cycle;
\draw(-3.304,3.163)--(-3.305,3.16)--(-3.304,3.16);
\filldraw[fill opacity=0.8,fill=gray!20,draw=none](-4.294,2.978)--(-4.282,2.966)--(-4.289,2.956)--cycle;
\draw(-4.282,2.966)--(-4.289,2.956);
\filldraw[fill opacity=0.8,fill=gray!20,draw=none](-4.258,2.96)--(-4.29,2.94)--(-4.296,2.946)--(-4.282,2.966)--cycle;
\draw(-4.296,2.946)--(-4.282,2.966);
\filldraw[fill opacity=0.8,fill=gray!20,draw=none](-4.29,2.94)--(-4.287,2.942)--(-4.287,2.936)--cycle;
\filldraw[fill opacity=0.8,fill=gray!20,draw=none](-4.242,2.912)--(-4.243,2.927)--(-4.235,2.952)--(-4.226,2.943)--(-4.219,2.887)--cycle;
\draw(-4.235,2.952)--(-4.226,2.943)--(-4.219,2.887)--(-4.242,2.912)--(-4.243,2.927);
\filldraw[fill opacity=0.8,fill=gray!20,draw=none](-4.261,2.924)--(-4.259,2.941)--(-4.247,2.958)--(-4.238,2.948)--(-4.242,2.938)--(-4.244,2.936)--cycle;
\draw(-4.259,2.941)--(-4.247,2.958);
\draw(-4.242,2.938)--(-4.244,2.936);
\filldraw[fill opacity=0.8,fill=gray!20,draw=none](-4.238,2.948)--(-4.237,2.946)--(-4.242,2.938)--cycle;
\draw(-4.237,2.946)--(-4.242,2.938);
\filldraw[fill opacity=0.8,fill=gray!20,draw=none](-4.235,2.952)--(-4.244,2.936)--(-4.237,2.946)--cycle;
\draw(-4.244,2.936)--(-4.237,2.946);
\filldraw[fill opacity=0.8,fill=gray!20,draw=none](-4.243,2.937)--(-4.225,2.968)--(-4.23,2.964)--(-4.279,2.895)--cycle;
\draw(-4.23,2.964)--(-4.279,2.895);
\filldraw[fill opacity=0.8,fill=gray!20,draw=none](-4.231,2.948)--(-4.235,2.952)--(-4.234,2.958)--cycle;
\draw(-4.231,2.948)--(-4.235,2.952);
\filldraw[fill opacity=0.8,fill=gray!20,draw=none](-4.261,2.924)--(-4.269,2.918)--(-4.273,2.922)--(-4.259,2.941)--cycle;
\draw(-4.273,2.922)--(-4.259,2.941);
\filldraw[fill opacity=0.8,fill=gray!20,draw=none](-4.311,2.882)--(-4.279,2.895)--(-4.23,2.964)--(-4.242,2.969)--(-4.279,2.937)--(-4.315,2.886)--cycle;
\draw(-4.279,2.895)--(-4.23,2.964);
\draw(-4.279,2.937)--(-4.315,2.886);
\filldraw[fill opacity=0.8,fill=gray!20,draw=none](-4.231,2.948)--(-4.234,2.958)--(-4.234,2.96)--(-4.226,2.943)--cycle;
\draw(-4.234,2.96)--(-4.226,2.943)--(-4.231,2.948);
\filldraw[fill opacity=0.8,fill=gray!20,draw=none](-4.225,2.968)--(-4.219,2.979)--(-4.23,2.964)--cycle;
\draw(-4.219,2.979)--(-4.23,2.964);
\filldraw[fill opacity=0.8,fill=gray!20,draw=none](-4.218,2.989)--(-4.219,2.979)--(-4.202,3.002)--cycle;
\draw(-4.219,2.979)--(-4.202,3.002);
\filldraw[fill opacity=0.8,fill=gray!20,draw=none](-4.234,2.958)--(-4.236,2.966)--(-4.234,2.96)--cycle;
\draw(-4.236,2.966)--(-4.234,2.96);
\filldraw[fill opacity=0.8,fill=gray!20,draw=none](-4.242,2.969)--(-4.236,2.966)--(-4.245,2.988)--cycle;
\draw(-4.236,2.966)--(-4.245,2.988);
\filldraw[fill opacity=0.8,fill=gray!20,draw=none](-4.23,2.964)--(-4.219,2.979)--(-4.218,2.989)--(-4.242,2.969)--cycle;
\draw(-4.23,2.964)--(-4.219,2.979);
\filldraw[fill opacity=0.8,fill=gray!20,draw=none](-4.242,2.969)--(-4.237,2.945)--(-4.234,2.938)--(-4.226,2.943)--(-4.236,2.966)--cycle;
\draw(-4.234,2.938)--(-4.226,2.943)--(-4.236,2.966);
\filldraw[fill opacity=0.8,fill=gray!20,draw=none](-4.356,2.959)--(-4.356,2.956)--(-4.358,2.941)--cycle;
\draw(-4.356,2.959)--(-4.356,2.956)--(-4.358,2.941);
\filldraw[fill opacity=0.8,fill=gray!20,draw=none](-4.379,2.864)--(-4.359,2.899)--(-4.353,2.912)--(-4.438,2.89)--cycle;
\draw(-4.379,2.864)--(-4.359,2.899);
\filldraw[fill opacity=0.8,fill=gray!20,draw=none](-4.364,2.906)--(-4.379,2.864)--(-4.438,2.89)--(-4.36,2.921)--cycle;
\draw(-4.364,2.906)--(-4.379,2.864);
\filldraw[fill opacity=0.8,fill=gray!20,draw=none](-4.357,2.944)--(-4.356,2.956)--(-4.378,2.967)--(-4.375,2.938)--cycle;
\draw(-4.357,2.944)--(-4.356,2.956)--(-4.378,2.967);
\filldraw[fill opacity=0.8,fill=gray!20,draw=none](-4.342,2.919)--(-4.356,2.939)--(-4.349,2.932)--cycle;
\draw(-4.356,2.939)--(-4.349,2.932)--(-4.342,2.919);
\filldraw[fill opacity=0.8,fill=gray!20,draw=none](-4.353,2.912)--(-4.352,2.912)--(-4.35,2.919)--cycle;
\draw(-4.352,2.912)--(-4.35,2.919);
\filldraw[fill opacity=0.8,fill=gray!20,draw=none](-4.359,2.899)--(-4.354,2.907)--(-4.352,2.912)--(-4.353,2.912)--cycle;
\draw(-4.359,2.899)--(-4.354,2.907)--(-4.352,2.912);
\filldraw[fill opacity=0.8,fill=gray!20,draw=none](-4.353,2.912)--(-4.36,2.921)--(-4.364,2.905)--(-4.359,2.899)--cycle;
\draw(-4.353,2.912)--(-4.36,2.921);
\filldraw[fill opacity=0.8,fill=gray!20,draw=none](-4.352,2.905)--(-4.342,2.919)--(-4.349,2.932)--(-4.359,2.919)--cycle;
\draw(-4.342,2.919)--(-4.349,2.932)--(-4.359,2.919);
\filldraw[fill opacity=0.8,fill=gray!20,draw=none](-4.36,2.919)--(-4.357,2.944)--(-4.371,2.939)--(-4.372,2.917)--(-4.37,2.916)--cycle;
\draw(-4.36,2.919)--(-4.357,2.944);
\draw(-4.372,2.917)--(-4.37,2.916);
\filldraw[fill opacity=0.8,fill=gray!20,draw=none](-4.352,2.934)--(-4.359,2.943)--(-4.356,2.939)--cycle;
\draw(-4.359,2.943)--(-4.356,2.939);
\filldraw[fill opacity=0.8,fill=gray!20,draw=none](-4.365,2.927)--(-4.358,2.921)--(-4.349,2.932)--cycle;
\draw(-4.358,2.921)--(-4.349,2.932);
\filldraw[fill opacity=0.8,fill=gray!20,draw=none](-4.357,2.941)--(-4.358,2.929)--(-4.349,2.932)--cycle;
\draw(-4.349,2.932)--(-4.357,2.941);
\filldraw[fill opacity=0.8,fill=gray!20,draw=none](-4.357,2.941)--(-4.37,2.939)--(-4.365,2.927)--(-4.358,2.929)--cycle;
\filldraw[fill opacity=0.8,fill=gray!20,draw=none](-4.359,2.919)--(-4.371,2.932)--(-4.364,2.926)--cycle;
\draw(-4.371,2.932)--(-4.364,2.926)--(-4.359,2.919);
\filldraw[fill opacity=0.8,fill=gray!20,draw=none](-4.395,2.95)--(-4.378,2.917)--(-4.371,2.916)--(-4.364,2.926)--cycle;
\draw(-4.371,2.916)--(-4.364,2.926)--(-4.395,2.95);
\filldraw[fill opacity=0.8,fill=gray!20,draw=none](-4.371,2.939)--(-4.375,2.938)--(-4.374,2.918)--(-4.372,2.917)--cycle;
\draw(-4.374,2.918)--(-4.372,2.917);
\filldraw[fill opacity=0.8,fill=gray!20,draw=none](-4.37,2.939)--(-4.375,2.938)--(-4.373,2.934)--(-4.365,2.927)--cycle;
\filldraw[fill opacity=0.8,fill=gray!20,draw=none](-4.36,2.921)--(-4.364,2.926)--(-4.371,2.916)--(-4.364,2.905)--cycle;
\draw(-4.36,2.921)--(-4.364,2.926)--(-4.371,2.916);
\filldraw[fill opacity=0.8,fill=gray!20,draw=none](-4.364,2.906)--(-4.36,2.921)--(-4.361,2.912)--cycle;
\draw(-4.36,2.921)--(-4.361,2.912)--(-4.364,2.906);
\filldraw[fill opacity=0.8,fill=gray!20,draw=none](-4.36,2.919)--(-4.37,2.916)--(-4.361,2.912)--cycle;
\draw(-4.37,2.916)--(-4.361,2.912)--(-4.36,2.919);
\filldraw[fill opacity=0.8,fill=gray!20,draw=none](-4.373,2.934)--(-4.365,2.92)--(-4.361,2.918)--(-4.358,2.921)--cycle;
\draw(-4.361,2.918)--(-4.358,2.921);
\filldraw[fill opacity=0.8,fill=gray!20](-4.387,2.915)--(-3.138,3.168)--(-3.154,3.215)--(-4.403,2.962)--cycle;
\filldraw[fill opacity=0.8,fill=gray!20,draw=none](-3.334,3.226)--(-3.327,3.224)--(-3.331,3.241)--(-3.354,3.239)--cycle;
\draw(-3.334,3.226)--(-3.327,3.224);
\filldraw[fill opacity=0.8,fill=gray!20,draw=none](-2.383,5.589)--(-2.381,5.599)--(-2.372,5.595)--(-2.383,5.589)--cycle;
\draw(-2.372,5.595)--(-2.383,5.589)--(-2.383,5.589)--(-2.381,5.599);
\filldraw[fill opacity=0.8,fill=gray!20](-2.225,5.831)--(-2.241,5.872)--(-2.225,5.854)--(-2.207,5.811)--cycle;
\filldraw[fill opacity=0.8,fill=gray!20,draw=none](-7.468,1.108)--(-7.494,1.089)--(-7.438,1.104)--cycle;
\filldraw[fill opacity=0.8,fill=gray!20,draw=none](-7.454,1.138)--(-7.494,1.136)--(-7.494,1.111)--(-7.438,1.104)--(-7.438,1.129)--cycle;
\draw(-7.454,1.138)--(-7.494,1.136)--(-7.494,1.111);
\draw(-7.438,1.104)--(-7.438,1.129);
\filldraw[fill opacity=0.8,fill=gray!20,draw=none](-7.562,1.025)--(-7.533,1.035)--(-7.531,1.051)--(-7.537,1.078)--(-7.588,1.061)--cycle;
\draw(-7.562,1.025)--(-7.533,1.035);
\draw(-7.537,1.078)--(-7.588,1.061);
\filldraw[fill opacity=0.8,fill=gray!20,draw=none](-4.442,3.102)--(-2.589,5.696)--(-2.562,5.723)--(-2.562,5.693)--(-4.426,3.083)--cycle;
\draw(-4.442,3.102)--(-2.589,5.696);
\draw(-2.562,5.693)--(-4.426,3.083);
\filldraw[fill opacity=0.8,fill=gray!20,draw=none](-4.614,2.741)--(-4.676,2.719)--(-4.61,2.707)--(-4.586,2.716)--cycle;
\draw(-4.614,2.741)--(-4.676,2.719);
\draw(-4.61,2.707)--(-4.586,2.716);
\filldraw[fill opacity=0.8,fill=gray!20,draw=none](-4.609,2.707)--(-4.633,2.708)--(-4.649,2.728)--cycle;
\filldraw[fill opacity=0.8,fill=gray!20,draw=none](-4.498,2.786)--(-4.527,2.794)--(-4.528,2.788)--cycle;
\filldraw[fill opacity=0.8,fill=gray!20,draw=none](-4.514,2.726)--(-4.543,2.714)--(-4.518,2.752)--cycle;
\draw(-4.543,2.714)--(-4.518,2.752);
\filldraw[fill opacity=0.8,fill=gray!20,draw=none](-4.496,2.706)--(-4.497,2.706)--(-4.497,2.707)--cycle;
\draw(-4.496,2.706)--(-4.497,2.706);
\filldraw[fill opacity=0.8,fill=gray!20,draw=none](-4.488,2.704)--(-4.491,2.702)--(-4.506,2.694)--(-4.497,2.706)--(-4.496,2.706)--cycle;
\draw(-4.491,2.702)--(-4.506,2.694);
\draw(-4.497,2.706)--(-4.496,2.706);
\filldraw[fill opacity=0.8,fill=gray!20,draw=none](-4.504,2.73)--(-4.514,2.726)--(-4.508,2.73)--cycle;
\filldraw[fill opacity=0.8,fill=gray!20,draw=none](-4.504,2.73)--(-4.514,2.726)--(-4.517,2.75)--cycle;
\filldraw[fill opacity=0.8,fill=gray!20,draw=none](-4.526,2.77)--(-4.528,2.768)--(-4.529,2.775)--cycle;
\draw(-4.526,2.77)--(-4.528,2.768);
\filldraw[fill opacity=0.8,fill=gray!20,draw=none](-4.517,2.686)--(-4.545,2.669)--(-4.525,2.687)--cycle;
\draw(-4.517,2.686)--(-4.545,2.669);
\filldraw[fill opacity=0.8,fill=gray!20,draw=none](-4.519,2.689)--(-4.523,2.689)--(-4.539,2.696)--(-4.545,2.704)--cycle;
\draw(-4.539,2.696)--(-4.545,2.704);
\filldraw[fill opacity=0.8,fill=gray!20,draw=none](-4.501,2.73)--(-4.524,2.728)--(-4.528,2.768)--(-4.526,2.77)--cycle;
\draw(-4.528,2.768)--(-4.526,2.77);
\filldraw[fill opacity=0.8,fill=gray!20,draw=none](-4.694,2.594)--(-5.174,2.296)--(-5.028,2.37)--(-4.637,2.612)--cycle;
\draw(-4.694,2.594)--(-5.174,2.296);
\draw(-5.028,2.37)--(-4.637,2.612);
\filldraw[fill opacity=0.8,fill=gray!20,draw=none](-4.524,2.728)--(-4.566,2.722)--(-4.528,2.768)--cycle;
\draw(-4.566,2.722)--(-4.528,2.768);
\filldraw[fill opacity=0.8,fill=gray!20,draw=none](-4.514,2.726)--(-4.513,2.72)--(-4.543,2.714)--cycle;
\filldraw[fill opacity=0.8,fill=gray!20,draw=none](-4.514,2.726)--(-4.517,2.723)--(-4.544,2.714)--cycle;
\draw(-4.517,2.723)--(-4.544,2.714);
\filldraw[fill opacity=0.8,fill=gray!20,draw=none](-4.574,2.72)--(-4.688,2.679)--(-4.635,2.681)--(-4.57,2.704)--cycle;
\draw(-4.574,2.72)--(-4.688,2.679);
\draw(-4.635,2.681)--(-4.57,2.704);
\filldraw[fill opacity=0.8,fill=gray!20,draw=none](-4.633,2.708)--(-4.62,2.691)--(-4.636,2.667)--(-4.654,2.686)--(-4.661,2.709)--cycle;
\draw(-4.62,2.691)--(-4.636,2.667);
\filldraw[fill opacity=0.8,fill=gray!20,draw=none](-4.513,2.777)--(-4.587,2.75)--(-4.578,2.723)--(-4.566,2.722)--(-4.502,2.745)--cycle;
\draw(-4.513,2.777)--(-4.587,2.75);
\draw(-4.566,2.722)--(-4.502,2.745);
\filldraw[fill opacity=0.8,fill=gray!20,draw=none](-4.583,2.728)--(-4.566,2.722)--(-4.574,2.722)--cycle;
\filldraw[fill opacity=0.8,fill=gray!20,draw=none](-4.576,2.723)--(-4.574,2.723)--(-4.565,2.741)--(-4.576,2.757)--(-4.608,2.764)--(-4.579,2.727)--cycle;
\draw(-4.576,2.723)--(-4.574,2.723);
\draw(-4.576,2.757)--(-4.608,2.764)--(-4.579,2.727);
\filldraw[fill opacity=0.8,fill=gray!20,draw=none](-4.579,2.727)--(-4.587,2.75)--(-4.614,2.741)--(-4.609,2.736)--cycle;
\draw(-4.587,2.75)--(-4.614,2.741);
\filldraw[fill opacity=0.8,fill=gray!20,draw=none](-4.633,2.708)--(-4.661,2.709)--(-4.661,2.71)--(-4.649,2.728)--cycle;
\draw(-4.661,2.71)--(-4.649,2.728);
\filldraw[fill opacity=0.8,fill=gray!20,draw=none](-4.649,2.728)--(-4.661,2.71)--(-4.659,2.744)--cycle;
\draw(-4.649,2.728)--(-4.661,2.71);
\filldraw[fill opacity=0.8,fill=gray!20,draw=none](-4.609,2.736)--(-4.61,2.734)--(-4.619,2.739)--cycle;
\filldraw[fill opacity=0.8,fill=gray!20,draw=none](-4.498,2.786)--(-4.526,2.799)--(-4.527,2.794)--cycle;
\draw(-4.498,2.786)--(-4.526,2.799);
\filldraw[fill opacity=0.8,fill=gray!20,draw=none](-4.61,2.734)--(-4.623,2.716)--(-4.634,2.722)--(-4.619,2.739)--cycle;
\draw(-4.634,2.722)--(-4.619,2.739);
\filldraw[fill opacity=0.8,fill=gray!20,draw=none](-4.619,2.739)--(-4.634,2.722)--(-4.646,2.759)--cycle;
\draw(-4.619,2.739)--(-4.634,2.722);
\filldraw[fill opacity=0.8,fill=gray!20,draw=none](-4.618,2.788)--(-4.624,2.786)--(-4.595,2.748)--(-4.587,2.75)--cycle;
\draw(-4.618,2.788)--(-4.624,2.786);
\draw(-4.595,2.748)--(-4.587,2.75);
\filldraw[fill opacity=0.8,fill=gray!20,draw=none](-4.654,2.686)--(-4.668,2.7)--(-4.661,2.71)--cycle;
\draw(-4.668,2.7)--(-4.661,2.71);
\filldraw[fill opacity=0.8,fill=gray!20,draw=none](-4.461,2.799)--(-4.56,2.817)--(-4.498,2.786)--cycle;
\draw(-4.56,2.817)--(-4.498,2.786);
\filldraw[fill opacity=0.8,fill=gray!20,draw=none](-4.659,2.744)--(-4.661,2.71)--(-4.668,2.7)--(-4.665,2.754)--cycle;
\draw(-4.661,2.71)--(-4.668,2.7);
\filldraw[fill opacity=0.8,fill=gray!20,draw=none](-4.605,2.793)--(-4.613,2.79)--(-4.614,2.783)--(-4.596,2.761)--cycle;
\draw(-4.605,2.793)--(-4.613,2.79);
\filldraw[fill opacity=0.8,fill=gray!20,draw=none](-4.605,2.793)--(-4.596,2.761)--(-4.581,2.758)--cycle;
\draw(-4.596,2.761)--(-4.581,2.758);
\filldraw[fill opacity=0.8,fill=gray!20,draw=none](-4.504,2.772)--(-4.484,2.755)--(-4.484,2.755)--(-4.498,2.786)--cycle;
\draw(-4.484,2.755)--(-4.498,2.786);
\filldraw[fill opacity=0.8,fill=gray!20,draw=none](-4.464,2.809)--(-4.474,2.839)--(-4.574,2.804)--(-4.553,2.762)--(-4.487,2.786)--cycle;
\draw(-4.474,2.839)--(-4.574,2.804);
\draw(-4.553,2.762)--(-4.487,2.786);
\filldraw[fill opacity=0.8,fill=gray!20,draw=none](-4.613,2.79)--(-4.618,2.788)--(-4.614,2.783)--cycle;
\draw(-4.613,2.79)--(-4.618,2.788);
\filldraw[fill opacity=0.8,fill=gray!20,draw=none](-4.464,2.809)--(-4.487,2.786)--(-4.46,2.796)--cycle;
\draw(-4.487,2.786)--(-4.46,2.796);
\filldraw[fill opacity=0.8,fill=gray!20,draw=none](-4.461,2.799)--(-4.46,2.799)--(-4.48,2.81)--(-4.487,2.812)--(-4.565,2.819)--(-4.56,2.817)--cycle;
\draw(-4.46,2.799)--(-4.48,2.81);
\draw(-4.565,2.819)--(-4.56,2.817);
\filldraw[fill opacity=0.8,fill=gray!20,draw=none](-4.555,2.814)--(-4.552,2.811)--(-4.545,2.814)--cycle;
\draw(-4.552,2.811)--(-4.545,2.814);
\filldraw[fill opacity=0.8,fill=gray!20,draw=none](-4.443,2.831)--(-4.449,2.844)--(-4.474,2.839)--(-4.464,2.809)--cycle;
\draw(-4.443,2.831)--(-4.449,2.844);
\filldraw[fill opacity=0.8,fill=gray!20,draw=none](-4.457,2.864)--(-4.469,2.896)--(-4.534,2.872)--(-4.501,2.83)--(-4.469,2.841)--cycle;
\draw(-4.457,2.864)--(-4.469,2.896)--(-4.534,2.872);
\draw(-4.501,2.83)--(-4.469,2.841);
\filldraw[fill opacity=0.8,fill=gray!20,draw=none](-4.451,2.848)--(-4.449,2.844)--(-4.443,2.831)--cycle;
\draw(-4.449,2.844)--(-4.443,2.831);
\filldraw[fill opacity=0.8,fill=gray!20,draw=none](-4.463,2.887)--(-7.366,1.087)--(-7.316,1.059)--(-4.435,2.844)--cycle;
\draw(-7.316,1.059)--(-4.435,2.844)--(-4.463,2.887)--(-7.366,1.087);
\filldraw[fill opacity=0.8,fill=gray!20,draw=none](-7.368,1.089)--(-7.37,1.085)--(-7.366,1.087)--cycle;
\draw(-7.37,1.085)--(-7.366,1.087);
\filldraw[fill opacity=0.8,fill=gray!20](-2.383,5.589)--(-2.43,5.595)--(-2.422,5.6)--(-2.383,5.589)--cycle;
\filldraw[fill opacity=0.8,fill=gray!20,draw=none](-7.646,1.738)--(-7.636,1.743)--(-7.636,1.798)--(-7.667,1.785)--cycle;
\draw(-7.646,1.738)--(-7.636,1.743);
\draw(-7.636,1.798)--(-7.667,1.785);
\filldraw[fill opacity=0.8,fill=gray!20,draw=none](-7.55,1.085)--(-7.55,1.08)--(-7.538,1.077)--cycle;
\draw(-7.55,1.085)--(-7.55,1.08);
\filldraw[fill opacity=0.8,fill=gray!20,draw=none](-7.536,1.077)--(-7.536,1.076)--(-7.528,1.071)--(-7.523,1.074)--cycle;
\filldraw[fill opacity=0.8,fill=gray!20,draw=none](-7.538,1.077)--(-7.536,1.076)--(-7.536,1.077)--cycle;
\filldraw[fill opacity=0.8,fill=gray!20,draw=none](-7.55,1.095)--(-7.55,1.085)--(-7.538,1.077)--(-7.536,1.077)--(-7.517,1.1)--cycle;
\draw(-7.55,1.095)--(-7.55,1.085);
\filldraw[fill opacity=0.8,fill=gray!20,draw=none](-7.53,1.084)--(-7.536,1.077)--(-7.523,1.074)--(-7.494,1.089)--cycle;
\filldraw[fill opacity=0.8,fill=gray!20,draw=none](-2.365,5.593)--(-2.341,5.599)--(-2.336,5.594)--(-2.354,5.592)--cycle;
\draw(-2.365,5.593)--(-2.341,5.599)--(-2.336,5.594)--(-2.354,5.592);
\filldraw[fill opacity=0.8,fill=gray!20,draw=none](-2.365,5.593)--(-2.354,5.592)--(-2.363,5.591)--cycle;
\draw(-2.354,5.592)--(-2.363,5.591);
\filldraw[fill opacity=0.8,fill=gray!20,draw=none](-2.363,5.591)--(-2.354,5.592)--(-2.344,5.591)--(-2.347,5.589)--(-2.359,5.589)--cycle;
\draw(-2.363,5.591)--(-2.354,5.592);
\draw(-2.347,5.589)--(-2.359,5.589);
\filldraw[fill opacity=0.8,fill=gray!20,draw=none](-2.354,5.592)--(-2.336,5.594)--(-2.34,5.591)--cycle;
\draw(-2.354,5.592)--(-2.336,5.594)--(-2.34,5.591);
\filldraw[fill opacity=0.8,fill=gray!20,draw=none](-2.359,5.589)--(-2.347,5.589)--(-2.349,5.588)--(-2.353,5.587)--cycle;
\draw(-2.359,5.589)--(-2.347,5.589);
\draw(-2.349,5.588)--(-2.353,5.587);
\filldraw[fill opacity=0.8,fill=gray!20](-2.336,5.594)--(-2.292,5.611)--(-2.307,5.601)--(-2.343,5.589)--cycle;
\filldraw[fill opacity=0.8,fill=gray!20,draw=none](-2.344,5.591)--(-2.34,5.591)--(-2.343,5.589)--(-2.347,5.589)--cycle;
\draw(-2.34,5.591)--(-2.343,5.589)--(-2.347,5.589);
\filldraw[fill opacity=0.8,fill=gray!20,draw=none](-2.349,5.589)--(-2.353,5.587)--(-2.349,5.588)--cycle;
\draw(-2.353,5.587)--(-2.349,5.588);
\filldraw[fill opacity=0.8,fill=gray!20,draw=none](-4.164,3.075)--(-2.293,5.694)--(-2.277,5.69)--(-4.146,3.074)--cycle;
\draw(-4.164,3.075)--(-2.293,5.694)--(-2.277,5.69)--(-4.146,3.074);
\filldraw[fill opacity=0.8,fill=gray!20,draw=none](-2.321,5.926)--(-2.333,5.938)--(-2.301,5.931)--(-2.286,5.92)--cycle;
\draw(-2.321,5.926)--(-2.333,5.938)--(-2.301,5.931)--(-2.286,5.92);
\filldraw[fill opacity=0.8,fill=gray!20,draw=none](-7.454,1.138)--(-7.438,1.129)--(-7.438,1.139)--cycle;
\draw(-7.438,1.129)--(-7.438,1.139)--(-7.454,1.138);
\filldraw[fill opacity=0.8,fill=gray!20,draw=none](-7.55,1.122)--(-7.55,1.095)--(-7.517,1.1)--(-7.506,1.113)--cycle;
\draw(-7.55,1.122)--(-7.55,1.095);
\filldraw[fill opacity=0.8,fill=gray!20,draw=none](-7.53,1.084)--(-7.494,1.089)--(-7.494,1.128)--cycle;
\draw(-7.494,1.089)--(-7.494,1.128);
\filldraw[fill opacity=0.8,fill=gray!20,draw=none](-4.227,2.851)--(-4.237,2.906)--(-4.228,2.897)--cycle;
\draw(-4.237,2.906)--(-4.228,2.897);
\filldraw[fill opacity=0.8,fill=gray!20,draw=none](-4.263,2.91)--(-4.249,2.899)--(-4.26,2.909)--cycle;
\draw(-4.263,2.91)--(-4.249,2.899);
\filldraw[fill opacity=0.8,fill=gray!20,draw=none](-4.241,2.862)--(-4.227,2.851)--(-4.233,2.885)--(-4.249,2.899)--cycle;
\draw(-4.241,2.862)--(-4.227,2.851);
\draw(-4.233,2.885)--(-4.249,2.899);
\filldraw[fill opacity=0.8,fill=gray!20,draw=none](-4.279,2.895)--(-4.241,2.862)--(-4.249,2.899)--(-4.263,2.91)--cycle;
\draw(-4.279,2.895)--(-4.241,2.862);
\draw(-4.249,2.899)--(-4.263,2.91);
\filldraw[fill opacity=0.8,fill=gray!20,draw=none](-4.227,2.832)--(-4.228,2.897)--(-4.219,2.887)--(-4.226,2.832)--cycle;
\draw(-4.228,2.897)--(-4.219,2.887)--(-4.226,2.832)--(-4.227,2.832);
\filldraw[fill opacity=0.8,fill=gray!20,draw=none](-4.231,2.894)--(-4.227,2.882)--(-4.219,2.887)--(-4.226,2.943)--(-4.234,2.938)--cycle;
\draw(-4.227,2.882)--(-4.219,2.887)--(-4.226,2.943)--(-4.234,2.938);
\filldraw[fill opacity=0.8,fill=gray!20,draw=none](-4.268,2.807)--(-4.292,2.836)--(-4.314,2.831)--cycle;
\draw(-4.292,2.836)--(-4.314,2.831);
\filldraw[fill opacity=0.8,fill=gray!20,draw=none](-4.334,2.886)--(-4.334,2.888)--(-4.339,2.884)--cycle;
\draw(-4.334,2.886)--(-4.334,2.888);
\filldraw[fill opacity=0.8,fill=gray!20,draw=none](-4.337,2.875)--(-4.328,2.867)--(-4.32,2.879)--cycle;
\draw(-4.328,2.867)--(-4.32,2.879);
\filldraw[fill opacity=0.8,fill=gray!20,draw=none](-4.338,2.865)--(-4.336,2.876)--(-4.338,2.873)--cycle;
\filldraw[fill opacity=0.8,fill=gray!20,draw=none](-4.337,2.875)--(-4.338,2.874)--(-4.33,2.865)--(-4.328,2.867)--cycle;
\draw(-4.33,2.865)--(-4.328,2.867);
\filldraw[fill opacity=0.8,fill=gray!20,draw=none](-4.338,2.865)--(-4.33,2.865)--(-4.338,2.874)--cycle;
\filldraw[fill opacity=0.8,fill=gray!20,draw=none](-4.337,2.875)--(-4.326,2.852)--(-4.273,2.839)--(-4.32,2.879)--cycle;
\draw(-4.273,2.839)--(-4.32,2.879);
\filldraw[fill opacity=0.8,fill=gray!20,draw=none](-4.311,2.882)--(-4.315,2.886)--(-4.32,2.879)--cycle;
\draw(-4.315,2.886)--(-4.32,2.879);
\filldraw[fill opacity=0.8,fill=gray!20,draw=none](-4.325,2.903)--(-4.315,2.886)--(-4.279,2.937)--cycle;
\draw(-4.315,2.886)--(-4.279,2.937);
\filldraw[fill opacity=0.8,fill=gray!20,draw=none](-4.344,2.873)--(-4.343,2.864)--(-4.339,2.884)--(-4.342,2.879)--cycle;
\draw(-4.339,2.884)--(-4.342,2.879);
\filldraw[fill opacity=0.8,fill=gray!20,draw=none](-4.334,2.888)--(-4.341,2.883)--(-4.338,2.876)--(-4.331,2.883)--cycle;
\draw(-4.338,2.876)--(-4.331,2.883);
\filldraw[fill opacity=0.8,fill=gray!20,draw=none](-4.339,2.895)--(-4.335,2.89)--(-4.334,2.886)--cycle;
\draw(-4.339,2.895)--(-4.335,2.89)--(-4.334,2.886);
\filldraw[fill opacity=0.8,fill=gray!20,draw=none](-4.338,2.885)--(-4.334,2.888)--(-4.34,2.896)--cycle;
\filldraw[fill opacity=0.8,fill=gray!20,draw=none](-4.334,2.888)--(-4.335,2.89)--(-4.339,2.884)--cycle;
\draw(-4.334,2.888)--(-4.335,2.89)--(-4.339,2.884);
\filldraw[fill opacity=0.8,fill=gray!20,draw=none](-4.339,2.895)--(-4.339,2.884)--(-4.335,2.89)--cycle;
\draw(-4.339,2.884)--(-4.335,2.89)--(-4.339,2.895);
\filldraw[fill opacity=0.8,fill=gray!20,draw=none](-4.338,2.873)--(-4.336,2.876)--(-4.335,2.879)--(-4.338,2.875)--cycle;
\draw(-4.335,2.879)--(-4.338,2.875);
\filldraw[fill opacity=0.8,fill=gray!20,draw=none](-4.337,2.875)--(-4.32,2.879)--(-4.334,2.89)--cycle;
\draw(-4.32,2.879)--(-4.334,2.89);
\filldraw[fill opacity=0.8,fill=gray!20,draw=none](-4.341,2.883)--(-4.337,2.875)--(-4.334,2.89)--(-4.339,2.894)--cycle;
\draw(-4.334,2.89)--(-4.339,2.894);
\filldraw[fill opacity=0.8,fill=gray!20,draw=none](-4.338,2.894)--(-4.338,2.885)--(-4.32,2.879)--(-4.315,2.886)--(-4.325,2.903)--cycle;
\draw(-4.32,2.879)--(-4.315,2.886);
\filldraw[fill opacity=0.8,fill=gray!20,draw=none](-4.341,2.883)--(-4.344,2.873)--(-4.337,2.875)--cycle;
\filldraw[fill opacity=0.8,fill=gray!20,draw=none](-4.379,2.864)--(-4.344,2.873)--(-4.339,2.894)--(-4.354,2.907)--cycle;
\draw(-4.339,2.894)--(-4.354,2.907)--(-4.379,2.864);
\filldraw[fill opacity=0.8,fill=gray!20,draw=none](-4.355,2.895)--(-4.365,2.901)--(-4.357,2.881)--(-4.344,2.876)--(-4.342,2.879)--cycle;
\draw(-4.344,2.876)--(-4.342,2.879);
\filldraw[fill opacity=0.8,fill=gray!20,draw=none](-4.352,2.905)--(-4.355,2.901)--(-4.35,2.885)--(-4.338,2.876)--cycle;
\filldraw[fill opacity=0.8,fill=gray!20,draw=none](-4.346,2.888)--(-4.345,2.882)--(-4.337,2.875)--(-4.32,2.879)--cycle;
\filldraw[fill opacity=0.8,fill=gray!20,draw=none](-4.344,2.873)--(-4.343,2.865)--(-4.338,2.865)--(-4.338,2.874)--cycle;
\filldraw[fill opacity=0.8,fill=gray!20,draw=none](-4.343,2.856)--(-4.342,2.845)--(-4.338,2.865)--(-4.338,2.873)--cycle;
\filldraw[fill opacity=0.8,fill=gray!20,draw=none](-4.337,2.875)--(-4.343,2.856)--(-4.326,2.852)--cycle;
\filldraw[fill opacity=0.8,fill=gray!20,draw=none](-4.337,2.875)--(-4.345,2.882)--(-4.338,2.874)--cycle;
\filldraw[fill opacity=0.8,fill=gray!20,draw=none](-4.343,2.865)--(-4.344,2.876)--(-4.357,2.881)--(-4.379,2.864)--cycle;
\filldraw[fill opacity=0.8,fill=gray!20,draw=none](-4.344,2.873)--(-4.348,2.857)--(-4.346,2.847)--(-4.343,2.864)--cycle;
\filldraw[fill opacity=0.8,fill=gray!20,draw=none](-4.337,2.875)--(-4.344,2.873)--(-4.348,2.857)--(-4.343,2.856)--cycle;
\filldraw[fill opacity=0.8,fill=gray!20,draw=none](-4.343,2.868)--(-4.338,2.873)--(-4.338,2.875)--(-4.343,2.869)--cycle;
\draw(-4.338,2.875)--(-4.343,2.869);
\filldraw[fill opacity=0.8,fill=gray!20,draw=none](-4.302,2.855)--(-4.292,2.836)--(-4.273,2.839)--cycle;
\draw(-4.292,2.836)--(-4.273,2.839);
\filldraw[fill opacity=0.8,fill=gray!20,draw=none](-4.344,2.873)--(-4.379,2.864)--(-4.348,2.857)--cycle;
\filldraw[fill opacity=0.8,fill=gray!20,draw=none](-4.348,2.857)--(-4.342,2.879)--(-4.349,2.868)--cycle;
\draw(-4.342,2.879)--(-4.349,2.868);
\filldraw[fill opacity=0.8,fill=gray!20,draw=none](-4.344,2.873)--(-4.338,2.874)--(-4.345,2.882)--cycle;
\filldraw[fill opacity=0.8,fill=gray!20,draw=none](-4.357,2.881)--(-4.352,2.87)--(-4.349,2.868)--(-4.344,2.876)--cycle;
\draw(-4.349,2.868)--(-4.344,2.876);
\filldraw[fill opacity=0.8,fill=gray!20,draw=none](-4.344,2.876)--(-4.345,2.882)--(-4.35,2.885)--(-4.357,2.881)--cycle;
\filldraw[fill opacity=0.8,fill=gray!20,draw=none](-4.344,2.881)--(-4.348,2.878)--(-4.346,2.871)--(-4.343,2.869)--(-4.338,2.876)--cycle;
\draw(-4.343,2.869)--(-4.338,2.876);
\filldraw[fill opacity=0.8,fill=gray!20,draw=none](-4.338,2.894)--(-4.346,2.888)--(-4.338,2.885)--cycle;
\filldraw[fill opacity=0.8,fill=gray!20,draw=none](-4.344,2.881)--(-4.35,2.885)--(-4.348,2.878)--cycle;
\filldraw[fill opacity=0.8,fill=gray!20,draw=none](-4.345,2.882)--(-4.349,2.886)--(-4.35,2.885)--cycle;
\filldraw[fill opacity=0.8,fill=gray!20,draw=none](-4.346,2.888)--(-4.349,2.886)--(-4.345,2.882)--cycle;
\filldraw[fill opacity=0.8,fill=gray!20,draw=none](-4.355,2.901)--(-4.352,2.905)--(-4.359,2.919)--(-4.361,2.918)--cycle;
\draw(-4.359,2.919)--(-4.361,2.918);
\filldraw[fill opacity=0.8,fill=gray!20,draw=none](-4.361,2.912)--(-4.372,2.917)--(-4.374,2.916)--(-4.379,2.864)--cycle;
\draw(-4.379,2.864)--(-4.361,2.912)--(-4.372,2.917);
\filldraw[fill opacity=0.8,fill=gray!20,draw=none](-4.37,2.912)--(-4.365,2.901)--(-4.355,2.895)--cycle;
\filldraw[fill opacity=0.8,fill=gray!20,draw=none](-4.371,2.916)--(-4.37,2.912)--(-4.364,2.905)--cycle;
\filldraw[fill opacity=0.8,fill=gray!20,draw=none](-4.372,2.917)--(-4.374,2.918)--(-4.374,2.916)--cycle;
\draw(-4.372,2.917)--(-4.374,2.918);
\filldraw[fill opacity=0.8,fill=gray!20](-4.379,2.864)--(-3.129,3.118)--(-3.138,3.168)--(-4.387,2.915)--cycle;
\filldraw[fill opacity=0.8,fill=gray!20,draw=none](-3.572,2.303)--(-3.594,2.29)--(-3.584,2.285)--cycle;
\draw(-3.594,2.29)--(-3.584,2.285);
\filldraw[fill opacity=0.8,fill=gray!20,draw=none](-3.21,2.996)--(-3.293,3.002)--(-3.285,2.987)--cycle;
\draw(-3.21,2.996)--(-3.293,3.002)--(-3.285,2.987);
\filldraw[fill opacity=0.8,fill=gray!20,draw=none](-7.468,1.108)--(-7.494,1.111)--(-7.494,1.089)--cycle;
\draw(-7.494,1.111)--(-7.494,1.089);
\filldraw[fill opacity=0.8,fill=gray!20](-3.029,3.228)--(-3.059,3.276)--(-3.119,3.265)--(-3.103,3.214)--cycle;
\filldraw[fill opacity=0.8,fill=gray!20,draw=none](-3.529,2.287)--(-3.512,2.292)--(-3.547,2.307)--cycle;
\draw(-3.529,2.287)--(-3.512,2.292)--(-3.547,2.307);
\filldraw[fill opacity=0.8,fill=gray!20](-3.005,3.117)--(-3.011,3.174)--(-3.093,3.158)--(-3.09,3.101)--cycle;
\filldraw[fill opacity=0.8,fill=gray!20,draw=none](-2.383,5.589)--(-2.372,5.595)--(-2.365,5.593)--(-2.383,5.589)--cycle;
\draw(-2.365,5.593)--(-2.383,5.589)--(-2.383,5.589)--(-2.372,5.595);
\filldraw[fill opacity=0.8,fill=gray!20](-2.424,5.94)--(-2.404,5.949)--(-2.38,5.95)--(-2.377,5.942)--cycle;
\filldraw[fill opacity=0.8,fill=gray!20](-2.377,5.942)--(-2.38,5.95)--(-2.357,5.948)--(-2.333,5.938)--cycle;
\filldraw[fill opacity=0.8,fill=gray!20,draw=none](-7.548,1.137)--(-7.55,1.136)--(-7.55,1.122)--(-7.506,1.113)--(-7.494,1.128)--(-7.494,1.136)--cycle;
\draw(-7.55,1.136)--(-7.55,1.122);
\draw(-7.494,1.128)--(-7.494,1.136)--(-7.548,1.137);
\filldraw[fill opacity=0.8,fill=gray!20](-2.241,5.872)--(-2.267,5.906)--(-2.254,5.892)--(-2.225,5.854)--cycle;
\filldraw[fill opacity=0.8,fill=gray!20,draw=none](-2.267,5.904)--(-2.264,5.903)--(-2.267,5.906)--(-2.286,5.92)--cycle;
\draw(-2.264,5.903)--(-2.267,5.906)--(-2.286,5.92);
\filldraw[fill opacity=0.8,fill=gray!20,draw=none](-2.267,5.904)--(-2.286,5.92)--(-2.272,5.906)--cycle;
\filldraw[fill opacity=0.8,fill=gray!20,draw=none](-2.281,5.961)--(-2.253,5.789)--(-2.267,5.802)--(-2.288,5.933)--cycle;
\draw(-2.281,5.961)--(-2.253,5.789)--(-2.267,5.802)--(-2.288,5.933);
\filldraw[fill opacity=0.8,fill=gray!20,draw=none](-2.621,7.831)--(-2.704,7.837)--(-2.704,7.894)--(-2.634,7.889)--cycle;
\draw(-2.621,7.831)--(-2.704,7.837)--(-2.704,7.894)--(-2.634,7.889);
\filldraw[fill opacity=0.8,fill=gray!20,draw=none](-2.665,7.778)--(-2.612,7.774)--(-2.627,7.736)--cycle;
\draw(-2.665,7.778)--(-2.612,7.774)--(-2.627,7.736);
\filldraw[fill opacity=0.8,fill=gray!20,draw=none](-2.621,7.831)--(-2.634,7.889)--(-2.596,7.886)--(-2.6,7.83)--cycle;
\draw(-2.634,7.889)--(-2.596,7.886)--(-2.6,7.83)--(-2.621,7.831);
\filldraw[fill opacity=0.8,fill=gray!20,draw=none](-2.56,7.82)--(-2.6,7.83)--(-2.596,7.886)--(-2.565,7.879)--cycle;
\draw(-2.56,7.82)--(-2.6,7.83)--(-2.596,7.886)--(-2.565,7.879);
\filldraw[fill opacity=0.8,fill=gray!20,draw=none](-2.557,7.76)--(-2.602,7.771)--(-2.609,7.789)--(-2.6,7.83)--(-2.56,7.82)--cycle;
\draw(-2.557,7.76)--(-2.602,7.771);
\draw(-2.609,7.789)--(-2.6,7.83)--(-2.56,7.82);
\filldraw[fill opacity=0.8,fill=gray!20,draw=none](-2.602,7.771)--(-2.612,7.774)--(-2.609,7.789)--cycle;
\draw(-2.602,7.771)--(-2.612,7.774)--(-2.609,7.789);
\filldraw[fill opacity=0.8,fill=gray!20,draw=none](-2.606,7.728)--(-2.627,7.736)--(-2.612,7.774)--(-2.602,7.771)--(-2.584,7.735)--cycle;
\draw(-2.627,7.736)--(-2.612,7.774)--(-2.602,7.771);
\filldraw[fill opacity=0.8,fill=gray!20,draw=none](-2.602,7.771)--(-2.557,7.76)--(-2.557,7.743)--(-2.574,7.716)--cycle;
\draw(-2.602,7.771)--(-2.557,7.76);
\draw(-2.557,7.743)--(-2.574,7.716);
\filldraw[fill opacity=0.8,fill=gray!20,draw=none](-2.606,7.728)--(-2.584,7.735)--(-2.574,7.716)--cycle;
\filldraw[fill opacity=0.8,fill=gray!20,draw=none](-2.601,7.881)--(-2.295,5.976)--(-2.319,5.941)--(-2.632,7.888)--cycle;
\draw(-2.319,5.941)--(-2.632,7.888)--(-2.601,7.881)--(-2.295,5.976);
\filldraw[fill opacity=0.8,fill=gray!20,draw=none](-7.548,1.137)--(-7.55,1.138)--(-7.55,1.136)--cycle;
\draw(-7.548,1.137)--(-7.55,1.138)--(-7.55,1.136);
\filldraw[fill opacity=0.8,fill=gray!20,draw=none](-7.536,.856)--(-7.533,.853)--(-7.527,.855)--(-7.528,.863)--cycle;
\draw(-7.533,.853)--(-7.527,.855);
\filldraw[fill opacity=0.8,fill=gray!20,draw=none](-2.527,5.888)--(-2.525,5.874)--(-2.529,5.874)--cycle;
\draw(-2.527,5.888)--(-2.525,5.874);
\filldraw[fill opacity=0.8,fill=gray!20,draw=none](-2.538,5.862)--(-2.512,5.895)--(-2.491,5.909)--(-2.5,5.896)--cycle;
\draw(-2.538,5.862)--(-2.512,5.895)--(-2.491,5.909)--(-2.5,5.896);
\filldraw[fill opacity=0.8,fill=gray!20,draw=none](-2.383,5.589)--(-2.365,5.593)--(-2.363,5.591)--(-2.383,5.589)--cycle;
\draw(-2.363,5.591)--(-2.383,5.589)--(-2.383,5.589)--(-2.365,5.593);
\filldraw[fill opacity=0.8,fill=gray!20,draw=none](-3.327,3.224)--(-3.293,3.216)--(-3.282,3.246)--(-3.331,3.241)--cycle;
\draw(-3.327,3.224)--(-3.293,3.216)--(-3.282,3.246);
\filldraw[fill opacity=0.8,fill=gray!20](-3.011,3.174)--(-3.029,3.228)--(-3.103,3.214)--(-3.093,3.158)--cycle;
\filldraw[fill opacity=0.8,fill=gray!20,draw=none](-2.543,5.826)--(-2.559,5.816)--(-2.541,5.859)--(-2.539,5.86)--cycle;
\draw(-2.543,5.826)--(-2.559,5.816)--(-2.541,5.859)--(-2.539,5.86);
\filldraw[fill opacity=0.8,fill=gray!20,draw=none](-3.296,3.003)--(-3.293,3.002)--(-3.294,3.005)--cycle;
\draw(-3.296,3.003)--(-3.293,3.002)--(-3.294,3.005);
\filldraw[fill opacity=0.8,fill=gray!20,draw=none](-3.685,2.153)--(-3.69,2.149)--(-3.687,2.148)--cycle;
\filldraw[fill opacity=0.8,fill=gray!20,draw=none](-3.968,2.349)--(-3.763,2.178)--(-3.69,2.149)--(-3.685,2.153)--(-3.684,2.157)--(-3.928,2.36)--cycle;
\draw(-3.968,2.349)--(-3.763,2.178);
\draw(-3.684,2.157)--(-3.928,2.36);
\filldraw[fill opacity=0.8,fill=gray!20](-2.459,5.933)--(-2.422,5.945)--(-2.404,5.949)--(-2.424,5.94)--cycle;
\filldraw[fill opacity=0.8,fill=gray!20,draw=none](-2.816,7.832)--(-2.823,7.888)--(-2.815,7.889)--(-2.812,7.832)--cycle;
\draw(-2.823,7.888)--(-2.815,7.889)--(-2.812,7.832)--(-2.816,7.832);
\filldraw[fill opacity=0.8,fill=gray!20,draw=none](-2.811,7.775)--(-2.816,7.832)--(-2.812,7.832)--(-2.802,7.776)--cycle;
\draw(-2.816,7.832)--(-2.812,7.832)--(-2.802,7.776)--(-2.811,7.775);
\filldraw[fill opacity=0.8,fill=gray!20,draw=none](-2.811,7.775)--(-2.864,7.764)--(-2.878,7.82)--(-2.816,7.832)--cycle;
\draw(-2.811,7.775)--(-2.864,7.764);
\draw(-2.878,7.82)--(-2.816,7.832);
\filldraw[fill opacity=0.8,fill=gray!20,draw=none](-2.816,7.832)--(-2.894,7.816)--(-2.9,7.873)--(-2.823,7.888)--cycle;
\draw(-2.816,7.832)--(-2.894,7.816)--(-2.9,7.873)--(-2.823,7.888);
\filldraw[fill opacity=0.8,fill=gray!20,draw=none](-2.811,7.721)--(-2.811,7.775)--(-2.802,7.776)--(-2.786,7.725)--cycle;
\draw(-2.811,7.775)--(-2.802,7.776)--(-2.786,7.725)--(-2.811,7.721);
\filldraw[fill opacity=0.8,fill=gray!20,draw=none](-2.811,7.721)--(-2.846,7.714)--(-2.85,7.72)--(-2.864,7.764)--(-2.811,7.775)--cycle;
\draw(-2.811,7.721)--(-2.846,7.714)--(-2.85,7.72);
\draw(-2.864,7.764)--(-2.811,7.775);
\filldraw[fill opacity=0.8,fill=gray!20,draw=none](-2.79,7.696)--(-2.812,7.693)--(-2.811,7.721)--(-2.786,7.725)--(-2.778,7.707)--cycle;
\draw(-2.811,7.721)--(-2.786,7.725)--(-2.778,7.707);
\filldraw[fill opacity=0.8,fill=gray!20,draw=none](-2.813,7.679)--(-2.846,7.714)--(-2.811,7.721)--cycle;
\draw(-2.813,7.679)--(-2.846,7.714)--(-2.811,7.721);
\filldraw[fill opacity=0.8,fill=gray!20,draw=none](-2.79,7.696)--(-2.81,7.676)--(-2.813,7.679)--(-2.812,7.693)--cycle;
\draw(-2.81,7.676)--(-2.813,7.679);
\filldraw[fill opacity=0.8,fill=gray!20,draw=none](-2.814,7.676)--(-2.829,7.675)--(-2.848,7.713)--(-2.846,7.714)--(-2.813,7.679)--cycle;
\draw(-2.848,7.713)--(-2.846,7.714)--(-2.813,7.679);
\filldraw[fill opacity=0.8,fill=gray!20,draw=none](-2.814,7.676)--(-2.813,7.679)--(-2.81,7.676)--cycle;
\draw(-2.813,7.679)--(-2.81,7.676);
\filldraw[fill opacity=0.8,fill=gray!20,draw=none](-2.816,7.857)--(-2.504,5.913)--(-2.535,5.938)--(-2.841,7.84)--cycle;
\draw(-2.535,5.938)--(-2.841,7.84)--(-2.816,7.857)--(-2.504,5.913);
\filldraw[fill opacity=0.8,fill=gray!20,draw=none](-2.526,5.787)--(-2.521,5.799)--(-2.513,5.802)--cycle;
\filldraw[fill opacity=0.8,fill=gray!20,draw=none](-2.526,5.787)--(-2.513,5.802)--(-2.528,5.781)--cycle;
\draw(-2.513,5.802)--(-2.528,5.781);
\filldraw[fill opacity=0.8,fill=gray!20,draw=none](-2.531,5.858)--(-2.529,5.874)--(-2.538,5.863)--cycle;
\draw(-2.529,5.874)--(-2.538,5.863);
\filldraw[fill opacity=0.8,fill=gray!20,draw=none](-2.525,5.874)--(-2.507,5.761)--(-2.513,5.745)--(-2.533,5.874)--cycle;
\draw(-2.525,5.874)--(-2.507,5.761)--(-2.513,5.745)--(-2.533,5.874);
\filldraw[fill opacity=0.8,fill=gray!20,draw=none](-7.51,.891)--(-7.491,.897)--(-7.473,.948)--(-7.504,.938)--cycle;
\draw(-7.51,.891)--(-7.491,.897);
\draw(-7.473,.948)--(-7.504,.938);
\filldraw[fill opacity=0.8,fill=gray!20,draw=none](-7.528,.863)--(-7.491,.897)--(-7.531,.884)--cycle;
\draw(-7.491,.897)--(-7.531,.884);
\filldraw[fill opacity=0.8,fill=gray!20,draw=none](-3.295,3.004)--(-3.294,3.005)--(-3.295,3.008)--cycle;
\draw(-3.294,3.005)--(-3.295,3.008);
\filldraw[fill opacity=0.8,fill=gray!20](-3.197,3.261)--(-3.195,3.305)--(-3.249,3.309)--(-3.274,3.267)--cycle;
\filldraw[fill opacity=0.8,fill=gray!20,draw=none](-3.21,2.996)--(-3.2,2.997)--(-3.201,3.042)--(-3.227,3.044)--(-3.294,3.005)--(-3.293,3.002)--cycle;
\draw(-3.2,2.997)--(-3.201,3.042)--(-3.227,3.044);
\draw(-3.294,3.005)--(-3.293,3.002)--(-3.21,2.996);
\filldraw[fill opacity=0.8,fill=gray!20,draw=none](-7.628,.822)--(-7.533,.853)--(-7.559,.875)--(-7.752,.813)--cycle;
\draw(-7.628,.822)--(-7.533,.853);
\draw(-7.559,.875)--(-7.752,.813);
\filldraw[fill opacity=0.8,fill=gray!20,draw=none](-7.53,.929)--(-7.473,.948)--(-7.491,.995)--(-7.542,.979)--cycle;
\draw(-7.53,.929)--(-7.473,.948);
\draw(-7.491,.995)--(-7.542,.979);
\filldraw[fill opacity=0.8,fill=gray!20](-2.425,5.59)--(-2.465,5.604)--(-2.474,5.614)--(-2.43,5.595)--cycle;
\filldraw[fill opacity=0.8,fill=gray!20](-2.383,5.589)--(-2.425,5.59)--(-2.43,5.595)--(-2.383,5.589)--cycle;
\filldraw[fill opacity=0.8,fill=gray!20](-2.512,5.895)--(-2.474,5.923)--(-2.459,5.933)--(-2.491,5.909)--cycle;
\filldraw[fill opacity=0.8,fill=gray!20,draw=none](-3.274,3.017)--(-3.28,3.041)--(-3.295,3.008)--(-3.294,3.005)--cycle;
\draw(-3.295,3.008)--(-3.294,3.005);
\filldraw[fill opacity=0.8,fill=gray!20,draw=none](-3.325,3.221)--(-3.327,3.224)--(-3.334,3.226)--cycle;
\draw(-3.327,3.224)--(-3.334,3.226);
\filldraw[fill opacity=0.8,fill=gray!20](-2.465,5.604)--(-2.499,5.628)--(-2.512,5.643)--(-2.474,5.614)--cycle;
\filldraw[fill opacity=0.8,fill=gray!20](-2.383,5.589)--(-2.409,5.586)--(-2.425,5.59)--(-2.383,5.589)--cycle;
\filldraw[fill opacity=0.8,fill=gray!20,draw=none](-2.383,5.589)--(-2.359,5.589)--(-2.353,5.587)--(-2.362,5.585)--(-2.383,5.589)--cycle;
\draw(-2.353,5.587)--(-2.362,5.585)--(-2.383,5.589)--(-2.383,5.589)--(-2.359,5.589);
\filldraw[fill opacity=0.8,fill=gray!20,draw=none](-2.383,5.589)--(-2.363,5.591)--(-2.359,5.589)--(-2.383,5.589)--cycle;
\draw(-2.359,5.589)--(-2.383,5.589)--(-2.383,5.589)--(-2.363,5.591);
\filldraw[fill opacity=0.8,fill=gray!20](-2.383,5.589)--(-2.386,5.584)--(-2.409,5.586)--(-2.383,5.589)--cycle;
\filldraw[fill opacity=0.8,fill=gray!20](-2.383,5.589)--(-2.362,5.585)--(-2.386,5.584)--(-2.383,5.589)--cycle;
\filldraw[fill opacity=0.8,fill=gray!20](-2.333,5.938)--(-2.357,5.948)--(-2.341,5.944)--(-2.301,5.931)--cycle;
\filldraw[fill opacity=0.8,fill=gray!20,draw=none](-3.685,2.153)--(-3.683,2.156)--(-3.684,2.157)--cycle;
\draw(-3.683,2.156)--(-3.684,2.157);
\filldraw[fill opacity=0.8,fill=gray!20](-8.023,1.626)--(-8.023,1.68)--(-7.919,1.673)--(-7.915,1.619)--cycle;
\filldraw[fill opacity=0.8,fill=gray!20,draw=none](-7.953,1.66)--(-7.667,1.785)--(-7.694,1.822)--(-7.991,1.692)--cycle;
\draw(-7.694,1.822)--(-7.991,1.692)--(-7.953,1.66)--(-7.667,1.785);
\filldraw[fill opacity=0.8,fill=gray!20,draw=none](-4.02,2.374)--(-3.905,2.275)--(-4.038,2.386)--cycle;
\draw(-3.905,2.275)--(-4.038,2.386);
\filldraw[fill opacity=0.8,fill=gray!20,draw=none](-4.038,2.386)--(-3.905,2.275)--(-3.763,2.178)--(-3.968,2.349)--cycle;
\draw(-4.038,2.386)--(-3.905,2.275);
\draw(-3.763,2.178)--(-3.968,2.349);
\filldraw[fill opacity=0.8,fill=gray!20,draw=none](-2.521,5.799)--(-2.489,5.845)--(-2.488,5.837)--(-2.513,5.802)--cycle;
\draw(-2.521,5.799)--(-2.489,5.845)--(-2.488,5.837)--(-2.513,5.802);
\filldraw[fill opacity=0.8,fill=gray!20](-3.119,3.265)--(-3.139,3.308)--(-3.195,3.305)--(-3.197,3.261)--cycle;
\filldraw[fill opacity=0.8,fill=gray!20](-2.499,5.628)--(-2.525,5.662)--(-2.541,5.68)--(-2.512,5.643)--cycle;
\filldraw[fill opacity=0.8,fill=gray!20,draw=none](-7.661,1.634)--(-7.635,1.645)--(-7.635,1.689)--(-7.698,1.662)--cycle;
\draw(-7.661,1.634)--(-7.635,1.645);
\draw(-7.635,1.689)--(-7.698,1.662);
\filldraw[fill opacity=0.8,fill=gray!20,draw=none](-3.928,2.36)--(-3.683,2.156)--(-3.661,2.198)--(-3.925,2.418)--cycle;
\draw(-3.928,2.36)--(-3.683,2.156);
\draw(-3.661,2.198)--(-3.925,2.418);
\filldraw[fill opacity=0.8,fill=gray!20,draw=none](-2.562,5.723)--(-2.589,5.696)--(-2.562,5.734)--cycle;
\draw(-2.589,5.696)--(-2.562,5.734);
\filldraw[fill opacity=0.8,fill=gray!20,draw=none](-7.536,.856)--(-7.528,.863)--(-7.531,.884)--(-7.559,.875)--cycle;
\draw(-7.531,.884)--(-7.559,.875);
\filldraw[fill opacity=0.8,fill=gray!20,draw=none](-3.325,3.221)--(-3.296,3.204)--(-3.293,3.216)--(-3.327,3.224)--cycle;
\draw(-3.296,3.204)--(-3.293,3.216)--(-3.327,3.224);
\filldraw[fill opacity=0.8,fill=gray!20,draw=none](-3.103,2.999)--(-3.1,3.014)--(-3.2,2.997)--(-3.2,2.995)--cycle;
\draw(-3.2,2.997)--(-3.2,2.995)--(-3.103,2.999)--(-3.1,3.014);
\filldraw[fill opacity=0.8,fill=gray!20,draw=none](-7.531,1.051)--(-7.533,1.035)--(-7.527,1.036)--cycle;
\draw(-7.533,1.035)--(-7.527,1.036);
\filldraw[fill opacity=0.8,fill=gray!20](-2.292,5.611)--(-2.254,5.639)--(-2.275,5.625)--(-2.307,5.601)--cycle;
\filldraw[fill opacity=0.8,fill=gray!20](-2.525,5.662)--(-2.541,5.704)--(-2.559,5.723)--(-2.541,5.68)--cycle;
\filldraw[fill opacity=0.8,fill=gray!20,draw=none](-3.306,3.184)--(-3.308,3.211)--(-3.325,3.221)--cycle;
\filldraw[fill opacity=0.8,fill=gray!20,draw=none](-2.561,5.736)--(-2.556,5.75)--(-2.548,5.761)--(-2.526,5.787)--(-2.528,5.781)--(-2.559,5.737)--cycle;
\draw(-2.556,5.75)--(-2.548,5.761);
\draw(-2.528,5.781)--(-2.559,5.737);
\filldraw[fill opacity=0.8,fill=gray!20,draw=none](-3.562,2.276)--(-3.594,2.29)--(-3.62,2.247)--(-3.605,2.24)--cycle;
\draw(-3.62,2.247)--(-3.605,2.24)--(-3.562,2.276)--(-3.594,2.29);
\filldraw[fill opacity=0.8,fill=gray!20,draw=none](-3.274,3.017)--(-3.227,3.044)--(-3.277,3.047)--(-3.28,3.041)--cycle;
\draw(-3.227,3.044)--(-3.277,3.047);
\filldraw[fill opacity=0.8,fill=gray!20,draw=none](-4.457,3.104)--(-4.443,3.1)--(-4.443,3.099)--(-4.471,3.098)--cycle;
\draw(-4.443,3.099)--(-4.471,3.098);
\filldraw[fill opacity=0.8,fill=gray!20,draw=none](-4.615,2.971)--(-4.597,3.02)--(-4.523,3.034)--(-4.528,3.008)--cycle;
\draw(-4.615,2.971)--(-4.597,3.02)--(-4.523,3.034)--(-4.528,3.008);
\filldraw[fill opacity=0.8,fill=gray!20,draw=none](-4.508,3.007)--(-4.542,2.995)--(-4.548,2.998)--(-4.515,3.03)--(-4.499,3.022)--cycle;
\draw(-4.542,2.995)--(-4.548,2.998);
\draw(-4.515,3.03)--(-4.499,3.022);
\filldraw[fill opacity=0.8,fill=gray!20,draw=none](-4.443,3.1)--(-4.457,3.104)--(-4.442,3.109)--cycle;
\filldraw[fill opacity=0.8,fill=gray!20,draw=none](-4.499,3.022)--(-4.515,3.03)--(-4.486,3.049)--cycle;
\draw(-4.499,3.022)--(-4.515,3.03);
\filldraw[fill opacity=0.8,fill=gray!20,draw=none](-4.524,2.996)--(-2.578,5.72)--(-2.561,5.736)--(-2.562,5.734)--(-4.511,3.005)--cycle;
\draw(-4.524,2.996)--(-2.578,5.72);
\draw(-2.562,5.734)--(-4.511,3.005);
\filldraw[fill opacity=0.8,fill=gray!20,draw=none](-3.1,3.014)--(-3.093,3.047)--(-3.201,3.042)--(-3.2,2.997)--cycle;
\draw(-3.1,3.014)--(-3.093,3.047)--(-3.201,3.042)--(-3.2,2.997);
\filldraw[fill opacity=0.8,fill=gray!20,draw=none](-7.528,1.021)--(-7.53,.989)--(-7.529,.983)--(-7.491,.995)--cycle;
\draw(-7.529,.983)--(-7.491,.995);
\filldraw[fill opacity=0.8,fill=gray!20,draw=none](-2.526,5.787)--(-2.548,5.761)--(-2.521,5.799)--cycle;
\draw(-2.548,5.761)--(-2.521,5.799);
\filldraw[fill opacity=0.8,fill=gray!20,draw=none](-2.545,5.739)--(-2.526,5.766)--(-2.521,5.799)--(-2.556,5.75)--cycle;
\draw(-2.545,5.739)--(-2.526,5.766);
\draw(-2.521,5.799)--(-2.556,5.75);
\filldraw[fill opacity=0.8,fill=gray!20](-2.541,5.704)--(-2.547,5.749)--(-2.566,5.77)--(-2.559,5.723)--cycle;
\filldraw[fill opacity=0.8,fill=gray!20](-2.254,5.639)--(-2.225,5.676)--(-2.251,5.659)--(-2.275,5.625)--cycle;
\filldraw[fill opacity=0.8,fill=gray!20,draw=none](-2.561,5.736)--(-2.578,5.72)--(-2.556,5.75)--cycle;
\draw(-2.578,5.72)--(-2.556,5.75);
\filldraw[fill opacity=0.8,fill=gray!20](-2.409,5.586)--(-2.433,5.596)--(-2.465,5.604)--(-2.425,5.59)--cycle;
\filldraw[fill opacity=0.8,fill=gray!20,draw=none](-7.559,.875)--(-7.51,.891)--(-7.504,.938)--(-7.602,.906)--cycle;
\draw(-7.559,.875)--(-7.51,.891);
\draw(-7.504,.938)--(-7.602,.906);
\filldraw[fill opacity=0.8,fill=gray!20,draw=none](-4.247,2.754)--(-3.636,2.243)--(-3.601,2.278)--(-4.232,2.805)--cycle;
\draw(-4.247,2.754)--(-3.636,2.243);
\draw(-3.601,2.278)--(-4.232,2.805);
\filldraw[fill opacity=0.8,fill=gray!20,draw=none](-3.601,2.278)--(-3.629,2.251)--(-3.62,2.247)--cycle;
\draw(-3.629,2.251)--(-3.62,2.247);
\filldraw[fill opacity=0.8,fill=gray!20,draw=none](-3.21,2.996)--(-3.2,2.995)--(-3.2,2.997)--cycle;
\draw(-3.21,2.996)--(-3.2,2.995)--(-3.2,2.997);
\filldraw[fill opacity=0.8,fill=gray!20,draw=none](-2.347,5.589)--(-2.343,5.589)--(-2.349,5.588)--cycle;
\draw(-2.347,5.589)--(-2.343,5.589)--(-2.349,5.588);
\filldraw[fill opacity=0.8,fill=gray!20,draw=none](-2.286,5.92)--(-2.301,5.931)--(-2.292,5.921)--(-2.27,5.904)--cycle;
\draw(-2.286,5.92)--(-2.301,5.931)--(-2.292,5.921)--(-2.27,5.904);
\filldraw[fill opacity=0.8,fill=gray!20](-2.474,5.923)--(-2.43,5.94)--(-2.422,5.945)--(-2.459,5.933)--cycle;
\filldraw[fill opacity=0.8,fill=gray!20](-3.2,3.209)--(-3.197,3.261)--(-3.274,3.267)--(-3.293,3.216)--cycle;
\filldraw[fill opacity=0.8,fill=gray!20](-8.023,1.569)--(-8.023,1.626)--(-7.915,1.619)--(-7.919,1.562)--cycle;
\filldraw[fill opacity=0.8,fill=gray!20,draw=none](-7.85,1.551)--(-7.853,1.546)--(-7.919,1.562)--(-7.915,1.619)--(-7.839,1.6)--cycle;
\draw(-7.853,1.546)--(-7.919,1.562)--(-7.915,1.619)--(-7.839,1.6);
\filldraw[fill opacity=0.8,fill=gray!20,draw=none](-7.915,1.619)--(-7.919,1.673)--(-7.853,1.656)--(-7.85,1.65)--(-7.839,1.6)--cycle;
\draw(-7.839,1.6)--(-7.915,1.619)--(-7.919,1.673)--(-7.853,1.656);
\filldraw[fill opacity=0.8,fill=gray!20,draw=none](-7.918,1.566)--(-7.635,1.689)--(-7.635,1.743)--(-7.927,1.616)--cycle;
\draw(-7.635,1.743)--(-7.927,1.616)--(-7.918,1.566)--(-7.635,1.689);
\filldraw[fill opacity=0.8,fill=gray!20,draw=none](-2.353,5.587)--(-2.349,5.589)--(-2.351,5.59)--(-2.362,5.585)--cycle;
\draw(-2.351,5.59)--(-2.362,5.585)--(-2.353,5.587);
\filldraw[fill opacity=0.8,fill=gray!20,draw=none](-2.362,5.585)--(-2.351,5.59)--(-2.351,5.594)--(-2.388,5.593)--(-2.386,5.584)--cycle;
\draw(-2.351,5.594)--(-2.388,5.593)--(-2.386,5.584)--(-2.362,5.585)--(-2.351,5.59);
\filldraw[fill opacity=0.8,fill=gray!20,draw=none](-4.164,3.075)--(-4.146,3.074)--(-4.162,3.051)--cycle;
\draw(-4.146,3.074)--(-4.162,3.051);
\filldraw[fill opacity=0.8,fill=gray!20,draw=none](-4.146,3.074)--(-2.277,5.69)--(-2.278,5.697)--(-4.156,3.068)--cycle;
\draw(-4.146,3.074)--(-2.277,5.69)--(-2.278,5.697)--(-4.156,3.068);
\filldraw[fill opacity=0.8,fill=gray!20,draw=none](-2.267,5.904)--(-2.259,5.898)--(-2.264,5.903)--cycle;
\draw(-2.259,5.898)--(-2.264,5.903);
\filldraw[fill opacity=0.8,fill=gray!20,draw=none](-2.349,5.589)--(-2.349,5.588)--(-2.343,5.589)--(-2.307,5.601)--(-2.33,5.597)--cycle;
\draw(-2.349,5.588)--(-2.343,5.589)--(-2.307,5.601)--(-2.33,5.597);
\filldraw[fill opacity=0.8,fill=gray!20,draw=none](-2.559,7.724)--(-2.574,7.716)--(-2.557,7.743)--cycle;
\draw(-2.574,7.716)--(-2.557,7.743);
\filldraw[fill opacity=0.8,fill=gray!20,draw=none](-2.561,7.708)--(-2.574,7.716)--(-2.559,7.724)--cycle;
\filldraw[fill opacity=0.8,fill=gray!20,draw=none](-2.587,7.868)--(-2.281,5.961)--(-2.288,5.933)--(-2.601,7.881)--cycle;
\draw(-2.288,5.933)--(-2.601,7.881)--(-2.587,7.868)--(-2.281,5.961);
\filldraw[fill opacity=0.8,fill=gray!20](-2.547,5.749)--(-2.541,5.796)--(-2.559,5.816)--(-2.566,5.77)--cycle;
\filldraw[fill opacity=0.8,fill=gray!20](-2.225,5.676)--(-2.207,5.719)--(-2.235,5.7)--(-2.251,5.659)--cycle;
\filldraw[fill opacity=0.8,fill=gray!20,draw=none](-3.28,3.041)--(-3.277,3.047)--(-3.282,3.048)--cycle;
\draw(-3.277,3.047)--(-3.282,3.048);
\filldraw[fill opacity=0.8,fill=gray!20,draw=none](-7.537,1.027)--(-7.528,1.021)--(-7.527,1.036)--(-7.538,1.033)--cycle;
\draw(-7.527,1.036)--(-7.538,1.033);
\filldraw[fill opacity=0.8,fill=gray!20,draw=none](-3.306,3.184)--(-3.302,3.176)--(-3.296,3.204)--(-3.308,3.211)--cycle;
\draw(-3.302,3.176)--(-3.296,3.204);
\filldraw[fill opacity=0.8,fill=gray!20](-2.404,5.949)--(-2.383,5.946)--(-2.383,5.946)--(-2.38,5.95)--cycle;
\filldraw[fill opacity=0.8,fill=gray!20](-2.38,5.95)--(-2.383,5.946)--(-2.383,5.946)--(-2.357,5.948)--cycle;
\filldraw[fill opacity=0.8,fill=gray!20,draw=none](-3.277,3.088)--(-3.282,3.048)--(-3.277,3.047)--(-3.266,3.062)--cycle;
\draw(-3.282,3.048)--(-3.277,3.047);
\filldraw[fill opacity=0.8,fill=gray!20,draw=none](-2.816,7.669)--(-2.823,7.664)--(-2.829,7.675)--(-2.814,7.676)--cycle;
\draw(-2.816,7.669)--(-2.823,7.664);
\filldraw[fill opacity=0.8,fill=gray!20,draw=none](-2.82,7.66)--(-2.821,7.66)--(-2.823,7.664)--(-2.816,7.669)--cycle;
\draw(-2.82,7.66)--(-2.821,7.66);
\draw(-2.823,7.664)--(-2.816,7.669);
\filldraw[fill opacity=0.8,fill=gray!20,draw=none](-2.841,7.84)--(-2.527,5.888)--(-2.529,5.874)--(-2.533,5.874)--(-2.847,7.824)--cycle;
\draw(-2.533,5.874)--(-2.847,7.824)--(-2.841,7.84)--(-2.527,5.888);
\filldraw[fill opacity=0.8,fill=gray!20](-2.301,5.931)--(-2.341,5.944)--(-2.336,5.939)--(-2.292,5.921)--cycle;
\filldraw[fill opacity=0.8,fill=gray!20](-3.103,3.214)--(-3.119,3.265)--(-3.197,3.261)--(-3.2,3.209)--cycle;
\filldraw[fill opacity=0.8,fill=gray!20,draw=none](-3.266,3.062)--(-3.277,3.047)--(-3.259,3.046)--cycle;
\draw(-3.277,3.047)--(-3.259,3.046);
\filldraw[fill opacity=0.8,fill=gray!20,draw=none](-7.927,1.616)--(-7.646,1.738)--(-7.667,1.785)--(-7.953,1.66)--cycle;
\draw(-7.667,1.785)--(-7.953,1.66)--(-7.927,1.616)--(-7.646,1.738);
\filldraw[fill opacity=0.8,fill=gray!20,draw=none](-2.278,5.889)--(-2.273,5.906)--(-2.292,5.921)--(-2.307,5.911)--(-2.285,5.889)--cycle;
\draw(-2.273,5.906)--(-2.292,5.921)--(-2.307,5.911)--(-2.285,5.889);
\filldraw[fill opacity=0.8,fill=gray!20,draw=none](-2.281,5.91)--(-2.259,5.773)--(-2.253,5.789)--(-2.272,5.906)--cycle;
\draw(-2.281,5.91)--(-2.259,5.773)--(-2.253,5.789)--(-2.272,5.906);
\filldraw[fill opacity=0.8,fill=gray!20,draw=none](-4.064,2.535)--(-3.661,2.198)--(-3.647,2.253)--(-4.107,2.637)--cycle;
\draw(-4.064,2.535)--(-3.661,2.198);
\draw(-3.647,2.253)--(-4.107,2.637);
\filldraw[fill opacity=0.8,fill=gray!20,draw=none](-4.976,2.661)--(-6.356,2.17)--(-6.721,1.992)--(-5.242,2.518)--cycle;
\draw(-4.976,2.661)--(-6.356,2.17);
\draw(-6.721,1.992)--(-5.242,2.518);
\filldraw[fill opacity=0.8,fill=gray!20](-2.422,5.945)--(-2.383,5.946)--(-2.383,5.946)--(-2.404,5.949)--cycle;
\filldraw[fill opacity=0.8,fill=gray!20,draw=none](-3.633,2.241)--(-3.66,2.203)--(-3.652,2.199)--cycle;
\draw(-3.66,2.203)--(-3.652,2.199);
\filldraw[fill opacity=0.8,fill=gray!20,draw=none](-3.66,2.203)--(-3.633,2.241)--(-3.647,2.253)--cycle;
\draw(-3.633,2.241)--(-3.647,2.253);
\filldraw[fill opacity=0.8,fill=gray!20](-3.093,3.047)--(-3.09,3.101)--(-3.201,3.096)--(-3.201,3.042)--cycle;
\filldraw[fill opacity=0.8,fill=gray!20,draw=none](-2.541,5.796)--(-2.534,5.815)--(-2.532,5.849)--(-2.541,5.859)--(-2.559,5.816)--cycle;
\draw(-2.532,5.849)--(-2.541,5.859)--(-2.559,5.816)--(-2.541,5.796)--(-2.534,5.815);
\filldraw[fill opacity=0.8,fill=gray!20](-2.386,5.584)--(-2.388,5.593)--(-2.433,5.596)--(-2.409,5.586)--cycle;
\filldraw[fill opacity=0.8,fill=gray!20](-2.207,5.719)--(-2.2,5.765)--(-2.23,5.745)--(-2.235,5.7)--cycle;
\filldraw[fill opacity=0.8,fill=gray!20](-2.357,5.948)--(-2.383,5.946)--(-2.383,5.946)--(-2.341,5.944)--cycle;
\filldraw[fill opacity=0.8,fill=gray!20,draw=none](-3.605,2.24)--(-3.629,2.251)--(-3.652,2.199)--(-3.633,2.191)--cycle;
\draw(-3.652,2.199)--(-3.633,2.191)--(-3.605,2.24)--(-3.629,2.251);
\filldraw[fill opacity=0.8,fill=gray!20,draw=none](-2.349,5.589)--(-2.33,5.597)--(-2.342,5.595)--(-2.351,5.59)--cycle;
\draw(-2.33,5.597)--(-2.342,5.595)--(-2.351,5.59);
\filldraw[fill opacity=0.8,fill=gray!20,draw=none](-3.266,3.062)--(-3.259,3.046)--(-3.201,3.042)--(-3.201,3.096)--(-3.237,3.098)--cycle;
\draw(-3.259,3.046)--(-3.201,3.042)--(-3.201,3.096)--(-3.237,3.098);
\filldraw[fill opacity=0.8,fill=gray!20,draw=none](-3.304,3.163)--(-3.302,3.176)--(-3.306,3.184)--cycle;
\draw(-3.304,3.163)--(-3.302,3.176);
\filldraw[fill opacity=0.8,fill=gray!20,draw=none](-2.259,5.898)--(-2.267,5.904)--(-2.272,5.906)--(-2.27,5.904)--(-2.254,5.892)--cycle;
\draw(-2.27,5.904)--(-2.254,5.892)--(-2.259,5.898);
\filldraw[fill opacity=0.8,fill=gray!20,draw=none](-7.537,1.027)--(-7.53,.989)--(-7.528,1.021)--cycle;
\filldraw[fill opacity=0.8,fill=gray!20,draw=none](-2.508,5.791)--(-2.473,5.84)--(-2.489,5.845)--(-2.521,5.799)--cycle;
\draw(-2.508,5.791)--(-2.473,5.84)--(-2.489,5.845)--(-2.521,5.799);
\filldraw[fill opacity=0.8,fill=gray!20,draw=none](-2.531,5.858)--(-2.518,5.851)--(-2.499,5.881)--(-2.512,5.895)--(-2.529,5.874)--cycle;
\draw(-2.518,5.851)--(-2.499,5.881)--(-2.512,5.895)--(-2.529,5.874);
\filldraw[fill opacity=0.8,fill=gray!20,draw=none](-2.816,7.669)--(-2.814,7.676)--(-2.81,7.676)--(-2.808,7.674)--cycle;
\draw(-2.81,7.676)--(-2.808,7.674)--(-2.816,7.669);
\filldraw[fill opacity=0.8,fill=gray!20,draw=none](-2.774,7.64)--(-2.82,7.66)--(-2.816,7.669)--(-2.808,7.674)--(-2.764,7.646)--cycle;
\draw(-2.816,7.669)--(-2.808,7.674)--(-2.764,7.646)--(-2.774,7.64)--(-2.82,7.66);
\filldraw[fill opacity=0.8,fill=gray!20,draw=none](-2.768,7.634)--(-2.806,7.647)--(-2.82,7.659)--(-2.82,7.66)--(-2.774,7.64)--cycle;
\draw(-2.82,7.66)--(-2.774,7.64)--(-2.768,7.634)--(-2.806,7.647);
\filldraw[fill opacity=0.8,fill=gray!20](-2.847,7.824)--(-2.513,5.745)--(-2.499,5.733)--(-2.833,7.812)--cycle;
\filldraw[fill opacity=0.8,fill=gray!20,draw=none](-2.539,5.86)--(-2.541,5.859)--(-2.538,5.862)--cycle;
\draw(-2.539,5.86)--(-2.541,5.859)--(-2.538,5.862);
\filldraw[fill opacity=0.8,fill=gray!20,draw=none](-3.277,3.088)--(-3.266,3.062)--(-3.237,3.098)--(-3.276,3.101)--cycle;
\draw(-3.237,3.098)--(-3.276,3.101);
\filldraw[fill opacity=0.8,fill=gray!20,draw=none](-3.288,3.166)--(-3.276,3.215)--(-3.293,3.216)--(-3.302,3.176)--cycle;
\draw(-3.276,3.215)--(-3.293,3.216)--(-3.302,3.176);
\filldraw[fill opacity=0.8,fill=gray!20](-2.433,5.596)--(-2.454,5.617)--(-2.499,5.628)--(-2.465,5.604)--cycle;
\filldraw[fill opacity=0.8,fill=gray!20,draw=none](-7.568,.971)--(-7.529,.983)--(-7.538,1.033)--(-7.577,1.02)--cycle;
\draw(-7.568,.971)--(-7.529,.983);
\draw(-7.538,1.033)--(-7.577,1.02);
\filldraw[fill opacity=0.8,fill=gray!20](-7.961,.885)--(-7.962,.939)--(-7.858,.932)--(-7.853,.877)--cycle;
\filldraw[fill opacity=0.8,fill=gray!20](-7.853,.877)--(-7.858,.932)--(-7.785,.914)--(-7.778,.859)--cycle;
\filldraw[fill opacity=0.8,fill=gray!20,draw=none](-7.897,.917)--(-7.562,1.025)--(-7.588,1.061)--(-7.933,.95)--cycle;
\draw(-7.588,1.061)--(-7.933,.95)--(-7.897,.917)--(-7.562,1.025);
\filldraw[fill opacity=0.8,fill=gray!20,draw=none](-3.304,3.16)--(-3.277,3.158)--(-3.302,3.176)--(-3.304,3.163)--cycle;
\draw(-3.304,3.16)--(-3.277,3.158);
\draw(-3.302,3.176)--(-3.304,3.163);
\filldraw[fill opacity=0.8,fill=gray!20,draw=none](-3.277,3.088)--(-3.276,3.101)--(-3.284,3.102)--cycle;
\draw(-3.276,3.101)--(-3.284,3.102);
\filldraw[fill opacity=0.8,fill=gray!20,draw=none](-3.288,3.166)--(-3.277,3.158)--(-3.201,3.153)--(-3.2,3.209)--(-3.276,3.215)--cycle;
\draw(-3.277,3.158)--(-3.201,3.153)--(-3.2,3.209)--(-3.276,3.215);
\filldraw[fill opacity=0.8,fill=gray!20,draw=none](-2.532,5.849)--(-2.531,5.858)--(-2.538,5.863)--(-2.541,5.859)--cycle;
\draw(-2.538,5.863)--(-2.541,5.859)--(-2.532,5.849);
\filldraw[fill opacity=0.8,fill=gray!20](-2.2,5.765)--(-2.207,5.811)--(-2.235,5.792)--(-2.23,5.745)--cycle;
\filldraw[fill opacity=0.8,fill=gray!20,draw=none](-3.287,3.138)--(-3.29,3.159)--(-3.304,3.16)--cycle;
\draw(-3.29,3.159)--(-3.304,3.16);
\filldraw[fill opacity=0.8,fill=gray!20,draw=none](-3.284,3.102)--(-3.276,3.101)--(-3.27,3.114)--(-3.287,3.138)--cycle;
\draw(-3.284,3.102)--(-3.276,3.101);
\filldraw[fill opacity=0.8,fill=gray!20](-2.43,5.94)--(-2.383,5.946)--(-2.383,5.946)--(-2.422,5.945)--cycle;
\filldraw[fill opacity=0.8,fill=gray!20](-2.307,5.601)--(-2.275,5.625)--(-2.325,5.616)--(-2.342,5.595)--cycle;
\filldraw[fill opacity=0.8,fill=gray!20,draw=none](-3.287,3.138)--(-3.27,3.114)--(-3.252,3.157)--(-3.29,3.159)--cycle;
\draw(-3.252,3.157)--(-3.29,3.159);
\filldraw[fill opacity=0.8,fill=gray!20,draw=none](-7.921,1.55)--(-7.828,1.561)--(-7.661,1.634)--(-7.698,1.662)--(-7.918,1.566)--cycle;
\draw(-7.828,1.561)--(-7.661,1.634);
\draw(-7.698,1.662)--(-7.918,1.566)--(-7.921,1.55);
\filldraw[fill opacity=0.8,fill=gray!20,draw=none](-3.27,3.114)--(-3.276,3.101)--(-3.26,3.1)--cycle;
\draw(-3.276,3.101)--(-3.26,3.1);
\filldraw[fill opacity=0.8,fill=gray!20](-3.093,3.158)--(-3.103,3.214)--(-3.2,3.209)--(-3.201,3.153)--cycle;
\filldraw[fill opacity=0.8,fill=gray!20](-3.09,3.101)--(-3.093,3.158)--(-3.201,3.153)--(-3.201,3.096)--cycle;
\filldraw[fill opacity=0.8,fill=gray!20](-2.341,5.944)--(-2.383,5.946)--(-2.383,5.946)--(-2.336,5.939)--cycle;
\filldraw[fill opacity=0.8,fill=gray!20,draw=none](-3.27,3.114)--(-3.26,3.1)--(-3.201,3.096)--(-3.201,3.153)--(-3.252,3.157)--cycle;
\draw(-3.26,3.1)--(-3.201,3.096)--(-3.201,3.153)--(-3.252,3.157);
\filldraw[fill opacity=0.8,fill=gray!20,draw=none](-2.526,5.766)--(-2.508,5.791)--(-2.521,5.799)--cycle;
\draw(-2.526,5.766)--(-2.508,5.791);
\filldraw[fill opacity=0.8,fill=gray!20](-2.499,5.881)--(-2.465,5.913)--(-2.474,5.923)--(-2.512,5.895)--cycle;
\filldraw[fill opacity=0.8,fill=gray!20](-2.207,5.811)--(-2.225,5.854)--(-2.251,5.837)--(-2.235,5.792)--cycle;
\filldraw[fill opacity=0.8,fill=gray!20,draw=none](-2.351,5.59)--(-2.342,5.595)--(-2.351,5.594)--cycle;
\draw(-2.351,5.59)--(-2.342,5.595)--(-2.351,5.594);
\filldraw[fill opacity=0.8,fill=gray!20,draw=none](-4.471,3.024)--(-4.487,3.039)--(-4.443,3.1)--(-4.44,3.099)--(-4.426,3.083)--(-4.467,3.025)--cycle;
\draw(-4.487,3.039)--(-4.443,3.1);
\draw(-4.426,3.083)--(-4.467,3.025);
\filldraw[fill opacity=0.8,fill=gray!20,draw=none](-4.44,3.099)--(-4.431,3.1)--(-4.43,3.088)--cycle;
\draw(-4.44,3.099)--(-4.431,3.1)--(-4.43,3.088);
\filldraw[fill opacity=0.8,fill=gray!20,draw=none](-4.443,3.1)--(-4.442,3.102)--(-4.44,3.099)--cycle;
\draw(-4.443,3.1)--(-4.442,3.102);
\filldraw[fill opacity=0.8,fill=gray!20,draw=none](-4.44,3.099)--(-4.443,3.1)--(-4.442,3.109)--(-4.434,3.109)--(-4.431,3.1)--cycle;
\draw(-4.442,3.109)--(-4.434,3.109)--(-4.431,3.1)--(-4.44,3.099);
\filldraw[fill opacity=0.8,fill=gray!20,draw=none](-4.416,3.074)--(-4.424,3.086)--(-4.409,3.107)--(-4.396,3.061)--cycle;
\draw(-4.424,3.086)--(-4.409,3.107);
\filldraw[fill opacity=0.8,fill=gray!20](-4.431,3.1)--(-4.434,3.109)--(-4.407,3.107)--(-4.377,3.096)--cycle;
\filldraw[fill opacity=0.8,fill=gray!20,draw=none](-4.495,3.012)--(-4.5,3.01)--(-4.499,3.022)--cycle;
\filldraw[fill opacity=0.8,fill=gray!20,draw=none](-4.508,3.007)--(-4.499,3.022)--(-4.485,3.015)--cycle;
\draw(-4.499,3.022)--(-4.485,3.015);
\filldraw[fill opacity=0.8,fill=gray!20,draw=none](-4.5,3.01)--(-4.506,3.007)--(-4.507,3.011)--(-4.499,3.022)--cycle;
\draw(-4.507,3.011)--(-4.499,3.022);
\filldraw[fill opacity=0.8,fill=gray!20,draw=none](-4.427,3.082)--(-4.424,3.086)--(-4.416,3.074)--cycle;
\draw(-4.427,3.082)--(-4.424,3.086);
\filldraw[fill opacity=0.8,fill=gray!20,draw=none](-4.423,3.075)--(-4.43,3.088)--(-4.431,3.1)--(-4.377,3.096)--(-4.352,3.07)--cycle;
\draw(-4.43,3.088)--(-4.431,3.1)--(-4.377,3.096)--(-4.352,3.07)--(-4.423,3.075);
\filldraw[fill opacity=0.8,fill=gray!20,draw=none](-4.467,3.025)--(-4.427,3.082)--(-4.416,3.074)--(-4.401,3.053)--cycle;
\draw(-4.467,3.025)--(-4.427,3.082);
\filldraw[fill opacity=0.8,fill=gray!20,draw=none](-4.489,3.017)--(-4.499,3.022)--(-4.486,3.049)--(-4.436,3.024)--cycle;
\draw(-4.489,3.017)--(-4.499,3.022);
\draw(-4.486,3.049)--(-4.436,3.024);
\filldraw[fill opacity=0.8,fill=gray!20,draw=none](-4.513,3.002)--(-4.511,3.005)--(-4.506,3.007)--cycle;
\draw(-4.513,3.002)--(-4.511,3.005);
\filldraw[fill opacity=0.8,fill=gray!20,draw=none](-4.511,3.005)--(-4.507,3.011)--(-4.506,3.007)--cycle;
\draw(-4.511,3.005)--(-4.507,3.011);
\filldraw[fill opacity=0.8,fill=gray!20,draw=none](-4.464,3.038)--(-4.486,3.049)--(-4.465,3.051)--(-4.412,3.024)--cycle;
\draw(-4.464,3.038)--(-4.486,3.049);
\draw(-4.465,3.051)--(-4.412,3.024);
\filldraw[fill opacity=0.8,fill=gray!20,draw=none](-4.442,3.109)--(-4.436,3.107)--(-4.434,3.109)--cycle;
\draw(-4.436,3.107)--(-4.434,3.109)--(-4.442,3.109);
\filldraw[fill opacity=0.8,fill=gray!20,draw=none](-4.434,3.109)--(-4.436,3.107)--(-4.429,3.105)--(-4.407,3.107)--cycle;
\draw(-4.429,3.105)--(-4.407,3.107)--(-4.434,3.109)--(-4.436,3.107);
\filldraw[fill opacity=0.8,fill=gray!20](-4.377,3.096)--(-4.407,3.107)--(-4.387,3.103)--(-4.34,3.086)--cycle;
\filldraw[fill opacity=0.8,fill=gray!20,draw=none](-4.407,3.107)--(-4.429,3.105)--(-4.427,3.104)--(-4.387,3.103)--cycle;
\draw(-4.427,3.104)--(-4.387,3.103)--(-4.407,3.107)--(-4.429,3.105);
\filldraw[fill opacity=0.8,fill=gray!20,draw=none](-4.515,2.997)--(-4.502,3)--(-2.545,5.739)--(-2.556,5.75)--(-4.514,3.009)--cycle;
\draw(-4.502,3)--(-2.545,5.739);
\draw(-2.556,5.75)--(-4.514,3.009);
\filldraw[fill opacity=0.8,fill=gray!20](-2.454,5.617)--(-2.47,5.649)--(-2.525,5.662)--(-2.499,5.628)--cycle;
\filldraw[fill opacity=0.8,fill=gray!20](-2.465,5.913)--(-2.425,5.935)--(-2.43,5.94)--(-2.474,5.923)--cycle;
\filldraw[fill opacity=0.8,fill=gray!20,draw=none](-2.245,5.841)--(-2.225,5.854)--(-2.254,5.892)--(-2.261,5.887)--cycle;
\draw(-2.245,5.841)--(-2.225,5.854)--(-2.254,5.892)--(-2.261,5.887);
\filldraw[fill opacity=0.8,fill=gray!20,draw=none](-2.563,7.694)--(-2.578,7.71)--(-2.574,7.716)--(-2.561,7.708)--cycle;
\draw(-2.563,7.694)--(-2.578,7.71)--(-2.574,7.716);
\filldraw[fill opacity=0.8,fill=gray!20,draw=none](-2.634,7.721)--(-2.604,7.71)--(-2.574,7.525)--(-2.625,7.568)--(-2.649,7.719)--cycle;
\draw(-2.604,7.71)--(-2.574,7.525);
\draw(-2.625,7.568)--(-2.649,7.719);
\filldraw[fill opacity=0.8,fill=gray!20,draw=none](-2.619,7.672)--(-2.578,7.71)--(-2.563,7.694)--(-2.564,7.691)--(-2.607,7.659)--cycle;
\draw(-2.564,7.691)--(-2.607,7.659)--(-2.619,7.672)--(-2.578,7.71)--(-2.563,7.694);
\filldraw[fill opacity=0.8,fill=gray!20](-2.666,7.645)--(-2.619,7.672)--(-2.607,7.659)--(-2.66,7.639)--cycle;
\filldraw[fill opacity=0.8,fill=gray!20](-2.66,7.639)--(-2.607,7.659)--(-2.625,7.648)--(-2.669,7.633)--cycle;
\filldraw[fill opacity=0.8,fill=gray!20,draw=none](-2.607,7.659)--(-2.564,7.691)--(-2.566,7.69)--(-2.571,7.687)--(-2.592,7.673)--(-2.598,7.668)--(-2.625,7.648)--cycle;
\draw(-2.566,7.69)--(-2.571,7.687);
\draw(-2.598,7.668)--(-2.625,7.648)--(-2.607,7.659)--(-2.564,7.691);
\filldraw[fill opacity=0.8,fill=gray!20,draw=none](-2.571,7.687)--(-2.566,7.69)--(-2.567,7.69)--cycle;
\draw(-2.571,7.687)--(-2.566,7.69);
\filldraw[fill opacity=0.8,fill=gray!20,draw=none](-2.593,7.853)--(-2.281,5.91)--(-2.272,5.906)--(-2.587,7.868)--cycle;
\draw(-2.272,5.906)--(-2.587,7.868)--(-2.593,7.853)--(-2.281,5.91);
\filldraw[fill opacity=0.8,fill=gray!20](-2.425,5.935)--(-2.383,5.946)--(-2.383,5.946)--(-2.43,5.94)--cycle;
\filldraw[fill opacity=0.8,fill=gray!20,draw=none](-2.261,5.887)--(-2.254,5.892)--(-2.27,5.904)--cycle;
\draw(-2.261,5.887)--(-2.254,5.892)--(-2.27,5.904);
\filldraw[fill opacity=0.8,fill=gray!20,draw=none](-2.351,5.594)--(-2.35,5.597)--(-2.364,5.614)--(-2.391,5.613)--(-2.388,5.593)--cycle;
\draw(-2.364,5.614)--(-2.391,5.613)--(-2.388,5.593)--(-2.351,5.594);
\filldraw[fill opacity=0.8,fill=gray!20,draw=none](-2.364,5.614)--(-2.364,5.617)--(-2.392,5.635)--(-2.391,5.613)--cycle;
\draw(-2.392,5.635)--(-2.391,5.613)--(-2.364,5.614);
\filldraw[fill opacity=0.8,fill=gray!20,draw=none](-4.178,3.111)--(-3.764,3.691)--(-3.764,3.635)--(-4.081,3.19)--cycle;
\draw(-4.178,3.111)--(-3.764,3.691);
\draw(-3.764,3.635)--(-4.081,3.19);
\filldraw[fill opacity=0.8,fill=gray!20,draw=none](-4.207,3.071)--(-4.178,3.111)--(-4.081,3.19)--(-4.164,3.075)--cycle;
\draw(-4.207,3.071)--(-4.178,3.111);
\draw(-4.081,3.19)--(-4.164,3.075);
\filldraw[fill opacity=0.8,fill=gray!20,draw=none](-4.156,3.068)--(-2.278,5.697)--(-2.294,5.715)--(-4.192,3.058)--cycle;
\draw(-4.156,3.068)--(-2.278,5.697)--(-2.294,5.715)--(-4.192,3.058);
\filldraw[fill opacity=0.8,fill=gray!20,draw=none](-2.285,5.889)--(-2.303,5.872)--(-2.284,5.757)--(-2.259,5.773)--(-2.278,5.889)--cycle;
\draw(-2.303,5.872)--(-2.284,5.757)--(-2.259,5.773)--(-2.278,5.889);
\filldraw[fill opacity=0.8,fill=gray!20,draw=none](-2.261,5.887)--(-2.27,5.904)--(-2.273,5.906)--(-2.28,5.883)--(-2.275,5.878)--cycle;
\draw(-2.27,5.904)--(-2.273,5.906);
\draw(-2.28,5.883)--(-2.275,5.878)--(-2.261,5.887);
\filldraw[fill opacity=0.8,fill=gray!20,draw=none](-2.531,5.858)--(-2.532,5.849)--(-2.525,5.841)--(-2.518,5.851)--cycle;
\draw(-2.532,5.849)--(-2.525,5.841)--(-2.518,5.851);
\filldraw[fill opacity=0.8,fill=gray!20](-2.362,5.93)--(-2.383,5.946)--(-2.383,5.946)--(-2.386,5.929)--cycle;
\filldraw[fill opacity=0.8,fill=gray!20](-2.386,5.929)--(-2.383,5.946)--(-2.383,5.946)--(-2.409,5.931)--cycle;
\filldraw[fill opacity=0.8,fill=gray!20](-2.343,5.934)--(-2.383,5.946)--(-2.383,5.946)--(-2.362,5.93)--cycle;
\filldraw[fill opacity=0.8,fill=gray!20](-2.409,5.931)--(-2.383,5.946)--(-2.383,5.946)--(-2.425,5.935)--cycle;
\filldraw[fill opacity=0.8,fill=gray!20](-2.336,5.939)--(-2.383,5.946)--(-2.383,5.946)--(-2.343,5.934)--cycle;
\filldraw[fill opacity=0.8,fill=gray!20](-2.292,5.921)--(-2.336,5.939)--(-2.343,5.934)--(-2.307,5.911)--cycle;
\filldraw[fill opacity=0.8,fill=gray!20](-2.388,5.593)--(-2.391,5.613)--(-2.454,5.617)--(-2.433,5.596)--cycle;
\filldraw[fill opacity=0.8,fill=gray!20,draw=none](-7.564,.919)--(-7.53,.929)--(-7.542,.979)--(-7.568,.971)--cycle;
\draw(-7.564,.919)--(-7.53,.929);
\draw(-7.542,.979)--(-7.568,.971);
\filldraw[fill opacity=0.8,fill=gray!20,draw=none](-2.301,5.62)--(-2.275,5.625)--(-2.251,5.659)--(-2.267,5.656)--cycle;
\draw(-2.301,5.62)--(-2.275,5.625)--(-2.251,5.659)--(-2.267,5.656);
\filldraw[fill opacity=0.8,fill=gray!20,draw=none](-2.35,5.597)--(-2.348,5.594)--(-2.342,5.595)--(-2.325,5.616)--(-2.341,5.615)--cycle;
\draw(-2.348,5.594)--(-2.342,5.595)--(-2.325,5.616)--(-2.341,5.615);
\filldraw[fill opacity=0.8,fill=gray!20,draw=none](-2.35,5.597)--(-2.351,5.594)--(-2.348,5.594)--cycle;
\draw(-2.351,5.594)--(-2.348,5.594);
\filldraw[fill opacity=0.8,fill=gray!20](-2.47,5.649)--(-2.48,5.689)--(-2.541,5.704)--(-2.525,5.662)--cycle;
\filldraw[fill opacity=0.8,fill=gray!20,draw=none](-2.35,5.597)--(-2.341,5.615)--(-2.364,5.614)--cycle;
\draw(-2.341,5.615)--(-2.364,5.614);
\filldraw[fill opacity=0.8,fill=gray!20,draw=none](-2.476,5.779)--(-2.444,5.825)--(-2.473,5.84)--(-2.508,5.791)--cycle;
\draw(-2.476,5.779)--(-2.444,5.825)--(-2.473,5.84)--(-2.508,5.791);
\filldraw[fill opacity=0.8,fill=gray!20](-2.454,5.87)--(-2.433,5.905)--(-2.465,5.913)--(-2.499,5.881)--cycle;
\filldraw[fill opacity=0.8,fill=gray!20,draw=none](-2.486,5.839)--(-2.464,5.845)--(-2.454,5.87)--(-2.499,5.881)--(-2.518,5.851)--cycle;
\draw(-2.464,5.845)--(-2.454,5.87)--(-2.499,5.881)--(-2.518,5.851);
\filldraw[fill opacity=0.8,fill=gray!20,draw=none](-2.723,7.728)--(-2.74,7.727)--(-2.793,7.748)--(-2.802,7.776)--(-2.705,7.781)--(-2.707,7.744)--cycle;
\draw(-2.723,7.728)--(-2.74,7.727);
\draw(-2.793,7.748)--(-2.802,7.776)--(-2.705,7.781)--(-2.707,7.744);
\filldraw[fill opacity=0.8,fill=gray!20,draw=none](-2.778,7.708)--(-2.786,7.725)--(-2.708,7.729)--(-2.708,7.716)--cycle;
\draw(-2.778,7.708)--(-2.786,7.725)--(-2.708,7.729)--(-2.708,7.716);
\filldraw[fill opacity=0.8,fill=gray!20,draw=none](-2.74,7.727)--(-2.786,7.725)--(-2.793,7.748)--cycle;
\draw(-2.74,7.727)--(-2.786,7.725)--(-2.793,7.748);
\filldraw[fill opacity=0.8,fill=gray!20,draw=none](-2.658,7.725)--(-2.708,7.729)--(-2.705,7.781)--(-2.665,7.778)--(-2.627,7.736)--(-2.628,7.731)--cycle;
\draw(-2.658,7.725)--(-2.708,7.729)--(-2.705,7.781)--(-2.665,7.778);
\draw(-2.627,7.736)--(-2.628,7.731);
\filldraw[fill opacity=0.8,fill=gray!20,draw=none](-2.723,7.728)--(-2.707,7.744)--(-2.708,7.729)--cycle;
\draw(-2.707,7.744)--(-2.708,7.729)--(-2.723,7.728);
\filldraw[fill opacity=0.8,fill=gray!20,draw=none](-2.701,7.72)--(-2.7,7.716)--(-2.708,7.71)--(-2.727,7.707)--cycle;
\draw(-2.701,7.72)--(-2.7,7.716);
\filldraw[fill opacity=0.8,fill=gray!20,draw=none](-2.708,7.719)--(-2.708,7.729)--(-2.658,7.725)--cycle;
\draw(-2.708,7.719)--(-2.708,7.729)--(-2.658,7.725);
\filldraw[fill opacity=0.8,fill=gray!20,draw=none](-2.665,7.778)--(-2.705,7.781)--(-2.704,7.837)--cycle;
\draw(-2.665,7.778)--(-2.705,7.781)--(-2.704,7.837);
\filldraw[fill opacity=0.8,fill=gray!20,draw=none](-2.676,7.889)--(-2.646,7.699)--(-2.694,7.682)--(-2.727,7.883)--cycle;
\draw(-2.694,7.682)--(-2.727,7.883)--(-2.676,7.889)--(-2.646,7.699);
\filldraw[fill opacity=0.8,fill=gray!20,draw=none](-2.7,7.716)--(-2.699,7.711)--(-2.708,7.71)--cycle;
\draw(-2.7,7.716)--(-2.699,7.711);
\filldraw[fill opacity=0.8,fill=gray!20,draw=none](-2.642,7.724)--(-2.634,7.721)--(-2.649,7.719)--(-2.65,7.726)--cycle;
\draw(-2.649,7.719)--(-2.65,7.726);
\filldraw[fill opacity=0.8,fill=gray!20,draw=none](-2.71,7.685)--(-2.708,7.719)--(-2.658,7.725)--(-2.642,7.724)--(-2.632,7.721)--(-2.656,7.681)--cycle;
\draw(-2.658,7.725)--(-2.642,7.724);
\draw(-2.632,7.721)--(-2.656,7.681)--(-2.71,7.685)--(-2.708,7.719);
\filldraw[fill opacity=0.8,fill=gray!20,draw=none](-2.727,7.883)--(-2.701,7.72)--(-2.727,7.707)--(-2.749,7.704)--(-2.776,7.872)--cycle;
\draw(-2.749,7.704)--(-2.776,7.872)--(-2.727,7.883)--(-2.701,7.72);
\filldraw[fill opacity=0.8,fill=gray!20,draw=none](-2.727,7.707)--(-2.748,7.697)--(-2.749,7.704)--cycle;
\draw(-2.748,7.697)--(-2.749,7.704);
\filldraw[fill opacity=0.8,fill=gray!20,draw=none](-2.754,7.699)--(-2.749,7.704)--(-2.748,7.7)--cycle;
\draw(-2.749,7.704)--(-2.748,7.7);
\filldraw[fill opacity=0.8,fill=gray!20,draw=none](-2.773,7.697)--(-2.75,7.712)--(-2.749,7.704)--(-2.754,7.699)--cycle;
\draw(-2.75,7.712)--(-2.749,7.704);
\filldraw[fill opacity=0.8,fill=gray!20,draw=none](-2.773,7.697)--(-2.754,7.699)--(-2.786,7.669)--(-2.789,7.688)--cycle;
\draw(-2.786,7.669)--(-2.789,7.688);
\filldraw[fill opacity=0.8,fill=gray!20,draw=none](-2.766,7.682)--(-2.778,7.708)--(-2.708,7.716)--(-2.71,7.685)--cycle;
\draw(-2.708,7.716)--(-2.71,7.685)--(-2.766,7.682)--(-2.778,7.708);
\filldraw[fill opacity=0.8,fill=gray!20,draw=none](-2.773,7.697)--(-2.789,7.688)--(-2.79,7.696)--cycle;
\draw(-2.789,7.688)--(-2.79,7.696);
\filldraw[fill opacity=0.8,fill=gray!20,draw=none](-2.808,7.674)--(-2.81,7.676)--(-2.778,7.707)--(-2.766,7.682)--cycle;
\draw(-2.778,7.707)--(-2.766,7.682)--(-2.808,7.674)--(-2.81,7.676);
\filldraw[fill opacity=0.8,fill=gray!20,draw=none](-2.776,7.872)--(-2.75,7.712)--(-2.773,7.697)--(-2.79,7.696)--(-2.816,7.857)--cycle;
\draw(-2.79,7.696)--(-2.816,7.857)--(-2.776,7.872)--(-2.75,7.712);
\filldraw[fill opacity=0.8,fill=gray!20](-2.764,7.646)--(-2.808,7.674)--(-2.766,7.682)--(-2.742,7.651)--cycle;
\filldraw[fill opacity=0.8,fill=gray!20](-2.717,7.632)--(-2.768,7.634)--(-2.774,7.64)--(-2.717,7.632)--cycle;
\filldraw[fill opacity=0.8,fill=gray!20](-2.717,7.632)--(-2.748,7.629)--(-2.768,7.634)--(-2.717,7.632)--cycle;
\filldraw[fill opacity=0.8,fill=gray!20,draw=none](-2.748,7.629)--(-2.774,7.639)--(-2.775,7.64)--(-2.806,7.647)--(-2.768,7.634)--cycle;
\draw(-2.806,7.647)--(-2.768,7.634)--(-2.748,7.629)--(-2.774,7.639);
\filldraw[fill opacity=0.8,fill=gray!20](-2.833,7.812)--(-2.499,5.733)--(-2.468,5.725)--(-2.802,7.805)--cycle;
\filldraw[fill opacity=0.8,fill=gray!20,draw=none](-2.534,5.815)--(-2.525,5.841)--(-2.532,5.849)--cycle;
\draw(-2.534,5.815)--(-2.525,5.841)--(-2.532,5.849);
\filldraw[fill opacity=0.8,fill=gray!20,draw=none](-7.935,.77)--(-7.955,.789)--(-7.962,.807)--(-7.962,.828)--(-7.858,.821)--(-7.87,.765)--cycle;
\draw(-7.962,.807)--(-7.962,.828)--(-7.858,.821)--(-7.87,.765)--(-7.935,.77);
\filldraw[fill opacity=0.8,fill=gray!20](-7.962,.828)--(-7.961,.885)--(-7.853,.877)--(-7.858,.821)--cycle;
\filldraw[fill opacity=0.8,fill=gray!20](-7.858,.821)--(-7.853,.877)--(-7.778,.859)--(-7.785,.803)--cycle;
\filldraw[fill opacity=0.8,fill=gray!20,draw=none](-7.779,.804)--(-7.781,.799)--(-7.785,.803)--(-7.778,.859)--(-7.771,.852)--cycle;
\draw(-7.781,.799)--(-7.785,.803)--(-7.778,.859)--(-7.771,.852);
\filldraw[fill opacity=0.8,fill=gray!20,draw=none](-7.868,.809)--(-7.752,.813)--(-7.559,.875)--(-7.602,.906)--(-7.865,.821)--cycle;
\draw(-7.752,.813)--(-7.559,.875);
\draw(-7.602,.906)--(-7.865,.821)--(-7.868,.809);
\filldraw[fill opacity=0.8,fill=gray!20,draw=none](-2.245,5.841)--(-2.261,5.887)--(-2.275,5.878)--(-2.251,5.837)--cycle;
\draw(-2.261,5.887)--(-2.275,5.878)--(-2.251,5.837)--(-2.245,5.841);
\filldraw[fill opacity=0.8,fill=gray!20](-2.433,5.905)--(-2.409,5.931)--(-2.425,5.935)--(-2.465,5.913)--cycle;
\filldraw[fill opacity=0.8,fill=gray!20,draw=none](-7.779,.902)--(-7.771,.852)--(-7.778,.859)--(-7.785,.914)--(-7.781,.91)--cycle;
\draw(-7.771,.852)--(-7.778,.859)--(-7.785,.914)--(-7.781,.91);
\filldraw[fill opacity=0.8,fill=gray!20,draw=none](-7.781,.91)--(-7.785,.914)--(-7.795,.941)--cycle;
\draw(-7.781,.91)--(-7.785,.914)--(-7.795,.941);
\filldraw[fill opacity=0.8,fill=gray!20,draw=none](-7.874,.872)--(-7.568,.971)--(-7.577,1.02)--(-7.897,.917)--cycle;
\draw(-7.577,1.02)--(-7.897,.917)--(-7.874,.872)--(-7.568,.971);
\filldraw[fill opacity=0.8,fill=gray!20,draw=none](-2.301,5.62)--(-2.267,5.656)--(-2.312,5.647)--(-2.325,5.616)--cycle;
\draw(-2.267,5.656)--(-2.312,5.647)--(-2.325,5.616)--(-2.301,5.62);
\filldraw[fill opacity=0.8,fill=gray!20](-2.307,5.911)--(-2.343,5.934)--(-2.362,5.93)--(-2.342,5.904)--cycle;
\filldraw[fill opacity=0.8,fill=gray!20,draw=none](-2.267,5.656)--(-2.251,5.659)--(-2.238,5.692)--cycle;
\draw(-2.267,5.656)--(-2.251,5.659)--(-2.238,5.692);
\filldraw[fill opacity=0.8,fill=gray!20,draw=none](-2.511,5.731)--(-2.491,5.758)--(-2.508,5.791)--(-2.545,5.739)--cycle;
\draw(-2.511,5.731)--(-2.491,5.758);
\draw(-2.508,5.791)--(-2.545,5.739);
\filldraw[fill opacity=0.8,fill=gray!20,draw=none](-2.481,5.701)--(-2.476,5.701)--(-2.473,5.705)--(-2.491,5.758)--(-2.511,5.731)--cycle;
\draw(-2.476,5.701)--(-2.473,5.705);
\draw(-2.491,5.758)--(-2.511,5.731);
\filldraw[fill opacity=0.8,fill=gray!20](-2.48,5.689)--(-2.484,5.734)--(-2.547,5.749)--(-2.541,5.704)--cycle;
\filldraw[fill opacity=0.8,fill=gray!20,draw=none](-8.018,1.512)--(-8.024,1.518)--(-8.023,1.569)--(-7.919,1.562)--(-7.931,1.506)--cycle;
\draw(-8.024,1.518)--(-8.023,1.569)--(-7.919,1.562)--(-7.931,1.506)--(-8.018,1.512);
\filldraw[fill opacity=0.8,fill=gray!20,draw=none](-7.939,1.486)--(-7.96,1.483)--(-7.955,1.485)--cycle;
\draw(-7.96,1.483)--(-7.955,1.485);
\filldraw[fill opacity=0.8,fill=gray!20,draw=none](-7.96,1.483)--(-8.005,1.498)--(-8.018,1.512)--(-7.931,1.506)--(-7.939,1.486)--cycle;
\draw(-8.018,1.512)--(-7.931,1.506)--(-7.939,1.486);
\filldraw[fill opacity=0.8,fill=gray!20,draw=none](-7.953,1.479)--(-7.724,1.578)--(-7.828,1.561)--(-7.927,1.517)--cycle;
\draw(-7.828,1.561)--(-7.927,1.517)--(-7.953,1.479)--(-7.724,1.578);
\filldraw[fill opacity=0.8,fill=gray!20,draw=none](-2.491,5.758)--(-2.476,5.779)--(-2.508,5.791)--cycle;
\draw(-2.491,5.758)--(-2.476,5.779);
\filldraw[fill opacity=0.8,fill=gray!20,draw=none](-2.532,5.702)--(-2.511,5.731)--(-2.545,5.739)--cycle;
\draw(-2.532,5.702)--(-2.511,5.731);
\filldraw[fill opacity=0.8,fill=gray!20,draw=none](-4.396,3.061)--(-4.394,3.06)--(-4.395,3.058)--cycle;
\draw(-4.394,3.06)--(-4.395,3.058);
\filldraw[fill opacity=0.8,fill=gray!20,draw=none](-4.416,3.074)--(-4.396,3.061)--(-4.395,3.058)--(-4.4,3.052)--cycle;
\draw(-4.395,3.058)--(-4.4,3.052);
\filldraw[fill opacity=0.8,fill=gray!20,draw=none](-4.384,3.042)--(-4.4,3.052)--(-4.416,3.074)--(-4.352,3.07)--(-4.341,3.047)--cycle;
\draw(-4.416,3.074)--(-4.352,3.07)--(-4.341,3.047);
\filldraw[fill opacity=0.8,fill=gray!20,draw=none](-4.436,3.024)--(-4.464,3.038)--(-4.412,3.024)--cycle;
\draw(-4.436,3.024)--(-4.464,3.038);
\filldraw[fill opacity=0.8,fill=gray!20,draw=none](-4.418,2.987)--(-4.425,2.998)--(-4.448,3.017)--(-4.47,3.017)--(-4.475,3.011)--(-4.438,2.89)--cycle;
\draw(-4.418,2.987)--(-4.425,2.998)--(-4.448,3.017)--(-4.47,3.017)--(-4.475,3.011);
\filldraw[fill opacity=0.8,fill=gray!20,draw=none](-4.412,3.024)--(-4.465,3.051)--(-4.441,3.029)--(-4.419,3.018)--cycle;
\draw(-4.412,3.024)--(-4.465,3.051);
\draw(-4.441,3.029)--(-4.419,3.018);
\filldraw[fill opacity=0.8,fill=gray!20,draw=none](-4.387,3.103)--(-4.427,3.104)--(-4.422,3.102)--(-4.381,3.096)--cycle;
\draw(-4.422,3.102)--(-4.381,3.096)--(-4.387,3.103)--(-4.427,3.104);
\filldraw[fill opacity=0.8,fill=gray!20,draw=none](-4.486,3.008)--(-4.487,3.007)--(-4.488,2.999)--cycle;
\draw(-4.486,3.008)--(-4.487,3.007)--(-4.488,2.999);
\filldraw[fill opacity=0.8,fill=gray!20,draw=none](-4.381,3.096)--(-4.422,3.102)--(-4.41,3.096)--(-4.39,3.09)--cycle;
\draw(-4.41,3.096)--(-4.39,3.09)--(-4.381,3.096)--(-4.422,3.102);
\filldraw[fill opacity=0.8,fill=gray!20,draw=none](-4.397,3.089)--(-4.39,3.09)--(-4.41,3.096)--cycle;
\draw(-4.397,3.089)--(-4.39,3.09)--(-4.41,3.096);
\filldraw[fill opacity=0.8,fill=gray!20,draw=none](-4.449,3.017)--(-2.532,5.702)--(-2.545,5.739)--(-4.502,3)--cycle;
\draw(-4.449,3.017)--(-2.532,5.702);
\draw(-2.545,5.739)--(-4.502,3);
\filldraw[fill opacity=0.8,fill=gray!20](-2.388,5.902)--(-2.386,5.929)--(-2.409,5.931)--(-2.433,5.905)--cycle;
\filldraw[fill opacity=0.8,fill=gray!20,draw=none](-2.392,5.635)--(-2.364,5.617)--(-2.294,5.715)--(-2.324,5.741)--(-2.396,5.64)--cycle;
\draw(-2.364,5.617)--(-2.294,5.715)--(-2.324,5.741)--(-2.396,5.64);
\filldraw[fill opacity=0.8,fill=gray!20](-2.342,5.904)--(-2.362,5.93)--(-2.386,5.929)--(-2.388,5.902)--cycle;
\filldraw[fill opacity=0.8,fill=gray!20](-2.391,5.865)--(-2.388,5.902)--(-2.433,5.905)--(-2.454,5.87)--cycle;
\filldraw[fill opacity=0.8,fill=gray!20,draw=none](-2.324,5.741)--(-2.363,5.771)--(-2.377,5.752)--cycle;
\draw(-2.324,5.741)--(-2.363,5.771)--(-2.377,5.752);
\filldraw[fill opacity=0.8,fill=gray!20,draw=none](-2.377,5.752)--(-2.373,5.73)--(-2.324,5.741)--(-2.331,5.785)--cycle;
\draw(-2.377,5.752)--(-2.373,5.73)--(-2.324,5.741)--(-2.331,5.785);
\filldraw[fill opacity=0.8,fill=gray!20,draw=none](-2.43,5.766)--(-2.423,5.725)--(-2.373,5.73)--(-2.377,5.752)--cycle;
\draw(-2.43,5.766)--(-2.423,5.725)--(-2.373,5.73)--(-2.377,5.752);
\filldraw[fill opacity=0.8,fill=gray!20,draw=none](-2.402,5.644)--(-2.396,5.64)--(-2.351,5.704)--(-2.377,5.752)--(-2.431,5.677)--cycle;
\draw(-2.396,5.64)--(-2.351,5.704);
\draw(-2.377,5.752)--(-2.431,5.677);
\filldraw[fill opacity=0.8,fill=gray!20](-2.391,5.613)--(-2.392,5.643)--(-2.47,5.649)--(-2.454,5.617)--cycle;
\filldraw[fill opacity=0.8,fill=gray!20,draw=none](-2.267,5.656)--(-2.238,5.692)--(-2.235,5.7)--(-2.295,5.688)--(-2.307,5.67)--(-2.312,5.647)--cycle;
\draw(-2.238,5.692)--(-2.235,5.7)--(-2.295,5.688);
\draw(-2.307,5.67)--(-2.312,5.647)--(-2.267,5.656);
\filldraw[fill opacity=0.8,fill=gray!20,draw=none](-2.476,5.779)--(-2.468,5.725)--(-2.423,5.725)--(-2.43,5.766)--cycle;
\draw(-2.476,5.779)--(-2.468,5.725)--(-2.423,5.725)--(-2.43,5.766);
\filldraw[fill opacity=0.8,fill=gray!20,draw=none](-2.473,5.705)--(-2.445,5.744)--(-2.476,5.779)--(-2.491,5.758)--cycle;
\draw(-2.473,5.705)--(-2.445,5.744);
\draw(-2.476,5.779)--(-2.491,5.758);
\filldraw[fill opacity=0.8,fill=gray!20,draw=none](-2.484,5.734)--(-2.482,5.757)--(-2.49,5.784)--(-2.541,5.796)--(-2.547,5.749)--cycle;
\draw(-2.49,5.784)--(-2.541,5.796)--(-2.547,5.749)--(-2.484,5.734)--(-2.482,5.757);
\filldraw[fill opacity=0.8,fill=gray!20,draw=none](-2.278,5.889)--(-2.285,5.889)--(-2.28,5.883)--cycle;
\draw(-2.285,5.889)--(-2.28,5.883);
\filldraw[fill opacity=0.8,fill=gray!20,draw=none](-2.303,5.873)--(-2.285,5.889)--(-2.307,5.911)--(-2.342,5.904)--(-2.326,5.87)--cycle;
\draw(-2.285,5.889)--(-2.307,5.911)--(-2.342,5.904)--(-2.326,5.87);
\filldraw[fill opacity=0.8,fill=gray!20,draw=none](-2.642,7.724)--(-2.65,7.726)--(-2.65,7.727)--cycle;
\draw(-2.65,7.726)--(-2.65,7.727);
\filldraw[fill opacity=0.8,fill=gray!20,draw=none](-2.658,7.725)--(-2.628,7.731)--(-2.631,7.723)--cycle;
\draw(-2.628,7.731)--(-2.631,7.723)--(-2.658,7.725);
\filldraw[fill opacity=0.8,fill=gray!20,draw=none](-2.627,7.722)--(-2.631,7.723)--(-2.627,7.736)--(-2.606,7.728)--cycle;
\draw(-2.627,7.722)--(-2.631,7.723)--(-2.627,7.736);
\filldraw[fill opacity=0.8,fill=gray!20,draw=none](-2.642,7.724)--(-2.631,7.723)--(-2.632,7.721)--cycle;
\draw(-2.642,7.724)--(-2.631,7.723)--(-2.632,7.721);
\filldraw[fill opacity=0.8,fill=gray!20,draw=none](-2.632,7.721)--(-2.631,7.723)--(-2.627,7.722)--cycle;
\draw(-2.632,7.721)--(-2.631,7.723)--(-2.627,7.722);
\filldraw[fill opacity=0.8,fill=gray!20,draw=none](-2.632,7.888)--(-2.604,7.71)--(-2.65,7.727)--(-2.676,7.889)--cycle;
\draw(-2.65,7.727)--(-2.676,7.889)--(-2.632,7.888)--(-2.604,7.71);
\filldraw[fill opacity=0.8,fill=gray!20,draw=none](-2.656,7.681)--(-2.632,7.721)--(-2.627,7.722)--(-2.578,7.71)--(-2.619,7.672)--cycle;
\draw(-2.627,7.722)--(-2.578,7.71)--(-2.619,7.672)--(-2.656,7.681)--(-2.632,7.721);
\filldraw[fill opacity=0.8,fill=gray!20,draw=none](-2.627,7.722)--(-2.606,7.728)--(-2.574,7.716)--(-2.578,7.71)--cycle;
\draw(-2.574,7.716)--(-2.578,7.71)--(-2.627,7.722);
\filldraw[fill opacity=0.8,fill=gray!20,draw=none](-2.618,7.836)--(-2.305,5.888)--(-2.278,5.889)--(-2.593,7.853)--cycle;
\draw(-2.278,5.889)--(-2.593,7.853)--(-2.618,7.836)--(-2.305,5.888);
\filldraw[fill opacity=0.8,fill=gray!20,draw=none](-2.341,5.615)--(-2.325,5.616)--(-2.312,5.647)--(-2.322,5.646)--cycle;
\draw(-2.341,5.615)--(-2.325,5.616)--(-2.312,5.647)--(-2.322,5.646);
\filldraw[fill opacity=0.8,fill=gray!20](-2.235,5.7)--(-2.23,5.745)--(-2.301,5.731)--(-2.304,5.686)--cycle;
\filldraw[fill opacity=0.8,fill=gray!20,draw=none](-2.329,5.868)--(-2.326,5.87)--(-2.342,5.904)--(-2.388,5.902)--(-2.391,5.865)--cycle;
\draw(-2.326,5.87)--(-2.342,5.904)--(-2.388,5.902)--(-2.391,5.865)--(-2.329,5.868);
\filldraw[fill opacity=0.8,fill=gray!20,draw=none](-2.342,5.856)--(-2.329,5.868)--(-2.391,5.865)--(-2.391,5.848)--cycle;
\draw(-2.329,5.868)--(-2.391,5.865)--(-2.391,5.848);
\filldraw[fill opacity=0.8,fill=gray!20](-2.685,7.65)--(-2.656,7.681)--(-2.619,7.672)--(-2.666,7.645)--cycle;
\filldraw[fill opacity=0.8,fill=gray!20](-2.717,7.632)--(-2.666,7.645)--(-2.66,7.639)--(-2.717,7.632)--cycle;
\filldraw[fill opacity=0.8,fill=gray!20](-2.717,7.632)--(-2.66,7.639)--(-2.669,7.633)--(-2.717,7.632)--cycle;
\filldraw[fill opacity=0.8,fill=gray!20](-2.717,7.632)--(-2.669,7.633)--(-2.691,7.628)--(-2.717,7.632)--cycle;
\filldraw[fill opacity=0.8,fill=gray!20](-2.669,7.633)--(-2.625,7.648)--(-2.668,7.639)--(-2.691,7.628)--cycle;
\filldraw[fill opacity=0.8,fill=gray!20,draw=none](-2.625,7.648)--(-2.598,7.668)--(-2.632,7.66)--(-2.661,7.648)--(-2.668,7.639)--cycle;
\draw(-2.661,7.648)--(-2.668,7.639)--(-2.625,7.648)--(-2.598,7.668);
\filldraw[fill opacity=0.8,fill=gray!20](-2.658,7.821)--(-2.324,5.741)--(-2.284,5.757)--(-2.618,7.836)--cycle;
\filldraw[fill opacity=0.8,fill=gray!20,draw=none](-2.49,5.784)--(-2.511,5.838)--(-2.525,5.841)--(-2.541,5.796)--cycle;
\draw(-2.511,5.838)--(-2.525,5.841)--(-2.541,5.796)--(-2.49,5.784);
\filldraw[fill opacity=0.8,fill=gray!20,draw=none](-2.364,5.614)--(-2.341,5.615)--(-2.322,5.646)--(-2.367,5.644)--cycle;
\draw(-2.364,5.614)--(-2.341,5.615);
\draw(-2.322,5.646)--(-2.367,5.644);
\filldraw[fill opacity=0.8,fill=gray!20,draw=none](-2.445,5.744)--(-2.405,5.801)--(-2.444,5.825)--(-2.476,5.779)--cycle;
\draw(-2.445,5.744)--(-2.405,5.801)--(-2.444,5.825)--(-2.476,5.779);
\filldraw[fill opacity=0.8,fill=gray!20,draw=none](-2.464,5.845)--(-2.486,5.839)--(-2.476,5.779)--(-2.43,5.766)--(-2.442,5.838)--cycle;
\draw(-2.486,5.839)--(-2.476,5.779);
\draw(-2.43,5.766)--(-2.442,5.838);
\filldraw[fill opacity=0.8,fill=gray!20,draw=none](-2.511,5.838)--(-2.47,5.828)--(-2.468,5.832)--(-2.518,5.851)--cycle;
\draw(-2.511,5.838)--(-2.47,5.828)--(-2.468,5.832);
\filldraw[fill opacity=0.8,fill=gray!20,draw=none](-2.511,5.838)--(-2.518,5.851)--(-2.525,5.841)--cycle;
\draw(-2.518,5.851)--(-2.525,5.841)--(-2.511,5.838);
\filldraw[fill opacity=0.8,fill=gray!20,draw=none](-7.718,.869)--(-7.564,.919)--(-7.568,.971)--(-7.715,.923)--cycle;
\draw(-7.718,.869)--(-7.564,.919);
\draw(-7.568,.971)--(-7.715,.923);
\filldraw[fill opacity=0.8,fill=gray!20,draw=none](-4.063,2.419)--(-4.038,2.386)--(-3.968,2.349)--(-4.049,2.417)--cycle;
\draw(-3.968,2.349)--(-4.049,2.417);
\filldraw[fill opacity=0.8,fill=gray!20,draw=none](-2.285,5.889)--(-2.305,5.888)--(-2.303,5.872)--cycle;
\draw(-2.305,5.888)--(-2.303,5.872);
\filldraw[fill opacity=0.8,fill=gray!20,draw=none](-2.303,5.873)--(-2.275,5.878)--(-2.285,5.889)--cycle;
\draw(-2.303,5.873)--(-2.275,5.878)--(-2.285,5.889);
\filldraw[fill opacity=0.8,fill=gray!20,draw=none](-7.922,.755)--(-7.935,.77)--(-7.87,.765)--(-7.879,.741)--cycle;
\draw(-7.935,.77)--(-7.87,.765)--(-7.879,.741);
\filldraw[fill opacity=0.8,fill=gray!20,draw=none](-7.795,.773)--(-7.785,.803)--(-7.781,.799)--cycle;
\draw(-7.795,.773)--(-7.785,.803)--(-7.781,.799);
\filldraw[fill opacity=0.8,fill=gray!20,draw=none](-7.897,.735)--(-7.628,.822)--(-7.752,.813)--(-7.874,.773)--cycle;
\draw(-7.752,.813)--(-7.874,.773)--(-7.897,.735)--(-7.628,.822);
\filldraw[fill opacity=0.8,fill=gray!20,draw=none](-2.364,5.617)--(-2.367,5.644)--(-2.392,5.643)--(-2.392,5.635)--cycle;
\draw(-2.367,5.644)--(-2.392,5.643)--(-2.392,5.635);
\filldraw[fill opacity=0.8,fill=gray!20,draw=none](-2.392,5.635)--(-2.372,5.606)--(-2.364,5.617)--cycle;
\draw(-2.372,5.606)--(-2.364,5.617);
\filldraw[fill opacity=0.8,fill=gray!20,draw=none](-5.555,2.354)--(-5.835,2.255)--(-5.563,2.355)--cycle;
\draw(-5.555,2.354)--(-5.835,2.255);
\filldraw[fill opacity=0.8,fill=gray!20,draw=none](-5.464,2.403)--(-6.081,2.184)--(-5.835,2.255)--(-5.555,2.354)--cycle;
\draw(-5.464,2.403)--(-6.081,2.184);
\draw(-5.835,2.255)--(-5.555,2.354);
\filldraw[fill opacity=0.8,fill=gray!20,draw=none](-2.273,5.874)--(-2.275,5.878)--(-2.303,5.872)--cycle;
\draw(-2.273,5.874)--(-2.275,5.878)--(-2.303,5.872);
\filldraw[fill opacity=0.8,fill=gray!20,draw=none](-4.1,3.187)--(-2.372,5.606)--(-2.396,5.64)--(-4.108,3.244)--cycle;
\draw(-4.1,3.187)--(-2.372,5.606);
\draw(-2.396,5.64)--(-4.108,3.244);
\filldraw[fill opacity=0.8,fill=gray!20](-2.23,5.745)--(-2.235,5.792)--(-2.304,5.779)--(-2.301,5.731)--cycle;
\filldraw[fill opacity=0.8,fill=gray!20,draw=none](-2.251,5.837)--(-2.273,5.874)--(-2.303,5.872)--(-2.325,5.868)--(-2.312,5.825)--cycle;
\draw(-2.303,5.872)--(-2.325,5.868)--(-2.312,5.825)--(-2.251,5.837)--(-2.273,5.874);
\filldraw[fill opacity=0.8,fill=gray!20,draw=none](-2.49,5.784)--(-2.48,5.781)--(-2.47,5.828)--(-2.511,5.838)--cycle;
\draw(-2.49,5.784)--(-2.48,5.781)--(-2.47,5.828)--(-2.511,5.838);
\filldraw[fill opacity=0.8,fill=gray!20](-2.235,5.792)--(-2.251,5.837)--(-2.312,5.825)--(-2.304,5.779)--cycle;
\filldraw[fill opacity=0.8,fill=gray!20,draw=none](-4.049,2.417)--(-3.968,2.349)--(-3.928,2.36)--(-4.044,2.457)--cycle;
\draw(-4.049,2.417)--(-3.968,2.349);
\draw(-3.928,2.36)--(-4.044,2.457);
\filldraw[fill opacity=0.8,fill=gray!20,draw=none](-2.455,5.687)--(-2.431,5.677)--(-2.363,5.771)--(-2.405,5.801)--(-2.476,5.701)--cycle;
\draw(-2.431,5.677)--(-2.363,5.771)--(-2.405,5.801)--(-2.476,5.701);
\filldraw[fill opacity=0.8,fill=gray!20](-2.392,5.643)--(-2.394,5.682)--(-2.48,5.689)--(-2.47,5.649)--cycle;
\filldraw[fill opacity=0.8,fill=gray!20,draw=none](-2.351,5.704)--(-2.324,5.741)--(-2.377,5.752)--cycle;
\draw(-2.351,5.704)--(-2.324,5.741);
\filldraw[fill opacity=0.8,fill=gray!20,draw=none](-2.455,5.687)--(-2.476,5.701)--(-2.481,5.701)--(-2.48,5.689)--cycle;
\draw(-2.481,5.701)--(-2.48,5.689)--(-2.455,5.687);
\filldraw[fill opacity=0.8,fill=gray!20](-4.352,3.07)--(-4.377,3.096)--(-4.34,3.086)--(-4.299,3.057)--cycle;
\filldraw[fill opacity=0.8,fill=gray!20](-4.34,3.086)--(-4.387,3.103)--(-4.381,3.096)--(-4.328,3.074)--cycle;
\filldraw[fill opacity=0.8,fill=gray!20,draw=none](-4.401,2.954)--(-4.403,2.962)--(-4.418,2.987)--(-4.438,2.89)--cycle;
\draw(-4.401,2.954)--(-4.403,2.962)--(-4.418,2.987);
\filldraw[fill opacity=0.8,fill=gray!20,draw=none](-4.438,2.89)--(-4.488,2.999)--(-4.469,3)--(-4.431,2.993)--(-4.418,2.987)--cycle;
\draw(-4.469,3)--(-4.431,2.993);
\filldraw[fill opacity=0.8,fill=gray!20,draw=none](-4.438,2.89)--(-4.488,2.985)--(-4.451,2.979)--(-4.407,2.959)--(-4.401,2.954)--cycle;
\draw(-4.488,2.985)--(-4.451,2.979)--(-4.407,2.959);
\filldraw[fill opacity=0.8,fill=gray!20,draw=none](-4.412,3.024)--(-4.419,3.018)--(-4.391,3.004)--cycle;
\draw(-4.419,3.018)--(-4.391,3.004);
\filldraw[fill opacity=0.8,fill=gray!20,draw=none](-4.341,3.047)--(-4.352,3.07)--(-4.299,3.057)--(-4.293,3.049)--cycle;
\draw(-4.341,3.047)--(-4.352,3.07)--(-4.299,3.057)--(-4.293,3.049);
\filldraw[fill opacity=0.8,fill=gray!20](-4.328,3.074)--(-4.381,3.096)--(-4.39,3.09)--(-4.346,3.062)--cycle;
\filldraw[fill opacity=0.8,fill=gray!20,draw=none](-4.391,3.004)--(-4.441,3.029)--(-4.416,2.986)--(-4.406,2.981)--cycle;
\draw(-4.391,3.004)--(-4.441,3.029);
\draw(-4.416,2.986)--(-4.406,2.981);
\filldraw[fill opacity=0.8,fill=gray!20,draw=none](-4.488,2.999)--(-4.474,3.001)--(-4.469,3)--cycle;
\draw(-4.488,2.999)--(-4.474,3.001)--(-4.469,3);
\filldraw[fill opacity=0.8,fill=gray!20,draw=none](-4.488,2.985)--(-4.487,2.975)--(-4.457,2.968)--(-4.451,2.979)--cycle;
\draw(-4.457,2.968)--(-4.451,2.979)--(-4.488,2.985);
\filldraw[fill opacity=0.8,fill=gray!20,draw=none](-4.474,3.001)--(-4.488,2.986)--(-4.478,2.976)--(-4.469,2.974)--(-4.432,2.989)--(-4.429,2.993)--cycle;
\draw(-4.432,2.989)--(-4.429,2.993)--(-4.474,3.001)--(-4.488,2.986);
\filldraw[fill opacity=0.8,fill=gray!20,draw=none](-4.36,3.064)--(-4.353,3.061)--(-4.346,3.062)--(-4.39,3.09)--(-4.397,3.089)--cycle;
\draw(-4.353,3.061)--(-4.346,3.062)--(-4.39,3.09)--(-4.397,3.089);
\filldraw[fill opacity=0.8,fill=gray!20,draw=none](-4.5,2.978)--(-4.489,2.962)--(-4.461,2.964)--(-4.457,2.968)--cycle;
\draw(-4.461,2.964)--(-4.457,2.968);
\filldraw[fill opacity=0.8,fill=gray!20,draw=none](-4.431,2.993)--(-4.429,2.993)--(-4.412,2.983)--cycle;
\draw(-4.431,2.993)--(-4.429,2.993)--(-4.412,2.983);
\filldraw[fill opacity=0.8,fill=gray!20,draw=none](-4.418,2.987)--(-4.429,2.993)--(-4.442,2.977)--(-4.411,2.962)--(-4.403,2.962)--cycle;
\draw(-4.418,2.987)--(-4.429,2.993)--(-4.442,2.977);
\filldraw[fill opacity=0.8,fill=gray!20,draw=none](-4.451,2.979)--(-4.461,2.964)--(-4.445,2.944)--(-4.417,2.939)--(-4.405,2.958)--cycle;
\draw(-4.417,2.939)--(-4.405,2.958)--(-4.451,2.979)--(-4.461,2.964);
\filldraw[fill opacity=0.8,fill=gray!20,draw=none](-4.469,2.974)--(-4.447,2.971)--(-4.432,2.989)--cycle;
\draw(-4.447,2.971)--(-4.432,2.989);
\filldraw[fill opacity=0.8,fill=gray!20,draw=none](-4.442,2.977)--(-4.447,2.971)--(-4.438,2.961)--(-4.411,2.962)--cycle;
\draw(-4.442,2.977)--(-4.447,2.971);
\filldraw[fill opacity=0.8,fill=gray!20,draw=none](-4.495,2.946)--(-4.426,2.971)--(-2.478,5.698)--(-2.511,5.731)--(-4.498,2.948)--cycle;
\draw(-4.426,2.971)--(-2.478,5.698);
\draw(-2.511,5.731)--(-4.498,2.948)--(-4.495,2.946);
\filldraw[fill opacity=0.8,fill=gray!20,draw=none](-2.322,5.646)--(-2.312,5.647)--(-2.307,5.67)--cycle;
\draw(-2.322,5.646)--(-2.312,5.647)--(-2.307,5.67);
\filldraw[fill opacity=0.8,fill=gray!20,draw=none](-2.367,5.644)--(-2.322,5.646)--(-2.307,5.67)--(-2.304,5.686)--(-2.358,5.684)--cycle;
\draw(-2.367,5.644)--(-2.322,5.646);
\draw(-2.307,5.67)--(-2.304,5.686)--(-2.358,5.684);
\filldraw[fill opacity=0.8,fill=gray!20,draw=none](-2.303,5.873)--(-2.326,5.87)--(-2.325,5.868)--cycle;
\draw(-2.326,5.87)--(-2.325,5.868)--(-2.303,5.873);
\filldraw[fill opacity=0.8,fill=gray!20,draw=none](-2.486,5.839)--(-2.468,5.832)--(-2.464,5.845)--cycle;
\draw(-2.468,5.832)--(-2.464,5.845);
\filldraw[fill opacity=0.8,fill=gray!20,draw=none](-2.442,5.838)--(-2.391,5.848)--(-2.391,5.865)--(-2.454,5.87)--(-2.464,5.845)--cycle;
\draw(-2.391,5.848)--(-2.391,5.865)--(-2.454,5.87)--(-2.464,5.845);
\filldraw[fill opacity=0.8,fill=gray!20](-2.742,7.651)--(-2.766,7.682)--(-2.71,7.685)--(-2.713,7.652)--cycle;
\filldraw[fill opacity=0.8,fill=gray!20](-2.717,7.632)--(-2.742,7.651)--(-2.713,7.652)--(-2.717,7.632)--cycle;
\filldraw[fill opacity=0.8,fill=gray!20](-2.717,7.632)--(-2.764,7.646)--(-2.742,7.651)--(-2.717,7.632)--cycle;
\filldraw[fill opacity=0.8,fill=gray!20](-2.717,7.632)--(-2.774,7.64)--(-2.764,7.646)--(-2.717,7.632)--cycle;
\filldraw[fill opacity=0.8,fill=gray!20](-2.717,7.632)--(-2.72,7.627)--(-2.748,7.629)--(-2.717,7.632)--cycle;
\filldraw[fill opacity=0.8,fill=gray!20,draw=none](-2.72,7.627)--(-2.723,7.637)--(-2.747,7.639)--(-2.774,7.639)--(-2.748,7.629)--cycle;
\draw(-2.774,7.639)--(-2.748,7.629)--(-2.72,7.627)--(-2.723,7.637)--(-2.747,7.639);
\filldraw[fill opacity=0.8,fill=gray!20,draw=none](-2.731,7.64)--(-2.747,7.639)--(-2.723,7.637)--(-2.724,7.638)--cycle;
\draw(-2.747,7.639)--(-2.723,7.637)--(-2.724,7.638);
\filldraw[fill opacity=0.8,fill=gray!20,draw=none](-2.802,7.805)--(-2.486,5.839)--(-2.444,5.851)--(-2.757,7.804)--cycle;
\draw(-2.444,5.851)--(-2.757,7.804)--(-2.802,7.805)--(-2.486,5.839);
\filldraw[fill opacity=0.8,fill=gray!20,draw=none](-2.393,5.854)--(-2.377,5.752)--(-2.331,5.785)--(-2.343,5.86)--cycle;
\draw(-2.393,5.854)--(-2.377,5.752);
\draw(-2.331,5.785)--(-2.343,5.86);
\filldraw[fill opacity=0.8,fill=gray!20,draw=none](-2.397,5.639)--(-2.396,5.64)--(-2.402,5.644)--cycle;
\draw(-2.397,5.639)--(-2.396,5.64);
\filldraw[fill opacity=0.8,fill=gray!20,draw=none](-2.367,5.644)--(-2.358,5.684)--(-2.394,5.682)--(-2.392,5.643)--cycle;
\draw(-2.358,5.684)--(-2.394,5.682)--(-2.392,5.643)--(-2.367,5.644);
\filldraw[fill opacity=0.8,fill=gray!20,draw=none](-4.275,3.028)--(-4.293,3.049)--(-4.29,3.047)--cycle;
\filldraw[fill opacity=0.8,fill=gray!20,draw=none](-4.293,3.049)--(-4.299,3.057)--(-4.288,3.045)--cycle;
\draw(-4.293,3.049)--(-4.299,3.057)--(-4.288,3.045);
\filldraw[fill opacity=0.8,fill=gray!20,draw=none](-4.288,3.045)--(-4.299,3.057)--(-4.34,3.086)--(-4.328,3.074)--(-4.287,3.043)--cycle;
\draw(-4.288,3.045)--(-4.299,3.057)--(-4.34,3.086)--(-4.328,3.074)--(-4.287,3.043);
\filldraw[fill opacity=0.8,fill=gray!20,draw=none](-4.275,3.028)--(-4.29,3.047)--(-4.288,3.045)--(-4.283,3.04)--(-4.262,3.012)--cycle;
\draw(-4.288,3.045)--(-4.283,3.04)--(-4.262,3.012);
\filldraw[fill opacity=0.8,fill=gray!20,draw=none](-4.379,2.864)--(-4.387,2.915)--(-4.401,2.954)--(-4.438,2.89)--cycle;
\draw(-4.379,2.864)--(-4.387,2.915)--(-4.401,2.954);
\filldraw[fill opacity=0.8,fill=gray!20,draw=none](-4.249,2.996)--(-4.275,3.028)--(-4.262,3.012)--(-4.248,2.995)--cycle;
\draw(-4.262,3.012)--(-4.248,2.995)--(-4.249,2.996);
\filldraw[fill opacity=0.8,fill=gray!20,draw=none](-4.288,3.045)--(-4.287,3.043)--(-4.283,3.04)--cycle;
\draw(-4.287,3.043)--(-4.283,3.04)--(-4.288,3.045);
\filldraw[fill opacity=0.8,fill=gray!20,draw=none](-4.282,3)--(-4.27,2.992)--(-4.262,3.012)--(-4.283,3.04)--(-4.304,3.026)--cycle;
\draw(-4.262,3.012)--(-4.283,3.04)--(-4.304,3.026);
\filldraw[fill opacity=0.8,fill=gray!20,draw=none](-4.318,3.034)--(-4.304,3.026)--(-4.283,3.04)--(-4.328,3.074)--(-4.346,3.062)--(-4.327,3.043)--cycle;
\draw(-4.304,3.026)--(-4.283,3.04)--(-4.328,3.074)--(-4.346,3.062)--(-4.327,3.043);
\filldraw[fill opacity=0.8,fill=gray!20,draw=none](-4.404,2.919)--(-4.378,2.917)--(-4.395,2.95)--(-4.405,2.958)--(-4.417,2.939)--cycle;
\draw(-4.395,2.95)--(-4.405,2.958)--(-4.417,2.939);
\filldraw[fill opacity=0.8,fill=gray!20,draw=none](-4.407,2.959)--(-4.405,2.958)--(-4.395,2.95)--cycle;
\draw(-4.407,2.959)--(-4.405,2.958)--(-4.395,2.95);
\filldraw[fill opacity=0.8,fill=gray!20,draw=none](-4.398,2.936)--(-4.392,2.933)--(-4.375,2.938)--(-4.39,2.962)--(-4.397,2.953)--(-4.403,2.944)--cycle;
\draw(-4.39,2.962)--(-4.397,2.953);
\filldraw[fill opacity=0.8,fill=gray!20,draw=none](-4.375,2.938)--(-4.395,2.932)--(-4.394,2.928)--(-4.374,2.918)--cycle;
\draw(-4.394,2.928)--(-4.374,2.918);
\filldraw[fill opacity=0.8,fill=gray!20,draw=none](-4.371,2.916)--(-4.372,2.915)--(-4.37,2.912)--cycle;
\draw(-4.371,2.916)--(-4.372,2.915);
\filldraw[fill opacity=0.8,fill=gray!20,draw=none](-4.392,2.933)--(-4.398,2.936)--(-4.395,2.932)--cycle;
\filldraw[fill opacity=0.8,fill=gray!20,draw=none](-4.375,2.938)--(-4.378,2.967)--(-4.399,2.942)--(-4.395,2.932)--cycle;
\filldraw[fill opacity=0.8,fill=gray!20,draw=none](-4.35,2.885)--(-4.355,2.901)--(-4.36,2.895)--(-4.359,2.892)--cycle;
\filldraw[fill opacity=0.8,fill=gray!20,draw=none](-4.364,2.906)--(-4.36,2.895)--(-4.355,2.901)--(-4.361,2.918)--(-4.362,2.915)--cycle;
\draw(-4.361,2.918)--(-4.362,2.915);
\filldraw[fill opacity=0.8,fill=gray!20,draw=none](-4.378,2.917)--(-4.378,2.916)--(-4.371,2.916)--cycle;
\filldraw[fill opacity=0.8,fill=gray!20,draw=none](-4.374,2.916)--(-4.378,2.916)--(-4.39,2.909)--(-4.379,2.864)--cycle;
\filldraw[fill opacity=0.8,fill=gray!20,draw=none](-4.357,2.881)--(-4.365,2.901)--(-4.377,2.908)--(-4.365,2.884)--cycle;
\filldraw[fill opacity=0.8,fill=gray!20,draw=none](-4.392,2.933)--(-4.365,2.92)--(-4.375,2.938)--cycle;
\filldraw[fill opacity=0.8,fill=gray!20,draw=none](-4.365,2.92)--(-4.362,2.915)--(-4.361,2.918)--cycle;
\draw(-4.362,2.915)--(-4.361,2.918);
\filldraw[fill opacity=0.8,fill=gray!20,draw=none](-4.378,2.916)--(-4.375,2.911)--(-4.371,2.916)--cycle;
\draw(-4.375,2.911)--(-4.371,2.916);
\filldraw[fill opacity=0.8,fill=gray!20,draw=none](-4.374,2.916)--(-4.374,2.918)--(-4.378,2.916)--cycle;
\filldraw[fill opacity=0.8,fill=gray!20,draw=none](-4.39,2.913)--(-4.39,2.909)--(-4.378,2.916)--cycle;
\filldraw[fill opacity=0.8,fill=gray!20,draw=none](-4.378,2.916)--(-4.378,2.917)--(-4.404,2.919)--(-4.402,2.916)--(-4.39,2.913)--cycle;
\filldraw[fill opacity=0.8,fill=gray!20,draw=none](-4.39,2.913)--(-4.378,2.916)--(-4.374,2.918)--(-4.394,2.928)--cycle;
\draw(-4.374,2.918)--(-4.394,2.928);
\filldraw[fill opacity=0.8,fill=gray!20,draw=none](-4.365,2.92)--(-4.373,2.924)--(-4.393,2.929)--(-4.366,2.911)--(-4.362,2.915)--cycle;
\draw(-4.366,2.911)--(-4.362,2.915);
\filldraw[fill opacity=0.8,fill=gray!20,draw=none](-4.37,2.912)--(-4.372,2.915)--(-4.377,2.908)--(-4.365,2.901)--cycle;
\draw(-4.372,2.915)--(-4.377,2.908);
\filldraw[fill opacity=0.8,fill=gray!20,draw=none](-4.364,2.906)--(-4.362,2.915)--(-4.366,2.911)--cycle;
\draw(-4.362,2.915)--(-4.366,2.911);
\filldraw[fill opacity=0.8,fill=gray!20,draw=none](-4.373,2.924)--(-4.392,2.933)--(-4.395,2.932)--(-4.393,2.929)--cycle;
\filldraw[fill opacity=0.8,fill=gray!20,draw=none](-4.378,2.916)--(-4.39,2.913)--(-4.376,2.908)--(-4.375,2.911)--cycle;
\draw(-4.376,2.908)--(-4.375,2.911);
\filldraw[fill opacity=0.8,fill=gray!20,draw=none](-4.379,2.864)--(-2.397,5.639)--(-2.402,5.644)--(-2.437,5.667)--(-4.418,2.894)--cycle;
\draw(-2.437,5.667)--(-4.418,2.894)--(-4.379,2.864)--(-2.397,5.639);
\filldraw[fill opacity=0.8,fill=gray!20](-2.713,7.652)--(-2.71,7.685)--(-2.656,7.681)--(-2.685,7.65)--cycle;
\filldraw[fill opacity=0.8,fill=gray!20](-2.717,7.632)--(-2.713,7.652)--(-2.685,7.65)--(-2.717,7.632)--cycle;
\filldraw[fill opacity=0.8,fill=gray!20](-2.717,7.632)--(-2.685,7.65)--(-2.666,7.645)--(-2.717,7.632)--cycle;
\filldraw[fill opacity=0.8,fill=gray!20](-2.717,7.632)--(-2.691,7.628)--(-2.72,7.627)--(-2.717,7.632)--cycle;
\filldraw[fill opacity=0.8,fill=gray!20](-2.691,7.628)--(-2.668,7.639)--(-2.723,7.637)--(-2.72,7.627)--cycle;
\filldraw[fill opacity=0.8,fill=gray!20,draw=none](-2.668,7.639)--(-2.661,7.648)--(-2.681,7.648)--(-2.724,7.638)--(-2.723,7.637)--cycle;
\draw(-2.724,7.638)--(-2.723,7.637)--(-2.668,7.639)--(-2.661,7.648);
\filldraw[fill opacity=0.8,fill=gray!20,draw=none](-2.757,7.804)--(-2.43,5.766)--(-2.381,5.782)--(-2.707,7.809)--cycle;
\draw(-2.381,5.782)--(-2.707,7.809)--(-2.757,7.804)--(-2.43,5.766);
\filldraw[fill opacity=0.8,fill=gray!20,draw=none](-2.43,5.766)--(-2.377,5.752)--(-2.381,5.782)--cycle;
\draw(-2.377,5.752)--(-2.381,5.782);
\filldraw[fill opacity=0.8,fill=gray!20,draw=none](-2.402,5.644)--(-2.431,5.677)--(-2.437,5.667)--cycle;
\draw(-2.431,5.677)--(-2.437,5.667);
\filldraw[fill opacity=0.8,fill=gray!20,draw=none](-2.295,5.688)--(-2.304,5.686)--(-2.307,5.67)--cycle;
\draw(-2.295,5.688)--(-2.304,5.686)--(-2.307,5.67);
\filldraw[fill opacity=0.8,fill=gray!20,draw=none](-4.044,2.457)--(-3.928,2.36)--(-3.925,2.418)--(-4.064,2.535)--cycle;
\draw(-4.044,2.457)--(-3.928,2.36);
\draw(-3.925,2.418)--(-4.064,2.535);
\filldraw[fill opacity=0.8,fill=gray!20,draw=none](-2.394,5.702)--(-2.394,5.727)--(-2.484,5.734)--(-2.481,5.701)--cycle;
\draw(-2.394,5.702)--(-2.394,5.727)--(-2.484,5.734)--(-2.481,5.701);
\filldraw[fill opacity=0.8,fill=gray!20,draw=none](-2.481,5.701)--(-2.478,5.698)--(-2.476,5.701)--cycle;
\draw(-2.478,5.698)--(-2.476,5.701);
\filldraw[fill opacity=0.8,fill=gray!20,draw=none](-2.478,5.697)--(-2.455,5.687)--(-2.476,5.701)--(-2.478,5.698)--cycle;
\draw(-2.476,5.701)--(-2.478,5.698);
\filldraw[fill opacity=0.8,fill=gray!20,draw=none](-2.455,5.687)--(-2.394,5.682)--(-2.394,5.702)--(-2.476,5.701)--cycle;
\draw(-2.455,5.687)--(-2.394,5.682)--(-2.394,5.702);
\filldraw[fill opacity=0.8,fill=gray!20,draw=none](-2.329,5.868)--(-2.325,5.868)--(-2.326,5.87)--cycle;
\draw(-2.329,5.868)--(-2.325,5.868)--(-2.326,5.87);
\filldraw[fill opacity=0.8,fill=gray!20,draw=none](-2.478,5.697)--(-2.478,5.698)--(-2.479,5.697)--cycle;
\draw(-2.478,5.698)--(-2.479,5.697);
\filldraw[fill opacity=0.8,fill=gray!20,draw=none](-4.391,3.004)--(-4.406,2.981)--(-4.378,2.967)--cycle;
\draw(-4.406,2.981)--(-4.378,2.967);
\filldraw[fill opacity=0.8,fill=gray!20,draw=none](-4.438,2.961)--(-4.435,2.958)--(-4.404,2.945)--(-4.39,2.962)--cycle;
\draw(-4.404,2.945)--(-4.39,2.962);
\filldraw[fill opacity=0.8,fill=gray!20,draw=none](-4.403,2.944)--(-4.397,2.953)--(-4.404,2.945)--cycle;
\draw(-4.397,2.953)--(-4.404,2.945);
\filldraw[fill opacity=0.8,fill=gray!20,draw=none](-4.399,2.942)--(-4.378,2.967)--(-4.416,2.986)--cycle;
\draw(-4.378,2.967)--(-4.416,2.986);
\filldraw[fill opacity=0.8,fill=gray!20,draw=none](-4.327,3.043)--(-4.346,3.062)--(-4.353,3.061)--cycle;
\draw(-4.327,3.043)--(-4.346,3.062)--(-4.353,3.061);
\filldraw[fill opacity=0.8,fill=gray!20,draw=none](-4.418,2.894)--(-2.453,5.646)--(-2.478,5.697)--(-2.479,5.697)--(-4.46,2.924)--cycle;
\draw(-2.479,5.697)--(-4.46,2.924)--(-4.418,2.894)--(-2.453,5.646);
\filldraw[fill opacity=0.8,fill=gray!20,draw=none](-2.478,5.697)--(-2.453,5.646)--(-2.431,5.677)--cycle;
\draw(-2.453,5.646)--(-2.431,5.677);
\filldraw[fill opacity=0.8,fill=gray!20,draw=none](-2.342,5.856)--(-2.322,5.859)--(-2.325,5.868)--(-2.329,5.868)--cycle;
\draw(-2.322,5.859)--(-2.325,5.868)--(-2.329,5.868);
\filldraw[fill opacity=0.8,fill=gray!20,draw=none](-2.442,5.838)--(-2.464,5.845)--(-2.469,5.832)--cycle;
\draw(-2.464,5.845)--(-2.469,5.832);
\filldraw[fill opacity=0.8,fill=gray!20,draw=none](-2.464,5.845)--(-2.442,5.838)--(-2.444,5.851)--cycle;
\draw(-2.442,5.838)--(-2.444,5.851);
\filldraw[fill opacity=0.8,fill=gray!20,draw=none](-4.063,2.419)--(-4.049,2.417)--(-4.084,2.446)--cycle;
\draw(-4.049,2.417)--(-4.084,2.446);
\filldraw[fill opacity=0.8,fill=gray!20,draw=none](-2.482,5.757)--(-2.48,5.781)--(-2.49,5.784)--cycle;
\draw(-2.482,5.757)--(-2.48,5.781)--(-2.49,5.784);
\filldraw[fill opacity=0.8,fill=gray!20,draw=none](-2.358,5.684)--(-2.304,5.686)--(-2.301,5.731)--(-2.337,5.73)--cycle;
\draw(-2.358,5.684)--(-2.304,5.686)--(-2.301,5.731)--(-2.337,5.73);
\filldraw[fill opacity=0.8,fill=gray!20,draw=none](-2.342,5.856)--(-2.377,5.823)--(-2.312,5.825)--(-2.322,5.859)--cycle;
\draw(-2.377,5.823)--(-2.312,5.825)--(-2.322,5.859);
\filldraw[fill opacity=0.8,fill=gray!20,draw=none](-2.397,5.822)--(-2.442,5.838)--(-2.469,5.832)--(-2.47,5.828)--cycle;
\draw(-2.469,5.832)--(-2.47,5.828)--(-2.397,5.822);
\filldraw[fill opacity=0.8,fill=gray!20,draw=none](-2.431,5.778)--(-2.464,5.827)--(-2.47,5.828)--(-2.48,5.781)--cycle;
\draw(-2.464,5.827)--(-2.47,5.828)--(-2.48,5.781)--(-2.431,5.778);
\filldraw[fill opacity=0.8,fill=gray!20,draw=none](-2.411,5.729)--(-2.431,5.778)--(-2.48,5.781)--(-2.484,5.734)--cycle;
\draw(-2.431,5.778)--(-2.48,5.781)--(-2.484,5.734)--(-2.411,5.729);
\filldraw[fill opacity=0.8,fill=gray!20,draw=none](-2.362,5.837)--(-2.342,5.856)--(-2.391,5.848)--cycle;
\filldraw[fill opacity=0.8,fill=gray!20,draw=none](-2.431,5.778)--(-2.394,5.775)--(-2.392,5.822)--(-2.464,5.827)--cycle;
\draw(-2.431,5.778)--(-2.394,5.775)--(-2.392,5.822)--(-2.464,5.827);
\filldraw[fill opacity=0.8,fill=gray!20,draw=none](-2.707,7.809)--(-2.393,5.854)--(-2.343,5.86)--(-2.658,7.821)--cycle;
\draw(-2.343,5.86)--(-2.658,7.821)--(-2.707,7.809)--(-2.393,5.854);
\filldraw[fill opacity=0.8,fill=gray!20,draw=none](-2.442,5.838)--(-2.416,5.829)--(-2.391,5.848)--cycle;
\filldraw[fill opacity=0.8,fill=gray!20,draw=none](-2.358,5.684)--(-2.337,5.73)--(-2.394,5.727)--(-2.394,5.682)--cycle;
\draw(-2.337,5.73)--(-2.394,5.727)--(-2.394,5.682)--(-2.358,5.684);
\filldraw[fill opacity=0.8,fill=gray!20,draw=none](-2.31,5.811)--(-2.312,5.825)--(-2.331,5.825)--cycle;
\draw(-2.31,5.811)--(-2.312,5.825)--(-2.331,5.825);
\filldraw[fill opacity=0.8,fill=gray!20](-2.301,5.731)--(-2.304,5.779)--(-2.394,5.775)--(-2.394,5.727)--cycle;
\filldraw[fill opacity=0.8,fill=gray!20,draw=none](-2.304,5.779)--(-2.31,5.811)--(-2.331,5.825)--(-2.377,5.823)--(-2.393,5.796)--(-2.394,5.775)--cycle;
\draw(-2.331,5.825)--(-2.377,5.823);
\draw(-2.393,5.796)--(-2.394,5.775)--(-2.304,5.779)--(-2.31,5.811);
\filldraw[fill opacity=0.8,fill=gray!20,draw=none](-2.377,5.823)--(-2.362,5.837)--(-2.391,5.848)--(-2.392,5.822)--cycle;
\draw(-2.391,5.848)--(-2.392,5.822)--(-2.377,5.823);
\filldraw[fill opacity=0.8,fill=gray!20,draw=none](-2.416,5.829)--(-2.397,5.822)--(-2.392,5.822)--(-2.391,5.848)--cycle;
\draw(-2.397,5.822)--(-2.392,5.822)--(-2.391,5.848);
\filldraw[fill opacity=0.8,fill=gray!20,draw=none](-4.084,2.446)--(-4.049,2.417)--(-4.044,2.457)--(-4.108,2.511)--cycle;
\draw(-4.084,2.446)--(-4.049,2.417);
\draw(-4.044,2.457)--(-4.108,2.511);
\filldraw[fill opacity=0.8,fill=gray!20,draw=none](-2.411,5.729)--(-2.394,5.727)--(-2.394,5.775)--(-2.431,5.778)--cycle;
\draw(-2.411,5.729)--(-2.394,5.727)--(-2.394,5.775)--(-2.431,5.778);
\filldraw[fill opacity=0.8,fill=gray!20,draw=none](-2.377,5.823)--(-2.392,5.822)--(-2.393,5.796)--cycle;
\draw(-2.377,5.823)--(-2.392,5.822)--(-2.393,5.796);
\filldraw[fill opacity=0.8,fill=gray!20,draw=none](-5.526,2.37)--(-5.59,2.33)--(-5.464,2.403)--cycle;
\draw(-5.526,2.37)--(-5.59,2.33);
\filldraw[fill opacity=0.8,fill=gray!20,draw=none](-4.084,2.446)--(-4.108,2.511)--(-4.143,2.54)--cycle;
\draw(-4.108,2.511)--(-4.143,2.54);
\filldraw[fill opacity=0.8,fill=gray!20,draw=none](-4.108,2.511)--(-4.044,2.457)--(-4.064,2.535)--(-4.148,2.604)--cycle;
\draw(-4.108,2.511)--(-4.044,2.457);
\draw(-4.064,2.535)--(-4.148,2.604);
\filldraw[fill opacity=0.8,fill=gray!20,draw=none](-7.975,.828)--(-7.962,.828)--(-7.962,.807)--cycle;
\draw(-7.975,.828)--(-7.962,.828)--(-7.962,.807);
\filldraw[fill opacity=0.8,fill=gray!20,draw=none](-7.975,.828)--(-7.98,.834)--(-7.989,.884)--(-7.961,.885)--(-7.962,.828)--cycle;
\draw(-7.989,.884)--(-7.961,.885)--(-7.962,.828)--(-7.975,.828);
\filldraw[fill opacity=0.8,fill=gray!20,draw=none](-7.989,.884)--(-7.98,.932)--(-7.975,.939)--(-7.962,.939)--(-7.961,.885)--cycle;
\draw(-7.975,.939)--(-7.962,.939)--(-7.961,.885)--(-7.989,.884);
\filldraw[fill opacity=0.8,fill=gray!20,draw=none](-7.975,.939)--(-7.962,.957)--(-7.962,.939)--cycle;
\draw(-7.962,.957)--(-7.962,.939)--(-7.975,.939);
\filldraw[fill opacity=0.8,fill=gray!20,draw=none](-4.143,2.54)--(-4.108,2.511)--(-4.148,2.604)--(-4.205,2.652)--cycle;
\draw(-4.143,2.54)--(-4.108,2.511);
\draw(-4.148,2.604)--(-4.205,2.652);
\filldraw[fill opacity=0.8,fill=gray!20](-8.104,.705)--(-8.133,.753)--(-8.06,.767)--(-8.044,.716)--cycle;
\filldraw[fill opacity=0.8,fill=gray!20,draw=none](-4.089,2.828)--(-4.07,2.831)--(-4.105,2.824)--cycle;
\draw(-4.07,2.831)--(-4.105,2.824);
\filldraw[fill opacity=0.8,fill=gray!20,draw=none](-4.161,2.823)--(-4.149,2.815)--(-4.07,2.831)--(-4.002,2.861)--(-4.149,2.831)--cycle;
\draw(-4.149,2.815)--(-4.07,2.831);
\draw(-4.002,2.861)--(-4.149,2.831);
\filldraw[fill opacity=0.8,fill=gray!20](-4.597,3.02)--(-4.567,3.06)--(-4.507,3.072)--(-4.523,3.034)--cycle;
\filldraw[fill opacity=0.8,fill=gray!20,draw=none](-8.066,1.567)--(-8.071,1.574)--(-8.079,1.624)--(-8.023,1.626)--(-8.023,1.569)--cycle;
\draw(-8.079,1.624)--(-8.023,1.626)--(-8.023,1.569)--(-8.066,1.567);
\filldraw[fill opacity=0.8,fill=gray!20,draw=none](-8.044,1.53)--(-8.066,1.567)--(-8.023,1.569)--(-8.024,1.518)--cycle;
\draw(-8.066,1.567)--(-8.023,1.569)--(-8.024,1.518);
\filldraw[fill opacity=0.8,fill=gray!20,draw=none](-4.148,2.604)--(-4.064,2.535)--(-4.107,2.637)--(-4.197,2.712)--cycle;
\draw(-4.148,2.604)--(-4.064,2.535);
\draw(-4.107,2.637)--(-4.197,2.712);
\filldraw[fill opacity=0.8,fill=gray!20,draw=none](-8.042,.713)--(-8.044,.716)--(-8.02,.717)--cycle;
\draw(-8.042,.713)--(-8.044,.716)--(-8.02,.717);
\filldraw[fill opacity=0.8,fill=gray!20](-8.151,.918)--(-8.133,.967)--(-8.06,.982)--(-8.069,.934)--cycle;
\filldraw[fill opacity=0.8,fill=gray!20](-8.066,.665)--(-8.104,.705)--(-8.044,.716)--(-8.024,.673)--cycle;
\filldraw[fill opacity=0.8,fill=gray!20,draw=none](-4.149,2.831)--(-4.002,2.861)--(-3.966,2.902)--(-4.136,2.867)--cycle;
\draw(-4.149,2.831)--(-4.002,2.861);
\draw(-3.966,2.902)--(-4.136,2.867);
\filldraw[fill opacity=0.8,fill=gray!20,draw=none](-4.143,2.54)--(-4.205,2.652)--(-4.246,2.687)--cycle;
\draw(-4.205,2.652)--(-4.246,2.687);
\filldraw[fill opacity=0.8,fill=gray!20,draw=none](-7.948,.985)--(-7.963,.986)--(-7.965,1.023)--(-7.938,1.021)--cycle;
\draw(-7.948,.985)--(-7.963,.986)--(-7.965,1.023)--(-7.938,1.021);
\filldraw[fill opacity=0.8,fill=gray!20](-8.06,.982)--(-8.044,1.019)--(-7.965,1.023)--(-7.963,.986)--cycle;
\filldraw[fill opacity=0.8,fill=gray!20](-7.914,.672)--(-7.889,.714)--(-7.836,.701)--(-7.876,.663)--cycle;
\filldraw[fill opacity=0.8,fill=gray!20,draw=none](-7.964,.676)--(-7.89,.713)--(-7.914,.672)--cycle;
\draw(-7.89,.713)--(-7.914,.672)--(-7.964,.676);
\filldraw[fill opacity=0.8,fill=gray!20,draw=none](-7.888,.714)--(-7.861,.73)--(-7.83,.709)--(-7.836,.701)--cycle;
\draw(-7.83,.709)--(-7.836,.701)--(-7.888,.714);
\filldraw[fill opacity=0.8,fill=gray!20,draw=none](-8.062,.971)--(-8.06,.982)--(-7.982,.985)--cycle;
\draw(-8.062,.971)--(-8.06,.982)--(-7.982,.985);
\filldraw[fill opacity=0.8,fill=gray!20,draw=none](-7.865,.821)--(-7.718,.869)--(-7.715,.923)--(-7.874,.872)--cycle;
\draw(-7.715,.923)--(-7.874,.872)--(-7.865,.821)--(-7.718,.869);
\filldraw[fill opacity=0.8,fill=gray!20,draw=none](-4.193,2.822)--(-4.174,2.806)--(-4.149,2.815)--cycle;
\draw(-4.193,2.822)--(-4.174,2.806);
\filldraw[fill opacity=0.8,fill=gray!20,draw=none](-4.161,2.823)--(-4.168,2.818)--(-4.149,2.815)--cycle;
\filldraw[fill opacity=0.8,fill=gray!20,draw=none](-7.922,.993)--(-7.947,.99)--(-7.938,1.021)--(-7.889,1.017)--(-7.888,1.016)--cycle;
\draw(-7.938,1.021)--(-7.889,1.017)--(-7.888,1.016);
\filldraw[fill opacity=0.8,fill=gray!20](-7.971,.643)--(-7.968,.676)--(-7.914,.672)--(-7.943,.641)--cycle;
\filldraw[fill opacity=0.8,fill=gray!20](-8,.642)--(-8.024,.673)--(-7.968,.676)--(-7.971,.643)--cycle;
\filldraw[fill opacity=0.8,fill=gray!20,draw=none](-4.739,2.603)--(-5.276,2.27)--(-5.174,2.296)--(-4.694,2.594)--cycle;
\draw(-4.739,2.603)--(-5.276,2.27);
\draw(-5.174,2.296)--(-4.694,2.594);
\filldraw[fill opacity=0.8,fill=gray!20,draw=none](-7.861,.73)--(-7.821,.753)--(-7.804,.749)--(-7.83,.709)--cycle;
\draw(-7.821,.753)--(-7.804,.749)--(-7.83,.709);
\filldraw[fill opacity=0.8,fill=gray!20,draw=none](-4.614,3.009)--(-4.626,2.98)--(-4.634,2.984)--cycle;
\draw(-4.626,2.98)--(-4.634,2.984);
\filldraw[fill opacity=0.8,fill=gray!20,draw=none](-4.641,2.955)--(-4.634,2.984)--(-4.626,2.98)--cycle;
\draw(-4.634,2.984)--(-4.626,2.98);
\filldraw[fill opacity=0.8,fill=gray!20,draw=none](-4.641,2.955)--(-4.666,2.944)--(-5.128,3.175)--(-4.755,3.045)--(-4.634,2.984)--cycle;
\draw(-4.666,2.944)--(-5.128,3.175);
\draw(-4.755,3.045)--(-4.634,2.984);
\filldraw[fill opacity=0.8,fill=gray!20,draw=none](-4.614,3.009)--(-4.634,2.984)--(-7.637,4.487)--(-7.624,4.523)--(-7.609,4.53)--(-4.606,3.027)--cycle;
\draw(-4.634,2.984)--(-7.637,4.487);
\draw(-7.609,4.53)--(-4.606,3.027);
\filldraw[fill opacity=0.8,fill=gray!20,draw=none](-4.649,2.949)--(-4.628,3)--(-4.597,3.02)--(-4.615,2.971)--cycle;
\draw(-4.649,2.949)--(-4.628,3)--(-4.597,3.02)--(-4.615,2.971);
\filldraw[fill opacity=0.8,fill=gray!20](-4.567,3.06)--(-4.53,3.089)--(-4.487,3.097)--(-4.507,3.072)--cycle;
\filldraw[fill opacity=0.8,fill=gray!20,draw=none](-4.602,2.848)--(-5.487,2.534)--(-5.319,2.539)--(-4.89,2.692)--cycle;
\draw(-4.602,2.848)--(-5.487,2.534);
\draw(-5.319,2.539)--(-4.89,2.692);
\filldraw[fill opacity=0.8,fill=gray!20,draw=none](-4.136,2.867)--(-3.966,2.902)--(-3.968,2.948)--(-4.146,2.912)--cycle;
\draw(-4.136,2.867)--(-3.966,2.902);
\draw(-3.968,2.948)--(-4.146,2.912);
\filldraw[fill opacity=0.8,fill=gray!20,draw=none](-4.506,3.073)--(-4.487,3.097)--(-4.471,3.098)--cycle;
\draw(-4.506,3.073)--(-4.487,3.097)--(-4.471,3.098);
\filldraw[fill opacity=0.8,fill=gray!20](-8.133,.967)--(-8.104,1.008)--(-8.044,1.019)--(-8.06,.982)--cycle;
\filldraw[fill opacity=0.8,fill=gray!20](-8.022,.637)--(-8.066,.665)--(-8.024,.673)--(-8,.642)--cycle;
\filldraw[fill opacity=0.8,fill=gray!20,draw=none](-7.922,.993)--(-7.906,1.004)--(-7.879,.998)--(-7.879,.997)--cycle;
\draw(-7.879,.998)--(-7.879,.997);
\filldraw[fill opacity=0.8,fill=gray!20](-8.213,1.66)--(-8.195,1.709)--(-8.121,1.723)--(-8.131,1.675)--cycle;
\filldraw[fill opacity=0.8,fill=gray!20,draw=none](-8.105,1.456)--(-8.105,1.457)--(-8.098,1.458)--cycle;
\draw(-8.105,1.456)--(-8.105,1.457)--(-8.098,1.458);
\filldraw[fill opacity=0.8,fill=gray!20](-7.943,.641)--(-7.914,.672)--(-7.876,.663)--(-7.924,.636)--cycle;
\filldraw[fill opacity=0.8,fill=gray!20](-8.127,1.406)--(-8.165,1.446)--(-8.105,1.457)--(-8.085,1.414)--cycle;
\filldraw[fill opacity=0.8,fill=gray!20,draw=none](-7.95,1.455)--(-7.942,1.477)--(-7.894,1.447)--(-7.897,1.442)--cycle;
\draw(-7.894,1.447)--(-7.897,1.442)--(-7.95,1.455)--(-7.942,1.477);
\filldraw[fill opacity=0.8,fill=gray!20](-8.164,.733)--(-8.186,.785)--(-8.151,.808)--(-8.133,.753)--cycle;
\filldraw[fill opacity=0.8,fill=gray!20](-8.186,.785)--(-8.194,.84)--(-8.158,.864)--(-8.151,.808)--cycle;
\filldraw[fill opacity=0.8,fill=gray!20,draw=none](-7.968,1.457)--(-7.945,1.47)--(-7.95,1.455)--cycle;
\draw(-7.945,1.47)--(-7.95,1.455)--(-7.968,1.457);
\filldraw[fill opacity=0.8,fill=gray!20](-8.129,.688)--(-8.164,.733)--(-8.133,.753)--(-8.104,.705)--cycle;
\filldraw[fill opacity=0.8,fill=gray!20,draw=none](-8.029,1.417)--(-8.029,1.422)--(-7.968,1.457)--(-7.95,1.455)--(-7.975,1.413)--cycle;
\draw(-7.968,1.457)--(-7.95,1.455)--(-7.975,1.413)--(-8.029,1.417)--(-8.029,1.422);
\filldraw[fill opacity=0.8,fill=gray!20,draw=none](-7.821,.753)--(-7.797,.769)--(-7.804,.749)--cycle;
\draw(-7.797,.769)--(-7.804,.749)--(-7.821,.753);
\filldraw[fill opacity=0.8,fill=gray!20,draw=none](-4.568,2.7)--(-4.576,2.669)--(-4.62,2.691)--(-4.609,2.707)--cycle;
\draw(-4.62,2.691)--(-4.609,2.707);
\filldraw[fill opacity=0.8,fill=gray!20,draw=none](-4.609,2.707)--(-4.62,2.691)--(-4.633,2.708)--cycle;
\draw(-4.609,2.707)--(-4.62,2.691);
\filldraw[fill opacity=0.8,fill=gray!20,draw=none](-4.523,2.685)--(-4.574,2.668)--(-4.694,2.594)--(-4.637,2.612)--(-4.521,2.684)--cycle;
\draw(-4.574,2.668)--(-4.694,2.594);
\draw(-4.637,2.612)--(-4.521,2.684);
\filldraw[fill opacity=0.8,fill=gray!20,draw=none](-4.576,2.669)--(-4.577,2.668)--(-4.636,2.667)--(-4.62,2.691)--cycle;
\draw(-4.636,2.667)--(-4.62,2.691);
\filldraw[fill opacity=0.8,fill=gray!20,draw=none](-4.628,2.725)--(-5.318,2.297)--(-5.276,2.27)--(-4.605,2.686)--cycle;
\draw(-4.628,2.725)--(-5.318,2.297);
\draw(-5.276,2.27)--(-4.605,2.686);
\filldraw[fill opacity=0.8,fill=gray!20,draw=none](-5.242,2.518)--(-5.178,2.549)--(-5.144,2.57)--cycle;
\draw(-5.178,2.549)--(-5.144,2.57);
\filldraw[fill opacity=0.8,fill=gray!20,draw=none](-7.938,1.474)--(-7.899,1.498)--(-7.866,1.49)--(-7.894,1.447)--cycle;
\draw(-7.899,1.498)--(-7.866,1.49)--(-7.894,1.447);
\filldraw[fill opacity=0.8,fill=gray!20](-8.121,1.723)--(-8.105,1.76)--(-8.027,1.764)--(-8.024,1.727)--cycle;
\filldraw[fill opacity=0.8,fill=gray!20,draw=none](-4.614,3.009)--(-4.606,3.027)--(-4.601,3.025)--cycle;
\draw(-4.606,3.027)--(-4.601,3.025);
\filldraw[fill opacity=0.8,fill=gray!20,draw=none](-4.582,3.05)--(-4.601,3.025)--(-4.611,3.029)--cycle;
\draw(-4.601,3.025)--(-4.611,3.029);
\filldraw[fill opacity=0.8,fill=gray!20](-4.628,3)--(-4.593,3.044)--(-4.567,3.06)--(-4.597,3.02)--cycle;
\filldraw[fill opacity=0.8,fill=gray!20](-7.975,1.413)--(-7.95,1.455)--(-7.897,1.442)--(-7.938,1.404)--cycle;
\filldraw[fill opacity=0.8,fill=gray!20,draw=none](-7.868,.809)--(-7.874,.773)--(-7.752,.813)--cycle;
\draw(-7.868,.809)--(-7.874,.773)--(-7.752,.813);
\filldraw[fill opacity=0.8,fill=gray!20,draw=none](-4.487,2.689)--(-4.493,2.685)--(-4.49,2.691)--cycle;
\draw(-4.487,2.689)--(-4.493,2.685);
\filldraw[fill opacity=0.8,fill=gray!20](-8.194,.84)--(-8.186,.896)--(-8.151,.918)--(-8.158,.864)--cycle;
\filldraw[fill opacity=0.8,fill=gray!20](-8.084,.653)--(-8.129,.688)--(-8.104,.705)--(-8.066,.665)--cycle;
\filldraw[fill opacity=0.8,fill=gray!20](-8.044,1.019)--(-8.024,1.044)--(-7.968,1.047)--(-7.965,1.023)--cycle;
\filldraw[fill opacity=0.8,fill=gray!20](-7.965,1.023)--(-7.968,1.047)--(-7.914,1.043)--(-7.889,1.017)--cycle;
\filldraw[fill opacity=0.8,fill=gray!20,draw=none](-4.277,2.753)--(-4.253,2.774)--(-4.26,2.772)--cycle;
\filldraw[fill opacity=0.8,fill=gray!20,draw=none](-4.205,2.652)--(-4.148,2.604)--(-4.197,2.712)--(-4.263,2.767)--cycle;
\draw(-4.205,2.652)--(-4.148,2.604);
\draw(-4.197,2.712)--(-4.263,2.767);
\filldraw[fill opacity=0.8,fill=gray!20,draw=none](-7.906,1.004)--(-7.888,1.016)--(-7.879,.998)--cycle;
\draw(-7.888,1.016)--(-7.879,.998);
\filldraw[fill opacity=0.8,fill=gray!20,draw=none](-8.005,1.735)--(-8.025,1.733)--(-8.027,1.764)--(-7.968,1.76)--cycle;
\draw(-8.025,1.733)--(-8.027,1.764)--(-7.968,1.76);
\filldraw[fill opacity=0.8,fill=gray!20,draw=none](-4.481,2.682)--(-4.494,2.684)--(-4.493,2.685)--(-4.487,2.689)--cycle;
\draw(-4.481,2.682)--(-4.494,2.684);
\draw(-4.493,2.685)--(-4.487,2.689);
\filldraw[fill opacity=0.8,fill=gray!20,draw=none](-7.879,.997)--(-7.879,.998)--(-7.878,.997)--cycle;
\draw(-7.879,.997)--(-7.879,.998);
\filldraw[fill opacity=0.8,fill=gray!20,draw=none](-8.123,1.715)--(-8.121,1.723)--(-8.068,1.725)--cycle;
\draw(-8.123,1.715)--(-8.121,1.723)--(-8.068,1.725);
\filldraw[fill opacity=0.8,fill=gray!20,draw=none](-7.878,.997)--(-7.879,.998)--(-7.888,1.016)--(-7.888,1.017)--(-7.836,1.004)--(-7.83,.997)--cycle;
\draw(-7.879,.998)--(-7.888,1.016);
\draw(-7.888,1.017)--(-7.836,1.004)--(-7.83,.997);
\filldraw[fill opacity=0.8,fill=gray!20,draw=none](-7.839,.98)--(-7.878,.997)--(-7.861,.997)--cycle;
\filldraw[fill opacity=0.8,fill=gray!20,draw=none](-7.921,1.55)--(-7.927,1.517)--(-7.828,1.561)--cycle;
\draw(-7.921,1.55)--(-7.927,1.517)--(-7.828,1.561);
\filldraw[fill opacity=0.8,fill=gray!20,draw=none](-4.161,2.823)--(-4.149,2.831)--(-4.193,2.822)--cycle;
\draw(-4.149,2.831)--(-4.193,2.822);
\filldraw[fill opacity=0.8,fill=gray!20,draw=none](-7.801,.766)--(-7.795,.773)--(-7.797,.769)--cycle;
\draw(-7.795,.773)--(-7.797,.769);
\filldraw[fill opacity=0.8,fill=gray!20](-8.061,1.383)--(-8.085,1.414)--(-8.029,1.417)--(-8.032,1.384)--cycle;
\filldraw[fill opacity=0.8,fill=gray!20](-8.032,1.384)--(-8.029,1.417)--(-7.975,1.413)--(-8.005,1.382)--cycle;
\filldraw[fill opacity=0.8,fill=gray!20,draw=none](-7.804,.749)--(-7.797,.769)--(-7.773,.79)--(-7.763,.779)--(-7.785,.728)--cycle;
\draw(-7.773,.79)--(-7.763,.779)--(-7.785,.728)--(-7.804,.749)--(-7.797,.769);
\filldraw[fill opacity=0.8,fill=gray!20](-7.836,.701)--(-7.804,.749)--(-7.785,.728)--(-7.819,.684)--cycle;
\filldraw[fill opacity=0.8,fill=gray!20,draw=none](-7.797,.769)--(-7.795,.773)--(-7.781,.799)--(-7.773,.79)--cycle;
\draw(-7.797,.769)--(-7.795,.773);
\draw(-7.781,.799)--(-7.773,.79);
\filldraw[fill opacity=0.8,fill=gray!20,draw=none](-4.605,2.97)--(-4.626,2.98)--(-4.614,3.009)--(-4.601,3.025)--(-4.542,2.995)--cycle;
\draw(-4.605,2.97)--(-4.626,2.98);
\draw(-4.601,3.025)--(-4.542,2.995);
\filldraw[fill opacity=0.8,fill=gray!20,draw=none](-4.252,2.797)--(-4.248,2.795)--(-4.248,2.801)--cycle;
\filldraw[fill opacity=0.8,fill=gray!20,draw=none](-7.96,1.74)--(-8.005,1.735)--(-7.988,1.747)--cycle;
\filldraw[fill opacity=0.8,fill=gray!20,draw=none](-7.889,1.496)--(-7.899,1.498)--(-7.878,1.511)--cycle;
\draw(-7.889,1.496)--(-7.899,1.498);
\filldraw[fill opacity=0.8,fill=gray!20](-8.195,1.709)--(-8.165,1.749)--(-8.105,1.76)--(-8.121,1.723)--cycle;
\filldraw[fill opacity=0.8,fill=gray!20](-8.104,1.008)--(-8.066,1.036)--(-8.024,1.044)--(-8.044,1.019)--cycle;
\filldraw[fill opacity=0.8,fill=gray!20](-8.186,.896)--(-8.164,.947)--(-8.133,.967)--(-8.151,.918)--cycle;
\filldraw[fill opacity=0.8,fill=gray!20](-7.876,.663)--(-7.836,.701)--(-7.819,.684)--(-7.865,.65)--cycle;
\filldraw[fill opacity=0.8,fill=gray!20](-7.974,.623)--(-8,.642)--(-7.971,.643)--(-7.974,.623)--cycle;
\filldraw[fill opacity=0.8,fill=gray!20](-7.974,.623)--(-7.971,.643)--(-7.943,.641)--(-7.974,.623)--cycle;
\filldraw[fill opacity=0.8,fill=gray!20,draw=none](-4.58,3.052)--(-4.572,3.058)--(-4.551,3.074)--(-4.559,3.066)--(-4.567,3.06)--cycle;
\draw(-4.559,3.066)--(-4.567,3.06)--(-4.58,3.052);
\filldraw[fill opacity=0.8,fill=gray!20,draw=none](-4.58,3.052)--(-4.582,3.05)--(-4.572,3.058)--cycle;
\draw(-4.58,3.052)--(-4.582,3.05);
\filldraw[fill opacity=0.8,fill=gray!20,draw=none](-4.548,2.998)--(-4.601,3.025)--(-4.575,3.06)--(-4.515,3.03)--cycle;
\draw(-4.548,2.998)--(-4.601,3.025);
\draw(-4.575,3.06)--(-4.515,3.03);
\filldraw[fill opacity=0.8,fill=gray!20](-8.031,.631)--(-8.084,.653)--(-8.066,.665)--(-8.022,.637)--cycle;
\filldraw[fill opacity=0.8,fill=gray!20,draw=none](-4.457,3.104)--(-4.471,3.098)--(-4.487,3.097)--(-4.467,3.106)--cycle;
\draw(-4.471,3.098)--(-4.487,3.097)--(-4.467,3.106);
\filldraw[fill opacity=0.8,fill=gray!20,draw=none](-4.252,2.785)--(-4.252,2.804)--(-4.241,2.796)--(-4.248,2.78)--cycle;
\draw(-4.241,2.796)--(-4.248,2.78)--(-4.252,2.785);
\filldraw[fill opacity=0.8,fill=gray!20,draw=none](-4.88,2.697)--(-4.89,2.692)--(-4.877,2.696)--cycle;
\draw(-4.89,2.692)--(-4.877,2.696);
\filldraw[fill opacity=0.8,fill=gray!20,draw=none](-5.144,2.57)--(-5.178,2.549)--(-4.925,2.659)--(-4.815,2.727)--cycle;
\draw(-5.144,2.57)--(-5.178,2.549);
\draw(-4.925,2.659)--(-4.815,2.727);
\filldraw[fill opacity=0.8,fill=gray!20](-8.083,1.378)--(-8.127,1.406)--(-8.085,1.414)--(-8.061,1.383)--cycle;
\filldraw[fill opacity=0.8,fill=gray!20,draw=none](-4.213,2.826)--(-4.193,2.822)--(-4.149,2.831)--(-4.136,2.867)--(-4.213,2.851)--cycle;
\draw(-4.193,2.822)--(-4.149,2.831);
\draw(-4.136,2.867)--(-4.213,2.851);
\filldraw[fill opacity=0.8,fill=gray!20,draw=none](-4.494,2.684)--(-4.495,2.684)--(-4.493,2.685)--cycle;
\draw(-4.494,2.684)--(-4.495,2.684)--(-4.493,2.685);
\filldraw[fill opacity=0.8,fill=gray!20](-4.53,3.089)--(-4.485,3.104)--(-4.463,3.108)--(-4.487,3.097)--cycle;
\filldraw[fill opacity=0.8,fill=gray!20](-7.974,.623)--(-8.022,.637)--(-8,.642)--(-7.974,.623)--cycle;
\filldraw[fill opacity=0.8,fill=gray!20,draw=none](-4.162,3.051)--(-4.146,3.074)--(-4.156,3.068)--(-4.186,3.025)--cycle;
\draw(-4.162,3.051)--(-4.146,3.074);
\draw(-4.156,3.068)--(-4.186,3.025);
\filldraw[fill opacity=0.8,fill=gray!20,draw=none](-7.888,1.016)--(-7.889,1.017)--(-7.888,1.017)--cycle;
\draw(-7.888,1.016)--(-7.889,1.017)--(-7.888,1.017);
\filldraw[fill opacity=0.8,fill=gray!20,draw=none](-7.888,1.017)--(-7.889,1.017)--(-7.914,1.043)--(-7.877,1.034)--cycle;
\draw(-7.888,1.017)--(-7.889,1.017)--(-7.914,1.043)--(-7.877,1.034);
\filldraw[fill opacity=0.8,fill=gray!20](-7.974,.623)--(-7.943,.641)--(-7.924,.636)--(-7.974,.623)--cycle;
\filldraw[fill opacity=0.8,fill=gray!20,draw=none](-4.509,2.695)--(-4.523,2.689)--(-4.517,2.686)--(-4.506,2.694)--cycle;
\draw(-4.517,2.686)--(-4.506,2.694);
\filldraw[fill opacity=0.8,fill=gray!20,draw=none](-4.489,2.678)--(-4.523,2.689)--(-4.5,2.686)--(-4.495,2.684)--cycle;
\draw(-4.5,2.686)--(-4.495,2.684)--(-4.489,2.678)--(-4.523,2.689);
\filldraw[fill opacity=0.8,fill=gray!20,draw=none](-4.494,2.683)--(-4.495,2.684)--(-4.494,2.684)--cycle;
\draw(-4.494,2.683)--(-4.495,2.684)--(-4.494,2.684);
\filldraw[fill opacity=0.8,fill=gray!20,draw=none](-4.476,2.677)--(-4.489,2.678)--(-4.494,2.683)--(-4.494,2.684)--(-4.481,2.682)--cycle;
\draw(-4.476,2.677)--(-4.489,2.678)--(-4.494,2.683);
\draw(-4.494,2.684)--(-4.481,2.682);
\filldraw[fill opacity=0.8,fill=gray!20,draw=none](-7.889,1.496)--(-7.878,1.511)--(-7.852,1.527)--(-7.866,1.49)--cycle;
\draw(-7.852,1.527)--(-7.866,1.49)--(-7.889,1.496);
\filldraw[fill opacity=0.8,fill=gray!20,draw=none](-7.888,1.017)--(-7.877,1.034)--(-7.876,1.034)--(-7.836,1.004)--cycle;
\draw(-7.877,1.034)--(-7.876,1.034)--(-7.836,1.004)--(-7.888,1.017);
\filldraw[fill opacity=0.8,fill=gray!20](-8.005,1.382)--(-7.975,1.413)--(-7.938,1.404)--(-7.985,1.377)--cycle;
\filldraw[fill opacity=0.8,fill=gray!20,draw=none](-4.252,2.797)--(-4.248,2.801)--(-4.249,2.811)--(-4.268,2.807)--cycle;
\draw(-4.249,2.811)--(-4.268,2.807);
\filldraw[fill opacity=0.8,fill=gray!20,draw=none](-4.162,3.051)--(-4.186,3.025)--(-4.202,3.002)--cycle;
\draw(-4.186,3.025)--(-4.202,3.002);
\filldraw[fill opacity=0.8,fill=gray!20,draw=none](-4.641,2.955)--(-4.626,2.98)--(-4.605,2.97)--cycle;
\draw(-4.626,2.98)--(-4.605,2.97);
\filldraw[fill opacity=0.8,fill=gray!20](-7.924,.636)--(-7.876,.663)--(-7.865,.65)--(-7.918,.63)--cycle;
\filldraw[fill opacity=0.8,fill=gray!20,draw=none](-7.779,.804)--(-7.78,.798)--(-7.781,.799)--cycle;
\draw(-7.78,.798)--(-7.781,.799);
\filldraw[fill opacity=0.8,fill=gray!20](-8.226,1.474)--(-8.248,1.526)--(-8.213,1.549)--(-8.195,1.494)--cycle;
\filldraw[fill opacity=0.8,fill=gray!20](-8.248,1.526)--(-8.255,1.581)--(-8.219,1.605)--(-8.213,1.549)--cycle;
\filldraw[fill opacity=0.8,fill=gray!20](-8.191,1.429)--(-8.226,1.474)--(-8.195,1.494)--(-8.165,1.446)--cycle;
\filldraw[fill opacity=0.8,fill=gray!20,draw=none](-7.858,1.524)--(-7.878,1.511)--(-7.853,1.546)--(-7.851,1.545)--cycle;
\draw(-7.853,1.546)--(-7.851,1.545);
\filldraw[fill opacity=0.8,fill=gray!20,draw=none](-7.839,.98)--(-7.861,.997)--(-7.83,.997)--(-7.807,.967)--cycle;
\draw(-7.83,.997)--(-7.807,.967);
\filldraw[fill opacity=0.8,fill=gray!20,draw=none](-4.551,3.074)--(-4.541,3.082)--(-4.53,3.089)--(-4.559,3.066)--cycle;
\draw(-4.541,3.082)--(-4.53,3.089)--(-4.559,3.066);
\filldraw[fill opacity=0.8,fill=gray!20,draw=none](-4.534,2.734)--(-4.574,2.72)--(-4.57,2.704)--(-4.517,2.723)--cycle;
\draw(-4.534,2.734)--(-4.574,2.72);
\draw(-4.57,2.704)--(-4.517,2.723);
\filldraw[fill opacity=0.8,fill=gray!20,draw=none](-4.546,2.705)--(-4.54,2.698)--(-4.549,2.705)--(-4.557,2.711)--cycle;
\draw(-4.546,2.705)--(-4.54,2.698);
\filldraw[fill opacity=0.8,fill=gray!20,draw=none](-4.488,2.704)--(-4.463,2.649)--(-4.425,2.631)--(-4.452,2.687)--cycle;
\draw(-4.488,2.704)--(-4.463,2.649);
\draw(-4.425,2.631)--(-4.452,2.687);
\filldraw[fill opacity=0.8,fill=gray!20,draw=none](-7.78,.798)--(-7.779,.804)--(-7.765,.845)--(-7.755,.835)--(-7.763,.779)--cycle;
\draw(-7.765,.845)--(-7.755,.835)--(-7.763,.779)--(-7.78,.798);
\filldraw[fill opacity=0.8,fill=gray!20,draw=none](-7.96,1.74)--(-7.988,1.747)--(-7.968,1.76)--(-7.95,1.758)--(-7.942,1.742)--cycle;
\draw(-7.968,1.76)--(-7.95,1.758)--(-7.942,1.742);
\filldraw[fill opacity=0.8,fill=gray!20,draw=none](-7.821,.967)--(-7.839,.98)--(-7.807,.967)--(-7.804,.963)--cycle;
\draw(-7.807,.967)--(-7.804,.963)--(-7.821,.967);
\filldraw[fill opacity=0.8,fill=gray!20,draw=none](-7.779,.804)--(-7.771,.852)--(-7.765,.845)--cycle;
\draw(-7.771,.852)--(-7.765,.845);
\filldraw[fill opacity=0.8,fill=gray!20,draw=none](-7.96,1.74)--(-7.942,1.742)--(-7.939,1.735)--cycle;
\draw(-7.942,1.742)--(-7.939,1.735);
\filldraw[fill opacity=0.8,fill=gray!20,draw=none](-4.547,3.078)--(-4.546,3.078)--(-4.495,3.098)--(-4.485,3.104)--(-4.53,3.089)--cycle;
\draw(-4.546,3.078)--(-4.495,3.098)--(-4.485,3.104)--(-4.53,3.089)--(-4.547,3.078);
\filldraw[fill opacity=0.8,fill=gray!20,draw=none](-4.551,3.074)--(-4.572,3.058)--(-4.549,3.076)--(-4.547,3.077)--(-4.547,3.078)--cycle;
\draw(-4.549,3.076)--(-4.547,3.077)--(-4.547,3.078);
\filldraw[fill opacity=0.8,fill=gray!20,draw=none](-4.551,3.074)--(-4.547,3.078)--(-4.541,3.082)--cycle;
\draw(-4.547,3.078)--(-4.541,3.082);
\filldraw[fill opacity=0.8,fill=gray!20,draw=none](-4.547,3.078)--(-4.547,3.077)--(-4.546,3.078)--cycle;
\draw(-4.547,3.078)--(-4.547,3.077)--(-4.546,3.078);
\filldraw[fill opacity=0.8,fill=gray!20,draw=none](-4.547,3.077)--(-4.547,3.077)--(-4.549,3.076)--cycle;
\draw(-4.547,3.077)--(-4.547,3.077)--(-4.549,3.076);
\filldraw[fill opacity=0.8,fill=gray!20,draw=none](-4.547,3.077)--(-4.544,3.078)--(-4.546,3.078)--(-4.547,3.077)--cycle;
\draw(-4.546,3.078)--(-4.547,3.077)--(-4.547,3.077);
\filldraw[fill opacity=0.8,fill=gray!20,draw=none](-4.515,3.03)--(-4.575,3.06)--(-4.561,3.086)--(-4.486,3.049)--cycle;
\draw(-4.515,3.03)--(-4.575,3.06);
\draw(-4.561,3.086)--(-4.486,3.049);
\filldraw[fill opacity=0.8,fill=gray!20](-8.164,.947)--(-8.129,.991)--(-8.104,1.008)--(-8.133,.967)--cycle;
\filldraw[fill opacity=0.8,fill=gray!20](-8.255,1.581)--(-8.248,1.637)--(-8.213,1.66)--(-8.219,1.605)--cycle;
\filldraw[fill opacity=0.8,fill=gray!20,draw=none](-7.801,.948)--(-7.821,.967)--(-7.812,.965)--cycle;
\draw(-7.821,.967)--(-7.812,.965);
\filldraw[fill opacity=0.8,fill=gray!20,draw=none](-8.149,.717)--(-8.161,.747)--(-8.176,.774)--(-8.186,.785)--(-8.164,.733)--cycle;
\draw(-8.176,.774)--(-8.186,.785)--(-8.164,.733)--(-8.149,.717);
\filldraw[fill opacity=0.8,fill=gray!20,draw=none](-8.073,.641)--(-8.113,.671)--(-8.125,.683)--(-8.126,.686)--(-8.084,.653)--cycle;
\draw(-8.126,.686)--(-8.084,.653)--(-8.073,.641)--(-8.113,.671)--(-8.125,.683);
\filldraw[fill opacity=0.8,fill=gray!20,draw=none](-8.125,.683)--(-8.119,.678)--(-8.142,.708)--(-8.149,.717)--(-8.164,.733)--(-8.129,.688)--cycle;
\draw(-8.119,.678)--(-8.142,.708);
\draw(-8.149,.717)--(-8.164,.733)--(-8.129,.688)--(-8.125,.683);
\filldraw[fill opacity=0.8,fill=gray!20,draw=none](-8.161,.747)--(-8.169,.766)--(-8.176,.774)--cycle;
\draw(-8.169,.766)--(-8.176,.774);
\filldraw[fill opacity=0.8,fill=gray!20,draw=none](-8.175,.773)--(-8.169,.766)--(-8.17,.773)--(-8.176,.799)--cycle;
\draw(-8.175,.773)--(-8.169,.766);
\filldraw[fill opacity=0.8,fill=gray!20,draw=none](-8.161,.747)--(-8.149,.717)--(-8.145,.712)--(-8.152,.73)--cycle;
\draw(-8.149,.717)--(-8.145,.712)--(-8.152,.73);
\filldraw[fill opacity=0.8,fill=gray!20,draw=none](-8.161,.747)--(-8.152,.73)--(-8.153,.734)--(-8.168,.765)--(-8.169,.766)--cycle;
\draw(-8.152,.73)--(-8.153,.734);
\draw(-8.168,.765)--(-8.169,.766);
\filldraw[fill opacity=0.8,fill=gray!20,draw=none](-8.17,.773)--(-8.169,.766)--(-8.168,.765)--cycle;
\draw(-8.169,.766)--(-8.168,.765);
\filldraw[fill opacity=0.8,fill=gray!20,draw=none](-8.137,.71)--(-8.148,.727)--(-8.152,.73)--(-8.145,.712)--cycle;
\draw(-8.152,.73)--(-8.145,.712)--(-8.137,.71);
\filldraw[fill opacity=0.8,fill=gray!20,draw=none](-8.148,.727)--(-8.153,.734)--(-8.152,.73)--cycle;
\draw(-8.153,.734)--(-8.152,.73);
\filldraw[fill opacity=0.8,fill=gray!20,draw=none](-8.175,.773)--(-8.176,.799)--(-8.184,.83)--(-8.194,.84)--(-8.186,.785)--cycle;
\draw(-8.184,.83)--(-8.194,.84)--(-8.186,.785)--(-8.175,.773);
\filldraw[fill opacity=0.8,fill=gray!20,draw=none](-8.176,.799)--(-8.178,.823)--(-8.184,.83)--cycle;
\draw(-8.178,.823)--(-8.184,.83);
\filldraw[fill opacity=0.8,fill=gray!20,draw=none](-8.176,.799)--(-8.17,.773)--(-8.178,.823)--cycle;
\filldraw[fill opacity=0.8,fill=gray!20,draw=none](-8.48,.673)--(-8.482,.676)--(-8.48,.676)--cycle;
\draw(-8.482,.676)--(-8.48,.676);
\filldraw[fill opacity=0.8,fill=gray!20,draw=none](-8.48,.676)--(-8.482,.676)--(-8.498,.695)--(-8.488,.723)--cycle;
\draw(-8.48,.676)--(-8.482,.676);
\filldraw[fill opacity=0.8,fill=gray!20,draw=none](-8.48,.673)--(-8.481,.629)--(-8.516,.627)--(-8.524,.674)--(-8.482,.676)--cycle;
\draw(-8.481,.629)--(-8.516,.627)--(-8.524,.674)--(-8.482,.676);
\filldraw[fill opacity=0.8,fill=gray!20,draw=none](-8.498,.695)--(-8.482,.676)--(-8.506,.675)--cycle;
\draw(-8.482,.676)--(-8.506,.675);
\filldraw[fill opacity=0.8,fill=gray!20,draw=none](-8.498,.695)--(-8.506,.675)--(-8.524,.674)--(-8.527,.721)--(-8.521,.722)--cycle;
\draw(-8.506,.675)--(-8.524,.674)--(-8.527,.721)--(-8.521,.722);
\filldraw[fill opacity=0.8,fill=gray!20,draw=none](-8.498,.695)--(-8.521,.722)--(-8.488,.723)--cycle;
\draw(-8.521,.722)--(-8.488,.723);
\filldraw[fill opacity=0.8,fill=gray!20](-8.577,.615)--(-8.593,.661)--(-8.524,.674)--(-8.516,.627)--cycle;
\filldraw[fill opacity=0.8,fill=gray!20](-8.593,.661)--(-8.598,.708)--(-8.527,.721)--(-8.524,.674)--cycle;
\filldraw[fill opacity=0.8,fill=gray!20,draw=none](-8.48,.673)--(-8.454,.63)--(-8.481,.629)--cycle;
\draw(-8.454,.63)--(-8.481,.629);
\filldraw[fill opacity=0.8,fill=gray!20,draw=none](-8.454,.628)--(-8.456,.63)--(-8.454,.63)--cycle;
\draw(-8.456,.63)--(-8.454,.63);
\filldraw[fill opacity=0.8,fill=gray!20,draw=none](-8.454,.628)--(-8.441,.587)--(-8.503,.585)--(-8.516,.627)--(-8.456,.63)--cycle;
\draw(-8.441,.587)--(-8.503,.585)--(-8.516,.627)--(-8.456,.63);
\filldraw[fill opacity=0.8,fill=gray!20](-8.553,.575)--(-8.577,.615)--(-8.516,.627)--(-8.503,.585)--cycle;
\filldraw[fill opacity=0.8,fill=gray!20,draw=none](-8.542,.653)--(-8.522,.626)--(-8.523,.618)--(-8.555,.631)--cycle;
\draw(-8.522,.626)--(-8.523,.618)--(-8.555,.631);
\filldraw[fill opacity=0.8,fill=gray!20,draw=none](-8.522,.614)--(-8.523,.618)--(-8.522,.626)--cycle;
\draw(-8.522,.614)--(-8.523,.618)--(-8.522,.626);
\filldraw[fill opacity=0.8,fill=gray!20,draw=none](-8.619,.635)--(-8.621,.642)--(-8.593,.661)--(-8.583,.633)--cycle;
\draw(-8.619,.635)--(-8.621,.642)--(-8.593,.661)--(-8.583,.633);
\filldraw[fill opacity=0.8,fill=gray!20,draw=none](-8.611,.649)--(-8.613,.672)--(-8.596,.696)--(-8.593,.661)--cycle;
\draw(-8.596,.696)--(-8.593,.661)--(-8.611,.649);
\filldraw[fill opacity=0.8,fill=gray!20,draw=none](-8.613,.672)--(-8.615,.697)--(-8.598,.708)--(-8.596,.696)--cycle;
\draw(-8.615,.697)--(-8.598,.708)--(-8.596,.696);
\filldraw[fill opacity=0.8,fill=gray!20,draw=none](-8.605,.703)--(-8.615,.697)--(-8.603,.747)--(-8.602,.747)--cycle;
\draw(-8.605,.703)--(-8.615,.697);
\draw(-8.603,.747)--(-8.602,.747);
\filldraw[fill opacity=0.8,fill=gray!20,draw=none](-8.559,.676)--(-8.542,.653)--(-8.555,.631)--(-8.595,.648)--cycle;
\draw(-8.555,.631)--(-8.595,.648);
\filldraw[fill opacity=0.8,fill=gray!20,draw=none](-8.603,.599)--(-8.619,.635)--(-8.583,.633)--(-8.577,.615)--cycle;
\draw(-8.583,.633)--(-8.577,.615)--(-8.603,.599)--(-8.619,.635);
\filldraw[fill opacity=0.8,fill=gray!20](-8.574,.561)--(-8.603,.599)--(-8.577,.615)--(-8.553,.575)--cycle;
\filldraw[fill opacity=0.8,fill=gray!20,draw=none](-8.534,.623)--(-8.523,.618)--(-8.522,.614)--(-8.526,.614)--cycle;
\draw(-8.534,.623)--(-8.523,.618)--(-8.522,.614);
\filldraw[fill opacity=0.8,fill=gray!20](-8.052,.758)--(-8.522,.606)--(-8.546,.651)--(-8.075,.803)--cycle;
\filldraw[fill opacity=0.8,fill=gray!20,draw=none](-8.539,.707)--(-8.532,.669)--(-8.556,.678)--cycle;
\draw(-8.532,.669)--(-8.556,.678);
\filldraw[fill opacity=0.8,fill=gray!20,draw=none](-8.563,.681)--(-8.532,.669)--(-8.542,.653)--cycle;
\draw(-8.563,.681)--(-8.532,.669);
\filldraw[fill opacity=0.8,fill=gray!20](-8.598,.708)--(-8.593,.753)--(-8.524,.766)--(-8.527,.721)--cycle;
\filldraw[fill opacity=0.8,fill=gray!20,draw=none](-8.605,.703)--(-8.602,.747)--(-8.593,.753)--(-8.598,.708)--cycle;
\draw(-8.602,.747)--(-8.593,.753)--(-8.598,.708)--(-8.605,.703);
\filldraw[fill opacity=0.8,fill=gray!20,draw=none](-8.602,.747)--(-8.583,.79)--(-8.577,.794)--(-8.593,.753)--cycle;
\draw(-8.583,.79)--(-8.577,.794)--(-8.593,.753)--(-8.602,.747);
\filldraw[fill opacity=0.8,fill=gray!20,draw=none](-8.6,.751)--(-8.542,.727)--(-8.539,.707)--(-8.556,.678)--(-8.625,.707)--cycle;
\draw(-8.6,.751)--(-8.542,.727);
\draw(-8.556,.678)--(-8.625,.707);
\filldraw[fill opacity=0.8,fill=gray!20](-8.075,.803)--(-8.546,.651)--(-8.554,.702)--(-8.084,.854)--cycle;
\filldraw[fill opacity=0.8,fill=gray!20](-7.865,.821)--(-7.874,.773)--(-7.897,.735)--(-7.933,.713)--(-7.974,.709)--(-8.016,.725)--(-8.052,.758)--(-8.075,.803)--(-8.084,.854)--(-8.075,.902)--(-8.052,.94)--(-8.016,.962)--(-7.974,.966)--(-7.933,.95)--(-7.897,.917)--(-7.874,.872)--cycle;
\filldraw[fill opacity=0.8,fill=gray!20](-7.865,.821)--(-7.874,.872)--(-7.897,.917)--(-7.933,.95)--(-7.974,.966)--(-8.016,.962)--(-8.052,.94)--(-8.075,.902)--(-8.084,.854)--(-8.075,.803)--(-8.052,.758)--(-8.016,.725)--(-7.974,.709)--(-7.933,.713)--(-7.897,.735)--(-7.874,.773)--cycle;
\filldraw[fill opacity=0.8,fill=gray!20](-8.145,1.394)--(-8.191,1.429)--(-8.165,1.446)--(-8.127,1.406)--cycle;
\filldraw[fill opacity=0.8,fill=gray!20](-8.027,1.764)--(-8.029,1.788)--(-7.975,1.784)--(-7.95,1.758)--cycle;
\filldraw[fill opacity=0.8,fill=gray!20](-8.105,1.76)--(-8.085,1.786)--(-8.029,1.788)--(-8.027,1.764)--cycle;
\filldraw[fill opacity=0.8,fill=gray!20,draw=none](-4.504,2.73)--(-4.489,2.706)--(-4.491,2.702)--(-4.509,2.695)--(-4.514,2.726)--cycle;
\draw(-4.489,2.706)--(-4.491,2.702);
\filldraw[fill opacity=0.8,fill=gray!20,draw=none](-4.504,2.73)--(-4.501,2.729)--(-4.517,2.723)--(-4.514,2.726)--cycle;
\draw(-4.501,2.729)--(-4.517,2.723);
\filldraw[fill opacity=0.8,fill=gray!20](-7.974,.623)--(-8.031,.631)--(-8.022,.637)--(-7.974,.623)--cycle;
\filldraw[fill opacity=0.8,fill=gray!20,draw=none](-7.916,1.725)--(-7.939,1.735)--(-7.942,1.742)--(-7.938,1.742)--cycle;
\draw(-7.939,1.735)--(-7.942,1.742);
\filldraw[fill opacity=0.8,fill=gray!20,draw=none](-4.282,2.749)--(-4.277,2.753)--(-4.26,2.772)--(-4.264,2.77)--(-4.265,2.77)--cycle;
\draw(-4.264,2.77)--(-4.265,2.77);
\filldraw[fill opacity=0.8,fill=gray!20,draw=none](-4.287,2.742)--(-4.277,2.753)--(-4.282,2.749)--cycle;
\filldraw[fill opacity=0.8,fill=gray!20,draw=none](-4.246,2.687)--(-4.205,2.652)--(-4.263,2.767)--(-4.318,2.813)--cycle;
\draw(-4.246,2.687)--(-4.205,2.652);
\draw(-4.263,2.767)--(-4.318,2.813);
\filldraw[fill opacity=0.8,fill=gray!20,draw=none](-4.457,3.104)--(-4.467,3.106)--(-4.463,3.108)--(-4.442,3.109)--cycle;
\draw(-4.467,3.106)--(-4.463,3.108)--(-4.442,3.109);
\filldraw[fill opacity=0.8,fill=gray!20,draw=none](-4.252,2.785)--(-4.248,2.78)--(-4.253,2.774)--cycle;
\draw(-4.252,2.785)--(-4.248,2.78)--(-4.253,2.774);
\filldraw[fill opacity=0.8,fill=gray!20,draw=none](-7.85,1.551)--(-7.851,1.545)--(-7.853,1.546)--cycle;
\draw(-7.851,1.545)--(-7.853,1.546);
\filldraw[fill opacity=0.8,fill=gray!20,draw=none](-8.184,.83)--(-8.176,.852)--(-8.175,.884)--(-8.186,.896)--(-8.194,.84)--cycle;
\draw(-8.175,.884)--(-8.186,.896)--(-8.194,.84)--(-8.184,.83);
\filldraw[fill opacity=0.8,fill=gray!20,draw=none](-8.184,.83)--(-8.178,.823)--(-8.176,.852)--cycle;
\draw(-8.184,.83)--(-8.178,.823);
\filldraw[fill opacity=0.8,fill=gray!20,draw=none](-8.176,.852)--(-8.17,.871)--(-8.169,.877)--(-8.175,.884)--cycle;
\draw(-8.169,.877)--(-8.175,.884);
\filldraw[fill opacity=0.8,fill=gray!20,draw=none](-8.176,.852)--(-8.178,.823)--(-8.17,.871)--cycle;
\filldraw[fill opacity=0.8,fill=gray!20,draw=none](-8.488,.723)--(-8.498,.749)--(-8.482,.768)--(-8.48,.768)--cycle;
\draw(-8.482,.768)--(-8.48,.768);
\filldraw[fill opacity=0.8,fill=gray!20,draw=none](-8.498,.749)--(-8.488,.723)--(-8.521,.722)--cycle;
\draw(-8.488,.723)--(-8.521,.722);
\filldraw[fill opacity=0.8,fill=gray!20,draw=none](-8.498,.749)--(-8.521,.722)--(-8.527,.721)--(-8.524,.766)--(-8.506,.767)--cycle;
\draw(-8.521,.722)--(-8.527,.721)--(-8.524,.766)--(-8.506,.767);
\filldraw[fill opacity=0.8,fill=gray!20,draw=none](-8.498,.749)--(-8.506,.767)--(-8.482,.768)--cycle;
\draw(-8.506,.767)--(-8.482,.768);
\filldraw[fill opacity=0.8,fill=gray!20,draw=none](-8.48,.771)--(-8.48,.768)--(-8.482,.768)--cycle;
\draw(-8.48,.768)--(-8.482,.768);
\filldraw[fill opacity=0.8,fill=gray!20,draw=none](-8.481,.795)--(-8.48,.771)--(-8.482,.768)--(-8.524,.766)--(-8.523,.772)--cycle;
\draw(-8.482,.768)--(-8.524,.766)--(-8.523,.772);
\filldraw[fill opacity=0.8,fill=gray!20,draw=none](-8.539,.707)--(-8.542,.727)--(-8.53,.722)--cycle;
\draw(-8.542,.727)--(-8.53,.722);
\filldraw[fill opacity=0.8,fill=gray!20,draw=none](-8.546,.733)--(-8.524,.757)--(-8.53,.722)--(-8.542,.727)--cycle;
\draw(-8.53,.722)--(-8.542,.727);
\filldraw[fill opacity=0.8,fill=gray!20,draw=none](-8.547,.762)--(-8.593,.753)--(-8.577,.794)--(-8.516,.806)--(-8.523,.772)--cycle;
\draw(-8.547,.762)--(-8.593,.753)--(-8.577,.794)--(-8.516,.806)--(-8.523,.772);
\filldraw[fill opacity=0.8,fill=gray!20,draw=none](-8.562,.762)--(-8.542,.727)--(-8.591,.748)--cycle;
\draw(-8.542,.727)--(-8.591,.748);
\filldraw[fill opacity=0.8,fill=gray!20,draw=none](-8.547,.762)--(-8.523,.772)--(-8.524,.766)--cycle;
\draw(-8.523,.772)--(-8.524,.766)--(-8.547,.762);
\filldraw[fill opacity=0.8,fill=gray!20,draw=none](-8.481,.795)--(-8.523,.772)--(-8.516,.806)--(-8.481,.807)--cycle;
\draw(-8.523,.772)--(-8.516,.806)--(-8.481,.807);
\filldraw[fill opacity=0.8,fill=gray!20,draw=none](-8.557,.798)--(-8.577,.794)--(-8.561,.816)--cycle;
\draw(-8.557,.798)--(-8.577,.794)--(-8.561,.816);
\filldraw[fill opacity=0.8,fill=gray!20,draw=none](-8.582,.791)--(-8.578,.797)--(-8.556,.823)--(-8.577,.794)--cycle;
\draw(-8.556,.823)--(-8.577,.794)--(-8.582,.791);
\filldraw[fill opacity=0.8,fill=gray!20,draw=none](-8.546,.733)--(-8.58,.794)--(-8.578,.797)--(-8.522,.773)--(-8.524,.757)--cycle;
\draw(-8.578,.797)--(-8.522,.773);
\filldraw[fill opacity=0.8,fill=gray!20](-8.084,.854)--(-8.554,.702)--(-8.546,.75)--(-8.075,.902)--cycle;
\filldraw[fill opacity=0.8,fill=gray!20,draw=none](-8.142,.708)--(-8.145,.712)--(-8.149,.717)--cycle;
\draw(-8.142,.708)--(-8.145,.712)--(-8.149,.717);
\filldraw[fill opacity=0.8,fill=gray!20,draw=none](-8.088,.678)--(-8.11,.695)--(-8.142,.708)--(-8.119,.678)--cycle;
\draw(-8.142,.708)--(-8.119,.678);
\filldraw[fill opacity=0.8,fill=gray!20,draw=none](-8.11,.695)--(-8.127,.708)--(-8.145,.712)--(-8.142,.708)--cycle;
\draw(-8.127,.708)--(-8.145,.712)--(-8.142,.708);
\filldraw[fill opacity=0.8,fill=gray!20,draw=none](-8.148,.727)--(-8.137,.71)--(-8.127,.708)--cycle;
\draw(-8.137,.71)--(-8.127,.708);
\filldraw[fill opacity=0.8,fill=gray!20,draw=none](-8.446,.602)--(-8.454,.628)--(-8.436,.616)--(-8.437,.601)--cycle;
\draw(-8.436,.616)--(-8.437,.601);
\filldraw[fill opacity=0.8,fill=gray!20,draw=none](-8.419,.595)--(-8.437,.601)--(-8.436,.616)--cycle;
\draw(-8.437,.601)--(-8.436,.616);
\filldraw[fill opacity=0.8,fill=gray!20,draw=none](-8.419,.595)--(-8.411,.586)--(-8.437,.588)--(-8.437,.601)--cycle;
\draw(-8.411,.586)--(-8.437,.588)--(-8.437,.601);
\filldraw[fill opacity=0.8,fill=gray!20,draw=none](-8.441,.587)--(-8.446,.602)--(-8.437,.601)--(-8.437,.588)--cycle;
\draw(-8.437,.601)--(-8.437,.588)--(-8.441,.587);
\filldraw[fill opacity=0.8,fill=gray!20,draw=none](-8.491,.56)--(-8.503,.585)--(-8.437,.588)--(-8.438,.57)--cycle;
\draw(-8.491,.56)--(-8.503,.585)--(-8.437,.588)--(-8.438,.57);
\filldraw[fill opacity=0.8,fill=gray!20](-8.016,.725)--(-8.487,.573)--(-8.522,.606)--(-8.052,.758)--cycle;
\filldraw[fill opacity=0.8,fill=gray!20,draw=none](-4.462,2.673)--(-4.469,2.673)--(-4.469,2.673)--(-4.476,2.677)--cycle;
\draw(-4.462,2.673)--(-4.469,2.673);
\filldraw[fill opacity=0.8,fill=gray!20,draw=none](-4.469,2.673)--(-4.489,2.678)--(-4.476,2.677)--cycle;
\draw(-4.469,2.673)--(-4.489,2.678)--(-4.476,2.677);
\filldraw[fill opacity=0.8,fill=gray!20](-7.974,.623)--(-7.924,.636)--(-7.918,.63)--(-7.974,.623)--cycle;
\filldraw[fill opacity=0.8,fill=gray!20,draw=none](-7.851,1.545)--(-7.85,1.551)--(-7.844,1.56)--(-7.846,1.544)--cycle;
\draw(-7.844,1.56)--(-7.846,1.544)--(-7.851,1.545);
\filldraw[fill opacity=0.8,fill=gray!20,draw=none](-7.858,1.524)--(-7.851,1.545)--(-7.846,1.544)--(-7.852,1.527)--cycle;
\draw(-7.851,1.545)--(-7.846,1.544)--(-7.852,1.527);
\filldraw[fill opacity=0.8,fill=gray!20,draw=none](-7.942,1.742)--(-7.95,1.758)--(-7.897,1.745)--(-7.894,1.741)--cycle;
\draw(-7.942,1.742)--(-7.95,1.758)--(-7.897,1.745)--(-7.894,1.741);
\filldraw[fill opacity=0.8,fill=gray!20,draw=none](-7.85,1.551)--(-7.839,1.6)--(-7.844,1.56)--cycle;
\draw(-7.839,1.6)--(-7.844,1.56);
\filldraw[fill opacity=0.8,fill=gray!20,draw=none](-7.788,.928)--(-7.78,.909)--(-7.781,.91)--(-7.795,.941)--(-7.797,.945)--cycle;
\draw(-7.78,.909)--(-7.781,.91);
\draw(-7.795,.941)--(-7.797,.945);
\filldraw[fill opacity=0.8,fill=gray!20,draw=none](-7.801,.948)--(-7.797,.945)--(-7.795,.941)--cycle;
\draw(-7.797,.945)--(-7.795,.941);
\filldraw[fill opacity=0.8,fill=gray!20,draw=none](-4.513,2.72)--(-4.509,2.695)--(-4.523,2.689)--(-4.539,2.696)--(-4.549,2.705)--(-4.543,2.714)--cycle;
\draw(-4.549,2.705)--(-4.543,2.714);
\filldraw[fill opacity=0.8,fill=gray!20,draw=none](-7.801,.948)--(-7.812,.965)--(-7.804,.963)--(-7.797,.945)--cycle;
\draw(-7.812,.965)--(-7.804,.963)--(-7.797,.945);
\filldraw[fill opacity=0.8,fill=gray!20,draw=none](-7.866,1.49)--(-7.852,1.527)--(-7.84,1.537)--(-7.824,1.52)--(-7.846,1.469)--cycle;
\draw(-7.84,1.537)--(-7.824,1.52)--(-7.846,1.469)--(-7.866,1.49)--(-7.852,1.527);
\filldraw[fill opacity=0.8,fill=gray!20](-7.897,1.442)--(-7.866,1.49)--(-7.846,1.469)--(-7.881,1.425)--cycle;
\filldraw[fill opacity=0.8,fill=gray!20,draw=none](-7.788,.928)--(-7.797,.945)--(-7.804,.963)--(-7.8,.958)--cycle;
\draw(-7.797,.945)--(-7.804,.963)--(-7.8,.958);
\filldraw[fill opacity=0.8,fill=gray!20,draw=none](-7.852,1.527)--(-7.846,1.544)--(-7.84,1.537)--cycle;
\draw(-7.852,1.527)--(-7.846,1.544)--(-7.84,1.537);
\filldraw[fill opacity=0.8,fill=gray!20](-8.024,1.044)--(-8,1.056)--(-7.971,1.057)--(-7.968,1.047)--cycle;
\filldraw[fill opacity=0.8,fill=gray!20](-7.968,1.047)--(-7.971,1.057)--(-7.943,1.055)--(-7.914,1.043)--cycle;
\filldraw[fill opacity=0.8,fill=gray!20,draw=none](-7.84,1.537)--(-7.846,1.544)--(-7.844,1.56)--(-7.832,1.592)--(-7.817,1.576)--(-7.82,1.554)--cycle;
\draw(-7.84,1.537)--(-7.846,1.544)--(-7.844,1.56);
\draw(-7.832,1.592)--(-7.817,1.576)--(-7.82,1.554);
\filldraw[fill opacity=0.8,fill=gray!20,draw=none](-4.488,2.704)--(-4.468,2.677)--(-4.48,2.677)--(-4.504,2.684)--(-4.491,2.702)--cycle;
\draw(-4.504,2.684)--(-4.491,2.702);
\filldraw[fill opacity=0.8,fill=gray!20,draw=none](-4.491,2.702)--(-4.504,2.684)--(-4.507,2.685)--(-4.508,2.69)--cycle;
\draw(-4.491,2.702)--(-4.504,2.684);
\filldraw[fill opacity=0.8,fill=gray!20,draw=none](-4.509,2.695)--(-4.508,2.69)--(-4.513,2.686)--(-4.523,2.689)--cycle;
\filldraw[fill opacity=0.8,fill=gray!20,draw=none](-4.513,2.686)--(-4.508,2.69)--(-4.507,2.685)--cycle;
\filldraw[fill opacity=0.8,fill=gray!20,draw=none](-4.469,2.673)--(-4.498,2.685)--(-4.516,2.689)--(-4.503,2.682)--(-4.489,2.678)--cycle;
\draw(-4.503,2.682)--(-4.489,2.678)--(-4.469,2.673)--(-4.498,2.685)--(-4.516,2.689);
\filldraw[fill opacity=0.8,fill=gray!20,draw=none](-7.779,.902)--(-7.773,.876)--(-7.771,.852)--cycle;
\filldraw[fill opacity=0.8,fill=gray!20,draw=none](-7.771,.852)--(-7.774,.902)--(-7.763,.89)--(-7.755,.835)--cycle;
\draw(-7.774,.902)--(-7.763,.89)--(-7.755,.835)--(-7.771,.852);
\filldraw[fill opacity=0.8,fill=gray!20,draw=none](-4.523,2.689)--(-4.527,2.687)--(-4.521,2.684)--(-4.517,2.686)--cycle;
\draw(-4.521,2.684)--(-4.517,2.686);
\filldraw[fill opacity=0.8,fill=gray!20,draw=none](-4.23,2.848)--(-4.227,2.832)--(-4.226,2.832)--(-4.219,2.887)--(-4.227,2.882)--cycle;
\draw(-4.227,2.832)--(-4.226,2.832)--(-4.219,2.887)--(-4.227,2.882);
\filldraw[fill opacity=0.8,fill=gray!20,draw=none](-4.26,2.842)--(-4.136,2.867)--(-4.146,2.912)--(-4.278,2.885)--cycle;
\draw(-4.26,2.842)--(-4.136,2.867);
\draw(-4.146,2.912)--(-4.278,2.885);
\filldraw[fill opacity=0.8,fill=gray!20](-8.211,4.316)--(-8.21,4.372)--(-8.106,4.365)--(-8.118,4.309)--cycle;
\filldraw[fill opacity=0.8,fill=gray!20](-8.21,4.372)--(-8.209,4.429)--(-8.101,4.421)--(-8.106,4.365)--cycle;
\filldraw[fill opacity=0.8,fill=gray!20](-8.106,4.365)--(-8.101,4.421)--(-8.026,4.403)--(-8.033,4.347)--cycle;
\filldraw[fill opacity=0.8,fill=gray!20](-8.118,4.309)--(-8.106,4.365)--(-8.033,4.347)--(-8.052,4.293)--cycle;
\filldraw[fill opacity=0.8,fill=gray!20](-8.101,4.421)--(-8.106,4.476)--(-8.033,4.458)--(-8.026,4.403)--cycle;
\filldraw[fill opacity=0.8,fill=gray!20,draw=none](-8.02,4.333)--(-8.033,4.347)--(-8.026,4.403)--(-8.003,4.379)--(-8.009,4.339)--cycle;
\draw(-8.02,4.333)--(-8.033,4.347)--(-8.026,4.403)--(-8.003,4.379)--(-8.009,4.339);
\filldraw[fill opacity=0.8,fill=gray!20,draw=none](-7.644,4.554)--(-7.998,4.392)--(-8.003,4.404)--(-7.643,4.576)--cycle;
\draw(-8.003,4.404)--(-7.643,4.576);
\filldraw[fill opacity=0.8,fill=gray!20,draw=none](-7.642,4.598)--(-7.642,4.576)--(-7.992,4.409)--(-7.989,4.463)--(-7.691,4.605)--cycle;
\draw(-7.642,4.576)--(-7.992,4.409);
\draw(-7.989,4.463)--(-7.691,4.605);
\filldraw[fill opacity=0.8,fill=gray!20,draw=none](-7.659,4.657)--(-7.65,4.625)--(-7.691,4.605)--(-7.773,4.623)--(-7.668,4.673)--cycle;
\draw(-7.65,4.625)--(-7.691,4.605);
\draw(-7.773,4.623)--(-7.668,4.673);
\filldraw[fill opacity=0.8,fill=gray!20,draw=none](-7.642,4.601)--(-7.642,4.598)--(-7.691,4.605)--(-7.649,4.625)--cycle;
\draw(-7.691,4.605)--(-7.649,4.625);
\filldraw[fill opacity=0.8,fill=gray!20](-8.209,4.429)--(-8.21,4.483)--(-8.106,4.476)--(-8.101,4.421)--cycle;
\filldraw[fill opacity=0.8,fill=gray!20](-8.21,4.483)--(-8.211,4.53)--(-8.118,4.523)--(-8.106,4.476)--cycle;
\filldraw[fill opacity=0.8,fill=gray!20](-8.106,4.476)--(-8.118,4.523)--(-8.052,4.507)--(-8.033,4.458)--cycle;
\filldraw[fill opacity=0.8,fill=gray!20,draw=none](-7.664,4.675)--(-7.985,4.522)--(-8.019,4.558)--(-7.7,4.711)--cycle;
\draw(-7.664,4.675)--(-7.985,4.522);
\draw(-8.019,4.558)--(-7.7,4.711);
\filldraw[fill opacity=0.8,fill=gray!20,draw=none](-7.659,4.657)--(-7.643,4.628)--(-7.65,4.625)--cycle;
\draw(-7.643,4.628)--(-7.65,4.625);
\filldraw[fill opacity=0.8,fill=gray!20,draw=none](-7.642,4.601)--(-7.649,4.625)--(-7.643,4.628)--cycle;
\draw(-7.649,4.625)--(-7.643,4.628);
\filldraw[fill opacity=0.8,fill=gray!20,draw=none](-7.65,4.601)--(-7.71,4.624)--(-7.715,4.655)--(-7.646,4.658)--(-7.608,4.63)--(-7.609,4.603)--cycle;
\draw(-7.71,4.624)--(-7.715,4.655)--(-7.646,4.658);
\draw(-7.608,4.63)--(-7.609,4.603)--(-7.65,4.601);
\filldraw[fill opacity=0.8,fill=gray!20,draw=none](-7.621,4.55)--(-7.629,4.552)--(-7.636,4.579)--(-7.631,4.602)--(-7.609,4.603)--(-7.611,4.551)--cycle;
\draw(-7.631,4.602)--(-7.609,4.603)--(-7.611,4.551)--(-7.621,4.55);
\filldraw[fill opacity=0.8,fill=gray!20,draw=none](-7.636,4.579)--(-7.642,4.601)--(-7.631,4.602)--cycle;
\draw(-7.642,4.601)--(-7.631,4.602);
\filldraw[fill opacity=0.8,fill=gray!20,draw=none](-7.642,4.598)--(-7.628,4.596)--(-7.629,4.582)--(-7.642,4.576)--cycle;
\draw(-7.629,4.582)--(-7.642,4.576);
\filldraw[fill opacity=0.8,fill=gray!20,draw=none](-7.642,4.554)--(-7.644,4.554)--(-7.643,4.576)--(-7.636,4.579)--cycle;
\draw(-7.643,4.576)--(-7.636,4.579);
\filldraw[fill opacity=0.8,fill=gray!20,draw=none](-7.636,4.579)--(-7.642,4.554)--(-7.695,4.564)--(-7.705,4.599)--(-7.642,4.601)--cycle;
\draw(-7.695,4.564)--(-7.705,4.599)--(-7.642,4.601);
\filldraw[fill opacity=0.8,fill=gray!20,draw=none](-7.61,4.576)--(-7.609,4.586)--(-7.601,4.583)--cycle;
\draw(-7.61,4.576)--(-7.609,4.586);
\filldraw[fill opacity=0.8,fill=gray!20,draw=none](-7.691,4.605)--(-8.014,4.451)--(-8.021,4.505)--(-7.773,4.623)--cycle;
\draw(-7.691,4.605)--(-8.014,4.451);
\draw(-8.021,4.505)--(-7.773,4.623);
\filldraw[fill opacity=0.8,fill=gray!20,draw=none](-7.65,4.601)--(-7.705,4.599)--(-7.71,4.624)--cycle;
\draw(-7.65,4.601)--(-7.705,4.599)--(-7.71,4.624);
\filldraw[fill opacity=0.8,fill=gray!20,draw=none](-7.567,4.55)--(-7.588,4.549)--(-7.611,4.551)--(-7.61,4.576)--(-7.601,4.583)--(-7.535,4.565)--cycle;
\draw(-7.588,4.549)--(-7.611,4.551)--(-7.61,4.576);
\filldraw[fill opacity=0.8,fill=gray!20,draw=none](-7.629,4.552)--(-7.642,4.554)--(-7.636,4.579)--cycle;
\filldraw[fill opacity=0.8,fill=gray!20,draw=none](-7.658,4.547)--(-7.644,4.554)--(-7.644,4.549)--(-7.648,4.534)--(-7.666,4.525)--cycle;
\draw(-7.648,4.534)--(-7.666,4.525);
\filldraw[fill opacity=0.8,fill=gray!20,draw=none](-7.641,4.555)--(-7.642,4.554)--(-7.636,4.579)--(-7.629,4.582)--cycle;
\draw(-7.636,4.579)--(-7.629,4.582);
\filldraw[fill opacity=0.8,fill=gray!20,draw=none](-7.716,4.542)--(-7.75,4.59)--(-7.705,4.599)--(-7.69,4.547)--cycle;
\draw(-7.75,4.59)--(-7.705,4.599)--(-7.69,4.547)--(-7.716,4.542);
\filldraw[fill opacity=0.8,fill=gray!20,draw=none](-7.601,4.583)--(-7.609,4.586)--(-7.609,4.603)--(-7.582,4.601)--cycle;
\draw(-7.609,4.586)--(-7.609,4.603)--(-7.582,4.601);
\filldraw[fill opacity=0.8,fill=gray!20,draw=none](-7.628,4.596)--(-7.607,4.593)--(-7.629,4.582)--cycle;
\draw(-7.607,4.593)--(-7.629,4.582);
\filldraw[fill opacity=0.8,fill=gray!20,draw=none](-7.601,4.583)--(-7.582,4.601)--(-7.515,4.596)--(-7.528,4.563)--cycle;
\draw(-7.582,4.601)--(-7.515,4.596)--(-7.528,4.563);
\filldraw[fill opacity=0.8,fill=gray!20,draw=none](-7.592,4.592)--(-7.605,4.572)--(-7.641,4.555)--(-7.629,4.582)--(-7.607,4.593)--cycle;
\draw(-7.629,4.582)--(-7.607,4.593);
\filldraw[fill opacity=0.8,fill=gray!20](-8.213,4.264)--(-8.211,4.316)--(-8.118,4.309)--(-8.137,4.258)--cycle;
\filldraw[fill opacity=0.8,fill=gray!20,draw=none](-7.618,4.535)--(-7.644,4.534)--(-7.633,4.55)--(-7.611,4.551)--(-7.612,4.536)--cycle;
\draw(-7.633,4.55)--(-7.611,4.551)--(-7.612,4.536);
\filldraw[fill opacity=0.8,fill=gray!20,draw=none](-7.585,4.541)--(-7.612,4.535)--(-7.611,4.551)--(-7.571,4.548)--cycle;
\draw(-7.612,4.535)--(-7.611,4.551)--(-7.571,4.548);
\filldraw[fill opacity=0.8,fill=gray!20,draw=none](-7.642,4.554)--(-7.621,4.55)--(-7.644,4.549)--cycle;
\draw(-7.621,4.55)--(-7.644,4.549);
\filldraw[fill opacity=0.8,fill=gray!20,draw=none](-7.644,4.534)--(-7.644,4.549)--(-7.633,4.55)--cycle;
\draw(-7.644,4.549)--(-7.633,4.55);
\filldraw[fill opacity=0.8,fill=gray!20,draw=none](-7.644,4.554)--(-7.618,4.565)--(-7.634,4.543)--(-7.641,4.536)--(-7.644,4.535)--cycle;
\draw(-7.641,4.536)--(-7.644,4.535);
\filldraw[fill opacity=0.8,fill=gray!20,draw=none](-7.642,4.554)--(-7.644,4.549)--(-7.69,4.547)--(-7.695,4.564)--cycle;
\draw(-7.644,4.549)--(-7.69,4.547)--(-7.695,4.564);
\filldraw[fill opacity=0.8,fill=gray!20,draw=none](-7.644,4.549)--(-7.644,4.535)--(-7.648,4.534)--cycle;
\draw(-7.644,4.535)--(-7.648,4.534);
\filldraw[fill opacity=0.8,fill=gray!20,draw=none](-7.998,4.392)--(-7.669,4.542)--(-7.673,4.521)--(-7.989,4.371)--cycle;
\draw(-7.673,4.521)--(-7.989,4.371);
\filldraw[fill opacity=0.8,fill=gray!20,draw=none](-7.669,4.542)--(-7.658,4.547)--(-7.666,4.525)--(-7.673,4.521)--cycle;
\draw(-7.666,4.525)--(-7.673,4.521);
\filldraw[fill opacity=0.8,fill=gray!20,draw=none](-7.644,4.534)--(-7.683,4.534)--(-7.69,4.547)--(-7.644,4.549)--cycle;
\draw(-7.683,4.534)--(-7.69,4.547)--(-7.644,4.549);
\filldraw[fill opacity=0.8,fill=gray!20,draw=none](-7.567,4.55)--(-7.571,4.548)--(-7.588,4.549)--cycle;
\draw(-7.571,4.548)--(-7.588,4.549);
\filldraw[fill opacity=0.8,fill=gray!20,draw=none](-7.618,4.535)--(-7.612,4.536)--(-7.612,4.535)--cycle;
\draw(-7.612,4.536)--(-7.612,4.535);
\filldraw[fill opacity=0.8,fill=gray!20,draw=none](-7.585,4.541)--(-7.571,4.548)--(-7.561,4.547)--cycle;
\draw(-7.571,4.548)--(-7.561,4.547);
\filldraw[fill opacity=0.8,fill=gray!20,draw=none](-7.567,4.55)--(-7.552,4.55)--(-7.561,4.547)--(-7.571,4.548)--cycle;
\draw(-7.561,4.547)--(-7.571,4.548);
\filldraw[fill opacity=0.8,fill=gray!20,draw=none](-7.552,4.55)--(-7.567,4.55)--(-7.535,4.565)--(-7.528,4.563)--(-7.53,4.558)--cycle;
\draw(-7.528,4.563)--(-7.53,4.558);
\filldraw[fill opacity=0.8,fill=gray!20,draw=none](-7.618,4.565)--(-7.605,4.572)--(-7.608,4.567)--(-7.634,4.543)--cycle;
\filldraw[fill opacity=0.8,fill=gray!20,draw=none](-7.688,4.529)--(-7.691,4.53)--(-7.71,4.543)--(-7.69,4.547)--(-7.681,4.528)--cycle;
\draw(-7.71,4.543)--(-7.69,4.547)--(-7.681,4.528);
\filldraw[fill opacity=0.8,fill=gray!20,draw=none](-7.694,4.546)--(-7.695,4.55)--(-7.681,4.58)--(-7.672,4.577)--(-7.685,4.545)--cycle;
\draw(-7.672,4.577)--(-7.685,4.545);
\filldraw[fill opacity=0.8,fill=gray!20,draw=none](-7.655,4.553)--(-7.685,4.545)--(-7.672,4.577)--(-7.637,4.574)--cycle;
\draw(-7.685,4.545)--(-7.672,4.577);
\filldraw[fill opacity=0.8,fill=gray!20,draw=none](-7.646,4.658)--(-7.607,4.659)--(-7.608,4.63)--cycle;
\draw(-7.646,4.658)--(-7.607,4.659)--(-7.608,4.63);
\filldraw[fill opacity=0.8,fill=gray!20,draw=none](-7.609,4.603)--(-7.607,4.659)--(-7.515,4.596)--cycle;
\draw(-7.515,4.596)--(-7.609,4.603)--(-7.607,4.659);
\filldraw[fill opacity=0.8,fill=gray!20,draw=none](-7.581,4.629)--(-7.584,4.604)--(-7.607,4.593)--(-7.642,4.598)--(-7.643,4.628)--(-7.603,4.647)--cycle;
\draw(-7.584,4.604)--(-7.607,4.593);
\draw(-7.643,4.628)--(-7.603,4.647);
\filldraw[fill opacity=0.8,fill=gray!20,draw=none](-7.586,4.591)--(-7.607,4.593)--(-7.573,4.609)--cycle;
\draw(-7.607,4.593)--(-7.573,4.609);
\filldraw[fill opacity=0.8,fill=gray!20,draw=none](-7.672,4.577)--(-7.657,4.615)--(-7.604,4.596)--(-7.613,4.572)--cycle;
\draw(-7.672,4.577)--(-7.657,4.615);
\draw(-7.604,4.596)--(-7.613,4.572);
\filldraw[fill opacity=0.8,fill=gray!20](-8.317,4.478)--(-8.307,4.526)--(-8.211,4.53)--(-8.21,4.483)--cycle;
\filldraw[fill opacity=0.8,fill=gray!20](-8.211,4.53)--(-8.213,4.567)--(-8.137,4.561)--(-8.118,4.523)--cycle;
\filldraw[fill opacity=0.8,fill=gray!20,draw=none](-8.082,4.514)--(-8.118,4.523)--(-8.137,4.561)--(-8.084,4.548)--(-8.07,4.531)--cycle;
\draw(-8.082,4.514)--(-8.118,4.523)--(-8.137,4.561)--(-8.084,4.548)--(-8.07,4.531);
\filldraw[fill opacity=0.8,fill=gray!20,draw=none](-7.659,4.657)--(-7.672,4.702)--(-7.607,4.683)--(-7.607,4.659)--cycle;
\draw(-7.607,4.683)--(-7.607,4.659)--(-7.659,4.657);
\filldraw[fill opacity=0.8,fill=gray!20,draw=none](-7.672,4.702)--(-7.675,4.713)--(-7.607,4.716)--(-7.607,4.683)--cycle;
\draw(-7.675,4.713)--(-7.607,4.716)--(-7.607,4.683);
\filldraw[fill opacity=0.8,fill=gray!20,draw=none](-7.607,4.659)--(-7.503,4.652)--(-7.515,4.596)--cycle;
\draw(-7.607,4.659)--(-7.503,4.652)--(-7.515,4.596);
\filldraw[fill opacity=0.8,fill=gray!20](-7.607,4.659)--(-7.607,4.716)--(-7.499,4.709)--(-7.503,4.652)--cycle;
\filldraw[fill opacity=0.8,fill=gray!20](-7.607,4.716)--(-7.607,4.77)--(-7.503,4.763)--(-7.499,4.709)--cycle;
\filldraw[fill opacity=0.8,fill=gray!20,draw=none](-7.603,4.683)--(-7.596,4.65)--(-7.643,4.628)--(-7.668,4.673)--(-7.628,4.692)--cycle;
\draw(-7.596,4.65)--(-7.643,4.628);
\draw(-7.668,4.673)--(-7.628,4.692);
\filldraw[fill opacity=0.8,fill=gray!20,draw=none](-7.717,4.716)--(-7.708,4.707)--(-8.037,4.55)--(-8.068,4.576)--(-7.752,4.726)--cycle;
\draw(-7.708,4.707)--(-8.037,4.55);
\draw(-8.068,4.576)--(-7.752,4.726);
\filldraw[fill opacity=0.8,fill=gray!20,draw=none](-7.717,4.716)--(-7.7,4.711)--(-7.708,4.707)--cycle;
\draw(-7.7,4.711)--(-7.708,4.707);
\filldraw[fill opacity=0.8,fill=gray!20,draw=none](-7.659,4.657)--(-7.715,4.655)--(-7.718,4.711)--(-7.705,4.712)--(-7.672,4.702)--cycle;
\draw(-7.659,4.657)--(-7.715,4.655)--(-7.718,4.711)--(-7.705,4.712);
\filldraw[fill opacity=0.8,fill=gray!20,draw=none](-7.705,4.712)--(-7.718,4.717)--(-7.717,4.732)--(-7.607,4.745)--(-7.607,4.716)--cycle;
\draw(-7.718,4.717)--(-7.717,4.732);
\draw(-7.607,4.745)--(-7.607,4.716)--(-7.705,4.712);
\filldraw[fill opacity=0.8,fill=gray!20,draw=none](-7.717,4.732)--(-7.715,4.765)--(-7.607,4.77)--(-7.607,4.745)--cycle;
\draw(-7.717,4.732)--(-7.715,4.765)--(-7.607,4.77)--(-7.607,4.745);
\filldraw[fill opacity=0.8,fill=gray!20,draw=none](-7.672,4.702)--(-7.705,4.712)--(-7.675,4.713)--cycle;
\draw(-7.705,4.712)--(-7.675,4.713);
\filldraw[fill opacity=0.8,fill=gray!20,draw=none](-7.681,4.58)--(-7.629,4.692)--(-7.627,4.692)--(-7.672,4.577)--cycle;
\draw(-7.629,4.692)--(-7.627,4.692)--(-7.672,4.577);
\filldraw[fill opacity=0.8,fill=gray!20,draw=none](-7.655,4.553)--(-7.637,4.574)--(-7.613,4.572)--(-7.616,4.564)--cycle;
\draw(-7.613,4.572)--(-7.616,4.564);
\filldraw[fill opacity=0.8,fill=gray!20,draw=none](-7.723,4.553)--(-7.742,4.552)--(-7.751,4.555)--(-7.767,4.565)--(-7.779,4.584)--(-7.75,4.59)--cycle;
\draw(-7.767,4.565)--(-7.779,4.584)--(-7.75,4.59);
\filldraw[fill opacity=0.8,fill=gray!20,draw=none](-7.723,4.553)--(-7.721,4.55)--(-7.736,4.55)--(-7.742,4.552)--cycle;
\filldraw[fill opacity=0.8,fill=gray!20,draw=none](-7.694,4.546)--(-7.74,4.553)--(-7.724,4.595)--(-7.71,4.59)--cycle;
\draw(-7.74,4.553)--(-7.724,4.595);
\filldraw[fill opacity=0.8,fill=gray!20,draw=none](-7.695,4.55)--(-7.71,4.59)--(-7.681,4.58)--cycle;
\filldraw[fill opacity=0.8,fill=gray!20,draw=none](-7.619,4.715)--(-7.608,4.702)--(-7.664,4.675)--(-7.7,4.711)--(-7.652,4.733)--cycle;
\draw(-7.608,4.702)--(-7.664,4.675);
\draw(-7.7,4.711)--(-7.652,4.733);
\filldraw[fill opacity=0.8,fill=gray!20](-7.779,4.584)--(-7.797,4.639)--(-7.715,4.655)--(-7.705,4.599)--cycle;
\filldraw[fill opacity=0.8,fill=gray!20,draw=none](-7.681,4.58)--(-7.724,4.595)--(-7.678,4.711)--(-7.629,4.692)--cycle;
\draw(-7.724,4.595)--(-7.678,4.711)--(-7.629,4.692);
\filldraw[fill opacity=0.8,fill=gray!20,draw=none](-7.715,4.765)--(-7.609,4.816)--(-7.607,4.77)--cycle;
\draw(-7.609,4.816)--(-7.607,4.77)--(-7.715,4.765);
\filldraw[fill opacity=0.8,fill=gray!20](-8.213,4.567)--(-8.216,4.591)--(-8.162,4.587)--(-8.137,4.561)--cycle;
\filldraw[fill opacity=0.8,fill=gray!20,draw=none](-8.137,4.561)--(-8.139,4.563)--(-8.118,4.573)--(-8.084,4.548)--cycle;
\draw(-8.118,4.573)--(-8.084,4.548)--(-8.137,4.561)--(-8.139,4.563);
\filldraw[fill opacity=0.8,fill=gray!20,draw=none](-8.139,4.563)--(-8.162,4.587)--(-8.124,4.578)--(-8.118,4.573)--cycle;
\draw(-8.139,4.563)--(-8.162,4.587)--(-8.124,4.578)--(-8.118,4.573);
\filldraw[fill opacity=0.8,fill=gray!20,draw=none](-7.735,4.734)--(-8.092,4.564)--(-8.104,4.581)--(-7.771,4.739)--cycle;
\draw(-7.735,4.734)--(-8.092,4.564);
\draw(-8.104,4.581)--(-7.771,4.739);
\filldraw[fill opacity=0.8,fill=gray!20,draw=none](-7.717,4.716)--(-7.752,4.726)--(-7.735,4.734)--cycle;
\draw(-7.752,4.726)--(-7.735,4.734);
\filldraw[fill opacity=0.8,fill=gray!20,draw=none](-7.803,4.695)--(-7.8,4.726)--(-7.771,4.739)--(-7.717,4.732)--(-7.718,4.711)--cycle;
\draw(-7.717,4.732)--(-7.718,4.711)--(-7.803,4.695)--(-7.8,4.726);
\filldraw[fill opacity=0.8,fill=gray!20,draw=none](-7.771,4.739)--(-7.715,4.765)--(-7.717,4.732)--cycle;
\draw(-7.715,4.765)--(-7.717,4.732);
\filldraw[fill opacity=0.8,fill=gray!20,draw=none](-7.663,4.748)--(-7.648,4.735)--(-7.7,4.711)--(-7.717,4.716)--(-7.735,4.734)--(-7.688,4.757)--cycle;
\draw(-7.648,4.735)--(-7.7,4.711);
\draw(-7.735,4.734)--(-7.688,4.757);
\filldraw[fill opacity=0.8,fill=gray!20,draw=none](-7.619,4.715)--(-7.652,4.733)--(-7.64,4.739)--cycle;
\draw(-7.652,4.733)--(-7.64,4.739);
\filldraw[fill opacity=0.8,fill=gray!20,draw=none](-7.663,4.748)--(-7.64,4.739)--(-7.648,4.735)--cycle;
\draw(-7.64,4.739)--(-7.648,4.735);
\filldraw[fill opacity=0.8,fill=gray!20](-7.797,4.639)--(-7.803,4.695)--(-7.718,4.711)--(-7.715,4.655)--cycle;
\filldraw[fill opacity=0.8,fill=gray!20,draw=none](-7.705,4.712)--(-7.718,4.711)--(-7.718,4.717)--cycle;
\draw(-7.705,4.712)--(-7.718,4.711)--(-7.718,4.717);
\filldraw[fill opacity=0.8,fill=gray!20,draw=none](-7.751,4.555)--(-7.716,4.542)--(-7.726,4.54)--cycle;
\draw(-7.716,4.542)--(-7.726,4.54);
\filldraw[fill opacity=0.8,fill=gray!20](-8.137,4.258)--(-8.118,4.309)--(-8.052,4.293)--(-8.084,4.245)--cycle;
\filldraw[fill opacity=0.8,fill=gray!20,draw=none](-7.682,4.509)--(-7.743,4.466)--(-7.937,4.374)--(-7.952,4.388)--(-7.673,4.521)--cycle;
\draw(-7.743,4.466)--(-7.937,4.374);
\draw(-7.952,4.388)--(-7.673,4.521);
\filldraw[fill opacity=0.8,fill=gray!20,draw=none](-7.742,4.556)--(-7.721,4.608)--(-7.718,4.61)--(-7.74,4.553)--cycle;
\draw(-7.718,4.61)--(-7.74,4.553);
\filldraw[fill opacity=0.8,fill=gray!20,draw=none](-7.721,4.608)--(-7.678,4.711)--(-7.718,4.61)--cycle;
\draw(-7.678,4.711)--(-7.718,4.61);
\filldraw[fill opacity=0.8,fill=gray!20,draw=none](-7.64,4.739)--(-7.619,4.715)--(-7.677,4.58)--(-7.716,4.593)--(-7.721,4.608)--(-7.663,4.748)--cycle;
\draw(-7.619,4.715)--(-7.677,4.58);
\filldraw[fill opacity=0.8,fill=gray!20,draw=none](-7.608,4.598)--(-7.618,4.575)--(-7.677,4.58)--(-7.661,4.617)--cycle;
\draw(-7.608,4.598)--(-7.618,4.575);
\draw(-7.677,4.58)--(-7.661,4.617);
\filldraw[fill opacity=0.8,fill=gray!20,draw=none](-7.677,4.58)--(-7.691,4.548)--(-7.747,4.557)--(-7.729,4.597)--cycle;
\draw(-7.677,4.58)--(-7.691,4.548);
\draw(-7.747,4.557)--(-7.729,4.597);
\filldraw[fill opacity=0.8,fill=gray!20,draw=none](-7.642,4.577)--(-7.661,4.556)--(-7.691,4.548)--(-7.677,4.58)--cycle;
\draw(-7.691,4.548)--(-7.677,4.58);
\filldraw[fill opacity=0.8,fill=gray!20,draw=none](-7.592,4.592)--(-7.586,4.591)--(-7.6,4.574)--(-7.605,4.572)--cycle;
\filldraw[fill opacity=0.8,fill=gray!20,draw=none](-7.6,4.574)--(-7.616,4.564)--(-7.613,4.572)--cycle;
\draw(-7.616,4.564)--(-7.613,4.572);
\filldraw[fill opacity=0.8,fill=gray!20,draw=none](-7.586,4.591)--(-7.6,4.574)--(-7.613,4.572)--(-7.604,4.596)--cycle;
\draw(-7.613,4.572)--(-7.604,4.596);
\filldraw[fill opacity=0.8,fill=gray!20,draw=none](-7.605,4.572)--(-7.6,4.574)--(-7.608,4.567)--cycle;
\filldraw[fill opacity=0.8,fill=gray!20,draw=none](-7.642,4.577)--(-7.618,4.575)--(-7.621,4.567)--(-7.661,4.556)--cycle;
\draw(-7.618,4.575)--(-7.621,4.567);
\filldraw[fill opacity=0.8,fill=gray!20,draw=none](-7.586,4.591)--(-7.577,4.589)--(-7.6,4.574)--cycle;
\filldraw[fill opacity=0.8,fill=gray!20,draw=none](-7.586,4.591)--(-7.581,4.591)--(-7.6,4.574)--cycle;
\filldraw[fill opacity=0.8,fill=gray!20,draw=none](-7.581,4.591)--(-7.605,4.576)--(-7.618,4.575)--(-7.608,4.598)--cycle;
\draw(-7.618,4.575)--(-7.608,4.598);
\filldraw[fill opacity=0.8,fill=gray!20,draw=none](-7.605,4.576)--(-7.621,4.567)--(-7.618,4.575)--cycle;
\draw(-7.621,4.567)--(-7.618,4.575);
\filldraw[fill opacity=0.8,fill=gray!20,draw=none](-7.615,4.581)--(-7.676,4.564)--(-7.668,4.586)--cycle;
\draw(-7.676,4.564)--(-7.668,4.586);
\filldraw[fill opacity=0.8,fill=gray!20,draw=none](-7.716,4.593)--(-7.726,4.596)--(-7.721,4.608)--cycle;
\filldraw[fill opacity=0.8,fill=gray!20,draw=none](-7.668,4.586)--(-7.676,4.564)--(-7.725,4.571)--(-7.714,4.601)--cycle;
\draw(-7.668,4.586)--(-7.676,4.564);
\draw(-7.725,4.571)--(-7.714,4.601);
\filldraw[fill opacity=0.8,fill=gray!20,draw=none](-7.657,4.615)--(-7.638,4.665)--(-7.589,4.634)--(-7.604,4.596)--cycle;
\draw(-7.657,4.615)--(-7.638,4.665);
\draw(-7.589,4.634)--(-7.604,4.596);
\filldraw[fill opacity=0.8,fill=gray!20,draw=none](-7.591,4.635)--(-7.608,4.598)--(-7.661,4.617)--(-7.64,4.666)--cycle;
\draw(-7.591,4.635)--(-7.608,4.598);
\draw(-7.661,4.617)--(-7.64,4.666);
\filldraw[fill opacity=0.8,fill=gray!20,draw=none](-7.639,4.663)--(-7.668,4.586)--(-7.714,4.601)--(-7.682,4.686)--cycle;
\draw(-7.639,4.663)--(-7.668,4.586);
\draw(-7.714,4.601)--(-7.682,4.686);
\filldraw[fill opacity=0.8,fill=gray!20,draw=none](-7.607,4.6)--(-7.615,4.581)--(-7.668,4.586)--(-7.656,4.617)--cycle;
\draw(-7.607,4.6)--(-7.615,4.581);
\draw(-7.668,4.586)--(-7.656,4.617);
\filldraw[fill opacity=0.8,fill=gray!20,draw=none](-7.524,4.764)--(-7.607,4.77)--(-7.609,4.817)--(-7.567,4.814)--cycle;
\draw(-7.524,4.764)--(-7.607,4.77)--(-7.609,4.817)--(-7.567,4.814);
\filldraw[fill opacity=0.8,fill=gray!20,draw=none](-7.586,4.675)--(-7.628,4.692)--(-7.601,4.705)--cycle;
\draw(-7.628,4.692)--(-7.601,4.705);
\filldraw[fill opacity=0.8,fill=gray!20,draw=none](-7.524,4.764)--(-7.551,4.796)--(-7.52,4.799)--(-7.512,4.795)--(-7.503,4.763)--cycle;
\draw(-7.512,4.795)--(-7.503,4.763)--(-7.524,4.764);
\filldraw[fill opacity=0.8,fill=gray!20,draw=none](-7.581,4.629)--(-7.603,4.647)--(-7.577,4.659)--cycle;
\draw(-7.603,4.647)--(-7.577,4.659);
\filldraw[fill opacity=0.8,fill=gray!20,draw=none](-7.602,4.675)--(-7.578,4.661)--(-7.577,4.659)--(-7.596,4.65)--cycle;
\draw(-7.577,4.659)--(-7.596,4.65);
\filldraw[fill opacity=0.8,fill=gray!20,draw=none](-7.602,4.675)--(-7.603,4.683)--(-7.586,4.675)--(-7.578,4.661)--cycle;
\filldraw[fill opacity=0.8,fill=gray!20,draw=none](-7.51,4.82)--(-7.518,4.803)--(-7.555,4.813)--(-7.573,4.82)--(-7.563,4.843)--cycle;
\draw(-7.573,4.82)--(-7.563,4.843)--(-7.51,4.82)--(-7.518,4.803);
\filldraw[fill opacity=0.8,fill=gray!20,draw=none](-7.601,4.705)--(-7.608,4.702)--(-7.619,4.715)--cycle;
\draw(-7.601,4.705)--(-7.608,4.702);
\filldraw[fill opacity=0.8,fill=gray!20,draw=none](-7.638,4.665)--(-7.627,4.692)--(-7.574,4.671)--(-7.589,4.634)--cycle;
\draw(-7.638,4.665)--(-7.627,4.692)--(-7.574,4.671)--(-7.589,4.634);
\filldraw[fill opacity=0.8,fill=gray!20,draw=none](-7.707,4.6)--(-7.668,4.679)--(-7.639,4.663)--(-7.673,4.588)--cycle;
\draw(-7.639,4.663)--(-7.673,4.588);
\filldraw[fill opacity=0.8,fill=gray!20,draw=none](-7.673,4.588)--(-7.659,4.619)--(-7.611,4.602)--(-7.619,4.583)--cycle;
\draw(-7.673,4.588)--(-7.659,4.619);
\draw(-7.611,4.602)--(-7.619,4.583);
\filldraw[fill opacity=0.8,fill=gray!20,draw=none](-7.72,4.573)--(-7.707,4.6)--(-7.673,4.588)--(-7.682,4.567)--cycle;
\draw(-7.673,4.588)--(-7.682,4.567);
\filldraw[fill opacity=0.8,fill=gray!20,draw=none](-7.682,4.567)--(-7.673,4.588)--(-7.619,4.583)--cycle;
\draw(-7.682,4.567)--(-7.673,4.588);
\filldraw[fill opacity=0.8,fill=gray!20,draw=none](-4.582,3.05)--(-4.611,3.029)--(-7.712,4.581)--(-7.68,4.614)--(-4.575,3.06)--cycle;
\draw(-4.611,3.029)--(-7.712,4.581)--(-7.68,4.614)--(-4.575,3.06);
\filldraw[fill opacity=0.8,fill=gray!20,draw=none](-7.844,1.56)--(-7.839,1.6)--(-7.832,1.592)--cycle;
\draw(-7.844,1.56)--(-7.839,1.6)--(-7.832,1.592);
\filldraw[fill opacity=0.8,fill=gray!20,draw=none](-7.868,1.705)--(-7.916,1.725)--(-7.938,1.742)--(-7.894,1.741)--(-7.866,1.704)--cycle;
\draw(-7.894,1.741)--(-7.866,1.704)--(-7.868,1.705);
\filldraw[fill opacity=0.8,fill=gray!20,draw=none](-7.779,.902)--(-7.781,.91)--(-7.78,.909)--cycle;
\draw(-7.781,.91)--(-7.78,.909);
\filldraw[fill opacity=0.8,fill=gray!20](-7.938,1.404)--(-7.897,1.442)--(-7.881,1.425)--(-7.926,1.392)--cycle;
\filldraw[fill opacity=0.8,fill=gray!20](-8.248,1.637)--(-8.226,1.688)--(-8.195,1.709)--(-8.213,1.66)--cycle;
\filldraw[fill opacity=0.8,fill=gray!20](-8.165,1.749)--(-8.127,1.777)--(-8.085,1.786)--(-8.105,1.76)--cycle;
\filldraw[fill opacity=0.8,fill=gray!20](-8.036,1.364)--(-8.032,1.384)--(-8.005,1.382)--(-8.036,1.364)--cycle;
\filldraw[fill opacity=0.8,fill=gray!20](-8.036,1.364)--(-8.061,1.383)--(-8.032,1.384)--(-8.036,1.364)--cycle;
\filldraw[fill opacity=0.8,fill=gray!20,draw=none](-4.557,2.711)--(-4.549,2.705)--(-4.559,2.712)--cycle;
\filldraw[fill opacity=0.8,fill=gray!20,draw=none](-4.636,2.667)--(-4.596,2.654)--(-4.574,2.668)--cycle;
\draw(-4.596,2.654)--(-4.574,2.668);
\filldraw[fill opacity=0.8,fill=gray!20,draw=none](-4.543,2.714)--(-4.579,2.661)--(-4.566,2.704)--cycle;
\draw(-4.543,2.714)--(-4.579,2.661);
\filldraw[fill opacity=0.8,fill=gray!20,draw=none](-4.559,2.712)--(-4.543,2.714)--(-4.554,2.709)--cycle;
\filldraw[fill opacity=0.8,fill=gray!20](-8.093,1.372)--(-8.145,1.394)--(-8.127,1.406)--(-8.083,1.378)--cycle;
\filldraw[fill opacity=0.8,fill=gray!20,draw=none](-4.488,2.704)--(-4.491,2.702)--(-4.489,2.706)--cycle;
\draw(-4.491,2.702)--(-4.489,2.706);
\filldraw[fill opacity=0.8,fill=gray!20,draw=none](-4.485,2.734)--(-4.491,2.732)--(-4.49,2.733)--cycle;
\draw(-4.485,2.734)--(-4.491,2.732);
\filldraw[fill opacity=0.8,fill=gray!20,draw=none](-4.49,2.733)--(-4.491,2.732)--(-4.494,2.731)--cycle;
\draw(-4.491,2.732)--(-4.494,2.731);
\filldraw[fill opacity=0.8,fill=gray!20,draw=none](-4.494,2.731)--(-4.497,2.728)--(-4.501,2.73)--cycle;
\draw(-4.494,2.731)--(-4.497,2.728);
\filldraw[fill opacity=0.8,fill=gray!20,draw=none](-4.5,2.729)--(-4.488,2.704)--(-4.488,2.704)--(-4.479,2.719)--cycle;
\draw(-4.5,2.729)--(-4.488,2.704);
\filldraw[fill opacity=0.8,fill=gray!20,draw=none](-4.479,2.719)--(-4.488,2.715)--(-4.489,2.715)--(-4.498,2.727)--(-4.497,2.728)--cycle;
\draw(-4.498,2.727)--(-4.497,2.728);
\filldraw[fill opacity=0.8,fill=gray!20,draw=none](-4.489,2.715)--(-4.518,2.702)--(-4.498,2.727)--cycle;
\draw(-4.518,2.702)--(-4.498,2.727);
\filldraw[fill opacity=0.8,fill=gray!20,draw=none](-4.49,2.72)--(-4.521,2.701)--(-4.506,2.694)--(-4.488,2.704)--cycle;
\draw(-4.49,2.72)--(-4.521,2.701);
\draw(-4.506,2.694)--(-4.488,2.704);
\filldraw[fill opacity=0.8,fill=gray!20,draw=none](-7.8,.958)--(-7.804,.963)--(-7.807,.967)--cycle;
\draw(-7.8,.958)--(-7.804,.963)--(-7.807,.967);
\filldraw[fill opacity=0.8,fill=gray!20,draw=none](-7.773,.876)--(-7.779,.902)--(-7.78,.909)--(-7.774,.902)--cycle;
\draw(-7.78,.909)--(-7.774,.902);
\filldraw[fill opacity=0.8,fill=gray!20,draw=none](-4.488,2.704)--(-4.488,2.704)--(-4.491,2.702)--cycle;
\draw(-4.488,2.704)--(-4.491,2.702);
\filldraw[fill opacity=0.8,fill=gray!20,draw=none](-4.488,2.705)--(-4.488,2.704)--(-4.488,2.704)--cycle;
\draw(-4.488,2.705)--(-4.488,2.704);
\filldraw[fill opacity=0.8,fill=gray!20,draw=none](-4.46,2.691)--(-4.47,2.688)--(-4.477,2.689)--(-4.488,2.704)--cycle;
\filldraw[fill opacity=0.8,fill=gray!20,draw=none](-7.916,1.725)--(-7.868,1.705)--(-7.899,1.712)--cycle;
\draw(-7.868,1.705)--(-7.899,1.712);
\filldraw[fill opacity=0.8,fill=gray!20,draw=none](-7.788,.928)--(-7.773,.901)--(-7.78,.909)--cycle;
\draw(-7.773,.901)--(-7.78,.909);
\filldraw[fill opacity=0.8,fill=gray!20,draw=none](-4.488,2.704)--(-4.452,2.687)--(-4.472,2.73)--cycle;
\draw(-4.452,2.687)--(-4.472,2.73);
\filldraw[fill opacity=0.8,fill=gray!20,draw=none](-4.491,2.702)--(-4.508,2.69)--(-4.509,2.695)--cycle;
\filldraw[fill opacity=0.8,fill=gray!20,draw=none](-4.253,2.774)--(-4.248,2.78)--(-4.264,2.77)--cycle;
\draw(-4.253,2.774)--(-4.248,2.78)--(-4.264,2.77);
\filldraw[fill opacity=0.8,fill=gray!20,draw=none](-4.523,2.685)--(-4.527,2.687)--(-4.574,2.668)--cycle;
\filldraw[fill opacity=0.8,fill=gray!20,draw=none](-7.8,.958)--(-7.807,.967)--(-7.836,1.004)--(-7.819,.987)--(-7.785,.942)--cycle;
\draw(-7.807,.967)--(-7.836,1.004)--(-7.819,.987)--(-7.785,.942)--(-7.8,.958);
\filldraw[fill opacity=0.8,fill=gray!20,draw=none](-4.227,2.832)--(-4.226,2.832)--(-4.227,2.829)--cycle;
\draw(-4.227,2.832)--(-4.226,2.832)--(-4.227,2.829);
\filldraw[fill opacity=0.8,fill=gray!20,draw=none](-4.227,2.829)--(-4.226,2.832)--(-4.227,2.832)--cycle;
\draw(-4.227,2.829)--(-4.226,2.832)--(-4.227,2.832);
\filldraw[fill opacity=0.8,fill=gray!20,draw=none](-4.257,2.836)--(-4.213,2.826)--(-4.213,2.851)--(-4.26,2.842)--cycle;
\draw(-4.213,2.851)--(-4.26,2.842);
\filldraw[fill opacity=0.8,fill=gray!20,draw=none](-4.501,2.73)--(-4.5,2.729)--(-4.479,2.719)--(-4.472,2.73)--(-4.483,2.753)--cycle;
\draw(-4.501,2.73)--(-4.5,2.729);
\draw(-4.472,2.73)--(-4.483,2.753);
\filldraw[fill opacity=0.8,fill=gray!20,draw=none](-4.472,2.73)--(-4.489,2.714)--(-4.488,2.704)--(-4.488,2.705)--cycle;
\draw(-4.488,2.704)--(-4.488,2.705);
\filldraw[fill opacity=0.8,fill=gray!20,draw=none](-4.523,2.689)--(-4.554,2.699)--(-4.551,2.702)--cycle;
\draw(-4.554,2.699)--(-4.551,2.702);
\filldraw[fill opacity=0.8,fill=gray!20,draw=none](-4.528,2.692)--(-4.523,2.689)--(-4.509,2.695)--(-4.521,2.701)--(-4.529,2.696)--cycle;
\draw(-4.521,2.701)--(-4.529,2.696);
\filldraw[fill opacity=0.8,fill=gray!20,draw=none](-4.523,2.689)--(-4.536,2.694)--(-4.539,2.696)--cycle;
\draw(-4.523,2.689)--(-4.536,2.694)--(-4.539,2.696);
\filldraw[fill opacity=0.8,fill=gray!20,draw=none](-4.516,2.689)--(-4.536,2.694)--(-4.503,2.682)--cycle;
\draw(-4.516,2.689)--(-4.536,2.694)--(-4.503,2.682);
\filldraw[fill opacity=0.8,fill=gray!20](-8.036,1.364)--(-8.083,1.378)--(-8.061,1.383)--(-8.036,1.364)--cycle;
\filldraw[fill opacity=0.8,fill=gray!20,draw=none](-7.84,1.624)--(-7.832,1.592)--(-7.839,1.6)--cycle;
\draw(-7.832,1.592)--(-7.839,1.6);
\filldraw[fill opacity=0.8,fill=gray!20,draw=none](-7.84,1.624)--(-7.839,1.6)--(-7.844,1.639)--cycle;
\draw(-7.839,1.6)--(-7.844,1.639);
\filldraw[fill opacity=0.8,fill=gray!20,draw=none](-7.85,1.65)--(-7.844,1.639)--(-7.839,1.6)--cycle;
\draw(-7.844,1.639)--(-7.839,1.6);
\filldraw[fill opacity=0.8,fill=gray!20,draw=none](-4.315,2.811)--(-4.265,2.769)--(-4.241,2.812)--(-4.251,2.821)--cycle;
\draw(-4.315,2.811)--(-4.265,2.769);
\draw(-4.241,2.812)--(-4.251,2.821);
\filldraw[fill opacity=0.8,fill=gray!20,draw=none](-4.253,2.796)--(-4.264,2.77)--(-4.264,2.77)--(-4.248,2.78)--(-4.241,2.796)--(-4.243,2.809)--cycle;
\draw(-4.264,2.77)--(-4.248,2.78)--(-4.241,2.796);
\filldraw[fill opacity=0.8,fill=gray!20,draw=none](-8.176,.885)--(-8.152,.906)--(-8.145,.926)--(-8.149,.931)--(-8.169,.936)--(-8.186,.896)--cycle;
\draw(-8.152,.906)--(-8.145,.926)--(-8.149,.931);
\draw(-8.169,.936)--(-8.186,.896)--(-8.176,.885);
\filldraw[fill opacity=0.8,fill=gray!20,draw=none](-8.176,.885)--(-8.168,.876)--(-8.153,.902)--(-8.152,.906)--cycle;
\draw(-8.176,.885)--(-8.168,.876);
\draw(-8.153,.902)--(-8.152,.906);
\filldraw[fill opacity=0.8,fill=gray!20,draw=none](-8.17,.871)--(-8.168,.876)--(-8.169,.877)--cycle;
\draw(-8.168,.876)--(-8.169,.877);
\filldraw[fill opacity=0.8,fill=gray!20,draw=none](-8.148,.909)--(-8.152,.906)--(-8.153,.902)--cycle;
\draw(-8.152,.906)--(-8.153,.902);
\filldraw[fill opacity=0.8,fill=gray!20,draw=none](-8.48,.771)--(-8.481,.795)--(-8.456,.809)--(-8.454,.809)--cycle;
\draw(-8.456,.809)--(-8.454,.809);
\filldraw[fill opacity=0.8,fill=gray!20,draw=none](-8.481,.795)--(-8.481,.807)--(-8.456,.809)--cycle;
\draw(-8.481,.807)--(-8.456,.809);
\filldraw[fill opacity=0.8,fill=gray!20,draw=none](-8.454,.81)--(-8.454,.809)--(-8.456,.809)--cycle;
\draw(-8.454,.809)--(-8.456,.809);
\filldraw[fill opacity=0.8,fill=gray!20,draw=none](-8.524,.757)--(-8.522,.773)--(-8.513,.77)--cycle;
\draw(-8.522,.773)--(-8.513,.77);
\filldraw[fill opacity=0.8,fill=gray!20,draw=none](-8.523,.782)--(-8.499,.796)--(-8.513,.77)--(-8.522,.773)--cycle;
\draw(-8.513,.77)--(-8.522,.773);
\filldraw[fill opacity=0.8,fill=gray!20,draw=none](-8.557,.798)--(-8.561,.816)--(-8.553,.828)--(-8.503,.837)--(-8.516,.806)--cycle;
\draw(-8.561,.816)--(-8.553,.828)--(-8.503,.837)--(-8.516,.806)--(-8.557,.798);
\filldraw[fill opacity=0.8,fill=gray!20,draw=none](-8.52,.834)--(-8.553,.828)--(-8.529,.845)--cycle;
\draw(-8.52,.834)--(-8.553,.828)--(-8.529,.845);
\filldraw[fill opacity=0.8,fill=gray!20,draw=none](-8.578,.797)--(-8.558,.824)--(-8.553,.828)--(-8.556,.823)--cycle;
\draw(-8.558,.824)--(-8.553,.828)--(-8.556,.823);
\filldraw[fill opacity=0.8,fill=gray!20,draw=none](-8.549,.831)--(-8.532,.824)--(-8.522,.773)--(-8.589,.801)--cycle;
\draw(-8.549,.831)--(-8.532,.824);
\draw(-8.522,.773)--(-8.589,.801);
\filldraw[fill opacity=0.8,fill=gray!20,draw=none](-8.446,.829)--(-8.454,.81)--(-8.456,.809)--(-8.516,.806)--(-8.511,.818)--cycle;
\draw(-8.456,.809)--(-8.516,.806)--(-8.511,.818);
\filldraw[fill opacity=0.8,fill=gray!20,draw=none](-8.523,.782)--(-8.529,.809)--(-8.505,.813)--(-8.492,.807)--(-8.499,.796)--cycle;
\draw(-8.505,.813)--(-8.492,.807);
\filldraw[fill opacity=0.8,fill=gray!20](-8.075,.902)--(-8.546,.75)--(-8.522,.788)--(-8.052,.94)--cycle;
\filldraw[fill opacity=0.8,fill=gray!20,draw=none](-7.878,1.693)--(-7.899,1.712)--(-7.889,1.71)--cycle;
\draw(-7.899,1.712)--(-7.889,1.71);
\filldraw[fill opacity=0.8,fill=gray!20,draw=none](-4.549,2.705)--(-4.54,2.698)--(-4.539,2.696)--cycle;
\draw(-4.54,2.698)--(-4.539,2.696);
\filldraw[fill opacity=0.8,fill=gray!20,draw=none](-4.502,2.745)--(-4.534,2.734)--(-4.517,2.723)--(-4.501,2.729)--cycle;
\draw(-4.502,2.745)--(-4.534,2.734);
\draw(-4.517,2.723)--(-4.501,2.729);
\filldraw[fill opacity=0.8,fill=gray!20,draw=none](-4.448,2.671)--(-4.464,2.672)--(-4.469,2.673)--(-4.462,2.673)--cycle;
\draw(-4.448,2.671)--(-4.464,2.672);
\draw(-4.469,2.673)--(-4.462,2.673);
\filldraw[fill opacity=0.8,fill=gray!20](-8.066,1.036)--(-8.022,1.051)--(-8,1.056)--(-8.024,1.044)--cycle;
\filldraw[fill opacity=0.8,fill=gray!20,draw=none](-8.125,.683)--(-8.129,.688)--(-8.126,.686)--cycle;
\draw(-8.125,.683)--(-8.129,.688)--(-8.126,.686);
\filldraw[fill opacity=0.8,fill=gray!20,draw=none](-8.125,.683)--(-8.113,.671)--(-8.119,.678)--cycle;
\draw(-8.125,.683)--(-8.113,.671)--(-8.119,.678);
\filldraw[fill opacity=0.8,fill=gray!20](-8.025,.625)--(-8.073,.641)--(-8.084,.653)--(-8.031,.631)--cycle;
\filldraw[fill opacity=0.8,fill=gray!20,draw=none](-8.071,.641)--(-8.061,.658)--(-8.113,.671)--(-8.073,.641)--cycle;
\draw(-8.061,.658)--(-8.113,.671)--(-8.073,.641)--(-8.071,.641);
\filldraw[fill opacity=0.8,fill=gray!20,draw=none](-8.061,.658)--(-8.088,.678)--(-8.119,.678)--(-8.113,.671)--cycle;
\draw(-8.119,.678)--(-8.113,.671)--(-8.061,.658);
\filldraw[fill opacity=0.8,fill=gray!20,draw=none](-8.071,.678)--(-8.11,.695)--(-8.088,.678)--cycle;
\filldraw[fill opacity=0.8,fill=gray!20,draw=none](-8.071,.678)--(-8.088,.678)--(-8.061,.658)--(-8.06,.659)--(-8.07,.677)--cycle;
\draw(-8.06,.659)--(-8.07,.677);
\filldraw[fill opacity=0.8,fill=gray!20,draw=none](-8.438,.57)--(-8.437,.588)--(-8.384,.584)--(-8.375,.581)--(-8.376,.58)--cycle;
\draw(-8.438,.57)--(-8.437,.588)--(-8.384,.584);
\draw(-8.375,.581)--(-8.376,.58);
\filldraw[fill opacity=0.8,fill=gray!20,draw=none](-8.384,.584)--(-8.411,.586)--(-8.419,.595)--cycle;
\draw(-8.384,.584)--(-8.411,.586);
\filldraw[fill opacity=0.8,fill=gray!20,draw=none](-8.398,.548)--(-8.44,.551)--(-8.438,.57)--(-8.376,.58)--(-8.393,.551)--cycle;
\draw(-8.398,.548)--(-8.44,.551)--(-8.438,.57);
\draw(-8.376,.58)--(-8.393,.551);
\filldraw[fill opacity=0.8,fill=gray!20,draw=none](-8.486,.549)--(-8.491,.56)--(-8.438,.57)--(-8.44,.551)--cycle;
\draw(-8.438,.57)--(-8.44,.551)--(-8.486,.549)--(-8.491,.56);
\filldraw[fill opacity=0.8,fill=gray!20](-7.974,.709)--(-8.445,.557)--(-8.487,.573)--(-8.016,.725)--cycle;
\filldraw[fill opacity=0.8,fill=gray!20,draw=none](-4.539,2.696)--(-4.551,2.702)--(-4.549,2.705)--cycle;
\draw(-4.551,2.702)--(-4.549,2.705);
\filldraw[fill opacity=0.8,fill=gray!20,draw=none](-4.554,2.709)--(-4.549,2.705)--(-4.539,2.696)--(-4.536,2.694)--(-4.554,2.707)--cycle;
\draw(-4.539,2.696)--(-4.536,2.694)--(-4.554,2.707);
\filldraw[fill opacity=0.8,fill=gray!20,draw=none](-4.67,2.942)--(-4.668,2.945)--(-4.666,2.944)--cycle;
\draw(-4.668,2.945)--(-4.666,2.944);
\filldraw[fill opacity=0.8,fill=gray!20,draw=none](-4.559,2.712)--(-4.554,2.709)--(-4.566,2.704)--(-4.564,2.712)--cycle;
\filldraw[fill opacity=0.8,fill=gray!20,draw=none](-4.559,2.712)--(-4.554,2.709)--(-4.554,2.707)--(-4.574,2.721)--cycle;
\draw(-4.554,2.707)--(-4.574,2.721);
\filldraw[fill opacity=0.8,fill=gray!20,draw=none](-7.95,1.758)--(-7.958,1.767)--(-7.948,1.777)--(-7.938,1.775)--(-7.897,1.745)--cycle;
\draw(-7.948,1.777)--(-7.938,1.775)--(-7.897,1.745)--(-7.95,1.758)--(-7.958,1.767);
\filldraw[fill opacity=0.8,fill=gray!20](-8.036,1.364)--(-8.005,1.382)--(-7.985,1.377)--(-8.036,1.364)--cycle;
\filldraw[fill opacity=0.8,fill=gray!20,draw=none](-4.67,2.942)--(-4.714,2.922)--(-4.793,2.961)--(-4.699,2.961)--(-4.668,2.945)--cycle;
\draw(-4.714,2.922)--(-4.793,2.961);
\draw(-4.699,2.961)--(-4.668,2.945);
\filldraw[fill opacity=0.8,fill=gray!20,draw=none](-4.442,3.109)--(-4.448,3.106)--(-4.438,3.104)--(-4.438,3.104)--(-4.436,3.107)--cycle;
\draw(-4.448,3.106)--(-4.438,3.104)--(-4.438,3.104)--(-4.436,3.107);
\filldraw[fill opacity=0.8,fill=gray!20,draw=none](-4.442,3.109)--(-4.463,3.108)--(-4.448,3.106)--cycle;
\draw(-4.442,3.109)--(-4.463,3.108)--(-4.448,3.106);
\filldraw[fill opacity=0.8,fill=gray!20,draw=none](-7.773,.901)--(-7.788,.928)--(-7.8,.958)--(-7.785,.942)--(-7.763,.89)--cycle;
\draw(-7.8,.958)--(-7.785,.942)--(-7.763,.89)--(-7.773,.901);
\filldraw[fill opacity=0.8,fill=gray!20](-8.129,.991)--(-8.084,1.024)--(-8.066,1.036)--(-8.104,1.008)--cycle;
\filldraw[fill opacity=0.8,fill=gray!20](-7.974,.623)--(-8.025,.625)--(-8.031,.631)--(-7.974,.623)--cycle;
\filldraw[fill opacity=0.8,fill=gray!20](-7.985,1.377)--(-7.938,1.404)--(-7.926,1.392)--(-7.979,1.371)--cycle;
\filldraw[fill opacity=0.8,fill=gray!20,draw=none](-4.564,2.712)--(-4.565,2.707)--(-4.569,2.711)--cycle;
\filldraw[fill opacity=0.8,fill=gray!20,draw=none](-4.694,2.911)--(-4.714,2.922)--(-4.67,2.942)--cycle;
\draw(-4.694,2.911)--(-4.714,2.922);
\filldraw[fill opacity=0.8,fill=gray!20,draw=none](-4.572,3.058)--(-4.582,3.05)--(-4.593,3.044)--(-4.549,3.076)--cycle;
\draw(-4.582,3.05)--(-4.593,3.044)--(-4.549,3.076);
\filldraw[fill opacity=0.8,fill=gray!20,draw=none](-4.523,2.689)--(-4.514,2.685)--(-4.527,2.676)--(-4.557,2.693)--(-4.554,2.699)--cycle;
\draw(-4.557,2.693)--(-4.554,2.699);
\filldraw[fill opacity=0.8,fill=gray!20,draw=none](-7.853,1.656)--(-7.878,1.693)--(-7.852,1.67)--(-7.846,1.655)--cycle;
\draw(-7.852,1.67)--(-7.846,1.655)--(-7.853,1.656);
\filldraw[fill opacity=0.8,fill=gray!20,draw=none](-7.853,1.656)--(-7.851,1.656)--(-7.85,1.65)--cycle;
\draw(-7.853,1.656)--(-7.851,1.656);
\filldraw[fill opacity=0.8,fill=gray!20](-7.974,.623)--(-7.927,.624)--(-7.949,.619)--(-7.974,.623)--cycle;
\filldraw[fill opacity=0.8,fill=gray!20](-7.974,.623)--(-7.918,.63)--(-7.927,.624)--(-7.974,.623)--cycle;
\filldraw[fill opacity=0.8,fill=gray!20](-7.974,.623)--(-8.006,.62)--(-8.025,.625)--(-7.974,.623)--cycle;
\filldraw[fill opacity=0.8,fill=gray!20](-7.974,.623)--(-7.978,.618)--(-8.006,.62)--(-7.974,.623)--cycle;
\filldraw[fill opacity=0.8,fill=gray!20](-7.974,.623)--(-7.949,.619)--(-7.978,.618)--(-7.974,.623)--cycle;
\filldraw[fill opacity=0.8,fill=gray!20](-7.914,1.043)--(-7.943,1.055)--(-7.924,1.05)--(-7.876,1.034)--cycle;
\filldraw[fill opacity=0.8,fill=gray!20,draw=none](-4.636,2.667)--(-4.739,2.603)--(-4.694,2.594)--(-4.596,2.654)--cycle;
\draw(-4.636,2.667)--(-4.739,2.603);
\draw(-4.694,2.594)--(-4.596,2.654);
\filldraw[fill opacity=0.8,fill=gray!20,draw=none](-4.342,2.845)--(-4.343,2.864)--(-4.352,2.87)--(-4.379,2.864)--cycle;
\draw(-4.352,2.87)--(-4.379,2.864);
\filldraw[fill opacity=0.8,fill=gray!20,draw=none](-4.343,2.856)--(-4.347,2.838)--(-4.347,2.822)--(-4.342,2.845)--cycle;
\filldraw[fill opacity=0.8,fill=gray!20,draw=none](-4.348,2.857)--(-4.351,2.841)--(-4.35,2.825)--(-4.346,2.847)--cycle;
\filldraw[fill opacity=0.8,fill=gray!20,draw=none](-4.246,2.687)--(-4.318,2.813)--(-4.379,2.864)--cycle;
\draw(-4.318,2.813)--(-4.379,2.864);
\filldraw[fill opacity=0.8,fill=gray!20,draw=none](-4.569,2.711)--(-4.565,2.707)--(-4.568,2.7)--(-4.609,2.707)--cycle;
\filldraw[fill opacity=0.8,fill=gray!20,draw=none](-7.85,1.65)--(-7.851,1.656)--(-7.846,1.655)--(-7.844,1.639)--cycle;
\draw(-7.851,1.656)--(-7.846,1.655)--(-7.844,1.639);
\filldraw[fill opacity=0.8,fill=gray!20,draw=none](-4.501,2.73)--(-4.498,2.727)--(-4.518,2.702)--(-4.522,2.701)--(-4.524,2.728)--cycle;
\draw(-4.498,2.727)--(-4.518,2.702);
\filldraw[fill opacity=0.8,fill=gray!20,draw=none](-4.484,2.752)--(-4.487,2.751)--(-4.502,2.741)--(-4.501,2.73)--cycle;
\draw(-4.484,2.752)--(-4.487,2.751);
\filldraw[fill opacity=0.8,fill=gray!20,draw=none](-4.497,2.728)--(-4.498,2.727)--(-4.501,2.73)--cycle;
\draw(-4.497,2.728)--(-4.498,2.727);
\filldraw[fill opacity=0.8,fill=gray!20,draw=none](-4.485,3.104)--(-4.451,3.104)--(-4.448,3.106)--(-4.463,3.108)--cycle;
\draw(-4.448,3.106)--(-4.463,3.108)--(-4.485,3.104)--(-4.451,3.104);
\filldraw[fill opacity=0.8,fill=gray!20,draw=none](-7.878,1.693)--(-7.889,1.71)--(-7.868,1.705)--(-7.866,1.704)--(-7.852,1.67)--cycle;
\draw(-7.889,1.71)--(-7.868,1.705);
\draw(-7.866,1.704)--(-7.852,1.67);
\filldraw[fill opacity=0.8,fill=gray!20,draw=none](-7.84,1.624)--(-7.844,1.639)--(-7.846,1.655)--(-7.841,1.649)--cycle;
\draw(-7.844,1.639)--(-7.846,1.655)--(-7.841,1.649);
\filldraw[fill opacity=0.8,fill=gray!20,draw=none](-4.523,2.689)--(-4.532,2.694)--(-4.535,2.692)--(-4.527,2.687)--cycle;
\draw(-4.532,2.694)--(-4.535,2.692);
\filldraw[fill opacity=0.8,fill=gray!20,draw=none](-4.513,2.686)--(-4.514,2.685)--(-4.523,2.689)--cycle;
\filldraw[fill opacity=0.8,fill=gray!20,draw=none](-4.65,2.948)--(-4.649,2.949)--(-4.649,2.95)--cycle;
\draw(-4.65,2.948)--(-4.649,2.949);
\filldraw[fill opacity=0.8,fill=gray!20,draw=none](-4.469,2.673)--(-4.469,2.673)--(-4.469,2.673)--cycle;
\draw(-4.469,2.673)--(-4.469,2.673);
\filldraw[fill opacity=0.8,fill=gray!20,draw=none](-4.464,2.672)--(-4.469,2.673)--(-4.469,2.673)--cycle;
\draw(-4.464,2.672)--(-4.469,2.673)--(-4.469,2.673);
\filldraw[fill opacity=0.8,fill=gray!20,draw=none](-4.243,2.809)--(-4.227,2.829)--(-4.227,2.832)--(-4.245,2.819)--cycle;
\draw(-4.227,2.832)--(-4.245,2.819);
\filldraw[fill opacity=0.8,fill=gray!20,draw=none](-4.514,2.669)--(-4.535,2.637)--(-4.58,2.658)--(-4.579,2.661)--(-4.557,2.693)--cycle;
\draw(-4.514,2.669)--(-4.535,2.637);
\draw(-4.579,2.661)--(-4.557,2.693);
\filldraw[fill opacity=0.8,fill=gray!20,draw=none](-4.513,2.686)--(-4.507,2.685)--(-4.506,2.68)--(-4.514,2.669)--(-4.527,2.676)--cycle;
\draw(-4.506,2.68)--(-4.514,2.669);
\filldraw[fill opacity=0.8,fill=gray!20](-7.918,.63)--(-7.865,.65)--(-7.883,.639)--(-7.927,.624)--cycle;
\filldraw[fill opacity=0.8,fill=gray!20,draw=none](-4.453,2.687)--(-4.46,2.676)--(-4.468,2.677)--(-4.477,2.689)--cycle;
\draw(-4.453,2.687)--(-4.46,2.676);
\filldraw[fill opacity=0.8,fill=gray!20,draw=none](-4.452,2.687)--(-4.423,2.686)--(-4.426,2.683)--(-4.429,2.681)--(-4.444,2.678)--(-4.46,2.676)--(-4.453,2.687)--cycle;
\draw(-4.423,2.686)--(-4.426,2.683);
\draw(-4.46,2.676)--(-4.453,2.687);
\filldraw[fill opacity=0.8,fill=gray!20,draw=none](-4.464,2.672)--(-4.445,2.672)--(-4.442,2.672)--(-4.444,2.681)--(-4.466,2.682)--(-4.469,2.673)--cycle;
\draw(-4.442,2.672)--(-4.444,2.681)--(-4.466,2.682);
\draw(-4.469,2.673)--(-4.464,2.672);
\filldraw[fill opacity=0.8,fill=gray!20,draw=none](-4.642,2.94)--(-4.641,2.95)--(-4.649,2.949)--(-4.65,2.948)--cycle;
\draw(-4.649,2.949)--(-4.65,2.948)--(-4.642,2.94);
\filldraw[fill opacity=0.8,fill=gray!20,draw=none](-4.641,2.927)--(-4.642,2.94)--(-4.65,2.948)--cycle;
\draw(-4.642,2.94)--(-4.65,2.948);
\filldraw[fill opacity=0.8,fill=gray!20,draw=none](-4.466,2.682)--(-4.498,2.685)--(-4.469,2.673)--cycle;
\draw(-4.466,2.682)--(-4.498,2.685)--(-4.469,2.673);
\filldraw[fill opacity=0.8,fill=gray!20,draw=none](-4.438,2.89)--(-4.438,2.891)--(-4.438,2.89)--cycle;
\filldraw[fill opacity=0.8,fill=gray!20,draw=none](-4.438,2.848)--(-4.431,2.839)--(-4.43,2.838)--(-4.438,2.89)--(-4.469,2.898)--(-4.463,2.887)--cycle;
\draw(-4.431,2.839)--(-4.43,2.838);
\draw(-4.469,2.898)--(-4.463,2.887)--(-4.438,2.848);
\filldraw[fill opacity=0.8,fill=gray!20,draw=none](-4.617,2.953)--(-4.608,2.979)--(-4.628,3)--(-4.649,2.949)--cycle;
\draw(-4.617,2.953)--(-4.608,2.979)--(-4.628,3)--(-4.649,2.949);
\filldraw[fill opacity=0.8,fill=gray!20,draw=none](-4.379,2.864)--(-4.418,2.894)--(-4.456,2.921)--(-4.47,2.93)--(-4.48,2.936)--(-4.438,2.89)--cycle;
\draw(-4.379,2.864)--(-4.418,2.894)--(-4.456,2.921);
\draw(-4.47,2.93)--(-4.48,2.936);
\filldraw[fill opacity=0.8,fill=gray!20,draw=none](-4.527,2.687)--(-4.535,2.692)--(-4.574,2.668)--cycle;
\draw(-4.535,2.692)--(-4.574,2.668);
\filldraw[fill opacity=0.8,fill=gray!20,draw=none](-4.504,2.684)--(-4.506,2.68)--(-4.507,2.685)--cycle;
\draw(-4.504,2.684)--(-4.506,2.68);
\filldraw[fill opacity=0.8,fill=gray!20,draw=none](-4.48,2.677)--(-4.5,2.678)--(-4.506,2.681)--(-4.504,2.684)--cycle;
\draw(-4.506,2.681)--(-4.504,2.684);
\filldraw[fill opacity=0.8,fill=gray!20,draw=none](-4.23,2.848)--(-4.231,2.828)--(-4.227,2.832)--cycle;
\draw(-4.231,2.828)--(-4.227,2.832);
\filldraw[fill opacity=0.8,fill=gray!20,draw=none](-4.445,2.672)--(-4.464,2.672)--(-4.448,2.671)--cycle;
\draw(-4.464,2.672)--(-4.448,2.671);
\filldraw[fill opacity=0.8,fill=gray!20,draw=none](-7.481,4.566)--(-7.504,4.559)--(-7.528,4.563)--(-7.515,4.596)--cycle;
\draw(-7.528,4.563)--(-7.515,4.596);
\filldraw[fill opacity=0.8,fill=gray!20,draw=none](-7.552,4.55)--(-7.53,4.558)--(-7.533,4.551)--cycle;
\draw(-7.53,4.558)--(-7.533,4.551);
\filldraw[fill opacity=0.8,fill=gray!20,draw=none](-7.504,4.559)--(-7.533,4.551)--(-7.528,4.563)--cycle;
\draw(-7.533,4.551)--(-7.528,4.563);
\filldraw[fill opacity=0.8,fill=gray!20,draw=none](-7.574,4.624)--(-7.549,4.606)--(-7.572,4.592)--cycle;
\filldraw[fill opacity=0.8,fill=gray!20,draw=none](-7.581,4.629)--(-7.568,4.619)--(-7.567,4.612)--(-7.584,4.604)--cycle;
\draw(-7.567,4.612)--(-7.584,4.604);
\filldraw[fill opacity=0.8,fill=gray!20,draw=none](-7.574,4.624)--(-7.572,4.592)--(-7.577,4.589)--(-7.604,4.596)--(-7.589,4.634)--cycle;
\draw(-7.604,4.596)--(-7.589,4.634);
\filldraw[fill opacity=0.8,fill=gray!20,draw=none](-7.564,4.616)--(-7.552,4.607)--(-7.581,4.591)--(-7.581,4.591)--cycle;
\filldraw[fill opacity=0.8,fill=gray!20,draw=none](-7.559,4.611)--(-7.581,4.591)--(-7.586,4.591)--(-7.573,4.609)--(-7.562,4.614)--cycle;
\draw(-7.573,4.609)--(-7.562,4.614);
\filldraw[fill opacity=0.8,fill=gray!20,draw=none](-7.564,4.616)--(-7.581,4.591)--(-7.608,4.598)--(-7.591,4.635)--cycle;
\draw(-7.608,4.598)--(-7.591,4.635);
\filldraw[fill opacity=0.8,fill=gray!20,draw=none](-7.6,4.574)--(-7.577,4.589)--(-7.554,4.583)--(-7.555,4.58)--cycle;
\draw(-7.554,4.583)--(-7.555,4.58);
\filldraw[fill opacity=0.8,fill=gray!20,draw=none](-7.59,4.578)--(-7.6,4.574)--(-7.581,4.591)--cycle;
\filldraw[fill opacity=0.8,fill=gray!20,draw=none](-7.59,4.578)--(-7.605,4.576)--(-7.582,4.59)--cycle;
\filldraw[fill opacity=0.8,fill=gray!20,draw=none](-7.594,4.635)--(-7.607,4.6)--(-7.656,4.617)--(-7.639,4.663)--cycle;
\draw(-7.594,4.635)--(-7.607,4.6);
\draw(-7.656,4.617)--(-7.639,4.663);
\filldraw[fill opacity=0.8,fill=gray!20,draw=none](-7.659,4.619)--(-7.639,4.663)--(-7.596,4.636)--(-7.611,4.602)--cycle;
\draw(-7.659,4.619)--(-7.639,4.663);
\draw(-7.596,4.636)--(-7.611,4.602);
\filldraw[fill opacity=0.8,fill=gray!20,draw=none](-7.577,4.589)--(-7.549,4.606)--(-7.546,4.604)--(-7.554,4.583)--cycle;
\draw(-7.546,4.604)--(-7.554,4.583);
\filldraw[fill opacity=0.8,fill=gray!20,draw=none](-7.59,4.578)--(-7.582,4.59)--(-7.581,4.591)--(-7.558,4.585)--(-7.559,4.582)--cycle;
\draw(-7.558,4.585)--(-7.559,4.582);
\filldraw[fill opacity=0.8,fill=gray!20,draw=none](-7.581,4.591)--(-7.552,4.607)--(-7.549,4.605)--(-7.558,4.585)--cycle;
\draw(-7.549,4.605)--(-7.558,4.585);
\filldraw[fill opacity=0.8,fill=gray!20,draw=none](-7.568,4.59)--(-7.57,4.588)--(-7.59,4.578)--(-7.581,4.591)--cycle;
\filldraw[fill opacity=0.8,fill=gray!20,draw=none](-7.557,4.604)--(-7.568,4.59)--(-7.581,4.591)--cycle;
\filldraw[fill opacity=0.8,fill=gray!20,draw=none](-7.555,4.607)--(-7.557,4.604)--(-7.581,4.591)--(-7.559,4.611)--cycle;
\filldraw[fill opacity=0.8,fill=gray!20,draw=none](-7.591,4.596)--(-7.568,4.59)--(-7.57,4.588)--(-7.615,4.581)--cycle;
\filldraw[fill opacity=0.8,fill=gray!20,draw=none](-7.591,4.596)--(-7.615,4.581)--(-7.607,4.6)--cycle;
\draw(-7.615,4.581)--(-7.607,4.6);
\filldraw[fill opacity=0.8,fill=gray!20,draw=none](-7.568,4.619)--(-7.562,4.614)--(-7.567,4.612)--cycle;
\draw(-7.562,4.614)--(-7.567,4.612);
\filldraw[fill opacity=0.8,fill=gray!20,draw=none](-7.564,4.613)--(-7.591,4.596)--(-7.607,4.6)--(-7.594,4.635)--cycle;
\draw(-7.607,4.6)--(-7.594,4.635);
\filldraw[fill opacity=0.8,fill=gray!20,draw=none](-7.566,4.614)--(-7.594,4.598)--(-7.611,4.602)--(-7.596,4.636)--cycle;
\draw(-7.611,4.602)--(-7.596,4.636);
\filldraw[fill opacity=0.8,fill=gray!20,draw=none](-7.594,4.598)--(-7.619,4.583)--(-7.611,4.602)--cycle;
\draw(-7.619,4.583)--(-7.611,4.602);
\filldraw[fill opacity=0.8,fill=gray!20,draw=none](-7.619,4.583)--(-7.594,4.598)--(-7.566,4.59)--cycle;
\filldraw[fill opacity=0.8,fill=gray!20,draw=none](-7.568,4.59)--(-7.591,4.596)--(-7.564,4.613)--(-7.556,4.608)--(-7.557,4.604)--cycle;
\draw(-7.556,4.608)--(-7.557,4.604);
\filldraw[fill opacity=0.8,fill=gray!20,draw=none](-7.568,4.59)--(-7.557,4.604)--(-7.563,4.589)--cycle;
\draw(-7.557,4.604)--(-7.563,4.589);
\filldraw[fill opacity=0.8,fill=gray!20,draw=none](-7.531,4.59)--(-7.55,4.582)--(-7.554,4.583)--(-7.546,4.604)--cycle;
\draw(-7.554,4.583)--(-7.546,4.604);
\filldraw[fill opacity=0.8,fill=gray!20,draw=none](-7.534,4.592)--(-7.554,4.584)--(-7.558,4.585)--(-7.549,4.605)--cycle;
\draw(-7.558,4.585)--(-7.549,4.605);
\filldraw[fill opacity=0.8,fill=gray!20,draw=none](-7.557,4.604)--(-7.555,4.607)--(-7.554,4.605)--cycle;
\filldraw[fill opacity=0.8,fill=gray!20,draw=none](-7.544,4.597)--(-7.563,4.589)--(-7.556,4.608)--cycle;
\draw(-7.563,4.589)--(-7.556,4.608);
\filldraw[fill opacity=0.8,fill=gray!20,draw=none](-7.557,4.604)--(-7.554,4.605)--(-7.564,4.59)--(-7.565,4.59)--(-7.568,4.59)--cycle;
\filldraw[fill opacity=0.8,fill=gray!20,draw=none](-7.594,4.598)--(-7.566,4.614)--(-7.558,4.608)--(-7.566,4.59)--cycle;
\draw(-7.558,4.608)--(-7.566,4.59);
\filldraw[fill opacity=0.8,fill=gray!20,draw=none](-7.55,4.582)--(-7.555,4.58)--(-7.554,4.583)--cycle;
\draw(-7.555,4.58)--(-7.554,4.583);
\filldraw[fill opacity=0.8,fill=gray!20,draw=none](-7.554,4.584)--(-7.559,4.582)--(-7.558,4.585)--cycle;
\draw(-7.559,4.582)--(-7.558,4.585);
\filldraw[fill opacity=0.8,fill=gray!20,draw=none](-7.568,4.59)--(-7.563,4.589)--(-7.57,4.588)--cycle;
\filldraw[fill opacity=0.8,fill=gray!20,draw=none](-7.568,4.59)--(-7.565,4.59)--(-7.57,4.588)--cycle;
\filldraw[fill opacity=0.8,fill=gray!20,draw=none](-4.575,3.06)--(-7.68,4.614)--(-7.649,4.631)--(-4.561,3.086)--cycle;
\draw(-4.575,3.06)--(-7.68,4.614)--(-7.649,4.631)--(-4.561,3.086);
\filldraw[fill opacity=0.8,fill=gray!20,draw=none](-4.215,3.011)--(-4.186,3.025)--(-4.156,3.068)--(-4.192,3.058)--(-4.212,3.029)--cycle;
\draw(-4.186,3.025)--(-4.156,3.068);
\draw(-4.192,3.058)--(-4.212,3.029);
\filldraw[fill opacity=0.8,fill=gray!20,draw=none](-4.544,3.078)--(-4.519,3.083)--(-4.521,3.087)--(-4.546,3.078)--cycle;
\draw(-4.521,3.087)--(-4.546,3.078);
\filldraw[fill opacity=0.8,fill=gray!20,draw=none](-4.519,3.083)--(-4.544,3.078)--(-4.516,3.076)--cycle;
\filldraw[fill opacity=0.8,fill=gray!20,draw=none](-7.504,4.559)--(-7.481,4.566)--(-7.467,4.553)--cycle;
\filldraw[fill opacity=0.8,fill=gray!20,draw=none](-7.55,4.582)--(-7.531,4.59)--(-7.516,4.578)--cycle;
\filldraw[fill opacity=0.8,fill=gray!20,draw=none](-7.554,4.584)--(-7.534,4.592)--(-7.52,4.58)--cycle;
\filldraw[fill opacity=0.8,fill=gray!20,draw=none](-7.544,4.597)--(-7.529,4.584)--(-7.563,4.589)--cycle;
\filldraw[fill opacity=0.8,fill=gray!20,draw=none](-7.561,4.589)--(-7.565,4.59)--(-7.557,4.593)--cycle;
\filldraw[fill opacity=0.8,fill=gray!20,draw=none](-7.549,4.601)--(-7.546,4.598)--(-7.557,4.593)--cycle;
\filldraw[fill opacity=0.8,fill=gray!20,draw=none](-7.554,4.605)--(-7.546,4.598)--(-7.564,4.59)--cycle;
\filldraw[fill opacity=0.8,fill=gray!20,draw=none](-7.549,4.601)--(-7.557,4.593)--(-7.566,4.59)--(-7.558,4.608)--cycle;
\draw(-7.566,4.59)--(-7.558,4.608);
\filldraw[fill opacity=0.8,fill=gray!20,draw=none](-7.561,4.589)--(-7.566,4.59)--(-7.557,4.593)--cycle;
\filldraw[fill opacity=0.8,fill=gray!20,draw=none](-7.543,4.587)--(-7.561,4.589)--(-7.557,4.593)--(-7.546,4.598)--(-7.537,4.59)--cycle;
\filldraw[fill opacity=0.8,fill=gray!20,draw=none](-7.54,4.592)--(-7.544,4.588)--(-7.561,4.589)--(-7.557,4.593)--(-7.546,4.598)--cycle;
\filldraw[fill opacity=0.8,fill=gray!20,draw=none](-7.543,4.587)--(-7.537,4.59)--(-7.531,4.585)--cycle;
\filldraw[fill opacity=0.8,fill=gray!20,draw=none](-7.54,4.592)--(-7.536,4.588)--(-7.544,4.588)--cycle;
\filldraw[fill opacity=0.8,fill=gray!20,draw=none](-7.546,4.589)--(-7.54,4.592)--(-7.537,4.589)--(-7.537,4.588)--cycle;
\draw(-7.546,4.589)--(-7.54,4.592);
\draw(-7.537,4.589)--(-7.537,4.588);
\filldraw[fill opacity=0.8,fill=gray!20,draw=none](-7.726,4.596)--(-7.729,4.597)--(-7.663,4.748)--cycle;
\draw(-7.729,4.597)--(-7.663,4.748);
\filldraw[fill opacity=0.8,fill=gray!20,draw=none](-7.715,4.765)--(-7.705,4.813)--(-7.609,4.817)--(-7.609,4.816)--cycle;
\draw(-7.715,4.765)--(-7.705,4.813)--(-7.609,4.817)--(-7.609,4.816);
\filldraw[fill opacity=0.8,fill=gray!20,draw=none](-7.567,4.814)--(-7.609,4.817)--(-7.61,4.842)--cycle;
\draw(-7.567,4.814)--(-7.609,4.817)--(-7.61,4.842);
\filldraw[fill opacity=0.8,fill=gray!20,draw=none](-7.688,4.757)--(-7.735,4.734)--(-7.771,4.739)--(-7.744,4.752)--cycle;
\draw(-7.688,4.757)--(-7.735,4.734);
\draw(-7.771,4.739)--(-7.744,4.752);
\filldraw[fill opacity=0.8,fill=gray!20,draw=none](-7.563,4.843)--(-7.564,4.84)--(-7.622,4.843)--(-7.612,4.865)--cycle;
\draw(-7.622,4.843)--(-7.612,4.865)--(-7.563,4.843)--(-7.564,4.84);
\filldraw[fill opacity=0.8,fill=gray!20,draw=none](-7.663,4.748)--(-7.688,4.757)--(-7.679,4.761)--cycle;
\draw(-7.688,4.757)--(-7.679,4.761);
\filldraw[fill opacity=0.8,fill=gray!20,draw=none](-7.705,4.813)--(-7.69,4.85)--(-7.638,4.853)--(-7.61,4.835)--(-7.609,4.817)--cycle;
\draw(-7.61,4.835)--(-7.609,4.817)--(-7.705,4.813)--(-7.69,4.85)--(-7.638,4.853);
\filldraw[fill opacity=0.8,fill=gray!20,draw=none](-7.704,4.763)--(-7.679,4.761)--(-7.688,4.757)--(-7.692,4.757)--cycle;
\draw(-7.679,4.761)--(-7.688,4.757);
\filldraw[fill opacity=0.8,fill=gray!20,draw=none](-7.704,4.763)--(-7.692,4.757)--(-7.744,4.752)--(-7.719,4.764)--cycle;
\draw(-7.744,4.752)--(-7.719,4.764);
\filldraw[fill opacity=0.8,fill=gray!20,draw=none](-7.679,4.761)--(-7.663,4.748)--(-7.711,4.639)--(-7.749,4.66)--(-7.704,4.763)--cycle;
\draw(-7.663,4.748)--(-7.711,4.639);
\draw(-7.749,4.66)--(-7.704,4.763);
\filldraw[fill opacity=0.8,fill=gray!20,draw=none](-7.709,4.638)--(-7.745,4.659)--(-7.719,4.726)--(-7.678,4.711)--cycle;
\draw(-7.745,4.659)--(-7.719,4.726)--(-7.678,4.711);
\filldraw[fill opacity=0.8,fill=gray!20,draw=none](-7.679,4.694)--(-7.725,4.571)--(-7.728,4.574)--cycle;
\draw(-7.679,4.694)--(-7.725,4.571);
\filldraw[fill opacity=0.8,fill=gray!20,draw=none](-7.707,4.6)--(-7.716,4.603)--(-7.682,4.687)--(-7.668,4.679)--cycle;
\filldraw[fill opacity=0.8,fill=gray!20,draw=none](-7.629,4.638)--(-7.54,4.592)--(-7.546,4.589)--(-7.635,4.631)--cycle;
\draw(-7.54,4.592)--(-7.546,4.589);
\filldraw[fill opacity=0.8,fill=gray!20,draw=none](-4.486,3.049)--(-7.649,4.631)--(-7.622,4.631)--(-4.484,3.06)--cycle;
\draw(-4.486,3.049)--(-7.649,4.631)--(-7.622,4.631)--(-4.484,3.06);
\filldraw[fill opacity=0.8,fill=gray!20,draw=none](-4.486,3.049)--(-4.484,3.06)--(-4.465,3.051)--cycle;
\draw(-4.484,3.06)--(-4.465,3.051);
\filldraw[fill opacity=0.8,fill=gray!20,draw=none](-7.84,1.648)--(-7.846,1.655)--(-7.852,1.67)--cycle;
\draw(-7.84,1.648)--(-7.846,1.655)--(-7.852,1.67);
\filldraw[fill opacity=0.8,fill=gray!20,draw=none](-4.47,2.662)--(-4.503,2.613)--(-4.506,2.68)--(-4.506,2.681)--cycle;
\draw(-4.47,2.662)--(-4.503,2.613);
\draw(-4.506,2.68)--(-4.506,2.681);
\filldraw[fill opacity=0.8,fill=gray!20,draw=none](-4.503,2.613)--(-4.783,2.193)--(-4.666,2.44)--(-4.506,2.68)--cycle;
\draw(-4.503,2.613)--(-4.783,2.193);
\draw(-4.666,2.44)--(-4.506,2.68);
\filldraw[fill opacity=0.8,fill=gray!20](-8.226,1.688)--(-8.191,1.732)--(-8.165,1.749)--(-8.195,1.709)--cycle;
\filldraw[fill opacity=0.8,fill=gray!20,draw=none](-7.958,1.767)--(-7.975,1.784)--(-7.948,1.777)--cycle;
\draw(-7.958,1.767)--(-7.975,1.784)--(-7.948,1.777);
\filldraw[fill opacity=0.8,fill=gray!20](-8.036,1.364)--(-8.093,1.372)--(-8.083,1.378)--(-8.036,1.364)--cycle;
\filldraw[fill opacity=0.8,fill=gray!20,draw=none](-7.832,1.592)--(-7.84,1.624)--(-7.841,1.649)--(-7.824,1.631)--(-7.817,1.576)--cycle;
\draw(-7.841,1.649)--(-7.824,1.631)--(-7.817,1.576)--(-7.832,1.592);
\filldraw[fill opacity=0.8,fill=gray!20,draw=none](-4.524,2.728)--(-4.522,2.702)--(-4.534,2.701)--(-4.569,2.719)--(-4.566,2.722)--cycle;
\draw(-4.569,2.719)--(-4.566,2.722);
\filldraw[fill opacity=0.8,fill=gray!20,draw=none](-4.489,2.715)--(-4.473,2.693)--(-4.478,2.687)--(-4.503,2.689)--(-4.522,2.696)--(-4.522,2.699)--(-4.518,2.702)--cycle;
\draw(-4.473,2.693)--(-4.478,2.687);
\draw(-4.522,2.699)--(-4.518,2.702);
\filldraw[fill opacity=0.8,fill=gray!20,draw=none](-4.576,2.723)--(-4.572,2.72)--(-4.566,2.722)--cycle;
\draw(-4.572,2.72)--(-4.566,2.722);
\filldraw[fill opacity=0.8,fill=gray!20,draw=none](-4.574,2.722)--(-4.566,2.722)--(-4.569,2.719)--cycle;
\draw(-4.566,2.722)--(-4.569,2.719);
\filldraw[fill opacity=0.8,fill=gray!20,draw=none](-4.534,2.701)--(-4.529,2.696)--(-4.518,2.703)--cycle;
\draw(-4.529,2.696)--(-4.518,2.703);
\filldraw[fill opacity=0.8,fill=gray!20,draw=none](-4.487,2.751)--(-4.502,2.745)--(-4.502,2.741)--cycle;
\draw(-4.487,2.751)--(-4.502,2.745);
\filldraw[fill opacity=0.8,fill=gray!20,draw=none](-4.46,2.796)--(-4.513,2.777)--(-4.502,2.745)--(-4.487,2.751)--cycle;
\draw(-4.46,2.796)--(-4.513,2.777);
\draw(-4.502,2.745)--(-4.487,2.751);
\filldraw[fill opacity=0.8,fill=gray!20,draw=none](-4.5,2.751)--(-4.55,2.72)--(-4.549,2.717)--(-4.534,2.701)--(-4.518,2.703)--(-4.49,2.72)--cycle;
\draw(-4.5,2.751)--(-4.55,2.72);
\draw(-4.518,2.703)--(-4.49,2.72);
\filldraw[fill opacity=0.8,fill=gray!20,draw=none](-4.518,2.702)--(-4.522,2.699)--(-4.522,2.701)--cycle;
\draw(-4.518,2.702)--(-4.522,2.699);
\filldraw[fill opacity=0.8,fill=gray!20,draw=none](-4.522,2.702)--(-4.522,2.701)--(-4.529,2.699)--(-4.534,2.701)--cycle;
\filldraw[fill opacity=0.8,fill=gray!20,draw=none](-4.529,2.699)--(-4.522,2.701)--(-4.522,2.699)--(-4.524,2.696)--cycle;
\draw(-4.522,2.699)--(-4.524,2.696);
\filldraw[fill opacity=0.8,fill=gray!20,draw=none](-4.522,2.696)--(-4.524,2.696)--(-4.522,2.699)--cycle;
\draw(-4.524,2.696)--(-4.522,2.699);
\filldraw[fill opacity=0.8,fill=gray!20](-4.498,2.685)--(-4.523,2.71)--(-4.577,2.723)--(-4.536,2.694)--cycle;
\filldraw[fill opacity=0.8,fill=gray!20,draw=none](-4.515,2.997)--(-4.514,3.009)--(-4.524,2.996)--cycle;
\draw(-4.514,3.009)--(-4.524,2.996);
\filldraw[fill opacity=0.8,fill=gray!20](-7.865,.65)--(-7.819,.684)--(-7.845,.667)--(-7.883,.639)--cycle;
\filldraw[fill opacity=0.8,fill=gray!20,draw=none](-4.215,3.011)--(-4.217,2.992)--(-4.202,3.002)--(-4.186,3.025)--cycle;
\draw(-4.202,3.002)--(-4.186,3.025);
\filldraw[fill opacity=0.8,fill=gray!20,draw=none](-4.48,2.677)--(-4.466,2.673)--(-4.464,2.671)--(-4.47,2.662)--(-4.5,2.678)--cycle;
\draw(-4.464,2.671)--(-4.47,2.662);
\filldraw[fill opacity=0.8,fill=gray!20,draw=none](-7.824,.992)--(-7.836,1.004)--(-7.876,1.034)--(-7.865,1.022)--(-7.823,.989)--cycle;
\draw(-7.824,.992)--(-7.836,1.004)--(-7.876,1.034)--(-7.865,1.022)--(-7.823,.989);
\filldraw[fill opacity=0.8,fill=gray!20,draw=none](-7.84,1.648)--(-7.852,1.67)--(-7.866,1.704)--(-7.846,1.683)--(-7.824,1.631)--cycle;
\draw(-7.852,1.67)--(-7.866,1.704)--(-7.846,1.683)--(-7.824,1.631)--(-7.84,1.648);
\filldraw[fill opacity=0.8,fill=gray!20,draw=none](-7.84,1.537)--(-7.82,1.554)--(-7.824,1.52)--cycle;
\draw(-7.82,1.554)--(-7.824,1.52)--(-7.84,1.537);
\filldraw[fill opacity=0.8,fill=gray!20](-8.216,4.22)--(-8.213,4.264)--(-8.137,4.258)--(-8.162,4.216)--cycle;
\filldraw[fill opacity=0.8,fill=gray!20,draw=none](-8.041,4.323)--(-8.033,4.347)--(-8.02,4.333)--cycle;
\draw(-8.041,4.323)--(-8.033,4.347)--(-8.02,4.333);
\filldraw[fill opacity=0.8,fill=gray!20,draw=none](-8.052,4.293)--(-8.041,4.323)--(-8.02,4.333)--(-8.011,4.323)--(-8.033,4.272)--cycle;
\draw(-8.02,4.333)--(-8.011,4.323)--(-8.033,4.272)--(-8.052,4.293)--(-8.041,4.323);
\filldraw[fill opacity=0.8,fill=gray!20,draw=none](-8.02,4.333)--(-8.009,4.339)--(-8.011,4.323)--cycle;
\draw(-8.009,4.339)--(-8.011,4.323)--(-8.02,4.333);
\filldraw[fill opacity=0.8,fill=gray!20,draw=none](-8.021,4.356)--(-8.036,4.306)--(-8.011,4.323)--(-8.003,4.379)--(-8.01,4.374)--cycle;
\draw(-8.036,4.306)--(-8.011,4.323)--(-8.003,4.379)--(-8.01,4.374);
\filldraw[fill opacity=0.8,fill=gray!20](-8.026,4.403)--(-8.033,4.458)--(-8.011,4.434)--(-8.003,4.379)--cycle;
\filldraw[fill opacity=0.8,fill=gray!20,draw=none](-8.021,4.356)--(-8.01,4.374)--(-8.016,4.37)--cycle;
\draw(-8.01,4.374)--(-8.016,4.37);
\filldraw[fill opacity=0.8,fill=gray!20,draw=none](-8.013,4.385)--(-8.018,4.357)--(-8.132,4.303)--(-8.127,4.333)--cycle;
\draw(-8.018,4.357)--(-8.132,4.303)--(-8.127,4.333);
\filldraw[fill opacity=0.8,fill=gray!20,draw=none](-8.038,4.326)--(-8.152,4.272)--(-8.132,4.303)--(-8.021,4.356)--cycle;
\draw(-8.038,4.326)--(-8.152,4.272)--(-8.132,4.303)--(-8.021,4.356);
\filldraw[fill opacity=0.8,fill=gray!20,draw=none](-8.041,4.303)--(-8.036,4.306)--(-8.021,4.356)--(-8.044,4.316)--(-8.045,4.307)--cycle;
\draw(-8.041,4.303)--(-8.036,4.306);
\draw(-8.044,4.316)--(-8.045,4.307);
\filldraw[fill opacity=0.8,fill=gray!20,draw=none](-8.013,4.385)--(-8.127,4.333)--(-8.125,4.346)--(-8.011,4.4)--cycle;
\draw(-8.127,4.333)--(-8.125,4.346)--(-8.011,4.4);
\filldraw[fill opacity=0.8,fill=gray!20,draw=none](-8.043,4.324)--(-8.071,4.311)--(-8.183,4.257)--(-8.152,4.272)--(-8.038,4.326)--cycle;
\draw(-8.071,4.311)--(-8.183,4.257)--(-8.152,4.272)--(-8.038,4.326);
\filldraw[fill opacity=0.8,fill=gray!20,draw=none](-7.837,4.422)--(-8.049,4.321)--(-8.031,4.329)--(-7.743,4.466)--cycle;
\draw(-7.837,4.422)--(-8.049,4.321);
\draw(-8.031,4.329)--(-7.743,4.466);
\filldraw[fill opacity=0.8,fill=gray!20,draw=none](-7.704,4.486)--(-7.709,4.483)--(-7.758,4.46)--(-7.837,4.422)--(-7.743,4.466)--(-7.695,4.49)--cycle;
\draw(-7.758,4.46)--(-7.837,4.422);
\draw(-7.743,4.466)--(-7.695,4.49);
\filldraw[fill opacity=0.8,fill=gray!20,draw=none](-7.695,4.49)--(-7.743,4.466)--(-7.682,4.509)--cycle;
\draw(-7.695,4.49)--(-7.743,4.466);
\filldraw[fill opacity=0.8,fill=gray!20,draw=none](-7.775,4.463)--(-7.802,4.461)--(-7.908,4.411)--(-7.837,4.422)--(-7.754,4.462)--cycle;
\draw(-7.802,4.461)--(-7.908,4.411);
\draw(-7.837,4.422)--(-7.754,4.462);
\filldraw[fill opacity=0.8,fill=gray!20,draw=none](-7.709,4.483)--(-7.745,4.466)--(-7.758,4.46)--cycle;
\draw(-7.745,4.466)--(-7.758,4.46);
\filldraw[fill opacity=0.8,fill=gray!20,draw=none](-7.734,4.419)--(-7.719,4.456)--(-7.676,4.411)--(-7.685,4.388)--cycle;
\draw(-7.734,4.419)--(-7.719,4.456);
\draw(-7.676,4.411)--(-7.685,4.388);
\filldraw[fill opacity=0.8,fill=gray!20,draw=none](-7.782,4.445)--(-7.779,4.453)--(-7.728,4.434)--(-7.734,4.419)--cycle;
\draw(-7.782,4.445)--(-7.779,4.453);
\draw(-7.728,4.434)--(-7.734,4.419);
\filldraw[fill opacity=0.8,fill=gray!20,draw=none](-7.775,4.463)--(-7.754,4.462)--(-7.745,4.466)--cycle;
\draw(-7.754,4.462)--(-7.745,4.466);
\filldraw[fill opacity=0.8,fill=gray!20,draw=none](-7.779,4.453)--(-7.766,4.487)--(-7.719,4.456)--(-7.728,4.434)--cycle;
\draw(-7.779,4.453)--(-7.766,4.487);
\draw(-7.719,4.456)--(-7.728,4.434);
\filldraw[fill opacity=0.8,fill=gray!20,draw=none](-7.693,4.429)--(-7.719,4.456)--(-7.709,4.483)--(-7.705,4.485)--(-7.692,4.459)--cycle;
\draw(-7.719,4.456)--(-7.709,4.483);
\filldraw[fill opacity=0.8,fill=gray!20,draw=none](-7.739,4.469)--(-7.709,4.483)--(-7.719,4.456)--cycle;
\draw(-7.709,4.483)--(-7.719,4.456);
\filldraw[fill opacity=0.8,fill=gray!20,draw=none](-7.739,4.469)--(-7.766,4.487)--(-7.749,4.528)--(-7.707,4.488)--(-7.709,4.483)--cycle;
\draw(-7.766,4.487)--(-7.749,4.528);
\draw(-7.707,4.488)--(-7.709,4.483);
\filldraw[fill opacity=0.8,fill=gray!20,draw=none](-7.709,4.483)--(-7.705,4.485)--(-7.724,4.476)--(-7.733,4.472)--(-7.745,4.466)--cycle;
\draw(-7.733,4.472)--(-7.745,4.466);
\filldraw[fill opacity=0.8,fill=gray!20,draw=none](-7.693,4.429)--(-7.692,4.459)--(-7.673,4.419)--(-7.676,4.411)--cycle;
\draw(-7.673,4.419)--(-7.676,4.411);
\filldraw[fill opacity=0.8,fill=gray!20,draw=none](-7.685,4.388)--(-7.676,4.411)--(-7.668,4.4)--(-7.662,4.371)--cycle;
\draw(-7.685,4.388)--(-7.676,4.411);
\filldraw[fill opacity=0.8,fill=gray!20,draw=none](-7.676,4.411)--(-7.673,4.419)--(-7.668,4.4)--cycle;
\draw(-7.676,4.411)--(-7.673,4.419);
\filldraw[fill opacity=0.8,fill=gray!20,draw=none](-7.794,4.465)--(-7.802,4.461)--(-7.775,4.463)--cycle;
\draw(-7.794,4.465)--(-7.802,4.461);
\filldraw[fill opacity=0.8,fill=gray!20,draw=none](-7.783,4.456)--(-7.754,4.527)--(-7.75,4.526)--(-7.779,4.453)--cycle;
\draw(-7.75,4.526)--(-7.779,4.453);
\filldraw[fill opacity=0.8,fill=gray!20,draw=none](-7.78,4.464)--(-7.775,4.463)--(-7.745,4.466)--(-7.743,4.467)--cycle;
\draw(-7.745,4.466)--(-7.743,4.467);
\filldraw[fill opacity=0.8,fill=gray!20,draw=none](-7.774,4.48)--(-7.783,4.456)--(-7.814,4.482)--(-7.795,4.532)--(-7.787,4.532)--cycle;
\draw(-7.814,4.482)--(-7.795,4.532);
\filldraw[fill opacity=0.8,fill=gray!20,draw=none](-7.77,4.47)--(-7.781,4.464)--(-7.78,4.464)--(-7.743,4.467)--(-7.731,4.473)--cycle;
\draw(-7.743,4.467)--(-7.731,4.473);
\filldraw[fill opacity=0.8,fill=gray!20,draw=none](-7.724,4.476)--(-7.725,4.475)--(-7.731,4.473)--(-7.733,4.472)--cycle;
\draw(-7.731,4.473)--(-7.733,4.472);
\filldraw[fill opacity=0.8,fill=gray!20,draw=none](-7.733,4.453)--(-7.723,4.475)--(-7.695,4.413)--cycle;
\draw(-7.733,4.453)--(-7.723,4.475);
\filldraw[fill opacity=0.8,fill=gray!20,draw=none](-7.75,4.464)--(-7.725,4.476)--(-7.723,4.475)--(-7.733,4.453)--cycle;
\draw(-7.723,4.475)--(-7.733,4.453);
\filldraw[fill opacity=0.8,fill=gray!20,draw=none](-7.773,4.468)--(-7.779,4.469)--(-7.787,4.468)--(-7.794,4.465)--(-7.781,4.464)--cycle;
\draw(-7.787,4.468)--(-7.794,4.465);
\filldraw[fill opacity=0.8,fill=gray!20,draw=none](-7.773,4.463)--(-7.77,4.47)--(-7.759,4.471)--(-7.733,4.453)--(-7.735,4.449)--cycle;
\draw(-7.733,4.453)--(-7.735,4.449);
\filldraw[fill opacity=0.8,fill=gray!20,draw=none](-7.811,4.37)--(-7.782,4.445)--(-7.734,4.419)--(-7.76,4.351)--cycle;
\draw(-7.734,4.419)--(-7.76,4.351)--(-7.811,4.37)--(-7.782,4.445);
\filldraw[fill opacity=0.8,fill=gray!20,draw=none](-7.754,4.368)--(-7.734,4.419)--(-7.685,4.388)--(-7.7,4.349)--cycle;
\draw(-7.754,4.368)--(-7.734,4.419);
\draw(-7.685,4.388)--(-7.7,4.349);
\filldraw[fill opacity=0.8,fill=gray!20,draw=none](-7.747,4.422)--(-7.733,4.453)--(-7.695,4.413)--(-7.703,4.395)--cycle;
\draw(-7.747,4.422)--(-7.733,4.453);
\draw(-7.695,4.413)--(-7.703,4.395);
\filldraw[fill opacity=0.8,fill=gray!20](-8.292,4.26)--(-8.307,4.311)--(-8.211,4.316)--(-8.213,4.264)--cycle;
\filldraw[fill opacity=0.8,fill=gray!20](-8.272,4.217)--(-8.292,4.26)--(-8.213,4.264)--(-8.216,4.22)--cycle;
\filldraw[fill opacity=0.8,fill=gray!20](-8.307,4.311)--(-8.317,4.367)--(-8.21,4.372)--(-8.211,4.316)--cycle;
\filldraw[fill opacity=0.8,fill=gray!20](-8.317,4.367)--(-8.321,4.424)--(-8.209,4.429)--(-8.21,4.372)--cycle;
\filldraw[fill opacity=0.8,fill=gray!20](-8.321,4.424)--(-8.317,4.478)--(-8.21,4.483)--(-8.209,4.429)--cycle;
\filldraw[fill opacity=0.8,fill=gray!20](-8.311,4.369)--(-8.32,4.417)--(-8.313,4.46)--(-8.293,4.491)--(-8.262,4.506)--(-8.225,4.501)--(-8.187,4.478)--(-8.155,4.441)--(-8.134,4.394)--(-8.125,4.346)--(-8.132,4.303)--(-8.152,4.272)--(-8.183,4.257)--(-8.22,4.262)--(-8.257,4.285)--(-8.29,4.322)--cycle;
\filldraw[fill opacity=0.8,fill=gray!20,draw=none](-7.779,4.469)--(-7.774,4.481)--(-7.759,4.471)--cycle;
\draw(-7.779,4.469)--(-7.774,4.481);
\filldraw[fill opacity=0.8,fill=gray!20,draw=none](-7.819,4.494)--(-7.968,4.423)--(-7.958,4.387)--(-7.805,4.46)--cycle;
\draw(-7.819,4.494)--(-7.968,4.423);
\draw(-7.958,4.387)--(-7.805,4.46);
\filldraw[fill opacity=0.8,fill=gray!20,draw=none](-7.82,4.505)--(-7.803,4.549)--(-7.79,4.543)--(-7.818,4.472)--cycle;
\draw(-7.79,4.543)--(-7.818,4.472);
\filldraw[fill opacity=0.8,fill=gray!20,draw=none](-7.812,4.492)--(-7.816,4.495)--(-7.819,4.494)--(-7.805,4.46)--(-7.801,4.462)--cycle;
\draw(-7.816,4.495)--(-7.819,4.494);
\draw(-7.805,4.46)--(-7.801,4.462);
\filldraw[fill opacity=0.8,fill=gray!20,draw=none](-7.812,4.492)--(-7.801,4.462)--(-7.785,4.469)--cycle;
\draw(-7.801,4.462)--(-7.785,4.469);
\filldraw[fill opacity=0.8,fill=gray!20,draw=none](-7.812,4.459)--(-7.809,4.466)--(-7.794,4.465)--(-7.779,4.453)--(-7.794,4.413)--cycle;
\draw(-7.779,4.453)--(-7.794,4.413);
\filldraw[fill opacity=0.8,fill=gray!20,draw=none](-7.784,4.442)--(-7.773,4.463)--(-7.735,4.449)--(-7.747,4.422)--cycle;
\draw(-7.735,4.449)--(-7.747,4.422);
\filldraw[fill opacity=0.8,fill=gray!20,draw=none](-7.75,4.464)--(-7.767,4.476)--(-7.754,4.503)--(-7.725,4.476)--cycle;
\filldraw[fill opacity=0.8,fill=gray!20,draw=none](-7.725,4.475)--(-7.729,4.474)--(-7.731,4.473)--cycle;
\draw(-7.729,4.474)--(-7.731,4.473);
\filldraw[fill opacity=0.8,fill=gray!20,draw=none](-7.725,4.475)--(-7.728,4.474)--(-7.729,4.474)--cycle;
\draw(-7.728,4.474)--(-7.729,4.474);
\filldraw[fill opacity=0.8,fill=gray!20,draw=none](-7.757,4.476)--(-7.77,4.47)--(-7.731,4.473)--(-7.728,4.474)--cycle;
\draw(-7.731,4.473)--(-7.728,4.474);
\filldraw[fill opacity=0.8,fill=gray!20,draw=none](-4.793,2.961)--(-7.761,4.447)--(-7.756,4.491)--(-4.699,2.961)--cycle;
\draw(-4.793,2.961)--(-7.761,4.447)--(-7.756,4.491)--(-4.699,2.961);
\filldraw[fill opacity=0.8,fill=gray!20,draw=none](-4.46,2.691)--(-4.453,2.687)--(-4.453,2.687)--(-4.47,2.688)--cycle;
\draw(-4.453,2.687)--(-4.453,2.687);
\filldraw[fill opacity=0.8,fill=gray!20,draw=none](-4.217,2.992)--(-4.218,2.989)--(-4.202,3.002)--cycle;
\filldraw[fill opacity=0.8,fill=gray!20](-8.036,1.364)--(-7.985,1.377)--(-7.979,1.371)--(-8.036,1.364)--cycle;
\filldraw[fill opacity=0.8,fill=gray!20,draw=none](-8.148,.909)--(-8.137,.924)--(-8.145,.926)--(-8.152,.906)--cycle;
\draw(-8.137,.924)--(-8.145,.926)--(-8.152,.906);
\filldraw[fill opacity=0.8,fill=gray!20,draw=none](-8.127,.922)--(-8.137,.924)--(-8.148,.909)--cycle;
\draw(-8.127,.922)--(-8.137,.924);
\filldraw[fill opacity=0.8,fill=gray!20,draw=none](-8.131,.931)--(-8.137,.924)--(-8.127,.922)--(-8.11,.932)--cycle;
\draw(-8.137,.924)--(-8.127,.922);
\filldraw[fill opacity=0.8,fill=gray!20,draw=none](-8.052,.94)--(-8.281,.866)--(-8.324,.863)--(-8.016,.962)--cycle;
\draw(-8.324,.863)--(-8.016,.962)--(-8.052,.94)--(-8.281,.866);
\filldraw[fill opacity=0.8,fill=gray!20,draw=none](-4.445,2.672)--(-4.442,2.672)--(-4.442,2.672)--cycle;
\draw(-4.442,2.672)--(-4.442,2.672);
\filldraw[fill opacity=0.8,fill=gray!20](-2.815,7.889)--(-2.812,7.943)--(-2.704,7.948)--(-2.704,7.894)--cycle;
\filldraw[fill opacity=0.8,fill=gray!20](-2.704,7.894)--(-2.704,7.948)--(-2.6,7.941)--(-2.596,7.886)--cycle;
\filldraw[fill opacity=0.8,fill=gray!20](-4.608,2.979)--(-4.577,3.026)--(-4.593,3.044)--(-4.628,3)--cycle;
\filldraw[fill opacity=0.8,fill=gray!20,draw=none](-4.436,3.107)--(-4.438,3.104)--(-4.438,3.104)--(-4.429,3.105)--cycle;
\draw(-4.436,3.107)--(-4.438,3.104)--(-4.438,3.104)--(-4.429,3.105);
\filldraw[fill opacity=0.8,fill=gray!20,draw=none](-4.403,2.68)--(-4.443,2.676)--(-4.442,2.672)--cycle;
\draw(-4.443,2.676)--(-4.442,2.672);
\filldraw[fill opacity=0.8,fill=gray!20,draw=none](-8.043,.671)--(-8.07,.677)--(-8.06,.659)--cycle;
\draw(-8.07,.677)--(-8.06,.659);
\filldraw[fill opacity=0.8,fill=gray!20,draw=none](-8.071,.678)--(-8.07,.677)--(-8.07,.678)--cycle;
\draw(-8.07,.677)--(-8.07,.678);
\filldraw[fill opacity=0.8,fill=gray!20,draw=none](-8.043,.671)--(-8.027,.682)--(-8.07,.678)--(-8.07,.677)--cycle;
\draw(-8.07,.678)--(-8.07,.677);
\filldraw[fill opacity=0.8,fill=gray!20,draw=none](-8.393,.551)--(-8.376,.58)--(-8.374,.58)--(-8.36,.569)--cycle;
\draw(-8.393,.551)--(-8.376,.58);
\filldraw[fill opacity=0.8,fill=gray!20,draw=none](-8.36,.569)--(-8.374,.58)--(-8.365,.581)--(-8.346,.576)--cycle;
\draw(-8.365,.581)--(-8.346,.576);
\filldraw[fill opacity=0.8,fill=gray!20,draw=none](-8.346,.576)--(-8.365,.581)--(-8.335,.583)--cycle;
\draw(-8.346,.576)--(-8.365,.581);
\filldraw[fill opacity=0.8,fill=gray!20](-7.933,.713)--(-8.403,.561)--(-8.445,.557)--(-7.974,.709)--cycle;
\filldraw[fill opacity=0.8,fill=gray!20,draw=none](-4.265,2.769)--(-4.247,2.754)--(-4.232,2.805)--(-4.241,2.812)--cycle;
\draw(-4.265,2.769)--(-4.247,2.754);
\draw(-4.232,2.805)--(-4.241,2.812);
\filldraw[fill opacity=0.8,fill=gray!20,draw=none](-4.245,2.819)--(-4.231,2.828)--(-4.23,2.848)--(-4.235,2.876)--(-4.255,2.864)--(-4.256,2.855)--cycle;
\draw(-4.245,2.819)--(-4.231,2.828);
\draw(-4.235,2.876)--(-4.255,2.864)--(-4.256,2.855);
\filldraw[fill opacity=0.8,fill=gray!20,draw=none](-4.479,2.719)--(-4.466,2.712)--(-4.469,2.712)--(-4.488,2.715)--cycle;
\filldraw[fill opacity=0.8,fill=gray!20,draw=none](-4.528,2.692)--(-4.529,2.696)--(-4.532,2.694)--cycle;
\draw(-4.529,2.696)--(-4.532,2.694);
\filldraw[fill opacity=0.8,fill=gray!20](-4.495,3.098)--(-4.438,3.104)--(-4.438,3.104)--(-4.485,3.104)--cycle;
\filldraw[fill opacity=0.8,fill=gray!20](-8.029,1.788)--(-8.032,1.798)--(-8.005,1.796)--(-7.975,1.784)--cycle;
\filldraw[fill opacity=0.8,fill=gray!20](-8.085,1.786)--(-8.061,1.797)--(-8.032,1.798)--(-8.029,1.788)--cycle;
\filldraw[fill opacity=0.8,fill=gray!20,draw=none](-4.468,2.677)--(-4.466,2.673)--(-4.48,2.677)--cycle;
\filldraw[fill opacity=0.8,fill=gray!20,draw=none](-4.452,2.687)--(-4.453,2.687)--(-4.453,2.687)--cycle;
\draw(-4.453,2.687)--(-4.453,2.687);
\filldraw[fill opacity=0.8,fill=gray!20](-7.819,.684)--(-7.785,.728)--(-7.816,.707)--(-7.845,.667)--cycle;
\filldraw[fill opacity=0.8,fill=gray!20,draw=none](-4.471,2.732)--(-4.49,2.72)--(-4.489,2.714)--cycle;
\draw(-4.471,2.732)--(-4.49,2.72);
\filldraw[fill opacity=0.8,fill=gray!20,draw=none](-4.488,2.715)--(-4.489,2.715)--(-4.489,2.715)--cycle;
\filldraw[fill opacity=0.8,fill=gray!20,draw=none](-8.206,1.453)--(-8.22,1.487)--(-8.232,1.509)--(-8.248,1.526)--(-8.226,1.474)--cycle;
\draw(-8.232,1.509)--(-8.248,1.526)--(-8.226,1.474)--(-8.206,1.453)--(-8.22,1.487);
\filldraw[fill opacity=0.8,fill=gray!20,draw=none](-8.134,1.382)--(-8.175,1.412)--(-8.183,1.421)--(-8.186,1.425)--(-8.145,1.394)--cycle;
\draw(-8.186,1.425)--(-8.145,1.394)--(-8.134,1.382)--(-8.175,1.412)--(-8.183,1.421);
\filldraw[fill opacity=0.8,fill=gray!20,draw=none](-8.183,1.421)--(-8.191,1.429)--(-8.186,1.425)--cycle;
\draw(-8.183,1.421)--(-8.191,1.429)--(-8.186,1.425);
\filldraw[fill opacity=0.8,fill=gray!20,draw=none](-8.183,1.421)--(-8.178,1.416)--(-8.206,1.453)--(-8.226,1.474)--(-8.191,1.429)--cycle;
\draw(-8.178,1.416)--(-8.206,1.453)--(-8.226,1.474)--(-8.191,1.429)--(-8.183,1.421);
\filldraw[fill opacity=0.8,fill=gray!20,draw=none](-8.22,1.487)--(-8.226,1.502)--(-8.232,1.509)--cycle;
\draw(-8.22,1.487)--(-8.226,1.502)--(-8.232,1.509);
\filldraw[fill opacity=0.8,fill=gray!20,draw=none](-8.231,1.508)--(-8.226,1.502)--(-8.228,1.518)--(-8.232,1.533)--cycle;
\draw(-8.231,1.508)--(-8.226,1.502)--(-8.228,1.518);
\filldraw[fill opacity=0.8,fill=gray!20,draw=none](-8.183,1.447)--(-8.194,1.464)--(-8.22,1.487)--(-8.206,1.453)--cycle;
\draw(-8.22,1.487)--(-8.206,1.453)--(-8.183,1.447);
\filldraw[fill opacity=0.8,fill=gray!20,draw=none](-8.194,1.464)--(-8.219,1.501)--(-8.226,1.502)--(-8.22,1.487)--cycle;
\draw(-8.219,1.501)--(-8.226,1.502)--(-8.22,1.487);
\filldraw[fill opacity=0.8,fill=gray!20,draw=none](-8.231,1.508)--(-8.232,1.533)--(-8.24,1.565)--(-8.255,1.581)--(-8.248,1.526)--cycle;
\draw(-8.24,1.565)--(-8.255,1.581)--(-8.248,1.526)--(-8.231,1.508);
\filldraw[fill opacity=0.8,fill=gray!20,draw=none](-8.232,1.533)--(-8.232,1.557)--(-8.24,1.565)--cycle;
\draw(-8.232,1.557)--(-8.24,1.565);
\filldraw[fill opacity=0.8,fill=gray!20,draw=none](-8.232,1.533)--(-8.228,1.518)--(-8.232,1.557)--cycle;
\draw(-8.228,1.518)--(-8.232,1.557);
\filldraw[fill opacity=0.8,fill=gray!20,draw=none](-8.221,1.501)--(-8.222,1.508)--(-8.228,1.518)--(-8.226,1.502)--cycle;
\draw(-8.228,1.518)--(-8.226,1.502)--(-8.221,1.501);
\filldraw[fill opacity=0.8,fill=gray!20,draw=none](-8.222,1.508)--(-8.232,1.557)--(-8.228,1.518)--cycle;
\draw(-8.232,1.557)--(-8.228,1.518);
\filldraw[fill opacity=0.8,fill=gray!20,draw=none](-8.5,1.389)--(-8.524,1.388)--(-8.509,1.437)--cycle;
\draw(-8.5,1.389)--(-8.524,1.388);
\filldraw[fill opacity=0.8,fill=gray!20,draw=none](-8.221,1.501)--(-8.219,1.501)--(-8.222,1.508)--cycle;
\draw(-8.221,1.501)--(-8.219,1.501);
\filldraw[fill opacity=0.8,fill=gray!20,draw=none](-8.5,1.386)--(-8.499,1.342)--(-8.516,1.342)--(-8.524,1.388)--(-8.502,1.389)--cycle;
\draw(-8.499,1.342)--(-8.516,1.342)--(-8.524,1.388)--(-8.502,1.389);
\filldraw[fill opacity=0.8,fill=gray!20,draw=none](-8.475,1.343)--(-8.499,1.342)--(-8.5,1.386)--cycle;
\draw(-8.475,1.343)--(-8.499,1.342);
\filldraw[fill opacity=0.8,fill=gray!20,draw=none](-8.475,1.342)--(-8.477,1.343)--(-8.475,1.343)--cycle;
\draw(-8.477,1.343)--(-8.475,1.343);
\filldraw[fill opacity=0.8,fill=gray!20,draw=none](-8.475,1.342)--(-8.46,1.301)--(-8.503,1.299)--(-8.516,1.342)--(-8.477,1.343)--cycle;
\draw(-8.46,1.301)--(-8.503,1.299)--(-8.516,1.342)--(-8.477,1.343);
\filldraw[fill opacity=0.8,fill=gray!20](-8.577,1.33)--(-8.593,1.375)--(-8.524,1.388)--(-8.516,1.342)--cycle;
\filldraw[fill opacity=0.8,fill=gray!20](-8.553,1.289)--(-8.577,1.33)--(-8.516,1.342)--(-8.503,1.299)--cycle;
\filldraw[fill opacity=0.8,fill=gray!20,draw=none](-8.5,1.386)--(-8.502,1.389)--(-8.5,1.389)--cycle;
\draw(-8.502,1.389)--(-8.5,1.389);
\filldraw[fill opacity=0.8,fill=gray!20,draw=none](-8.524,1.388)--(-8.527,1.436)--(-8.509,1.437)--cycle;
\draw(-8.524,1.388)--(-8.527,1.436)--(-8.509,1.437);
\filldraw[fill opacity=0.8,fill=gray!20](-8.593,1.375)--(-8.598,1.422)--(-8.527,1.436)--(-8.524,1.388)--cycle;
\filldraw[fill opacity=0.8,fill=gray!20,draw=none](-8.593,1.32)--(-8.602,1.369)--(-8.593,1.375)--(-8.577,1.33)--cycle;
\draw(-8.602,1.369)--(-8.593,1.375)--(-8.577,1.33)--(-8.593,1.32);
\filldraw[fill opacity=0.8,fill=gray!20,draw=none](-8.599,1.371)--(-8.611,1.413)--(-8.598,1.422)--(-8.593,1.375)--cycle;
\draw(-8.611,1.413)--(-8.598,1.422)--(-8.593,1.375)--(-8.599,1.371);
\filldraw[fill opacity=0.8,fill=gray!20,draw=none](-8.554,1.393)--(-8.535,1.371)--(-8.534,1.366)--(-8.568,1.359)--cycle;
\draw(-8.534,1.366)--(-8.568,1.359);
\filldraw[fill opacity=0.8,fill=gray!20,draw=none](-8.535,1.371)--(-8.571,1.411)--(-8.542,1.417)--cycle;
\draw(-8.571,1.411)--(-8.542,1.417);
\filldraw[fill opacity=0.8,fill=gray!20,draw=none](-8.544,1.421)--(-8.542,1.417)--(-8.545,1.417)--cycle;
\draw(-8.542,1.417)--(-8.545,1.417);
\filldraw[fill opacity=0.8,fill=gray!20,draw=none](-8.545,1.432)--(-8.598,1.422)--(-8.593,1.467)--(-8.524,1.481)--(-8.526,1.451)--cycle;
\draw(-8.545,1.432)--(-8.598,1.422)--(-8.593,1.467)--(-8.524,1.481)--(-8.526,1.451);
\filldraw[fill opacity=0.8,fill=gray!20,draw=none](-8.556,1.445)--(-8.544,1.421)--(-8.545,1.417)--(-8.588,1.407)--cycle;
\draw(-8.545,1.417)--(-8.588,1.407);
\filldraw[fill opacity=0.8,fill=gray!20,draw=none](-8.602,1.419)--(-8.61,1.456)--(-8.593,1.467)--(-8.598,1.422)--cycle;
\draw(-8.61,1.456)--(-8.593,1.467)--(-8.598,1.422)--(-8.602,1.419);
\filldraw[fill opacity=0.8,fill=gray!20,draw=none](-8.607,1.388)--(-8.604,1.388)--(-8.599,1.371)--(-8.605,1.367)--cycle;
\draw(-8.599,1.371)--(-8.605,1.367);
\filldraw[fill opacity=0.8,fill=gray!20](-8.593,1.467)--(-8.577,1.508)--(-8.516,1.52)--(-8.524,1.481)--cycle;
\filldraw[fill opacity=0.8,fill=gray!20,draw=none](-8.595,1.466)--(-8.605,1.488)--(-8.603,1.491)--(-8.577,1.508)--(-8.593,1.467)--cycle;
\draw(-8.605,1.488)--(-8.603,1.491)--(-8.577,1.508)--(-8.593,1.467)--(-8.595,1.466);
\filldraw[fill opacity=0.8,fill=gray!20,draw=none](-8.603,1.484)--(-8.595,1.466)--(-8.61,1.456)--cycle;
\draw(-8.595,1.466)--(-8.61,1.456);
\filldraw[fill opacity=0.8,fill=gray!20,draw=none](-8.603,1.484)--(-8.546,1.496)--(-8.538,1.468)--(-8.54,1.464)--(-8.612,1.448)--cycle;
\draw(-8.603,1.484)--(-8.546,1.496);
\draw(-8.54,1.464)--(-8.612,1.448);
\filldraw[fill opacity=0.8,fill=gray!20,draw=none](-8.692,1.43)--(-8.608,1.449)--(-8.612,1.402)--(-8.684,1.385)--cycle;
\draw(-8.692,1.43)--(-8.608,1.449);
\draw(-8.612,1.402)--(-8.684,1.385);
\filldraw[fill opacity=0.8,fill=gray!20,draw=none](-9.238,1.341)--(-8.603,1.484)--(-8.612,1.448)--(-9.323,1.288)--cycle;
\draw(-9.238,1.341)--(-8.603,1.484);
\draw(-8.612,1.448)--(-9.323,1.288);
\filldraw[fill opacity=0.8,fill=gray!20,draw=none](-8.602,1.419)--(-8.628,1.402)--(-8.621,1.449)--(-8.61,1.456)--cycle;
\draw(-8.602,1.419)--(-8.628,1.402)--(-8.621,1.449)--(-8.61,1.456);
\filldraw[fill opacity=0.8,fill=gray!20,draw=none](-8.608,1.449)--(-8.563,1.459)--(-8.556,1.445)--(-8.588,1.407)--(-8.612,1.402)--cycle;
\draw(-8.608,1.449)--(-8.563,1.459);
\draw(-8.588,1.407)--(-8.612,1.402);
\filldraw[fill opacity=0.8,fill=gray!20,draw=none](-8.608,1.402)--(-8.604,1.388)--(-8.607,1.388)--cycle;
\filldraw[fill opacity=0.8,fill=gray!20,draw=none](-8.606,1.373)--(-8.607,1.388)--(-8.588,1.407)--(-8.571,1.411)--(-8.554,1.393)--(-8.568,1.359)--(-8.569,1.359)--cycle;
\draw(-8.588,1.407)--(-8.571,1.411);
\draw(-8.568,1.359)--(-8.569,1.359);
\filldraw[fill opacity=0.8,fill=gray!20](-8.145,1.541)--(-8.554,1.363)--(-8.563,1.413)--(-8.154,1.591)--cycle;
\filldraw[fill opacity=0.8,fill=gray!20,draw=none](-8.535,1.371)--(-8.532,1.367)--(-8.534,1.366)--cycle;
\draw(-8.532,1.367)--(-8.534,1.366);
\filldraw[fill opacity=0.8,fill=gray!20,draw=none](-8.569,1.359)--(-8.532,1.367)--(-8.513,1.325)--cycle;
\draw(-8.569,1.359)--(-8.532,1.367);
\filldraw[fill opacity=0.8,fill=gray!20,draw=none](-8.574,1.275)--(-8.579,1.281)--(-8.588,1.323)--(-8.577,1.33)--(-8.553,1.289)--cycle;
\draw(-8.588,1.323)--(-8.577,1.33)--(-8.553,1.289)--(-8.574,1.275)--(-8.579,1.281);
\filldraw[fill opacity=0.8,fill=gray!20,draw=none](-8.492,1.276)--(-8.503,1.299)--(-8.437,1.302)--(-8.438,1.288)--cycle;
\draw(-8.492,1.276)--(-8.503,1.299)--(-8.437,1.302)--(-8.438,1.288);
\filldraw[fill opacity=0.8,fill=gray!20](-8.521,1.256)--(-8.553,1.289)--(-8.503,1.299)--(-8.486,1.263)--cycle;
\filldraw[fill opacity=0.8,fill=gray!20,draw=none](-8.536,1.247)--(-8.572,1.273)--(-8.571,1.277)--(-8.553,1.289)--(-8.521,1.256)--cycle;
\draw(-8.571,1.277)--(-8.553,1.289)--(-8.521,1.256)--(-8.536,1.247)--(-8.572,1.273);
\filldraw[fill opacity=0.8,fill=gray!20,draw=none](-8.572,1.273)--(-8.574,1.275)--(-8.571,1.277)--cycle;
\draw(-8.572,1.273)--(-8.574,1.275)--(-8.571,1.277);
\filldraw[fill opacity=0.8,fill=gray!20,draw=none](-8.573,1.274)--(-8.596,1.302)--(-8.579,1.306)--(-8.48,1.29)--(-8.478,1.288)--(-8.566,1.268)--cycle;
\draw(-8.596,1.302)--(-8.579,1.306);
\draw(-8.478,1.288)--(-8.566,1.268);
\filldraw[fill opacity=0.8,fill=gray!20,draw=none](-9.272,1.201)--(-8.61,1.349)--(-8.584,1.305)--(-9.203,1.166)--cycle;
\draw(-9.272,1.201)--(-8.61,1.349);
\draw(-8.584,1.305)--(-9.203,1.166);
\filldraw[fill opacity=0.8,fill=gray!20,draw=none](-8.585,1.307)--(-8.593,1.3)--(-8.603,1.313)--(-8.588,1.323)--cycle;
\draw(-8.593,1.3)--(-8.603,1.313)--(-8.588,1.323);
\filldraw[fill opacity=0.8,fill=gray!20,draw=none](-8.585,1.307)--(-8.61,1.349)--(-8.569,1.359)--(-8.513,1.325)--(-8.511,1.321)--(-8.579,1.306)--cycle;
\draw(-8.61,1.349)--(-8.569,1.359);
\draw(-8.511,1.321)--(-8.579,1.306);
\filldraw[fill opacity=0.8,fill=gray!20](-8.119,1.497)--(-8.528,1.318)--(-8.554,1.363)--(-8.145,1.541)--cycle;
\filldraw[fill opacity=0.8,fill=gray!20](-7.918,1.566)--(-7.927,1.517)--(-7.953,1.479)--(-7.991,1.455)--(-8.036,1.45)--(-8.081,1.465)--(-8.119,1.497)--(-8.145,1.541)--(-8.154,1.591)--(-8.145,1.64)--(-8.119,1.679)--(-8.081,1.702)--(-8.036,1.707)--(-7.991,1.692)--(-7.953,1.66)--(-7.927,1.616)--cycle;
\filldraw[fill opacity=0.8,fill=gray!20](-7.918,1.566)--(-7.927,1.616)--(-7.953,1.66)--(-7.991,1.692)--(-8.036,1.707)--(-8.081,1.702)--(-8.119,1.679)--(-8.145,1.64)--(-8.154,1.591)--(-8.145,1.541)--(-8.119,1.497)--(-8.081,1.465)--(-8.036,1.45)--(-7.991,1.455)--(-7.953,1.479)--(-7.927,1.517)--cycle;
\filldraw[fill opacity=0.8,fill=gray!20,draw=none](-4.488,2.715)--(-4.463,2.711)--(-4.46,2.709)--(-4.473,2.693)--(-4.489,2.715)--cycle;
\draw(-4.46,2.709)--(-4.473,2.693);
\filldraw[fill opacity=0.8,fill=gray!20,draw=none](-4.452,2.687)--(-4.414,2.7)--(-4.423,2.686)--cycle;
\draw(-4.414,2.7)--(-4.423,2.686);
\filldraw[fill opacity=0.8,fill=gray!20,draw=none](-4.548,2.7)--(-4.587,2.697)--(-4.57,2.718)--cycle;
\draw(-4.587,2.697)--(-4.57,2.718);
\filldraw[fill opacity=0.8,fill=gray!20,draw=none](-4.57,2.708)--(-4.636,2.667)--(-4.574,2.668)--(-4.532,2.694)--cycle;
\draw(-4.57,2.708)--(-4.636,2.667);
\draw(-4.574,2.668)--(-4.532,2.694);
\filldraw[fill opacity=0.8,fill=gray!20,draw=none](-7.868,1.705)--(-7.866,1.704)--(-7.866,1.704)--cycle;
\draw(-7.868,1.705)--(-7.866,1.704)--(-7.866,1.704);
\filldraw[fill opacity=0.8,fill=gray!20,draw=none](-4.241,2.812)--(-4.243,2.809)--(-4.243,2.808)--cycle;
\filldraw[fill opacity=0.8,fill=gray!20](-8.006,.62)--(-8.035,.632)--(-8.073,.641)--(-8.025,.625)--cycle;
\filldraw[fill opacity=0.8,fill=gray!20,draw=none](-4.577,2.668)--(-4.579,2.661)--(-4.586,2.65)--(-4.636,2.667)--cycle;
\draw(-4.579,2.661)--(-4.586,2.65);
\filldraw[fill opacity=0.8,fill=gray!20,draw=none](-8.134,1.415)--(-8.156,1.432)--(-8.203,1.452)--(-8.206,1.453)--(-8.178,1.416)--cycle;
\draw(-8.203,1.452)--(-8.206,1.453)--(-8.178,1.416);
\filldraw[fill opacity=0.8,fill=gray!20,draw=none](-8.156,1.432)--(-8.173,1.445)--(-8.203,1.452)--cycle;
\draw(-8.173,1.445)--(-8.203,1.452);
\filldraw[fill opacity=0.8,fill=gray!20,draw=none](-8.194,1.464)--(-8.183,1.447)--(-8.173,1.445)--cycle;
\draw(-8.183,1.447)--(-8.173,1.445);
\filldraw[fill opacity=0.8,fill=gray!20,draw=none](-8.465,1.316)--(-8.475,1.342)--(-8.437,1.313)--(-8.437,1.311)--cycle;
\draw(-8.437,1.313)--(-8.437,1.311);
\filldraw[fill opacity=0.8,fill=gray!20,draw=none](-8.434,1.31)--(-8.437,1.311)--(-8.437,1.313)--cycle;
\draw(-8.437,1.311)--(-8.437,1.313);
\filldraw[fill opacity=0.8,fill=gray!20,draw=none](-8.434,1.31)--(-8.426,1.301)--(-8.437,1.302)--(-8.437,1.311)--cycle;
\draw(-8.426,1.301)--(-8.437,1.302)--(-8.437,1.311);
\filldraw[fill opacity=0.8,fill=gray!20,draw=none](-8.46,1.301)--(-8.465,1.316)--(-8.437,1.311)--(-8.437,1.302)--cycle;
\draw(-8.437,1.311)--(-8.437,1.302)--(-8.46,1.301);
\filldraw[fill opacity=0.8,fill=gray!20,draw=none](-8.513,1.325)--(-8.508,1.322)--(-8.511,1.321)--cycle;
\draw(-8.508,1.322)--(-8.511,1.321);
\filldraw[fill opacity=0.8,fill=gray!20,draw=none](-8.579,1.306)--(-8.508,1.322)--(-8.48,1.29)--cycle;
\draw(-8.579,1.306)--(-8.508,1.322);
\filldraw[fill opacity=0.8,fill=gray!20](-8.081,1.465)--(-8.49,1.286)--(-8.528,1.318)--(-8.119,1.497)--cycle;
\filldraw[fill opacity=0.8,fill=gray!20,draw=none](-4.248,2.971)--(-4.242,2.969)--(-4.245,2.988)--(-4.246,2.991)--cycle;
\draw(-4.245,2.988)--(-4.246,2.991);
\filldraw[fill opacity=0.8,fill=gray!20,draw=none](-4.218,2.989)--(-4.217,2.992)--(-4.259,2.964)--(-4.279,2.937)--cycle;
\draw(-4.259,2.964)--(-4.279,2.937);
\filldraw[fill opacity=0.8,fill=gray!20,draw=none](-4.23,2.848)--(-4.227,2.882)--(-4.235,2.876)--cycle;
\draw(-4.227,2.882)--(-4.235,2.876);
\filldraw[fill opacity=0.8,fill=gray!20,draw=none](-4.46,2.676)--(-4.464,2.671)--(-4.468,2.677)--cycle;
\draw(-4.46,2.676)--(-4.464,2.671);
\filldraw[fill opacity=0.8,fill=gray!20](-7.866,1.704)--(-7.897,1.745)--(-7.881,1.728)--(-7.846,1.683)--cycle;
\filldraw[fill opacity=0.8,fill=gray!20,draw=none](-4.576,2.723)--(-4.596,2.725)--(-4.586,2.716)--(-4.572,2.72)--cycle;
\draw(-4.586,2.716)--(-4.572,2.72);
\filldraw[fill opacity=0.8,fill=gray!20,draw=none](-4.577,3.026)--(-4.536,3.065)--(-4.547,3.077)--(-4.549,3.076)--(-4.593,3.044)--cycle;
\draw(-4.549,3.076)--(-4.593,3.044)--(-4.577,3.026)--(-4.536,3.065)--(-4.547,3.077);
\filldraw[fill opacity=0.8,fill=gray!20,draw=none](-4.248,2.971)--(-4.246,2.991)--(-4.248,2.995)--(-4.256,2.99)--cycle;
\draw(-4.246,2.991)--(-4.248,2.995)--(-4.256,2.99);
\filldraw[fill opacity=0.8,fill=gray!20,draw=none](-4.252,2.973)--(-4.248,2.971)--(-4.256,2.99)--(-4.262,2.986)--cycle;
\draw(-4.256,2.99)--(-4.262,2.986);
\filldraw[fill opacity=0.8,fill=gray!20,draw=none](-4.217,2.992)--(-4.212,3.029)--(-4.259,2.964)--cycle;
\draw(-4.212,3.029)--(-4.259,2.964);
\filldraw[fill opacity=0.8,fill=gray!20,draw=none](-8.24,1.565)--(-8.228,1.597)--(-8.226,1.613)--(-8.232,1.62)--(-8.252,1.603)--(-8.255,1.581)--cycle;
\draw(-8.228,1.597)--(-8.226,1.613)--(-8.232,1.62);
\draw(-8.252,1.603)--(-8.255,1.581)--(-8.24,1.565);
\filldraw[fill opacity=0.8,fill=gray!20,draw=none](-8.24,1.565)--(-8.232,1.557)--(-8.228,1.597)--cycle;
\draw(-8.24,1.565)--(-8.232,1.557)--(-8.228,1.597);
\filldraw[fill opacity=0.8,fill=gray!20,draw=none](-8.222,1.606)--(-8.228,1.597)--(-8.232,1.557)--cycle;
\draw(-8.228,1.597)--(-8.232,1.557);
\filldraw[fill opacity=0.8,fill=gray!20,draw=none](-8.222,1.606)--(-8.221,1.612)--(-8.226,1.613)--(-8.228,1.597)--cycle;
\draw(-8.221,1.612)--(-8.226,1.613)--(-8.228,1.597);
\filldraw[fill opacity=0.8,fill=gray!20,draw=none](-8.509,1.437)--(-8.517,1.462)--(-8.502,1.482)--(-8.5,1.482)--cycle;
\draw(-8.502,1.482)--(-8.5,1.482);
\filldraw[fill opacity=0.8,fill=gray!20,draw=none](-8.517,1.462)--(-8.509,1.437)--(-8.527,1.436)--(-8.526,1.451)--cycle;
\draw(-8.509,1.437)--(-8.527,1.436)--(-8.526,1.451);
\filldraw[fill opacity=0.8,fill=gray!20,draw=none](-8.545,1.432)--(-8.526,1.451)--(-8.527,1.436)--cycle;
\draw(-8.526,1.451)--(-8.527,1.436)--(-8.545,1.432);
\filldraw[fill opacity=0.8,fill=gray!20,draw=none](-8.517,1.462)--(-8.526,1.451)--(-8.524,1.481)--cycle;
\draw(-8.526,1.451)--(-8.524,1.481);
\filldraw[fill opacity=0.8,fill=gray!20,draw=none](-8.517,1.462)--(-8.524,1.481)--(-8.502,1.482)--cycle;
\draw(-8.524,1.481)--(-8.502,1.482);
\filldraw[fill opacity=0.8,fill=gray!20,draw=none](-8.5,1.484)--(-8.5,1.482)--(-8.502,1.482)--cycle;
\draw(-8.5,1.482)--(-8.502,1.482);
\filldraw[fill opacity=0.8,fill=gray!20,draw=none](-8.5,1.509)--(-8.5,1.484)--(-8.502,1.482)--(-8.524,1.481)--(-8.521,1.496)--cycle;
\draw(-8.502,1.482)--(-8.524,1.481)--(-8.521,1.496);
\filldraw[fill opacity=0.8,fill=gray!20,draw=none](-8.544,1.421)--(-8.563,1.459)--(-8.537,1.465)--cycle;
\draw(-8.563,1.459)--(-8.537,1.465);
\filldraw[fill opacity=0.8,fill=gray!20,draw=none](-8.538,1.468)--(-8.537,1.465)--(-8.54,1.464)--cycle;
\draw(-8.537,1.465)--(-8.54,1.464);
\filldraw[fill opacity=0.8,fill=gray!20](-8.154,1.591)--(-8.563,1.413)--(-8.554,1.461)--(-8.145,1.64)--cycle;
\filldraw[fill opacity=0.8,fill=gray!20,draw=none](-4.253,2.796)--(-4.243,2.809)--(-4.245,2.816)--cycle;
\filldraw[fill opacity=0.8,fill=gray!20](-8.084,1.024)--(-8.031,1.045)--(-8.022,1.051)--(-8.066,1.036)--cycle;
\filldraw[fill opacity=0.8,fill=gray!20,draw=none](-4.89,2.692)--(-4.815,2.727)--(-4.747,2.769)--cycle;
\draw(-4.815,2.727)--(-4.747,2.769);
\filldraw[fill opacity=0.8,fill=gray!20,draw=none](-6.15,.414)--(-6.097,.436)--(-6.068,.406)--(-6.074,.376)--(-6.102,.327)--(-6.145,.291)--(-6.156,.288)--cycle;
\draw(-6.068,.406)--(-6.074,.376)--(-6.102,.327)--(-6.145,.291)--(-6.156,.288);
\filldraw[fill opacity=0.8,fill=gray!20,draw=none](-6.035,.349)--(-6.041,.342)--(-6.089,.349)--(-6.074,.376)--(-6.034,.359)--cycle;
\draw(-6.089,.349)--(-6.074,.376)--(-6.034,.359);
\filldraw[fill opacity=0.8,fill=gray!20,draw=none](-6.041,.383)--(-6.034,.359)--(-6.074,.376)--(-6.065,.423)--cycle;
\draw(-6.034,.359)--(-6.074,.376)--(-6.065,.423);
\filldraw[fill opacity=0.8,fill=gray!20,draw=none](-6.041,.342)--(-6.067,.311)--(-6.102,.327)--(-6.089,.349)--cycle;
\draw(-6.067,.311)--(-6.102,.327)--(-6.089,.349);
\filldraw[fill opacity=0.8,fill=gray!20,draw=none](-6.123,.282)--(-6.145,.291)--(-6.102,.327)--(-6.077,.316)--cycle;
\draw(-6.123,.282)--(-6.145,.291)--(-6.102,.327)--(-6.077,.316);
\filldraw[fill opacity=0.8,fill=gray!20,draw=none](-6.097,.436)--(-6.067,.449)--(-6.064,.432)--(-6.068,.406)--cycle;
\draw(-6.067,.449)--(-6.064,.432)--(-6.068,.406);
\filldraw[fill opacity=0.8,fill=gray!20,draw=none](-6.041,.383)--(-6.065,.423)--(-6.064,.432)--(-6.054,.428)--cycle;
\draw(-6.065,.423)--(-6.064,.432)--(-6.054,.428);
\filldraw[fill opacity=0.8,fill=gray!20,draw=none](-6.062,.44)--(-6.057,.429)--(-6.064,.432)--(-6.067,.449)--cycle;
\draw(-6.057,.429)--(-6.064,.432)--(-6.067,.449);
\filldraw[fill opacity=0.8,fill=gray!20,draw=none](-6.062,.44)--(-6.054,.428)--(-6.057,.429)--cycle;
\draw(-6.054,.428)--(-6.057,.429);
\filldraw[fill opacity=0.8,fill=gray!20,draw=none](-6.123,.282)--(-6.077,.316)--(-6.067,.311)--cycle;
\draw(-6.077,.316)--(-6.067,.311);
\filldraw[fill opacity=0.8,fill=gray!20,draw=none](-6.09,.348)--(-6.067,.328)--(-6.067,.29)--(-6.123,.282)--(-6.123,.341)--cycle;
\draw(-6.067,.328)--(-6.067,.29);
\draw(-6.123,.282)--(-6.123,.341);
\filldraw[fill opacity=0.8,fill=gray!20,draw=none](-6.19,.273)--(-6.195,.275)--(-6.145,.291)--(-6.123,.282)--cycle;
\draw(-6.19,.273)--(-6.195,.275)--(-6.145,.291)--(-6.123,.282);
\filldraw[fill opacity=0.8,fill=gray!20,draw=none](-6.224,.357)--(-6.151,.391)--(-6.156,.288)--(-6.192,.276)--cycle;
\draw(-6.156,.288)--(-6.192,.276);
\filldraw[fill opacity=0.8,fill=gray!20,draw=none](-6.067,.25)--(-6.067,.234)--(-6.123,.235)--(-6.123,.26)--cycle;
\draw(-6.067,.25)--(-6.067,.234)--(-6.123,.235)--(-6.123,.26);
\filldraw[fill opacity=0.8,fill=gray!20,draw=none](-6.178,.268)--(-6.19,.273)--(-6.154,.278)--cycle;
\draw(-6.178,.268)--(-6.19,.273);
\filldraw[fill opacity=0.8,fill=gray!20,draw=none](-6.178,.268)--(-6.154,.278)--(-6.123,.282)--cycle;
\filldraw[fill opacity=0.8,fill=gray!20,draw=none](-6.136,.354)--(-6.123,.333)--(-6.123,.282)--(-6.178,.268)--(-6.178,.342)--cycle;
\draw(-6.123,.333)--(-6.123,.282);
\draw(-6.178,.268)--(-6.178,.342);
\filldraw[fill opacity=0.8,fill=gray!20](-6.94,.325)--(-6.939,.382)--(-6.835,.374)--(-6.847,.319)--cycle;
\filldraw[fill opacity=0.8,fill=gray!20](-7.037,.321)--(-7.046,.377)--(-6.939,.382)--(-6.94,.325)--cycle;
\filldraw[fill opacity=0.8,fill=gray!20](-6.942,.273)--(-6.94,.325)--(-6.847,.319)--(-6.866,.268)--cycle;
\filldraw[fill opacity=0.8,fill=gray!20](-7.021,.27)--(-7.037,.321)--(-6.94,.325)--(-6.942,.273)--cycle;
\filldraw[fill opacity=0.8,fill=gray!20,draw=none](-6.144,.366)--(-6.123,.37)--(-6.123,.333)--cycle;
\draw(-6.123,.37)--(-6.123,.333);
\filldraw[fill opacity=0.8,fill=gray!20,draw=none](-6.111,.367)--(-6.09,.348)--(-6.123,.341)--(-6.123,.37)--cycle;
\draw(-6.123,.341)--(-6.123,.37);
\filldraw[fill opacity=0.8,fill=gray!20,draw=none](-6.041,.342)--(-6.036,.342)--(-6.037,.336)--(-6.067,.311)--cycle;
\filldraw[fill opacity=0.8,fill=gray!20,draw=none](-6.061,.353)--(-6.039,.337)--(-6.067,.326)--(-6.067,.353)--cycle;
\draw(-6.067,.326)--(-6.067,.353);
\filldraw[fill opacity=0.8,fill=gray!20,draw=none](-6.09,.348)--(-6.067,.353)--(-6.067,.328)--cycle;
\draw(-6.067,.353)--(-6.067,.328);
\filldraw[fill opacity=0.8,fill=gray!20,draw=none](-6.09,.348)--(-6.111,.367)--(-6.067,.357)--(-6.067,.353)--cycle;
\draw(-6.067,.357)--(-6.067,.353);
\filldraw[fill opacity=0.8,fill=gray!20,draw=none](-6.061,.353)--(-6.067,.353)--(-6.067,.357)--cycle;
\draw(-6.067,.353)--(-6.067,.357);
\filldraw[fill opacity=0.8,fill=gray!20,draw=none](-6.264,.361)--(-6.253,.371)--(-6.233,.379)--(-6.192,.276)--(-6.195,.275)--(-6.245,.28)--(-6.284,.305)--cycle;
\draw(-6.192,.276)--(-6.195,.275)--(-6.245,.28)--(-6.284,.305);
\filldraw[fill opacity=0.8,fill=gray!20,draw=none](-6.224,.357)--(-6.233,.379)--(-6.15,.414)--(-6.151,.391)--cycle;
\filldraw[fill opacity=0.8,fill=gray!20,draw=none](-6.233,.275)--(-6.245,.28)--(-6.195,.275)--(-6.178,.268)--cycle;
\draw(-6.233,.275)--(-6.245,.28)--(-6.195,.275)--(-6.178,.268);
\filldraw[fill opacity=0.8,fill=gray!20,draw=none](-6.123,.26)--(-6.123,.235)--(-6.163,.233)--(-6.178,.242)--(-6.178,.268)--cycle;
\draw(-6.123,.26)--(-6.123,.235)--(-6.163,.233);
\draw(-6.178,.242)--(-6.178,.268);
\filldraw[fill opacity=0.8,fill=gray!20](-6.103,.172)--(-6.159,.173)--(-6.209,.179)--(-6.244,.189)--(-6.259,.201)--(-6.252,.214)--(-6.223,.225)--(-6.178,.232)--(-6.123,.235)--(-6.067,.234)--(-6.017,.228)--(-5.982,.218)--(-5.967,.206)--(-5.974,.193)--(-6.002,.183)--(-6.047,.175)--cycle;
\filldraw[fill opacity=0.8,fill=gray!20,draw=none](-6.223,.271)--(-6.208,.272)--(-6.178,.268)--cycle;
\filldraw[fill opacity=0.8,fill=gray!20,draw=none](-6.223,.271)--(-6.233,.275)--(-6.208,.272)--cycle;
\draw(-6.223,.271)--(-6.233,.275);
\filldraw[fill opacity=0.8,fill=gray!20,draw=none](-6.185,.36)--(-6.178,.37)--(-6.178,.268)--(-6.207,.27)--cycle;
\draw(-6.178,.37)--(-6.178,.268);
\filldraw[fill opacity=0.8,fill=gray!20,draw=none](-6.264,.361)--(-6.284,.305)--(-6.287,.307)--(-6.301,.327)--cycle;
\draw(-6.284,.305)--(-6.287,.307)--(-6.301,.327);
\filldraw[fill opacity=0.8,fill=gray!20,draw=none](-6.22,.313)--(-6.192,.333)--(-6.2,.3)--(-6.223,.292)--(-6.223,.308)--cycle;
\draw(-6.223,.292)--(-6.223,.308);
\filldraw[fill opacity=0.8,fill=gray!20,draw=none](-6.22,.313)--(-6.194,.348)--(-6.187,.353)--(-6.192,.333)--cycle;
\filldraw[fill opacity=0.8,fill=gray!20,draw=none](-6.194,.348)--(-6.138,.327)--(-6.135,.312)--(-6.22,.313)--cycle;
\draw(-6.135,.312)--(-6.22,.313);
\filldraw[fill opacity=0.8,fill=gray!20,draw=none](-6.2,.3)--(-6.207,.27)--(-6.223,.271)--(-6.223,.292)--cycle;
\draw(-6.223,.271)--(-6.223,.292);
\filldraw[fill opacity=0.8,fill=gray!20,draw=none](-6.138,.327)--(-6.108,.316)--(-6.108,.312)--(-6.135,.312)--cycle;
\draw(-6.108,.316)--(-6.108,.312)--(-6.135,.312);
\filldraw[fill opacity=0.8,fill=gray!20](-6.939,.382)--(-6.938,.439)--(-6.831,.431)--(-6.835,.374)--cycle;
\filldraw[fill opacity=0.8,fill=gray!20](-7.046,.377)--(-7.05,.434)--(-6.938,.439)--(-6.939,.382)--cycle;
\filldraw[fill opacity=0.8,fill=gray!20,draw=none](-6.301,.327)--(-6.264,.361)--(-6.227,.361)--(-6.208,.354)--(-6.222,.313)--(-6.287,.313)--cycle;
\draw(-6.264,.361)--(-6.227,.361);
\draw(-6.222,.313)--(-6.287,.313);
\filldraw[fill opacity=0.8,fill=gray!20,draw=none](-6.208,.354)--(-6.194,.348)--(-6.22,.313)--(-6.222,.313)--cycle;
\draw(-6.22,.313)--(-6.222,.313);
\filldraw[fill opacity=0.8,fill=gray!20,draw=none](-6.223,.292)--(-6.223,.271)--(-6.238,.281)--cycle;
\draw(-6.223,.292)--(-6.223,.271);
\filldraw[fill opacity=0.8,fill=gray!20,draw=none](-6.223,.31)--(-6.223,.292)--(-6.238,.281)--(-6.241,.283)--cycle;
\draw(-6.223,.31)--(-6.223,.292);
\filldraw[fill opacity=0.8,fill=gray!20,draw=none](-6.22,.313)--(-6.223,.308)--(-6.223,.31)--cycle;
\draw(-6.223,.308)--(-6.223,.31);
\filldraw[fill opacity=0.8,fill=gray!20,draw=none](-6.148,.312)--(-6.135,.312)--(-6.138,.308)--cycle;
\draw(-6.148,.312)--(-6.135,.312);
\filldraw[fill opacity=0.8,fill=gray!20,draw=none](-6.109,.303)--(-6.108,.312)--(-6.108,.316)--cycle;
\draw(-6.109,.303)--(-6.108,.312)--(-6.108,.316);
\filldraw[fill opacity=0.8,fill=gray!20,draw=none](-6.138,.308)--(-6.135,.312)--(-6.108,.312)--(-6.109,.303)--(-6.114,.3)--cycle;
\draw(-6.135,.312)--(-6.108,.312)--(-6.109,.303);
\filldraw[fill opacity=0.8,fill=gray!20,draw=none](-4.666,2.44)--(-6.1,.294)--(-6.146,.314)--(-4.579,2.661)--cycle;
\draw(-4.666,2.44)--(-6.1,.294)--(-6.146,.314)--(-4.579,2.661);
\filldraw[fill opacity=0.8,fill=gray!20,draw=none](-4.451,3.104)--(-4.438,3.104)--(-4.438,3.104)--(-4.448,3.106)--cycle;
\draw(-4.451,3.104)--(-4.438,3.104)--(-4.438,3.104)--(-4.448,3.106);
\filldraw[fill opacity=0.8,fill=gray!20,draw=none](-4.572,2.72)--(-4.574,2.721)--(-4.576,2.723)--cycle;
\filldraw[fill opacity=0.8,fill=gray!20](-7.927,.624)--(-7.883,.639)--(-7.925,.63)--(-7.949,.619)--cycle;
\filldraw[fill opacity=0.8,fill=gray!20,draw=none](-4.434,2.676)--(-4.427,2.681)--(-4.429,2.681)--(-4.444,2.678)--(-4.443,2.676)--cycle;
\draw(-4.427,2.681)--(-4.429,2.681);
\draw(-4.444,2.678)--(-4.443,2.676);
\filldraw[fill opacity=0.8,fill=gray!20,draw=none](-4.434,2.676)--(-4.403,2.68)--(-4.39,2.682)--(-4.389,2.683)--(-4.427,2.681)--cycle;
\draw(-4.39,2.682)--(-4.389,2.683)--(-4.427,2.681);
\filldraw[fill opacity=0.8,fill=gray!20,draw=none](-4.428,2.679)--(-4.446,2.652)--(-4.464,2.671)--(-4.46,2.676)--cycle;
\draw(-4.428,2.679)--(-4.446,2.652);
\draw(-4.464,2.671)--(-4.46,2.676);
\filldraw[fill opacity=0.8,fill=gray!20,draw=none](-4.466,2.712)--(-4.463,2.711)--(-4.469,2.712)--cycle;
\filldraw[fill opacity=0.8,fill=gray!20,draw=none](-4.535,2.637)--(-4.666,2.44)--(-4.58,2.658)--cycle;
\draw(-4.535,2.637)--(-4.666,2.44);
\filldraw[fill opacity=0.8,fill=gray!20,draw=none](-4.331,2.853)--(-4.314,2.831)--(-4.292,2.836)--(-4.297,2.845)--cycle;
\draw(-4.314,2.831)--(-4.292,2.836);
\filldraw[fill opacity=0.8,fill=gray!20,draw=none](-4.343,2.856)--(-4.342,2.845)--(-4.314,2.831)--(-4.331,2.853)--cycle;
\filldraw[fill opacity=0.8,fill=gray!20,draw=none](-4.379,2.864)--(-4.315,2.811)--(-4.251,2.821)--(-4.273,2.839)--cycle;
\draw(-4.379,2.864)--(-4.315,2.811);
\draw(-4.251,2.821)--(-4.273,2.839);
\filldraw[fill opacity=0.8,fill=gray!20](-8.127,1.777)--(-8.083,1.792)--(-8.061,1.797)--(-8.085,1.786)--cycle;
\filldraw[fill opacity=0.8,fill=gray!20,draw=none](-4.257,2.836)--(-4.26,2.842)--(-4.273,2.839)--cycle;
\draw(-4.26,2.842)--(-4.273,2.839);
\filldraw[fill opacity=0.8,fill=gray!20,draw=none](-4.642,2.94)--(-4.628,2.925)--(-4.617,2.953)--(-4.641,2.95)--cycle;
\draw(-4.642,2.94)--(-4.628,2.925)--(-4.617,2.953);
\filldraw[fill opacity=0.8,fill=gray!20,draw=none](-4.534,2.701)--(-4.55,2.7)--(-4.532,2.694)--(-4.529,2.696)--cycle;
\draw(-4.532,2.694)--(-4.529,2.696);
\filldraw[fill opacity=0.8,fill=gray!20,draw=none](-4.516,2.979)--(-4.502,3)--(-4.515,2.997)--cycle;
\draw(-4.516,2.979)--(-4.502,3);
\filldraw[fill opacity=0.8,fill=gray!20,draw=none](-4.694,2.909)--(-4.696,2.912)--(-4.694,2.911)--cycle;
\draw(-4.696,2.912)--(-4.694,2.911);
\filldraw[fill opacity=0.8,fill=gray!20](-7.785,.728)--(-7.763,.779)--(-7.797,.757)--(-7.816,.707)--cycle;
\filldraw[fill opacity=0.8,fill=gray!20,draw=none](-4.248,2.993)--(-4.249,2.996)--(-4.249,2.995)--cycle;
\draw(-4.249,2.996)--(-4.249,2.995);
\filldraw[fill opacity=0.8,fill=gray!20,draw=none](-4.694,2.909)--(-4.7,2.884)--(-4.744,2.906)--(-4.714,2.922)--(-4.696,2.912)--cycle;
\draw(-4.7,2.884)--(-4.744,2.906);
\draw(-4.714,2.922)--(-4.696,2.912);
\filldraw[fill opacity=0.8,fill=gray!20,draw=none](-4.463,2.711)--(-4.459,2.71)--(-4.46,2.709)--cycle;
\draw(-4.459,2.71)--(-4.46,2.709);
\filldraw[fill opacity=0.8,fill=gray!20,draw=none](-4.574,2.721)--(-4.577,2.723)--(-4.577,2.724)--cycle;
\draw(-4.574,2.721)--(-4.577,2.723)--(-4.577,2.724);
\filldraw[fill opacity=0.8,fill=gray!20,draw=none](-4.446,2.652)--(-4.507,2.561)--(-4.503,2.613)--(-4.464,2.671)--cycle;
\draw(-4.446,2.652)--(-4.507,2.561);
\draw(-4.503,2.613)--(-4.464,2.671);
\filldraw[fill opacity=0.8,fill=gray!20,draw=none](-4.579,2.727)--(-4.609,2.736)--(-4.596,2.725)--(-4.578,2.723)--cycle;
\filldraw[fill opacity=0.8,fill=gray!20,draw=none](-4.248,2.993)--(-4.249,2.995)--(-4.248,2.995)--(-4.246,2.991)--cycle;
\draw(-4.249,2.995)--(-4.248,2.995)--(-4.246,2.991);
\filldraw[fill opacity=0.8,fill=gray!20,draw=none](-4.577,2.724)--(-4.577,2.723)--(-4.579,2.727)--cycle;
\draw(-4.577,2.724)--(-4.577,2.723)--(-4.579,2.727);
\filldraw[fill opacity=0.8,fill=gray!20,draw=none](-4.421,2.718)--(-4.446,2.708)--(-4.46,2.709)--(-4.459,2.71)--cycle;
\draw(-4.46,2.709)--(-4.459,2.71);
\filldraw[fill opacity=0.8,fill=gray!20](-2.812,7.943)--(-2.802,7.991)--(-2.705,7.995)--(-2.704,7.948)--cycle;
\filldraw[fill opacity=0.8,fill=gray!20](-2.704,7.948)--(-2.705,7.995)--(-2.612,7.988)--(-2.6,7.941)--cycle;
\filldraw[fill opacity=0.8,fill=gray!20,draw=none](-4.248,2.993)--(-4.246,2.991)--(-4.245,2.988)--cycle;
\draw(-4.246,2.991)--(-4.245,2.988);
\filldraw[fill opacity=0.8,fill=gray!20,draw=none](-4.265,2.77)--(-4.264,2.77)--(-4.264,2.77)--cycle;
\draw(-4.265,2.77)--(-4.264,2.77);
\filldraw[fill opacity=0.8,fill=gray!20,draw=none](-4.192,3.058)--(-4.1,3.187)--(-4.108,3.244)--(-4.25,3.045)--cycle;
\draw(-4.192,3.058)--(-4.1,3.187);
\draw(-4.108,3.244)--(-4.25,3.045);
\filldraw[fill opacity=0.8,fill=gray!20,draw=none](-4.307,2.721)--(-4.287,2.742)--(-4.282,2.749)--(-4.298,2.735)--(-4.308,2.72)--cycle;
\draw(-4.298,2.735)--(-4.308,2.72)--(-4.307,2.721);
\filldraw[fill opacity=0.8,fill=gray!20,draw=none](-4.534,2.701)--(-4.548,2.7)--(-4.57,2.718)--(-4.569,2.719)--cycle;
\draw(-4.57,2.718)--(-4.569,2.719);
\filldraw[fill opacity=0.8,fill=gray!20,draw=none](-4.574,2.722)--(-4.569,2.719)--(-4.576,2.711)--(-4.593,2.722)--cycle;
\draw(-4.569,2.719)--(-4.576,2.711);
\filldraw[fill opacity=0.8,fill=gray!20](-8.191,1.732)--(-8.145,1.766)--(-8.127,1.777)--(-8.165,1.749)--cycle;
\filldraw[fill opacity=0.8,fill=gray!20](-8.087,1.366)--(-8.134,1.382)--(-8.145,1.394)--(-8.093,1.372)--cycle;
\filldraw[fill opacity=0.8,fill=gray!20](-8.036,1.364)--(-8.087,1.366)--(-8.093,1.372)--(-8.036,1.364)--cycle;
\filldraw[fill opacity=0.8,fill=gray!20,draw=none](-4.446,2.708)--(-4.468,2.699)--(-4.46,2.709)--cycle;
\draw(-4.468,2.699)--(-4.46,2.709);
\filldraw[fill opacity=0.8,fill=gray!20,draw=none](-4.282,2.749)--(-4.265,2.77)--(-4.279,2.76)--(-4.298,2.735)--cycle;
\draw(-4.265,2.77)--(-4.279,2.76)--(-4.298,2.735);
\filldraw[fill opacity=0.8,fill=gray!20,draw=none](-4.684,2.877)--(-4.685,2.876)--(-4.7,2.884)--(-4.694,2.909)--cycle;
\draw(-4.685,2.876)--(-4.7,2.884);
\filldraw[fill opacity=0.8,fill=gray!20](-8,1.056)--(-7.974,1.052)--(-7.974,1.052)--(-7.971,1.057)--cycle;
\filldraw[fill opacity=0.8,fill=gray!20](-7.971,1.057)--(-7.974,1.052)--(-7.974,1.052)--(-7.943,1.055)--cycle;
\filldraw[fill opacity=0.8,fill=gray!20,draw=none](-4.576,2.723)--(-4.579,2.727)--(-4.577,2.723)--cycle;
\draw(-4.579,2.727)--(-4.577,2.723)--(-4.576,2.723);
\filldraw[fill opacity=0.8,fill=gray!20,draw=none](-8.183,1.421)--(-8.175,1.412)--(-8.178,1.416)--cycle;
\draw(-8.183,1.421)--(-8.175,1.412)--(-8.178,1.416);
\filldraw[fill opacity=0.8,fill=gray!20,draw=none](-8.124,1.38)--(-8.113,1.39)--(-8.121,1.399)--(-8.175,1.412)--(-8.134,1.382)--cycle;
\draw(-8.113,1.39)--(-8.121,1.399)--(-8.175,1.412)--(-8.134,1.382)--(-8.124,1.38);
\filldraw[fill opacity=0.8,fill=gray!20,draw=none](-8.121,1.399)--(-8.127,1.41)--(-8.134,1.415)--(-8.178,1.416)--(-8.175,1.412)--cycle;
\draw(-8.178,1.416)--(-8.175,1.412)--(-8.121,1.399)--(-8.127,1.41);
\filldraw[fill opacity=0.8,fill=gray!20,draw=none](-8.103,1.397)--(-8.084,1.411)--(-8.111,1.417)--(-8.13,1.415)--(-8.121,1.399)--cycle;
\draw(-8.13,1.415)--(-8.121,1.399)--(-8.103,1.397);
\filldraw[fill opacity=0.8,fill=gray!20,draw=none](-8.156,1.432)--(-8.127,1.41)--(-8.133,1.422)--cycle;
\draw(-8.127,1.41)--(-8.133,1.422);
\filldraw[fill opacity=0.8,fill=gray!20,draw=none](-8.111,1.417)--(-8.133,1.422)--(-8.13,1.415)--cycle;
\draw(-8.133,1.422)--(-8.13,1.415);
\filldraw[fill opacity=0.8,fill=gray!20,draw=none](-8.4,1.299)--(-8.426,1.301)--(-8.434,1.31)--cycle;
\draw(-8.4,1.299)--(-8.426,1.301);
\filldraw[fill opacity=0.8,fill=gray!20,draw=none](-8.438,1.288)--(-8.437,1.302)--(-8.4,1.299)--(-8.39,1.295)--cycle;
\draw(-8.438,1.288)--(-8.437,1.302)--(-8.4,1.299);
\filldraw[fill opacity=0.8,fill=gray!20,draw=none](-8.409,1.263)--(-8.44,1.265)--(-8.438,1.288)--(-8.39,1.295)--(-8.378,1.291)--(-8.386,1.276)--cycle;
\draw(-8.409,1.263)--(-8.44,1.265)--(-8.438,1.288);
\draw(-8.378,1.291)--(-8.386,1.276);
\filldraw[fill opacity=0.8,fill=gray!20,draw=none](-8.486,1.263)--(-8.492,1.276)--(-8.438,1.288)--(-8.44,1.265)--cycle;
\draw(-8.438,1.288)--(-8.44,1.265)--(-8.486,1.263)--(-8.492,1.276);
\filldraw[fill opacity=0.8,fill=gray!20,draw=none](-8.48,1.29)--(-8.473,1.289)--(-8.478,1.288)--cycle;
\draw(-8.473,1.289)--(-8.478,1.288);
\filldraw[fill opacity=0.8,fill=gray!20,draw=none](-9.492,1.08)--(-9.472,1.115)--(-8.6,.751)--(-8.625,.707)--(-9.494,1.07)--cycle;
\draw(-9.472,1.115)--(-8.6,.751);
\draw(-8.625,.707)--(-9.494,1.07);
\filldraw[fill opacity=0.8,fill=gray!20](-8.466,1.237)--(-8.486,1.263)--(-8.44,1.265)--(-8.442,1.238)--cycle;
\filldraw[fill opacity=0.8,fill=gray!20](-8.485,1.233)--(-8.521,1.256)--(-8.486,1.263)--(-8.466,1.237)--cycle;
\filldraw[fill opacity=0.8,fill=gray!20](-8.492,1.228)--(-8.536,1.247)--(-8.521,1.256)--(-8.485,1.233)--cycle;
\filldraw[fill opacity=0.8,fill=gray!20,draw=none](-8.529,1.239)--(-8.544,1.251)--(-8.545,1.253)--(-8.536,1.247)--cycle;
\draw(-8.545,1.253)--(-8.536,1.247)--(-8.529,1.239);
\filldraw[fill opacity=0.8,fill=gray!20,draw=none](-8.544,1.251)--(-8.573,1.274)--(-8.574,1.275)--(-8.545,1.253)--cycle;
\draw(-8.573,1.274)--(-8.574,1.275)--(-8.545,1.253);
\filldraw[fill opacity=0.8,fill=gray!20,draw=none](-8.573,1.274)--(-8.566,1.268)--(-8.568,1.268)--cycle;
\draw(-8.566,1.268)--(-8.568,1.268);
\filldraw[fill opacity=0.8,fill=gray!20,draw=none](-8.566,1.268)--(-8.544,1.251)--(-8.544,1.249)--cycle;
\filldraw[fill opacity=0.8,fill=gray!20,draw=none](-8.544,1.251)--(-8.529,1.239)--(-8.527,1.236)--(-8.544,1.249)--cycle;
\draw(-8.529,1.239)--(-8.527,1.236)--(-8.544,1.249);
\filldraw[fill opacity=0.8,fill=gray!20,draw=none](-8.566,1.268)--(-8.473,1.289)--(-8.443,1.272)--(-8.494,1.26)--(-8.544,1.249)--cycle;
\draw(-8.566,1.268)--(-8.473,1.289);
\draw(-8.494,1.26)--(-8.544,1.249);
\filldraw[fill opacity=0.8,fill=gray!20](-8.036,1.45)--(-8.445,1.271)--(-8.49,1.286)--(-8.081,1.465)--cycle;
\filldraw[fill opacity=0.8,fill=gray!20,draw=none](-4.586,2.65)--(-4.61,2.615)--(-4.648,2.648)--(-4.636,2.667)--cycle;
\draw(-4.586,2.65)--(-4.61,2.615);
\draw(-4.648,2.648)--(-4.636,2.667);
\filldraw[fill opacity=0.8,fill=gray!20,draw=none](-4.583,2.728)--(-4.574,2.722)--(-4.593,2.722)--(-4.61,2.734)--(-4.609,2.736)--cycle;
\filldraw[fill opacity=0.8,fill=gray!20](-8.036,1.364)--(-8.01,1.36)--(-8.039,1.359)--(-8.036,1.364)--cycle;
\filldraw[fill opacity=0.8,fill=gray!20](-8.036,1.364)--(-7.988,1.365)--(-8.01,1.36)--(-8.036,1.364)--cycle;
\filldraw[fill opacity=0.8,fill=gray!20](-8.036,1.364)--(-7.979,1.371)--(-7.988,1.365)--(-8.036,1.364)--cycle;
\filldraw[fill opacity=0.8,fill=gray!20](-8.036,1.364)--(-8.067,1.361)--(-8.087,1.366)--(-8.036,1.364)--cycle;
\filldraw[fill opacity=0.8,fill=gray!20](-8.036,1.364)--(-8.039,1.359)--(-8.067,1.361)--(-8.036,1.364)--cycle;
\filldraw[fill opacity=0.8,fill=gray!20](-7.975,1.784)--(-8.005,1.796)--(-7.985,1.791)--(-7.938,1.775)--cycle;
\filldraw[fill opacity=0.8,fill=gray!20](-2.9,7.873)--(-2.894,7.927)--(-2.812,7.943)--(-2.815,7.889)--cycle;
\filldraw[fill opacity=0.8,fill=gray!20,draw=none](-4.429,3.105)--(-4.438,3.104)--(-4.438,3.104)--(-4.427,3.104)--cycle;
\draw(-4.429,3.105)--(-4.438,3.104)--(-4.438,3.104)--(-4.427,3.104);
\filldraw[fill opacity=0.8,fill=gray!20,draw=none](-4.259,2.964)--(-4.253,2.973)--(-4.262,2.986)--(-4.27,2.992)--(-4.293,2.985)--(-4.324,2.942)--cycle;
\draw(-4.259,2.964)--(-4.253,2.973);
\draw(-4.293,2.985)--(-4.324,2.942);
\filldraw[fill opacity=0.8,fill=gray!20,draw=none](-4.276,2.977)--(-4.248,2.995)--(-4.262,3.012)--cycle;
\draw(-4.276,2.977)--(-4.248,2.995)--(-4.262,3.012);
\filldraw[fill opacity=0.8,fill=gray!20,draw=none](-4.265,2.77)--(-4.264,2.77)--(-4.276,2.769)--(-4.279,2.76)--cycle;
\draw(-4.276,2.769)--(-4.279,2.76)--(-4.265,2.77);
\filldraw[fill opacity=0.8,fill=gray!20,draw=none](-4.641,2.927)--(-4.63,2.902)--(-4.628,2.925)--(-4.642,2.94)--cycle;
\draw(-4.63,2.902)--(-4.628,2.925)--(-4.642,2.94);
\filldraw[fill opacity=0.8,fill=gray!20,draw=none](-4.253,2.796)--(-4.274,2.769)--(-4.264,2.77)--cycle;
\filldraw[fill opacity=0.8,fill=gray!20,draw=none](-4.544,3.078)--(-4.547,3.077)--(-4.536,3.065)--(-4.516,3.076)--cycle;
\draw(-4.547,3.077)--(-4.536,3.065)--(-4.516,3.076);
\filldraw[fill opacity=0.8,fill=gray!20,draw=none](-4.457,2.811)--(-4.48,2.81)--(-4.46,2.799)--cycle;
\draw(-4.48,2.81)--(-4.46,2.799);
\filldraw[fill opacity=0.8,fill=gray!20,draw=none](-4.439,2.813)--(-4.457,2.811)--(-4.46,2.799)--cycle;
\filldraw[fill opacity=0.8,fill=gray!20,draw=none](-4.457,2.811)--(-4.493,2.808)--(-4.576,2.757)--(-4.55,2.72)--(-4.45,2.782)--cycle;
\draw(-4.493,2.808)--(-4.576,2.757);
\draw(-4.55,2.72)--(-4.45,2.782);
\filldraw[fill opacity=0.8,fill=gray!20,draw=none](-4.574,2.723)--(-4.526,2.711)--(-4.563,2.745)--cycle;
\draw(-4.574,2.723)--(-4.526,2.711);
\filldraw[fill opacity=0.8,fill=gray!20](-7.876,1.034)--(-7.924,1.05)--(-7.918,1.044)--(-7.865,1.022)--cycle;
\filldraw[fill opacity=0.8,fill=gray!20,draw=none](-4.302,2.855)--(-4.273,2.839)--(-4.26,2.842)--(-4.278,2.885)--(-4.315,2.877)--cycle;
\draw(-4.273,2.839)--(-4.26,2.842);
\draw(-4.278,2.885)--(-4.315,2.877);
\filldraw[fill opacity=0.8,fill=gray!20,draw=none](-4.484,2.755)--(-4.487,2.751)--(-4.484,2.752)--cycle;
\draw(-4.487,2.751)--(-4.484,2.752);
\filldraw[fill opacity=0.8,fill=gray!20,draw=none](-4.484,2.755)--(-4.484,2.752)--(-4.483,2.753)--(-4.483,2.754)--cycle;
\draw(-4.483,2.753)--(-4.483,2.754);
\filldraw[fill opacity=0.8,fill=gray!20,draw=none](-4.484,2.755)--(-4.483,2.754)--(-4.484,2.755)--cycle;
\draw(-4.483,2.754)--(-4.484,2.755);
\filldraw[fill opacity=0.8,fill=gray!20,draw=none](-4.45,2.782)--(-4.5,2.751)--(-4.49,2.72)--(-4.471,2.732)--cycle;
\draw(-4.45,2.782)--(-4.5,2.751);
\draw(-4.49,2.72)--(-4.471,2.732);
\filldraw[fill opacity=0.8,fill=gray!20,draw=none](-4.231,2.894)--(-4.23,2.88)--(-4.227,2.882)--cycle;
\draw(-4.23,2.88)--(-4.227,2.882);
\filldraw[fill opacity=0.8,fill=gray!20,draw=none](-4.534,2.701)--(-4.529,2.699)--(-4.541,2.695)--(-4.548,2.7)--cycle;
\filldraw[fill opacity=0.8,fill=gray!20,draw=none](-4.549,2.717)--(-4.544,2.701)--(-4.534,2.701)--cycle;
\filldraw[fill opacity=0.8,fill=gray!20,draw=none](-4.519,3.083)--(-4.491,3.09)--(-4.489,3.092)--(-4.495,3.098)--(-4.521,3.087)--cycle;
\draw(-4.491,3.09)--(-4.489,3.092)--(-4.495,3.098)--(-4.521,3.087);
\filldraw[fill opacity=0.8,fill=gray!20,draw=none](-4.61,2.615)--(-6.052,.455)--(-6.068,.455)--(-6.086,.496)--(-4.648,2.648)--cycle;
\draw(-4.61,2.615)--(-6.052,.455);
\draw(-6.086,.496)--(-4.648,2.648);
\filldraw[fill opacity=0.8,fill=gray!20,draw=none](-4.648,2.648)--(-6.114,.454)--(-6.146,.487)--(-4.668,2.7)--cycle;
\draw(-4.648,2.648)--(-6.114,.454);
\draw(-6.146,.487)--(-4.668,2.7);
\filldraw[fill opacity=0.8,fill=gray!20,draw=none](-4.878,2.385)--(-5.764,1.058)--(-5.763,1.117)--(-4.781,2.588)--cycle;
\draw(-4.878,2.385)--(-5.764,1.058);
\draw(-5.763,1.117)--(-4.781,2.588);
\filldraw[fill opacity=0.8,fill=gray!20,draw=none](-5.003,2.202)--(-5.001,2.227)--(-5.057,2.143)--(-5.03,2.128)--cycle;
\draw(-5.001,2.227)--(-5.057,2.143);
\filldraw[fill opacity=0.8,fill=gray!20,draw=none](-5.003,2.202)--(-5.03,2.128)--(-5.022,2.123)--(-5.008,2.144)--cycle;
\draw(-5.022,2.123)--(-5.008,2.144);
\filldraw[fill opacity=0.8,fill=gray!20,draw=none](-5.03,2.128)--(-5.057,2.143)--(-5.177,1.963)--(-5.09,2.021)--(-5.043,2.092)--cycle;
\draw(-5.057,2.143)--(-5.177,1.963);
\draw(-5.09,2.021)--(-5.043,2.092);
\filldraw[fill opacity=0.8,fill=gray!20,draw=none](-4.997,2.133)--(-5.008,2.144)--(-5.022,2.123)--cycle;
\draw(-5.008,2.144)--(-5.022,2.123);
\filldraw[fill opacity=0.8,fill=gray!20,draw=none](-4.977,2.115)--(-4.997,2.133)--(-5.022,2.123)--(-5.09,2.021)--cycle;
\draw(-5.022,2.123)--(-5.09,2.021);
\filldraw[fill opacity=0.8,fill=gray!20,draw=none](-5.03,2.128)--(-5.043,2.092)--(-5.022,2.123)--cycle;
\draw(-5.043,2.092)--(-5.022,2.123);
\filldraw[fill opacity=0.8,fill=gray!20,draw=none](-4.731,2.675)--(-4.781,2.588)--(-6.14,.553)--(-6.141,.553)--(-6.105,.631)--(-4.789,2.6)--cycle;
\draw(-4.781,2.588)--(-6.14,.553);
\draw(-6.105,.631)--(-4.789,2.6);
\filldraw[fill opacity=0.8,fill=gray!20,draw=none](-5.203,1.969)--(-5.575,1.413)--(-5.474,1.519)--(-5.177,1.963)--cycle;
\draw(-5.203,1.969)--(-5.575,1.413);
\draw(-5.474,1.519)--(-5.177,1.963);
\filldraw[fill opacity=0.8,fill=gray!20,draw=none](-5.177,1.963)--(-5.474,1.519)--(-5.295,1.715)--(-5.09,2.021)--cycle;
\draw(-5.177,1.963)--(-5.474,1.519);
\draw(-5.295,1.715)--(-5.09,2.021);
\filldraw[fill opacity=0.8,fill=gray!20,draw=none](-4.529,2.699)--(-4.524,2.696)--(-5.342,1.723)--(-4.839,2.398)--(-4.608,2.673)--cycle;
\draw(-4.524,2.696)--(-5.342,1.723);
\draw(-4.839,2.398)--(-4.608,2.673);
\filldraw[fill opacity=0.8,fill=gray!20](-8.022,1.051)--(-7.974,1.052)--(-7.974,1.052)--(-8,1.056)--cycle;
\filldraw[fill opacity=0.8,fill=gray!20](-7.979,1.371)--(-7.926,1.392)--(-7.944,1.38)--(-7.988,1.365)--cycle;
\filldraw[fill opacity=0.8,fill=gray!20,draw=none](-4.836,2.228)--(-4.904,2.181)--(-4.956,2.132)--(-4.971,2.11)--(-4.931,2.12)--cycle;
\draw(-4.956,2.132)--(-4.971,2.11);
\filldraw[fill opacity=0.8,fill=gray!20,draw=none](-4.904,2.181)--(-4.941,2.155)--(-4.956,2.132)--cycle;
\draw(-4.941,2.155)--(-4.956,2.132);
\filldraw[fill opacity=0.8,fill=gray!20,draw=none](-5.165,2.038)--(-5.583,1.412)--(-5.575,1.413)--(-5.203,1.969)--cycle;
\draw(-5.165,2.038)--(-5.583,1.412);
\draw(-5.575,1.413)--(-5.203,1.969);
\filldraw[fill opacity=0.8,fill=gray!20,draw=none](-4.997,2.133)--(-4.971,2.11)--(-4.941,2.155)--cycle;
\draw(-4.971,2.11)--(-4.941,2.155);
\filldraw[fill opacity=0.8,fill=gray!20,draw=none](-4.931,2.12)--(-4.971,2.11)--(-5.064,1.971)--cycle;
\draw(-4.971,2.11)--(-5.064,1.971);
\filldraw[fill opacity=0.8,fill=gray!20,draw=none](-4.977,2.115)--(-5.09,2.021)--(-5.295,1.715)--(-5.064,1.971)--(-4.971,2.11)--cycle;
\draw(-5.09,2.021)--(-5.295,1.715);
\draw(-5.064,1.971)--(-4.971,2.11);
\filldraw[fill opacity=0.8,fill=gray!20,draw=none](-5.764,1.058)--(-5.94,.795)--(-5.888,.931)--(-5.763,1.117)--cycle;
\draw(-5.764,1.058)--(-5.94,.795);
\draw(-5.888,.931)--(-5.763,1.117);
\filldraw[fill opacity=0.8,fill=gray!20,draw=none](-5.94,.795)--(-6.068,.604)--(-6.058,.676)--(-5.888,.931)--cycle;
\draw(-5.94,.795)--(-6.068,.604);
\draw(-6.058,.676)--(-5.888,.931);
\filldraw[fill opacity=0.8,fill=gray!20,draw=none](-5.474,1.519)--(-5.953,.802)--(-5.862,.865)--(-5.295,1.715)--cycle;
\draw(-5.474,1.519)--(-5.953,.802);
\draw(-5.862,.865)--(-5.295,1.715);
\filldraw[fill opacity=0.8,fill=gray!20,draw=none](-5.583,1.412)--(-6.105,.631)--(-6.038,.72)--(-5.575,1.413)--cycle;
\draw(-5.583,1.412)--(-6.105,.631);
\draw(-6.038,.72)--(-5.575,1.413);
\filldraw[fill opacity=0.8,fill=gray!20,draw=none](-5.575,1.413)--(-6.038,.72)--(-5.953,.802)--(-5.474,1.519)--cycle;
\draw(-5.575,1.413)--(-6.038,.72);
\draw(-5.953,.802)--(-5.474,1.519);
\filldraw[fill opacity=0.8,fill=gray!20,draw=none](-6,.943)--(-6.022,.952)--(-5.979,.988)--(-5.954,.976)--cycle;
\draw(-6,.943)--(-6.022,.952)--(-5.979,.988)--(-5.954,.976);
\filldraw[fill opacity=0.8,fill=gray!20,draw=none](-5.988,.919)--(-5.964,.948)--(-5.944,.95)--(-5.944,.91)--cycle;
\draw(-5.944,.95)--(-5.944,.91);
\filldraw[fill opacity=0.8,fill=gray!20,draw=none](-5.935,.979)--(-5.944,.972)--(-5.979,.988)--cycle;
\draw(-5.944,.972)--(-5.979,.988);
\filldraw[fill opacity=0.8,fill=gray!20,draw=none](-5.964,.948)--(-6,.904)--(-6,.943)--cycle;
\draw(-6,.904)--(-6,.943);
\filldraw[fill opacity=0.8,fill=gray!20,draw=none](-6,.943)--(-5.954,.976)--(-5.944,.972)--cycle;
\draw(-5.954,.976)--(-5.944,.972);
\filldraw[fill opacity=0.8,fill=gray!20,draw=none](-5.944,.977)--(-5.944,.972)--(-5.964,.948)--(-6,.943)--(-6,.952)--cycle;
\draw(-5.944,.977)--(-5.944,.972);
\draw(-6,.943)--(-6,.952);
\filldraw[fill opacity=0.8,fill=gray!20,draw=none](-5.935,.979)--(-5.907,.986)--(-5.894,.983)--(-5.894,.971)--(-5.944,.926)--(-5.944,.972)--cycle;
\draw(-5.894,.983)--(-5.894,.971);
\draw(-5.944,.926)--(-5.944,.972);
\filldraw[fill opacity=0.8,fill=gray!20,draw=none](-5.964,.948)--(-5.944,.972)--(-5.944,.95)--cycle;
\draw(-5.944,.972)--(-5.944,.95);
\filldraw[fill opacity=0.8,fill=gray!20,draw=none](-5.953,.922)--(-6,.943)--(-5.944,.972)--(-5.931,.967)--cycle;
\draw(-5.953,.922)--(-6,.943);
\draw(-5.944,.972)--(-5.931,.967);
\filldraw[fill opacity=0.8,fill=gray!20,draw=none](-5.922,.99)--(-5.907,.986)--(-5.935,.979)--cycle;
\filldraw[fill opacity=0.8,fill=gray!20,draw=none](-5.932,.982)--(-5.922,.99)--(-5.916,.976)--(-5.932,.979)--cycle;
\filldraw[fill opacity=0.8,fill=gray!20,draw=none](-5.932,.979)--(-5.916,.976)--(-5.909,.957)--(-5.931,.967)--cycle;
\draw(-5.909,.957)--(-5.931,.967);
\filldraw[fill opacity=0.8,fill=gray!20,draw=none](-5.917,.95)--(-5.894,.95)--(-5.894,.888)--(-5.944,.895)--(-5.944,.926)--cycle;
\draw(-5.894,.95)--(-5.894,.888)--(-5.944,.895)--(-5.944,.926);
\filldraw[fill opacity=0.8,fill=gray!20,draw=none](-5.917,.95)--(-5.894,.971)--(-5.894,.95)--cycle;
\draw(-5.894,.971)--(-5.894,.95);
\filldraw[fill opacity=0.8,fill=gray!20,draw=none](-5.923,.909)--(-5.953,.922)--(-5.931,.967)--(-5.909,.957)--cycle;
\draw(-5.923,.909)--(-5.953,.922);
\draw(-5.931,.967)--(-5.909,.957);
\filldraw[fill opacity=0.8,fill=gray!20,draw=none](-4.522,2.696)--(-4.524,2.632)--(-5.954,.931)--(-5.956,.932)--(-5.342,1.723)--(-4.524,2.696)--cycle;
\draw(-4.524,2.632)--(-5.954,.931)--(-5.956,.932);
\draw(-5.342,1.723)--(-4.524,2.696);
\filldraw[fill opacity=0.8,fill=gray!20,draw=none](-4.248,2.971)--(-4.237,2.945)--(-4.242,2.969)--cycle;
\filldraw[fill opacity=0.8,fill=gray!20,draw=none](-8.232,1.62)--(-8.248,1.637)--(-8.252,1.603)--cycle;
\draw(-8.232,1.62)--(-8.248,1.637)--(-8.252,1.603);
\filldraw[fill opacity=0.8,fill=gray!20,draw=none](-8.232,1.62)--(-8.22,1.63)--(-8.206,1.666)--(-8.231,1.675)--(-8.248,1.637)--cycle;
\draw(-8.22,1.63)--(-8.206,1.666);
\draw(-8.231,1.675)--(-8.248,1.637)--(-8.232,1.62);
\filldraw[fill opacity=0.8,fill=gray!20,draw=none](-8.232,1.62)--(-8.226,1.613)--(-8.22,1.63)--cycle;
\draw(-8.232,1.62)--(-8.226,1.613)--(-8.22,1.63);
\filldraw[fill opacity=0.8,fill=gray!20,draw=none](-8.221,1.612)--(-8.214,1.634)--(-8.22,1.63)--(-8.226,1.613)--cycle;
\draw(-8.22,1.63)--(-8.226,1.613)--(-8.221,1.612);
\filldraw[fill opacity=0.8,fill=gray!20,draw=none](-8.214,1.634)--(-8.203,1.666)--(-8.206,1.666)--(-8.22,1.63)--cycle;
\draw(-8.206,1.666)--(-8.22,1.63);
\filldraw[fill opacity=0.8,fill=gray!20,draw=none](-8.219,1.612)--(-8.221,1.612)--(-8.222,1.606)--cycle;
\draw(-8.219,1.612)--(-8.221,1.612);
\filldraw[fill opacity=0.8,fill=gray!20,draw=none](-8.214,1.634)--(-8.194,1.646)--(-8.183,1.662)--(-8.203,1.666)--cycle;
\draw(-8.183,1.662)--(-8.203,1.666);
\filldraw[fill opacity=0.8,fill=gray!20,draw=none](-8.214,1.634)--(-8.221,1.612)--(-8.219,1.612)--(-8.194,1.646)--cycle;
\draw(-8.221,1.612)--(-8.219,1.612);
\filldraw[fill opacity=0.8,fill=gray!20,draw=none](-8.5,1.484)--(-8.5,1.509)--(-8.477,1.522)--(-8.475,1.522)--cycle;
\draw(-8.477,1.522)--(-8.475,1.522);
\filldraw[fill opacity=0.8,fill=gray!20,draw=none](-8.5,1.509)--(-8.521,1.496)--(-8.516,1.52)--(-8.499,1.521)--cycle;
\draw(-8.521,1.496)--(-8.516,1.52)--(-8.499,1.521);
\filldraw[fill opacity=0.8,fill=gray!20,draw=none](-8.5,1.509)--(-8.499,1.521)--(-8.477,1.522)--cycle;
\draw(-8.499,1.521)--(-8.477,1.522);
\filldraw[fill opacity=0.8,fill=gray!20,draw=none](-8.538,1.468)--(-8.546,1.496)--(-8.518,1.503)--cycle;
\draw(-8.546,1.496)--(-8.518,1.503);
\filldraw[fill opacity=0.8,fill=gray!20,draw=none](-8.518,1.504)--(-8.518,1.503)--(-8.523,1.501)--cycle;
\draw(-8.518,1.503)--(-8.523,1.501);
\filldraw[fill opacity=0.8,fill=gray!20](-8.145,1.64)--(-8.554,1.461)--(-8.528,1.5)--(-8.119,1.679)--cycle;
\filldraw[fill opacity=0.8,fill=gray!20,draw=none](-4.274,2.769)--(-4.253,2.796)--(-4.245,2.816)--(-4.245,2.819)--(-4.261,2.809)--(-4.276,2.769)--cycle;
\draw(-4.245,2.819)--(-4.261,2.809)--(-4.276,2.769);
\filldraw[fill opacity=0.8,fill=gray!20,draw=none](-4.494,2.684)--(-4.478,2.687)--(-4.503,2.689)--(-4.498,2.685)--cycle;
\draw(-4.503,2.689)--(-4.498,2.685)--(-4.494,2.684);
\filldraw[fill opacity=0.8,fill=gray!20,draw=none](-4.494,2.684)--(-4.444,2.681)--(-4.445,2.684)--(-4.478,2.687)--cycle;
\draw(-4.494,2.684)--(-4.444,2.681)--(-4.445,2.684);
\filldraw[fill opacity=0.8,fill=gray!20,draw=none](-4.483,2.682)--(-4.524,2.632)--(-4.522,2.696)--cycle;
\draw(-4.483,2.682)--(-4.524,2.632);
\filldraw[fill opacity=0.8,fill=gray!20,draw=none](-8.091,.854)--(-8.079,.91)--(-8.127,.922)--(-8.148,.909)--(-8.153,.902)--(-8.164,.872)--cycle;
\draw(-8.153,.902)--(-8.164,.872)--(-8.091,.854)--(-8.079,.91)--(-8.127,.922);
\filldraw[fill opacity=0.8,fill=gray!20,draw=none](-8.071,.935)--(-8.11,.932)--(-8.127,.922)--(-8.079,.91)--(-8.07,.934)--cycle;
\draw(-8.127,.922)--(-8.079,.91)--(-8.07,.934);
\filldraw[fill opacity=0.8,fill=gray!20,draw=none](-8.281,.866)--(-8.448,.812)--(-8.436,.82)--(-8.411,.835)--(-8.324,.863)--cycle;
\draw(-8.281,.866)--(-8.448,.812);
\draw(-8.411,.835)--(-8.324,.863);
\filldraw[fill opacity=0.8,fill=gray!20,draw=none](-8.016,.962)--(-8.287,.875)--(-8.338,.849)--(-7.974,.966)--cycle;
\draw(-8.338,.849)--(-7.974,.966)--(-8.016,.962)--(-8.287,.875);
\filldraw[fill opacity=0.8,fill=gray!20,draw=none](-4.394,2.681)--(-4.39,2.682)--(-4.403,2.68)--cycle;
\draw(-4.394,2.681)--(-4.39,2.682);
\filldraw[fill opacity=0.8,fill=gray!20,draw=none](-7.7,4.349)--(-7.685,4.388)--(-7.662,4.371)--(-7.654,4.337)--cycle;
\draw(-7.7,4.349)--(-7.685,4.388);
\filldraw[fill opacity=0.8,fill=gray!20,draw=none](-7.662,4.371)--(-7.668,4.4)--(-7.641,4.36)--(-7.642,4.357)--cycle;
\draw(-7.641,4.36)--(-7.642,4.357);
\filldraw[fill opacity=0.8,fill=gray!20,draw=none](-7.654,4.337)--(-7.662,4.371)--(-7.642,4.357)--(-7.65,4.336)--cycle;
\draw(-7.642,4.357)--(-7.65,4.336);
\filldraw[fill opacity=0.8,fill=gray!20,draw=none](-7.766,4.378)--(-7.747,4.422)--(-7.703,4.395)--(-7.718,4.361)--cycle;
\draw(-7.766,4.378)--(-7.747,4.422);
\draw(-7.703,4.395)--(-7.718,4.361);
\filldraw[fill opacity=0.8,fill=gray!20](-8.029,4.079)--(-7.993,4.062)--(-8.214,3.475)--cycle;
\filldraw[fill opacity=0.8,fill=gray!20,draw=none](-8.486,3.585)--(-8.531,3.349)--(-8.631,3.387)--(-8.548,3.608)--cycle;
\draw(-8.486,3.585)--(-8.531,3.349)--(-8.631,3.387)--(-8.548,3.608);
\filldraw[fill opacity=0.8,fill=gray!20,draw=none](-8.548,3.608)--(-8.673,3.656)--(-8.668,3.654)--(-8.553,3.612)--(-8.434,3.566)--(-8.428,3.564)--cycle;
\draw(-8.673,3.656)--(-8.668,3.654)--(-8.553,3.612)--(-8.434,3.566)--(-8.428,3.564);
\filldraw[fill opacity=0.8,fill=gray!20](-8.553,3.612)--(-8.668,3.654)--(-8.447,4.24)--cycle;
\filldraw[fill opacity=0.8,fill=gray!20,draw=none](-8.584,3.621)--(-8.631,3.387)--(-8.727,3.422)--(-8.644,3.643)--cycle;
\draw(-8.584,3.621)--(-8.631,3.387)--(-8.727,3.422)--(-8.644,3.643);
\filldraw[fill opacity=0.8,fill=gray!20](-8.447,4.24)--(-8.332,4.198)--(-8.553,3.612)--cycle;
\filldraw[fill opacity=0.8,fill=gray!20](-8.434,3.566)--(-8.553,3.612)--(-8.332,4.198)--cycle;
\filldraw[fill opacity=0.8,fill=gray!20,draw=none](-8.644,3.643)--(-8.602,3.629)--(-8.548,3.608)--cycle;
\filldraw[fill opacity=0.8,fill=gray!20](-8.506,4.009)--(-8.41,3.974)--(-8.631,3.387)--cycle;
\filldraw[fill opacity=0.8,fill=gray!20](-8.332,4.198)--(-8.213,4.153)--(-8.434,3.566)--cycle;
\filldraw[fill opacity=0.8,fill=gray!20,draw=none](-8.486,3.585)--(-8.448,3.57)--(-8.531,3.349)--cycle;
\draw(-8.448,3.57)--(-8.531,3.349)--(-8.486,3.585);
\filldraw[fill opacity=0.8,fill=gray!20](-8.327,3.524)--(-8.434,3.566)--(-8.213,4.153)--cycle;
\filldraw[fill opacity=0.8,fill=gray!20,draw=none](-8.448,3.57)--(-8.496,3.589)--(-8.428,3.564)--(-8.327,3.524)--(-8.323,3.523)--cycle;
\draw(-8.428,3.564)--(-8.327,3.524)--(-8.323,3.523);
\filldraw[fill opacity=0.8,fill=gray!20,draw=none](-8.005,4.032)--(-8.215,3.475)--(-8.214,3.475)--(-7.993,4.062)--cycle;
\draw(-8.215,3.475)--(-8.214,3.475)--(-7.993,4.062)--(-8.005,4.032);
\filldraw[fill opacity=0.8,fill=gray!20,draw=none](-8.393,3.548)--(-8.443,3.314)--(-8.531,3.349)--(-8.448,3.57)--cycle;
\draw(-8.393,3.548)--(-8.443,3.314)--(-8.531,3.349)--(-8.448,3.57);
\filldraw[fill opacity=0.8,fill=gray!20](-8.213,4.153)--(-8.106,4.111)--(-8.327,3.524)--cycle;
\filldraw[fill opacity=0.8,fill=gray!20,draw=none](-8.359,3.535)--(-8.448,3.57)--(-8.394,3.55)--cycle;
\filldraw[fill opacity=0.8,fill=gray!20,draw=none](-8.359,3.535)--(-8.394,3.55)--(-8.323,3.523)--(-8.25,3.492)--(-8.248,3.492)--cycle;
\draw(-8.323,3.523)--(-8.25,3.492)--(-8.248,3.492);
\filldraw[fill opacity=0.8,fill=gray!20](-8.31,3.936)--(-8.222,3.901)--(-8.443,3.314)--cycle;
\filldraw[fill opacity=0.8,fill=gray!20](-8.668,3.654)--(-8.761,3.686)--(-8.54,4.273)--cycle;
\filldraw[fill opacity=0.8,fill=gray!20](-8.54,4.273)--(-8.447,4.24)--(-8.668,3.654)--cycle;
\filldraw[fill opacity=0.8,fill=gray!20,draw=none](-8.644,3.643)--(-8.763,3.687)--(-8.761,3.686)--(-8.673,3.656)--(-8.602,3.629)--cycle;
\draw(-8.763,3.687)--(-8.761,3.686)--(-8.673,3.656);
\filldraw[fill opacity=0.8,fill=gray!20,draw=none](-8.673,3.653)--(-8.727,3.422)--(-8.804,3.449)--(-8.721,3.67)--cycle;
\draw(-8.673,3.653)--(-8.727,3.422)--(-8.804,3.449)--(-8.721,3.67);
\filldraw[fill opacity=0.8,fill=gray!20,draw=none](-8.721,3.67)--(-8.695,3.662)--(-8.644,3.643)--cycle;
\filldraw[fill opacity=0.8,fill=gray!20](-8.583,4.036)--(-8.506,4.009)--(-8.727,3.422)--cycle;
\filldraw[fill opacity=0.8,fill=gray!20](-8.25,3.492)--(-8.327,3.524)--(-8.106,4.111)--cycle;
\filldraw[fill opacity=0.8,fill=gray!20](-8.214,3.475)--(-8.25,3.492)--(-8.029,4.079)--cycle;
\filldraw[fill opacity=0.8,fill=gray!20,draw=none](-8.261,3.494)--(-8.255,3.494)--(-8.248,3.492)--(-8.214,3.475)--(-8.215,3.475)--cycle;
\draw(-8.248,3.492)--(-8.214,3.475)--(-8.215,3.475);
\filldraw[fill opacity=0.8,fill=gray!20,draw=none](-8.265,3.494)--(-8.261,3.494)--(-8.215,3.475)--(-8.225,3.475)--(-8.228,3.476)--cycle;
\draw(-8.215,3.475)--(-8.225,3.475)--(-8.228,3.476);
\filldraw[fill opacity=0.8,fill=gray!20,draw=none](-8.005,4.032)--(-8.225,3.475)--(-8.215,3.475)--cycle;
\draw(-8.005,4.032)--(-8.225,3.475)--(-8.215,3.475);
\filldraw[fill opacity=0.8,fill=gray!20](-7.993,4.062)--(-8.004,4.062)--(-8.225,3.475)--cycle;
\filldraw[fill opacity=0.8,fill=gray!20,draw=none](-8.319,3.519)--(-8.379,3.288)--(-8.443,3.314)--(-8.359,3.535)--cycle;
\draw(-8.319,3.519)--(-8.379,3.288)--(-8.443,3.314)--(-8.359,3.535);
\filldraw[fill opacity=0.8,fill=gray!20](-8.106,4.111)--(-8.029,4.079)--(-8.25,3.492)--cycle;
\filldraw[fill opacity=0.8,fill=gray!20,draw=none](-8.295,3.509)--(-8.359,3.535)--(-8.311,3.516)--cycle;
\filldraw[fill opacity=0.8,fill=gray!20,draw=none](-8.261,3.494)--(-8.295,3.509)--(-8.311,3.516)--(-8.255,3.494)--cycle;
\filldraw[fill opacity=0.8,fill=gray!20](-8.222,3.901)--(-8.157,3.874)--(-8.379,3.288)--cycle;
\filldraw[fill opacity=0.8,fill=gray!20,draw=none](-8.721,3.67)--(-8.817,3.704)--(-8.817,3.704)--(-8.763,3.687)--(-8.695,3.662)--cycle;
\draw(-8.817,3.704)--(-8.817,3.704)--(-8.763,3.687);
\filldraw[fill opacity=0.8,fill=gray!20](-8.761,3.686)--(-8.817,3.704)--(-8.595,4.29)--cycle;
\filldraw[fill opacity=0.8,fill=gray!20](-8.595,4.29)--(-8.54,4.273)--(-8.761,3.686)--cycle;
\filldraw[fill opacity=0.8,fill=gray!20,draw=none](-8.738,3.676)--(-8.804,3.449)--(-8.851,3.464)--(-8.767,3.685)--cycle;
\draw(-8.738,3.676)--(-8.804,3.449)--(-8.851,3.464)--(-8.767,3.685);
\filldraw[fill opacity=0.8,fill=gray!20,draw=none](-8.767,3.685)--(-8.762,3.685)--(-8.721,3.67)--cycle;
\filldraw[fill opacity=0.8,fill=gray!20](-8.629,4.05)--(-8.583,4.036)--(-8.804,3.449)--cycle;
\filldraw[fill opacity=0.8,fill=gray!20,draw=none](-8.265,3.494)--(-8.295,3.509)--(-8.261,3.494)--cycle;
\filldraw[fill opacity=0.8,fill=gray!20](-8.349,3.273)--(-8.379,3.288)--(-8.157,3.874)--cycle;
\filldraw[fill opacity=0.8,fill=gray!20](-8.157,3.874)--(-8.128,3.86)--(-8.349,3.273)--cycle;
\filldraw[fill opacity=0.8,fill=gray!20,draw=none](-8.771,3.685)--(-8.851,3.464)--(-8.86,3.464)--(-8.776,3.685)--cycle;
\draw(-8.771,3.685)--(-8.851,3.464)--(-8.86,3.464)--(-8.776,3.685);
\filldraw[fill opacity=0.8,fill=gray!20,draw=none](-8.767,3.685)--(-8.792,3.693)--(-8.811,3.702)--(-8.762,3.685)--cycle;
\filldraw[fill opacity=0.8,fill=gray!20,draw=none](-8.771,3.685)--(-8.767,3.685)--(-8.851,3.464)--cycle;
\draw(-8.767,3.685)--(-8.851,3.464)--(-8.771,3.685);
\filldraw[fill opacity=0.8,fill=gray!20,draw=none](-8.321,3.509)--(-8.398,3.536)--(-8.494,3.571)--(-8.594,3.609)--(-8.682,3.644)--(-8.747,3.671)--(-8.776,3.685)--(-8.767,3.685)--(-8.721,3.67)--(-8.644,3.643)--(-8.548,3.608)--(-8.448,3.57)--(-8.359,3.535)--(-8.295,3.509)--(-8.265,3.494)--(-8.275,3.495)--cycle;
\filldraw[fill opacity=0.8,fill=gray!20](-8.672,3.349)--(-8.675,3.373)--(-8.621,3.369)--(-8.596,3.343)--cycle;
\filldraw[fill opacity=0.8,fill=gray!20](-8.596,3.343)--(-8.621,3.369)--(-8.583,3.36)--(-8.543,3.33)--cycle;
\filldraw[fill opacity=0.8,fill=gray!20](-8.675,3.373)--(-8.678,3.383)--(-8.65,3.381)--(-8.621,3.369)--cycle;
\filldraw[fill opacity=0.8,fill=gray!20](-8.621,3.369)--(-8.65,3.381)--(-8.631,3.376)--(-8.583,3.36)--cycle;
\filldraw[fill opacity=0.8,fill=gray!20](-8.577,3.305)--(-8.596,3.343)--(-8.543,3.33)--(-8.512,3.289)--cycle;
\filldraw[fill opacity=0.8,fill=gray!20](-8.512,3.289)--(-8.543,3.33)--(-8.527,3.313)--(-8.492,3.268)--cycle;
\filldraw[fill opacity=0.8,fill=gray!20](-8.543,3.33)--(-8.583,3.36)--(-8.572,3.347)--(-8.527,3.313)--cycle;
\filldraw[fill opacity=0.8,fill=gray!20](-8.731,3.37)--(-8.707,3.381)--(-8.678,3.383)--(-8.675,3.373)--cycle;
\filldraw[fill opacity=0.8,fill=gray!20](-8.678,3.383)--(-8.682,3.378)--(-8.682,3.378)--(-8.65,3.381)--cycle;
\filldraw[fill opacity=0.8,fill=gray!20](-8.583,3.36)--(-8.631,3.376)--(-8.625,3.37)--(-8.572,3.347)--cycle;
\filldraw[fill opacity=0.8,fill=gray!20](-8.65,3.381)--(-8.682,3.378)--(-8.682,3.378)--(-8.631,3.376)--cycle;
\filldraw[fill opacity=0.8,fill=gray!20](-8.631,3.376)--(-8.682,3.378)--(-8.682,3.378)--(-8.625,3.37)--cycle;
\filldraw[fill opacity=0.8,fill=gray!20](-8.625,3.37)--(-8.682,3.378)--(-8.682,3.378)--(-8.634,3.363)--cycle;
\filldraw[fill opacity=0.8,fill=gray!20](-8.67,3.312)--(-8.672,3.349)--(-8.596,3.343)--(-8.577,3.305)--cycle;
\filldraw[fill opacity=0.8,fill=gray!20](-8.572,3.347)--(-8.625,3.37)--(-8.634,3.363)--(-8.59,3.336)--cycle;
\filldraw[fill opacity=0.8,fill=gray!20](-8.577,3.35)--(-8.677,3.388)--(-8.766,3.423)--(-8.83,3.45)--(-8.86,3.464)--(-8.851,3.464)--(-8.804,3.449)--(-8.727,3.422)--(-8.631,3.387)--(-8.531,3.349)--(-8.443,3.314)--(-8.379,3.288)--(-8.349,3.273)--(-8.358,3.274)--(-8.404,3.288)--(-8.481,3.315)--cycle;
\filldraw[fill opacity=0.8,fill=gray!20](-8.358,3.274)--(-8.349,3.273)--(-8.128,3.86)--cycle;
\filldraw[fill opacity=0.8,fill=gray!20](-8.128,3.86)--(-8.137,3.86)--(-8.358,3.274)--cycle;
\filldraw[fill opacity=0.8,fill=gray!20,draw=none](-8.275,3.495)--(-8.265,3.494)--(-8.249,3.487)--cycle;
\filldraw[fill opacity=0.8,fill=gray!20](-8.404,3.288)--(-8.358,3.274)--(-8.137,3.86)--cycle;
\filldraw[fill opacity=0.8,fill=gray!20,draw=none](-8.275,3.495)--(-8.249,3.487)--(-8.228,3.476)--(-8.281,3.493)--(-8.286,3.495)--cycle;
\draw(-8.228,3.476)--(-8.281,3.493)--(-8.286,3.495);
\filldraw[fill opacity=0.8,fill=gray!20](-8.137,3.86)--(-8.183,3.875)--(-8.404,3.288)--cycle;
\filldraw[fill opacity=0.8,fill=gray!20,draw=none](-8.321,3.509)--(-8.275,3.495)--(-8.279,3.495)--cycle;
\filldraw[fill opacity=0.8,fill=gray!20,draw=none](-8.267,3.522)--(-8.278,3.492)--(-8.225,3.475)--(-8.004,4.062)--cycle;
\draw(-8.278,3.492)--(-8.225,3.475)--(-8.004,4.062)--(-8.267,3.522);
\filldraw[fill opacity=0.8,fill=gray!20,draw=none](-7.99,3.835)--(-8.082,3.83)--(-8.085,3.887)--(-7.974,3.892)--(-7.974,3.877)--cycle;
\draw(-7.99,3.835)--(-8.082,3.83)--(-8.085,3.887)--(-7.974,3.892)--(-7.974,3.877);
\filldraw[fill opacity=0.8,fill=gray!20](-7.974,3.892)--(-7.975,3.946)--(-7.87,3.939)--(-7.866,3.885)--cycle;
\filldraw[fill opacity=0.8,fill=gray!20,draw=none](-8.486,3.585)--(-8.548,3.608)--(-8.41,3.974)--cycle;
\draw(-8.548,3.608)--(-8.41,3.974)--(-8.486,3.585);
\filldraw[fill opacity=0.8,fill=gray!20,draw=none](-8.486,3.585)--(-8.41,3.974)--(-8.31,3.936)--(-8.448,3.57)--cycle;
\draw(-8.486,3.585)--(-8.41,3.974)--(-8.31,3.936)--(-8.448,3.57);
\filldraw[fill opacity=0.8,fill=gray!20,draw=none](-8.393,3.548)--(-8.448,3.57)--(-8.31,3.936)--cycle;
\draw(-8.448,3.57)--(-8.31,3.936)--(-8.393,3.548);
\filldraw[fill opacity=0.8,fill=gray!20,draw=none](-8.584,3.621)--(-8.644,3.643)--(-8.506,4.009)--cycle;
\draw(-8.644,3.643)--(-8.506,4.009)--(-8.584,3.621);
\filldraw[fill opacity=0.8,fill=gray!20,draw=none](-8.319,3.519)--(-8.359,3.535)--(-8.222,3.901)--cycle;
\draw(-8.359,3.535)--(-8.222,3.901)--(-8.319,3.519);
\filldraw[fill opacity=0.8,fill=gray!20,draw=none](-8.673,3.653)--(-8.721,3.67)--(-8.583,4.036)--cycle;
\draw(-8.721,3.67)--(-8.583,4.036)--(-8.673,3.653);
\filldraw[fill opacity=0.8,fill=gray!20,draw=none](-8.738,3.676)--(-8.767,3.685)--(-8.629,4.05)--cycle;
\draw(-8.767,3.685)--(-8.629,4.05)--(-8.738,3.676);
\filldraw[fill opacity=0.8,fill=gray!20,draw=none](-8.771,3.685)--(-8.639,4.05)--(-8.629,4.05)--(-8.767,3.685)--cycle;
\draw(-8.771,3.685)--(-8.639,4.05)--(-8.629,4.05)--(-8.767,3.685);
\filldraw[fill opacity=0.8,fill=gray!20](-8.26,3.902)--(-8.183,3.875)--(-8.137,3.86)--(-8.128,3.86)--(-8.157,3.874)--(-8.222,3.901)--(-8.31,3.936)--(-8.41,3.974)--(-8.506,4.009)--(-8.583,4.036)--(-8.629,4.05)--(-8.639,4.05)--(-8.609,4.036)--(-8.545,4.009)--(-8.456,3.974)--(-8.356,3.937)--cycle;
\filldraw[fill opacity=0.8,fill=gray!20](-8.481,3.315)--(-8.404,3.288)--(-8.183,3.875)--cycle;
\filldraw[fill opacity=0.8,fill=gray!20](-8.183,3.875)--(-8.26,3.902)--(-8.481,3.315)--cycle;
\filldraw[fill opacity=0.8,fill=gray!20,draw=none](-8.379,3.528)--(-8.321,3.509)--(-8.279,3.495)--(-8.286,3.495)--(-8.374,3.525)--cycle;
\draw(-8.286,3.495)--(-8.374,3.525)--(-8.379,3.528);
\filldraw[fill opacity=0.8,fill=gray!20](-8.004,4.062)--(-8.06,4.079)--(-8.281,3.493)--cycle;
\filldraw[fill opacity=0.8,fill=gray!20](-8.085,3.887)--(-8.082,3.941)--(-7.975,3.946)--(-7.974,3.892)--cycle;
\filldraw[fill opacity=0.8,fill=gray!20](-8.082,3.941)--(-8.072,3.989)--(-7.976,3.993)--(-7.975,3.946)--cycle;
\filldraw[fill opacity=0.8,fill=gray!20,draw=none](-7.818,4.373)--(-7.805,4.385)--(-7.811,4.37)--cycle;
\draw(-7.805,4.385)--(-7.811,4.37)--(-7.818,4.373);
\filldraw[fill opacity=0.8,fill=gray!20,draw=none](-7.818,4.373)--(-7.852,4.385)--(-7.82,4.467)--(-7.816,4.467)--(-7.794,4.413)--(-7.805,4.385)--cycle;
\draw(-7.818,4.373)--(-7.852,4.385)--(-7.82,4.467);
\draw(-7.794,4.413)--(-7.805,4.385);
\filldraw[fill opacity=0.8,fill=gray!20,draw=none](-7.818,4.372)--(-7.784,4.442)--(-7.747,4.422)--(-7.776,4.356)--cycle;
\draw(-7.747,4.422)--(-7.776,4.356);
\filldraw[fill opacity=0.8,fill=gray!20,draw=none](-7.689,4.385)--(-7.695,4.413)--(-7.665,4.368)--cycle;
\filldraw[fill opacity=0.8,fill=gray!20,draw=none](-7.703,4.395)--(-7.695,4.413)--(-7.689,4.385)--cycle;
\draw(-7.703,4.395)--(-7.695,4.413);
\filldraw[fill opacity=0.8,fill=gray!20,draw=none](-7.718,4.361)--(-7.703,4.395)--(-7.689,4.385)--(-7.683,4.352)--cycle;
\draw(-7.718,4.361)--(-7.703,4.395);
\filldraw[fill opacity=0.8,fill=gray!20,draw=none](-7.642,4.357)--(-7.641,4.36)--(-7.639,4.355)--cycle;
\draw(-7.642,4.357)--(-7.641,4.36);
\filldraw[fill opacity=0.8,fill=gray!20,draw=none](-7.683,4.352)--(-7.689,4.385)--(-7.665,4.368)--(-7.673,4.349)--cycle;
\draw(-7.665,4.368)--(-7.673,4.349);
\filldraw[fill opacity=0.8,fill=gray!20,draw=none](-4.744,2.906)--(-7.755,4.412)--(-7.761,4.447)--(-4.714,2.922)--cycle;
\draw(-4.744,2.906)--(-7.755,4.412)--(-7.761,4.447)--(-4.714,2.922);
\filldraw[fill opacity=0.8,fill=gray!20,draw=none](-4.427,3.104)--(-4.426,3.103)--(-4.422,3.102)--cycle;
\draw(-4.426,3.103)--(-4.422,3.102);
\filldraw[fill opacity=0.8,fill=gray!20,draw=none](-4.427,3.104)--(-4.438,3.104)--(-4.438,3.104)--(-4.426,3.103)--cycle;
\draw(-4.427,3.104)--(-4.438,3.104)--(-4.438,3.104)--(-4.426,3.103);
\filldraw[fill opacity=0.8,fill=gray!20](-7.978,.618)--(-7.981,.628)--(-8.035,.632)--(-8.006,.62)--cycle;
\filldraw[fill opacity=0.8,fill=gray!20,draw=none](-7.763,.779)--(-7.759,.804)--(-7.792,.802)--(-7.797,.757)--cycle;
\draw(-7.792,.802)--(-7.797,.757)--(-7.763,.779)--(-7.759,.804);
\filldraw[fill opacity=0.8,fill=gray!20](-7.943,1.055)--(-7.974,1.052)--(-7.974,1.052)--(-7.924,1.05)--cycle;
\filldraw[fill opacity=0.8,fill=gray!20](-4.489,3.092)--(-4.438,3.104)--(-4.438,3.104)--(-4.495,3.098)--cycle;
\filldraw[fill opacity=0.8,fill=gray!20,draw=none](-4.549,2.717)--(-4.552,2.719)--(-4.57,2.708)--(-4.55,2.7)--(-4.544,2.701)--cycle;
\draw(-4.552,2.719)--(-4.57,2.708);
\filldraw[fill opacity=0.8,fill=gray!20,draw=none](-4.38,2.685)--(-4.388,2.683)--(-4.39,2.682)--(-4.394,2.681)--cycle;
\draw(-4.38,2.685)--(-4.388,2.683);
\draw(-4.39,2.682)--(-4.394,2.681);
\filldraw[fill opacity=0.8,fill=gray!20,draw=none](-8.11,.695)--(-8.071,.678)--(-8.07,.678)--(-8.079,.696)--(-8.127,.708)--cycle;
\draw(-8.07,.678)--(-8.079,.696)--(-8.127,.708);
\filldraw[fill opacity=0.8,fill=gray!20,draw=none](-8.027,.682)--(-8.014,.691)--(-8.079,.696)--(-8.07,.678)--cycle;
\draw(-8.014,.691)--(-8.079,.696)--(-8.07,.678);
\filldraw[fill opacity=0.8,fill=gray!20,draw=none](-8.014,.691)--(-7.993,.704)--(-7.987,.718)--(-7.987,.736)--(-8.091,.743)--(-8.079,.696)--cycle;
\draw(-7.987,.718)--(-7.987,.736)--(-8.091,.743)--(-8.079,.696)--(-8.014,.691);
\filldraw[fill opacity=0.8,fill=gray!20,draw=none](-8.335,.583)--(-8.365,.581)--(-8.374,.583)--(-8.358,.625)--(-8.303,.612)--(-8.314,.595)--cycle;
\draw(-8.365,.581)--(-8.374,.583)--(-8.358,.625)--(-8.303,.612)--(-8.314,.595);
\filldraw[fill opacity=0.8,fill=gray!20,draw=none](-8.374,.58)--(-8.375,.581)--(-8.374,.583)--(-8.365,.581)--cycle;
\draw(-8.375,.581)--(-8.374,.583)--(-8.365,.581);
\filldraw[fill opacity=0.8,fill=gray!20,draw=none](-8.3,.605)--(-8.314,.595)--(-8.303,.612)--(-8.299,.607)--cycle;
\draw(-8.314,.595)--(-8.303,.612)--(-8.299,.607);
\filldraw[fill opacity=0.8,fill=gray!20](-7.897,.735)--(-8.368,.583)--(-8.403,.561)--(-7.933,.713)--cycle;
\filldraw[fill opacity=0.8,fill=gray!20,draw=none](-4.576,2.711)--(-4.587,2.697)--(-4.623,2.716)--(-4.61,2.734)--cycle;
\draw(-4.576,2.711)--(-4.587,2.697);
\filldraw[fill opacity=0.8,fill=gray!20,draw=none](-4.548,2.7)--(-4.541,2.695)--(-4.608,2.673)--(-4.587,2.697)--cycle;
\draw(-4.608,2.673)--(-4.587,2.697);
\filldraw[fill opacity=0.8,fill=gray!20,draw=none](-4.307,2.721)--(-4.307,2.72)--(-4.311,2.718)--cycle;
\draw(-4.307,2.721)--(-4.307,2.72);
\filldraw[fill opacity=0.8,fill=gray!20,draw=none](-4.23,2.88)--(-4.231,2.894)--(-4.242,2.933)--(-4.261,2.92)--(-4.255,2.864)--cycle;
\draw(-4.242,2.933)--(-4.261,2.92)--(-4.255,2.864)--(-4.23,2.88);
\filldraw[fill opacity=0.8,fill=gray!20,draw=none](-4.231,2.894)--(-4.234,2.938)--(-4.242,2.933)--cycle;
\draw(-4.234,2.938)--(-4.242,2.933);
\filldraw[fill opacity=0.8,fill=gray!20,draw=none](-4.637,2.888)--(-4.636,2.916)--(-4.641,2.927)--cycle;
\filldraw[fill opacity=0.8,fill=gray!20,draw=none](-7.675,4.506)--(-7.692,4.491)--(-7.695,4.49)--(-7.682,4.509)--(-7.655,4.527)--cycle;
\draw(-7.692,4.491)--(-7.695,4.49);
\filldraw[fill opacity=0.8,fill=gray!20,draw=none](-7.691,4.492)--(-7.705,4.485)--(-7.695,4.49)--(-7.686,4.494)--cycle;
\draw(-7.695,4.49)--(-7.686,4.494);
\filldraw[fill opacity=0.8,fill=gray!20,draw=none](-7.675,4.506)--(-7.686,4.494)--(-7.692,4.491)--cycle;
\draw(-7.686,4.494)--(-7.692,4.491);
\filldraw[fill opacity=0.8,fill=gray!20,draw=none](-7.709,4.483)--(-7.704,4.486)--(-7.705,4.485)--cycle;
\filldraw[fill opacity=0.8,fill=gray!20,draw=none](-7.709,4.483)--(-7.707,4.488)--(-7.705,4.485)--cycle;
\draw(-7.709,4.483)--(-7.707,4.488);
\filldraw[fill opacity=0.8,fill=gray!20,draw=none](-7.69,4.514)--(-7.691,4.492)--(-7.705,4.485)--(-7.707,4.488)--(-7.702,4.5)--cycle;
\draw(-7.707,4.488)--(-7.702,4.5);
\filldraw[fill opacity=0.8,fill=gray!20,draw=none](-7.714,4.495)--(-7.702,4.5)--(-7.707,4.488)--cycle;
\draw(-7.702,4.5)--(-7.707,4.488);
\filldraw[fill opacity=0.8,fill=gray!20,draw=none](-7.691,4.492)--(-7.692,4.459)--(-7.705,4.485)--cycle;
\filldraw[fill opacity=0.8,fill=gray!20,draw=none](-7.691,4.492)--(-7.723,4.476)--(-7.724,4.476)--(-7.705,4.485)--cycle;
\filldraw[fill opacity=0.8,fill=gray!20,draw=none](-7.689,4.529)--(-7.69,4.514)--(-7.702,4.5)--(-7.691,4.53)--cycle;
\draw(-7.702,4.5)--(-7.691,4.53);
\filldraw[fill opacity=0.8,fill=gray!20,draw=none](-7.714,4.495)--(-7.749,4.528)--(-7.743,4.546)--(-7.691,4.53)--(-7.702,4.5)--cycle;
\draw(-7.749,4.528)--(-7.743,4.546);
\draw(-7.691,4.53)--(-7.702,4.5);
\filldraw[fill opacity=0.8,fill=gray!20,draw=none](-7.697,4.529)--(-7.698,4.518)--(-7.708,4.507)--(-7.709,4.508)--(-7.699,4.531)--cycle;
\draw(-7.708,4.507)--(-7.709,4.508)--(-7.699,4.531);
\filldraw[fill opacity=0.8,fill=gray!20,draw=none](-7.729,4.54)--(-7.699,4.531)--(-7.702,4.524)--cycle;
\draw(-7.699,4.531)--(-7.702,4.524);
\filldraw[fill opacity=0.8,fill=gray!20,draw=none](-7.754,4.527)--(-7.754,4.528)--(-7.749,4.528)--(-7.75,4.526)--cycle;
\draw(-7.749,4.528)--(-7.75,4.526);
\filldraw[fill opacity=0.8,fill=gray!20,draw=none](-7.754,4.528)--(-7.745,4.549)--(-7.741,4.549)--(-7.749,4.528)--cycle;
\draw(-7.741,4.549)--(-7.749,4.528);
\filldraw[fill opacity=0.8,fill=gray!20,draw=none](-7.747,4.545)--(-7.729,4.54)--(-7.702,4.524)--(-7.709,4.508)--(-7.754,4.527)--cycle;
\draw(-7.702,4.524)--(-7.709,4.508)--(-7.754,4.527);
\filldraw[fill opacity=0.8,fill=gray!20,draw=none](-7.72,4.513)--(-7.744,4.523)--(-7.709,4.508)--(-7.699,4.503)--cycle;
\draw(-7.744,4.523)--(-7.709,4.508)--(-7.699,4.503);
\filldraw[fill opacity=0.8,fill=gray!20,draw=none](-7.698,4.518)--(-7.699,4.503)--(-7.708,4.507)--cycle;
\draw(-7.699,4.503)--(-7.708,4.507);
\filldraw[fill opacity=0.8,fill=gray!20,draw=none](-7.781,4.539)--(-7.786,4.541)--(-7.759,4.529)--(-7.754,4.527)--cycle;
\draw(-7.786,4.541)--(-7.759,4.529)--(-7.754,4.527);
\filldraw[fill opacity=0.8,fill=gray!20,draw=none](-7.747,4.545)--(-7.754,4.527)--(-7.759,4.529)--(-7.751,4.547)--cycle;
\draw(-7.754,4.527)--(-7.759,4.529)--(-7.751,4.547);
\filldraw[fill opacity=0.8,fill=gray!20,draw=none](-7.759,4.551)--(-7.752,4.546)--(-7.759,4.529)--(-7.799,4.547)--(-7.797,4.55)--cycle;
\draw(-7.752,4.546)--(-7.759,4.529)--(-7.799,4.547)--(-7.797,4.55);
\filldraw[fill opacity=0.8,fill=gray!20,draw=none](-7.729,4.517)--(-7.737,4.52)--(-7.739,4.521)--(-7.744,4.523)--(-7.751,4.526)--(-7.744,4.523)--cycle;
\draw(-7.751,4.526)--(-7.744,4.523);
\filldraw[fill opacity=0.8,fill=gray!20,draw=none](-7.719,4.53)--(-7.723,4.53)--(-7.741,4.533)--(-7.741,4.537)--(-7.732,4.539)--cycle;
\draw(-7.741,4.537)--(-7.732,4.539);
\filldraw[fill opacity=0.8,fill=gray!20,draw=none](-7.729,4.517)--(-7.72,4.513)--(-7.737,4.52)--cycle;
\filldraw[fill opacity=0.8,fill=gray!20,draw=none](-7.723,4.476)--(-7.725,4.475)--(-7.724,4.476)--cycle;
\filldraw[fill opacity=0.8,fill=gray!20,draw=none](-7.725,4.476)--(-7.724,4.476)--(-7.723,4.475)--cycle;
\filldraw[fill opacity=0.8,fill=gray!20,draw=none](-7.725,4.476)--(-7.754,4.503)--(-7.744,4.523)--(-7.739,4.521)--(-7.724,4.476)--cycle;
\filldraw[fill opacity=0.8,fill=gray!20,draw=none](-7.741,4.533)--(-7.723,4.53)--(-7.741,4.531)--cycle;
\filldraw[fill opacity=0.8,fill=gray!20,draw=none](-7.744,4.523)--(-7.739,4.521)--(-7.744,4.523)--cycle;
\filldraw[fill opacity=0.8,fill=gray!20,draw=none](-7.744,4.523)--(-7.742,4.528)--(-7.739,4.521)--cycle;
\filldraw[fill opacity=0.8,fill=gray!20,draw=none](-5.128,3.175)--(-7.756,4.491)--(-7.738,4.538)--(-4.755,3.045)--cycle;
\draw(-5.128,3.175)--(-7.756,4.491)--(-7.738,4.538)--(-4.755,3.045);
\filldraw[fill opacity=0.8,fill=gray!20](-7.949,.619)--(-7.925,.63)--(-7.981,.628)--(-7.978,.618)--cycle;
\filldraw[fill opacity=0.8,fill=gray!20,draw=none](-4.413,2.701)--(-4.415,2.699)--(-4.414,2.7)--cycle;
\draw(-4.415,2.699)--(-4.414,2.7);
\filldraw[fill opacity=0.8,fill=gray!20,draw=none](-4.413,2.701)--(-4.388,2.728)--(-4.404,2.705)--(-4.426,2.683)--(-4.415,2.699)--cycle;
\draw(-4.388,2.728)--(-4.404,2.705);
\draw(-4.426,2.683)--(-4.415,2.699);
\filldraw[fill opacity=0.8,fill=gray!20,draw=none](-4.379,2.741)--(-4.388,2.728)--(-4.413,2.701)--cycle;
\draw(-4.379,2.741)--(-4.388,2.728);
\filldraw[fill opacity=0.8,fill=gray!20,draw=none](-4.427,2.681)--(-4.414,2.694)--(-4.413,2.704)--(-4.438,2.695)--(-4.445,2.689)--(-4.444,2.681)--cycle;
\draw(-4.445,2.689)--(-4.444,2.681)--(-4.427,2.681);
\filldraw[fill opacity=0.8,fill=gray!20,draw=none](-4.478,2.687)--(-4.445,2.684)--(-4.446,2.692)--cycle;
\draw(-4.445,2.684)--(-4.446,2.692);
\filldraw[fill opacity=0.8,fill=gray!20,draw=none](-4.438,2.695)--(-4.446,2.692)--(-4.445,2.689)--cycle;
\draw(-4.446,2.692)--(-4.445,2.689);
\filldraw[fill opacity=0.8,fill=gray!20,draw=none](-4.421,2.718)--(-4.418,2.719)--(-4.455,2.676)--(-4.473,2.693)--(-4.468,2.699)--cycle;
\draw(-4.418,2.719)--(-4.455,2.676);
\draw(-4.473,2.693)--(-4.468,2.699);
\filldraw[fill opacity=0.8,fill=gray!20,draw=none](-4.38,2.685)--(-4.357,2.698)--(-4.386,2.686)--(-4.389,2.683)--cycle;
\draw(-4.386,2.686)--(-4.389,2.683)--(-4.38,2.685);
\filldraw[fill opacity=0.8,fill=gray!20,draw=none](-4.307,2.72)--(-4.308,2.72)--(-4.311,2.718)--cycle;
\draw(-4.307,2.72)--(-4.308,2.72)--(-4.311,2.718);
\filldraw[fill opacity=0.8,fill=gray!20,draw=none](-4.334,2.712)--(-4.311,2.718)--(-4.308,2.72)--(-4.328,2.716)--cycle;
\draw(-4.311,2.718)--(-4.308,2.72)--(-4.328,2.716);
\filldraw[fill opacity=0.8,fill=gray!20,draw=none](-4.334,2.712)--(-4.357,2.698)--(-4.311,2.718)--cycle;
\filldraw[fill opacity=0.8,fill=gray!20,draw=none](-4.647,2.818)--(-4.62,2.817)--(-4.625,2.835)--(-4.642,2.826)--cycle;
\filldraw[fill opacity=0.8,fill=gray!20,draw=none](-4.619,2.812)--(-4.614,2.81)--(-4.622,2.823)--cycle;
\draw(-4.619,2.812)--(-4.614,2.81);
\filldraw[fill opacity=0.8,fill=gray!20,draw=none](-4.465,2.891)--(-4.747,2.769)--(-4.815,2.727)--(-4.564,2.825)--(-4.463,2.887)--cycle;
\draw(-4.747,2.769)--(-4.815,2.727);
\draw(-4.564,2.825)--(-4.463,2.887)--(-4.465,2.891);
\filldraw[fill opacity=0.8,fill=gray!20,draw=none](-4.237,2.945)--(-4.236,2.937)--(-4.234,2.938)--cycle;
\draw(-4.236,2.937)--(-4.234,2.938);
\filldraw[fill opacity=0.8,fill=gray!20](-7.926,1.392)--(-7.881,1.425)--(-7.906,1.408)--(-7.944,1.38)--cycle;
\filldraw[fill opacity=0.8,fill=gray!20](-2.596,7.886)--(-2.6,7.941)--(-2.527,7.923)--(-2.52,7.868)--cycle;
\filldraw[fill opacity=0.8,fill=gray!20,draw=none](-4.42,3.086)--(-4.422,3.093)--(-4.438,3.104)--(-4.438,3.104)--(-4.441,3.085)--cycle;
\draw(-4.422,3.093)--(-4.438,3.104)--(-4.438,3.104)--(-4.441,3.085)--(-4.42,3.086);
\filldraw[fill opacity=0.8,fill=gray!20](-4.441,3.085)--(-4.438,3.104)--(-4.438,3.104)--(-4.469,3.087)--cycle;
\filldraw[fill opacity=0.8,fill=gray!20,draw=none](-4.425,3.1)--(-4.438,3.104)--(-4.438,3.104)--(-4.422,3.093)--cycle;
\draw(-4.425,3.1)--(-4.438,3.104)--(-4.438,3.104)--(-4.422,3.093);
\filldraw[fill opacity=0.8,fill=gray!20](-4.469,3.087)--(-4.438,3.104)--(-4.438,3.104)--(-4.489,3.092)--cycle;
\filldraw[fill opacity=0.8,fill=gray!20,draw=none](-4.426,3.103)--(-4.438,3.104)--(-4.438,3.104)--(-4.425,3.1)--cycle;
\draw(-4.426,3.103)--(-4.438,3.104)--(-4.438,3.104)--(-4.425,3.1);
\filldraw[fill opacity=0.8,fill=gray!20,draw=none](-4.422,3.102)--(-4.426,3.103)--(-4.425,3.1)--(-4.41,3.096)--cycle;
\draw(-4.422,3.102)--(-4.426,3.103);
\draw(-4.425,3.1)--(-4.41,3.096);
\filldraw[fill opacity=0.8,fill=gray!20,draw=none](-4.426,2.682)--(-4.428,2.679)--(-4.444,2.678)--cycle;
\draw(-4.426,2.682)--(-4.428,2.679);
\filldraw[fill opacity=0.8,fill=gray!20,draw=none](-4.429,2.681)--(-4.444,2.681)--(-4.444,2.678)--cycle;
\draw(-4.429,2.681)--(-4.444,2.681)--(-4.444,2.678);
\filldraw[fill opacity=0.8,fill=gray!20,draw=none](-7.759,.804)--(-7.755,.835)--(-7.791,.811)--(-7.792,.802)--cycle;
\draw(-7.759,.804)--(-7.755,.835)--(-7.791,.811)--(-7.792,.802);
\filldraw[fill opacity=0.8,fill=gray!20,draw=none](-7.889,1.737)--(-7.897,1.745)--(-7.938,1.775)--(-7.926,1.763)--(-7.886,1.732)--cycle;
\draw(-7.889,1.737)--(-7.897,1.745)--(-7.938,1.775)--(-7.926,1.763)--(-7.886,1.732);
\filldraw[fill opacity=0.8,fill=gray!20,draw=none](-4.248,2.971)--(-4.252,2.926)--(-4.236,2.937)--(-4.237,2.945)--cycle;
\draw(-4.252,2.926)--(-4.236,2.937);
\filldraw[fill opacity=0.8,fill=gray!20,draw=none](-4.311,2.737)--(-4.328,2.716)--(-4.308,2.72)--(-4.298,2.735)--cycle;
\draw(-4.328,2.716)--(-4.308,2.72)--(-4.298,2.735);
\filldraw[fill opacity=0.8,fill=gray!20,draw=none](-4.5,2.985)--(-4.494,2.986)--(-4.488,2.985)--cycle;
\draw(-4.5,2.985)--(-4.494,2.986)--(-4.488,2.985);
\filldraw[fill opacity=0.8,fill=gray!20,draw=none](-4.5,2.985)--(-4.5,2.978)--(-4.494,2.986)--cycle;
\draw(-4.5,2.978)--(-4.494,2.986)--(-4.5,2.985);
\filldraw[fill opacity=0.8,fill=gray!20,draw=none](-4.498,2.978)--(-4.498,2.983)--(-4.516,2.979)--cycle;
\filldraw[fill opacity=0.8,fill=gray!20,draw=none](-4.488,2.999)--(-4.539,2.964)--(-4.484,2.99)--(-4.474,3.001)--cycle;
\draw(-4.484,2.99)--(-4.474,3.001)--(-4.488,2.999);
\filldraw[fill opacity=0.8,fill=gray!20,draw=none](-4.499,3.056)--(-4.499,3.058)--(-4.516,3.076)--(-4.536,3.065)--cycle;
\draw(-4.516,3.076)--(-4.536,3.065)--(-4.499,3.056);
\filldraw[fill opacity=0.8,fill=gray!20,draw=none](-7.535,4.545)--(-7.533,4.551)--(-7.504,4.559)--(-7.467,4.553)--(-7.481,4.532)--cycle;
\draw(-7.467,4.553)--(-7.481,4.532)--(-7.535,4.545)--(-7.533,4.551);
\filldraw[fill opacity=0.8,fill=gray!20,draw=none](-7.645,4.534)--(-7.682,4.509)--(-7.673,4.521)--(-7.644,4.535)--cycle;
\draw(-7.673,4.521)--(-7.644,4.535);
\filldraw[fill opacity=0.8,fill=gray!20,draw=none](-7.649,4.528)--(-7.62,4.534)--(-7.637,4.487)--(-7.675,4.506)--cycle;
\draw(-7.637,4.487)--(-7.675,4.506);
\filldraw[fill opacity=0.8,fill=gray!20,draw=none](-7.649,4.528)--(-7.645,4.534)--(-7.618,4.535)--cycle;
\filldraw[fill opacity=0.8,fill=gray!20,draw=none](-7.62,4.534)--(-7.649,4.528)--(-7.633,4.542)--(-7.619,4.535)--cycle;
\draw(-7.633,4.542)--(-7.619,4.535);
\filldraw[fill opacity=0.8,fill=gray!20,draw=none](-7.649,4.528)--(-7.675,4.506)--(-7.655,4.527)--(-7.645,4.534)--cycle;
\filldraw[fill opacity=0.8,fill=gray!20,draw=none](-7.645,4.534)--(-7.644,4.535)--(-7.641,4.536)--cycle;
\draw(-7.644,4.535)--(-7.641,4.536);
\filldraw[fill opacity=0.8,fill=gray!20,draw=none](-7.649,4.528)--(-7.645,4.534)--(-7.641,4.536)--(-7.638,4.538)--cycle;
\draw(-7.641,4.536)--(-7.638,4.538);
\filldraw[fill opacity=0.8,fill=gray!20,draw=none](-7.649,4.528)--(-7.678,4.523)--(-7.683,4.534)--(-7.645,4.534)--cycle;
\draw(-7.678,4.523)--(-7.683,4.534);
\filldraw[fill opacity=0.8,fill=gray!20,draw=none](-7.638,4.538)--(-7.641,4.536)--(-7.634,4.543)--cycle;
\draw(-7.638,4.538)--(-7.641,4.536);
\filldraw[fill opacity=0.8,fill=gray!20,draw=none](-7.624,4.523)--(-7.619,4.535)--(-7.609,4.53)--cycle;
\draw(-7.619,4.535)--(-7.609,4.53);
\filldraw[fill opacity=0.8,fill=gray!20,draw=none](-7.683,4.522)--(-7.649,4.528)--(-7.675,4.506)--(-7.69,4.514)--cycle;
\draw(-7.675,4.506)--(-7.69,4.514);
\filldraw[fill opacity=0.8,fill=gray!20,draw=none](-7.644,4.514)--(-7.653,4.523)--(-7.649,4.528)--(-7.618,4.535)--(-7.612,4.535)--(-7.612,4.528)--cycle;
\draw(-7.612,4.535)--(-7.612,4.528);
\filldraw[fill opacity=0.8,fill=gray!20,draw=none](-7.614,4.507)--(-7.612,4.535)--(-7.561,4.547)--(-7.535,4.545)--(-7.56,4.503)--cycle;
\draw(-7.561,4.547)--(-7.535,4.545)--(-7.56,4.503)--(-7.614,4.507)--(-7.612,4.535);
\filldraw[fill opacity=0.8,fill=gray!20,draw=none](-7.644,4.514)--(-7.649,4.511)--(-7.659,4.514)--(-7.653,4.523)--cycle;
\filldraw[fill opacity=0.8,fill=gray!20,draw=none](-7.631,4.506)--(-7.638,4.508)--(-7.644,4.514)--(-7.612,4.528)--(-7.614,4.507)--cycle;
\draw(-7.612,4.528)--(-7.614,4.507)--(-7.631,4.506);
\filldraw[fill opacity=0.8,fill=gray!20,draw=none](-7.682,4.523)--(-7.689,4.529)--(-7.681,4.528)--(-7.678,4.523)--cycle;
\draw(-7.681,4.528)--(-7.678,4.523);
\filldraw[fill opacity=0.8,fill=gray!20,draw=none](-7.649,4.528)--(-7.678,4.523)--(-7.682,4.523)--(-7.655,4.553)--(-7.633,4.542)--cycle;
\draw(-7.655,4.553)--(-7.633,4.542);
\filldraw[fill opacity=0.8,fill=gray!20,draw=none](-7.659,4.514)--(-7.682,4.496)--(-7.686,4.494)--(-7.675,4.506)--(-7.649,4.528)--cycle;
\draw(-7.682,4.496)--(-7.686,4.494);
\filldraw[fill opacity=0.8,fill=gray!20,draw=none](-7.665,4.511)--(-7.658,4.507)--(-7.691,4.492)--(-7.69,4.514)--(-7.689,4.515)--cycle;
\filldraw[fill opacity=0.8,fill=gray!20,draw=none](-7.658,4.507)--(-7.676,4.52)--(-7.649,4.511)--cycle;
\filldraw[fill opacity=0.8,fill=gray!20,draw=none](-7.665,4.511)--(-7.689,4.515)--(-7.683,4.522)--(-7.676,4.52)--cycle;
\filldraw[fill opacity=0.8,fill=gray!20,draw=none](-7.624,4.544)--(-7.653,4.523)--(-7.649,4.528)--(-7.638,4.538)--(-7.621,4.546)--cycle;
\draw(-7.638,4.538)--(-7.621,4.546);
\filldraw[fill opacity=0.8,fill=gray!20,draw=none](-7.621,4.546)--(-7.638,4.538)--(-7.634,4.543)--(-7.608,4.567)--cycle;
\draw(-7.621,4.546)--(-7.638,4.538);
\filldraw[fill opacity=0.8,fill=gray!20,draw=none](-7.649,4.511)--(-7.659,4.514)--(-7.624,4.544)--(-7.634,4.518)--cycle;
\draw(-7.624,4.544)--(-7.634,4.518);
\filldraw[fill opacity=0.8,fill=gray!20,draw=none](-7.631,4.506)--(-7.614,4.507)--(-7.614,4.503)--cycle;
\draw(-7.631,4.506)--(-7.614,4.507)--(-7.614,4.503);
\filldraw[fill opacity=0.8,fill=gray!20,draw=none](-7.59,4.49)--(-7.615,4.495)--(-7.614,4.507)--(-7.609,4.507)--cycle;
\draw(-7.615,4.495)--(-7.614,4.507)--(-7.609,4.507);
\filldraw[fill opacity=0.8,fill=gray!20,draw=none](-7.59,4.49)--(-7.609,4.507)--(-7.56,4.503)--(-7.575,4.487)--cycle;
\draw(-7.609,4.507)--(-7.56,4.503)--(-7.575,4.487);
\filldraw[fill opacity=0.8,fill=gray!20,draw=none](-7.692,4.459)--(-7.691,4.492)--(-7.649,4.511)--(-7.638,4.508)--(-7.673,4.419)--cycle;
\draw(-7.638,4.508)--(-7.673,4.419);
\filldraw[fill opacity=0.8,fill=gray!20,draw=none](-7.638,4.508)--(-7.631,4.506)--(-7.636,4.506)--cycle;
\draw(-7.631,4.506)--(-7.636,4.506);
\filldraw[fill opacity=0.8,fill=gray!20,draw=none](-7.668,4.4)--(-7.673,4.419)--(-7.638,4.507)--(-7.593,4.484)--(-7.641,4.36)--cycle;
\draw(-7.673,4.419)--(-7.638,4.507);
\draw(-7.593,4.484)--(-7.641,4.36);
\filldraw[fill opacity=0.8,fill=gray!20,draw=none](-7.663,4.509)--(-7.659,4.514)--(-7.649,4.511)--(-7.657,4.507)--cycle;
\filldraw[fill opacity=0.8,fill=gray!20,draw=none](-7.648,4.505)--(-7.649,4.505)--(-7.631,4.506)--(-7.614,4.503)--(-7.614,4.5)--cycle;
\draw(-7.649,4.505)--(-7.631,4.506);
\draw(-7.614,4.503)--(-7.614,4.5);
\filldraw[fill opacity=0.8,fill=gray!20,draw=none](-7.624,4.544)--(-7.621,4.546)--(-7.62,4.547)--cycle;
\draw(-7.621,4.546)--(-7.62,4.547);
\filldraw[fill opacity=0.8,fill=gray!20,draw=none](-7.62,4.547)--(-7.621,4.546)--(-7.608,4.567)--(-7.6,4.574)--cycle;
\draw(-7.62,4.547)--(-7.621,4.546);
\filldraw[fill opacity=0.8,fill=gray!20,draw=none](-7.627,4.502)--(-7.636,4.506)--(-7.638,4.508)--(-7.616,4.564)--(-7.605,4.571)--cycle;
\draw(-7.638,4.508)--(-7.616,4.564);
\filldraw[fill opacity=0.8,fill=gray!20,draw=none](-7.659,4.514)--(-7.676,4.52)--(-7.678,4.523)--(-7.649,4.528)--cycle;
\draw(-7.676,4.52)--(-7.678,4.523);
\filldraw[fill opacity=0.8,fill=gray!20,draw=none](-7.624,4.544)--(-7.659,4.514)--(-7.653,4.523)--cycle;
\filldraw[fill opacity=0.8,fill=gray!20,draw=none](-7.691,4.53)--(-7.703,4.533)--(-7.716,4.542)--(-7.71,4.543)--cycle;
\draw(-7.716,4.542)--(-7.71,4.543);
\filldraw[fill opacity=0.8,fill=gray!20,draw=none](-7.69,4.514)--(-7.69,4.524)--(-7.683,4.522)--cycle;
\filldraw[fill opacity=0.8,fill=gray!20,draw=none](-7.683,4.522)--(-7.69,4.514)--(-7.699,4.518)--cycle;
\draw(-7.69,4.514)--(-7.699,4.518);
\filldraw[fill opacity=0.8,fill=gray!20,draw=none](-7.682,4.523)--(-7.678,4.523)--(-7.683,4.522)--cycle;
\filldraw[fill opacity=0.8,fill=gray!20,draw=none](-7.676,4.52)--(-7.69,4.524)--(-7.689,4.529)--cycle;
\filldraw[fill opacity=0.8,fill=gray!20,draw=none](-7.691,4.525)--(-7.698,4.518)--(-7.697,4.529)--cycle;
\filldraw[fill opacity=0.8,fill=gray!20,draw=none](-7.682,4.523)--(-7.683,4.522)--(-7.699,4.518)--(-7.723,4.53)--cycle;
\draw(-7.699,4.518)--(-7.723,4.53);
\filldraw[fill opacity=0.8,fill=gray!20,draw=none](-7.666,4.507)--(-7.663,4.505)--(-7.676,4.499)--cycle;
\filldraw[fill opacity=0.8,fill=gray!20,draw=none](-7.637,4.532)--(-7.644,4.513)--(-7.663,4.505)--(-7.666,4.507)--cycle;
\draw(-7.637,4.532)--(-7.644,4.513);
\filldraw[fill opacity=0.8,fill=gray!20,draw=none](-7.663,4.509)--(-7.657,4.507)--(-7.663,4.505)--(-7.667,4.505)--cycle;
\draw(-7.663,4.505)--(-7.667,4.505);
\filldraw[fill opacity=0.8,fill=gray!20,draw=none](-7.638,4.508)--(-7.649,4.511)--(-7.644,4.514)--cycle;
\filldraw[fill opacity=0.8,fill=gray!20,draw=none](-7.649,4.511)--(-7.634,4.518)--(-7.638,4.508)--cycle;
\draw(-7.634,4.518)--(-7.638,4.508);
\filldraw[fill opacity=0.8,fill=gray!20,draw=none](-7.657,4.507)--(-7.649,4.511)--(-7.638,4.508)--(-7.636,4.506)--(-7.651,4.505)--cycle;
\draw(-7.636,4.506)--(-7.651,4.505);
\filldraw[fill opacity=0.8,fill=gray!20,draw=none](-7.636,4.506)--(-7.638,4.507)--(-7.638,4.508)--cycle;
\draw(-7.638,4.507)--(-7.638,4.508);
\filldraw[fill opacity=0.8,fill=gray!20,draw=none](-7.644,4.513)--(-7.657,4.485)--(-7.682,4.496)--cycle;
\draw(-7.644,4.513)--(-7.657,4.485)--(-7.682,4.496);
\filldraw[fill opacity=0.8,fill=gray!20,draw=none](-7.648,4.505)--(-7.614,4.5)--(-7.615,4.488)--cycle;
\draw(-7.614,4.5)--(-7.615,4.488);
\filldraw[fill opacity=0.8,fill=gray!20,draw=none](-7.615,4.495)--(-7.627,4.502)--(-7.605,4.571)--(-7.6,4.574)--(-7.555,4.58)--(-7.59,4.49)--cycle;
\draw(-7.555,4.58)--(-7.59,4.49);
\filldraw[fill opacity=0.8,fill=gray!20,draw=none](-7.608,4.552)--(-7.643,4.502)--(-7.648,4.505)--(-7.637,4.532)--cycle;
\draw(-7.648,4.505)--(-7.637,4.532);
\filldraw[fill opacity=0.8,fill=gray!20,draw=none](-7.666,4.509)--(-7.667,4.504)--(-7.668,4.502)--(-7.682,4.496)--cycle;
\draw(-7.668,4.502)--(-7.682,4.496);
\filldraw[fill opacity=0.8,fill=gray!20,draw=none](-7.668,4.503)--(-7.668,4.502)--(-7.667,4.505)--cycle;
\draw(-7.668,4.503)--(-7.668,4.502);
\filldraw[fill opacity=0.8,fill=gray!20,draw=none](-7.668,4.502)--(-7.669,4.504)--(-7.663,4.505)--cycle;
\draw(-7.668,4.502)--(-7.669,4.504)--(-7.663,4.505);
\filldraw[fill opacity=0.8,fill=gray!20,draw=none](-7.615,4.495)--(-7.59,4.49)--(-7.593,4.484)--cycle;
\draw(-7.59,4.49)--(-7.593,4.484);
\filldraw[fill opacity=0.8,fill=gray!20,draw=none](-4.749,2.899)--(-7.612,4.332)--(-7.623,4.347)--(-4.744,2.906)--cycle;
\draw(-4.749,2.899)--(-7.612,4.332);
\draw(-7.623,4.347)--(-4.744,2.906);
\filldraw[fill opacity=0.8,fill=gray!20,draw=none](-7.561,4.547)--(-7.552,4.55)--(-7.533,4.551)--(-7.535,4.545)--cycle;
\draw(-7.533,4.551)--(-7.535,4.545)--(-7.561,4.547);
\filldraw[fill opacity=0.8,fill=gray!20](-7.56,4.503)--(-7.535,4.545)--(-7.481,4.532)--(-7.522,4.494)--cycle;
\filldraw[fill opacity=0.8,fill=gray!20,draw=none](-7.612,4.332)--(-7.639,4.355)--(-7.623,4.347)--cycle;
\draw(-7.639,4.355)--(-7.623,4.347);
\filldraw[fill opacity=0.8,fill=gray!20,draw=none](-7.598,4.47)--(-7.592,4.485)--(-7.557,4.474)--cycle;
\draw(-7.598,4.47)--(-7.592,4.485);
\filldraw[fill opacity=0.8,fill=gray!20,draw=none](-7.639,4.355)--(-7.641,4.36)--(-7.598,4.47)--(-7.557,4.474)--(-7.612,4.332)--cycle;
\draw(-7.641,4.36)--(-7.598,4.47);
\draw(-7.557,4.474)--(-7.612,4.332);
\filldraw[fill opacity=0.8,fill=gray!20,draw=none](-7.601,4.486)--(-7.615,4.488)--(-7.615,4.495)--(-7.612,4.495)--cycle;
\draw(-7.615,4.488)--(-7.615,4.495);
\filldraw[fill opacity=0.8,fill=gray!20,draw=none](-7.564,4.498)--(-7.56,4.503)--(-7.522,4.494)--(-7.557,4.474)--cycle;
\draw(-7.564,4.498)--(-7.56,4.503)--(-7.522,4.494)--(-7.557,4.474);
\filldraw[fill opacity=0.8,fill=gray!20,draw=none](-7.587,4.474)--(-7.564,4.498)--(-7.557,4.474)--(-7.568,4.468)--cycle;
\draw(-7.587,4.474)--(-7.564,4.498);
\draw(-7.557,4.474)--(-7.568,4.468);
\filldraw[fill opacity=0.8,fill=gray!20,draw=none](-7.59,4.49)--(-7.575,4.487)--(-7.58,4.481)--cycle;
\draw(-7.575,4.487)--(-7.58,4.481);
\filldraw[fill opacity=0.8,fill=gray!20,draw=none](-7.592,4.485)--(-7.555,4.58)--(-7.55,4.582)--(-7.516,4.578)--(-7.557,4.474)--cycle;
\draw(-7.592,4.485)--(-7.555,4.58);
\draw(-7.516,4.578)--(-7.557,4.474);
\filldraw[fill opacity=0.8,fill=gray!20,draw=none](-7.601,4.486)--(-7.612,4.495)--(-7.59,4.49)--(-7.58,4.481)--cycle;
\filldraw[fill opacity=0.8,fill=gray!20,draw=none](-7.608,4.552)--(-7.569,4.581)--(-7.559,4.582)--(-7.601,4.486)--(-7.616,4.488)--(-7.643,4.502)--cycle;
\draw(-7.559,4.582)--(-7.601,4.486);
\filldraw[fill opacity=0.8,fill=gray!20,draw=none](-7.691,4.525)--(-7.678,4.523)--(-7.673,4.512)--cycle;
\draw(-7.678,4.523)--(-7.673,4.512);
\filldraw[fill opacity=0.8,fill=gray!20,draw=none](-7.691,4.53)--(-7.688,4.529)--(-7.689,4.529)--cycle;
\filldraw[fill opacity=0.8,fill=gray!20,draw=none](-7.659,4.514)--(-7.676,4.52)--(-7.689,4.529)--(-7.689,4.533)--(-7.685,4.545)--(-7.616,4.564)--(-7.624,4.544)--cycle;
\draw(-7.689,4.533)--(-7.685,4.545);
\draw(-7.616,4.564)--(-7.624,4.544);
\filldraw[fill opacity=0.8,fill=gray!20,draw=none](-7.663,4.509)--(-7.659,4.514)--(-7.624,4.544)--(-7.62,4.547)--(-7.616,4.548)--cycle;
\draw(-7.62,4.547)--(-7.616,4.548);
\filldraw[fill opacity=0.8,fill=gray!20,draw=none](-7.616,4.548)--(-7.62,4.547)--(-7.605,4.566)--cycle;
\draw(-7.616,4.548)--(-7.62,4.547);
\filldraw[fill opacity=0.8,fill=gray!20,draw=none](-7.637,4.532)--(-7.616,4.548)--(-7.608,4.552)--cycle;
\draw(-7.616,4.548)--(-7.608,4.552);
\filldraw[fill opacity=0.8,fill=gray!20,draw=none](-7.59,4.578)--(-7.608,4.552)--(-7.616,4.548)--(-7.605,4.566)--(-7.6,4.574)--cycle;
\draw(-7.608,4.552)--(-7.616,4.548);
\filldraw[fill opacity=0.8,fill=gray!20,draw=none](-7.605,4.576)--(-7.569,4.581)--(-7.637,4.532)--(-7.621,4.567)--cycle;
\draw(-7.637,4.532)--(-7.621,4.567);
\filldraw[fill opacity=0.8,fill=gray!20,draw=none](-7.682,4.523)--(-7.718,4.529)--(-7.719,4.53)--(-7.689,4.529)--cycle;
\filldraw[fill opacity=0.8,fill=gray!20,draw=none](-7.721,4.55)--(-7.716,4.542)--(-7.716,4.542)--(-7.736,4.55)--cycle;
\draw(-7.716,4.542)--(-7.716,4.542);
\filldraw[fill opacity=0.8,fill=gray!20,draw=none](-7.703,4.533)--(-7.726,4.54)--(-7.716,4.542)--cycle;
\draw(-7.726,4.54)--(-7.716,4.542);
\filldraw[fill opacity=0.8,fill=gray!20,draw=none](-7.743,4.546)--(-7.74,4.553)--(-7.694,4.546)--(-7.689,4.533)--(-7.691,4.53)--cycle;
\draw(-7.743,4.546)--(-7.74,4.553);
\draw(-7.689,4.533)--(-7.691,4.53);
\filldraw[fill opacity=0.8,fill=gray!20,draw=none](-7.689,4.529)--(-7.691,4.53)--(-7.689,4.533)--cycle;
\draw(-7.691,4.53)--(-7.689,4.533);
\filldraw[fill opacity=0.8,fill=gray!20,draw=none](-7.703,4.533)--(-7.691,4.53)--(-7.689,4.529)--(-7.697,4.529)--cycle;
\filldraw[fill opacity=0.8,fill=gray!20,draw=none](-7.694,4.546)--(-7.685,4.545)--(-7.689,4.533)--cycle;
\draw(-7.685,4.545)--(-7.689,4.533);
\filldraw[fill opacity=0.8,fill=gray!20,draw=none](-7.703,4.533)--(-7.699,4.531)--(-7.729,4.54)--(-7.726,4.54)--cycle;
\draw(-7.729,4.54)--(-7.726,4.54);
\filldraw[fill opacity=0.8,fill=gray!20,draw=none](-7.661,4.556)--(-7.691,4.525)--(-7.697,4.529)--(-7.696,4.537)--(-7.691,4.548)--cycle;
\draw(-7.696,4.537)--(-7.691,4.548);
\filldraw[fill opacity=0.8,fill=gray!20,draw=none](-7.701,4.55)--(-7.691,4.548)--(-7.696,4.537)--cycle;
\draw(-7.691,4.548)--(-7.696,4.537);
\filldraw[fill opacity=0.8,fill=gray!20,draw=none](-7.729,4.54)--(-7.751,4.547)--(-7.749,4.552)--cycle;
\draw(-7.751,4.547)--(-7.749,4.552);
\filldraw[fill opacity=0.8,fill=gray!20,draw=none](-7.787,4.542)--(-7.752,4.545)--(-7.748,4.542)--(-7.754,4.528)--(-7.757,4.529)--cycle;
\filldraw[fill opacity=0.8,fill=gray!20,draw=none](-7.752,4.545)--(-7.747,4.545)--(-7.748,4.542)--cycle;
\filldraw[fill opacity=0.8,fill=gray!20,draw=none](-7.759,4.551)--(-7.749,4.552)--(-7.752,4.546)--cycle;
\draw(-7.749,4.552)--(-7.752,4.546);
\filldraw[fill opacity=0.8,fill=gray!20,draw=none](-7.747,4.545)--(-7.752,4.545)--(-7.758,4.549)--(-7.745,4.549)--cycle;
\filldraw[fill opacity=0.8,fill=gray!20,draw=none](-7.781,4.539)--(-7.787,4.542)--(-7.79,4.543)--(-7.786,4.541)--cycle;
\draw(-7.79,4.543)--(-7.786,4.541);
\filldraw[fill opacity=0.8,fill=gray!20,draw=none](-7.752,4.545)--(-7.787,4.542)--(-7.79,4.543)--(-7.789,4.547)--(-7.758,4.549)--cycle;
\draw(-7.79,4.543)--(-7.789,4.547);
\filldraw[fill opacity=0.8,fill=gray!20,draw=none](-7.74,4.549)--(-7.726,4.54)--(-7.741,4.537)--cycle;
\draw(-7.726,4.54)--(-7.741,4.537);
\filldraw[fill opacity=0.8,fill=gray!20,draw=none](-7.736,4.555)--(-7.701,4.55)--(-7.696,4.537)--(-7.699,4.531)--(-7.729,4.54)--(-7.741,4.547)--cycle;
\draw(-7.696,4.537)--(-7.699,4.531);
\filldraw[fill opacity=0.8,fill=gray!20,draw=none](-7.697,4.529)--(-7.699,4.531)--(-7.696,4.537)--cycle;
\draw(-7.699,4.531)--(-7.696,4.537);
\filldraw[fill opacity=0.8,fill=gray!20,draw=none](-7.699,4.531)--(-7.697,4.529)--(-7.719,4.53)--(-7.732,4.539)--(-7.729,4.54)--cycle;
\draw(-7.732,4.539)--(-7.729,4.54);
\filldraw[fill opacity=0.8,fill=gray!20,draw=none](-7.682,4.523)--(-7.723,4.53)--(-7.738,4.538)--(-7.676,4.564)--(-7.655,4.553)--cycle;
\draw(-7.723,4.53)--(-7.738,4.538);
\draw(-7.676,4.564)--(-7.655,4.553);
\filldraw[fill opacity=0.8,fill=gray!20,draw=none](-7.666,4.509)--(-7.663,4.511)--(-7.667,4.505)--(-7.667,4.504)--cycle;
\filldraw[fill opacity=0.8,fill=gray!20,draw=none](-7.663,4.511)--(-7.659,4.514)--(-7.667,4.505)--cycle;
\filldraw[fill opacity=0.8,fill=gray!20,draw=none](-7.697,4.519)--(-7.673,4.512)--(-7.666,4.507)--(-7.676,4.499)--(-7.682,4.496)--(-7.699,4.503)--(-7.698,4.518)--cycle;
\draw(-7.682,4.496)--(-7.699,4.503);
\filldraw[fill opacity=0.8,fill=gray!20,draw=none](-7.659,4.514)--(-7.663,4.509)--(-7.673,4.512)--(-7.676,4.52)--cycle;
\draw(-7.673,4.512)--(-7.676,4.52);
\filldraw[fill opacity=0.8,fill=gray!20,draw=none](-7.697,4.519)--(-7.691,4.525)--(-7.673,4.512)--cycle;
\filldraw[fill opacity=0.8,fill=gray!20,draw=none](-7.704,4.519)--(-7.718,4.529)--(-7.691,4.525)--(-7.673,4.512)--(-7.672,4.509)--cycle;
\draw(-7.673,4.512)--(-7.672,4.509);
\filldraw[fill opacity=0.8,fill=gray!20,draw=none](-7.661,4.556)--(-7.621,4.567)--(-7.637,4.532)--(-7.666,4.507)--(-7.691,4.525)--cycle;
\draw(-7.621,4.567)--(-7.637,4.532);
\filldraw[fill opacity=0.8,fill=gray!20,draw=none](-7.634,4.53)--(-7.639,4.516)--(-7.657,4.508)--(-7.66,4.509)--cycle;
\draw(-7.634,4.53)--(-7.639,4.516);
\draw(-7.657,4.508)--(-7.66,4.509);
\filldraw[fill opacity=0.8,fill=gray!20,draw=none](-7.567,4.577)--(-7.58,4.544)--(-7.639,4.516)--(-7.634,4.53)--cycle;
\draw(-7.567,4.577)--(-7.58,4.544);
\draw(-7.639,4.516)--(-7.634,4.53);
\filldraw[fill opacity=0.8,fill=gray!20,draw=none](-7.663,4.509)--(-7.667,4.505)--(-7.669,4.504)--(-7.673,4.512)--cycle;
\draw(-7.667,4.505)--(-7.669,4.504)--(-7.673,4.512);
\filldraw[fill opacity=0.8,fill=gray!20,draw=none](-7.66,4.509)--(-7.666,4.503)--(-7.668,4.503)--(-7.667,4.505)--(-7.663,4.509)--(-7.662,4.51)--cycle;
\draw(-7.666,4.503)--(-7.668,4.503);
\filldraw[fill opacity=0.8,fill=gray!20,draw=none](-7.66,4.509)--(-7.662,4.51)--(-7.637,4.532)--(-7.608,4.552)--(-7.605,4.554)--cycle;
\draw(-7.608,4.552)--(-7.605,4.554);
\filldraw[fill opacity=0.8,fill=gray!20,draw=none](-7.68,4.502)--(-7.704,4.519)--(-7.672,4.509)--(-7.669,4.504)--cycle;
\draw(-7.672,4.509)--(-7.669,4.504)--(-7.68,4.502);
\filldraw[fill opacity=0.8,fill=gray!20,draw=none](-7.673,4.5)--(-7.682,4.496)--(-7.709,4.508)--(-7.707,4.512)--cycle;
\draw(-7.709,4.508)--(-7.707,4.512);
\filldraw[fill opacity=0.8,fill=gray!20,draw=none](-7.682,4.496)--(-7.692,4.5)--(-7.669,4.504)--(-7.668,4.502)--cycle;
\draw(-7.692,4.5)--(-7.669,4.504)--(-7.668,4.502);
\filldraw[fill opacity=0.8,fill=gray!20,draw=none](-7.671,4.565)--(-7.687,4.519)--(-7.692,4.521)--(-7.676,4.564)--cycle;
\draw(-7.687,4.519)--(-7.692,4.521)--(-7.676,4.564);
\filldraw[fill opacity=0.8,fill=gray!20,draw=none](-7.676,4.564)--(-7.692,4.521)--(-7.738,4.538)--(-7.725,4.571)--cycle;
\draw(-7.676,4.564)--(-7.692,4.521)--(-7.738,4.538)--(-7.725,4.571);
\filldraw[fill opacity=0.8,fill=gray!20,draw=none](-7.671,4.565)--(-7.615,4.581)--(-7.634,4.53)--(-7.66,4.509)--(-7.687,4.519)--cycle;
\draw(-7.615,4.581)--(-7.634,4.53);
\draw(-7.66,4.509)--(-7.687,4.519);
\filldraw[fill opacity=0.8,fill=gray!20,draw=none](-7.738,4.538)--(-7.732,4.548)--(-7.682,4.567)--(-7.676,4.564)--cycle;
\draw(-7.738,4.538)--(-7.732,4.548);
\draw(-7.682,4.567)--(-7.676,4.564);
\filldraw[fill opacity=0.8,fill=gray!20,draw=none](-7.657,4.508)--(-7.658,4.508)--(-7.649,4.515)--(-7.651,4.51)--cycle;
\draw(-7.649,4.515)--(-7.651,4.51);
\filldraw[fill opacity=0.8,fill=gray!20,draw=none](-7.603,4.533)--(-7.651,4.51)--(-7.649,4.515)--(-7.581,4.565)--cycle;
\draw(-7.651,4.51)--(-7.649,4.515);
\filldraw[fill opacity=0.8,fill=gray!20,draw=none](-7.607,4.472)--(-7.61,4.465)--(-7.657,4.485)--(-7.65,4.5)--cycle;
\draw(-7.607,4.472)--(-7.61,4.465)--(-7.657,4.485)--(-7.65,4.5);
\filldraw[fill opacity=0.8,fill=gray!20,draw=none](-7.616,4.488)--(-7.603,4.482)--(-7.607,4.472)--(-7.633,4.489)--(-7.637,4.493)--cycle;
\draw(-7.603,4.482)--(-7.607,4.472);
\filldraw[fill opacity=0.8,fill=gray!20,draw=none](-7.601,4.486)--(-7.603,4.482)--(-7.616,4.488)--cycle;
\draw(-7.601,4.486)--(-7.603,4.482);
\filldraw[fill opacity=0.8,fill=gray!20,draw=none](-7.572,4.46)--(-7.576,4.45)--(-7.61,4.465)--(-7.601,4.486)--cycle;
\draw(-7.572,4.46)--(-7.576,4.45)--(-7.61,4.465)--(-7.601,4.486);
\filldraw[fill opacity=0.8,fill=gray!20,draw=none](-7.612,4.474)--(-7.617,4.475)--(-7.615,4.488)--(-7.58,4.481)--(-7.589,4.472)--cycle;
\draw(-7.617,4.475)--(-7.615,4.488);
\draw(-7.58,4.481)--(-7.589,4.472)--(-7.612,4.474);
\filldraw[fill opacity=0.8,fill=gray!20,draw=none](-7.554,4.584)--(-7.532,4.581)--(-7.58,4.467)--(-7.601,4.486)--(-7.559,4.582)--cycle;
\draw(-7.601,4.486)--(-7.559,4.582);
\filldraw[fill opacity=0.8,fill=gray!20,draw=none](-7.603,4.533)--(-7.581,4.565)--(-7.575,4.57)--(-7.588,4.54)--cycle;
\draw(-7.575,4.57)--(-7.588,4.54);
\filldraw[fill opacity=0.8,fill=gray!20,draw=none](-7.532,4.581)--(-7.52,4.58)--(-7.572,4.46)--(-7.58,4.467)--cycle;
\draw(-7.52,4.58)--(-7.572,4.46);
\filldraw[fill opacity=0.8,fill=gray!20,draw=none](-7.529,4.584)--(-7.571,4.473)--(-7.602,4.486)--(-7.563,4.589)--cycle;
\draw(-7.529,4.584)--(-7.571,4.473)--(-7.602,4.486)--(-7.563,4.589);
\filldraw[fill opacity=0.8,fill=gray!20,draw=none](-7.58,4.544)--(-7.602,4.486)--(-7.644,4.503)--(-7.639,4.516)--cycle;
\draw(-7.58,4.544)--(-7.602,4.486)--(-7.644,4.503)--(-7.639,4.516);
\filldraw[fill opacity=0.8,fill=gray!20,draw=none](-7.658,4.508)--(-7.66,4.509)--(-7.605,4.554)--(-7.6,4.556)--cycle;
\draw(-7.605,4.554)--(-7.6,4.556);
\filldraw[fill opacity=0.8,fill=gray!20,draw=none](-7.59,4.578)--(-7.57,4.588)--(-7.59,4.561)--(-7.608,4.552)--cycle;
\draw(-7.59,4.561)--(-7.608,4.552);
\filldraw[fill opacity=0.8,fill=gray!20,draw=none](-7.619,4.54)--(-7.6,4.556)--(-7.59,4.561)--cycle;
\draw(-7.6,4.556)--(-7.59,4.561);
\filldraw[fill opacity=0.8,fill=gray!20,draw=none](-7.563,4.589)--(-7.567,4.577)--(-7.634,4.53)--(-7.615,4.581)--cycle;
\draw(-7.563,4.589)--(-7.567,4.577);
\draw(-7.634,4.53)--(-7.615,4.581);
\filldraw[fill opacity=0.8,fill=gray!20,draw=none](-7.611,4.489)--(-7.569,4.583)--(-7.561,4.589)--(-7.531,4.585)--(-7.58,4.477)--cycle;
\draw(-7.611,4.489)--(-7.569,4.583);
\draw(-7.531,4.585)--(-7.58,4.477);
\filldraw[fill opacity=0.8,fill=gray!20,draw=none](-7.649,4.515)--(-7.619,4.54)--(-7.59,4.561)--(-7.581,4.565)--cycle;
\draw(-7.59,4.561)--(-7.581,4.565);
\filldraw[fill opacity=0.8,fill=gray!20,draw=none](-7.57,4.588)--(-7.565,4.59)--(-7.564,4.59)--(-7.581,4.565)--(-7.59,4.561)--cycle;
\draw(-7.581,4.565)--(-7.59,4.561);
\filldraw[fill opacity=0.8,fill=gray!20,draw=none](-7.649,4.515)--(-7.619,4.583)--(-7.566,4.59)--(-7.575,4.57)--cycle;
\draw(-7.649,4.515)--(-7.619,4.583);
\draw(-7.566,4.59)--(-7.575,4.57);
\filldraw[fill opacity=0.8,fill=gray!20,draw=none](-7.564,4.59)--(-7.536,4.588)--(-7.535,4.587)--(-7.581,4.565)--cycle;
\draw(-7.535,4.587)--(-7.581,4.565);
\filldraw[fill opacity=0.8,fill=gray!20,draw=none](-7.481,4.532)--(-7.467,4.553)--(-7.459,4.539)--(-7.461,4.52)--(-7.465,4.515)--cycle;
\draw(-7.461,4.52)--(-7.465,4.515)--(-7.481,4.532)--(-7.467,4.553);
\filldraw[fill opacity=0.8,fill=gray!20,draw=none](-7.658,4.508)--(-7.701,4.524)--(-7.682,4.567)--(-7.619,4.583)--(-7.649,4.515)--cycle;
\draw(-7.701,4.524)--(-7.682,4.567);
\draw(-7.619,4.583)--(-7.649,4.515);
\filldraw[fill opacity=0.8,fill=gray!20,draw=none](-7.751,4.555)--(-7.741,4.549)--(-7.758,4.549)--(-7.764,4.56)--cycle;
\draw(-7.758,4.549)--(-7.764,4.56);
\filldraw[fill opacity=0.8,fill=gray!20,draw=none](-7.742,4.556)--(-7.745,4.549)--(-7.749,4.552)--(-7.747,4.557)--cycle;
\draw(-7.749,4.552)--(-7.747,4.557);
\filldraw[fill opacity=0.8,fill=gray!20,draw=none](-7.751,4.555)--(-7.764,4.56)--(-7.767,4.565)--cycle;
\draw(-7.764,4.56)--(-7.767,4.565);
\filldraw[fill opacity=0.8,fill=gray!20,draw=none](-7.779,4.566)--(-7.759,4.551)--(-7.775,4.551)--(-7.781,4.56)--cycle;
\filldraw[fill opacity=0.8,fill=gray!20,draw=none](-7.775,4.551)--(-7.797,4.55)--(-7.789,4.57)--cycle;
\draw(-7.797,4.55)--(-7.789,4.57);
\filldraw[fill opacity=0.8,fill=gray!20,draw=none](-7.788,4.572)--(-7.793,4.575)--(-7.779,4.584)--(-7.77,4.57)--cycle;
\draw(-7.793,4.575)--(-7.779,4.584)--(-7.77,4.57);
\filldraw[fill opacity=0.8,fill=gray!20,draw=none](-7.779,4.566)--(-7.788,4.572)--(-7.77,4.57)--(-7.767,4.564)--cycle;
\draw(-7.77,4.57)--(-7.767,4.564);
\filldraw[fill opacity=0.8,fill=gray!20,draw=none](-7.779,4.566)--(-7.767,4.564)--(-7.759,4.551)--cycle;
\draw(-7.767,4.564)--(-7.759,4.551);
\filldraw[fill opacity=0.8,fill=gray!20,draw=none](-7.779,4.566)--(-7.744,4.658)--(-7.711,4.639)--(-7.749,4.552)--(-7.759,4.551)--cycle;
\draw(-7.711,4.639)--(-7.749,4.552);
\filldraw[fill opacity=0.8,fill=gray!20,draw=none](-7.797,4.573)--(-7.804,4.576)--(-7.786,4.606)--(-7.779,4.584)--cycle;
\draw(-7.786,4.606)--(-7.779,4.584)--(-7.797,4.573);
\filldraw[fill opacity=0.8,fill=gray!20,draw=none](-7.744,4.658)--(-7.781,4.56)--(-7.789,4.57)--(-7.749,4.66)--cycle;
\draw(-7.789,4.57)--(-7.749,4.66);
\filldraw[fill opacity=0.8,fill=gray!20,draw=none](-7.781,4.566)--(-7.779,4.566)--(-7.759,4.551)--(-7.758,4.549)--cycle;
\draw(-7.759,4.551)--(-7.758,4.549);
\filldraw[fill opacity=0.8,fill=gray!20,draw=none](-7.813,4.579)--(-7.817,4.58)--(-7.831,4.613)--(-7.831,4.616)--(-7.797,4.639)--(-7.789,4.616)--cycle;
\draw(-7.817,4.58)--(-7.831,4.613);
\draw(-7.831,4.616)--(-7.797,4.639)--(-7.789,4.616);
\filldraw[fill opacity=0.8,fill=gray!20,draw=none](-7.804,4.576)--(-7.813,4.579)--(-7.789,4.616)--(-7.786,4.606)--cycle;
\draw(-7.789,4.616)--(-7.786,4.606);
\filldraw[fill opacity=0.8,fill=gray!20,draw=none](-7.788,4.572)--(-7.789,4.57)--(-7.797,4.573)--cycle;
\draw(-7.788,4.572)--(-7.789,4.57);
\filldraw[fill opacity=0.8,fill=gray!20,draw=none](-7.781,4.566)--(-7.79,4.572)--(-7.788,4.572)--(-7.779,4.566)--cycle;
\filldraw[fill opacity=0.8,fill=gray!20,draw=none](-7.721,4.608)--(-7.745,4.549)--(-7.758,4.549)--(-7.781,4.566)--(-7.777,4.577)--cycle;
\draw(-7.781,4.566)--(-7.777,4.577);
\filldraw[fill opacity=0.8,fill=gray!20,draw=none](-7.709,4.638)--(-7.721,4.608)--(-7.777,4.577)--(-7.745,4.659)--cycle;
\draw(-7.777,4.577)--(-7.745,4.659);
\filldraw[fill opacity=0.8,fill=gray!20,draw=none](-7.569,4.583)--(-7.566,4.59)--(-7.561,4.589)--cycle;
\draw(-7.569,4.583)--(-7.566,4.59);
\filldraw[fill opacity=0.8,fill=gray!20,draw=none](-7.635,4.631)--(-7.546,4.589)--(-7.583,4.573)--(-7.634,4.568)--(-7.683,4.577)--(-7.686,4.579)--cycle;
\draw(-7.546,4.589)--(-7.583,4.573)--(-7.634,4.568)--(-7.683,4.577)--(-7.686,4.579);
\filldraw[fill opacity=0.8,fill=gray!20,draw=none](-7.58,4.544)--(-7.596,4.537)--(-7.582,4.565)--(-7.581,4.565)--(-7.558,4.576)--cycle;
\draw(-7.58,4.544)--(-7.596,4.537);
\draw(-7.581,4.565)--(-7.558,4.576);
\filldraw[fill opacity=0.8,fill=gray!20,draw=none](-7.557,4.474)--(-7.522,4.494)--(-7.511,4.482)--(-7.547,4.467)--cycle;
\draw(-7.557,4.474)--(-7.522,4.494)--(-7.511,4.482)--(-7.547,4.467);
\filldraw[fill opacity=0.8,fill=gray!20,draw=none](-7.594,4.335)--(-7.612,4.332)--(-7.56,4.465)--(-7.547,4.467)--(-7.542,4.464)--(-7.591,4.338)--cycle;
\draw(-7.612,4.332)--(-7.56,4.465);
\draw(-7.542,4.464)--(-7.591,4.338);
\filldraw[fill opacity=0.8,fill=gray!20,draw=none](-7.547,4.467)--(-7.54,4.468)--(-7.542,4.464)--cycle;
\draw(-7.54,4.468)--(-7.542,4.464);
\filldraw[fill opacity=0.8,fill=gray!20,draw=none](-7.547,4.467)--(-7.557,4.474)--(-7.516,4.578)--(-7.506,4.564)--cycle;
\draw(-7.557,4.474)--(-7.516,4.578);
\filldraw[fill opacity=0.8,fill=gray!20,draw=none](-7.56,4.465)--(-7.557,4.474)--(-7.547,4.467)--cycle;
\draw(-7.56,4.465)--(-7.557,4.474);
\filldraw[fill opacity=0.8,fill=gray!20,draw=none](-7.554,4.457)--(-7.56,4.443)--(-7.576,4.45)--(-7.568,4.468)--cycle;
\draw(-7.554,4.457)--(-7.56,4.443)--(-7.576,4.45)--(-7.568,4.468);
\filldraw[fill opacity=0.8,fill=gray!20,draw=none](-7.569,4.467)--(-7.557,4.474)--(-7.547,4.467)--(-7.563,4.461)--cycle;
\draw(-7.547,4.467)--(-7.563,4.461)--(-7.569,4.467)--(-7.557,4.474);
\filldraw[fill opacity=0.8,fill=gray!20,draw=none](-7.551,4.464)--(-7.561,4.462)--(-7.547,4.467)--cycle;
\draw(-7.561,4.462)--(-7.547,4.467);
\filldraw[fill opacity=0.8,fill=gray!20,draw=none](-7.524,4.525)--(-7.554,4.457)--(-7.568,4.468)--(-7.523,4.571)--cycle;
\draw(-7.524,4.525)--(-7.554,4.457);
\draw(-7.568,4.468)--(-7.523,4.571);
\filldraw[fill opacity=0.8,fill=gray!20,draw=none](-7.509,4.564)--(-7.509,4.56)--(-7.524,4.525)--(-7.523,4.571)--(-7.52,4.58)--cycle;
\draw(-7.509,4.56)--(-7.524,4.525);
\draw(-7.523,4.571)--(-7.52,4.58);
\filldraw[fill opacity=0.8,fill=gray!20,draw=none](-7.518,4.569)--(-7.556,4.466)--(-7.571,4.473)--(-7.529,4.584)--cycle;
\draw(-7.518,4.569)--(-7.556,4.466)--(-7.571,4.473)--(-7.529,4.584);
\filldraw[fill opacity=0.8,fill=gray!20,draw=none](-7.542,4.562)--(-7.531,4.585)--(-7.522,4.572)--cycle;
\draw(-7.542,4.562)--(-7.531,4.585);
\filldraw[fill opacity=0.8,fill=gray!20,draw=none](-7.536,4.588)--(-7.522,4.572)--(-7.535,4.566)--(-7.537,4.588)--cycle;
\draw(-7.535,4.566)--(-7.537,4.588);
\filldraw[fill opacity=0.8,fill=gray!20,draw=none](-7.537,4.586)--(-7.535,4.566)--(-7.58,4.544)--cycle;
\draw(-7.537,4.586)--(-7.535,4.566);
\filldraw[fill opacity=0.8,fill=gray!20,draw=none](-7.533,4.584)--(-7.55,4.559)--(-7.58,4.544)--(-7.558,4.576)--(-7.535,4.587)--cycle;
\draw(-7.533,4.584)--(-7.55,4.559)--(-7.58,4.544);
\draw(-7.558,4.576)--(-7.535,4.587);
\filldraw[fill opacity=0.8,fill=gray!20,draw=none](-7.546,4.589)--(-7.537,4.588)--(-7.537,4.586)--(-7.58,4.544)--(-7.583,4.573)--cycle;
\draw(-7.537,4.588)--(-7.537,4.586);
\draw(-7.58,4.544)--(-7.583,4.573)--(-7.546,4.589);
\filldraw[fill opacity=0.8,fill=gray!20,draw=none](-7.596,4.537)--(-7.657,4.508)--(-7.658,4.508)--(-7.649,4.515)--(-7.582,4.565)--cycle;
\draw(-7.596,4.537)--(-7.657,4.508);
\filldraw[fill opacity=0.8,fill=gray!20,draw=none](-7.583,4.573)--(-7.58,4.544)--(-7.629,4.521)--(-7.634,4.568)--cycle;
\draw(-7.629,4.521)--(-7.634,4.568)--(-7.583,4.573)--(-7.58,4.544);
\filldraw[fill opacity=0.8,fill=gray!20,draw=none](-7.621,4.551)--(-7.618,4.549)--(-7.581,4.544)--(-7.55,4.559)--(-7.533,4.584)--(-7.612,4.621)--cycle;
\draw(-7.621,4.551)--(-7.618,4.549)--(-7.581,4.544)--(-7.55,4.559)--(-7.533,4.584);
\filldraw[fill opacity=0.8,fill=gray!20,draw=none](-7.657,4.507)--(-7.651,4.505)--(-7.663,4.505)--cycle;
\draw(-7.651,4.505)--(-7.663,4.505);
\filldraw[fill opacity=0.8,fill=gray!20,draw=none](-7.616,4.488)--(-7.637,4.493)--(-7.648,4.505)--cycle;
\filldraw[fill opacity=0.8,fill=gray!20,draw=none](-7.633,4.489)--(-7.65,4.5)--(-7.648,4.505)--cycle;
\draw(-7.65,4.5)--(-7.648,4.505);
\filldraw[fill opacity=0.8,fill=gray!20,draw=none](-7.658,4.495)--(-7.665,4.499)--(-7.668,4.502)--(-7.663,4.505)--(-7.651,4.505)--(-7.648,4.505)--(-7.617,4.489)--cycle;
\draw(-7.665,4.499)--(-7.668,4.502);
\draw(-7.663,4.505)--(-7.651,4.505);
\filldraw[fill opacity=0.8,fill=gray!20,draw=none](-7.682,4.496)--(-7.673,4.5)--(-7.658,4.494)--(-7.662,4.487)--cycle;
\draw(-7.658,4.494)--(-7.662,4.487);
\filldraw[fill opacity=0.8,fill=gray!20,draw=none](-7.669,4.502)--(-7.668,4.502)--(-7.665,4.499)--cycle;
\draw(-7.668,4.502)--(-7.665,4.499);
\filldraw[fill opacity=0.8,fill=gray!20,draw=none](-7.648,4.505)--(-7.651,4.505)--(-7.649,4.505)--cycle;
\draw(-7.651,4.505)--(-7.649,4.505);
\filldraw[fill opacity=0.8,fill=gray!20,draw=none](-7.662,4.487)--(-7.658,4.495)--(-7.617,4.475)--(-7.62,4.469)--cycle;
\draw(-7.662,4.487)--(-7.658,4.495);
\draw(-7.617,4.475)--(-7.62,4.469);
\filldraw[fill opacity=0.8,fill=gray!20,draw=none](-7.682,4.496)--(-7.669,4.502)--(-7.665,4.499)--(-7.654,4.484)--cycle;
\draw(-7.665,4.499)--(-7.654,4.484);
\filldraw[fill opacity=0.8,fill=gray!20,draw=none](-7.658,4.495)--(-7.663,4.496)--(-7.665,4.499)--cycle;
\draw(-7.663,4.496)--(-7.665,4.499);
\filldraw[fill opacity=0.8,fill=gray!20,draw=none](-7.663,4.496)--(-7.673,4.5)--(-7.657,4.508)--(-7.653,4.506)--(-7.658,4.495)--cycle;
\draw(-7.653,4.506)--(-7.658,4.495);
\filldraw[fill opacity=0.8,fill=gray!20,draw=none](-7.63,4.473)--(-7.654,4.484)--(-7.663,4.496)--(-7.617,4.489)--(-7.615,4.488)--(-7.617,4.474)--cycle;
\draw(-7.654,4.484)--(-7.663,4.496);
\draw(-7.615,4.488)--(-7.617,4.474)--(-7.63,4.473);
\filldraw[fill opacity=0.8,fill=gray!20,draw=none](-7.658,4.495)--(-7.653,4.506)--(-7.611,4.489)--(-7.617,4.475)--cycle;
\draw(-7.658,4.495)--(-7.653,4.506);
\draw(-7.611,4.489)--(-7.617,4.475);
\filldraw[fill opacity=0.8,fill=gray!20,draw=none](-7.612,4.474)--(-7.617,4.474)--(-7.617,4.475)--cycle;
\draw(-7.612,4.474)--(-7.617,4.474)--(-7.617,4.475);
\filldraw[fill opacity=0.8,fill=gray!20,draw=none](-7.611,4.489)--(-7.587,4.48)--(-7.571,4.473)--(-7.563,4.47)--cycle;
\draw(-7.587,4.48)--(-7.571,4.473)--(-7.563,4.47);
\filldraw[fill opacity=0.8,fill=gray!20,draw=none](-7.58,4.477)--(-7.542,4.562)--(-7.528,4.569)--(-7.53,4.55)--(-7.566,4.47)--cycle;
\draw(-7.58,4.477)--(-7.542,4.562);
\draw(-7.53,4.55)--(-7.566,4.47);
\filldraw[fill opacity=0.8,fill=gray!20,draw=none](-7.528,4.569)--(-7.522,4.572)--(-7.524,4.563)--(-7.53,4.55)--cycle;
\draw(-7.524,4.563)--(-7.53,4.55);
\filldraw[fill opacity=0.8,fill=gray!20,draw=none](-7.522,4.572)--(-7.524,4.563)--(-7.533,4.544)--(-7.535,4.566)--cycle;
\draw(-7.533,4.544)--(-7.535,4.566);
\filldraw[fill opacity=0.8,fill=gray!20,draw=none](-7.524,4.563)--(-7.522,4.572)--(-7.52,4.571)--cycle;
\filldraw[fill opacity=0.8,fill=gray!20,draw=none](-4.465,3.051)--(-7.622,4.631)--(-7.603,4.612)--(-4.457,3.037)--cycle;
\draw(-4.465,3.051)--(-7.622,4.631)--(-7.603,4.612)--(-4.457,3.037);
\filldraw[fill opacity=0.8,fill=gray!20,draw=none](-4.465,3.051)--(-4.457,3.037)--(-4.441,3.029)--cycle;
\draw(-4.457,3.037)--(-4.441,3.029);
\filldraw[fill opacity=0.8,fill=gray!20,draw=none](-8.071,.641)--(-8.035,.632)--(-8.06,.658)--(-8.061,.658)--cycle;
\draw(-8.071,.641)--(-8.035,.632)--(-8.06,.658)--(-8.061,.658);
\filldraw[fill opacity=0.8,fill=gray!20,draw=none](-4.27,2.992)--(-4.232,3.002)--(-4.192,3.058)--(-4.25,3.045)--(-4.282,3)--cycle;
\draw(-4.232,3.002)--(-4.192,3.058);
\draw(-4.25,3.045)--(-4.282,3);
\filldraw[fill opacity=0.8,fill=gray!20,draw=none](-4.636,2.667)--(-4.645,2.654)--(-4.654,2.686)--cycle;
\draw(-4.636,2.667)--(-4.645,2.654);
\filldraw[fill opacity=0.8,fill=gray!20,draw=none](-4.388,2.683)--(-4.389,2.683)--(-4.39,2.682)--cycle;
\draw(-4.388,2.683)--(-4.389,2.683)--(-4.39,2.682);
\filldraw[fill opacity=0.8,fill=gray!20,draw=none](-8.149,.931)--(-8.164,.947)--(-8.169,.936)--cycle;
\draw(-8.149,.931)--(-8.164,.947)--(-8.169,.936);
\filldraw[fill opacity=0.8,fill=gray!20](-2.894,7.927)--(-2.876,7.976)--(-2.802,7.991)--(-2.812,7.943)--cycle;
\filldraw[fill opacity=0.8,fill=gray!20,draw=none](-4.519,3.083)--(-4.516,3.076)--(-4.491,3.09)--cycle;
\draw(-4.516,3.076)--(-4.491,3.09);
\filldraw[fill opacity=0.8,fill=gray!20,draw=none](-8.067,1.422)--(-8.111,1.417)--(-8.084,1.411)--cycle;
\filldraw[fill opacity=0.8,fill=gray!20,draw=none](-8.39,1.295)--(-8.377,1.298)--(-8.374,1.297)--(-8.378,1.291)--cycle;
\draw(-8.377,1.298)--(-8.374,1.297)--(-8.378,1.291);
\filldraw[fill opacity=0.8,fill=gray!20,draw=none](-8.377,1.298)--(-8.373,1.298)--(-8.374,1.297)--cycle;
\draw(-8.373,1.298)--(-8.374,1.297)--(-8.377,1.298);
\filldraw[fill opacity=0.8,fill=gray!20,draw=none](-8.357,1.293)--(-8.374,1.297)--(-8.373,1.298)--(-8.346,1.3)--cycle;
\draw(-8.357,1.293)--(-8.374,1.297)--(-8.373,1.298);
\filldraw[fill opacity=0.8,fill=gray!20,draw=none](-8.386,1.276)--(-8.374,1.297)--(-8.357,1.293)--cycle;
\draw(-8.386,1.276)--(-8.374,1.297)--(-8.357,1.293);
\filldraw[fill opacity=0.8,fill=gray!20,draw=none](-8.443,1.272)--(-8.434,1.274)--(-8.433,1.273)--(-8.442,1.271)--cycle;
\draw(-8.434,1.274)--(-8.433,1.273)--(-8.442,1.271);
\filldraw[fill opacity=0.8,fill=gray!20,draw=none](-8.425,1.274)--(-8.425,1.273)--(-8.433,1.273)--(-8.434,1.274)--cycle;
\draw(-8.425,1.273)--(-8.433,1.273)--(-8.434,1.274);
\filldraw[fill opacity=0.8,fill=gray!20](-7.991,1.455)--(-8.4,1.276)--(-8.445,1.271)--(-8.036,1.45)--cycle;
\filldraw[fill opacity=0.8,fill=gray!20,draw=none](-4.331,2.914)--(-4.279,2.937)--(-4.259,2.964)--(-4.324,2.942)--(-4.336,2.924)--cycle;
\draw(-4.279,2.937)--(-4.259,2.964);
\draw(-4.324,2.942)--(-4.336,2.924);
\filldraw[fill opacity=0.8,fill=gray!20,draw=none](-4.252,2.926)--(-4.248,2.971)--(-4.27,2.981)--(-4.279,2.974)--(-4.261,2.92)--cycle;
\draw(-4.27,2.981)--(-4.279,2.974)--(-4.261,2.92)--(-4.252,2.926);
\filldraw[fill opacity=0.8,fill=gray!20,draw=none](-4.465,2.686)--(-4.48,2.685)--(-4.473,2.693)--cycle;
\draw(-4.48,2.685)--(-4.473,2.693);
\filldraw[fill opacity=0.8,fill=gray!20,draw=none](-4.404,2.705)--(-4.402,2.682)--(-4.389,2.683)--(-4.368,2.708)--(-4.402,2.707)--cycle;
\draw(-4.402,2.682)--(-4.389,2.683)--(-4.368,2.708)--(-4.402,2.707);
\filldraw[fill opacity=0.8,fill=gray!20,draw=none](-4.245,2.819)--(-4.256,2.855)--(-4.261,2.809)--cycle;
\draw(-4.256,2.855)--(-4.261,2.809)--(-4.245,2.819);
\filldraw[fill opacity=0.8,fill=gray!20,draw=none](-4.357,2.698)--(-4.328,2.716)--(-4.368,2.708)--(-4.386,2.686)--cycle;
\draw(-4.328,2.716)--(-4.368,2.708)--(-4.386,2.686);
\filldraw[fill opacity=0.8,fill=gray!20,draw=none](-4.478,2.687)--(-4.446,2.692)--(-4.447,2.705)--(-4.523,2.71)--(-4.503,2.689)--cycle;
\draw(-4.446,2.692)--(-4.447,2.705)--(-4.523,2.71)--(-4.503,2.689);
\filldraw[fill opacity=0.8,fill=gray!20,draw=none](-4.478,2.687)--(-4.483,2.682)--(-4.503,2.689)--cycle;
\draw(-4.478,2.687)--(-4.483,2.682);
\filldraw[fill opacity=0.8,fill=gray!20](-8.145,.926)--(-8.113,.974)--(-8.129,.991)--(-8.164,.947)--cycle;
\filldraw[fill opacity=0.8,fill=gray!20](-7.755,.835)--(-7.763,.89)--(-7.797,.867)--(-7.791,.811)--cycle;
\filldraw[fill opacity=0.8,fill=gray!20,draw=none](-4.397,3.089)--(-4.41,3.096)--(-4.425,3.1)--(-4.422,3.093)--(-4.412,3.086)--cycle;
\draw(-4.41,3.096)--(-4.425,3.1);
\draw(-4.422,3.093)--(-4.412,3.086)--(-4.397,3.089);
\filldraw[fill opacity=0.8,fill=gray!20,draw=none](-4.404,2.705)--(-4.427,2.681)--(-4.402,2.682)--cycle;
\draw(-4.427,2.681)--(-4.402,2.682);
\filldraw[fill opacity=0.8,fill=gray!20,draw=none](-4.498,2.983)--(-4.496,3.002)--(-4.502,3)--(-4.516,2.979)--cycle;
\draw(-4.502,3)--(-4.516,2.979);
\filldraw[fill opacity=0.8,fill=gray!20](-7.881,1.425)--(-7.846,1.469)--(-7.877,1.449)--(-7.906,1.408)--cycle;
\filldraw[fill opacity=0.8,fill=gray!20,draw=none](-4.592,2.747)--(-4.628,2.725)--(-4.605,2.686)--(-4.552,2.719)--cycle;
\draw(-4.592,2.747)--(-4.628,2.725);
\draw(-4.605,2.686)--(-4.552,2.719);
\filldraw[fill opacity=0.8,fill=gray!20](-8.031,1.045)--(-7.974,1.052)--(-7.974,1.052)--(-8.022,1.051)--cycle;
\filldraw[fill opacity=0.8,fill=gray!20,draw=none](-4.25,2.977)--(-4.232,3.002)--(-4.27,2.992)--cycle;
\draw(-4.25,2.977)--(-4.232,3.002);
\filldraw[fill opacity=0.8,fill=gray!20,draw=none](-4.343,2.864)--(-4.343,2.869)--(-4.345,2.871)--(-4.352,2.87)--cycle;
\draw(-4.345,2.871)--(-4.352,2.87);
\filldraw[fill opacity=0.8,fill=gray!20,draw=none](-4.347,2.838)--(-4.338,2.873)--(-4.345,2.867)--(-4.347,2.856)--cycle;
\filldraw[fill opacity=0.8,fill=gray!20,draw=none](-4.343,2.869)--(-4.331,2.853)--(-4.297,2.845)--(-4.302,2.855)--(-4.335,2.873)--(-4.343,2.872)--cycle;
\draw(-4.335,2.873)--(-4.343,2.872);
\filldraw[fill opacity=0.8,fill=gray!20,draw=none](-4.418,2.72)--(-4.418,2.719)--(-4.421,2.718)--cycle;
\draw(-4.418,2.72)--(-4.418,2.719);
\filldraw[fill opacity=0.8,fill=gray!20,draw=none](-4.311,2.737)--(-4.298,2.735)--(-4.279,2.76)--(-4.293,2.757)--cycle;
\draw(-4.298,2.735)--(-4.279,2.76)--(-4.293,2.757);
\filldraw[fill opacity=0.8,fill=gray!20](-7.883,.639)--(-7.845,.667)--(-7.905,.656)--(-7.925,.63)--cycle;
\filldraw[fill opacity=0.8,fill=gray!20,draw=none](-4.383,2.752)--(-4.403,2.732)--(-4.418,2.719)--(-4.418,2.72)--cycle;
\draw(-4.418,2.719)--(-4.418,2.72);
\filldraw[fill opacity=0.8,fill=gray!20](-8.067,1.361)--(-8.096,1.373)--(-8.134,1.382)--(-8.087,1.366)--cycle;
\filldraw[fill opacity=0.8,fill=gray!20,draw=none](-4.488,2.985)--(-4.494,2.986)--(-4.5,2.978)--cycle;
\draw(-4.488,2.985)--(-4.494,2.986)--(-4.5,2.978);
\filldraw[fill opacity=0.8,fill=gray!20,draw=none](-4.488,2.985)--(-4.488,2.982)--(-4.487,2.975)--cycle;
\draw(-4.488,2.985)--(-4.488,2.982)--(-4.487,2.975);
\filldraw[fill opacity=0.8,fill=gray!20,draw=none](-4.498,2.983)--(-4.498,2.978)--(-4.477,2.977)--(-4.469,2.99)--cycle;
\draw(-4.477,2.977)--(-4.469,2.99);
\filldraw[fill opacity=0.8,fill=gray!20,draw=none](-4.576,2.909)--(-4.517,2.951)--(-4.5,2.978)--cycle;
\draw(-4.517,2.951)--(-4.5,2.978);
\filldraw[fill opacity=0.8,fill=gray!20,draw=none](-4.498,2.983)--(-4.488,2.986)--(-4.484,2.99)--cycle;
\draw(-4.488,2.986)--(-4.484,2.99);
\filldraw[fill opacity=0.8,fill=gray!20,draw=none](-4.498,2.983)--(-4.469,2.99)--(-4.449,3.017)--(-4.496,3.002)--cycle;
\draw(-4.469,2.99)--(-4.449,3.017);
\filldraw[fill opacity=0.8,fill=gray!20,draw=none](-4.587,2.697)--(-4.598,2.685)--(-4.614,2.701)--cycle;
\draw(-4.587,2.697)--(-4.598,2.685);
\filldraw[fill opacity=0.8,fill=gray!20,draw=none](-4.587,2.697)--(-4.614,2.701)--(-4.625,2.713)--(-4.623,2.716)--cycle;
\filldraw[fill opacity=0.8,fill=gray!20,draw=none](-4.438,2.695)--(-4.413,2.704)--(-4.413,2.706)--(-4.425,2.706)--cycle;
\draw(-4.413,2.706)--(-4.425,2.706);
\filldraw[fill opacity=0.8,fill=gray!20,draw=none](-4.379,2.744)--(-4.38,2.741)--(-4.379,2.741)--cycle;
\draw(-4.38,2.741)--(-4.379,2.741);
\filldraw[fill opacity=0.8,fill=gray!20,draw=none](-4.383,2.752)--(-4.38,2.755)--(-4.389,2.744)--(-4.403,2.732)--cycle;
\draw(-4.38,2.755)--(-4.389,2.744);
\filldraw[fill opacity=0.8,fill=gray!20,draw=none](-4.379,2.756)--(-4.38,2.755)--(-4.383,2.752)--cycle;
\draw(-4.379,2.756)--(-4.38,2.755);
\filldraw[fill opacity=0.8,fill=gray!20,draw=none](-4.352,2.808)--(-4.368,2.783)--(-4.38,2.755)--(-4.379,2.756)--cycle;
\draw(-4.38,2.755)--(-4.379,2.756);
\filldraw[fill opacity=0.8,fill=gray!20,draw=none](-4.408,2.721)--(-4.425,2.706)--(-4.413,2.706)--cycle;
\draw(-4.425,2.706)--(-4.413,2.706);
\filldraw[fill opacity=0.8,fill=gray!20,draw=none](-4.389,2.744)--(-4.43,2.695)--(-4.445,2.687)--(-4.418,2.719)--cycle;
\draw(-4.389,2.744)--(-4.43,2.695);
\draw(-4.445,2.687)--(-4.418,2.719);
\filldraw[fill opacity=0.8,fill=gray!20,draw=none](-4.654,2.686)--(-4.645,2.654)--(-4.648,2.648)--(-4.668,2.7)--cycle;
\draw(-4.645,2.654)--(-4.648,2.648);
\filldraw[fill opacity=0.8,fill=gray!20,draw=none](-4.426,2.683)--(-4.426,2.682)--(-4.429,2.681)--cycle;
\draw(-4.426,2.683)--(-4.426,2.682);
\filldraw[fill opacity=0.8,fill=gray!20,draw=none](-8.194,1.646)--(-8.173,1.659)--(-8.183,1.662)--cycle;
\draw(-8.173,1.659)--(-8.183,1.662);
\filldraw[fill opacity=0.8,fill=gray!20,draw=none](-8.178,1.668)--(-8.183,1.662)--(-8.173,1.659)--(-8.156,1.669)--cycle;
\draw(-8.183,1.662)--(-8.173,1.659);
\filldraw[fill opacity=0.8,fill=gray!20,draw=none](-8.465,1.543)--(-8.475,1.523)--(-8.477,1.522)--(-8.516,1.52)--(-8.51,1.534)--cycle;
\draw(-8.477,1.522)--(-8.516,1.52)--(-8.51,1.534);
\filldraw[fill opacity=0.8,fill=gray!20,draw=none](-8.475,1.523)--(-8.465,1.543)--(-8.437,1.548)--(-8.437,1.546)--cycle;
\draw(-8.437,1.548)--(-8.437,1.546);
\filldraw[fill opacity=0.8,fill=gray!20,draw=none](-8.518,1.504)--(-8.524,1.517)--(-8.486,1.525)--cycle;
\draw(-8.524,1.517)--(-8.486,1.525);
\filldraw[fill opacity=0.8,fill=gray!20,draw=none](-8.119,1.679)--(-8.461,1.529)--(-8.507,1.513)--(-8.49,1.524)--(-8.081,1.702)--cycle;
\draw(-8.507,1.513)--(-8.49,1.524)--(-8.081,1.702)--(-8.119,1.679)--(-8.461,1.529);
\filldraw[fill opacity=0.8,fill=gray!20,draw=none](-4.683,2.875)--(-4.685,2.876)--(-4.684,2.877)--cycle;
\draw(-4.683,2.875)--(-4.685,2.876);
\filldraw[fill opacity=0.8,fill=gray!20,draw=none](-4.253,2.973)--(-4.25,2.977)--(-4.262,2.986)--cycle;
\draw(-4.253,2.973)--(-4.25,2.977);
\filldraw[fill opacity=0.8,fill=gray!20,draw=none](-4.252,2.973)--(-4.262,2.986)--(-4.27,2.981)--cycle;
\draw(-4.262,2.986)--(-4.27,2.981);
\filldraw[fill opacity=0.8,fill=gray!20](-7.924,1.05)--(-7.974,1.052)--(-7.974,1.052)--(-7.918,1.044)--cycle;
\filldraw[fill opacity=0.8,fill=gray!20,draw=none](-4.68,2.872)--(-4.685,2.876)--(-4.683,2.875)--cycle;
\draw(-4.685,2.876)--(-4.683,2.875);
\filldraw[fill opacity=0.8,fill=gray!20,draw=none](-4.55,2.72)--(-4.552,2.719)--(-4.549,2.717)--cycle;
\draw(-4.55,2.72)--(-4.552,2.719);
\filldraw[fill opacity=0.8,fill=gray!20,draw=none](-4.404,2.705)--(-4.438,2.653)--(-4.446,2.652)--(-4.426,2.683)--cycle;
\draw(-4.404,2.705)--(-4.438,2.653);
\draw(-4.446,2.652)--(-4.426,2.683);
\filldraw[fill opacity=0.8,fill=gray!20](-8.145,1.766)--(-8.093,1.786)--(-8.083,1.792)--(-8.127,1.777)--cycle;
\filldraw[fill opacity=0.8,fill=gray!20](-2.802,7.991)--(-2.786,8.028)--(-2.708,8.032)--(-2.705,7.995)--cycle;
\filldraw[fill opacity=0.8,fill=gray!20](-2.705,7.995)--(-2.708,8.032)--(-2.631,8.026)--(-2.612,7.988)--cycle;
\filldraw[fill opacity=0.8,fill=gray!20,draw=none](-7.824,.992)--(-7.823,.989)--(-7.819,.987)--cycle;
\draw(-7.823,.989)--(-7.819,.987)--(-7.824,.992);
\filldraw[fill opacity=0.8,fill=gray!20,draw=none](-4.465,2.686)--(-4.455,2.676)--(-4.535,2.58)--(-4.524,2.632)--(-4.48,2.685)--cycle;
\draw(-4.455,2.676)--(-4.535,2.58);
\draw(-4.524,2.632)--(-4.48,2.685);
\filldraw[fill opacity=0.8,fill=gray!20,draw=none](-4.68,2.872)--(-4.651,2.849)--(-4.684,2.866)--(-4.7,2.884)--(-4.685,2.876)--cycle;
\draw(-4.651,2.849)--(-4.684,2.866);
\draw(-4.7,2.884)--(-4.685,2.876);
\filldraw[fill opacity=0.8,fill=gray!20](-2.6,7.941)--(-2.612,7.988)--(-2.547,7.972)--(-2.527,7.923)--cycle;
\filldraw[fill opacity=0.8,fill=gray!20,draw=none](-4.328,2.716)--(-4.293,2.757)--(-4.353,2.746)--(-4.368,2.708)--cycle;
\draw(-4.293,2.757)--(-4.353,2.746)--(-4.368,2.708)--(-4.328,2.716);
\filldraw[fill opacity=0.8,fill=gray!20](-7.988,1.365)--(-7.944,1.38)--(-7.987,1.372)--(-8.01,1.36)--cycle;
\filldraw[fill opacity=0.8,fill=gray!20](-8.113,.974)--(-8.073,1.012)--(-8.084,1.024)--(-8.129,.991)--cycle;
\filldraw[fill opacity=0.8,fill=gray!20](-7.763,.89)--(-7.785,.942)--(-7.816,.922)--(-7.797,.867)--cycle;
\filldraw[fill opacity=0.8,fill=gray!20,draw=none](-4.302,2.855)--(-4.315,2.877)--(-4.335,2.873)--cycle;
\draw(-4.315,2.877)--(-4.335,2.873);
\filldraw[fill opacity=0.8,fill=gray!20,draw=none](-4.634,2.841)--(-4.651,2.849)--(-4.68,2.872)--cycle;
\draw(-4.634,2.841)--(-4.651,2.849);
\filldraw[fill opacity=0.8,fill=gray!20,draw=none](-4.331,2.914)--(-4.325,2.903)--(-4.279,2.937)--cycle;
\filldraw[fill opacity=0.8,fill=gray!20,draw=none](-4.506,2.979)--(-4.539,2.964)--(-4.551,2.95)--cycle;
\draw(-4.539,2.964)--(-4.551,2.95);
\filldraw[fill opacity=0.8,fill=gray!20,draw=none](-4.556,2.944)--(-4.552,2.948)--(-4.539,2.964)--cycle;
\draw(-4.552,2.948)--(-4.539,2.964);
\filldraw[fill opacity=0.8,fill=gray!20,draw=none](-4.601,2.9)--(-4.556,2.944)--(-4.539,2.964)--cycle;
\filldraw[fill opacity=0.8,fill=gray!20,draw=none](-4.576,2.757)--(-4.592,2.747)--(-4.552,2.719)--(-4.55,2.72)--cycle;
\draw(-4.576,2.757)--(-4.592,2.747);
\draw(-4.552,2.719)--(-4.55,2.72);
\filldraw[fill opacity=0.8,fill=gray!20](-7.846,1.469)--(-7.824,1.52)--(-7.859,1.498)--(-7.877,1.449)--cycle;
\filldraw[fill opacity=0.8,fill=gray!20,draw=none](-4.484,3.071)--(-4.469,3.087)--(-4.489,3.092)--(-4.516,3.076)--cycle;
\draw(-4.484,3.071)--(-4.469,3.087)--(-4.489,3.092)--(-4.516,3.076);
\filldraw[fill opacity=0.8,fill=gray!20,draw=none](-4.637,2.888)--(-4.635,2.869)--(-4.634,2.869)--(-4.63,2.902)--(-4.636,2.916)--cycle;
\draw(-4.635,2.869)--(-4.634,2.869)--(-4.63,2.902);
\filldraw[fill opacity=0.8,fill=gray!20,draw=none](-8.17,.871)--(-8.178,.823)--(-8.171,.816)--(-8.164,.872)--(-8.168,.876)--cycle;
\draw(-8.178,.823)--(-8.171,.816)--(-8.164,.872)--(-8.168,.876);
\filldraw[fill opacity=0.8,fill=gray!20,draw=none](-8.168,.876)--(-8.164,.872)--(-8.153,.902)--cycle;
\draw(-8.168,.876)--(-8.164,.872)--(-8.153,.902);
\filldraw[fill opacity=0.8,fill=gray!20,draw=none](-8.014,.905)--(-8.027,.92)--(-8.07,.934)--(-8.079,.91)--cycle;
\draw(-8.07,.934)--(-8.079,.91)--(-8.014,.905);
\filldraw[fill opacity=0.8,fill=gray!20,draw=none](-7.987,.847)--(-7.987,.868)--(-7.993,.886)--(-8.014,.905)--(-8.079,.91)--(-8.091,.854)--cycle;
\draw(-8.014,.905)--(-8.079,.91)--(-8.091,.854)--(-7.987,.847)--(-7.987,.868);
\filldraw[fill opacity=0.8,fill=gray!20](-8.434,.77)--(-8.436,.81)--(-8.358,.804)--(-8.348,.764)--cycle;
\filldraw[fill opacity=0.8,fill=gray!20,draw=none](-8.448,.812)--(-8.454,.81)--(-8.436,.82)--cycle;
\draw(-8.448,.812)--(-8.454,.81);
\filldraw[fill opacity=0.8,fill=gray!20,draw=none](-8.454,.809)--(-8.446,.829)--(-8.437,.831)--(-8.436,.81)--cycle;
\draw(-8.437,.831)--(-8.436,.81)--(-8.454,.809);
\filldraw[fill opacity=0.8,fill=gray!20,draw=none](-8.436,.82)--(-8.437,.831)--(-8.419,.833)--cycle;
\draw(-8.436,.82)--(-8.437,.831);
\filldraw[fill opacity=0.8,fill=gray!20,draw=none](-8.436,.82)--(-8.419,.833)--(-8.411,.835)--cycle;
\draw(-8.419,.833)--(-8.411,.835);
\filldraw[fill opacity=0.8,fill=gray!20,draw=none](-8.436,.81)--(-8.436,.82)--(-8.419,.833)--(-8.384,.836)--(-8.374,.836)--(-8.358,.804)--cycle;
\draw(-8.384,.836)--(-8.374,.836)--(-8.358,.804)--(-8.436,.81)--(-8.436,.82);
\filldraw[fill opacity=0.8,fill=gray!20,draw=none](-8.454,.63)--(-8.48,.673)--(-8.48,.676)--(-8.434,.678)--(-8.436,.631)--cycle;
\draw(-8.48,.676)--(-8.434,.678)--(-8.436,.631)--(-8.454,.63);
\filldraw[fill opacity=0.8,fill=gray!20,draw=none](-8.48,.676)--(-8.488,.723)--(-8.434,.726)--(-8.434,.678)--cycle;
\draw(-8.488,.723)--(-8.434,.726)--(-8.434,.678)--(-8.48,.676);
\filldraw[fill opacity=0.8,fill=gray!20](-8.436,.631)--(-8.434,.678)--(-8.348,.672)--(-8.358,.625)--cycle;
\filldraw[fill opacity=0.8,fill=gray!20](-8.434,.678)--(-8.434,.726)--(-8.344,.719)--(-8.348,.672)--cycle;
\filldraw[fill opacity=0.8,fill=gray!20,draw=none](-8.488,.723)--(-8.48,.768)--(-8.434,.77)--(-8.434,.726)--cycle;
\draw(-8.48,.768)--(-8.434,.77)--(-8.434,.726)--(-8.488,.723);
\filldraw[fill opacity=0.8,fill=gray!20](-8.434,.726)--(-8.434,.77)--(-8.348,.764)--(-8.344,.719)--cycle;
\filldraw[fill opacity=0.8,fill=gray!20,draw=none](-8.384,.584)--(-8.419,.595)--(-8.436,.616)--(-8.436,.631)--(-8.358,.625)--(-8.374,.583)--cycle;
\draw(-8.436,.616)--(-8.436,.631)--(-8.358,.625)--(-8.374,.583)--(-8.384,.584);
\filldraw[fill opacity=0.8,fill=gray!20,draw=none](-8.454,.628)--(-8.454,.63)--(-8.436,.631)--(-8.436,.616)--cycle;
\draw(-8.454,.63)--(-8.436,.631)--(-8.436,.616);
\filldraw[fill opacity=0.8,fill=gray!20,draw=none](-8.48,.768)--(-8.48,.771)--(-8.454,.809)--(-8.436,.81)--(-8.434,.77)--cycle;
\draw(-8.454,.809)--(-8.436,.81)--(-8.434,.77)--(-8.48,.768);
\filldraw[fill opacity=0.8,fill=gray!20](-8.521,.542)--(-8.553,.575)--(-8.503,.585)--(-8.486,.549)--cycle;
\filldraw[fill opacity=0.8,fill=gray!20,draw=none](-8.52,.607)--(-8.522,.614)--(-8.522,.626)--(-8.519,.663)--(-8.504,.711)--(-8.48,.756)--(-8.45,.79)--(-8.42,.808)--(-8.394,.807)--(-8.376,.789)--(-8.374,.781)--(-8.37,.765)--cycle;
\draw(-8.52,.607)--(-8.522,.614);
\draw(-8.522,.626)--(-8.519,.663)--(-8.504,.711)--(-8.48,.756)--(-8.45,.79)--(-8.42,.808)--(-8.394,.807)--(-8.376,.789);
\draw(-8.374,.781)--(-8.37,.765);
\filldraw[fill opacity=0.8,fill=gray!20,draw=none](-8.358,.804)--(-8.374,.836)--(-8.365,.833)--(-8.335,.82)--(-8.314,.805)--(-8.303,.79)--cycle;
\draw(-8.314,.805)--(-8.303,.79)--(-8.358,.804)--(-8.374,.836)--(-8.365,.833);
\filldraw[fill opacity=0.8,fill=gray!20,draw=none](-8.436,.82)--(-8.454,.81)--(-8.499,.796)--(-8.453,.822)--(-8.419,.833)--cycle;
\draw(-8.454,.81)--(-8.499,.796);
\draw(-8.453,.822)--(-8.419,.833);
\filldraw[fill opacity=0.8,fill=gray!20,draw=none](-8.492,.807)--(-8.45,.79)--(-8.48,.756)--(-8.513,.77)--cycle;
\draw(-8.492,.807)--(-8.45,.79)--(-8.48,.756)--(-8.513,.77);
\filldraw[fill opacity=0.8,fill=gray!20,draw=none](-8.492,.814)--(-8.465,.818)--(-8.484,.804)--(-8.492,.807)--cycle;
\draw(-8.484,.804)--(-8.492,.807);
\filldraw[fill opacity=0.8,fill=gray!20,draw=none](-8.484,.804)--(-8.465,.818)--(-8.453,.822)--cycle;
\draw(-8.465,.818)--(-8.453,.822);
\filldraw[fill opacity=0.8,fill=gray!20,draw=none](-8.465,.818)--(-8.449,.82)--(-8.42,.808)--(-8.45,.79)--(-8.484,.804)--cycle;
\draw(-8.449,.82)--(-8.42,.808)--(-8.45,.79)--(-8.484,.804);
\filldraw[fill opacity=0.8,fill=gray!20,draw=none](-8.384,.836)--(-8.375,.837)--(-8.374,.836)--cycle;
\draw(-8.375,.837)--(-8.374,.836)--(-8.384,.836);
\filldraw[fill opacity=0.8,fill=gray!20,draw=none](-8.365,.833)--(-8.374,.836)--(-8.376,.837)--cycle;
\draw(-8.365,.833)--(-8.374,.836)--(-8.376,.837);
\filldraw[fill opacity=0.8,fill=gray!20,draw=none](-8.43,.819)--(-8.411,.814)--(-8.394,.807)--(-8.42,.808)--(-8.449,.82)--cycle;
\draw(-8.411,.814)--(-8.394,.807)--(-8.42,.808)--(-8.449,.82);
\filldraw[fill opacity=0.8,fill=gray!20,draw=none](-8.398,.8)--(-8.376,.789)--(-8.394,.807)--(-8.411,.814)--cycle;
\draw(-8.376,.789)--(-8.394,.807)--(-8.411,.814);
\filldraw[fill opacity=0.8,fill=gray!20](-7.974,.966)--(-8.445,.814)--(-8.403,.798)--(-7.933,.95)--cycle;
\filldraw[fill opacity=0.8,fill=gray!20,draw=none](-4.501,2.949)--(-4.498,2.948)--(-4.489,2.962)--(-4.498,2.978)--cycle;
\draw(-4.501,2.949)--(-4.498,2.948)--(-4.489,2.962);
\filldraw[fill opacity=0.8,fill=gray!20,draw=none](-4.504,2.969)--(-4.508,2.965)--(-4.503,2.968)--cycle;
\draw(-4.508,2.965)--(-4.503,2.968)--(-4.504,2.969);
\filldraw[fill opacity=0.8,fill=gray!20,draw=none](-4.504,2.969)--(-4.503,2.968)--(-4.501,2.961)--cycle;
\draw(-4.504,2.969)--(-4.503,2.968)--(-4.501,2.961);
\filldraw[fill opacity=0.8,fill=gray!20,draw=none](-4.501,2.961)--(-4.503,2.968)--(-4.508,2.965)--cycle;
\draw(-4.501,2.961)--(-4.503,2.968)--(-4.508,2.965);
\filldraw[fill opacity=0.8,fill=gray!20,draw=none](-4.5,2.978)--(-4.511,2.96)--(-4.489,2.962)--cycle;
\draw(-4.5,2.978)--(-4.511,2.96);
\filldraw[fill opacity=0.8,fill=gray!20,draw=none](-4.489,2.962)--(-4.478,2.976)--(-4.48,2.978)--(-4.498,2.978)--cycle;
\draw(-4.489,2.962)--(-4.478,2.976);
\filldraw[fill opacity=0.8,fill=gray!20,draw=none](-4.498,2.983)--(-4.506,2.979)--(-4.551,2.95)--(-4.552,2.948)--(-4.499,2.972)--(-4.488,2.986)--cycle;
\draw(-4.551,2.95)--(-4.552,2.948);
\draw(-4.499,2.972)--(-4.488,2.986);
\filldraw[fill opacity=0.8,fill=gray!20,draw=none](-8.148,.727)--(-8.127,.708)--(-8.079,.696)--(-8.091,.743)--(-8.164,.761)--(-8.153,.734)--cycle;
\draw(-8.127,.708)--(-8.079,.696)--(-8.091,.743)--(-8.164,.761)--(-8.153,.734);
\filldraw[fill opacity=0.8,fill=gray!20,draw=none](-7.974,.736)--(-7.987,.736)--(-7.987,.718)--cycle;
\draw(-7.974,.736)--(-7.987,.736)--(-7.987,.718);
\filldraw[fill opacity=0.8,fill=gray!20,draw=none](-7.974,.736)--(-7.969,.743)--(-7.96,.791)--(-7.988,.79)--(-7.987,.736)--cycle;
\draw(-7.96,.791)--(-7.988,.79)--(-7.987,.736)--(-7.974,.736);
\filldraw[fill opacity=0.8,fill=gray!20](-8.358,.625)--(-8.348,.672)--(-8.287,.657)--(-8.303,.612)--cycle;
\filldraw[fill opacity=0.8,fill=gray!20,draw=none](-8.278,.643)--(-8.299,.607)--(-8.303,.612)--(-8.287,.657)--(-8.277,.646)--cycle;
\draw(-8.299,.607)--(-8.303,.612)--(-8.287,.657)--(-8.277,.646);
\filldraw[fill opacity=0.8,fill=gray!20](-7.874,.773)--(-8.344,.622)--(-8.368,.583)--(-7.897,.735)--cycle;
\filldraw[fill opacity=0.8,fill=gray!20](-8.061,1.797)--(-8.036,1.793)--(-8.036,1.793)--(-8.032,1.798)--cycle;
\filldraw[fill opacity=0.8,fill=gray!20](-8.032,1.798)--(-8.036,1.793)--(-8.036,1.793)--(-8.005,1.796)--cycle;
\filldraw[fill opacity=0.8,fill=gray!20,draw=none](-4.684,2.866)--(-4.749,2.899)--(-4.744,2.906)--(-4.7,2.884)--cycle;
\draw(-4.684,2.866)--(-4.749,2.899);
\draw(-4.744,2.906)--(-4.7,2.884);
\filldraw[fill opacity=0.8,fill=gray!20,draw=none](-4.668,2.7)--(-4.684,2.677)--(-4.673,2.726)--cycle;
\draw(-4.668,2.7)--(-4.684,2.677);
\filldraw[fill opacity=0.8,fill=gray!20,draw=none](-4.668,2.7)--(-4.673,2.726)--(-4.664,2.762)--cycle;
\filldraw[fill opacity=0.8,fill=gray!20,draw=none](-4.364,2.792)--(-4.38,2.771)--(-4.389,2.744)--(-4.38,2.755)--cycle;
\draw(-4.389,2.744)--(-4.38,2.755);
\filldraw[fill opacity=0.8,fill=gray!20,draw=none](-4.379,2.744)--(-4.363,2.79)--(-4.379,2.767)--(-4.388,2.728)--(-4.38,2.741)--cycle;
\draw(-4.363,2.79)--(-4.379,2.767);
\draw(-4.388,2.728)--(-4.38,2.741);
\filldraw[fill opacity=0.8,fill=gray!20,draw=none](-4.353,2.806)--(-4.363,2.79)--(-4.379,2.744)--cycle;
\draw(-4.353,2.806)--(-4.363,2.79);
\filldraw[fill opacity=0.8,fill=gray!20,draw=none](-7.938,1.775)--(-7.948,1.778)--(-7.958,1.776)--(-7.926,1.763)--cycle;
\draw(-7.958,1.776)--(-7.926,1.763)--(-7.938,1.775)--(-7.948,1.778);
\filldraw[fill opacity=0.8,fill=gray!20,draw=none](-4.623,2.716)--(-4.625,2.713)--(-4.634,2.722)--cycle;
\filldraw[fill opacity=0.8,fill=gray!20,draw=none](-4.42,3.086)--(-4.412,3.086)--(-4.422,3.093)--cycle;
\draw(-4.42,3.086)--(-4.412,3.086)--(-4.422,3.093);
\filldraw[fill opacity=0.8,fill=gray!20,draw=none](-4.637,2.872)--(-4.635,2.869)--(-4.637,2.888)--cycle;
\draw(-4.637,2.872)--(-4.635,2.869);
\filldraw[fill opacity=0.8,fill=gray!20](-8.073,1.012)--(-8.025,1.039)--(-8.031,1.045)--(-8.084,1.024)--cycle;
\filldraw[fill opacity=0.8,fill=gray!20](-7.785,.942)--(-7.819,.987)--(-7.845,.97)--(-7.816,.922)--cycle;
\filldraw[fill opacity=0.8,fill=gray!20,draw=none](-8.111,1.417)--(-8.067,1.422)--(-8.053,1.43)--(-8.141,1.437)--(-8.133,1.422)--cycle;
\draw(-8.053,1.43)--(-8.141,1.437)--(-8.133,1.422);
\filldraw[fill opacity=0.8,fill=gray!20,draw=none](-8.053,1.43)--(-8.047,1.434)--(-8.049,1.477)--(-8.153,1.484)--(-8.141,1.437)--cycle;
\draw(-8.047,1.434)--(-8.049,1.477)--(-8.153,1.484)--(-8.141,1.437)--(-8.053,1.43);
\filldraw[fill opacity=0.8,fill=gray!20,draw=none](-8.028,1.446)--(-8.006,1.479)--(-8.049,1.477)--(-8.047,1.434)--cycle;
\draw(-8.006,1.479)--(-8.049,1.477)--(-8.047,1.434);
\filldraw[fill opacity=0.8,fill=gray!20,draw=none](-7.953,1.479)--(-8.246,1.35)--(-8.212,1.358)--(-7.991,1.455)--cycle;
\draw(-8.212,1.358)--(-7.991,1.455)--(-7.953,1.479)--(-8.246,1.35);
\filldraw[fill opacity=0.8,fill=gray!20](-4.279,2.76)--(-4.261,2.809)--(-4.343,2.793)--(-4.353,2.746)--cycle;
\filldraw[fill opacity=0.8,fill=gray!20,draw=none](-4.598,2.685)--(-4.839,2.398)--(-4.625,2.713)--cycle;
\draw(-4.598,2.685)--(-4.839,2.398);
\filldraw[fill opacity=0.8,fill=gray!20](-8.083,1.792)--(-8.036,1.793)--(-8.036,1.793)--(-8.061,1.797)--cycle;
\filldraw[fill opacity=0.8,fill=gray!20,draw=none](-4.507,2.561)--(-4.912,1.955)--(-4.783,2.193)--(-4.503,2.613)--cycle;
\draw(-4.507,2.561)--(-4.912,1.955);
\draw(-4.783,2.193)--(-4.503,2.613);
\filldraw[fill opacity=0.8,fill=gray!20,draw=none](-2.56,7.82)--(-2.565,7.879)--(-2.52,7.868)--(-2.527,7.812)--cycle;
\draw(-2.565,7.879)--(-2.52,7.868)--(-2.527,7.812)--(-2.56,7.82);
\filldraw[fill opacity=0.8,fill=gray!20,draw=none](-4.632,2.846)--(-4.634,2.866)--(-4.635,2.869)--(-4.637,2.872)--cycle;
\draw(-4.632,2.846)--(-4.634,2.866);
\draw(-4.635,2.869)--(-4.637,2.872);
\filldraw[fill opacity=0.8,fill=gray!20,draw=none](-4.488,2.986)--(-4.494,2.978)--(-4.478,2.976)--cycle;
\draw(-4.488,2.986)--(-4.494,2.978);
\filldraw[fill opacity=0.8,fill=gray!20,draw=none](-4.684,2.677)--(-4.878,2.385)--(-4.781,2.588)--(-4.664,2.762)--cycle;
\draw(-4.684,2.677)--(-4.878,2.385);
\draw(-4.781,2.588)--(-4.664,2.762);
\filldraw[fill opacity=0.8,fill=gray!20,draw=none](-4.789,2.6)--(-4.979,2.316)--(-5.042,2.21)--(-4.898,2.427)--cycle;
\draw(-4.789,2.6)--(-4.979,2.316);
\draw(-5.042,2.21)--(-4.898,2.427);
\filldraw[fill opacity=0.8,fill=gray!20,draw=none](-4.922,2.378)--(-4.898,2.427)--(-4.952,2.346)--(-4.985,2.266)--cycle;
\draw(-4.898,2.427)--(-4.952,2.346);
\filldraw[fill opacity=0.8,fill=gray!20,draw=none](-4.979,2.316)--(-5.165,2.038)--(-5.203,1.969)--(-5.042,2.21)--cycle;
\draw(-4.979,2.316)--(-5.165,2.038);
\draw(-5.203,1.969)--(-5.042,2.21);
\filldraw[fill opacity=0.8,fill=gray!20,draw=none](-4.922,2.378)--(-4.985,2.266)--(-5.001,2.227)--(-4.987,2.248)--cycle;
\draw(-5.001,2.227)--(-4.987,2.248);
\filldraw[fill opacity=0.8,fill=gray!20,draw=none](-4.985,2.266)--(-4.952,2.346)--(-5.042,2.21)--(-5.057,2.143)--(-5.036,2.174)--cycle;
\draw(-4.952,2.346)--(-5.042,2.21);
\draw(-5.057,2.143)--(-5.036,2.174);
\filldraw[fill opacity=0.8,fill=gray!20,draw=none](-4.985,2.266)--(-5.036,2.174)--(-5.001,2.227)--cycle;
\draw(-5.036,2.174)--(-5.001,2.227);
\filldraw[fill opacity=0.8,fill=gray!20,draw=none](-5.042,2.21)--(-5.203,1.969)--(-5.177,1.963)--(-5.057,2.143)--cycle;
\draw(-5.042,2.21)--(-5.203,1.969);
\draw(-5.177,1.963)--(-5.057,2.143);
\filldraw[fill opacity=0.8,fill=gray!20,draw=none](-4.987,2.248)--(-5.001,2.227)--(-5.003,2.202)--cycle;
\draw(-4.987,2.248)--(-5.001,2.227);
\filldraw[fill opacity=0.8,fill=gray!20](-6.879,1.013)--(-6.878,1.069)--(-6.773,1.062)--(-6.785,1.006)--cycle;
\filldraw[fill opacity=0.8,fill=gray!20](-6.881,.961)--(-6.879,1.013)--(-6.785,1.006)--(-6.805,.955)--cycle;
\filldraw[fill opacity=0.8,fill=gray!20,draw=none](-6.785,1.006)--(-6.773,1.062)--(-6.722,1.049)--(-6.707,1.027)--(-6.72,.99)--cycle;
\draw(-6.707,1.027)--(-6.72,.99)--(-6.785,1.006)--(-6.773,1.062)--(-6.722,1.049);
\filldraw[fill opacity=0.8,fill=gray!20](-6.805,.955)--(-6.785,1.006)--(-6.72,.99)--(-6.751,.942)--cycle;
\filldraw[fill opacity=0.8,fill=gray!20,draw=none](-6.002,1.055)--(-5.963,1.06)--(-5.948,1.055)--(-5.951,1.037)--(-5.97,1.004)--cycle;
\draw(-5.948,1.055)--(-5.951,1.037)--(-5.97,1.004);
\filldraw[fill opacity=0.8,fill=gray!20,draw=none](-6.126,.965)--(-6.104,1.043)--(-6.002,1.055)--(-5.97,1.004)--(-5.971,1.003)--cycle;
\draw(-5.97,1.004)--(-5.971,1.003);
\filldraw[fill opacity=0.8,fill=gray!20,draw=none](-6.126,.965)--(-5.971,1.003)--(-5.979,.988)--(-6.022,.952)--(-6.072,.936)--(-6.122,.941)--(-6.132,.947)--cycle;
\draw(-5.971,1.003)--(-5.979,.988)--(-6.022,.952)--(-6.072,.936)--(-6.122,.941)--(-6.132,.947);
\filldraw[fill opacity=0.8,fill=gray!20,draw=none](-6.002,1.055)--(-6.016,1.077)--(-5.963,1.06)--cycle;
\filldraw[fill opacity=0.8,fill=gray!20,draw=none](-6.048,1.05)--(-6.044,1.086)--(-6.016,1.077)--(-6.002,1.055)--cycle;
\filldraw[fill opacity=0.8,fill=gray!20,draw=none](-6.002,.942)--(-6.037,1.014)--(-6.003,1.012)--(-6,1.011)--(-6,.943)--cycle;
\draw(-6,1.011)--(-6,.943);
\filldraw[fill opacity=0.8,fill=gray!20,draw=none](-6.056,.929)--(-6.072,.936)--(-6.022,.952)--(-6.01,.947)--cycle;
\draw(-6.056,.929)--(-6.072,.936)--(-6.022,.952)--(-6.01,.947);
\filldraw[fill opacity=0.8,fill=gray!20,draw=none](-6.056,.929)--(-6.01,.947)--(-6,.943)--cycle;
\draw(-6.01,.947)--(-6,.943);
\filldraw[fill opacity=0.8,fill=gray!20,draw=none](-6.016,.971)--(-6.002,.942)--(-6.056,.929)--(-6.056,.966)--cycle;
\draw(-6.056,.929)--(-6.056,.966);
\filldraw[fill opacity=0.8,fill=gray!20](-6.878,1.069)--(-6.877,1.126)--(-6.769,1.119)--(-6.773,1.062)--cycle;
\filldraw[fill opacity=0.8,fill=gray!20,draw=none](-6.032,1.004)--(-6.016,.971)--(-6.056,.966)--(-6.056,.996)--cycle;
\draw(-6.056,.966)--(-6.056,.996);
\filldraw[fill opacity=0.8,fill=gray!20,draw=none](-6.037,1.014)--(-6.043,1.028)--(-6.04,1.034)--(-6.003,1.012)--cycle;
\filldraw[fill opacity=0.8,fill=gray!20,draw=none](-6.694,1.1)--(-6.094,1.076)--(-6.109,1.025)--(-6.693,1.049)--cycle;
\draw(-6.694,1.1)--(-6.094,1.076);
\draw(-6.109,1.025)--(-6.693,1.049);
\filldraw[fill opacity=0.8,fill=gray!20,draw=none](-6.69,1.048)--(-6.109,1.025)--(-6.151,.979)--(-6.687,1)--cycle;
\draw(-6.69,1.048)--(-6.109,1.025);
\draw(-6.151,.979)--(-6.687,1);
\filldraw[fill opacity=0.8,fill=gray!20,draw=none](-6.108,1.042)--(-6.104,1.043)--(-6.132,.947)--(-6.154,.961)--cycle;
\draw(-6.132,.947)--(-6.154,.961);
\filldraw[fill opacity=0.8,fill=gray!20,draw=none](-6.133,.999)--(-6.109,1.025)--(-6.051,1.023)--(-6.056,1.015)--(-6.109,.977)--(-6.145,.979)--cycle;
\draw(-6.109,1.025)--(-6.051,1.023);
\draw(-6.109,.977)--(-6.145,.979);
\filldraw[fill opacity=0.8,fill=gray!20,draw=none](-6.097,.951)--(-6.081,.976)--(-6.056,.996)--(-6.056,.966)--cycle;
\draw(-6.056,.996)--(-6.056,.966);
\filldraw[fill opacity=0.8,fill=gray!20,draw=none](-6.037,1.014)--(-6.032,1.004)--(-6.056,.996)--(-6.056,1.016)--cycle;
\draw(-6.056,.996)--(-6.056,1.016);
\filldraw[fill opacity=0.8,fill=gray!20,draw=none](-6.045,1.023)--(-6.042,1.025)--(-6.037,1.014)--(-6.049,1.015)--cycle;
\filldraw[fill opacity=0.8,fill=gray!20,draw=none](-6.048,1.05)--(-6.108,1.042)--(-6.078,1.096)--(-6.044,1.086)--cycle;
\filldraw[fill opacity=0.8,fill=gray!20,draw=none](-6.062,1.064)--(-6.046,1.041)--(-6.043,1.028)--(-6.045,1.023)--(-6.069,1.024)--cycle;
\draw(-6.045,1.023)--(-6.069,1.024);
\filldraw[fill opacity=0.8,fill=gray!20,draw=none](-6.081,.976)--(-6.056,1.015)--(-6.056,.996)--cycle;
\draw(-6.056,1.015)--(-6.056,.996);
\filldraw[fill opacity=0.8,fill=gray!20,draw=none](-6.056,1.015)--(-6.051,1.023)--(-6.045,1.023)--cycle;
\draw(-6.051,1.023)--(-6.045,1.023);
\filldraw[fill opacity=0.8,fill=gray!20,draw=none](-6.048,1.038)--(-6.043,1.028)--(-6.049,1.015)--(-6.056,1.016)--(-6.056,1.043)--cycle;
\draw(-6.056,1.016)--(-6.056,1.043);
\filldraw[fill opacity=0.8,fill=gray!20,draw=none](-6.045,1.023)--(-6.043,1.028)--(-6.042,1.025)--cycle;
\filldraw[fill opacity=0.8,fill=gray!20,draw=none](-6.043,1.028)--(-6.042,1.023)--(-6.045,1.023)--cycle;
\draw(-6.042,1.023)--(-6.045,1.023);
\filldraw[fill opacity=0.8,fill=gray!20,draw=none](-6.056,1.015)--(-6.045,1.023)--(-6.042,1.023)--(-6.046,.975)--(-6.081,.976)--cycle;
\draw(-6.045,1.023)--(-6.042,1.023);
\draw(-6.046,.975)--(-6.081,.976);
\filldraw[fill opacity=0.8,fill=gray!20,draw=none](-6.043,1.028)--(-6.046,1.041)--(-6.04,1.034)--cycle;
\filldraw[fill opacity=0.8,fill=gray!20,draw=none](-6.043,1.028)--(-6.04,1.034)--(-6.032,1.022)--(-6.042,1.023)--cycle;
\draw(-6.032,1.022)--(-6.042,1.023);
\filldraw[fill opacity=0.8,fill=gray!20,draw=none](-6.043,1.028)--(-6.048,1.038)--(-6.04,1.034)--cycle;
\filldraw[fill opacity=0.8,fill=gray!20,draw=none](-6.044,1.006)--(-6.042,1.023)--(-6.032,1.022)--(-6.023,.989)--cycle;
\draw(-6.042,1.023)--(-6.032,1.022);
\filldraw[fill opacity=0.8,fill=gray!20,draw=none](-6.085,.994)--(-6.056,1.015)--(-6.081,.976)--(-6.091,.977)--cycle;
\draw(-6.081,.976)--(-6.091,.977);
\filldraw[fill opacity=0.8,fill=gray!20,draw=none](-6.103,1.104)--(-6.078,1.096)--(-6.092,1.071)--cycle;
\filldraw[fill opacity=0.8,fill=gray!20](-6.877,1.126)--(-6.878,1.18)--(-6.773,1.173)--(-6.769,1.119)--cycle;
\filldraw[fill opacity=0.8,fill=gray!20](-6.878,1.18)--(-6.879,1.227)--(-6.785,1.22)--(-6.773,1.173)--cycle;
\filldraw[fill opacity=0.8,fill=gray!20,draw=none](-6.673,1.146)--(-6.125,1.124)--(-6.103,1.104)--(-6.094,1.076)--(-6.694,1.1)--cycle;
\draw(-6.673,1.146)--(-6.125,1.124);
\draw(-6.094,1.076)--(-6.694,1.1);
\filldraw[fill opacity=0.8,fill=gray!20,draw=none](-6.671,1.181)--(-6.19,1.162)--(-6.125,1.124)--(-6.693,1.147)--cycle;
\draw(-6.671,1.181)--(-6.19,1.162);
\draw(-6.125,1.124)--(-6.693,1.147);
\filldraw[fill opacity=0.8,fill=gray!20,draw=none](-6.178,.98)--(-6.151,.979)--(-6.167,.971)--cycle;
\draw(-6.178,.98)--(-6.151,.979);
\filldraw[fill opacity=0.8,fill=gray!20,draw=none](-6.193,1.12)--(-6.187,1.131)--(-6.103,1.104)--(-6.092,1.071)--(-6.141,.985)--(-6.167,.971)--(-6.193,1.011)--(-6.203,1.064)--cycle;
\draw(-6.167,.971)--(-6.193,1.011)--(-6.203,1.064)--(-6.193,1.12)--(-6.187,1.131);
\filldraw[fill opacity=0.8,fill=gray!20,draw=none](-6.094,1.076)--(-6.07,1.075)--(-6.062,1.064)--(-6.069,1.024)--(-6.109,1.025)--cycle;
\draw(-6.094,1.076)--(-6.07,1.075);
\draw(-6.069,1.024)--(-6.109,1.025);
\filldraw[fill opacity=0.8,fill=gray!20,draw=none](-6.141,.985)--(-6.154,.961)--(-6.164,.967)--(-6.167,.971)--cycle;
\draw(-6.154,.961)--(-6.164,.967)--(-6.167,.971);
\filldraw[fill opacity=0.8,fill=gray!20,draw=none](-6.161,.966)--(-6.167,.971)--(-6.151,.979)--(-6.091,.977)--(-6.101,.961)--(-6.121,.945)--cycle;
\draw(-6.151,.979)--(-6.091,.977);
\filldraw[fill opacity=0.8,fill=gray!20,draw=none](-6.085,.994)--(-6.091,.977)--(-6.109,.977)--cycle;
\draw(-6.091,.977)--(-6.109,.977);
\filldraw[fill opacity=0.8,fill=gray!20,draw=none](-6.091,.977)--(-6.081,.976)--(-6.101,.961)--cycle;
\draw(-6.091,.977)--(-6.081,.976);
\filldraw[fill opacity=0.8,fill=gray!20,draw=none](-6.056,1.043)--(-6.056,1.015)--(-6.081,.976)--(-6.101,.961)--(-6.101,1.043)--cycle;
\draw(-6.056,1.043)--(-6.056,1.015);
\draw(-6.101,.961)--(-6.101,1.043);
\filldraw[fill opacity=0.8,fill=gray!20,draw=none](-6.044,1.006)--(-6.023,.989)--(-6.018,.974)--(-6.046,.975)--cycle;
\draw(-6.018,.974)--(-6.046,.975);
\filldraw[fill opacity=0.8,fill=gray!20,draw=none](-6.111,.936)--(-6.122,.941)--(-6.072,.936)--(-6.056,.929)--cycle;
\draw(-6.111,.936)--(-6.122,.941)--(-6.072,.936)--(-6.056,.929);
\filldraw[fill opacity=0.8,fill=gray!20,draw=none](-6,.919)--(-6,.896)--(-6.04,.894)--(-6.056,.903)--(-6.056,.92)--cycle;
\draw(-6,.919)--(-6,.896)--(-6.04,.894);
\draw(-6.056,.903)--(-6.056,.92);
\filldraw[fill opacity=0.8,fill=gray!20,draw=none](-6,.921)--(-6,.919)--(-6.056,.92)--(-6.056,.929)--cycle;
\draw(-6,.921)--(-6,.919);
\draw(-6.056,.92)--(-6.056,.929);
\filldraw[fill opacity=0.8,fill=gray!20,draw=none](-6.101,.932)--(-6.085,.933)--(-6.056,.929)--cycle;
\filldraw[fill opacity=0.8,fill=gray!20,draw=none](-6.101,.932)--(-6.111,.936)--(-6.085,.933)--cycle;
\draw(-6.101,.932)--(-6.111,.936);
\filldraw[fill opacity=0.8,fill=gray!20,draw=none](-6.097,.951)--(-6.056,.966)--(-6.056,.929)--(-6.101,.932)--(-6.101,.946)--cycle;
\draw(-6.056,.966)--(-6.056,.929);
\draw(-6.101,.932)--(-6.101,.946);
\filldraw[fill opacity=0.8,fill=gray!20,draw=none](-6.081,.976)--(-6.101,.946)--(-6.101,.961)--cycle;
\draw(-6.101,.946)--(-6.101,.961);
\filldraw[fill opacity=0.8,fill=gray!20,draw=none](-6.113,.941)--(-6.101,.95)--(-6.101,.932)--cycle;
\draw(-6.101,.95)--(-6.101,.932);
\filldraw[fill opacity=0.8,fill=gray!20](-6.96,.957)--(-6.975,1.009)--(-6.879,1.013)--(-6.881,.961)--cycle;
\filldraw[fill opacity=0.8,fill=gray!20](-6.884,.917)--(-6.881,.961)--(-6.805,.955)--(-6.83,.913)--cycle;
\filldraw[fill opacity=0.8,fill=gray!20,draw=none](-6.959,.956)--(-6.96,.957)--(-6.881,.961)--(-6.881,.954)--cycle;
\draw(-6.959,.956)--(-6.96,.957)--(-6.881,.961)--(-6.881,.954);
\filldraw[fill opacity=0.8,fill=gray!20,draw=none](-6.939,.914)--(-6.959,.956)--(-6.881,.954)--(-6.884,.917)--cycle;
\draw(-6.881,.954)--(-6.884,.917)--(-6.939,.914)--(-6.959,.956);
\filldraw[fill opacity=0.8,fill=gray!20](-6.83,.913)--(-6.805,.955)--(-6.751,.942)--(-6.792,.904)--cycle;
\filldraw[fill opacity=0.8,fill=gray!20,draw=none](-5.988,.919)--(-5.944,.91)--(-5.944,.895)--(-6,.896)--(-6,.904)--cycle;
\draw(-5.944,.91)--(-5.944,.895)--(-6,.896)--(-6,.904);
\filldraw[fill opacity=0.8,fill=gray!20,draw=none](-6.002,.942)--(-6,.94)--(-6,.921)--(-6.056,.929)--cycle;
\draw(-6,.94)--(-6,.921);
\filldraw[fill opacity=0.8,fill=gray!20,draw=none](-6.068,.455)--(-6.114,.454)--(-6.086,.496)--cycle;
\draw(-6.114,.454)--(-6.086,.496);
\filldraw[fill opacity=0.8,fill=gray!20,draw=none](-6.113,.941)--(-6.121,.945)--(-6.101,.961)--cycle;
\filldraw[fill opacity=0.8,fill=gray!20,draw=none](-6.101,1.079)--(-6.101,.95)--(-6.113,.941)--(-6.129,.952)--(-6.129,1.045)--cycle;
\draw(-6.101,1.079)--(-6.101,.95);
\draw(-6.129,.952)--(-6.129,1.045);
\filldraw[fill opacity=0.8,fill=gray!20,draw=none](-6.113,.941)--(-6.101,.961)--(-6.081,.976)--(-6.046,.975)--(-6.065,.938)--(-6.112,.94)--cycle;
\draw(-6.081,.976)--(-6.046,.975);
\draw(-6.065,.938)--(-6.112,.94);
\filldraw[fill opacity=0.8,fill=gray!20,draw=none](-6.046,.975)--(-6.018,.974)--(-6.03,.937)--(-6.065,.938)--cycle;
\draw(-6.046,.975)--(-6.018,.974);
\draw(-6.03,.937)--(-6.065,.938);
\filldraw[fill opacity=0.8,fill=gray!20,draw=none](-4.625,2.713)--(-4.839,2.398)--(-6.043,.965)--(-6.079,1.001)--(-4.634,2.722)--cycle;
\draw(-4.839,2.398)--(-6.043,.965)--(-6.079,1.001)--(-4.634,2.722);
\filldraw[fill opacity=0.8,fill=gray!20,draw=none](-4.497,3.057)--(-4.484,3.071)--(-4.516,3.076)--cycle;
\draw(-4.497,3.057)--(-4.484,3.071);
\filldraw[fill opacity=0.8,fill=gray!20,draw=none](-7.948,1.778)--(-7.985,1.791)--(-7.979,1.785)--(-7.958,1.776)--cycle;
\draw(-7.948,1.778)--(-7.985,1.791)--(-7.979,1.785)--(-7.958,1.776);
\filldraw[fill opacity=0.8,fill=gray!20,draw=none](-4.387,3.073)--(-4.36,3.064)--(-4.397,3.089)--cycle;
\filldraw[fill opacity=0.8,fill=gray!20,draw=none](-4.377,3.057)--(-4.397,3.089)--(-4.412,3.086)--(-4.389,3.054)--cycle;
\draw(-4.397,3.089)--(-4.412,3.086)--(-4.389,3.054)--(-4.377,3.057);
\filldraw[fill opacity=0.8,fill=gray!20,draw=none](-4.634,2.722)--(-4.646,2.707)--(-4.648,2.741)--cycle;
\draw(-4.634,2.722)--(-4.646,2.707);
\filldraw[fill opacity=0.8,fill=gray!20,draw=none](-4.634,2.722)--(-4.648,2.741)--(-4.65,2.772)--cycle;
\filldraw[fill opacity=0.8,fill=gray!20,draw=none](-4.343,2.856)--(-4.331,2.853)--(-4.343,2.869)--cycle;
\filldraw[fill opacity=0.8,fill=gray!20](-8.039,1.359)--(-8.042,1.369)--(-8.096,1.373)--(-8.067,1.361)--cycle;
\filldraw[fill opacity=0.8,fill=gray!20](-7.824,1.52)--(-7.817,1.576)--(-7.853,1.552)--(-7.859,1.498)--cycle;
\filldraw[fill opacity=0.8,fill=gray!20](-8.005,1.796)--(-8.036,1.793)--(-8.036,1.793)--(-7.985,1.791)--cycle;
\filldraw[fill opacity=0.8,fill=gray!20,draw=none](-4.27,2.992)--(-4.282,3)--(-4.293,2.985)--cycle;
\draw(-4.282,3)--(-4.293,2.985);
\filldraw[fill opacity=0.8,fill=gray!20,draw=none](-4.282,3)--(-4.271,2.988)--(-4.27,2.992)--cycle;
\filldraw[fill opacity=0.8,fill=gray!20,draw=none](-4.565,2.741)--(-4.563,2.745)--(-4.576,2.757)--cycle;
\filldraw[fill opacity=0.8,fill=gray!20](-2.876,7.976)--(-2.846,8.017)--(-2.786,8.028)--(-2.802,7.991)--cycle;
\filldraw[fill opacity=0.8,fill=gray!20,draw=none](-4.647,2.818)--(-4.642,2.826)--(-4.652,2.821)--(-4.649,2.818)--cycle;
\filldraw[fill opacity=0.8,fill=gray!20,draw=none](-4.626,2.861)--(-4.624,2.866)--(-4.618,2.879)--(-4.601,2.9)--cycle;
\draw(-4.618,2.879)--(-4.601,2.9);
\filldraw[fill opacity=0.8,fill=gray!20,draw=none](-4.61,2.885)--(-4.626,2.861)--(-4.601,2.9)--cycle;
\filldraw[fill opacity=0.8,fill=gray!20,draw=none](-4.601,2.9)--(-4.611,2.888)--(-4.576,2.921)--(-4.559,2.942)--cycle;
\draw(-4.601,2.9)--(-4.611,2.888);
\filldraw[fill opacity=0.8,fill=gray!20,draw=none](-4.601,2.9)--(-4.559,2.942)--(-4.556,2.944)--cycle;
\filldraw[fill opacity=0.8,fill=gray!20](-8.025,1.039)--(-7.974,1.052)--(-7.974,1.052)--(-8.031,1.045)--cycle;
\filldraw[fill opacity=0.8,fill=gray!20](-7.819,.987)--(-7.865,1.022)--(-7.883,1.01)--(-7.845,.97)--cycle;
\filldraw[fill opacity=0.8,fill=gray!20,draw=none](-2.557,7.76)--(-2.56,7.82)--(-2.527,7.812)--(-2.547,7.758)--cycle;
\draw(-2.56,7.82)--(-2.527,7.812)--(-2.547,7.758)--(-2.557,7.76);
\filldraw[fill opacity=0.8,fill=gray!20,draw=none](-4.414,2.694)--(-4.404,2.705)--(-4.404,2.707)--(-4.405,2.707)--(-4.413,2.704)--cycle;
\draw(-4.404,2.707)--(-4.405,2.707);
\filldraw[fill opacity=0.8,fill=gray!20](-8.01,1.36)--(-7.987,1.372)--(-8.042,1.369)--(-8.039,1.359)--cycle;
\filldraw[fill opacity=0.8,fill=gray!20,draw=none](-4.438,2.695)--(-4.425,2.706)--(-4.447,2.705)--(-4.446,2.692)--cycle;
\draw(-4.425,2.706)--(-4.447,2.705)--(-4.446,2.692);
\filldraw[fill opacity=0.8,fill=gray!20,draw=none](-4.43,2.695)--(-4.446,2.677)--(-4.455,2.676)--(-4.445,2.687)--cycle;
\draw(-4.43,2.695)--(-4.446,2.677);
\draw(-4.455,2.676)--(-4.445,2.687);
\filldraw[fill opacity=0.8,fill=gray!20,draw=none](-4.38,2.771)--(-4.401,2.743)--(-4.408,2.721)--(-4.389,2.744)--cycle;
\draw(-4.408,2.721)--(-4.389,2.744);
\filldraw[fill opacity=0.8,fill=gray!20,draw=none](-4.413,2.704)--(-4.405,2.707)--(-4.413,2.706)--cycle;
\draw(-4.405,2.707)--(-4.413,2.706);
\filldraw[fill opacity=0.8,fill=gray!20,draw=none](-4.404,2.707)--(-4.405,2.707)--(-4.404,2.707)--cycle;
\draw(-4.405,2.707)--(-4.404,2.707);
\filldraw[fill opacity=0.8,fill=gray!20,draw=none](-4.4,2.729)--(-4.408,2.721)--(-4.413,2.706)--(-4.405,2.707)--(-4.404,2.707)--cycle;
\draw(-4.413,2.706)--(-4.405,2.707);
\filldraw[fill opacity=0.8,fill=gray!20,draw=none](-4.425,2.706)--(-4.408,2.721)--(-4.402,2.744)--(-4.449,2.742)--(-4.447,2.705)--cycle;
\draw(-4.402,2.744)--(-4.449,2.742)--(-4.447,2.705)--(-4.425,2.706);
\filldraw[fill opacity=0.8,fill=gray!20,draw=none](-4.401,2.743)--(-4.447,2.684)--(-4.446,2.677)--(-4.408,2.721)--cycle;
\draw(-4.446,2.677)--(-4.408,2.721);
\filldraw[fill opacity=0.8,fill=gray!20,draw=none](-4.408,2.721)--(-4.4,2.729)--(-4.398,2.738)--(-4.402,2.744)--cycle;
\filldraw[fill opacity=0.8,fill=gray!20,draw=none](-4.379,2.767)--(-4.442,2.673)--(-4.438,2.653)--(-4.388,2.728)--cycle;
\draw(-4.379,2.767)--(-4.442,2.673);
\draw(-4.438,2.653)--(-4.388,2.728);
\filldraw[fill opacity=0.8,fill=gray!20,draw=none](-4.347,2.822)--(-4.347,2.838)--(-4.35,2.827)--(-4.351,2.811)--(-4.349,2.813)--cycle;
\draw(-4.351,2.811)--(-4.349,2.813);
\filldraw[fill opacity=0.8,fill=gray!20,draw=none](-2.864,7.764)--(-2.876,7.762)--(-2.894,7.816)--(-2.878,7.82)--cycle;
\draw(-2.864,7.764)--(-2.876,7.762)--(-2.894,7.816)--(-2.878,7.82);
\filldraw[fill opacity=0.8,fill=gray!20,draw=none](-4.64,2.816)--(-4.647,2.806)--(-4.645,2.827)--(-4.638,2.836)--cycle;
\draw(-4.645,2.827)--(-4.638,2.836);
\filldraw[fill opacity=0.8,fill=gray!20,draw=none](-4.64,2.816)--(-4.664,2.762)--(-4.651,2.797)--(-4.639,2.82)--(-4.634,2.831)--cycle;
\filldraw[fill opacity=0.8,fill=gray!20,draw=none](-4.64,2.816)--(-4.65,2.772)--(-4.647,2.806)--cycle;
\filldraw[fill opacity=0.8,fill=gray!20,draw=none](-4.349,2.813)--(-4.351,2.811)--(-4.352,2.808)--cycle;
\draw(-4.349,2.813)--(-4.351,2.811);
\filldraw[fill opacity=0.8,fill=gray!20,draw=none](-4.638,2.836)--(-4.621,2.869)--(-4.61,2.885)--cycle;
\filldraw[fill opacity=0.8,fill=gray!20,draw=none](-8.194,1.646)--(-8.219,1.612)--(-8.153,1.595)--(-8.141,1.651)--(-8.173,1.659)--cycle;
\draw(-8.219,1.612)--(-8.153,1.595)--(-8.141,1.651)--(-8.173,1.659);
\filldraw[fill opacity=0.8,fill=gray!20,draw=none](-8.156,1.669)--(-8.173,1.659)--(-8.141,1.651)--(-8.133,1.671)--cycle;
\draw(-8.173,1.659)--(-8.141,1.651)--(-8.133,1.671);
\filldraw[fill opacity=0.8,fill=gray!20,draw=none](-8.053,1.645)--(-8.067,1.659)--(-8.111,1.674)--(-8.133,1.671)--(-8.141,1.651)--cycle;
\draw(-8.133,1.671)--(-8.141,1.651)--(-8.053,1.645);
\filldraw[fill opacity=0.8,fill=gray!20,draw=none](-8.571,1.506)--(-8.524,1.517)--(-8.518,1.504)--(-8.523,1.501)--(-8.546,1.496)--cycle;
\draw(-8.571,1.506)--(-8.524,1.517);
\draw(-8.523,1.501)--(-8.546,1.496);
\filldraw[fill opacity=0.8,fill=gray!20,draw=none](-8.461,1.529)--(-8.528,1.5)--(-8.507,1.513)--cycle;
\draw(-8.461,1.529)--(-8.528,1.5)--(-8.507,1.513);
\filldraw[fill opacity=0.8,fill=gray!20,draw=none](-8.436,1.524)--(-8.437,1.546)--(-8.434,1.548)--(-8.4,1.552)--(-8.377,1.55)--(-8.373,1.549)--(-8.358,1.518)--cycle;
\draw(-8.4,1.552)--(-8.377,1.55);
\draw(-8.373,1.549)--(-8.358,1.518)--(-8.436,1.524)--(-8.437,1.546);
\filldraw[fill opacity=0.8,fill=gray!20,draw=none](-8.475,1.522)--(-8.475,1.523)--(-8.437,1.546)--(-8.436,1.524)--cycle;
\draw(-8.437,1.546)--(-8.436,1.524)--(-8.475,1.522);
\filldraw[fill opacity=0.8,fill=gray!20,draw=none](-8.475,1.343)--(-8.5,1.386)--(-8.5,1.389)--(-8.434,1.392)--(-8.436,1.345)--cycle;
\draw(-8.5,1.389)--(-8.434,1.392)--(-8.436,1.345)--(-8.475,1.343);
\filldraw[fill opacity=0.8,fill=gray!20,draw=none](-8.5,1.389)--(-8.509,1.437)--(-8.434,1.44)--(-8.434,1.392)--cycle;
\draw(-8.509,1.437)--(-8.434,1.44)--(-8.434,1.392)--(-8.5,1.389);
\filldraw[fill opacity=0.8,fill=gray!20](-8.436,1.345)--(-8.434,1.392)--(-8.348,1.386)--(-8.358,1.34)--cycle;
\filldraw[fill opacity=0.8,fill=gray!20](-8.434,1.392)--(-8.434,1.44)--(-8.344,1.433)--(-8.348,1.386)--cycle;
\filldraw[fill opacity=0.8,fill=gray!20,draw=none](-8.509,1.437)--(-8.5,1.482)--(-8.434,1.485)--(-8.434,1.44)--cycle;
\draw(-8.5,1.482)--(-8.434,1.485)--(-8.434,1.44)--(-8.509,1.437);
\filldraw[fill opacity=0.8,fill=gray!20](-8.434,1.44)--(-8.434,1.485)--(-8.348,1.478)--(-8.344,1.433)--cycle;
\filldraw[fill opacity=0.8,fill=gray!20,draw=none](-8.377,1.298)--(-8.4,1.299)--(-8.434,1.31)--(-8.437,1.313)--(-8.436,1.345)--(-8.358,1.34)--(-8.373,1.298)--cycle;
\draw(-8.377,1.298)--(-8.4,1.299);
\draw(-8.437,1.313)--(-8.436,1.345)--(-8.358,1.34)--(-8.373,1.298);
\filldraw[fill opacity=0.8,fill=gray!20,draw=none](-8.475,1.342)--(-8.475,1.343)--(-8.436,1.345)--(-8.437,1.313)--cycle;
\draw(-8.475,1.343)--(-8.436,1.345)--(-8.437,1.313);
\filldraw[fill opacity=0.8,fill=gray!20,draw=none](-8.5,1.482)--(-8.5,1.484)--(-8.475,1.522)--(-8.436,1.524)--(-8.434,1.485)--cycle;
\draw(-8.475,1.522)--(-8.436,1.524)--(-8.434,1.485)--(-8.5,1.482);
\filldraw[fill opacity=0.8,fill=gray!20](-8.434,1.485)--(-8.436,1.524)--(-8.358,1.518)--(-8.348,1.478)--cycle;
\filldraw[fill opacity=0.8,fill=gray!20,draw=none](-8.39,1.295)--(-8.4,1.299)--(-8.377,1.298)--cycle;
\draw(-8.4,1.299)--(-8.377,1.298);
\filldraw[fill opacity=0.8,fill=gray!20,draw=none](-8.475,1.523)--(-8.475,1.522)--(-8.477,1.522)--cycle;
\draw(-8.475,1.522)--(-8.477,1.522);
\filldraw[fill opacity=0.8,fill=gray!20,draw=none](-8.425,1.274)--(-8.434,1.274)--(-8.456,1.293)--(-8.478,1.329)--(-8.495,1.375)--(-8.504,1.426)--(-8.504,1.472)--(-8.495,1.508)--(-8.479,1.527)--(-8.467,1.527)--(-8.465,1.526)--cycle;
\draw(-8.434,1.274)--(-8.456,1.293)--(-8.478,1.329)--(-8.495,1.375)--(-8.504,1.426)--(-8.504,1.472)--(-8.495,1.508)--(-8.479,1.527)--(-8.467,1.527);
\filldraw[fill opacity=0.8,fill=gray!20,draw=none](-8.377,1.55)--(-8.4,1.552)--(-8.39,1.553)--cycle;
\draw(-8.377,1.55)--(-8.4,1.552);
\filldraw[fill opacity=0.8,fill=gray!20,draw=none](-8.518,1.504)--(-8.486,1.525)--(-8.479,1.527)--(-8.495,1.508)--(-8.518,1.503)--cycle;
\draw(-8.486,1.525)--(-8.479,1.527)--(-8.495,1.508)--(-8.518,1.503);
\filldraw[fill opacity=0.8,fill=gray!20,draw=none](-8.491,1.523)--(-8.472,1.526)--(-8.467,1.527)--(-8.479,1.527)--(-8.486,1.525)--cycle;
\draw(-8.467,1.527)--(-8.479,1.527)--(-8.486,1.525);
\filldraw[fill opacity=0.8,fill=gray!20](-8.081,1.702)--(-8.49,1.524)--(-8.445,1.529)--(-8.036,1.707)--cycle;
\filldraw[fill opacity=0.8,fill=gray!20,draw=none](-4.664,2.762)--(-4.674,2.747)--(-4.651,2.797)--cycle;
\draw(-4.664,2.762)--(-4.674,2.747);
\filldraw[fill opacity=0.8,fill=gray!20,draw=none](-7.889,1.737)--(-7.886,1.732)--(-7.881,1.728)--cycle;
\draw(-7.886,1.732)--(-7.881,1.728)--(-7.889,1.737);
\filldraw[fill opacity=0.8,fill=gray!20,draw=none](-4.353,2.809)--(-4.354,2.804)--(-4.353,2.806)--cycle;
\draw(-4.354,2.804)--(-4.353,2.806);
\filldraw[fill opacity=0.8,fill=gray!20,draw=none](-4.353,2.809)--(-4.354,2.832)--(-4.363,2.793)--(-4.354,2.804)--cycle;
\filldraw[fill opacity=0.8,fill=gray!20,draw=none](-4.352,2.808)--(-4.351,2.811)--(-4.361,2.799)--(-4.368,2.783)--cycle;
\draw(-4.351,2.811)--(-4.361,2.799);
\filldraw[fill opacity=0.8,fill=gray!20,draw=none](-4.35,2.825)--(-4.351,2.841)--(-4.354,2.832)--(-4.353,2.809)--cycle;
\filldraw[fill opacity=0.8,fill=gray!20,draw=none](-4.556,2.944)--(-4.559,2.942)--(-4.554,2.946)--(-4.552,2.948)--cycle;
\draw(-4.554,2.946)--(-4.552,2.948);
\filldraw[fill opacity=0.8,fill=gray!20,draw=none](-4.35,2.827)--(-4.355,2.805)--(-4.351,2.811)--cycle;
\draw(-4.355,2.805)--(-4.351,2.811);
\filldraw[fill opacity=0.8,fill=gray!20,draw=none](-4.628,2.841)--(-4.627,2.842)--(-4.63,2.851)--(-4.632,2.846)--cycle;
\filldraw[fill opacity=0.8,fill=gray!20,draw=none](-4.63,2.851)--(-4.634,2.869)--(-4.632,2.846)--cycle;
\draw(-4.634,2.869)--(-4.632,2.846);
\filldraw[fill opacity=0.8,fill=gray!20,draw=none](-4.629,2.84)--(-4.629,2.84)--(-4.635,2.842)--(-4.634,2.841)--cycle;
\filldraw[fill opacity=0.8,fill=gray!20,draw=none](-4.635,2.842)--(-4.637,2.842)--(-4.634,2.841)--cycle;
\draw(-4.637,2.842)--(-4.634,2.841);
\filldraw[fill opacity=0.8,fill=gray!20,draw=none](-4.635,2.842)--(-4.639,2.855)--(-4.651,2.849)--(-4.637,2.842)--cycle;
\draw(-4.651,2.849)--(-4.637,2.842);
\filldraw[fill opacity=0.8,fill=gray!20,draw=none](-4.635,2.842)--(-4.638,2.836)--(-4.66,2.809)--(-4.636,2.846)--cycle;
\draw(-4.638,2.836)--(-4.66,2.809);
\filldraw[fill opacity=0.8,fill=gray!20,draw=none](-4.634,2.831)--(-4.639,2.82)--(-4.631,2.838)--cycle;
\filldraw[fill opacity=0.8,fill=gray!20,draw=none](-4.651,2.797)--(-4.639,2.825)--(-4.631,2.838)--cycle;
\draw(-4.639,2.825)--(-4.631,2.838);
\filldraw[fill opacity=0.8,fill=gray!20,draw=none](-4.647,2.806)--(-4.651,2.797)--(-4.657,2.783)--(-4.7,2.727)--(-4.682,2.759)--cycle;
\filldraw[fill opacity=0.8,fill=gray!20,draw=none](-4.657,2.783)--(-4.674,2.747)--(-4.781,2.588)--(-4.7,2.727)--cycle;
\draw(-4.674,2.747)--(-4.781,2.588);
\filldraw[fill opacity=0.8,fill=gray!20,draw=none](-4.647,2.806)--(-4.65,2.772)--(-4.707,2.704)--(-4.682,2.759)--cycle;
\draw(-4.65,2.772)--(-4.707,2.704);
\filldraw[fill opacity=0.8,fill=gray!20,draw=none](-4.622,2.823)--(-4.614,2.81)--(-4.603,2.808)--(-4.627,2.839)--cycle;
\draw(-4.614,2.81)--(-4.603,2.808);
\filldraw[fill opacity=0.8,fill=gray!20,draw=none](-4.634,2.866)--(-4.63,2.851)--(-4.624,2.866)--(-4.632,2.868)--cycle;
\draw(-4.624,2.866)--(-4.632,2.868);
\filldraw[fill opacity=0.8,fill=gray!20,draw=none](-4.63,2.851)--(-4.635,2.842)--(-4.636,2.846)--(-4.626,2.861)--cycle;
\filldraw[fill opacity=0.8,fill=gray!20,draw=none](-4.628,2.841)--(-4.627,2.839)--(-4.627,2.842)--cycle;
\filldraw[fill opacity=0.8,fill=gray!20,draw=none](-4.629,2.84)--(-4.627,2.839)--(-4.629,2.84)--cycle;
\filldraw[fill opacity=0.8,fill=gray!20,draw=none](-4.628,2.841)--(-4.631,2.838)--(-4.627,2.842)--cycle;
\filldraw[fill opacity=0.8,fill=gray!20,draw=none](-4.628,2.841)--(-4.627,2.842)--(-4.597,2.879)--(-4.576,2.909)--cycle;
\filldraw[fill opacity=0.8,fill=gray!20,draw=none](-4.627,2.842)--(-4.615,2.855)--(-4.597,2.879)--cycle;
\filldraw[fill opacity=0.8,fill=gray!20,draw=none](-4.63,2.851)--(-4.626,2.861)--(-4.621,2.869)--cycle;
\filldraw[fill opacity=0.8,fill=gray!20,draw=none](-4.627,2.842)--(-4.627,2.839)--(-4.626,2.839)--(-4.614,2.837)--(-4.618,2.852)--cycle;
\filldraw[fill opacity=0.8,fill=gray!20,draw=none](-4.611,2.86)--(-4.596,2.879)--(-4.576,2.909)--cycle;
\draw(-4.596,2.879)--(-4.576,2.909);
\filldraw[fill opacity=0.8,fill=gray!20,draw=none](-4.627,2.842)--(-4.629,2.84)--(-4.623,2.843)--(-4.615,2.855)--cycle;
\filldraw[fill opacity=0.8,fill=gray!20,draw=none](-4.63,2.851)--(-4.627,2.842)--(-4.618,2.852)--(-4.62,2.865)--(-4.624,2.866)--cycle;
\draw(-4.62,2.865)--(-4.624,2.866);
\filldraw[fill opacity=0.8,fill=gray!20,draw=none](-4.635,2.842)--(-4.635,2.842)--(-4.629,2.84)--(-4.627,2.841)--cycle;
\filldraw[fill opacity=0.8,fill=gray!20,draw=none](-4.618,2.852)--(-4.614,2.837)--(-4.597,2.835)--(-4.573,2.853)--(-4.608,2.862)--cycle;
\draw(-4.573,2.853)--(-4.608,2.862);
\filldraw[fill opacity=0.8,fill=gray!20,draw=none](-4.629,2.84)--(-4.627,2.839)--(-4.626,2.839)--(-4.597,2.835)--(-4.618,2.846)--cycle;
\draw(-4.597,2.835)--(-4.618,2.846);
\filldraw[fill opacity=0.8,fill=gray!20,draw=none](-4.629,2.84)--(-4.631,2.838)--(-4.639,2.826)--(-4.629,2.836)--(-4.623,2.843)--cycle;
\draw(-4.631,2.838)--(-4.639,2.826);
\filldraw[fill opacity=0.8,fill=gray!20,draw=none](-4.576,2.909)--(-4.596,2.879)--(-4.541,2.917)--(-4.52,2.947)--cycle;
\draw(-4.576,2.909)--(-4.596,2.879);
\draw(-4.541,2.917)--(-4.52,2.947);
\filldraw[fill opacity=0.8,fill=gray!20,draw=none](-4.576,2.909)--(-4.52,2.947)--(-4.517,2.951)--cycle;
\draw(-4.52,2.947)--(-4.517,2.951);
\filldraw[fill opacity=0.8,fill=gray!20,draw=none](-4.646,2.707)--(-4.656,2.695)--(-4.657,2.763)--(-4.65,2.772)--cycle;
\draw(-4.646,2.707)--(-4.656,2.695);
\draw(-4.657,2.763)--(-4.65,2.772);
\filldraw[fill opacity=0.8,fill=gray!20,draw=none](-4.604,2.831)--(-4.57,2.821)--(-4.59,2.828)--(-4.627,2.839)--cycle;
\filldraw[fill opacity=0.8,fill=gray!20,draw=none](-4.626,2.839)--(-4.574,2.824)--(-4.597,2.835)--cycle;
\draw(-4.574,2.824)--(-4.597,2.835);
\filldraw[fill opacity=0.8,fill=gray!20,draw=none](-4.576,2.757)--(-4.548,2.75)--(-4.603,2.8)--cycle;
\draw(-4.576,2.757)--(-4.548,2.75);
\filldraw[fill opacity=0.8,fill=gray!20,draw=none](-4.603,2.8)--(-4.608,2.809)--(-4.614,2.81)--cycle;
\draw(-4.608,2.809)--(-4.614,2.81);
\filldraw[fill opacity=0.8,fill=gray!20,draw=none](-4.495,2.946)--(-4.498,2.948)--(-4.501,2.949)--cycle;
\draw(-4.495,2.946)--(-4.498,2.948)--(-4.501,2.949);
\filldraw[fill opacity=0.8,fill=gray!20,draw=none](-4.483,2.95)--(-4.482,2.943)--(-4.478,2.928)--cycle;
\draw(-4.483,2.95)--(-4.482,2.943)--(-4.478,2.928);
\filldraw[fill opacity=0.8,fill=gray!20,draw=none](-4.442,2.965)--(-4.483,2.95)--(-4.48,2.936)--(-4.47,2.93)--(-4.451,2.937)--(-4.435,2.958)--cycle;
\draw(-4.48,2.936)--(-4.47,2.93);
\draw(-4.451,2.937)--(-4.435,2.958);
\filldraw[fill opacity=0.8,fill=gray!20,draw=none](-4.478,2.914)--(-4.487,2.931)--(-4.52,2.91)--cycle;
\draw(-4.478,2.914)--(-4.487,2.931)--(-4.52,2.91);
\filldraw[fill opacity=0.8,fill=gray!20,draw=none](-4.488,2.934)--(-4.52,2.91)--(-4.487,2.931)--cycle;
\draw(-4.52,2.91)--(-4.487,2.931)--(-4.488,2.934);
\filldraw[fill opacity=0.8,fill=gray!20,draw=none](-4.488,2.934)--(-4.487,2.931)--(-4.478,2.914)--cycle;
\draw(-4.488,2.934)--(-4.487,2.931)--(-4.478,2.914);
\filldraw[fill opacity=0.8,fill=gray!20,draw=none](-4.511,2.96)--(-4.571,2.871)--(-4.563,2.866)--(-4.52,2.878)--(-4.516,2.88)--(-4.461,2.964)--cycle;
\draw(-4.511,2.96)--(-4.571,2.871);
\draw(-4.516,2.88)--(-4.461,2.964);
\filldraw[fill opacity=0.8,fill=gray!20,draw=none](-4.626,2.839)--(-4.61,2.818)--(-4.614,2.837)--cycle;
\filldraw[fill opacity=0.8,fill=gray!20,draw=none](-4.646,2.817)--(-4.647,2.806)--(-4.682,2.759)--(-4.679,2.766)--cycle;
\filldraw[fill opacity=0.8,fill=gray!20,draw=none](-4.647,2.806)--(-4.682,2.759)--(-4.675,2.772)--(-4.639,2.825)--cycle;
\draw(-4.675,2.772)--(-4.639,2.825);
\filldraw[fill opacity=0.8,fill=gray!20,draw=none](-4.617,2.881)--(-4.632,2.868)--(-4.622,2.866)--cycle;
\draw(-4.632,2.868)--(-4.622,2.866);
\filldraw[fill opacity=0.8,fill=gray!20,draw=none](-4.633,2.851)--(-4.644,2.834)--(-4.64,2.854)--(-4.618,2.879)--cycle;
\draw(-4.64,2.854)--(-4.618,2.879);
\filldraw[fill opacity=0.8,fill=gray!20,draw=none](-4.634,2.847)--(-4.636,2.847)--(-4.635,2.842)--(-4.631,2.842)--cycle;
\filldraw[fill opacity=0.8,fill=gray!20,draw=none](-4.682,2.759)--(-4.731,2.675)--(-4.789,2.6)--(-4.693,2.745)--cycle;
\draw(-4.789,2.6)--(-4.693,2.745);
\filldraw[fill opacity=0.8,fill=gray!20,draw=none](-4.656,2.695)--(-4.922,2.378)--(-4.879,2.455)--(-4.764,2.636)--(-4.657,2.763)--cycle;
\draw(-4.656,2.695)--(-4.922,2.378);
\draw(-4.764,2.636)--(-4.657,2.763);
\filldraw[fill opacity=0.8,fill=gray!20,draw=none](-4.611,2.86)--(-4.629,2.836)--(-4.617,2.847)--(-4.596,2.879)--cycle;
\draw(-4.617,2.847)--(-4.596,2.879);
\filldraw[fill opacity=0.8,fill=gray!20,draw=none](-4.634,2.847)--(-4.631,2.842)--(-4.627,2.841)--(-4.618,2.846)--(-4.619,2.846)--cycle;
\draw(-4.618,2.846)--(-4.619,2.846);
\filldraw[fill opacity=0.8,fill=gray!20,draw=none](-4.646,2.817)--(-4.679,2.766)--(-4.66,2.809)--(-4.645,2.827)--cycle;
\draw(-4.66,2.809)--(-4.645,2.827);
\filldraw[fill opacity=0.8,fill=gray!20,draw=none](-4.639,2.826)--(-4.675,2.772)--(-4.629,2.836)--cycle;
\draw(-4.639,2.826)--(-4.675,2.772);
\filldraw[fill opacity=0.8,fill=gray!20,draw=none](-4.57,2.821)--(-4.567,2.82)--(-4.574,2.824)--(-4.59,2.828)--cycle;
\draw(-4.567,2.82)--(-4.574,2.824);
\filldraw[fill opacity=0.8,fill=gray!20,draw=none](-4.565,2.819)--(-4.567,2.82)--(-4.57,2.821)--cycle;
\draw(-4.565,2.819)--(-4.567,2.82);
\filldraw[fill opacity=0.8,fill=gray!20,draw=none](-4.626,2.861)--(-4.627,2.86)--(-4.633,2.851)--(-4.63,2.856)--cycle;
\filldraw[fill opacity=0.8,fill=gray!20,draw=none](-4.597,2.835)--(-4.614,2.837)--(-4.611,2.823)--cycle;
\filldraw[fill opacity=0.8,fill=gray!20,draw=none](-4.618,2.852)--(-4.608,2.862)--(-4.62,2.865)--cycle;
\draw(-4.608,2.862)--(-4.62,2.865);
\filldraw[fill opacity=0.8,fill=gray!20,draw=none](-4.596,2.879)--(-4.605,2.865)--(-4.571,2.871)--(-4.541,2.917)--cycle;
\draw(-4.596,2.879)--(-4.605,2.865);
\draw(-4.571,2.871)--(-4.541,2.917);
\filldraw[fill opacity=0.8,fill=gray!20,draw=none](-4.617,2.881)--(-4.622,2.866)--(-4.614,2.864)--(-4.605,2.865)--(-4.568,2.906)--(-4.589,2.905)--cycle;
\draw(-4.622,2.866)--(-4.614,2.864);
\filldraw[fill opacity=0.8,fill=gray!20,draw=none](-4.624,2.866)--(-4.626,2.861)--(-4.627,2.86)--cycle;
\filldraw[fill opacity=0.8,fill=gray!20,draw=none](-4.624,2.866)--(-4.626,2.861)--(-4.63,2.856)--(-4.622,2.873)--cycle;
\filldraw[fill opacity=0.8,fill=gray!20,draw=none](-4.614,2.864)--(-4.608,2.862)--(-4.605,2.865)--cycle;
\draw(-4.614,2.864)--(-4.608,2.862);
\filldraw[fill opacity=0.8,fill=gray!20,draw=none](-4.431,2.839)--(-4.435,2.844)--(-4.461,2.828)--(-4.45,2.782)--cycle;
\draw(-4.431,2.839)--(-4.435,2.844)--(-4.461,2.828);
\filldraw[fill opacity=0.8,fill=gray!20,draw=none](-4.457,2.811)--(-4.461,2.828)--(-4.493,2.808)--cycle;
\draw(-4.461,2.828)--(-4.493,2.808);
\filldraw[fill opacity=0.8,fill=gray!20,draw=none](-4.449,2.844)--(-4.451,2.848)--(-4.474,2.839)--cycle;
\draw(-4.449,2.844)--(-4.451,2.848)--(-4.474,2.839);
\filldraw[fill opacity=0.8,fill=gray!20,draw=none](-4.457,2.864)--(-4.469,2.841)--(-4.451,2.848)--cycle;
\draw(-4.469,2.841)--(-4.451,2.848)--(-4.457,2.864);
\filldraw[fill opacity=0.8,fill=gray!20,draw=none](-4.644,2.834)--(-4.66,2.809)--(-5.053,2.341)--(-4.747,2.727)--(-4.64,2.854)--cycle;
\draw(-4.66,2.809)--(-5.053,2.341);
\draw(-4.747,2.727)--(-4.64,2.854);
\filldraw[fill opacity=0.8,fill=gray!20,draw=none](-4.639,2.855)--(-4.636,2.847)--(-4.619,2.846)--(-4.638,2.855)--cycle;
\draw(-4.619,2.846)--(-4.638,2.855);
\filldraw[fill opacity=0.8,fill=gray!20,draw=none](-4.693,2.745)--(-4.729,2.69)--(-4.764,2.636)--cycle;
\draw(-4.693,2.745)--(-4.729,2.69);
\filldraw[fill opacity=0.8,fill=gray!20,draw=none](-4.764,2.636)--(-4.729,2.69)--(-4.789,2.6)--(-4.841,2.518)--cycle;
\draw(-4.729,2.69)--(-4.789,2.6);
\filldraw[fill opacity=0.8,fill=gray!20,draw=none](-4.707,2.704)--(-5.219,2.094)--(-5.378,1.955)--(-4.66,2.809)--cycle;
\draw(-4.707,2.704)--(-5.219,2.094);
\draw(-5.378,1.955)--(-4.66,2.809);
\filldraw[fill opacity=0.8,fill=gray!20,draw=none](-4.513,2.826)--(-4.57,2.855)--(-4.602,2.838)--(-4.574,2.824)--cycle;
\draw(-4.513,2.826)--(-4.57,2.855);
\draw(-4.602,2.838)--(-4.574,2.824);
\filldraw[fill opacity=0.8,fill=gray!20,draw=none](-4.563,2.866)--(-4.59,2.882)--(-4.608,2.862)--(-4.571,2.853)--cycle;
\draw(-4.608,2.862)--(-4.571,2.853);
\filldraw[fill opacity=0.8,fill=gray!20,draw=none](-4.605,2.865)--(-4.617,2.847)--(-4.58,2.86)--(-4.577,2.861)--(-4.571,2.871)--cycle;
\draw(-4.605,2.865)--(-4.617,2.847);
\draw(-4.577,2.861)--(-4.571,2.871);
\filldraw[fill opacity=0.8,fill=gray!20,draw=none](-4.57,2.855)--(-4.581,2.86)--(-4.636,2.857)--(-4.638,2.855)--(-4.602,2.838)--cycle;
\draw(-4.57,2.855)--(-4.581,2.86);
\draw(-4.638,2.855)--(-4.602,2.838);
\filldraw[fill opacity=0.8,fill=gray!20,draw=none](-4.629,2.836)--(-4.653,2.802)--(-4.628,2.832)--(-4.617,2.847)--cycle;
\draw(-4.628,2.832)--(-4.617,2.847);
\filldraw[fill opacity=0.8,fill=gray!20,draw=none](-4.653,2.802)--(-4.675,2.772)--(-4.693,2.745)--(-4.764,2.636)--(-4.879,2.455)--(-4.628,2.832)--cycle;
\draw(-4.675,2.772)--(-4.693,2.745);
\draw(-4.879,2.455)--(-4.628,2.832);
\filldraw[fill opacity=0.8,fill=gray!20,draw=none](-4.586,2.833)--(-4.597,2.835)--(-4.611,2.823)--(-4.61,2.818)--(-4.603,2.808)--(-4.589,2.804)--cycle;
\draw(-4.603,2.808)--(-4.589,2.804);
\filldraw[fill opacity=0.8,fill=gray!20,draw=none](-4.597,2.835)--(-4.557,2.83)--(-4.559,2.85)--(-4.573,2.853)--cycle;
\draw(-4.557,2.83)--(-4.559,2.85)--(-4.573,2.853);
\filldraw[fill opacity=0.8,fill=gray!20,draw=none](-4.58,2.86)--(-4.617,2.847)--(-4.628,2.832)--cycle;
\draw(-4.617,2.847)--(-4.628,2.832);
\filldraw[fill opacity=0.8,fill=gray!20,draw=none](-4.59,2.882)--(-4.563,2.866)--(-4.557,2.874)--(-4.555,2.907)--(-4.568,2.906)--cycle;
\draw(-4.557,2.874)--(-4.555,2.907);
\filldraw[fill opacity=0.8,fill=gray!20,draw=none](-4.682,2.759)--(-4.693,2.745)--(-4.675,2.772)--cycle;
\draw(-4.693,2.745)--(-4.675,2.772);
\filldraw[fill opacity=0.8,fill=gray!20,draw=none](-4.581,2.86)--(-4.601,2.87)--(-4.625,2.865)--(-4.636,2.857)--cycle;
\draw(-4.581,2.86)--(-4.601,2.87);
\filldraw[fill opacity=0.8,fill=gray!20,draw=none](-4.563,2.866)--(-4.581,2.86)--(-4.57,2.855)--cycle;
\draw(-4.581,2.86)--(-4.57,2.855);
\filldraw[fill opacity=0.8,fill=gray!20,draw=none](-4.58,2.86)--(-4.628,2.832)--(-4.898,2.427)--(-4.987,2.248)--(-4.578,2.861)--cycle;
\draw(-4.628,2.832)--(-4.898,2.427);
\draw(-4.987,2.248)--(-4.578,2.861);
\filldraw[fill opacity=0.8,fill=gray!20,draw=none](-4.634,2.866)--(-4.634,2.869)--(-4.635,2.869)--cycle;
\draw(-4.634,2.866)--(-4.634,2.869)--(-4.635,2.869);
\filldraw[fill opacity=0.8,fill=gray!20,draw=none](-4.634,2.866)--(-4.632,2.868)--(-4.634,2.869)--cycle;
\draw(-4.632,2.868)--(-4.634,2.869);
\filldraw[fill opacity=0.8,fill=gray!20,draw=none](-4.625,2.865)--(-4.601,2.87)--(-4.611,2.875)--(-4.615,2.874)--cycle;
\draw(-4.601,2.87)--(-4.611,2.875);
\filldraw[fill opacity=0.8,fill=gray!20,draw=none](-4.603,2.8)--(-4.548,2.75)--(-4.543,2.748)--(-4.555,2.796)--(-4.608,2.809)--cycle;
\draw(-4.548,2.75)--(-4.543,2.748)--(-4.555,2.796)--(-4.608,2.809);
\filldraw[fill opacity=0.8,fill=gray!20,draw=none](-4.487,2.812)--(-4.517,2.822)--(-4.567,2.82)--(-4.565,2.819)--cycle;
\draw(-4.567,2.82)--(-4.565,2.819);
\filldraw[fill opacity=0.8,fill=gray!20,draw=none](-4.506,2.823)--(-4.513,2.826)--(-4.574,2.824)--(-4.567,2.82)--cycle;
\draw(-4.506,2.823)--(-4.513,2.826);
\draw(-4.574,2.824)--(-4.567,2.82);
\filldraw[fill opacity=0.8,fill=gray!20,draw=none](-4.455,2.823)--(-4.45,2.843)--(-4.52,2.878)--(-4.563,2.866)--(-4.57,2.855)--(-4.484,2.812)--cycle;
\draw(-4.45,2.843)--(-4.52,2.878);
\draw(-4.57,2.855)--(-4.484,2.812);
\filldraw[fill opacity=0.8,fill=gray!20,draw=none](-4.586,2.833)--(-4.589,2.804)--(-4.555,2.796)--(-4.557,2.83)--cycle;
\draw(-4.589,2.804)--(-4.555,2.796)--(-4.557,2.83);
\filldraw[fill opacity=0.8,fill=gray!20,draw=none](-4.439,2.813)--(-4.418,2.827)--(-4.429,2.833)--(-4.455,2.823)--(-4.457,2.811)--cycle;
\draw(-4.418,2.827)--(-4.429,2.833);
\filldraw[fill opacity=0.8,fill=gray!20,draw=none](-4.567,2.877)--(-4.611,2.875)--(-4.583,2.861)--cycle;
\draw(-4.611,2.875)--(-4.583,2.861);
\filldraw[fill opacity=0.8,fill=gray!20,draw=none](-4.537,2.887)--(-4.528,2.905)--(-4.555,2.907)--(-4.557,2.874)--cycle;
\draw(-4.528,2.905)--(-4.555,2.907)--(-4.557,2.874);
\filldraw[fill opacity=0.8,fill=gray!20,draw=none](-4.551,2.878)--(-4.537,2.887)--(-4.557,2.897)--(-4.563,2.896)--(-4.611,2.875)--cycle;
\draw(-4.537,2.887)--(-4.557,2.897);
\filldraw[fill opacity=0.8,fill=gray!20,draw=none](-5.053,2.341)--(-6.008,1.205)--(-6.013,1.211)--(-5.998,1.238)--(-4.747,2.727)--cycle;
\draw(-5.053,2.341)--(-6.008,1.205);
\draw(-5.998,1.238)--(-4.747,2.727);
\filldraw[fill opacity=0.8,fill=gray!20,draw=none](-5.219,2.094)--(-5.87,1.32)--(-5.861,1.38)--(-5.378,1.955)--cycle;
\draw(-5.219,2.094)--(-5.87,1.32);
\draw(-5.861,1.38)--(-5.378,1.955);
\filldraw[fill opacity=0.8,fill=gray!20,draw=none](-5.87,1.32)--(-5.988,1.179)--(-6.03,1.178)--(-5.861,1.38)--cycle;
\draw(-5.87,1.32)--(-5.988,1.179);
\draw(-6.03,1.178)--(-5.861,1.38);
\filldraw[fill opacity=0.8,fill=gray!20,draw=none](-5.988,1.179)--(-6.037,1.121)--(-6.04,1.124)--(-6.034,1.173)--(-6.03,1.178)--cycle;
\draw(-5.988,1.179)--(-6.037,1.121);
\draw(-6.034,1.173)--(-6.03,1.178);
\filldraw[fill opacity=0.8,fill=gray!20,draw=none](-4.764,2.636)--(-4.841,2.518)--(-4.898,2.427)--(-4.879,2.455)--cycle;
\draw(-4.898,2.427)--(-4.879,2.455);
\filldraw[fill opacity=0.8,fill=gray!20,draw=none](-5.93,1.179)--(-6.016,1.077)--(-6.041,1.116)--(-5.988,1.179)--cycle;
\draw(-5.93,1.179)--(-6.016,1.077);
\draw(-6.041,1.116)--(-5.988,1.179);
\filldraw[fill opacity=0.8,fill=gray!20,draw=none](-5.934,1.018)--(-5.932,.982)--(-5.935,.979)--(-5.979,.988)--(-5.951,1.037)--(-5.94,1.032)--cycle;
\draw(-5.979,.988)--(-5.951,1.037)--(-5.94,1.032);
\filldraw[fill opacity=0.8,fill=gray!20,draw=none](-5.94,1.032)--(-5.951,1.037)--(-5.95,1.045)--cycle;
\draw(-5.94,1.032)--(-5.951,1.037)--(-5.95,1.045);
\filldraw[fill opacity=0.8,fill=gray!20,draw=none](-5.945,1.039)--(-5.95,1.045)--(-5.948,1.055)--cycle;
\draw(-5.95,1.045)--(-5.948,1.055);
\filldraw[fill opacity=0.8,fill=gray!20,draw=none](-5.963,1.06)--(-5.947,1.062)--(-5.948,1.055)--cycle;
\draw(-5.947,1.062)--(-5.948,1.055);
\filldraw[fill opacity=0.8,fill=gray!20,draw=none](-5.934,1.03)--(-5.951,1.074)--(-5.937,1.091)--cycle;
\draw(-5.951,1.074)--(-5.937,1.091);
\filldraw[fill opacity=0.8,fill=gray!20](-4.447,2.705)--(-4.449,2.742)--(-4.543,2.748)--(-4.523,2.71)--cycle;
\filldraw[fill opacity=0.8,fill=gray!20,draw=none](-4.35,2.885)--(-4.359,2.892)--(-4.379,2.864)--cycle;
\draw(-4.359,2.892)--(-4.379,2.864);
\filldraw[fill opacity=0.8,fill=gray!20,draw=none](-4.39,2.909)--(-4.428,2.889)--(-4.379,2.864)--cycle;
\draw(-4.428,2.889)--(-4.379,2.864);
\filldraw[fill opacity=0.8,fill=gray!20,draw=none](-4.418,2.843)--(-4.432,2.838)--(-4.424,2.84)--cycle;
\draw(-4.432,2.838)--(-4.424,2.84)--(-4.418,2.843);
\filldraw[fill opacity=0.8,fill=gray!20,draw=none](-4.379,2.864)--(-4.427,2.889)--(-4.418,2.827)--cycle;
\draw(-4.379,2.864)--(-4.427,2.889);
\filldraw[fill opacity=0.8,fill=gray!20,draw=none](-4.43,2.838)--(-4.431,2.839)--(-4.432,2.838)--cycle;
\draw(-4.43,2.838)--(-4.431,2.839);
\filldraw[fill opacity=0.8,fill=gray!20,draw=none](-4.438,2.848)--(-4.435,2.844)--(-4.431,2.839)--cycle;
\draw(-4.438,2.848)--(-4.435,2.844)--(-4.431,2.839);
\filldraw[fill opacity=0.8,fill=gray!20,draw=none](-5.934,1.03)--(-5.94,1.032)--(-5.945,1.039)--(-5.948,1.055)--(-5.947,1.062)--cycle;
\draw(-5.934,1.03)--(-5.94,1.032);
\draw(-5.948,1.055)--(-5.947,1.062);
\filldraw[fill opacity=0.8,fill=gray!20,draw=none](-5.342,1.723)--(-5.934,1.018)--(-5.937,1.091)--(-4.839,2.398)--cycle;
\draw(-5.342,1.723)--(-5.934,1.018);
\draw(-5.937,1.091)--(-4.839,2.398);
\filldraw[fill opacity=0.8,fill=gray!20,draw=none](-4.467,2.891)--(-4.474,2.914)--(-4.469,2.896)--cycle;
\draw(-4.474,2.914)--(-4.469,2.896)--(-4.467,2.891);
\filldraw[fill opacity=0.8,fill=gray!20,draw=none](-4.451,2.848)--(-4.452,2.851)--(-4.451,2.848)--(-4.449,2.844)--cycle;
\draw(-4.452,2.851)--(-4.451,2.848)--(-4.449,2.844);
\filldraw[fill opacity=0.8,fill=gray!20,draw=none](-4.457,2.864)--(-4.452,2.851)--(-4.451,2.848)--cycle;
\draw(-4.457,2.864)--(-4.452,2.851);
\filldraw[fill opacity=0.8,fill=gray!20,draw=none](-4.427,2.889)--(-4.447,2.898)--(-4.516,2.88)--(-4.52,2.878)--(-4.418,2.827)--cycle;
\draw(-4.427,2.889)--(-4.447,2.898);
\draw(-4.52,2.878)--(-4.418,2.827);
\filldraw[fill opacity=0.8,fill=gray!20](-4.449,2.742)--(-4.451,2.788)--(-4.555,2.796)--(-4.543,2.748)--cycle;
\filldraw[fill opacity=0.8,fill=gray!20,draw=none](-4.526,2.711)--(-4.523,2.71)--(-4.543,2.748)--(-4.576,2.757)--cycle;
\draw(-4.526,2.711)--(-4.523,2.71)--(-4.543,2.748)--(-4.576,2.757);
\filldraw[fill opacity=0.8,fill=gray!20,draw=none](-4.457,2.811)--(-4.455,2.823)--(-4.484,2.812)--(-4.48,2.81)--cycle;
\draw(-4.484,2.812)--(-4.48,2.81);
\filldraw[fill opacity=0.8,fill=gray!20,draw=none](-4.48,2.81)--(-4.484,2.812)--(-4.487,2.812)--cycle;
\draw(-4.48,2.81)--(-4.484,2.812);
\filldraw[fill opacity=0.8,fill=gray!20,draw=none](-4.487,2.812)--(-4.484,2.812)--(-4.506,2.823)--(-4.517,2.822)--cycle;
\draw(-4.484,2.812)--(-4.506,2.823);
\filldraw[fill opacity=0.8,fill=gray!20,draw=none](-4.429,2.833)--(-4.45,2.843)--(-4.455,2.823)--cycle;
\draw(-4.429,2.833)--(-4.45,2.843);
\filldraw[fill opacity=0.8,fill=gray!20,draw=none](-4.398,2.911)--(-4.4,2.904)--(-4.39,2.909)--(-4.39,2.913)--cycle;
\filldraw[fill opacity=0.8,fill=gray!20,draw=none](-4.4,2.904)--(-4.398,2.911)--(-4.427,2.905)--(-4.433,2.891)--(-4.428,2.889)--cycle;
\draw(-4.433,2.891)--(-4.428,2.889);
\filldraw[fill opacity=0.8,fill=gray!20,draw=none](-5.932,.982)--(-5.934,1.018)--(-5.922,.99)--cycle;
\filldraw[fill opacity=0.8,fill=gray!20,draw=none](-6.068,.604)--(-6.114,.534)--(-6.14,.553)--(-6.058,.676)--cycle;
\draw(-6.068,.604)--(-6.114,.534);
\draw(-6.14,.553)--(-6.058,.676);
\filldraw[fill opacity=0.8,fill=gray!20,draw=none](-5.295,1.715)--(-5.993,.67)--(-5.951,.66)--(-5.819,.841)--(-5.064,1.971)--cycle;
\draw(-5.295,1.715)--(-5.993,.67);
\draw(-5.819,.841)--(-5.064,1.971);
\filldraw[fill opacity=0.8,fill=gray!20,draw=none](-6.114,.534)--(-6.146,.487)--(-6.142,.549)--(-6.14,.553)--cycle;
\draw(-6.114,.534)--(-6.146,.487);
\draw(-6.142,.549)--(-6.14,.553);
\filldraw[fill opacity=0.8,fill=gray!20,draw=none](-6.14,.553)--(-6.142,.549)--(-6.141,.553)--cycle;
\draw(-6.14,.553)--(-6.142,.549);
\filldraw[fill opacity=0.8,fill=gray!20,draw=none](-6.147,.454)--(-6.142,.554)--(-6.102,.529)--(-6.074,.486)--(-6.068,.455)--cycle;
\draw(-6.142,.554)--(-6.102,.529)--(-6.074,.486)--(-6.068,.455);
\filldraw[fill opacity=0.8,fill=gray!20,draw=none](-5.947,1.063)--(-5.988,1.023)--(-5.992,1.026)--(-5.951,1.074)--cycle;
\draw(-5.992,1.026)--(-5.951,1.074);
\filldraw[fill opacity=0.8,fill=gray!20,draw=none](-5.948,1.062)--(-5.963,1.06)--(-6.016,1.077)--(-6.027,1.094)--(-6.027,1.178)--(-5.971,1.177)--(-5.951,1.146)--(-5.945,1.117)--cycle;
\draw(-5.971,1.177)--(-5.951,1.146)--(-5.945,1.117);
\filldraw[fill opacity=0.8,fill=gray!20,draw=none](-6.008,1.205)--(-6.03,1.178)--(-6.032,1.178)--(-6.013,1.211)--cycle;
\draw(-6.008,1.205)--(-6.03,1.178);
\filldraw[fill opacity=0.8,fill=gray!20,draw=none](-4.922,2.378)--(-5.93,1.179)--(-5.988,1.179)--(-4.805,2.587)--cycle;
\draw(-4.922,2.378)--(-5.93,1.179);
\draw(-5.988,1.179)--(-4.805,2.587);
\filldraw[fill opacity=0.8,fill=gray!20,draw=none](-6.034,1.178)--(-6.031,1.204)--(-6.02,1.215)--(-5.979,1.19)--(-5.971,1.177)--cycle;
\draw(-6.02,1.215)--(-5.979,1.19)--(-5.971,1.177);
\filldraw[fill opacity=0.8,fill=gray!20,draw=none](-6.034,1.178)--(-6.041,1.116)--(-6.07,1.163)--(-6.056,1.178)--cycle;
\filldraw[fill opacity=0.8,fill=gray!20,draw=none](-6.013,1.211)--(-6.02,1.215)--(-6.019,1.215)--(-6.016,1.213)--cycle;
\draw(-6.013,1.211)--(-6.02,1.215);
\draw(-6.019,1.215)--(-6.016,1.213);
\filldraw[fill opacity=0.8,fill=gray!20,draw=none](-6.034,1.178)--(-6.056,1.178)--(-6.031,1.204)--cycle;
\filldraw[fill opacity=0.8,fill=gray!20,draw=none](-6.011,1.215)--(-6.032,1.178)--(-6.043,1.177)--(-6.033,1.196)--(-6.02,1.212)--cycle;
\draw(-6.033,1.196)--(-6.02,1.212);
\filldraw[fill opacity=0.8,fill=gray!20,draw=none](-5.954,1.179)--(-5.979,1.19)--(-6.013,1.211)--(-6.016,1.213)--(-6,1.207)--cycle;
\draw(-5.954,1.179)--(-5.979,1.19)--(-6.013,1.211);
\draw(-6.016,1.213)--(-6,1.207);
\filldraw[fill opacity=0.8,fill=gray!20,draw=none](-6.04,1.124)--(-6.037,1.121)--(-6.041,1.116)--cycle;
\draw(-6.037,1.121)--(-6.041,1.116);
\filldraw[fill opacity=0.8,fill=gray!20,draw=none](-6.03,1.178)--(-6.034,1.173)--(-6.032,1.178)--cycle;
\draw(-6.03,1.178)--(-6.034,1.173);
\filldraw[fill opacity=0.8,fill=gray!20,draw=none](-6.027,1.094)--(-6.041,1.116)--(-6.034,1.178)--(-6.027,1.178)--cycle;
\filldraw[fill opacity=0.8,fill=gray!20,draw=none](-6.041,1.116)--(-6.044,1.086)--(-6.077,1.096)--(-6.109,1.124)--(-6.07,1.163)--cycle;
\filldraw[fill opacity=0.8,fill=gray!20,draw=none](-6.04,1.124)--(-6.059,1.144)--(-6.034,1.173)--cycle;
\draw(-6.059,1.144)--(-6.034,1.173);
\filldraw[fill opacity=0.8,fill=gray!20,draw=none](-6.056,1.154)--(-6.041,1.177)--(-6.032,1.178)--(-6.034,1.173)--(-6.059,1.144)--(-6.059,1.148)--cycle;
\draw(-6.034,1.173)--(-6.059,1.144);
\filldraw[fill opacity=0.8,fill=gray!20,draw=none](-6.041,1.116)--(-6.016,1.077)--(-6.044,1.086)--cycle;
\filldraw[fill opacity=0.8,fill=gray!20,draw=none](-6.085,1.088)--(-6.07,1.075)--(-6.09,1.076)--cycle;
\draw(-6.07,1.075)--(-6.09,1.076);
\filldraw[fill opacity=0.8,fill=gray!20,draw=none](-5.797,1.136)--(-5.906,.996)--(-5.954,.994)--(-5.835,1.135)--cycle;
\draw(-5.954,.994)--(-5.835,1.135);
\filldraw[fill opacity=0.8,fill=gray!20,draw=none](-5.954,.994)--(-5.944,.99)--(-5.944,.977)--(-6,.952)--(-6,.992)--cycle;
\draw(-5.944,.99)--(-5.944,.977);
\draw(-6,.952)--(-6,.992);
\filldraw[fill opacity=0.8,fill=gray!20,draw=none](-5.935,.979)--(-5.944,.972)--(-5.944,.977)--cycle;
\draw(-5.944,.972)--(-5.944,.977);
\filldraw[fill opacity=0.8,fill=gray!20,draw=none](-5.935,.979)--(-5.932,.979)--(-5.931,.967)--(-5.944,.972)--cycle;
\draw(-5.931,.967)--(-5.944,.972);
\filldraw[fill opacity=0.8,fill=gray!20,draw=none](-5.922,.99)--(-5.935,.979)--(-5.944,.977)--(-5.944,.995)--cycle;
\draw(-5.944,.977)--(-5.944,.995);
\filldraw[fill opacity=0.8,fill=gray!20,draw=none](-5.954,.994)--(-5.944,.995)--(-5.944,.99)--cycle;
\draw(-5.944,.995)--(-5.944,.99);
\filldraw[fill opacity=0.8,fill=gray!20,draw=none](-5.932,.982)--(-5.932,.979)--(-5.935,.979)--cycle;
\filldraw[fill opacity=0.8,fill=gray!20,draw=none](-5.949,1.061)--(-5.963,1.005)--(-5.988,1.023)--cycle;
\filldraw[fill opacity=0.8,fill=gray!20,draw=none](-5.906,1.127)--(-5.951,1.146)--(-5.979,1.19)--(-5.944,1.174)--cycle;
\draw(-5.906,1.127)--(-5.951,1.146)--(-5.979,1.19)--(-5.944,1.174);
\filldraw[fill opacity=0.8,fill=gray!20,draw=none](-6.002,.942)--(-6,.943)--(-6,.94)--cycle;
\draw(-6,.943)--(-6,.94);
\filldraw[fill opacity=0.8,fill=gray!20,draw=none](-5.953,.802)--(-6.061,.64)--(-6.032,.611)--(-5.862,.865)--cycle;
\draw(-5.953,.802)--(-6.061,.64);
\draw(-6.032,.611)--(-5.862,.865);
\filldraw[fill opacity=0.8,fill=gray!20,draw=none](-6.013,.893)--(-6.004,.896)--(-6,.896)--(-5.944,.895)--(-5.894,.888)--(-5.859,.879)--(-5.844,.867)--(-5.851,.854)--(-5.879,.843)--(-5.895,.841)--cycle;
\draw(-6.004,.896)--(-6,.896)--(-5.944,.895)--(-5.894,.888)--(-5.859,.879)--(-5.844,.867)--(-5.851,.854)--(-5.879,.843)--(-5.895,.841);
\filldraw[fill opacity=0.8,fill=gray!20,draw=none](-5.928,.893)--(-6.018,.932)--(-6.002,.942)--(-6,.943)--(-5.915,.905)--cycle;
\draw(-6,.943)--(-5.915,.905);
\filldraw[fill opacity=0.8,fill=gray!20,draw=none](-5.983,.993)--(-6,.992)--(-6,1.011)--cycle;
\draw(-6,.992)--(-6,1.011);
\filldraw[fill opacity=0.8,fill=gray!20,draw=none](-5.954,.994)--(-5.983,.993)--(-6,1.011)--(-6,1.012)--cycle;
\draw(-6,1.011)--(-6,1.012);
\filldraw[fill opacity=0.8,fill=gray!20,draw=none](-5.971,1.01)--(-6.003,1.012)--(-5.992,1.026)--cycle;
\draw(-6.003,1.012)--(-5.992,1.026);
\filldraw[fill opacity=0.8,fill=gray!20,draw=none](-5.906,.996)--(-5.956,.932)--(-5.981,.937)--(-5.97,.975)--(-5.954,.994)--cycle;
\draw(-5.956,.932)--(-5.981,.937);
\draw(-5.97,.975)--(-5.954,.994);
\filldraw[fill opacity=0.8,fill=gray!20,draw=none](-5.922,.99)--(-5.94,1.032)--(-5.894,1.012)--cycle;
\draw(-5.94,1.032)--(-5.894,1.012);
\filldraw[fill opacity=0.8,fill=gray!20,draw=none](-5.949,1.061)--(-5.947,1.063)--(-5.934,1.03)--(-5.934,1.018)--(-5.952,.996)--(-5.963,1.005)--cycle;
\draw(-5.934,1.018)--(-5.952,.996);
\filldraw[fill opacity=0.8,fill=gray!20,draw=none](-5.948,1.062)--(-5.945,1.117)--(-5.941,1.093)--(-5.947,1.062)--cycle;
\draw(-5.945,1.117)--(-5.941,1.093)--(-5.947,1.062);
\filldraw[fill opacity=0.8,fill=gray!20,draw=none](-5.432,1.503)--(-5.843,1.013)--(-5.862,1.001)--(-5.468,1.47)--cycle;
\draw(-5.432,1.503)--(-5.843,1.013);
\draw(-5.862,1.001)--(-5.468,1.47);
\filldraw[fill opacity=0.8,fill=gray!20,draw=none](-5.894,1.012)--(-5.934,1.03)--(-5.947,1.062)--(-5.941,1.093)--(-5.877,1.065)--cycle;
\draw(-5.894,1.012)--(-5.934,1.03);
\draw(-5.947,1.062)--(-5.941,1.093)--(-5.877,1.065);
\filldraw[fill opacity=0.8,fill=gray!20,draw=none](-5.784,1.093)--(-5.865,.997)--(-5.894,.99)--(-5.904,.99)--(-5.811,1.101)--cycle;
\draw(-5.784,1.093)--(-5.865,.997);
\draw(-5.904,.99)--(-5.811,1.101);
\filldraw[fill opacity=0.8,fill=gray!20,draw=none](-5.877,1.065)--(-5.941,1.093)--(-5.951,1.146)--(-5.894,1.122)--cycle;
\draw(-5.877,1.065)--(-5.941,1.093)--(-5.951,1.146)--(-5.894,1.122);
\filldraw[fill opacity=0.8,fill=gray!20,draw=none](-5.994,1.012)--(-5.971,1.01)--(-5.952,.996)--cycle;
\filldraw[fill opacity=0.8,fill=gray!20,draw=none](-6.016,1.077)--(-6.04,1.048)--(-6.052,1.075)--(-6.057,1.098)--(-6.041,1.116)--cycle;
\draw(-6.016,1.077)--(-6.04,1.048);
\draw(-6.057,1.098)--(-6.041,1.116);
\filldraw[fill opacity=0.8,fill=gray!20,draw=none](-6.023,1.073)--(-6,1.025)--(-6,1.011)--(-6.04,1.034)--cycle;
\draw(-6,1.025)--(-6,1.011);
\filldraw[fill opacity=0.8,fill=gray!20,draw=none](-5.964,1.186)--(-5.964,1.015)--(-6,1.025)--(-6,1.207)--cycle;
\draw(-6,1.025)--(-6,1.207);
\filldraw[fill opacity=0.8,fill=gray!20,draw=none](-5.894,.99)--(-5.907,.986)--(-5.904,.99)--cycle;
\draw(-5.907,.986)--(-5.904,.99);
\filldraw[fill opacity=0.8,fill=gray!20,draw=none](-6,1.207)--(-6.018,1.214)--(-6.019,1.215)--(-6.02,1.21)--cycle;
\draw(-6,1.207)--(-6.018,1.214);
\filldraw[fill opacity=0.8,fill=gray!20,draw=none](-6.04,1.124)--(-6.041,1.116)--(-6.055,1.1)--(-6.061,1.122)--(-6.062,1.141)--(-6.059,1.144)--cycle;
\draw(-6.041,1.116)--(-6.055,1.1);
\draw(-6.062,1.141)--(-6.059,1.144);
\filldraw[fill opacity=0.8,fill=gray!20,draw=none](-6.077,1.096)--(-6.103,1.104)--(-6.109,1.124)--cycle;
\filldraw[fill opacity=0.8,fill=gray!20,draw=none](-6.103,1.104)--(-6.085,1.088)--(-6.09,1.076)--(-6.094,1.076)--cycle;
\draw(-6.09,1.076)--(-6.094,1.076);
\filldraw[fill opacity=0.8,fill=gray!20,draw=none](-5.968,1.016)--(-6,1.016)--(-6,1.025)--cycle;
\draw(-6,1.016)--(-6,1.025);
\filldraw[fill opacity=0.8,fill=gray!20,draw=none](-6.056,1.154)--(-6.043,1.177)--(-6.041,1.177)--cycle;
\filldraw[fill opacity=0.8,fill=gray!20,draw=none](-6.052,1.177)--(-6,1.176)--(-6,1.124)--(-6.011,1.101)--(-6.045,1.121)--(-6.056,1.143)--(-6.056,1.174)--cycle;
\draw(-6,1.176)--(-6,1.124);
\draw(-6.056,1.143)--(-6.056,1.174);
\filldraw[fill opacity=0.8,fill=gray!20,draw=none](-5.964,1.2)--(-5.964,1.186)--(-6,1.207)--cycle;
\filldraw[fill opacity=0.8,fill=gray!20,draw=none](-5.964,1.186)--(-5.964,1.252)--(-5.944,1.252)--(-5.944,1.174)--cycle;
\draw(-5.964,1.252)--(-5.944,1.252)--(-5.944,1.174);
\filldraw[fill opacity=0.8,fill=gray!20,draw=none](-5.894,1.246)--(-5.894,1.147)--(-5.944,1.174)--(-5.944,1.252)--cycle;
\draw(-5.944,1.174)--(-5.944,1.252)--(-5.894,1.246)--(-5.894,1.147);
\filldraw[fill opacity=0.8,fill=gray!20,draw=none](-5.916,1.169)--(-5.942,1.173)--(-5.954,1.179)--(-6,1.207)--(-5.926,1.174)--cycle;
\draw(-5.942,1.173)--(-5.954,1.179);
\draw(-6,1.207)--(-5.926,1.174);
\filldraw[fill opacity=0.8,fill=gray!20,draw=none](-6.011,1.101)--(-6,1.124)--(-6,1.095)--cycle;
\draw(-6,1.124)--(-6,1.095);
\filldraw[fill opacity=0.8,fill=gray!20,draw=none](-6.023,1.073)--(-6.011,1.101)--(-6,1.095)--(-6,1.025)--cycle;
\draw(-6,1.095)--(-6,1.025);
\filldraw[fill opacity=0.8,fill=gray!20,draw=none](-6.055,1.1)--(-6.061,1.093)--(-6.061,1.122)--cycle;
\draw(-6.055,1.1)--(-6.061,1.093);
\filldraw[fill opacity=0.8,fill=gray!20,draw=none](-6.116,1.117)--(-6.109,1.124)--(-6.075,1.122)--(-6.052,1.075)--(-6.07,1.075)--cycle;
\draw(-6.109,1.124)--(-6.075,1.122);
\draw(-6.052,1.075)--(-6.07,1.075);
\filldraw[fill opacity=0.8,fill=gray!20,draw=none](-6.04,1.048)--(-6.046,1.041)--(-6.052,1.075)--cycle;
\draw(-6.04,1.048)--(-6.046,1.041);
\filldraw[fill opacity=0.8,fill=gray!20,draw=none](-6.04,1.034)--(-6.07,1.075)--(-6.023,1.073)--cycle;
\draw(-6.07,1.075)--(-6.023,1.073);
\filldraw[fill opacity=0.8,fill=gray!20,draw=none](-6.048,1.038)--(-6.056,1.043)--(-6.056,1.054)--cycle;
\draw(-6.056,1.043)--(-6.056,1.054);
\filldraw[fill opacity=0.8,fill=gray!20,draw=none](-6.046,1.044)--(-6.076,1.074)--(-6.057,1.098)--cycle;
\draw(-6.076,1.074)--(-6.057,1.098);
\filldraw[fill opacity=0.8,fill=gray!20,draw=none](-6.075,1.122)--(-6.051,1.121)--(-6.04,1.074)--(-6.052,1.075)--cycle;
\draw(-6.075,1.122)--(-6.051,1.121);
\draw(-6.04,1.074)--(-6.052,1.075);
\filldraw[fill opacity=0.8,fill=gray!20,draw=none](-6.061,1.122)--(-6.066,1.136)--(-6.062,1.141)--cycle;
\draw(-6.066,1.136)--(-6.062,1.141);
\filldraw[fill opacity=0.8,fill=gray!20,draw=none](-6.045,1.121)--(-6.056,1.127)--(-6.056,1.143)--cycle;
\draw(-6.056,1.127)--(-6.056,1.143);
\filldraw[fill opacity=0.8,fill=gray!20,draw=none](-6.065,1.139)--(-6.059,1.148)--(-6.059,1.144)--(-6.066,1.136)--cycle;
\draw(-6.059,1.144)--(-6.066,1.136);
\filldraw[fill opacity=0.8,fill=gray!20,draw=none](-6.065,1.139)--(-6.056,1.154)--(-6.056,1.127)--cycle;
\draw(-6.056,1.154)--(-6.056,1.127);
\filldraw[fill opacity=0.8,fill=gray!20,draw=none](-6.056,1.154)--(-6.059,1.148)--(-6.059,1.148)--cycle;
\filldraw[fill opacity=0.8,fill=gray!20,draw=none](-6.059,1.148)--(-6.073,1.155)--(-6.056,1.174)--(-6.056,1.154)--cycle;
\draw(-6.056,1.174)--(-6.056,1.154);
\filldraw[fill opacity=0.8,fill=gray!20,draw=none](-6.043,1.177)--(-6.056,1.154)--(-6.059,1.148)--(-6.059,1.165)--(-6.05,1.176)--cycle;
\draw(-6.059,1.165)--(-6.05,1.176);
\filldraw[fill opacity=0.8,fill=gray!20,draw=none](-6.076,1.152)--(-6.067,1.141)--(-6.066,1.136)--(-6.07,1.122)--(-6.109,1.124)--cycle;
\draw(-6.07,1.122)--(-6.109,1.124);
\filldraw[fill opacity=0.8,fill=gray!20,draw=none](-6.066,1.136)--(-6.061,1.122)--(-6.07,1.122)--cycle;
\draw(-6.061,1.122)--(-6.07,1.122);
\filldraw[fill opacity=0.8,fill=gray!20,draw=none](-6.061,1.122)--(-6.061,1.093)--(-6.076,1.074)--(-6.068,1.133)--(-6.066,1.136)--cycle;
\draw(-6.061,1.093)--(-6.076,1.074);
\draw(-6.068,1.133)--(-6.066,1.136);
\filldraw[fill opacity=0.8,fill=gray!20,draw=none](-6.059,1.148)--(-6.065,1.139)--(-6.076,1.152)--(-6.073,1.155)--cycle;
\filldraw[fill opacity=0.8,fill=gray!20,draw=none](-6.059,1.157)--(-6.059,1.148)--(-6.065,1.139)--cycle;
\filldraw[fill opacity=0.8,fill=gray!20,draw=none](-6.065,1.139)--(-6.059,1.157)--(-5.973,1.154)--(-5.968,1.118)--(-6.051,1.121)--cycle;
\draw(-6.059,1.157)--(-5.973,1.154)--(-5.968,1.118)--(-6.051,1.121);
\filldraw[fill opacity=0.8,fill=gray!20,draw=none](-6.011,1.215)--(-6.02,1.212)--(-5.998,1.238)--cycle;
\draw(-6.02,1.212)--(-5.998,1.238);
\filldraw[fill opacity=0.8,fill=gray!20,draw=none](-6.052,1.177)--(-6.02,1.209)--(-6,1.207)--(-6,1.176)--cycle;
\draw(-6,1.207)--(-6,1.176);
\filldraw[fill opacity=0.8,fill=gray!20,draw=none](-5.964,1.057)--(-5.964,1.186)--(-5.944,1.174)--(-5.944,1.058)--cycle;
\draw(-5.944,1.174)--(-5.944,1.058);
\filldraw[fill opacity=0.8,fill=gray!20,draw=none](-5.953,1.178)--(-5.926,1.174)--(-6,1.207)--(-6.02,1.21)--(-6.02,1.201)--(-5.997,1.189)--cycle;
\draw(-5.926,1.174)--(-6,1.207);
\filldraw[fill opacity=0.8,fill=gray!20,draw=none](-6.02,1.209)--(-6,1.229)--(-6,1.207)--cycle;
\draw(-6,1.229)--(-6,1.207);
\filldraw[fill opacity=0.8,fill=gray!20,draw=none](-5.993,1.236)--(-5.964,1.2)--(-6,1.207)--(-6,1.229)--cycle;
\draw(-6,1.207)--(-6,1.229);
\filldraw[fill opacity=0.8,fill=gray!20,draw=none](-5.993,1.236)--(-5.984,1.244)--(-5.965,1.252)--(-5.964,1.252)--(-5.964,1.2)--cycle;
\draw(-5.965,1.252)--(-5.964,1.252);
\filldraw[fill opacity=0.8,fill=gray!20,draw=none](-6.065,1.139)--(-6.067,1.141)--(-6.071,1.156)--(-6.069,1.158)--(-6.059,1.157)--cycle;
\draw(-6.069,1.158)--(-6.059,1.157);
\filldraw[fill opacity=0.8,fill=gray!20,draw=none](-6.067,1.141)--(-6.051,1.121)--(-6.061,1.122)--cycle;
\draw(-6.051,1.121)--(-6.061,1.122);
\filldraw[fill opacity=0.8,fill=gray!20,draw=none](-6.065,1.139)--(-6.066,1.136)--(-6.068,1.133)--cycle;
\draw(-6.066,1.136)--(-6.068,1.133);
\filldraw[fill opacity=0.8,fill=gray!20,draw=none](-6.076,1.152)--(-6.056,1.127)--(-6.056,1.043)--(-6.101,1.043)--(-6.101,1.126)--cycle;
\draw(-6.056,1.127)--(-6.056,1.043);
\draw(-6.101,1.043)--(-6.101,1.126);
\filldraw[fill opacity=0.8,fill=gray!20,draw=none](-6.076,1.152)--(-6.071,1.156)--(-6.067,1.141)--cycle;
\filldraw[fill opacity=0.8,fill=gray!20,draw=none](-6.101,1.099)--(-6.101,1.079)--(-6.129,1.045)--(-6.129,1.069)--cycle;
\draw(-6.101,1.099)--(-6.101,1.079);
\draw(-6.129,1.045)--(-6.129,1.069);
\filldraw[fill opacity=0.8,fill=gray!20,draw=none](-6.045,1.175)--(-5.981,1.172)--(-5.973,1.154)--(-6.069,1.158)--cycle;
\draw(-6.045,1.175)--(-5.981,1.172)--(-5.973,1.154)--(-6.069,1.158);
\filldraw[fill opacity=0.8,fill=gray!20,draw=none](-6.985,1.175)--(-6.977,1.215)--(-6.878,1.212)--(-6.878,1.18)--cycle;
\draw(-6.878,1.212)--(-6.878,1.18)--(-6.985,1.175)--(-6.977,1.215);
\filldraw[fill opacity=0.8,fill=gray!20,draw=none](-6.116,1.117)--(-6.125,1.124)--(-6.109,1.124)--cycle;
\draw(-6.125,1.124)--(-6.109,1.124);
\filldraw[fill opacity=0.8,fill=gray!20,draw=none](-6.056,1.178)--(-6.109,1.124)--(-6.148,1.158)--(-6.116,1.18)--cycle;
\filldraw[fill opacity=0.8,fill=gray!20,draw=none](-6.103,1.104)--(-6.122,1.11)--(-6.109,1.124)--cycle;
\filldraw[fill opacity=0.8,fill=gray!20,draw=none](-6.171,1.151)--(-6.148,1.158)--(-6.109,1.124)--(-6.125,1.124)--cycle;
\draw(-6.109,1.124)--(-6.125,1.124);
\filldraw[fill opacity=0.8,fill=gray!20,draw=none](-6.148,1.158)--(-6.14,1.16)--(-6.081,1.158)--(-6.076,1.152)--(-6.109,1.124)--cycle;
\draw(-6.14,1.16)--(-6.081,1.158);
\filldraw[fill opacity=0.8,fill=gray!20,draw=none](-6.109,1.124)--(-6.122,1.11)--(-6.187,1.131)--(-6.186,1.132)--(-6.148,1.158)--cycle;
\draw(-6.187,1.131)--(-6.186,1.132);
\filldraw[fill opacity=0.8,fill=gray!20,draw=none](-6.116,1.18)--(-6.146,1.159)--(-6.151,1.161)--(-6.163,1.171)--(-6.151,1.18)--cycle;
\draw(-6.163,1.171)--(-6.151,1.18);
\filldraw[fill opacity=0.8,fill=gray!20,draw=none](-6.151,1.161)--(-6.167,1.166)--(-6.164,1.169)--(-6.163,1.171)--cycle;
\draw(-6.167,1.166)--(-6.164,1.169)--(-6.163,1.171);
\filldraw[fill opacity=0.8,fill=gray!20,draw=none](-6.101,1.126)--(-6.101,1.099)--(-6.129,1.069)--(-6.129,1.092)--cycle;
\draw(-6.101,1.126)--(-6.101,1.099);
\draw(-6.129,1.069)--(-6.129,1.092);
\filldraw[fill opacity=0.8,fill=gray!20,draw=none](-6.059,1.157)--(-6.065,1.139)--(-6.068,1.133)--(-6.107,1.086)--(-6.095,1.122)--(-6.059,1.165)--cycle;
\draw(-6.068,1.133)--(-6.107,1.086)--(-6.095,1.122)--(-6.059,1.165);
\filldraw[fill opacity=0.8,fill=gray!20,draw=none](-5.883,1.242)--(-5.86,1.212)--(-5.859,1.21)--(-5.859,1.096)--(-5.894,1.122)--(-5.894,1.246)--cycle;
\draw(-5.859,1.21)--(-5.859,1.096);
\draw(-5.894,1.122)--(-5.894,1.246)--(-5.883,1.242);
\filldraw[fill opacity=0.8,fill=gray!20,draw=none](-5.951,1.244)--(-5.901,1.241)--(-5.883,1.242)--(-5.894,1.246)--(-5.944,1.252)--(-5.965,1.252)--cycle;
\draw(-5.883,1.242)--(-5.894,1.246)--(-5.944,1.252)--(-5.965,1.252);
\filldraw[fill opacity=0.8,fill=gray!20,draw=none](-6.043,1.174)--(-6.021,1.173)--(-5.99,1.171)--(-5.981,1.172)--(-6.045,1.175)--cycle;
\draw(-6.021,1.173)--(-5.99,1.171)--(-5.981,1.172)--(-6.045,1.175);
\filldraw[fill opacity=0.8,fill=gray!20,draw=none](-6.022,1.16)--(-5.997,1.156)--(-5.99,1.171)--(-6.021,1.173)--cycle;
\draw(-5.997,1.156)--(-5.99,1.171)--(-6.021,1.173);
\filldraw[fill opacity=0.8,fill=gray!20,draw=none](-4.552,2.948)--(-6.067,1.146)--(-6.027,1.154)--(-4.499,2.972)--cycle;
\draw(-4.552,2.948)--(-6.067,1.146)--(-6.027,1.154)--(-4.499,2.972);
\filldraw[fill opacity=0.8,fill=gray!20](-4.523,3.013)--(-4.498,3.056)--(-4.536,3.065)--(-4.577,3.026)--cycle;
\filldraw[fill opacity=0.8,fill=gray!20,draw=none](-4.363,2.793)--(-4.364,2.789)--(-4.354,2.804)--cycle;
\draw(-4.364,2.789)--(-4.354,2.804);
\filldraw[fill opacity=0.8,fill=gray!20,draw=none](-4.438,2.653)--(-4.519,2.533)--(-4.507,2.561)--(-4.446,2.652)--cycle;
\draw(-4.438,2.653)--(-4.519,2.533);
\draw(-4.507,2.561)--(-4.446,2.652);
\filldraw[fill opacity=0.8,fill=gray!20,draw=none](-4.352,2.891)--(-4.346,2.888)--(-4.325,2.903)--(-4.336,2.924)--(-4.357,2.896)--cycle;
\draw(-4.336,2.924)--(-4.357,2.896);
\filldraw[fill opacity=0.8,fill=gray!20,draw=none](-4.492,3.011)--(-4.488,3.022)--(-4.499,3.055)--(-4.523,3.013)--cycle;
\draw(-4.499,3.055)--(-4.523,3.013)--(-4.492,3.011);
\filldraw[fill opacity=0.8,fill=gray!20,draw=none](-4.441,3.029)--(-4.499,3.058)--(-4.496,3.026)--(-4.432,2.994)--cycle;
\draw(-4.441,3.029)--(-4.499,3.058);
\draw(-4.496,3.026)--(-4.432,2.994);
\filldraw[fill opacity=0.8,fill=gray!20,draw=none](-4.441,3.029)--(-4.432,2.994)--(-4.416,2.986)--cycle;
\draw(-4.432,2.994)--(-4.416,2.986);
\filldraw[fill opacity=0.8,fill=gray!20,draw=none](-2.85,7.72)--(-2.876,7.762)--(-2.864,7.764)--cycle;
\draw(-2.85,7.72)--(-2.876,7.762)--(-2.864,7.764);
\filldraw[fill opacity=0.8,fill=gray!20](-7.978,1.032)--(-7.974,1.052)--(-7.974,1.052)--(-8.006,1.034)--cycle;
\filldraw[fill opacity=0.8,fill=gray!20](-7.949,1.033)--(-7.974,1.052)--(-7.974,1.052)--(-7.978,1.032)--cycle;
\filldraw[fill opacity=0.8,fill=gray!20](-8.006,1.034)--(-7.974,1.052)--(-7.974,1.052)--(-8.025,1.039)--cycle;
\filldraw[fill opacity=0.8,fill=gray!20](-7.981,.628)--(-7.984,.652)--(-8.06,.658)--(-8.035,.632)--cycle;
\filldraw[fill opacity=0.8,fill=gray!20](-7.927,1.038)--(-7.974,1.052)--(-7.974,1.052)--(-7.949,1.033)--cycle;
\filldraw[fill opacity=0.8,fill=gray!20](-7.918,1.044)--(-7.974,1.052)--(-7.974,1.052)--(-7.927,1.038)--cycle;
\filldraw[fill opacity=0.8,fill=gray!20](-7.865,1.022)--(-7.918,1.044)--(-7.927,1.038)--(-7.883,1.01)--cycle;
\filldraw[fill opacity=0.8,fill=gray!20,draw=none](-4.445,2.944)--(-4.461,2.964)--(-4.471,2.948)--cycle;
\draw(-4.461,2.964)--(-4.471,2.948);
\filldraw[fill opacity=0.8,fill=gray!20,draw=none](-4.456,2.921)--(-4.46,2.924)--(-4.47,2.93)--cycle;
\draw(-4.456,2.921)--(-4.46,2.924)--(-4.47,2.93);
\filldraw[fill opacity=0.8,fill=gray!20,draw=none](-4.47,2.93)--(-4.46,2.924)--(-4.451,2.937)--cycle;
\draw(-4.47,2.93)--(-4.46,2.924)--(-4.451,2.937);
\filldraw[fill opacity=0.8,fill=gray!20,draw=none](-4.445,2.944)--(-4.471,2.948)--(-4.526,2.865)--(-4.51,2.857)--(-4.443,2.901)--(-4.429,2.922)--cycle;
\draw(-4.471,2.948)--(-4.526,2.865);
\draw(-4.443,2.901)--(-4.429,2.922);
\filldraw[fill opacity=0.8,fill=gray!20,draw=none](-4.447,2.898)--(-4.484,2.917)--(-4.493,2.917)--(-4.528,2.906)--(-4.542,2.889)--(-4.521,2.879)--cycle;
\draw(-4.447,2.898)--(-4.484,2.917);
\draw(-4.542,2.889)--(-4.521,2.879);
\filldraw[fill opacity=0.8,fill=gray!20,draw=none](-4.451,2.788)--(-4.451,2.812)--(-4.482,2.843)--(-4.487,2.845)--(-4.559,2.85)--(-4.555,2.796)--cycle;
\draw(-4.487,2.845)--(-4.559,2.85)--(-4.555,2.796)--(-4.451,2.788)--(-4.451,2.812);
\filldraw[fill opacity=0.8,fill=gray!20,draw=none](-4.563,2.866)--(-4.571,2.853)--(-4.559,2.85)--(-4.558,2.863)--cycle;
\draw(-4.571,2.853)--(-4.559,2.85)--(-4.558,2.863);
\filldraw[fill opacity=0.8,fill=gray!20,draw=none](-4.554,2.85)--(-4.547,2.864)--(-4.558,2.863)--(-4.559,2.85)--cycle;
\draw(-4.558,2.863)--(-4.559,2.85)--(-4.554,2.85);
\filldraw[fill opacity=0.8,fill=gray!20,draw=none](-4.554,2.861)--(-4.571,2.871)--(-4.579,2.86)--cycle;
\draw(-4.571,2.871)--(-4.579,2.86);
\filldraw[fill opacity=0.8,fill=gray!20,draw=none](-4.58,2.86)--(-4.578,2.861)--(-4.577,2.861)--cycle;
\draw(-4.578,2.861)--(-4.577,2.861);
\filldraw[fill opacity=0.8,fill=gray!20,draw=none](-4.563,2.866)--(-4.555,2.877)--(-4.567,2.877)--(-4.583,2.861)--(-4.581,2.86)--cycle;
\draw(-4.583,2.861)--(-4.581,2.86);
\filldraw[fill opacity=0.8,fill=gray!20,draw=none](-4.52,2.878)--(-4.521,2.879)--(-4.551,2.878)--(-4.557,2.874)--(-4.563,2.866)--cycle;
\draw(-4.52,2.878)--(-4.521,2.879);
\filldraw[fill opacity=0.8,fill=gray!20,draw=none](-4.563,2.866)--(-4.558,2.863)--(-4.526,2.865)--(-4.517,2.879)--cycle;
\draw(-4.526,2.865)--(-4.517,2.879);
\filldraw[fill opacity=0.8,fill=gray!20,draw=none](-4.563,2.866)--(-4.558,2.863)--(-4.557,2.874)--cycle;
\draw(-4.558,2.863)--(-4.557,2.874);
\filldraw[fill opacity=0.8,fill=gray!20,draw=none](-4.543,2.873)--(-4.554,2.85)--(-4.487,2.845)--cycle;
\draw(-4.554,2.85)--(-4.487,2.845);
\filldraw[fill opacity=0.8,fill=gray!20,draw=none](-4.554,2.861)--(-4.579,2.86)--(-4.987,2.248)--(-5.003,2.202)--(-5.008,2.144)--(-4.537,2.85)--cycle;
\draw(-4.579,2.86)--(-4.987,2.248);
\draw(-5.008,2.144)--(-4.537,2.85);
\filldraw[fill opacity=0.8,fill=gray!20,draw=none](-4.551,2.878)--(-4.521,2.879)--(-4.537,2.887)--cycle;
\draw(-4.521,2.879)--(-4.537,2.887);
\filldraw[fill opacity=0.8,fill=gray!20,draw=none](-4.558,2.863)--(-4.537,2.85)--(-4.526,2.865)--cycle;
\draw(-4.537,2.85)--(-4.526,2.865);
\filldraw[fill opacity=0.8,fill=gray!20,draw=none](-4.547,2.864)--(-4.537,2.887)--(-4.557,2.874)--(-4.558,2.863)--cycle;
\draw(-4.557,2.874)--(-4.558,2.863);
\filldraw[fill opacity=0.8,fill=gray!20,draw=none](-4.551,2.878)--(-4.555,2.877)--(-4.557,2.874)--cycle;
\filldraw[fill opacity=0.8,fill=gray!20,draw=none](-4.528,2.905)--(-4.528,2.906)--(-4.552,2.908)--(-4.555,2.907)--cycle;
\draw(-4.555,2.907)--(-4.528,2.905);
\filldraw[fill opacity=0.8,fill=gray!20,draw=none](-4.445,2.944)--(-4.429,2.922)--(-4.417,2.939)--cycle;
\draw(-4.429,2.922)--(-4.417,2.939);
\filldraw[fill opacity=0.8,fill=gray!20,draw=none](-4.435,2.939)--(-4.446,2.962)--(-4.452,2.965)--(-4.463,2.952)--cycle;
\draw(-4.452,2.965)--(-4.463,2.952);
\filldraw[fill opacity=0.8,fill=gray!20,draw=none](-4.52,2.878)--(-4.517,2.879)--(-4.516,2.88)--cycle;
\draw(-4.517,2.879)--(-4.516,2.88);
\filldraw[fill opacity=0.8,fill=gray!20,draw=none](-4.516,2.88)--(-4.521,2.879)--(-4.52,2.878)--cycle;
\draw(-4.521,2.879)--(-4.52,2.878);
\filldraw[fill opacity=0.8,fill=gray!20,draw=none](-4.543,2.873)--(-4.51,2.857)--(-4.512,2.886)--(-4.525,2.905)--(-4.528,2.905)--cycle;
\draw(-4.525,2.905)--(-4.528,2.905);
\filldraw[fill opacity=0.8,fill=gray!20,draw=none](-4.528,2.906)--(-4.557,2.897)--(-4.542,2.889)--cycle;
\draw(-4.557,2.897)--(-4.542,2.889);
\filldraw[fill opacity=0.8,fill=gray!20,draw=none](-4.528,2.905)--(-4.503,2.903)--(-4.504,2.904)--(-4.528,2.906)--cycle;
\draw(-4.528,2.905)--(-4.503,2.903);
\filldraw[fill opacity=0.8,fill=gray!20,draw=none](-4.783,2.193)--(-6.016,.347)--(-6.019,.349)--(-6.023,.354)--(-6.029,.4)--(-4.666,2.44)--cycle;
\draw(-4.783,2.193)--(-6.016,.347);
\draw(-6.029,.4)--(-4.666,2.44);
\filldraw[fill opacity=0.8,fill=gray!20,draw=none](-4.535,2.58)--(-4.737,2.339)--(-4.796,2.281)--(-4.904,2.181)--(-4.524,2.632)--cycle;
\draw(-4.535,2.58)--(-4.737,2.339);
\draw(-4.904,2.181)--(-4.524,2.632);
\filldraw[fill opacity=0.8,fill=gray!20,draw=none](-4.41,2.919)--(-4.404,2.919)--(-4.417,2.939)--(-4.427,2.924)--cycle;
\draw(-4.417,2.939)--(-4.427,2.924);
\filldraw[fill opacity=0.8,fill=gray!20,draw=none](-4.398,2.911)--(-4.402,2.916)--(-4.41,2.919)--(-4.43,2.92)--(-4.443,2.901)--cycle;
\draw(-4.43,2.92)--(-4.443,2.901);
\filldraw[fill opacity=0.8,fill=gray!20,draw=none](-4.41,2.919)--(-4.427,2.924)--(-4.43,2.92)--cycle;
\draw(-4.427,2.924)--(-4.43,2.92);
\filldraw[fill opacity=0.8,fill=gray!20,draw=none](-4.414,2.934)--(-4.484,2.917)--(-4.433,2.891)--cycle;
\draw(-4.484,2.917)--(-4.433,2.891);
\filldraw[fill opacity=0.8,fill=gray!20,draw=none](-4.398,2.911)--(-4.443,2.901)--(-4.493,2.826)--(-4.443,2.809)--(-4.387,2.893)--cycle;
\draw(-4.443,2.901)--(-4.493,2.826);
\draw(-4.443,2.809)--(-4.387,2.893);
\filldraw[fill opacity=0.8,fill=gray!20,draw=none](-4.513,2.855)--(-4.482,2.843)--(-4.443,2.901)--cycle;
\draw(-4.482,2.843)--(-4.443,2.901);
\filldraw[fill opacity=0.8,fill=gray!20,draw=none](-4.483,2.845)--(-4.451,2.842)--(-4.451,2.862)--(-4.461,2.877)--(-4.503,2.903)--(-4.525,2.905)--cycle;
\draw(-4.483,2.845)--(-4.451,2.842)--(-4.451,2.862);
\draw(-4.503,2.903)--(-4.525,2.905);
\filldraw[fill opacity=0.8,fill=gray!20,draw=none](-4.879,2.455)--(-4.805,2.587)--(-4.764,2.636)--cycle;
\draw(-4.805,2.587)--(-4.764,2.636);
\filldraw[fill opacity=0.8,fill=gray!20,draw=none](-6.036,.92)--(-6.056,.929)--(-6.002,.942)--cycle;
\draw(-6.036,.92)--(-6.056,.929);
\filldraw[fill opacity=0.8,fill=gray!20,draw=none](-5.994,1.012)--(-5.952,.996)--(-5.984,.958)--(-6.003,.973)--(-6.008,1.007)--(-6.003,1.012)--cycle;
\draw(-5.952,.996)--(-5.984,.958);
\draw(-6.008,1.007)--(-6.003,1.012);
\filldraw[fill opacity=0.8,fill=gray!20,draw=none](-5.968,1.016)--(-5.964,1.015)--(-5.964,.998)--(-6,1.012)--(-6,1.016)--cycle;
\draw(-6,1.012)--(-6,1.016);
\filldraw[fill opacity=0.8,fill=gray!20,draw=none](-6.003,.973)--(-6.023,.989)--(-6.008,1.007)--cycle;
\draw(-6.023,.989)--(-6.008,1.007);
\filldraw[fill opacity=0.8,fill=gray!20,draw=none](-5.901,.881)--(-5.925,.871)--(-6.036,.92)--(-6.018,.932)--cycle;
\draw(-5.925,.871)--(-6.036,.92);
\filldraw[fill opacity=0.8,fill=gray!20,draw=none](-6.04,1.034)--(-6.023,1.073)--(-5.967,1.071)--(-5.968,1.02)--(-6.032,1.022)--cycle;
\draw(-6.023,1.073)--(-5.967,1.071)--(-5.968,1.02)--(-6.032,1.022);
\filldraw[fill opacity=0.8,fill=gray!20,draw=none](-6.032,1.022)--(-5.968,1.02)--(-5.973,.972)--(-6.018,.974)--cycle;
\draw(-6.032,1.022)--(-5.968,1.02)--(-5.973,.972)--(-6.018,.974);
\filldraw[fill opacity=0.8,fill=gray!20,draw=none](-6.046,1.044)--(-6.046,1.041)--(-6.079,1.001)--(-6.102,1.044)--(-6.076,1.074)--cycle;
\draw(-6.046,1.041)--(-6.079,1.001)--(-6.102,1.044)--(-6.076,1.074);
\filldraw[fill opacity=0.8,fill=gray!20,draw=none](-6.051,1.121)--(-6.045,1.121)--(-6.023,1.073)--(-6.04,1.074)--cycle;
\draw(-6.051,1.121)--(-6.045,1.121);
\draw(-6.023,1.073)--(-6.04,1.074);
\filldraw[fill opacity=0.8,fill=gray!20,draw=none](-6.011,1.101)--(-6.04,1.034)--(-6.048,1.038)--(-6.056,1.054)--(-6.056,1.127)--cycle;
\draw(-6.056,1.054)--(-6.056,1.127);
\filldraw[fill opacity=0.8,fill=gray!20,draw=none](-6.045,1.121)--(-5.968,1.118)--(-5.967,1.071)--(-6.023,1.073)--cycle;
\draw(-6.045,1.121)--(-5.968,1.118)--(-5.967,1.071)--(-6.023,1.073);
\filldraw[fill opacity=0.8,fill=gray!20,draw=none](-6.083,.92)--(-6.056,.903)--(-6.056,.893)--(-6.101,.885)--(-6.101,.921)--cycle;
\draw(-6.056,.903)--(-6.056,.893)--(-6.101,.885)--(-6.101,.921);
\filldraw[fill opacity=0.8,fill=gray!20,draw=none](-6.083,.92)--(-6.056,.92)--(-6.056,.903)--cycle;
\draw(-6.056,.92)--(-6.056,.903);
\filldraw[fill opacity=0.8,fill=gray!20,draw=none](-5.979,.943)--(-5.993,.942)--(-5.996,.944)--(-5.97,.975)--cycle;
\draw(-5.996,.944)--(-5.97,.975);
\filldraw[fill opacity=0.8,fill=gray!20,draw=none](-6.003,.973)--(-5.984,.958)--(-5.999,.94)--cycle;
\draw(-5.984,.958)--(-5.999,.94);
\filldraw[fill opacity=0.8,fill=gray!20,draw=none](-5.979,.943)--(-5.981,.935)--(-6.03,.937)--cycle;
\draw(-5.979,.943)--(-5.981,.935)--(-6.03,.937);
\filldraw[fill opacity=0.8,fill=gray!20,draw=none](-5.989,.938)--(-5.999,.94)--(-5.996,.944)--cycle;
\draw(-5.989,.938)--(-5.999,.94)--(-5.996,.944);
\filldraw[fill opacity=0.8,fill=gray!20,draw=none](-6.018,.974)--(-5.973,.972)--(-5.979,.943)--(-6.03,.937)--cycle;
\draw(-6.018,.974)--(-5.973,.972)--(-5.979,.943);
\filldraw[fill opacity=0.8,fill=gray!20,draw=none](-6.003,.973)--(-5.999,.94)--(-6.043,.965)--(-6.023,.989)--cycle;
\draw(-5.999,.94)--(-6.043,.965)--(-6.023,.989);
\filldraw[fill opacity=0.8,fill=gray!20,draw=none](-6.056,.929)--(-6.056,.92)--(-6.083,.92)--(-6.101,.932)--cycle;
\draw(-6.056,.929)--(-6.056,.92);
\filldraw[fill opacity=0.8,fill=gray!20,draw=none](-6.069,.918)--(-6.101,.932)--(-6.056,.929)--(-6.031,.918)--cycle;
\draw(-6.069,.918)--(-6.101,.932);
\draw(-6.056,.929)--(-6.031,.918);
\filldraw[fill opacity=0.8,fill=gray!20,draw=none](-6.032,.905)--(-6.049,.909)--(-6.069,.918)--(-6.031,.918)--(-5.983,.897)--cycle;
\draw(-6.049,.909)--(-6.069,.918);
\draw(-6.031,.918)--(-5.983,.897);
\filldraw[fill opacity=0.8,fill=gray!20,draw=none](-6.112,.94)--(-5.981,.935)--(-5.99,.914)--(-6.065,.917)--cycle;
\draw(-6.112,.94)--(-5.981,.935)--(-5.99,.914)--(-6.065,.917);
\filldraw[fill opacity=0.8,fill=gray!20,draw=none](-5.979,.943)--(-5.981,.937)--(-5.989,.938)--(-5.993,.942)--cycle;
\draw(-5.981,.937)--(-5.989,.938);
\filldraw[fill opacity=0.8,fill=gray!20,draw=none](-5.981,.937)--(-5.979,.943)--(-5.973,.972)--(-5.968,1.02)--(-5.967,1.071)--(-5.968,1.118)--(-5.973,1.154)--(-5.981,1.172)--(-5.99,1.171)--(-5.997,1.156)--(-5.999,1.149)--cycle;
\draw(-5.979,.943)--(-5.973,.972)--(-5.968,1.02)--(-5.967,1.071)--(-5.968,1.118)--(-5.973,1.154)--(-5.981,1.172)--(-5.99,1.171)--(-5.997,1.156);
\filldraw[fill opacity=0.8,fill=gray!20,draw=none](-4.494,2.978)--(-6.027,1.154)--(-5.981,1.145)--(-4.447,2.971)--cycle;
\draw(-4.494,2.978)--(-6.027,1.154)--(-5.981,1.145)--(-4.447,2.971);
\filldraw[fill opacity=0.8,fill=gray!20,draw=none](-4.276,2.977)--(-4.271,2.988)--(-4.304,3.026)--(-4.308,3.023)--(-4.279,2.974)--cycle;
\draw(-4.304,3.026)--(-4.308,3.023)--(-4.279,2.974)--(-4.276,2.977);
\filldraw[fill opacity=0.8,fill=gray!20,draw=none](-4.559,2.942)--(-4.585,2.912)--(-4.575,2.921)--(-4.554,2.946)--cycle;
\draw(-4.575,2.921)--(-4.554,2.946);
\filldraw[fill opacity=0.8,fill=gray!20,draw=none](-8.206,1.666)--(-8.206,1.667)--(-8.226,1.688)--(-8.231,1.675)--cycle;
\draw(-8.206,1.666)--(-8.206,1.667)--(-8.226,1.688)--(-8.231,1.675);
\filldraw[fill opacity=0.8,fill=gray!20](-7.845,.667)--(-7.816,.707)--(-7.889,.693)--(-7.905,.656)--cycle;
\filldraw[fill opacity=0.8,fill=gray!20,draw=none](-4.444,3.061)--(-4.441,3.085)--(-4.469,3.087)--(-4.484,3.071)--cycle;
\draw(-4.444,3.061)--(-4.441,3.085)--(-4.469,3.087)--(-4.484,3.071);
\filldraw[fill opacity=0.8,fill=gray!20,draw=none](-4.347,2.838)--(-4.347,2.856)--(-4.349,2.849)--(-4.35,2.827)--cycle;
\filldraw[fill opacity=0.8,fill=gray!20,draw=none](-4.351,2.841)--(-4.348,2.857)--(-4.349,2.868)--(-4.351,2.866)--(-4.353,2.858)--cycle;
\draw(-4.349,2.868)--(-4.351,2.866);
\filldraw[fill opacity=0.8,fill=gray!20,draw=none](-4.351,2.841)--(-4.353,2.858)--(-4.354,2.852)--(-4.354,2.832)--cycle;
\filldraw[fill opacity=0.8,fill=gray!20,draw=none](-4.354,2.832)--(-4.354,2.861)--(-4.388,2.811)--(-4.379,2.767)--(-4.364,2.789)--cycle;
\draw(-4.354,2.861)--(-4.388,2.811);
\draw(-4.379,2.767)--(-4.364,2.789);
\filldraw[fill opacity=0.8,fill=gray!20,draw=none](-4.351,2.866)--(-4.354,2.861)--(-4.354,2.852)--cycle;
\draw(-4.351,2.866)--(-4.354,2.861);
\filldraw[fill opacity=0.8,fill=gray!20,draw=none](-4.359,2.87)--(-4.381,2.889)--(-4.388,2.891)--(-4.443,2.809)--(-4.427,2.789)--(-4.411,2.776)--(-4.355,2.86)--cycle;
\draw(-4.388,2.891)--(-4.443,2.809);
\draw(-4.411,2.776)--(-4.355,2.86);
\filldraw[fill opacity=0.8,fill=gray!20,draw=none](-4.35,2.827)--(-4.348,2.863)--(-4.382,2.823)--(-4.377,2.78)--(-4.355,2.805)--cycle;
\draw(-4.348,2.863)--(-4.382,2.823);
\draw(-4.377,2.78)--(-4.355,2.805);
\filldraw[fill opacity=0.8,fill=gray!20,draw=none](-4.478,2.976)--(-4.477,2.977)--(-4.48,2.978)--cycle;
\draw(-4.478,2.976)--(-4.477,2.977);
\filldraw[fill opacity=0.8,fill=gray!20](-8.206,1.667)--(-8.175,1.715)--(-8.191,1.732)--(-8.226,1.688)--cycle;
\filldraw[fill opacity=0.8,fill=gray!20](-7.817,1.576)--(-7.824,1.631)--(-7.859,1.609)--(-7.853,1.552)--cycle;
\filldraw[fill opacity=0.8,fill=gray!20](-2.612,7.988)--(-2.631,8.026)--(-2.578,8.013)--(-2.547,7.972)--cycle;
\filldraw[fill opacity=0.8,fill=gray!20,draw=none](-4.611,2.888)--(-4.618,2.88)--(-4.597,2.898)--(-4.576,2.921)--cycle;
\draw(-4.611,2.888)--(-4.618,2.88);
\filldraw[fill opacity=0.8,fill=gray!20,draw=none](-4.585,2.912)--(-4.597,2.898)--(-4.589,2.905)--(-4.575,2.921)--cycle;
\draw(-4.589,2.905)--(-4.575,2.921);
\filldraw[fill opacity=0.8,fill=gray!20,draw=none](-4.632,2.868)--(-4.589,2.905)--(-4.63,2.902)--(-4.634,2.869)--cycle;
\draw(-4.63,2.902)--(-4.634,2.869)--(-4.632,2.868);
\filldraw[fill opacity=0.8,fill=gray!20](-4.389,3.054)--(-4.412,3.086)--(-4.441,3.085)--(-4.444,3.052)--cycle;
\filldraw[fill opacity=0.8,fill=gray!20,draw=none](-4.345,2.867)--(-4.343,2.868)--(-4.343,2.869)--(-4.344,2.867)--cycle;
\draw(-4.343,2.869)--(-4.344,2.867);
\filldraw[fill opacity=0.8,fill=gray!20,draw=none](-4.343,2.869)--(-4.343,2.872)--(-4.345,2.871)--cycle;
\draw(-4.343,2.872)--(-4.345,2.871);
\filldraw[fill opacity=0.8,fill=gray!20,draw=none](-4.346,2.871)--(-4.344,2.867)--(-4.343,2.869)--cycle;
\draw(-4.344,2.867)--(-4.343,2.869);
\filldraw[fill opacity=0.8,fill=gray!20](-8.095,.797)--(-8.091,.854)--(-8.164,.872)--(-8.171,.816)--cycle;
\filldraw[fill opacity=0.8,fill=gray!20](-8.348,.764)--(-8.358,.804)--(-8.303,.79)--(-8.287,.749)--cycle;
\filldraw[fill opacity=0.8,fill=gray!20,draw=none](-8.278,.741)--(-8.277,.739)--(-8.287,.749)--(-8.303,.79)--(-8.299,.786)--cycle;
\draw(-8.277,.739)--(-8.287,.749)--(-8.303,.79)--(-8.299,.786);
\filldraw[fill opacity=0.8,fill=gray!20,draw=none](-8.3,.787)--(-8.299,.786)--(-8.303,.79)--(-8.314,.805)--cycle;
\draw(-8.299,.786)--(-8.303,.79)--(-8.314,.805);
\filldraw[fill opacity=0.8,fill=gray!20,draw=none](-8.379,.776)--(-8.37,.765)--(-8.374,.781)--cycle;
\draw(-8.37,.765)--(-8.374,.781);
\filldraw[fill opacity=0.8,fill=gray!20](-7.933,.95)--(-8.403,.798)--(-8.368,.765)--(-7.897,.917)--cycle;
\filldraw[fill opacity=0.8,fill=gray!20](-7.925,.63)--(-7.905,.656)--(-7.984,.652)--(-7.981,.628)--cycle;
\filldraw[fill opacity=0.8,fill=gray!20](-8.091,.743)--(-8.095,.797)--(-8.171,.816)--(-8.164,.761)--cycle;
\filldraw[fill opacity=0.8,fill=gray!20](-7.987,.736)--(-7.988,.79)--(-8.095,.797)--(-8.091,.743)--cycle;
\filldraw[fill opacity=0.8,fill=gray!20](-8.348,.672)--(-8.344,.719)--(-8.281,.704)--(-8.287,.657)--cycle;
\filldraw[fill opacity=0.8,fill=gray!20,draw=none](-8.277,.646)--(-8.287,.657)--(-8.281,.704)--(-8.27,.691)--cycle;
\draw(-8.277,.646)--(-8.287,.657)--(-8.281,.704)--(-8.27,.691);
\filldraw[fill opacity=0.8,fill=gray!20](-7.865,.821)--(-8.336,.67)--(-8.344,.622)--(-7.874,.773)--cycle;
\filldraw[fill opacity=0.8,fill=gray!20](-8.093,1.786)--(-8.036,1.793)--(-8.036,1.793)--(-8.083,1.792)--cycle;
\filldraw[fill opacity=0.8,fill=gray!20,draw=none](-4.346,2.871)--(-4.354,2.88)--(-4.348,2.863)--(-4.344,2.867)--cycle;
\draw(-4.348,2.863)--(-4.344,2.867);
\filldraw[fill opacity=0.8,fill=gray!20,draw=none](-4.345,2.867)--(-4.344,2.867)--(-4.348,2.863)--cycle;
\draw(-4.344,2.867)--(-4.348,2.863);
\filldraw[fill opacity=0.8,fill=gray!20,draw=none](-4.345,2.867)--(-4.348,2.863)--(-4.349,2.849)--cycle;
\filldraw[fill opacity=0.8,fill=gray!20,draw=none](-4.346,2.871)--(-4.35,2.885)--(-4.359,2.892)--(-4.354,2.88)--cycle;
\filldraw[fill opacity=0.8,fill=gray!20,draw=none](-4.346,2.888)--(-4.352,2.891)--(-4.349,2.886)--cycle;
\filldraw[fill opacity=0.8,fill=gray!20,draw=none](-4.639,2.855)--(-4.64,2.856)--(-4.649,2.861)--(-4.684,2.866)--(-4.651,2.849)--cycle;
\draw(-4.64,2.856)--(-4.649,2.861);
\draw(-4.684,2.866)--(-4.651,2.849);
\filldraw[fill opacity=0.8,fill=gray!20](-2.907,7.742)--(-2.929,7.794)--(-2.894,7.816)--(-2.876,7.762)--cycle;
\filldraw[fill opacity=0.8,fill=gray!20](-2.929,7.794)--(-2.936,7.849)--(-2.9,7.873)--(-2.894,7.816)--cycle;
\filldraw[fill opacity=0.8,fill=gray!20,draw=none](-8.124,1.38)--(-8.096,1.373)--(-8.113,1.39)--cycle;
\draw(-8.124,1.38)--(-8.096,1.373)--(-8.113,1.39);
\filldraw[fill opacity=0.8,fill=gray!20,draw=none](-4.352,2.87)--(-4.351,2.866)--(-4.349,2.868)--cycle;
\draw(-4.351,2.866)--(-4.349,2.868);
\filldraw[fill opacity=0.8,fill=gray!20](-7.944,1.38)--(-7.906,1.408)--(-7.966,1.397)--(-7.987,1.372)--cycle;
\filldraw[fill opacity=0.8,fill=gray!20](-2.872,7.697)--(-2.907,7.742)--(-2.876,7.762)--(-2.846,7.714)--cycle;
\filldraw[fill opacity=0.8,fill=gray!20](-2.658,7.821)--(-2.618,7.836)--(-2.593,7.853)--(-2.587,7.868)--(-2.601,7.881)--(-2.632,7.888)--(-2.676,7.889)--(-2.727,7.883)--(-2.776,7.872)--(-2.816,7.857)--(-2.841,7.84)--(-2.847,7.824)--(-2.833,7.812)--(-2.802,7.805)--(-2.757,7.804)--(-2.707,7.809)--cycle;
\filldraw[fill opacity=0.8,fill=gray!20,draw=none](-4.35,2.885)--(-4.349,2.886)--(-4.357,2.896)--(-4.359,2.892)--cycle;
\draw(-4.357,2.896)--(-4.359,2.892);
\filldraw[fill opacity=0.8,fill=gray!20,draw=none](-4.555,2.907)--(-4.628,2.925)--(-4.63,2.902)--cycle;
\draw(-4.555,2.907)--(-4.628,2.925)--(-4.63,2.902);
\filldraw[fill opacity=0.8,fill=gray!20,draw=none](-8.17,.773)--(-8.168,.765)--(-8.164,.761)--(-8.171,.816)--(-8.178,.823)--cycle;
\draw(-8.168,.765)--(-8.164,.761)--(-8.171,.816)--(-8.178,.823);
\filldraw[fill opacity=0.8,fill=gray!20](-4.543,2.963)--(-4.523,3.013)--(-4.577,3.026)--(-4.608,2.979)--cycle;
\filldraw[fill opacity=0.8,fill=gray!20](-4.261,2.809)--(-4.255,2.864)--(-4.34,2.847)--(-4.343,2.793)--cycle;
\filldraw[fill opacity=0.8,fill=gray!20,draw=none](-4.404,2.707)--(-4.404,2.707)--(-4.368,2.708)--(-4.361,2.725)--cycle;
\draw(-4.404,2.707)--(-4.368,2.708)--(-4.361,2.725);
\filldraw[fill opacity=0.8,fill=gray!20,draw=none](-4.352,2.87)--(-4.362,2.875)--(-4.355,2.86)--(-4.351,2.866)--cycle;
\draw(-4.355,2.86)--(-4.351,2.866);
\filldraw[fill opacity=0.8,fill=gray!20,draw=none](-4.364,2.792)--(-4.361,2.799)--(-4.377,2.78)--(-4.38,2.771)--cycle;
\draw(-4.361,2.799)--(-4.377,2.78);
\filldraw[fill opacity=0.8,fill=gray!20,draw=none](-4.352,2.87)--(-4.357,2.881)--(-4.365,2.884)--(-4.362,2.875)--cycle;
\filldraw[fill opacity=0.8,fill=gray!20,draw=none](-8.156,1.432)--(-8.133,1.422)--(-8.141,1.437)--(-8.173,1.445)--cycle;
\draw(-8.133,1.422)--(-8.141,1.437)--(-8.173,1.445);
\filldraw[fill opacity=0.8,fill=gray!20,draw=none](-8.006,1.479)--(-8.001,1.485)--(-7.992,1.533)--(-8.049,1.531)--(-8.049,1.477)--cycle;
\draw(-7.992,1.533)--(-8.049,1.531)--(-8.049,1.477)--(-8.006,1.479);
\filldraw[fill opacity=0.8,fill=gray!20,draw=none](-8.304,1.326)--(-8.358,1.34)--(-8.348,1.386)--(-8.287,1.371)--(-8.303,1.328)--cycle;
\draw(-8.304,1.326)--(-8.358,1.34)--(-8.348,1.386)--(-8.287,1.371)--(-8.303,1.328);
\filldraw[fill opacity=0.8,fill=gray!20,draw=none](-8.305,1.325)--(-8.346,1.3)--(-8.373,1.298)--(-8.358,1.34)--(-8.304,1.326)--cycle;
\draw(-8.373,1.298)--(-8.358,1.34)--(-8.304,1.326);
\filldraw[fill opacity=0.8,fill=gray!20,draw=none](-8.283,1.362)--(-8.303,1.328)--(-8.287,1.371)--(-8.282,1.365)--cycle;
\draw(-8.303,1.328)--(-8.287,1.371)--(-8.282,1.365);
\filldraw[fill opacity=0.8,fill=gray!20](-7.927,1.517)--(-8.336,1.339)--(-8.362,1.3)--(-7.953,1.479)--cycle;
\filldraw[fill opacity=0.8,fill=gray!20,draw=none](-4.387,3.073)--(-4.377,3.057)--(-4.355,3.061)--(-4.36,3.064)--cycle;
\draw(-4.377,3.057)--(-4.355,3.061);
\filldraw[fill opacity=0.8,fill=gray!20](-7.985,1.791)--(-8.036,1.793)--(-8.036,1.793)--(-7.979,1.785)--cycle;
\filldraw[fill opacity=0.8,fill=gray!20,draw=none](-4.359,2.87)--(-4.365,2.884)--(-4.381,2.889)--cycle;
\filldraw[fill opacity=0.8,fill=gray!20,draw=none](-4.365,2.884)--(-4.377,2.908)--(-4.388,2.891)--cycle;
\draw(-4.377,2.908)--(-4.388,2.891);
\filldraw[fill opacity=0.8,fill=gray!20,draw=none](-4.446,2.677)--(-4.553,2.549)--(-4.535,2.58)--(-4.455,2.676)--cycle;
\draw(-4.446,2.677)--(-4.553,2.549);
\draw(-4.535,2.58)--(-4.455,2.676);
\filldraw[fill opacity=0.8,fill=gray!20,draw=none](-4.402,2.744)--(-4.415,2.79)--(-4.451,2.788)--(-4.449,2.742)--cycle;
\draw(-4.415,2.79)--(-4.451,2.788)--(-4.449,2.742)--(-4.402,2.744);
\filldraw[fill opacity=0.8,fill=gray!20,draw=none](-4.388,2.811)--(-4.414,2.772)--(-4.413,2.762)--(-4.398,2.738)--(-4.379,2.767)--cycle;
\draw(-4.388,2.811)--(-4.414,2.772);
\draw(-4.398,2.738)--(-4.379,2.767);
\filldraw[fill opacity=0.8,fill=gray!20,draw=none](-4.402,2.744)--(-4.395,2.758)--(-4.4,2.791)--(-4.415,2.79)--cycle;
\draw(-4.4,2.791)--(-4.415,2.79);
\filldraw[fill opacity=0.8,fill=gray!20,draw=none](-4.427,2.789)--(-4.414,2.772)--(-4.411,2.776)--cycle;
\draw(-4.414,2.772)--(-4.411,2.776);
\filldraw[fill opacity=0.8,fill=gray!20,draw=none](-4.4,2.791)--(-4.4,2.801)--(-4.424,2.842)--(-4.427,2.843)--(-4.451,2.842)--(-4.451,2.788)--cycle;
\draw(-4.427,2.843)--(-4.451,2.842)--(-4.451,2.788)--(-4.4,2.791);
\filldraw[fill opacity=0.8,fill=gray!20,draw=none](-4.552,2.55)--(-4.737,2.339)--(-4.535,2.58)--cycle;
\draw(-4.737,2.339)--(-4.535,2.58);
\filldraw[fill opacity=0.8,fill=gray!20,draw=none](-4.427,2.789)--(-4.446,2.804)--(-4.796,2.281)--(-4.633,2.444)--(-4.414,2.772)--cycle;
\draw(-4.446,2.804)--(-4.796,2.281);
\draw(-4.633,2.444)--(-4.414,2.772);
\filldraw[fill opacity=0.8,fill=gray!20,draw=none](-4.443,2.809)--(-4.446,2.804)--(-4.427,2.789)--cycle;
\draw(-4.443,2.809)--(-4.446,2.804);
\filldraw[fill opacity=0.8,fill=gray!20,draw=none](-4.493,2.826)--(-4.941,2.155)--(-4.904,2.181)--(-4.796,2.281)--(-4.443,2.809)--cycle;
\draw(-4.493,2.826)--(-4.941,2.155);
\draw(-4.796,2.281)--(-4.443,2.809);
\filldraw[fill opacity=0.8,fill=gray!20,draw=none](-4.36,2.895)--(-4.642,2.513)--(-4.348,2.863)--cycle;
\draw(-4.642,2.513)--(-4.348,2.863);
\filldraw[fill opacity=0.8,fill=gray!20](-4.555,2.907)--(-4.543,2.963)--(-4.608,2.979)--(-4.628,2.925)--cycle;
\filldraw[fill opacity=0.8,fill=gray!20,draw=none](-4.306,3.024)--(-4.304,3.026)--(-4.318,3.034)--cycle;
\draw(-4.306,3.024)--(-4.304,3.026);
\filldraw[fill opacity=0.8,fill=gray!20,draw=none](-8.153,.734)--(-8.164,.761)--(-8.168,.765)--cycle;
\draw(-8.153,.734)--(-8.164,.761)--(-8.168,.765);
\filldraw[fill opacity=0.8,fill=gray!20,draw=none](-4.404,2.705)--(-4.402,2.707)--(-4.404,2.707)--cycle;
\draw(-4.402,2.707)--(-4.404,2.707);
\filldraw[fill opacity=0.8,fill=gray!20,draw=none](-4.36,3.064)--(-4.355,3.061)--(-4.353,3.061)--cycle;
\draw(-4.355,3.061)--(-4.353,3.061);
\filldraw[fill opacity=0.8,fill=gray!20,draw=none](-4.344,3.043)--(-4.321,3.037)--(-4.327,3.043)--(-4.353,3.061)--(-4.355,3.061)--cycle;
\draw(-4.321,3.037)--(-4.327,3.043);
\draw(-4.353,3.061)--(-4.355,3.061);
\filldraw[fill opacity=0.8,fill=gray!20,draw=none](-4.624,2.866)--(-4.622,2.873)--(-4.618,2.879)--cycle;
\filldraw[fill opacity=0.8,fill=gray!20](-8.175,1.715)--(-8.134,1.753)--(-8.145,1.766)--(-8.191,1.732)--cycle;
\filldraw[fill opacity=0.8,fill=gray!20](-7.824,1.631)--(-7.846,1.683)--(-7.877,1.663)--(-7.859,1.609)--cycle;
\filldraw[fill opacity=0.8,fill=gray!20,draw=none](-4.329,3.019)--(-4.355,3.061)--(-4.389,3.054)--(-4.368,3.011)--cycle;
\draw(-4.355,3.061)--(-4.389,3.054)--(-4.368,3.011)--(-4.329,3.019);
\filldraw[fill opacity=0.8,fill=gray!20](-2.936,7.849)--(-2.929,7.905)--(-2.894,7.927)--(-2.9,7.873)--cycle;
\filldraw[fill opacity=0.8,fill=gray!20,draw=none](-2.557,7.76)--(-2.547,7.758)--(-2.557,7.743)--cycle;
\draw(-2.557,7.76)--(-2.547,7.758)--(-2.557,7.743);
\filldraw[fill opacity=0.8,fill=gray!20,draw=none](-4.318,3.034)--(-4.327,3.043)--(-4.321,3.037)--cycle;
\draw(-4.327,3.043)--(-4.321,3.037);
\filldraw[fill opacity=0.8,fill=gray!20](-2.786,8.028)--(-2.766,8.053)--(-2.71,8.056)--(-2.708,8.032)--cycle;
\filldraw[fill opacity=0.8,fill=gray!20](-2.708,8.032)--(-2.71,8.056)--(-2.656,8.052)--(-2.631,8.026)--cycle;
\filldraw[fill opacity=0.8,fill=gray!20,draw=none](-4.395,2.733)--(-4.4,2.729)--(-4.404,2.707)--(-4.384,2.716)--cycle;
\filldraw[fill opacity=0.8,fill=gray!20,draw=none](-4.306,3.024)--(-4.318,3.034)--(-4.321,3.037)--(-4.308,3.023)--cycle;
\draw(-4.321,3.037)--(-4.308,3.023)--(-4.306,3.024);
\filldraw[fill opacity=0.8,fill=gray!20,draw=none](-4.447,2.971)--(-4.449,2.968)--(-4.445,2.961)--(-4.438,2.961)--cycle;
\draw(-4.447,2.971)--(-4.449,2.968);
\filldraw[fill opacity=0.8,fill=gray!20,draw=none](-4.364,2.906)--(-4.368,2.883)--(-4.36,2.895)--cycle;
\filldraw[fill opacity=0.8,fill=gray!20,draw=none](-8.157,1.539)--(-8.153,1.595)--(-8.219,1.612)--(-8.222,1.606)--(-8.232,1.557)--cycle;
\draw(-8.232,1.557)--(-8.157,1.539)--(-8.153,1.595)--(-8.219,1.612);
\filldraw[fill opacity=0.8,fill=gray!20,draw=none](-8.049,1.588)--(-8.047,1.639)--(-8.053,1.645)--(-8.141,1.651)--(-8.153,1.595)--cycle;
\draw(-8.053,1.645)--(-8.141,1.651)--(-8.153,1.595)--(-8.049,1.588)--(-8.047,1.639);
\filldraw[fill opacity=0.8,fill=gray!20,draw=none](-8.304,1.505)--(-8.358,1.518)--(-8.373,1.549)--(-8.346,1.537)--(-8.305,1.506)--cycle;
\draw(-8.304,1.505)--(-8.358,1.518)--(-8.373,1.549);
\filldraw[fill opacity=0.8,fill=gray!20](-8.036,1.707)--(-8.445,1.529)--(-8.4,1.514)--(-7.991,1.692)--cycle;
\filldraw[fill opacity=0.8,fill=gray!20,draw=none](-4.344,3.043)--(-4.329,3.019)--(-4.308,3.023)--(-4.321,3.037)--cycle;
\draw(-4.329,3.019)--(-4.308,3.023)--(-4.321,3.037);
\filldraw[fill opacity=0.8,fill=gray!20,draw=none](-4.496,3.055)--(-4.444,3.052)--(-4.444,3.061)--(-4.484,3.071)--(-4.497,3.057)--cycle;
\draw(-4.496,3.055)--(-4.444,3.052)--(-4.444,3.061);
\draw(-4.484,3.071)--(-4.497,3.057);
\filldraw[fill opacity=0.8,fill=gray!20,draw=none](-2.829,7.675)--(-2.843,7.675)--(-2.872,7.697)--(-2.848,7.713)--cycle;
\draw(-2.843,7.675)--(-2.872,7.697)--(-2.848,7.713);
\filldraw[fill opacity=0.8,fill=gray!20,draw=none](-4.618,2.88)--(-4.747,2.727)--(-4.597,2.898)--cycle;
\draw(-4.618,2.88)--(-4.747,2.727);
\filldraw[fill opacity=0.8,fill=gray!20,draw=none](-4.513,2.855)--(-4.549,2.831)--(-5.008,2.144)--(-4.997,2.133)--(-4.941,2.155)--(-4.482,2.843)--cycle;
\draw(-4.549,2.831)--(-5.008,2.144);
\draw(-4.941,2.155)--(-4.482,2.843);
\filldraw[fill opacity=0.8,fill=gray!20,draw=none](-5.797,1.136)--(-5.835,1.135)--(-5.342,1.723)--cycle;
\draw(-5.835,1.135)--(-5.342,1.723);
\filldraw[fill opacity=0.8,fill=gray!20,draw=none](-5.894,1.147)--(-5.894,1.122)--(-5.944,1.174)--cycle;
\draw(-5.894,1.147)--(-5.894,1.122);
\filldraw[fill opacity=0.8,fill=gray!20,draw=none](-5.894,1.122)--(-5.906,1.127)--(-5.913,1.136)--cycle;
\draw(-5.894,1.122)--(-5.906,1.127);
\filldraw[fill opacity=0.8,fill=gray!20,draw=none](-5.894,.99)--(-5.865,.997)--(-5.872,.989)--cycle;
\draw(-5.865,.997)--(-5.872,.989);
\filldraw[fill opacity=0.8,fill=gray!20,draw=none](-5.894,1.012)--(-5.877,1.065)--(-5.859,1.057)--cycle;
\draw(-5.877,1.065)--(-5.859,1.057);
\filldraw[fill opacity=0.8,fill=gray!20,draw=none](-5.886,.978)--(-5.864,.964)--(-5.894,.888)--(-5.894,.888)--(-5.894,.971)--cycle;
\draw(-5.894,.888)--(-5.894,.888)--(-5.894,.971);
\filldraw[fill opacity=0.8,fill=gray!20,draw=none](-5.847,1.149)--(-5.844,1.117)--(-5.844,1.1)--(-5.859,1.057)--(-5.859,1.21)--cycle;
\draw(-5.844,1.117)--(-5.844,1.1);
\draw(-5.859,1.057)--(-5.859,1.21);
\filldraw[fill opacity=0.8,fill=gray!20,draw=none](-5.862,1.131)--(-5.845,1.104)--(-5.844,1.1)--(-5.894,1.122)--(-5.913,1.136)--(-5.944,1.174)--(-5.873,1.143)--cycle;
\draw(-5.844,1.1)--(-5.894,1.122);
\draw(-5.944,1.174)--(-5.873,1.143);
\filldraw[fill opacity=0.8,fill=gray!20,draw=none](-5.894,1.122)--(-5.894,1.012)--(-5.902,.988)--(-5.907,.986)--(-5.944,.995)--(-5.944,1.174)--cycle;
\draw(-5.894,1.122)--(-5.894,1.012);
\draw(-5.944,.995)--(-5.944,1.174);
\filldraw[fill opacity=0.8,fill=gray!20,draw=none](-5.843,1.013)--(-5.846,1.011)--(-5.859,1)--(-5.866,.996)--(-5.862,1.001)--cycle;
\draw(-5.843,1.013)--(-5.846,1.011);
\draw(-5.866,.996)--(-5.862,1.001);
\filldraw[fill opacity=0.8,fill=gray!20,draw=none](-5.859,1)--(-5.872,.989)--(-5.866,.996)--cycle;
\draw(-5.872,.989)--(-5.866,.996);
\filldraw[fill opacity=0.8,fill=gray!20,draw=none](-5.859,1.096)--(-5.859,1.057)--(-5.894,1.122)--cycle;
\draw(-5.859,1.096)--(-5.859,1.057);
\filldraw[fill opacity=0.8,fill=gray!20,draw=none](-5.842,1.049)--(-5.877,1.065)--(-5.894,1.122)--(-5.844,1.1)--cycle;
\draw(-5.842,1.049)--(-5.877,1.065);
\draw(-5.894,1.122)--(-5.844,1.1);
\filldraw[fill opacity=0.8,fill=gray!20,draw=none](-5.859,1.057)--(-5.859,1)--(-5.872,.989)--(-5.894,.99)--(-5.894,1.122)--cycle;
\draw(-5.859,1.057)--(-5.859,1);
\draw(-5.894,.99)--(-5.894,1.122);
\filldraw[fill opacity=0.8,fill=gray!20,draw=none](-5.842,1.028)--(-5.849,1.012)--(-5.859,1.057)--(-5.842,1.049)--cycle;
\draw(-5.859,1.057)--(-5.842,1.049);
\filldraw[fill opacity=0.8,fill=gray!20,draw=none](-5.844,1.1)--(-5.844,1.03)--(-5.849,1.012)--(-5.859,1.057)--cycle;
\draw(-5.844,1.1)--(-5.844,1.03);
\filldraw[fill opacity=0.8,fill=gray!20,draw=none](-5.839,1.089)--(-5.835,1.047)--(-5.842,1.049)--(-5.844,1.1)--(-5.841,1.099)--cycle;
\draw(-5.835,1.047)--(-5.842,1.049);
\draw(-5.844,1.1)--(-5.841,1.099);
\filldraw[fill opacity=0.8,fill=gray!20,draw=none](-5.845,1.104)--(-5.847,1.079)--(-5.845,1.032)--(-5.844,1.03)--(-5.844,1.1)--cycle;
\draw(-5.844,1.03)--(-5.844,1.1);
\filldraw[fill opacity=0.8,fill=gray!20,draw=none](-5.845,1.104)--(-5.842,1.1)--(-5.844,1.1)--cycle;
\filldraw[fill opacity=0.8,fill=gray!20,draw=none](-5.842,1.1)--(-5.841,1.099)--(-5.844,1.1)--cycle;
\draw(-5.841,1.099)--(-5.844,1.1);
\filldraw[fill opacity=0.8,fill=gray!20,draw=none](-5.845,1.104)--(-5.844,1.1)--(-5.844,1.117)--cycle;
\draw(-5.844,1.1)--(-5.844,1.117);
\filldraw[fill opacity=0.8,fill=gray!20,draw=none](-4.368,2.883)--(-4.364,2.906)--(-4.366,2.911)--(-5.901,1.085)--(-5.878,1.042)--(-4.642,2.513)--cycle;
\draw(-4.366,2.911)--(-5.901,1.085)--(-5.878,1.042)--(-4.642,2.513);
\filldraw[fill opacity=0.8,fill=gray!20](-8.035,1.003)--(-8.006,1.034)--(-8.025,1.039)--(-8.073,1.012)--cycle;
\filldraw[fill opacity=0.8,fill=gray!20](-7.988,.79)--(-7.987,.847)--(-8.091,.854)--(-8.095,.797)--cycle;
\filldraw[fill opacity=0.8,fill=gray!20,draw=none](-7.96,.791)--(-7.969,.841)--(-7.974,.847)--(-7.987,.847)--(-7.988,.79)--cycle;
\draw(-7.974,.847)--(-7.987,.847)--(-7.988,.79)--(-7.96,.791);
\filldraw[fill opacity=0.8,fill=gray!20,draw=none](-7.974,.847)--(-7.987,.868)--(-7.987,.847)--cycle;
\draw(-7.987,.868)--(-7.987,.847)--(-7.974,.847);
\filldraw[fill opacity=0.8,fill=gray!20](-8.344,.719)--(-8.348,.764)--(-8.287,.749)--(-8.281,.704)--cycle;
\filldraw[fill opacity=0.8,fill=gray!20,draw=none](-8.27,.691)--(-8.281,.704)--(-8.287,.749)--(-8.277,.739)--cycle;
\draw(-8.27,.691)--(-8.281,.704)--(-8.287,.749)--(-8.277,.739);
\filldraw[fill opacity=0.8,fill=gray!20](-7.897,.917)--(-8.368,.765)--(-8.344,.72)--(-7.874,.872)--cycle;
\filldraw[fill opacity=0.8,fill=gray!20,draw=none](-4.395,2.733)--(-4.384,2.716)--(-4.361,2.725)--(-4.353,2.746)--(-4.382,2.745)--cycle;
\draw(-4.361,2.725)--(-4.353,2.746)--(-4.382,2.745);
\filldraw[fill opacity=0.8,fill=gray!20,draw=none](-4.446,2.962)--(-4.449,2.968)--(-4.452,2.965)--cycle;
\draw(-4.449,2.968)--(-4.452,2.965);
\filldraw[fill opacity=0.8,fill=gray!20,draw=none](-4.402,2.916)--(-4.387,2.893)--(-4.376,2.908)--cycle;
\draw(-4.387,2.893)--(-4.376,2.908);
\filldraw[fill opacity=0.8,fill=gray!20,draw=none](-4.393,2.929)--(-4.375,2.9)--(-4.366,2.911)--cycle;
\draw(-4.375,2.9)--(-4.366,2.911);
\filldraw[fill opacity=0.8,fill=gray!20](-7.874,.872)--(-8.344,.72)--(-8.336,.67)--(-7.865,.821)--cycle;
\filldraw[fill opacity=0.8,fill=gray!20,draw=none](-4.442,2.965)--(-4.435,2.958)--(-4.426,2.971)--cycle;
\draw(-4.435,2.958)--(-4.426,2.971);
\filldraw[fill opacity=0.8,fill=gray!20,draw=none](-4.438,2.961)--(-4.442,2.961)--(-4.435,2.958)--cycle;
\filldraw[fill opacity=0.8,fill=gray!20,draw=none](-4.442,2.961)--(-4.445,2.961)--(-4.435,2.939)--(-4.42,2.932)--(-4.413,2.934)--(-4.404,2.945)--cycle;
\draw(-4.413,2.934)--(-4.404,2.945);
\filldraw[fill opacity=0.8,fill=gray!20,draw=none](-4.502,2.96)--(-4.492,2.977)--(-4.492,3.011)--(-4.523,3.013)--(-4.543,2.963)--cycle;
\draw(-4.492,3.011)--(-4.523,3.013)--(-4.543,2.963)--(-4.502,2.96);
\filldraw[fill opacity=0.8,fill=gray!20,draw=none](-7.639,4.516)--(-7.644,4.503)--(-7.657,4.508)--cycle;
\draw(-7.639,4.516)--(-7.644,4.503)--(-7.657,4.508);
\filldraw[fill opacity=0.8,fill=gray!20,draw=none](-7.653,4.506)--(-7.651,4.51)--(-7.588,4.54)--(-7.611,4.489)--cycle;
\draw(-7.653,4.506)--(-7.651,4.51);
\draw(-7.588,4.54)--(-7.611,4.489);
\filldraw[fill opacity=0.8,fill=gray!20](-7.522,4.494)--(-7.481,4.532)--(-7.465,4.515)--(-7.511,4.482)--cycle;
\filldraw[fill opacity=0.8,fill=gray!20,draw=none](-7.629,4.521)--(-7.644,4.514)--(-7.62,4.525)--(-7.596,4.537)--cycle;
\draw(-7.629,4.521)--(-7.644,4.514);
\draw(-7.62,4.525)--(-7.596,4.537);
\filldraw[fill opacity=0.8,fill=gray!20,draw=none](-7.581,4.544)--(-7.629,4.521)--(-7.596,4.537)--(-7.55,4.559)--cycle;
\draw(-7.596,4.537)--(-7.55,4.559)--(-7.581,4.544)--(-7.629,4.521);
\filldraw[fill opacity=0.8,fill=gray!20,draw=none](-7.658,4.508)--(-7.659,4.507)--(-7.666,4.503)--(-7.66,4.509)--cycle;
\draw(-7.659,4.507)--(-7.666,4.503);
\filldraw[fill opacity=0.8,fill=gray!20,draw=none](-7.741,4.549)--(-7.74,4.549)--(-7.741,4.537)--(-7.75,4.536)--(-7.758,4.549)--cycle;
\draw(-7.741,4.537)--(-7.75,4.536)--(-7.758,4.549);
\filldraw[fill opacity=0.8,fill=gray!20,draw=none](-7.718,4.529)--(-7.723,4.53)--(-7.719,4.53)--cycle;
\filldraw[fill opacity=0.8,fill=gray!20,draw=none](-7.704,4.519)--(-7.74,4.53)--(-7.741,4.531)--(-7.723,4.53)--(-7.718,4.529)--cycle;
\filldraw[fill opacity=0.8,fill=gray!20,draw=none](-7.739,4.521)--(-7.742,4.528)--(-7.737,4.537)--(-7.701,4.524)--(-7.709,4.508)--cycle;
\draw(-7.701,4.524)--(-7.709,4.508);
\filldraw[fill opacity=0.8,fill=gray!20,draw=none](-7.657,4.508)--(-7.673,4.5)--(-7.707,4.512)--(-7.701,4.524)--cycle;
\draw(-7.707,4.512)--(-7.701,4.524);
\filldraw[fill opacity=0.8,fill=gray!20,draw=none](-7.657,4.508)--(-7.691,4.517)--(-7.737,4.537)--(-7.687,4.519)--(-7.657,4.508)--cycle;
\draw(-7.687,4.519)--(-7.657,4.508);
\filldraw[fill opacity=0.8,fill=gray!20,draw=none](-7.657,4.508)--(-7.662,4.506)--(-7.684,4.514)--(-7.691,4.517)--cycle;
\filldraw[fill opacity=0.8,fill=gray!20,draw=none](-7.617,4.474)--(-7.589,4.472)--(-7.599,4.466)--cycle;
\draw(-7.617,4.474)--(-7.589,4.472)--(-7.599,4.466);
\filldraw[fill opacity=0.8,fill=gray!20,draw=none](-7.589,4.472)--(-7.587,4.474)--(-7.568,4.468)--(-7.569,4.467)--cycle;
\draw(-7.568,4.468)--(-7.569,4.467)--(-7.589,4.472)--(-7.587,4.474);
\filldraw[fill opacity=0.8,fill=gray!20,draw=none](-7.591,4.47)--(-7.589,4.472)--(-7.569,4.467)--(-7.586,4.463)--cycle;
\draw(-7.591,4.47)--(-7.589,4.472)--(-7.569,4.467)--(-7.586,4.463);
\filldraw[fill opacity=0.8,fill=gray!20,draw=none](-7.599,4.466)--(-7.591,4.47)--(-7.586,4.463)--cycle;
\draw(-7.599,4.466)--(-7.591,4.47);
\filldraw[fill opacity=0.8,fill=gray!20,draw=none](-7.599,4.466)--(-7.617,4.474)--(-7.611,4.489)--(-7.58,4.477)--(-7.586,4.463)--cycle;
\draw(-7.617,4.474)--(-7.611,4.489);
\draw(-7.58,4.477)--(-7.586,4.463);
\filldraw[fill opacity=0.8,fill=gray!20,draw=none](-7.617,4.468)--(-7.592,4.464)--(-7.587,4.461)--(-7.589,4.455)--cycle;
\draw(-7.587,4.461)--(-7.589,4.455);
\filldraw[fill opacity=0.8,fill=gray!20,draw=none](-7.586,4.463)--(-7.569,4.467)--(-7.563,4.461)--(-7.568,4.46)--cycle;
\draw(-7.586,4.463)--(-7.569,4.467)--(-7.563,4.461)--(-7.568,4.46);
\filldraw[fill opacity=0.8,fill=gray!20,draw=none](-7.599,4.466)--(-7.586,4.463)--(-7.587,4.461)--cycle;
\draw(-7.586,4.463)--(-7.587,4.461);
\filldraw[fill opacity=0.8,fill=gray!20,draw=none](-7.586,4.463)--(-7.568,4.46)--(-7.581,4.459)--cycle;
\draw(-7.568,4.46)--(-7.581,4.459);
\filldraw[fill opacity=0.8,fill=gray!20,draw=none](-7.581,4.459)--(-7.563,4.461)--(-7.573,4.455)--cycle;
\draw(-7.581,4.459)--(-7.563,4.461)--(-7.573,4.455);
\filldraw[fill opacity=0.8,fill=gray!20,draw=none](-7.581,4.459)--(-7.586,4.463)--(-7.58,4.477)--(-7.566,4.47)--(-7.572,4.455)--cycle;
\draw(-7.586,4.463)--(-7.58,4.477);
\draw(-7.566,4.47)--(-7.572,4.455);
\filldraw[fill opacity=0.8,fill=gray!20,draw=none](-7.567,4.47)--(-7.627,4.492)--(-7.662,4.506)--(-7.657,4.508)--(-7.653,4.506)--(-7.611,4.489)--(-7.58,4.477)--(-7.566,4.47)--cycle;
\filldraw[fill opacity=0.8,fill=gray!20,draw=none](-7.566,4.47)--(-7.58,4.477)--(-7.563,4.47)--cycle;
\filldraw[fill opacity=0.8,fill=gray!20,draw=none](-7.695,4.413)--(-7.662,4.487)--(-7.62,4.469)--(-7.665,4.368)--cycle;
\draw(-7.695,4.413)--(-7.662,4.487);
\draw(-7.62,4.469)--(-7.665,4.368);
\filldraw[fill opacity=0.8,fill=gray!20,draw=none](-7.65,4.336)--(-7.642,4.357)--(-7.639,4.355)--(-7.631,4.334)--cycle;
\draw(-7.65,4.336)--(-7.642,4.357);
\filldraw[fill opacity=0.8,fill=gray!20,draw=none](-7.644,4.514)--(-7.682,4.496)--(-7.682,4.496)--(-7.62,4.525)--cycle;
\draw(-7.644,4.514)--(-7.682,4.496);
\draw(-7.682,4.496)--(-7.62,4.525);
\filldraw[fill opacity=0.8,fill=gray!20,draw=none](-7.657,4.508)--(-7.657,4.508)--(-7.644,4.503)--(-7.631,4.498)--cycle;
\draw(-7.657,4.508)--(-7.644,4.503)--(-7.631,4.498);
\filldraw[fill opacity=0.8,fill=gray!20,draw=none](-7.657,4.508)--(-7.659,4.507)--(-7.658,4.508)--cycle;
\draw(-7.657,4.508)--(-7.659,4.507);
\filldraw[fill opacity=0.8,fill=gray!20,draw=none](-7.657,4.508)--(-7.651,4.51)--(-7.653,4.506)--cycle;
\draw(-7.651,4.51)--(-7.653,4.506);
\filldraw[fill opacity=0.8,fill=gray!20,draw=none](-7.653,4.506)--(-7.631,4.498)--(-7.602,4.486)--(-7.587,4.48)--cycle;
\draw(-7.631,4.498)--(-7.602,4.486)--(-7.587,4.48);
\filldraw[fill opacity=0.8,fill=gray!20,draw=none](-7.57,4.47)--(-7.572,4.47)--(-7.627,4.492)--(-7.567,4.47)--cycle;
\filldraw[fill opacity=0.8,fill=gray!20,draw=none](-7.535,4.566)--(-7.533,4.544)--(-7.566,4.478)--(-7.572,4.47)--(-7.573,4.471)--(-7.58,4.544)--cycle;
\draw(-7.535,4.566)--(-7.533,4.544);
\draw(-7.573,4.471)--(-7.58,4.544);
\filldraw[fill opacity=0.8,fill=gray!20,draw=none](-7.618,4.472)--(-7.617,4.474)--(-7.599,4.466)--cycle;
\draw(-7.618,4.472)--(-7.617,4.474);
\filldraw[fill opacity=0.8,fill=gray!20,draw=none](-7.617,4.468)--(-7.62,4.469)--(-7.618,4.472)--(-7.599,4.466)--(-7.592,4.464)--cycle;
\draw(-7.62,4.469)--(-7.618,4.472);
\filldraw[fill opacity=0.8,fill=gray!20,draw=none](-7.618,4.468)--(-7.617,4.474)--(-7.599,4.466)--(-7.606,4.463)--cycle;
\draw(-7.618,4.468)--(-7.617,4.474);
\draw(-7.599,4.466)--(-7.606,4.463);
\filldraw[fill opacity=0.8,fill=gray!20,draw=none](-7.63,4.473)--(-7.617,4.474)--(-7.618,4.468)--cycle;
\draw(-7.63,4.473)--(-7.617,4.474)--(-7.618,4.468);
\filldraw[fill opacity=0.8,fill=gray!20,draw=none](-7.606,4.463)--(-7.599,4.466)--(-7.586,4.463)--(-7.599,4.46)--cycle;
\draw(-7.606,4.463)--(-7.599,4.466);
\draw(-7.586,4.463)--(-7.599,4.46);
\filldraw[fill opacity=0.8,fill=gray!20,draw=none](-7.587,4.461)--(-7.586,4.463)--(-7.581,4.459)--cycle;
\draw(-7.587,4.461)--(-7.586,4.463);
\filldraw[fill opacity=0.8,fill=gray!20,draw=none](-7.58,4.451)--(-7.589,4.455)--(-7.587,4.461)--(-7.581,4.459)--(-7.575,4.454)--cycle;
\draw(-7.589,4.455)--(-7.587,4.461);
\filldraw[fill opacity=0.8,fill=gray!20,draw=none](-7.599,4.46)--(-7.586,4.463)--(-7.581,4.459)--(-7.594,4.457)--cycle;
\draw(-7.599,4.46)--(-7.586,4.463);
\draw(-7.581,4.459)--(-7.594,4.457);
\filldraw[fill opacity=0.8,fill=gray!20,draw=none](-7.575,4.454)--(-7.581,4.459)--(-7.572,4.455)--cycle;
\filldraw[fill opacity=0.8,fill=gray!20,draw=none](-7.594,4.457)--(-7.581,4.459)--(-7.573,4.455)--(-7.587,4.455)--cycle;
\draw(-7.594,4.457)--(-7.581,4.459);
\draw(-7.573,4.455)--(-7.587,4.455);
\filldraw[fill opacity=0.8,fill=gray!20,draw=none](-7.58,4.544)--(-7.573,4.471)--(-7.627,4.492)--(-7.629,4.521)--cycle;
\draw(-7.58,4.544)--(-7.573,4.471);
\draw(-7.627,4.492)--(-7.629,4.521);
\filldraw[fill opacity=0.8,fill=gray!20,draw=none](-7.543,4.465)--(-7.551,4.464)--(-7.547,4.467)--(-7.538,4.471)--cycle;
\draw(-7.547,4.467)--(-7.538,4.471);
\filldraw[fill opacity=0.8,fill=gray!20,draw=none](-7.547,4.467)--(-7.506,4.564)--(-7.505,4.563)--(-7.505,4.558)--(-7.54,4.468)--cycle;
\draw(-7.505,4.558)--(-7.54,4.468);
\filldraw[fill opacity=0.8,fill=gray!20,draw=none](-7.542,4.462)--(-7.54,4.468)--(-7.544,4.465)--cycle;
\draw(-7.542,4.462)--(-7.54,4.468);
\filldraw[fill opacity=0.8,fill=gray!20,draw=none](-7.543,4.465)--(-7.538,4.471)--(-7.511,4.482)--(-7.529,4.47)--(-7.541,4.465)--cycle;
\draw(-7.538,4.471)--(-7.511,4.482)--(-7.529,4.47)--(-7.541,4.465);
\filldraw[fill opacity=0.8,fill=gray!20,draw=none](-7.511,4.482)--(-7.465,4.515)--(-7.474,4.509)--(-7.505,4.488)--(-7.529,4.47)--cycle;
\draw(-7.505,4.488)--(-7.529,4.47)--(-7.511,4.482)--(-7.465,4.515)--(-7.474,4.509);
\filldraw[fill opacity=0.8,fill=gray!20,draw=none](-7.634,4.568)--(-7.629,4.521)--(-7.675,4.521)--(-7.68,4.542)--(-7.683,4.577)--cycle;
\draw(-7.68,4.542)--(-7.683,4.577)--(-7.634,4.568)--(-7.629,4.521);
\filldraw[fill opacity=0.8,fill=gray!20,draw=none](-7.618,4.549)--(-7.655,4.531)--(-7.675,4.521)--(-7.629,4.521)--(-7.581,4.544)--cycle;
\draw(-7.629,4.521)--(-7.581,4.544)--(-7.618,4.549)--(-7.655,4.531);
\filldraw[fill opacity=0.8,fill=gray!20,draw=none](-7.469,4.514)--(-7.474,4.509)--(-7.465,4.515)--(-7.461,4.52)--cycle;
\draw(-7.474,4.509)--(-7.465,4.515)--(-7.461,4.52);
\filldraw[fill opacity=0.8,fill=gray!20,draw=none](-7.563,4.461)--(-7.561,4.462)--(-7.551,4.464)--(-7.553,4.461)--(-7.573,4.455)--cycle;
\draw(-7.553,4.461)--(-7.573,4.455)--(-7.563,4.461)--(-7.561,4.462);
\filldraw[fill opacity=0.8,fill=gray!20,draw=none](-7.545,4.467)--(-7.544,4.465)--(-7.54,4.468)--(-7.505,4.558)--(-7.511,4.548)--(-7.53,4.507)--(-7.534,4.498)--(-7.545,4.468)--cycle;
\draw(-7.54,4.468)--(-7.505,4.558);
\draw(-7.534,4.498)--(-7.545,4.468);
\filldraw[fill opacity=0.8,fill=gray!20,draw=none](-7.566,4.47)--(-7.563,4.47)--(-7.556,4.466)--(-7.559,4.466)--cycle;
\draw(-7.563,4.47)--(-7.556,4.466)--(-7.559,4.466);
\filldraw[fill opacity=0.8,fill=gray!20,draw=none](-7.527,4.547)--(-7.524,4.563)--(-7.52,4.571)--(-7.52,4.57)--(-7.52,4.566)--cycle;
\filldraw[fill opacity=0.8,fill=gray!20,draw=none](-7.532,4.536)--(-7.56,4.466)--(-7.556,4.466)--(-7.522,4.557)--cycle;
\draw(-7.56,4.466)--(-7.556,4.466)--(-7.522,4.557);
\filldraw[fill opacity=0.8,fill=gray!20,draw=none](-7.52,4.566)--(-7.532,4.536)--(-7.522,4.557)--(-7.518,4.569)--cycle;
\draw(-7.522,4.557)--(-7.518,4.569);
\filldraw[fill opacity=0.8,fill=gray!20,draw=none](-7.512,4.556)--(-7.53,4.518)--(-7.555,4.464)--(-7.553,4.46)--(-7.509,4.56)--cycle;
\draw(-7.553,4.46)--(-7.509,4.56);
\filldraw[fill opacity=0.8,fill=gray!20,draw=none](-7.58,4.451)--(-7.575,4.454)--(-7.573,4.452)--(-7.575,4.449)--cycle;
\draw(-7.573,4.452)--(-7.575,4.449);
\filldraw[fill opacity=0.8,fill=gray!20,draw=none](-7.587,4.455)--(-7.573,4.455)--(-7.583,4.453)--cycle;
\draw(-7.587,4.455)--(-7.573,4.455)--(-7.583,4.453);
\filldraw[fill opacity=0.8,fill=gray!20,draw=none](-7.575,4.454)--(-7.572,4.455)--(-7.573,4.452)--cycle;
\draw(-7.572,4.455)--(-7.573,4.452);
\filldraw[fill opacity=0.8,fill=gray!20,draw=none](-7.522,4.572)--(-7.521,4.57)--(-7.524,4.563)--cycle;
\draw(-7.521,4.57)--(-7.524,4.563);
\filldraw[fill opacity=0.8,fill=gray!20,draw=none](-7.566,4.47)--(-7.521,4.57)--(-7.536,4.547)--(-7.57,4.47)--cycle;
\draw(-7.566,4.47)--(-7.521,4.57);
\draw(-7.536,4.547)--(-7.57,4.47);
\filldraw[fill opacity=0.8,fill=gray!20,draw=none](-7.524,4.563)--(-7.527,4.547)--(-7.532,4.536)--(-7.533,4.544)--cycle;
\draw(-7.532,4.536)--(-7.533,4.544);
\filldraw[fill opacity=0.8,fill=gray!20,draw=none](-7.572,4.455)--(-7.566,4.47)--(-7.57,4.47)--(-7.575,4.459)--cycle;
\draw(-7.572,4.455)--(-7.566,4.47);
\draw(-7.57,4.47)--(-7.575,4.459);
\filldraw[fill opacity=0.8,fill=gray!20,draw=none](-7.567,4.47)--(-7.566,4.47)--(-7.56,4.467)--cycle;
\filldraw[fill opacity=0.8,fill=gray!20,draw=none](-7.569,4.469)--(-7.57,4.47)--(-7.567,4.47)--(-7.56,4.467)--(-7.559,4.466)--(-7.561,4.467)--(-7.566,4.468)--cycle;
\draw(-7.559,4.466)--(-7.561,4.467)--(-7.566,4.468);
\filldraw[fill opacity=0.8,fill=gray!20,draw=none](-7.533,4.544)--(-7.532,4.536)--(-7.551,4.493)--(-7.568,4.469)--(-7.57,4.47)--cycle;
\draw(-7.533,4.544)--(-7.532,4.536);
\filldraw[fill opacity=0.8,fill=gray!20,draw=none](-7.527,4.547)--(-7.52,4.566)--(-7.517,4.547)--(-7.53,4.518)--(-7.531,4.528)--cycle;
\draw(-7.53,4.518)--(-7.531,4.528);
\filldraw[fill opacity=0.8,fill=gray!20,draw=none](-7.52,4.566)--(-7.531,4.545)--(-7.561,4.467)--(-7.56,4.466)--cycle;
\draw(-7.531,4.545)--(-7.561,4.467)--(-7.56,4.466);
\filldraw[fill opacity=0.8,fill=gray!20,draw=none](-7.512,4.556)--(-7.517,4.547)--(-7.53,4.518)--cycle;
\filldraw[fill opacity=0.8,fill=gray!20,draw=none](-7.527,4.547)--(-7.531,4.528)--(-7.532,4.536)--cycle;
\draw(-7.531,4.528)--(-7.532,4.536);
\filldraw[fill opacity=0.8,fill=gray!20,draw=none](-7.551,4.493)--(-7.532,4.536)--(-7.531,4.528)--(-7.532,4.52)--cycle;
\draw(-7.532,4.536)--(-7.531,4.528);
\filldraw[fill opacity=0.8,fill=gray!20,draw=none](-7.511,4.548)--(-7.517,4.538)--(-7.53,4.507)--cycle;
\filldraw[fill opacity=0.8,fill=gray!20,draw=none](-7.532,4.515)--(-7.531,4.528)--(-7.53,4.518)--cycle;
\draw(-7.531,4.528)--(-7.53,4.518);
\filldraw[fill opacity=0.8,fill=gray!20,draw=none](-7.517,4.547)--(-7.516,4.538)--(-7.529,4.508)--(-7.53,4.518)--cycle;
\draw(-7.529,4.508)--(-7.53,4.518);
\filldraw[fill opacity=0.8,fill=gray!20,draw=none](-7.572,4.47)--(-7.57,4.47)--(-7.536,4.547)--(-7.574,4.519)--(-7.574,4.519)--cycle;
\draw(-7.57,4.47)--(-7.536,4.547);
\draw(-7.574,4.519)--(-7.574,4.519);
\filldraw[fill opacity=0.8,fill=gray!20,draw=none](-7.567,4.516)--(-7.569,4.511)--(-7.566,4.468)--(-7.561,4.467)--(-7.531,4.545)--cycle;
\draw(-7.567,4.516)--(-7.569,4.511);
\draw(-7.566,4.468)--(-7.561,4.467)--(-7.531,4.545);
\filldraw[fill opacity=0.8,fill=gray!20,draw=none](-7.558,4.472)--(-7.56,4.472)--(-7.551,4.493)--(-7.532,4.52)--(-7.532,4.515)--(-7.544,4.491)--cycle;
\filldraw[fill opacity=0.8,fill=gray!20,draw=none](-7.53,4.518)--(-7.517,4.547)--(-7.521,4.54)--(-7.528,4.527)--(-7.544,4.491)--cycle;
\draw(-7.528,4.527)--(-7.544,4.491);
\filldraw[fill opacity=0.8,fill=gray!20,draw=none](-7.741,4.533)--(-7.749,4.535)--(-7.75,4.536)--(-7.741,4.537)--cycle;
\draw(-7.749,4.535)--(-7.75,4.536)--(-7.741,4.537);
\filldraw[fill opacity=0.8,fill=gray!20,draw=none](-7.74,4.53)--(-7.704,4.519)--(-7.68,4.502)--(-7.692,4.5)--(-7.733,4.518)--(-7.74,4.526)--cycle;
\draw(-7.68,4.502)--(-7.692,4.5);
\draw(-7.733,4.518)--(-7.74,4.526);
\filldraw[fill opacity=0.8,fill=gray!20,draw=none](-7.685,4.515)--(-7.704,4.505)--(-7.682,4.496)--(-7.66,4.507)--cycle;
\draw(-7.682,4.496)--(-7.66,4.507);
\filldraw[fill opacity=0.8,fill=gray!20,draw=none](-7.555,4.464)--(-7.53,4.518)--(-7.544,4.491)--(-7.556,4.464)--cycle;
\draw(-7.544,4.491)--(-7.556,4.464);
\filldraw[fill opacity=0.8,fill=gray!20,draw=none](-7.531,4.503)--(-7.533,4.501)--(-7.532,4.515)--(-7.53,4.518)--(-7.529,4.508)--cycle;
\draw(-7.53,4.518)--(-7.529,4.508);
\filldraw[fill opacity=0.8,fill=gray!20,draw=none](-7.548,4.468)--(-7.552,4.465)--(-7.542,4.467)--cycle;
\draw(-7.552,4.465)--(-7.542,4.467);
\filldraw[fill opacity=0.8,fill=gray!20,draw=none](-7.544,4.491)--(-7.532,4.515)--(-7.533,4.506)--cycle;
\filldraw[fill opacity=0.8,fill=gray!20,draw=none](-7.516,4.538)--(-7.526,4.495)--(-7.528,4.49)--(-7.529,4.508)--cycle;
\draw(-7.528,4.49)--(-7.529,4.508);
\filldraw[fill opacity=0.8,fill=gray!20,draw=none](-7.517,4.538)--(-7.523,4.524)--(-7.534,4.498)--cycle;
\draw(-7.523,4.524)--(-7.534,4.498);
\filldraw[fill opacity=0.8,fill=gray!20,draw=none](-7.569,4.469)--(-7.663,4.506)--(-7.663,4.506)--(-7.627,4.492)--cycle;
\filldraw[fill opacity=0.8,fill=gray!20,draw=none](-7.573,4.471)--(-7.573,4.469)--(-7.582,4.46)--(-7.597,4.459)--(-7.625,4.471)--(-7.627,4.492)--cycle;
\draw(-7.573,4.471)--(-7.573,4.469);
\draw(-7.625,4.471)--(-7.627,4.492);
\filldraw[fill opacity=0.8,fill=gray!20,draw=none](-7.627,4.492)--(-7.626,4.481)--(-7.672,4.504)--(-7.673,4.509)--cycle;
\draw(-7.627,4.492)--(-7.626,4.481);
\filldraw[fill opacity=0.8,fill=gray!20,draw=none](-7.547,4.481)--(-7.551,4.477)--(-7.544,4.491)--(-7.533,4.506)--(-7.533,4.501)--cycle;
\filldraw[fill opacity=0.8,fill=gray!20,draw=none](-7.533,4.499)--(-7.529,4.508)--(-7.528,4.49)--(-7.534,4.482)--cycle;
\draw(-7.529,4.508)--(-7.528,4.49);
\filldraw[fill opacity=0.8,fill=gray!20,draw=none](-7.551,4.468)--(-7.545,4.468)--(-7.541,4.478)--cycle;
\draw(-7.545,4.468)--(-7.541,4.478);
\filldraw[fill opacity=0.8,fill=gray!20,draw=none](-7.545,4.47)--(-7.51,4.484)--(-7.505,4.488)--(-7.509,4.487)--(-7.532,4.481)--cycle;
\draw(-7.51,4.484)--(-7.505,4.488);
\filldraw[fill opacity=0.8,fill=gray!20,draw=none](-7.619,4.486)--(-7.624,4.488)--(-7.667,4.507)--(-7.569,4.469)--(-7.567,4.468)--(-7.581,4.473)--cycle;
\draw(-7.567,4.468)--(-7.581,4.473);
\filldraw[fill opacity=0.8,fill=gray!20,draw=none](-7.566,4.478)--(-7.57,4.47)--(-7.572,4.47)--(-7.572,4.47)--cycle;
\filldraw[fill opacity=0.8,fill=gray!20,draw=none](-7.572,4.47)--(-7.574,4.519)--(-7.592,4.477)--cycle;
\draw(-7.574,4.519)--(-7.592,4.477);
\filldraw[fill opacity=0.8,fill=gray!20,draw=none](-7.569,4.511)--(-7.583,4.474)--(-7.566,4.468)--cycle;
\draw(-7.569,4.511)--(-7.583,4.474)--(-7.566,4.468);
\filldraw[fill opacity=0.8,fill=gray!20,draw=none](-7.551,4.493)--(-7.563,4.467)--(-7.568,4.469)--cycle;
\filldraw[fill opacity=0.8,fill=gray!20,draw=none](-7.56,4.465)--(-7.556,4.464)--(-7.552,4.473)--cycle;
\draw(-7.556,4.464)--(-7.552,4.473);
\filldraw[fill opacity=0.8,fill=gray!20,draw=none](-7.556,4.469)--(-7.551,4.468)--(-7.541,4.478)--(-7.531,4.503)--cycle;
\draw(-7.541,4.478)--(-7.531,4.503);
\filldraw[fill opacity=0.8,fill=gray!20,draw=none](-7.562,4.466)--(-7.559,4.465)--(-7.552,4.473)--(-7.544,4.491)--cycle;
\draw(-7.552,4.473)--(-7.544,4.491);
\filldraw[fill opacity=0.8,fill=gray!20,draw=none](-7.56,4.472)--(-7.553,4.473)--(-7.556,4.469)--(-7.561,4.466)--(-7.563,4.467)--cycle;
\filldraw[fill opacity=0.8,fill=gray!20,draw=none](-7.543,4.516)--(-7.564,4.467)--(-7.562,4.466)--(-7.544,4.491)--(-7.528,4.527)--cycle;
\draw(-7.544,4.491)--(-7.528,4.527);
\filldraw[fill opacity=0.8,fill=gray!20,draw=none](-7.558,4.472)--(-7.544,4.491)--(-7.553,4.473)--cycle;
\filldraw[fill opacity=0.8,fill=gray!20,draw=none](-7.572,4.47)--(-7.573,4.469)--(-7.573,4.471)--cycle;
\draw(-7.573,4.469)--(-7.573,4.471);
\filldraw[fill opacity=0.8,fill=gray!20,draw=none](-7.57,4.47)--(-7.57,4.47)--(-7.572,4.47)--cycle;
\filldraw[fill opacity=0.8,fill=gray!20,draw=none](-7.57,4.469)--(-7.564,4.467)--(-7.543,4.516)--(-7.553,4.509)--(-7.575,4.474)--cycle;
\filldraw[fill opacity=0.8,fill=gray!20,draw=none](-7.564,4.47)--(-7.556,4.469)--(-7.531,4.503)--(-7.523,4.524)--(-7.546,4.506)--(-7.566,4.473)--cycle;
\draw(-7.531,4.503)--(-7.523,4.524);
\filldraw[fill opacity=0.8,fill=gray!20,draw=none](-7.531,4.503)--(-7.533,4.499)--(-7.533,4.501)--cycle;
\filldraw[fill opacity=0.8,fill=gray!20,draw=none](-7.555,4.47)--(-7.553,4.473)--(-7.533,4.501)--(-7.534,4.482)--(-7.537,4.479)--cycle;
\filldraw[fill opacity=0.8,fill=gray!20,draw=none](-4.414,2.966)--(-4.416,2.986)--(-7.597,4.578)--(-7.603,4.534)--(-4.433,2.947)--cycle;
\draw(-4.416,2.986)--(-7.597,4.578)--(-7.603,4.534)--(-4.433,2.947);
\filldraw[fill opacity=0.8,fill=gray!20,draw=none](-4.409,2.933)--(-4.403,2.944)--(-4.404,2.945)--(-4.413,2.934)--cycle;
\draw(-4.404,2.945)--(-4.413,2.934);
\filldraw[fill opacity=0.8,fill=gray!20,draw=none](-4.398,2.936)--(-4.403,2.944)--(-4.406,2.939)--cycle;
\filldraw[fill opacity=0.8,fill=gray!20,draw=none](-4.395,2.932)--(-4.398,2.936)--(-4.406,2.939)--(-4.407,2.937)--(-4.397,2.931)--cycle;
\filldraw[fill opacity=0.8,fill=gray!20,draw=none](-4.398,2.911)--(-4.39,2.913)--(-4.394,2.928)--cycle;
\filldraw[fill opacity=0.8,fill=gray!20,draw=none](-4.399,2.93)--(-4.397,2.931)--(-4.407,2.937)--(-4.409,2.933)--cycle;
\filldraw[fill opacity=0.8,fill=gray!20,draw=none](-4.399,2.942)--(-4.416,2.986)--(-4.413,2.937)--(-4.406,2.934)--cycle;
\draw(-4.413,2.937)--(-4.406,2.934);
\filldraw[fill opacity=0.8,fill=gray!20](-4.279,2.974)--(-4.308,3.023)--(-4.368,3.011)--(-4.353,2.96)--cycle;
\filldraw[fill opacity=0.8,fill=gray!20](-8.134,1.753)--(-8.087,1.78)--(-8.093,1.786)--(-8.145,1.766)--cycle;
\filldraw[fill opacity=0.8,fill=gray!20](-7.846,1.683)--(-7.881,1.728)--(-7.906,1.711)--(-7.877,1.663)--cycle;
\filldraw[fill opacity=0.8,fill=gray!20](-4.255,2.864)--(-4.261,2.92)--(-4.343,2.904)--(-4.34,2.847)--cycle;
\filldraw[fill opacity=0.8,fill=gray!20,draw=none](-2.559,7.724)--(-2.557,7.743)--(-2.547,7.758)--(-2.53,7.74)--cycle;
\draw(-2.557,7.743)--(-2.547,7.758)--(-2.53,7.74);
\filldraw[fill opacity=0.8,fill=gray!20](-2.547,7.758)--(-2.527,7.812)--(-2.505,7.788)--(-2.527,7.737)--cycle;
\filldraw[fill opacity=0.8,fill=gray!20](-7.883,1.01)--(-7.927,1.038)--(-7.949,1.033)--(-7.925,1.002)--cycle;
\filldraw[fill opacity=0.8,fill=gray!20,draw=none](-4.649,2.861)--(-4.73,2.902)--(-4.749,2.899)--(-4.684,2.866)--cycle;
\draw(-4.649,2.861)--(-4.73,2.902);
\draw(-4.749,2.899)--(-4.684,2.866);
\filldraw[fill opacity=0.8,fill=gray!20](-7.816,.707)--(-7.797,.757)--(-7.88,.741)--(-7.889,.693)--cycle;
\filldraw[fill opacity=0.8,fill=gray!20](-2.527,7.812)--(-2.52,7.868)--(-2.498,7.844)--(-2.505,7.788)--cycle;
\filldraw[fill opacity=0.8,fill=gray!20,draw=none](-4.401,2.743)--(-4.417,2.767)--(-4.593,2.504)--(-4.518,2.591)--cycle;
\draw(-4.417,2.767)--(-4.593,2.504);
\filldraw[fill opacity=0.8,fill=gray!20,draw=none](-4.38,2.771)--(-4.377,2.78)--(-4.448,2.695)--(-4.447,2.684)--cycle;
\draw(-4.377,2.78)--(-4.448,2.695);
\filldraw[fill opacity=0.8,fill=gray!20,draw=none](-4.399,2.942)--(-4.404,2.937)--(-4.394,2.928)--cycle;
\filldraw[fill opacity=0.8,fill=gray!20,draw=none](-4.395,2.932)--(-4.397,2.931)--(-4.393,2.929)--cycle;
\filldraw[fill opacity=0.8,fill=gray!20](-2.929,7.905)--(-2.907,7.956)--(-2.876,7.976)--(-2.894,7.927)--cycle;
\filldraw[fill opacity=0.8,fill=gray!20](-2.846,8.017)--(-2.808,8.045)--(-2.766,8.053)--(-2.786,8.028)--cycle;
\filldraw[fill opacity=0.8,fill=gray!20,draw=none](-8.194,1.464)--(-8.173,1.445)--(-8.141,1.437)--(-8.153,1.484)--(-8.219,1.501)--cycle;
\draw(-8.173,1.445)--(-8.141,1.437)--(-8.153,1.484)--(-8.219,1.501);
\filldraw[fill opacity=0.8,fill=gray!20](-8.049,1.477)--(-8.049,1.531)--(-8.157,1.539)--(-8.153,1.484)--cycle;
\filldraw[fill opacity=0.8,fill=gray!20](-8.348,1.386)--(-8.344,1.433)--(-8.281,1.418)--(-8.287,1.371)--cycle;
\filldraw[fill opacity=0.8,fill=gray!20,draw=none](-8.282,1.365)--(-8.287,1.371)--(-8.281,1.418)--(-8.274,1.41)--cycle;
\draw(-8.282,1.365)--(-8.287,1.371)--(-8.281,1.418)--(-8.274,1.41);
\filldraw[fill opacity=0.8,fill=gray!20](-7.918,1.566)--(-8.327,1.387)--(-8.336,1.339)--(-7.927,1.517)--cycle;
\filldraw[fill opacity=0.8,fill=gray!20,draw=none](-4.414,2.966)--(-4.433,2.947)--(-4.413,2.937)--cycle;
\draw(-4.433,2.947)--(-4.413,2.937);
\filldraw[fill opacity=0.8,fill=gray!20,draw=none](-4.404,2.937)--(-4.406,2.934)--(-4.394,2.928)--cycle;
\draw(-4.406,2.934)--(-4.394,2.928);
\filldraw[fill opacity=0.8,fill=gray!20,draw=none](-4.404,2.919)--(-4.41,2.919)--(-4.402,2.916)--cycle;
\filldraw[fill opacity=0.8,fill=gray!20,draw=none](-4.398,2.911)--(-4.394,2.928)--(-4.399,2.93)--(-4.414,2.934)--(-4.427,2.905)--cycle;
\draw(-4.394,2.928)--(-4.399,2.93);
\filldraw[fill opacity=0.8,fill=gray!20,draw=none](-4.451,2.812)--(-4.451,2.82)--(-4.461,2.843)--(-4.483,2.845)--cycle;
\draw(-4.451,2.812)--(-4.451,2.82);
\draw(-4.461,2.843)--(-4.483,2.845);
\filldraw[fill opacity=0.8,fill=gray!20,draw=none](-4.451,2.82)--(-4.451,2.842)--(-4.461,2.843)--cycle;
\draw(-4.451,2.82)--(-4.451,2.842)--(-4.461,2.843);
\filldraw[fill opacity=0.8,fill=gray!20,draw=none](-4.427,2.843)--(-4.451,2.862)--(-4.451,2.842)--cycle;
\draw(-4.451,2.862)--(-4.451,2.842)--(-4.427,2.843);
\filldraw[fill opacity=0.8,fill=gray!20,draw=none](-4.393,2.929)--(-4.397,2.931)--(-4.414,2.926)--(-4.571,2.668)--(-4.375,2.9)--cycle;
\draw(-4.571,2.668)--(-4.375,2.9);
\filldraw[fill opacity=0.8,fill=gray!20](-8.087,1.78)--(-8.036,1.793)--(-8.036,1.793)--(-8.093,1.786)--cycle;
\filldraw[fill opacity=0.8,fill=gray!20](-7.881,1.728)--(-7.926,1.763)--(-7.944,1.751)--(-7.906,1.711)--cycle;
\filldraw[fill opacity=0.8,fill=gray!20](-4.261,2.92)--(-4.279,2.974)--(-4.353,2.96)--(-4.343,2.904)--cycle;
\filldraw[fill opacity=0.8,fill=gray!20,draw=none](-4.499,3.056)--(-4.498,3.056)--(-4.497,3.057)--(-4.499,3.058)--cycle;
\draw(-4.499,3.056)--(-4.498,3.056)--(-4.497,3.057);
\filldraw[fill opacity=0.8,fill=gray!20,draw=none](-4.499,3.058)--(-7.603,4.612)--(-7.597,4.578)--(-4.496,3.026)--cycle;
\draw(-4.499,3.058)--(-7.603,4.612)--(-7.597,4.578)--(-4.496,3.026);
\filldraw[fill opacity=0.8,fill=gray!20,draw=none](-4.399,2.93)--(-4.413,2.937)--(-4.414,2.934)--cycle;
\draw(-4.399,2.93)--(-4.413,2.937);
\filldraw[fill opacity=0.8,fill=gray!20,draw=none](-4.399,2.93)--(-4.409,2.933)--(-4.414,2.926)--cycle;
\filldraw[fill opacity=0.8,fill=gray!20,draw=none](-4.496,3.055)--(-4.497,3.057)--(-4.498,3.056)--cycle;
\draw(-4.497,3.057)--(-4.498,3.056)--(-4.496,3.055);
\filldraw[fill opacity=0.8,fill=gray!20](-8.042,1.369)--(-8.045,1.393)--(-8.121,1.399)--(-8.096,1.373)--cycle;
\filldraw[fill opacity=0.8,fill=gray!20](-7.926,1.763)--(-7.979,1.785)--(-7.988,1.779)--(-7.944,1.751)--cycle;
\filldraw[fill opacity=0.8,fill=gray!20](-7.979,1.785)--(-8.036,1.793)--(-8.036,1.793)--(-7.988,1.779)--cycle;
\filldraw[fill opacity=0.8,fill=gray!20](-8.039,1.773)--(-8.036,1.793)--(-8.036,1.793)--(-8.067,1.775)--cycle;
\filldraw[fill opacity=0.8,fill=gray!20](-8.067,1.775)--(-8.036,1.793)--(-8.036,1.793)--(-8.087,1.78)--cycle;
\filldraw[fill opacity=0.8,fill=gray!20](-7.988,1.779)--(-8.036,1.793)--(-8.036,1.793)--(-8.01,1.774)--cycle;
\filldraw[fill opacity=0.8,fill=gray!20](-8.01,1.774)--(-8.036,1.793)--(-8.036,1.793)--(-8.039,1.773)--cycle;
\filldraw[fill opacity=0.8,fill=gray!20](-2.52,7.868)--(-2.527,7.923)--(-2.505,7.899)--(-2.498,7.844)--cycle;
\filldraw[fill opacity=0.8,fill=gray!20,draw=none](-4.414,2.772)--(-4.417,2.767)--(-4.413,2.762)--cycle;
\draw(-4.414,2.772)--(-4.417,2.767);
\filldraw[fill opacity=0.8,fill=gray!20,draw=none](-4.796,2.281)--(-4.827,2.233)--(-4.713,2.333)--(-4.652,2.415)--(-4.633,2.444)--cycle;
\draw(-4.796,2.281)--(-4.827,2.233);
\draw(-4.652,2.415)--(-4.633,2.444);
\filldraw[fill opacity=0.8,fill=gray!20,draw=none](-4.904,2.181)--(-4.827,2.233)--(-4.796,2.281)--cycle;
\draw(-4.827,2.233)--(-4.796,2.281);
\filldraw[fill opacity=0.8,fill=gray!20,draw=none](-4.836,2.228)--(-4.931,2.12)--(-4.897,2.129)--(-4.827,2.233)--cycle;
\draw(-4.897,2.129)--(-4.827,2.233);
\filldraw[fill opacity=0.8,fill=gray!20,draw=none](-4.931,2.12)--(-5.064,1.971)--(-5.819,.841)--(-5.168,1.724)--(-4.897,2.129)--cycle;
\draw(-5.064,1.971)--(-5.819,.841);
\draw(-5.168,1.724)--(-4.897,2.129);
\filldraw[fill opacity=0.8,fill=gray!20,draw=none](-4.796,2.281)--(-4.754,2.319)--(-5.784,1.093)--(-5.811,1.101)--(-4.948,2.128)--cycle;
\draw(-4.754,2.319)--(-5.784,1.093);
\draw(-5.811,1.101)--(-4.948,2.128);
\filldraw[fill opacity=0.8,fill=gray!20,draw=none](-4.796,2.281)--(-4.948,2.128)--(-4.904,2.181)--cycle;
\draw(-4.948,2.128)--(-4.904,2.181);
\filldraw[fill opacity=0.8,fill=gray!20,draw=none](-5.902,.988)--(-5.903,.985)--(-5.907,.986)--cycle;
\filldraw[fill opacity=0.8,fill=gray!20,draw=none](-5.894,1.012)--(-5.894,.983)--(-5.903,.985)--cycle;
\draw(-5.894,1.012)--(-5.894,.983);
\filldraw[fill opacity=0.8,fill=gray!20,draw=none](-5.886,.978)--(-5.894,.971)--(-5.894,.983)--cycle;
\draw(-5.894,.971)--(-5.894,.983);
\filldraw[fill opacity=0.8,fill=gray!20,draw=none](-5.859,.989)--(-5.859,.976)--(-5.864,.964)--(-5.894,.983)--(-5.894,.99)--cycle;
\draw(-5.859,.989)--(-5.859,.976);
\draw(-5.894,.983)--(-5.894,.99);
\filldraw[fill opacity=0.8,fill=gray!20,draw=none](-5.888,.97)--(-5.904,.955)--(-5.909,.957)--(-5.916,.976)--cycle;
\draw(-5.904,.955)--(-5.909,.957);
\filldraw[fill opacity=0.8,fill=gray!20,draw=none](-5.864,.964)--(-5.859,.96)--(-5.859,.879)--(-5.894,.888)--cycle;
\draw(-5.859,.96)--(-5.859,.879)--(-5.894,.888);
\filldraw[fill opacity=0.8,fill=gray!20,draw=none](-5.844,.93)--(-5.844,.867)--(-5.859,.879)--(-5.859,.96)--cycle;
\draw(-5.844,.93)--(-5.844,.867)--(-5.859,.879)--(-5.859,.96);
\filldraw[fill opacity=0.8,fill=gray!20,draw=none](-5.879,.89)--(-5.923,.909)--(-5.909,.957)--(-5.851,.932)--cycle;
\draw(-5.879,.89)--(-5.923,.909);
\draw(-5.909,.957)--(-5.851,.932);
\filldraw[fill opacity=0.8,fill=gray!20,draw=none](-5.894,.99)--(-5.872,.989)--(-5.888,.97)--(-5.916,.976)--(-5.907,.986)--cycle;
\draw(-5.872,.989)--(-5.888,.97);
\draw(-5.916,.976)--(-5.907,.986);
\filldraw[fill opacity=0.8,fill=gray!20,draw=none](-5.859,1)--(-5.859,.989)--(-5.872,.989)--cycle;
\draw(-5.859,1)--(-5.859,.989);
\filldraw[fill opacity=0.8,fill=gray!20,draw=none](-5.859,1)--(-5.846,1.011)--(-5.851,1.004)--cycle;
\draw(-5.846,1.011)--(-5.851,1.004);
\filldraw[fill opacity=0.8,fill=gray!20,draw=none](-5.849,1.012)--(-5.859,.976)--(-5.859,1.057)--cycle;
\draw(-5.859,.976)--(-5.859,1.057);
\filldraw[fill opacity=0.8,fill=gray!20,draw=none](-5.884,.946)--(-5.904,.955)--(-5.859,.997)--(-5.856,.996)--cycle;
\draw(-5.884,.946)--(-5.904,.955);
\draw(-5.859,.997)--(-5.856,.996);
\filldraw[fill opacity=0.8,fill=gray!20,draw=none](-5.854,1.001)--(-5.856,.996)--(-5.859,.997)--cycle;
\draw(-5.856,.996)--(-5.859,.997);
\filldraw[fill opacity=0.8,fill=gray!20,draw=none](-5.859,1)--(-5.851,1.004)--(-5.854,1.001)--(-5.879,.981)--(-5.872,.989)--cycle;
\draw(-5.851,1.004)--(-5.854,1.001);
\draw(-5.879,.981)--(-5.872,.989);
\filldraw[fill opacity=0.8,fill=gray!20,draw=none](-5.854,1.001)--(-5.885,.964)--(-5.913,.94)--(-5.879,.981)--cycle;
\draw(-5.854,1.001)--(-5.885,.964)--(-5.913,.94)--(-5.879,.981);
\filldraw[fill opacity=0.8,fill=gray!20,draw=none](-5.888,.97)--(-5.916,.976)--(-5.922,.99)--(-5.894,1.012)--(-5.859,.997)--cycle;
\draw(-5.894,1.012)--(-5.859,.997);
\filldraw[fill opacity=0.8,fill=gray!20,draw=none](-5.849,1.012)--(-5.854,1.001)--(-5.859,.997)--(-5.894,1.012)--(-5.859,1.057)--cycle;
\draw(-5.859,.997)--(-5.894,1.012);
\filldraw[fill opacity=0.8,fill=gray!20,draw=none](-5.836,1.043)--(-5.842,1.028)--(-5.842,1.049)--(-5.835,1.047)--cycle;
\draw(-5.842,1.049)--(-5.835,1.047);
\filldraw[fill opacity=0.8,fill=gray!20,draw=none](-4.382,2.823)--(-5.878,1.042)--(-5.873,1)--(-4.377,2.78)--cycle;
\draw(-4.382,2.823)--(-5.878,1.042)--(-5.873,1)--(-4.377,2.78);
\filldraw[fill opacity=0.8,fill=gray!20,draw=none](-4.488,3.022)--(-4.481,3.044)--(-4.496,3.055)--(-4.498,3.056)--(-4.499,3.055)--cycle;
\draw(-4.496,3.055)--(-4.498,3.056)--(-4.499,3.055);
\filldraw[fill opacity=0.8,fill=gray!20,draw=none](-8.222,1.508)--(-8.219,1.501)--(-8.153,1.484)--(-8.157,1.539)--(-8.232,1.557)--cycle;
\draw(-8.219,1.501)--(-8.153,1.484)--(-8.157,1.539)--(-8.232,1.557);
\filldraw[fill opacity=0.8,fill=gray!20](-8.049,1.531)--(-8.049,1.588)--(-8.153,1.595)--(-8.157,1.539)--cycle;
\filldraw[fill opacity=0.8,fill=gray!20,draw=none](-8.006,1.59)--(-8.028,1.627)--(-8.047,1.639)--(-8.049,1.588)--cycle;
\draw(-8.047,1.639)--(-8.049,1.588)--(-8.006,1.59);
\filldraw[fill opacity=0.8,fill=gray!20,draw=none](-8.348,1.478)--(-8.358,1.518)--(-8.304,1.505)--(-8.303,1.503)--(-8.287,1.463)--cycle;
\draw(-8.303,1.503)--(-8.287,1.463)--(-8.348,1.478)--(-8.358,1.518)--(-8.304,1.505);
\filldraw[fill opacity=0.8,fill=gray!20](-7.991,1.692)--(-8.4,1.514)--(-8.362,1.482)--(-7.953,1.66)--cycle;
\filldraw[fill opacity=0.8,fill=gray!20,draw=none](-4.447,3.017)--(-4.444,3.052)--(-4.496,3.055)--cycle;
\draw(-4.447,3.017)--(-4.444,3.052)--(-4.496,3.055);
\filldraw[fill opacity=0.8,fill=gray!20](-8.06,.961)--(-8.035,1.003)--(-8.073,1.012)--(-8.113,.974)--cycle;
\filldraw[fill opacity=0.8,fill=gray!20](-7.906,1.408)--(-7.877,1.449)--(-7.951,1.434)--(-7.966,1.397)--cycle;
\filldraw[fill opacity=0.8,fill=gray!20,draw=none](-4.4,2.729)--(-4.382,2.745)--(-4.397,2.744)--cycle;
\draw(-4.382,2.745)--(-4.397,2.744);
\filldraw[fill opacity=0.8,fill=gray!20](-2.631,8.026)--(-2.656,8.052)--(-2.619,8.043)--(-2.578,8.013)--cycle;
\filldraw[fill opacity=0.8,fill=gray!20,draw=none](-8.061,.658)--(-8.06,.658)--(-8.06,.659)--cycle;
\draw(-8.061,.658)--(-8.06,.658)--(-8.06,.659);
\filldraw[fill opacity=0.8,fill=gray!20,draw=none](-4.51,2.857)--(-4.487,2.845)--(-4.483,2.845)--(-4.512,2.886)--cycle;
\draw(-4.487,2.845)--(-4.483,2.845);
\filldraw[fill opacity=0.8,fill=gray!20,draw=none](-4.482,2.843)--(-4.483,2.845)--(-4.487,2.845)--cycle;
\draw(-4.483,2.845)--(-4.487,2.845);
\filldraw[fill opacity=0.8,fill=gray!20,draw=none](-5.916,1.169)--(-5.903,1.161)--(-5.873,1.143)--(-5.942,1.173)--cycle;
\draw(-5.873,1.143)--(-5.942,1.173);
\filldraw[fill opacity=0.8,fill=gray!20,draw=none](-4.409,2.933)--(-4.413,2.934)--(-5.937,1.121)--(-5.901,1.085)--(-4.571,2.668)--cycle;
\draw(-4.413,2.934)--(-5.937,1.121)--(-5.901,1.085)--(-4.571,2.668);
\filldraw[fill opacity=0.8,fill=gray!20](-7.981,.999)--(-7.978,1.032)--(-8.006,1.034)--(-8.035,1.003)--cycle;
\filldraw[fill opacity=0.8,fill=gray!20,draw=none](-7.984,.652)--(-7.984,.658)--(-8.043,.671)--(-8.06,.659)--(-8.06,.658)--cycle;
\draw(-8.06,.659)--(-8.06,.658)--(-7.984,.652)--(-7.984,.658);
\filldraw[fill opacity=0.8,fill=gray!20](-7.987,1.372)--(-7.966,1.397)--(-8.045,1.393)--(-8.042,1.369)--cycle;
\filldraw[fill opacity=0.8,fill=gray!20,draw=none](-4.435,2.939)--(-4.463,2.952)--(-4.493,2.917)--(-4.426,2.919)--cycle;
\draw(-4.463,2.952)--(-4.493,2.917);
\filldraw[fill opacity=0.8,fill=gray!20,draw=none](-4.435,2.939)--(-4.426,2.919)--(-4.416,2.931)--cycle;
\draw(-4.426,2.919)--(-4.416,2.931);
\filldraw[fill opacity=0.8,fill=gray!20,draw=none](-4.414,2.934)--(-4.413,2.937)--(-4.433,2.947)--(-4.487,2.918)--(-4.484,2.917)--cycle;
\draw(-4.413,2.937)--(-4.433,2.947);
\draw(-4.487,2.918)--(-4.484,2.917);
\filldraw[fill opacity=0.8,fill=gray!20](-4.368,3.011)--(-4.389,3.054)--(-4.444,3.052)--(-4.447,3.008)--cycle;
\filldraw[fill opacity=0.8,fill=gray!20,draw=none](-4.42,2.932)--(-4.416,2.931)--(-4.413,2.934)--cycle;
\draw(-4.416,2.931)--(-4.413,2.934);
\filldraw[fill opacity=0.8,fill=gray!20,draw=none](-6.043,1.177)--(-6.05,1.176)--(-6.033,1.196)--cycle;
\draw(-6.05,1.176)--(-6.033,1.196);
\filldraw[fill opacity=0.8,fill=gray!20,draw=none](-4.597,2.898)--(-4.747,2.727)--(-6.05,1.176)--(-6.043,1.174)--(-4.589,2.905)--cycle;
\draw(-4.747,2.727)--(-6.05,1.176);
\draw(-6.043,1.174)--(-4.589,2.905);
\filldraw[fill opacity=0.8,fill=gray!20](-7.925,1.002)--(-7.949,1.033)--(-7.978,1.032)--(-7.981,.999)--cycle;
\filldraw[fill opacity=0.8,fill=gray!20,draw=none](-7.992,1.533)--(-8.001,1.584)--(-8.006,1.59)--(-8.049,1.588)--(-8.049,1.531)--cycle;
\draw(-8.006,1.59)--(-8.049,1.588)--(-8.049,1.531)--(-7.992,1.533);
\filldraw[fill opacity=0.8,fill=gray!20](-8.344,1.433)--(-8.348,1.478)--(-8.287,1.463)--(-8.281,1.418)--cycle;
\filldraw[fill opacity=0.8,fill=gray!20,draw=none](-8.274,1.41)--(-8.281,1.418)--(-8.287,1.463)--(-8.282,1.458)--cycle;
\draw(-8.274,1.41)--(-8.281,1.418)--(-8.287,1.463)--(-8.282,1.458);
\filldraw[fill opacity=0.8,fill=gray!20,draw=none](-8.283,1.461)--(-8.282,1.458)--(-8.287,1.463)--(-8.303,1.503)--cycle;
\draw(-8.282,1.458)--(-8.287,1.463)--(-8.303,1.503);
\filldraw[fill opacity=0.8,fill=gray!20](-7.927,1.616)--(-8.336,1.437)--(-8.327,1.387)--(-7.918,1.566)--cycle;
\filldraw[fill opacity=0.8,fill=gray!20,draw=none](-4.387,2.776)--(-4.394,2.753)--(-4.393,2.744)--(-4.353,2.746)--(-4.343,2.793)--(-4.381,2.792)--cycle;
\draw(-4.393,2.744)--(-4.353,2.746)--(-4.343,2.793)--(-4.381,2.792);
\filldraw[fill opacity=0.8,fill=gray!20](-7.845,.97)--(-7.883,1.01)--(-7.925,1.002)--(-7.905,.959)--cycle;
\filldraw[fill opacity=0.8,fill=gray!20](-7.953,1.66)--(-8.362,1.482)--(-8.336,1.437)--(-7.927,1.616)--cycle;
\filldraw[fill opacity=0.8,fill=gray!20,draw=none](-4.447,2.684)--(-4.518,2.591)--(-4.446,2.677)--cycle;
\draw(-4.518,2.591)--(-4.446,2.677);
\filldraw[fill opacity=0.8,fill=gray!20,draw=none](-4.401,2.743)--(-4.518,2.591)--(-4.424,2.699)--(-4.398,2.738)--cycle;
\draw(-4.424,2.699)--(-4.398,2.738);
\filldraw[fill opacity=0.8,fill=gray!20,draw=none](-4.398,2.738)--(-4.397,2.744)--(-4.402,2.744)--cycle;
\draw(-4.397,2.744)--(-4.402,2.744);
\filldraw[fill opacity=0.8,fill=gray!20,draw=none](-8.137,.924)--(-8.097,.97)--(-8.113,.974)--(-8.145,.926)--cycle;
\draw(-8.097,.97)--(-8.113,.974)--(-8.145,.926)--(-8.137,.924);
\filldraw[fill opacity=0.8,fill=gray!20](-7.797,.757)--(-7.791,.811)--(-7.876,.795)--(-7.88,.741)--cycle;
\filldraw[fill opacity=0.8,fill=gray!20](-7.905,.656)--(-7.889,.693)--(-7.986,.689)--(-7.984,.652)--cycle;
\filldraw[fill opacity=0.8,fill=gray!20,draw=none](-4.488,3.022)--(-4.484,3.01)--(-4.447,3.008)--(-4.447,3.017)--(-4.481,3.044)--cycle;
\draw(-4.484,3.01)--(-4.447,3.008)--(-4.447,3.017);
\filldraw[fill opacity=0.8,fill=gray!20](-2.527,7.923)--(-2.547,7.972)--(-2.527,7.951)--(-2.505,7.899)--cycle;
\filldraw[fill opacity=0.8,fill=gray!20,draw=none](-4.639,2.855)--(-4.638,2.855)--(-4.64,2.856)--cycle;
\draw(-4.638,2.855)--(-4.64,2.856);
\filldraw[fill opacity=0.8,fill=gray!20](-2.907,7.956)--(-2.872,8)--(-2.846,8.017)--(-2.876,7.976)--cycle;
\filldraw[fill opacity=0.8,fill=gray!20,draw=none](-4.519,2.533)--(-5.963,.371)--(-5.963,.37)--(-5.953,.396)--(-4.507,2.561)--cycle;
\draw(-4.519,2.533)--(-5.963,.371);
\draw(-5.953,.396)--(-4.507,2.561);
\filldraw[fill opacity=0.8,fill=gray!20,draw=none](-4.912,1.955)--(-5.984,.35)--(-6.001,.37)--(-4.783,2.193)--cycle;
\draw(-4.912,1.955)--(-5.984,.35);
\draw(-6.001,.37)--(-4.783,2.193);
\filldraw[fill opacity=0.8,fill=gray!20,draw=none](-6.068,.455)--(-6.052,.455)--(-6.062,.44)--cycle;
\draw(-6.052,.455)--(-6.062,.44);
\filldraw[fill opacity=0.8,fill=gray!20,draw=none](-6.023,.354)--(-6.041,.383)--(-6.029,.4)--cycle;
\draw(-6.041,.383)--(-6.029,.4);
\filldraw[fill opacity=0.8,fill=gray!20,draw=none](-6.017,.315)--(-6.017,.228)--(-6.067,.234)--(-6.067,.257)--cycle;
\draw(-6.017,.315)--(-6.017,.228)--(-6.067,.234)--(-6.067,.257);
\filldraw[fill opacity=0.8,fill=gray!20,draw=none](-5.984,.35)--(-5.985,.348)--(-6.016,.346)--(-6.001,.37)--cycle;
\draw(-5.984,.35)--(-5.985,.348);
\draw(-6.016,.346)--(-6.001,.37);
\filldraw[fill opacity=0.8,fill=gray!20,draw=none](-6.017,.352)--(-6.034,.359)--(-6.054,.428)--(-6,.404)--cycle;
\draw(-6.017,.352)--(-6.034,.359);
\draw(-6.054,.428)--(-6,.404);
\filldraw[fill opacity=0.8,fill=gray!20,draw=none](-6.067,.29)--(-6.067,.25)--(-6.123,.26)--(-6.123,.282)--cycle;
\draw(-6.067,.29)--(-6.067,.25);
\draw(-6.123,.26)--(-6.123,.282);
\filldraw[fill opacity=0.8,fill=gray!20,draw=none](-6,.404)--(-6.054,.428)--(-6.067,.449)--(-6.074,.486)--(-6.017,.461)--cycle;
\draw(-6,.404)--(-6.054,.428);
\draw(-6.067,.449)--(-6.074,.486)--(-6.017,.461);
\filldraw[fill opacity=0.8,fill=gray!20,draw=none](-5.982,.396)--(-5.982,.351)--(-6.014,.318)--(-6.017,.322)--(-6.017,.352)--cycle;
\draw(-5.982,.396)--(-5.982,.351);
\draw(-6.017,.322)--(-6.017,.352);
\filldraw[fill opacity=0.8,fill=gray!20,draw=none](-6.039,.337)--(-6.017,.322)--(-6.017,.315)--(-6.067,.257)--(-6.067,.326)--cycle;
\draw(-6.017,.322)--(-6.017,.315);
\draw(-6.067,.257)--(-6.067,.326);
\filldraw[fill opacity=0.8,fill=gray!20,draw=none](-6.114,.454)--(-6.068,.455)--(-6.067,.449)--(-6.097,.436)--cycle;
\draw(-6.068,.455)--(-6.067,.449);
\filldraw[fill opacity=0.8,fill=gray!20,draw=none](-6.022,.348)--(-6.037,.336)--(-6.034,.359)--(-6.023,.354)--cycle;
\draw(-6.034,.359)--(-6.023,.354);
\filldraw[fill opacity=0.8,fill=gray!20,draw=none](-6.023,.354)--(-6.022,.348)--(-6.061,.353)--(-6.041,.383)--cycle;
\draw(-6.061,.353)--(-6.041,.383);
\filldraw[fill opacity=0.8,fill=gray!20,draw=none](-6.035,.349)--(-6.036,.342)--(-6.041,.342)--cycle;
\filldraw[fill opacity=0.8,fill=gray!20,draw=none](-6.023,.354)--(-6.019,.347)--(-6.022,.348)--cycle;
\filldraw[fill opacity=0.8,fill=gray!20,draw=none](-6.022,.348)--(-6.023,.354)--(-6.017,.352)--cycle;
\draw(-6.023,.354)--(-6.017,.352);
\filldraw[fill opacity=0.8,fill=gray!20,draw=none](-6.016,.347)--(-6.019,.347)--(-6.019,.349)--cycle;
\filldraw[fill opacity=0.8,fill=gray!20,draw=none](-6.017,.461)--(-6.029,.466)--(-6.067,.513)--cycle;
\draw(-6.017,.461)--(-6.029,.466);
\filldraw[fill opacity=0.8,fill=gray!20,draw=none](-6.025,.469)--(-6.017,.461)--(-6.017,.322)--(-6.067,.357)--(-6.067,.367)--cycle;
\draw(-6.017,.461)--(-6.017,.322);
\draw(-6.067,.357)--(-6.067,.367);
\filldraw[fill opacity=0.8,fill=gray!20,draw=none](-5.982,.272)--(-5.982,.218)--(-6.017,.228)--(-6.017,.322)--cycle;
\draw(-5.982,.272)--(-5.982,.218)--(-6.017,.228)--(-6.017,.322);
\filldraw[fill opacity=0.8,fill=gray!20,draw=none](-5.967,.213)--(-5.967,.206)--(-5.982,.218)--(-5.982,.272)--cycle;
\draw(-5.967,.213)--(-5.967,.206)--(-5.982,.218)--(-5.982,.272);
\filldraw[fill opacity=0.8,fill=gray!20,draw=none](-5.982,.303)--(-5.982,.272)--(-5.989,.281)--cycle;
\draw(-5.982,.303)--(-5.982,.272);
\filldraw[fill opacity=0.8,fill=gray!20,draw=none](-6.033,.242)--(-6.123,.282)--(-6.067,.311)--(-6.003,.284)--cycle;
\draw(-6.033,.242)--(-6.123,.282);
\draw(-6.067,.311)--(-6.003,.284);
\filldraw[fill opacity=0.8,fill=gray!20,draw=none](-6.037,.336)--(-6.041,.3)--(-6.067,.311)--cycle;
\draw(-6.041,.3)--(-6.067,.311);
\filldraw[fill opacity=0.8,fill=gray!20,draw=none](-6.022,.348)--(-6.019,.347)--(-6.017,.345)--(-6.021,.34)--cycle;
\draw(-6.017,.345)--(-6.021,.34);
\filldraw[fill opacity=0.8,fill=gray!20,draw=none](-5.982,.351)--(-5.982,.303)--(-5.989,.281)--(-6.014,.318)--cycle;
\draw(-5.982,.351)--(-5.982,.303);
\filldraw[fill opacity=0.8,fill=gray!20,draw=none](-6.016,.347)--(-6.017,.345)--(-6.019,.347)--cycle;
\draw(-6.016,.347)--(-6.017,.345);
\filldraw[fill opacity=0.8,fill=gray!20,draw=none](-5.985,.348)--(-5.992,.338)--(-6.017,.345)--(-6.016,.346)--cycle;
\draw(-5.985,.348)--(-5.992,.338);
\draw(-6.017,.345)--(-6.016,.346);
\filldraw[fill opacity=0.8,fill=gray!20,draw=none](-5.992,.338)--(-5.993,.336)--(-6.021,.34)--(-6.017,.345)--cycle;
\draw(-5.992,.338)--(-5.993,.336);
\draw(-6.021,.34)--(-6.017,.345);
\filldraw[fill opacity=0.8,fill=gray!20,draw=none](-5.993,.336)--(-6.021,.34)--(-6.022,.348)--(-6.017,.352)--(-5.99,.34)--cycle;
\draw(-6.017,.352)--(-5.99,.34);
\filldraw[fill opacity=0.8,fill=gray!20,draw=none](-5.963,.37)--(-5.979,.357)--(-5.953,.396)--cycle;
\draw(-5.979,.357)--(-5.953,.396);
\filldraw[fill opacity=0.8,fill=gray!20,draw=none](-5.97,.365)--(-5.979,.353)--(-5.984,.35)--(-5.979,.357)--cycle;
\draw(-5.984,.35)--(-5.979,.357);
\filldraw[fill opacity=0.8,fill=gray!20,draw=none](-6.017,.352)--(-6,.404)--(-5.982,.396)--cycle;
\draw(-6,.404)--(-5.982,.396);
\filldraw[fill opacity=0.8,fill=gray!20,draw=none](-5.982,.396)--(-6,.404)--(-6.017,.461)--cycle;
\draw(-5.982,.396)--(-6,.404);
\filldraw[fill opacity=0.8,fill=gray!20,draw=none](-5.982,.435)--(-5.982,.396)--(-6.017,.352)--(-6.017,.461)--cycle;
\draw(-5.982,.435)--(-5.982,.396);
\draw(-6.017,.352)--(-6.017,.461);
\filldraw[fill opacity=0.8,fill=gray!20,draw=none](-5.982,.575)--(-5.982,.435)--(-6.017,.461)--(-6.017,.585)--cycle;
\draw(-6.017,.461)--(-6.017,.585)--(-5.982,.575)--(-5.982,.435);
\filldraw[fill opacity=0.8,fill=gray!20,draw=none](-5.967,.439)--(-5.982,.396)--(-5.982,.435)--cycle;
\draw(-5.982,.396)--(-5.982,.435);
\filldraw[fill opacity=0.8,fill=gray!20,draw=none](-5.976,.57)--(-5.969,.525)--(-5.967,.506)--(-5.967,.439)--(-5.982,.435)--(-5.982,.575)--cycle;
\draw(-5.967,.506)--(-5.967,.439);
\draw(-5.982,.435)--(-5.982,.575)--(-5.976,.57);
\filldraw[fill opacity=0.8,fill=gray!20,draw=none](-5.964,.389)--(-5.982,.396)--(-6.017,.461)--(-5.967,.439)--cycle;
\draw(-5.964,.389)--(-5.982,.396);
\draw(-6.017,.461)--(-5.967,.439);
\filldraw[fill opacity=0.8,fill=gray!20,draw=none](-5.979,.353)--(-5.974,.343)--(-5.982,.303)--(-5.982,.351)--cycle;
\draw(-5.982,.303)--(-5.982,.351);
\filldraw[fill opacity=0.8,fill=gray!20,draw=none](-5.979,.353)--(-5.993,.336)--(-5.984,.35)--cycle;
\draw(-5.993,.336)--(-5.984,.35);
\filldraw[fill opacity=0.8,fill=gray!20,draw=none](-6.003,.284)--(-6.014,.289)--(-6.021,.34)--(-5.978,.335)--cycle;
\draw(-6.003,.284)--(-6.014,.289);
\filldraw[fill opacity=0.8,fill=gray!20,draw=none](-5.973,.357)--(-5.979,.353)--(-5.97,.365)--(-5.963,.37)--(-5.964,.369)--(-5.968,.362)--cycle;
\draw(-5.964,.369)--(-5.968,.362);
\filldraw[fill opacity=0.8,fill=gray!20,draw=none](-5.973,.357)--(-5.982,.351)--(-5.982,.396)--cycle;
\draw(-5.982,.351)--(-5.982,.396);
\filldraw[fill opacity=0.8,fill=gray!20,draw=none](-6.017,.307)--(-6.014,.289)--(-6.027,.294)--cycle;
\draw(-6.014,.289)--(-6.027,.294);
\filldraw[fill opacity=0.8,fill=gray!20,draw=none](-5.993,.336)--(-5.99,.34)--(-5.978,.335)--cycle;
\draw(-5.99,.34)--(-5.978,.335);
\filldraw[fill opacity=0.8,fill=gray!20,draw=none](-5.979,.353)--(-5.973,.357)--(-5.972,.352)--(-5.974,.343)--cycle;
\filldraw[fill opacity=0.8,fill=gray!20,draw=none](-5.973,.357)--(-5.972,.352)--(-5.978,.335)--(-5.99,.34)--cycle;
\draw(-5.978,.335)--(-5.99,.34);
\filldraw[fill opacity=0.8,fill=gray!20,draw=none](-5.968,.362)--(-5.972,.352)--(-5.973,.357)--cycle;
\filldraw[fill opacity=0.8,fill=gray!20,draw=none](-5.973,.357)--(-5.97,.359)--(-5.972,.352)--cycle;
\filldraw[fill opacity=0.8,fill=gray!20,draw=none](-5.979,.353)--(-5.97,.359)--(-6,.314)--(-6.022,.293)--(-5.993,.336)--cycle;
\draw(-5.97,.359)--(-6,.314)--(-6.022,.293)--(-5.993,.336);
\filldraw[fill opacity=0.8,fill=gray!20,draw=none](-5.973,.357)--(-5.99,.34)--(-6.017,.352)--(-5.982,.396)--cycle;
\draw(-5.99,.34)--(-6.017,.352);
\filldraw[fill opacity=0.8,fill=gray!20,draw=none](-6.03,.341)--(-6.021,.34)--(-6.017,.307)--(-6.027,.294)--(-6.041,.3)--(-6.037,.336)--cycle;
\draw(-6.027,.294)--(-6.041,.3);
\filldraw[fill opacity=0.8,fill=gray!20,draw=none](-5.993,.336)--(-6.022,.293)--(-6.057,.286)--(-6.021,.34)--cycle;
\draw(-5.993,.336)--(-6.022,.293)--(-6.057,.286)--(-6.021,.34);
\filldraw[fill opacity=0.8,fill=gray!20,draw=none](-5.965,.371)--(-5.968,.362)--(-5.973,.357)--(-5.982,.396)--(-5.964,.389)--cycle;
\draw(-5.982,.396)--(-5.964,.389);
\filldraw[fill opacity=0.8,fill=gray!20,draw=none](-5.967,.439)--(-5.967,.376)--(-5.97,.359)--(-5.973,.357)--(-5.982,.396)--cycle;
\draw(-5.967,.439)--(-5.967,.376);
\filldraw[fill opacity=0.8,fill=gray!20,draw=none](-5.973,.357)--(-5.968,.362)--(-5.97,.359)--cycle;
\draw(-5.968,.362)--(-5.97,.359);
\filldraw[fill opacity=0.8,fill=gray!20,draw=none](-5.965,.371)--(-5.964,.389)--(-5.96,.387)--cycle;
\draw(-5.964,.389)--(-5.96,.387);
\filldraw[fill opacity=0.8,fill=gray!20,draw=none](-5.957,.405)--(-5.96,.387)--(-5.964,.389)--(-5.967,.439)--(-5.96,.436)--cycle;
\draw(-5.96,.387)--(-5.964,.389);
\draw(-5.967,.439)--(-5.96,.436);
\filldraw[fill opacity=0.8,fill=gray!20,draw=none](-6.029,.466)--(-6.074,.486)--(-6.102,.529)--(-6.067,.513)--cycle;
\draw(-6.029,.466)--(-6.074,.486)--(-6.102,.529)--(-6.067,.513);
\filldraw[fill opacity=0.8,fill=gray!20,draw=none](-6.017,.487)--(-6.049,.41)--(-6.067,.458)--(-6.067,.513)--cycle;
\draw(-6.067,.458)--(-6.067,.513);
\filldraw[fill opacity=0.8,fill=gray!20,draw=none](-6.017,.523)--(-6.017,.488)--(-6.017,.487)--(-6.067,.513)--(-6.067,.535)--cycle;
\draw(-6.017,.523)--(-6.017,.488);
\draw(-6.067,.513)--(-6.067,.535);
\filldraw[fill opacity=0.8,fill=gray!20,draw=none](-6.025,.469)--(-6.017,.488)--(-6.017,.461)--cycle;
\draw(-6.017,.488)--(-6.017,.461);
\filldraw[fill opacity=0.8,fill=gray!20,draw=none](-5.969,.467)--(-5.97,.454)--(-5.967,.439)--(-5.967,.465)--cycle;
\draw(-5.967,.439)--(-5.967,.465);
\filldraw[fill opacity=0.8,fill=gray!20,draw=none](-5.97,.454)--(-5.967,.439)--(-6.017,.461)--(-6.067,.513)--(-5.987,.479)--cycle;
\draw(-5.967,.439)--(-6.017,.461);
\draw(-6.067,.513)--(-5.987,.479);
\filldraw[fill opacity=0.8,fill=gray!20,draw=none](-5.97,.454)--(-5.971,.441)--(-5.97,.384)--(-5.967,.376)--(-5.967,.439)--cycle;
\draw(-5.967,.376)--(-5.967,.439);
\filldraw[fill opacity=0.8,fill=gray!20,draw=none](-4.442,2.673)--(-5.996,.345)--(-6,.314)--(-4.438,2.653)--cycle;
\draw(-4.442,2.673)--(-5.996,.345)--(-6,.314)--(-4.438,2.653);
\filldraw[fill opacity=0.8,fill=gray!20,draw=none](-4.433,2.947)--(-4.463,2.962)--(-4.507,2.928)--(-4.487,2.918)--cycle;
\draw(-4.433,2.947)--(-4.463,2.962);
\draw(-4.507,2.928)--(-4.487,2.918);
\filldraw[fill opacity=0.8,fill=gray!20,draw=none](-2.561,7.708)--(-2.559,7.724)--(-2.53,7.74)--(-2.527,7.737)--(-2.553,7.704)--cycle;
\draw(-2.53,7.74)--(-2.527,7.737)--(-2.553,7.704);
\filldraw[fill opacity=0.8,fill=gray!20,draw=none](-4.394,2.753)--(-4.397,2.744)--(-4.393,2.744)--cycle;
\draw(-4.397,2.744)--(-4.393,2.744);
\filldraw[fill opacity=0.8,fill=gray!20,draw=none](-4.395,2.758)--(-4.402,2.744)--(-4.397,2.744)--(-4.394,2.753)--cycle;
\draw(-4.402,2.744)--(-4.397,2.744);
\filldraw[fill opacity=0.8,fill=gray!20,draw=none](-4.625,2.865)--(-4.649,2.861)--(-4.638,2.855)--cycle;
\draw(-4.649,2.861)--(-4.638,2.855);
\filldraw[fill opacity=0.8,fill=gray!20](-8.096,1.744)--(-8.067,1.775)--(-8.087,1.78)--(-8.134,1.753)--cycle;
\filldraw[fill opacity=0.8,fill=gray!20,draw=none](-2.823,7.664)--(-2.826,7.662)--(-2.843,7.675)--(-2.829,7.675)--cycle;
\draw(-2.823,7.664)--(-2.826,7.662)--(-2.843,7.675);
\filldraw[fill opacity=0.8,fill=gray!20,draw=none](-4.395,2.758)--(-4.394,2.753)--(-4.387,2.776)--cycle;
\filldraw[fill opacity=0.8,fill=gray!20,draw=none](-4.493,2.917)--(-4.504,2.904)--(-4.465,2.873)--(-4.426,2.919)--cycle;
\draw(-4.493,2.917)--(-4.504,2.904);
\draw(-4.465,2.873)--(-4.426,2.919);
\filldraw[fill opacity=0.8,fill=gray!20,draw=none](-8.043,.671)--(-7.984,.658)--(-7.986,.687)--(-8.027,.682)--cycle;
\draw(-7.984,.658)--(-7.986,.687);
\filldraw[fill opacity=0.8,fill=gray!20,draw=none](-8.131,.931)--(-8.11,.932)--(-8.061,.961)--(-8.097,.97)--cycle;
\draw(-8.061,.961)--(-8.097,.97);
\filldraw[fill opacity=0.8,fill=gray!20,draw=none](-4.625,2.865)--(-4.615,2.874)--(-4.654,2.864)--(-4.649,2.861)--cycle;
\draw(-4.654,2.864)--(-4.649,2.861);
\filldraw[fill opacity=0.8,fill=gray!20](-7.944,1.751)--(-7.988,1.779)--(-8.01,1.774)--(-7.987,1.743)--cycle;
\filldraw[fill opacity=0.8,fill=gray!20](-7.816,.922)--(-7.845,.97)--(-7.905,.959)--(-7.889,.908)--cycle;
\filldraw[fill opacity=0.8,fill=gray!20,draw=none](-4.395,2.758)--(-4.387,2.776)--(-4.386,2.782)--(-4.394,2.791)--(-4.4,2.791)--cycle;
\draw(-4.394,2.791)--(-4.4,2.791);
\filldraw[fill opacity=0.8,fill=gray!20](-7.791,.811)--(-7.797,.867)--(-7.88,.852)--(-7.876,.795)--cycle;
\filldraw[fill opacity=0.8,fill=gray!20](-7.877,1.449)--(-7.859,1.498)--(-7.941,1.482)--(-7.951,1.434)--cycle;
\filldraw[fill opacity=0.8,fill=gray!20](-2.766,8.053)--(-2.742,8.065)--(-2.713,8.066)--(-2.71,8.056)--cycle;
\filldraw[fill opacity=0.8,fill=gray!20](-2.71,8.056)--(-2.713,8.066)--(-2.685,8.064)--(-2.656,8.052)--cycle;
\filldraw[fill opacity=0.8,fill=gray!20,draw=none](-4.73,2.902)--(-7.578,4.327)--(-7.612,4.332)--(-4.749,2.899)--cycle;
\draw(-4.73,2.902)--(-7.578,4.327);
\draw(-7.612,4.332)--(-4.749,2.899);
\filldraw[fill opacity=0.8,fill=gray!20,draw=none](-4.492,3.011)--(-4.484,3.01)--(-4.488,3.022)--cycle;
\draw(-4.492,3.011)--(-4.484,3.01);
\filldraw[fill opacity=0.8,fill=gray!20,draw=none](-7.639,4.355)--(-7.612,4.332)--(-7.621,4.309)--cycle;
\draw(-7.612,4.332)--(-7.621,4.309);
\filldraw[fill opacity=0.8,fill=gray!20,draw=none](-7.657,4.347)--(-7.665,4.368)--(-7.639,4.345)--cycle;
\filldraw[fill opacity=0.8,fill=gray!20,draw=none](-7.612,4.332)--(-7.639,4.345)--(-7.665,4.368)--(-7.639,4.355)--cycle;
\draw(-7.612,4.332)--(-7.639,4.345);
\draw(-7.665,4.368)--(-7.639,4.355);
\filldraw[fill opacity=0.8,fill=gray!20,draw=none](-7.665,4.368)--(-7.62,4.469)--(-7.589,4.455)--(-7.639,4.345)--cycle;
\draw(-7.665,4.368)--(-7.62,4.469);
\draw(-7.589,4.455)--(-7.639,4.345);
\filldraw[fill opacity=0.8,fill=gray!20,draw=none](-7.639,4.345)--(-7.594,4.445)--(-7.58,4.451)--(-7.575,4.449)--(-7.62,4.348)--cycle;
\draw(-7.639,4.345)--(-7.594,4.445);
\draw(-7.575,4.449)--(-7.62,4.348);
\filldraw[fill opacity=0.8,fill=gray!20,draw=none](-7.585,4.364)--(-7.591,4.338)--(-7.542,4.462)--(-7.544,4.465)--(-7.548,4.46)--(-7.58,4.379)--cycle;
\draw(-7.591,4.338)--(-7.542,4.462);
\draw(-7.548,4.46)--(-7.58,4.379);
\filldraw[fill opacity=0.8,fill=gray!20,draw=none](-7.62,4.348)--(-7.575,4.449)--(-7.578,4.451)--(-7.611,4.377)--cycle;
\draw(-7.62,4.348)--(-7.575,4.449);
\draw(-7.578,4.451)--(-7.611,4.377);
\filldraw[fill opacity=0.8,fill=gray!20,draw=none](-4.615,2.874)--(-4.612,2.876)--(-7.678,4.41)--(-7.71,4.393)--(-4.654,2.864)--cycle;
\draw(-4.612,2.876)--(-7.678,4.41)--(-7.71,4.393)--(-4.654,2.864);
\filldraw[fill opacity=0.8,fill=gray!20](-2.547,7.972)--(-2.578,8.013)--(-2.562,7.996)--(-2.527,7.951)--cycle;
\filldraw[fill opacity=0.8,fill=gray!20,draw=none](-4.552,2.908)--(-4.511,2.926)--(-4.508,2.929)--(-4.502,2.96)--(-4.543,2.963)--(-4.554,2.908)--cycle;
\draw(-4.502,2.96)--(-4.543,2.963)--(-4.554,2.908);
\filldraw[fill opacity=0.8,fill=gray!20,draw=none](-7.551,4.464)--(-7.541,4.465)--(-7.553,4.461)--cycle;
\draw(-7.541,4.465)--(-7.553,4.461);
\filldraw[fill opacity=0.8,fill=gray!20,draw=none](-7.583,4.453)--(-7.573,4.455)--(-7.529,4.47)--(-7.569,4.462)--(-7.573,4.461)--(-7.587,4.454)--cycle;
\draw(-7.583,4.453)--(-7.573,4.455)--(-7.529,4.47)--(-7.569,4.462);
\draw(-7.573,4.461)--(-7.587,4.454);
\filldraw[fill opacity=0.8,fill=gray!20,draw=none](-7.555,4.464)--(-7.564,4.444)--(-7.56,4.443)--(-7.553,4.46)--cycle;
\draw(-7.564,4.444)--(-7.56,4.443)--(-7.553,4.46);
\filldraw[fill opacity=0.8,fill=gray!20,draw=none](-7.544,4.465)--(-7.546,4.467)--(-7.548,4.46)--cycle;
\draw(-7.546,4.467)--(-7.548,4.46);
\filldraw[fill opacity=0.8,fill=gray!20,draw=none](-7.545,4.47)--(-7.548,4.468)--(-7.542,4.467)--(-7.529,4.47)--(-7.51,4.484)--cycle;
\draw(-7.542,4.467)--(-7.529,4.47)--(-7.51,4.484);
\filldraw[fill opacity=0.8,fill=gray!20,draw=none](-7.545,4.467)--(-7.545,4.468)--(-7.546,4.467)--cycle;
\draw(-7.545,4.468)--(-7.546,4.467);
\filldraw[fill opacity=0.8,fill=gray!20,draw=none](-7.555,4.464)--(-7.556,4.464)--(-7.564,4.445)--(-7.564,4.444)--cycle;
\draw(-7.556,4.464)--(-7.564,4.445)--(-7.564,4.444);
\filldraw[fill opacity=0.8,fill=gray!20,draw=none](-7.563,4.465)--(-7.569,4.462)--(-7.557,4.464)--cycle;
\draw(-7.569,4.462)--(-7.557,4.464);
\filldraw[fill opacity=0.8,fill=gray!20,draw=none](-7.562,4.466)--(-7.564,4.463)--(-7.561,4.464)--(-7.559,4.465)--cycle;
\filldraw[fill opacity=0.8,fill=gray!20,draw=none](-7.563,4.465)--(-7.557,4.464)--(-7.552,4.465)--(-7.532,4.481)--(-7.537,4.479)--cycle;
\draw(-7.557,4.464)--(-7.552,4.465);
\filldraw[fill opacity=0.8,fill=gray!20,draw=none](-7.575,4.449)--(-7.572,4.455)--(-7.575,4.459)--(-7.578,4.451)--cycle;
\draw(-7.575,4.449)--(-7.572,4.455);
\draw(-7.575,4.459)--(-7.578,4.451);
\filldraw[fill opacity=0.8,fill=gray!20,draw=none](-7.587,4.454)--(-7.577,4.459)--(-7.585,4.461)--(-7.6,4.46)--cycle;
\draw(-7.587,4.454)--(-7.577,4.459);
\draw(-7.585,4.461)--(-7.6,4.46);
\filldraw[fill opacity=0.8,fill=gray!20,draw=none](-7.563,4.462)--(-7.572,4.453)--(-7.575,4.449)--(-7.564,4.445)--(-7.557,4.46)--cycle;
\draw(-7.575,4.449)--(-7.564,4.445)--(-7.557,4.46);
\filldraw[fill opacity=0.8,fill=gray!20,draw=none](-7.557,4.462)--(-7.559,4.46)--(-7.565,4.445)--(-7.556,4.441)--(-7.548,4.46)--cycle;
\draw(-7.556,4.441)--(-7.548,4.46);
\filldraw[fill opacity=0.8,fill=gray!20,draw=none](-7.564,4.463)--(-7.565,4.462)--(-7.557,4.46)--(-7.556,4.464)--cycle;
\draw(-7.557,4.46)--(-7.556,4.464);
\filldraw[fill opacity=0.8,fill=gray!20,draw=none](-7.56,4.465)--(-7.561,4.464)--(-7.556,4.464)--cycle;
\filldraw[fill opacity=0.8,fill=gray!20,draw=none](-7.558,4.463)--(-7.548,4.46)--(-7.546,4.467)--(-7.556,4.466)--cycle;
\draw(-7.548,4.46)--(-7.546,4.467);
\filldraw[fill opacity=0.8,fill=gray!20,draw=none](-7.551,4.468)--(-7.553,4.466)--(-7.546,4.467)--(-7.545,4.468)--cycle;
\draw(-7.546,4.467)--(-7.545,4.468);
\filldraw[fill opacity=0.8,fill=gray!20,draw=none](-7.57,4.47)--(-7.57,4.47)--(-7.572,4.47)--cycle;
\filldraw[fill opacity=0.8,fill=gray!20,draw=none](-7.584,4.458)--(-7.575,4.459)--(-7.57,4.47)--(-7.572,4.47)--cycle;
\draw(-7.575,4.459)--(-7.57,4.47);
\filldraw[fill opacity=0.8,fill=gray!20,draw=none](-7.585,4.461)--(-7.582,4.46)--(-7.572,4.47)--(-7.592,4.477)--(-7.595,4.472)--cycle;
\draw(-7.592,4.477)--(-7.595,4.472);
\filldraw[fill opacity=0.8,fill=gray!20,draw=none](-7.572,4.47)--(-7.562,4.466)--(-7.564,4.463)--(-7.572,4.461)--(-7.573,4.469)--cycle;
\draw(-7.572,4.461)--(-7.573,4.469);
\filldraw[fill opacity=0.8,fill=gray!20,draw=none](-7.573,4.469)--(-7.573,4.465)--(-7.575,4.461)--(-7.582,4.46)--cycle;
\draw(-7.573,4.469)--(-7.573,4.465);
\filldraw[fill opacity=0.8,fill=gray!20,draw=none](-7.569,4.469)--(-7.566,4.468)--(-7.567,4.468)--cycle;
\draw(-7.566,4.468)--(-7.567,4.468);
\filldraw[fill opacity=0.8,fill=gray!20,draw=none](-7.547,4.481)--(-7.553,4.473)--(-7.551,4.477)--cycle;
\filldraw[fill opacity=0.8,fill=gray!20,draw=none](-7.569,4.462)--(-7.563,4.465)--(-7.568,4.466)--(-7.571,4.462)--cycle;
\draw(-7.568,4.466)--(-7.571,4.462)--(-7.569,4.462);
\filldraw[fill opacity=0.8,fill=gray!20,draw=none](-7.562,4.466)--(-7.564,4.467)--(-7.565,4.463)--(-7.564,4.463)--cycle;
\filldraw[fill opacity=0.8,fill=gray!20,draw=none](-7.553,4.473)--(-7.555,4.47)--(-7.556,4.469)--cycle;
\filldraw[fill opacity=0.8,fill=gray!20,draw=none](-7.564,4.467)--(-7.57,4.469)--(-7.565,4.463)--cycle;
\filldraw[fill opacity=0.8,fill=gray!20,draw=none](-7.563,4.465)--(-7.537,4.479)--(-7.562,4.472)--(-7.568,4.466)--cycle;
\draw(-7.562,4.472)--(-7.568,4.466);
\filldraw[fill opacity=0.8,fill=gray!20,draw=none](-7.558,4.465)--(-7.553,4.466)--(-7.551,4.468)--(-7.556,4.469)--cycle;
\filldraw[fill opacity=0.8,fill=gray!20,draw=none](-7.575,4.461)--(-7.573,4.465)--(-7.572,4.461)--cycle;
\draw(-7.573,4.465)--(-7.572,4.461);
\filldraw[fill opacity=0.8,fill=gray!20,draw=none](-7.564,4.463)--(-7.572,4.453)--(-7.572,4.461)--cycle;
\draw(-7.572,4.453)--(-7.572,4.461);
\filldraw[fill opacity=0.8,fill=gray!20,draw=none](-7.584,4.458)--(-7.588,4.455)--(-7.578,4.451)--(-7.575,4.459)--cycle;
\draw(-7.578,4.451)--(-7.575,4.459);
\filldraw[fill opacity=0.8,fill=gray!20,draw=none](-7.577,4.459)--(-7.571,4.462)--(-7.585,4.461)--cycle;
\draw(-7.577,4.459)--(-7.571,4.462)--(-7.585,4.461);
\filldraw[fill opacity=0.8,fill=gray!20,draw=none](-7.575,4.461)--(-7.576,4.46)--(-7.591,4.456)--(-7.597,4.459)--cycle;
\filldraw[fill opacity=0.8,fill=gray!20,draw=none](-7.569,4.462)--(-7.571,4.462)--(-7.573,4.461)--cycle;
\draw(-7.569,4.462)--(-7.571,4.462)--(-7.573,4.461);
\filldraw[fill opacity=0.8,fill=gray!20,draw=none](-7.575,4.461)--(-7.572,4.461)--(-7.576,4.46)--cycle;
\filldraw[fill opacity=0.8,fill=gray!20,draw=none](-7.572,4.453)--(-7.576,4.449)--(-7.575,4.449)--cycle;
\draw(-7.576,4.449)--(-7.575,4.449);
\filldraw[fill opacity=0.8,fill=gray!20,draw=none](-7.572,4.461)--(-7.571,4.448)--(-7.591,4.456)--cycle;
\draw(-7.572,4.461)--(-7.571,4.448);
\filldraw[fill opacity=0.8,fill=gray!20,draw=none](-7.663,4.496)--(-7.658,4.495)--(-7.658,4.494)--cycle;
\draw(-7.658,4.495)--(-7.658,4.494);
\filldraw[fill opacity=0.8,fill=gray!20,draw=none](-7.672,4.504)--(-7.626,4.481)--(-7.625,4.471)--(-7.669,4.49)--cycle;
\draw(-7.626,4.481)--(-7.625,4.471);
\filldraw[fill opacity=0.8,fill=gray!20,draw=none](-7.568,4.457)--(-7.563,4.462)--(-7.566,4.462)--cycle;
\filldraw[fill opacity=0.8,fill=gray!20,draw=none](-7.557,4.462)--(-7.558,4.463)--(-7.559,4.46)--cycle;
\filldraw[fill opacity=0.8,fill=gray!20,draw=none](-7.562,4.466)--(-7.559,4.465)--(-7.564,4.463)--cycle;
\filldraw[fill opacity=0.8,fill=gray!20,draw=none](-7.595,4.472)--(-7.585,4.461)--(-7.571,4.462)--(-7.569,4.464)--cycle;
\draw(-7.585,4.461)--(-7.571,4.462)--(-7.569,4.464);
\filldraw[fill opacity=0.8,fill=gray!20,draw=none](-7.571,4.447)--(-7.565,4.445)--(-7.556,4.466)--(-7.558,4.465)--cycle;
\filldraw[fill opacity=0.8,fill=gray!20,draw=none](-7.555,4.47)--(-7.557,4.466)--(-7.559,4.465)--(-7.561,4.466)--cycle;
\filldraw[fill opacity=0.8,fill=gray!20,draw=none](-7.564,4.463)--(-7.559,4.465)--(-7.557,4.464)--(-7.566,4.448)--(-7.568,4.446)--(-7.571,4.448)--(-7.572,4.453)--cycle;
\draw(-7.571,4.448)--(-7.572,4.453);
\filldraw[fill opacity=0.8,fill=gray!20,draw=none](-7.564,4.463)--(-7.568,4.463)--(-7.565,4.462)--cycle;
\filldraw[fill opacity=0.8,fill=gray!20,draw=none](-7.56,4.465)--(-7.558,4.465)--(-7.556,4.469)--(-7.564,4.47)--cycle;
\filldraw[fill opacity=0.8,fill=gray!20,draw=none](-7.555,4.47)--(-7.537,4.479)--(-7.545,4.47)--(-7.557,4.466)--cycle;
\filldraw[fill opacity=0.8,fill=gray!20,draw=none](-7.568,4.457)--(-7.566,4.462)--(-7.568,4.463)--(-7.581,4.461)--(-7.583,4.453)--(-7.576,4.449)--cycle;
\draw(-7.583,4.453)--(-7.576,4.449);
\filldraw[fill opacity=0.8,fill=gray!20,draw=none](-7.621,4.487)--(-7.625,4.48)--(-7.595,4.472)--(-7.592,4.477)--cycle;
\draw(-7.595,4.472)--(-7.592,4.477);
\filldraw[fill opacity=0.8,fill=gray!20,draw=none](-7.595,4.472)--(-7.569,4.464)--(-7.568,4.466)--(-7.57,4.469)--(-7.594,4.477)--(-7.601,4.478)--cycle;
\draw(-7.569,4.464)--(-7.568,4.466);
\filldraw[fill opacity=0.8,fill=gray!20,draw=none](-7.57,4.469)--(-7.577,4.471)--(-7.58,4.466)--(-7.568,4.463)--(-7.565,4.463)--cycle;
\filldraw[fill opacity=0.8,fill=gray!20,draw=none](-7.545,4.47)--(-7.551,4.462)--(-7.559,4.465)--cycle;
\filldraw[fill opacity=0.8,fill=gray!20,draw=none](-7.58,4.466)--(-7.581,4.461)--(-7.568,4.463)--cycle;
\filldraw[fill opacity=0.8,fill=gray!20,draw=none](-7.575,4.449)--(-7.571,4.447)--(-7.558,4.465)--(-7.571,4.464)--(-7.572,4.462)--cycle;
\filldraw[fill opacity=0.8,fill=gray!20,draw=none](-7.557,4.464)--(-7.551,4.462)--(-7.557,4.455)--(-7.566,4.448)--cycle;
\filldraw[fill opacity=0.8,fill=gray!20,draw=none](-7.568,4.466)--(-7.564,4.47)--(-7.566,4.473)--(-7.575,4.474)--cycle;
\draw(-7.568,4.466)--(-7.564,4.47);
\filldraw[fill opacity=0.8,fill=gray!20,draw=none](-7.56,4.465)--(-7.564,4.47)--(-7.567,4.471)--(-7.569,4.468)--cycle;
\filldraw[fill opacity=0.8,fill=gray!20,draw=none](-7.56,4.465)--(-7.569,4.468)--(-7.571,4.464)--cycle;
\filldraw[fill opacity=0.8,fill=gray!20,draw=none](-7.621,4.487)--(-7.592,4.477)--(-7.574,4.519)--(-7.616,4.497)--cycle;
\draw(-7.592,4.477)--(-7.574,4.519);
\filldraw[fill opacity=0.8,fill=gray!20,draw=none](-7.62,4.486)--(-7.583,4.474)--(-7.567,4.516)--cycle;
\draw(-7.62,4.486)--(-7.583,4.474)--(-7.567,4.516);
\filldraw[fill opacity=0.8,fill=gray!20,draw=none](-7.57,4.469)--(-7.575,4.474)--(-7.594,4.477)--cycle;
\filldraw[fill opacity=0.8,fill=gray!20,draw=none](-7.57,4.469)--(-7.575,4.474)--(-7.577,4.471)--cycle;
\filldraw[fill opacity=0.8,fill=gray!20,draw=none](-7.564,4.47)--(-7.562,4.472)--(-7.566,4.473)--cycle;
\draw(-7.564,4.47)--(-7.562,4.472);
\filldraw[fill opacity=0.8,fill=gray!20,draw=none](-7.564,4.47)--(-7.566,4.473)--(-7.567,4.471)--cycle;
\filldraw[fill opacity=0.8,fill=gray!20,draw=none](-7.619,4.486)--(-7.581,4.473)--(-7.583,4.474)--(-7.617,4.485)--cycle;
\draw(-7.581,4.473)--(-7.583,4.474)--(-7.617,4.485);
\filldraw[fill opacity=0.8,fill=gray!20,draw=none](-7.595,4.472)--(-7.601,4.478)--(-7.605,4.479)--(-7.624,4.48)--(-7.625,4.48)--cycle;
\filldraw[fill opacity=0.8,fill=gray!20,draw=none](-4.463,2.962)--(-7.603,4.534)--(-7.62,4.486)--(-4.507,2.928)--cycle;
\draw(-4.463,2.962)--(-7.603,4.534)--(-7.62,4.486)--(-4.507,2.928);
\filldraw[fill opacity=0.8,fill=gray!20](-7.797,.867)--(-7.816,.922)--(-7.889,.908)--(-7.88,.852)--cycle;
\filldraw[fill opacity=0.8,fill=gray!20](-4.353,2.96)--(-4.368,3.011)--(-4.447,3.008)--(-4.449,2.956)--cycle;
\filldraw[fill opacity=0.8,fill=gray!20,draw=none](-2.821,7.66)--(-2.826,7.662)--(-2.823,7.664)--cycle;
\draw(-2.821,7.66)--(-2.826,7.662)--(-2.823,7.664);
\filldraw[fill opacity=0.8,fill=gray!20,draw=none](-4.492,2.977)--(-4.479,3.001)--(-4.484,3.01)--(-4.492,3.011)--cycle;
\draw(-4.484,3.01)--(-4.492,3.011);
\filldraw[fill opacity=0.8,fill=gray!20,draw=none](-4.387,2.776)--(-4.381,2.792)--(-4.383,2.791)--cycle;
\draw(-4.381,2.792)--(-4.383,2.791);
\filldraw[fill opacity=0.8,fill=gray!20](-8.121,1.702)--(-8.096,1.744)--(-8.134,1.753)--(-8.175,1.715)--cycle;
\filldraw[fill opacity=0.8,fill=gray!20](-2.808,8.045)--(-2.764,8.06)--(-2.742,8.065)--(-2.766,8.053)--cycle;
\filldraw[fill opacity=0.8,fill=gray!20,draw=none](-4.4,2.828)--(-4.4,2.801)--(-4.394,2.791)--(-4.343,2.793)--(-4.343,2.796)--cycle;
\draw(-4.394,2.791)--(-4.343,2.793)--(-4.343,2.796);
\filldraw[fill opacity=0.8,fill=gray!20,draw=none](-4.454,2.956)--(-4.449,2.956)--(-4.447,3.008)--(-4.484,3.01)--cycle;
\draw(-4.454,2.956)--(-4.449,2.956)--(-4.447,3.008)--(-4.484,3.01);
\filldraw[fill opacity=0.8,fill=gray!20,draw=none](-2.563,7.694)--(-2.561,7.708)--(-2.553,7.704)--(-2.562,7.693)--cycle;
\draw(-2.553,7.704)--(-2.562,7.693)--(-2.563,7.694);
\filldraw[fill opacity=0.8,fill=gray!20,draw=none](-7.985,.999)--(-8.035,1.003)--(-8.059,.962)--cycle;
\draw(-7.985,.999)--(-8.035,1.003)--(-8.059,.962);
\filldraw[fill opacity=0.8,fill=gray!20,draw=none](-4.4,2.828)--(-4.343,2.796)--(-4.34,2.847)--(-4.399,2.845)--cycle;
\draw(-4.343,2.796)--(-4.34,2.847)--(-4.399,2.845);
\filldraw[fill opacity=0.8,fill=gray!20,draw=none](-4.386,2.782)--(-4.383,2.791)--(-4.394,2.791)--cycle;
\draw(-4.383,2.791)--(-4.394,2.791);
\filldraw[fill opacity=0.8,fill=gray!20](-8.042,1.74)--(-8.039,1.773)--(-8.067,1.775)--(-8.096,1.744)--cycle;
\filldraw[fill opacity=0.8,fill=gray!20,draw=none](-5.97,.454)--(-5.966,.448)--(-5.96,.436)--(-5.967,.439)--cycle;
\draw(-5.96,.436)--(-5.967,.439);
\filldraw[fill opacity=0.8,fill=gray!20,draw=none](-4.593,2.504)--(-6.01,.382)--(-5.996,.345)--(-4.424,2.699)--cycle;
\draw(-4.593,2.504)--(-6.01,.382)--(-5.996,.345)--(-4.424,2.699);
\filldraw[fill opacity=0.8,fill=gray!20](-2.872,8)--(-2.826,8.033)--(-2.808,8.045)--(-2.846,8.017)--cycle;
\filldraw[fill opacity=0.8,fill=gray!20,draw=none](-8.11,.932)--(-8.07,.935)--(-8.06,.961)--(-8.061,.961)--cycle;
\draw(-8.07,.935)--(-8.06,.961)--(-8.061,.961);
\filldraw[fill opacity=0.8,fill=gray!20,draw=none](-2.823,7.658)--(-2.826,7.662)--(-2.82,7.66)--cycle;
\draw(-2.823,7.658)--(-2.826,7.662)--(-2.82,7.66);
\filldraw[fill opacity=0.8,fill=gray!20,draw=none](-2.823,7.658)--(-2.831,7.662)--(-2.856,7.68)--(-2.872,7.697)--(-2.826,7.662)--cycle;
\draw(-2.831,7.662)--(-2.856,7.68)--(-2.872,7.697)--(-2.826,7.662)--(-2.823,7.658);
\filldraw[fill opacity=0.8,fill=gray!20](-2.656,8.052)--(-2.685,8.064)--(-2.666,8.059)--(-2.619,8.043)--cycle;
\filldraw[fill opacity=0.8,fill=gray!20](-7.987,1.743)--(-8.01,1.774)--(-8.039,1.773)--(-8.042,1.74)--cycle;
\filldraw[fill opacity=0.8,fill=gray!20,draw=none](-4.492,2.977)--(-4.492,2.959)--(-4.454,2.956)--(-4.479,3.001)--cycle;
\draw(-4.492,2.959)--(-4.454,2.956);
\filldraw[fill opacity=0.8,fill=gray!20,draw=none](-8.027,.682)--(-7.986,.687)--(-7.986,.689)--(-8.014,.691)--cycle;
\draw(-7.986,.687)--(-7.986,.689)--(-8.014,.691);
\filldraw[fill opacity=0.8,fill=gray!20,draw=none](-2.806,7.647)--(-2.815,7.65)--(-2.823,7.658)--(-2.82,7.659)--cycle;
\draw(-2.806,7.647)--(-2.815,7.65)--(-2.823,7.658);
\filldraw[fill opacity=0.8,fill=gray!20,draw=none](-7.929,.958)--(-7.907,.962)--(-7.925,1.002)--(-7.981,.999)--(-7.984,.955)--cycle;
\draw(-7.907,.962)--(-7.925,1.002)--(-7.981,.999)--(-7.984,.955)--(-7.929,.958);
\filldraw[fill opacity=0.8,fill=gray!20,draw=none](-4.484,2.917)--(-4.487,2.918)--(-4.493,2.917)--cycle;
\draw(-4.484,2.917)--(-4.487,2.918);
\filldraw[fill opacity=0.8,fill=gray!20,draw=none](-8.103,1.397)--(-8.045,1.393)--(-8.046,1.402)--(-8.084,1.411)--cycle;
\draw(-8.103,1.397)--(-8.045,1.393)--(-8.046,1.402);
\filldraw[fill opacity=0.8,fill=gray!20,draw=none](-4.528,2.906)--(-4.511,2.926)--(-4.552,2.908)--cycle;
\filldraw[fill opacity=0.8,fill=gray!20,draw=none](-4.487,2.918)--(-4.507,2.928)--(-4.559,2.898)--(-4.557,2.897)--cycle;
\draw(-4.487,2.918)--(-4.507,2.928);
\draw(-4.559,2.898)--(-4.557,2.897);
\filldraw[fill opacity=0.8,fill=gray!20,draw=none](-4.4,2.791)--(-4.394,2.791)--(-4.4,2.801)--cycle;
\draw(-4.4,2.791)--(-4.394,2.791);
\filldraw[fill opacity=0.8,fill=gray!20](-7.906,1.711)--(-7.944,1.751)--(-7.987,1.743)--(-7.966,1.7)--cycle;
\filldraw[fill opacity=0.8,fill=gray!20](-2.856,7.68)--(-2.887,7.721)--(-2.907,7.742)--(-2.872,7.697)--cycle;
\filldraw[fill opacity=0.8,fill=gray!20,draw=none](-8.183,1.662)--(-8.145,1.708)--(-8.175,1.715)--(-8.206,1.667)--cycle;
\draw(-8.145,1.708)--(-8.175,1.715)--(-8.206,1.667)--(-8.183,1.662);
\filldraw[fill opacity=0.8,fill=gray!20](-7.859,1.498)--(-7.853,1.552)--(-7.938,1.536)--(-7.941,1.482)--cycle;
\filldraw[fill opacity=0.8,fill=gray!20](-7.966,1.397)--(-7.951,1.434)--(-8.047,1.43)--(-8.045,1.393)--cycle;
\filldraw[fill opacity=0.8,fill=gray!20,draw=none](-7.967,.69)--(-7.889,.693)--(-7.887,.704)--cycle;
\draw(-7.967,.69)--(-7.889,.693)--(-7.887,.704);
\filldraw[fill opacity=0.8,fill=gray!20,draw=none](-4.552,2.908)--(-4.554,2.908)--(-4.555,2.907)--cycle;
\draw(-4.554,2.908)--(-4.555,2.907);
\filldraw[fill opacity=0.8,fill=gray!20,draw=none](-4.502,2.96)--(-4.492,2.959)--(-4.492,2.977)--cycle;
\draw(-4.502,2.96)--(-4.492,2.959);
\filldraw[fill opacity=0.8,fill=gray!20,draw=none](-4.4,2.828)--(-4.424,2.842)--(-4.4,2.801)--cycle;
\filldraw[fill opacity=0.8,fill=gray!20,draw=none](-4.615,2.874)--(-4.611,2.875)--(-4.612,2.876)--cycle;
\draw(-4.611,2.875)--(-4.612,2.876);
\filldraw[fill opacity=0.8,fill=gray!20,draw=none](-8.203,1.666)--(-8.206,1.667)--(-8.206,1.666)--cycle;
\draw(-8.203,1.666)--(-8.206,1.667)--(-8.206,1.666);
\filldraw[fill opacity=0.8,fill=gray!20,draw=none](-4.443,2.9)--(-4.343,2.904)--(-4.353,2.96)--(-4.448,2.956)--cycle;
\draw(-4.443,2.9)--(-4.343,2.904)--(-4.353,2.96)--(-4.448,2.956);
\filldraw[fill opacity=0.8,fill=gray!20,draw=none](-7.984,.955)--(-7.981,.999)--(-7.985,.999)--(-8.059,.962)--(-8.06,.961)--cycle;
\draw(-8.059,.962)--(-8.06,.961)--(-7.984,.955)--(-7.981,.999)--(-7.985,.999);
\filldraw[fill opacity=0.8,fill=gray!20,draw=none](-4.526,2.865)--(-4.529,2.862)--(-4.513,2.855)--(-4.51,2.857)--cycle;
\draw(-4.526,2.865)--(-4.529,2.862);
\filldraw[fill opacity=0.8,fill=gray!20](-2.887,7.721)--(-2.907,7.77)--(-2.929,7.794)--(-2.907,7.742)--cycle;
\filldraw[fill opacity=0.8,fill=gray!20,draw=none](-4.425,2.843)--(-4.34,2.847)--(-4.343,2.904)--(-4.443,2.9)--cycle;
\draw(-4.425,2.843)--(-4.34,2.847)--(-4.343,2.904)--(-4.443,2.9);
\filldraw[fill opacity=0.8,fill=gray!20,draw=none](-7.967,.69)--(-7.887,.704)--(-7.88,.741)--(-7.932,.738)--(-7.986,.709)--(-7.986,.689)--cycle;
\draw(-7.887,.704)--(-7.88,.741)--(-7.932,.738);
\draw(-7.986,.709)--(-7.986,.689)--(-7.967,.69);
\filldraw[fill opacity=0.8,fill=gray!20,draw=none](-4.563,2.896)--(-4.596,2.891)--(-4.612,2.876)--(-4.611,2.875)--cycle;
\draw(-4.612,2.876)--(-4.611,2.875);
\filldraw[fill opacity=0.8,fill=gray!20,draw=none](-4.51,2.926)--(-4.507,2.928)--(-4.508,2.929)--cycle;
\draw(-4.507,2.928)--(-4.508,2.929);
\filldraw[fill opacity=0.8,fill=gray!20,draw=none](-4.509,2.927)--(-4.449,2.953)--(-4.449,2.956)--(-4.502,2.96)--cycle;
\draw(-4.449,2.953)--(-4.449,2.956)--(-4.502,2.96);
\filldraw[fill opacity=0.8,fill=gray!20](-2.578,8.013)--(-2.619,8.043)--(-2.607,8.031)--(-2.562,7.996)--cycle;
\filldraw[fill opacity=0.8,fill=gray!20,draw=none](-4.596,2.891)--(-4.631,2.885)--(-4.612,2.876)--cycle;
\draw(-4.631,2.885)--(-4.612,2.876);
\filldraw[fill opacity=0.8,fill=gray!20,draw=none](-4.529,2.862)--(-4.549,2.831)--(-4.513,2.855)--cycle;
\draw(-4.529,2.862)--(-4.549,2.831);
\filldraw[fill opacity=0.8,fill=gray!20,draw=none](-4.528,2.906)--(-4.451,2.899)--(-4.449,2.953)--(-4.511,2.926)--cycle;
\draw(-4.451,2.899)--(-4.449,2.953);
\filldraw[fill opacity=0.8,fill=gray!20,draw=none](-7.993,.704)--(-8.014,.691)--(-8.001,.69)--cycle;
\draw(-8.014,.691)--(-8.001,.69);
\filldraw[fill opacity=0.8,fill=gray!20,draw=none](-2.563,7.694)--(-2.562,7.693)--(-2.564,7.691)--cycle;
\draw(-2.563,7.694)--(-2.562,7.693)--(-2.564,7.691);
\filldraw[fill opacity=0.8,fill=gray!20,draw=none](-4.511,2.926)--(-4.509,2.927)--(-4.508,2.929)--cycle;
\filldraw[fill opacity=0.8,fill=gray!20,draw=none](-7.738,4.52)--(-7.744,4.523)--(-7.699,4.503)--(-7.688,4.499)--cycle;
\draw(-7.699,4.503)--(-7.688,4.499);
\filldraw[fill opacity=0.8,fill=gray!20,draw=none](-7.675,4.493)--(-7.643,4.479)--(-7.61,4.465)--(-7.588,4.455)--cycle;
\draw(-7.643,4.479)--(-7.61,4.465)--(-7.588,4.455);
\filldraw[fill opacity=0.8,fill=gray!20,draw=none](-7.62,4.454)--(-7.646,4.473)--(-7.63,4.473)--(-7.618,4.468)--(-7.62,4.454)--cycle;
\draw(-7.618,4.468)--(-7.62,4.454)--(-7.62,4.454)--(-7.646,4.473)--(-7.63,4.473);
\filldraw[fill opacity=0.8,fill=gray!20](-7.62,4.454)--(-7.668,4.468)--(-7.646,4.473)--(-7.62,4.454)--cycle;
\filldraw[fill opacity=0.8,fill=gray!20,draw=none](-7.671,4.491)--(-7.702,4.504)--(-7.588,4.455)--(-7.576,4.45)--(-7.571,4.447)--cycle;
\draw(-7.588,4.455)--(-7.576,4.45)--(-7.571,4.447);
\filldraw[fill opacity=0.8,fill=gray!20,draw=none](-7.62,4.454)--(-7.618,4.468)--(-7.606,4.463)--(-7.62,4.454)--cycle;
\draw(-7.606,4.463)--(-7.62,4.454)--(-7.62,4.454)--(-7.618,4.468);
\filldraw[fill opacity=0.8,fill=gray!20,draw=none](-7.723,4.475)--(-7.722,4.477)--(-7.682,4.496)--(-7.662,4.487)--(-7.695,4.413)--cycle;
\draw(-7.723,4.475)--(-7.722,4.477);
\draw(-7.662,4.487)--(-7.695,4.413);
\filldraw[fill opacity=0.8,fill=gray!20](-7.62,4.454)--(-7.677,4.462)--(-7.668,4.468)--(-7.62,4.454)--cycle;
\filldraw[fill opacity=0.8,fill=gray!20,draw=none](-7.62,4.454)--(-7.606,4.463)--(-7.599,4.46)--(-7.62,4.454)--cycle;
\draw(-7.599,4.46)--(-7.62,4.454)--(-7.62,4.454)--(-7.606,4.463);
\filldraw[fill opacity=0.8,fill=gray!20,draw=none](-7.594,4.445)--(-7.589,4.455)--(-7.58,4.451)--cycle;
\draw(-7.594,4.445)--(-7.589,4.455);
\filldraw[fill opacity=0.8,fill=gray!20,draw=none](-7.62,4.454)--(-7.599,4.46)--(-7.594,4.457)--(-7.62,4.454)--cycle;
\draw(-7.594,4.457)--(-7.62,4.454)--(-7.62,4.454)--(-7.599,4.46);
\filldraw[fill opacity=0.8,fill=gray!20,draw=none](-7.629,4.473)--(-7.571,4.447)--(-7.56,4.443)--(-7.564,4.445)--(-7.575,4.449)--cycle;
\draw(-7.571,4.447)--(-7.56,4.443)--(-7.564,4.445)--(-7.575,4.449);
\filldraw[fill opacity=0.8,fill=gray!20](-7.62,4.454)--(-7.671,4.456)--(-7.677,4.462)--(-7.62,4.454)--cycle;
\filldraw[fill opacity=0.8,fill=gray!20,draw=none](-7.62,4.454)--(-7.587,4.455)--(-7.583,4.453)--(-7.595,4.45)--(-7.62,4.454)--cycle;
\draw(-7.583,4.453)--(-7.595,4.45)--(-7.62,4.454)--(-7.62,4.454)--(-7.587,4.455);
\filldraw[fill opacity=0.8,fill=gray!20,draw=none](-7.62,4.454)--(-7.594,4.457)--(-7.587,4.455)--(-7.62,4.454)--cycle;
\draw(-7.587,4.455)--(-7.62,4.454)--(-7.62,4.454)--(-7.594,4.457);
\filldraw[fill opacity=0.8,fill=gray!20](-7.62,4.454)--(-7.595,4.45)--(-7.624,4.449)--(-7.62,4.454)--cycle;
\filldraw[fill opacity=0.8,fill=gray!20](-7.62,4.454)--(-7.651,4.451)--(-7.671,4.456)--(-7.62,4.454)--cycle;
\filldraw[fill opacity=0.8,fill=gray!20](-7.62,4.454)--(-7.624,4.449)--(-7.651,4.451)--(-7.62,4.454)--cycle;
\filldraw[fill opacity=0.8,fill=gray!20,draw=none](-8.082,4.514)--(-8.07,4.531)--(-8.052,4.507)--cycle;
\draw(-8.07,4.531)--(-8.052,4.507)--(-8.082,4.514);
\filldraw[fill opacity=0.8,fill=gray!20,draw=none](-8.019,4.443)--(-8.033,4.458)--(-8.052,4.507)--(-8.037,4.491)--cycle;
\draw(-8.019,4.443)--(-8.033,4.458)--(-8.052,4.507)--(-8.037,4.491);
\filldraw[fill opacity=0.8,fill=gray!20,draw=none](-7.985,4.522)--(-8.041,4.495)--(-8.07,4.534)--(-8.019,4.558)--cycle;
\draw(-7.985,4.522)--(-8.041,4.495);
\draw(-8.07,4.534)--(-8.019,4.558);
\filldraw[fill opacity=0.8,fill=gray!20,draw=none](-7.745,4.549)--(-7.742,4.556)--(-7.74,4.553)--(-7.741,4.549)--cycle;
\draw(-7.74,4.553)--(-7.741,4.549);
\filldraw[fill opacity=0.8,fill=gray!20,draw=none](-7.742,4.556)--(-7.736,4.555)--(-7.741,4.547)--(-7.745,4.549)--cycle;
\filldraw[fill opacity=0.8,fill=gray!20,draw=none](-7.758,4.549)--(-7.775,4.548)--(-7.784,4.56)--(-7.781,4.566)--cycle;
\draw(-7.784,4.56)--(-7.781,4.566);
\filldraw[fill opacity=0.8,fill=gray!20,draw=none](-7.596,4.458)--(-7.627,4.472)--(-7.64,4.478)--(-7.671,4.491)--(-7.629,4.473)--(-7.576,4.45)--cycle;
\filldraw[fill opacity=0.8,fill=gray!20,draw=none](-7.583,4.453)--(-7.587,4.454)--(-7.595,4.45)--cycle;
\draw(-7.587,4.454)--(-7.595,4.45)--(-7.583,4.453);
\filldraw[fill opacity=0.8,fill=gray!20,draw=none](-7.624,4.449)--(-7.624,4.45)--(-7.663,4.462)--(-7.669,4.462)--(-7.676,4.461)--(-7.651,4.451)--cycle;
\draw(-7.663,4.462)--(-7.669,4.462);
\draw(-7.676,4.461)--(-7.651,4.451)--(-7.624,4.449)--(-7.624,4.45);
\filldraw[fill opacity=0.8,fill=gray!20,draw=none](-7.595,4.45)--(-7.587,4.454)--(-7.591,4.456)--(-7.624,4.456)--(-7.626,4.456)--(-7.624,4.449)--cycle;
\draw(-7.626,4.456)--(-7.624,4.449)--(-7.595,4.45)--(-7.587,4.454);
\filldraw[fill opacity=0.8,fill=gray!20,draw=none](-7.624,4.45)--(-7.626,4.456)--(-7.652,4.461)--(-7.663,4.462)--cycle;
\draw(-7.624,4.45)--(-7.626,4.456);
\draw(-7.652,4.461)--(-7.663,4.462);
\filldraw[fill opacity=0.8,fill=gray!20,draw=none](-7.668,4.468)--(-7.699,4.488)--(-7.682,4.496)--(-7.654,4.484)--(-7.646,4.473)--cycle;
\draw(-7.654,4.484)--(-7.646,4.473)--(-7.668,4.468)--(-7.699,4.488);
\filldraw[fill opacity=0.8,fill=gray!20,draw=none](-7.63,4.473)--(-7.646,4.473)--(-7.654,4.484)--cycle;
\draw(-7.63,4.473)--(-7.646,4.473)--(-7.654,4.484);
\filldraw[fill opacity=0.8,fill=gray!20,draw=none](-7.624,4.456)--(-7.626,4.456)--(-7.626,4.456)--cycle;
\draw(-7.626,4.456)--(-7.626,4.456);
\filldraw[fill opacity=0.8,fill=gray!20,draw=none](-7.663,4.464)--(-7.669,4.49)--(-7.625,4.471)--(-7.623,4.453)--cycle;
\draw(-7.625,4.471)--(-7.623,4.453);
\filldraw[fill opacity=0.8,fill=gray!20,draw=none](-7.597,4.459)--(-7.624,4.456)--(-7.625,4.471)--cycle;
\draw(-7.624,4.456)--(-7.625,4.471);
\filldraw[fill opacity=0.8,fill=gray!20,draw=none](-7.591,4.456)--(-7.6,4.46)--(-7.627,4.459)--(-7.626,4.456)--cycle;
\draw(-7.6,4.46)--(-7.627,4.459)--(-7.626,4.456);
\filldraw[fill opacity=0.8,fill=gray!20,draw=none](-7.571,4.447)--(-7.577,4.442)--(-7.583,4.411)--(-7.58,4.379)--(-7.556,4.441)--cycle;
\draw(-7.58,4.379)--(-7.556,4.441);
\filldraw[fill opacity=0.8,fill=gray!20,draw=none](-7.597,4.459)--(-7.588,4.455)--(-7.614,4.429)--(-7.616,4.428)--(-7.621,4.427)--(-7.624,4.456)--cycle;
\draw(-7.621,4.427)--(-7.624,4.456);
\filldraw[fill opacity=0.8,fill=gray!20,draw=none](-7.611,4.377)--(-7.578,4.451)--(-7.588,4.455)--(-7.614,4.429)--(-7.615,4.427)--cycle;
\draw(-7.611,4.377)--(-7.578,4.451);
\draw(-7.614,4.429)--(-7.615,4.427);
\filldraw[fill opacity=0.8,fill=gray!20,draw=none](-7.682,4.496)--(-7.698,4.489)--(-7.728,4.474)--(-7.725,4.475)--(-7.686,4.494)--(-7.682,4.496)--cycle;
\draw(-7.698,4.489)--(-7.728,4.474);
\draw(-7.686,4.494)--(-7.682,4.496);
\filldraw[fill opacity=0.8,fill=gray!20,draw=none](-7.675,4.493)--(-7.702,4.504)--(-7.738,4.52)--(-7.688,4.499)--(-7.657,4.485)--(-7.643,4.479)--cycle;
\draw(-7.688,4.499)--(-7.657,4.485)--(-7.643,4.479);
\filldraw[fill opacity=0.8,fill=gray!20,draw=none](-7.682,4.496)--(-7.722,4.477)--(-7.709,4.508)--cycle;
\draw(-7.722,4.477)--(-7.709,4.508);
\filldraw[fill opacity=0.8,fill=gray!20,draw=none](-7.724,4.476)--(-7.739,4.521)--(-7.709,4.508)--(-7.722,4.477)--cycle;
\draw(-7.709,4.508)--(-7.722,4.477);
\filldraw[fill opacity=0.8,fill=gray!20,draw=none](-7.626,4.456)--(-7.627,4.459)--(-7.652,4.461)--cycle;
\draw(-7.626,4.456)--(-7.627,4.459)--(-7.652,4.461);
\filldraw[fill opacity=0.8,fill=gray!20,draw=none](-7.6,4.46)--(-7.585,4.461)--(-7.595,4.472)--(-7.625,4.48)--(-7.629,4.477)--(-7.628,4.472)--cycle;
\draw(-7.6,4.46)--(-7.585,4.461);
\draw(-7.629,4.477)--(-7.628,4.472);
\filldraw[fill opacity=0.8,fill=gray!20,draw=none](-7.593,4.457)--(-7.584,4.458)--(-7.582,4.46)--(-7.598,4.464)--(-7.6,4.46)--cycle;
\draw(-7.598,4.464)--(-7.6,4.46);
\filldraw[fill opacity=0.8,fill=gray!20,draw=none](-7.584,4.458)--(-7.593,4.457)--(-7.588,4.455)--cycle;
\filldraw[fill opacity=0.8,fill=gray!20,draw=none](-7.663,4.464)--(-7.669,4.462)--(-7.652,4.461)--cycle;
\draw(-7.669,4.462)--(-7.652,4.461);
\filldraw[fill opacity=0.8,fill=gray!20,draw=none](-7.662,4.458)--(-7.663,4.464)--(-7.623,4.453)--(-7.623,4.446)--cycle;
\draw(-7.623,4.453)--(-7.623,4.446);
\filldraw[fill opacity=0.8,fill=gray!20,draw=none](-7.588,4.455)--(-7.571,4.448)--(-7.577,4.442)--(-7.614,4.429)--cycle;
\filldraw[fill opacity=0.8,fill=gray!20,draw=none](-7.588,4.455)--(-7.6,4.46)--(-7.614,4.429)--cycle;
\draw(-7.6,4.46)--(-7.614,4.429);
\filldraw[fill opacity=0.8,fill=gray!20,draw=none](-7.585,4.461)--(-7.595,4.472)--(-7.598,4.464)--cycle;
\draw(-7.595,4.472)--(-7.598,4.464);
\filldraw[fill opacity=0.8,fill=gray!20,draw=none](-7.634,4.475)--(-7.663,4.464)--(-7.652,4.461)--(-7.627,4.459)--(-7.628,4.472)--cycle;
\draw(-7.652,4.461)--(-7.627,4.459)--(-7.628,4.472);
\filldraw[fill opacity=0.8,fill=gray!20,draw=none](-7.6,4.46)--(-7.628,4.472)--(-7.627,4.459)--cycle;
\draw(-7.628,4.472)--(-7.627,4.459)--(-7.6,4.46);
\filldraw[fill opacity=0.8,fill=gray!20,draw=none](-7.596,4.458)--(-7.576,4.45)--(-7.575,4.449)--(-7.579,4.451)--cycle;
\draw(-7.575,4.449)--(-7.579,4.451);
\filldraw[fill opacity=0.8,fill=gray!20,draw=none](-7.71,4.489)--(-7.735,4.489)--(-7.757,4.476)--(-7.728,4.474)--(-7.701,4.487)--cycle;
\draw(-7.728,4.474)--(-7.701,4.487);
\filldraw[fill opacity=0.8,fill=gray!20,draw=none](-7.671,4.456)--(-7.701,4.466)--(-7.722,4.481)--(-7.677,4.462)--cycle;
\draw(-7.722,4.481)--(-7.677,4.462)--(-7.671,4.456)--(-7.701,4.466);
\filldraw[fill opacity=0.8,fill=gray!20,draw=none](-7.651,4.451)--(-7.676,4.461)--(-7.701,4.466)--(-7.671,4.456)--cycle;
\draw(-7.701,4.466)--(-7.671,4.456)--(-7.651,4.451)--(-7.676,4.461);
\filldraw[fill opacity=0.8,fill=gray!20,draw=none](-7.657,4.435)--(-7.662,4.458)--(-7.623,4.446)--(-7.621,4.427)--(-7.649,4.431)--cycle;
\draw(-7.623,4.446)--(-7.621,4.427);
\filldraw[fill opacity=0.8,fill=gray!20,draw=none](-7.571,4.448)--(-7.571,4.444)--(-7.577,4.442)--cycle;
\draw(-7.571,4.448)--(-7.571,4.444);
\filldraw[fill opacity=0.8,fill=gray!20,draw=none](-7.568,4.446)--(-7.571,4.444)--(-7.571,4.448)--cycle;
\draw(-7.571,4.444)--(-7.571,4.448);
\filldraw[fill opacity=0.8,fill=gray!20,draw=none](-7.571,4.447)--(-7.575,4.449)--(-7.577,4.442)--cycle;
\filldraw[fill opacity=0.8,fill=gray!20,draw=none](-7.617,4.467)--(-7.618,4.468)--(-7.596,4.458)--(-7.579,4.451)--(-7.583,4.453)--cycle;
\draw(-7.579,4.451)--(-7.583,4.453);
\filldraw[fill opacity=0.8,fill=gray!20,draw=none](-7.618,4.468)--(-7.627,4.472)--(-7.596,4.458)--cycle;
\filldraw[fill opacity=0.8,fill=gray!20,draw=none](-7.625,4.48)--(-7.626,4.477)--(-7.624,4.471)--(-7.6,4.46)--(-7.595,4.472)--cycle;
\draw(-7.6,4.46)--(-7.595,4.472);
\filldraw[fill opacity=0.8,fill=gray!20,draw=none](-7.624,4.471)--(-7.615,4.428)--(-7.614,4.429)--(-7.6,4.46)--cycle;
\draw(-7.614,4.429)--(-7.6,4.46);
\filldraw[fill opacity=0.8,fill=gray!20,draw=none](-7.615,4.428)--(-7.615,4.427)--(-7.614,4.429)--cycle;
\draw(-7.615,4.427)--(-7.614,4.429);
\filldraw[fill opacity=0.8,fill=gray!20,draw=none](-7.634,4.475)--(-7.64,4.478)--(-7.627,4.472)--cycle;
\filldraw[fill opacity=0.8,fill=gray!20,draw=none](-7.634,4.475)--(-7.627,4.472)--(-7.618,4.468)--(-7.624,4.471)--(-7.629,4.473)--cycle;
\filldraw[fill opacity=0.8,fill=gray!20,draw=none](-7.618,4.468)--(-7.617,4.467)--(-7.624,4.471)--cycle;
\filldraw[fill opacity=0.8,fill=gray!20,draw=none](-4.51,2.926)--(-4.508,2.929)--(-7.62,4.486)--(-7.647,4.443)--(-4.559,2.898)--cycle;
\draw(-4.508,2.929)--(-7.62,4.486)--(-7.647,4.443)--(-4.559,2.898);
\filldraw[fill opacity=0.8,fill=gray!20,draw=none](-2.823,7.658)--(-2.815,7.65)--(-2.831,7.662)--cycle;
\draw(-2.823,7.658)--(-2.815,7.65)--(-2.831,7.662);
\filldraw[fill opacity=0.8,fill=gray!20,draw=none](-4.424,2.842)--(-4.4,2.828)--(-4.399,2.845)--(-4.425,2.843)--cycle;
\draw(-4.399,2.845)--(-4.425,2.843);
\filldraw[fill opacity=0.8,fill=gray!20,draw=none](-7.993,.704)--(-8.001,.69)--(-7.986,.689)--(-7.986,.709)--cycle;
\draw(-8.001,.69)--(-7.986,.689)--(-7.986,.709);
\filldraw[fill opacity=0.8,fill=gray!20](-7.877,1.663)--(-7.906,1.711)--(-7.966,1.7)--(-7.951,1.649)--cycle;
\filldraw[fill opacity=0.8,fill=gray!20](-7.853,1.552)--(-7.859,1.609)--(-7.941,1.593)--(-7.938,1.536)--cycle;
\filldraw[fill opacity=0.8,fill=gray!20,draw=none](-2.564,7.691)--(-2.562,7.693)--(-2.566,7.69)--cycle;
\draw(-2.564,7.691)--(-2.562,7.693)--(-2.566,7.69);
\filldraw[fill opacity=0.8,fill=gray!20,draw=none](-4.503,2.903)--(-4.451,2.899)--(-4.504,2.904)--cycle;
\draw(-4.503,2.903)--(-4.451,2.899);
\filldraw[fill opacity=0.8,fill=gray!20,draw=none](-4.504,2.904)--(-5.981,1.145)--(-5.937,1.121)--(-4.465,2.873)--cycle;
\draw(-4.504,2.904)--(-5.981,1.145)--(-5.937,1.121)--(-4.465,2.873);
\filldraw[fill opacity=0.8,fill=gray!20](-2.907,7.77)--(-2.913,7.825)--(-2.936,7.849)--(-2.929,7.794)--cycle;
\filldraw[fill opacity=0.8,fill=gray!20,draw=none](-2.567,7.69)--(-2.566,7.69)--(-2.562,7.693)--(-2.556,7.7)--cycle;
\draw(-2.566,7.69)--(-2.562,7.693)--(-2.556,7.7);
\filldraw[fill opacity=0.8,fill=gray!20,draw=none](-4.557,2.897)--(-4.559,2.898)--(-4.563,2.896)--cycle;
\draw(-4.557,2.897)--(-4.559,2.898);
\filldraw[fill opacity=0.8,fill=gray!20,draw=none](-4.461,2.877)--(-4.477,2.901)--(-4.503,2.903)--cycle;
\draw(-4.477,2.901)--(-4.503,2.903);
\filldraw[fill opacity=0.8,fill=gray!20,draw=none](-4.596,2.891)--(-4.563,2.896)--(-4.559,2.898)--(-4.578,2.907)--cycle;
\draw(-4.559,2.898)--(-4.578,2.907);
\filldraw[fill opacity=0.8,fill=gray!20,draw=none](-7.984,.947)--(-7.984,.955)--(-8.06,.961)--(-8.07,.935)--cycle;
\draw(-7.984,.947)--(-7.984,.955)--(-8.06,.961)--(-8.07,.935);
\filldraw[fill opacity=0.8,fill=gray!20,draw=none](-7.657,4.435)--(-7.649,4.431)--(-7.657,4.433)--cycle;
\filldraw[fill opacity=0.8,fill=gray!20,draw=none](-4.596,2.891)--(-4.578,2.907)--(-7.647,4.443)--(-7.678,4.41)--(-4.631,2.885)--cycle;
\draw(-4.578,2.907)--(-7.647,4.443)--(-7.678,4.41)--(-4.631,2.885);
\filldraw[fill opacity=0.8,fill=gray!20,draw=none](-2.567,7.69)--(-2.556,7.7)--(-2.553,7.704)--(-2.545,7.725)--(-2.558,7.716)--(-2.577,7.69)--cycle;
\draw(-2.556,7.7)--(-2.553,7.704);
\draw(-2.545,7.725)--(-2.558,7.716)--(-2.577,7.69);
\filldraw[fill opacity=0.8,fill=gray!20,draw=none](-2.553,7.704)--(-2.527,7.737)--(-2.545,7.725)--cycle;
\draw(-2.553,7.704)--(-2.527,7.737)--(-2.545,7.725);
\filldraw[fill opacity=0.8,fill=gray!20,draw=none](-8.067,1.422)--(-8.084,1.411)--(-8.046,1.402)--(-8.047,1.424)--cycle;
\draw(-8.046,1.402)--(-8.047,1.424);
\filldraw[fill opacity=0.8,fill=gray!20,draw=none](-4.424,2.842)--(-4.425,2.843)--(-4.427,2.843)--cycle;
\draw(-4.425,2.843)--(-4.427,2.843);
\filldraw[fill opacity=0.8,fill=gray!20,draw=none](-4.427,2.843)--(-4.425,2.843)--(-4.443,2.9)--(-4.451,2.899)--(-4.451,2.862)--cycle;
\draw(-4.427,2.843)--(-4.425,2.843);
\draw(-4.443,2.9)--(-4.451,2.899)--(-4.451,2.862);
\filldraw[fill opacity=0.8,fill=gray!20](-7.859,1.609)--(-7.877,1.663)--(-7.951,1.649)--(-7.941,1.593)--cycle;
\filldraw[fill opacity=0.8,fill=gray!20](-2.826,8.033)--(-2.774,8.054)--(-2.764,8.06)--(-2.808,8.045)--cycle;
\filldraw[fill opacity=0.8,fill=gray!20,draw=none](-4.443,2.9)--(-4.448,2.956)--(-4.449,2.956)--(-4.451,2.899)--cycle;
\draw(-4.448,2.956)--(-4.449,2.956)--(-4.451,2.899)--(-4.443,2.9);
\filldraw[fill opacity=0.8,fill=gray!20,draw=none](-4.552,2.55)--(-4.553,2.549)--(-4.683,2.394)--(-4.807,2.256)--(-4.737,2.339)--cycle;
\draw(-4.553,2.549)--(-4.683,2.394);
\draw(-4.807,2.256)--(-4.737,2.339);
\filldraw[fill opacity=0.8,fill=gray!20,draw=none](-4.713,2.333)--(-4.827,2.233)--(-5.168,1.724)--cycle;
\draw(-4.827,2.233)--(-5.168,1.724);
\filldraw[fill opacity=0.8,fill=gray!20,draw=none](-4.683,2.394)--(-4.799,2.256)--(-4.9,2.146)--(-4.807,2.256)--cycle;
\draw(-4.683,2.394)--(-4.799,2.256);
\draw(-4.9,2.146)--(-4.807,2.256);
\filldraw[fill opacity=0.8,fill=gray!20,draw=none](-4.796,2.281)--(-4.737,2.339)--(-4.754,2.319)--cycle;
\draw(-4.737,2.339)--(-4.754,2.319);
\filldraw[fill opacity=0.8,fill=gray!20,draw=none](-4.799,2.256)--(-5.432,1.503)--(-5.468,1.47)--(-4.9,2.146)--cycle;
\draw(-4.799,2.256)--(-5.432,1.503);
\draw(-5.468,1.47)--(-4.9,2.146);
\filldraw[fill opacity=0.8,fill=gray!20,draw=none](-4.447,2.684)--(-4.448,2.695)--(-5.873,1)--(-5.885,.964)--(-4.518,2.591)--cycle;
\draw(-4.448,2.695)--(-5.873,1)--(-5.885,.964)--(-4.518,2.591);
\filldraw[fill opacity=0.8,fill=gray!20,draw=none](-7.929,.958)--(-7.905,.959)--(-7.907,.962)--cycle;
\draw(-7.929,.958)--(-7.905,.959)--(-7.907,.962);
\filldraw[fill opacity=0.8,fill=gray!20,draw=none](-4.461,2.877)--(-4.451,2.871)--(-4.451,2.899)--(-4.477,2.901)--cycle;
\draw(-4.451,2.871)--(-4.451,2.899)--(-4.477,2.901);
\filldraw[fill opacity=0.8,fill=gray!20,draw=none](-8.071,.935)--(-8.07,.934)--(-8.07,.935)--cycle;
\draw(-8.07,.934)--(-8.07,.935);
\filldraw[fill opacity=0.8,fill=gray!20,draw=none](-7.993,.704)--(-7.986,.709)--(-7.987,.718)--cycle;
\draw(-7.986,.709)--(-7.987,.718);
\filldraw[fill opacity=0.8,fill=gray!20,draw=none](-7.967,.904)--(-7.889,.908)--(-7.905,.959)--(-7.984,.955)--(-7.986,.907)--cycle;
\draw(-7.967,.904)--(-7.889,.908)--(-7.905,.959)--(-7.984,.955)--(-7.986,.907);
\filldraw[fill opacity=0.8,fill=gray!20,draw=none](-8.027,.92)--(-8.043,.939)--(-8.07,.935)--(-8.07,.934)--cycle;
\draw(-8.07,.935)--(-8.07,.934);
\filldraw[fill opacity=0.8,fill=gray!20,draw=none](-8.104,1.701)--(-8.043,1.735)--(-8.042,1.74)--(-8.096,1.744)--(-8.121,1.702)--cycle;
\draw(-8.043,1.735)--(-8.042,1.74)--(-8.096,1.744)--(-8.121,1.702)--(-8.104,1.701);
\filldraw[fill opacity=0.8,fill=gray!20](-2.527,7.737)--(-2.505,7.788)--(-2.54,7.765)--(-2.558,7.716)--cycle;
\filldraw[fill opacity=0.8,fill=gray!20](-2.913,7.825)--(-2.907,7.881)--(-2.929,7.905)--(-2.936,7.849)--cycle;
\filldraw[fill opacity=0.8,fill=gray!20,draw=none](-4.461,2.877)--(-4.451,2.862)--(-4.451,2.871)--cycle;
\draw(-4.451,2.862)--(-4.451,2.871);
\filldraw[fill opacity=0.8,fill=gray!20,draw=none](-8.178,1.668)--(-8.156,1.669)--(-8.127,1.687)--(-8.121,1.702)--(-8.145,1.708)--cycle;
\draw(-8.127,1.687)--(-8.121,1.702)--(-8.145,1.708);
\filldraw[fill opacity=0.8,fill=gray!20,draw=none](-7.969,.743)--(-7.97,.737)--(-7.88,.741)--(-7.876,.795)--(-7.927,.792)--cycle;
\draw(-7.97,.737)--(-7.88,.741)--(-7.876,.795)--(-7.927,.792);
\filldraw[fill opacity=0.8,fill=gray!20](-2.742,8.065)--(-2.717,8.061)--(-2.717,8.061)--(-2.713,8.066)--cycle;
\filldraw[fill opacity=0.8,fill=gray!20](-2.713,8.066)--(-2.717,8.061)--(-2.717,8.061)--(-2.685,8.064)--cycle;
\filldraw[fill opacity=0.8,fill=gray!20,draw=none](-7.932,.738)--(-7.974,.736)--(-7.987,.718)--(-7.986,.709)--cycle;
\draw(-7.932,.738)--(-7.974,.736);
\draw(-7.987,.718)--(-7.986,.709);
\filldraw[fill opacity=0.8,fill=gray!20,draw=none](-8.027,.92)--(-7.986,.907)--(-7.984,.947)--(-8.043,.939)--cycle;
\draw(-7.986,.907)--(-7.984,.947);
\filldraw[fill opacity=0.8,fill=gray!20,draw=none](-7.974,1.699)--(-7.967,1.701)--(-7.987,1.743)--(-8.042,1.74)--(-8.045,1.696)--cycle;
\draw(-7.967,1.701)--(-7.987,1.743)--(-8.042,1.74)--(-8.045,1.696)--(-7.974,1.699);
\filldraw[fill opacity=0.8,fill=gray!20](-2.619,8.043)--(-2.666,8.059)--(-2.66,8.053)--(-2.607,8.031)--cycle;
\filldraw[fill opacity=0.8,fill=gray!20,draw=none](-2.571,7.687)--(-2.567,7.69)--(-2.577,7.69)--(-2.587,7.676)--cycle;
\draw(-2.577,7.69)--(-2.587,7.676)--(-2.571,7.687);
\filldraw[fill opacity=0.8,fill=gray!20](-2.764,8.06)--(-2.717,8.061)--(-2.717,8.061)--(-2.742,8.065)--cycle;
\filldraw[fill opacity=0.8,fill=gray!20,draw=none](-8.003,1.432)--(-7.951,1.434)--(-7.949,1.442)--cycle;
\draw(-8.003,1.432)--(-7.951,1.434)--(-7.949,1.442);
\filldraw[fill opacity=0.8,fill=gray!20,draw=none](-8.067,1.422)--(-8.047,1.424)--(-8.047,1.43)--(-8.053,1.43)--cycle;
\draw(-8.047,1.424)--(-8.047,1.43)--(-8.053,1.43);
\filldraw[fill opacity=0.8,fill=gray!20](-2.505,7.788)--(-2.498,7.844)--(-2.534,7.82)--(-2.54,7.765)--cycle;
\filldraw[fill opacity=0.8,fill=gray!20](-2.907,7.881)--(-2.887,7.935)--(-2.907,7.956)--(-2.929,7.905)--cycle;
\filldraw[fill opacity=0.8,fill=gray!20](-2.685,8.064)--(-2.717,8.061)--(-2.717,8.061)--(-2.666,8.059)--cycle;
\filldraw[fill opacity=0.8,fill=gray!20,draw=none](-2.592,7.673)--(-2.571,7.687)--(-2.58,7.681)--cycle;
\draw(-2.571,7.687)--(-2.58,7.681);
\filldraw[fill opacity=0.8,fill=gray!20,draw=none](-2.774,7.64)--(-2.777,7.641)--(-2.802,7.647)--(-2.806,7.647)--cycle;
\draw(-2.774,7.64)--(-2.777,7.641)--(-2.802,7.647);
\filldraw[fill opacity=0.8,fill=gray!20,draw=none](-2.802,7.647)--(-2.815,7.65)--(-2.806,7.647)--cycle;
\draw(-2.802,7.647)--(-2.815,7.65)--(-2.806,7.647);
\filldraw[fill opacity=0.8,fill=gray!20,draw=none](-8.038,1.43)--(-8.003,1.432)--(-7.949,1.442)--(-7.941,1.482)--(-7.969,1.48)--(-8.028,1.446)--cycle;
\draw(-8.038,1.43)--(-8.003,1.432);
\draw(-7.949,1.442)--(-7.941,1.482)--(-7.969,1.48);
\filldraw[fill opacity=0.8,fill=gray!20,draw=none](-8.014,.905)--(-7.986,.903)--(-7.986,.907)--(-8.027,.92)--cycle;
\draw(-8.014,.905)--(-7.986,.903)--(-7.986,.907);
\filldraw[fill opacity=0.8,fill=gray!20,draw=none](-7.969,.87)--(-7.932,.849)--(-7.88,.852)--(-7.889,.908)--(-7.968,.904)--cycle;
\draw(-7.932,.849)--(-7.88,.852)--(-7.889,.908)--(-7.968,.904);
\filldraw[fill opacity=0.8,fill=gray!20,draw=none](-2.802,7.647)--(-2.792,7.656)--(-2.802,7.667)--(-2.856,7.68)--(-2.815,7.65)--cycle;
\draw(-2.792,7.656)--(-2.802,7.667)--(-2.856,7.68)--(-2.815,7.65)--(-2.802,7.647);
\filldraw[fill opacity=0.8,fill=gray!20,draw=none](-7.969,.743)--(-7.974,.736)--(-7.97,.737)--cycle;
\draw(-7.974,.736)--(-7.97,.737);
\filldraw[fill opacity=0.8,fill=gray!20,draw=none](-8.156,1.669)--(-8.133,1.671)--(-8.127,1.687)--cycle;
\draw(-8.133,1.671)--(-8.127,1.687);
\filldraw[fill opacity=0.8,fill=gray!20,draw=none](-7.96,.791)--(-7.876,.795)--(-7.88,.852)--(-7.938,.849)--cycle;
\draw(-7.96,.791)--(-7.876,.795)--(-7.88,.852)--(-7.938,.849);
\filldraw[fill opacity=0.8,fill=gray!20,draw=none](-7.969,.743)--(-7.927,.792)--(-7.96,.791)--cycle;
\draw(-7.927,.792)--(-7.96,.791);
\filldraw[fill opacity=0.8,fill=gray!20,draw=none](-8.104,1.701)--(-8.045,1.696)--(-8.043,1.735)--cycle;
\draw(-8.104,1.701)--(-8.045,1.696)--(-8.043,1.735);
\filldraw[fill opacity=0.8,fill=gray!20](-2.498,7.844)--(-2.505,7.899)--(-2.54,7.876)--(-2.534,7.82)--cycle;
\filldraw[fill opacity=0.8,fill=gray!20](-2.887,7.935)--(-2.856,7.983)--(-2.872,8)--(-2.907,7.956)--cycle;
\filldraw[fill opacity=0.8,fill=gray!20,draw=none](-8.053,1.43)--(-8.047,1.43)--(-8.047,1.434)--cycle;
\draw(-8.053,1.43)--(-8.047,1.43)--(-8.047,1.434);
\filldraw[fill opacity=0.8,fill=gray!20,draw=none](-8.038,1.43)--(-8.028,1.446)--(-8.047,1.434)--(-8.047,1.43)--cycle;
\draw(-8.047,1.434)--(-8.047,1.43)--(-8.038,1.43);
\filldraw[fill opacity=0.8,fill=gray!20](-2.774,8.054)--(-2.717,8.061)--(-2.717,8.061)--(-2.764,8.06)--cycle;
\filldraw[fill opacity=0.8,fill=gray!20,draw=none](-7.993,.886)--(-8.001,.904)--(-8.014,.905)--cycle;
\draw(-8.001,.904)--(-8.014,.905);
\filldraw[fill opacity=0.8,fill=gray!20,draw=none](-8.104,1.701)--(-8.121,1.702)--(-8.127,1.687)--cycle;
\draw(-8.104,1.701)--(-8.121,1.702)--(-8.127,1.687);
\filldraw[fill opacity=0.8,fill=gray!20,draw=none](-2.802,7.647)--(-2.777,7.641)--(-2.792,7.656)--cycle;
\draw(-2.802,7.647)--(-2.777,7.641)--(-2.792,7.656);
\filldraw[fill opacity=0.8,fill=gray!20,draw=none](-7.993,.886)--(-7.986,.879)--(-7.986,.903)--(-8.001,.904)--cycle;
\draw(-7.986,.879)--(-7.986,.903)--(-8.001,.904);
\filldraw[fill opacity=0.8,fill=gray!20](-2.666,8.059)--(-2.717,8.061)--(-2.717,8.061)--(-2.66,8.053)--cycle;
\filldraw[fill opacity=0.8,fill=gray!20,draw=none](-7.967,.904)--(-7.986,.907)--(-7.986,.903)--cycle;
\draw(-7.986,.907)--(-7.986,.903)--(-7.967,.904);
\filldraw[fill opacity=0.8,fill=gray!20,draw=none](-8.111,1.674)--(-8.046,1.684)--(-8.045,1.696)--(-8.104,1.701)--(-8.127,1.687)--(-8.13,1.68)--cycle;
\draw(-8.046,1.684)--(-8.045,1.696)--(-8.104,1.701);
\draw(-8.127,1.687)--(-8.13,1.68);
\filldraw[fill opacity=0.8,fill=gray!20,draw=none](-7.969,.87)--(-7.968,.904)--(-7.986,.903)--(-7.986,.879)--cycle;
\draw(-7.968,.904)--(-7.986,.903)--(-7.986,.879);
\filldraw[fill opacity=0.8,fill=gray!20,draw=none](-8.028,1.446)--(-7.969,1.48)--(-8.006,1.479)--cycle;
\draw(-7.969,1.48)--(-8.006,1.479);
\filldraw[fill opacity=0.8,fill=gray!20,draw=none](-7.97,.848)--(-7.932,.849)--(-7.969,.87)--cycle;
\draw(-7.97,.848)--(-7.932,.849);
\filldraw[fill opacity=0.8,fill=gray!20,draw=none](-8.111,1.674)--(-8.13,1.68)--(-8.133,1.671)--cycle;
\draw(-8.13,1.68)--(-8.133,1.671);
\filldraw[fill opacity=0.8,fill=gray!20,draw=none](-7.96,.791)--(-7.949,.819)--(-7.969,.841)--cycle;
\filldraw[fill opacity=0.8,fill=gray!20,draw=none](-7.974,1.699)--(-7.966,1.7)--(-7.967,1.701)--cycle;
\draw(-7.974,1.699)--(-7.966,1.7)--(-7.967,1.701);
\filldraw[fill opacity=0.8,fill=gray!20](-2.505,7.899)--(-2.527,7.951)--(-2.558,7.931)--(-2.54,7.876)--cycle;
\filldraw[fill opacity=0.8,fill=gray!20](-2.856,7.983)--(-2.815,8.021)--(-2.826,8.033)--(-2.872,8)--cycle;
\filldraw[fill opacity=0.8,fill=gray!20,draw=none](-7.969,.841)--(-7.949,.819)--(-7.938,.849)--(-7.97,.848)--cycle;
\draw(-7.938,.849)--(-7.97,.848);
\filldraw[fill opacity=0.8,fill=gray!20,draw=none](-8.041,1.652)--(-8.003,1.646)--(-7.951,1.649)--(-7.966,1.7)--(-8.045,1.696)--(-8.046,1.667)--cycle;
\draw(-8.003,1.646)--(-7.951,1.649)--(-7.966,1.7)--(-8.045,1.696)--(-8.046,1.667);
\filldraw[fill opacity=0.8,fill=gray!20,draw=none](-7.987,.868)--(-7.986,.879)--(-7.993,.886)--cycle;
\draw(-7.987,.868)--(-7.986,.879);
\filldraw[fill opacity=0.8,fill=gray!20,draw=none](-7.974,.847)--(-7.97,.848)--(-7.969,.87)--(-7.986,.879)--(-7.987,.868)--cycle;
\draw(-7.974,.847)--(-7.97,.848);
\draw(-7.986,.879)--(-7.987,.868);
\filldraw[fill opacity=0.8,fill=gray!20,draw=none](-2.592,7.673)--(-2.58,7.681)--(-2.587,7.676)--(-2.595,7.67)--cycle;
\draw(-2.58,7.681)--(-2.587,7.676)--(-2.595,7.67);
\filldraw[fill opacity=0.8,fill=gray!20,draw=none](-8.001,1.485)--(-8.002,1.479)--(-7.941,1.482)--(-7.938,1.536)--(-7.963,1.535)--cycle;
\draw(-8.002,1.479)--(-7.941,1.482)--(-7.938,1.536)--(-7.963,1.535);
\filldraw[fill opacity=0.8,fill=gray!20,draw=none](-7.969,.841)--(-7.97,.848)--(-7.974,.847)--cycle;
\draw(-7.97,.848)--(-7.974,.847);
\filldraw[fill opacity=0.8,fill=gray!20,draw=none](-8.067,1.659)--(-8.084,1.678)--(-8.111,1.674)--cycle;
\filldraw[fill opacity=0.8,fill=gray!20](-2.802,7.667)--(-2.822,7.705)--(-2.887,7.721)--(-2.856,7.68)--cycle;
\filldraw[fill opacity=0.8,fill=gray!20](-2.527,7.951)--(-2.562,7.996)--(-2.587,7.979)--(-2.558,7.931)--cycle;
\filldraw[fill opacity=0.8,fill=gray!20](-2.815,8.021)--(-2.768,8.048)--(-2.774,8.054)--(-2.826,8.033)--cycle;
\filldraw[fill opacity=0.8,fill=gray!20,draw=none](-2.592,7.673)--(-2.595,7.67)--(-2.598,7.668)--cycle;
\draw(-2.595,7.67)--(-2.598,7.668);
\filldraw[fill opacity=0.8,fill=gray!20,draw=none](-8.067,1.659)--(-8.047,1.653)--(-8.046,1.684)--(-8.084,1.678)--cycle;
\draw(-8.047,1.653)--(-8.046,1.684);
\filldraw[fill opacity=0.8,fill=gray!20,draw=none](-2.632,7.66)--(-2.598,7.668)--(-2.595,7.67)--(-2.606,7.671)--cycle;
\draw(-2.598,7.668)--(-2.595,7.67);
\filldraw[fill opacity=0.8,fill=gray!20,draw=none](-8.001,1.485)--(-8.006,1.479)--(-8.002,1.479)--cycle;
\draw(-8.006,1.479)--(-8.002,1.479);
\filldraw[fill opacity=0.8,fill=gray!20,draw=none](-2.774,7.639)--(-2.774,7.64)--(-2.775,7.64)--cycle;
\draw(-2.774,7.639)--(-2.774,7.64);
\filldraw[fill opacity=0.8,fill=gray!20,draw=none](-2.747,7.639)--(-2.769,7.64)--(-2.774,7.64)--(-2.774,7.639)--cycle;
\draw(-2.747,7.639)--(-2.769,7.64);
\draw(-2.774,7.64)--(-2.774,7.639);
\filldraw[fill opacity=0.8,fill=gray!20,draw=none](-2.769,7.64)--(-2.777,7.641)--(-2.774,7.64)--cycle;
\draw(-2.769,7.64)--(-2.777,7.641)--(-2.774,7.64);
\filldraw[fill opacity=0.8,fill=gray!20,draw=none](-2.595,7.67)--(-2.587,7.676)--(-2.612,7.671)--cycle;
\draw(-2.595,7.67)--(-2.587,7.676)--(-2.612,7.671);
\filldraw[fill opacity=0.8,fill=gray!20](-2.562,7.996)--(-2.607,8.031)--(-2.625,8.019)--(-2.587,7.979)--cycle;
\filldraw[fill opacity=0.8,fill=gray!20](-2.768,8.048)--(-2.717,8.061)--(-2.717,8.061)--(-2.774,8.054)--cycle;
\filldraw[fill opacity=0.8,fill=gray!20,draw=none](-8.001,1.485)--(-7.963,1.535)--(-7.992,1.533)--cycle;
\draw(-7.963,1.535)--(-7.992,1.533);
\filldraw[fill opacity=0.8,fill=gray!20,draw=none](-8.003,1.612)--(-7.969,1.591)--(-7.941,1.593)--(-7.951,1.649)--(-8.003,1.646)--cycle;
\draw(-7.969,1.591)--(-7.941,1.593)--(-7.951,1.649)--(-8.003,1.646);
\filldraw[fill opacity=0.8,fill=gray!20,draw=none](-7.992,1.533)--(-7.938,1.536)--(-7.941,1.593)--(-7.974,1.591)--cycle;
\draw(-7.992,1.533)--(-7.938,1.536)--(-7.941,1.593)--(-7.974,1.591);
\filldraw[fill opacity=0.8,fill=gray!20](-2.669,8.047)--(-2.717,8.061)--(-2.717,8.061)--(-2.691,8.042)--cycle;
\filldraw[fill opacity=0.8,fill=gray!20](-2.691,8.042)--(-2.717,8.061)--(-2.717,8.061)--(-2.72,8.041)--cycle;
\filldraw[fill opacity=0.8,fill=gray!20](-2.72,8.041)--(-2.717,8.061)--(-2.717,8.061)--(-2.748,8.043)--cycle;
\filldraw[fill opacity=0.8,fill=gray!20](-2.748,8.043)--(-2.717,8.061)--(-2.717,8.061)--(-2.768,8.048)--cycle;
\filldraw[fill opacity=0.8,fill=gray!20](-2.66,8.053)--(-2.717,8.061)--(-2.717,8.061)--(-2.669,8.047)--cycle;
\filldraw[fill opacity=0.8,fill=gray!20](-2.607,8.031)--(-2.66,8.053)--(-2.669,8.047)--(-2.625,8.019)--cycle;
\filldraw[fill opacity=0.8,fill=gray!20,draw=none](-2.769,7.64)--(-2.76,7.645)--(-2.786,7.65)--(-2.777,7.641)--cycle;
\draw(-2.786,7.65)--(-2.777,7.641)--(-2.769,7.64);
\filldraw[fill opacity=0.8,fill=gray!20,draw=none](-2.599,7.674)--(-2.587,7.676)--(-2.579,7.688)--cycle;
\draw(-2.599,7.674)--(-2.587,7.676)--(-2.579,7.688);
\filldraw[fill opacity=0.8,fill=gray!20,draw=none](-2.76,7.645)--(-2.769,7.64)--(-2.747,7.639)--(-2.731,7.64)--cycle;
\draw(-2.769,7.64)--(-2.747,7.639);
\filldraw[fill opacity=0.8,fill=gray!20,draw=none](-8.053,1.645)--(-8.047,1.644)--(-8.047,1.653)--(-8.067,1.659)--cycle;
\draw(-8.053,1.645)--(-8.047,1.644)--(-8.047,1.653);
\filldraw[fill opacity=0.8,fill=gray!20,draw=none](-8.041,1.652)--(-8.046,1.667)--(-8.047,1.653)--cycle;
\draw(-8.046,1.667)--(-8.047,1.653);
\filldraw[fill opacity=0.8,fill=gray!20,draw=none](-8.003,1.646)--(-8.047,1.653)--(-8.047,1.644)--cycle;
\draw(-8.047,1.653)--(-8.047,1.644)--(-8.003,1.646);
\filldraw[fill opacity=0.8,fill=gray!20,draw=none](-8.028,1.627)--(-8.003,1.612)--(-8.003,1.646)--(-8.038,1.645)--cycle;
\draw(-8.003,1.646)--(-8.038,1.645);
\filldraw[fill opacity=0.8,fill=gray!20,draw=none](-2.612,7.671)--(-2.599,7.674)--(-2.579,7.688)--(-2.558,7.716)--(-2.632,7.702)--(-2.64,7.681)--cycle;
\draw(-2.612,7.671)--(-2.599,7.674);
\draw(-2.579,7.688)--(-2.558,7.716)--(-2.632,7.702)--(-2.64,7.681);
\filldraw[fill opacity=0.8,fill=gray!20,draw=none](-8.001,1.584)--(-7.992,1.533)--(-7.984,1.561)--cycle;
\filldraw[fill opacity=0.8,fill=gray!20](-2.822,7.705)--(-2.834,7.752)--(-2.907,7.77)--(-2.887,7.721)--cycle;
\filldraw[fill opacity=0.8,fill=gray!20,draw=none](-8.002,1.59)--(-7.969,1.591)--(-8.003,1.612)--cycle;
\draw(-8.002,1.59)--(-7.969,1.591);
\filldraw[fill opacity=0.8,fill=gray!20,draw=none](-8.001,1.584)--(-7.984,1.561)--(-7.974,1.591)--(-8.002,1.59)--cycle;
\draw(-7.974,1.591)--(-8.002,1.59);
\filldraw[fill opacity=0.8,fill=gray!20,draw=none](-2.632,7.66)--(-2.606,7.671)--(-2.612,7.671)--(-2.647,7.665)--(-2.655,7.655)--cycle;
\draw(-2.612,7.671)--(-2.647,7.665)--(-2.655,7.655);
\filldraw[fill opacity=0.8,fill=gray!20,draw=none](-8.006,1.59)--(-8.002,1.59)--(-8.003,1.612)--(-8.028,1.627)--cycle;
\draw(-8.006,1.59)--(-8.002,1.59);
\filldraw[fill opacity=0.8,fill=gray!20,draw=none](-8.047,1.639)--(-8.047,1.644)--(-8.053,1.645)--cycle;
\draw(-8.047,1.639)--(-8.047,1.644)--(-8.053,1.645);
\filldraw[fill opacity=0.8,fill=gray!20,draw=none](-8.028,1.627)--(-8.038,1.645)--(-8.047,1.644)--(-8.047,1.639)--cycle;
\draw(-8.038,1.645)--(-8.047,1.644)--(-8.047,1.639);
\filldraw[fill opacity=0.8,fill=gray!20,draw=none](-2.76,7.645)--(-2.73,7.661)--(-2.802,7.667)--(-2.786,7.65)--cycle;
\draw(-2.73,7.661)--(-2.802,7.667)--(-2.786,7.65);
\filldraw[fill opacity=0.8,fill=gray!20,draw=none](-8.001,1.584)--(-8.002,1.59)--(-8.006,1.59)--cycle;
\draw(-8.002,1.59)--(-8.006,1.59);
\filldraw[fill opacity=0.8,fill=gray!20,draw=none](-2.632,7.66)--(-2.655,7.655)--(-2.661,7.648)--cycle;
\draw(-2.655,7.655)--(-2.661,7.648);
\filldraw[fill opacity=0.8,fill=gray!20,draw=none](-2.76,7.645)--(-2.731,7.64)--(-2.724,7.64)--(-2.726,7.661)--(-2.73,7.661)--cycle;
\draw(-2.724,7.64)--(-2.726,7.661)--(-2.73,7.661);
\filldraw[fill opacity=0.8,fill=gray!20,draw=none](-2.731,7.64)--(-2.724,7.638)--(-2.724,7.64)--cycle;
\draw(-2.724,7.638)--(-2.724,7.64);
\filldraw[fill opacity=0.8,fill=gray!20,draw=none](-2.681,7.648)--(-2.725,7.648)--(-2.724,7.638)--cycle;
\draw(-2.725,7.648)--(-2.724,7.638);
\filldraw[fill opacity=0.8,fill=gray!20,draw=none](-2.661,7.648)--(-2.656,7.654)--(-2.681,7.648)--cycle;
\draw(-2.661,7.648)--(-2.656,7.654);
\filldraw[fill opacity=0.8,fill=gray!20](-2.777,8.012)--(-2.748,8.043)--(-2.768,8.048)--(-2.815,8.021)--cycle;
\filldraw[fill opacity=0.8,fill=gray!20,draw=none](-2.612,7.671)--(-2.64,7.681)--(-2.647,7.665)--cycle;
\draw(-2.64,7.681)--(-2.647,7.665)--(-2.612,7.671);
\filldraw[fill opacity=0.8,fill=gray!20,draw=none](-2.681,7.648)--(-2.656,7.654)--(-2.647,7.665)--(-2.726,7.661)--(-2.725,7.648)--cycle;
\draw(-2.656,7.654)--(-2.647,7.665)--(-2.726,7.661)--(-2.725,7.648);
\filldraw[fill opacity=0.8,fill=gray!20](-2.625,8.019)--(-2.669,8.047)--(-2.691,8.042)--(-2.668,8.011)--cycle;
\filldraw[fill opacity=0.8,fill=gray!20](-2.558,7.716)--(-2.54,7.765)--(-2.622,7.75)--(-2.632,7.702)--cycle;
\filldraw[fill opacity=0.8,fill=gray!20](-2.834,7.752)--(-2.838,7.806)--(-2.913,7.825)--(-2.907,7.77)--cycle;
\filldraw[fill opacity=0.8,fill=gray!20](-2.802,7.97)--(-2.777,8.012)--(-2.815,8.021)--(-2.856,7.983)--cycle;
\filldraw[fill opacity=0.8,fill=gray!20](-2.723,8.008)--(-2.72,8.041)--(-2.748,8.043)--(-2.777,8.012)--cycle;
\filldraw[fill opacity=0.8,fill=gray!20](-2.726,7.661)--(-2.728,7.698)--(-2.822,7.705)--(-2.802,7.667)--cycle;
\filldraw[fill opacity=0.8,fill=gray!20](-2.838,7.806)--(-2.834,7.863)--(-2.907,7.881)--(-2.913,7.825)--cycle;
\filldraw[fill opacity=0.8,fill=gray!20](-2.668,8.011)--(-2.691,8.042)--(-2.72,8.041)--(-2.723,8.008)--cycle;
\filldraw[fill opacity=0.8,fill=gray!20](-2.587,7.979)--(-2.625,8.019)--(-2.668,8.011)--(-2.647,7.968)--cycle;
\filldraw[fill opacity=0.8,fill=gray!20](-2.822,7.919)--(-2.802,7.97)--(-2.856,7.983)--(-2.887,7.935)--cycle;
\filldraw[fill opacity=0.8,fill=gray!20](-2.54,7.765)--(-2.534,7.82)--(-2.619,7.804)--(-2.622,7.75)--cycle;
\filldraw[fill opacity=0.8,fill=gray!20,draw=none](-2.687,7.663)--(-2.647,7.665)--(-2.632,7.702)--(-2.728,7.698)--(-2.727,7.684)--cycle;
\draw(-2.687,7.663)--(-2.647,7.665)--(-2.632,7.702)--(-2.728,7.698)--(-2.727,7.684);
\filldraw[fill opacity=0.8,fill=gray!20](-2.834,7.863)--(-2.822,7.919)--(-2.887,7.935)--(-2.907,7.881)--cycle;
\filldraw[fill opacity=0.8,fill=gray!20,draw=none](-2.687,7.663)--(-2.727,7.684)--(-2.726,7.661)--cycle;
\draw(-2.727,7.684)--(-2.726,7.661)--(-2.687,7.663);
\filldraw[fill opacity=0.8,fill=gray!20](-2.558,7.931)--(-2.587,7.979)--(-2.647,7.968)--(-2.632,7.916)--cycle;
\filldraw[fill opacity=0.8,fill=gray!20](-2.534,7.82)--(-2.54,7.876)--(-2.622,7.86)--(-2.619,7.804)--cycle;
\filldraw[fill opacity=0.8,fill=gray!20](-2.54,7.876)--(-2.558,7.931)--(-2.632,7.916)--(-2.622,7.86)--cycle;
\filldraw[fill opacity=0.8,fill=gray!20](-2.726,7.964)--(-2.723,8.008)--(-2.777,8.012)--(-2.802,7.97)--cycle;
\filldraw[fill opacity=0.8,fill=gray!20](-2.728,7.698)--(-2.73,7.745)--(-2.834,7.752)--(-2.822,7.705)--cycle;
\filldraw[fill opacity=0.8,fill=gray!20](-2.647,7.968)--(-2.668,8.011)--(-2.723,8.008)--(-2.726,7.964)--cycle;
\filldraw[fill opacity=0.8,fill=gray!20](-2.632,7.702)--(-2.622,7.75)--(-2.73,7.745)--(-2.728,7.698)--cycle;
\filldraw[fill opacity=0.8,fill=gray!20](-2.728,7.912)--(-2.726,7.964)--(-2.802,7.97)--(-2.822,7.919)--cycle;
\filldraw[fill opacity=0.8,fill=gray!20](-2.73,7.745)--(-2.73,7.799)--(-2.838,7.806)--(-2.834,7.752)--cycle;
\filldraw[fill opacity=0.8,fill=gray!20](-2.632,7.916)--(-2.647,7.968)--(-2.726,7.964)--(-2.728,7.912)--cycle;
\filldraw[fill opacity=0.8,fill=gray!20](-2.622,7.75)--(-2.619,7.804)--(-2.73,7.799)--(-2.73,7.745)--cycle;
\filldraw[fill opacity=0.8,fill=gray!20](-2.73,7.799)--(-2.73,7.856)--(-2.834,7.863)--(-2.838,7.806)--cycle;
\filldraw[fill opacity=0.8,fill=gray!20](-2.73,7.856)--(-2.728,7.912)--(-2.822,7.919)--(-2.834,7.863)--cycle;
\filldraw[fill opacity=0.8,fill=gray!20](-2.622,7.86)--(-2.632,7.916)--(-2.728,7.912)--(-2.73,7.856)--cycle;
\filldraw[fill opacity=0.8,fill=gray!20](-2.619,7.804)--(-2.622,7.86)--(-2.73,7.856)--(-2.73,7.799)--cycle;
\filldraw[fill opacity=0.8,fill=gray!20,draw=none](-8.484,.804)--(-8.499,.796)--(-8.522,.788)--(-8.487,.811)--(-8.465,.818)--cycle;
\draw(-8.499,.796)--(-8.522,.788)--(-8.487,.811)--(-8.465,.818);
\filldraw[fill opacity=0.8,fill=gray!20,draw=none](-8.287,.875)--(-8.487,.811)--(-8.445,.814)--(-8.338,.849)--cycle;
\draw(-8.287,.875)--(-8.487,.811)--(-8.445,.814)--(-8.338,.849);
\filldraw[fill opacity=0.8,fill=gray!20,draw=none](-4.713,2.333)--(-4.698,2.346)--(-4.652,2.415)--cycle;
\draw(-4.698,2.346)--(-4.652,2.415);
\filldraw[fill opacity=0.8,fill=gray!20,draw=none](-8.246,1.35)--(-8.362,1.3)--(-8.4,1.276)--(-8.212,1.358)--cycle;
\draw(-8.246,1.35)--(-8.362,1.3)--(-8.4,1.276)--(-8.212,1.358);
\filldraw[fill opacity=0.8,fill=gray!20,draw=none](-6.245,.412)--(-6.231,.453)--(-6.114,.454)--(-6.097,.436)--(-6.233,.379)--cycle;
\filldraw[fill opacity=0.8,fill=gray!20,draw=none](-6.111,.367)--(-6.123,.37)--(-6.123,.378)--cycle;
\draw(-6.123,.37)--(-6.123,.378);
\filldraw[fill opacity=0.8,fill=gray!20,draw=none](-6.068,.455)--(-6.062,.44)--(-6.107,.373)--(-6.148,.385)--(-6.157,.39)--(-6.114,.454)--cycle;
\draw(-6.062,.44)--(-6.107,.373);
\draw(-6.157,.39)--(-6.114,.454);
\filldraw[fill opacity=0.8,fill=gray!20,draw=none](-5.996,.484)--(-5.987,.479)--(-6.034,.499)--(-6.064,.52)--(-6.036,.508)--cycle;
\draw(-5.987,.479)--(-6.034,.499);
\draw(-6.064,.52)--(-6.036,.508);
\filldraw[fill opacity=0.8,fill=gray!20,draw=none](-5.969,.467)--(-5.967,.465)--(-5.967,.506)--cycle;
\draw(-5.967,.465)--(-5.967,.506);
\filldraw[fill opacity=0.8,fill=gray!20,draw=none](-4.713,2.333)--(-5.168,1.724)--(-6.039,.419)--(-6.01,.382)--(-4.698,2.346)--cycle;
\draw(-5.168,1.724)--(-6.039,.419)--(-6.01,.382)--(-4.698,2.346);
\filldraw[fill opacity=0.8,fill=gray!20,draw=none](-8.384,.584)--(-8.374,.583)--(-8.375,.581)--cycle;
\draw(-8.384,.584)--(-8.374,.583)--(-8.375,.581);
\filldraw[fill opacity=0.8,fill=gray!20,draw=none](-8.376,.58)--(-8.375,.581)--(-8.374,.58)--cycle;
\draw(-8.376,.58)--(-8.375,.581);
\filldraw[fill opacity=0.8,fill=gray!20](-8.442,.524)--(-8.44,.551)--(-8.395,.548)--(-8.419,.522)--cycle;
\filldraw[fill opacity=0.8,fill=gray!20](-8.466,.522)--(-8.486,.549)--(-8.44,.551)--(-8.442,.524)--cycle;
\filldraw[fill opacity=0.8,fill=gray!20,draw=none](-8.524,.757)--(-8.513,.77)--(-8.48,.756)--(-8.504,.711)--(-8.53,.722)--cycle;
\draw(-8.513,.77)--(-8.48,.756)--(-8.504,.711)--(-8.53,.722);
\filldraw[fill opacity=0.8,fill=gray!20,draw=none](-8.539,.707)--(-8.53,.722)--(-8.504,.711)--(-8.519,.663)--(-8.532,.669)--cycle;
\draw(-8.53,.722)--(-8.504,.711)--(-8.519,.663)--(-8.532,.669);
\filldraw[fill opacity=0.8,fill=gray!20](-8.485,.519)--(-8.521,.542)--(-8.486,.549)--(-8.466,.522)--cycle;
\filldraw[fill opacity=0.8,fill=gray!20,draw=none](-8.395,.548)--(-8.393,.551)--(-8.36,.569)--(-8.345,.557)--(-8.363,.54)--cycle;
\draw(-8.345,.557)--(-8.363,.54)--(-8.395,.548)--(-8.393,.551);
\filldraw[fill opacity=0.8,fill=gray!20,draw=none](-8.398,.548)--(-8.393,.551)--(-8.395,.548)--cycle;
\draw(-8.393,.551)--(-8.395,.548)--(-8.398,.548);
\filldraw[fill opacity=0.8,fill=gray!20](-8.419,.522)--(-8.395,.548)--(-8.363,.54)--(-8.403,.518)--cycle;
\filldraw[fill opacity=0.8,fill=gray!20,draw=none](-8.437,.831)--(-8.437,.84)--(-8.384,.836)--cycle;
\draw(-8.437,.831)--(-8.437,.84)--(-8.384,.836);
\filldraw[fill opacity=0.8,fill=gray!20,draw=none](-8.446,.829)--(-8.441,.84)--(-8.437,.84)--(-8.437,.831)--cycle;
\draw(-8.441,.84)--(-8.437,.84)--(-8.437,.831);
\filldraw[fill opacity=0.8,fill=gray!20,draw=none](-8.446,.829)--(-8.511,.818)--(-8.503,.837)--(-8.441,.84)--cycle;
\draw(-8.511,.818)--(-8.503,.837)--(-8.441,.84);
\filldraw[fill opacity=0.8,fill=gray!20,draw=none](-8.542,.653)--(-8.532,.669)--(-8.519,.663)--(-8.522,.626)--cycle;
\draw(-8.532,.669)--(-8.519,.663)--(-8.522,.626);
\filldraw[fill opacity=0.8,fill=gray!20,draw=none](-8.36,.569)--(-8.346,.576)--(-8.329,.572)--(-8.345,.557)--cycle;
\draw(-8.346,.576)--(-8.329,.572)--(-8.345,.557);
\filldraw[fill opacity=0.8,fill=gray!20,draw=none](-8.346,.576)--(-8.335,.583)--(-8.322,.584)--(-8.329,.572)--cycle;
\draw(-8.322,.584)--(-8.329,.572)--(-8.346,.576);
\filldraw[fill opacity=0.8,fill=gray!20](-8.536,.532)--(-8.574,.561)--(-8.553,.575)--(-8.521,.542)--cycle;
\filldraw[fill opacity=0.8,fill=gray!20,draw=none](-8.384,.836)--(-8.437,.84)--(-8.438,.85)--(-8.376,.837)--(-8.375,.837)--cycle;
\draw(-8.384,.836)--(-8.437,.84)--(-8.438,.85);
\draw(-8.376,.837)--(-8.375,.837);
\filldraw[fill opacity=0.8,fill=gray!20](-8.503,.837)--(-8.486,.858)--(-8.44,.86)--(-8.437,.84)--cycle;
\filldraw[fill opacity=0.8,fill=gray!20,draw=none](-8.335,.583)--(-8.314,.595)--(-8.322,.584)--cycle;
\draw(-8.314,.595)--(-8.322,.584);
\filldraw[fill opacity=0.8,fill=gray!20,draw=none](-8.322,.584)--(-8.314,.595)--(-8.298,.607)--(-8.287,.594)--(-8.298,.58)--cycle;
\draw(-8.322,.584)--(-8.314,.595);
\draw(-8.298,.607)--(-8.287,.594)--(-8.298,.58);
\filldraw[fill opacity=0.8,fill=gray!20,draw=none](-8.329,.572)--(-8.322,.584)--(-8.298,.58)--(-8.316,.558)--cycle;
\draw(-8.298,.58)--(-8.316,.558)--(-8.329,.572)--(-8.322,.584);
\filldraw[fill opacity=0.8,fill=gray!20](-8.445,.507)--(-8.466,.522)--(-8.442,.524)--(-8.445,.507)--cycle;
\filldraw[fill opacity=0.8,fill=gray!20](-8.445,.507)--(-8.442,.524)--(-8.419,.522)--(-8.445,.507)--cycle;
\filldraw[fill opacity=0.8,fill=gray!20,draw=none](-8.52,.834)--(-8.529,.845)--(-8.521,.851)--(-8.486,.858)--(-8.503,.837)--cycle;
\draw(-8.529,.845)--(-8.521,.851)--(-8.486,.858)--(-8.503,.837)--(-8.52,.834);
\filldraw[fill opacity=0.8,fill=gray!20](-8.363,.54)--(-8.329,.572)--(-8.316,.558)--(-8.354,.53)--cycle;
\filldraw[fill opacity=0.8,fill=gray!20](-8.492,.514)--(-8.536,.532)--(-8.521,.542)--(-8.485,.519)--cycle;
\filldraw[fill opacity=0.8,fill=gray!20,draw=none](-8.616,.695)--(-8.615,.703)--(-8.613,.702)--cycle;
\draw(-8.615,.703)--(-8.613,.702);
\filldraw[fill opacity=0.8,fill=gray!20,draw=none](-8.611,.649)--(-8.621,.642)--(-8.623,.653)--(-8.616,.695)--(-8.615,.697)--cycle;
\draw(-8.611,.649)--(-8.621,.642)--(-8.623,.653);
\draw(-8.616,.695)--(-8.615,.697);
\filldraw[fill opacity=0.8,fill=gray!20,draw=none](-8.613,.702)--(-8.615,.697)--(-8.616,.695)--cycle;
\draw(-8.615,.697)--(-8.616,.695);
\filldraw[fill opacity=0.8,fill=gray!20,draw=none](-8.616,.695)--(-8.613,.702)--(-8.563,.681)--(-8.559,.676)--(-8.595,.648)--(-8.622,.659)--cycle;
\draw(-8.613,.702)--(-8.563,.681);
\draw(-8.595,.648)--(-8.622,.659);
\filldraw[fill opacity=0.8,fill=gray!20](-8.445,.507)--(-8.485,.519)--(-8.466,.522)--(-8.445,.507)--cycle;
\filldraw[fill opacity=0.8,fill=gray!20,draw=none](-8.526,.614)--(-8.522,.614)--(-8.52,.607)--cycle;
\draw(-8.522,.614)--(-8.52,.607);
\filldraw[fill opacity=0.8,fill=gray!20](-8.344,.72)--(-8.368,.765)--(-8.403,.798)--(-8.445,.814)--(-8.487,.811)--(-8.522,.788)--(-8.546,.75)--(-8.554,.702)--(-8.546,.651)--(-8.522,.606)--(-8.487,.573)--(-8.445,.557)--(-8.403,.561)--(-8.368,.583)--(-8.344,.622)--(-8.336,.67)--cycle;
\filldraw[fill opacity=0.8,fill=gray!20,draw=none](-8.438,.85)--(-8.44,.86)--(-8.395,.857)--(-8.376,.837)--cycle;
\draw(-8.438,.85)--(-8.44,.86)--(-8.395,.857)--(-8.376,.837);
\filldraw[fill opacity=0.8,fill=gray!20](-8.445,.507)--(-8.419,.522)--(-8.403,.518)--(-8.445,.507)--cycle;
\filldraw[fill opacity=0.8,fill=gray!20,draw=none](-8.636,.712)--(-8.615,.703)--(-8.622,.659)--(-8.625,.66)--cycle;
\draw(-8.636,.712)--(-8.615,.703);
\draw(-8.622,.659)--(-8.625,.66);
\filldraw[fill opacity=0.8,fill=gray!20,draw=none](-8.6,.595)--(-8.598,.601)--(-8.609,.623)--(-8.619,.635)--(-8.603,.599)--cycle;
\draw(-8.619,.635)--(-8.603,.599)--(-8.6,.595);
\filldraw[fill opacity=0.8,fill=gray!20,draw=none](-8.609,.623)--(-8.617,.637)--(-8.621,.642)--(-8.619,.635)--cycle;
\draw(-8.617,.637)--(-8.621,.642)--(-8.619,.635);
\filldraw[fill opacity=0.8,fill=gray!20,draw=none](-8.622,.659)--(-8.623,.653)--(-8.624,.657)--cycle;
\draw(-8.623,.653)--(-8.624,.657);
\filldraw[fill opacity=0.8,fill=gray!20,draw=none](-8.617,.637)--(-8.624,.657)--(-8.621,.642)--cycle;
\draw(-8.624,.657)--(-8.621,.642)--(-8.617,.637);
\filldraw[fill opacity=0.8,fill=gray!20,draw=none](-8.622,.659)--(-8.534,.623)--(-8.526,.614)--(-8.58,.611)--(-8.639,.635)--cycle;
\draw(-8.622,.659)--(-8.534,.623);
\draw(-8.58,.611)--(-8.639,.635);
\filldraw[fill opacity=0.8,fill=gray!20](-8.403,.518)--(-8.363,.54)--(-8.354,.53)--(-8.398,.513)--cycle;
\filldraw[fill opacity=0.8,fill=gray!20,draw=none](-8.374,.837)--(-8.376,.837)--(-8.395,.857)--(-8.363,.849)--(-8.345,.836)--cycle;
\draw(-8.376,.837)--(-8.395,.857)--(-8.363,.849)--(-8.345,.836);
\filldraw[fill opacity=0.8,fill=gray!20,draw=none](-8.365,.833)--(-8.374,.837)--(-8.345,.836)--(-8.329,.825)--cycle;
\draw(-8.345,.836)--(-8.329,.825)--(-8.365,.833);
\filldraw[fill opacity=0.8,fill=gray!20,draw=none](-8.526,.614)--(-8.52,.607)--(-8.516,.59)--(-8.52,.586)--(-8.58,.611)--cycle;
\draw(-8.52,.607)--(-8.516,.59);
\draw(-8.52,.586)--(-8.58,.611);
\filldraw[fill opacity=0.8,fill=gray!20,draw=none](-8.41,.616)--(-8.44,.582)--(-8.47,.564)--(-8.496,.564)--(-8.514,.582)--(-8.516,.59)--(-8.52,.607)--(-8.37,.765)--(-8.367,.753)--(-8.371,.708)--(-8.386,.66)--cycle;
\draw(-8.516,.59)--(-8.52,.607);
\draw(-8.37,.765)--(-8.367,.753)--(-8.371,.708)--(-8.386,.66)--(-8.41,.616)--(-8.44,.582)--(-8.47,.564)--(-8.496,.564)--(-8.514,.582);
\filldraw[fill opacity=0.8,fill=gray!20,draw=none](-8.578,.797)--(-8.582,.791)--(-8.583,.79)--cycle;
\draw(-8.582,.791)--(-8.583,.79);
\filldraw[fill opacity=0.8,fill=gray!20,draw=none](-8.598,.805)--(-8.583,.799)--(-8.562,.762)--(-8.591,.748)--(-8.628,.763)--cycle;
\draw(-8.598,.805)--(-8.583,.799);
\draw(-8.591,.748)--(-8.628,.763);
\filldraw[fill opacity=0.8,fill=gray!20,draw=none](-8.3,.605)--(-8.299,.607)--(-8.298,.607)--cycle;
\draw(-8.299,.607)--(-8.298,.607);
\filldraw[fill opacity=0.8,fill=gray!20,draw=none](-8.335,.82)--(-8.365,.833)--(-8.346,.829)--cycle;
\draw(-8.365,.833)--(-8.346,.829);
\filldraw[fill opacity=0.8,fill=gray!20,draw=none](-8.288,.615)--(-8.298,.607)--(-8.299,.607)--(-8.278,.643)--cycle;
\draw(-8.298,.607)--(-8.299,.607);
\filldraw[fill opacity=0.8,fill=gray!20,draw=none](-8.298,.607)--(-8.272,.629)--(-8.287,.594)--cycle;
\draw(-8.272,.629)--(-8.287,.594)--(-8.298,.607);
\filldraw[fill opacity=0.8,fill=gray!20](-8.445,.507)--(-8.492,.514)--(-8.485,.519)--(-8.445,.507)--cycle;
\filldraw[fill opacity=0.8,fill=gray!20](-8.445,.507)--(-8.403,.518)--(-8.398,.513)--(-8.445,.507)--cycle;
\filldraw[fill opacity=0.8,fill=gray!20,draw=none](-8.602,.747)--(-8.621,.734)--(-8.603,.777)--(-8.583,.79)--cycle;
\draw(-8.602,.747)--(-8.621,.734)--(-8.603,.777)--(-8.583,.79);
\filldraw[fill opacity=0.8,fill=gray!20,draw=none](-8.613,.702)--(-8.616,.695)--(-8.628,.688)--(-8.621,.734)--(-8.603,.747)--cycle;
\draw(-8.616,.695)--(-8.628,.688)--(-8.621,.734)--(-8.603,.747);
\filldraw[fill opacity=0.8,fill=gray!20,draw=none](-8.556,.825)--(-8.53,.846)--(-8.526,.848)--(-8.529,.845)--(-8.553,.828)--cycle;
\draw(-8.53,.846)--(-8.526,.848);
\draw(-8.529,.845)--(-8.553,.828)--(-8.556,.825);
\filldraw[fill opacity=0.8,fill=gray!20,draw=none](-8.526,.848)--(-8.521,.851)--(-8.529,.845)--cycle;
\draw(-8.526,.848)--(-8.521,.851)--(-8.529,.845);
\filldraw[fill opacity=0.8,fill=gray!20](-8.521,.851)--(-8.485,.864)--(-8.466,.867)--(-8.486,.858)--cycle;
\filldraw[fill opacity=0.8,fill=gray!20,draw=none](-8.526,.848)--(-8.52,.849)--(-8.497,.857)--(-8.492,.859)--(-8.485,.864)--(-8.521,.851)--cycle;
\draw(-8.497,.857)--(-8.492,.859)--(-8.485,.864)--(-8.521,.851)--(-8.526,.848);
\filldraw[fill opacity=0.8,fill=gray!20,draw=none](-8.526,.848)--(-8.53,.846)--(-8.52,.849)--cycle;
\draw(-8.526,.848)--(-8.53,.846);
\filldraw[fill opacity=0.8,fill=gray!20,draw=none](-9.436,1.215)--(-9.436,1.216)--(-8.619,1.4)--(-8.599,1.352)--(-9.354,1.182)--cycle;
\draw(-9.436,1.216)--(-8.619,1.4);
\draw(-8.599,1.352)--(-9.354,1.182);
\filldraw[fill opacity=0.8,fill=gray!20](-9.653,1.136)--(-9.656,1.193)--(-9.545,1.198)--(-9.545,1.141)--cycle;
\filldraw[fill opacity=0.8,fill=gray!20,draw=none](-9.459,1.221)--(-9.442,1.262)--(-8.692,1.43)--(-8.684,1.385)--(-9.456,1.212)--cycle;
\draw(-9.442,1.262)--(-8.692,1.43);
\draw(-8.684,1.385)--(-9.456,1.212);
\filldraw[fill opacity=0.8,fill=gray!20,draw=none](-9.354,1.182)--(-9.272,1.201)--(-9.203,1.166)--(-9.266,1.152)--cycle;
\draw(-9.354,1.182)--(-9.272,1.201);
\draw(-9.203,1.166)--(-9.266,1.152);
\filldraw[fill opacity=0.8,fill=gray!20,draw=none](-9.266,1.152)--(-9.203,1.166)--(-9.155,1.136)--(-9.189,1.128)--cycle;
\draw(-9.266,1.152)--(-9.203,1.166);
\draw(-9.155,1.136)--(-9.189,1.128);
\filldraw[fill opacity=0.8,fill=gray!20,draw=none](-9.3,1.144)--(-9.266,1.152)--(-9.189,1.128)--cycle;
\draw(-9.3,1.144)--(-9.266,1.152);
\filldraw[fill opacity=0.8,fill=gray!20,draw=none](-9.411,1.169)--(-9.354,1.182)--(-9.266,1.152)--(-9.3,1.144)--cycle;
\draw(-9.411,1.169)--(-9.354,1.182);
\draw(-9.266,1.152)--(-9.3,1.144);
\filldraw[fill opacity=0.8,fill=gray!20,draw=none](-9.444,1.181)--(-9.436,1.215)--(-9.354,1.182)--(-9.41,1.17)--cycle;
\draw(-9.354,1.182)--(-9.41,1.17);
\filldraw[fill opacity=0.8,fill=gray!20,draw=none](-9.622,1.194)--(-9.656,1.203)--(-9.653,1.247)--(-9.545,1.252)--(-9.545,1.198)--cycle;
\draw(-9.656,1.203)--(-9.653,1.247)--(-9.545,1.252)--(-9.545,1.198)--(-9.622,1.194);
\filldraw[fill opacity=0.8,fill=gray!20,draw=none](-9.479,1.232)--(-9.469,1.209)--(-9.471,1.192)--(-9.545,1.198)--(-9.545,1.249)--cycle;
\draw(-9.471,1.192)--(-9.545,1.198)--(-9.545,1.249);
\filldraw[fill opacity=0.8,fill=gray!20,draw=none](-9.461,1.187)--(-9.463,1.21)--(-9.436,1.216)--(-9.444,1.181)--cycle;
\draw(-9.463,1.21)--(-9.436,1.216);
\filldraw[fill opacity=0.8,fill=gray!20,draw=none](-9.467,1.228)--(-9.545,1.249)--(-9.545,1.252)--(-9.465,1.246)--cycle;
\draw(-9.545,1.249)--(-9.545,1.252)--(-9.465,1.246);
\filldraw[fill opacity=0.8,fill=gray!20,draw=none](-9.523,1.25)--(-9.545,1.252)--(-9.547,1.299)--(-9.495,1.295)--(-9.47,1.282)--cycle;
\draw(-9.523,1.25)--(-9.545,1.252)--(-9.547,1.299)--(-9.495,1.295);
\filldraw[fill opacity=0.8,fill=gray!20,draw=none](-9.653,1.247)--(-9.646,1.279)--(-9.547,1.299)--(-9.545,1.252)--cycle;
\draw(-9.547,1.299)--(-9.545,1.252)--(-9.653,1.247)--(-9.646,1.279);
\filldraw[fill opacity=0.8,fill=gray!20,draw=none](-9.459,1.221)--(-9.456,1.212)--(-9.463,1.21)--cycle;
\draw(-9.456,1.212)--(-9.463,1.21);
\filldraw[fill opacity=0.8,fill=gray!20,draw=none](-9.468,1.256)--(-9.472,1.246)--(-9.523,1.25)--(-9.474,1.28)--cycle;
\draw(-9.472,1.246)--(-9.523,1.25);
\filldraw[fill opacity=0.8,fill=gray!20,draw=none](-9.459,1.163)--(-9.46,1.162)--(-9.483,1.193)--(-9.462,1.192)--cycle;
\draw(-9.483,1.193)--(-9.462,1.192);
\filldraw[fill opacity=0.8,fill=gray!20,draw=none](-9.469,1.209)--(-9.462,1.192)--(-9.471,1.192)--cycle;
\draw(-9.462,1.192)--(-9.471,1.192);
\filldraw[fill opacity=0.8,fill=gray!20,draw=none](-9.461,1.187)--(-9.503,1.201)--(-9.463,1.21)--cycle;
\draw(-9.503,1.201)--(-9.463,1.21);
\filldraw[fill opacity=0.8,fill=gray!20,draw=none](-9.46,1.162)--(-9.498,1.137)--(-9.545,1.141)--(-9.545,1.198)--(-9.483,1.193)--cycle;
\draw(-9.498,1.137)--(-9.545,1.141)--(-9.545,1.198)--(-9.483,1.193);
\filldraw[fill opacity=0.8,fill=gray!20,draw=none](-9.469,1.209)--(-9.479,1.232)--(-9.467,1.228)--cycle;
\filldraw[fill opacity=0.8,fill=gray!20,draw=none](-9.455,1.268)--(-9.433,1.297)--(-9.238,1.341)--(-9.323,1.288)--(-9.456,1.258)--cycle;
\draw(-9.433,1.297)--(-9.238,1.341);
\draw(-9.323,1.288)--(-9.456,1.258);
\filldraw[fill opacity=0.8,fill=gray!20,draw=none](-9.468,1.256)--(-9.442,1.262)--(-9.459,1.221)--cycle;
\draw(-9.468,1.256)--(-9.442,1.262);
\filldraw[fill opacity=0.8,fill=gray!20,draw=none](-9.455,1.268)--(-9.456,1.258)--(-9.464,1.257)--cycle;
\draw(-9.456,1.258)--(-9.464,1.257);
\filldraw[fill opacity=0.8,fill=gray!20,draw=none](-9.468,1.256)--(-9.474,1.28)--(-9.47,1.282)--(-9.459,1.275)--cycle;
\filldraw[fill opacity=0.8,fill=gray!20,draw=none](-9.453,1.29)--(-9.455,1.268)--(-9.464,1.257)--(-9.468,1.256)--cycle;
\draw(-9.464,1.257)--(-9.468,1.256);
\filldraw[fill opacity=0.8,fill=gray!20,draw=none](-9.453,1.29)--(-9.452,1.291)--(-9.449,1.277)--(-9.455,1.268)--cycle;
\filldraw[fill opacity=0.8,fill=gray!20,draw=none](-9.472,1.246)--(-9.459,1.275)--(-9.447,1.269)--(-9.441,1.244)--cycle;
\draw(-9.447,1.269)--(-9.441,1.244)--(-9.472,1.246);
\filldraw[fill opacity=0.8,fill=gray!20,draw=none](-9.535,1.241)--(-9.468,1.256)--(-9.459,1.221)--(-9.463,1.21)--(-9.58,1.184)--cycle;
\draw(-9.535,1.241)--(-9.468,1.256);
\draw(-9.463,1.21)--(-9.58,1.184);
\filldraw[fill opacity=0.8,fill=gray!20,draw=none](-9.469,1.209)--(-9.467,1.228)--(-9.443,1.222)--(-9.445,1.191)--(-9.462,1.192)--cycle;
\draw(-9.445,1.191)--(-9.462,1.192);
\filldraw[fill opacity=0.8,fill=gray!20,draw=none](-9.445,1.191)--(-9.443,1.222)--(-9.439,1.221)--(-9.437,1.19)--cycle;
\draw(-9.439,1.221)--(-9.437,1.19)--(-9.445,1.191);
\filldraw[fill opacity=0.8,fill=gray!20,draw=none](-9.419,1.219)--(-9.426,1.198)--(-9.437,1.19)--(-9.439,1.221)--cycle;
\draw(-9.437,1.19)--(-9.439,1.221);
\filldraw[fill opacity=0.8,fill=gray!20,draw=none](-9.443,1.222)--(-9.467,1.228)--(-9.465,1.246)--(-9.441,1.244)--cycle;
\draw(-9.465,1.246)--(-9.441,1.244);
\filldraw[fill opacity=0.8,fill=gray!20,draw=none](-9.443,1.222)--(-9.441,1.244)--(-9.439,1.221)--cycle;
\draw(-9.441,1.244)--(-9.439,1.221);
\filldraw[fill opacity=0.8,fill=gray!20,draw=none](-9.419,1.219)--(-9.439,1.221)--(-9.441,1.244)--(-9.413,1.237)--cycle;
\draw(-9.439,1.221)--(-9.441,1.244)--(-9.413,1.237);
\filldraw[fill opacity=0.8,fill=gray!20,draw=none](-9.401,1.217)--(-9.426,1.198)--(-9.419,1.219)--cycle;
\filldraw[fill opacity=0.8,fill=gray!20,draw=none](-9.47,1.282)--(-9.453,1.292)--(-9.447,1.269)--cycle;
\draw(-9.453,1.292)--(-9.447,1.269);
\filldraw[fill opacity=0.8,fill=gray!20,draw=none](-9.47,1.282)--(-9.495,1.295)--(-9.453,1.292)--cycle;
\draw(-9.495,1.295)--(-9.453,1.292);
\filldraw[fill opacity=0.8,fill=gray!20,draw=none](-9.457,1.292)--(-9.452,1.293)--(-9.453,1.29)--(-9.468,1.256)--(-9.51,1.246)--cycle;
\draw(-9.457,1.292)--(-9.452,1.293);
\draw(-9.468,1.256)--(-9.51,1.246);
\filldraw[fill opacity=0.8,fill=gray!20,draw=none](-9.387,1.227)--(-9.401,1.217)--(-9.419,1.219)--(-9.413,1.237)--(-9.393,1.232)--cycle;
\draw(-9.413,1.237)--(-9.393,1.232);
\filldraw[fill opacity=0.8,fill=gray!20,draw=none](-9.457,1.292)--(-9.51,1.246)--(-9.535,1.241)--cycle;
\draw(-9.51,1.246)--(-9.535,1.241);
\filldraw[fill opacity=0.8,fill=gray!20,draw=none](-9.542,1.225)--(-9.529,1.255)--(-9.479,1.241)--(-9.487,1.222)--cycle;
\draw(-9.542,1.225)--(-9.529,1.255);
\draw(-9.479,1.241)--(-9.487,1.222);
\filldraw[fill opacity=0.8,fill=gray!20,draw=none](-9.552,1.227)--(-9.59,1.237)--(-9.585,1.249)--(-9.543,1.261)--(-9.529,1.255)--(-9.535,1.241)--cycle;
\draw(-9.59,1.237)--(-9.585,1.249);
\draw(-9.529,1.255)--(-9.535,1.241);
\filldraw[fill opacity=0.8,fill=gray!20,draw=none](-9.529,1.255)--(-9.524,1.266)--(-9.48,1.283)--(-9.465,1.275)--(-9.479,1.241)--cycle;
\draw(-9.529,1.255)--(-9.524,1.266);
\draw(-9.465,1.275)--(-9.479,1.241);
\filldraw[fill opacity=0.8,fill=gray!20,draw=none](-9.543,1.261)--(-9.524,1.266)--(-9.529,1.255)--cycle;
\draw(-9.524,1.266)--(-9.529,1.255);
\filldraw[fill opacity=0.8,fill=gray!20,draw=none](-9.481,1.294)--(-9.547,1.299)--(-9.498,1.305)--cycle;
\draw(-9.481,1.294)--(-9.547,1.299);
\filldraw[fill opacity=0.8,fill=gray!20,draw=none](-9.543,1.261)--(-9.575,1.272)--(-9.567,1.291)--(-9.508,1.303)--(-9.524,1.266)--cycle;
\draw(-9.575,1.272)--(-9.567,1.291);
\draw(-9.508,1.303)--(-9.524,1.266);
\filldraw[fill opacity=0.8,fill=gray!20,draw=none](-9.524,1.266)--(-9.51,1.3)--(-9.48,1.283)--cycle;
\draw(-9.524,1.266)--(-9.51,1.3);
\filldraw[fill opacity=0.8,fill=gray!20,draw=none](-9.585,1.249)--(-9.575,1.272)--(-9.543,1.261)--cycle;
\draw(-9.585,1.249)--(-9.575,1.272);
\filldraw[fill opacity=0.8,fill=gray!20](-8.767,3.093)--(-8.776,3.149)--(-8.669,3.154)--(-8.67,3.098)--cycle;
\filldraw[fill opacity=0.8,fill=gray!20](-8.776,3.149)--(-8.78,3.206)--(-8.668,3.211)--(-8.669,3.154)--cycle;
\filldraw[fill opacity=0.8,fill=gray!20,draw=none](-9.622,1.194)--(-9.656,1.193)--(-9.656,1.203)--cycle;
\draw(-9.622,1.194)--(-9.656,1.193)--(-9.656,1.203);
\filldraw[fill opacity=0.8,fill=gray!20](-8.84,3.079)--(-8.859,3.133)--(-8.776,3.149)--(-8.767,3.093)--cycle;
\filldraw[fill opacity=0.8,fill=gray!20](-8.859,3.133)--(-8.865,3.19)--(-8.78,3.206)--(-8.776,3.149)--cycle;
\filldraw[fill opacity=0.8,fill=gray!20](-8.811,3.03)--(-8.84,3.079)--(-8.767,3.093)--(-8.751,3.042)--cycle;
\filldraw[fill opacity=0.8,fill=gray!20](-8.761,3.152)--(-8.763,3.199)--(-8.671,3.203)--(-8.671,3.156)--cycle;
\filldraw[fill opacity=0.8,fill=gray!20,draw=none](-9.646,1.279)--(-9.643,1.294)--(-9.547,1.299)--cycle;
\draw(-9.646,1.279)--(-9.643,1.294)--(-9.547,1.299);
\filldraw[fill opacity=0.8,fill=gray!20](-9.643,1.294)--(-9.627,1.332)--(-9.549,1.335)--(-9.547,1.299)--cycle;
\filldraw[fill opacity=0.8,fill=gray!20,draw=none](-9.631,1.242)--(-9.615,1.279)--(-9.567,1.291)--(-9.585,1.249)--cycle;
\draw(-9.631,1.242)--(-9.615,1.279);
\draw(-9.567,1.291)--(-9.585,1.249);
\filldraw[fill opacity=0.8,fill=gray!20,draw=none](-9.585,1.263)--(-9.488,1.285)--(-9.475,1.28)--(-9.535,1.241)--(-9.564,1.234)--cycle;
\draw(-9.585,1.263)--(-9.488,1.285);
\draw(-9.535,1.241)--(-9.564,1.234);
\filldraw[fill opacity=0.8,fill=gray!20,draw=none](-9.487,1.222)--(-9.479,1.241)--(-9.434,1.231)--cycle;
\draw(-9.487,1.222)--(-9.479,1.241);
\filldraw[fill opacity=0.8,fill=gray!20,draw=none](-9.452,1.291)--(-9.451,1.293)--(-9.433,1.297)--(-9.449,1.277)--cycle;
\draw(-9.451,1.293)--(-9.433,1.297);
\filldraw[fill opacity=0.8,fill=gray!20,draw=none](-9.441,1.244)--(-9.449,1.276)--(-9.417,1.27)--(-9.372,1.237)--(-9.368,1.226)--cycle;
\draw(-9.372,1.237)--(-9.368,1.226)--(-9.441,1.244)--(-9.449,1.276);
\filldraw[fill opacity=0.8,fill=gray!20,draw=none](-9.479,1.241)--(-9.465,1.275)--(-9.426,1.249)--(-9.434,1.231)--cycle;
\draw(-9.479,1.241)--(-9.465,1.275);
\draw(-9.426,1.249)--(-9.434,1.231);
\filldraw[fill opacity=0.8,fill=gray!20,draw=none](-9.533,1.272)--(-8.491,.837)--(-8.492,.807)--(-9.563,1.254)--cycle;
\draw(-8.492,.807)--(-9.563,1.254)--(-9.533,1.272)--(-8.491,.837);
\filldraw[fill opacity=0.8,fill=gray!20](-8.486,.858)--(-8.466,.867)--(-8.442,.869)--(-8.44,.86)--cycle;
\filldraw[fill opacity=0.8,fill=gray!20](-8.44,.86)--(-8.442,.869)--(-8.419,.867)--(-8.395,.857)--cycle;
\filldraw[fill opacity=0.8,fill=gray!20,draw=none](-8.516,.59)--(-8.515,.583)--(-8.52,.586)--cycle;
\draw(-8.516,.59)--(-8.515,.583)--(-8.52,.586);
\filldraw[fill opacity=0.8,fill=gray!20,draw=none](-8.514,.582)--(-8.515,.583)--(-8.516,.59)--cycle;
\draw(-8.514,.582)--(-8.515,.583)--(-8.516,.59);
\filldraw[fill opacity=0.8,fill=gray!20,draw=none](-8.529,.809)--(-8.532,.824)--(-8.505,.813)--cycle;
\draw(-8.532,.824)--(-8.505,.813);
\filldraw[fill opacity=0.8,fill=gray!20,draw=none](-8.58,.794)--(-8.583,.799)--(-8.578,.797)--cycle;
\draw(-8.583,.799)--(-8.578,.797);
\filldraw[fill opacity=0.8,fill=gray!20,draw=none](-8.557,.834)--(-8.549,.831)--(-8.589,.801)--(-8.603,.807)--cycle;
\draw(-8.557,.834)--(-8.549,.831);
\draw(-8.589,.801)--(-8.603,.807);
\filldraw[fill opacity=0.8,fill=gray!20,draw=none](-8.578,.797)--(-8.583,.79)--(-8.603,.777)--(-8.574,.814)--(-8.558,.824)--cycle;
\draw(-8.583,.79)--(-8.603,.777)--(-8.574,.814)--(-8.558,.824);
\filldraw[fill opacity=0.8,fill=gray!20,draw=none](-8.335,.82)--(-8.346,.829)--(-8.329,.825)--(-8.322,.815)--cycle;
\draw(-8.346,.829)--(-8.329,.825)--(-8.322,.815);
\filldraw[fill opacity=0.8,fill=gray!20,draw=none](-8.609,.623)--(-8.598,.601)--(-8.597,.606)--cycle;
\filldraw[fill opacity=0.8,fill=gray!20,draw=none](-9.627,1.029)--(-9.643,1.08)--(-9.633,1.08)--(-9.549,1.034)--(-9.549,1.032)--cycle;
\draw(-9.549,1.034)--(-9.549,1.032)--(-9.627,1.029)--(-9.643,1.08)--(-9.633,1.08);
\filldraw[fill opacity=0.8,fill=gray!20,draw=none](-9.56,.988)--(-9.616,1.005)--(-9.627,1.029)--(-9.549,1.032)--(-9.551,.988)--cycle;
\draw(-9.616,1.005)--(-9.627,1.029)--(-9.549,1.032)--(-9.551,.988)--(-9.56,.988);
\filldraw[fill opacity=0.8,fill=gray!20,draw=none](-9.551,.988)--(-9.549,1.032)--(-9.546,1.032)--(-9.492,.993)--(-9.497,.984)--cycle;
\draw(-9.492,.993)--(-9.497,.984)--(-9.551,.988)--(-9.549,1.032)--(-9.546,1.032);
\filldraw[fill opacity=0.8,fill=gray!20,draw=none](-9.552,.986)--(-9.551,.988)--(-9.497,.984)--(-9.506,.976)--cycle;
\draw(-9.552,.986)--(-9.551,.988)--(-9.497,.984)--(-9.506,.976);
\filldraw[fill opacity=0.8,fill=gray!20,draw=none](-9.488,.989)--(-9.488,.982)--(-9.497,.984)--(-9.492,.993)--cycle;
\draw(-9.488,.982)--(-9.497,.984)--(-9.492,.993);
\filldraw[fill opacity=0.8,fill=gray!20,draw=none](-9.506,.976)--(-9.497,.984)--(-9.478,.98)--(-9.469,.97)--cycle;
\draw(-9.506,.976)--(-9.497,.984)--(-9.478,.98);
\filldraw[fill opacity=0.8,fill=gray!20,draw=none](-9.488,.989)--(-9.478,.98)--(-9.488,.982)--cycle;
\draw(-9.478,.98)--(-9.488,.982);
\filldraw[fill opacity=0.8,fill=gray!20](-9.628,1.048)--(-8.515,.583)--(-8.496,.564)--(-9.609,1.029)--cycle;
\filldraw[fill opacity=0.8,fill=gray!20,draw=none](-8.335,.82)--(-8.322,.815)--(-8.314,.805)--cycle;
\draw(-8.322,.815)--(-8.314,.805);
\filldraw[fill opacity=0.8,fill=gray!20,draw=none](-8.625,.66)--(-8.622,.659)--(-8.624,.657)--cycle;
\draw(-8.625,.66)--(-8.622,.659);
\filldraw[fill opacity=0.8,fill=gray!20,draw=none](-8.622,.659)--(-8.624,.657)--(-8.628,.688)--(-8.616,.695)--cycle;
\draw(-8.624,.657)--(-8.628,.688)--(-8.616,.695);
\filldraw[fill opacity=0.5,fill=gray!20](-7.856,.814)--(-8.035,.846)--(-8.182,.442)--(-8.022,.358)--cycle;
\filldraw[fill opacity=0.5,fill=gray!20](-7.731,.778)--(-7.856,.814)--(-8.022,.358)--(-7.904,.301)--cycle;
\filldraw[fill opacity=0.5,fill=gray!20](-7.618,.743)--(-7.731,.778)--(-7.904,.301)--(-7.798,.25)--cycle;
\filldraw[fill opacity=0.5,fill=gray!20](-7.522,.71)--(-7.618,.743)--(-7.798,.25)--(-7.706,.208)--cycle;
\filldraw[fill opacity=0.5,fill=gray!20](-7.447,.68)--(-7.522,.71)--(-7.706,.208)--(-7.632,.174)--cycle;
\filldraw[fill opacity=0.5,fill=gray!20](-7.396,.655)--(-7.447,.68)--(-7.632,.174)--(-7.579,.152)--cycle;
\filldraw[fill opacity=0.5,fill=gray!20](-7.369,.635)--(-7.396,.655)--(-7.579,.152)--(-7.549,.142)--cycle;
\filldraw[fill opacity=0.5,fill=gray!20](-7.395,.612)--(-7.856,.814)--(-8.022,.358)--(-7.561,.157)--cycle;
\filldraw[fill opacity=0.5,fill=gray!20](-8.022,.358)--(-8.182,.442)--(-8.417,.082)--(-8.287,-.046)--cycle;
\filldraw[fill opacity=0.5,fill=gray!20](-7.904,.301)--(-8.022,.358)--(-8.286,-.046)--(-8.181,-.122)--cycle;
\filldraw[fill opacity=0.5,fill=gray!20](-7.798,.25)--(-7.904,.301)--(-8.181,-.122)--(-8.083,-.187)--cycle;
\filldraw[fill opacity=0.5,fill=gray!20](-7.706,.208)--(-7.798,.25)--(-8.083,-.187)--(-7.997,-.239)--cycle;
\filldraw[fill opacity=0.5,fill=gray!20](-7.632,.174)--(-7.706,.208)--(-7.997,-.239)--(-7.925,-.275)--cycle;
\filldraw[fill opacity=0.5,fill=gray!20](-7.579,.152)--(-7.632,.174)--(-7.925,-.275)--(-7.87,-.294)--cycle;
\filldraw[fill opacity=0.5,fill=gray!20](-7.549,.142)--(-7.579,.152)--(-7.87,-.294)--(-7.835,-.296)--cycle;
\filldraw[fill opacity=0.5,fill=gray!20](-7.561,.157)--(-8.022,.358)--(-8.287,-.046)--(-7.825,-.247)--cycle;
\filldraw[fill opacity=0.8,fill=gray!20,draw=none](-8.492,.814)--(-8.491,.834)--(-8.48,.833)--(-8.449,.82)--cycle;
\draw(-8.48,.833)--(-8.449,.82);
\filldraw[fill opacity=0.8,fill=gray!20](-8.487,.509)--(-8.527,.522)--(-8.536,.532)--(-8.492,.514)--cycle;
\filldraw[fill opacity=0.8,fill=gray!20](-8.445,.507)--(-8.487,.509)--(-8.492,.514)--(-8.445,.507)--cycle;
\filldraw[fill opacity=0.8,fill=gray!20,draw=none](-8.288,.615)--(-8.278,.643)--(-8.277,.646)--(-8.269,.637)--(-8.272,.629)--cycle;
\draw(-8.277,.646)--(-8.269,.637)--(-8.272,.629);
\filldraw[fill opacity=0.8,fill=gray!20,draw=none](-8.3,.787)--(-8.314,.805)--(-8.322,.815)--(-8.318,.81)--cycle;
\draw(-8.314,.805)--(-8.322,.815);
\filldraw[fill opacity=0.8,fill=gray!20,draw=none](-9.502,1.098)--(-9.515,1.082)--(-9.547,1.084)--(-9.545,1.141)--(-9.512,1.138)--cycle;
\draw(-9.515,1.082)--(-9.547,1.084)--(-9.545,1.141)--(-9.512,1.138);
\filldraw[fill opacity=0.8,fill=gray!20,draw=none](-9.502,1.098)--(-9.512,1.138)--(-9.47,1.135)--cycle;
\draw(-9.512,1.138)--(-9.47,1.135);
\filldraw[fill opacity=0.8,fill=gray!20,draw=none](-9.46,1.162)--(-9.459,1.16)--(-9.47,1.135)--(-9.498,1.137)--cycle;
\draw(-9.47,1.135)--(-9.498,1.137);
\filldraw[fill opacity=0.8,fill=gray!20,draw=none](-9.458,1.162)--(-9.459,1.163)--(-9.449,1.161)--cycle;
\filldraw[fill opacity=0.8,fill=gray!20,draw=none](-9.46,1.162)--(-9.459,1.163)--(-9.458,1.161)--(-9.459,1.16)--cycle;
\filldraw[fill opacity=0.8,fill=gray!20,draw=none](-9.459,1.163)--(-9.457,1.164)--(-9.458,1.161)--cycle;
\filldraw[fill opacity=0.8,fill=gray!20,draw=none](-9.458,1.162)--(-9.617,1.176)--(-9.557,1.189)--(-9.459,1.163)--cycle;
\draw(-9.617,1.176)--(-9.557,1.189);
\filldraw[fill opacity=0.8,fill=gray!20,draw=none](-9.459,1.163)--(-9.461,1.187)--(-9.444,1.181)--(-9.449,1.161)--cycle;
\filldraw[fill opacity=0.8,fill=gray!20,draw=none](-9.457,1.164)--(-9.459,1.163)--(-9.462,1.192)--(-9.445,1.191)--cycle;
\draw(-9.462,1.192)--(-9.445,1.191);
\filldraw[fill opacity=0.8,fill=gray!20,draw=none](-9.459,1.163)--(-9.557,1.189)--(-9.503,1.201)--(-9.461,1.187)--cycle;
\draw(-9.557,1.189)--(-9.503,1.201);
\filldraw[fill opacity=0.8,fill=gray!20,draw=none](-9.535,1.241)--(-9.58,1.184)--(-9.617,1.176)--cycle;
\draw(-9.58,1.184)--(-9.617,1.176);
\filldraw[fill opacity=0.8,fill=gray!20,draw=none](-9.617,1.176)--(-9.603,1.208)--(-9.552,1.203)--cycle;
\draw(-9.617,1.176)--(-9.603,1.208);
\filldraw[fill opacity=0.8,fill=gray!20,draw=none](-9.603,1.208)--(-9.59,1.237)--(-9.542,1.225)--(-9.552,1.203)--cycle;
\draw(-9.603,1.208)--(-9.59,1.237);
\draw(-9.542,1.225)--(-9.552,1.203);
\filldraw[fill opacity=0.8,fill=gray!20,draw=none](-9.653,1.191)--(-9.631,1.242)--(-9.585,1.249)--(-9.617,1.176)--cycle;
\draw(-9.585,1.249)--(-9.617,1.176)--(-9.653,1.191)--(-9.631,1.242);
\filldraw[fill opacity=0.8,fill=gray!20,draw=none](-9.617,1.222)--(-9.535,1.241)--(-9.617,1.176)--cycle;
\draw(-9.617,1.176)--(-9.617,1.222)--(-9.535,1.241);
\filldraw[fill opacity=0.8,fill=gray!20,draw=none](-9.593,1.22)--(-8.598,.805)--(-8.628,.763)--(-9.617,1.176)--cycle;
\draw(-8.628,.763)--(-9.617,1.176)--(-9.593,1.22)--(-8.598,.805);
\filldraw[fill opacity=0.8,fill=gray!20](-8.395,.857)--(-8.419,.867)--(-8.403,.863)--(-8.363,.849)--cycle;
\filldraw[fill opacity=0.8,fill=gray!20](-8.527,.522)--(-8.561,.547)--(-8.574,.561)--(-8.536,.532)--cycle;
\filldraw[fill opacity=0.8,fill=gray!20](-8.445,.507)--(-8.406,.508)--(-8.424,.504)--(-8.445,.507)--cycle;
\filldraw[fill opacity=0.8,fill=gray!20](-8.445,.507)--(-8.398,.513)--(-8.406,.508)--(-8.445,.507)--cycle;
\filldraw[fill opacity=0.8,fill=gray!20](-8.445,.507)--(-8.471,.505)--(-8.487,.509)--(-8.445,.507)--cycle;
\filldraw[fill opacity=0.8,fill=gray!20](-8.445,.507)--(-8.424,.504)--(-8.448,.503)--(-8.445,.507)--cycle;
\filldraw[fill opacity=0.8,fill=gray!20](-8.445,.507)--(-8.448,.503)--(-8.471,.505)--(-8.445,.507)--cycle;
\filldraw[fill opacity=0.8,fill=gray!20,draw=none](-8.278,.643)--(-8.277,.646)--(-8.277,.646)--cycle;
\draw(-8.277,.646)--(-8.277,.646);
\filldraw[fill opacity=0.8,fill=gray!20,draw=none](-8.277,.646)--(-8.277,.646)--(-8.27,.691)--(-8.263,.684)--cycle;
\draw(-8.277,.646)--(-8.277,.646);
\draw(-8.27,.691)--(-8.263,.684);
\filldraw[fill opacity=0.8,fill=gray!20,draw=none](-8.277,.646)--(-8.263,.684)--(-8.262,.683)--(-8.269,.637)--cycle;
\draw(-8.263,.684)--(-8.262,.683)--(-8.269,.637)--(-8.277,.646);
\filldraw[fill opacity=0.8,fill=gray!20](-8.398,.513)--(-8.354,.53)--(-8.369,.52)--(-8.406,.508)--cycle;
\filldraw[fill opacity=0.8,fill=gray!20](-8.561,.547)--(-8.587,.581)--(-8.603,.599)--(-8.574,.561)--cycle;
\filldraw[fill opacity=0.8,fill=gray!20,draw=none](-8.641,.714)--(-8.636,.712)--(-8.625,.66)--(-8.63,.663)--cycle;
\draw(-8.641,.714)--(-8.636,.712);
\draw(-8.625,.66)--(-8.63,.663);
\filldraw[fill opacity=0.8,fill=gray!20,draw=none](-8.322,.815)--(-8.329,.825)--(-8.316,.81)--(-8.298,.788)--cycle;
\draw(-8.322,.815)--(-8.329,.825)--(-8.316,.81)--(-8.298,.788);
\filldraw[fill opacity=0.8,fill=gray!20,draw=none](-9.643,1.08)--(-9.653,1.136)--(-9.591,1.139)--(-9.546,1.102)--(-9.547,1.084)--cycle;
\draw(-9.546,1.102)--(-9.547,1.084)--(-9.643,1.08)--(-9.653,1.136)--(-9.591,1.139);
\filldraw[fill opacity=0.8,fill=gray!20,draw=none](-9.633,1.08)--(-9.547,1.084)--(-9.549,1.034)--cycle;
\draw(-9.633,1.08)--(-9.547,1.084)--(-9.549,1.034);
\filldraw[fill opacity=0.8,fill=gray!20,draw=none](-9.508,1.043)--(-9.548,1.037)--(-9.547,1.084)--(-9.532,1.083)--cycle;
\draw(-9.548,1.037)--(-9.547,1.084)--(-9.532,1.083);
\filldraw[fill opacity=0.8,fill=gray!20,draw=none](-9.497,1.071)--(-9.51,1.047)--(-9.532,1.083)--(-9.497,1.081)--cycle;
\draw(-9.532,1.083)--(-9.497,1.081);
\filldraw[fill opacity=0.8,fill=gray!20,draw=none](-9.497,1.071)--(-9.497,1.045)--(-9.508,1.043)--(-9.51,1.047)--cycle;
\filldraw[fill opacity=0.8,fill=gray!20,draw=none](-9.508,1.043)--(-9.499,1.029)--(-9.549,1.032)--(-9.548,1.037)--cycle;
\draw(-9.499,1.029)--(-9.549,1.032)--(-9.548,1.037);
\filldraw[fill opacity=0.8,fill=gray!20,draw=none](-9.498,1.025)--(-9.506,1.003)--(-9.546,1.032)--(-9.499,1.029)--cycle;
\draw(-9.546,1.032)--(-9.499,1.029);
\filldraw[fill opacity=0.8,fill=gray!20,draw=none](-9.497,1.045)--(-9.497,1.029)--(-9.499,1.029)--(-9.508,1.043)--cycle;
\draw(-9.497,1.029)--(-9.499,1.029);
\filldraw[fill opacity=0.8,fill=gray!20,draw=none](-9.498,1.025)--(-9.499,1.029)--(-9.497,1.029)--cycle;
\draw(-9.499,1.029)--(-9.497,1.029);
\filldraw[fill opacity=0.8,fill=gray!20,draw=none](-9.632,1.127)--(-8.641,.714)--(-8.63,.663)--(-9.636,1.082)--cycle;
\draw(-8.63,.663)--(-9.636,1.082)--(-9.632,1.127)--(-8.641,.714);
\filldraw[fill opacity=0.8,fill=gray!20,draw=none](-8.556,.825)--(-8.574,.814)--(-8.536,.842)--(-8.53,.846)--cycle;
\draw(-8.556,.825)--(-8.574,.814)--(-8.536,.842)--(-8.53,.846);
\filldraw[fill opacity=0.8,fill=gray!20,draw=none](-9.412,1.157)--(-9.412,1.163)--(-9.39,1.165)--(-9.3,1.144)--cycle;
\filldraw[fill opacity=0.8,fill=gray!20,draw=none](-9.412,1.163)--(-9.411,1.169)--(-9.39,1.165)--cycle;
\filldraw[fill opacity=0.8,fill=gray!20,draw=none](-9.444,1.181)--(-9.41,1.17)--(-9.449,1.161)--cycle;
\draw(-9.41,1.17)--(-9.449,1.161);
\filldraw[fill opacity=0.8,fill=gray!20,draw=none](-9.45,1.161)--(-9.411,1.169)--(-9.412,1.163)--cycle;
\draw(-9.45,1.161)--(-9.411,1.169);
\filldraw[fill opacity=0.8,fill=gray!20,draw=none](-9.45,1.161)--(-9.412,1.163)--(-9.412,1.157)--cycle;
\filldraw[fill opacity=0.8,fill=gray!20,draw=none](-9.457,1.164)--(-9.445,1.191)--(-9.437,1.19)--(-9.438,1.176)--cycle;
\draw(-9.445,1.191)--(-9.437,1.19)--(-9.438,1.176);
\filldraw[fill opacity=0.8,fill=gray!20,draw=none](-9.438,1.176)--(-9.437,1.19)--(-9.43,1.188)--cycle;
\draw(-9.438,1.176)--(-9.437,1.19)--(-9.43,1.188);
\filldraw[fill opacity=0.8,fill=gray!20,draw=none](-9.43,1.188)--(-9.437,1.19)--(-9.426,1.198)--cycle;
\draw(-9.43,1.188)--(-9.437,1.19);
\filldraw[fill opacity=0.8,fill=gray!20,draw=none](-9.552,1.203)--(-9.532,1.224)--(-9.487,1.222)--cycle;
\filldraw[fill opacity=0.8,fill=gray!20,draw=none](-9.552,1.203)--(-9.542,1.225)--(-9.532,1.224)--cycle;
\draw(-9.552,1.203)--(-9.542,1.225);
\filldraw[fill opacity=0.8,fill=gray!20,draw=none](-9.552,1.227)--(-9.535,1.241)--(-9.542,1.225)--cycle;
\draw(-9.535,1.241)--(-9.542,1.225);
\filldraw[fill opacity=0.8,fill=gray!20,draw=none](-9.608,1.258)--(-9.585,1.263)--(-9.564,1.234)--(-9.617,1.222)--cycle;
\draw(-9.564,1.234)--(-9.617,1.222)--(-9.608,1.258)--(-9.585,1.263);
\filldraw[fill opacity=0.8,fill=gray!20,draw=none](-9.563,1.254)--(-8.557,.834)--(-8.603,.807)--(-9.593,1.22)--cycle;
\draw(-8.603,.807)--(-9.593,1.22)--(-9.563,1.254)--(-8.557,.834);
\filldraw[fill opacity=0.8,fill=gray!20](-8.354,.53)--(-8.316,.558)--(-8.337,.544)--(-8.369,.52)--cycle;
\filldraw[fill opacity=0.8,fill=gray!20,draw=none](-8.329,.825)--(-8.345,.836)--(-8.342,.83)--(-8.316,.81)--cycle;
\draw(-8.342,.83)--(-8.316,.81)--(-8.329,.825)--(-8.345,.836);
\filldraw[fill opacity=0.8,fill=gray!20,draw=none](-8.278,.741)--(-8.299,.786)--(-8.298,.785)--(-8.288,.767)--cycle;
\draw(-8.299,.786)--(-8.298,.785);
\filldraw[fill opacity=0.8,fill=gray!20,draw=none](-8.3,.787)--(-8.298,.785)--(-8.299,.786)--cycle;
\draw(-8.298,.785)--(-8.299,.786);
\filldraw[fill opacity=0.8,fill=gray!20,draw=none](-8.298,.785)--(-8.3,.787)--(-8.318,.81)--(-8.298,.788)--(-8.287,.773)--cycle;
\draw(-8.298,.788)--(-8.287,.773)--(-8.298,.785);
\filldraw[fill opacity=0.8,fill=gray!20,draw=none](-8.43,.819)--(-8.449,.82)--(-8.48,.833)--cycle;
\draw(-8.449,.82)--(-8.48,.833);
\filldraw[fill opacity=0.5,fill=gray!20](-7.8,1.289)--(-7.984,1.269)--(-8.035,.846)--(-7.856,.814)--cycle;
\filldraw[fill opacity=0.5,fill=gray!20](-7.671,1.276)--(-7.8,1.289)--(-7.856,.814)--(-7.731,.778)--cycle;
\filldraw[fill opacity=0.5,fill=gray!20](-7.557,1.258)--(-7.671,1.276)--(-7.731,.778)--(-7.618,.743)--cycle;
\filldraw[fill opacity=0.5,fill=gray!20](-7.46,1.235)--(-7.557,1.258)--(-7.618,.743)--(-7.522,.71)--cycle;
\filldraw[fill opacity=0.5,fill=gray!20](-7.384,1.209)--(-7.46,1.235)--(-7.522,.71)--(-7.447,.68)--cycle;
\filldraw[fill opacity=0.5,fill=gray!20](-7.333,1.18)--(-7.384,1.209)--(-7.447,.68)--(-7.396,.655)--cycle;
\filldraw[fill opacity=0.5,fill=gray!20](-7.308,1.149)--(-7.333,1.18)--(-7.396,.655)--(-7.369,.635)--cycle;
\filldraw[fill opacity=0.5,fill=gray!20](-7.338,1.088)--(-7.8,1.289)--(-7.856,.814)--(-7.395,.612)--cycle;
\filldraw[fill opacity=0.8,fill=gray!20,draw=none](-8.288,.767)--(-8.298,.785)--(-8.293,.78)--cycle;
\draw(-8.298,.785)--(-8.293,.78);
\filldraw[fill opacity=0.8,fill=gray!20,draw=none](-8.271,.715)--(-8.27,.691)--(-8.277,.739)--(-8.277,.738)--cycle;
\draw(-8.277,.739)--(-8.277,.738);
\filldraw[fill opacity=0.8,fill=gray!20,draw=none](-8.27,.691)--(-8.272,.733)--(-8.269,.73)--(-8.262,.683)--cycle;
\draw(-8.272,.733)--(-8.269,.73)--(-8.262,.683)--(-8.27,.691);
\filldraw[fill opacity=0.8,fill=gray!20](-8.316,.558)--(-8.287,.594)--(-8.313,.577)--(-8.337,.544)--cycle;
\filldraw[fill opacity=0.8,fill=gray!20,draw=none](-9.506,1.003)--(-9.498,1.025)--(-9.489,.999)--(-9.492,.993)--cycle;
\draw(-9.489,.999)--(-9.492,.993);
\filldraw[fill opacity=0.8,fill=gray!20,draw=none](-9.488,.989)--(-9.492,.993)--(-9.489,.999)--cycle;
\draw(-9.492,.993)--(-9.489,.999);
\filldraw[fill opacity=0.8,fill=gray!20,draw=none](-9.636,1.082)--(-8.625,.66)--(-8.624,.657)--(-8.639,.635)--(-9.628,1.048)--cycle;
\draw(-8.639,.635)--(-9.628,1.048)--(-9.636,1.082)--(-8.625,.66);
\filldraw[fill opacity=0.8,fill=gray!20,draw=none](-8.278,.741)--(-8.277,.738)--(-8.277,.739)--cycle;
\draw(-8.277,.738)--(-8.277,.739);
\filldraw[fill opacity=0.8,fill=gray!20](-8.471,.505)--(-8.495,.514)--(-8.527,.522)--(-8.487,.509)--cycle;
\filldraw[fill opacity=0.8,fill=gray!20,draw=none](-8.277,.738)--(-8.278,.741)--(-8.288,.767)--(-8.272,.738)--(-8.269,.73)--cycle;
\draw(-8.272,.738)--(-8.269,.73)--(-8.277,.738);
\filldraw[fill opacity=0.8,fill=gray!20,draw=none](-8.345,.836)--(-8.363,.849)--(-8.354,.839)--(-8.342,.83)--cycle;
\draw(-8.345,.836)--(-8.363,.849)--(-8.354,.839)--(-8.342,.83);
\filldraw[fill opacity=0.8,fill=gray!20,draw=none](-8.617,.637)--(-8.603,.622)--(-8.609,.668)--(-8.628,.688)--(-8.624,.657)--cycle;
\draw(-8.617,.637)--(-8.603,.622)--(-8.609,.668)--(-8.628,.688)--(-8.624,.657);
\filldraw[fill opacity=0.8,fill=gray!20,draw=none](-8.271,.715)--(-8.277,.738)--(-8.272,.733)--cycle;
\draw(-8.277,.738)--(-8.272,.733);
\filldraw[fill opacity=0.8,fill=gray!20](-8.406,.508)--(-8.369,.52)--(-8.404,.513)--(-8.424,.504)--cycle;
\filldraw[fill opacity=0.8,fill=gray!20,draw=none](-8.6,.595)--(-8.588,.583)--(-8.598,.601)--cycle;
\draw(-8.6,.595)--(-8.588,.583);
\filldraw[fill opacity=0.8,fill=gray!20](-8.609,.668)--(-8.603,.715)--(-8.621,.734)--(-8.628,.688)--cycle;
\filldraw[fill opacity=0.8,fill=gray!20,draw=none](-8.287,.594)--(-8.272,.629)--(-8.28,.63)--(-8.298,.618)--(-8.313,.577)--cycle;
\draw(-8.28,.63)--(-8.298,.618)--(-8.313,.577)--(-8.287,.594)--(-8.272,.629);
\filldraw[fill opacity=0.8,fill=gray!20](-8.442,.869)--(-8.445,.864)--(-8.445,.864)--(-8.419,.867)--cycle;
\filldraw[fill opacity=0.8,fill=gray!20](-8.466,.867)--(-8.445,.864)--(-8.445,.864)--(-8.442,.869)--cycle;
\filldraw[fill opacity=0.8,fill=gray!20,draw=none](-8.52,.849)--(-8.514,.85)--(-8.497,.857)--cycle;
\draw(-8.514,.85)--(-8.497,.857);
\filldraw[fill opacity=0.8,fill=gray!20,draw=none](-8.495,.849)--(-8.498,.857)--(-8.514,.85)--cycle;
\draw(-8.498,.857)--(-8.514,.85);
\filldraw[fill opacity=0.8,fill=gray!20,draw=none](-8.509,.855)--(-8.423,.819)--(-8.43,.819)--(-8.48,.833)--(-8.491,.837)--cycle;
\draw(-8.509,.855)--(-8.423,.819);
\draw(-8.48,.833)--(-8.491,.837);
\filldraw[fill opacity=0.8,fill=gray!20,draw=none](-8.491,.834)--(-8.491,.837)--(-8.48,.833)--cycle;
\draw(-8.491,.837)--(-8.48,.833);
\filldraw[fill opacity=0.8,fill=gray!20,draw=none](-8.288,.767)--(-8.293,.78)--(-8.287,.773)--(-8.272,.738)--cycle;
\draw(-8.293,.78)--(-8.287,.773)--(-8.272,.738);
\filldraw[fill opacity=0.8,fill=gray!20](-8.363,.849)--(-8.403,.863)--(-8.398,.858)--(-8.354,.839)--cycle;
\filldraw[fill opacity=0.8,fill=gray!20](-8.485,.864)--(-8.445,.864)--(-8.445,.864)--(-8.466,.867)--cycle;
\filldraw[fill opacity=0.8,fill=gray!20,draw=none](-8.272,.629)--(-8.269,.637)--(-8.28,.63)--cycle;
\draw(-8.272,.629)--(-8.269,.637)--(-8.28,.63);
\filldraw[fill opacity=0.8,fill=gray!20,draw=none](-8.609,.623)--(-8.597,.606)--(-8.603,.622)--(-8.617,.637)--cycle;
\draw(-8.597,.606)--(-8.603,.622)--(-8.617,.637);
\filldraw[fill opacity=0.8,fill=gray!20,draw=none](-8.564,.603)--(-8.558,.611)--(-8.603,.622)--(-8.597,.606)--cycle;
\draw(-8.558,.611)--(-8.603,.622)--(-8.597,.606);
\filldraw[fill opacity=0.8,fill=gray!20,draw=none](-9.478,.98)--(-9.488,.989)--(-9.489,.999)--(-9.472,1.027)--(-9.419,1.014)--(-9.46,.975)--cycle;
\draw(-9.489,.999)--(-9.472,1.027)--(-9.419,1.014)--(-9.46,.975)--(-9.478,.98);
\filldraw[fill opacity=0.8,fill=gray!20,draw=none](-9.478,.98)--(-9.46,.975)--(-9.469,.97)--cycle;
\draw(-9.478,.98)--(-9.46,.975)--(-9.469,.97);
\filldraw[fill opacity=0.8,fill=gray!20,draw=none](-9.443,.97)--(-9.454,.969)--(-9.46,.975)--(-9.426,1.007)--(-9.406,.994)--(-9.433,.974)--cycle;
\draw(-9.454,.969)--(-9.46,.975)--(-9.426,1.007);
\draw(-9.406,.994)--(-9.433,.974);
\filldraw[fill opacity=0.8,fill=gray!20,draw=none](-9.469,.97)--(-9.46,.975)--(-9.454,.969)--cycle;
\draw(-9.469,.97)--(-9.46,.975)--(-9.454,.969);
\filldraw[fill opacity=0.8,fill=gray!20,draw=none](-9.145,1.127)--(-9.189,1.128)--(-9.155,1.136)--cycle;
\draw(-9.189,1.128)--(-9.155,1.136);
\filldraw[fill opacity=0.8,fill=gray!20,draw=none](-9.189,1.128)--(-9.145,1.127)--(-9.134,1.116)--cycle;
\filldraw[fill opacity=0.8,fill=gray!20,draw=none](-9.134,1.116)--(-8.509,.855)--(-8.491,.837)--(-9.158,1.116)--cycle;
\draw(-9.134,1.116)--(-8.509,.855);
\draw(-8.491,.837)--(-9.158,1.116);
\filldraw[fill opacity=0.8,fill=gray!20,draw=none](-9.134,1.116)--(-9.158,1.116)--(-9.189,1.128)--cycle;
\draw(-9.158,1.116)--(-9.189,1.128);
\filldraw[fill opacity=0.8,fill=gray!20,draw=none](-9.492,1.08)--(-9.483,1.12)--(-9.472,1.115)--cycle;
\draw(-9.483,1.12)--(-9.472,1.115);
\filldraw[fill opacity=0.8,fill=gray!20,draw=none](-9.591,1.139)--(-9.545,1.141)--(-9.546,1.102)--cycle;
\draw(-9.591,1.139)--(-9.545,1.141)--(-9.546,1.102);
\filldraw[fill opacity=0.8,fill=gray!20,draw=none](-9.515,1.082)--(-9.47,1.135)--(-9.441,1.133)--(-9.453,1.077)--cycle;
\draw(-9.47,1.135)--(-9.441,1.133)--(-9.453,1.077)--(-9.515,1.082);
\filldraw[fill opacity=0.8,fill=gray!20,draw=none](-9.492,1.08)--(-9.494,1.07)--(-9.497,1.071)--cycle;
\draw(-9.494,1.07)--(-9.497,1.071);
\filldraw[fill opacity=0.8,fill=gray!20,draw=none](-9.497,1.045)--(-9.497,1.081)--(-9.453,1.077)--(-9.463,1.051)--cycle;
\draw(-9.497,1.081)--(-9.453,1.077)--(-9.463,1.051);
\filldraw[fill opacity=0.8,fill=gray!20,draw=none](-9.617,1.176)--(-9.483,1.12)--(-9.492,1.08)--(-9.497,1.071)--(-9.632,1.127)--cycle;
\draw(-9.497,1.071)--(-9.632,1.127)--(-9.617,1.176)--(-9.483,1.12);
\filldraw[fill opacity=0.8,fill=gray!20,draw=none](-9.44,1.135)--(-9.442,1.16)--(-9.412,1.157)--(-9.415,1.118)--(-9.426,1.116)--cycle;
\draw(-9.415,1.118)--(-9.426,1.116);
\filldraw[fill opacity=0.8,fill=gray!20](-9.735,1.12)--(-9.741,1.176)--(-9.656,1.193)--(-9.653,1.136)--cycle;
\filldraw[fill opacity=0.8,fill=gray!20](-9.717,1.066)--(-9.735,1.12)--(-9.653,1.136)--(-9.643,1.08)--cycle;
\filldraw[fill opacity=0.8,fill=gray!20,draw=none](-9.44,1.135)--(-9.441,1.135)--(-9.449,1.161)--(-9.442,1.16)--cycle;
\filldraw[fill opacity=0.8,fill=gray!20,draw=none](-9.441,1.135)--(-9.458,1.159)--(-9.45,1.161)--(-9.449,1.161)--cycle;
\draw(-9.458,1.159)--(-9.45,1.161);
\filldraw[fill opacity=0.8,fill=gray!20,draw=none](-9.456,1.134)--(-9.47,1.135)--(-9.458,1.161)--cycle;
\draw(-9.456,1.134)--(-9.47,1.135);
\filldraw[fill opacity=0.8,fill=gray!20,draw=none](-9.617,1.176)--(-9.449,1.161)--(-9.608,1.125)--cycle;
\draw(-9.449,1.161)--(-9.608,1.125)--(-9.617,1.176);
\filldraw[fill opacity=0.8,fill=gray!20](-9.741,1.176)--(-9.735,1.231)--(-9.653,1.247)--(-9.656,1.193)--cycle;
\filldraw[fill opacity=0.8,fill=gray!20](-9.688,1.017)--(-9.717,1.066)--(-9.643,1.08)--(-9.627,1.029)--cycle;
\filldraw[fill opacity=0.8,fill=gray!20,draw=none](-9.456,1.134)--(-9.458,1.161)--(-9.457,1.164)--(-9.438,1.176)--(-9.441,1.133)--cycle;
\draw(-9.438,1.176)--(-9.441,1.133)--(-9.456,1.134);
\filldraw[fill opacity=0.8,fill=gray!20,draw=none](-9.432,1.114)--(-9.3,1.144)--(-9.189,1.128)--(-9.388,1.084)--cycle;
\draw(-9.432,1.114)--(-9.3,1.144);
\draw(-9.189,1.128)--(-9.388,1.084);
\filldraw[fill opacity=0.8,fill=gray!20,draw=none](-9.383,1.085)--(-9.189,1.128)--(-9.134,1.116)--(-9.321,1.074)--cycle;
\draw(-9.383,1.085)--(-9.189,1.128);
\draw(-9.134,1.116)--(-9.321,1.074);
\filldraw[fill opacity=0.8,fill=gray!20,draw=none](-9.412,1.157)--(-9.3,1.144)--(-9.415,1.118)--cycle;
\draw(-9.3,1.144)--(-9.415,1.118);
\filldraw[fill opacity=0.8,fill=gray!20,draw=none](-9.439,1.113)--(-9.444,1.122)--(-9.441,1.133)--(-9.44,1.133)--cycle;
\draw(-9.444,1.122)--(-9.441,1.133)--(-9.44,1.133);
\filldraw[fill opacity=0.8,fill=gray!20,draw=none](-9.439,1.113)--(-9.436,1.073)--(-9.453,1.077)--(-9.444,1.122)--cycle;
\draw(-9.436,1.073)--(-9.453,1.077)--(-9.444,1.122);
\filldraw[fill opacity=0.8,fill=gray!20,draw=none](-9.608,1.125)--(-9.458,1.159)--(-9.44,1.135)--(-9.439,1.113)--(-9.591,1.079)--cycle;
\draw(-9.439,1.113)--(-9.591,1.079)--(-9.608,1.125)--(-9.458,1.159);
\filldraw[fill opacity=0.8,fill=gray!20,draw=none](-9.44,1.135)--(-9.426,1.116)--(-9.434,1.114)--cycle;
\draw(-9.426,1.116)--(-9.434,1.114);
\filldraw[fill opacity=0.8,fill=gray!20,draw=none](-9.441,1.133)--(-9.438,1.176)--(-9.43,1.188)--(-9.362,1.171)--(-9.368,1.115)--cycle;
\draw(-9.43,1.188)--(-9.362,1.171)--(-9.368,1.115)--(-9.441,1.133)--(-9.438,1.176);
\filldraw[fill opacity=0.8,fill=gray!20,draw=none](-9.44,1.135)--(-9.434,1.114)--(-9.439,1.113)--cycle;
\draw(-9.434,1.114)--(-9.439,1.113);
\filldraw[fill opacity=0.8,fill=gray!20,draw=none](-9.439,1.113)--(-9.432,1.114)--(-9.388,1.084)--(-9.42,1.076)--cycle;
\draw(-9.439,1.113)--(-9.432,1.114);
\draw(-9.388,1.084)--(-9.42,1.076);
\filldraw[fill opacity=0.8,fill=gray!20,draw=none](-9.392,1.083)--(-9.383,1.085)--(-9.321,1.074)--(-9.39,1.058)--cycle;
\draw(-9.392,1.083)--(-9.383,1.085);
\draw(-9.321,1.074)--(-9.39,1.058);
\filldraw[fill opacity=0.8,fill=gray!20,draw=none](-9.418,1.077)--(-9.392,1.083)--(-9.39,1.058)--cycle;
\draw(-9.418,1.077)--(-9.392,1.083);
\filldraw[fill opacity=0.8,fill=gray!20,draw=none](-9.463,1.051)--(-9.453,1.077)--(-9.388,1.061)--(-9.39,1.058)--cycle;
\draw(-9.463,1.051)--(-9.453,1.077)--(-9.388,1.061)--(-9.39,1.058);
\filldraw[fill opacity=0.8,fill=gray!20,draw=none](-9.497,1.029)--(-9.497,1.045)--(-9.463,1.051)--(-9.472,1.027)--cycle;
\draw(-9.463,1.051)--(-9.472,1.027)--(-9.497,1.029);
\filldraw[fill opacity=0.8,fill=gray!20,draw=none](-9.434,1.017)--(-9.472,1.027)--(-9.463,1.051)--(-9.39,1.058)--(-9.408,1.031)--cycle;
\draw(-9.434,1.017)--(-9.472,1.027)--(-9.463,1.051);
\draw(-9.39,1.058)--(-9.408,1.031);
\filldraw[fill opacity=0.8,fill=gray!20,draw=none](-9.398,1.064)--(-9.436,1.073)--(-9.437,1.086)--cycle;
\draw(-9.398,1.064)--(-9.436,1.073);
\filldraw[fill opacity=0.8,fill=gray!20,draw=none](-9.569,1.043)--(-9.418,1.077)--(-9.39,1.058)--(-9.546,1.023)--cycle;
\draw(-9.39,1.058)--(-9.546,1.023)--(-9.569,1.043)--(-9.418,1.077);
\filldraw[fill opacity=0.8,fill=gray!20,draw=none](-9.398,1.064)--(-9.437,1.086)--(-9.44,1.133)--(-9.368,1.115)--(-9.388,1.061)--cycle;
\draw(-9.44,1.133)--(-9.368,1.115)--(-9.388,1.061)--(-9.398,1.064);
\filldraw[fill opacity=0.8,fill=gray!20,draw=none](-9.591,1.079)--(-9.439,1.113)--(-9.42,1.076)--(-9.569,1.043)--cycle;
\draw(-9.42,1.076)--(-9.569,1.043)--(-9.591,1.079)--(-9.439,1.113);
\filldraw[fill opacity=0.8,fill=gray!20,draw=none](-9.572,1.156)--(-9.552,1.203)--(-9.487,1.222)--(-9.525,1.136)--cycle;
\draw(-9.487,1.222)--(-9.525,1.136)--(-9.572,1.156)--(-9.552,1.203);
\filldraw[fill opacity=0.8,fill=gray!20,draw=none](-9.617,1.176)--(-9.552,1.203)--(-9.572,1.156)--cycle;
\draw(-9.552,1.203)--(-9.572,1.156)--(-9.617,1.176);
\filldraw[fill opacity=0.8,fill=gray!20,draw=none](-9.385,1.058)--(-9.389,1.059)--(-9.134,1.116)--(-9.144,1.109)--(-9.373,1.057)--cycle;
\draw(-9.389,1.059)--(-9.134,1.116);
\draw(-9.144,1.109)--(-9.373,1.057);
\filldraw[fill opacity=0.8,fill=gray!20,draw=none](-9.385,1.058)--(-9.373,1.057)--(-9.375,1.057)--cycle;
\draw(-9.373,1.057)--(-9.375,1.057);
\filldraw[fill opacity=0.8,fill=gray!20,draw=none](-9.356,1.077)--(-9.379,1.052)--(-9.388,1.061)--(-9.368,1.115)--(-9.347,1.092)--cycle;
\draw(-9.379,1.052)--(-9.388,1.061)--(-9.368,1.115)--(-9.347,1.092);
\filldraw[fill opacity=0.8,fill=gray!20,draw=none](-9.408,1.031)--(-9.39,1.058)--(-9.385,1.058)--(-9.379,1.052)--cycle;
\draw(-9.408,1.031)--(-9.39,1.058);
\draw(-9.385,1.058)--(-9.379,1.052);
\filldraw[fill opacity=0.8,fill=gray!20,draw=none](-9.558,1.15)--(-9.454,1.105)--(-9.475,1.068)--(-9.513,1.036)--(-9.546,1.025)--cycle;
\draw(-9.454,1.105)--(-9.475,1.068)--(-9.513,1.036)--(-9.546,1.025);
\filldraw[fill opacity=0.8,fill=gray!20,draw=none](-9.501,1.031)--(-9.521,1.029)--(-9.494,1.035)--(-9.471,1.036)--cycle;
\draw(-9.521,1.029)--(-9.494,1.035);
\filldraw[fill opacity=0.8,fill=gray!20,draw=none](-9.498,1.025)--(-9.497,1.029)--(-9.472,1.027)--(-9.489,.999)--cycle;
\draw(-9.497,1.029)--(-9.472,1.027)--(-9.489,.999);
\filldraw[fill opacity=0.8,fill=gray!20,draw=none](-9.494,1.035)--(-9.39,1.058)--(-9.385,1.058)--(-9.375,1.057)--(-9.465,1.037)--cycle;
\draw(-9.494,1.035)--(-9.39,1.058);
\draw(-9.375,1.057)--(-9.465,1.037);
\filldraw[fill opacity=0.8,fill=gray!20,draw=none](-9.501,1.031)--(-9.471,1.036)--(-9.465,1.037)--(-9.483,1.033)--cycle;
\draw(-9.465,1.037)--(-9.483,1.033);
\filldraw[fill opacity=0.8,fill=gray!20,draw=none](-9.475,1.068)--(-9.439,1.053)--(-9.493,1.028)--(-9.513,1.036)--cycle;
\draw(-9.493,1.028)--(-9.513,1.036)--(-9.475,1.068)--(-9.439,1.053);
\filldraw[fill opacity=0.8,fill=gray!20,draw=none](-9.501,1.031)--(-9.493,1.028)--(-9.542,1.014)--(-9.556,1.021)--cycle;
\draw(-9.501,1.031)--(-9.493,1.028);
\draw(-9.542,1.014)--(-9.556,1.021);
\filldraw[fill opacity=0.8,fill=gray!20,draw=none](-9.546,1.023)--(-9.521,1.029)--(-9.483,1.033)--(-9.524,1.023)--cycle;
\draw(-9.483,1.033)--(-9.524,1.023)--(-9.546,1.023)--(-9.521,1.029);
\filldraw[fill opacity=0.8,fill=gray!20,draw=none](-9.55,1.027)--(-9.546,1.023)--(-9.541,1.023)--cycle;
\draw(-9.55,1.027)--(-9.546,1.023)--(-9.541,1.023);
\filldraw[fill opacity=0.8,fill=gray!20,draw=none](-9.513,1.036)--(-9.501,1.031)--(-9.556,1.021)--(-9.558,1.021)--cycle;
\draw(-9.556,1.021)--(-9.558,1.021)--(-9.513,1.036)--(-9.501,1.031);
\filldraw[fill opacity=0.8,fill=gray!20,draw=none](-9.499,1.124)--(-9.617,1.176)--(-9.608,1.125)--(-9.591,1.079)--(-9.569,1.043)--(-9.55,1.027)--(-9.541,1.023)--(-9.524,1.023)--(-9.508,1.042)--(-9.499,1.078)--cycle;
\draw(-9.617,1.176)--(-9.608,1.125)--(-9.591,1.079)--(-9.569,1.043)--(-9.55,1.027);
\draw(-9.541,1.023)--(-9.524,1.023)--(-9.508,1.042)--(-9.499,1.078)--(-9.499,1.124);
\filldraw[fill opacity=0.8,fill=gray!20,draw=none](-9.558,1.15)--(-9.546,1.025)--(-9.558,1.021)--(-9.603,1.026)--(-9.609,1.03)--cycle;
\draw(-9.546,1.025)--(-9.558,1.021)--(-9.603,1.026)--(-9.609,1.03);
\filldraw[fill opacity=0.8,fill=gray!20,draw=none](-9.558,1.021)--(-9.533,1.011)--(-9.546,1.013)--(-9.603,1.026)--cycle;
\draw(-9.603,1.026)--(-9.558,1.021)--(-9.533,1.011);
\filldraw[fill opacity=0.8,fill=gray!20,draw=none](-9.609,1.03)--(-9.603,1.026)--(-9.606,1.029)--cycle;
\draw(-9.609,1.03)--(-9.603,1.026);
\filldraw[fill opacity=0.8,fill=gray!20](-9.609,1.029)--(-8.496,.564)--(-8.47,.564)--(-9.583,1.028)--cycle;
\filldraw[fill opacity=0.8,fill=gray!20](-8.603,.715)--(-8.587,.76)--(-8.603,.777)--(-8.621,.734)--cycle;
\filldraw[fill opacity=0.8,fill=gray!20](-8.419,.867)--(-8.445,.864)--(-8.445,.864)--(-8.403,.863)--cycle;
\filldraw[fill opacity=0.8,fill=gray!20](-8.269,.637)--(-8.262,.683)--(-8.292,.664)--(-8.298,.618)--cycle;
\filldraw[fill opacity=0.8,fill=gray!20](-8.448,.503)--(-8.451,.511)--(-8.495,.514)--(-8.471,.505)--cycle;
\filldraw[fill opacity=0.5,fill=gray!20](-8.287,-.046)--(-8.417,.082)--(-8.723,-.207)--(-8.631,-.371)--cycle;
\filldraw[fill opacity=0.5,fill=gray!20](-8.181,-.122)--(-8.286,-.046)--(-8.631,-.371)--(-8.541,-.463)--cycle;
\filldraw[fill opacity=0.5,fill=gray!20](-8.083,-.187)--(-8.181,-.122)--(-8.541,-.463)--(-8.456,-.54)--cycle;
\filldraw[fill opacity=0.5,fill=gray!20](-7.997,-.239)--(-8.083,-.187)--(-8.456,-.54)--(-8.377,-.598)--cycle;
\filldraw[fill opacity=0.5,fill=gray!20](-7.925,-.275)--(-7.997,-.239)--(-8.377,-.598)--(-8.308,-.637)--cycle;
\filldraw[fill opacity=0.5,fill=gray!20](-7.87,-.294)--(-7.925,-.275)--(-8.308,-.637)--(-8.25,-.653)--cycle;
\filldraw[fill opacity=0.5,fill=gray!20](-7.825,-.247)--(-8.287,-.046)--(-8.631,-.371)--(-8.169,-.573)--cycle;
\filldraw[fill opacity=0.8,fill=gray!20](-8.424,.504)--(-8.404,.513)--(-8.451,.511)--(-8.448,.503)--cycle;
\filldraw[fill opacity=0.8,fill=gray!20](-8.495,.514)--(-8.516,.536)--(-8.561,.547)--(-8.527,.522)--cycle;
\filldraw[fill opacity=0.8,fill=gray!20,draw=none](-8.423,.819)--(-8.411,.814)--(-8.43,.819)--cycle;
\draw(-8.423,.819)--(-8.411,.814);
\filldraw[fill opacity=0.8,fill=gray!20,draw=none](-8.52,.849)--(-8.53,.846)--(-8.536,.842)--(-8.514,.85)--cycle;
\draw(-8.53,.846)--(-8.536,.842)--(-8.514,.85);
\filldraw[fill opacity=0.8,fill=gray!20](-8.587,.76)--(-8.561,.799)--(-8.574,.814)--(-8.603,.777)--cycle;
\filldraw[fill opacity=0.8,fill=gray!20](-8.262,.683)--(-8.269,.73)--(-8.298,.711)--(-8.292,.664)--cycle;
\filldraw[fill opacity=0.8,fill=gray!20,draw=none](-8.427,.815)--(-8.398,.8)--(-8.411,.814)--(-8.423,.819)--cycle;
\draw(-8.411,.814)--(-8.423,.819);
\filldraw[fill opacity=0.8,fill=gray!20](-8.492,.859)--(-8.445,.864)--(-8.445,.864)--(-8.485,.864)--cycle;
\filldraw[fill opacity=0.8,fill=gray!20](-8.369,.52)--(-8.337,.544)--(-8.387,.534)--(-8.404,.513)--cycle;
\filldraw[fill opacity=0.8,fill=gray!20](-8.403,.863)--(-8.445,.864)--(-8.445,.864)--(-8.398,.858)--cycle;
\filldraw[fill opacity=0.8,fill=gray!20,draw=none](-8.427,.815)--(-8.423,.819)--(-8.498,.85)--cycle;
\draw(-8.423,.819)--(-8.498,.85);
\filldraw[fill opacity=0.8,fill=gray!20](-8.561,.799)--(-8.527,.831)--(-8.536,.842)--(-8.574,.814)--cycle;
\filldraw[fill opacity=0.8,fill=gray!20](-8.269,.73)--(-8.287,.773)--(-8.313,.756)--(-8.298,.711)--cycle;
\filldraw[fill opacity=0.8,fill=gray!20,draw=none](-8.598,.601)--(-8.588,.583)--(-8.587,.581)--(-8.597,.606)--cycle;
\draw(-8.588,.583)--(-8.587,.581)--(-8.597,.606);
\filldraw[fill opacity=0.8,fill=gray!20](-8.516,.536)--(-8.532,.568)--(-8.587,.581)--(-8.561,.547)--cycle;
\filldraw[fill opacity=0.8,fill=gray!20](-8.287,.773)--(-8.316,.81)--(-8.337,.796)--(-8.313,.756)--cycle;
\filldraw[fill opacity=0.8,fill=gray!20,draw=none](-8.527,.831)--(-8.495,.849)--(-8.514,.85)--(-8.536,.842)--cycle;
\draw(-8.514,.85)--(-8.536,.842)--(-8.527,.831)--(-8.495,.849);
\filldraw[fill opacity=0.8,fill=gray!20,draw=none](-8.495,.849)--(-8.487,.854)--(-8.492,.858)--(-8.497,.857)--(-8.498,.857)--cycle;
\draw(-8.495,.849)--(-8.487,.854)--(-8.492,.858);
\draw(-8.497,.857)--(-8.498,.857);
\filldraw[fill opacity=0.8,fill=gray!20,draw=none](-8.492,.858)--(-8.492,.859)--(-8.497,.857)--cycle;
\draw(-8.492,.858)--(-8.492,.859)--(-8.497,.857);
\filldraw[fill opacity=0.8,fill=gray!20](-8.487,.854)--(-8.445,.864)--(-8.445,.864)--(-8.492,.859)--cycle;
\filldraw[fill opacity=0.8,fill=gray!20](-8.316,.81)--(-8.354,.839)--(-8.369,.829)--(-8.337,.796)--cycle;
\filldraw[fill opacity=0.8,fill=gray!20](-8.442,1.238)--(-8.44,1.265)--(-8.395,1.262)--(-8.419,1.236)--cycle;
\filldraw[fill opacity=0.8,fill=gray!20](-8.398,.858)--(-8.445,.864)--(-8.445,.864)--(-8.406,.853)--cycle;
\filldraw[fill opacity=0.8,fill=gray!20](-8.406,.853)--(-8.445,.864)--(-8.445,.864)--(-8.424,.849)--cycle;
\filldraw[fill opacity=0.8,fill=gray!20](-8.424,.849)--(-8.445,.864)--(-8.445,.864)--(-8.448,.848)--cycle;
\filldraw[fill opacity=0.8,fill=gray!20](-8.448,.848)--(-8.445,.864)--(-8.445,.864)--(-8.471,.85)--cycle;
\filldraw[fill opacity=0.8,fill=gray!20](-8.471,.85)--(-8.445,.864)--(-8.445,.864)--(-8.487,.854)--cycle;
\filldraw[fill opacity=0.8,fill=gray!20](-8.354,.839)--(-8.398,.858)--(-8.406,.853)--(-8.369,.829)--cycle;
\filldraw[fill opacity=0.8,fill=gray!20](-8.451,.511)--(-8.453,.531)--(-8.516,.536)--(-8.495,.514)--cycle;
\filldraw[fill opacity=0.8,fill=gray!20,draw=none](-8.513,1.325)--(-8.532,1.367)--(-8.495,1.375)--(-8.478,1.329)--(-8.508,1.322)--cycle;
\draw(-8.532,1.367)--(-8.495,1.375)--(-8.478,1.329)--(-8.508,1.322);
\filldraw[fill opacity=0.8,fill=gray!20,draw=none](-8.535,1.371)--(-8.542,1.417)--(-8.504,1.426)--(-8.495,1.375)--(-8.532,1.367)--cycle;
\draw(-8.542,1.417)--(-8.504,1.426)--(-8.495,1.375)--(-8.532,1.367);
\filldraw[fill opacity=0.8,fill=gray!20,draw=none](-8.409,1.263)--(-8.386,1.276)--(-8.395,1.262)--cycle;
\draw(-8.386,1.276)--(-8.395,1.262)--(-8.409,1.263);
\filldraw[fill opacity=0.8,fill=gray!20,draw=none](-8.395,1.262)--(-8.386,1.276)--(-8.357,1.293)--(-8.329,1.286)--(-8.363,1.254)--cycle;
\draw(-8.357,1.293)--(-8.329,1.286)--(-8.363,1.254)--(-8.395,1.262)--(-8.386,1.276);
\filldraw[fill opacity=0.8,fill=gray!20](-8.337,.544)--(-8.313,.577)--(-8.374,.565)--(-8.387,.534)--cycle;
\filldraw[fill opacity=0.8,fill=gray!20](-8.577,1.508)--(-8.553,1.542)--(-8.503,1.552)--(-8.516,1.52)--cycle;
\filldraw[fill opacity=0.8,fill=gray!20,draw=none](-8.398,.8)--(-8.394,.796)--(-8.375,.788)--(-8.376,.789)--cycle;
\draw(-8.394,.796)--(-8.375,.788)--(-8.376,.789);
\filldraw[fill opacity=0.8,fill=gray!20,draw=none](-8.376,.789)--(-8.375,.788)--(-8.374,.781)--cycle;
\draw(-8.376,.789)--(-8.375,.788)--(-8.374,.781);
\filldraw[fill opacity=0.8,fill=gray!20,draw=none](-8.558,.611)--(-8.544,.622)--(-8.546,.652)--(-8.609,.668)--(-8.603,.622)--cycle;
\draw(-8.544,.622)--(-8.546,.652)--(-8.609,.668)--(-8.603,.622)--(-8.558,.611);
\filldraw[fill opacity=0.8,fill=gray!20,draw=none](-8.539,.623)--(-8.512,.65)--(-8.546,.652)--(-8.544,.622)--cycle;
\draw(-8.512,.65)--(-8.546,.652)--(-8.544,.622);
\filldraw[fill opacity=0.8,fill=gray!20,draw=none](-9.385,1.058)--(-9.39,1.058)--(-9.389,1.059)--cycle;
\draw(-9.39,1.058)--(-9.389,1.059);
\filldraw[fill opacity=0.8,fill=gray!20,draw=none](-9.39,1.058)--(-9.388,1.061)--(-9.385,1.058)--cycle;
\draw(-9.39,1.058)--(-9.388,1.061)--(-9.385,1.058);
\filldraw[fill opacity=0.8,fill=gray!20,draw=none](-9.415,.988)--(-9.433,.974)--(-9.403,.996)--(-9.411,.991)--cycle;
\draw(-9.433,.974)--(-9.403,.996)--(-9.411,.991);
\filldraw[fill opacity=0.8,fill=gray!20,draw=none](-9.454,1.105)--(-9.405,1.083)--(-9.427,1.06)--(-9.439,1.053)--(-9.475,1.068)--cycle;
\draw(-9.439,1.053)--(-9.475,1.068)--(-9.454,1.105);
\filldraw[fill opacity=0.8,fill=gray!20,draw=none](-9.524,1.023)--(-9.465,1.037)--(-9.427,1.06)--(-9.508,1.042)--cycle;
\draw(-9.427,1.06)--(-9.508,1.042)--(-9.524,1.023)--(-9.465,1.037);
\filldraw[fill opacity=0.8,fill=gray!20,draw=none](-9.493,1.028)--(-9.47,1.017)--(-9.479,1.015)--(-9.533,1.011)--(-9.542,1.014)--cycle;
\draw(-9.493,1.028)--(-9.47,1.017);
\draw(-9.533,1.011)--(-9.542,1.014);
\filldraw[fill opacity=0.8,fill=gray!20](-9.583,1.028)--(-8.47,.564)--(-8.44,.582)--(-9.552,1.046)--cycle;
\filldraw[fill opacity=0.8,fill=gray!20,draw=none](-8.357,1.293)--(-8.346,1.3)--(-8.319,1.301)--(-8.329,1.286)--cycle;
\draw(-8.319,1.301)--(-8.329,1.286)--(-8.357,1.293);
\filldraw[fill opacity=0.8,fill=gray!20,draw=none](-8.544,1.421)--(-8.537,1.465)--(-8.504,1.472)--(-8.504,1.426)--(-8.542,1.417)--cycle;
\draw(-8.537,1.465)--(-8.504,1.472)--(-8.504,1.426)--(-8.542,1.417);
\filldraw[fill opacity=0.8,fill=gray!20,draw=none](-8.396,.795)--(-8.379,.776)--(-8.374,.781)--(-8.375,.788)--(-8.394,.796)--cycle;
\draw(-8.374,.781)--(-8.375,.788)--(-8.394,.796);
\filldraw[fill opacity=0.8,fill=gray!20](-8.404,.513)--(-8.387,.534)--(-8.453,.531)--(-8.451,.511)--cycle;
\filldraw[fill opacity=0.8,fill=gray!20,draw=none](-8.495,.824)--(-8.483,.837)--(-8.495,.849)--(-8.527,.831)--cycle;
\draw(-8.495,.849)--(-8.527,.831)--(-8.495,.824)--(-8.483,.837);
\filldraw[fill opacity=0.8,fill=gray!20,draw=none](-8.475,.822)--(-8.475,.83)--(-8.483,.837)--(-8.495,.824)--cycle;
\draw(-8.483,.837)--(-8.495,.824)--(-8.475,.822);
\filldraw[fill opacity=0.8,fill=gray!20,draw=none](-9.367,1.213)--(-9.134,1.116)--(-9.189,1.128)--(-9.375,1.206)--cycle;
\draw(-9.367,1.213)--(-9.134,1.116);
\draw(-9.189,1.128)--(-9.375,1.206);
\filldraw[fill opacity=0.8,fill=gray!20,draw=none](-9.367,1.213)--(-9.375,1.206)--(-9.401,1.217)--cycle;
\draw(-9.375,1.206)--(-9.401,1.217);
\filldraw[fill opacity=0.8,fill=gray!20,draw=none](-9.43,1.188)--(-9.426,1.198)--(-9.387,1.227)--(-9.367,1.213)--(-9.362,1.171)--cycle;
\draw(-9.367,1.213)--(-9.362,1.171)--(-9.43,1.188);
\filldraw[fill opacity=0.8,fill=gray!20,draw=none](-9.399,1.227)--(-9.367,1.213)--(-9.401,1.217)--(-9.418,1.224)--cycle;
\draw(-9.399,1.227)--(-9.367,1.213);
\draw(-9.401,1.217)--(-9.418,1.224);
\filldraw[fill opacity=0.8,fill=gray!20,draw=none](-9.399,1.227)--(-9.418,1.224)--(-9.434,1.231)--cycle;
\draw(-9.418,1.224)--(-9.434,1.231);
\filldraw[fill opacity=0.8,fill=gray!20,draw=none](-9.483,1.117)--(-9.434,1.231)--(-9.414,1.239)--(-9.399,1.227)--(-9.452,1.104)--cycle;
\draw(-9.399,1.227)--(-9.452,1.104)--(-9.483,1.117)--(-9.434,1.231);
\filldraw[fill opacity=0.8,fill=gray!20,draw=none](-9.414,1.239)--(-9.434,1.231)--(-9.426,1.249)--cycle;
\draw(-9.434,1.231)--(-9.426,1.249);
\filldraw[fill opacity=0.8,fill=gray!20,draw=none](-9.507,1.271)--(-9.399,1.227)--(-9.434,1.231)--(-9.533,1.272)--cycle;
\draw(-9.434,1.231)--(-9.533,1.272)--(-9.507,1.271)--(-9.399,1.227);
\filldraw[fill opacity=0.8,fill=gray!20,draw=none](-9.363,1.181)--(-9.367,1.213)--(-9.356,1.197)--(-9.349,1.178)--cycle;
\draw(-9.363,1.181)--(-9.367,1.213);
\filldraw[fill opacity=0.8,fill=gray!20,draw=none](-9.45,1.103)--(-9.452,1.104)--(-9.448,1.115)--cycle;
\draw(-9.45,1.103)--(-9.452,1.104)--(-9.448,1.115);
\filldraw[fill opacity=0.8,fill=gray!20,draw=none](-9.454,1.105)--(-9.449,1.113)--(-9.398,1.091)--(-9.405,1.083)--cycle;
\draw(-9.454,1.105)--(-9.449,1.113)--(-9.398,1.091);
\filldraw[fill opacity=0.8,fill=gray!20,draw=none](-9.407,1.208)--(-9.399,1.227)--(-9.389,1.211)--(-9.392,1.203)--cycle;
\draw(-9.407,1.208)--(-9.399,1.227);
\draw(-9.389,1.211)--(-9.392,1.203);
\filldraw[fill opacity=0.8,fill=gray!20,draw=none](-9.525,1.136)--(-9.487,1.222)--(-9.434,1.231)--(-9.483,1.117)--cycle;
\draw(-9.434,1.231)--(-9.483,1.117)--(-9.525,1.136)--(-9.487,1.222);
\filldraw[fill opacity=0.8,fill=gray!20,draw=none](-9.481,1.294)--(-9.483,1.295)--(-9.455,1.296)--(-9.453,1.292)--cycle;
\draw(-9.455,1.296)--(-9.453,1.292)--(-9.481,1.294);
\filldraw[fill opacity=0.8,fill=gray!20,draw=none](-9.483,1.295)--(-9.498,1.305)--(-9.491,1.306)--(-9.455,1.296)--cycle;
\filldraw[fill opacity=0.8,fill=gray!20,draw=none](-9.452,1.293)--(-9.453,1.292)--(-9.455,1.296)--cycle;
\draw(-9.453,1.292)--(-9.455,1.296);
\filldraw[fill opacity=0.8,fill=gray!20,draw=none](-9.48,1.283)--(-9.51,1.3)--(-9.508,1.303)--(-9.448,1.313)--(-9.457,1.292)--cycle;
\draw(-9.51,1.3)--(-9.508,1.303);
\draw(-9.448,1.313)--(-9.457,1.292);
\filldraw[fill opacity=0.8,fill=gray!20,draw=none](-9.491,1.306)--(-9.462,1.31)--(-9.455,1.296)--cycle;
\draw(-9.462,1.31)--(-9.455,1.296);
\filldraw[fill opacity=0.8,fill=gray!20,draw=none](-9.452,1.294)--(-9.457,1.292)--(-9.455,1.296)--cycle;
\draw(-9.457,1.292)--(-9.455,1.296);
\filldraw[fill opacity=0.8,fill=gray!20,draw=none](-9.452,1.294)--(-9.455,1.296)--(-9.448,1.313)--(-9.394,1.321)--cycle;
\draw(-9.455,1.296)--(-9.448,1.313);
\filldraw[fill opacity=0.8,fill=gray!20,draw=none](-9.442,1.31)--(-9.452,1.293)--(-9.455,1.296)--(-9.462,1.31)--cycle;
\draw(-9.455,1.296)--(-9.462,1.31);
\filldraw[fill opacity=0.8,fill=gray!20,draw=none](-9.592,1.277)--(-9.442,1.31)--(-9.452,1.294)--(-9.457,1.292)--(-9.608,1.258)--cycle;
\draw(-9.457,1.292)--(-9.608,1.258)--(-9.592,1.277)--(-9.442,1.31);
\filldraw[fill opacity=0.8,fill=gray!20,draw=none](-9.44,1.163)--(-9.449,1.113)--(-9.454,1.105)--(-9.558,1.15)--(-9.507,1.27)--(-9.475,1.25)--(-9.449,1.211)--cycle;
\draw(-9.507,1.27)--(-9.475,1.25)--(-9.449,1.211)--(-9.44,1.163)--(-9.449,1.113)--(-9.454,1.105);
\filldraw[fill opacity=0.8,fill=gray!20,draw=none](-9.423,1.172)--(-9.407,1.208)--(-9.392,1.203)--(-9.406,1.17)--cycle;
\draw(-9.423,1.172)--(-9.407,1.208);
\draw(-9.392,1.203)--(-9.406,1.17);
\filldraw[fill opacity=0.8,fill=gray!20,draw=none](-9.475,1.25)--(-9.458,1.243)--(-9.449,1.236)--(-9.411,1.194)--(-9.449,1.211)--cycle;
\draw(-9.411,1.194)--(-9.449,1.211)--(-9.475,1.25)--(-9.458,1.243);
\filldraw[fill opacity=0.8,fill=gray!20,draw=none](-9.507,1.27)--(-9.505,1.269)--(-9.458,1.243)--(-9.475,1.25)--cycle;
\draw(-9.458,1.243)--(-9.475,1.25)--(-9.507,1.27);
\filldraw[fill opacity=0.8,fill=gray!20,draw=none](-9.488,1.252)--(-8.394,.796)--(-8.398,.8)--(-8.498,.85)--(-9.507,1.271)--cycle;
\draw(-8.498,.85)--(-9.507,1.271)--(-9.488,1.252)--(-8.394,.796);
\filldraw[fill opacity=0.8,fill=gray!20,draw=none](-8.544,.571)--(-8.597,.606)--(-8.587,.581)--cycle;
\draw(-8.597,.606)--(-8.587,.581)--(-8.544,.571);
\filldraw[fill opacity=0.8,fill=gray!20,draw=none](-8.305,1.325)--(-8.325,1.301)--(-8.346,1.3)--cycle;
\filldraw[fill opacity=0.8,fill=gray!20](-8.419,1.236)--(-8.395,1.262)--(-8.363,1.254)--(-8.403,1.232)--cycle;
\filldraw[fill opacity=0.5,fill=gray!20](-7.856,1.752)--(-8.035,1.681)--(-7.984,1.269)--(-7.8,1.289)--cycle;
\filldraw[fill opacity=0.5,fill=gray!20](-7.731,1.761)--(-7.856,1.752)--(-7.8,1.289)--(-7.671,1.276)--cycle;
\filldraw[fill opacity=0.5,fill=gray!20](-7.618,1.759)--(-7.731,1.761)--(-7.671,1.276)--(-7.557,1.258)--cycle;
\filldraw[fill opacity=0.5,fill=gray!20](-7.522,1.746)--(-7.618,1.759)--(-7.557,1.258)--(-7.46,1.235)--cycle;
\filldraw[fill opacity=0.5,fill=gray!20](-7.447,1.723)--(-7.522,1.746)--(-7.46,1.235)--(-7.384,1.209)--cycle;
\filldraw[fill opacity=0.5,fill=gray!20](-7.396,1.691)--(-7.447,1.723)--(-7.384,1.209)--(-7.333,1.18)--cycle;
\filldraw[fill opacity=0.5,fill=gray!20](-7.395,1.551)--(-7.856,1.752)--(-7.8,1.289)--(-7.338,1.088)--cycle;
\filldraw[fill opacity=0.8,fill=gray!20,draw=none](-8.437,1.546)--(-8.437,1.554)--(-8.426,1.554)--cycle;
\draw(-8.437,1.546)--(-8.437,1.554)--(-8.426,1.554);
\filldraw[fill opacity=0.8,fill=gray!20,draw=none](-8.48,1.29)--(-8.508,1.322)--(-8.478,1.329)--(-8.456,1.293)--(-8.473,1.289)--cycle;
\draw(-8.508,1.322)--(-8.478,1.329)--(-8.456,1.293)--(-8.473,1.289);
\filldraw[fill opacity=0.8,fill=gray!20,draw=none](-8.465,1.543)--(-8.46,1.553)--(-8.437,1.554)--(-8.437,1.548)--cycle;
\draw(-8.46,1.553)--(-8.437,1.554)--(-8.437,1.548);
\filldraw[fill opacity=0.8,fill=gray!20,draw=none](-8.465,1.543)--(-8.51,1.534)--(-8.503,1.552)--(-8.46,1.553)--cycle;
\draw(-8.51,1.534)--(-8.503,1.552)--(-8.46,1.553);
\filldraw[fill opacity=0.8,fill=gray!20,draw=none](-8.434,1.548)--(-8.426,1.554)--(-8.4,1.552)--cycle;
\draw(-8.426,1.554)--(-8.4,1.552);
\filldraw[fill opacity=0.8,fill=gray!20,draw=none](-8.396,.795)--(-8.394,.796)--(-8.397,.797)--cycle;
\draw(-8.394,.796)--(-8.397,.797);
\filldraw[fill opacity=0.8,fill=gray!20](-8.516,.788)--(-8.495,.824)--(-8.527,.831)--(-8.561,.799)--cycle;
\filldraw[fill opacity=0.8,fill=gray!20,draw=none](-8.469,.785)--(-8.466,.795)--(-8.475,.822)--(-8.495,.824)--(-8.516,.788)--cycle;
\draw(-8.475,.822)--(-8.495,.824)--(-8.516,.788)--(-8.469,.785);
\filldraw[fill opacity=0.8,fill=gray!20,draw=none](-9.343,1.064)--(-9.144,1.109)--(-9.217,1.108)--(-9.349,1.078)--cycle;
\draw(-9.343,1.064)--(-9.144,1.109);
\draw(-9.217,1.108)--(-9.349,1.078);
\filldraw[fill opacity=0.8,fill=gray!20,draw=none](-9.347,1.092)--(-9.368,1.115)--(-9.362,1.171)--(-9.341,1.149)--(-9.341,1.134)--(-9.346,1.094)--cycle;
\draw(-9.347,1.092)--(-9.368,1.115)--(-9.362,1.171)--(-9.341,1.149);
\draw(-9.341,1.134)--(-9.346,1.094);
\filldraw[fill opacity=0.8,fill=gray!20,draw=none](-9.341,1.149)--(-9.362,1.171)--(-9.363,1.181)--(-9.349,1.178)--(-9.344,1.162)--cycle;
\draw(-9.341,1.149)--(-9.362,1.171)--(-9.363,1.181);
\filldraw[fill opacity=0.8,fill=gray!20,draw=none](-9.436,1.142)--(-9.423,1.172)--(-9.406,1.17)--(-9.416,1.147)--cycle;
\draw(-9.436,1.142)--(-9.423,1.172);
\draw(-9.406,1.17)--(-9.416,1.147);
\filldraw[fill opacity=0.8,fill=gray!20,draw=none](-9.449,1.211)--(-9.411,1.194)--(-9.406,1.187)--(-9.405,1.185)--(-9.405,1.183)--(-9.415,1.152)--(-9.44,1.163)--cycle;
\draw(-9.415,1.152)--(-9.44,1.163)--(-9.449,1.211)--(-9.411,1.194);
\filldraw[fill opacity=0.8,fill=gray!20,draw=none](-9.448,1.112)--(-9.448,1.115)--(-9.436,1.142)--(-9.416,1.147)--(-9.42,1.139)--cycle;
\draw(-9.448,1.115)--(-9.436,1.142);
\draw(-9.416,1.147)--(-9.42,1.139);
\filldraw[fill opacity=0.8,fill=gray!20,draw=none](-9.405,1.177)--(-9.406,1.17)--(-9.389,1.211)--(-9.398,1.196)--cycle;
\draw(-9.406,1.17)--(-9.389,1.211);
\filldraw[fill opacity=0.8,fill=gray!20,draw=none](-9.405,1.177)--(-9.398,1.196)--(-9.404,1.187)--cycle;
\filldraw[fill opacity=0.8,fill=gray!20,draw=none](-9.48,1.218)--(-8.432,.78)--(-8.396,.795)--(-8.397,.797)--(-9.488,1.252)--cycle;
\draw(-8.397,.797)--(-9.488,1.252)--(-9.48,1.218)--(-8.432,.78);
\filldraw[fill opacity=0.5,fill=gray!20,draw=none](-8.171,.473)--(-8.169,.436)--(-8.182,.442)--cycle;
\draw(-8.169,.436)--(-8.182,.442)--(-8.171,.473);
\filldraw[fill opacity=0.5,fill=gray!20](-8.182,.442)--(-8.009,.366)--(-8.244,.007)--(-8.417,.082)--cycle;
\filldraw[fill opacity=0.8,fill=gray!20,draw=none](-8.396,.795)--(-8.432,.78)--(-8.39,.763)--(-8.379,.776)--cycle;
\draw(-8.432,.78)--(-8.39,.763);
\filldraw[fill opacity=0.8,fill=gray!20,draw=none](-8.474,.846)--(-8.471,.85)--(-8.487,.854)--(-8.495,.849)--cycle;
\draw(-8.474,.846)--(-8.471,.85)--(-8.487,.854)--(-8.495,.849);
\filldraw[fill opacity=0.5,fill=gray!20,draw=none](-8.171,.473)--(-8.035,.846)--(-7.862,.771)--(-8.009,.366)--(-8.169,.436)--cycle;
\draw(-8.171,.473)--(-8.035,.846)--(-7.862,.771)--(-8.009,.366)--(-8.169,.436);
\filldraw[fill opacity=0.8,fill=gray!20,draw=none](-8.574,.591)--(-8.564,.603)--(-8.597,.606)--cycle;
\filldraw[fill opacity=0.8,fill=gray!20,draw=none](-8.426,1.554)--(-8.437,1.554)--(-8.44,1.575)--(-8.401,1.572)--cycle;
\draw(-8.426,1.554)--(-8.437,1.554)--(-8.44,1.575)--(-8.401,1.572);
\filldraw[fill opacity=0.8,fill=gray!20](-8.503,1.552)--(-8.486,1.573)--(-8.44,1.575)--(-8.437,1.554)--cycle;
\filldraw[fill opacity=0.8,fill=gray!20,draw=none](-8.483,.837)--(-8.474,.846)--(-8.495,.849)--cycle;
\draw(-8.483,.837)--(-8.474,.846);
\filldraw[fill opacity=0.8,fill=gray!20](-8.369,.829)--(-8.406,.853)--(-8.424,.849)--(-8.404,.823)--cycle;
\filldraw[fill opacity=0.8,fill=gray!20](-8.313,.577)--(-8.298,.618)--(-8.366,.605)--(-8.374,.565)--cycle;
\filldraw[fill opacity=0.8,fill=gray!20,draw=none](-8.39,1.553)--(-8.4,1.552)--(-8.426,1.554)--(-8.419,1.558)--cycle;
\draw(-8.4,1.552)--(-8.426,1.554);
\filldraw[fill opacity=0.8,fill=gray!20,draw=none](-8.379,.776)--(-8.39,.763)--(-8.367,.753)--(-8.37,.765)--cycle;
\draw(-8.39,.763)--(-8.367,.753)--(-8.37,.765);
\filldraw[fill opacity=0.8,fill=gray!20,draw=none](-8.538,1.468)--(-8.518,1.503)--(-8.495,1.508)--(-8.504,1.472)--(-8.537,1.465)--cycle;
\draw(-8.518,1.503)--(-8.495,1.508)--(-8.504,1.472)--(-8.537,1.465);
\filldraw[fill opacity=0.8,fill=gray!20,draw=none](-8.325,1.301)--(-8.305,1.325)--(-8.303,1.326)--(-8.319,1.301)--cycle;
\draw(-8.303,1.326)--(-8.319,1.301);
\filldraw[fill opacity=0.8,fill=gray!20,draw=none](-8.305,1.325)--(-8.304,1.326)--(-8.303,1.326)--cycle;
\draw(-8.304,1.326)--(-8.303,1.326);
\filldraw[fill opacity=0.8,fill=gray!20,draw=none](-8.304,1.326)--(-8.303,1.328)--(-8.303,1.326)--cycle;
\draw(-8.303,1.328)--(-8.303,1.326)--(-8.304,1.326);
\filldraw[fill opacity=0.8,fill=gray!20,draw=none](-8.292,1.334)--(-8.297,1.32)--(-8.303,1.326)--cycle;
\draw(-8.297,1.32)--(-8.303,1.326);
\filldraw[fill opacity=0.8,fill=gray!20,draw=none](-8.292,1.334)--(-8.303,1.326)--(-8.303,1.328)--(-8.283,1.362)--cycle;
\draw(-8.303,1.326)--(-8.303,1.328);
\filldraw[fill opacity=0.8,fill=gray!20,draw=none](-8.319,1.301)--(-8.303,1.326)--(-8.287,1.309)--(-8.297,1.296)--cycle;
\draw(-8.319,1.301)--(-8.303,1.326)--(-8.287,1.309)--(-8.297,1.296);
\filldraw[fill opacity=0.5,fill=gray!20](-7.984,1.269)--(-7.811,1.193)--(-7.862,.771)--(-8.035,.846)--cycle;
\filldraw[fill opacity=0.8,fill=gray!20,draw=none](-8.329,1.286)--(-8.319,1.301)--(-8.297,1.296)--(-8.316,1.272)--cycle;
\draw(-8.297,1.296)--(-8.316,1.272)--(-8.329,1.286)--(-8.319,1.301);
\filldraw[fill opacity=0.8,fill=gray!20,draw=none](-8.593,1.32)--(-8.603,1.313)--(-8.621,1.356)--(-8.602,1.369)--cycle;
\draw(-8.593,1.32)--(-8.603,1.313)--(-8.621,1.356)--(-8.602,1.369);
\filldraw[fill opacity=0.8,fill=gray!20,draw=none](-8.605,1.367)--(-8.606,1.373)--(-8.569,1.359)--(-8.599,1.352)--cycle;
\draw(-8.569,1.359)--(-8.599,1.352);
\filldraw[fill opacity=0.8,fill=gray!20](-8.553,1.542)--(-8.521,1.566)--(-8.486,1.573)--(-8.503,1.552)--cycle;
\filldraw[fill opacity=0.8,fill=gray!20](-8.445,1.221)--(-8.466,1.237)--(-8.442,1.238)--(-8.445,1.221)--cycle;
\filldraw[fill opacity=0.8,fill=gray!20](-8.445,1.221)--(-8.442,1.238)--(-8.419,1.236)--(-8.445,1.221)--cycle;
\filldraw[fill opacity=0.8,fill=gray!20](-8.363,1.254)--(-8.329,1.286)--(-8.316,1.272)--(-8.354,1.244)--cycle;
\filldraw[fill opacity=0.8,fill=gray!20,draw=none](-8.574,.591)--(-8.544,.571)--(-8.532,.568)--(-8.541,.601)--(-8.564,.603)--cycle;
\draw(-8.544,.571)--(-8.532,.568)--(-8.541,.601);
\filldraw[fill opacity=0.8,fill=gray!20,draw=none](-8.475,.83)--(-8.465,.821)--(-8.451,.82)--(-8.448,.848)--(-8.471,.85)--(-8.474,.847)--cycle;
\draw(-8.465,.821)--(-8.451,.82)--(-8.448,.848)--(-8.471,.85)--(-8.474,.847);
\filldraw[fill opacity=0.8,fill=gray!20](-8.453,.531)--(-8.455,.562)--(-8.532,.568)--(-8.516,.536)--cycle;
\filldraw[fill opacity=0.8,fill=gray!20,draw=none](-8.39,1.553)--(-8.419,1.558)--(-8.401,1.572)--(-8.395,1.571)--(-8.378,1.554)--cycle;
\draw(-8.401,1.572)--(-8.395,1.571)--(-8.378,1.554);
\filldraw[fill opacity=0.8,fill=gray!20](-8.445,1.221)--(-8.485,1.233)--(-8.466,1.237)--(-8.445,1.221)--cycle;
\filldraw[fill opacity=0.8,fill=gray!20](-8.546,.652)--(-8.542,.7)--(-8.603,.715)--(-8.609,.668)--cycle;
\filldraw[fill opacity=0.8,fill=gray!20](-8.404,.823)--(-8.424,.849)--(-8.448,.848)--(-8.451,.82)--cycle;
\filldraw[fill opacity=0.8,fill=gray!20,draw=none](-8.475,.83)--(-8.474,.847)--(-8.483,.837)--cycle;
\draw(-8.474,.847)--(-8.483,.837);
\filldraw[fill opacity=0.8,fill=gray!20,draw=none](-8.512,.65)--(-8.504,.655)--(-8.486,.696)--(-8.542,.7)--(-8.546,.652)--cycle;
\draw(-8.486,.696)--(-8.542,.7)--(-8.546,.652)--(-8.512,.65);
\filldraw[fill opacity=0.8,fill=gray!20,draw=none](-9.434,1.017)--(-9.408,1.031)--(-9.419,1.014)--cycle;
\draw(-9.408,1.031)--(-9.419,1.014)--(-9.434,1.017);
\filldraw[fill opacity=0.8,fill=gray!20,draw=none](-9.419,1.014)--(-9.408,1.031)--(-9.379,1.052)--(-9.368,1.04)--(-9.403,.996)--cycle;
\draw(-9.379,1.052)--(-9.368,1.04)--(-9.403,.996)--(-9.419,1.014)--(-9.408,1.031);
\filldraw[fill opacity=0.8,fill=gray!20,draw=none](-9.426,1.007)--(-9.419,1.014)--(-9.403,.996)--(-9.406,.994)--cycle;
\draw(-9.426,1.007)--(-9.419,1.014)--(-9.403,.996)--(-9.406,.994);
\filldraw[fill opacity=0.8,fill=gray!20,draw=none](-9.437,1.043)--(-9.399,1.051)--(-9.385,1.07)--(-9.413,1.064)--cycle;
\draw(-9.437,1.043)--(-9.399,1.051);
\draw(-9.385,1.07)--(-9.413,1.064);
\filldraw[fill opacity=0.8,fill=gray!20,draw=none](-9.465,1.037)--(-9.437,1.043)--(-9.413,1.064)--(-9.427,1.06)--cycle;
\draw(-9.465,1.037)--(-9.437,1.043);
\draw(-9.413,1.064)--(-9.427,1.06);
\filldraw[fill opacity=0.8,fill=gray!20,draw=none](-9.388,1.023)--(-9.411,.991)--(-9.403,.996)--(-9.368,1.04)--(-9.381,1.032)--cycle;
\draw(-9.411,.991)--(-9.403,.996)--(-9.368,1.04)--(-9.381,1.032);
\filldraw[fill opacity=0.8,fill=gray!20,draw=none](-9.439,1.053)--(-9.42,1.044)--(-9.426,1.039)--(-9.47,1.017)--(-9.493,1.028)--cycle;
\draw(-9.439,1.053)--(-9.42,1.044);
\draw(-9.47,1.017)--(-9.493,1.028);
\filldraw[fill opacity=0.8,fill=gray!20,draw=none](-9.415,1.152)--(-9.388,1.14)--(-9.387,1.132)--(-9.388,1.117)--(-9.392,1.102)--(-9.398,1.091)--(-9.433,1.106)--cycle;
\draw(-9.415,1.152)--(-9.388,1.14);
\draw(-9.398,1.091)--(-9.433,1.106);
\filldraw[fill opacity=0.8,fill=gray!20,draw=none](-9.508,1.042)--(-9.413,1.064)--(-9.392,1.102)--(-9.499,1.078)--cycle;
\draw(-9.392,1.102)--(-9.499,1.078)--(-9.508,1.042)--(-9.413,1.064);
\filldraw[fill opacity=0.8,fill=gray!20](-9.552,1.046)--(-8.44,.582)--(-8.41,.616)--(-9.523,1.08)--cycle;
\filldraw[fill opacity=0.8,fill=gray!20,draw=none](-8.377,1.55)--(-8.39,1.553)--(-8.378,1.554)--(-8.374,1.55)--cycle;
\draw(-8.378,1.554)--(-8.374,1.55)--(-8.377,1.55);
\filldraw[fill opacity=0.8,fill=gray!20,draw=none](-8.377,1.55)--(-8.374,1.55)--(-8.373,1.549)--cycle;
\draw(-8.377,1.55)--(-8.374,1.55)--(-8.373,1.549);
\filldraw[fill opacity=0.8,fill=gray!20](-8.532,.746)--(-8.516,.788)--(-8.561,.799)--(-8.587,.76)--cycle;
\filldraw[fill opacity=0.8,fill=gray!20,draw=none](-8.473,.742)--(-8.469,.749)--(-8.469,.785)--(-8.516,.788)--(-8.532,.746)--cycle;
\draw(-8.469,.785)--(-8.516,.788)--(-8.532,.746)--(-8.473,.742);
\filldraw[fill opacity=0.8,fill=gray!20,draw=none](-9.356,1.077)--(-9.217,1.108)--(-9.334,1.113)--cycle;
\draw(-9.356,1.077)--(-9.217,1.108);
\filldraw[fill opacity=0.8,fill=gray!20,draw=none](-9.44,1.163)--(-9.415,1.152)--(-9.433,1.106)--(-9.449,1.113)--cycle;
\draw(-9.433,1.106)--(-9.449,1.113)--(-9.44,1.163)--(-9.415,1.152);
\filldraw[fill opacity=0.8,fill=gray!20,draw=none](-9.45,1.103)--(-9.448,1.112)--(-9.42,1.139)--(-9.438,1.098)--cycle;
\draw(-9.42,1.139)--(-9.438,1.098)--(-9.45,1.103);
\filldraw[fill opacity=0.8,fill=gray!20,draw=none](-9.438,1.098)--(-9.42,1.139)--(-9.415,1.152)--(-9.405,1.183)--(-9.442,1.099)--cycle;
\draw(-9.405,1.183)--(-9.442,1.099)--(-9.438,1.098)--(-9.42,1.139);
\filldraw[fill opacity=0.8,fill=gray!20,draw=none](-9.405,1.177)--(-9.42,1.139)--(-9.406,1.17)--cycle;
\draw(-9.42,1.139)--(-9.406,1.17);
\filldraw[fill opacity=0.8,fill=gray!20,draw=none](-9.415,1.152)--(-9.405,1.177)--(-9.404,1.187)--(-9.404,1.186)--(-9.405,1.183)--cycle;
\draw(-9.404,1.186)--(-9.405,1.183);
\filldraw[fill opacity=0.8,fill=gray!20,draw=none](-9.405,1.184)--(-9.388,1.14)--(-9.415,1.152)--cycle;
\draw(-9.388,1.14)--(-9.415,1.152);
\filldraw[fill opacity=0.8,fill=gray!20,draw=none](-9.413,1.064)--(-9.356,1.077)--(-9.346,1.093)--(-9.349,1.111)--(-9.392,1.102)--cycle;
\draw(-9.413,1.064)--(-9.356,1.077);
\draw(-9.349,1.111)--(-9.392,1.102);
\filldraw[fill opacity=0.8,fill=gray!20,draw=none](-9.349,1.116)--(-9.349,1.114)--(-9.351,1.089)--(-9.346,1.092)--(-9.341,1.134)--cycle;
\draw(-9.351,1.089)--(-9.346,1.092)--(-9.341,1.134);
\filldraw[fill opacity=0.8,fill=gray!20,draw=none](-9.346,1.093)--(-9.334,1.113)--(-9.341,1.113)--(-9.349,1.111)--cycle;
\draw(-9.341,1.113)--(-9.349,1.111);
\filldraw[fill opacity=0.8,fill=gray!20,draw=none](-9.442,1.099)--(-9.405,1.183)--(-9.405,1.185)--(-9.443,1.156)--(-9.463,1.109)--cycle;
\draw(-9.443,1.156)--(-9.463,1.109)--(-9.442,1.099)--(-9.405,1.183);
\filldraw[fill opacity=0.8,fill=gray!20,draw=none](-9.405,1.185)--(-9.405,1.184)--(-9.405,1.183)--cycle;
\filldraw[fill opacity=0.8,fill=gray!20,draw=none](-9.405,1.183)--(-9.404,1.186)--(-9.405,1.185)--cycle;
\draw(-9.405,1.183)--(-9.404,1.186);
\filldraw[fill opacity=0.8,fill=gray!20](-9.484,1.173)--(-8.371,.708)--(-8.367,.753)--(-9.48,1.218)--cycle;
\filldraw[fill opacity=0.8,fill=gray!20,draw=none](-8.346,1.537)--(-8.373,1.549)--(-8.374,1.55)--(-8.357,1.546)--cycle;
\draw(-8.373,1.549)--(-8.374,1.55)--(-8.357,1.546);
\filldraw[fill opacity=0.8,fill=gray!20,draw=none](-8.357,1.546)--(-8.374,1.55)--(-8.378,1.554)--(-8.37,1.553)--cycle;
\draw(-8.357,1.546)--(-8.374,1.55)--(-8.378,1.554);
\filldraw[fill opacity=0.8,fill=gray!20](-8.445,1.221)--(-8.419,1.236)--(-8.403,1.232)--(-8.445,1.221)--cycle;
\filldraw[fill opacity=0.8,fill=gray!20,draw=none](-8.443,1.272)--(-8.473,1.289)--(-8.456,1.293)--(-8.434,1.274)--cycle;
\draw(-8.473,1.289)--(-8.456,1.293)--(-8.434,1.274);
\filldraw[fill opacity=0.8,fill=gray!20](-8.337,.796)--(-8.369,.829)--(-8.404,.823)--(-8.387,.787)--cycle;
\filldraw[fill opacity=0.8,fill=gray!20,draw=none](-8.607,1.388)--(-8.608,1.402)--(-8.588,1.407)--cycle;
\draw(-8.608,1.402)--(-8.588,1.407);
\filldraw[fill opacity=0.8,fill=gray!20](-8.298,.618)--(-8.292,.664)--(-8.363,.65)--(-8.366,.605)--cycle;
\filldraw[fill opacity=0.8,fill=gray!20](-8.387,.534)--(-8.374,.565)--(-8.455,.562)--(-8.453,.531)--cycle;
\filldraw[fill opacity=0.8,fill=gray!20](-8.403,1.232)--(-8.363,1.254)--(-8.354,1.244)--(-8.398,1.227)--cycle;
\filldraw[fill opacity=0.8,fill=gray!20](-8.542,.7)--(-8.532,.746)--(-8.587,.76)--(-8.603,.715)--cycle;
\filldraw[fill opacity=0.8,fill=gray!20,draw=none](-8.378,1.554)--(-8.395,1.571)--(-8.363,1.564)--(-8.347,1.552)--cycle;
\draw(-8.378,1.554)--(-8.395,1.571)--(-8.363,1.564)--(-8.347,1.552);
\filldraw[fill opacity=0.8,fill=gray!20,draw=none](-8.605,1.367)--(-8.619,1.4)--(-8.608,1.402)--cycle;
\draw(-8.619,1.4)--(-8.608,1.402);
\filldraw[fill opacity=0.8,fill=gray!20,draw=none](-8.607,1.388)--(-8.605,1.367)--(-8.621,1.356)--(-8.626,1.386)--cycle;
\draw(-8.605,1.367)--(-8.621,1.356)--(-8.626,1.386);
\filldraw[fill opacity=0.8,fill=gray!20,draw=none](-8.608,1.402)--(-8.607,1.388)--(-8.626,1.386)--(-8.628,1.402)--(-8.611,1.413)--cycle;
\draw(-8.626,1.386)--(-8.628,1.402)--(-8.611,1.413);
\filldraw[fill opacity=0.8,fill=gray!20,draw=none](-8.603,1.491)--(-8.594,1.503)--(-8.573,1.514)--(-8.577,1.508)--cycle;
\draw(-8.573,1.514)--(-8.577,1.508)--(-8.603,1.491)--(-8.594,1.503);
\filldraw[fill opacity=0.8,fill=gray!20](-8.336,1.437)--(-8.362,1.482)--(-8.4,1.514)--(-8.445,1.529)--(-8.49,1.524)--(-8.528,1.5)--(-8.554,1.461)--(-8.563,1.413)--(-8.554,1.363)--(-8.528,1.318)--(-8.49,1.286)--(-8.445,1.271)--(-8.4,1.276)--(-8.362,1.3)--(-8.336,1.339)--(-8.327,1.387)--cycle;
\filldraw[fill opacity=0.8,fill=gray!20](-8.445,1.221)--(-8.492,1.228)--(-8.485,1.233)--(-8.445,1.221)--cycle;
\filldraw[fill opacity=0.8,fill=gray!20,draw=none](-8.281,1.365)--(-8.282,1.365)--(-8.274,1.41)--(-8.267,1.402)--cycle;
\draw(-8.281,1.365)--(-8.282,1.365);
\draw(-8.274,1.41)--(-8.267,1.402);
\filldraw[fill opacity=0.8,fill=gray!20,draw=none](-8.283,1.362)--(-8.282,1.365)--(-8.281,1.365)--cycle;
\draw(-8.282,1.365)--(-8.281,1.365);
\filldraw[fill opacity=0.8,fill=gray!20,draw=none](-8.292,1.334)--(-8.283,1.362)--(-8.281,1.365)--(-8.269,1.351)--cycle;
\draw(-8.281,1.365)--(-8.269,1.351);
\filldraw[fill opacity=0.8,fill=gray!20](-8.313,.756)--(-8.337,.796)--(-8.387,.787)--(-8.374,.744)--cycle;
\filldraw[fill opacity=0.5,fill=gray!20](-8.417,.082)--(-8.244,.007)--(-8.55,-.283)--(-8.723,-.207)--cycle;
\filldraw[fill opacity=0.8,fill=gray!20](-8.292,.664)--(-8.298,.711)--(-8.366,.697)--(-8.363,.65)--cycle;
\filldraw[fill opacity=0.8,fill=gray!20,draw=none](-8.594,1.503)--(-8.574,1.528)--(-8.553,1.542)--(-8.573,1.514)--cycle;
\draw(-8.594,1.503)--(-8.574,1.528)--(-8.553,1.542)--(-8.573,1.514);
\filldraw[fill opacity=0.8,fill=gray!20,draw=none](-8.297,1.32)--(-8.292,1.334)--(-8.269,1.351)--(-8.287,1.309)--cycle;
\draw(-8.269,1.351)--(-8.287,1.309)--(-8.297,1.32);
\filldraw[fill opacity=0.8,fill=gray!20,draw=none](-8.281,1.365)--(-8.267,1.402)--(-8.262,1.398)--(-8.269,1.351)--cycle;
\draw(-8.267,1.402)--(-8.262,1.398)--(-8.269,1.351)--(-8.281,1.365);
\filldraw[fill opacity=0.8,fill=gray!20,draw=none](-8.564,.603)--(-8.541,.601)--(-8.542,.607)--(-8.558,.611)--cycle;
\draw(-8.541,.601)--(-8.542,.607)--(-8.558,.611);
\filldraw[fill opacity=0.8,fill=gray!20](-8.445,1.221)--(-8.403,1.232)--(-8.398,1.227)--(-8.445,1.221)--cycle;
\filldraw[fill opacity=0.8,fill=gray!20,draw=none](-8.585,1.307)--(-8.579,1.306)--(-8.584,1.305)--cycle;
\draw(-8.579,1.306)--(-8.584,1.305);
\filldraw[fill opacity=0.8,fill=gray!20,draw=none](-8.357,1.546)--(-8.37,1.553)--(-8.347,1.552)--(-8.329,1.539)--cycle;
\draw(-8.347,1.552)--(-8.329,1.539)--(-8.357,1.546);
\filldraw[fill opacity=0.8,fill=gray!20,draw=none](-8.346,1.537)--(-8.357,1.546)--(-8.329,1.539)--(-8.319,1.526)--cycle;
\draw(-8.357,1.546)--(-8.329,1.539)--(-8.319,1.526);
\filldraw[fill opacity=0.8,fill=gray!20,draw=none](-8.585,1.307)--(-8.579,1.281)--(-8.593,1.3)--cycle;
\draw(-8.579,1.281)--(-8.593,1.3);
\filldraw[fill opacity=0.8,fill=gray!20,draw=none](-8.305,1.506)--(-8.346,1.537)--(-8.325,1.528)--cycle;
\filldraw[fill opacity=0.8,fill=gray!20,draw=none](-8.475,.822)--(-8.465,.821)--(-8.475,.83)--cycle;
\draw(-8.475,.822)--(-8.465,.821);
\filldraw[fill opacity=0.8,fill=gray!20](-8.44,1.575)--(-8.442,1.583)--(-8.419,1.581)--(-8.395,1.571)--cycle;
\filldraw[fill opacity=0.8,fill=gray!20](-8.486,1.573)--(-8.466,1.582)--(-8.442,1.583)--(-8.44,1.575)--cycle;
\filldraw[fill opacity=0.8,fill=gray!20](-8.298,.711)--(-8.313,.756)--(-8.374,.744)--(-8.366,.697)--cycle;
\filldraw[fill opacity=0.8,fill=gray!20,draw=none](-8.486,.696)--(-8.482,.7)--(-8.473,.742)--(-8.532,.746)--(-8.542,.7)--cycle;
\draw(-8.473,.742)--(-8.532,.746)--(-8.542,.7)--(-8.486,.696);
\filldraw[fill opacity=0.8,fill=gray!20,draw=none](-9.399,1.051)--(-9.343,1.064)--(-9.349,1.078)--(-9.385,1.07)--cycle;
\draw(-9.399,1.051)--(-9.343,1.064);
\draw(-9.349,1.078)--(-9.385,1.07);
\filldraw[fill opacity=0.8,fill=gray!20,draw=none](-9.379,1.052)--(-9.351,1.081)--(-9.368,1.04)--cycle;
\draw(-9.351,1.081)--(-9.368,1.04)--(-9.379,1.052);
\filldraw[fill opacity=0.8,fill=gray!20,draw=none](-9.356,1.077)--(-9.347,1.092)--(-9.346,1.092)--(-9.351,1.081)--cycle;
\draw(-9.347,1.092)--(-9.346,1.092)--(-9.351,1.081);
\filldraw[fill opacity=0.8,fill=gray!20,draw=none](-9.365,1.069)--(-9.381,1.032)--(-9.368,1.04)--(-9.346,1.092)--(-9.356,1.085)--cycle;
\draw(-9.381,1.032)--(-9.368,1.04)--(-9.346,1.092)--(-9.356,1.085);
\filldraw[fill opacity=0.8,fill=gray!20,draw=none](-9.427,1.06)--(-9.436,1.051)--(-9.439,1.053)--cycle;
\draw(-9.436,1.051)--(-9.439,1.053);
\filldraw[fill opacity=0.8,fill=gray!20,draw=none](-9.398,1.091)--(-9.396,1.089)--(-9.425,1.046)--(-9.436,1.051)--cycle;
\draw(-9.398,1.091)--(-9.396,1.089);
\draw(-9.425,1.046)--(-9.436,1.051);
\filldraw[fill opacity=0.8,fill=gray!20,draw=none](-9.396,1.089)--(-9.391,1.087)--(-9.394,1.08)--(-9.42,1.044)--(-9.425,1.046)--cycle;
\draw(-9.396,1.089)--(-9.391,1.087);
\draw(-9.42,1.044)--(-9.425,1.046);
\filldraw[fill opacity=0.8,fill=gray!20,draw=none](-9.463,1.109)--(-9.443,1.156)--(-9.499,1.124)--cycle;
\draw(-9.499,1.124)--(-9.463,1.109)--(-9.443,1.156);
\filldraw[fill opacity=0.8,fill=gray!20,draw=none](-9.428,1.094)--(-9.392,1.102)--(-9.388,1.117)--(-9.499,1.124)--cycle;
\draw(-9.428,1.094)--(-9.392,1.102);
\filldraw[fill opacity=0.8,fill=gray!20,draw=none](-9.499,1.078)--(-9.428,1.094)--(-9.499,1.124)--cycle;
\draw(-9.499,1.124)--(-9.499,1.078)--(-9.428,1.094);
\filldraw[fill opacity=0.8,fill=gray!20](-9.523,1.08)--(-8.41,.616)--(-8.386,.66)--(-9.499,1.124)--cycle;
\filldraw[fill opacity=0.5,fill=gray!20](-8.631,-.371)--(-8.723,-.207)--(-9.079,-.407)--(-9.031,-.596)--cycle;
\filldraw[fill opacity=0.5,fill=gray!20](-8.541,-.463)--(-8.631,-.371)--(-9.031,-.596)--(-8.961,-.699)--cycle;
\filldraw[fill opacity=0.5,fill=gray!20](-8.456,-.54)--(-8.541,-.463)--(-8.961,-.699)--(-8.889,-.783)--cycle;
\filldraw[fill opacity=0.5,fill=gray!20](-8.377,-.598)--(-8.456,-.54)--(-8.889,-.783)--(-8.819,-.846)--cycle;
\filldraw[fill opacity=0.5,fill=gray!20](-8.308,-.637)--(-8.377,-.598)--(-8.819,-.846)--(-8.753,-.886)--cycle;
\filldraw[fill opacity=0.5,fill=gray!20](-8.25,-.653)--(-8.308,-.637)--(-8.753,-.886)--(-8.693,-.902)--cycle;
\filldraw[fill opacity=0.5,fill=gray!20](-8.169,-.573)--(-8.631,-.371)--(-9.031,-.596)--(-8.57,-.798)--cycle;
\filldraw[fill opacity=0.8,fill=gray!20,draw=none](-9.347,1.092)--(-9.346,1.094)--(-9.346,1.092)--cycle;
\draw(-9.346,1.094)--(-9.346,1.092)--(-9.347,1.092);
\filldraw[fill opacity=0.8,fill=gray!20,draw=none](-9.349,1.114)--(-9.356,1.085)--(-9.351,1.089)--cycle;
\draw(-9.356,1.085)--(-9.351,1.089);
\filldraw[fill opacity=0.8,fill=gray!20,draw=none](-9.349,1.111)--(-9.341,1.113)--(-9.349,1.114)--cycle;
\draw(-9.349,1.111)--(-9.341,1.113);
\filldraw[fill opacity=0.8,fill=gray!20,draw=none](-9.392,1.102)--(-9.349,1.111)--(-9.349,1.114)--(-9.388,1.117)--cycle;
\draw(-9.392,1.102)--(-9.349,1.111);
\filldraw[fill opacity=0.8,fill=gray!20,draw=none](-9.389,1.108)--(-9.391,1.087)--(-9.398,1.091)--cycle;
\draw(-9.391,1.087)--(-9.398,1.091);
\filldraw[fill opacity=0.8,fill=gray!20,draw=none](-9.388,1.117)--(-9.389,1.108)--(-9.392,1.102)--cycle;
\filldraw[fill opacity=0.8,fill=gray!20](-9.499,1.124)--(-8.386,.66)--(-8.371,.708)--(-9.484,1.173)--cycle;
\filldraw[fill opacity=0.8,fill=gray!20,draw=none](-8.573,1.274)--(-8.583,1.282)--(-8.597,1.302)--(-8.596,1.302)--cycle;
\draw(-8.597,1.302)--(-8.596,1.302);
\filldraw[fill opacity=0.8,fill=gray!20,draw=none](-8.455,.562)--(-8.455,.577)--(-8.541,.601)--(-8.532,.568)--cycle;
\draw(-8.541,.601)--(-8.532,.568)--(-8.455,.562)--(-8.455,.577);
\filldraw[fill opacity=0.8,fill=gray!20,draw=none](-8.725,1.273)--(-8.597,1.302)--(-8.572,1.267)--(-8.739,1.229)--cycle;
\draw(-8.725,1.273)--(-8.597,1.302);
\draw(-8.572,1.267)--(-8.739,1.229);
\filldraw[fill opacity=0.8,fill=gray!20,draw=none](-8.304,1.505)--(-8.303,1.505)--(-8.303,1.503)--cycle;
\draw(-8.304,1.505)--(-8.303,1.505)--(-8.303,1.503);
\filldraw[fill opacity=0.8,fill=gray!20,draw=none](-8.283,1.461)--(-8.303,1.503)--(-8.303,1.505)--(-8.297,1.498)--cycle;
\draw(-8.303,1.503)--(-8.303,1.505)--(-8.297,1.498);
\filldraw[fill opacity=0.8,fill=gray!20,draw=none](-8.304,1.505)--(-8.305,1.506)--(-8.303,1.505)--cycle;
\draw(-8.303,1.505)--(-8.304,1.505);
\filldraw[fill opacity=0.8,fill=gray!20,draw=none](-8.305,1.506)--(-8.325,1.528)--(-8.319,1.526)--(-8.303,1.505)--cycle;
\draw(-8.319,1.526)--(-8.303,1.505);
\filldraw[fill opacity=0.8,fill=gray!20](-8.303,1.505)--(-8.329,1.539)--(-8.316,1.525)--(-8.287,1.487)--cycle;
\filldraw[fill opacity=0.8,fill=gray!20,draw=none](-8.274,1.41)--(-8.282,1.458)--(-8.276,1.451)--cycle;
\draw(-8.282,1.458)--(-8.276,1.451);
\filldraw[fill opacity=0.8,fill=gray!20,draw=none](-8.274,1.41)--(-8.276,1.451)--(-8.269,1.444)--(-8.262,1.398)--cycle;
\draw(-8.276,1.451)--(-8.269,1.444)--(-8.262,1.398)--(-8.274,1.41);
\filldraw[fill opacity=0.8,fill=gray!20](-8.521,1.566)--(-8.485,1.578)--(-8.466,1.582)--(-8.486,1.573)--cycle;
\filldraw[fill opacity=0.8,fill=gray!20,draw=none](-8.409,.786)--(-8.387,.787)--(-8.404,.823)--(-8.451,.82)--(-8.451,.811)--cycle;
\draw(-8.409,.786)--(-8.387,.787)--(-8.404,.823)--(-8.451,.82)--(-8.451,.811);
\filldraw[fill opacity=0.8,fill=gray!20,draw=none](-8.466,.795)--(-8.457,.821)--(-8.475,.822)--cycle;
\draw(-8.457,.821)--(-8.475,.822);
\filldraw[fill opacity=0.8,fill=gray!20,draw=none](-8.558,.611)--(-8.542,.607)--(-8.544,.622)--cycle;
\draw(-8.558,.611)--(-8.542,.607)--(-8.544,.622);
\filldraw[fill opacity=0.8,fill=gray!20,draw=none](-8.281,1.457)--(-8.282,1.458)--(-8.283,1.461)--cycle;
\draw(-8.281,1.457)--(-8.282,1.458);
\filldraw[fill opacity=0.8,fill=gray!20](-8.374,.565)--(-8.366,.605)--(-8.456,.601)--(-8.455,.562)--cycle;
\filldraw[fill opacity=0.8,fill=gray!20](-8.574,1.528)--(-8.536,1.556)--(-8.521,1.566)--(-8.553,1.542)--cycle;
\filldraw[fill opacity=0.8,fill=gray!20](-8.487,1.223)--(-8.527,1.236)--(-8.536,1.247)--(-8.492,1.228)--cycle;
\filldraw[fill opacity=0.8,fill=gray!20](-8.445,1.221)--(-8.487,1.223)--(-8.492,1.228)--(-8.445,1.221)--cycle;
\filldraw[fill opacity=0.8,fill=gray!20,draw=none](-8.281,1.457)--(-8.283,1.461)--(-8.297,1.498)--(-8.287,1.487)--(-8.269,1.444)--cycle;
\draw(-8.297,1.498)--(-8.287,1.487)--(-8.269,1.444)--(-8.281,1.457);
\filldraw[fill opacity=0.8,fill=gray!20,draw=none](-8.395,1.571)--(-8.399,1.573)--(-8.4,1.576)--(-8.363,1.564)--cycle;
\draw(-8.4,1.576)--(-8.363,1.564)--(-8.395,1.571)--(-8.399,1.573);
\filldraw[fill opacity=0.8,fill=gray!20](-8.445,1.221)--(-8.406,1.222)--(-8.424,1.218)--(-8.445,1.221)--cycle;
\filldraw[fill opacity=0.8,fill=gray!20](-8.445,1.221)--(-8.398,1.227)--(-8.406,1.222)--(-8.445,1.221)--cycle;
\filldraw[fill opacity=0.8,fill=gray!20](-8.445,1.221)--(-8.471,1.219)--(-8.487,1.223)--(-8.445,1.221)--cycle;
\filldraw[fill opacity=0.8,fill=gray!20](-8.445,1.221)--(-8.424,1.218)--(-8.448,1.217)--(-8.445,1.221)--cycle;
\filldraw[fill opacity=0.8,fill=gray!20](-8.445,1.221)--(-8.448,1.217)--(-8.471,1.219)--(-8.445,1.221)--cycle;
\filldraw[fill opacity=0.8,fill=gray!20,draw=none](-8.466,.795)--(-8.462,.784)--(-8.453,.784)--(-8.451,.82)--(-8.457,.821)--cycle;
\draw(-8.462,.784)--(-8.453,.784)--(-8.451,.82)--(-8.457,.821);
\filldraw[fill opacity=0.8,fill=gray!20](-8.398,1.227)--(-8.354,1.244)--(-8.369,1.234)--(-8.406,1.222)--cycle;
\filldraw[fill opacity=0.8,fill=gray!20,draw=none](-8.587,1.264)--(-8.566,1.268)--(-8.544,1.249)--cycle;
\draw(-8.587,1.264)--(-8.566,1.268);
\filldraw[fill opacity=0.8,fill=gray!20,draw=none](-8.566,1.268)--(-8.544,1.249)--(-8.561,1.261)--(-8.573,1.274)--cycle;
\draw(-8.544,1.249)--(-8.561,1.261)--(-8.573,1.274);
\filldraw[fill opacity=0.8,fill=gray!20](-8.561,1.261)--(-8.587,1.295)--(-8.603,1.313)--(-8.574,1.275)--cycle;
\filldraw[fill opacity=0.8,fill=gray!20,draw=none](-8.607,1.485)--(-8.602,1.49)--(-8.586,1.503)--(-8.571,1.506)--(-8.546,1.496)--(-8.608,1.482)--cycle;
\draw(-8.586,1.503)--(-8.571,1.506);
\draw(-8.546,1.496)--(-8.608,1.482);
\filldraw[fill opacity=0.8,fill=gray!20,draw=none](-8.603,1.484)--(-8.61,1.456)--(-8.621,1.449)--(-8.605,1.488)--cycle;
\draw(-8.61,1.456)--(-8.621,1.449)--(-8.605,1.488);
\filldraw[fill opacity=0.8,fill=gray!20,draw=none](-8.583,1.282)--(-8.573,1.274)--(-8.568,1.268)--(-8.572,1.267)--cycle;
\draw(-8.568,1.268)--(-8.572,1.267);
\filldraw[fill opacity=0.8,fill=gray!20,draw=none](-8.52,1.235)--(-8.521,1.249)--(-8.544,1.249)--(-8.527,1.236)--cycle;
\draw(-8.544,1.249)--(-8.527,1.236)--(-8.52,1.235);
\filldraw[fill opacity=0.8,fill=gray!20,draw=none](-9.094,1.125)--(-8.433,1.273)--(-8.411,1.273)--(-9.03,1.134)--cycle;
\draw(-9.094,1.125)--(-8.433,1.273)--(-8.411,1.273)--(-9.03,1.134);
\filldraw[fill opacity=0.8,fill=gray!20,draw=none](-8.399,1.573)--(-8.419,1.581)--(-8.403,1.577)--(-8.4,1.576)--cycle;
\draw(-8.399,1.573)--(-8.419,1.581)--(-8.403,1.577)--(-8.4,1.576);
\filldraw[fill opacity=0.8,fill=gray!20,draw=none](-8.443,1.272)--(-8.442,1.271)--(-8.494,1.26)--cycle;
\draw(-8.442,1.271)--(-8.494,1.26);
\filldraw[fill opacity=0.8,fill=gray!20,draw=none](-8.455,.577)--(-8.456,.601)--(-8.542,.607)--(-8.541,.601)--cycle;
\draw(-8.455,.577)--(-8.456,.601)--(-8.542,.607)--(-8.541,.601);
\filldraw[fill opacity=0.8,fill=gray!20](-8.587,1.295)--(-8.603,1.337)--(-8.621,1.356)--(-8.603,1.313)--cycle;
\filldraw[fill opacity=0.8,fill=gray!20](-8.354,1.244)--(-8.316,1.272)--(-8.337,1.258)--(-8.369,1.234)--cycle;
\filldraw[fill opacity=0.8,fill=gray!20,draw=none](-8.409,.786)--(-8.451,.811)--(-8.453,.784)--cycle;
\draw(-8.451,.811)--(-8.453,.784)--(-8.409,.786);
\filldraw[fill opacity=0.8,fill=gray!20](-8.329,1.539)--(-8.363,1.564)--(-8.354,1.553)--(-8.316,1.525)--cycle;
\filldraw[fill opacity=0.8,fill=gray!20,draw=none](-8.456,.601)--(-8.456,.634)--(-8.539,.623)--(-8.543,.618)--(-8.542,.607)--cycle;
\draw(-8.543,.618)--(-8.542,.607)--(-8.456,.601)--(-8.456,.634);
\filldraw[fill opacity=0.8,fill=gray!20,draw=none](-8.539,.623)--(-8.544,.622)--(-8.543,.618)--cycle;
\draw(-8.544,.622)--(-8.543,.618);
\filldraw[fill opacity=0.8,fill=gray!20](-8.603,1.337)--(-8.609,1.382)--(-8.628,1.402)--(-8.621,1.356)--cycle;
\filldraw[fill opacity=0.8,fill=gray!20](-8.316,1.272)--(-8.287,1.309)--(-8.313,1.292)--(-8.337,1.258)--cycle;
\filldraw[fill opacity=0.8,fill=gray!20,draw=none](-8.539,.623)--(-8.509,.627)--(-8.507,.65)--(-8.512,.65)--cycle;
\draw(-8.507,.65)--(-8.512,.65);
\filldraw[fill opacity=0.8,fill=gray!20,draw=none](-8.429,.742)--(-8.374,.744)--(-8.387,.787)--(-8.453,.784)--(-8.453,.769)--cycle;
\draw(-8.429,.742)--(-8.374,.744)--(-8.387,.787)--(-8.453,.784)--(-8.453,.769);
\filldraw[fill opacity=0.8,fill=gray!20,draw=none](-8.469,.785)--(-8.462,.784)--(-8.466,.795)--cycle;
\draw(-8.469,.785)--(-8.462,.784);
\filldraw[fill opacity=0.8,fill=gray!20,draw=none](-8.425,1.274)--(-8.423,1.273)--(-8.425,1.273)--cycle;
\draw(-8.423,1.273)--(-8.425,1.273);
\filldraw[fill opacity=0.8,fill=gray!20,draw=none](-8.386,1.328)--(-8.395,1.292)--(-8.411,1.273)--(-8.423,1.273)--(-8.425,1.274)--(-8.465,1.526)--(-8.456,1.526)--(-8.434,1.507)--(-8.412,1.471)--(-8.395,1.425)--(-8.386,1.374)--cycle;
\draw(-8.456,1.526)--(-8.434,1.507)--(-8.412,1.471)--(-8.395,1.425)--(-8.386,1.374)--(-8.386,1.328)--(-8.395,1.292)--(-8.411,1.273)--(-8.423,1.273);
\filldraw[fill opacity=0.8,fill=gray!20](-8.471,1.219)--(-8.495,1.229)--(-8.527,1.236)--(-8.487,1.223)--cycle;
\filldraw[fill opacity=0.5,fill=gray!20,draw=none](-7.994,1.348)--(-7.828,1.33)--(-7.811,1.193)--(-7.984,1.269)--cycle;
\draw(-7.828,1.33)--(-7.811,1.193)--(-7.984,1.269)--(-7.994,1.348);
\filldraw[fill opacity=0.8,fill=gray!20](-8.366,.605)--(-8.363,.65)--(-8.456,.646)--(-8.456,.601)--cycle;
\filldraw[fill opacity=0.8,fill=gray!20,draw=none](-9.145,1.127)--(-9.155,1.136)--(-8.587,1.264)--(-8.544,1.249)--(-9.094,1.125)--cycle;
\draw(-9.155,1.136)--(-8.587,1.264);
\draw(-8.544,1.249)--(-9.094,1.125);
\filldraw[fill opacity=0.8,fill=gray!20](-8.536,1.556)--(-8.492,1.573)--(-8.485,1.578)--(-8.521,1.566)--cycle;
\filldraw[fill opacity=0.8,fill=gray!20](-8.406,1.222)--(-8.369,1.234)--(-8.404,1.227)--(-8.424,1.218)--cycle;
\filldraw[fill opacity=0.8,fill=gray!20,draw=none](-8.469,.749)--(-8.453,.778)--(-8.453,.784)--(-8.469,.785)--cycle;
\draw(-8.453,.778)--(-8.453,.784)--(-8.469,.785);
\filldraw[fill opacity=0.8,fill=gray!20](-8.609,1.382)--(-8.603,1.429)--(-8.621,1.449)--(-8.628,1.402)--cycle;
\filldraw[fill opacity=0.8,fill=gray!20](-8.287,1.309)--(-8.269,1.351)--(-8.298,1.333)--(-8.313,1.292)--cycle;
\filldraw[fill opacity=0.8,fill=gray!20,draw=none](-9.394,1.321)--(-8.602,1.499)--(-8.608,1.482)--(-9.436,1.296)--cycle;
\draw(-9.394,1.321)--(-8.602,1.499);
\draw(-8.608,1.482)--(-9.436,1.296);
\filldraw[fill opacity=0.8,fill=gray!20](-8.603,1.429)--(-8.587,1.474)--(-8.603,1.491)--(-8.621,1.449)--cycle;
\filldraw[fill opacity=0.8,fill=gray!20,draw=none](-8.587,1.474)--(-8.569,1.502)--(-8.594,1.503)--(-8.603,1.491)--cycle;
\draw(-8.594,1.503)--(-8.603,1.491)--(-8.587,1.474)--(-8.569,1.502);
\filldraw[fill opacity=0.8,fill=gray!20,draw=none](-8.602,1.49)--(-8.607,1.487)--(-8.602,1.499)--(-8.591,1.502)--cycle;
\draw(-8.602,1.499)--(-8.591,1.502);
\filldraw[fill opacity=0.8,fill=gray!20,draw=none](-8.602,1.49)--(-8.591,1.502)--(-8.586,1.503)--cycle;
\draw(-8.591,1.502)--(-8.586,1.503);
\filldraw[fill opacity=0.8,fill=gray!20,draw=none](-9.447,1.296)--(-9.394,1.321)--(-9.436,1.296)--(-9.452,1.293)--cycle;
\draw(-9.436,1.296)--(-9.452,1.293);
\filldraw[fill opacity=0.8,fill=gray!20,draw=none](-9.452,1.291)--(-9.452,1.293)--(-9.451,1.293)--cycle;
\draw(-9.452,1.293)--(-9.451,1.293);
\filldraw[fill opacity=0.8,fill=gray!20,draw=none](-9.447,1.296)--(-9.452,1.293)--(-9.457,1.292)--cycle;
\draw(-9.452,1.293)--(-9.457,1.292);
\filldraw[fill opacity=0.8,fill=gray!20,draw=none](-9.449,1.276)--(-9.453,1.29)--(-9.452,1.292)--(-9.443,1.289)--(-9.417,1.27)--cycle;
\draw(-9.449,1.276)--(-9.453,1.29);
\draw(-9.452,1.292)--(-9.443,1.289);
\filldraw[fill opacity=0.8,fill=gray!20,draw=none](-9.452,1.291)--(-9.453,1.29)--(-9.452,1.293)--cycle;
\filldraw[fill opacity=0.8,fill=gray!20,draw=none](-9.453,1.29)--(-9.453,1.292)--(-9.452,1.292)--cycle;
\draw(-9.453,1.29)--(-9.453,1.292)--(-9.452,1.292);
\filldraw[fill opacity=0.8,fill=gray!20,draw=none](-9.443,1.289)--(-9.453,1.292)--(-9.452,1.294)--cycle;
\draw(-9.443,1.289)--(-9.453,1.292);
\filldraw[fill opacity=0.8,fill=gray!20,draw=none](-9.465,1.275)--(-9.457,1.292)--(-9.394,1.321)--(-9.426,1.249)--cycle;
\draw(-9.465,1.275)--(-9.457,1.292);
\draw(-9.394,1.321)--(-9.426,1.249);
\filldraw[fill opacity=0.8,fill=gray!20,draw=none](-9.442,1.31)--(-9.423,1.315)--(-9.447,1.296)--(-9.452,1.294)--cycle;
\draw(-9.442,1.31)--(-9.423,1.315);
\filldraw[fill opacity=0.8,fill=gray!20,draw=none](-9.423,1.315)--(-9.394,1.321)--(-9.447,1.296)--cycle;
\draw(-9.423,1.315)--(-9.394,1.321);
\filldraw[fill opacity=0.8,fill=gray!20,draw=none](-9.417,1.27)--(-9.404,1.3)--(-9.383,1.264)--cycle;
\draw(-9.417,1.27)--(-9.404,1.3);
\filldraw[fill opacity=0.8,fill=gray!20,draw=none](-9.404,1.3)--(-9.394,1.321)--(-9.357,1.325)--(-9.383,1.264)--cycle;
\draw(-9.404,1.3)--(-9.394,1.321);
\draw(-9.357,1.325)--(-9.383,1.264);
\filldraw[fill opacity=0.8,fill=gray!20,draw=none](-9.443,1.289)--(-9.452,1.294)--(-9.442,1.31)--(-9.415,1.311)--(-9.388,1.276)--cycle;
\draw(-9.415,1.311)--(-9.388,1.276)--(-9.443,1.289);
\filldraw[fill opacity=0.8,fill=gray!20,draw=none](-9.57,1.277)--(-8.515,1.514)--(-8.486,1.525)--(-9.592,1.277)--cycle;
\draw(-8.486,1.525)--(-9.592,1.277)--(-9.57,1.277)--(-8.515,1.514);
\filldraw[fill opacity=0.8,fill=gray!20](-8.442,1.583)--(-8.445,1.579)--(-8.445,1.579)--(-8.419,1.581)--cycle;
\filldraw[fill opacity=0.8,fill=gray!20](-8.466,1.582)--(-8.445,1.579)--(-8.445,1.579)--(-8.442,1.583)--cycle;
\filldraw[fill opacity=0.8,fill=gray!20,draw=none](-8.509,.627)--(-8.456,.634)--(-8.456,.646)--(-8.507,.65)--cycle;
\draw(-8.456,.634)--(-8.456,.646)--(-8.507,.65);
\filldraw[fill opacity=0.8,fill=gray!20,draw=none](-8.363,1.564)--(-8.4,1.576)--(-8.402,1.576)--(-8.398,1.572)--(-8.354,1.553)--cycle;
\draw(-8.402,1.576)--(-8.398,1.572)--(-8.354,1.553)--(-8.363,1.564)--(-8.4,1.576);
\filldraw[fill opacity=0.8,fill=gray!20,draw=none](-8.469,.749)--(-8.469,.742)--(-8.455,.74)--(-8.453,.778)--cycle;
\draw(-8.469,.742)--(-8.455,.74)--(-8.453,.778);
\filldraw[fill opacity=0.8,fill=gray!20](-8.485,1.578)--(-8.445,1.579)--(-8.445,1.579)--(-8.466,1.582)--cycle;
\filldraw[fill opacity=0.8,fill=gray!20](-8.366,.697)--(-8.374,.744)--(-8.455,.74)--(-8.456,.693)--cycle;
\filldraw[fill opacity=0.8,fill=gray!20](-8.419,1.581)--(-8.445,1.579)--(-8.445,1.579)--(-8.403,1.577)--cycle;
\filldraw[fill opacity=0.8,fill=gray!20](-8.363,.65)--(-8.366,.697)--(-8.456,.693)--(-8.456,.646)--cycle;
\filldraw[fill opacity=0.8,fill=gray!20](-8.269,1.351)--(-8.262,1.398)--(-8.292,1.378)--(-8.298,1.333)--cycle;
\filldraw[fill opacity=0.8,fill=gray!20](-8.448,1.217)--(-8.451,1.225)--(-8.495,1.229)--(-8.471,1.219)--cycle;
\filldraw[fill opacity=0.8,fill=gray!20,draw=none](-8.504,.655)--(-8.512,.65)--(-8.507,.65)--cycle;
\draw(-8.512,.65)--(-8.507,.65);
\filldraw[fill opacity=0.8,fill=gray!20,draw=none](-8.429,.742)--(-8.453,.769)--(-8.455,.74)--cycle;
\draw(-8.453,.769)--(-8.455,.74)--(-8.429,.742);
\filldraw[fill opacity=0.8,fill=gray!20,draw=none](-8.504,.655)--(-8.507,.65)--(-8.456,.646)--(-8.456,.686)--cycle;
\draw(-8.507,.65)--(-8.456,.646)--(-8.456,.686);
\filldraw[fill opacity=0.8,fill=gray!20](-8.424,1.218)--(-8.404,1.227)--(-8.451,1.225)--(-8.448,1.217)--cycle;
\filldraw[fill opacity=0.8,fill=gray!20,draw=none](-8.504,.655)--(-8.481,.67)--(-8.483,.695)--(-8.486,.696)--cycle;
\draw(-8.483,.695)--(-8.486,.696);
\filldraw[fill opacity=0.8,fill=gray!20,draw=none](-8.491,1.523)--(-8.497,1.521)--(-8.472,1.526)--cycle;
\filldraw[fill opacity=0.8,fill=gray!20,draw=none](-8.602,1.49)--(-8.607,1.485)--(-8.607,1.487)--cycle;
\filldraw[fill opacity=0.5,fill=gray!20,draw=none](-7.857,1.333)--(-7.994,1.348)--(-8.006,1.446)--cycle;
\draw(-7.994,1.348)--(-8.006,1.446);
\filldraw[fill opacity=0.8,fill=gray!20,draw=none](-8.473,.742)--(-8.469,.742)--(-8.469,.749)--cycle;
\draw(-8.473,.742)--(-8.469,.742);
\filldraw[fill opacity=0.8,fill=gray!20,draw=none](-9.203,1.166)--(-8.725,1.273)--(-8.739,1.229)--(-9.155,1.136)--cycle;
\draw(-9.203,1.166)--(-8.725,1.273);
\draw(-8.739,1.229)--(-9.155,1.136);
\filldraw[fill opacity=0.8,fill=gray!20,draw=none](-8.482,.7)--(-8.455,.733)--(-8.455,.74)--(-8.473,.742)--cycle;
\draw(-8.455,.733)--(-8.455,.74)--(-8.473,.742);
\filldraw[fill opacity=0.8,fill=gray!20](-8.262,1.398)--(-8.269,1.444)--(-8.298,1.425)--(-8.292,1.378)--cycle;
\filldraw[fill opacity=0.8,fill=gray!20,draw=none](-8.521,1.249)--(-8.521,1.251)--(-8.561,1.261)--(-8.544,1.249)--cycle;
\draw(-8.521,1.251)--(-8.561,1.261)--(-8.544,1.249);
\filldraw[fill opacity=0.8,fill=gray!20,draw=none](-8.521,1.251)--(-8.519,1.255)--(-8.532,1.282)--(-8.587,1.295)--(-8.561,1.261)--cycle;
\draw(-8.519,1.255)--(-8.532,1.282)--(-8.587,1.295)--(-8.561,1.261)--(-8.521,1.251);
\filldraw[fill opacity=0.8,fill=gray!20,draw=none](-8.501,1.269)--(-8.501,1.28)--(-8.532,1.282)--(-8.519,1.255)--cycle;
\draw(-8.501,1.28)--(-8.532,1.282)--(-8.519,1.255);
\filldraw[fill opacity=0.8,fill=gray!20,draw=none](-9.03,1.134)--(-8.411,1.273)--(-8.395,1.292)--(-9.006,1.155)--cycle;
\draw(-9.03,1.134)--(-8.411,1.273)--(-8.395,1.292)--(-9.006,1.155);
\filldraw[fill opacity=0.8,fill=gray!20,draw=none](-8.481,.67)--(-8.456,.686)--(-8.456,.693)--(-8.483,.695)--cycle;
\draw(-8.456,.686)--(-8.456,.693)--(-8.483,.695);
\filldraw[fill opacity=0.8,fill=gray!20,draw=none](-8.482,.7)--(-8.486,.696)--(-8.483,.695)--cycle;
\draw(-8.486,.696)--(-8.483,.695);
\filldraw[fill opacity=0.8,fill=gray!20](-8.492,1.573)--(-8.445,1.579)--(-8.445,1.579)--(-8.485,1.578)--cycle;
\filldraw[fill opacity=0.8,fill=gray!20,draw=none](-8.482,.7)--(-8.483,.695)--(-8.456,.693)--(-8.455,.733)--cycle;
\draw(-8.483,.695)--(-8.456,.693)--(-8.455,.733);
\filldraw[fill opacity=0.8,fill=gray!20,draw=none](-8.52,1.235)--(-8.495,1.229)--(-8.515,1.249)--(-8.521,1.249)--cycle;
\draw(-8.52,1.235)--(-8.495,1.229)--(-8.515,1.249);
\filldraw[fill opacity=0.8,fill=gray!20,draw=none](-8.497,1.521)--(-8.515,1.514)--(-8.457,1.527)--(-8.467,1.527)--cycle;
\draw(-8.515,1.514)--(-8.457,1.527)--(-8.467,1.527);
\filldraw[fill opacity=0.8,fill=gray!20,draw=none](-8.465,1.526)--(-8.467,1.527)--(-8.465,1.527)--cycle;
\draw(-8.467,1.527)--(-8.465,1.527);
\filldraw[fill opacity=0.8,fill=gray!20,draw=none](-8.569,1.502)--(-8.561,1.514)--(-8.574,1.528)--(-8.594,1.503)--cycle;
\draw(-8.569,1.502)--(-8.561,1.514)--(-8.574,1.528)--(-8.594,1.503);
\filldraw[fill opacity=0.8,fill=gray!20](-8.369,1.234)--(-8.337,1.258)--(-8.387,1.248)--(-8.404,1.227)--cycle;
\filldraw[fill opacity=0.8,fill=gray!20,draw=none](-8.4,1.576)--(-8.403,1.577)--(-8.402,1.576)--cycle;
\draw(-8.4,1.576)--(-8.403,1.577)--(-8.402,1.576);
\filldraw[fill opacity=0.8,fill=gray!20](-8.403,1.577)--(-8.445,1.579)--(-8.445,1.579)--(-8.398,1.572)--cycle;
\filldraw[fill opacity=0.8,fill=gray!20,draw=none](-8.465,1.526)--(-8.465,1.527)--(-8.457,1.527)--(-8.456,1.526)--cycle;
\draw(-8.465,1.527)--(-8.457,1.527)--(-8.456,1.526);
\filldraw[fill opacity=0.8,fill=gray!20](-8.561,1.514)--(-8.527,1.546)--(-8.536,1.556)--(-8.574,1.528)--cycle;
\filldraw[fill opacity=0.8,fill=gray!20](-8.269,1.444)--(-8.287,1.487)--(-8.313,1.47)--(-8.298,1.425)--cycle;
\filldraw[fill opacity=0.5,fill=gray!20](-8.022,2.171)--(-8.182,2.053)--(-8.035,1.681)--(-7.856,1.752)--cycle;
\filldraw[fill opacity=0.5,fill=gray!20](-7.904,2.2)--(-8.022,2.171)--(-7.856,1.752)--(-7.731,1.761)--cycle;
\filldraw[fill opacity=0.5,fill=gray!20](-7.798,2.212)--(-7.904,2.2)--(-7.731,1.761)--(-7.618,1.759)--cycle;
\filldraw[fill opacity=0.5,fill=gray!20](-7.706,2.209)--(-7.798,2.212)--(-7.618,1.759)--(-7.522,1.746)--cycle;
\filldraw[fill opacity=0.5,fill=gray!20](-7.632,2.189)--(-7.706,2.209)--(-7.522,1.746)--(-7.447,1.723)--cycle;
\filldraw[fill opacity=0.5,fill=gray!20](-7.561,1.97)--(-8.022,2.171)--(-7.856,1.752)--(-7.395,1.551)--cycle;
\filldraw[fill opacity=0.8,fill=gray!20](-8.527,1.546)--(-8.487,1.568)--(-8.492,1.573)--(-8.536,1.556)--cycle;
\filldraw[fill opacity=0.8,fill=gray!20](-8.287,1.487)--(-8.316,1.525)--(-8.337,1.511)--(-8.313,1.47)--cycle;
\filldraw[fill opacity=0.5,fill=gray!20](-8.723,-.207)--(-8.55,-.283)--(-8.906,-.482)--(-9.079,-.407)--cycle;
\filldraw[fill opacity=0.5,fill=gray!20,draw=none](-7.857,1.333)--(-8.006,1.446)--(-8.035,1.681)--(-7.862,1.605)--(-7.828,1.33)--cycle;
\draw(-8.006,1.446)--(-8.035,1.681)--(-7.862,1.605)--(-7.828,1.33);
\filldraw[fill opacity=0.8,fill=gray!20,draw=none](-8.536,1.461)--(-8.541,1.483)--(-8.569,1.502)--(-8.587,1.474)--cycle;
\draw(-8.569,1.502)--(-8.587,1.474)--(-8.536,1.461);
\filldraw[fill opacity=0.8,fill=gray!20,draw=none](-8.542,1.414)--(-8.533,1.457)--(-8.536,1.461)--(-8.587,1.474)--(-8.603,1.429)--cycle;
\draw(-8.536,1.461)--(-8.587,1.474)--(-8.603,1.429)--(-8.542,1.414)--(-8.533,1.457);
\filldraw[fill opacity=0.8,fill=gray!20,draw=none](-9.48,1.283)--(-9.457,1.292)--(-9.465,1.275)--cycle;
\draw(-9.457,1.292)--(-9.465,1.275);
\filldraw[fill opacity=0.8,fill=gray!20,draw=none](-9.488,1.285)--(-9.457,1.292)--(-9.475,1.28)--cycle;
\draw(-9.488,1.285)--(-9.457,1.292);
\filldraw[fill opacity=0.8,fill=gray!20,draw=none](-9.396,1.255)--(-9.443,1.289)--(-9.388,1.276)--(-9.383,1.264)--cycle;
\draw(-9.443,1.289)--(-9.388,1.276)--(-9.383,1.264);
\filldraw[fill opacity=0.8,fill=gray!20,draw=none](-9.499,1.124)--(-9.617,1.176)--(-9.572,1.156)--(-9.525,1.136)--(-9.483,1.117)--(-9.452,1.104)--(-9.438,1.098)--(-9.442,1.099)--(-9.463,1.109)--cycle;
\draw(-9.617,1.176)--(-9.572,1.156)--(-9.525,1.136)--(-9.483,1.117)--(-9.452,1.104)--(-9.438,1.098)--(-9.442,1.099)--(-9.463,1.109)--(-9.499,1.124);
\filldraw[fill opacity=0.8,fill=gray!20,draw=none](-9.499,1.124)--(-9.484,1.173)--(-9.48,1.218)--(-9.488,1.252)--(-9.507,1.271)--(-9.533,1.272)--(-9.563,1.254)--(-9.593,1.22)--(-9.617,1.176)--(-9.632,1.127)--(-9.636,1.082)--(-9.628,1.048)--(-9.611,1.031)--(-9.606,1.029)--(-9.583,1.028)--(-9.552,1.046)--(-9.523,1.08)--cycle;
\draw(-9.606,1.029)--(-9.583,1.028)--(-9.552,1.046)--(-9.523,1.08)--(-9.499,1.124)--(-9.484,1.173)--(-9.48,1.218)--(-9.488,1.252)--(-9.507,1.271)--(-9.533,1.272)--(-9.563,1.254)--(-9.593,1.22)--(-9.617,1.176)--(-9.632,1.127)--(-9.636,1.082)--(-9.628,1.048)--(-9.611,1.031);
\filldraw[fill opacity=0.8,fill=gray!20,draw=none](-9.396,1.255)--(-9.383,1.264)--(-9.372,1.237)--cycle;
\draw(-9.383,1.264)--(-9.372,1.237);
\filldraw[fill opacity=0.8,fill=gray!20,draw=none](-9.378,1.258)--(-9.383,1.264)--(-9.357,1.325)--(-9.349,1.301)--(-9.362,1.273)--cycle;
\draw(-9.383,1.264)--(-9.357,1.325);
\draw(-9.349,1.301)--(-9.362,1.273);
\filldraw[fill opacity=0.8,fill=gray!20,draw=none](-9.387,1.227)--(-9.393,1.232)--(-9.383,1.23)--cycle;
\draw(-9.393,1.232)--(-9.383,1.23);
\filldraw[fill opacity=0.8,fill=gray!20,draw=none](-9.387,1.227)--(-9.383,1.23)--(-9.368,1.226)--(-9.367,1.213)--cycle;
\draw(-9.383,1.23)--(-9.368,1.226)--(-9.367,1.213);
\filldraw[fill opacity=0.8,fill=gray!20,draw=none](-9.378,1.258)--(-9.362,1.273)--(-9.372,1.25)--cycle;
\draw(-9.362,1.273)--(-9.372,1.25);
\filldraw[fill opacity=0.8,fill=gray!20,draw=none](-9.371,1.258)--(-9.388,1.276)--(-9.415,1.311)--(-9.397,1.291)--(-9.372,1.259)--cycle;
\draw(-9.371,1.258)--(-9.388,1.276)--(-9.415,1.311);
\filldraw[fill opacity=0.8,fill=gray!20,draw=none](-9.558,1.15)--(-9.57,1.275)--(-9.558,1.279)--(-9.513,1.274)--(-9.507,1.27)--cycle;
\draw(-9.57,1.275)--(-9.558,1.279)--(-9.513,1.274)--(-9.507,1.27);
\filldraw[fill opacity=0.8,fill=gray!20,draw=none](-9.507,1.27)--(-9.513,1.274)--(-9.505,1.269)--cycle;
\draw(-9.507,1.27)--(-9.513,1.274);
\filldraw[fill opacity=0.8,fill=gray!20,draw=none](-9.513,1.274)--(-9.497,1.267)--(-9.454,1.241)--(-9.458,1.243)--cycle;
\draw(-9.513,1.274)--(-9.497,1.267);
\draw(-9.454,1.241)--(-9.458,1.243);
\filldraw[fill opacity=0.8,fill=gray!20,draw=none](-9.497,1.267)--(-9.486,1.262)--(-9.445,1.239)--(-9.43,1.23)--(-9.454,1.241)--cycle;
\draw(-9.497,1.267)--(-9.486,1.262);
\draw(-9.43,1.23)--(-9.454,1.241);
\filldraw[fill opacity=0.8,fill=gray!20,draw=none](-9.558,1.279)--(-9.556,1.278)--(-9.508,1.266)--(-9.486,1.262)--(-9.513,1.274)--cycle;
\draw(-9.486,1.262)--(-9.513,1.274)--(-9.558,1.279)--(-9.556,1.278);
\filldraw[fill opacity=0.8,fill=gray!20,draw=none](-9.57,1.275)--(-9.566,1.273)--(-9.556,1.278)--(-9.558,1.279)--cycle;
\draw(-9.556,1.278)--(-9.558,1.279)--(-9.57,1.275);
\filldraw[fill opacity=0.8,fill=gray!20](-9.547,1.257)--(-8.434,1.507)--(-8.457,1.527)--(-9.57,1.277)--cycle;
\filldraw[fill opacity=0.8,fill=gray!20](-8.487,1.568)--(-8.445,1.579)--(-8.445,1.579)--(-8.492,1.573)--cycle;
\filldraw[fill opacity=0.8,fill=gray!20](-8.316,1.525)--(-8.354,1.553)--(-8.369,1.544)--(-8.337,1.511)--cycle;
\filldraw[fill opacity=0.8,fill=gray!20](-8.354,1.553)--(-8.398,1.572)--(-8.406,1.567)--(-8.369,1.544)--cycle;
\filldraw[fill opacity=0.8,fill=gray!20](-8.398,1.572)--(-8.445,1.579)--(-8.445,1.579)--(-8.406,1.567)--cycle;
\filldraw[fill opacity=0.8,fill=gray!20](-8.406,1.567)--(-8.445,1.579)--(-8.445,1.579)--(-8.424,1.563)--cycle;
\filldraw[fill opacity=0.8,fill=gray!20](-8.471,1.564)--(-8.445,1.579)--(-8.445,1.579)--(-8.487,1.568)--cycle;
\filldraw[fill opacity=0.8,fill=gray!20](-8.448,1.562)--(-8.445,1.579)--(-8.445,1.579)--(-8.471,1.564)--cycle;
\filldraw[fill opacity=0.8,fill=gray!20](-8.451,1.225)--(-8.453,1.246)--(-8.516,1.25)--(-8.495,1.229)--cycle;
\filldraw[fill opacity=0.8,fill=gray!20](-8.424,1.563)--(-8.445,1.579)--(-8.445,1.579)--(-8.448,1.562)--cycle;
\filldraw[fill opacity=0.8,fill=gray!20](-8.337,1.258)--(-8.313,1.292)--(-8.374,1.28)--(-8.387,1.248)--cycle;
\filldraw[fill opacity=0.8,fill=gray!20](-8.404,1.227)--(-8.387,1.248)--(-8.453,1.246)--(-8.451,1.225)--cycle;
\filldraw[fill opacity=0.8,fill=gray!20](-8.532,1.282)--(-8.542,1.322)--(-8.603,1.337)--(-8.587,1.295)--cycle;
\filldraw[fill opacity=0.8,fill=gray!20,draw=none](-8.501,1.28)--(-8.49,1.304)--(-8.494,1.318)--(-8.542,1.322)--(-8.532,1.282)--cycle;
\draw(-8.494,1.318)--(-8.542,1.322)--(-8.532,1.282)--(-8.501,1.28);
\filldraw[fill opacity=0.8,fill=gray!20,draw=none](-9.006,1.155)--(-8.395,1.292)--(-8.386,1.328)--(-9.027,1.184)--cycle;
\draw(-9.006,1.155)--(-8.395,1.292)--(-8.386,1.328)--(-9.027,1.184);
\filldraw[fill opacity=0.8,fill=gray!20](-8.495,1.538)--(-8.471,1.564)--(-8.487,1.568)--(-8.527,1.546)--cycle;
\filldraw[fill opacity=0.8,fill=gray!20,draw=none](-8.521,1.249)--(-8.515,1.249)--(-8.516,1.25)--(-8.521,1.251)--cycle;
\draw(-8.515,1.249)--(-8.516,1.25)--(-8.521,1.251);
\filldraw[fill opacity=0.8,fill=gray!20](-8.369,1.544)--(-8.406,1.567)--(-8.424,1.563)--(-8.404,1.537)--cycle;
\filldraw[fill opacity=0.8,fill=gray!20](-8.313,1.292)--(-8.298,1.333)--(-8.366,1.319)--(-8.374,1.28)--cycle;
\filldraw[fill opacity=0.8,fill=gray!20](-8.542,1.322)--(-8.546,1.367)--(-8.609,1.382)--(-8.603,1.337)--cycle;
\filldraw[fill opacity=0.8,fill=gray!20,draw=none](-8.521,1.251)--(-8.516,1.25)--(-8.519,1.255)--cycle;
\draw(-8.521,1.251)--(-8.516,1.25)--(-8.519,1.255);
\filldraw[fill opacity=0.8,fill=gray!20,draw=none](-8.529,1.506)--(-8.513,1.508)--(-8.495,1.538)--(-8.527,1.546)--(-8.561,1.514)--cycle;
\draw(-8.513,1.508)--(-8.495,1.538)--(-8.527,1.546)--(-8.561,1.514)--(-8.529,1.506);
\filldraw[fill opacity=0.8,fill=gray!20,draw=none](-8.507,1.411)--(-8.52,1.447)--(-8.533,1.457)--(-8.542,1.414)--cycle;
\draw(-8.533,1.457)--(-8.542,1.414)--(-8.507,1.411);
\filldraw[fill opacity=0.8,fill=gray!20,draw=none](-9.414,1.239)--(-9.426,1.249)--(-9.417,1.27)--(-9.383,1.264)--(-9.39,1.248)--cycle;
\draw(-9.426,1.249)--(-9.417,1.27);
\draw(-9.383,1.264)--(-9.39,1.248);
\filldraw[fill opacity=0.8,fill=gray!20,draw=none](-9.372,1.237)--(-9.388,1.276)--(-9.371,1.258)--cycle;
\draw(-9.372,1.237)--(-9.388,1.276)--(-9.371,1.258);
\filldraw[fill opacity=0.8,fill=gray!20,draw=none](-9.399,1.227)--(-9.39,1.248)--(-9.378,1.258)--(-9.372,1.25)--(-9.389,1.211)--cycle;
\draw(-9.399,1.227)--(-9.39,1.248);
\draw(-9.372,1.25)--(-9.389,1.211);
\filldraw[fill opacity=0.8,fill=gray!20,draw=none](-9.378,1.258)--(-9.39,1.248)--(-9.383,1.264)--cycle;
\draw(-9.39,1.248)--(-9.383,1.264);
\filldraw[fill opacity=0.8,fill=gray!20,draw=none](-9.371,1.251)--(-9.37,1.255)--(-9.372,1.259)--(-9.374,1.255)--(-9.378,1.245)--cycle;
\draw(-9.371,1.251)--(-9.37,1.255);
\draw(-9.374,1.255)--(-9.378,1.245);
\filldraw[fill opacity=0.8,fill=gray!20,draw=none](-9.37,1.255)--(-9.349,1.301)--(-9.372,1.259)--cycle;
\draw(-9.37,1.255)--(-9.349,1.301);
\filldraw[fill opacity=0.8,fill=gray!20,draw=none](-9.372,1.251)--(-9.371,1.258)--(-9.368,1.255)--(-9.367,1.253)--(-9.363,1.243)--cycle;
\draw(-9.371,1.258)--(-9.368,1.255);
\draw(-9.367,1.253)--(-9.363,1.243);
\filldraw[fill opacity=0.8,fill=gray!20,draw=none](-9.449,1.236)--(-9.458,1.243)--(-9.454,1.241)--cycle;
\draw(-9.458,1.243)--(-9.454,1.241);
\filldraw[fill opacity=0.8,fill=gray!20,draw=none](-9.368,1.255)--(-9.371,1.258)--(-9.372,1.259)--(-9.369,1.256)--cycle;
\draw(-9.368,1.255)--(-9.371,1.258);
\filldraw[fill opacity=0.8,fill=gray!20,draw=none](-9.412,1.21)--(-9.411,1.207)--(-9.449,1.236)--(-9.454,1.241)--(-9.43,1.23)--cycle;
\draw(-9.454,1.241)--(-9.43,1.23);
\filldraw[fill opacity=0.8,fill=gray!20](-9.525,1.221)--(-8.412,1.471)--(-8.434,1.507)--(-9.547,1.257)--cycle;
\filldraw[fill opacity=0.8,fill=gray!20,draw=none](-8.507,1.25)--(-8.501,1.25)--(-8.501,1.269)--(-8.519,1.255)--(-8.516,1.25)--cycle;
\draw(-8.519,1.255)--(-8.516,1.25)--(-8.507,1.25);
\filldraw[fill opacity=0.8,fill=gray!20](-8.451,1.535)--(-8.448,1.562)--(-8.471,1.564)--(-8.495,1.538)--cycle;
\filldraw[fill opacity=0.8,fill=gray!20](-8.546,1.367)--(-8.542,1.414)--(-8.603,1.429)--(-8.609,1.382)--cycle;
\filldraw[fill opacity=0.8,fill=gray!20](-8.404,1.537)--(-8.424,1.563)--(-8.448,1.562)--(-8.451,1.535)--cycle;
\filldraw[fill opacity=0.8,fill=gray!20,draw=none](-8.507,1.25)--(-8.453,1.246)--(-8.453,1.257)--cycle;
\draw(-8.507,1.25)--(-8.453,1.246)--(-8.453,1.257);
\filldraw[fill opacity=0.8,fill=gray!20](-8.337,1.511)--(-8.369,1.544)--(-8.404,1.537)--(-8.387,1.501)--cycle;
\filldraw[fill opacity=0.8,fill=gray!20,draw=none](-8.501,1.25)--(-8.453,1.257)--(-8.455,1.276)--(-8.487,1.279)--(-8.501,1.269)--cycle;
\draw(-8.453,1.257)--(-8.455,1.276)--(-8.487,1.279);
\filldraw[fill opacity=0.8,fill=gray!20](-8.298,1.333)--(-8.292,1.378)--(-8.363,1.364)--(-8.366,1.319)--cycle;
\filldraw[fill opacity=0.8,fill=gray!20,draw=none](-8.529,1.506)--(-8.561,1.514)--(-8.569,1.502)--cycle;
\draw(-8.529,1.506)--(-8.561,1.514)--(-8.569,1.502);
\filldraw[fill opacity=0.8,fill=gray!20,draw=none](-8.387,1.248)--(-8.378,1.271)--(-8.453,1.257)--(-8.453,1.246)--cycle;
\draw(-8.453,1.257)--(-8.453,1.246)--(-8.387,1.248)--(-8.378,1.271);
\filldraw[fill opacity=0.8,fill=gray!20,draw=none](-8.541,1.483)--(-8.545,1.504)--(-8.569,1.502)--cycle;
\filldraw[fill opacity=0.8,fill=gray!20,draw=none](-8.494,1.318)--(-8.49,1.351)--(-8.495,1.363)--(-8.546,1.367)--(-8.542,1.322)--cycle;
\draw(-8.495,1.363)--(-8.546,1.367)--(-8.542,1.322)--(-8.494,1.318);
\filldraw[fill opacity=0.8,fill=gray!20,draw=none](-9.245,1.135)--(-8.386,1.328)--(-8.386,1.374)--(-9.321,1.164)--cycle;
\draw(-9.245,1.135)--(-8.386,1.328)--(-8.386,1.374)--(-9.321,1.164);
\filldraw[fill opacity=0.8,fill=gray!20](-8.313,1.47)--(-8.337,1.511)--(-8.387,1.501)--(-8.374,1.458)--cycle;
\filldraw[fill opacity=0.8,fill=gray!20](-8.292,1.378)--(-8.298,1.425)--(-8.366,1.412)--(-8.363,1.364)--cycle;
\filldraw[fill opacity=0.5,fill=gray!20](-8.182,2.053)--(-8.009,1.978)--(-7.862,1.605)--(-8.035,1.681)--cycle;
\filldraw[fill opacity=0.5,fill=gray!20](-9.031,-.596)--(-9.079,-.407)--(-9.461,-.504)--(-9.461,-.705)--cycle;
\filldraw[fill opacity=0.5,fill=gray!20](-8.961,-.699)--(-9.031,-.596)--(-9.461,-.705)--(-9.411,-.813)--cycle;
\filldraw[fill opacity=0.5,fill=gray!20](-8.889,-.783)--(-8.961,-.699)--(-9.411,-.813)--(-9.355,-.901)--cycle;
\filldraw[fill opacity=0.5,fill=gray!20,draw=none](-9.189,-.915)--(-8.841,-.827)--(-8.889,-.783)--(-9.237,-.871)--cycle;
\draw(-8.841,-.827)--(-8.889,-.783)--(-9.237,-.871);
\filldraw[fill opacity=0.5,fill=gray!20,draw=none](-9.519,-.895)--(-9.355,-.901)--(-9.411,-.813)--(-9.575,-.807)--cycle;
\draw(-9.519,-.895)--(-9.355,-.901)--(-9.411,-.813)--(-9.575,-.807);
\filldraw[fill opacity=0.5,fill=gray!20,draw=none](-9.189,-.915)--(-9.237,-.871)--(-9.355,-.901)--(-9.312,-.946)--cycle;
\draw(-9.237,-.871)--(-9.355,-.901)--(-9.312,-.946);
\filldraw[fill opacity=0.8,fill=gray!20,draw=none](-9.439,-.963)--(-9.436,-.942)--(-9.183,-.951)--(-9.182,-.957)--(-9.184,-.971)--cycle;
\draw(-9.439,-.963)--(-9.436,-.942);
\draw(-9.183,-.951)--(-9.182,-.957)--(-9.184,-.971);
\filldraw[fill opacity=0.5,fill=gray!20,draw=none](-9.13,-.925)--(-8.819,-.846)--(-8.841,-.827)--(-9.149,-.905)--cycle;
\draw(-9.13,-.925)--(-8.819,-.846)--(-8.841,-.827);
\filldraw[fill opacity=0.8,fill=gray!20,draw=none](-9.439,-.963)--(-9.184,-.971)--(-9.191,-1.013)--(-9.22,-1.062)--(-9.262,-1.098)--(-9.312,-1.114)--(-9.363,-1.109)--(-9.405,-1.083)--(-9.433,-1.039)--(-9.443,-.986)--cycle;
\draw(-9.184,-.971)--(-9.191,-1.013)--(-9.22,-1.062)--(-9.262,-1.098)--(-9.312,-1.114)--(-9.363,-1.109)--(-9.405,-1.083)--(-9.433,-1.039)--(-9.443,-.986)--(-9.439,-.963);
\filldraw[fill opacity=0.5,fill=gray!20,draw=none](-9.129,-.926)--(-8.814,-.85)--(-8.819,-.846)--(-9.13,-.925)--cycle;
\draw(-8.814,-.85)--(-8.819,-.846)--(-9.13,-.925);
\filldraw[fill opacity=0.5,fill=gray!20,draw=none](-9.129,-.926)--(-9.067,-.966)--(-8.753,-.886)--(-8.814,-.85)--cycle;
\draw(-9.067,-.966)--(-8.753,-.886)--(-8.814,-.85);
\filldraw[fill opacity=0.5,fill=gray!20,draw=none](-9.005,-.98)--(-8.698,-.9)--(-8.753,-.886)--(-9.067,-.966)--cycle;
\draw(-8.698,-.9)--(-8.753,-.886)--(-9.067,-.966);
\filldraw[fill opacity=0.8,fill=gray!20,draw=none](-9.132,-.951)--(-9.127,-.908)--(-9.113,-.868)--(-9.092,-.846)--(-9.067,-.846)--(-9.042,-.868)--(-9.021,-.908)--(-9.006,-.961)--(-9.005,-.98)--cycle;
\draw(-9.132,-.951)--(-9.127,-.908)--(-9.113,-.868)--(-9.092,-.846)--(-9.067,-.846)--(-9.042,-.868)--(-9.021,-.908)--(-9.006,-.961)--(-9.005,-.98);
\filldraw[fill opacity=0.8,fill=gray!20,draw=none](-9.107,-1.081)--(-9.113,-1.07)--(-9.127,-1.018)--(-9.128,-1.006)--(-9.005,-.98)--(-9.002,-1.009)--cycle;
\draw(-9.107,-1.081)--(-9.113,-1.07)--(-9.127,-1.018)--(-9.128,-1.006);
\draw(-9.005,-.98)--(-9.002,-1.009);
\filldraw[fill opacity=0.8,fill=gray!20,draw=none](-9.18,-1.018)--(-9.191,-1.013)--(-9.182,-.957)--(-9.165,-.964)--cycle;
\draw(-9.18,-1.018)--(-9.191,-1.013)--(-9.182,-.957)--(-9.165,-.964);
\filldraw[fill opacity=0.8,fill=gray!20,draw=none](-9.194,-1.074)--(-9.22,-1.062)--(-9.191,-1.013)--(-9.17,-1.022)--cycle;
\draw(-9.194,-1.074)--(-9.22,-1.062)--(-9.191,-1.013)--(-9.17,-1.022);
\filldraw[fill opacity=0.5,fill=gray!20,draw=none](-9.169,-.935)--(-9.13,-.925)--(-9.149,-.905)--(-9.189,-.915)--cycle;
\draw(-9.169,-.935)--(-9.13,-.925);
\filldraw[fill opacity=0.8,fill=gray!20,draw=none](-9.173,-.934)--(-9.177,-.959)--(-9.182,-.957)--(-9.191,-.904)--(-9.18,-.909)--cycle;
\draw(-9.177,-.959)--(-9.182,-.957)--(-9.191,-.904)--(-9.18,-.909);
\filldraw[fill opacity=0.8,fill=gray!20,draw=none](-9.17,-1.016)--(-9.179,-1.018)--(-9.18,-1.018)--(-9.165,-.964)--cycle;
\draw(-9.179,-1.018)--(-9.18,-1.018);
\filldraw[fill opacity=0.5,fill=gray!20,draw=none](-9.432,-.946)--(-9.361,-.956)--(-9.346,-.91)--(-9.355,-.901)--(-9.473,-.897)--cycle;
\draw(-9.346,-.91)--(-9.355,-.901)--(-9.473,-.897);
\filldraw[fill opacity=0.5,fill=gray!20,draw=none](-9.361,-.956)--(-9.326,-.96)--(-9.308,-.951)--(-9.346,-.91)--cycle;
\draw(-9.308,-.951)--(-9.346,-.91);
\filldraw[fill opacity=0.8,fill=gray!20,draw=none](-9.358,-.945)--(-9.358,-.843)--(-9.312,-.829)--(-9.262,-.834)--(-9.22,-.86)--(-9.191,-.904)--(-9.183,-.951)--cycle;
\draw(-9.358,-.843)--(-9.312,-.829)--(-9.262,-.834)--(-9.22,-.86)--(-9.191,-.904)--(-9.183,-.951);
\filldraw[fill opacity=0.8,fill=gray!20,draw=none](-9.173,-.934)--(-9.165,-.964)--(-9.177,-.959)--cycle;
\draw(-9.165,-.964)--(-9.177,-.959);
\filldraw[fill opacity=0.8,fill=gray!20,draw=none](-9.123,-.863)--(-9.115,-.849)--(-9.092,-.846)--(-9.109,-.864)--cycle;
\draw(-9.115,-.849)--(-9.092,-.846)--(-9.109,-.864);
\filldraw[fill opacity=0.8,fill=gray!20,draw=none](-9.165,-.964)--(-9.132,-.961)--(-9.127,-1.018)--(-9.17,-1.022)--cycle;
\draw(-9.165,-.964)--(-9.132,-.961)--(-9.127,-1.018)--(-9.17,-1.022);
\filldraw[fill opacity=0.8,fill=gray!20,draw=none](-9.128,-1.006)--(-9.132,-.961)--(-9.132,-.951)--(-9.005,-.98)--cycle;
\draw(-9.128,-1.006)--(-9.132,-.961)--(-9.132,-.951);
\filldraw[fill opacity=0.8,fill=gray!20,draw=none](-9.173,-.934)--(-9.18,-.909)--(-9.17,-.913)--cycle;
\draw(-9.18,-.909)--(-9.17,-.913);
\filldraw[fill opacity=0.5,fill=gray!20,draw=none](-9.282,-.953)--(-9.256,-.948)--(-9.185,-.92)--(-9.189,-.915)--(-9.312,-.946)--(-9.308,-.951)--cycle;
\draw(-9.312,-.946)--(-9.308,-.951);
\filldraw[fill opacity=0.8,fill=gray!20,draw=none](-9.123,-.863)--(-9.149,-.862)--(-9.135,-.851)--(-9.115,-.849)--cycle;
\draw(-9.135,-.851)--(-9.115,-.849);
\filldraw[fill opacity=0.8,fill=gray!20,draw=none](-9.166,-1.056)--(-9.135,-1.018)--(-9.127,-1.018)--(-9.113,-1.07)--(-9.164,-1.076)--cycle;
\draw(-9.135,-1.018)--(-9.127,-1.018)--(-9.113,-1.07)--(-9.164,-1.076);
\filldraw[fill opacity=0.8,fill=gray!20,draw=none](-9.241,-.797)--(-9.241,-.843)--(-9.296,-.81)--(-9.296,-.8)--cycle;
\draw(-9.296,-.81)--(-9.296,-.8)--(-9.241,-.797)--(-9.241,-.843);
\filldraw[fill opacity=0.8,fill=gray!20,draw=none](-9.184,-.798)--(-9.184,-.854)--(-9.205,-.85)--(-9.241,-.805)--(-9.241,-.797)--cycle;
\draw(-9.241,-.805)--(-9.241,-.797)--(-9.184,-.798)--(-9.184,-.854);
\filldraw[fill opacity=0.8,fill=gray!20,draw=none](-9.174,-.8)--(-9.16,-.845)--(-9.166,-.853)--(-9.184,-.854)--(-9.184,-.798)--cycle;
\draw(-9.184,-.854)--(-9.184,-.798)--(-9.174,-.8);
\filldraw[fill opacity=0.8,fill=gray!20,draw=none](-9.235,-.802)--(-9.196,-.798)--(-9.241,-.797)--(-9.247,-.797)--cycle;
\draw(-9.196,-.798)--(-9.241,-.797)--(-9.247,-.797);
\filldraw[fill opacity=0.8,fill=gray!20,draw=none](-9.309,-.802)--(-9.296,-.8)--(-9.296,-.81)--cycle;
\draw(-9.309,-.802)--(-9.296,-.8)--(-9.296,-.81);
\filldraw[fill opacity=0.8,fill=gray!20,draw=none](-9.294,-.809)--(-9.235,-.802)--(-9.247,-.797)--(-9.296,-.8)--(-9.309,-.802)--cycle;
\draw(-9.247,-.797)--(-9.296,-.8)--(-9.309,-.802);
\filldraw[fill opacity=0.8,fill=gray!20,draw=none](-9.17,-.913)--(-9.191,-.904)--(-9.22,-.86)--(-9.194,-.871)--cycle;
\draw(-9.17,-.913)--(-9.191,-.904)--(-9.22,-.86)--(-9.194,-.871);
\filldraw[fill opacity=0.8,fill=gray!20,draw=none](-9.108,-.849)--(-9.076,-.847)--(-9.074,-.846)--(-9.092,-.846)--(-9.115,-.849)--cycle;
\draw(-9.074,-.846)--(-9.092,-.846)--(-9.115,-.849);
\filldraw[fill opacity=0.8,fill=gray!20,draw=none](-9.184,-.876)--(-9.22,-.86)--(-9.262,-.834)--(-9.241,-.843)--cycle;
\draw(-9.184,-.876)--(-9.22,-.86)--(-9.262,-.834)--(-9.241,-.843);
\filldraw[fill opacity=0.8,fill=gray!20,draw=none](-9.159,-.933)--(-9.129,-.926)--(-9.132,-.961)--(-9.142,-.962)--cycle;
\draw(-9.129,-.926)--(-9.132,-.961)--(-9.142,-.962);
\filldraw[fill opacity=0.5,fill=gray!20,draw=none](-9.11,-.964)--(-9.076,-.968)--(-9.067,-.966)--(-9.13,-.925)--(-9.157,-.932)--cycle;
\draw(-9.076,-.968)--(-9.067,-.966);
\draw(-9.13,-.925)--(-9.157,-.932);
\filldraw[fill opacity=0.8,fill=gray!20,draw=none](-9.159,-.933)--(-9.17,-.913)--(-9.127,-.908)--(-9.129,-.926)--cycle;
\draw(-9.17,-.913)--(-9.127,-.908)--(-9.129,-.926);
\filldraw[fill opacity=0.8,fill=gray!20,draw=none](-9.16,-.845)--(-9.174,-.8)--(-9.135,-.804)--(-9.135,-.813)--cycle;
\draw(-9.174,-.8)--(-9.135,-.804)--(-9.135,-.813);
\filldraw[fill opacity=0.8,fill=gray!20,draw=none](-9.131,-.805)--(-9.135,-.813)--(-9.135,-.804)--cycle;
\draw(-9.135,-.813)--(-9.135,-.804)--(-9.131,-.805);
\filldraw[fill opacity=0.8,fill=gray!20,draw=none](-9.235,-.802)--(-9.208,-.814)--(-9.131,-.805)--(-9.135,-.804)--(-9.184,-.798)--(-9.196,-.798)--cycle;
\draw(-9.131,-.805)--(-9.135,-.804)--(-9.184,-.798)--(-9.196,-.798);
\filldraw[fill opacity=0.8,fill=gray!20,draw=none](-9.184,-.876)--(-9.113,-.868)--(-9.127,-.908)--(-9.17,-.913)--cycle;
\draw(-9.184,-.876)--(-9.113,-.868)--(-9.127,-.908)--(-9.17,-.913);
\filldraw[fill opacity=0.8,fill=gray!20,draw=none](-9.155,-.861)--(-9.109,-.864)--(-9.113,-.868)--(-9.184,-.876)--cycle;
\draw(-9.109,-.864)--(-9.113,-.868)--(-9.184,-.876);
\filldraw[fill opacity=0.8,fill=gray!20,draw=none](-9.108,-.849)--(-9.115,-.849)--(-9.135,-.851)--cycle;
\draw(-9.115,-.849)--(-9.135,-.851);
\filldraw[fill opacity=0.8,fill=gray!20,draw=none](-9.1,-.85)--(-9.1,-.954)--(-9.135,-.928)--(-9.135,-.851)--cycle;
\draw(-9.1,-.85)--(-9.1,-.954);
\draw(-9.135,-.928)--(-9.135,-.851);
\filldraw[fill opacity=0.8,fill=gray!20,draw=none](-9.155,-.861)--(-9.16,-.845)--(-9.135,-.813)--(-9.135,-.851)--cycle;
\draw(-9.135,-.813)--(-9.135,-.851);
\filldraw[fill opacity=0.5,fill=gray!20,draw=none](-9.256,-.948)--(-9.173,-.931)--(-9.185,-.92)--cycle;
\filldraw[fill opacity=0.8,fill=gray!20,draw=none](-9.149,-.862)--(-9.155,-.861)--(-9.135,-.851)--cycle;
\filldraw[fill opacity=0.8,fill=gray!20,draw=none](-9.155,-.861)--(-9.135,-.851)--(-9.135,-.928)--cycle;
\draw(-9.135,-.851)--(-9.135,-.928);
\filldraw[fill opacity=0.8,fill=gray!20,draw=none](-9.168,-.935)--(-9.159,-.933)--(-9.142,-.962)--(-9.165,-.964)--cycle;
\draw(-9.142,-.962)--(-9.165,-.964);
\filldraw[fill opacity=0.8,fill=gray!20,draw=none](-9.168,-.935)--(-9.165,-.964)--(-9.173,-.934)--(-9.173,-.932)--cycle;
\filldraw[fill opacity=0.8,fill=gray!20,draw=none](-9.281,-.962)--(-9.168,-.935)--(-9.165,-.964)--(-9.296,-.979)--cycle;
\draw(-9.165,-.964)--(-9.296,-.979);
\filldraw[fill opacity=0.8,fill=gray!20,draw=none](-9.006,-1.07)--(-9.021,-1.111)--(-9.042,-1.132)--(-9.067,-1.132)--(-9.092,-1.11)--(-9.107,-1.081)--(-9.002,-1.009)--(-9.001,-1.018)--cycle;
\draw(-9.002,-1.009)--(-9.001,-1.018)--(-9.006,-1.07)--(-9.021,-1.111)--(-9.042,-1.132)--(-9.067,-1.132)--(-9.092,-1.11)--(-9.107,-1.081);
\filldraw[fill opacity=0.8,fill=gray!20,draw=none](-9.155,-.861)--(-9.149,-.853)--(-9.135,-.851)--cycle;
\draw(-9.149,-.853)--(-9.135,-.851);
\filldraw[fill opacity=0.8,fill=gray!20,draw=none](-9.1,-.85)--(-9.076,-.847)--(-9.108,-.849)--cycle;
\draw(-9.1,-.85)--(-9.076,-.847);
\filldraw[fill opacity=0.8,fill=gray!20,draw=none](-9.1,-.824)--(-9.1,-.85)--(-9.135,-.851)--cycle;
\draw(-9.1,-.824)--(-9.1,-.85);
\filldraw[fill opacity=0.8,fill=gray!20,draw=none](-9.131,-.805)--(-9.1,-.814)--(-9.1,-.824)--(-9.135,-.851)--(-9.135,-.813)--cycle;
\draw(-9.131,-.805)--(-9.1,-.814)--(-9.1,-.824);
\draw(-9.135,-.851)--(-9.135,-.813);
\filldraw[fill opacity=0.8,fill=gray!20,draw=none](-9.126,-.853)--(-9.1,-.85)--(-9.108,-.849)--(-9.135,-.851)--(-9.149,-.853)--cycle;
\draw(-9.126,-.853)--(-9.1,-.85);
\draw(-9.135,-.851)--(-9.149,-.853);
\filldraw[fill opacity=0.8,fill=gray!20,draw=none](-9.135,-.928)--(-9.17,-.913)--(-9.194,-.871)--(-9.184,-.876)--cycle;
\draw(-9.135,-.928)--(-9.17,-.913);
\draw(-9.194,-.871)--(-9.184,-.876);
\filldraw[fill opacity=0.8,fill=gray!20,draw=none](-9.229,-.82)--(-9.241,-.822)--(-9.241,-.805)--cycle;
\draw(-9.241,-.822)--(-9.241,-.805);
\filldraw[fill opacity=0.8,fill=gray!20,draw=none](-9.235,-.802)--(-9.294,-.809)--(-9.267,-.82)--(-9.208,-.814)--cycle;
\filldraw[fill opacity=0.8,fill=gray!20,draw=none](-9.17,-1.022)--(-9.135,-1.018)--(-9.184,-1.078)--cycle;
\draw(-9.17,-1.022)--(-9.135,-1.018);
\filldraw[fill opacity=0.8,fill=gray!20,draw=none](-9.17,-1.022)--(-9.179,-1.018)--(-9.171,-1.016)--cycle;
\draw(-9.17,-1.022)--(-9.179,-1.018);
\filldraw[fill opacity=0.8,fill=gray!20,draw=none](-9.135,-1.038)--(-9.17,-1.022)--(-9.171,-1.016)--(-9.108,-1.003)--cycle;
\draw(-9.135,-1.038)--(-9.17,-1.022);
\filldraw[fill opacity=0.8,fill=gray!20,draw=none](-9.296,-.979)--(-9.165,-.964)--(-9.17,-1.022)--(-9.241,-1.03)--cycle;
\draw(-9.296,-.979)--(-9.165,-.964);
\draw(-9.17,-1.022)--(-9.241,-1.03);
\filldraw[fill opacity=0.8,fill=gray!20,draw=none](-9.113,-.969)--(-9.1,-.993)--(-9.165,-.964)--(-9.168,-.935)--cycle;
\draw(-9.1,-.993)--(-9.165,-.964);
\filldraw[fill opacity=0.8,fill=gray!20,draw=none](-9.241,-1.107)--(-9.262,-1.098)--(-9.22,-1.062)--(-9.184,-1.078)--cycle;
\draw(-9.241,-1.107)--(-9.262,-1.098)--(-9.22,-1.062)--(-9.184,-1.078);
\filldraw[fill opacity=0.8,fill=gray!20,draw=none](-9.184,-1.14)--(-9.184,-1.155)--(-9.241,-1.154)--(-9.241,-1.129)--cycle;
\draw(-9.184,-1.14)--(-9.184,-1.155)--(-9.241,-1.154)--(-9.241,-1.129);
\filldraw[fill opacity=0.5,fill=gray!20,draw=none](-9.1,-.974)--(-9.096,-.973)--(-9.146,-.939)--(-9.176,-.944)--cycle;
\draw(-9.1,-.974)--(-9.096,-.973);
\filldraw[fill opacity=0.8,fill=gray!20,draw=none](-9.358,-.945)--(-9.436,-.942)--(-9.433,-.93)--(-9.405,-.881)--(-9.363,-.845)--(-9.358,-.843)--cycle;
\draw(-9.436,-.942)--(-9.433,-.93)--(-9.405,-.881)--(-9.363,-.845)--(-9.358,-.843);
\filldraw[fill opacity=0.8,fill=gray!20,draw=none](-9.241,-.843)--(-9.262,-.834)--(-9.312,-.829)--(-9.308,-.831)--cycle;
\draw(-9.241,-.843)--(-9.262,-.834)--(-9.312,-.829)--(-9.308,-.831);
\filldraw[fill opacity=0.8,fill=gray!20,draw=none](-9.296,-1.121)--(-9.312,-1.114)--(-9.262,-1.098)--(-9.241,-1.107)--cycle;
\draw(-9.296,-1.121)--(-9.312,-1.114)--(-9.262,-1.098)--(-9.241,-1.107);
\filldraw[fill opacity=0.8,fill=gray!20,draw=none](-9.266,-1.126)--(-9.241,-1.129)--(-9.241,-1.154)--(-9.296,-1.157)--(-9.296,-1.147)--cycle;
\draw(-9.241,-1.129)--(-9.241,-1.154)--(-9.296,-1.157)--(-9.296,-1.147);
\filldraw[fill opacity=0.8,fill=gray!20](-9.22,-1.218)--(-9.277,-1.216)--(-9.327,-1.21)--(-9.362,-1.2)--(-9.377,-1.188)--(-9.369,-1.176)--(-9.341,-1.165)--(-9.296,-1.157)--(-9.241,-1.154)--(-9.184,-1.155)--(-9.135,-1.162)--(-9.1,-1.171)--(-9.085,-1.183)--(-9.092,-1.196)--(-9.12,-1.207)--(-9.165,-1.214)--cycle;
\filldraw[fill opacity=0.8,fill=gray!20,draw=none](-9.185,-1.06)--(-9.186,-1.055)--(-9.17,-1.022)--(-9.163,-1.025)--cycle;
\draw(-9.17,-1.022)--(-9.163,-1.025);
\filldraw[fill opacity=0.8,fill=gray!20,draw=none](-9.241,-1.03)--(-9.17,-1.022)--(-9.176,-1.045)--cycle;
\draw(-9.241,-1.03)--(-9.17,-1.022);
\filldraw[fill opacity=0.8,fill=gray!20,draw=none](-9.168,-.935)--(-9.173,-.932)--(-9.17,-.913)--cycle;
\filldraw[fill opacity=0.5,fill=gray!20,draw=none](-9.28,-.962)--(-9.218,-.947)--(-9.169,-.935)--(-9.173,-.931)--(-9.256,-.948)--(-9.296,-.964)--cycle;
\draw(-9.218,-.947)--(-9.169,-.935);
\filldraw[fill opacity=0.5,fill=gray!20,draw=none](-9.241,-.955)--(-9.146,-.939)--(-9.157,-.932)--(-9.252,-.956)--cycle;
\draw(-9.157,-.932)--(-9.252,-.956);
\filldraw[fill opacity=0.8,fill=gray!20,draw=none](-9.17,-.913)--(-9.159,-.933)--(-9.168,-.935)--cycle;
\filldraw[fill opacity=0.5,fill=gray!20,draw=none](-9.256,-.948)--(-9.302,-.957)--(-9.296,-.964)--cycle;
\draw(-9.302,-.957)--(-9.296,-.964);
\filldraw[fill opacity=0.5,fill=gray!20,draw=none](-9.326,-.96)--(-9.296,-.964)--(-9.308,-.951)--cycle;
\draw(-9.296,-.964)--(-9.308,-.951);
\filldraw[fill opacity=0.5,fill=gray!20,draw=none](-9.282,-.953)--(-9.308,-.951)--(-9.302,-.957)--cycle;
\draw(-9.308,-.951)--(-9.302,-.957);
\filldraw[fill opacity=0.8,fill=gray!20,draw=none](-9.241,-.921)--(-9.17,-.913)--(-9.168,-.935)--(-9.281,-.962)--cycle;
\draw(-9.241,-.921)--(-9.17,-.913);
\filldraw[fill opacity=0.8,fill=gray!20,draw=none](-9.161,-1.014)--(-9.17,-1.016)--(-9.165,-.964)--(-9.152,-.97)--cycle;
\draw(-9.165,-.964)--(-9.152,-.97);
\filldraw[fill opacity=0.8,fill=gray!20,draw=none](-9.108,-1.003)--(-9.161,-1.014)--(-9.152,-.97)--(-9.1,-.993)--cycle;
\draw(-9.152,-.97)--(-9.1,-.993);
\filldraw[fill opacity=0.8,fill=gray!20,draw=none](-9.284,-.966)--(-9.241,-.971)--(-9.241,-1.107)--(-9.296,-1.147)--(-9.296,-.979)--cycle;
\draw(-9.241,-.971)--(-9.241,-1.107);
\draw(-9.296,-1.147)--(-9.296,-.979);
\filldraw[fill opacity=0.8,fill=gray!20,draw=none](-9.184,-1.078)--(-9.194,-1.074)--(-9.186,-1.055)--cycle;
\draw(-9.184,-1.078)--(-9.194,-1.074);
\filldraw[fill opacity=0.8,fill=gray!20,draw=none](-9.351,-1.114)--(-9.363,-1.109)--(-9.312,-1.114)--(-9.296,-1.121)--cycle;
\draw(-9.351,-1.114)--(-9.363,-1.109)--(-9.312,-1.114)--(-9.296,-1.121);
\filldraw[fill opacity=0.8,fill=gray!20,draw=none](-9.266,-1.126)--(-9.241,-1.107)--(-9.241,-1.129)--cycle;
\draw(-9.241,-1.107)--(-9.241,-1.129);
\filldraw[fill opacity=0.8,fill=gray!20,draw=none](-9.184,-1.103)--(-9.184,-1.14)--(-9.241,-1.129)--(-9.241,-1.124)--cycle;
\draw(-9.184,-1.103)--(-9.184,-1.14);
\draw(-9.241,-1.129)--(-9.241,-1.124);
\filldraw[fill opacity=0.8,fill=gray!20,draw=none](-9.085,-.873)--(-9.085,-.95)--(-9.1,-.954)--(-9.1,-.85)--cycle;
\draw(-9.085,-.873)--(-9.085,-.95);
\draw(-9.1,-.954)--(-9.1,-.85);
\filldraw[fill opacity=0.8,fill=gray!20,draw=none](-9.099,-.874)--(-9.085,-.873)--(-9.076,-.847)--(-9.126,-.853)--cycle;
\draw(-9.099,-.874)--(-9.085,-.873);
\draw(-9.076,-.847)--(-9.126,-.853);
\filldraw[fill opacity=0.8,fill=gray!20,draw=none](-9.085,-.836)--(-9.085,-.873)--(-9.1,-.85)--(-9.1,-.822)--cycle;
\draw(-9.085,-.836)--(-9.085,-.873);
\draw(-9.1,-.85)--(-9.1,-.822);
\filldraw[fill opacity=0.8,fill=gray!20,draw=none](-9.085,-.826)--(-9.085,-.836)--(-9.1,-.822)--(-9.1,-.814)--cycle;
\draw(-9.1,-.822)--(-9.1,-.814)--(-9.085,-.826)--(-9.085,-.836);
\filldraw[fill opacity=0.8,fill=gray!20,draw=none](-9.09,-.836)--(-9.085,-.826)--(-9.1,-.814)--(-9.131,-.805)--(-9.155,-.808)--cycle;
\draw(-9.09,-.836)--(-9.085,-.826)--(-9.1,-.814)--(-9.131,-.805);
\filldraw[fill opacity=0.8,fill=gray!20,draw=none](-9.076,-.847)--(-9.067,-.846)--(-9.074,-.846)--cycle;
\draw(-9.076,-.847)--(-9.067,-.846)--(-9.074,-.846);
\filldraw[fill opacity=0.8,fill=gray!20,draw=none](-9.139,-.953)--(-9.168,-.935)--(-9.17,-.913)--(-9.146,-.923)--cycle;
\draw(-9.17,-.913)--(-9.146,-.923);
\filldraw[fill opacity=0.8,fill=gray!20,draw=none](-9.176,-.898)--(-9.17,-.913)--(-9.241,-.921)--cycle;
\draw(-9.17,-.913)--(-9.241,-.921);
\filldraw[fill opacity=0.5,fill=gray!20,draw=none](-9.154,-.953)--(-9.176,-.944)--(-9.241,-.955)--cycle;
\filldraw[fill opacity=0.8,fill=gray!20,draw=none](-9.241,-.843)--(-9.241,-.921)--(-9.296,-.94)--(-9.296,-.836)--cycle;
\draw(-9.241,-.843)--(-9.241,-.921);
\draw(-9.296,-.94)--(-9.296,-.836);
\filldraw[fill opacity=0.8,fill=gray!20,draw=none](-9.229,-.82)--(-9.184,-.876)--(-9.241,-.921)--(-9.241,-.822)--cycle;
\draw(-9.241,-.921)--(-9.241,-.822);
\filldraw[fill opacity=0.8,fill=gray!20,draw=none](-9.296,-.836)--(-9.296,-.94)--(-9.341,-.932)--(-9.341,-.854)--cycle;
\draw(-9.296,-.836)--(-9.296,-.94);
\draw(-9.341,-.932)--(-9.341,-.854);
\filldraw[fill opacity=0.8,fill=gray!20,draw=none](-9.211,-1.113)--(-9.241,-1.124)--(-9.241,-1.03)--cycle;
\draw(-9.241,-1.124)--(-9.241,-1.03);
\filldraw[fill opacity=0.8,fill=gray!20,draw=none](-9.184,-1.078)--(-9.218,-1.095)--(-9.241,-1.03)--cycle;
\filldraw[fill opacity=0.8,fill=gray!20,draw=none](-9.241,-.957)--(-9.241,-.971)--(-9.296,-.964)--cycle;
\draw(-9.241,-.957)--(-9.241,-.971);
\filldraw[fill opacity=0.8,fill=gray!20,draw=none](-9.241,-.921)--(-9.28,-.962)--(-9.296,-.964)--(-9.296,-.94)--cycle;
\draw(-9.296,-.964)--(-9.296,-.94);
\filldraw[fill opacity=0.8,fill=gray!20,draw=none](-9.284,-.966)--(-9.296,-.979)--(-9.296,-.964)--cycle;
\draw(-9.296,-.979)--(-9.296,-.964);
\filldraw[fill opacity=0.5,fill=gray!20,draw=none](-9.28,-.962)--(-9.296,-.964)--(-9.295,-.965)--cycle;
\draw(-9.296,-.964)--(-9.295,-.965);
\filldraw[fill opacity=0.5,fill=gray!20,draw=none](-9.364,-.964)--(-9.309,-.965)--(-9.296,-.964)--(-9.361,-.956)--cycle;
\filldraw[fill opacity=0.8,fill=gray!20,draw=none](-9.184,-1.078)--(-9.185,-1.06)--(-9.163,-1.025)--(-9.146,-1.033)--cycle;
\draw(-9.163,-1.025)--(-9.146,-1.033);
\filldraw[fill opacity=0.8,fill=gray!20,draw=none](-9.241,-1.03)--(-9.176,-1.045)--(-9.184,-1.078)--cycle;
\filldraw[fill opacity=0.8,fill=gray!20,draw=none](-9.184,-.961)--(-9.184,-1.078)--(-9.241,-1.03)--(-9.241,-.957)--cycle;
\draw(-9.184,-.961)--(-9.184,-1.078);
\draw(-9.241,-1.03)--(-9.241,-.957);
\filldraw[fill opacity=0.8,fill=gray!20,draw=none](-9.296,-.964)--(-9.296,-.979)--(-9.309,-.965)--cycle;
\draw(-9.296,-.964)--(-9.296,-.979);
\filldraw[fill opacity=0.8,fill=gray!20,draw=none](-9.281,-.962)--(-9.296,-.979)--(-9.391,-.989)--cycle;
\draw(-9.296,-.979)--(-9.391,-.989);
\filldraw[fill opacity=0.8,fill=gray!20,draw=none](-9.296,-.979)--(-9.296,-1.052)--(-9.341,-1.067)--(-9.341,-1.041)--cycle;
\draw(-9.296,-.979)--(-9.296,-1.052);
\draw(-9.341,-1.067)--(-9.341,-1.041);
\filldraw[fill opacity=0.8,fill=gray!20,draw=none](-9.296,-.979)--(-9.241,-1.03)--(-9.341,-1.041)--cycle;
\draw(-9.241,-1.03)--(-9.341,-1.041);
\filldraw[fill opacity=0.8,fill=gray!20,draw=none](-9.427,-.993)--(-9.296,-.979)--(-9.341,-1.041)--(-9.412,-1.049)--cycle;
\draw(-9.427,-.993)--(-9.296,-.979);
\draw(-9.341,-1.041)--(-9.412,-1.049);
\filldraw[fill opacity=0.8,fill=gray!20,draw=none](-9.419,-.963)--(-9.309,-.965)--(-9.306,-.968)--(-9.391,-.989)--(-9.427,-.993)--cycle;
\draw(-9.391,-.989)--(-9.427,-.993);
\filldraw[fill opacity=0.5,fill=gray!20,draw=none](-9.309,-.965)--(-9.37,-.964)--(-9.316,-.966)--cycle;
\draw(-9.37,-.964)--(-9.316,-.966);
\filldraw[fill opacity=0.5,fill=gray!20,draw=none](-9.419,-.962)--(-9.37,-.964)--(-9.364,-.964)--(-9.361,-.956)--(-9.432,-.946)--cycle;
\draw(-9.419,-.962)--(-9.37,-.964);
\filldraw[fill opacity=0.8,fill=gray!20,draw=none](-9.309,-.965)--(-9.419,-.963)--(-9.412,-.939)--(-9.341,-.932)--cycle;
\draw(-9.412,-.939)--(-9.341,-.932);
\filldraw[fill opacity=0.8,fill=gray!20,draw=none](-9.309,-.965)--(-9.296,-.979)--(-9.341,-1.041)--(-9.341,-.966)--cycle;
\draw(-9.341,-1.041)--(-9.341,-.966);
\filldraw[fill opacity=0.8,fill=gray!20,draw=none](-9.309,-.965)--(-9.293,-.965)--(-9.306,-.968)--cycle;
\filldraw[fill opacity=0.5,fill=gray!20,draw=none](-9.366,-.967)--(-9.316,-.966)--(-9.364,-.964)--cycle;
\draw(-9.316,-.966)--(-9.364,-.964);
\filldraw[fill opacity=0.5,fill=gray!20,draw=none](-9.369,-.974)--(-9.341,-.966)--(-9.366,-.967)--cycle;
\filldraw[fill opacity=0.5,fill=gray!20,draw=none](-9.409,-.969)--(-9.366,-.967)--(-9.364,-.964)--(-9.419,-.962)--cycle;
\draw(-9.364,-.964)--(-9.419,-.962);
\filldraw[fill opacity=0.8,fill=gray!20,draw=none](-9.353,-.872)--(-9.341,-.932)--(-9.369,-.896)--cycle;
\filldraw[fill opacity=0.8,fill=gray!20,draw=none](-9.353,-.872)--(-9.341,-.854)--(-9.341,-.932)--cycle;
\draw(-9.341,-.854)--(-9.341,-.932);
\filldraw[fill opacity=0.8,fill=gray!20,draw=none](-9.369,-.896)--(-9.184,-.876)--(-9.176,-.898)--(-9.241,-.921)--(-9.341,-.932)--cycle;
\draw(-9.369,-.896)--(-9.184,-.876);
\draw(-9.241,-.921)--(-9.341,-.932);
\filldraw[fill opacity=0.8,fill=gray!20,draw=none](-9.166,-1.056)--(-9.164,-1.076)--(-9.184,-1.078)--cycle;
\draw(-9.164,-1.076)--(-9.184,-1.078);
\filldraw[fill opacity=0.8,fill=gray!20,draw=none](-9.184,-1.078)--(-9.113,-1.07)--(-9.109,-1.078)--(-9.143,-1.094)--cycle;
\draw(-9.184,-1.078)--(-9.113,-1.07)--(-9.109,-1.078);
\filldraw[fill opacity=0.5,fill=gray!20,draw=none](-9.135,-.96)--(-9.154,-.953)--(-9.184,-.954)--cycle;
\filldraw[fill opacity=0.8,fill=gray!20,draw=none](-9.184,-1.078)--(-9.146,-1.033)--(-9.135,-1.038)--cycle;
\draw(-9.146,-1.033)--(-9.135,-1.038);
\filldraw[fill opacity=0.8,fill=gray!20,draw=none](-9.135,-1.012)--(-9.135,-1.038)--(-9.184,-1.078)--(-9.184,-.988)--cycle;
\draw(-9.135,-1.012)--(-9.135,-1.038);
\draw(-9.184,-1.078)--(-9.184,-.988);
\filldraw[fill opacity=0.5,fill=gray!20,draw=none](-9.218,-.947)--(-9.295,-.965)--(-9.294,-.966)--cycle;
\draw(-9.295,-.965)--(-9.294,-.966)--(-9.218,-.947);
\filldraw[fill opacity=0.8,fill=gray!20,draw=none](-9.184,-.876)--(-9.184,-.961)--(-9.241,-.957)--(-9.241,-.921)--cycle;
\draw(-9.184,-.876)--(-9.184,-.961);
\draw(-9.241,-.957)--(-9.241,-.921);
\filldraw[fill opacity=0.5,fill=gray!20,draw=none](-9.241,-.955)--(-9.252,-.956)--(-9.272,-.961)--cycle;
\draw(-9.252,-.956)--(-9.272,-.961);
\filldraw[fill opacity=0.8,fill=gray!20,draw=none](-9.241,-.921)--(-9.241,-.957)--(-9.28,-.962)--cycle;
\draw(-9.241,-.921)--(-9.241,-.957);
\filldraw[fill opacity=0.8,fill=gray!20,draw=none](-9.135,-.977)--(-9.135,-1.012)--(-9.184,-.988)--(-9.184,-.961)--cycle;
\draw(-9.135,-.977)--(-9.135,-1.012);
\draw(-9.184,-.988)--(-9.184,-.961);
\filldraw[fill opacity=0.8,fill=gray!20,draw=none](-9.135,-.96)--(-9.135,-.977)--(-9.184,-.961)--(-9.184,-.954)--cycle;
\draw(-9.135,-.96)--(-9.135,-.977);
\draw(-9.184,-.961)--(-9.184,-.954);
\filldraw[fill opacity=0.8,fill=gray!20,draw=none](-9.113,-.969)--(-9.139,-.953)--(-9.146,-.923)--(-9.135,-.928)--cycle;
\draw(-9.146,-.923)--(-9.135,-.928);
\filldraw[fill opacity=0.8,fill=gray!20,draw=none](-9.16,-.845)--(-9.135,-.928)--(-9.135,-.96)--(-9.184,-.954)--(-9.184,-.876)--cycle;
\draw(-9.135,-.928)--(-9.135,-.96);
\draw(-9.184,-.954)--(-9.184,-.876);
\filldraw[fill opacity=0.8,fill=gray!20,draw=none](-9.17,-.858)--(-9.184,-.876)--(-9.184,-.854)--cycle;
\draw(-9.184,-.876)--(-9.184,-.854);
\filldraw[fill opacity=0.8,fill=gray!20,draw=none](-9.166,-.853)--(-9.17,-.858)--(-9.184,-.854)--cycle;
\filldraw[fill opacity=0.8,fill=gray!20,draw=none](-9.155,-.861)--(-9.182,-.86)--(-9.182,-.856)--(-9.149,-.853)--cycle;
\draw(-9.182,-.856)--(-9.149,-.853);
\filldraw[fill opacity=0.8,fill=gray!20,draw=none](-9.184,-.854)--(-9.184,-.876)--(-9.205,-.85)--cycle;
\draw(-9.184,-.854)--(-9.184,-.876);
\filldraw[fill opacity=0.8,fill=gray!20,draw=none](-9.21,-.83)--(-9.125,-.821)--(-9.155,-.808)--(-9.239,-.817)--cycle;
\filldraw[fill opacity=0.8,fill=gray!20,draw=none](-9.06,-.852)--(-9.067,-.846)--(-9.076,-.847)--cycle;
\draw(-9.06,-.852)--(-9.067,-.846)--(-9.076,-.847);
\filldraw[fill opacity=0.8,fill=gray!20,draw=none](-9.135,-1.115)--(-9.092,-1.11)--(-9.067,-1.132)--(-9.076,-1.133)--cycle;
\draw(-9.135,-1.115)--(-9.092,-1.11)--(-9.067,-1.132)--(-9.076,-1.133);
\filldraw[fill opacity=0.8,fill=gray!20,draw=none](-9.085,-.873)--(-9.042,-.868)--(-9.06,-.852)--(-9.076,-.847)--cycle;
\draw(-9.085,-.873)--(-9.042,-.868)--(-9.06,-.852);
\filldraw[fill opacity=0.8,fill=gray!20,draw=none](-9.155,-1.1)--(-9.109,-1.078)--(-9.092,-1.11)--(-9.135,-1.115)--cycle;
\draw(-9.109,-1.078)--(-9.092,-1.11)--(-9.135,-1.115);
\filldraw[fill opacity=0.8,fill=gray!20,draw=none](-9.155,-1.1)--(-9.174,-1.157)--(-9.184,-1.155)--(-9.184,-1.078)--cycle;
\draw(-9.174,-1.157)--(-9.184,-1.155)--(-9.184,-1.078);
\filldraw[fill opacity=0.8,fill=gray!20,draw=none](-9.248,-1.142)--(-9.296,-1.121)--(-9.241,-1.107)--(-9.206,-1.123)--cycle;
\draw(-9.248,-1.142)--(-9.296,-1.121);
\draw(-9.241,-1.107)--(-9.206,-1.123);
\filldraw[fill opacity=0.8,fill=gray!20,draw=none](-9.184,-1.078)--(-9.212,-1.111)--(-9.218,-1.095)--cycle;
\filldraw[fill opacity=0.8,fill=gray!20,draw=none](-9.217,-1.118)--(-9.241,-1.107)--(-9.184,-1.078)--cycle;
\draw(-9.217,-1.118)--(-9.241,-1.107);
\filldraw[fill opacity=0.8,fill=gray!20,draw=none](-9.184,-1.078)--(-9.184,-1.103)--(-9.211,-1.113)--(-9.212,-1.111)--cycle;
\draw(-9.184,-1.078)--(-9.184,-1.103);
\filldraw[fill opacity=0.8,fill=gray!20,draw=none](-9.321,-1.131)--(-9.296,-1.121)--(-9.296,-1.147)--cycle;
\draw(-9.296,-1.121)--(-9.296,-1.147);
\filldraw[fill opacity=0.8,fill=gray!20,draw=none](-9.341,-1.118)--(-9.326,-1.117)--(-9.296,-1.121)--cycle;
\filldraw[fill opacity=0.8,fill=gray!20,draw=none](-9.341,-1.118)--(-9.351,-1.114)--(-9.326,-1.117)--cycle;
\draw(-9.341,-1.118)--(-9.351,-1.114);
\filldraw[fill opacity=0.8,fill=gray!20,draw=none](-9.296,-1.052)--(-9.296,-1.121)--(-9.341,-1.118)--(-9.341,-1.067)--cycle;
\draw(-9.296,-1.052)--(-9.296,-1.121);
\draw(-9.341,-1.118)--(-9.341,-1.067);
\filldraw[fill opacity=0.8,fill=gray!20,draw=none](-9.296,-1.121)--(-9.321,-1.131)--(-9.341,-1.118)--cycle;
\filldraw[fill opacity=0.8,fill=gray!20,draw=none](-9.306,-1.134)--(-9.341,-1.118)--(-9.296,-1.121)--(-9.231,-1.15)--cycle;
\draw(-9.306,-1.134)--(-9.341,-1.118);
\draw(-9.296,-1.121)--(-9.231,-1.15);
\filldraw[fill opacity=0.8,fill=gray!20,draw=none](-9.184,-1.078)--(-9.143,-1.094)--(-9.206,-1.123)--cycle;
\filldraw[fill opacity=0.8,fill=gray!20,draw=none](-9.135,-1.038)--(-9.155,-1.1)--(-9.184,-1.078)--cycle;
\filldraw[fill opacity=0.8,fill=gray!20,draw=none](-9.155,-1.1)--(-9.135,-1.038)--(-9.135,-1.115)--cycle;
\draw(-9.135,-1.038)--(-9.135,-1.115);
\filldraw[fill opacity=0.8,fill=gray!20,draw=none](-9.155,-1.1)--(-9.135,-1.115)--(-9.135,-1.162)--(-9.174,-1.157)--cycle;
\draw(-9.135,-1.115)--(-9.135,-1.162)--(-9.174,-1.157);
\filldraw[fill opacity=0.8,fill=gray!20,draw=none](-9.321,-1.131)--(-9.296,-1.147)--(-9.296,-1.157)--(-9.341,-1.165)--(-9.341,-1.14)--cycle;
\draw(-9.296,-1.147)--(-9.296,-1.157)--(-9.341,-1.165)--(-9.341,-1.14);
\filldraw[fill opacity=0.8,fill=gray!20,draw=none](-9.135,-1.115)--(-9.108,-1.123)--(-9.1,-1.136)--cycle;
\filldraw[fill opacity=0.8,fill=gray!20,draw=none](-9.108,-1.123)--(-9.076,-1.133)--(-9.1,-1.136)--cycle;
\draw(-9.076,-1.133)--(-9.1,-1.136);
\filldraw[fill opacity=0.8,fill=gray!20,draw=none](-9.1,-1.107)--(-9.1,-1.136)--(-9.135,-1.115)--(-9.135,-1.1)--cycle;
\draw(-9.1,-1.107)--(-9.1,-1.136);
\draw(-9.135,-1.115)--(-9.135,-1.1);
\filldraw[fill opacity=0.8,fill=gray!20,draw=none](-9.108,-1.003)--(-9.1,-1.001)--(-9.1,-1.107)--(-9.135,-1.1)--(-9.135,-1.038)--cycle;
\draw(-9.1,-1.001)--(-9.1,-1.107);
\draw(-9.135,-1.1)--(-9.135,-1.038);
\filldraw[fill opacity=0.8,fill=gray!20,draw=none](-9.296,-.94)--(-9.296,-.964)--(-9.309,-.965)--(-9.341,-.932)--cycle;
\draw(-9.296,-.94)--(-9.296,-.964);
\filldraw[fill opacity=0.5,fill=gray!20,draw=none](-9.309,-.965)--(-9.295,-.965)--(-9.296,-.964)--cycle;
\draw(-9.295,-.965)--(-9.296,-.964);
\filldraw[fill opacity=0.8,fill=gray!20,draw=none](-9.309,-.965)--(-9.341,-.932)--(-9.241,-.921)--(-9.281,-.962)--(-9.293,-.965)--cycle;
\draw(-9.341,-.932)--(-9.241,-.921);
\filldraw[fill opacity=0.8,fill=gray!20,draw=none](-9.309,-.965)--(-9.341,-.966)--(-9.341,-.932)--cycle;
\draw(-9.341,-.966)--(-9.341,-.932);
\filldraw[fill opacity=0.8,fill=gray!20,draw=none](-9.124,-.885)--(-9.088,-.891)--(-9.085,-.873)--(-9.132,-.878)--cycle;
\draw(-9.085,-.873)--(-9.132,-.878);
\filldraw[fill opacity=0.8,fill=gray!20,draw=none](-9.092,-.916)--(-9.085,-.925)--(-9.085,-.873)--cycle;
\draw(-9.085,-.925)--(-9.085,-.873);
\filldraw[fill opacity=0.8,fill=gray!20,draw=none](-9.088,-.891)--(-9.025,-.9)--(-9.042,-.868)--(-9.085,-.873)--cycle;
\draw(-9.025,-.9)--(-9.042,-.868)--(-9.085,-.873);
\filldraw[fill opacity=0.8,fill=gray!20,draw=none](-9.184,-.876)--(-9.241,-.843)--(-9.217,-.854)--cycle;
\draw(-9.241,-.843)--(-9.217,-.854);
\filldraw[fill opacity=0.8,fill=gray!20,draw=none](-9.206,-.859)--(-9.182,-.86)--(-9.184,-.876)--cycle;
\filldraw[fill opacity=0.8,fill=gray!20,draw=none](-9.206,-.859)--(-9.182,-.856)--(-9.182,-.86)--cycle;
\draw(-9.206,-.859)--(-9.182,-.856);
\filldraw[fill opacity=0.8,fill=gray!20,draw=none](-9.195,-.86)--(-9.126,-.853)--(-9.149,-.853)--(-9.206,-.859)--cycle;
\draw(-9.195,-.86)--(-9.126,-.853);
\draw(-9.149,-.853)--(-9.206,-.859);
\filldraw[fill opacity=0.8,fill=gray!20,draw=none](-9.21,-.83)--(-9.153,-.855)--(-9.12,-.85)--(-9.092,-.839)--(-9.09,-.836)--(-9.125,-.821)--cycle;
\draw(-9.153,-.855)--(-9.12,-.85)--(-9.092,-.839)--(-9.09,-.836);
\filldraw[fill opacity=0.8,fill=gray!20,draw=none](-9.071,-.893)--(-9.124,-.885)--(-9.092,-.916)--cycle;
\filldraw[fill opacity=0.8,fill=gray!20,draw=none](-9.092,-.839)--(-9.092,-.916)--(-9.085,-.873)--(-9.085,-.826)--cycle;
\draw(-9.085,-.873)--(-9.085,-.826)--(-9.092,-.839)--(-9.092,-.916);
\filldraw[fill opacity=0.8,fill=gray!20,draw=none](-9.119,-1.136)--(-9.1,-1.161)--(-9.1,-1.171)--(-9.135,-1.162)--(-9.135,-1.137)--cycle;
\draw(-9.1,-1.161)--(-9.1,-1.171)--(-9.135,-1.162)--(-9.135,-1.137);
\filldraw[fill opacity=0.8,fill=gray!20,draw=none](-9.1,-1.136)--(-9.067,-1.132)--(-9.042,-1.132)--(-9.065,-1.135)--cycle;
\draw(-9.1,-1.136)--(-9.067,-1.132)--(-9.042,-1.132)--(-9.065,-1.135);
\filldraw[fill opacity=0.8,fill=gray!20,draw=none](-9.1,-1.136)--(-9.1,-1.161)--(-9.135,-1.115)--cycle;
\draw(-9.1,-1.136)--(-9.1,-1.161);
\filldraw[fill opacity=0.8,fill=gray!20,draw=none](-9.1,-1.136)--(-9.081,-1.135)--(-9.085,-1.137)--cycle;
\filldraw[fill opacity=0.8,fill=gray!20,draw=none](-9.1,-.974)--(-9.1,-.993)--(-9.108,-1.003)--(-9.135,-1.009)--(-9.135,-.971)--cycle;
\draw(-9.1,-.974)--(-9.1,-.993);
\draw(-9.135,-1.009)--(-9.135,-.971);
\filldraw[fill opacity=0.8,fill=gray!20,draw=none](-9.1,-.974)--(-9.135,-.971)--(-9.135,-.96)--cycle;
\draw(-9.135,-.971)--(-9.135,-.96);
\filldraw[fill opacity=0.8,fill=gray!20,draw=none](-9.095,-.98)--(-9.1,-.993)--(-9.1,-.974)--cycle;
\draw(-9.1,-.993)--(-9.1,-.974);
\filldraw[fill opacity=0.8,fill=gray!20,draw=none](-9.11,-.973)--(-9.113,-.969)--(-9.103,-.975)--cycle;
\filldraw[fill opacity=0.8,fill=gray!20,draw=none](-9.11,-.973)--(-9.103,-.975)--(-9.07,-.995)--(-9.086,-.999)--(-9.1,-.993)--cycle;
\draw(-9.086,-.999)--(-9.1,-.993);
\filldraw[fill opacity=0.5,fill=gray!20,draw=none](-9.103,-.975)--(-9.113,-.969)--(-9.135,-.96)--(-9.184,-.954)--(-9.241,-.955)--(-9.272,-.961)--(-9.294,-.966)--(-9.231,-1.007)--cycle;
\draw(-9.272,-.961)--(-9.294,-.966)--(-9.231,-1.007)--(-9.103,-.975);
\filldraw[fill opacity=0.8,fill=gray!20,draw=none](-9.108,-1.003)--(-9.135,-1.038)--(-9.135,-1.009)--cycle;
\draw(-9.135,-1.038)--(-9.135,-1.009);
\filldraw[fill opacity=0.8,fill=gray!20,draw=none](-9.098,-1.001)--(-9.085,-1.06)--(-9.085,-1.137)--(-9.1,-1.136)--(-9.1,-1.001)--cycle;
\draw(-9.085,-1.06)--(-9.085,-1.137);
\draw(-9.1,-1.136)--(-9.1,-1.001);
\filldraw[fill opacity=0.8,fill=gray!20,draw=none](-9.092,-1.118)--(-9.184,-1.078)--(-9.135,-1.038)--(-9.085,-1.06)--cycle;
\draw(-9.092,-1.118)--(-9.184,-1.078);
\draw(-9.135,-1.038)--(-9.085,-1.06);
\filldraw[fill opacity=0.8,fill=gray!20,draw=none](-9.155,-1.1)--(-9.135,-1.115)--(-9.206,-1.123)--cycle;
\draw(-9.135,-1.115)--(-9.206,-1.123);
\filldraw[fill opacity=0.8,fill=gray!20,draw=none](-9.119,-1.136)--(-9.135,-1.137)--(-9.135,-1.115)--cycle;
\draw(-9.135,-1.137)--(-9.135,-1.115);
\filldraw[fill opacity=0.8,fill=gray!20,draw=none](-9.206,-1.123)--(-9.135,-1.115)--(-9.1,-1.136)--(-9.195,-1.146)--cycle;
\draw(-9.206,-1.123)--(-9.135,-1.115);
\draw(-9.1,-1.136)--(-9.195,-1.146);
\filldraw[fill opacity=0.8,fill=gray!20,draw=none](-9.085,-1.137)--(-9.085,-1.183)--(-9.1,-1.171)--(-9.1,-1.161)--cycle;
\draw(-9.085,-1.137)--(-9.085,-1.183)--(-9.1,-1.171)--(-9.1,-1.161);
\filldraw[fill opacity=0.8,fill=gray!20,draw=none](-9.085,-1.137)--(-9.1,-1.161)--(-9.1,-1.136)--cycle;
\draw(-9.1,-1.161)--(-9.1,-1.136);
\filldraw[fill opacity=0.8,fill=gray!20,draw=none](-9.195,-1.146)--(-9.1,-1.136)--(-9.085,-1.137)--(-9.155,-1.145)--cycle;
\draw(-9.195,-1.146)--(-9.1,-1.136);
\draw(-9.085,-1.137)--(-9.155,-1.145);
\filldraw[fill opacity=0.8,fill=gray!20,draw=none](-9.081,-1.135)--(-9.065,-1.135)--(-9.085,-1.137)--cycle;
\draw(-9.065,-1.135)--(-9.085,-1.137);
\filldraw[fill opacity=0.8,fill=gray!20,draw=none](-9.182,-.86)--(-9.155,-.861)--(-9.184,-.876)--cycle;
\filldraw[fill opacity=0.8,fill=gray!20,draw=none](-9.155,-.881)--(-9.099,-.874)--(-9.126,-.853)--(-9.195,-.86)--cycle;
\draw(-9.155,-.881)--(-9.099,-.874);
\draw(-9.126,-.853)--(-9.195,-.86);
\filldraw[fill opacity=0.8,fill=gray!20,draw=none](-9.12,-.85)--(-9.12,-.858)--(-9.1,-.899)--(-9.092,-.894)--(-9.092,-.839)--cycle;
\draw(-9.092,-.894)--(-9.092,-.839)--(-9.12,-.85)--(-9.12,-.858);
\filldraw[fill opacity=0.8,fill=gray!20,draw=none](-9.155,-1.145)--(-9.217,-1.118)--(-9.184,-1.078)--(-9.092,-1.118)--cycle;
\draw(-9.155,-1.145)--(-9.217,-1.118);
\draw(-9.184,-1.078)--(-9.092,-1.118);
\filldraw[fill opacity=0.8,fill=gray!20,draw=none](-9.092,-.916)--(-9.085,-.95)--(-9.085,-.925)--cycle;
\draw(-9.085,-.95)--(-9.085,-.925);
\filldraw[fill opacity=0.8,fill=gray!20,draw=none](-9.049,-.966)--(-9.135,-.928)--(-9.184,-.876)--(-9.092,-.916)--cycle;
\draw(-9.049,-.966)--(-9.135,-.928);
\draw(-9.184,-.876)--(-9.092,-.916);
\filldraw[fill opacity=0.8,fill=gray!20,draw=none](-9.124,-.885)--(-9.132,-.878)--(-9.155,-.881)--cycle;
\draw(-9.132,-.878)--(-9.155,-.881);
\filldraw[fill opacity=0.8,fill=gray!20,draw=none](-9.125,-.85)--(-9.165,-.857)--(-9.165,-.867)--(-9.141,-.883)--(-9.12,-.874)--(-9.12,-.858)--cycle;
\draw(-9.125,-.85)--(-9.165,-.857)--(-9.165,-.867);
\draw(-9.12,-.874)--(-9.12,-.858);
\filldraw[fill opacity=0.8,fill=gray!20,draw=none](-9.125,-.85)--(-9.12,-.858)--(-9.12,-.85)--cycle;
\draw(-9.12,-.858)--(-9.12,-.85)--(-9.125,-.85);
\filldraw[fill opacity=0.5,fill=gray!20](-8.57,-.798)--(-9.031,-.596)--(-9.461,-.705)--(-9,-.906)--cycle;
\filldraw[fill opacity=0.8,fill=gray!20](-8.298,1.425)--(-8.313,1.47)--(-8.374,1.458)--(-8.366,1.412)--cycle;
\filldraw[fill opacity=0.8,fill=gray!20,draw=none](-8.495,1.363)--(-8.499,1.401)--(-8.507,1.411)--(-8.542,1.414)--(-8.546,1.367)--cycle;
\draw(-8.507,1.411)--(-8.542,1.414)--(-8.546,1.367)--(-8.495,1.363);
\filldraw[fill opacity=0.8,fill=gray!20,draw=none](-9.414,1.239)--(-9.39,1.248)--(-9.399,1.227)--cycle;
\draw(-9.39,1.248)--(-9.399,1.227);
\filldraw[fill opacity=0.8,fill=gray!20,draw=none](-9.356,1.197)--(-9.367,1.213)--(-9.368,1.226)--(-9.364,1.221)--cycle;
\draw(-9.367,1.213)--(-9.368,1.226)--(-9.364,1.221);
\filldraw[fill opacity=0.8,fill=gray!20,draw=none](-9.368,1.226)--(-9.372,1.237)--(-9.372,1.251)--(-9.363,1.243)--(-9.346,1.203)--cycle;
\draw(-9.363,1.243)--(-9.346,1.203)--(-9.368,1.226)--(-9.372,1.237);
\filldraw[fill opacity=0.8,fill=gray!20,draw=none](-9.356,1.197)--(-9.364,1.221)--(-9.346,1.203)--(-9.343,1.179)--cycle;
\draw(-9.364,1.221)--(-9.346,1.203)--(-9.343,1.179);
\filldraw[fill opacity=0.8,fill=gray!20,draw=none](-9.389,1.211)--(-9.371,1.251)--(-9.378,1.245)--(-9.396,1.204)--cycle;
\draw(-9.389,1.211)--(-9.371,1.251);
\draw(-9.378,1.245)--(-9.396,1.204);
\filldraw[fill opacity=0.8,fill=gray!20,draw=none](-9.404,1.187)--(-9.389,1.211)--(-9.396,1.204)--(-9.404,1.187)--cycle;
\draw(-9.396,1.204)--(-9.404,1.187);
\filldraw[fill opacity=0.8,fill=gray!20,draw=none](-9.449,1.236)--(-9.411,1.207)--(-9.405,1.192)--(-9.411,1.194)--cycle;
\draw(-9.405,1.192)--(-9.411,1.194);
\filldraw[fill opacity=0.8,fill=gray!20,draw=none](-9.359,1.208)--(-9.354,1.197)--(-9.346,1.203)--(-9.367,1.253)--cycle;
\draw(-9.354,1.197)--(-9.346,1.203)--(-9.367,1.253);
\filldraw[fill opacity=0.8,fill=gray!20,draw=none](-9.406,1.187)--(-9.411,1.194)--(-9.409,1.194)--cycle;
\draw(-9.411,1.194)--(-9.409,1.194);
\filldraw[fill opacity=0.8,fill=gray!20,draw=none](-9.405,1.186)--(-9.404,1.186)--(-9.396,1.204)--(-9.405,1.206)--cycle;
\draw(-9.404,1.186)--(-9.396,1.204);
\filldraw[fill opacity=0.8,fill=gray!20,draw=none](-9.406,1.217)--(-9.412,1.21)--(-9.411,1.207)--(-9.396,1.204)--(-9.394,1.208)--cycle;
\draw(-9.396,1.204)--(-9.394,1.208);
\filldraw[fill opacity=0.8,fill=gray!20,draw=none](-9.382,1.245)--(-9.406,1.217)--(-9.394,1.208)--(-9.378,1.245)--cycle;
\draw(-9.394,1.208)--(-9.378,1.245);
\filldraw[fill opacity=0.8,fill=gray!20,draw=none](-9.382,1.245)--(-9.378,1.245)--(-9.374,1.255)--cycle;
\draw(-9.378,1.245)--(-9.374,1.255);
\filldraw[fill opacity=0.8,fill=gray!20,draw=none](-9.404,1.187)--(-9.404,1.187)--(-9.404,1.186)--cycle;
\draw(-9.404,1.187)--(-9.404,1.186);
\filldraw[fill opacity=0.8,fill=gray!20,draw=none](-9.405,1.186)--(-9.406,1.187)--(-9.409,1.194)--(-9.405,1.192)--cycle;
\draw(-9.409,1.194)--(-9.405,1.192);
\filldraw[fill opacity=0.8,fill=gray!20,draw=none](-9.405,1.186)--(-9.405,1.192)--(-9.411,1.207)--(-9.415,1.207)--(-9.426,1.193)--(-9.433,1.177)--cycle;
\draw(-9.426,1.193)--(-9.433,1.177);
\filldraw[fill opacity=0.8,fill=gray!20,draw=none](-9.405,1.192)--(-9.405,1.206)--(-9.411,1.207)--cycle;
\filldraw[fill opacity=0.8,fill=gray!20,draw=none](-9.411,1.207)--(-9.405,1.202)--(-9.402,1.199)--(-9.393,1.187)--(-9.405,1.192)--cycle;
\draw(-9.393,1.187)--(-9.405,1.192);
\filldraw[fill opacity=0.8,fill=gray!20,draw=none](-9.412,1.21)--(-9.415,1.207)--(-9.411,1.207)--cycle;
\filldraw[fill opacity=0.8,fill=gray!20,draw=none](-9.412,1.21)--(-9.405,1.202)--(-9.411,1.207)--cycle;
\filldraw[fill opacity=0.8,fill=gray!20](-9.508,1.175)--(-8.395,1.425)--(-8.412,1.471)--(-9.525,1.221)--cycle;
\filldraw[fill opacity=0.8,fill=gray!20,draw=none](-8.541,1.483)--(-8.527,1.474)--(-8.516,1.503)--(-8.529,1.506)--(-8.545,1.504)--cycle;
\draw(-8.527,1.474)--(-8.516,1.503)--(-8.529,1.506);
\filldraw[fill opacity=0.8,fill=gray!20,draw=none](-8.501,1.269)--(-8.487,1.279)--(-8.501,1.28)--cycle;
\draw(-8.487,1.279)--(-8.501,1.28);
\filldraw[fill opacity=0.8,fill=gray!20,draw=none](-8.378,1.271)--(-8.374,1.28)--(-8.455,1.276)--(-8.453,1.257)--cycle;
\draw(-8.378,1.271)--(-8.374,1.28)--(-8.455,1.276)--(-8.453,1.257);
\filldraw[fill opacity=0.8,fill=gray!20,draw=none](-8.452,1.518)--(-8.451,1.535)--(-8.495,1.538)--(-8.513,1.508)--cycle;
\draw(-8.452,1.518)--(-8.451,1.535)--(-8.495,1.538)--(-8.513,1.508);
\filldraw[fill opacity=0.8,fill=gray!20](-8.387,1.501)--(-8.404,1.537)--(-8.451,1.535)--(-8.453,1.498)--cycle;
\filldraw[fill opacity=0.8,fill=gray!20,draw=none](-9.344,1.162)--(-9.356,1.197)--(-9.35,1.188)--cycle;
\filldraw[fill opacity=0.8,fill=gray!20,draw=none](-9.344,1.162)--(-9.35,1.188)--(-9.343,1.179)--(-9.339,1.147)--cycle;
\draw(-9.343,1.179)--(-9.339,1.147);
\filldraw[fill opacity=0.8,fill=gray!20,draw=none](-9.341,1.149)--(-9.344,1.162)--(-9.339,1.147)--cycle;
\draw(-9.339,1.147)--(-9.341,1.149);
\filldraw[fill opacity=0.8,fill=gray!20,draw=none](-9.341,1.149)--(-9.339,1.147)--(-9.341,1.134)--cycle;
\draw(-9.341,1.149)--(-9.339,1.147)--(-9.341,1.134);
\filldraw[fill opacity=0.8,fill=gray!20,draw=none](-9.341,1.145)--(-9.339,1.147)--(-9.343,1.179)--(-9.344,1.179)--cycle;
\draw(-9.341,1.145)--(-9.339,1.147)--(-9.343,1.179);
\filldraw[fill opacity=0.8,fill=gray!20,draw=none](-9.353,1.179)--(-9.343,1.179)--(-9.346,1.203)--(-9.354,1.197)--cycle;
\draw(-9.343,1.179)--(-9.346,1.203)--(-9.354,1.197);
\filldraw[fill opacity=0.8,fill=gray!20,draw=none](-9.351,1.157)--(-9.347,1.142)--(-9.341,1.145)--(-9.344,1.179)--(-9.353,1.179)--cycle;
\draw(-9.347,1.142)--(-9.341,1.145);
\filldraw[fill opacity=0.8,fill=gray!20,draw=none](-9.405,1.185)--(-9.404,1.186)--(-9.405,1.186)--cycle;
\filldraw[fill opacity=0.8,fill=gray!20,draw=none](-9.406,1.187)--(-9.384,1.149)--(-9.382,1.138)--(-9.388,1.14)--cycle;
\draw(-9.382,1.138)--(-9.388,1.14);
\filldraw[fill opacity=0.8,fill=gray!20,draw=none](-9.405,1.185)--(-9.405,1.186)--(-9.433,1.177)--(-9.443,1.156)--cycle;
\draw(-9.433,1.177)--(-9.443,1.156);
\filldraw[fill opacity=0.8,fill=gray!20,draw=none](-9.384,1.15)--(-9.384,1.149)--(-9.405,1.186)--(-9.405,1.192)--(-9.393,1.187)--cycle;
\draw(-9.405,1.192)--(-9.393,1.187);
\filldraw[fill opacity=0.8,fill=gray!20,draw=none](-9.465,1.156)--(-9.443,1.156)--(-9.433,1.177)--(-9.441,1.179)--cycle;
\draw(-9.443,1.156)--(-9.433,1.177);
\filldraw[fill opacity=0.8,fill=gray!20,draw=none](-9.443,1.156)--(-9.465,1.156)--(-9.499,1.124)--cycle;
\filldraw[fill opacity=0.8,fill=gray!20,draw=none](-9.441,1.179)--(-9.433,1.177)--(-9.426,1.193)--cycle;
\draw(-9.433,1.177)--(-9.426,1.193);
\filldraw[fill opacity=0.8,fill=gray!20](-9.499,1.124)--(-8.386,1.374)--(-8.395,1.425)--(-9.508,1.175)--cycle;
\filldraw[fill opacity=0.8,fill=gray!20,draw=none](-8.374,1.28)--(-8.366,1.319)--(-8.401,1.318)--(-8.455,1.298)--(-8.455,1.276)--cycle;
\draw(-8.455,1.298)--(-8.455,1.276)--(-8.374,1.28)--(-8.366,1.319)--(-8.401,1.318);
\filldraw[fill opacity=0.8,fill=gray!20,draw=none](-8.49,1.304)--(-8.501,1.28)--(-8.487,1.279)--(-8.484,1.281)--cycle;
\draw(-8.501,1.28)--(-8.487,1.279);
\filldraw[fill opacity=0.8,fill=gray!20,draw=none](-8.529,1.506)--(-8.516,1.503)--(-8.513,1.508)--cycle;
\draw(-8.529,1.506)--(-8.516,1.503)--(-8.513,1.508);
\filldraw[fill opacity=0.8,fill=gray!20,draw=none](-8.487,1.279)--(-8.455,1.276)--(-8.455,1.298)--cycle;
\draw(-8.487,1.279)--(-8.455,1.276)--(-8.455,1.298);
\filldraw[fill opacity=0.8,fill=gray!20,draw=none](-8.536,1.461)--(-8.532,1.46)--(-8.527,1.474)--(-8.541,1.483)--cycle;
\draw(-8.536,1.461)--(-8.532,1.46)--(-8.527,1.474);
\filldraw[fill opacity=0.8,fill=gray!20,draw=none](-8.49,1.304)--(-8.484,1.281)--(-8.455,1.298)--(-8.456,1.315)--(-8.484,1.317)--cycle;
\draw(-8.455,1.298)--(-8.456,1.315)--(-8.484,1.317);
\filldraw[fill opacity=0.8,fill=gray!20,draw=none](-8.453,1.498)--(-8.452,1.518)--(-8.513,1.508)--(-8.516,1.503)--cycle;
\draw(-8.513,1.508)--(-8.516,1.503)--(-8.453,1.498)--(-8.452,1.518);
\filldraw[fill opacity=0.8,fill=gray!20,draw=none](-8.454,1.462)--(-8.453,1.498)--(-8.516,1.503)--(-8.527,1.474)--cycle;
\draw(-8.454,1.462)--(-8.453,1.498)--(-8.516,1.503)--(-8.527,1.474);
\filldraw[fill opacity=0.8,fill=gray!20,draw=none](-8.49,1.304)--(-8.484,1.317)--(-8.494,1.318)--cycle;
\draw(-8.484,1.317)--(-8.494,1.318);
\filldraw[fill opacity=0.8,fill=gray!20](-8.374,1.458)--(-8.387,1.501)--(-8.453,1.498)--(-8.455,1.455)--cycle;
\filldraw[fill opacity=0.8,fill=gray!20,draw=none](-8.366,1.319)--(-8.363,1.364)--(-8.446,1.361)--(-8.456,1.354)--(-8.456,1.315)--cycle;
\draw(-8.456,1.354)--(-8.456,1.315)--(-8.366,1.319)--(-8.363,1.364)--(-8.446,1.361);
\filldraw[fill opacity=0.8,fill=gray!20,draw=none](-8.524,1.46)--(-8.523,1.474)--(-8.527,1.474)--(-8.532,1.46)--cycle;
\draw(-8.527,1.474)--(-8.532,1.46)--(-8.524,1.46);
\filldraw[fill opacity=0.8,fill=gray!20,draw=none](-8.524,1.46)--(-8.455,1.455)--(-8.454,1.462)--(-8.523,1.474)--cycle;
\draw(-8.524,1.46)--(-8.455,1.455)--(-8.454,1.462);
\filldraw[fill opacity=0.8,fill=gray!20,draw=none](-8.533,1.457)--(-8.532,1.46)--(-8.536,1.461)--cycle;
\draw(-8.533,1.457)--(-8.532,1.46)--(-8.536,1.461);
\filldraw[fill opacity=0.8,fill=gray!20,draw=none](-8.52,1.447)--(-8.524,1.46)--(-8.532,1.46)--(-8.533,1.457)--cycle;
\draw(-8.524,1.46)--(-8.532,1.46)--(-8.533,1.457);
\filldraw[fill opacity=0.8,fill=gray!20,draw=none](-8.49,1.351)--(-8.494,1.318)--(-8.475,1.317)--cycle;
\draw(-8.494,1.318)--(-8.475,1.317);
\filldraw[fill opacity=0.8,fill=gray!20,draw=none](-8.401,1.318)--(-8.456,1.315)--(-8.455,1.298)--cycle;
\draw(-8.401,1.318)--(-8.456,1.315)--(-8.455,1.298);
\filldraw[fill opacity=0.8,fill=gray!20,draw=none](-8.52,1.447)--(-8.496,1.431)--(-8.489,1.457)--(-8.524,1.46)--cycle;
\draw(-8.489,1.457)--(-8.524,1.46);
\filldraw[fill opacity=0.8,fill=gray!20,draw=none](-8.507,1.411)--(-8.5,1.411)--(-8.496,1.431)--(-8.52,1.447)--cycle;
\draw(-8.507,1.411)--(-8.5,1.411);
\filldraw[fill opacity=0.8,fill=gray!20,draw=none](-8.49,1.351)--(-8.475,1.317)--(-8.456,1.315)--(-8.456,1.36)--(-8.489,1.363)--cycle;
\draw(-8.475,1.317)--(-8.456,1.315)--(-8.456,1.36)--(-8.489,1.363);
\filldraw[fill opacity=0.8,fill=gray!20](-8.366,1.412)--(-8.374,1.458)--(-8.455,1.455)--(-8.456,1.408)--cycle;
\filldraw[fill opacity=0.8,fill=gray!20](-8.363,1.364)--(-8.366,1.412)--(-8.456,1.408)--(-8.456,1.36)--cycle;
\filldraw[fill opacity=0.8,fill=gray!20,draw=none](-8.49,1.351)--(-8.489,1.363)--(-8.495,1.363)--cycle;
\draw(-8.489,1.363)--(-8.495,1.363);
\filldraw[fill opacity=0.8,fill=gray!20,draw=none](-8.496,1.431)--(-8.463,1.408)--(-8.456,1.408)--(-8.455,1.455)--(-8.489,1.457)--cycle;
\draw(-8.463,1.408)--(-8.456,1.408)--(-8.455,1.455)--(-8.489,1.457);
\filldraw[fill opacity=0.8,fill=gray!20,draw=none](-8.495,1.363)--(-8.47,1.361)--(-8.499,1.401)--cycle;
\draw(-8.495,1.363)--(-8.47,1.361);
\filldraw[fill opacity=0.8,fill=gray!20,draw=none](-8.499,1.401)--(-8.5,1.411)--(-8.507,1.411)--cycle;
\draw(-8.5,1.411)--(-8.507,1.411);
\filldraw[fill opacity=0.8,fill=gray!20,draw=none](-8.5,1.411)--(-8.463,1.408)--(-8.496,1.431)--cycle;
\draw(-8.5,1.411)--(-8.463,1.408);
\filldraw[fill opacity=0.8,fill=gray!20,draw=none](-8.499,1.401)--(-8.47,1.361)--(-8.456,1.36)--(-8.456,1.408)--(-8.5,1.411)--cycle;
\draw(-8.47,1.361)--(-8.456,1.36)--(-8.456,1.408)--(-8.5,1.411);
\filldraw[fill opacity=0.5,fill=gray!20](-9.079,-.407)--(-8.906,-.482)--(-9.288,-.579)--(-9.461,-.504)--cycle;
\filldraw[fill opacity=0.8,fill=gray!20,draw=none](-8.446,1.361)--(-8.456,1.36)--(-8.456,1.354)--cycle;
\draw(-8.446,1.361)--(-8.456,1.36)--(-8.456,1.354);
\filldraw[fill opacity=0.5,fill=gray!20](-8.287,2.518)--(-8.417,2.361)--(-8.182,2.053)--(-8.022,2.171)--cycle;
\filldraw[fill opacity=0.5,fill=gray!20](-8.181,2.563)--(-8.286,2.518)--(-8.022,2.171)--(-7.904,2.2)--cycle;
\filldraw[fill opacity=0.5,fill=gray!20](-8.083,2.588)--(-8.181,2.563)--(-7.904,2.2)--(-7.798,2.212)--cycle;
\filldraw[fill opacity=0.5,fill=gray!20](-7.997,2.591)--(-8.083,2.588)--(-7.798,2.212)--(-7.706,2.209)--cycle;
\filldraw[fill opacity=0.5,fill=gray!20](-7.825,2.317)--(-8.287,2.518)--(-8.022,2.171)--(-7.561,1.97)--cycle;
\filldraw[fill opacity=0.8,fill=gray!20,draw=none](-6.147,.454)--(-6.274,.485)--(-6.284,.512)--(-6.245,.544)--(-6.195,.561)--(-6.145,.555)--(-6.142,.554)--cycle;
\draw(-6.284,.512)--(-6.245,.544)--(-6.195,.561)--(-6.145,.555)--(-6.142,.554);
\filldraw[fill opacity=0.8,fill=gray!20,draw=none](-6.141,.553)--(-6.145,.555)--(-6.142,.554)--cycle;
\draw(-6.141,.553)--(-6.145,.555)--(-6.142,.554);
\filldraw[fill opacity=0.8,fill=gray!20,draw=none](-6.049,.41)--(-6.067,.367)--(-6.067,.458)--cycle;
\draw(-6.067,.367)--(-6.067,.458);
\filldraw[fill opacity=0.8,fill=gray!20,draw=none](-6.114,.454)--(-6.118,.448)--(-6.161,.464)--(-6.146,.487)--cycle;
\draw(-6.114,.454)--(-6.118,.448);
\draw(-6.161,.464)--(-6.146,.487);
\filldraw[fill opacity=0.8,fill=gray!20,draw=none](-6.147,.454)--(-6.253,.452)--(-6.264,.46)--(-6.274,.485)--cycle;
\filldraw[fill opacity=0.8,fill=gray!20,draw=none](-6.118,.448)--(-6.142,.412)--(-6.177,.44)--(-6.161,.464)--cycle;
\draw(-6.118,.448)--(-6.142,.412);
\draw(-6.177,.44)--(-6.161,.464);
\filldraw[fill opacity=0.8,fill=gray!20,draw=none](-6.067,.472)--(-6.067,.434)--(-6.123,.429)--(-6.123,.471)--cycle;
\draw(-6.067,.472)--(-6.067,.434);
\draw(-6.123,.429)--(-6.123,.471);
\filldraw[fill opacity=0.8,fill=gray!20,draw=none](-6.178,.498)--(-6.151,.552)--(-6.14,.555)--(-6.142,.549)--(-6.181,.491)--cycle;
\draw(-6.142,.549)--(-6.181,.491);
\filldraw[fill opacity=0.8,fill=gray!20,draw=none](-6.095,.526)--(-6.102,.529)--(-6.141,.553)--(-6.142,.554)--(-6.133,.55)--cycle;
\draw(-6.095,.526)--(-6.102,.529)--(-6.141,.553);
\draw(-6.142,.554)--(-6.133,.55);
\filldraw[fill opacity=0.8,fill=gray!20,draw=none](-6.017,.585)--(-6.017,.523)--(-6.067,.535)--(-6.067,.591)--cycle;
\draw(-6.067,.535)--(-6.067,.591)--(-6.017,.585)--(-6.017,.523);
\filldraw[fill opacity=0.8,fill=gray!20,draw=none](-6.159,.551)--(-6.156,.55)--(-6.142,.554)--(-6.145,.555)--(-6.16,.557)--cycle;
\draw(-6.142,.554)--(-6.145,.555)--(-6.16,.557);
\filldraw[fill opacity=0.8,fill=gray!20,draw=none](-6.14,.555)--(-6.156,.55)--(-6.15,.563)--(-6.105,.631)--cycle;
\draw(-6.15,.563)--(-6.105,.631);
\filldraw[fill opacity=0.8,fill=gray!20,draw=none](-6.146,.487)--(-6.161,.464)--(-6.184,.481)--(-6.183,.489)--(-6.142,.549)--cycle;
\draw(-6.146,.487)--(-6.161,.464);
\draw(-6.183,.489)--(-6.142,.549);
\filldraw[fill opacity=0.8,fill=gray!20,draw=none](-6.077,.518)--(-6.095,.526)--(-6.133,.55)--(-6.123,.546)--cycle;
\draw(-6.077,.518)--(-6.095,.526);
\draw(-6.133,.55)--(-6.123,.546);
\filldraw[fill opacity=0.8,fill=gray!20,draw=none](-6.178,.498)--(-6.156,.55)--(-6.151,.552)--cycle;
\filldraw[fill opacity=0.8,fill=gray!20,draw=none](-6.123,.546)--(-6.142,.554)--(-6.156,.55)--cycle;
\draw(-6.123,.546)--(-6.142,.554);
\filldraw[fill opacity=0.8,fill=gray!20,draw=none](-6.067,.591)--(-6.067,.513)--(-6.123,.546)--(-6.123,.593)--cycle;
\draw(-6.123,.546)--(-6.123,.593)--(-6.067,.591)--(-6.067,.513);
\filldraw[fill opacity=0.8,fill=gray!20,draw=none](-6.159,.551)--(-6.127,.592)--(-6.123,.593)--(-6.123,.546)--cycle;
\draw(-6.127,.592)--(-6.123,.593)--(-6.123,.546);
\filldraw[fill opacity=0.8,fill=gray!20,draw=none](-6.129,.583)--(-6.107,.571)--(-6.078,.568)--(-6.014,.567)--(-6.002,.567)--(-5.976,.57)--(-5.982,.575)--(-6.017,.585)--(-6.067,.591)--(-6.123,.593)--(-6.127,.592)--cycle;
\draw(-5.976,.57)--(-5.982,.575)--(-6.017,.585)--(-6.067,.591)--(-6.123,.593)--(-6.127,.592);
\filldraw[fill opacity=0.8,fill=gray!20,draw=none](-6.039,.501)--(-6.051,.507)--(-6.102,.537)--(-6.082,.528)--cycle;
\draw(-6.039,.501)--(-6.051,.507);
\draw(-6.102,.537)--(-6.082,.528);
\filldraw[fill opacity=0.8,fill=gray!20,draw=none](-6.051,.507)--(-6.077,.518)--(-6.123,.546)--(-6.102,.537)--cycle;
\draw(-6.051,.507)--(-6.077,.518);
\draw(-6.123,.546)--(-6.102,.537);
\filldraw[fill opacity=0.8,fill=gray!20,draw=none](-6.067,.513)--(-6.067,.472)--(-6.123,.471)--(-6.123,.546)--cycle;
\draw(-6.067,.513)--(-6.067,.472);
\draw(-6.123,.471)--(-6.123,.546);
\filldraw[fill opacity=0.8,fill=gray!20,draw=none](-6.159,.551)--(-6.123,.546)--(-6.123,.471)--(-6.178,.479)--(-6.178,.526)--cycle;
\draw(-6.123,.546)--(-6.123,.471);
\draw(-6.178,.479)--(-6.178,.526);
\filldraw[fill opacity=0.8,fill=gray!20,draw=none](-6.161,.464)--(-6.177,.44)--(-6.185,.459)--(-6.184,.481)--cycle;
\draw(-6.161,.464)--(-6.177,.44);
\filldraw[fill opacity=0.8,fill=gray!20,draw=none](-6.034,.499)--(-6.039,.501)--(-6.082,.528)--(-6.064,.52)--cycle;
\draw(-6.034,.499)--(-6.039,.501);
\draw(-6.082,.528)--(-6.064,.52);
\filldraw[fill opacity=0.8,fill=gray!20,draw=none](-6.094,.518)--(-6.041,.508)--(-6.036,.508)--(-6.082,.528)--(-6.142,.538)--(-6.103,.521)--cycle;
\draw(-6.036,.508)--(-6.082,.528);
\draw(-6.142,.538)--(-6.103,.521);
\filldraw[fill opacity=0.8,fill=gray!20,draw=none](-5.819,.841)--(-6.08,.45)--(-6.039,.419)--(-5.168,1.724)--cycle;
\draw(-5.819,.841)--(-6.08,.45)--(-6.039,.419)--(-5.168,1.724);
\filldraw[fill opacity=0.8,fill=gray!20](-8.606,4.29)--(-8.595,4.29)--(-8.817,3.704)--cycle;
\filldraw[fill opacity=0.8,fill=gray!20](-8.153,4.112)--(-8.06,4.079)--(-8.004,4.062)--(-7.993,4.062)--(-8.029,4.079)--(-8.106,4.111)--(-8.213,4.153)--(-8.332,4.198)--(-8.447,4.24)--(-8.54,4.273)--(-8.595,4.29)--(-8.606,4.29)--(-8.571,4.273)--(-8.494,4.241)--(-8.387,4.199)--(-8.267,4.154)--cycle;
\filldraw[fill opacity=0.5,fill=gray!20](-8.417,2.361)--(-8.244,2.286)--(-8.009,1.978)--(-8.182,2.053)--cycle;
\filldraw[fill opacity=0.8,fill=gray!20,draw=none](-9.145,1.127)--(-9.094,1.125)--(-9.134,1.116)--cycle;
\draw(-9.094,1.125)--(-9.134,1.116);
\filldraw[fill opacity=0.8,fill=gray!20,draw=none](-9.134,1.116)--(-9.094,1.125)--(-9.03,1.134)--(-9.144,1.109)--cycle;
\draw(-9.134,1.116)--(-9.094,1.125);
\draw(-9.03,1.134)--(-9.144,1.109);
\filldraw[fill opacity=0.5,fill=gray!20](-9.461,-.705)--(-9.461,-.504)--(-9.844,-.49)--(-9.891,-.69)--cycle;
\filldraw[fill opacity=0.8,fill=gray!20,draw=none](-9.144,1.109)--(-9.03,1.134)--(-9.006,1.155)--(-9.183,1.115)--cycle;
\draw(-9.144,1.109)--(-9.03,1.134);
\draw(-9.006,1.155)--(-9.183,1.115);
\filldraw[fill opacity=0.8,fill=gray!20,draw=none](-9.183,1.115)--(-9.006,1.155)--(-9.027,1.184)--(-9.245,1.135)--cycle;
\draw(-9.183,1.115)--(-9.006,1.155);
\draw(-9.027,1.184)--(-9.245,1.135);
\filldraw[fill opacity=0.8,fill=gray!20](-9.735,1.231)--(-9.717,1.28)--(-9.643,1.294)--(-9.653,1.247)--cycle;
\filldraw[fill opacity=0.5,fill=gray!20](-9.461,-.504)--(-9.288,-.579)--(-9.671,-.566)--(-9.844,-.49)--cycle;
\filldraw[fill opacity=0.8,fill=gray!20](-9.65,.978)--(-9.688,1.017)--(-9.627,1.029)--(-9.607,.986)--cycle;
\filldraw[fill opacity=0.8,fill=gray!20,draw=none](-9.144,1.109)--(-9.183,1.115)--(-9.217,1.108)--cycle;
\draw(-9.183,1.115)--(-9.217,1.108);
\filldraw[fill opacity=0.8,fill=gray!20,draw=none](-9.505,1.31)--(-9.498,1.305)--(-9.547,1.299)--(-9.548,1.321)--cycle;
\draw(-9.547,1.299)--(-9.548,1.321);
\filldraw[fill opacity=0.8,fill=gray!20,draw=none](-9.56,.988)--(-9.607,.986)--(-9.616,1.005)--cycle;
\draw(-9.56,.988)--(-9.607,.986)--(-9.616,1.005);
\filldraw[fill opacity=0.8,fill=gray!20](-9.583,.954)--(-9.607,.986)--(-9.551,.988)--(-9.555,.955)--cycle;
\filldraw[fill opacity=0.8,fill=gray!20,draw=none](-9.555,.955)--(-9.552,.986)--(-9.506,.976)--(-9.527,.953)--cycle;
\draw(-9.506,.976)--(-9.527,.953)--(-9.555,.955)--(-9.552,.986);
\filldraw[fill opacity=0.8,fill=gray!20](-9.717,1.28)--(-9.688,1.32)--(-9.627,1.332)--(-9.643,1.294)--cycle;
\filldraw[fill opacity=0.8,fill=gray!20](-9.605,.95)--(-9.65,.978)--(-9.607,.986)--(-9.583,.954)--cycle;
\filldraw[fill opacity=0.8,fill=gray!20,draw=none](-9.505,1.31)--(-9.548,1.321)--(-9.549,1.335)--(-9.544,1.335)--cycle;
\draw(-9.548,1.321)--(-9.549,1.335)--(-9.544,1.335);
\filldraw[fill opacity=0.8,fill=gray!20,draw=none](-9.505,1.31)--(-9.491,1.306)--(-9.498,1.305)--cycle;
\filldraw[fill opacity=0.8,fill=gray!20](-8.751,3.042)--(-8.767,3.093)--(-8.67,3.098)--(-8.672,3.046)--cycle;
\filldraw[fill opacity=0.8,fill=gray!20](-8.752,3.105)--(-8.761,3.152)--(-8.671,3.156)--(-8.672,3.109)--cycle;
\filldraw[fill opacity=0.8,fill=gray!20](-8.739,3.062)--(-8.752,3.105)--(-8.672,3.109)--(-8.674,3.065)--cycle;
\filldraw[fill opacity=0.8,fill=gray!20,draw=none](-9.567,1.291)--(-9.547,1.336)--(-9.501,1.319)--(-9.508,1.303)--cycle;
\draw(-9.567,1.291)--(-9.547,1.336);
\draw(-9.501,1.319)--(-9.508,1.303);
\filldraw[fill opacity=0.8,fill=gray!20](-8.67,3.098)--(-8.669,3.154)--(-8.565,3.146)--(-8.577,3.091)--cycle;
\filldraw[fill opacity=0.8,fill=gray!20](-8.669,3.154)--(-8.668,3.211)--(-8.561,3.203)--(-8.565,3.146)--cycle;
\filldraw[fill opacity=0.8,fill=gray!20](-8.731,2.999)--(-8.751,3.042)--(-8.672,3.046)--(-8.675,3.002)--cycle;
\filldraw[fill opacity=0.8,fill=gray!20](-8.672,3.109)--(-8.671,3.156)--(-8.584,3.149)--(-8.594,3.103)--cycle;
\filldraw[fill opacity=0.8,fill=gray!20](-8.671,3.156)--(-8.671,3.203)--(-8.581,3.197)--(-8.584,3.149)--cycle;
\filldraw[fill opacity=0.8,fill=gray!20](-8.672,3.046)--(-8.67,3.098)--(-8.577,3.091)--(-8.596,3.04)--cycle;
\filldraw[fill opacity=0.8,fill=gray!20](-8.674,3.065)--(-8.672,3.109)--(-8.594,3.103)--(-8.61,3.061)--cycle;
\filldraw[fill opacity=0.8,fill=gray!20,draw=none](-9.508,1.303)--(-8.767,3.007)--(-8.741,2.997)--(-8.721,2.983)--(-9.448,1.313)--cycle;
\draw(-9.508,1.303)--(-8.767,3.007);
\draw(-8.721,2.983)--(-9.448,1.313);
\filldraw[fill opacity=0.8,fill=gray!20,draw=none](-9.547,1.336)--(-8.811,3.028)--(-8.808,3.027)--(-8.768,3.002)--(-9.501,1.319)--cycle;
\draw(-9.547,1.336)--(-8.811,3.028);
\draw(-8.768,3.002)--(-9.501,1.319);
\filldraw[fill opacity=0.8,fill=gray!20,draw=none](-9.491,1.306)--(-9.505,1.31)--(-9.544,1.335)--(-9.472,1.33)--(-9.462,1.31)--cycle;
\draw(-9.544,1.335)--(-9.472,1.33)--(-9.462,1.31);
\filldraw[fill opacity=0.8,fill=gray!20](-8.675,3.002)--(-8.672,3.046)--(-8.596,3.04)--(-8.621,2.998)--cycle;
\filldraw[fill opacity=0.8,fill=gray!20,draw=none](-9.448,1.313)--(-8.723,2.979)--(-8.696,2.974)--(-8.68,2.963)--(-9.394,1.321)--cycle;
\draw(-9.448,1.313)--(-8.723,2.979);
\draw(-8.68,2.963)--(-9.394,1.321);
\filldraw[fill opacity=0.8,fill=gray!20](-9.77,1.097)--(-9.777,1.153)--(-9.741,1.176)--(-9.735,1.12)--cycle;
\filldraw[fill opacity=0.8,fill=gray!20](-9.748,1.045)--(-9.77,1.097)--(-9.735,1.12)--(-9.717,1.066)--cycle;
\filldraw[fill opacity=0.8,fill=gray!20](-9.713,1.001)--(-9.748,1.045)--(-9.717,1.066)--(-9.688,1.017)--cycle;
\filldraw[fill opacity=0.8,fill=gray!20,draw=none](-9.527,.953)--(-9.506,.976)--(-9.469,.97)--(-9.507,.949)--cycle;
\draw(-9.469,.97)--(-9.507,.949)--(-9.527,.953)--(-9.506,.976);
\filldraw[fill opacity=0.8,fill=gray!20](-9.777,1.153)--(-9.77,1.208)--(-9.735,1.231)--(-9.741,1.176)--cycle;
\filldraw[fill opacity=0.8,fill=gray!20](-9.668,.966)--(-9.713,1.001)--(-9.688,1.017)--(-9.65,.978)--cycle;
\filldraw[fill opacity=0.8,fill=gray!20,draw=none](-9.546,1.335)--(-9.549,1.335)--(-9.549,1.337)--cycle;
\draw(-9.546,1.335)--(-9.549,1.335)--(-9.549,1.337);
\filldraw[fill opacity=0.8,fill=gray!20,draw=none](-9.573,1.334)--(-9.585,1.347)--(-9.549,1.337)--(-9.549,1.335)--cycle;
\draw(-9.549,1.337)--(-9.549,1.335)--(-9.573,1.334);
\filldraw[fill opacity=0.8,fill=gray!20,draw=none](-9.573,1.334)--(-9.627,1.332)--(-9.609,1.354)--(-9.585,1.347)--cycle;
\draw(-9.573,1.334)--(-9.627,1.332)--(-9.609,1.354);
\filldraw[fill opacity=0.8,fill=gray!20,draw=none](-9.615,1.279)--(-9.574,1.373)--(-9.547,1.337)--(-9.567,1.291)--cycle;
\draw(-9.615,1.279)--(-9.574,1.373);
\draw(-9.547,1.337)--(-9.567,1.291);
\filldraw[fill opacity=0.8,fill=gray!20,draw=none](-9.574,1.373)--(-8.929,2.854)--(-8.843,2.953)--(-9.547,1.337)--cycle;
\draw(-9.574,1.373)--(-8.929,2.854);
\draw(-8.843,2.953)--(-9.547,1.337);
\filldraw[fill opacity=0.8,fill=gray!20,draw=none](-9.546,1.335)--(-9.549,1.337)--(-9.551,1.359)--(-9.497,1.356)--(-9.472,1.33)--cycle;
\draw(-9.549,1.337)--(-9.551,1.359)--(-9.497,1.356)--(-9.472,1.33)--(-9.546,1.335);
\filldraw[fill opacity=0.8,fill=gray!20,draw=none](-9.499,1.124)--(-9.508,1.175)--(-9.525,1.221)--(-9.547,1.257)--(-9.566,1.273)--(-9.575,1.277)--(-9.592,1.277)--(-9.608,1.258)--(-9.617,1.222)--(-9.617,1.176)--cycle;
\draw(-9.499,1.124)--(-9.508,1.175)--(-9.525,1.221)--(-9.547,1.257)--(-9.566,1.273);
\draw(-9.575,1.277)--(-9.592,1.277)--(-9.608,1.258)--(-9.617,1.222)--(-9.617,1.176);
\filldraw[fill opacity=0.8,fill=gray!20,draw=none](-9.499,1.124)--(-9.544,1.144)--(-9.591,1.164)--(-9.633,1.183)--(-9.659,1.194)--(-9.666,1.197)--(-9.678,1.202)--(-9.674,1.201)--(-9.653,1.191)--(-9.617,1.176)--cycle;
\draw(-9.499,1.124)--(-9.544,1.144)--(-9.591,1.164)--(-9.633,1.183)--(-9.659,1.194);
\draw(-9.666,1.197)--(-9.678,1.202)--(-9.674,1.201)--(-9.653,1.191)--(-9.617,1.176);
\filldraw[fill opacity=0.8,fill=gray!20](-8.865,3.19)--(-8.859,3.244)--(-8.776,3.26)--(-8.78,3.206)--cycle;
\filldraw[fill opacity=0.8,fill=gray!20,draw=none](-9.585,1.347)--(-9.609,1.354)--(-9.607,1.357)--(-9.594,1.358)--cycle;
\draw(-9.609,1.354)--(-9.607,1.357)--(-9.594,1.358);
\filldraw[fill opacity=0.8,fill=gray!20,draw=none](-9.594,1.358)--(-9.607,1.357)--(-9.601,1.36)--cycle;
\draw(-9.594,1.358)--(-9.607,1.357)--(-9.601,1.36);
\filldraw[fill opacity=0.8,fill=gray!20,draw=none](-9.65,1.349)--(-9.626,1.357)--(-9.605,1.359)--(-9.604,1.358)--(-9.607,1.357)--cycle;
\draw(-9.604,1.358)--(-9.607,1.357)--(-9.65,1.349)--(-9.626,1.357);
\filldraw[fill opacity=0.8,fill=gray!20,draw=none](-9.605,1.359)--(-9.601,1.36)--(-9.604,1.358)--cycle;
\draw(-9.601,1.36)--(-9.604,1.358);
\filldraw[fill opacity=0.8,fill=gray!20,draw=none](-9.674,1.201)--(-9.605,1.359)--(-9.601,1.36)--(-9.582,1.354)--(-9.653,1.191)--cycle;
\draw(-9.582,1.354)--(-9.653,1.191)--(-9.674,1.201)--(-9.605,1.359);
\filldraw[fill opacity=0.8,fill=gray!20,draw=none](-9.601,1.36)--(-9.578,1.362)--(-9.582,1.354)--cycle;
\draw(-9.578,1.362)--(-9.582,1.354);
\filldraw[fill opacity=0.8,fill=gray!20,draw=none](-9.585,1.347)--(-9.594,1.358)--(-9.551,1.359)--(-9.549,1.337)--cycle;
\draw(-9.594,1.358)--(-9.551,1.359)--(-9.549,1.337);
\filldraw[fill opacity=0.8,fill=gray!20,draw=none](-9.442,1.31)--(-9.462,1.31)--(-9.472,1.33)--(-9.436,1.321)--cycle;
\draw(-9.462,1.31)--(-9.472,1.33)--(-9.436,1.321);
\filldraw[fill opacity=0.8,fill=gray!20,draw=none](-9.394,1.321)--(-8.68,2.963)--(-8.666,2.961)--(-8.65,2.947)--(-9.357,1.325)--cycle;
\draw(-9.394,1.321)--(-8.68,2.963);
\draw(-8.65,2.947)--(-9.357,1.325);
\filldraw[fill opacity=0.8,fill=gray!20](-9.77,1.208)--(-9.748,1.26)--(-9.717,1.28)--(-9.735,1.231)--cycle;
\filldraw[fill opacity=0.8,fill=gray!20](-9.688,1.32)--(-9.65,1.349)--(-9.607,1.357)--(-9.627,1.332)--cycle;
\filldraw[fill opacity=0.8,fill=gray!20](-9.558,.936)--(-9.583,.954)--(-9.555,.955)--(-9.558,.936)--cycle;
\filldraw[fill opacity=0.8,fill=gray!20](-9.558,.936)--(-9.555,.955)--(-9.527,.953)--(-9.558,.936)--cycle;
\filldraw[fill opacity=0.8,fill=gray!20](-9.615,.944)--(-9.668,.966)--(-9.65,.978)--(-9.605,.95)--cycle;
\filldraw[fill opacity=0.5,fill=gray!20](-8.631,2.768)--(-8.723,2.584)--(-8.417,2.361)--(-8.287,2.518)--cycle;
\filldraw[fill opacity=0.8,fill=gray!20](-9.558,.936)--(-9.605,.95)--(-9.583,.954)--(-9.558,.936)--cycle;
\filldraw[fill opacity=0.8,fill=gray!20,draw=none](-9.217,1.108)--(-9.183,1.115)--(-9.245,1.135)--(-9.301,1.122)--cycle;
\draw(-9.217,1.108)--(-9.183,1.115);
\draw(-9.245,1.135)--(-9.301,1.122);
\filldraw[fill opacity=0.8,fill=gray!20](-9.472,1.33)--(-9.497,1.356)--(-9.46,1.346)--(-9.419,1.317)--cycle;
\filldraw[fill opacity=0.8,fill=gray!20](-9.558,.936)--(-9.527,.953)--(-9.507,.949)--(-9.558,.936)--cycle;
\filldraw[fill opacity=0.8,fill=gray!20,draw=none](-9.469,.97)--(-9.454,.969)--(-9.448,.963)--(-9.463,.957)--cycle;
\draw(-9.454,.969)--(-9.448,.963)--(-9.463,.957);
\filldraw[fill opacity=0.8,fill=gray!20,draw=none](-9.507,.949)--(-9.469,.97)--(-9.463,.957)--(-9.501,.942)--cycle;
\draw(-9.463,.957)--(-9.501,.942)--(-9.507,.949)--(-9.469,.97);
\filldraw[fill opacity=0.8,fill=gray!20,draw=none](-9.601,1.36)--(-9.605,1.361)--(-8.878,3.03)--(-8.862,3.045)--(-8.852,3.031)--(-9.578,1.362)--cycle;
\draw(-9.605,1.361)--(-8.878,3.03);
\draw(-8.852,3.031)--(-9.578,1.362);
\filldraw[fill opacity=0.8,fill=gray!20,draw=none](-9.442,1.31)--(-9.436,1.321)--(-9.419,1.317)--(-9.415,1.311)--cycle;
\draw(-9.436,1.321)--(-9.419,1.317)--(-9.415,1.311);
\filldraw[fill opacity=0.8,fill=gray!20](-9.748,1.26)--(-9.713,1.304)--(-9.688,1.32)--(-9.717,1.28)--cycle;
\filldraw[fill opacity=0.8,fill=gray!20,draw=none](-9.217,1.108)--(-9.301,1.122)--(-9.341,1.113)--cycle;
\draw(-9.301,1.122)--(-9.341,1.113);
\filldraw[fill opacity=0.8,fill=gray!20](-9.558,.936)--(-9.615,.944)--(-9.605,.95)--(-9.558,.936)--cycle;
\filldraw[fill opacity=0.8,fill=gray!20](-9.558,.936)--(-9.507,.949)--(-9.501,.942)--(-9.558,.936)--cycle;
\filldraw[fill opacity=0.5,fill=gray!20](-8.723,2.584)--(-8.55,2.508)--(-8.244,2.286)--(-8.417,2.361)--cycle;
\filldraw[fill opacity=0.8,fill=gray!20,draw=none](-9.594,1.358)--(-9.601,1.36)--(-9.583,1.368)--(-9.555,1.369)--(-9.551,1.359)--cycle;
\draw(-9.601,1.36)--(-9.583,1.368)--(-9.555,1.369)--(-9.551,1.359)--(-9.594,1.358);
\filldraw[fill opacity=0.8,fill=gray!20](-9.551,1.359)--(-9.555,1.369)--(-9.527,1.367)--(-9.497,1.356)--cycle;
\filldraw[fill opacity=0.8,fill=gray!20,draw=none](-9.357,1.325)--(-9.34,1.323)--(-9.349,1.301)--cycle;
\draw(-9.34,1.323)--(-9.349,1.301);
\filldraw[fill opacity=0.8,fill=gray!20](-8.577,3.091)--(-8.565,3.146)--(-8.492,3.129)--(-8.512,3.075)--cycle;
\filldraw[fill opacity=0.8,fill=gray!20](-8.596,3.04)--(-8.577,3.091)--(-8.512,3.075)--(-8.543,3.027)--cycle;
\filldraw[fill opacity=0.8,fill=gray!20](-8.621,2.998)--(-8.596,3.04)--(-8.543,3.027)--(-8.583,2.988)--cycle;
\filldraw[fill opacity=0.8,fill=gray!20](-8.594,3.103)--(-8.584,3.149)--(-8.524,3.134)--(-8.54,3.09)--cycle;
\filldraw[fill opacity=0.8,fill=gray!20](-8.61,3.061)--(-8.594,3.103)--(-8.54,3.09)--(-8.566,3.05)--cycle;
\filldraw[fill opacity=0.8,fill=gray!20,draw=none](-9.357,1.325)--(-8.65,2.947)--(-8.648,2.914)--(-9.34,1.323)--cycle;
\draw(-9.357,1.325)--(-8.65,2.947);
\draw(-8.648,2.914)--(-9.34,1.323);
\filldraw[fill opacity=0.8,fill=gray!20,draw=none](-9.451,.966)--(-9.454,.969)--(-9.443,.97)--cycle;
\draw(-9.451,.966)--(-9.454,.969);
\filldraw[fill opacity=0.8,fill=gray!20](-9.609,.937)--(-9.656,.954)--(-9.668,.966)--(-9.615,.944)--cycle;
\filldraw[fill opacity=0.8,fill=gray!20](-9.713,1.304)--(-9.668,1.337)--(-9.65,1.349)--(-9.688,1.32)--cycle;
\filldraw[fill opacity=0.8,fill=gray!20](-9.558,.936)--(-9.609,.937)--(-9.615,.944)--(-9.558,.936)--cycle;
\filldraw[fill opacity=0.8,fill=gray!20,draw=none](-9.451,.966)--(-9.443,.97)--(-9.439,.97)--(-9.448,.963)--cycle;
\draw(-9.439,.97)--(-9.448,.963)--(-9.451,.966);
\filldraw[fill opacity=0.8,fill=gray!20](-9.656,.954)--(-9.697,.983)--(-9.713,1.001)--(-9.668,.966)--cycle;
\filldraw[fill opacity=0.8,fill=gray!20](-9.558,.936)--(-9.589,.933)--(-9.609,.937)--(-9.558,.936)--cycle;
\filldraw[fill opacity=0.8,fill=gray!20](-9.558,.936)--(-9.511,.936)--(-9.533,.932)--(-9.558,.936)--cycle;
\filldraw[fill opacity=0.8,fill=gray!20](-9.497,1.356)--(-9.527,1.367)--(-9.507,1.363)--(-9.46,1.346)--cycle;
\filldraw[fill opacity=0.8,fill=gray!20](-9.558,.936)--(-9.561,.931)--(-9.589,.933)--(-9.558,.936)--cycle;
\filldraw[fill opacity=0.8,fill=gray!20](-9.558,.936)--(-9.533,.932)--(-9.561,.931)--(-9.558,.936)--cycle;
\filldraw[fill opacity=0.8,fill=gray!20](-9.558,.936)--(-9.501,.942)--(-9.511,.936)--(-9.558,.936)--cycle;
\filldraw[fill opacity=0.8,fill=gray!20](-9.697,.983)--(-9.728,1.024)--(-9.748,1.045)--(-9.713,1.001)--cycle;
\filldraw[fill opacity=0.8,fill=gray!20,draw=none](-9.501,.942)--(-9.463,.957)--(-9.47,.95)--(-9.511,.936)--cycle;
\draw(-9.47,.95)--(-9.511,.936)--(-9.501,.942)--(-9.463,.957);
\filldraw[fill opacity=0.8,fill=gray!20,draw=none](-9.605,1.359)--(-9.605,1.361)--(-9.601,1.36)--cycle;
\draw(-9.605,1.359)--(-9.605,1.361);
\filldraw[fill opacity=0.8,fill=gray!20,draw=none](-9.605,1.359)--(-9.626,1.357)--(-9.608,1.363)--cycle;
\draw(-9.626,1.357)--(-9.608,1.363);
\filldraw[fill opacity=0.8,fill=gray!20,draw=none](-9.678,1.202)--(-9.608,1.363)--(-9.6,1.37)--(-9.674,1.201)--cycle;
\draw(-9.6,1.37)--(-9.674,1.201)--(-9.678,1.202)--(-9.608,1.363);
\filldraw[fill opacity=0.8,fill=gray!20,draw=none](-9.605,1.359)--(-9.608,1.363)--(-9.605,1.364)--(-9.583,1.368)--(-9.601,1.36)--cycle;
\draw(-9.608,1.363)--(-9.605,1.364)--(-9.583,1.368)--(-9.601,1.36);
\filldraw[fill opacity=0.8,fill=gray!20,draw=none](-9.443,.97)--(-9.433,.974)--(-9.439,.97)--cycle;
\draw(-9.433,.974)--(-9.439,.97);
\filldraw[fill opacity=0.8,fill=gray!20,draw=none](-9.469,.951)--(-9.463,.957)--(-9.448,.963)--(-9.466,.951)--(-9.468,.95)--cycle;
\draw(-9.463,.957)--(-9.448,.963)--(-9.466,.951)--(-9.468,.95);
\filldraw[fill opacity=0.8,fill=gray!20,draw=none](-9.397,1.291)--(-9.415,1.311)--(-9.412,1.309)--cycle;
\filldraw[fill opacity=0.8,fill=gray!20,draw=none](-9.415,1.311)--(-9.419,1.317)--(-9.411,1.308)--cycle;
\draw(-9.415,1.311)--(-9.419,1.317)--(-9.411,1.308);
\filldraw[fill opacity=0.8,fill=gray!20,draw=none](-9.608,1.363)--(-8.987,2.789)--(-8.921,2.932)--(-9.6,1.37)--cycle;
\draw(-9.608,1.363)--(-8.987,2.789);
\draw(-8.921,2.932)--(-9.6,1.37);
\filldraw[fill opacity=0.8,fill=gray!20](-9.728,1.024)--(-9.748,1.074)--(-9.77,1.097)--(-9.748,1.045)--cycle;
\filldraw[fill opacity=0.8,fill=gray!20,draw=none](-9.448,.963)--(-9.439,.97)--(-9.428,.98)--(-9.428,.98)--(-9.466,.951)--cycle;
\draw(-9.428,.98)--(-9.428,.98)--(-9.466,.951)--(-9.448,.963)--(-9.439,.97);
\filldraw[fill opacity=0.8,fill=gray!20,draw=none](-9.411,1.308)--(-9.419,1.317)--(-9.46,1.346)--(-9.448,1.334)--(-9.41,1.305)--cycle;
\draw(-9.411,1.308)--(-9.419,1.317)--(-9.46,1.346)--(-9.448,1.334)--(-9.41,1.305);
\filldraw[fill opacity=0.8,fill=gray!20,draw=none](-9.436,.973)--(-9.439,.97)--(-9.433,.974)--(-9.415,.988)--cycle;
\draw(-9.439,.97)--(-9.433,.974);
\filldraw[fill opacity=0.8,fill=gray!20](-9.748,1.074)--(-9.754,1.129)--(-9.777,1.153)--(-9.77,1.097)--cycle;
\filldraw[fill opacity=0.8,fill=gray!20,draw=none](-9.349,1.116)--(-9.341,1.134)--(-9.339,1.147)--(-9.341,1.145)--cycle;
\draw(-9.341,1.134)--(-9.339,1.147)--(-9.341,1.145);
\filldraw[fill opacity=0.8,fill=gray!20,draw=none](-9.349,1.115)--(-9.349,1.116)--(-9.341,1.145)--(-9.347,1.142)--cycle;
\draw(-9.341,1.145)--(-9.347,1.142);
\filldraw[fill opacity=0.8,fill=gray!20,draw=none](-9.347,1.137)--(-9.301,1.122)--(-9.245,1.135)--(-9.321,1.164)--(-9.346,1.159)--cycle;
\draw(-9.301,1.122)--(-9.245,1.135);
\draw(-9.321,1.164)--(-9.346,1.159);
\filldraw[fill opacity=0.8,fill=gray!20](-9.589,.933)--(-9.618,.944)--(-9.656,.954)--(-9.609,.937)--cycle;
\filldraw[fill opacity=0.8,fill=gray!20](-8.829,3.138)--(-8.834,3.185)--(-8.763,3.199)--(-8.761,3.152)--cycle;
\filldraw[fill opacity=0.8,fill=gray!20](-8.814,3.093)--(-8.829,3.138)--(-8.761,3.152)--(-8.752,3.105)--cycle;
\filldraw[fill opacity=0.8,fill=gray!20](-8.78,3.206)--(-8.776,3.26)--(-8.669,3.265)--(-8.668,3.211)--cycle;
\filldraw[fill opacity=0.8,fill=gray!20](-8.834,3.185)--(-8.829,3.231)--(-8.761,3.244)--(-8.763,3.199)--cycle;
\filldraw[fill opacity=0.8,fill=gray!20,draw=none](-9.668,1.337)--(-9.615,1.358)--(-9.61,1.361)--(-9.608,1.363)--(-9.609,1.363)--(-9.65,1.349)--cycle;
\draw(-9.609,1.363)--(-9.65,1.349)--(-9.668,1.337)--(-9.615,1.358)--(-9.61,1.361);
\filldraw[fill opacity=0.8,fill=gray!20](-9.511,.936)--(-9.466,.951)--(-9.509,.943)--(-9.533,.932)--cycle;
\filldraw[fill opacity=0.8,fill=gray!20,draw=none](-9.372,1.259)--(-9.349,1.301)--(-9.364,1.277)--(-9.372,1.26)--cycle;
\draw(-9.364,1.277)--(-9.372,1.26);
\filldraw[fill opacity=0.8,fill=gray!20,draw=none](-9.349,1.301)--(-9.34,1.323)--(-9.347,1.317)--(-9.364,1.277)--cycle;
\draw(-9.349,1.301)--(-9.34,1.323);
\draw(-9.347,1.317)--(-9.364,1.277);
\filldraw[fill opacity=0.8,fill=gray!20,draw=none](-8.696,2.974)--(-8.677,2.97)--(-8.68,2.963)--cycle;
\draw(-8.677,2.97)--(-8.68,2.963);
\filldraw[fill opacity=0.8,fill=gray!20,draw=none](-8.677,2.969)--(-8.677,2.97)--(-8.675,2.968)--cycle;
\draw(-8.677,2.969)--(-8.677,2.97);
\filldraw[fill opacity=0.8,fill=gray!20](-8.678,2.969)--(-8.675,3.002)--(-8.621,2.998)--(-8.65,2.967)--cycle;
\filldraw[fill opacity=0.8,fill=gray!20](-8.631,3.025)--(-8.61,3.061)--(-8.566,3.05)--(-8.6,3.018)--cycle;
\filldraw[fill opacity=0.8,fill=gray!20](-8.65,2.967)--(-8.621,2.998)--(-8.583,2.988)--(-8.631,2.962)--cycle;
\filldraw[fill opacity=0.8,fill=gray!20,draw=none](-8.65,2.942)--(-8.637,2.947)--(-8.635,2.944)--(-8.648,2.914)--cycle;
\draw(-8.635,2.944)--(-8.648,2.914);
\filldraw[fill opacity=0.8,fill=gray!20,draw=none](-9.34,1.323)--(-8.635,2.943)--(-8.668,2.877)--(-9.347,1.317)--cycle;
\draw(-9.34,1.323)--(-8.635,2.943);
\draw(-8.668,2.877)--(-9.347,1.317);
\filldraw[fill opacity=0.8,fill=gray!20,draw=none](-9.436,.973)--(-9.415,.988)--(-9.412,.991)--(-9.428,.98)--cycle;
\draw(-9.412,.991)--(-9.428,.98);
\filldraw[fill opacity=0.8,fill=gray!20](-9.754,1.129)--(-9.748,1.185)--(-9.77,1.208)--(-9.777,1.153)--cycle;
\filldraw[fill opacity=0.8,fill=gray!20](-9.583,1.368)--(-9.558,1.364)--(-9.558,1.364)--(-9.555,1.369)--cycle;
\filldraw[fill opacity=0.8,fill=gray!20](-9.555,1.369)--(-9.558,1.364)--(-9.558,1.364)--(-9.527,1.367)--cycle;
\filldraw[fill opacity=0.8,fill=gray!20,draw=none](-9.397,1.291)--(-9.412,1.309)--(-9.411,1.308)--(-9.403,1.299)--(-9.384,1.276)--cycle;
\draw(-9.411,1.308)--(-9.403,1.299)--(-9.384,1.276);
\filldraw[fill opacity=0.8,fill=gray!20,draw=none](-9.611,1.031)--(-9.609,1.029)--(-9.606,1.029)--cycle;
\draw(-9.611,1.031)--(-9.609,1.029)--(-9.606,1.029);
\filldraw[fill opacity=0.8,fill=gray!20](-9.46,1.346)--(-9.507,1.363)--(-9.501,1.356)--(-9.448,1.334)--cycle;
\filldraw[fill opacity=0.8,fill=gray!20,draw=none](-9.546,1.013)--(-9.587,1.019)--(-9.603,1.026)--cycle;
\draw(-9.587,1.019)--(-9.603,1.026);
\filldraw[fill opacity=0.8,fill=gray!20,draw=none](-9.558,1.15)--(-9.609,1.03)--(-9.641,1.05)--(-9.667,1.089)--(-9.676,1.137)--(-9.667,1.187)--(-9.662,1.195)--cycle;
\draw(-9.609,1.03)--(-9.641,1.05)--(-9.667,1.089)--(-9.676,1.137)--(-9.667,1.187)--(-9.662,1.195);
\filldraw[fill opacity=0.8,fill=gray!20](-9.466,.951)--(-9.428,.98)--(-9.489,.968)--(-9.509,.943)--cycle;
\filldraw[fill opacity=0.8,fill=gray!20,draw=none](-9.546,1.013)--(-9.529,1.009)--(-9.48,.988)--(-9.528,.994)--(-9.587,1.019)--cycle;
\draw(-9.529,1.009)--(-9.48,.988);
\draw(-9.528,.994)--(-9.587,1.019);
\filldraw[fill opacity=0.8,fill=gray!20,draw=none](-9.603,1.026)--(-9.528,.994)--(-9.542,1.003)--(-9.564,1.016)--(-9.625,1.043)--cycle;
\draw(-9.603,1.026)--(-9.528,.994);
\draw(-9.564,1.016)--(-9.625,1.043);
\filldraw[fill opacity=0.8,fill=gray!20](-9.605,1.364)--(-9.558,1.364)--(-9.558,1.364)--(-9.583,1.368)--cycle;
\filldraw[fill opacity=0.8,fill=gray!20,draw=none](-9.415,.988)--(-9.411,.991)--(-9.412,.991)--cycle;
\draw(-9.411,.991)--(-9.412,.991);
\filldraw[fill opacity=0.8,fill=gray!20,draw=none](-9.406,1.001)--(-9.412,.991)--(-9.411,.991)--(-9.388,1.023)--cycle;
\draw(-9.412,.991)--(-9.411,.991);
\filldraw[fill opacity=0.8,fill=gray!20,draw=none](-9.609,1.03)--(-9.606,1.029)--(-9.625,1.043)--(-9.641,1.05)--cycle;
\draw(-9.625,1.043)--(-9.641,1.05)--(-9.609,1.03);
\filldraw[fill opacity=0.8,fill=gray!20](-9.748,1.185)--(-9.728,1.239)--(-9.748,1.26)--(-9.77,1.208)--cycle;
\filldraw[fill opacity=0.8,fill=gray!20](-9.527,1.367)--(-9.558,1.364)--(-9.558,1.364)--(-9.507,1.363)--cycle;
\filldraw[fill opacity=0.8,fill=gray!20](-9.561,.931)--(-9.565,.941)--(-9.618,.944)--(-9.589,.933)--cycle;
\filldraw[fill opacity=0.8,fill=gray!20,draw=none](-9.558,1.15)--(-9.662,1.195)--(-9.641,1.232)--(-9.603,1.264)--(-9.57,1.275)--cycle;
\draw(-9.662,1.195)--(-9.641,1.232)--(-9.603,1.264)--(-9.57,1.275);
\filldraw[fill opacity=0.8,fill=gray!20,draw=none](-9.412,.991)--(-9.395,1.023)--(-9.401,1.017)--(-9.428,.98)--cycle;
\draw(-9.401,1.017)--(-9.428,.98)--(-9.412,.991);
\filldraw[fill opacity=0.8,fill=gray!20,draw=none](-9.349,1.115)--(-9.341,1.113)--(-9.301,1.122)--(-9.347,1.137)--cycle;
\draw(-9.341,1.113)--(-9.301,1.122);
\filldraw[fill opacity=0.8,fill=gray!20,draw=none](-9.411,1.308)--(-9.41,1.305)--(-9.403,1.299)--cycle;
\draw(-9.41,1.305)--(-9.403,1.299)--(-9.411,1.308);
\filldraw[fill opacity=0.8,fill=gray!20](-9.533,.932)--(-9.509,.943)--(-9.565,.941)--(-9.561,.931)--cycle;
\filldraw[fill opacity=0.8,fill=gray!20](-9.618,.944)--(-9.643,.97)--(-9.697,.983)--(-9.656,.954)--cycle;
\filldraw[fill opacity=0.5,fill=gray!20](-9.891,-.69)--(-9.844,-.49)--(-10.2,-.368)--(-10.292,-.553)--cycle;
\filldraw[fill opacity=0.8,fill=gray!20,draw=none](-9.369,1.256)--(-9.397,1.291)--(-9.388,1.279)--cycle;
\filldraw[fill opacity=0.8,fill=gray!20,draw=none](-9.469,.951)--(-9.468,.95)--(-9.47,.95)--cycle;
\draw(-9.468,.95)--(-9.47,.95);
\filldraw[fill opacity=0.8,fill=gray!20](-9.728,1.239)--(-9.697,1.286)--(-9.713,1.304)--(-9.748,1.26)--cycle;
\filldraw[fill opacity=0.8,fill=gray!20,draw=none](-9.607,1.363)--(-9.605,1.364)--(-9.609,1.363)--cycle;
\draw(-9.607,1.363)--(-9.605,1.364)--(-9.609,1.363);
\filldraw[fill opacity=0.8,fill=gray!20,draw=none](-8.987,2.789)--(-8.883,3.03)--(-8.869,3.051)--(-8.921,2.932)--cycle;
\draw(-8.987,2.789)--(-8.883,3.03);
\draw(-8.869,3.051)--(-8.921,2.932);
\filldraw[fill opacity=0.8,fill=gray!20,draw=none](-9.663,1.196)--(-9.594,1.355)--(-9.599,1.36)--(-9.608,1.364)--(-9.678,1.202)--cycle;
\draw(-9.608,1.364)--(-9.678,1.202)--(-9.663,1.196)--(-9.594,1.355);
\filldraw[fill opacity=0.8,fill=gray!20,draw=none](-9.61,1.361)--(-9.607,1.363)--(-9.608,1.363)--cycle;
\draw(-9.61,1.361)--(-9.607,1.363);
\filldraw[fill opacity=0.8,fill=gray!20](-9.615,1.358)--(-9.558,1.364)--(-9.558,1.364)--(-9.605,1.364)--cycle;
\filldraw[fill opacity=0.8,fill=gray!20,draw=none](-9.599,1.36)--(-9.607,1.366)--(-9.608,1.364)--cycle;
\draw(-9.607,1.366)--(-9.608,1.364);
\filldraw[fill opacity=0.8,fill=gray!20,draw=none](-9.594,1.355)--(-9.558,1.364)--(-9.558,1.364)--(-9.613,1.358)--cycle;
\draw(-9.594,1.355)--(-9.558,1.364)--(-9.558,1.364)--(-9.613,1.358);
\filldraw[fill opacity=0.8,fill=gray!20,draw=none](-9.599,1.36)--(-9.593,1.357)--(-8.886,2.981)--(-8.883,3.03)--(-9.607,1.366)--cycle;
\draw(-9.593,1.357)--(-8.886,2.981);
\draw(-8.883,3.03)--(-9.607,1.366);
\filldraw[fill opacity=0.8,fill=gray!20,draw=none](-9.479,1.015)--(-9.52,1.007)--(-9.529,1.009)--(-9.533,1.011)--cycle;
\draw(-9.529,1.009)--(-9.533,1.011);
\filldraw[fill opacity=0.8,fill=gray!20,draw=none](-9.533,1.011)--(-9.529,1.009)--(-9.546,1.013)--cycle;
\draw(-9.533,1.011)--(-9.529,1.009);
\filldraw[fill opacity=0.8,fill=gray!20,draw=none](-9.349,1.114)--(-9.341,1.113)--(-9.349,1.115)--cycle;
\filldraw[fill opacity=0.8,fill=gray!20,draw=none](-9.52,1.007)--(-9.524,1.007)--(-9.529,1.009)--cycle;
\draw(-9.524,1.007)--(-9.529,1.009);
\filldraw[fill opacity=0.8,fill=gray!20,draw=none](-9.641,1.05)--(-9.564,1.016)--(-9.57,1.026)--(-9.586,1.049)--(-9.591,1.056)--(-9.667,1.089)--cycle;
\draw(-9.591,1.056)--(-9.667,1.089)--(-9.641,1.05)--(-9.564,1.016);
\filldraw[fill opacity=0.8,fill=gray!20](-9.507,1.363)--(-9.558,1.364)--(-9.558,1.364)--(-9.501,1.356)--cycle;
\filldraw[fill opacity=0.8,fill=gray!20,draw=none](-9.406,1.001)--(-9.388,1.023)--(-9.383,1.031)--(-9.394,1.023)--cycle;
\draw(-9.383,1.031)--(-9.394,1.023);
\filldraw[fill opacity=0.8,fill=gray!20,draw=none](-9.388,1.023)--(-9.381,1.032)--(-9.383,1.031)--cycle;
\draw(-9.381,1.032)--(-9.383,1.031);
\filldraw[fill opacity=0.8,fill=gray!20,draw=none](-9.349,1.116)--(-9.365,1.079)--(-9.356,1.085)--cycle;
\draw(-9.365,1.079)--(-9.356,1.085);
\filldraw[fill opacity=0.8,fill=gray!20,draw=none](-9.379,1.045)--(-9.383,1.031)--(-9.381,1.032)--(-9.365,1.069)--cycle;
\draw(-9.383,1.031)--(-9.381,1.032);
\filldraw[fill opacity=0.8,fill=gray!20](-9.697,1.286)--(-9.656,1.325)--(-9.668,1.337)--(-9.713,1.304)--cycle;
\filldraw[fill opacity=0.8,fill=gray!20,draw=none](-9.388,1.117)--(-9.349,1.114)--(-9.349,1.115)--(-9.386,1.124)--cycle;
\filldraw[fill opacity=0.8,fill=gray!20,draw=none](-9.566,1.273)--(-9.57,1.277)--(-9.575,1.277)--cycle;
\draw(-9.566,1.273)--(-9.57,1.277)--(-9.575,1.277);
\filldraw[fill opacity=0.8,fill=gray!20,draw=none](-9.387,1.132)--(-9.388,1.14)--(-9.386,1.139)--cycle;
\draw(-9.388,1.14)--(-9.386,1.139);
\filldraw[fill opacity=0.8,fill=gray!20,draw=none](-9.386,1.124)--(-9.349,1.115)--(-9.347,1.137)--(-9.38,1.148)--cycle;
\filldraw[fill opacity=0.8,fill=gray!20,draw=none](-9.384,1.149)--(-9.347,1.137)--(-9.347,1.142)--(-9.351,1.157)--(-9.384,1.15)--cycle;
\draw(-9.351,1.157)--(-9.384,1.15);
\filldraw[fill opacity=0.8,fill=gray!20,draw=none](-9.358,1.095)--(-9.349,1.115)--(-9.347,1.142)--(-9.355,1.137)--cycle;
\draw(-9.347,1.142)--(-9.355,1.137);
\filldraw[fill opacity=0.5,fill=gray!20](-9.844,-.49)--(-9.671,-.566)--(-10.027,-.444)--(-10.2,-.368)--cycle;
\filldraw[fill opacity=0.8,fill=gray!20,draw=none](-9.365,1.069)--(-9.356,1.085)--(-9.359,1.083)--cycle;
\draw(-9.356,1.085)--(-9.359,1.083);
\filldraw[fill opacity=0.8,fill=gray!20,draw=none](-9.389,1.027)--(-9.383,1.031)--(-9.379,1.045)--cycle;
\draw(-9.389,1.027)--(-9.383,1.031);
\filldraw[fill opacity=0.8,fill=gray!20,draw=none](-9.371,1.258)--(-9.388,1.279)--(-9.384,1.276)--(-9.372,1.26)--cycle;
\draw(-9.384,1.276)--(-9.372,1.26);
\filldraw[fill opacity=0.8,fill=gray!20,draw=none](-9.372,1.259)--(-9.372,1.26)--(-9.374,1.255)--cycle;
\draw(-9.372,1.26)--(-9.374,1.255);
\filldraw[fill opacity=0.8,fill=gray!20,draw=none](-9.591,1.259)--(-9.566,1.248)--(-9.554,1.248)--(-9.506,1.256)--(-9.556,1.278)--cycle;
\draw(-9.591,1.259)--(-9.566,1.248);
\draw(-9.506,1.256)--(-9.556,1.278);
\filldraw[fill opacity=0.8,fill=gray!20,draw=none](-9.508,1.266)--(-9.556,1.278)--(-9.542,1.271)--cycle;
\draw(-9.556,1.278)--(-9.542,1.271);
\filldraw[fill opacity=0.8,fill=gray!20,draw=none](-9.369,1.256)--(-9.371,1.258)--(-9.372,1.26)--(-9.368,1.255)--cycle;
\draw(-9.372,1.26)--(-9.368,1.255);
\filldraw[fill opacity=0.8,fill=gray!20](-9.643,.97)--(-9.663,1.008)--(-9.728,1.024)--(-9.697,.983)--cycle;
\filldraw[fill opacity=0.8,fill=gray!20,draw=none](-9.368,1.255)--(-9.369,1.256)--(-9.368,1.255)--cycle;
\draw(-9.368,1.255)--(-9.368,1.255);
\filldraw[fill opacity=0.8,fill=gray!20,draw=none](-9.368,1.255)--(-9.368,1.255)--(-9.367,1.253)--cycle;
\draw(-9.368,1.255)--(-9.368,1.255)--(-9.367,1.253);
\filldraw[fill opacity=0.8,fill=gray!20,draw=none](-9.379,1.045)--(-9.365,1.069)--(-9.359,1.083)--(-9.37,1.076)--cycle;
\draw(-9.359,1.083)--(-9.37,1.076);
\filldraw[fill opacity=0.8,fill=gray!20,draw=none](-9.57,1.275)--(-9.603,1.264)--(-9.591,1.259)--(-9.566,1.273)--cycle;
\draw(-9.57,1.275)--(-9.603,1.264)--(-9.591,1.259);
\filldraw[fill opacity=0.8,fill=gray!20,draw=none](-9.359,1.208)--(-9.367,1.253)--(-9.368,1.255)--(-9.376,1.249)--cycle;
\draw(-9.367,1.253)--(-9.368,1.255)--(-9.376,1.249);
\filldraw[fill opacity=0.8,fill=gray!20](-9.656,1.325)--(-9.609,1.351)--(-9.615,1.358)--(-9.668,1.337)--cycle;
\filldraw[fill opacity=0.8,fill=gray!20,draw=none](-9.406,1.217)--(-9.374,1.255)--(-9.412,1.221)--cycle;
\filldraw[fill opacity=0.8,fill=gray!20,draw=none](-9.372,1.26)--(-9.384,1.276)--(-9.39,1.261)--cycle;
\draw(-9.372,1.26)--(-9.384,1.276);
\filldraw[fill opacity=0.8,fill=gray!20,draw=none](-9.412,1.221)--(-9.374,1.255)--(-9.372,1.26)--(-9.389,1.261)--(-9.399,1.255)--(-9.414,1.222)--cycle;
\draw(-9.374,1.255)--(-9.372,1.26);
\draw(-9.399,1.255)--(-9.414,1.222);
\filldraw[fill opacity=0.8,fill=gray!20,draw=none](-9.368,1.255)--(-9.372,1.26)--(-9.39,1.261)--(-9.399,1.234)--cycle;
\draw(-9.399,1.234)--(-9.368,1.255)--(-9.372,1.26);
\filldraw[fill opacity=0.8,fill=gray!20,draw=none](-9.426,1.039)--(-9.462,1.014)--(-9.47,1.017)--cycle;
\draw(-9.462,1.014)--(-9.47,1.017);
\filldraw[fill opacity=0.8,fill=gray!20,draw=none](-9.47,1.017)--(-9.468,1.016)--(-9.479,1.015)--cycle;
\draw(-9.47,1.017)--(-9.468,1.016);
\filldraw[fill opacity=0.8,fill=gray!20,draw=none](-9.372,1.26)--(-9.364,1.277)--(-9.389,1.261)--cycle;
\draw(-9.372,1.26)--(-9.364,1.277);
\filldraw[fill opacity=0.8,fill=gray!20](-8.676,3.029)--(-8.674,3.065)--(-8.61,3.061)--(-8.631,3.025)--cycle;
\filldraw[fill opacity=0.8,fill=gray!20,draw=none](-9.389,1.261)--(-9.396,1.262)--(-9.399,1.255)--cycle;
\draw(-9.396,1.262)--(-9.399,1.255);
\filldraw[fill opacity=0.8,fill=gray!20,draw=none](-9.391,1.258)--(-9.384,1.276)--(-9.403,1.299)--(-9.427,1.284)--cycle;
\draw(-9.384,1.276)--(-9.403,1.299)--(-9.427,1.284);
\filldraw[fill opacity=0.8,fill=gray!20,draw=none](-9.389,1.261)--(-9.364,1.277)--(-8.649,2.92)--(-8.683,2.902)--(-9.396,1.262)--cycle;
\draw(-9.364,1.277)--(-8.649,2.92);
\draw(-8.683,2.902)--(-9.396,1.262);
\filldraw[fill opacity=0.8,fill=gray!20,draw=none](-9.479,1.015)--(-9.468,1.016)--(-9.438,1.003)--(-9.452,.998)--(-9.52,1.007)--cycle;
\draw(-9.468,1.016)--(-9.438,1.003);
\filldraw[fill opacity=0.8,fill=gray!20,draw=none](-9.351,1.157)--(-9.35,1.14)--(-9.347,1.142)--cycle;
\draw(-9.35,1.14)--(-9.347,1.142);
\filldraw[fill opacity=0.8,fill=gray!20,draw=none](-9.347,1.142)--(-9.346,1.159)--(-9.351,1.157)--cycle;
\draw(-9.346,1.159)--(-9.351,1.157);
\filldraw[fill opacity=0.8,fill=gray!20,draw=none](-9.52,1.007)--(-9.452,.998)--(-9.48,.988)--(-9.524,1.007)--cycle;
\draw(-9.48,.988)--(-9.524,1.007);
\filldraw[fill opacity=0.8,fill=gray!20,draw=none](-9.389,1.027)--(-9.379,1.045)--(-9.373,1.067)--(-9.392,1.039)--(-9.399,1.02)--cycle;
\draw(-9.392,1.039)--(-9.399,1.02)--(-9.389,1.027);
\filldraw[fill opacity=0.8,fill=gray!20,draw=none](-9.667,1.089)--(-9.591,1.056)--(-9.601,1.104)--(-9.676,1.137)--cycle;
\draw(-9.601,1.104)--(-9.676,1.137)--(-9.667,1.089)--(-9.591,1.056);
\filldraw[fill opacity=0.8,fill=gray!20,draw=none](-9.365,1.106)--(-9.364,1.081)--(-9.358,1.095)--(-9.355,1.137)--cycle;
\filldraw[fill opacity=0.8,fill=gray!20,draw=none](-9.609,1.351)--(-9.594,1.355)--(-9.613,1.358)--(-9.615,1.358)--cycle;
\draw(-9.613,1.358)--(-9.615,1.358)--(-9.609,1.351)--(-9.594,1.355);
\filldraw[fill opacity=0.8,fill=gray!20,draw=none](-9.429,1.285)--(-9.427,1.284)--(-9.403,1.299)--(-9.448,1.334)--(-9.466,1.322)--(-9.432,1.286)--cycle;
\draw(-9.427,1.284)--(-9.403,1.299)--(-9.448,1.334)--(-9.466,1.322)--(-9.432,1.286);
\filldraw[fill opacity=0.8,fill=gray!20,draw=none](-9.594,1.355)--(-9.593,1.357)--(-9.599,1.36)--cycle;
\draw(-9.594,1.355)--(-9.593,1.357);
\filldraw[fill opacity=0.8,fill=gray!20,draw=none](-9.365,1.159)--(-9.354,1.137)--(-9.35,1.14)--(-9.353,1.179)--(-9.364,1.178)--cycle;
\draw(-9.354,1.137)--(-9.35,1.14);
\filldraw[fill opacity=0.8,fill=gray!20,draw=none](-9.394,1.08)--(-9.412,1.041)--(-9.42,1.044)--cycle;
\draw(-9.412,1.041)--(-9.42,1.044);
\filldraw[fill opacity=0.8,fill=gray!20,draw=none](-9.42,1.044)--(-9.418,1.044)--(-9.426,1.039)--cycle;
\draw(-9.42,1.044)--(-9.418,1.044);
\filldraw[fill opacity=0.8,fill=gray!20,draw=none](-9.426,1.039)--(-9.418,1.044)--(-9.398,1.035)--(-9.42,1.016)--(-9.438,1.003)--(-9.462,1.014)--cycle;
\draw(-9.418,1.044)--(-9.398,1.035);
\draw(-9.438,1.003)--(-9.462,1.014);
\filldraw[fill opacity=0.8,fill=gray!20,draw=none](-9.365,1.106)--(-9.373,1.08)--(-9.375,1.073)--(-9.365,1.079)--(-9.364,1.081)--cycle;
\draw(-9.375,1.073)--(-9.365,1.079);
\filldraw[fill opacity=0.8,fill=gray!20,draw=none](-9.387,1.132)--(-9.386,1.139)--(-9.365,1.13)--(-9.373,1.08)--(-9.382,1.083)--cycle;
\draw(-9.386,1.139)--(-9.365,1.13);
\draw(-9.373,1.08)--(-9.382,1.083);
\filldraw[fill opacity=0.8,fill=gray!20,draw=none](-9.373,1.178)--(-9.353,1.179)--(-9.354,1.197)--(-9.375,1.184)--cycle;
\draw(-9.354,1.197)--(-9.375,1.184);
\filldraw[fill opacity=0.8,fill=gray!20,draw=none](-9.387,1.132)--(-9.382,1.083)--(-9.391,1.087)--cycle;
\draw(-9.382,1.083)--(-9.391,1.087);
\filldraw[fill opacity=0.8,fill=gray!20](-9.565,.941)--(-9.567,.965)--(-9.643,.97)--(-9.618,.944)--cycle;
\filldraw[fill opacity=0.8,fill=gray!20,draw=none](-9.581,1.351)--(-9.558,1.364)--(-9.558,1.364)--(-9.594,1.355)--cycle;
\draw(-9.581,1.351)--(-9.558,1.364)--(-9.558,1.364)--(-9.594,1.355);
\filldraw[fill opacity=0.8,fill=gray!20,draw=none](-9.565,1.345)--(-9.561,1.345)--(-9.558,1.364)--(-9.558,1.364)--(-9.581,1.351)--cycle;
\draw(-9.565,1.345)--(-9.561,1.345)--(-9.558,1.364)--(-9.558,1.364)--(-9.581,1.351);
\filldraw[fill opacity=0.8,fill=gray!20](-9.533,1.346)--(-9.558,1.364)--(-9.558,1.364)--(-9.561,1.345)--cycle;
\filldraw[fill opacity=0.8,fill=gray!20](-9.511,1.35)--(-9.558,1.364)--(-9.558,1.364)--(-9.533,1.346)--cycle;
\filldraw[fill opacity=0.8,fill=gray!20](-9.448,1.334)--(-9.501,1.356)--(-9.511,1.35)--(-9.466,1.322)--cycle;
\filldraw[fill opacity=0.8,fill=gray!20](-9.501,1.356)--(-9.558,1.364)--(-9.558,1.364)--(-9.511,1.35)--cycle;
\filldraw[fill opacity=0.8,fill=gray!20,draw=none](-9.359,1.208)--(-9.356,1.196)--(-9.354,1.197)--cycle;
\draw(-9.356,1.196)--(-9.354,1.197);
\filldraw[fill opacity=0.8,fill=gray!20,draw=none](-9.388,1.117)--(-9.386,1.124)--(-9.439,1.138)--(-9.499,1.124)--cycle;
\draw(-9.439,1.138)--(-9.499,1.124);
\filldraw[fill opacity=0.8,fill=gray!20,draw=none](-9.641,1.232)--(-9.626,1.225)--(-9.614,1.226)--(-9.566,1.248)--(-9.603,1.264)--cycle;
\draw(-9.566,1.248)--(-9.603,1.264)--(-9.641,1.232)--(-9.626,1.225);
\filldraw[fill opacity=0.8,fill=gray!20,draw=none](-9.391,1.087)--(-9.389,1.087)--(-9.394,1.08)--cycle;
\draw(-9.391,1.087)--(-9.389,1.087);
\filldraw[fill opacity=0.8,fill=gray!20](-9.428,.98)--(-9.399,1.02)--(-9.473,1.006)--(-9.489,.968)--cycle;
\filldraw[fill opacity=0.8,fill=gray!20,draw=none](-9.382,1.206)--(-9.362,1.192)--(-9.356,1.196)--(-9.359,1.208)--(-9.373,1.242)--(-9.393,1.239)--cycle;
\draw(-9.362,1.192)--(-9.356,1.196);
\filldraw[fill opacity=0.8,fill=gray!20,draw=none](-9.386,1.124)--(-9.382,1.138)--(-9.384,1.149)--(-9.385,1.15)--(-9.439,1.138)--cycle;
\draw(-9.385,1.15)--(-9.439,1.138);
\filldraw[fill opacity=0.8,fill=gray!20,draw=none](-9.394,1.08)--(-9.389,1.087)--(-9.373,1.08)--(-9.376,1.072)--(-9.389,1.046)--(-9.398,1.035)--(-9.412,1.041)--cycle;
\draw(-9.389,1.087)--(-9.373,1.08);
\draw(-9.398,1.035)--(-9.412,1.041);
\filldraw[fill opacity=0.8,fill=gray!20](-9.509,.943)--(-9.489,.968)--(-9.567,.965)--(-9.565,.941)--cycle;
\filldraw[fill opacity=0.8,fill=gray!20,draw=none](-9.382,1.138)--(-9.38,1.148)--(-9.384,1.149)--cycle;
\filldraw[fill opacity=0.8,fill=gray!20,draw=none](-9.384,1.149)--(-9.383,1.149)--(-9.38,1.137)--(-9.382,1.138)--cycle;
\draw(-9.38,1.137)--(-9.382,1.138);
\filldraw[fill opacity=0.8,fill=gray!20,draw=none](-9.508,1.266)--(-9.477,1.258)--(-9.486,1.262)--cycle;
\draw(-9.477,1.258)--(-9.486,1.262);
\filldraw[fill opacity=0.8,fill=gray!20,draw=none](-9.445,1.239)--(-9.486,1.262)--(-9.478,1.258)--cycle;
\draw(-9.486,1.262)--(-9.478,1.258);
\filldraw[fill opacity=0.8,fill=gray!20,draw=none](-9.667,1.187)--(-9.626,1.225)--(-9.641,1.232)--cycle;
\draw(-9.626,1.225)--(-9.641,1.232)--(-9.667,1.187);
\filldraw[fill opacity=0.8,fill=gray!20,draw=none](-9.659,1.194)--(-9.663,1.196)--(-9.666,1.197)--cycle;
\draw(-9.659,1.194)--(-9.663,1.196)--(-9.666,1.197);
\filldraw[fill opacity=0.8,fill=gray!20,draw=none](-9.373,1.242)--(-9.376,1.249)--(-9.393,1.239)--cycle;
\draw(-9.376,1.249)--(-9.393,1.239);
\filldraw[fill opacity=0.8,fill=gray!20,draw=none](-9.676,1.137)--(-9.601,1.104)--(-9.591,1.154)--(-9.667,1.187)--cycle;
\draw(-9.591,1.154)--(-9.667,1.187)--(-9.676,1.137)--(-9.601,1.104);
\filldraw[fill opacity=0.8,fill=gray!20,draw=none](-9.48,1.259)--(-9.508,1.266)--(-9.542,1.271)--(-9.528,1.265)--cycle;
\draw(-9.542,1.271)--(-9.528,1.265);
\filldraw[fill opacity=0.8,fill=gray!20,draw=none](-9.365,1.159)--(-9.365,1.13)--(-9.354,1.137)--cycle;
\draw(-9.365,1.13)--(-9.354,1.137);
\filldraw[fill opacity=0.8,fill=gray!20](-9.663,1.008)--(-9.675,1.056)--(-9.748,1.074)--(-9.728,1.024)--cycle;
\filldraw[fill opacity=0.8,fill=gray!20,draw=none](-9.365,1.106)--(-9.355,1.137)--(-9.365,1.13)--cycle;
\draw(-9.355,1.137)--(-9.365,1.13);
\filldraw[fill opacity=0.8,fill=gray!20,draw=none](-9.383,1.149)--(-9.374,1.134)--(-9.38,1.137)--cycle;
\draw(-9.374,1.134)--(-9.38,1.137);
\filldraw[fill opacity=0.8,fill=gray!20,draw=none](-9.384,1.15)--(-9.383,1.149)--(-9.384,1.149)--cycle;
\filldraw[fill opacity=0.8,fill=gray!20,draw=none](-9.384,1.149)--(-9.384,1.15)--(-9.385,1.15)--cycle;
\draw(-9.384,1.15)--(-9.385,1.15);
\filldraw[fill opacity=0.8,fill=gray!20,draw=none](-9.373,1.067)--(-9.37,1.076)--(-9.381,1.069)--(-9.392,1.039)--cycle;
\draw(-9.37,1.076)--(-9.381,1.069)--(-9.392,1.039);
\filldraw[fill opacity=0.8,fill=gray!20,draw=none](-9.383,1.149)--(-9.393,1.187)--(-9.373,1.178)--(-9.365,1.13)--(-9.374,1.134)--cycle;
\draw(-9.393,1.187)--(-9.373,1.178);
\draw(-9.365,1.13)--(-9.374,1.134);
\filldraw[fill opacity=0.8,fill=gray!20,draw=none](-9.395,1.023)--(-9.394,1.023)--(-9.399,1.02)--(-9.401,1.017)--cycle;
\draw(-9.394,1.023)--(-9.399,1.02)--(-9.401,1.017);
\filldraw[fill opacity=0.8,fill=gray!20,draw=none](-8.929,2.854)--(-8.855,3.024)--(-8.811,3.028)--(-8.843,2.953)--cycle;
\draw(-8.929,2.854)--(-8.855,3.024);
\draw(-8.811,3.028)--(-8.843,2.953);
\filldraw[fill opacity=0.8,fill=gray!20,draw=none](-9.633,1.183)--(-9.563,1.343)--(-9.589,1.366)--(-9.663,1.196)--cycle;
\draw(-9.589,1.366)--(-9.663,1.196)--(-9.633,1.183)--(-9.563,1.343);
\filldraw[fill opacity=0.8,fill=gray!20,draw=none](-9.465,1.156)--(-9.441,1.179)--(-9.449,1.181)--(-9.482,1.163)--(-9.485,1.157)--cycle;
\draw(-9.482,1.163)--(-9.485,1.157);
\filldraw[fill opacity=0.8,fill=gray!20,draw=none](-9.465,1.156)--(-9.485,1.157)--(-9.499,1.124)--cycle;
\draw(-9.485,1.157)--(-9.499,1.124);
\filldraw[fill opacity=0.8,fill=gray!20](-9.381,1.069)--(-9.375,1.124)--(-9.46,1.107)--(-9.463,1.053)--cycle;
\filldraw[fill opacity=0.8,fill=gray!20,draw=none](-9.667,1.187)--(-9.651,1.18)--(-9.621,1.223)--(-9.626,1.225)--cycle;
\draw(-9.667,1.187)--(-9.651,1.18);
\draw(-9.621,1.223)--(-9.626,1.225);
\filldraw[fill opacity=0.8,fill=gray!20,draw=none](-9.412,1.21)--(-9.406,1.217)--(-9.412,1.221)--(-9.415,1.219)--cycle;
\filldraw[fill opacity=0.8,fill=gray!20,draw=none](-9.589,1.347)--(-9.581,1.351)--(-9.594,1.355)--(-9.609,1.351)--cycle;
\draw(-9.594,1.355)--(-9.609,1.351)--(-9.589,1.347)--(-9.581,1.351);
\filldraw[fill opacity=0.8,fill=gray!20,draw=none](-9.562,1.343)--(-9.561,1.345)--(-9.576,1.346)--cycle;
\draw(-9.562,1.343)--(-9.561,1.345)--(-9.576,1.346);
\filldraw[fill opacity=0.8,fill=gray!20,draw=none](-9.563,1.343)--(-8.901,2.864)--(-8.904,2.898)--(-8.912,2.922)--(-9.589,1.366)--cycle;
\draw(-9.563,1.343)--(-8.901,2.864);
\draw(-8.912,2.922)--(-9.589,1.366);
\filldraw[fill opacity=0.8,fill=gray!20,draw=none](-9.402,1.199)--(-9.391,1.186)--(-9.393,1.187)--cycle;
\draw(-9.391,1.186)--(-9.393,1.187);
\filldraw[fill opacity=0.8,fill=gray!20,draw=none](-9.404,1.201)--(-9.405,1.202)--(-9.43,1.23)--(-9.426,1.229)--(-9.42,1.223)--cycle;
\draw(-9.43,1.23)--(-9.426,1.229);
\filldraw[fill opacity=0.8,fill=gray!20,draw=none](-9.445,1.239)--(-9.426,1.229)--(-9.43,1.23)--cycle;
\draw(-9.426,1.229)--(-9.43,1.23);
\filldraw[fill opacity=0.8,fill=gray!20,draw=none](-9.445,1.239)--(-9.478,1.258)--(-9.464,1.252)--(-9.423,1.228)--(-9.426,1.229)--cycle;
\draw(-9.478,1.258)--(-9.464,1.252);
\draw(-9.423,1.228)--(-9.426,1.229);
\filldraw[fill opacity=0.8,fill=gray!20,draw=none](-9.41,1.209)--(-9.42,1.223)--(-9.415,1.219)--cycle;
\filldraw[fill opacity=0.8,fill=gray!20,draw=none](-9.415,1.207)--(-9.412,1.21)--(-9.415,1.219)--(-9.42,1.208)--cycle;
\draw(-9.415,1.219)--(-9.42,1.208);
\filldraw[fill opacity=0.8,fill=gray!20,draw=none](-9.402,1.199)--(-9.405,1.202)--(-9.404,1.201)--cycle;
\filldraw[fill opacity=0.8,fill=gray!20,draw=none](-9.402,1.199)--(-9.404,1.201)--(-9.374,1.178)--(-9.391,1.186)--cycle;
\draw(-9.374,1.178)--(-9.391,1.186);
\filldraw[fill opacity=0.8,fill=gray!20](-9.618,1.316)--(-9.589,1.347)--(-9.609,1.351)--(-9.656,1.325)--cycle;
\filldraw[fill opacity=0.8,fill=gray!20,draw=none](-9.415,1.207)--(-9.42,1.208)--(-9.426,1.193)--cycle;
\draw(-9.42,1.208)--(-9.426,1.193);
\filldraw[fill opacity=0.8,fill=gray!20,draw=none](-9.404,1.201)--(-9.41,1.209)--(-9.415,1.219)--(-9.375,1.181)--(-9.373,1.178)--(-9.374,1.178)--cycle;
\draw(-9.373,1.178)--(-9.374,1.178);
\filldraw[fill opacity=0.8,fill=gray!20,draw=none](-9.382,1.206)--(-9.375,1.184)--(-9.362,1.192)--cycle;
\draw(-9.375,1.184)--(-9.362,1.192);
\filldraw[fill opacity=0.8,fill=gray!20,draw=none](-9.373,1.08)--(-9.365,1.106)--(-9.365,1.13)--cycle;
\filldraw[fill opacity=0.8,fill=gray!20,draw=none](-9.48,1.259)--(-9.528,1.265)--(-9.511,1.258)--(-9.464,1.252)--(-9.477,1.258)--cycle;
\draw(-9.528,1.265)--(-9.511,1.258);
\draw(-9.464,1.252)--(-9.477,1.258);
\filldraw[fill opacity=0.8,fill=gray!20,draw=none](-9.42,1.223)--(-9.426,1.229)--(-9.423,1.228)--cycle;
\draw(-9.426,1.229)--(-9.423,1.228);
\filldraw[fill opacity=0.8,fill=gray!20,draw=none](-9.431,1.246)--(-9.404,1.244)--(-9.385,1.289)--(-9.429,1.285)--(-9.44,1.261)--cycle;
\draw(-9.404,1.244)--(-9.385,1.289);
\draw(-9.429,1.285)--(-9.44,1.261);
\filldraw[fill opacity=0.8,fill=gray!20,draw=none](-9.398,1.238)--(-9.391,1.258)--(-9.427,1.284)--(-9.428,1.283)--(-9.401,1.238)--cycle;
\draw(-9.427,1.284)--(-9.428,1.283)--(-9.401,1.238);
\filldraw[fill opacity=0.8,fill=gray!20,draw=none](-9.499,1.316)--(-9.466,1.322)--(-9.511,1.35)--(-9.533,1.346)--(-9.514,1.322)--cycle;
\draw(-9.499,1.316)--(-9.466,1.322)--(-9.511,1.35)--(-9.533,1.346)--(-9.514,1.322);
\filldraw[fill opacity=0.8,fill=gray!20,draw=none](-9.409,1.018)--(-9.399,1.02)--(-9.394,1.035)--(-9.396,1.037)--cycle;
\draw(-9.409,1.018)--(-9.399,1.02)--(-9.394,1.035);
\filldraw[fill opacity=0.8,fill=gray!20,draw=none](-9.412,1.221)--(-9.414,1.222)--(-9.415,1.219)--cycle;
\draw(-9.414,1.222)--(-9.415,1.219);
\filldraw[fill opacity=0.8,fill=gray!20,draw=none](-9.419,1.226)--(-9.415,1.219)--(-9.414,1.222)--cycle;
\draw(-9.415,1.219)--(-9.414,1.222);
\filldraw[fill opacity=0.8,fill=gray!20,draw=none](-9.431,1.246)--(-9.419,1.226)--(-9.414,1.222)--(-9.404,1.244)--cycle;
\draw(-9.414,1.222)--(-9.404,1.244);
\filldraw[fill opacity=0.8,fill=gray!20,draw=none](-9.389,1.205)--(-9.376,1.183)--(-9.375,1.181)--(-9.42,1.223)--(-9.423,1.228)--(-9.398,1.216)--cycle;
\draw(-9.423,1.228)--(-9.398,1.216);
\filldraw[fill opacity=0.8,fill=gray!20](-9.675,1.056)--(-9.679,1.11)--(-9.754,1.129)--(-9.748,1.074)--cycle;
\filldraw[fill opacity=0.8,fill=gray!20,draw=none](-9.464,1.252)--(-9.455,1.248)--(-9.447,1.244)--(-9.419,1.226)--(-9.423,1.228)--cycle;
\draw(-9.464,1.252)--(-9.455,1.248);
\draw(-9.419,1.226)--(-9.423,1.228);
\filldraw[fill opacity=0.8,fill=gray!20,draw=none](-9.365,1.13)--(-9.364,1.178)--(-9.381,1.178)--(-9.375,1.124)--cycle;
\draw(-9.381,1.178)--(-9.375,1.124)--(-9.365,1.13);
\filldraw[fill opacity=0.8,fill=gray!20,draw=none](-9.565,1.345)--(-9.581,1.351)--(-9.589,1.347)--cycle;
\draw(-9.581,1.351)--(-9.589,1.347)--(-9.565,1.345);
\filldraw[fill opacity=0.8,fill=gray!20,draw=none](-9.449,1.181)--(-9.441,1.179)--(-9.426,1.193)--cycle;
\filldraw[fill opacity=0.8,fill=gray!20,draw=none](-9.449,1.181)--(-9.426,1.193)--(-9.415,1.219)--(-9.467,1.198)--(-9.472,1.186)--cycle;
\draw(-9.426,1.193)--(-9.415,1.219);
\draw(-9.467,1.198)--(-9.472,1.186);
\filldraw[fill opacity=0.8,fill=gray!20,draw=none](-9.415,1.219)--(-9.431,1.246)--(-9.446,1.247)--(-9.467,1.198)--cycle;
\draw(-9.446,1.247)--(-9.467,1.198);
\filldraw[fill opacity=0.8,fill=gray!20,draw=none](-9.385,1.289)--(-8.709,2.842)--(-8.76,2.823)--(-9.429,1.285)--cycle;
\draw(-9.385,1.289)--(-8.709,2.842);
\draw(-8.76,2.823)--(-9.429,1.285);
\filldraw[fill opacity=0.8,fill=gray!20,draw=none](-9.375,1.073)--(-9.365,1.13)--(-9.375,1.124)--(-9.381,1.069)--cycle;
\draw(-9.365,1.13)--(-9.375,1.124)--(-9.381,1.069)--(-9.375,1.073);
\filldraw[fill opacity=0.8,fill=gray!20,draw=none](-9.373,1.08)--(-9.372,1.079)--(-9.376,1.072)--cycle;
\draw(-9.373,1.08)--(-9.372,1.079);
\filldraw[fill opacity=0.8,fill=gray!20,draw=none](-9.167,.991)--(-9.191,1.001)--(-9.182,1.057)--(-9.157,1.047)--cycle;
\draw(-9.167,.991)--(-9.191,1.001)--(-9.182,1.057)--(-9.157,1.047);
\filldraw[fill opacity=0.8,fill=gray!20,draw=none](-9.196,.941)--(-9.22,.952)--(-9.191,1.001)--(-9.167,.991)--cycle;
\draw(-9.196,.941)--(-9.22,.952)--(-9.191,1.001)--(-9.167,.991);
\filldraw[fill opacity=0.8,fill=gray!20,draw=none](-9.365,1.13)--(-9.195,1.056)--(-9.204,1.006)--(-9.373,1.08)--cycle;
\draw(-9.365,1.13)--(-9.195,1.056);
\draw(-9.204,1.006)--(-9.373,1.08);
\filldraw[fill opacity=0.8,fill=gray!20,draw=none](-9.396,1.037)--(-9.394,1.035)--(-9.389,1.046)--cycle;
\draw(-9.394,1.035)--(-9.389,1.046);
\filldraw[fill opacity=0.8,fill=gray!20,draw=none](-9.372,1.079)--(-9.204,1.006)--(-9.229,.961)--(-9.396,1.034)--cycle;
\draw(-9.372,1.079)--(-9.204,1.006);
\draw(-9.229,.961)--(-9.396,1.034);
\filldraw[fill opacity=0.8,fill=gray!20,draw=none](-9.42,1.016)--(-9.409,1.018)--(-9.398,1.035)--cycle;
\draw(-9.42,1.016)--(-9.409,1.018);
\filldraw[fill opacity=0.8,fill=gray!20,draw=none](-9.447,1.244)--(-9.412,1.223)--(-9.419,1.226)--cycle;
\draw(-9.412,1.223)--(-9.419,1.226);
\filldraw[fill opacity=0.8,fill=gray!20,draw=none](-9.398,1.035)--(-9.395,1.033)--(-9.42,1.016)--cycle;
\draw(-9.398,1.035)--(-9.395,1.033);
\filldraw[fill opacity=0.8,fill=gray!20,draw=none](-9.389,1.046)--(-9.396,1.034)--(-9.398,1.035)--cycle;
\draw(-9.396,1.034)--(-9.398,1.035);
\filldraw[fill opacity=0.8,fill=gray!20,draw=none](-9.42,1.016)--(-9.398,1.035)--(-9.396,1.037)--(-9.416,1.062)--(-9.463,1.053)--(-9.473,1.006)--cycle;
\draw(-9.416,1.062)--(-9.463,1.053)--(-9.473,1.006)--(-9.42,1.016);
\filldraw[fill opacity=0.8,fill=gray!20,draw=none](-9.373,1.178)--(-9.204,1.104)--(-9.195,1.056)--(-9.365,1.13)--cycle;
\draw(-9.373,1.178)--(-9.204,1.104);
\draw(-9.195,1.056)--(-9.365,1.13);
\filldraw[fill opacity=0.8,fill=gray!20](-9.643,1.273)--(-9.618,1.316)--(-9.656,1.325)--(-9.697,1.286)--cycle;
\filldraw[fill opacity=0.8,fill=gray!20,draw=none](-9.396,1.037)--(-9.389,1.046)--(-9.381,1.069)--(-9.416,1.062)--cycle;
\draw(-9.389,1.046)--(-9.381,1.069)--(-9.416,1.062);
\filldraw[fill opacity=0.8,fill=gray!20,draw=none](-9.614,1.226)--(-9.626,1.225)--(-9.621,1.223)--cycle;
\draw(-9.626,1.225)--(-9.621,1.223);
\filldraw[fill opacity=0.8,fill=gray!20,draw=none](-8.808,3.027)--(-8.764,3.014)--(-8.768,3.002)--cycle;
\draw(-8.764,3.014)--(-8.768,3.002);
\filldraw[fill opacity=0.8,fill=gray!20,draw=none](-8.767,3.007)--(-8.764,3.014)--(-8.741,2.997)--cycle;
\draw(-8.767,3.007)--(-8.764,3.014);
\filldraw[fill opacity=0.8,fill=gray!20](-8.773,2.991)--(-8.811,3.03)--(-8.751,3.042)--(-8.731,2.999)--cycle;
\filldraw[fill opacity=0.8,fill=gray!20,draw=none](-8.751,3.06)--(-8.78,3.1)--(-8.752,3.105)--(-8.739,3.062)--cycle;
\draw(-8.78,3.1)--(-8.752,3.105)--(-8.739,3.062)--(-8.751,3.06);
\filldraw[fill opacity=0.8,fill=gray!20,draw=none](-9.591,1.164)--(-9.522,1.323)--(-9.563,1.343)--(-9.633,1.183)--cycle;
\draw(-9.563,1.343)--(-9.633,1.183)--(-9.591,1.164)--(-9.522,1.323);
\filldraw[fill opacity=0.8,fill=gray!20,draw=none](-9.565,1.312)--(-9.562,1.343)--(-9.576,1.346)--(-9.589,1.347)--(-9.618,1.316)--cycle;
\draw(-9.576,1.346)--(-9.589,1.347)--(-9.618,1.316)--(-9.565,1.312)--(-9.562,1.343);
\filldraw[fill opacity=0.8,fill=gray!20](-9.567,.965)--(-9.569,1.001)--(-9.663,1.008)--(-9.643,.97)--cycle;
\filldraw[fill opacity=0.8,fill=gray!20,draw=none](-9.395,1.033)--(-9.229,.961)--(-9.267,.929)--(-9.438,1.003)--cycle;
\draw(-9.395,1.033)--(-9.229,.961);
\draw(-9.267,.929)--(-9.438,1.003);
\filldraw[fill opacity=0.8,fill=gray!20,draw=none](-9.449,1.181)--(-9.472,1.186)--(-9.482,1.163)--cycle;
\draw(-9.472,1.186)--(-9.482,1.163);
\filldraw[fill opacity=0.8,fill=gray!20,draw=none](-9.651,1.18)--(-9.62,1.167)--(-9.593,1.211)--(-9.621,1.223)--cycle;
\draw(-9.651,1.18)--(-9.62,1.167);
\draw(-9.593,1.211)--(-9.621,1.223);
\filldraw[fill opacity=0.8,fill=gray!20](-9.679,1.11)--(-9.675,1.167)--(-9.748,1.185)--(-9.754,1.129)--cycle;
\filldraw[fill opacity=0.8,fill=gray!20,draw=none](-9.514,1.322)--(-9.533,1.346)--(-9.561,1.345)--(-9.563,1.329)--cycle;
\draw(-9.514,1.322)--(-9.533,1.346)--(-9.561,1.345)--(-9.563,1.329);
\filldraw[fill opacity=0.8,fill=gray!20,draw=none](-9.438,1.003)--(-9.414,.993)--(-9.452,.998)--cycle;
\draw(-9.438,1.003)--(-9.414,.993);
\filldraw[fill opacity=0.8,fill=gray!20,draw=none](-9.511,1.258)--(-9.506,1.256)--(-9.496,1.254)--(-9.455,1.248)--(-9.464,1.252)--cycle;
\draw(-9.511,1.258)--(-9.506,1.256);
\draw(-9.455,1.248)--(-9.464,1.252);
\filldraw[fill opacity=0.8,fill=gray!20,draw=none](-9.476,1.3)--(-9.432,1.286)--(-9.466,1.322)--(-9.499,1.316)--cycle;
\draw(-9.432,1.286)--(-9.466,1.322)--(-9.499,1.316);
\filldraw[fill opacity=0.8,fill=gray!20,draw=none](-9.594,1.228)--(-9.614,1.226)--(-9.621,1.223)--(-9.609,1.218)--cycle;
\draw(-9.621,1.223)--(-9.609,1.218);
\filldraw[fill opacity=0.8,fill=gray!20](-9.663,1.223)--(-9.643,1.273)--(-9.697,1.286)--(-9.728,1.239)--cycle;
\filldraw[fill opacity=0.8,fill=gray!20](-9.489,.968)--(-9.473,1.006)--(-9.569,1.001)--(-9.567,.965)--cycle;
\filldraw[fill opacity=0.8,fill=gray!20,draw=none](-9.499,1.124)--(-9.434,1.274)--(-9.494,1.258)--(-9.544,1.144)--cycle;
\draw(-9.494,1.258)--(-9.544,1.144)--(-9.499,1.124)--(-9.434,1.274);
\filldraw[fill opacity=0.8,fill=gray!20,draw=none](-9.447,1.244)--(-9.453,1.248)--(-9.438,1.241)--(-9.42,1.23)--(-9.398,1.216)--(-9.412,1.223)--cycle;
\draw(-9.453,1.248)--(-9.438,1.241);
\draw(-9.398,1.216)--(-9.412,1.223);
\filldraw[fill opacity=0.8,fill=gray!20,draw=none](-9.431,1.246)--(-9.44,1.261)--(-9.446,1.247)--cycle;
\draw(-9.44,1.261)--(-9.446,1.247);
\filldraw[fill opacity=0.8,fill=gray!20,draw=none](-9.614,1.226)--(-9.594,1.228)--(-9.563,1.246)--(-9.566,1.248)--cycle;
\draw(-9.563,1.246)--(-9.566,1.248);
\filldraw[fill opacity=0.8,fill=gray!20,draw=none](-9.522,1.323)--(-8.846,2.876)--(-8.878,2.918)--(-9.563,1.343)--cycle;
\draw(-9.522,1.323)--(-8.846,2.876);
\draw(-8.878,2.918)--(-9.563,1.343);
\filldraw[fill opacity=0.8,fill=gray!20](-9.675,1.167)--(-9.663,1.223)--(-9.728,1.239)--(-9.748,1.185)--cycle;
\filldraw[fill opacity=0.8,fill=gray!20,draw=none](-9.373,1.178)--(-9.375,1.184)--(-9.381,1.18)--(-9.381,1.178)--cycle;
\draw(-9.375,1.184)--(-9.381,1.18)--(-9.381,1.178);
\filldraw[fill opacity=0.8,fill=gray!20,draw=none](-9.375,1.181)--(-9.37,1.177)--(-9.373,1.178)--cycle;
\draw(-9.37,1.177)--(-9.373,1.178);
\filldraw[fill opacity=0.8,fill=gray!20,draw=none](-9.452,.998)--(-9.414,.993)--(-9.267,.929)--(-9.312,.914)--(-9.48,.988)--cycle;
\draw(-9.414,.993)--(-9.267,.929);
\draw(-9.312,.914)--(-9.48,.988);
\filldraw[fill opacity=0.8,fill=gray!20,draw=none](-9.376,1.183)--(-9.375,1.184)--(-9.393,1.239)--(-9.399,1.234)--(-9.389,1.205)--cycle;
\draw(-9.376,1.183)--(-9.375,1.184);
\draw(-9.393,1.239)--(-9.399,1.234)--(-9.389,1.205);
\filldraw[fill opacity=0.8,fill=gray!20,draw=none](-9.375,1.181)--(-9.39,1.213)--(-9.229,1.143)--(-9.204,1.104)--(-9.37,1.177)--cycle;
\draw(-9.39,1.213)--(-9.229,1.143);
\draw(-9.204,1.104)--(-9.37,1.177);
\filldraw[fill opacity=0.8,fill=gray!20,draw=none](-9.376,1.183)--(-9.389,1.205)--(-9.381,1.18)--cycle;
\draw(-9.389,1.205)--(-9.381,1.18)--(-9.376,1.183);
\filldraw[fill opacity=0.8,fill=gray!20,draw=none](-9.38,1.193)--(-9.376,1.183)--(-9.389,1.205)--cycle;
\filldraw[fill opacity=0.8,fill=gray!20,draw=none](-9.427,1.284)--(-9.431,1.285)--(-9.428,1.283)--cycle;
\draw(-9.431,1.285)--(-9.428,1.283)--(-9.427,1.284);
\filldraw[fill opacity=0.8,fill=gray!20,draw=none](-9.455,1.248)--(-9.453,1.248)--(-9.447,1.244)--cycle;
\draw(-9.455,1.248)--(-9.453,1.248);
\filldraw[fill opacity=0.8,fill=gray!20,draw=none](-9.434,1.274)--(-9.429,1.285)--(-9.471,1.311)--(-9.494,1.258)--cycle;
\draw(-9.434,1.274)--(-9.429,1.285);
\draw(-9.471,1.311)--(-9.494,1.258);
\filldraw[fill opacity=0.8,fill=gray!20,draw=none](-9.429,1.285)--(-8.787,2.761)--(-8.843,2.753)--(-9.471,1.311)--cycle;
\draw(-9.429,1.285)--(-8.787,2.761);
\draw(-8.843,2.753)--(-9.471,1.311);
\filldraw[fill opacity=0.8,fill=gray!20,draw=none](-9.429,1.285)--(-9.432,1.286)--(-9.431,1.285)--cycle;
\draw(-9.432,1.286)--(-9.431,1.285);
\filldraw[fill opacity=0.8,fill=gray!20,draw=none](-9.447,1.279)--(-9.428,1.283)--(-9.432,1.286)--(-9.476,1.3)--cycle;
\draw(-9.447,1.279)--(-9.428,1.283)--(-9.432,1.286);
\filldraw[fill opacity=0.8,fill=gray!20,draw=none](-9.438,1.241)--(-9.401,1.238)--(-9.428,1.283)--(-9.489,1.271)--cycle;
\draw(-9.401,1.238)--(-9.428,1.283)--(-9.489,1.271);
\filldraw[fill opacity=0.8,fill=gray!20,draw=none](-9.496,1.254)--(-9.446,1.245)--(-9.455,1.248)--cycle;
\draw(-9.446,1.245)--(-9.455,1.248);
\filldraw[fill opacity=0.8,fill=gray!20,draw=none](-9.398,1.238)--(-9.401,1.238)--(-9.399,1.234)--cycle;
\draw(-9.401,1.238)--(-9.399,1.234);
\filldraw[fill opacity=0.8,fill=gray!20](-9.375,1.124)--(-9.381,1.18)--(-9.463,1.164)--(-9.46,1.107)--cycle;
\filldraw[fill opacity=0.8,fill=gray!20,draw=none](-9.554,1.248)--(-9.566,1.248)--(-9.563,1.246)--cycle;
\draw(-9.566,1.248)--(-9.563,1.246);
\filldraw[fill opacity=0.8,fill=gray!20,draw=none](-9.381,1.18)--(-9.392,1.213)--(-9.42,1.23)--(-9.473,1.22)--(-9.463,1.164)--cycle;
\draw(-9.42,1.23)--(-9.473,1.22)--(-9.463,1.164)--(-9.381,1.18)--(-9.392,1.213);
\filldraw[fill opacity=0.8,fill=gray!20,draw=none](-9.38,1.193)--(-9.389,1.205)--(-9.396,1.216)--(-9.39,1.213)--cycle;
\draw(-9.396,1.216)--(-9.39,1.213);
\filldraw[fill opacity=0.8,fill=gray!20,draw=none](-9.48,.988)--(-9.516,.988)--(-9.528,.994)--cycle;
\draw(-9.516,.988)--(-9.528,.994);
\filldraw[fill opacity=0.8,fill=gray!20,draw=none](-9.48,.988)--(-9.312,.914)--(-9.358,.919)--(-9.516,.988)--cycle;
\draw(-9.48,.988)--(-9.312,.914);
\draw(-9.358,.919)--(-9.516,.988);
\filldraw[fill opacity=0.8,fill=gray!20,draw=none](-8.732,3.047)--(-8.739,3.062)--(-8.674,3.065)--(-8.676,3.03)--cycle;
\draw(-8.732,3.047)--(-8.739,3.062)--(-8.674,3.065)--(-8.676,3.03);
\filldraw[fill opacity=0.8,fill=gray!20,draw=none](-9.544,1.144)--(-8.803,2.846)--(-8.846,2.876)--(-9.591,1.164)--cycle;
\draw(-8.846,2.876)--(-9.591,1.164)--(-9.544,1.144)--(-8.803,2.846);
\filldraw[fill opacity=0.8,fill=gray!20,draw=none](-9.509,1.314)--(-9.514,1.322)--(-9.563,1.329)--(-9.565,1.312)--cycle;
\draw(-9.563,1.329)--(-9.565,1.312)--(-9.509,1.314)--(-9.514,1.322);
\filldraw[fill opacity=0.8,fill=gray!20,draw=none](-9.594,1.228)--(-9.609,1.218)--(-9.573,1.202)--(-9.535,1.232)--cycle;
\draw(-9.609,1.218)--(-9.573,1.202);
\filldraw[fill opacity=0.8,fill=gray!20,draw=none](-9.594,1.228)--(-9.535,1.232)--(-9.533,1.233)--(-9.563,1.246)--cycle;
\draw(-9.533,1.233)--(-9.563,1.246);
\filldraw[fill opacity=0.8,fill=gray!20,draw=none](-9.42,1.23)--(-9.399,1.234)--(-9.401,1.238)--(-9.438,1.241)--cycle;
\draw(-9.42,1.23)--(-9.399,1.234)--(-9.401,1.238);
\filldraw[fill opacity=0.8,fill=gray!20,draw=none](-9.392,1.213)--(-9.399,1.234)--(-9.42,1.23)--cycle;
\draw(-9.392,1.213)--(-9.399,1.234)--(-9.42,1.23);
\filldraw[fill opacity=0.8,fill=gray!20,draw=none](-9.42,1.23)--(-9.438,1.241)--(-9.48,1.245)--(-9.473,1.22)--cycle;
\draw(-9.48,1.245)--(-9.473,1.22)--(-9.42,1.23);
\filldraw[fill opacity=0.8,fill=gray!20,draw=none](-9.554,1.248)--(-9.563,1.246)--(-9.528,1.231)--(-9.48,1.245)--(-9.493,1.25)--cycle;
\draw(-9.563,1.246)--(-9.528,1.231);
\draw(-9.48,1.245)--(-9.493,1.25);
\filldraw[fill opacity=0.8,fill=gray!20,draw=none](-9.389,1.205)--(-9.398,1.216)--(-9.396,1.216)--cycle;
\draw(-9.398,1.216)--(-9.396,1.216);
\filldraw[fill opacity=0.8,fill=gray!20,draw=none](-9.554,1.248)--(-9.527,1.249)--(-9.504,1.255)--(-9.506,1.256)--cycle;
\draw(-9.504,1.255)--(-9.506,1.256);
\filldraw[fill opacity=0.8,fill=gray!20,draw=none](-9.507,1.315)--(-9.499,1.316)--(-9.514,1.322)--(-9.511,1.318)--cycle;
\draw(-9.507,1.315)--(-9.499,1.316);
\draw(-9.514,1.322)--(-9.511,1.318);
\filldraw[fill opacity=0.8,fill=gray!20,draw=none](-9.476,1.3)--(-9.499,1.316)--(-9.509,1.314)--(-9.506,1.309)--cycle;
\draw(-9.499,1.316)--(-9.509,1.314)--(-9.506,1.309);
\filldraw[fill opacity=0.8,fill=gray!20,draw=none](-9.42,1.23)--(-9.397,1.217)--(-9.396,1.216)--(-9.398,1.216)--cycle;
\draw(-9.396,1.216)--(-9.398,1.216);
\filldraw[fill opacity=0.8,fill=gray!20,draw=none](-9.506,1.256)--(-9.504,1.255)--(-9.496,1.254)--cycle;
\draw(-9.506,1.256)--(-9.504,1.255);
\filldraw[fill opacity=0.8,fill=gray!20](-9.569,1.001)--(-9.571,1.048)--(-9.675,1.056)--(-9.663,1.008)--cycle;
\filldraw[fill opacity=0.8,fill=gray!20](-9.567,1.268)--(-9.565,1.312)--(-9.618,1.316)--(-9.643,1.273)--cycle;
\filldraw[fill opacity=0.8,fill=gray!20,draw=none](-9.496,1.254)--(-9.504,1.255)--(-9.48,1.245)--(-9.438,1.241)--(-9.446,1.245)--cycle;
\draw(-9.504,1.255)--(-9.48,1.245);
\draw(-9.438,1.241)--(-9.446,1.245);
\filldraw[fill opacity=0.8,fill=gray!20,draw=none](-9.447,1.279)--(-9.476,1.3)--(-9.506,1.309)--(-9.489,1.271)--cycle;
\draw(-9.506,1.309)--(-9.489,1.271)--(-9.447,1.279);
\filldraw[fill opacity=0.8,fill=gray!20,draw=none](-9.507,1.315)--(-9.511,1.318)--(-9.509,1.314)--cycle;
\draw(-9.511,1.318)--(-9.509,1.314)--(-9.507,1.315);
\filldraw[fill opacity=0.8,fill=gray!20,draw=none](-9.422,1.234)--(-9.267,1.167)--(-9.229,1.143)--(-9.396,1.216)--cycle;
\draw(-9.422,1.234)--(-9.267,1.167);
\draw(-9.229,1.143)--(-9.396,1.216);
\filldraw[fill opacity=0.8,fill=gray!20,draw=none](-9.527,1.249)--(-9.493,1.25)--(-9.504,1.255)--cycle;
\draw(-9.493,1.25)--(-9.504,1.255);
\filldraw[fill opacity=0.8,fill=gray!20,draw=none](-9.438,1.241)--(-9.422,1.234)--(-9.397,1.217)--cycle;
\draw(-9.438,1.241)--(-9.422,1.234);
\filldraw[fill opacity=0.8,fill=gray!20](-9.489,1.271)--(-9.509,1.314)--(-9.565,1.312)--(-9.567,1.268)--cycle;
\filldraw[fill opacity=0.8,fill=gray!20,draw=none](-9.542,1.003)--(-9.528,.994)--(-9.526,.993)--cycle;
\draw(-9.528,.994)--(-9.526,.993);
\filldraw[fill opacity=0.8,fill=gray!20,draw=none](-9.556,1.002)--(-9.473,1.006)--(-9.463,1.053)--(-9.571,1.048)--(-9.57,1.026)--cycle;
\draw(-9.556,1.002)--(-9.473,1.006)--(-9.463,1.053)--(-9.571,1.048)--(-9.57,1.026);
\filldraw[fill opacity=0.8,fill=gray!20,draw=none](-9.542,1.003)--(-9.526,.993)--(-9.516,.988)--(-9.539,1.006)--(-9.562,1.015)--cycle;
\draw(-9.526,.993)--(-9.516,.988);
\draw(-9.539,1.006)--(-9.562,1.015);
\filldraw[fill opacity=0.8,fill=gray!20,draw=none](-9.48,1.245)--(-9.477,1.243)--(-9.436,1.24)--(-9.438,1.241)--cycle;
\draw(-9.48,1.245)--(-9.477,1.243);
\draw(-9.436,1.24)--(-9.438,1.241);
\filldraw[fill opacity=0.8,fill=gray!20,draw=none](-9.438,1.241)--(-9.489,1.271)--(-9.48,1.245)--cycle;
\draw(-9.489,1.271)--(-9.48,1.245);
\filldraw[fill opacity=0.8,fill=gray!20](-9.463,1.053)--(-9.46,1.107)--(-9.571,1.102)--(-9.571,1.048)--cycle;
\filldraw[fill opacity=0.8,fill=gray!20,draw=none](-9.62,1.167)--(-9.591,1.154)--(-9.586,1.16)--(-9.57,1.186)--(-9.564,1.198)--(-9.593,1.211)--cycle;
\draw(-9.62,1.167)--(-9.591,1.154);
\draw(-9.564,1.198)--(-9.593,1.211);
\filldraw[fill opacity=0.8,fill=gray!20,draw=none](-9.477,1.243)--(-9.312,1.171)--(-9.267,1.167)--(-9.436,1.24)--cycle;
\draw(-9.477,1.243)--(-9.312,1.171);
\draw(-9.267,1.167)--(-9.436,1.24);
\filldraw[fill opacity=0.8,fill=gray!20,draw=none](-9.542,1.003)--(-9.56,1.014)--(-9.563,1.016)--(-9.564,1.016)--cycle;
\draw(-9.563,1.016)--(-9.564,1.016);
\filldraw[fill opacity=0.8,fill=gray!20,draw=none](-9.556,1.002)--(-9.57,1.026)--(-9.569,1.001)--cycle;
\draw(-9.57,1.026)--(-9.569,1.001)--(-9.556,1.002);
\filldraw[fill opacity=0.8,fill=gray!20](-9.569,1.216)--(-9.567,1.268)--(-9.643,1.273)--(-9.663,1.223)--cycle;
\filldraw[fill opacity=0.8,fill=gray!20,draw=none](-9.586,1.049)--(-9.632,1.107)--(-9.679,1.11)--(-9.675,1.056)--cycle;
\draw(-9.632,1.107)--(-9.679,1.11)--(-9.675,1.056)--(-9.586,1.049);
\filldraw[fill opacity=0.8,fill=gray!20,draw=none](-9.516,.988)--(-9.508,.985)--(-9.524,.999)--(-9.539,1.006)--cycle;
\draw(-9.516,.988)--(-9.508,.985);
\draw(-9.524,.999)--(-9.539,1.006);
\filldraw[fill opacity=0.8,fill=gray!20,draw=none](-9.542,1.217)--(-9.473,1.22)--(-9.489,1.271)--cycle;
\draw(-9.542,1.217)--(-9.473,1.22)--(-9.489,1.271);
\filldraw[fill opacity=0.8,fill=gray!20,draw=none](-9.573,1.202)--(-9.564,1.198)--(-9.526,1.23)--(-9.533,1.233)--cycle;
\draw(-9.573,1.202)--(-9.564,1.198);
\draw(-9.526,1.23)--(-9.533,1.233);
\filldraw[fill opacity=0.8,fill=gray!20,draw=none](-9.542,1.217)--(-9.489,1.271)--(-9.567,1.268)--(-9.569,1.216)--cycle;
\draw(-9.489,1.271)--(-9.567,1.268)--(-9.569,1.216)--(-9.542,1.217);
\filldraw[fill opacity=0.8,fill=gray!20,draw=none](-9.564,1.016)--(-9.563,1.016)--(-9.57,1.026)--cycle;
\draw(-9.564,1.016)--(-9.563,1.016);
\filldraw[fill opacity=0.8,fill=gray!20,draw=none](-9.56,1.014)--(-9.562,1.015)--(-9.563,1.016)--cycle;
\draw(-9.562,1.015)--(-9.563,1.016);
\filldraw[fill opacity=0.8,fill=gray!20,draw=none](-9.586,1.049)--(-9.571,1.048)--(-9.571,1.102)--(-9.632,1.107)--cycle;
\draw(-9.586,1.049)--(-9.571,1.048)--(-9.571,1.102)--(-9.632,1.107);
\filldraw[fill opacity=0.8,fill=gray!20,draw=none](-9.563,1.016)--(-9.539,1.006)--(-9.547,1.037)--(-9.59,1.055)--cycle;
\draw(-9.563,1.016)--(-9.539,1.006);
\draw(-9.547,1.037)--(-9.59,1.055);
\filldraw[fill opacity=0.8,fill=gray!20](-9.571,1.159)--(-9.569,1.216)--(-9.663,1.223)--(-9.675,1.167)--cycle;
\filldraw[fill opacity=0.8,fill=gray!20,draw=none](-9.632,1.107)--(-9.586,1.16)--(-9.675,1.167)--(-9.679,1.11)--cycle;
\draw(-9.586,1.16)--(-9.675,1.167)--(-9.679,1.11)--(-9.632,1.107);
\filldraw[fill opacity=0.8,fill=gray!20,draw=none](-9.528,1.231)--(-9.504,1.221)--(-9.471,1.241)--(-9.48,1.245)--cycle;
\draw(-9.528,1.231)--(-9.504,1.221);
\draw(-9.471,1.241)--(-9.48,1.245);
\filldraw[fill opacity=0.8,fill=gray!20,draw=none](-9.508,.985)--(-9.358,.919)--(-9.396,.943)--(-9.524,.999)--cycle;
\draw(-9.508,.985)--(-9.358,.919);
\draw(-9.396,.943)--(-9.524,.999);
\filldraw[fill opacity=0.8,fill=gray!20,draw=none](-9.586,1.049)--(-9.59,1.055)--(-9.591,1.056)--cycle;
\draw(-9.59,1.055)--(-9.591,1.056);
\filldraw[fill opacity=0.8,fill=gray!20,draw=none](-9.591,1.056)--(-9.59,1.055)--(-9.591,1.1)--(-9.601,1.104)--cycle;
\draw(-9.591,1.056)--(-9.59,1.055);
\draw(-9.591,1.1)--(-9.601,1.104);
\filldraw[fill opacity=0.8,fill=gray!20,draw=none](-9.463,1.164)--(-9.473,1.22)--(-9.556,1.216)--(-9.57,1.186)--(-9.571,1.159)--cycle;
\draw(-9.57,1.186)--(-9.571,1.159)--(-9.463,1.164)--(-9.473,1.22)--(-9.556,1.216);
\filldraw[fill opacity=0.8,fill=gray!20](-9.46,1.107)--(-9.463,1.164)--(-9.571,1.159)--(-9.571,1.102)--cycle;
\filldraw[fill opacity=0.8,fill=gray!20,draw=none](-9.59,1.055)--(-9.547,1.037)--(-9.537,1.077)--(-9.591,1.1)--cycle;
\draw(-9.59,1.055)--(-9.547,1.037);
\draw(-9.537,1.077)--(-9.591,1.1);
\filldraw[fill opacity=0.8,fill=gray!20,draw=none](-9.632,1.107)--(-9.571,1.102)--(-9.571,1.159)--(-9.586,1.16)--cycle;
\draw(-9.632,1.107)--(-9.571,1.102)--(-9.571,1.159)--(-9.586,1.16);
\filldraw[fill opacity=0.8,fill=gray!20,draw=none](-9.564,1.198)--(-9.563,1.197)--(-9.512,1.224)--(-9.526,1.23)--cycle;
\draw(-9.564,1.198)--(-9.563,1.197);
\draw(-9.512,1.224)--(-9.526,1.23);
\filldraw[fill opacity=0.8,fill=gray!20,draw=none](-9.504,1.221)--(-9.358,1.157)--(-9.312,1.171)--(-9.471,1.241)--cycle;
\draw(-9.504,1.221)--(-9.358,1.157);
\draw(-9.312,1.171)--(-9.471,1.241);
\filldraw[fill opacity=0.8,fill=gray!20,draw=none](-9.601,1.104)--(-9.59,1.131)--(-9.59,1.154)--(-9.591,1.154)--cycle;
\draw(-9.59,1.154)--(-9.591,1.154);
\filldraw[fill opacity=0.8,fill=gray!20,draw=none](-9.601,1.104)--(-9.555,1.084)--(-9.538,1.131)--(-9.583,1.151)--cycle;
\draw(-9.601,1.104)--(-9.555,1.084);
\draw(-9.538,1.131)--(-9.583,1.151);
\filldraw[fill opacity=0.8,fill=gray!20,draw=none](-9.556,1.216)--(-9.569,1.216)--(-9.57,1.186)--cycle;
\draw(-9.556,1.216)--(-9.569,1.216)--(-9.57,1.186);
\filldraw[fill opacity=0.8,fill=gray!20,draw=none](-9.57,1.186)--(-9.563,1.197)--(-9.564,1.198)--cycle;
\draw(-9.563,1.197)--(-9.564,1.198);
\filldraw[fill opacity=0.8,fill=gray!20,draw=none](-9.591,1.154)--(-9.59,1.154)--(-9.586,1.16)--cycle;
\draw(-9.591,1.154)--(-9.59,1.154);
\filldraw[fill opacity=0.8,fill=gray!20,draw=none](-9.59,1.154)--(-9.448,1.092)--(-9.473,1.158)--(-9.563,1.197)--cycle;
\draw(-9.59,1.154)--(-9.448,1.092);
\draw(-9.473,1.158)--(-9.563,1.197);
\filldraw[fill opacity=0.8,fill=gray!20,draw=none](-9.59,1.131)--(-9.583,1.151)--(-9.59,1.154)--cycle;
\draw(-9.583,1.151)--(-9.59,1.154);
\filldraw[fill opacity=0.8,fill=gray!20,draw=none](-9.563,1.197)--(-9.396,1.125)--(-9.358,1.157)--(-9.512,1.224)--cycle;
\draw(-9.563,1.197)--(-9.396,1.125);
\draw(-9.358,1.157)--(-9.512,1.224);
\filldraw[fill opacity=0.8,fill=gray!20,draw=none](-9.539,1.006)--(-9.396,.943)--(-9.421,.982)--(-9.547,1.037)--cycle;
\draw(-9.539,1.006)--(-9.396,.943);
\draw(-9.421,.982)--(-9.547,1.037);
\filldraw[fill opacity=0.8,fill=gray!20,draw=none](-9.547,1.037)--(-9.421,.982)--(-9.43,1.03)--(-9.537,1.077)--cycle;
\draw(-9.547,1.037)--(-9.421,.982);
\draw(-9.43,1.03)--(-9.537,1.077);
\filldraw[fill opacity=0.8,fill=gray!20,draw=none](-9.555,1.084)--(-9.43,1.03)--(-9.421,1.08)--(-9.538,1.131)--cycle;
\draw(-9.555,1.084)--(-9.43,1.03);
\draw(-9.421,1.08)--(-9.538,1.131);
\filldraw[fill opacity=0.8,fill=gray!20](-9.031,1.325)--(-9.034,1.382)--(-8.923,1.387)--(-8.923,1.33)--cycle;
\filldraw[fill opacity=0.8,fill=gray!20](-9.021,1.269)--(-9.031,1.325)--(-8.923,1.33)--(-8.925,1.274)--cycle;
\filldraw[fill opacity=0.8,fill=gray!20,draw=none](-8.85,1.325)--(-8.923,1.33)--(-8.923,1.387)--(-8.841,1.381)--cycle;
\draw(-8.85,1.325)--(-8.923,1.33)--(-8.923,1.387)--(-8.841,1.381);
\filldraw[fill opacity=0.8,fill=gray!20,draw=none](-8.877,1.27)--(-8.925,1.274)--(-8.923,1.33)--(-8.85,1.325)--cycle;
\draw(-8.877,1.27)--(-8.925,1.274)--(-8.923,1.33)--(-8.85,1.325);
\filldraw[fill opacity=0.8,fill=gray!20,draw=none](-8.841,1.381)--(-8.923,1.387)--(-8.923,1.441)--(-8.85,1.436)--cycle;
\draw(-8.841,1.381)--(-8.923,1.387)--(-8.923,1.441)--(-8.85,1.436);
\filldraw[fill opacity=0.8,fill=gray!20](-9.034,1.382)--(-9.031,1.436)--(-8.923,1.441)--(-8.923,1.387)--cycle;
\filldraw[fill opacity=0.8,fill=gray!20](-9.005,1.218)--(-9.021,1.269)--(-8.925,1.274)--(-8.927,1.222)--cycle;
\filldraw[fill opacity=0.8,fill=gray!20,draw=none](-8.919,1.221)--(-8.927,1.222)--(-8.925,1.274)--(-8.877,1.27)--cycle;
\draw(-8.919,1.221)--(-8.927,1.222)--(-8.925,1.274)--(-8.877,1.27);
\filldraw[fill opacity=0.8,fill=gray!20](-9.095,1.255)--(-9.113,1.309)--(-9.031,1.325)--(-9.021,1.269)--cycle;
\filldraw[fill opacity=0.8,fill=gray!20](-9.113,1.309)--(-9.119,1.366)--(-9.034,1.382)--(-9.031,1.325)--cycle;
\filldraw[fill opacity=0.8,fill=gray!20](-9.031,1.436)--(-9.021,1.484)--(-8.925,1.488)--(-8.923,1.441)--cycle;
\filldraw[fill opacity=0.8,fill=gray!20,draw=none](-8.85,1.436)--(-8.923,1.441)--(-8.925,1.488)--(-8.877,1.484)--cycle;
\draw(-8.85,1.436)--(-8.923,1.441)--(-8.925,1.488)--(-8.877,1.484);
\filldraw[fill opacity=0.8,fill=gray!20](-9.119,1.366)--(-9.113,1.42)--(-9.031,1.436)--(-9.034,1.382)--cycle;
\filldraw[fill opacity=0.8,fill=gray!20](-9.066,1.206)--(-9.095,1.255)--(-9.021,1.269)--(-9.005,1.218)--cycle;
\filldraw[fill opacity=0.8,fill=gray!20,draw=none](-8.994,1.193)--(-9.005,1.218)--(-8.927,1.222)--(-8.927,1.215)--cycle;
\draw(-8.994,1.193)--(-9.005,1.218)--(-8.927,1.222)--(-8.927,1.215);
\filldraw[fill opacity=0.8,fill=gray!20,draw=none](-8.927,1.215)--(-8.927,1.222)--(-8.919,1.221)--cycle;
\draw(-8.927,1.215)--(-8.927,1.222)--(-8.919,1.221);
\filldraw[fill opacity=0.8,fill=gray!20](-9.113,1.42)--(-9.095,1.469)--(-9.021,1.484)--(-9.031,1.436)--cycle;
\filldraw[fill opacity=0.8,fill=gray!20,draw=none](-9.448,1.092)--(-9.421,1.08)--(-9.396,1.125)--(-9.473,1.158)--cycle;
\draw(-9.448,1.092)--(-9.421,1.08);
\draw(-9.396,1.125)--(-9.473,1.158);
\filldraw[fill opacity=0.8,fill=gray!20,draw=none](-9.057,1.201)--(-9.061,1.207)--(-9.005,1.218)--(-8.994,1.193)--cycle;
\draw(-9.061,1.207)--(-9.005,1.218)--(-8.994,1.193);
\filldraw[fill opacity=0.8,fill=gray!20](-9.021,1.484)--(-9.005,1.521)--(-8.927,1.525)--(-8.925,1.488)--cycle;
\filldraw[fill opacity=0.8,fill=gray!20,draw=none](-8.877,1.484)--(-8.925,1.488)--(-8.927,1.525)--(-8.919,1.524)--cycle;
\draw(-8.877,1.484)--(-8.925,1.488)--(-8.927,1.525)--(-8.919,1.524);
\filldraw[fill opacity=0.8,fill=gray!20](-9.433,1.084)--(-9.405,1.134)--(-9.363,1.169)--(-9.312,1.186)--(-9.262,1.18)--(-9.22,1.154)--(-9.191,1.111)--(-9.182,1.057)--(-9.191,1.001)--(-9.22,.952)--(-9.262,.916)--(-9.312,.9)--(-9.363,.905)--(-9.405,.932)--(-9.433,.975)--(-9.443,1.029)--cycle;
\filldraw[fill opacity=0.8,fill=gray!20,draw=none](-9.238,.906)--(-9.262,.916)--(-9.22,.952)--(-9.196,.941)--cycle;
\draw(-9.238,.906)--(-9.262,.916)--(-9.22,.952)--(-9.196,.941);
\filldraw[fill opacity=0.8,fill=gray!20,draw=none](-9.195,1.056)--(-9.174,1.047)--(-9.174,1.025)--(-9.176,1.017)--(-9.183,.997)--(-9.204,1.006)--cycle;
\draw(-9.195,1.056)--(-9.174,1.047);
\draw(-9.183,.997)--(-9.204,1.006);
\filldraw[fill opacity=0.8,fill=gray!20,draw=none](-9.204,1.006)--(-9.179,.995)--(-9.191,.974)--(-9.21,.953)--(-9.229,.961)--cycle;
\draw(-9.204,1.006)--(-9.179,.995);
\draw(-9.21,.953)--(-9.229,.961);
\filldraw[fill opacity=0.8,fill=gray!20,draw=none](-9.157,1.047)--(-9.182,1.057)--(-9.191,1.111)--(-9.167,1.1)--cycle;
\draw(-9.157,1.047)--(-9.182,1.057)--(-9.191,1.111)--(-9.167,1.1);
\filldraw[fill opacity=0.8,fill=gray!20,draw=none](-9.229,.961)--(-9.206,.951)--(-9.215,.943)--(-9.252,.922)--(-9.267,.929)--cycle;
\draw(-9.229,.961)--(-9.206,.951);
\draw(-9.252,.922)--(-9.267,.929);
\filldraw[fill opacity=0.8,fill=gray!20](-9.095,1.469)--(-9.066,1.509)--(-9.005,1.521)--(-9.021,1.484)--cycle;
\filldraw[fill opacity=0.8,fill=gray!20,draw=none](-9.204,1.104)--(-9.179,1.093)--(-9.174,1.065)--(-9.174,1.047)--(-9.195,1.056)--cycle;
\draw(-9.204,1.104)--(-9.179,1.093);
\draw(-9.174,1.047)--(-9.195,1.056);
\filldraw[fill opacity=0.8,fill=gray!20,draw=none](-9.288,.889)--(-9.312,.9)--(-9.262,.916)--(-9.238,.906)--cycle;
\draw(-9.288,.889)--(-9.312,.9)--(-9.262,.916)--(-9.238,.906);
\filldraw[fill opacity=0.8,fill=gray!20,draw=none](-9.267,.929)--(-9.243,.918)--(-9.288,.904)--(-9.312,.914)--cycle;
\draw(-9.267,.929)--(-9.243,.918);
\draw(-9.288,.904)--(-9.312,.914);
\filldraw[fill opacity=0.8,fill=gray!20,draw=none](-9.194,1.14)--(-9.173,1.103)--(-9.191,1.111)--(-9.22,1.154)--(-9.197,1.144)--cycle;
\draw(-9.173,1.103)--(-9.191,1.111)--(-9.22,1.154)--(-9.197,1.144);
\filldraw[fill opacity=0.8,fill=gray!20,draw=none](-9.125,1.27)--(-9.137,1.293)--(-9.113,1.309)--(-9.095,1.256)--cycle;
\draw(-9.137,1.293)--(-9.113,1.309)--(-9.095,1.256);
\filldraw[fill opacity=0.8,fill=gray!20,draw=none](-9.137,1.293)--(-9.138,1.295)--(-9.113,1.311)--(-9.113,1.309)--cycle;
\draw(-9.113,1.311)--(-9.113,1.309)--(-9.137,1.293);
\filldraw[fill opacity=0.8,fill=gray!20,draw=none](-9.11,1.245)--(-9.113,1.249)--(-9.125,1.27)--(-9.095,1.256)--(-9.095,1.255)--cycle;
\draw(-9.095,1.256)--(-9.095,1.255)--(-9.11,1.245);
\filldraw[fill opacity=0.8,fill=gray!20,draw=none](-9.087,1.228)--(-9.101,1.251)--(-9.095,1.255)--(-9.073,1.218)--cycle;
\draw(-9.101,1.251)--(-9.095,1.255)--(-9.073,1.218);
\filldraw[fill opacity=0.8,fill=gray!20,draw=none](-9.138,1.295)--(-9.147,1.348)--(-9.119,1.366)--(-9.113,1.311)--cycle;
\draw(-9.147,1.348)--(-9.119,1.366)--(-9.113,1.311);
\filldraw[fill opacity=0.8,fill=gray!20,draw=none](-9.229,1.143)--(-9.207,1.133)--(-9.204,1.129)--(-9.185,1.096)--(-9.204,1.104)--cycle;
\draw(-9.229,1.143)--(-9.207,1.133);
\draw(-9.185,1.096)--(-9.204,1.104);
\filldraw[fill opacity=0.8,fill=gray!20,draw=none](-9.057,1.201)--(-9.06,1.201)--(-9.066,1.206)--(-9.061,1.207)--cycle;
\draw(-9.06,1.201)--(-9.066,1.206)--(-9.061,1.207);
\filldraw[fill opacity=0.8,fill=gray!20,draw=none](-9.068,1.205)--(-9.079,1.215)--(-9.087,1.228)--(-9.073,1.218)--(-9.066,1.206)--cycle;
\draw(-9.073,1.218)--(-9.066,1.206)--(-9.068,1.205);
\filldraw[fill opacity=0.8,fill=gray!20,draw=none](-8.985,1.356)--(-8.995,1.365)--(-9.002,1.373)--cycle;
\draw(-8.985,1.356)--(-8.995,1.365)--(-9.002,1.373);
\filldraw[fill opacity=0.8,fill=gray!20,draw=none](-9.087,1.228)--(-9.105,1.241)--(-9.109,1.246)--(-9.101,1.251)--cycle;
\draw(-9.109,1.246)--(-9.101,1.251);
\filldraw[fill opacity=0.8,fill=gray!20,draw=none](-9.087,1.228)--(-9.079,1.215)--(-9.096,1.231)--(-9.105,1.241)--cycle;
\filldraw[fill opacity=0.8,fill=gray!20,draw=none](-8.985,1.356)--(-9.096,1.231)--(-9.105,1.241)--(-8.995,1.365)--cycle;
\draw(-9.105,1.241)--(-8.995,1.365)--(-8.985,1.356);
\filldraw[fill opacity=0.8,fill=gray!20,draw=none](-8.985,1.356)--(-8.995,1.365)--(-9.002,1.373)--cycle;
\draw(-8.985,1.356)--(-8.995,1.365)--(-9.002,1.373);
\filldraw[fill opacity=0.8,fill=gray!20,draw=none](-8.985,1.356)--(-9.096,1.231)--(-9.105,1.241)--(-8.995,1.365)--cycle;
\draw(-9.105,1.241)--(-8.995,1.365)--(-8.985,1.356);
\filldraw[fill opacity=0.8,fill=gray!20,draw=none](-8.985,1.356)--(-8.995,1.365)--(-9.002,1.373)--cycle;
\draw(-8.985,1.356)--(-8.995,1.365)--(-9.002,1.373);
\filldraw[fill opacity=0.8,fill=gray!20,draw=none](-8.985,1.356)--(-9.096,1.231)--(-9.105,1.241)--(-8.995,1.365)--cycle;
\draw(-9.105,1.241)--(-8.995,1.365)--(-8.985,1.356);
\filldraw[fill opacity=0.8,fill=gray!20,draw=none](-9.147,1.348)--(-9.137,1.404)--(-9.113,1.42)--(-9.119,1.366)--cycle;
\draw(-9.137,1.404)--(-9.113,1.42)--(-9.119,1.366)--(-9.147,1.348);
\filldraw[fill opacity=0.8,fill=gray!20,draw=none](-9.068,1.205)--(-9.066,1.206)--(-9.06,1.201)--cycle;
\draw(-9.068,1.205)--(-9.066,1.206)--(-9.06,1.201);
\filldraw[fill opacity=0.8,fill=gray!20,draw=none](-9.005,1.521)--(-8.994,1.536)--(-8.927,1.528)--(-8.927,1.525)--cycle;
\draw(-8.927,1.528)--(-8.927,1.525)--(-9.005,1.521)--(-8.994,1.536);
\filldraw[fill opacity=0.8,fill=gray!20,draw=none](-8.919,1.524)--(-8.927,1.525)--(-8.927,1.528)--cycle;
\draw(-8.919,1.524)--(-8.927,1.525)--(-8.927,1.528);
\filldraw[fill opacity=0.8,fill=gray!20](-9.31,.995)--(-9.312,1.043)--(-9.22,1.047)--(-9.22,.999)--cycle;
\filldraw[fill opacity=0.8,fill=gray!20](-9.31,.995)--(-9.312,1.043)--(-9.22,1.047)--(-9.22,.999)--cycle;
\filldraw[fill opacity=0.8,fill=gray!20,draw=none](-9.105,1.241)--(-9.11,1.245)--(-9.109,1.246)--cycle;
\draw(-9.11,1.245)--(-9.109,1.246);
\filldraw[fill opacity=0.8,fill=gray!20,draw=none](-9.11,1.245)--(-9.105,1.241)--(-9.106,1.24)--cycle;
\draw(-9.105,1.241)--(-9.106,1.24);
\filldraw[fill opacity=0.8,fill=gray!20,draw=none](-9.113,1.249)--(-9.002,1.373)--(-8.995,1.365)--(-9.106,1.24)--cycle;
\draw(-9.002,1.373)--(-8.995,1.365)--(-9.106,1.24);
\filldraw[fill opacity=0.8,fill=gray!20,draw=none](-9.11,1.245)--(-9.113,1.249)--(-9.002,1.373)--(-8.995,1.365)--(-9.105,1.241)--cycle;
\draw(-9.002,1.373)--(-8.995,1.365)--(-9.105,1.241);
\filldraw[fill opacity=0.8,fill=gray!20,draw=none](-9.113,1.249)--(-9.002,1.373)--(-8.995,1.365)--(-9.106,1.24)--cycle;
\draw(-9.002,1.373)--(-8.995,1.365)--(-9.106,1.24);
\filldraw[fill opacity=0.8,fill=gray!20,draw=none](-9.338,.895)--(-9.363,.905)--(-9.312,.9)--(-9.288,.889)--cycle;
\draw(-9.338,.895)--(-9.363,.905)--(-9.312,.9)--(-9.288,.889);
\filldraw[fill opacity=0.8,fill=gray!20,draw=none](-9.312,.914)--(-9.288,.904)--(-9.333,.909)--(-9.358,.919)--cycle;
\draw(-9.312,.914)--(-9.288,.904);
\draw(-9.333,.909)--(-9.358,.919);
\filldraw[fill opacity=0.8,fill=gray!20,draw=none](-9.176,1.017)--(-9.179,.995)--(-9.183,.997)--cycle;
\draw(-9.179,.995)--(-9.183,.997);
\filldraw[fill opacity=0.8,fill=gray!20,draw=none](-9.191,.974)--(-9.205,.95)--(-9.21,.953)--cycle;
\draw(-9.205,.95)--(-9.21,.953);
\filldraw[fill opacity=0.8,fill=gray!20,draw=none](-9.21,.952)--(-9.221,.952)--(-9.22,.999)--(-9.188,.997)--cycle;
\draw(-9.21,.952)--(-9.221,.952)--(-9.22,.999)--(-9.188,.997);
\filldraw[fill opacity=0.8,fill=gray!20,draw=none](-9.188,.997)--(-9.22,.999)--(-9.22,1.047)--(-9.18,1.044)--cycle;
\draw(-9.188,.997)--(-9.22,.999)--(-9.22,1.047)--(-9.18,1.044);
\filldraw[fill opacity=0.8,fill=gray!20,draw=none](-9.188,.997)--(-9.22,.999)--(-9.22,1.047)--(-9.18,1.044)--cycle;
\draw(-9.188,.997)--(-9.22,.999)--(-9.22,1.047)--(-9.18,1.044);
\filldraw[fill opacity=0.8,fill=gray!20](-9.301,.949)--(-9.31,.995)--(-9.22,.999)--(-9.221,.952)--cycle;
\filldraw[fill opacity=0.8,fill=gray!20,draw=none](-9.21,.952)--(-9.221,.952)--(-9.22,.999)--(-9.188,.997)--cycle;
\draw(-9.21,.952)--(-9.221,.952)--(-9.22,.999)--(-9.188,.997);
\filldraw[fill opacity=0.8,fill=gray!20](-9.301,.949)--(-9.31,.995)--(-9.22,.999)--(-9.221,.952)--cycle;
\filldraw[fill opacity=0.8,fill=gray!20,draw=none](-9.222,1.16)--(-9.206,1.148)--(-9.22,1.154)--(-9.241,1.167)--(-9.244,1.172)--(-9.238,1.17)--cycle;
\draw(-9.206,1.148)--(-9.22,1.154)--(-9.241,1.167);
\draw(-9.244,1.172)--(-9.238,1.17);
\filldraw[fill opacity=0.8,fill=gray!20,draw=none](-9.251,1.156)--(-9.253,1.16)--(-9.246,1.157)--(-9.231,1.148)--(-9.215,1.137)--(-9.229,1.143)--cycle;
\draw(-9.253,1.16)--(-9.246,1.157);
\draw(-9.215,1.137)--(-9.229,1.143);
\filldraw[fill opacity=0.8,fill=gray!20,draw=none](-9.215,.943)--(-9.227,.932)--(-9.246,.92)--(-9.252,.922)--cycle;
\draw(-9.246,.92)--(-9.252,.922);
\filldraw[fill opacity=0.8,fill=gray!20,draw=none](-9.066,1.509)--(-9.06,1.513)--(-8.994,1.536)--(-9.005,1.521)--cycle;
\draw(-8.994,1.536)--(-9.005,1.521)--(-9.066,1.509)--(-9.06,1.513);
\filldraw[fill opacity=0.8,fill=gray!20,draw=none](-9.137,1.404)--(-9.111,1.459)--(-9.095,1.469)--(-9.113,1.42)--cycle;
\draw(-9.111,1.459)--(-9.095,1.469)--(-9.113,1.42)--(-9.137,1.404);
\filldraw[fill opacity=0.8,fill=gray!20,draw=none](-9.18,1.044)--(-9.22,1.047)--(-9.22,1.092)--(-9.188,1.09)--cycle;
\draw(-9.18,1.044)--(-9.22,1.047)--(-9.22,1.092)--(-9.188,1.09);
\filldraw[fill opacity=0.8,fill=gray!20,draw=none](-9.274,1.045)--(-9.247,1.091)--(-9.22,1.092)--(-9.22,1.047)--cycle;
\draw(-9.247,1.091)--(-9.22,1.092)--(-9.22,1.047)--(-9.274,1.045);
\filldraw[fill opacity=0.8,fill=gray!20,draw=none](-9.18,1.044)--(-9.22,1.047)--(-9.22,1.092)--(-9.188,1.09)--cycle;
\draw(-9.18,1.044)--(-9.22,1.047)--(-9.22,1.092)--(-9.188,1.09);
\filldraw[fill opacity=0.8,fill=gray!20,draw=none](-9.274,1.045)--(-9.312,1.043)--(-9.31,1.088)--(-9.247,1.091)--cycle;
\draw(-9.274,1.045)--(-9.312,1.043)--(-9.31,1.088)--(-9.247,1.091);
\filldraw[fill opacity=0.8,fill=gray!20,draw=none](-9.274,1.045)--(-9.247,1.091)--(-9.22,1.092)--(-9.22,1.047)--cycle;
\draw(-9.247,1.091)--(-9.22,1.092)--(-9.22,1.047)--(-9.274,1.045);
\filldraw[fill opacity=0.8,fill=gray!20,draw=none](-9.174,1.065)--(-9.17,1.045)--(-9.174,1.047)--cycle;
\draw(-9.17,1.045)--(-9.174,1.047);
\filldraw[fill opacity=0.8,fill=gray!20,draw=none](-9.174,1.047)--(-9.17,1.045)--(-9.174,1.025)--cycle;
\draw(-9.174,1.047)--(-9.17,1.045);
\filldraw[fill opacity=0.8,fill=gray!20,draw=none](-9.206,.951)--(-9.205,.95)--(-9.215,.943)--cycle;
\draw(-9.206,.951)--(-9.205,.95);
\filldraw[fill opacity=0.8,fill=gray!20,draw=none](-9.222,.938)--(-9.221,.952)--(-9.21,.952)--cycle;
\draw(-9.222,.938)--(-9.221,.952)--(-9.21,.952);
\filldraw[fill opacity=0.8,fill=gray!20,draw=none](-9.288,.906)--(-9.301,.949)--(-9.221,.952)--(-9.222,.938)--cycle;
\draw(-9.288,.906)--(-9.301,.949)--(-9.221,.952)--(-9.222,.938);
\filldraw[fill opacity=0.8,fill=gray!20,draw=none](-9.288,.906)--(-9.301,.949)--(-9.221,.952)--(-9.222,.938)--cycle;
\draw(-9.288,.906)--(-9.301,.949)--(-9.221,.952)--(-9.222,.938);
\filldraw[fill opacity=0.8,fill=gray!20,draw=none](-9.222,.938)--(-9.221,.952)--(-9.21,.952)--cycle;
\draw(-9.222,.938)--(-9.221,.952)--(-9.21,.952);
\filldraw[fill opacity=0.8,fill=gray!20,draw=none](-9.194,1.14)--(-9.186,1.129)--(-9.167,1.1)--(-9.173,1.103)--cycle;
\draw(-9.167,1.1)--(-9.173,1.103);
\filldraw[fill opacity=0.8,fill=gray!20,draw=none](-9.241,1.167)--(-9.262,1.18)--(-9.244,1.172)--cycle;
\draw(-9.241,1.167)--(-9.262,1.18)--(-9.244,1.172);
\filldraw[fill opacity=0.8,fill=gray!20,draw=none](-9.274,1.045)--(-9.312,1.043)--(-9.31,1.088)--(-9.247,1.091)--cycle;
\draw(-9.274,1.045)--(-9.312,1.043)--(-9.31,1.088)--(-9.247,1.091);
\filldraw[fill opacity=0.8,fill=gray!20,draw=none](-9.204,1.129)--(-9.196,1.118)--(-9.179,1.093)--(-9.185,1.096)--cycle;
\draw(-9.179,1.093)--(-9.185,1.096);
\filldraw[fill opacity=0.8,fill=gray!20,draw=none](-9.096,1.231)--(-9.158,1.161)--(-9.174,1.163)--(-9.105,1.241)--cycle;
\draw(-9.174,1.163)--(-9.105,1.241);
\filldraw[fill opacity=0.8,fill=gray!20,draw=none](-9.096,1.231)--(-9.158,1.161)--(-9.174,1.163)--(-9.105,1.241)--cycle;
\draw(-9.174,1.163)--(-9.105,1.241);
\filldraw[fill opacity=0.8,fill=gray!20,draw=none](-9.096,1.231)--(-9.158,1.161)--(-9.174,1.163)--(-9.105,1.241)--cycle;
\draw(-9.174,1.163)--(-9.105,1.241);
\filldraw[fill opacity=0.8,fill=gray!20,draw=none](-9.105,1.241)--(-9.096,1.231)--(-9.111,1.245)--(-9.11,1.245)--cycle;
\draw(-9.111,1.245)--(-9.11,1.245);
\filldraw[fill opacity=0.8,fill=gray!20,draw=none](-9.227,.932)--(-9.243,.918)--(-9.246,.92)--cycle;
\draw(-9.243,.918)--(-9.246,.92);
\filldraw[fill opacity=0.8,fill=gray!20,draw=none](-9.251,1.156)--(-9.267,1.167)--(-9.253,1.16)--cycle;
\draw(-9.267,1.167)--(-9.253,1.16);
\filldraw[fill opacity=0.8,fill=gray!20,draw=none](-9.348,.901)--(-9.381,.921)--(-9.405,.932)--(-9.363,.905)--(-9.341,.896)--cycle;
\draw(-9.381,.921)--(-9.405,.932)--(-9.363,.905)--(-9.341,.896);
\filldraw[fill opacity=0.8,fill=gray!20](-9.363,.937)--(-9.378,.982)--(-9.31,.995)--(-9.301,.949)--cycle;
\filldraw[fill opacity=0.8,fill=gray!20](-9.378,.982)--(-9.383,1.029)--(-9.312,1.043)--(-9.31,.995)--cycle;
\filldraw[fill opacity=0.8,fill=gray!20](-9.363,.937)--(-9.378,.982)--(-9.31,.995)--(-9.301,.949)--cycle;
\filldraw[fill opacity=0.8,fill=gray!20](-9.378,.982)--(-9.383,1.029)--(-9.312,1.043)--(-9.31,.995)--cycle;
\filldraw[fill opacity=0.8,fill=gray!20,draw=none](-9.358,.919)--(-9.342,.913)--(-9.362,.926)--(-9.373,.933)--(-9.396,.943)--cycle;
\draw(-9.358,.919)--(-9.342,.913);
\draw(-9.373,.933)--(-9.396,.943);
\filldraw[fill opacity=0.8,fill=gray!20,draw=none](-9.11,1.245)--(-9.114,1.247)--(-9.113,1.249)--cycle;
\filldraw[fill opacity=0.8,fill=gray!20,draw=none](-9.188,1.164)--(-9.113,1.249)--(-9.106,1.24)--(-9.174,1.163)--cycle;
\draw(-9.106,1.24)--(-9.174,1.163);
\filldraw[fill opacity=0.8,fill=gray!20,draw=none](-9.188,1.164)--(-9.114,1.247)--(-9.11,1.245)--(-9.106,1.24)--(-9.174,1.163)--cycle;
\draw(-9.106,1.24)--(-9.174,1.163);
\filldraw[fill opacity=0.8,fill=gray!20,draw=none](-9.188,1.164)--(-9.113,1.249)--(-9.106,1.24)--(-9.174,1.163)--cycle;
\draw(-9.106,1.24)--(-9.174,1.163);
\filldraw[fill opacity=0.8,fill=gray!20,draw=none](-9.21,1.123)--(-9.201,1.09)--(-9.22,1.092)--(-9.221,1.131)--(-9.217,1.131)--cycle;
\draw(-9.201,1.09)--(-9.22,1.092)--(-9.221,1.131)--(-9.217,1.131);
\filldraw[fill opacity=0.8,fill=gray!20,draw=none](-9.247,1.091)--(-9.235,1.13)--(-9.221,1.131)--(-9.22,1.092)--cycle;
\draw(-9.235,1.13)--(-9.221,1.131)--(-9.22,1.092)--(-9.247,1.091);
\filldraw[fill opacity=0.8,fill=gray!20,draw=none](-9.21,1.123)--(-9.201,1.09)--(-9.22,1.092)--(-9.221,1.131)--(-9.217,1.131)--cycle;
\draw(-9.201,1.09)--(-9.22,1.092)--(-9.221,1.131)--(-9.217,1.131);
\filldraw[fill opacity=0.8,fill=gray!20,draw=none](-9.247,1.091)--(-9.235,1.13)--(-9.221,1.131)--(-9.22,1.092)--cycle;
\draw(-9.235,1.13)--(-9.221,1.131)--(-9.22,1.092)--(-9.247,1.091);
\filldraw[fill opacity=0.8,fill=gray!20](-9.383,1.029)--(-9.378,1.075)--(-9.31,1.088)--(-9.312,1.043)--cycle;
\filldraw[fill opacity=0.8,fill=gray!20](-9.383,1.029)--(-9.378,1.075)--(-9.31,1.088)--(-9.312,1.043)--cycle;
\filldraw[fill opacity=0.8,fill=gray!20,draw=none](-9.355,.924)--(-9.363,.937)--(-9.301,.949)--(-9.288,.906)--cycle;
\draw(-9.355,.924)--(-9.363,.937)--(-9.301,.949)--(-9.288,.906);
\filldraw[fill opacity=0.8,fill=gray!20,draw=none](-9.355,.924)--(-9.363,.937)--(-9.301,.949)--(-9.288,.906)--cycle;
\draw(-9.355,.924)--(-9.363,.937)--(-9.301,.949)--(-9.288,.906);
\filldraw[fill opacity=0.8,fill=gray!20,draw=none](-9.11,1.245)--(-9.111,1.245)--(-9.113,1.249)--cycle;
\draw(-9.11,1.245)--(-9.111,1.245);
\filldraw[fill opacity=0.8,fill=gray!20,draw=none](-9.247,1.091)--(-9.31,1.088)--(-9.301,1.127)--(-9.235,1.13)--cycle;
\draw(-9.247,1.091)--(-9.31,1.088)--(-9.301,1.127)--(-9.235,1.13);
\filldraw[fill opacity=0.8,fill=gray!20,draw=none](-9.247,1.091)--(-9.31,1.088)--(-9.301,1.127)--(-9.235,1.13)--cycle;
\draw(-9.247,1.091)--(-9.31,1.088)--(-9.301,1.127)--(-9.235,1.13);
\filldraw[fill opacity=0.8,fill=gray!20,draw=none](-9.21,1.123)--(-9.201,1.113)--(-9.188,1.09)--(-9.201,1.09)--cycle;
\draw(-9.188,1.09)--(-9.201,1.09);
\filldraw[fill opacity=0.8,fill=gray!20,draw=none](-9.21,1.123)--(-9.201,1.113)--(-9.188,1.09)--(-9.201,1.09)--cycle;
\draw(-9.188,1.09)--(-9.201,1.09);
\filldraw[fill opacity=0.8,fill=gray!20,draw=none](-9.222,1.16)--(-9.202,1.148)--(-9.197,1.144)--(-9.206,1.148)--cycle;
\draw(-9.197,1.144)--(-9.206,1.148);
\filldraw[fill opacity=0.8,fill=gray!20,draw=none](-9.312,1.171)--(-9.288,1.161)--(-9.284,1.16)--(-9.254,1.161)--(-9.267,1.167)--cycle;
\draw(-9.312,1.171)--(-9.288,1.161);
\draw(-9.254,1.161)--(-9.267,1.167);
\filldraw[fill opacity=0.8,fill=gray!20,draw=none](-9.111,1.459)--(-9.068,1.508)--(-9.066,1.509)--(-9.095,1.469)--cycle;
\draw(-9.068,1.508)--(-9.066,1.509)--(-9.095,1.469)--(-9.111,1.459);
\filldraw[fill opacity=0.8,fill=gray!20,draw=none](-9.251,1.171)--(-9.246,1.171)--(-9.244,1.172)--(-9.262,1.18)--(-9.312,1.186)--(-9.288,1.175)--cycle;
\draw(-9.244,1.172)--(-9.262,1.18)--(-9.312,1.186)--(-9.288,1.175);
\filldraw[fill opacity=0.8,fill=gray!20,draw=none](-9.231,1.148)--(-9.227,1.146)--(-9.217,1.14)--(-9.206,1.133)--(-9.215,1.137)--cycle;
\draw(-9.206,1.133)--(-9.215,1.137);
\filldraw[fill opacity=0.8,fill=gray!20,draw=none](-9.158,1.161)--(-9.186,1.129)--(-9.194,1.14)--(-9.174,1.163)--cycle;
\draw(-9.194,1.14)--(-9.174,1.163);
\filldraw[fill opacity=0.8,fill=gray!20,draw=none](-9.158,1.161)--(-9.186,1.129)--(-9.194,1.14)--(-9.174,1.163)--cycle;
\draw(-9.194,1.14)--(-9.174,1.163);
\filldraw[fill opacity=0.8,fill=gray!20,draw=none](-9.158,1.161)--(-9.186,1.129)--(-9.194,1.14)--(-9.174,1.163)--cycle;
\draw(-9.194,1.14)--(-9.174,1.163);
\filldraw[fill opacity=0.8,fill=gray!20,draw=none](-9.202,1.148)--(-9.188,1.164)--(-9.174,1.163)--(-9.193,1.142)--cycle;
\draw(-9.174,1.163)--(-9.193,1.142);
\filldraw[fill opacity=0.8,fill=gray!20,draw=none](-9.202,1.148)--(-9.188,1.164)--(-9.174,1.163)--(-9.193,1.142)--cycle;
\draw(-9.174,1.163)--(-9.193,1.142);
\filldraw[fill opacity=0.8,fill=gray!20,draw=none](-9.202,1.148)--(-9.188,1.164)--(-9.174,1.163)--(-9.193,1.142)--cycle;
\draw(-9.174,1.163)--(-9.193,1.142);
\filldraw[fill opacity=0.8,fill=gray!20](-9.378,1.075)--(-9.363,1.115)--(-9.301,1.127)--(-9.31,1.088)--cycle;
\filldraw[fill opacity=0.8,fill=gray!20](-9.378,1.075)--(-9.363,1.115)--(-9.301,1.127)--(-9.31,1.088)--cycle;
\filldraw[fill opacity=0.8,fill=gray!20,draw=none](-9.186,1.129)--(-9.196,1.118)--(-9.204,1.129)--(-9.194,1.14)--cycle;
\draw(-9.204,1.129)--(-9.194,1.14);
\filldraw[fill opacity=0.8,fill=gray!20,draw=none](-9.186,1.129)--(-9.196,1.118)--(-9.204,1.129)--(-9.194,1.14)--cycle;
\draw(-9.204,1.129)--(-9.194,1.14);
\filldraw[fill opacity=0.8,fill=gray!20,draw=none](-9.186,1.129)--(-9.196,1.118)--(-9.204,1.129)--(-9.194,1.14)--cycle;
\draw(-9.204,1.129)--(-9.194,1.14);
\filldraw[fill opacity=0.8,fill=gray!20,draw=none](-9.194,1.14)--(-9.197,1.144)--(-9.197,1.144)--cycle;
\draw(-9.197,1.144)--(-9.197,1.144);
\filldraw[fill opacity=0.8,fill=gray!20,draw=none](-9.206,1.133)--(-9.21,1.139)--(-9.202,1.148)--(-9.193,1.142)--(-9.203,1.131)--cycle;
\draw(-9.193,1.142)--(-9.203,1.131);
\filldraw[fill opacity=0.8,fill=gray!20,draw=none](-9.206,1.133)--(-9.21,1.139)--(-9.202,1.148)--(-9.193,1.142)--(-9.203,1.131)--cycle;
\draw(-9.193,1.142)--(-9.203,1.131);
\filldraw[fill opacity=0.8,fill=gray!20,draw=none](-9.206,1.133)--(-9.21,1.139)--(-9.202,1.148)--(-9.193,1.142)--(-9.203,1.131)--cycle;
\draw(-9.193,1.142)--(-9.203,1.131);
\filldraw[fill opacity=0.8,fill=gray!20,draw=none](-9.186,1.129)--(-9.194,1.14)--(-9.197,1.144)--(-9.196,1.143)--cycle;
\draw(-9.197,1.144)--(-9.196,1.143);
\filldraw[fill opacity=0.8,fill=gray!20,draw=none](-9.196,1.118)--(-9.201,1.113)--(-9.21,1.123)--(-9.204,1.129)--cycle;
\draw(-9.21,1.123)--(-9.204,1.129);
\filldraw[fill opacity=0.8,fill=gray!20,draw=none](-9.196,1.118)--(-9.201,1.113)--(-9.21,1.123)--(-9.204,1.129)--cycle;
\draw(-9.21,1.123)--(-9.204,1.129);
\filldraw[fill opacity=0.8,fill=gray!20,draw=none](-9.196,1.118)--(-9.201,1.113)--(-9.21,1.123)--(-9.204,1.129)--cycle;
\draw(-9.21,1.123)--(-9.204,1.129);
\filldraw[fill opacity=0.8,fill=gray!20,draw=none](-9.207,1.133)--(-9.205,1.132)--(-9.196,1.118)--cycle;
\draw(-9.207,1.133)--(-9.205,1.132);
\filldraw[fill opacity=0.8,fill=gray!20,draw=none](-9.201,1.113)--(-9.28,1.023)--(-9.29,1.033)--(-9.21,1.123)--cycle;
\draw(-9.28,1.023)--(-9.29,1.033)--(-9.21,1.123);
\filldraw[fill opacity=0.8,fill=gray!20,draw=none](-9.201,1.113)--(-9.28,1.023)--(-9.29,1.033)--(-9.21,1.123)--cycle;
\draw(-9.28,1.023)--(-9.29,1.033)--(-9.21,1.123);
\filldraw[fill opacity=0.8,fill=gray!20,draw=none](-9.201,1.113)--(-9.28,1.023)--(-9.29,1.033)--(-9.21,1.123)--cycle;
\draw(-9.28,1.023)--(-9.29,1.033)--(-9.21,1.123);
\filldraw[fill opacity=0.8,fill=gray!20,draw=none](-9.21,1.123)--(-9.217,1.131)--(-9.212,1.13)--cycle;
\draw(-9.217,1.131)--(-9.212,1.13);
\filldraw[fill opacity=0.8,fill=gray!20,draw=none](-9.21,1.123)--(-9.217,1.131)--(-9.212,1.13)--cycle;
\draw(-9.217,1.131)--(-9.212,1.13);
\filldraw[fill opacity=0.8,fill=gray!20,draw=none](-9.217,1.14)--(-9.212,1.137)--(-9.206,1.133)--(-9.206,1.133)--cycle;
\draw(-9.206,1.133)--(-9.206,1.133);
\filldraw[fill opacity=0.8,fill=gray!20,draw=none](-9.214,1.134)--(-9.212,1.137)--(-9.206,1.133)--(-9.204,1.129)--(-9.208,1.125)--cycle;
\draw(-9.204,1.129)--(-9.208,1.125);
\filldraw[fill opacity=0.8,fill=gray!20,draw=none](-9.214,1.134)--(-9.212,1.137)--(-9.206,1.133)--(-9.204,1.129)--(-9.208,1.125)--cycle;
\draw(-9.204,1.129)--(-9.208,1.125);
\filldraw[fill opacity=0.8,fill=gray!20,draw=none](-9.214,1.134)--(-9.212,1.137)--(-9.206,1.133)--(-9.204,1.129)--(-9.208,1.125)--cycle;
\draw(-9.204,1.129)--(-9.208,1.125);
\filldraw[fill opacity=0.8,fill=gray!20,draw=none](-9.214,1.134)--(-9.212,1.13)--(-9.221,1.131)--(-9.222,1.141)--cycle;
\draw(-9.212,1.13)--(-9.221,1.131)--(-9.222,1.141);
\filldraw[fill opacity=0.8,fill=gray!20,draw=none](-9.214,1.134)--(-9.212,1.13)--(-9.221,1.131)--(-9.222,1.141)--cycle;
\draw(-9.212,1.13)--(-9.221,1.131)--(-9.222,1.141);
\filldraw[fill opacity=0.8,fill=gray!20,draw=none](-9.214,1.134)--(-9.208,1.125)--(-9.29,1.033)--(-9.297,1.041)--cycle;
\draw(-9.208,1.125)--(-9.29,1.033)--(-9.297,1.041);
\filldraw[fill opacity=0.8,fill=gray!20,draw=none](-9.214,1.134)--(-9.208,1.125)--(-9.29,1.033)--(-9.297,1.041)--cycle;
\draw(-9.208,1.125)--(-9.29,1.033)--(-9.297,1.041);
\filldraw[fill opacity=0.8,fill=gray!20,draw=none](-9.214,1.134)--(-9.208,1.125)--(-9.29,1.033)--(-9.297,1.041)--cycle;
\draw(-9.208,1.125)--(-9.29,1.033)--(-9.297,1.041);
\filldraw[fill opacity=0.8,fill=gray!20,draw=none](-9.201,1.113)--(-9.21,1.123)--(-9.212,1.13)--(-9.21,1.13)--cycle;
\draw(-9.212,1.13)--(-9.21,1.13);
\filldraw[fill opacity=0.8,fill=gray!20,draw=none](-9.201,1.113)--(-9.21,1.123)--(-9.212,1.13)--(-9.21,1.13)--cycle;
\draw(-9.212,1.13)--(-9.21,1.13);
\filldraw[fill opacity=0.8,fill=gray!20,draw=none](-9.396,.943)--(-9.377,.934)--(-9.385,.953)--(-9.397,.971)--(-9.421,.982)--cycle;
\draw(-9.396,.943)--(-9.377,.934);
\draw(-9.397,.971)--(-9.421,.982);
\filldraw[fill opacity=0.8,fill=gray!20,draw=none](-9.206,1.133)--(-9.203,1.131)--(-9.204,1.129)--cycle;
\draw(-9.203,1.131)--(-9.204,1.129);
\filldraw[fill opacity=0.8,fill=gray!20,draw=none](-9.206,1.133)--(-9.203,1.131)--(-9.204,1.129)--cycle;
\draw(-9.203,1.131)--(-9.204,1.129);
\filldraw[fill opacity=0.8,fill=gray!20,draw=none](-9.206,1.133)--(-9.203,1.131)--(-9.204,1.129)--cycle;
\draw(-9.203,1.131)--(-9.204,1.129);
\filldraw[fill opacity=0.8,fill=gray!20,draw=none](-9.409,.964)--(-9.433,.975)--(-9.405,.932)--(-9.381,.921)--cycle;
\draw(-9.409,.964)--(-9.433,.975)--(-9.405,.932)--(-9.381,.921);
\filldraw[fill opacity=0.8,fill=gray!20,draw=none](-9.202,1.148)--(-9.196,1.143)--(-9.197,1.144)--cycle;
\draw(-9.196,1.143)--(-9.197,1.144);
\filldraw[fill opacity=0.8,fill=gray!20,draw=none](-9.342,.913)--(-9.336,.91)--(-9.349,.919)--(-9.36,.925)--(-9.362,.926)--cycle;
\draw(-9.342,.913)--(-9.336,.91);
\filldraw[fill opacity=0.8,fill=gray!20,draw=none](-9.212,1.137)--(-9.205,1.132)--(-9.206,1.133)--cycle;
\draw(-9.205,1.132)--(-9.206,1.133);
\filldraw[fill opacity=0.8,fill=gray!20,draw=none](-9.212,1.137)--(-9.21,1.139)--(-9.206,1.133)--cycle;
\filldraw[fill opacity=0.8,fill=gray!20,draw=none](-9.212,1.137)--(-9.21,1.139)--(-9.206,1.133)--cycle;
\filldraw[fill opacity=0.8,fill=gray!20,draw=none](-9.212,1.137)--(-9.21,1.139)--(-9.206,1.133)--cycle;
\filldraw[fill opacity=0.8,fill=gray!20,draw=none](-9.235,1.13)--(-9.235,1.145)--(-9.222,1.141)--(-9.221,1.131)--cycle;
\draw(-9.222,1.141)--(-9.221,1.131)--(-9.235,1.13);
\filldraw[fill opacity=0.8,fill=gray!20,draw=none](-9.235,1.13)--(-9.235,1.145)--(-9.222,1.141)--(-9.221,1.131)--cycle;
\draw(-9.222,1.141)--(-9.221,1.131)--(-9.235,1.13);
\filldraw[fill opacity=0.8,fill=gray!20,draw=none](-9.235,1.13)--(-9.281,1.128)--(-9.264,1.152)--(-9.235,1.145)--cycle;
\draw(-9.235,1.13)--(-9.281,1.128);
\filldraw[fill opacity=0.8,fill=gray!20,draw=none](-9.235,1.13)--(-9.281,1.128)--(-9.264,1.152)--(-9.235,1.145)--cycle;
\draw(-9.235,1.13)--(-9.281,1.128);
\filldraw[fill opacity=0.8,fill=gray!20,draw=none](-9.214,1.134)--(-9.21,1.13)--(-9.212,1.13)--cycle;
\draw(-9.21,1.13)--(-9.212,1.13);
\filldraw[fill opacity=0.8,fill=gray!20,draw=none](-9.214,1.134)--(-9.21,1.13)--(-9.212,1.13)--cycle;
\draw(-9.21,1.13)--(-9.212,1.13);
\filldraw[fill opacity=0.8,fill=gray!20,draw=none](-9.358,1.157)--(-9.333,1.146)--(-9.288,1.161)--(-9.312,1.171)--cycle;
\draw(-9.358,1.157)--(-9.333,1.146);
\draw(-9.288,1.161)--(-9.312,1.171);
\filldraw[fill opacity=0.8,fill=gray!20,draw=none](-9.284,1.16)--(-9.259,1.158)--(-9.253,1.16)--(-9.254,1.161)--cycle;
\draw(-9.253,1.16)--(-9.254,1.161);
\filldraw[fill opacity=0.8,fill=gray!20,draw=none](-9.288,1.175)--(-9.312,1.186)--(-9.363,1.169)--(-9.338,1.159)--cycle;
\draw(-9.288,1.175)--(-9.312,1.186)--(-9.363,1.169)--(-9.338,1.159);
\filldraw[fill opacity=0.8,fill=gray!20,draw=none](-9.259,1.158)--(-9.243,1.156)--(-9.253,1.16)--cycle;
\draw(-9.243,1.156)--(-9.253,1.16);
\filldraw[fill opacity=0.8,fill=gray!20,draw=none](-9.068,1.508)--(-9.06,1.513)--(-9.066,1.509)--cycle;
\draw(-9.06,1.513)--(-9.066,1.509)--(-9.068,1.508);
\filldraw[fill opacity=0.8,fill=gray!20,draw=none](-9.348,.901)--(-9.341,.896)--(-9.338,.895)--cycle;
\draw(-9.341,.896)--(-9.338,.895);
\filldraw[fill opacity=0.8,fill=gray!20,draw=none](-9.281,1.128)--(-9.301,1.127)--(-9.288,1.159)--(-9.264,1.152)--cycle;
\draw(-9.281,1.128)--(-9.301,1.127)--(-9.288,1.159);
\filldraw[fill opacity=0.8,fill=gray!20,draw=none](-9.281,1.128)--(-9.301,1.127)--(-9.288,1.159)--(-9.264,1.152)--cycle;
\draw(-9.281,1.128)--(-9.301,1.127)--(-9.288,1.159);
\filldraw[fill opacity=0.8,fill=gray!20,draw=none](-9.336,.91)--(-9.333,.909)--(-9.349,.919)--cycle;
\draw(-9.336,.91)--(-9.333,.909);
\filldraw[fill opacity=0.8,fill=gray!20,draw=none](-9.297,1.041)--(-9.29,1.033)--(-9.28,1.023)--cycle;
\draw(-9.297,1.041)--(-9.29,1.033)--(-9.28,1.023);
\filldraw[fill opacity=0.8,fill=gray!20,draw=none](-9.297,1.041)--(-9.29,1.033)--(-9.28,1.023)--cycle;
\draw(-9.297,1.041)--(-9.29,1.033)--(-9.28,1.023);
\filldraw[fill opacity=0.8,fill=gray!20,draw=none](-9.297,1.041)--(-9.29,1.033)--(-9.28,1.023)--cycle;
\draw(-9.297,1.041)--(-9.29,1.033)--(-9.28,1.023);
\filldraw[fill opacity=0.8,fill=gray!20,draw=none](-9.247,1.171)--(-9.238,1.17)--(-9.244,1.172)--cycle;
\draw(-9.238,1.17)--(-9.244,1.172);
\filldraw[fill opacity=0.8,fill=gray!20,draw=none](-9.231,1.148)--(-9.232,1.149)--(-9.227,1.146)--cycle;
\filldraw[fill opacity=0.8,fill=gray!20,draw=none](-9.363,1.115)--(-9.355,1.126)--(-9.288,1.159)--(-9.301,1.127)--cycle;
\draw(-9.288,1.159)--(-9.301,1.127)--(-9.363,1.115)--(-9.355,1.126);
\filldraw[fill opacity=0.8,fill=gray!20,draw=none](-9.363,1.115)--(-9.355,1.126)--(-9.288,1.159)--(-9.301,1.127)--cycle;
\draw(-9.288,1.159)--(-9.301,1.127)--(-9.363,1.115)--(-9.355,1.126);
\filldraw[fill opacity=0.8,fill=gray!20,draw=none](-9.421,.982)--(-9.401,.973)--(-9.401,.992)--(-9.406,1.019)--(-9.43,1.03)--cycle;
\draw(-9.421,.982)--(-9.401,.973);
\draw(-9.406,1.019)--(-9.43,1.03);
\filldraw[fill opacity=0.8,fill=gray!20,draw=none](-9.246,1.157)--(-9.243,1.156)--(-9.232,1.149)--(-9.231,1.148)--cycle;
\draw(-9.246,1.157)--(-9.243,1.156);
\filldraw[fill opacity=0.8,fill=gray!20,draw=none](-9.419,1.018)--(-9.443,1.029)--(-9.433,.975)--(-9.409,.964)--cycle;
\draw(-9.419,1.018)--(-9.443,1.029)--(-9.433,.975)--(-9.409,.964);
\filldraw[fill opacity=0.8,fill=gray!20,draw=none](-9.396,1.125)--(-9.373,1.115)--(-9.36,1.124)--(-9.349,1.133)--(-9.336,1.147)--(-9.358,1.157)--cycle;
\draw(-9.396,1.125)--(-9.373,1.115);
\draw(-9.336,1.147)--(-9.358,1.157);
\filldraw[fill opacity=0.8,fill=gray!20,draw=none](-9.251,1.171)--(-9.247,1.171)--(-9.246,1.171)--cycle;
\filldraw[fill opacity=0.8,fill=gray!20,draw=none](-9.338,1.159)--(-9.363,1.169)--(-9.405,1.134)--(-9.381,1.123)--cycle;
\draw(-9.338,1.159)--(-9.363,1.169)--(-9.405,1.134)--(-9.381,1.123);
\filldraw[fill opacity=0.8,fill=gray!20,draw=none](-9.389,.975)--(-9.396,1.021)--(-9.383,1.029)--(-9.378,.982)--cycle;
\draw(-9.396,1.021)--(-9.383,1.029)--(-9.378,.982)--(-9.389,.975);
\filldraw[fill opacity=0.8,fill=gray!20,draw=none](-9.389,.975)--(-9.396,1.021)--(-9.383,1.029)--(-9.378,.982)--cycle;
\draw(-9.396,1.021)--(-9.383,1.029)--(-9.378,.982)--(-9.389,.975);
\filldraw[fill opacity=0.8,fill=gray!20,draw=none](-9.367,.934)--(-9.389,.975)--(-9.378,.982)--(-9.363,.937)--cycle;
\draw(-9.389,.975)--(-9.378,.982)--(-9.363,.937)--(-9.367,.934);
\filldraw[fill opacity=0.8,fill=gray!20,draw=none](-9.367,.934)--(-9.389,.975)--(-9.378,.982)--(-9.363,.937)--cycle;
\draw(-9.389,.975)--(-9.378,.982)--(-9.363,.937)--(-9.367,.934);
\filldraw[fill opacity=0.8,fill=gray!20,draw=none](-9.367,.934)--(-9.363,.937)--(-9.355,.924)--cycle;
\draw(-9.367,.934)--(-9.363,.937)--(-9.355,.924);
\filldraw[fill opacity=0.8,fill=gray!20,draw=none](-9.367,.934)--(-9.363,.937)--(-9.355,.924)--cycle;
\draw(-9.367,.934)--(-9.363,.937)--(-9.355,.924);
\filldraw[fill opacity=0.8,fill=gray!20,draw=none](-9.43,1.03)--(-9.409,1.021)--(-9.403,1.04)--(-9.397,1.07)--(-9.421,1.08)--cycle;
\draw(-9.43,1.03)--(-9.409,1.021);
\draw(-9.397,1.07)--(-9.421,1.08);
\filldraw[fill opacity=0.8,fill=gray!20,draw=none](-9.377,.934)--(-9.372,.932)--(-9.385,.953)--cycle;
\draw(-9.377,.934)--(-9.372,.932);
\filldraw[fill opacity=0.8,fill=gray!20,draw=none](-9.362,.926)--(-9.36,.925)--(-9.362,.926)--cycle;
\filldraw[fill opacity=0.8,fill=gray!20,draw=none](-9.362,.926)--(-9.362,.926)--(-9.372,.932)--(-9.373,.933)--cycle;
\draw(-9.372,.932)--(-9.373,.933);
\filldraw[fill opacity=0.8,fill=gray!20,draw=none](-9.421,1.08)--(-9.401,1.071)--(-9.383,1.094)--(-9.372,1.114)--(-9.396,1.125)--cycle;
\draw(-9.421,1.08)--(-9.401,1.071);
\draw(-9.372,1.114)--(-9.396,1.125);
\filldraw[fill opacity=0.8,fill=gray!20,draw=none](-9.396,1.021)--(-9.389,1.068)--(-9.378,1.075)--(-9.383,1.029)--cycle;
\draw(-9.389,1.068)--(-9.378,1.075)--(-9.383,1.029)--(-9.396,1.021);
\filldraw[fill opacity=0.8,fill=gray!20,draw=none](-9.396,1.021)--(-9.389,1.068)--(-9.378,1.075)--(-9.383,1.029)--cycle;
\draw(-9.389,1.068)--(-9.378,1.075)--(-9.383,1.029)--(-9.396,1.021);
\filldraw[fill opacity=0.8,fill=gray!20,draw=none](-9.409,1.074)--(-9.433,1.084)--(-9.443,1.029)--(-9.419,1.018)--cycle;
\draw(-9.409,1.074)--(-9.433,1.084)--(-9.443,1.029)--(-9.419,1.018);
\filldraw[fill opacity=0.8,fill=gray!20,draw=none](-9.381,1.123)--(-9.405,1.134)--(-9.433,1.084)--(-9.409,1.074)--cycle;
\draw(-9.381,1.123)--(-9.405,1.134)--(-9.433,1.084)--(-9.409,1.074);
\filldraw[fill opacity=0.8,fill=gray!20,draw=none](-9.389,1.068)--(-9.367,1.113)--(-9.363,1.115)--(-9.378,1.075)--cycle;
\draw(-9.367,1.113)--(-9.363,1.115)--(-9.378,1.075)--(-9.389,1.068);
\filldraw[fill opacity=0.8,fill=gray!20,draw=none](-9.389,1.068)--(-9.367,1.113)--(-9.363,1.115)--(-9.378,1.075)--cycle;
\draw(-9.367,1.113)--(-9.363,1.115)--(-9.378,1.075)--(-9.389,1.068);
\filldraw[fill opacity=0.8,fill=gray!20,draw=none](-9.401,.973)--(-9.397,.971)--(-9.401,.992)--cycle;
\draw(-9.401,.973)--(-9.397,.971);
\filldraw[fill opacity=0.8,fill=gray!20,draw=none](-9.349,1.133)--(-9.333,1.146)--(-9.336,1.147)--cycle;
\draw(-9.333,1.146)--(-9.336,1.147);
\filldraw[fill opacity=0.8,fill=gray!20,draw=none](-9.367,1.113)--(-9.355,1.126)--(-9.363,1.115)--cycle;
\draw(-9.355,1.126)--(-9.363,1.115)--(-9.367,1.113);
\filldraw[fill opacity=0.8,fill=gray!20,draw=none](-9.367,1.113)--(-9.355,1.126)--(-9.363,1.115)--cycle;
\draw(-9.355,1.126)--(-9.363,1.115)--(-9.367,1.113);
\filldraw[fill opacity=0.8,fill=gray!20,draw=none](-9.409,1.021)--(-9.406,1.019)--(-9.403,1.04)--cycle;
\draw(-9.409,1.021)--(-9.406,1.019);
\filldraw[fill opacity=0.8,fill=gray!20,draw=none](-9.401,1.071)--(-9.397,1.07)--(-9.383,1.094)--cycle;
\draw(-9.401,1.071)--(-9.397,1.07);
\filldraw[fill opacity=0.8,fill=gray!20,draw=none](-9.373,1.115)--(-9.372,1.114)--(-9.36,1.124)--cycle;
\draw(-9.373,1.115)--(-9.372,1.114);
\filldraw[fill opacity=0.5,fill=gray!20](-7.862,1.605)--(-8.009,1.978)--(-8.244,2.286)--(-8.55,2.508)--(-8.906,2.631)--(-9.288,2.644)--(-9.671,2.547)--(-10.027,2.347)--(-10.333,2.058)--(-10.567,1.698)--(-10.715,1.294)--(-10.765,.871)--(-10.715,.46)--(-10.567,.087)--(-10.333,-.221)--(-10.027,-.444)--(-9.671,-.566)--(-9.288,-.579)--(-8.906,-.482)--(-8.55,-.283)--(-8.244,.007)--(-8.009,.366)--(-7.862,.771)--(-7.811,1.193)--cycle;
\filldraw[fill opacity=0.8,fill=gray!20,draw=none](-8.448,3.57)--(-8.548,3.608)--(-8.496,3.589)--cycle;
\filldraw[fill opacity=0.5,fill=gray!20](-9.411,-.813)--(-9.461,-.705)--(-9.891,-.69)--(-9.862,-.797)--cycle;
\filldraw[fill opacity=0.5,fill=gray!20,draw=none](-9.519,-.895)--(-9.575,-.807)--(-9.862,-.797)--(-9.82,-.885)--cycle;
\draw(-9.575,-.807)--(-9.862,-.797)--(-9.82,-.885)--(-9.519,-.895);
\filldraw[fill opacity=0.8,fill=gray!20,draw=none](-9.071,-.893)--(-9.092,-.916)--(-9.021,-.908)--(-9.025,-.9)--cycle;
\draw(-9.092,-.916)--(-9.021,-.908)--(-9.025,-.9);
\filldraw[fill opacity=0.8,fill=gray!20,draw=none](-9.248,-.857)--(-9.312,-.829)--(-9.363,-.845)--(-9.306,-.87)--cycle;
\draw(-9.248,-.857)--(-9.312,-.829)--(-9.363,-.845)--(-9.306,-.87);
\filldraw[fill opacity=0.8,fill=gray!20,draw=none](-9.234,-.846)--(-9.241,-.843)--(-9.308,-.831)--(-9.283,-.841)--cycle;
\draw(-9.234,-.846)--(-9.241,-.843);
\draw(-9.308,-.831)--(-9.283,-.841);
\filldraw[fill opacity=0.8,fill=gray!20,draw=none](-9.266,-.828)--(-9.241,-.843)--(-9.296,-.836)--cycle;
\filldraw[fill opacity=0.8,fill=gray!20,draw=none](-9.092,-.916)--(-9.124,-.885)--(-9.155,-.881)--cycle;
\filldraw[fill opacity=0.8,fill=gray!20,draw=none](-9.092,-.916)--(-9.184,-.876)--(-9.217,-.854)--(-9.144,-.886)--cycle;
\draw(-9.092,-.916)--(-9.184,-.876);
\draw(-9.217,-.854)--(-9.144,-.886);
\filldraw[fill opacity=0.8,fill=gray!20,draw=none](-9.306,-.87)--(-9.206,-.859)--(-9.184,-.876)--(-9.369,-.896)--cycle;
\draw(-9.306,-.87)--(-9.206,-.859);
\draw(-9.184,-.876)--(-9.369,-.896);
\filldraw[fill opacity=0.8,fill=gray!20,draw=none](-9.364,-.872)--(-9.359,-.847)--(-9.363,-.845)--(-9.396,-.873)--cycle;
\draw(-9.359,-.847)--(-9.363,-.845)--(-9.396,-.873);
\filldraw[fill opacity=0.8,fill=gray!20,draw=none](-9.266,-.828)--(-9.296,-.836)--(-9.296,-.81)--cycle;
\draw(-9.296,-.836)--(-9.296,-.81);
\filldraw[fill opacity=0.8,fill=gray!20,draw=none](-9.256,-.892)--(-9.155,-.881)--(-9.231,-.864)--cycle;
\draw(-9.256,-.892)--(-9.155,-.881);
\filldraw[fill opacity=0.8,fill=gray!20,draw=none](-9.231,-.864)--(-9.206,-.859)--(-9.306,-.87)--cycle;
\draw(-9.206,-.859)--(-9.306,-.87);
\filldraw[fill opacity=0.8,fill=gray!20,draw=none](-9.206,-.859)--(-9.234,-.846)--(-9.283,-.841)--(-9.231,-.864)--cycle;
\draw(-9.206,-.859)--(-9.234,-.846);
\draw(-9.283,-.841)--(-9.231,-.864);
\filldraw[fill opacity=0.8,fill=gray!20,draw=none](-9.256,-.892)--(-9.359,-.847)--(-9.369,-.896)--(-9.277,-.936)--cycle;
\draw(-9.256,-.892)--(-9.359,-.847);
\draw(-9.369,-.896)--(-9.277,-.936);
\filldraw[fill opacity=0.8,fill=gray!20,draw=none](-9.296,-.836)--(-9.341,-.854)--(-9.341,-.832)--cycle;
\draw(-9.341,-.854)--(-9.341,-.832);
\filldraw[fill opacity=0.8,fill=gray!20,draw=none](-9.309,-.802)--(-9.296,-.81)--(-9.296,-.836)--(-9.341,-.832)--(-9.341,-.808)--cycle;
\draw(-9.296,-.81)--(-9.296,-.836);
\draw(-9.341,-.832)--(-9.341,-.808)--(-9.309,-.802);
\filldraw[fill opacity=0.8,fill=gray!20,draw=none](-9.364,-.872)--(-9.396,-.873)--(-9.405,-.881)--(-9.369,-.896)--cycle;
\draw(-9.396,-.873)--(-9.405,-.881)--(-9.369,-.896);
\filldraw[fill opacity=0.5,fill=gray!20,draw=none](-9.309,-.965)--(-9.316,-.966)--(-9.294,-.966)--(-9.295,-.965)--cycle;
\draw(-9.316,-.966)--(-9.294,-.966)--(-9.295,-.965);
\filldraw[fill opacity=0.8,fill=gray!20,draw=none](-9.394,-.921)--(-9.369,-.896)--(-9.405,-.881)--(-9.411,-.891)--cycle;
\draw(-9.369,-.896)--(-9.405,-.881)--(-9.411,-.891);
\filldraw[fill opacity=0.8,fill=gray!20,draw=none](-9.394,-.921)--(-9.411,-.891)--(-9.433,-.93)--(-9.412,-.939)--cycle;
\draw(-9.411,-.891)--(-9.433,-.93)--(-9.412,-.939);
\filldraw[fill opacity=0.8,fill=gray!20,draw=none](-9.394,-1.069)--(-9.412,-1.049)--(-9.341,-1.041)--cycle;
\draw(-9.412,-1.049)--(-9.341,-1.041);
\filldraw[fill opacity=0.8,fill=gray!20,draw=none](-9.394,-1.069)--(-9.366,-1.054)--(-9.369,-1.098)--cycle;
\filldraw[fill opacity=0.8,fill=gray!20,draw=none](-9.354,-.864)--(-9.364,-.816)--(-9.341,-.808)--(-9.341,-.854)--cycle;
\draw(-9.364,-.816)--(-9.341,-.808)--(-9.341,-.854);
\filldraw[fill opacity=0.8,fill=gray!20,draw=none](-9.364,-.816)--(-9.354,-.864)--(-9.369,-.874)--(-9.369,-.818)--cycle;
\draw(-9.369,-.874)--(-9.369,-.818)--(-9.364,-.816);
\filldraw[fill opacity=0.5,fill=gray!20,draw=none](-9.385,-.978)--(-9.369,-.974)--(-9.366,-.967)--(-9.394,-.968)--cycle;
\filldraw[fill opacity=0.8,fill=gray!20,draw=none](-9.372,-.875)--(-9.373,-.868)--(-9.371,-.821)--(-9.369,-.818)--(-9.369,-.874)--cycle;
\draw(-9.371,-.821)--(-9.369,-.818)--(-9.369,-.874);
\filldraw[fill opacity=0.8,fill=gray!20,draw=none](-9.352,-.829)--(-9.267,-.82)--(-9.309,-.802)--(-9.341,-.808)--(-9.369,-.818)--(-9.371,-.821)--cycle;
\draw(-9.309,-.802)--(-9.341,-.808)--(-9.369,-.818)--(-9.371,-.821);
\filldraw[fill opacity=0.8,fill=gray!20,draw=none](-9.353,-.872)--(-9.354,-.864)--(-9.341,-.854)--cycle;
\filldraw[fill opacity=0.8,fill=gray!20,draw=none](-9.354,-.864)--(-9.353,-.872)--(-9.369,-.896)--(-9.369,-.874)--cycle;
\draw(-9.369,-.896)--(-9.369,-.874);
\filldraw[fill opacity=0.8,fill=gray!20,draw=none](-9.369,-.896)--(-9.341,-.932)--(-9.365,-.934)--cycle;
\draw(-9.341,-.932)--(-9.365,-.934);
\filldraw[fill opacity=0.8,fill=gray!20,draw=none](-9.394,-.921)--(-9.369,-.896)--(-9.365,-.934)--(-9.398,-.938)--cycle;
\draw(-9.365,-.934)--(-9.398,-.938);
\filldraw[fill opacity=0.8,fill=gray!20,draw=none](-9.341,-.932)--(-9.341,-.966)--(-9.369,-.974)--(-9.369,-.896)--cycle;
\draw(-9.341,-.932)--(-9.341,-.966);
\draw(-9.369,-.974)--(-9.369,-.896);
\filldraw[fill opacity=0.8,fill=gray!20,draw=none](-9.155,-.881)--(-9.206,-.859)--(-9.231,-.864)--(-9.213,-.872)--cycle;
\draw(-9.155,-.881)--(-9.206,-.859);
\draw(-9.231,-.864)--(-9.213,-.872);
\filldraw[fill opacity=0.8,fill=gray!20,draw=none](-9.231,-.864)--(-9.195,-.86)--(-9.206,-.859)--cycle;
\draw(-9.231,-.864)--(-9.195,-.86);
\filldraw[fill opacity=0.8,fill=gray!20,draw=none](-9.394,-.921)--(-9.398,-.938)--(-9.412,-.939)--cycle;
\draw(-9.398,-.938)--(-9.412,-.939);
\filldraw[fill opacity=0.5,fill=gray!20,draw=none](-9.392,-.98)--(-9.385,-.978)--(-9.394,-.968)--(-9.409,-.969)--cycle;
\filldraw[fill opacity=0.8,fill=gray!20,draw=none](-9.388,-.95)--(-9.433,-.93)--(-9.443,-.986)--(-9.395,-1.007)--cycle;
\draw(-9.388,-.95)--(-9.433,-.93)--(-9.443,-.986)--(-9.395,-1.007);
\filldraw[fill opacity=0.5,fill=gray!20,draw=none](-9.682,-.914)--(-9.432,-.946)--(-9.473,-.897)--(-9.732,-.888)--cycle;
\draw(-9.473,-.897)--(-9.732,-.888);
\filldraw[fill opacity=0.8,fill=gray!20,draw=none](-9.341,-1.067)--(-9.341,-1.118)--(-9.369,-1.098)--cycle;
\draw(-9.341,-1.067)--(-9.341,-1.118);
\filldraw[fill opacity=0.8,fill=gray!20,draw=none](-9.341,-1.041)--(-9.341,-1.067)--(-9.369,-1.098)--cycle;
\draw(-9.341,-1.041)--(-9.341,-1.067);
\filldraw[fill opacity=0.8,fill=gray!20,draw=none](-9.341,-1.041)--(-9.241,-1.03)--(-9.184,-1.078)--(-9.369,-1.098)--cycle;
\draw(-9.341,-1.041)--(-9.241,-1.03);
\draw(-9.184,-1.078)--(-9.369,-1.098);
\filldraw[fill opacity=0.8,fill=gray!20,draw=none](-9.369,-1.098)--(-9.184,-1.078)--(-9.206,-1.123)--(-9.306,-1.134)--cycle;
\draw(-9.369,-1.098)--(-9.184,-1.078);
\draw(-9.206,-1.123)--(-9.306,-1.134);
\filldraw[fill opacity=0.8,fill=gray!20,draw=none](-9.366,-1.054)--(-9.341,-1.041)--(-9.369,-1.098)--cycle;
\filldraw[fill opacity=0.8,fill=gray!20,draw=none](-9.341,-.968)--(-9.341,-1.041)--(-9.369,-1.098)--(-9.369,-.981)--cycle;
\draw(-9.341,-.968)--(-9.341,-1.041);
\draw(-9.369,-1.098)--(-9.369,-.981);
\filldraw[fill opacity=0.5,fill=gray!20,draw=none](-9.419,-.962)--(-9.432,-.946)--(-9.81,-.898)--(-9.769,-.95)--cycle;
\draw(-9.81,-.898)--(-9.769,-.95)--(-9.419,-.962);
\filldraw[fill opacity=0.8,fill=gray!20,draw=none](-9.379,-1.014)--(-9.443,-.986)--(-9.433,-1.039)--(-9.377,-1.064)--cycle;
\draw(-9.379,-1.014)--(-9.443,-.986)--(-9.433,-1.039)--(-9.377,-1.064);
\filldraw[fill opacity=0.8,fill=gray!20,draw=none](-9.456,-.962)--(-9.43,-.962)--(-9.427,-.993)--(-9.46,-.996)--cycle;
\draw(-9.427,-.993)--(-9.46,-.996)--(-9.456,-.962);
\filldraw[fill opacity=0.8,fill=gray!20,draw=none](-9.43,-.962)--(-9.419,-.963)--(-9.427,-.993)--cycle;
\filldraw[fill opacity=0.5,fill=gray!20,draw=none](-9.443,-.97)--(-9.409,-.969)--(-9.419,-.962)--(-9.458,-.961)--cycle;
\draw(-9.419,-.962)--(-9.458,-.961);
\filldraw[fill opacity=0.8,fill=gray!20,draw=none](-9.456,-.962)--(-9.455,-.944)--(-9.412,-.939)--(-9.419,-.963)--cycle;
\draw(-9.456,-.962)--(-9.455,-.944)--(-9.412,-.939);
\filldraw[fill opacity=0.8,fill=gray!20,draw=none](-9.441,-.904)--(-9.389,-.898)--(-9.412,-.939)--(-9.455,-.944)--cycle;
\draw(-9.412,-.939)--(-9.455,-.944)--(-9.441,-.904)--(-9.389,-.898);
\filldraw[fill opacity=0.8,fill=gray!20,draw=none](-9.369,-.993)--(-9.369,-1.098)--(-9.377,-1.064)--(-9.377,-.997)--cycle;
\draw(-9.369,-.993)--(-9.369,-1.098);
\draw(-9.377,-1.064)--(-9.377,-.997);
\filldraw[fill opacity=0.8,fill=gray!20,draw=none](-9.369,-.974)--(-9.369,-.993)--(-9.377,-.997)--(-9.377,-.987)--cycle;
\draw(-9.369,-.974)--(-9.369,-.993);
\draw(-9.377,-.997)--(-9.377,-.987);
\filldraw[fill opacity=0.8,fill=gray!20,draw=none](-9.396,-1.088)--(-9.363,-1.109)--(-9.306,-1.134)--cycle;
\draw(-9.396,-1.088)--(-9.363,-1.109)--(-9.306,-1.134);
\filldraw[fill opacity=0.8,fill=gray!20,draw=none](-9.206,-1.123)--(-9.195,-1.146)--(-9.231,-1.15)--cycle;
\draw(-9.195,-1.146)--(-9.231,-1.15);
\filldraw[fill opacity=0.8,fill=gray!20,draw=none](-9.341,-.966)--(-9.341,-.968)--(-9.369,-.981)--(-9.369,-.974)--cycle;
\draw(-9.341,-.966)--(-9.341,-.968);
\draw(-9.369,-.981)--(-9.369,-.974);
\filldraw[fill opacity=0.8,fill=gray!20,draw=none](-9.366,-1.001)--(-9.367,-.997)--(-9.377,-.987)--(-9.377,-.997)--cycle;
\draw(-9.377,-.987)--(-9.377,-.997);
\filldraw[fill opacity=0.5,fill=gray!20,draw=none](-9.369,-.974)--(-9.378,-.989)--(-9.358,-1.003)--(-9.231,-1.007)--(-9.294,-.966)--(-9.316,-.966)--(-9.341,-.966)--cycle;
\draw(-9.358,-1.003)--(-9.231,-1.007)--(-9.294,-.966)--(-9.316,-.966);
\filldraw[fill opacity=0.5,fill=gray!20,draw=none](-9.369,-.974)--(-9.392,-.98)--(-9.378,-.989)--cycle;
\filldraw[fill opacity=0.8,fill=gray!20,draw=none](-9.369,-.896)--(-9.369,-.974)--(-9.377,-.987)--(-9.377,-.955)--cycle;
\draw(-9.369,-.896)--(-9.369,-.974);
\draw(-9.377,-.987)--(-9.377,-.955);
\filldraw[fill opacity=0.8,fill=gray!20,draw=none](-9.366,-1.001)--(-9.377,-.997)--(-9.377,-1.064)--(-9.362,-1.021)--cycle;
\draw(-9.377,-.997)--(-9.377,-1.064);
\filldraw[fill opacity=0.8,fill=gray!20,draw=none](-9.451,-.995)--(-9.427,-.993)--(-9.412,-1.049)--cycle;
\draw(-9.451,-.995)--(-9.427,-.993);
\filldraw[fill opacity=0.8,fill=gray!20,draw=none](-9.349,-1.005)--(-9.394,-.995)--(-9.395,-1.007)--(-9.362,-1.021)--cycle;
\draw(-9.395,-1.007)--(-9.362,-1.021);
\filldraw[fill opacity=0.5,fill=gray!20,draw=none](-9.327,-1.009)--(-9.277,-1.015)--(-9.22,-1.019)--(-9.189,-1.021)--(-9.167,-1.022)--(-9.231,-1.007)--(-9.362,-1.003)--cycle;
\draw(-9.189,-1.021)--(-9.167,-1.022)--(-9.231,-1.007)--(-9.362,-1.003);
\filldraw[fill opacity=0.8,fill=gray!20,draw=none](-9.389,-.898)--(-9.369,-.896)--(-9.412,-.939)--cycle;
\draw(-9.389,-.898)--(-9.369,-.896);
\filldraw[fill opacity=0.8,fill=gray!20,draw=none](-9.369,-.896)--(-9.394,-.921)--(-9.377,-.955)--cycle;
\filldraw[fill opacity=0.8,fill=gray!20,draw=none](-9.382,-.892)--(-9.374,-.885)--(-9.369,-.896)--cycle;
\filldraw[fill opacity=0.8,fill=gray!20,draw=none](-9.378,-.998)--(-9.377,-.955)--(-9.388,-.95)--(-9.394,-.995)--cycle;
\draw(-9.377,-.955)--(-9.388,-.95);
\filldraw[fill opacity=0.5,fill=gray!20,draw=none](-9.38,-.993)--(-9.378,-.989)--(-9.409,-.969)--(-9.419,-.969)--cycle;
\filldraw[fill opacity=0.8,fill=gray!20,draw=none](-9.419,-.882)--(-9.412,-.881)--(-9.369,-.896)--(-9.441,-.904)--cycle;
\draw(-9.369,-.896)--(-9.441,-.904)--(-9.419,-.882)--(-9.412,-.881);
\filldraw[fill opacity=0.8,fill=gray!20,draw=none](-9.374,-.885)--(-9.369,-.896)--(-9.377,-.955)--cycle;
\filldraw[fill opacity=0.5,fill=gray!20,draw=none](-9.394,-1.001)--(-9.385,-1.002)--(-9.38,-.993)--(-9.419,-.969)--(-9.443,-.97)--cycle;
\draw(-9.394,-1.001)--(-9.385,-1.002);
\filldraw[fill opacity=0.8,fill=gray!20,draw=none](-9.417,-.987)--(-9.417,-.882)--(-9.419,-.882)--(-9.441,-.904)--(-9.455,-.944)--(-9.456,-.962)--cycle;
\draw(-9.417,-.882)--(-9.419,-.882)--(-9.441,-.904)--(-9.455,-.944)--(-9.456,-.962);
\filldraw[fill opacity=0.8,fill=gray!20,draw=none](-9.374,-.885)--(-9.373,-.876)--(-9.372,-.875)--(-9.369,-.896)--cycle;
\filldraw[fill opacity=0.8,fill=gray!20,draw=none](-9.372,-.875)--(-9.369,-.874)--(-9.369,-.896)--cycle;
\draw(-9.369,-.874)--(-9.369,-.896);
\filldraw[fill opacity=0.8,fill=gray!20,draw=none](-9.374,-.885)--(-9.306,-.87)--(-9.369,-.896)--cycle;
\filldraw[fill opacity=0.8,fill=gray!20,draw=none](-9.18,-.858)--(-9.165,-.857)--(-9.153,-.855)--(-9.239,-.817)--(-9.267,-.82)--cycle;
\draw(-9.18,-.858)--(-9.165,-.857)--(-9.153,-.855);
\filldraw[fill opacity=0.8,fill=gray!20,draw=none](-9.279,-.902)--(-9.277,-.936)--(-9.092,-.916)--(-9.155,-.881)--(-9.256,-.892)--cycle;
\draw(-9.277,-.936)--(-9.092,-.916);
\draw(-9.155,-.881)--(-9.256,-.892);
\filldraw[fill opacity=0.8,fill=gray!20,draw=none](-9.197,-.879)--(-9.248,-.857)--(-9.306,-.87)--(-9.244,-.897)--cycle;
\draw(-9.197,-.879)--(-9.248,-.857);
\draw(-9.306,-.87)--(-9.244,-.897);
\filldraw[fill opacity=0.8,fill=gray!20,draw=none](-9.374,-.885)--(-9.382,-.892)--(-9.412,-.881)--(-9.377,-.877)--cycle;
\draw(-9.412,-.881)--(-9.377,-.877);
\filldraw[fill opacity=0.8,fill=gray!20,draw=none](-9.371,-.821)--(-9.373,-.876)--(-9.377,-.877)--(-9.377,-.831)--cycle;
\draw(-9.377,-.877)--(-9.377,-.831)--(-9.371,-.821);
\filldraw[fill opacity=0.8,fill=gray!20,draw=none](-9.18,-.858)--(-9.267,-.82)--(-9.375,-.832)--(-9.362,-.843)--(-9.327,-.853)--(-9.277,-.859)--(-9.22,-.86)--cycle;
\draw(-9.375,-.832)--(-9.362,-.843)--(-9.327,-.853)--(-9.277,-.859)--(-9.22,-.86)--(-9.18,-.858);
\filldraw[fill opacity=0.8,fill=gray!20,draw=none](-9.373,-.876)--(-9.373,-.868)--(-9.372,-.875)--cycle;
\filldraw[fill opacity=0.8,fill=gray!20,draw=none](-9.374,-.885)--(-9.363,-.876)--(-9.306,-.87)--cycle;
\draw(-9.363,-.876)--(-9.306,-.87);
\filldraw[fill opacity=0.8,fill=gray!20,draw=none](-9.373,-.876)--(-9.377,-.955)--(-9.377,-.877)--cycle;
\draw(-9.377,-.955)--(-9.377,-.877);
\filldraw[fill opacity=0.8,fill=gray!20,draw=none](-9.374,-.885)--(-9.377,-.877)--(-9.363,-.876)--cycle;
\draw(-9.377,-.877)--(-9.363,-.876);
\filldraw[fill opacity=0.8,fill=gray!20,draw=none](-9.298,-.872)--(-9.266,-.868)--(-9.306,-.87)--(-9.329,-.872)--cycle;
\draw(-9.298,-.872)--(-9.266,-.868);
\draw(-9.306,-.87)--(-9.329,-.872);
\filldraw[fill opacity=0.8,fill=gray!20,draw=none](-9.266,-.868)--(-9.231,-.864)--(-9.306,-.87)--cycle;
\draw(-9.266,-.868)--(-9.231,-.864);
\filldraw[fill opacity=0.8,fill=gray!20,draw=none](-9.352,-.829)--(-9.371,-.821)--(-9.377,-.831)--(-9.375,-.832)--cycle;
\draw(-9.371,-.821)--(-9.377,-.831)--(-9.375,-.832);
\filldraw[fill opacity=0.8,fill=gray!20,draw=none](-9.155,-.881)--(-9.195,-.86)--(-9.231,-.864)--cycle;
\draw(-9.195,-.86)--(-9.231,-.864);
\filldraw[fill opacity=0.8,fill=gray!20,draw=none](-9.362,-.879)--(-9.298,-.872)--(-9.329,-.872)--(-9.377,-.877)--cycle;
\draw(-9.362,-.879)--(-9.298,-.872);
\draw(-9.329,-.872)--(-9.377,-.877);
\filldraw[fill opacity=0.8,fill=gray!20,draw=none](-9.277,-.936)--(-9.369,-.896)--(-9.377,-.955)--(-9.327,-.977)--cycle;
\draw(-9.277,-.936)--(-9.369,-.896);
\draw(-9.377,-.955)--(-9.327,-.977);
\filldraw[fill opacity=0.8,fill=gray!20,draw=none](-9.366,-1.001)--(-9.349,-1.005)--(-9.327,-.977)--(-9.377,-.955)--cycle;
\draw(-9.327,-.977)--(-9.377,-.955);
\filldraw[fill opacity=0.8,fill=gray!20,draw=none](-9.377,-.877)--(-9.377,-.955)--(-9.362,-.983)--(-9.362,-.879)--cycle;
\draw(-9.377,-.877)--(-9.377,-.955);
\draw(-9.362,-.983)--(-9.362,-.879);
\filldraw[fill opacity=0.8,fill=gray!20,draw=none](-9.327,-.899)--(-9.256,-.892)--(-9.231,-.864)--(-9.362,-.879)--cycle;
\draw(-9.327,-.899)--(-9.256,-.892);
\draw(-9.231,-.864)--(-9.362,-.879);
\filldraw[fill opacity=0.8,fill=gray!20,draw=none](-9.306,-1.134)--(-9.231,-1.15)--(-9.266,-1.154)--cycle;
\draw(-9.231,-1.15)--(-9.266,-1.154);
\filldraw[fill opacity=0.8,fill=gray!20,draw=none](-9.266,-1.154)--(-9.231,-1.15)--(-9.256,-1.156)--cycle;
\draw(-9.266,-1.154)--(-9.231,-1.15);
\filldraw[fill opacity=0.8,fill=gray!20,draw=none](-9.256,-1.156)--(-9.306,-1.134)--(-9.231,-1.15)--(-9.213,-1.158)--cycle;
\draw(-9.256,-1.156)--(-9.306,-1.134);
\draw(-9.231,-1.15)--(-9.213,-1.158);
\filldraw[fill opacity=0.8,fill=gray!20,draw=none](-9.213,-1.158)--(-9.231,-1.15)--(-9.155,-1.145)--cycle;
\draw(-9.213,-1.158)--(-9.231,-1.15);
\filldraw[fill opacity=0.8,fill=gray!20,draw=none](-9.231,-1.15)--(-9.195,-1.146)--(-9.155,-1.145)--(-9.256,-1.156)--cycle;
\draw(-9.231,-1.15)--(-9.195,-1.146);
\draw(-9.155,-1.145)--(-9.256,-1.156);
\filldraw[fill opacity=0.8,fill=gray!20,draw=none](-9.231,-1.15)--(-9.248,-1.142)--(-9.206,-1.123)--(-9.155,-1.145)--cycle;
\draw(-9.231,-1.15)--(-9.248,-1.142);
\draw(-9.206,-1.123)--(-9.155,-1.145);
\filldraw[fill opacity=0.8,fill=gray!20,draw=none](-9.306,-1.134)--(-9.206,-1.123)--(-9.231,-1.15)--cycle;
\draw(-9.306,-1.134)--(-9.206,-1.123);
\filldraw[fill opacity=0.8,fill=gray!20,draw=none](-9.394,-.921)--(-9.412,-.939)--(-9.377,-.955)--cycle;
\draw(-9.412,-.939)--(-9.377,-.955);
\filldraw[fill opacity=0.8,fill=gray!20,draw=none](-9.362,-1.021)--(-9.379,-1.014)--(-9.377,-1.064)--(-9.327,-1.086)--cycle;
\draw(-9.362,-1.021)--(-9.379,-1.014);
\draw(-9.377,-1.064)--(-9.327,-1.086);
\filldraw[fill opacity=0.8,fill=gray!20,draw=none](-9.377,-1.064)--(-9.362,-1.06)--(-9.362,-1.021)--cycle;
\draw(-9.362,-1.06)--(-9.362,-1.021);
\filldraw[fill opacity=0.5,fill=gray!20,draw=none](-9.362,-1.003)--(-9.358,-1.003)--(-9.367,-.997)--cycle;
\draw(-9.362,-1.003)--(-9.358,-1.003);
\filldraw[fill opacity=0.8,fill=gray!20,draw=none](-9.377,-.955)--(-9.377,-.987)--(-9.362,-1.003)--(-9.362,-.983)--cycle;
\draw(-9.377,-.955)--(-9.377,-.987);
\draw(-9.362,-1.003)--(-9.362,-.983);
\filldraw[fill opacity=0.8,fill=gray!20,draw=none](-9.378,-.998)--(-9.366,-1.001)--(-9.377,-.955)--cycle;
\filldraw[fill opacity=0.8,fill=gray!20,draw=none](-9.367,-.997)--(-9.362,-1.021)--(-9.362,-1.003)--cycle;
\draw(-9.362,-1.021)--(-9.362,-1.003);
\filldraw[fill opacity=0.5,fill=gray!20,draw=none](-9.38,-.993)--(-9.375,-.996)--(-9.362,-1.003)--(-9.367,-.997)--(-9.378,-.989)--cycle;
\filldraw[fill opacity=0.5,fill=gray!20,draw=none](-9.385,-1.002)--(-9.365,-1.002)--(-9.38,-.993)--cycle;
\draw(-9.385,-1.002)--(-9.365,-1.002);
\filldraw[fill opacity=0.5,fill=gray!20,draw=none](-9.375,-.996)--(-9.365,-1.002)--(-9.362,-1.003)--cycle;
\draw(-9.365,-1.002)--(-9.362,-1.003);
\filldraw[fill opacity=0.8,fill=gray!20,draw=none](-9.362,-.853)--(-9.362,-1.03)--(-9.327,-1.025)--(-9.327,-.899)--cycle;
\draw(-9.362,-.853)--(-9.362,-1.03);
\draw(-9.327,-1.025)--(-9.327,-.899);
\filldraw[fill opacity=0.8,fill=gray!20,draw=none](-9.362,-1.03)--(-9.362,-1.06)--(-9.327,-1.086)--(-9.327,-1.025)--cycle;
\draw(-9.362,-1.03)--(-9.362,-1.06);
\draw(-9.327,-1.086)--(-9.327,-1.025);
\filldraw[fill opacity=0.8,fill=gray!20,draw=none](-9.417,-1)--(-9.401,-1.007)--(-9.401,-.882)--(-9.417,-.882)--cycle;
\draw(-9.401,-.882)--(-9.417,-.882);
\filldraw[fill opacity=0.8,fill=gray!20,draw=none](-9.377,-.877)--(-9.362,-.879)--(-9.362,-.853)--cycle;
\draw(-9.362,-.879)--(-9.362,-.853);
\filldraw[fill opacity=0.8,fill=gray!20,draw=none](-9.375,-.832)--(-9.377,-.839)--(-9.377,-.877)--(-9.362,-.853)--(-9.362,-.843)--cycle;
\draw(-9.377,-.839)--(-9.377,-.877);
\draw(-9.362,-.853)--(-9.362,-.843)--(-9.375,-.832);
\filldraw[fill opacity=0.8,fill=gray!20,draw=none](-9.362,-.879)--(-9.377,-.877)--(-9.397,-.88)--cycle;
\draw(-9.377,-.877)--(-9.397,-.88);
\filldraw[fill opacity=0.8,fill=gray!20,draw=none](-9.375,-.832)--(-9.377,-.831)--(-9.377,-.839)--cycle;
\draw(-9.375,-.832)--(-9.377,-.831)--(-9.377,-.839);
\filldraw[fill opacity=0.8,fill=gray!20,draw=none](-9.394,-.882)--(-9.362,-.879)--(-9.397,-.88)--(-9.419,-.882)--cycle;
\draw(-9.397,-.88)--(-9.419,-.882)--(-9.394,-.882)--(-9.362,-.879);
\filldraw[fill opacity=0.5,fill=gray!20,draw=none](-9.108,-.964)--(-9.11,-.964)--(-9.106,-.967)--cycle;
\filldraw[fill opacity=0.5,fill=gray!20,draw=none](-9.108,-.964)--(-9.106,-.967)--(-9.096,-.973)--(-9.076,-.968)--cycle;
\draw(-9.096,-.973)--(-9.076,-.968);
\filldraw[fill opacity=0.8,fill=gray!20,draw=none](-9.1,-.954)--(-9.1,-.974)--(-9.135,-.96)--(-9.135,-.928)--cycle;
\draw(-9.1,-.954)--(-9.1,-.974);
\draw(-9.135,-.96)--(-9.135,-.928);
\filldraw[fill opacity=0.5,fill=gray!20,draw=none](-9.103,-.975)--(-9.1,-.974)--(-9.113,-.969)--cycle;
\draw(-9.103,-.975)--(-9.1,-.974);
\filldraw[fill opacity=0.5,fill=gray!20,draw=none](-9.081,-.976)--(-9.076,-.968)--(-9.096,-.973)--cycle;
\draw(-9.076,-.968)--(-9.096,-.973);
\filldraw[fill opacity=0.8,fill=gray!20,draw=none](-9.085,-1.06)--(-9.135,-1.038)--(-9.108,-1.003)--(-9.098,-1.001)--cycle;
\draw(-9.085,-1.06)--(-9.135,-1.038);
\filldraw[fill opacity=0.8,fill=gray!20,draw=none](-9.108,-1.003)--(-9.1,-.993)--(-9.1,-1.001)--cycle;
\draw(-9.1,-.993)--(-9.1,-1.001);
\filldraw[fill opacity=0.8,fill=gray!20,draw=none](-9.098,-1.001)--(-9.1,-1.001)--(-9.1,-.993)--cycle;
\draw(-9.1,-1.001)--(-9.1,-.993);
\filldraw[fill opacity=0.8,fill=gray!20,draw=none](-9.108,-1.003)--(-9.1,-.993)--(-9.086,-.999)--cycle;
\draw(-9.1,-.993)--(-9.086,-.999);
\filldraw[fill opacity=0.5,fill=gray!20,draw=none](-9.094,-.977)--(-9.083,-.98)--(-9.081,-.976)--(-9.096,-.973)--(-9.1,-.974)--cycle;
\draw(-9.096,-.973)--(-9.1,-.974);
\filldraw[fill opacity=0.8,fill=gray!20,draw=none](-9.085,-.95)--(-9.095,-.98)--(-9.1,-.974)--(-9.1,-.954)--cycle;
\draw(-9.1,-.974)--(-9.1,-.954);
\filldraw[fill opacity=0.8,fill=gray!20,draw=none](-9.094,-.977)--(-9.083,-.98)--(-9.083,-.987)--(-9.095,-.98)--cycle;
\filldraw[fill opacity=0.5,fill=gray!20,draw=none](-9.094,-.977)--(-9.1,-.974)--(-9.103,-.975)--cycle;
\draw(-9.1,-.974)--(-9.103,-.975);
\filldraw[fill opacity=0.8,fill=gray!20,draw=none](-9.094,-.977)--(-9.103,-.975)--(-9.113,-.969)--(-9.135,-.928)--(-9.085,-.95)--cycle;
\draw(-9.135,-.928)--(-9.085,-.95);
\filldraw[fill opacity=0.8,fill=gray!20,draw=none](-9.094,-.977)--(-9.085,-.982)--(-9.085,-.999)--(-9.098,-1.001)--(-9.1,-.993)--cycle;
\draw(-9.085,-.982)--(-9.085,-.999);
\filldraw[fill opacity=0.8,fill=gray!20,draw=none](-9.087,-1.049)--(-9.098,-1.001)--(-9.086,-.999)--(-9.039,-1.02)--cycle;
\draw(-9.086,-.999)--(-9.039,-1.02);
\filldraw[fill opacity=0.8,fill=gray!20,draw=none](-9.085,-.999)--(-9.085,-1.047)--(-9.087,-1.049)--(-9.098,-1.001)--cycle;
\draw(-9.085,-.999)--(-9.085,-1.047);
\filldraw[fill opacity=0.8,fill=gray!20,draw=none](-9.094,-.977)--(-9.095,-.98)--(-9.103,-.975)--cycle;
\filldraw[fill opacity=0.8,fill=gray!20,draw=none](-9.07,-.995)--(-9.036,-1.016)--(-9.035,-1.017)--(-9.039,-1.02)--(-9.086,-.999)--cycle;
\draw(-9.039,-1.02)--(-9.086,-.999);
\filldraw[fill opacity=0.8,fill=gray!20,draw=none](-9.049,-1.075)--(-9.085,-1.06)--(-9.087,-1.049)--(-9.039,-1.02)--(-9.034,-1.021)--cycle;
\draw(-9.049,-1.075)--(-9.085,-1.06);
\draw(-9.039,-1.02)--(-9.034,-1.021);
\filldraw[fill opacity=0.8,fill=gray!20,draw=none](-9.085,-1.047)--(-9.085,-1.06)--(-9.087,-1.049)--cycle;
\draw(-9.085,-1.047)--(-9.085,-1.06);
\filldraw[fill opacity=0.8,fill=gray!20,draw=none](-9.092,-1.002)--(-9.092,-1.118)--(-9.085,-1.06)--(-9.085,-.999)--cycle;
\draw(-9.092,-1.002)--(-9.092,-1.118);
\draw(-9.085,-1.06)--(-9.085,-.999);
\filldraw[fill opacity=0.8,fill=gray!20,draw=none](-9.043,-.99)--(-9.043,-.99)--(-9.07,-.995)--(-9.083,-.987)--(-9.083,-.98)--cycle;
\filldraw[fill opacity=0.8,fill=gray!20,draw=none](-9.12,-1.008)--(-9.12,-1.009)--(-9.092,-1.002)--(-9.092,-.994)--cycle;
\draw(-9.12,-1.008)--(-9.12,-1.009);
\draw(-9.092,-1.002)--(-9.092,-.994);
\filldraw[fill opacity=0.5,fill=gray!20,draw=none](-9.145,-1.016)--(-9.096,-1.001)--(-9.085,-.982)--(-9.094,-.977)--(-9.103,-.975)--(-9.231,-1.007)--(-9.167,-1.022)--cycle;
\draw(-9.103,-.975)--(-9.231,-1.007)--(-9.167,-1.022)--(-9.145,-1.016);
\filldraw[fill opacity=0.5,fill=gray!20,draw=none](-9.083,-.98)--(-9.094,-.977)--(-9.085,-.982)--cycle;
\filldraw[fill opacity=0.8,fill=gray!20,draw=none](-9.083,-.98)--(-9.094,-.977)--(-9.085,-.95)--cycle;
\filldraw[fill opacity=0.8,fill=gray!20,draw=none](-9.094,-.977)--(-9.085,-.95)--(-9.085,-.982)--cycle;
\draw(-9.085,-.95)--(-9.085,-.982);
\filldraw[fill opacity=0.8,fill=gray!20,draw=none](-9.144,-.886)--(-9.155,-.881)--(-9.213,-.872)--(-9.197,-.879)--cycle;
\draw(-9.144,-.886)--(-9.155,-.881);
\draw(-9.213,-.872)--(-9.197,-.879);
\filldraw[fill opacity=0.5,fill=gray!20,draw=none](-9.081,-.976)--(-9.031,-.987)--(-9.005,-.98)--(-9.067,-.966)--(-9.076,-.968)--cycle;
\draw(-9.067,-.966)--(-9.076,-.968);
\filldraw[fill opacity=0.5,fill=gray!20,draw=none](-9.083,-.98)--(-9.043,-.99)--(-9.031,-.987)--(-9.081,-.976)--cycle;
\filldraw[fill opacity=0.8,fill=gray!20,draw=none](-9.043,-.99)--(-9.083,-.98)--(-9.085,-.95)--(-9.049,-.966)--cycle;
\draw(-9.085,-.95)--(-9.049,-.966);
\filldraw[fill opacity=0.8,fill=gray!20,draw=none](-9.092,-.916)--(-9.092,-1.002)--(-9.085,-.999)--(-9.085,-.95)--cycle;
\draw(-9.092,-.916)--(-9.092,-1.002);
\draw(-9.085,-.999)--(-9.085,-.95);
\filldraw[fill opacity=0.8,fill=gray!20,draw=none](-9.12,-.973)--(-9.12,-1.008)--(-9.092,-.994)--(-9.092,-.916)--cycle;
\draw(-9.12,-.973)--(-9.12,-1.008);
\draw(-9.092,-.994)--(-9.092,-.916);
\filldraw[fill opacity=0.8,fill=gray!20,draw=none](-9.22,-.984)--(-9.12,-.973)--(-9.06,-.941)--(-9.072,-.914)--(-9.277,-.936)--cycle;
\draw(-9.22,-.984)--(-9.12,-.973);
\draw(-9.072,-.914)--(-9.277,-.936);
\filldraw[fill opacity=0.8,fill=gray!20,draw=none](-9.108,-.904)--(-9.092,-.916)--(-9.144,-.886)--(-9.127,-.893)--cycle;
\draw(-9.144,-.886)--(-9.127,-.893);
\filldraw[fill opacity=0.8,fill=gray!20,draw=none](-9.105,-.888)--(-9.108,-.904)--(-9.092,-.916)--cycle;
\filldraw[fill opacity=0.8,fill=gray!20,draw=none](-9.1,-.899)--(-9.092,-.916)--(-9.092,-.894)--cycle;
\draw(-9.092,-.916)--(-9.092,-.894);
\filldraw[fill opacity=0.8,fill=gray!20,draw=none](-9.277,-1.138)--(-9.405,-1.083)--(-9.396,-1.088)--(-9.306,-1.134)--(-9.256,-1.156)--cycle;
\draw(-9.277,-1.138)--(-9.405,-1.083)--(-9.396,-1.088);
\draw(-9.306,-1.134)--(-9.256,-1.156);
\filldraw[fill opacity=0.8,fill=gray!20,draw=none](-9.321,-1.131)--(-9.341,-1.14)--(-9.341,-1.118)--cycle;
\draw(-9.341,-1.14)--(-9.341,-1.118);
\filldraw[fill opacity=0.8,fill=gray!20,draw=none](-9.337,-1.007)--(-9.349,-1.005)--(-9.362,-1.021)--(-9.344,-1.029)--cycle;
\draw(-9.362,-1.021)--(-9.344,-1.029);
\filldraw[fill opacity=0.5,fill=gray!20,draw=none](-9.327,-1.009)--(-9.362,-1.003)--(-9.385,-1.002)--cycle;
\draw(-9.362,-1.003)--(-9.385,-1.002);
\filldraw[fill opacity=0.8,fill=gray!20,draw=none](-9.327,-1.025)--(-9.327,-1.086)--(-9.277,-1.138)--(-9.277,-1.022)--cycle;
\draw(-9.327,-1.025)--(-9.327,-1.086);
\draw(-9.277,-1.138)--(-9.277,-1.022);
\filldraw[fill opacity=0.8,fill=gray!20,draw=none](-9.306,-.915)--(-9.327,-.977)--(-9.327,-1.025)--(-9.277,-1.022)--(-9.277,-.936)--cycle;
\draw(-9.327,-.977)--(-9.327,-1.025);
\draw(-9.277,-1.022)--(-9.277,-.936);
\filldraw[fill opacity=0.8,fill=gray!20,draw=none](-9.293,-1.017)--(-9.337,-1.007)--(-9.343,-1.026)--cycle;
\filldraw[fill opacity=0.5,fill=gray!20,draw=none](-9.332,-1.016)--(-9.293,-1.017)--(-9.337,-1.007)--(-9.385,-1.002)--(-9.394,-1.001)--cycle;
\draw(-9.385,-1.002)--(-9.394,-1.001);
\filldraw[fill opacity=0.8,fill=gray!20,draw=none](-9.401,-1)--(-9.372,-1.007)--(-9.372,-.902)--(-9.394,-.882)--(-9.401,-.882)--cycle;
\draw(-9.372,-.902)--(-9.394,-.882)--(-9.401,-.882);
\filldraw[fill opacity=0.8,fill=gray!20,draw=none](-9.127,-.893)--(-9.144,-.886)--(-9.197,-.879)--(-9.178,-.887)--cycle;
\draw(-9.127,-.893)--(-9.144,-.886);
\draw(-9.197,-.879)--(-9.178,-.887);
\filldraw[fill opacity=0.8,fill=gray!20,draw=none](-9.049,-.966)--(-9.006,-.961)--(-9.021,-.908)--(-9.072,-.914)--cycle;
\draw(-9.049,-.966)--(-9.006,-.961)--(-9.021,-.908)--(-9.072,-.914);
\filldraw[fill opacity=0.8,fill=gray!20,draw=none](-9.108,-.904)--(-9.105,-.888)--(-9.12,-.858)--(-9.12,-.896)--cycle;
\draw(-9.12,-.858)--(-9.12,-.896);
\filldraw[fill opacity=0.8,fill=gray!20,draw=none](-9.056,-.932)--(-9.082,-.921)--(-9.12,-.896)--(-9.099,-.905)--cycle;
\draw(-9.12,-.896)--(-9.099,-.905)--(-9.056,-.932)--(-9.082,-.921);
\filldraw[fill opacity=0.8,fill=gray!20,draw=none](-9.082,-.921)--(-9.127,-.893)--(-9.12,-.896)--cycle;
\draw(-9.127,-.893)--(-9.12,-.896);
\filldraw[fill opacity=0.8,fill=gray!20,draw=none](-9.369,-.904)--(-9.327,-.899)--(-9.362,-.879)--(-9.394,-.882)--cycle;
\draw(-9.362,-.879)--(-9.394,-.882)--(-9.369,-.904)--(-9.327,-.899);
\filldraw[fill opacity=0.8,fill=gray!20,draw=none](-9.178,-.887)--(-9.197,-.879)--(-9.244,-.897)--(-9.227,-.904)--cycle;
\draw(-9.178,-.887)--(-9.197,-.879);
\draw(-9.244,-.897)--(-9.227,-.904);
\filldraw[fill opacity=0.8,fill=gray!20,draw=none](-9.108,-.904)--(-9.12,-.896)--(-9.12,-.973)--cycle;
\draw(-9.12,-.896)--(-9.12,-.973);
\filldraw[fill opacity=0.8,fill=gray!20,draw=none](-9.33,-.852)--(-9.362,-.843)--(-9.362,-.853)--(-9.327,-.899)--(-9.327,-.861)--cycle;
\draw(-9.33,-.852)--(-9.362,-.843)--(-9.362,-.853);
\draw(-9.327,-.899)--(-9.327,-.861);
\filldraw[fill opacity=0.8,fill=gray!20,draw=none](-9.25,-.904)--(-9.256,-.892)--(-9.277,-.936)--cycle;
\filldraw[fill opacity=0.8,fill=gray!20,draw=none](-9.244,-.919)--(-9.22,-.907)--(-9.256,-.892)--cycle;
\draw(-9.22,-.907)--(-9.256,-.892);
\filldraw[fill opacity=0.8,fill=gray!20,draw=none](-9.306,-.915)--(-9.256,-.892)--(-9.327,-.899)--cycle;
\draw(-9.256,-.892)--(-9.327,-.899);
\filldraw[fill opacity=0.5,fill=gray!20,draw=none](-9.043,-.99)--(-9.083,-.98)--(-9.085,-.982)--(-9.062,-.995)--cycle;
\filldraw[fill opacity=0.8,fill=gray!20,draw=none](-9.043,-.99)--(-9.036,-1.016)--(-9.07,-.995)--cycle;
\filldraw[fill opacity=0.5,fill=gray!20,draw=none](-9.097,-1.004)--(-9.092,-1.003)--(-9.062,-.995)--(-9.085,-.982)--cycle;
\draw(-9.097,-1.004)--(-9.092,-1.003);
\filldraw[fill opacity=0.8,fill=gray!20,draw=none](-9.296,-1.05)--(-9.362,-1.021)--(-9.327,-1.086)--(-9.291,-1.101)--cycle;
\draw(-9.296,-1.05)--(-9.362,-1.021);
\draw(-9.327,-1.086)--(-9.291,-1.101);
\filldraw[fill opacity=0.8,fill=gray!20,draw=none](-9.152,-1.018)--(-9.12,-1.009)--(-9.12,-.973)--cycle;
\draw(-9.12,-1.009)--(-9.12,-.973);
\filldraw[fill opacity=0.5,fill=gray!20,draw=none](-9.145,-1.016)--(-9.097,-1.004)--(-9.096,-1.001)--cycle;
\draw(-9.145,-1.016)--(-9.097,-1.004);
\filldraw[fill opacity=0.5,fill=gray!20,draw=none](-9.322,-1.011)--(-9.308,-1.012)--(-9.327,-1.009)--(-9.337,-1.007)--cycle;
\filldraw[fill opacity=0.8,fill=gray!20,draw=none](-9.181,-1.021)--(-9.168,-1.022)--(-9.043,-.99)--(-9.049,-.966)--(-9.22,-.984)--cycle;
\draw(-9.049,-.966)--(-9.22,-.984);
\filldraw[fill opacity=0.8,fill=gray!20,draw=none](-9.165,-.893)--(-9.165,-1.021)--(-9.152,-1.018)--(-9.12,-.973)--(-9.12,-.896)--cycle;
\draw(-9.165,-.893)--(-9.165,-1.021);
\draw(-9.12,-.973)--(-9.12,-.896);
\filldraw[fill opacity=0.8,fill=gray!20,draw=none](-9.12,-1.009)--(-9.12,-1.083)--(-9.092,-1.118)--(-9.092,-1.002)--cycle;
\draw(-9.12,-1.009)--(-9.12,-1.083);
\draw(-9.092,-1.118)--(-9.092,-1.002);
\filldraw[fill opacity=0.8,fill=gray!20,draw=none](-9.033,-.988)--(-9.005,-.98)--(-9.006,-.961)--(-9.049,-.966)--cycle;
\draw(-9.005,-.98)--(-9.006,-.961)--(-9.049,-.966);
\filldraw[fill opacity=0.5,fill=gray!20,draw=none](-9.031,-.987)--(-9.03,-.987)--(-9.004,-.98)--(-9.005,-.98)--cycle;
\draw(-9.03,-.987)--(-9.004,-.98);
\filldraw[fill opacity=0.5,fill=gray!20,draw=none](-9.043,-.99)--(-9.042,-.99)--(-9.03,-.987)--(-9.031,-.987)--cycle;
\draw(-9.042,-.99)--(-9.03,-.987);
\filldraw[fill opacity=0.8,fill=gray!20,draw=none](-9.043,-.99)--(-9.033,-.988)--(-9.049,-.966)--cycle;
\filldraw[fill opacity=0.8,fill=gray!20,draw=none](-9.12,-.973)--(-9.049,-.966)--(-9.06,-.941)--cycle;
\draw(-9.12,-.973)--(-9.049,-.966);
\filldraw[fill opacity=0.8,fill=gray!20,draw=none](-9.108,-.904)--(-9.082,-.921)--(-9.092,-.916)--cycle;
\draw(-9.082,-.921)--(-9.092,-.916);
\filldraw[fill opacity=0.8,fill=gray!20,draw=none](-9.028,-.975)--(-9.049,-.966)--(-9.092,-.916)--(-9.056,-.932)--cycle;
\draw(-9.092,-.916)--(-9.056,-.932)--(-9.028,-.975)--(-9.049,-.966);
\filldraw[fill opacity=0.8,fill=gray!20,draw=none](-9.108,-.904)--(-9.12,-.973)--(-9.092,-.916)--cycle;
\filldraw[fill opacity=0.8,fill=gray!20,draw=none](-9.043,-.99)--(-9.042,-.99)--(-9.043,-.99)--cycle;
\filldraw[fill opacity=0.8,fill=gray!20,draw=none](-9.026,-.987)--(-9.042,-.99)--(-9.043,-.99)--(-9.049,-.966)--(-9.028,-.975)--cycle;
\draw(-9.049,-.966)--(-9.028,-.975)--(-9.026,-.987);
\filldraw[fill opacity=0.5,fill=gray!20,draw=none](-9.043,-.99)--(-9.062,-.995)--(-9.062,-.995)--(-9.042,-.99)--cycle;
\draw(-9.062,-.995)--(-9.042,-.99);
\filldraw[fill opacity=0.5,fill=gray!20,draw=none](-9.092,-1.003)--(-9.062,-.995)--(-9.062,-.995)--cycle;
\draw(-9.092,-1.003)--(-9.062,-.995);
\filldraw[fill opacity=0.8,fill=gray!20,draw=none](-9.152,-1.018)--(-9.155,-1.022)--(-9.12,-1.024)--(-9.12,-1.009)--cycle;
\draw(-9.12,-1.024)--(-9.12,-1.009);
\filldraw[fill opacity=0.8,fill=gray!20,draw=none](-9.163,-1.017)--(-9.149,-1.018)--(-9.026,-.987)--(-9.028,-.975)--(-9.056,-.932)--(-9.099,-.905)--(-9.149,-.9)--(-9.163,-.905)--cycle;
\draw(-9.026,-.987)--(-9.028,-.975)--(-9.056,-.932)--(-9.099,-.905)--(-9.149,-.9)--(-9.163,-.905);
\filldraw[fill opacity=0.8,fill=gray!20,draw=none](-9.135,-.897)--(-9.12,-.896)--(-9.127,-.893)--(-9.178,-.887)--(-9.165,-.893)--cycle;
\draw(-9.12,-.896)--(-9.127,-.893);
\draw(-9.178,-.887)--(-9.165,-.893);
\filldraw[fill opacity=0.8,fill=gray!20,draw=none](-9.165,-.893)--(-9.12,-.896)--(-9.12,-.874)--cycle;
\draw(-9.12,-.896)--(-9.12,-.874);
\filldraw[fill opacity=0.8,fill=gray!20,draw=none](-9.099,-.905)--(-9.12,-.896)--(-9.153,-.898)--(-9.149,-.9)--cycle;
\draw(-9.153,-.898)--(-9.149,-.9)--(-9.099,-.905)--(-9.12,-.896);
\filldraw[fill opacity=0.8,fill=gray!20,draw=none](-9.22,-.907)--(-9.22,-1.02)--(-9.165,-1.021)--(-9.165,-.893)--cycle;
\draw(-9.22,-.907)--(-9.22,-1.02);
\draw(-9.165,-1.021)--(-9.165,-.893);
\filldraw[fill opacity=0.8,fill=gray!20,draw=none](-9.19,-.903)--(-9.165,-.893)--(-9.178,-.887)--(-9.227,-.904)--(-9.22,-.907)--cycle;
\draw(-9.165,-.893)--(-9.178,-.887);
\draw(-9.227,-.904)--(-9.22,-.907);
\filldraw[fill opacity=0.8,fill=gray!20,draw=none](-9.306,-.915)--(-9.287,-.858)--(-9.327,-.853)--(-9.327,-.899)--cycle;
\draw(-9.287,-.858)--(-9.327,-.853)--(-9.327,-.899);
\filldraw[fill opacity=0.8,fill=gray!20,draw=none](-9.165,-.867)--(-9.165,-.893)--(-9.141,-.883)--cycle;
\draw(-9.165,-.867)--(-9.165,-.893);
\filldraw[fill opacity=0.8,fill=gray!20,draw=none](-9.33,-.852)--(-9.327,-.861)--(-9.327,-.853)--cycle;
\draw(-9.327,-.861)--(-9.327,-.853)--(-9.33,-.852);
\filldraw[fill opacity=0.8,fill=gray!20,draw=none](-9.306,-.915)--(-9.327,-.899)--(-9.327,-.977)--cycle;
\draw(-9.327,-.899)--(-9.327,-.977);
\filldraw[fill opacity=0.8,fill=gray!20,draw=none](-9.279,-.902)--(-9.352,-.936)--(-9.348,-.944)--(-9.277,-.936)--cycle;
\draw(-9.352,-.936)--(-9.348,-.944)--(-9.277,-.936);
\filldraw[fill opacity=0.8,fill=gray!20,draw=none](-9.287,-.858)--(-9.306,-.915)--(-9.277,-.936)--(-9.277,-.859)--cycle;
\draw(-9.277,-.936)--(-9.277,-.859)--(-9.287,-.858);
\filldraw[fill opacity=0.8,fill=gray!20,draw=none](-9.135,-.897)--(-9.165,-.893)--(-9.153,-.898)--cycle;
\draw(-9.165,-.893)--(-9.153,-.898);
\filldraw[fill opacity=0.8,fill=gray!20,draw=none](-9.249,-.903)--(-9.265,-.859)--(-9.277,-.859)--(-9.277,-.936)--cycle;
\draw(-9.265,-.859)--(-9.277,-.859)--(-9.277,-.936);
\filldraw[fill opacity=0.8,fill=gray!20,draw=none](-9.22,-.86)--(-9.22,-.907)--(-9.165,-.893)--(-9.165,-.857)--cycle;
\draw(-9.165,-.893)--(-9.165,-.857)--(-9.22,-.86)--(-9.22,-.907);
\filldraw[fill opacity=0.8,fill=gray!20,draw=none](-9.149,-.9)--(-9.165,-.893)--(-9.21,-.911)--(-9.199,-.916)--cycle;
\draw(-9.21,-.911)--(-9.199,-.916)--(-9.149,-.9)--(-9.165,-.893);
\filldraw[fill opacity=0.8,fill=gray!20,draw=none](-9.265,-.859)--(-9.244,-.919)--(-9.22,-.907)--(-9.22,-.86)--cycle;
\draw(-9.22,-.907)--(-9.22,-.86)--(-9.265,-.859);
\filldraw[fill opacity=0.5,fill=gray!20](-9,-.906)--(-9.461,-.705)--(-9.891,-.69)--(-9.43,-.892)--cycle;
\filldraw[fill opacity=0.8,fill=gray!20](-8.668,3.211)--(-8.669,3.265)--(-8.565,3.257)--(-8.561,3.203)--cycle;
\filldraw[fill opacity=0.8,fill=gray!20](-8.669,3.265)--(-8.67,3.312)--(-8.577,3.305)--(-8.565,3.257)--cycle;
\filldraw[fill opacity=0.8,fill=gray!20](-8.776,3.26)--(-8.767,3.308)--(-8.67,3.312)--(-8.669,3.265)--cycle;
\filldraw[fill opacity=0.5,fill=gray!20](-7.835,-.296)--(-7.87,-.294)--(-8.25,-.653)--(-8.207,-.648)--cycle;
\filldraw[fill opacity=0.8,fill=gray!20](-8.565,3.146)--(-8.561,3.203)--(-8.485,3.185)--(-8.492,3.129)--cycle;
\filldraw[fill opacity=0.8,fill=gray!20](-8.561,3.203)--(-8.565,3.257)--(-8.492,3.239)--(-8.485,3.185)--cycle;
\filldraw[fill opacity=0.8,fill=gray!20](-8.763,3.199)--(-8.761,3.244)--(-8.671,3.248)--(-8.671,3.203)--cycle;
\filldraw[fill opacity=0.8,fill=gray!20](-8.671,3.203)--(-8.671,3.248)--(-8.584,3.242)--(-8.581,3.197)--cycle;
\filldraw[fill opacity=0.8,fill=gray!20](-8.859,3.244)--(-8.84,3.293)--(-8.767,3.308)--(-8.776,3.26)--cycle;
\filldraw[fill opacity=0.8,fill=gray!20](-8.671,3.248)--(-8.672,3.287)--(-8.594,3.281)--(-8.584,3.242)--cycle;
\filldraw[fill opacity=0.8,fill=gray!20](-8.761,3.244)--(-8.752,3.284)--(-8.672,3.287)--(-8.671,3.248)--cycle;
\filldraw[fill opacity=0.8,fill=gray!20](-8.767,3.308)--(-8.751,3.345)--(-8.672,3.349)--(-8.67,3.312)--cycle;
\filldraw[fill opacity=0.8,fill=gray!20](-8.399,4.351)--(-8.406,4.408)--(-8.321,4.424)--(-8.317,4.367)--cycle;
\filldraw[fill opacity=0.8,fill=gray!20](-8.381,4.297)--(-8.399,4.351)--(-8.317,4.367)--(-8.307,4.311)--cycle;
\filldraw[fill opacity=0.8,fill=gray!20](-8.565,3.257)--(-8.577,3.305)--(-8.512,3.289)--(-8.492,3.239)--cycle;
\filldraw[fill opacity=0.8,fill=gray!20](-8.406,4.408)--(-8.399,4.462)--(-8.317,4.478)--(-8.321,4.424)--cycle;
\filldraw[fill opacity=0.8,fill=gray!20](-8.584,3.149)--(-8.581,3.197)--(-8.518,3.181)--(-8.524,3.134)--cycle;
\filldraw[fill opacity=0.8,fill=gray!20](-8.352,4.249)--(-8.381,4.297)--(-8.307,4.311)--(-8.292,4.26)--cycle;
\filldraw[fill opacity=0.8,fill=gray!20](-8.581,3.197)--(-8.584,3.242)--(-8.524,3.227)--(-8.518,3.181)--cycle;
\filldraw[fill opacity=0.8,fill=gray!20,draw=none](-8.725,2.992)--(-8.731,2.999)--(-8.675,3.002)--(-8.677,2.984)--cycle;
\draw(-8.725,2.992)--(-8.731,2.999)--(-8.675,3.002)--(-8.677,2.984);
\filldraw[fill opacity=0.8,fill=gray!20](-8.829,3.231)--(-8.814,3.272)--(-8.752,3.284)--(-8.761,3.244)--cycle;
\filldraw[fill opacity=0.8,fill=gray!20](-8.84,3.293)--(-8.811,3.334)--(-8.751,3.345)--(-8.767,3.308)--cycle;
\filldraw[fill opacity=0.8,fill=gray!20,draw=none](-8.745,3.056)--(-8.751,3.06)--(-8.739,3.062)--(-8.735,3.053)--cycle;
\draw(-8.751,3.06)--(-8.739,3.062)--(-8.735,3.053);
\filldraw[fill opacity=0.8,fill=gray!20,draw=none](-8.741,2.997)--(-8.718,2.989)--(-8.721,2.983)--cycle;
\draw(-8.718,2.989)--(-8.721,2.983);
\filldraw[fill opacity=0.8,fill=gray!20,draw=none](-8.696,2.974)--(-8.723,2.979)--(-8.718,2.989)--cycle;
\draw(-8.723,2.979)--(-8.718,2.989);
\filldraw[fill opacity=0.8,fill=gray!20](-8.729,2.963)--(-8.773,2.991)--(-8.731,2.999)--(-8.707,2.967)--cycle;
\filldraw[fill opacity=0.8,fill=gray!20](-8.399,4.462)--(-8.381,4.511)--(-8.307,4.526)--(-8.317,4.478)--cycle;
\filldraw[fill opacity=0.8,fill=gray!20,draw=none](-8.808,3.027)--(-8.81,3.029)--(-8.792,3.07)--(-8.755,3.059)--(-8.746,3.054)--(-8.764,3.014)--cycle;
\draw(-8.81,3.029)--(-8.792,3.07);
\draw(-8.746,3.054)--(-8.764,3.014);
\filldraw[fill opacity=0.8,fill=gray!20,draw=none](-8.792,3.07)--(-8.789,3.079)--(-8.755,3.059)--cycle;
\draw(-8.792,3.07)--(-8.789,3.079);
\filldraw[fill opacity=0.8,fill=gray!20,draw=none](-8.843,3.051)--(-8.862,3.065)--(-8.84,3.079)--(-8.822,3.049)--cycle;
\draw(-8.862,3.065)--(-8.84,3.079)--(-8.822,3.049);
\filldraw[fill opacity=0.8,fill=gray!20,draw=none](-8.813,3.029)--(-8.843,3.051)--(-8.822,3.049)--(-8.811,3.03)--cycle;
\draw(-8.822,3.049)--(-8.811,3.03)--(-8.813,3.029);
\filldraw[fill opacity=0.8,fill=gray!20,draw=none](-8.808,3.027)--(-8.811,3.028)--(-8.81,3.029)--cycle;
\draw(-8.811,3.028)--(-8.81,3.029);
\filldraw[fill opacity=0.8,fill=gray!20,draw=none](-8.855,3.024)--(-8.847,3.042)--(-8.813,3.03)--(-8.811,3.028)--cycle;
\draw(-8.855,3.024)--(-8.847,3.042);
\filldraw[fill opacity=0.8,fill=gray!20,draw=none](-8.813,3.03)--(-8.81,3.029)--(-8.811,3.028)--cycle;
\draw(-8.81,3.029)--(-8.811,3.028);
\filldraw[fill opacity=0.8,fill=gray!20](-8.791,2.979)--(-8.837,3.014)--(-8.811,3.03)--(-8.773,2.991)--cycle;
\filldraw[fill opacity=0.8,fill=gray!20,draw=none](-8.813,3.03)--(-8.843,3.051)--(-8.832,3.077)--(-8.805,3.078)--(-8.793,3.069)--(-8.81,3.029)--cycle;
\draw(-8.843,3.051)--(-8.832,3.077);
\draw(-8.793,3.069)--(-8.81,3.029);
\filldraw[fill opacity=0.8,fill=gray!20,draw=none](-8.805,3.078)--(-8.789,3.079)--(-8.793,3.069)--cycle;
\draw(-8.789,3.079)--(-8.793,3.069);
\filldraw[fill opacity=0.8,fill=gray!20,draw=none](-8.751,3.06)--(-8.79,3.053)--(-8.814,3.093)--(-8.78,3.1)--cycle;
\draw(-8.751,3.06)--(-8.79,3.053)--(-8.814,3.093)--(-8.78,3.1);
\filldraw[fill opacity=0.8,fill=gray!20](-8.752,3.284)--(-8.739,3.315)--(-8.674,3.318)--(-8.672,3.287)--cycle;
\filldraw[fill opacity=0.8,fill=gray!20](-8.672,3.287)--(-8.674,3.318)--(-8.61,3.313)--(-8.594,3.281)--cycle;
\filldraw[fill opacity=0.8,fill=gray!20,draw=none](-8.745,3.056)--(-8.735,3.053)--(-8.732,3.047)--cycle;
\draw(-8.735,3.053)--(-8.732,3.047);
\filldraw[fill opacity=0.5,fill=gray!20](-7.369,1.65)--(-7.396,1.691)--(-7.333,1.18)--(-7.308,1.149)--cycle;
\filldraw[fill opacity=0.8,fill=gray!20](-8.584,3.242)--(-8.594,3.281)--(-8.54,3.268)--(-8.524,3.227)--cycle;
\filldraw[fill opacity=0.8,fill=gray!20](-8.314,4.209)--(-8.352,4.249)--(-8.292,4.26)--(-8.272,4.217)--cycle;
\filldraw[fill opacity=0.8,fill=gray!20,draw=none](-8.741,2.997)--(-8.764,3.014)--(-8.696,3.17)--(-8.649,3.149)--(-8.718,2.989)--cycle;
\draw(-8.764,3.014)--(-8.696,3.17)--(-8.649,3.149)--(-8.718,2.989);
\filldraw[fill opacity=0.8,fill=gray!20,draw=none](-8.755,3.059)--(-8.745,3.056)--(-8.746,3.054)--cycle;
\draw(-8.745,3.056)--(-8.746,3.054);
\filldraw[fill opacity=0.8,fill=gray!20,draw=none](-8.758,3.02)--(-8.79,3.053)--(-8.751,3.06)--(-8.732,3.047)--(-8.723,3.027)--cycle;
\draw(-8.732,3.047)--(-8.723,3.027)--(-8.758,3.02)--(-8.79,3.053)--(-8.751,3.06);
\filldraw[fill opacity=0.8,fill=gray!20,draw=none](-8.755,3.059)--(-8.789,3.079)--(-8.741,3.189)--(-8.696,3.17)--(-8.745,3.056)--cycle;
\draw(-8.789,3.079)--(-8.741,3.189)--(-8.696,3.17)--(-8.745,3.056);
\filldraw[fill opacity=0.8,fill=gray!20](-8.587,3.122)--(-8.565,3.113)--(-8.562,3.111)--(-8.576,3.117)--(-8.607,3.131)--(-8.649,3.149)--(-8.696,3.17)--(-8.741,3.189)--(-8.776,3.205)--(-8.798,3.214)--(-8.802,3.216)--(-8.787,3.209)--(-8.757,3.196)--(-8.715,3.178)--(-8.668,3.157)--(-8.623,3.138)--cycle;
\filldraw[fill opacity=0.8,fill=gray!20,draw=none](-8.696,2.974)--(-8.718,2.989)--(-8.718,2.99)--(-8.677,2.97)--cycle;
\draw(-8.718,2.989)--(-8.718,2.99);
\filldraw[fill opacity=0.8,fill=gray!20,draw=none](-8.707,2.967)--(-8.725,2.992)--(-8.677,2.984)--(-8.678,2.969)--cycle;
\draw(-8.677,2.984)--(-8.678,2.969)--(-8.707,2.967)--(-8.725,2.992);
\filldraw[fill opacity=0.8,fill=gray!20,draw=none](-8.702,3.027)--(-8.699,3.033)--(-8.689,3.028)--(-8.685,3.023)--cycle;
\draw(-8.702,3.027)--(-8.699,3.033);
\filldraw[fill opacity=0.8,fill=gray!20,draw=none](-8.699,3.033)--(-8.698,3.037)--(-8.689,3.028)--cycle;
\draw(-8.699,3.033)--(-8.698,3.037);
\filldraw[fill opacity=0.8,fill=gray!20,draw=none](-8.723,3.027)--(-8.732,3.047)--(-8.676,3.03)--(-8.676,3.029)--cycle;
\draw(-8.676,3.03)--(-8.676,3.029)--(-8.723,3.027)--(-8.732,3.047);
\filldraw[fill opacity=0.8,fill=gray!20,draw=none](-8.718,2.99)--(-8.702,3.027)--(-8.685,3.023)--(-8.663,3.001)--(-8.677,2.97)--cycle;
\draw(-8.718,2.99)--(-8.702,3.027);
\draw(-8.663,3.001)--(-8.677,2.97);
\filldraw[fill opacity=0.8,fill=gray!20,draw=none](-8.656,3.018)--(-8.677,3.022)--(-8.676,3.029)--(-8.668,3.028)--cycle;
\draw(-8.677,3.022)--(-8.676,3.029)--(-8.668,3.028);
\filldraw[fill opacity=0.8,fill=gray!20,draw=none](-8.689,3.028)--(-8.676,3.029)--(-8.677,3.022)--cycle;
\draw(-8.689,3.028)--(-8.676,3.029)--(-8.677,3.022);
\filldraw[fill opacity=0.8,fill=gray!20,draw=none](-8.656,3.018)--(-8.668,3.028)--(-8.631,3.025)--(-8.641,3.015)--cycle;
\draw(-8.668,3.028)--(-8.631,3.025)--(-8.641,3.015);
\filldraw[fill opacity=0.8,fill=gray!20,draw=none](-8.698,3.037)--(-8.649,3.149)--(-8.607,3.131)--(-8.663,3.001)--cycle;
\draw(-8.698,3.037)--(-8.649,3.149)--(-8.607,3.131)--(-8.663,3.001);
\filldraw[fill opacity=0.8,fill=gray!20](-8.871,3.059)--(-8.893,3.111)--(-8.859,3.133)--(-8.84,3.079)--cycle;
\filldraw[fill opacity=0.8,fill=gray!20](-8.893,3.111)--(-8.901,3.166)--(-8.865,3.19)--(-8.859,3.133)--cycle;
\filldraw[fill opacity=0.8,fill=gray!20](-8.307,4.526)--(-8.292,4.563)--(-8.213,4.567)--(-8.211,4.53)--cycle;
\filldraw[fill opacity=0.8,fill=gray!20](-8.162,4.216)--(-8.137,4.258)--(-8.084,4.245)--(-8.124,4.207)--cycle;
\filldraw[fill opacity=0.8,fill=gray!20,draw=none](-8.703,3)--(-8.723,3.027)--(-8.689,3.028)--(-8.677,3.022)--(-8.679,3.001)--cycle;
\draw(-8.677,3.022)--(-8.679,3.001)--(-8.703,3)--(-8.723,3.027)--(-8.689,3.028);
\filldraw[fill opacity=0.8,fill=gray!20](-8.814,3.272)--(-8.79,3.305)--(-8.739,3.315)--(-8.752,3.284)--cycle;
\filldraw[fill opacity=0.8,fill=gray!20](-8.901,3.166)--(-8.893,3.222)--(-8.859,3.244)--(-8.865,3.19)--cycle;
\filldraw[fill opacity=0.8,fill=gray!20](-8.751,3.345)--(-8.731,3.37)--(-8.675,3.373)--(-8.672,3.349)--cycle;
\filldraw[fill opacity=0.5,fill=gray!20](-7.579,2.153)--(-7.632,2.189)--(-7.447,1.723)--(-7.396,1.691)--cycle;
\filldraw[fill opacity=0.8,fill=gray!20,draw=none](-8.816,3.092)--(-8.824,3.095)--(-8.847,3.127)--(-8.829,3.138)--(-8.814,3.093)--cycle;
\draw(-8.847,3.127)--(-8.829,3.138)--(-8.814,3.093)--(-8.816,3.092);
\filldraw[fill opacity=0.8,fill=gray!20,draw=none](-8.82,3.089)--(-8.814,3.093)--(-8.805,3.078)--cycle;
\draw(-8.82,3.089)--(-8.814,3.093)--(-8.805,3.078);
\filldraw[fill opacity=0.8,fill=gray!20,draw=none](-8.816,3.092)--(-8.82,3.089)--(-8.824,3.095)--cycle;
\draw(-8.816,3.092)--(-8.82,3.089);
\filldraw[fill opacity=0.8,fill=gray!20,draw=none](-8.832,3.077)--(-8.776,3.205)--(-8.741,3.189)--(-8.789,3.079)--cycle;
\draw(-8.832,3.077)--(-8.776,3.205)--(-8.741,3.189)--(-8.789,3.079);
\filldraw[fill opacity=0.8,fill=gray!20](-8.721,2.997)--(-8.758,3.02)--(-8.723,3.027)--(-8.703,3)--cycle;
\filldraw[fill opacity=0.8,fill=gray!20,draw=none](-8.68,2.963)--(-8.678,2.967)--(-8.671,2.965)--(-8.666,2.961)--cycle;
\draw(-8.68,2.963)--(-8.678,2.967);
\filldraw[fill opacity=0.8,fill=gray!20,draw=none](-8.678,2.967)--(-8.677,2.969)--(-8.675,2.968)--(-8.671,2.965)--cycle;
\draw(-8.678,2.967)--(-8.677,2.969);
\filldraw[fill opacity=0.8,fill=gray!20,draw=none](-8.675,2.968)--(-8.65,2.967)--(-8.659,2.962)--cycle;
\draw(-8.675,2.968)--(-8.65,2.967)--(-8.659,2.962);
\filldraw[fill opacity=0.8,fill=gray!20,draw=none](-8.659,2.962)--(-8.65,2.967)--(-8.631,2.962)--(-8.646,2.958)--cycle;
\draw(-8.659,2.962)--(-8.65,2.967)--(-8.631,2.962)--(-8.646,2.958);
\filldraw[fill opacity=0.8,fill=gray!20,draw=none](-8.659,2.962)--(-8.675,2.968)--(-8.677,2.97)--(-8.659,3.01)--(-8.64,2.97)--(-8.646,2.958)--cycle;
\draw(-8.677,2.97)--(-8.659,3.01);
\draw(-8.64,2.97)--(-8.646,2.958);
\filldraw[fill opacity=0.8,fill=gray!20,draw=none](-8.654,3)--(-8.659,3.01)--(-8.656,3.018)--(-8.638,3.002)--cycle;
\draw(-8.659,3.01)--(-8.656,3.018);
\filldraw[fill opacity=0.8,fill=gray!20,draw=none](-8.679,3.001)--(-8.677,3.022)--(-8.641,3.015)--(-8.656,2.999)--cycle;
\draw(-8.641,3.015)--(-8.656,2.999)--(-8.679,3.001)--(-8.677,3.022);
\filldraw[fill opacity=0.8,fill=gray!20,draw=none](-8.847,3.042)--(-8.843,3.051)--(-8.813,3.03)--cycle;
\draw(-8.847,3.042)--(-8.843,3.051);
\filldraw[fill opacity=0.8,fill=gray!20,draw=none](-8.862,3.045)--(-8.854,3.052)--(-8.843,3.051)--(-8.852,3.031)--cycle;
\draw(-8.843,3.051)--(-8.852,3.031);
\filldraw[fill opacity=0.8,fill=gray!20,draw=none](-8.854,3.052)--(-8.862,3.045)--(-8.868,3.054)--cycle;
\filldraw[fill opacity=0.8,fill=gray!20,draw=none](-8.813,3.029)--(-8.837,3.014)--(-8.871,3.059)--(-8.862,3.065)--cycle;
\draw(-8.813,3.029)--(-8.837,3.014)--(-8.871,3.059)--(-8.862,3.065);
\filldraw[fill opacity=0.8,fill=gray!20](-8.219,4.187)--(-8.216,4.22)--(-8.162,4.216)--(-8.191,4.185)--cycle;
\filldraw[fill opacity=0.8,fill=gray!20](-8.248,4.186)--(-8.272,4.217)--(-8.216,4.22)--(-8.219,4.187)--cycle;
\filldraw[fill opacity=0.8,fill=gray!20,draw=none](-8.646,3.009)--(-8.631,3.025)--(-8.6,3.018)--(-8.626,3.003)--cycle;
\draw(-8.646,3.009)--(-8.631,3.025)--(-8.6,3.018)--(-8.626,3.003);
\filldraw[fill opacity=0.8,fill=gray!20](-8.594,3.281)--(-8.61,3.313)--(-8.566,3.302)--(-8.54,3.268)--cycle;
\filldraw[fill opacity=0.8,fill=gray!20](-8.543,3.027)--(-8.512,3.075)--(-8.492,3.054)--(-8.527,3.01)--cycle;
\filldraw[fill opacity=0.8,fill=gray!20](-8.512,3.075)--(-8.492,3.129)--(-8.47,3.105)--(-8.492,3.054)--cycle;
\filldraw[fill opacity=0.8,fill=gray!20](-8.858,3.119)--(-8.864,3.166)--(-8.834,3.185)--(-8.829,3.138)--cycle;
\filldraw[fill opacity=0.8,fill=gray!20](-8.492,3.129)--(-8.485,3.185)--(-8.462,3.16)--(-8.47,3.105)--cycle;
\filldraw[fill opacity=0.8,fill=gray!20](-8.381,4.511)--(-8.352,4.552)--(-8.292,4.563)--(-8.307,4.526)--cycle;
\filldraw[fill opacity=0.8,fill=gray!20,draw=none](-8.656,3.018)--(-8.607,3.131)--(-8.576,3.117)--(-8.63,2.995)--cycle;
\draw(-8.656,3.018)--(-8.607,3.131)--(-8.576,3.117)--(-8.63,2.995);
\filldraw[fill opacity=0.8,fill=gray!20,draw=none](-8.854,3.052)--(-8.868,3.054)--(-8.846,3.103)--(-8.824,3.095)--(-8.835,3.07)--cycle;
\draw(-8.868,3.054)--(-8.846,3.103);
\draw(-8.824,3.095)--(-8.835,3.07);
\filldraw[fill opacity=0.8,fill=gray!20,draw=none](-8.811,3.039)--(-8.84,3.076)--(-8.82,3.089)--(-8.805,3.078)--(-8.79,3.053)--cycle;
\draw(-8.805,3.078)--(-8.79,3.053)--(-8.811,3.039)--(-8.84,3.076)--(-8.82,3.089);
\filldraw[fill opacity=0.8,fill=gray!20](-8.893,3.222)--(-8.871,3.273)--(-8.84,3.293)--(-8.859,3.244)--cycle;
\filldraw[fill opacity=0.8,fill=gray!20](-8.811,3.334)--(-8.773,3.362)--(-8.731,3.37)--(-8.751,3.345)--cycle;
\filldraw[fill opacity=0.8,fill=gray!20,draw=none](-8.671,2.965)--(-8.675,2.968)--(-8.659,2.962)--cycle;
\filldraw[fill opacity=0.8,fill=gray!20,draw=none](-8.666,2.961)--(-8.671,2.965)--(-8.659,2.962)--(-8.652,2.959)--cycle;
\filldraw[fill opacity=0.8,fill=gray!20,draw=none](-8.666,2.961)--(-8.652,2.959)--(-8.646,2.957)--(-8.65,2.947)--cycle;
\draw(-8.646,2.957)--(-8.65,2.947);
\filldraw[fill opacity=0.8,fill=gray!20,draw=none](-8.682,2.949)--(-8.678,2.969)--(-8.675,2.968)--(-8.659,2.962)--(-8.682,2.949)--cycle;
\draw(-8.659,2.962)--(-8.682,2.949)--(-8.682,2.949)--(-8.678,2.969)--(-8.675,2.968);
\filldraw[fill opacity=0.8,fill=gray!20](-8.682,2.949)--(-8.707,2.967)--(-8.678,2.969)--(-8.682,2.949)--cycle;
\filldraw[fill opacity=0.8,fill=gray!20](-8.583,2.988)--(-8.543,3.027)--(-8.527,3.01)--(-8.572,2.976)--cycle;
\filldraw[fill opacity=0.8,fill=gray!20](-8.738,2.957)--(-8.791,2.979)--(-8.773,2.991)--(-8.729,2.963)--cycle;
\filldraw[fill opacity=0.8,fill=gray!20](-8.27,4.181)--(-8.314,4.209)--(-8.272,4.217)--(-8.248,4.186)--cycle;
\filldraw[fill opacity=0.5,fill=gray!20](-8.541,2.825)--(-8.631,2.768)--(-8.286,2.518)--(-8.181,2.563)--cycle;
\filldraw[fill opacity=0.5,fill=gray!20](-8.456,2.859)--(-8.541,2.825)--(-8.181,2.563)--(-8.083,2.588)--cycle;
\filldraw[fill opacity=0.5,fill=gray!20](-8.169,2.567)--(-8.631,2.768)--(-8.287,2.518)--(-7.825,2.317)--cycle;
\filldraw[fill opacity=0.5,fill=gray!20](-7.369,.62)--(-7.369,.635)--(-7.549,.142)--(-7.543,.143)--cycle;
\filldraw[fill opacity=0.5,fill=gray!20](-7.543,.143)--(-7.549,.142)--(-7.835,-.296)--(-7.819,-.28)--cycle;
\filldraw[fill opacity=0.8,fill=gray!20](-8.864,3.166)--(-8.858,3.212)--(-8.829,3.231)--(-8.834,3.185)--cycle;
\filldraw[fill opacity=0.8,fill=gray!20](-8.773,3.01)--(-8.811,3.039)--(-8.79,3.053)--(-8.758,3.02)--cycle;
\filldraw[fill opacity=0.8,fill=gray!20](-8.739,3.315)--(-8.723,3.336)--(-8.676,3.338)--(-8.674,3.318)--cycle;
\filldraw[fill opacity=0.8,fill=gray!20](-8.674,3.318)--(-8.676,3.338)--(-8.631,3.335)--(-8.61,3.313)--cycle;
\filldraw[fill opacity=0.8,fill=gray!20](-8.682,2.949)--(-8.729,2.963)--(-8.707,2.967)--(-8.682,2.949)--cycle;
\filldraw[fill opacity=0.8,fill=gray!20](-8.485,3.185)--(-8.492,3.239)--(-8.47,3.216)--(-8.462,3.16)--cycle;
\filldraw[fill opacity=0.8,fill=gray!20](-8.191,4.185)--(-8.162,4.216)--(-8.124,4.207)--(-8.172,4.18)--cycle;
\filldraw[fill opacity=0.8,fill=gray!20,draw=none](-8.654,3)--(-8.638,3.002)--(-8.63,2.995)--(-8.64,2.97)--cycle;
\draw(-8.63,2.995)--(-8.64,2.97);
\filldraw[fill opacity=0.8,fill=gray!20](-8.631,2.962)--(-8.583,2.988)--(-8.572,2.976)--(-8.625,2.956)--cycle;
\filldraw[fill opacity=0.8,fill=gray!20](-8.412,4.277)--(-8.434,4.329)--(-8.399,4.351)--(-8.381,4.297)--cycle;
\filldraw[fill opacity=0.8,fill=gray!20](-8.434,4.329)--(-8.442,4.384)--(-8.406,4.408)--(-8.399,4.351)--cycle;
\filldraw[fill opacity=0.8,fill=gray!20,draw=none](-8.656,2.999)--(-8.646,3.009)--(-8.626,3.003)--(-8.639,2.995)--cycle;
\draw(-8.626,3.003)--(-8.639,2.995)--(-8.656,2.999)--(-8.646,3.009);
\filldraw[fill opacity=0.8,fill=gray!20](-8.54,3.09)--(-8.524,3.134)--(-8.505,3.115)--(-8.523,3.072)--cycle;
\filldraw[fill opacity=0.8,fill=gray!20](-8.566,3.05)--(-8.54,3.09)--(-8.523,3.072)--(-8.552,3.035)--cycle;
\filldraw[fill opacity=0.8,fill=gray!20](-8.377,4.232)--(-8.412,4.277)--(-8.381,4.297)--(-8.352,4.249)--cycle;
\filldraw[fill opacity=0.8,fill=gray!20](-8.524,3.134)--(-8.518,3.181)--(-8.499,3.161)--(-8.505,3.115)--cycle;
\filldraw[fill opacity=0.8,fill=gray!20,draw=none](-8.849,3.098)--(-8.846,3.103)--(-8.849,3.096)--cycle;
\draw(-8.846,3.103)--(-8.849,3.096);
\filldraw[fill opacity=0.8,fill=gray!20,draw=none](-8.82,3.089)--(-8.84,3.076)--(-8.858,3.119)--(-8.847,3.127)--cycle;
\draw(-8.82,3.089)--(-8.84,3.076)--(-8.858,3.119)--(-8.847,3.127);
\filldraw[fill opacity=0.8,fill=gray!20](-8.858,3.212)--(-8.84,3.255)--(-8.814,3.272)--(-8.829,3.231)--cycle;
\filldraw[fill opacity=0.8,fill=gray!20](-8.79,3.305)--(-8.758,3.329)--(-8.723,3.336)--(-8.739,3.315)--cycle;
\filldraw[fill opacity=0.8,fill=gray!20](-8.682,2.985)--(-8.703,3)--(-8.679,3.001)--(-8.682,2.985)--cycle;
\filldraw[fill opacity=0.8,fill=gray!20](-8.682,2.985)--(-8.679,3.001)--(-8.656,2.999)--(-8.682,2.985)--cycle;
\filldraw[fill opacity=0.8,fill=gray!20](-8.6,3.018)--(-8.566,3.05)--(-8.552,3.035)--(-8.59,3.007)--cycle;
\filldraw[fill opacity=0.8,fill=gray!20](-8.729,2.991)--(-8.773,3.01)--(-8.758,3.02)--(-8.721,2.997)--cycle;
\filldraw[fill opacity=0.8,fill=gray!20,draw=none](-8.854,3.052)--(-8.835,3.07)--(-8.843,3.051)--cycle;
\draw(-8.835,3.07)--(-8.843,3.051);
\filldraw[fill opacity=0.8,fill=gray!20,draw=none](-8.846,3.103)--(-8.798,3.214)--(-8.776,3.205)--(-8.824,3.095)--cycle;
\draw(-8.846,3.103)--(-8.798,3.214)--(-8.776,3.205)--(-8.824,3.095);
\filldraw[fill opacity=0.8,fill=gray!20](-8.442,4.384)--(-8.434,4.44)--(-8.399,4.462)--(-8.406,4.408)--cycle;
\filldraw[fill opacity=0.5,fill=gray!20](-7.31,1.118)--(-7.308,1.149)--(-7.369,.635)--(-7.369,.62)--cycle;
\filldraw[fill opacity=0.8,fill=gray!20,draw=none](-8.652,2.959)--(-8.646,2.958)--(-8.646,2.957)--cycle;
\draw(-8.646,2.958)--(-8.646,2.957);
\filldraw[fill opacity=0.8,fill=gray!20,draw=none](-8.682,2.949)--(-8.659,2.962)--(-8.646,2.958)--(-8.682,2.949)--cycle;
\draw(-8.646,2.958)--(-8.682,2.949)--(-8.682,2.949)--(-8.659,2.962);
\filldraw[fill opacity=0.8,fill=gray!20,draw=none](-8.652,2.959)--(-8.659,2.962)--(-8.646,2.958)--cycle;
\filldraw[fill opacity=0.8,fill=gray!20](-8.292,4.563)--(-8.272,4.588)--(-8.216,4.591)--(-8.213,4.567)--cycle;
\filldraw[fill opacity=0.8,fill=gray!20](-8.332,4.197)--(-8.377,4.232)--(-8.352,4.249)--(-8.314,4.209)--cycle;
\filldraw[fill opacity=0.8,fill=gray!20](-8.492,3.239)--(-8.512,3.289)--(-8.492,3.268)--(-8.47,3.216)--cycle;
\filldraw[fill opacity=0.8,fill=gray!20](-8.682,2.985)--(-8.721,2.997)--(-8.703,3)--(-8.682,2.985)--cycle;
\filldraw[fill opacity=0.8,fill=gray!20](-8.518,3.181)--(-8.524,3.227)--(-8.505,3.207)--(-8.499,3.161)--cycle;
\filldraw[fill opacity=0.8,fill=gray!20](-8.871,3.273)--(-8.837,3.317)--(-8.811,3.334)--(-8.84,3.293)--cycle;
\filldraw[fill opacity=0.8,fill=gray!20](-8.61,3.313)--(-8.631,3.335)--(-8.6,3.327)--(-8.566,3.302)--cycle;
\filldraw[fill opacity=0.8,fill=gray!20](-8.682,2.985)--(-8.656,2.999)--(-8.639,2.995)--(-8.682,2.985)--cycle;
\filldraw[fill opacity=0.8,fill=gray!20](-8.817,3.704)--(-8.827,3.704)--(-8.606,4.29)--cycle;
\filldraw[fill opacity=0.8,fill=gray!20](-8.682,2.949)--(-8.738,2.957)--(-8.729,2.963)--(-8.682,2.949)--cycle;
\filldraw[fill opacity=0.8,fill=gray!20,draw=none](-8.792,3.693)--(-8.826,3.703)--(-8.827,3.704)--(-8.817,3.704)--(-8.811,3.702)--cycle;
\draw(-8.826,3.703)--(-8.827,3.704)--(-8.817,3.704);
\filldraw[fill opacity=0.8,fill=gray!20,draw=none](-8.626,3.003)--(-8.6,3.018)--(-8.59,3.007)--(-8.592,3.007)--cycle;
\draw(-8.626,3.003)--(-8.6,3.018)--(-8.59,3.007)--(-8.592,3.007);
\filldraw[fill opacity=0.5,fill=gray!20](-8.207,-.648)--(-8.25,-.653)--(-8.693,-.902)--(-8.641,-.892)--cycle;
\filldraw[fill opacity=0.8,fill=gray!20,draw=none](-8.65,2.942)--(-8.65,2.947)--(-8.646,2.957)--(-8.643,2.955)--(-8.637,2.947)--cycle;
\draw(-8.65,2.947)--(-8.646,2.957);
\filldraw[fill opacity=0.8,fill=gray!20,draw=none](-8.646,2.957)--(-8.646,2.958)--(-8.643,2.955)--cycle;
\draw(-8.646,2.957)--(-8.646,2.958);
\filldraw[fill opacity=0.8,fill=gray!20,draw=none](-8.643,2.955)--(-8.646,2.958)--(-8.64,2.954)--cycle;
\filldraw[fill opacity=0.8,fill=gray!20,draw=none](-8.637,2.947)--(-8.643,2.955)--(-8.64,2.954)--(-8.634,2.949)--cycle;
\filldraw[fill opacity=0.8,fill=gray!20](-8.682,2.949)--(-8.631,2.962)--(-8.625,2.956)--(-8.682,2.949)--cycle;
\filldraw[fill opacity=0.8,fill=gray!20](-8.084,4.245)--(-8.052,4.293)--(-8.033,4.272)--(-8.067,4.228)--cycle;
\filldraw[fill opacity=0.8,fill=gray!20](-8.524,3.227)--(-8.54,3.268)--(-8.523,3.251)--(-8.505,3.207)--cycle;
\filldraw[fill opacity=0.8,fill=gray!20](-8.84,3.255)--(-8.811,3.291)--(-8.79,3.305)--(-8.814,3.272)--cycle;
\filldraw[fill opacity=0.8,fill=gray!20](-8.352,4.552)--(-8.314,4.58)--(-8.272,4.588)--(-8.292,4.563)--cycle;
\filldraw[fill opacity=0.8,fill=gray!20](-8.434,4.44)--(-8.412,4.491)--(-8.381,4.511)--(-8.399,4.462)--cycle;
\filldraw[fill opacity=0.8,fill=gray!20](-8.124,4.207)--(-8.084,4.245)--(-8.067,4.228)--(-8.113,4.194)--cycle;
\filldraw[fill opacity=0.8,fill=gray!20](-8.222,4.167)--(-8.248,4.186)--(-8.219,4.187)--(-8.222,4.167)--cycle;
\filldraw[fill opacity=0.8,fill=gray!20](-8.222,4.167)--(-8.219,4.187)--(-8.191,4.185)--(-8.222,4.167)--cycle;
\filldraw[fill opacity=0.8,fill=gray!20,draw=none](-8.64,2.954)--(-8.625,2.956)--(-8.633,2.95)--cycle;
\draw(-8.64,2.954)--(-8.625,2.956)--(-8.633,2.95);
\filldraw[fill opacity=0.8,fill=gray!20,draw=none](-8.632,2.951)--(-8.633,2.95)--(-8.629,2.953)--cycle;
\draw(-8.633,2.95)--(-8.629,2.953);
\filldraw[fill opacity=0.8,fill=gray!20,draw=none](-8.646,2.958)--(-8.626,3.003)--(-8.619,2.979)--(-8.633,2.948)--cycle;
\draw(-8.646,2.958)--(-8.626,3.003);
\draw(-8.619,2.979)--(-8.633,2.948);
\filldraw[fill opacity=0.8,fill=gray!20,draw=none](-8.626,3.003)--(-8.592,3.007)--(-8.618,2.997)--cycle;
\draw(-8.592,3.007)--(-8.618,2.997);
\filldraw[fill opacity=0.8,fill=gray!20,draw=none](-8.527,3.01)--(-8.492,3.054)--(-8.516,3.038)--(-8.525,3.03)--(-8.552,2.993)--cycle;
\draw(-8.525,3.03)--(-8.552,2.993)--(-8.527,3.01)--(-8.492,3.054)--(-8.516,3.038);
\filldraw[fill opacity=0.8,fill=gray!20,draw=none](-8.574,3.117)--(-8.561,3.111)--(-8.578,3.073)--(-8.583,3.075)--cycle;
\draw(-8.578,3.073)--(-8.583,3.075);
\filldraw[fill opacity=0.8,fill=gray!20,draw=none](-8.574,3.117)--(-8.561,3.111)--(-8.578,3.073)--(-8.583,3.075)--cycle;
\draw(-8.578,3.073)--(-8.583,3.075);
\filldraw[fill opacity=0.8,fill=gray!20,draw=none](-8.611,2.998)--(-8.618,2.997)--(-8.608,3)--cycle;
\draw(-8.618,2.997)--(-8.608,3);
\filldraw[fill opacity=0.8,fill=gray!20,draw=none](-8.601,2.965)--(-8.572,2.976)--(-8.59,2.965)--(-8.592,2.964)--cycle;
\draw(-8.601,2.965)--(-8.572,2.976)--(-8.59,2.965)--(-8.592,2.964);
\filldraw[fill opacity=0.8,fill=gray!20,draw=none](-8.611,2.998)--(-8.608,3)--(-8.59,3.007)--(-8.602,3)--cycle;
\draw(-8.608,3)--(-8.59,3.007)--(-8.602,3);
\filldraw[fill opacity=0.8,fill=gray!20,draw=none](-8.581,3.068)--(-8.583,3.075)--(-8.578,3.073)--cycle;
\draw(-8.583,3.075)--(-8.578,3.073);
\filldraw[fill opacity=0.8,fill=gray!20,draw=none](-8.579,3.016)--(-8.552,3.035)--(-8.574,3.022)--(-8.588,3.01)--cycle;
\draw(-8.579,3.016)--(-8.552,3.035)--(-8.574,3.022)--(-8.588,3.01);
\filldraw[fill opacity=0.8,fill=gray!20,draw=none](-8.581,3.068)--(-8.583,3.075)--(-8.578,3.073)--cycle;
\draw(-8.583,3.075)--(-8.578,3.073);
\filldraw[fill opacity=0.8,fill=gray!20,draw=none](-8.626,3.003)--(-8.576,3.117)--(-8.562,3.111)--(-8.619,2.979)--cycle;
\draw(-8.626,3.003)--(-8.576,3.117)--(-8.562,3.111)--(-8.619,2.979);
\filldraw[fill opacity=0.8,fill=gray!20,draw=none](-8.639,2.995)--(-8.626,3.003)--(-8.618,2.997)--(-8.634,2.99)--cycle;
\draw(-8.618,2.997)--(-8.634,2.99)--(-8.639,2.995)--(-8.626,3.003);
\filldraw[fill opacity=0.8,fill=gray!20,draw=none](-8.786,3.209)--(-8.765,3.245)--(-8.727,3.277)--(-8.682,3.292)--(-8.637,3.287)--(-8.598,3.263)--(-8.573,3.224)--(-8.564,3.176)--(-8.573,3.126)--(-8.577,3.118)--cycle;
\draw(-8.786,3.209)--(-8.765,3.245)--(-8.727,3.277)--(-8.682,3.292)--(-8.637,3.287)--(-8.598,3.263)--(-8.573,3.224)--(-8.564,3.176)--(-8.573,3.126)--(-8.577,3.118);
\filldraw[fill opacity=0.8,fill=gray!20,draw=none](-8.505,3.067)--(-8.516,3.038)--(-8.492,3.054)--(-8.47,3.105)--(-8.486,3.095)--cycle;
\draw(-8.516,3.038)--(-8.492,3.054)--(-8.47,3.105)--(-8.486,3.095);
\filldraw[fill opacity=0.8,fill=gray!20,draw=none](-8.52,3.103)--(-8.534,3.065)--(-8.523,3.072)--(-8.505,3.115)--(-8.518,3.106)--cycle;
\draw(-8.534,3.065)--(-8.523,3.072)--(-8.505,3.115)--(-8.518,3.106);
\filldraw[fill opacity=0.8,fill=gray!20,draw=none](-8.511,3.129)--(-8.511,3.111)--(-8.505,3.115)--(-8.499,3.161)--(-8.5,3.16)--cycle;
\draw(-8.511,3.111)--(-8.505,3.115)--(-8.499,3.161)--(-8.5,3.16);
\filldraw[fill opacity=0.8,fill=gray!20,draw=none](-8.488,3.119)--(-8.488,3.093)--(-8.47,3.105)--(-8.462,3.16)--(-8.478,3.15)--cycle;
\draw(-8.488,3.093)--(-8.47,3.105)--(-8.462,3.16)--(-8.478,3.15);
\filldraw[fill opacity=0.8,fill=gray!20,draw=none](-8.511,3.129)--(-8.5,3.16)--(-8.511,3.153)--cycle;
\draw(-8.5,3.16)--(-8.511,3.153);
\filldraw[fill opacity=0.8,fill=gray!20,draw=none](-8.511,3.129)--(-8.518,3.106)--(-8.511,3.111)--cycle;
\draw(-8.518,3.106)--(-8.511,3.111);
\filldraw[fill opacity=0.8,fill=gray!20,draw=none](-8.519,3.106)--(-8.518,3.106)--(-8.511,3.129)--(-8.511,3.153)--cycle;
\draw(-8.519,3.106)--(-8.518,3.106);
\filldraw[fill opacity=0.8,fill=gray!20,draw=none](-8.52,3.103)--(-8.518,3.106)--(-8.519,3.106)--cycle;
\draw(-8.518,3.106)--(-8.519,3.106);
\filldraw[fill opacity=0.8,fill=gray!20,draw=none](-8.505,3.067)--(-8.486,3.095)--(-8.498,3.087)--cycle;
\draw(-8.486,3.095)--(-8.498,3.087);
\filldraw[fill opacity=0.8,fill=gray!20,draw=none](-8.488,3.119)--(-8.497,3.093)--(-8.498,3.087)--(-8.488,3.093)--cycle;
\draw(-8.498,3.087)--(-8.488,3.093);
\filldraw[fill opacity=0.8,fill=gray!20,draw=none](-8.488,3.119)--(-8.478,3.15)--(-8.489,3.143)--cycle;
\draw(-8.478,3.15)--(-8.489,3.143);
\filldraw[fill opacity=0.8,fill=gray!20,draw=none](-8.497,3.093)--(-8.488,3.119)--(-8.489,3.143)--cycle;
\filldraw[fill opacity=0.8,fill=gray!20,draw=none](-8.311,3.013)--(-8.315,3.015)--(-8.305,3.07)--(-8.292,3.065)--cycle;
\draw(-8.311,3.013)--(-8.315,3.015)--(-8.305,3.07)--(-8.292,3.065);
\filldraw[fill opacity=0.8,fill=gray!20,draw=none](-8.341,2.964)--(-8.344,2.965)--(-8.315,3.015)--(-8.302,3.009)--cycle;
\draw(-8.341,2.964)--(-8.344,2.965)--(-8.315,3.015)--(-8.302,3.009);
\filldraw[fill opacity=0.8,fill=gray!20,draw=none](-8.543,3.059)--(-8.534,3.065)--(-8.52,3.103)--cycle;
\draw(-8.543,3.059)--(-8.534,3.065);
\filldraw[fill opacity=0.8,fill=gray!20,draw=none](-8.505,3.067)--(-8.498,3.087)--(-8.5,3.086)--(-8.513,3.059)--(-8.516,3.053)--cycle;
\draw(-8.498,3.087)--(-8.5,3.086);
\draw(-8.513,3.059)--(-8.516,3.053);
\filldraw[fill opacity=0.8,fill=gray!20,draw=none](-8.497,3.093)--(-8.5,3.086)--(-8.498,3.087)--cycle;
\draw(-8.5,3.086)--(-8.498,3.087);
\filldraw[fill opacity=0.8,fill=gray!20,draw=none](-8.327,3.019)--(-8.309,3.051)--(-8.315,3.015)--(-8.334,2.982)--cycle;
\draw(-8.309,3.051)--(-8.315,3.015)--(-8.334,2.982);
\filldraw[fill opacity=0.8,fill=gray!20,draw=none](-8.353,2.974)--(-8.327,3.019)--(-8.332,2.992)--cycle;
\filldraw[fill opacity=0.8,fill=gray!20,draw=none](-8.333,2.961)--(-8.341,2.964)--(-8.302,3.009)--cycle;
\draw(-8.333,2.961)--(-8.341,2.964);
\filldraw[fill opacity=0.8,fill=gray!20,draw=none](-8.313,3.043)--(-8.318,3.069)--(-8.312,3.106)--(-8.305,3.07)--(-8.309,3.051)--cycle;
\draw(-8.312,3.106)--(-8.305,3.07)--(-8.309,3.051);
\filldraw[fill opacity=0.8,fill=gray!20,draw=none](-8.313,3.043)--(-8.327,3.019)--(-8.318,3.069)--cycle;
\filldraw[fill opacity=0.8,fill=gray!20,draw=none](-8.299,3.026)--(-8.307,3.011)--(-8.311,3.013)--(-8.292,3.065)--cycle;
\draw(-8.307,3.011)--(-8.311,3.013);
\filldraw[fill opacity=0.8,fill=gray!20,draw=none](-8.299,3.026)--(-8.302,3.009)--(-8.307,3.011)--cycle;
\draw(-8.302,3.009)--(-8.307,3.011);
\filldraw[fill opacity=0.8,fill=gray!20,draw=none](-8.288,3.007)--(-8.26,3.04)--(-8.246,3.034)--(-8.273,2.985)--(-8.294,2.994)--cycle;
\draw(-8.26,3.04)--(-8.246,3.034)--(-8.273,2.985)--(-8.294,2.994);
\filldraw[fill opacity=0.8,fill=gray!20,draw=none](-8.265,2.976)--(-8.289,2.936)--(-8.273,2.985)--(-8.259,3.012)--cycle;
\draw(-8.289,2.936)--(-8.273,2.985)--(-8.259,3.012);
\filldraw[fill opacity=0.8,fill=gray!20,draw=none](-8.292,3.008)--(-8.322,3.01)--(-8.321,3.016)--(-8.302,3.038)--cycle;
\draw(-8.292,3.008)--(-8.322,3.01);
\filldraw[fill opacity=0.8,fill=gray!20,draw=none](-8.296,3.002)--(-8.323,2.988)--(-8.322,3.01)--(-8.295,3.008)--cycle;
\draw(-8.322,3.01)--(-8.295,3.008);
\filldraw[fill opacity=0.8,fill=gray!20,draw=none](-8.302,3.038)--(-8.321,3.016)--(-8.312,3.066)--cycle;
\filldraw[fill opacity=0.8,fill=gray!20,draw=none](-8.248,3.004)--(-8.265,2.976)--(-8.259,3.012)--(-8.246,3.034)--(-8.227,3.056)--cycle;
\draw(-8.259,3.012)--(-8.246,3.034)--(-8.227,3.056);
\filldraw[fill opacity=0.8,fill=gray!20,draw=none](-8.236,3.067)--(-8.222,3.076)--(-8.213,3.072)--(-8.246,3.034)--(-8.247,3.035)--cycle;
\draw(-8.222,3.076)--(-8.213,3.072)--(-8.246,3.034)--(-8.247,3.035);
\filldraw[fill opacity=0.8,fill=gray!20,draw=none](-8.263,3.043)--(-8.245,3.061)--(-8.236,3.067)--(-8.247,3.035)--(-8.262,3.041)--cycle;
\draw(-8.247,3.035)--(-8.262,3.041);
\filldraw[fill opacity=0.8,fill=gray!20,draw=none](-8.301,2.978)--(-8.323,2.966)--(-8.323,2.988)--(-8.296,3.002)--cycle;
\filldraw[fill opacity=0.8,fill=gray!20,draw=none](-8.304,2.999)--(-8.309,2.991)--(-8.313,2.993)--(-8.302,3.009)--(-8.3,3.008)--cycle;
\draw(-8.302,3.009)--(-8.3,3.008);
\filldraw[fill opacity=0.8,fill=gray!20,draw=none](-8.283,3.037)--(-8.298,3.027)--(-8.302,3.038)--(-8.288,3.054)--cycle;
\filldraw[fill opacity=0.8,fill=gray!20,draw=none](-8.298,3.029)--(-8.3,3.008)--(-8.302,3.009)--(-8.299,3.026)--cycle;
\draw(-8.3,3.008)--(-8.302,3.009);
\filldraw[fill opacity=0.8,fill=gray!20,draw=none](-8.288,3.007)--(-8.294,2.994)--(-8.297,2.996)--cycle;
\draw(-8.294,2.994)--(-8.297,2.996);
\filldraw[fill opacity=0.8,fill=gray!20,draw=none](-8.288,3.054)--(-8.302,3.038)--(-8.312,3.066)--(-8.292,3.065)--cycle;
\draw(-8.312,3.066)--(-8.292,3.065);
\filldraw[fill opacity=0.8,fill=gray!20,draw=none](-8.296,3.002)--(-8.295,3.008)--(-8.286,3.007)--cycle;
\draw(-8.295,3.008)--(-8.286,3.007);
\filldraw[fill opacity=0.8,fill=gray!20,draw=none](-8.298,3.029)--(-8.299,3.026)--(-8.297,3.038)--cycle;
\filldraw[fill opacity=0.8,fill=gray!20,draw=none](-8.283,3.037)--(-8.279,3.026)--(-8.288,3.007)--(-8.292,3.008)--(-8.298,3.027)--cycle;
\draw(-8.288,3.007)--(-8.292,3.008);
\filldraw[fill opacity=0.8,fill=gray!20,draw=none](-8.353,2.974)--(-8.332,2.992)--(-8.334,2.982)--(-8.344,2.965)--(-8.371,2.942)--cycle;
\draw(-8.334,2.982)--(-8.344,2.965)--(-8.371,2.942);
\filldraw[fill opacity=0.8,fill=gray!20,draw=none](-8.307,2.952)--(-8.286,2.991)--(-8.273,2.985)--(-8.291,2.932)--(-8.309,2.94)--cycle;
\draw(-8.286,2.991)--(-8.273,2.985)--(-8.291,2.932)--(-8.309,2.94);
\filldraw[fill opacity=0.8,fill=gray!20,draw=none](-8.307,2.952)--(-8.301,2.978)--(-8.288,2.991)--(-8.286,2.991)--cycle;
\draw(-8.288,2.991)--(-8.286,2.991);
\filldraw[fill opacity=0.8,fill=gray!20,draw=none](-8.297,2.996)--(-8.288,2.991)--(-8.301,2.978)--cycle;
\draw(-8.297,2.996)--(-8.288,2.991);
\filldraw[fill opacity=0.8,fill=gray!20,draw=none](-8.301,2.978)--(-8.296,3.002)--(-8.286,3.007)--(-8.251,3.005)--cycle;
\draw(-8.286,3.007)--(-8.251,3.005);
\filldraw[fill opacity=0.8,fill=gray!20,draw=none](-8.27,3.045)--(-8.26,3.04)--(-8.288,3.007)--cycle;
\draw(-8.27,3.045)--(-8.26,3.04);
\filldraw[fill opacity=0.8,fill=gray!20,draw=none](-8.263,3.043)--(-8.262,3.041)--(-8.264,3.042)--cycle;
\draw(-8.262,3.041)--(-8.264,3.042);
\filldraw[fill opacity=0.8,fill=gray!20,draw=none](-8.283,3.037)--(-8.288,3.054)--(-8.28,3.064)--(-8.245,3.061)--cycle;
\draw(-8.28,3.064)--(-8.245,3.061);
\filldraw[fill opacity=0.8,fill=gray!20,draw=none](-8.257,3.062)--(-8.28,3.064)--(-8.296,3.083)--(-8.299,3.1)--(-8.295,3.11)--cycle;
\draw(-8.257,3.062)--(-8.28,3.064);
\filldraw[fill opacity=0.8,fill=gray!20,draw=none](-8.281,3.06)--(-8.276,3.069)--(-8.263,3.043)--(-8.264,3.042)--(-8.282,3.05)--cycle;
\draw(-8.264,3.042)--(-8.282,3.05);
\filldraw[fill opacity=0.8,fill=gray!20,draw=none](-8.245,3.061)--(-8.257,3.062)--(-8.266,3.074)--(-8.235,3.077)--(-8.234,3.072)--cycle;
\draw(-8.245,3.061)--(-8.257,3.062);
\draw(-8.235,3.077)--(-8.234,3.072);
\filldraw[fill opacity=0.8,fill=gray!20,draw=none](-8.266,3.074)--(-8.295,3.11)--(-8.292,3.119)--(-8.237,3.115)--(-8.235,3.077)--cycle;
\draw(-8.292,3.119)--(-8.237,3.115)--(-8.235,3.077);
\filldraw[fill opacity=0.8,fill=gray!20,draw=none](-8.245,3.061)--(-8.263,3.043)--(-8.266,3.048)--cycle;
\filldraw[fill opacity=0.8,fill=gray!20,draw=none](-8.236,3.023)--(-8.242,3.019)--(-8.227,3.056)--(-8.213,3.072)--(-8.203,3.078)--cycle;
\draw(-8.227,3.056)--(-8.213,3.072)--(-8.203,3.078);
\filldraw[fill opacity=0.8,fill=gray!20,draw=none](-8.236,3.023)--(-8.248,3.004)--(-8.242,3.019)--cycle;
\filldraw[fill opacity=0.8,fill=gray!20,draw=none](-8.248,3.004)--(-8.272,3.006)--(-8.283,3.037)--(-8.266,3.048)--(-8.239,3.019)--cycle;
\draw(-8.248,3.004)--(-8.272,3.006);
\filldraw[fill opacity=0.8,fill=gray!20,draw=none](-8.298,3.007)--(-8.28,3.033)--(-8.279,3.026)--(-8.288,3.007)--(-8.297,2.996)--(-8.301,2.997)--cycle;
\draw(-8.297,2.996)--(-8.301,2.997);
\filldraw[fill opacity=0.8,fill=gray!20,draw=none](-8.279,3.026)--(-8.272,3.006)--(-8.288,3.007)--cycle;
\draw(-8.272,3.006)--(-8.288,3.007);
\filldraw[fill opacity=0.8,fill=gray!20,draw=none](-8.284,2.946)--(-8.265,2.976)--(-8.27,2.949)--(-8.281,2.921)--cycle;
\filldraw[fill opacity=0.8,fill=gray!20,draw=none](-8.265,2.976)--(-8.248,3.004)--(-8.27,2.949)--cycle;
\filldraw[fill opacity=0.8,fill=gray!20,draw=none](-8.27,2.949)--(-8.307,2.952)--(-8.301,2.978)--(-8.251,3.005)--(-8.248,3.004)--cycle;
\draw(-8.27,2.949)--(-8.307,2.952);
\draw(-8.251,3.005)--(-8.248,3.004);
\filldraw[fill opacity=0.8,fill=gray!20,draw=none](-8.288,3.054)--(-8.292,3.065)--(-8.28,3.064)--cycle;
\draw(-8.292,3.065)--(-8.28,3.064);
\filldraw[fill opacity=0.8,fill=gray!20,draw=none](-8.298,3.007)--(-8.288,3.042)--(-8.283,3.05)--(-8.271,3.045)--cycle;
\draw(-8.283,3.05)--(-8.271,3.045);
\filldraw[fill opacity=0.8,fill=gray!20,draw=none](-8.28,3.033)--(-8.271,3.045)--(-8.27,3.045)--(-8.279,3.026)--cycle;
\draw(-8.271,3.045)--(-8.27,3.045);
\filldraw[fill opacity=0.8,fill=gray!20,draw=none](-8.286,3.052)--(-8.298,3.029)--(-8.297,3.038)--(-8.292,3.065)--(-8.283,3.061)--cycle;
\draw(-8.292,3.065)--(-8.283,3.061);
\filldraw[fill opacity=0.8,fill=gray!20,draw=none](-8.286,3.052)--(-8.298,3.007)--(-8.3,3.008)--(-8.298,3.029)--cycle;
\draw(-8.298,3.007)--(-8.3,3.008);
\filldraw[fill opacity=0.8,fill=gray!20,draw=none](-8.284,2.946)--(-8.281,2.921)--(-8.295,2.886)--(-8.291,2.932)--(-8.289,2.936)--cycle;
\draw(-8.295,2.886)--(-8.291,2.932)--(-8.289,2.936);
\filldraw[fill opacity=0.8,fill=gray!20,draw=none](-8.307,2.952)--(-8.323,2.953)--(-8.323,2.966)--(-8.301,2.978)--cycle;
\draw(-8.307,2.952)--(-8.323,2.953);
\filldraw[fill opacity=0.8,fill=gray!20,draw=none](-8.344,2.955)--(-8.378,2.926)--(-8.386,2.93)--(-8.344,2.965)--(-8.337,2.962)--cycle;
\draw(-8.378,2.926)--(-8.386,2.93)--(-8.344,2.965)--(-8.337,2.962);
\filldraw[fill opacity=0.8,fill=gray!20,draw=none](-8.344,2.955)--(-8.337,2.962)--(-8.336,2.962)--cycle;
\draw(-8.337,2.962)--(-8.336,2.962);
\filldraw[fill opacity=0.8,fill=gray!20,draw=none](-8.323,2.966)--(-8.323,2.953)--(-8.343,2.955)--(-8.343,2.956)--cycle;
\draw(-8.323,2.953)--(-8.343,2.955)--(-8.343,2.956);
\filldraw[fill opacity=0.8,fill=gray!20,draw=none](-8.343,2.951)--(-8.343,2.955)--(-8.322,2.953)--cycle;
\draw(-8.343,2.951)--(-8.343,2.955)--(-8.322,2.953);
\filldraw[fill opacity=0.8,fill=gray!20,draw=none](-8.572,2.976)--(-8.53,3.007)--(-8.534,3.005)--(-8.552,2.993)--(-8.59,2.965)--cycle;
\draw(-8.534,3.005)--(-8.552,2.993)--(-8.59,2.965)--(-8.572,2.976)--(-8.53,3.007);
\filldraw[fill opacity=0.8,fill=gray!20,draw=none](-8.315,2.974)--(-8.308,2.99)--(-8.301,2.997)--(-8.297,2.996)--(-8.301,2.978)--(-8.316,2.963)--cycle;
\draw(-8.301,2.997)--(-8.297,2.996);
\filldraw[fill opacity=0.8,fill=gray!20,draw=none](-8.304,2.999)--(-8.3,3.008)--(-8.298,3.007)--cycle;
\draw(-8.3,3.008)--(-8.298,3.007);
\filldraw[fill opacity=0.8,fill=gray!20,draw=none](-8.31,3.011)--(-8.288,3.042)--(-8.301,2.997)--(-8.313,3.002)--cycle;
\draw(-8.301,2.997)--(-8.313,3.002);
\filldraw[fill opacity=0.8,fill=gray!20,draw=none](-8.561,3.111)--(-8.517,3.092)--(-8.544,3.058)--(-8.578,3.073)--cycle;
\draw(-8.544,3.058)--(-8.578,3.073);
\filldraw[fill opacity=0.8,fill=gray!20,draw=none](-8.561,3.111)--(-8.517,3.092)--(-8.544,3.058)--(-8.578,3.073)--cycle;
\draw(-8.544,3.058)--(-8.578,3.073);
\filldraw[fill opacity=0.8,fill=gray!20,draw=none](-8.559,3.042)--(-8.569,3.035)--(-8.581,3.068)--(-8.578,3.073)--(-8.544,3.058)--cycle;
\draw(-8.578,3.073)--(-8.544,3.058);
\filldraw[fill opacity=0.8,fill=gray!20,draw=none](-8.559,3.042)--(-8.569,3.035)--(-8.581,3.068)--(-8.578,3.073)--(-8.544,3.058)--cycle;
\draw(-8.578,3.073)--(-8.544,3.058);
\filldraw[fill opacity=0.8,fill=gray!20](-8.552,3.035)--(-8.523,3.072)--(-8.549,3.055)--(-8.574,3.022)--cycle;
\filldraw[fill opacity=0.8,fill=gray!20,draw=none](-8.581,3.068)--(-8.599,3.033)--(-8.637,3.049)--(-8.598,3.082)--(-8.583,3.075)--cycle;
\draw(-8.599,3.033)--(-8.637,3.049)--(-8.598,3.082)--(-8.583,3.075);
\filldraw[fill opacity=0.8,fill=gray!20,draw=none](-8.581,3.068)--(-8.599,3.033)--(-8.637,3.049)--(-8.598,3.082)--(-8.583,3.075)--cycle;
\draw(-8.599,3.033)--(-8.637,3.049)--(-8.598,3.082)--(-8.583,3.075);
\filldraw[fill opacity=0.8,fill=gray!20,draw=none](-8.59,3.007)--(-8.579,3.016)--(-8.588,3.01)--(-8.605,2.998)--cycle;
\draw(-8.588,3.01)--(-8.605,2.998)--(-8.59,3.007)--(-8.579,3.016);
\filldraw[fill opacity=0.8,fill=gray!20,draw=none](-8.682,2.949)--(-8.64,2.954)--(-8.633,2.95)--(-8.634,2.95)--(-8.682,2.949)--cycle;
\draw(-8.633,2.95)--(-8.634,2.95)--(-8.682,2.949)--(-8.682,2.949)--(-8.64,2.954);
\filldraw[fill opacity=0.8,fill=gray!20](-8.682,2.949)--(-8.634,2.95)--(-8.656,2.945)--(-8.682,2.949)--cycle;
\filldraw[fill opacity=0.8,fill=gray!20,draw=none](-8.308,2.99)--(-8.315,2.974)--(-8.323,2.977)--(-8.313,2.993)--cycle;
\filldraw[fill opacity=0.8,fill=gray!20,draw=none](-8.304,2.999)--(-8.308,2.99)--(-8.309,2.991)--cycle;
\filldraw[fill opacity=0.8,fill=gray!20,draw=none](-8.313,3.002)--(-8.301,2.997)--(-8.314,2.984)--cycle;
\draw(-8.313,3.002)--(-8.301,2.997);
\filldraw[fill opacity=0.8,fill=gray!20,draw=none](-8.27,2.949)--(-8.248,3.004)--(-8.237,3.004)--(-8.25,2.948)--cycle;
\draw(-8.248,3.004)--(-8.237,3.004)--(-8.25,2.948)--(-8.27,2.949);
\filldraw[fill opacity=0.8,fill=gray!20,draw=none](-8.266,3.048)--(-8.245,3.061)--(-8.233,3.06)--(-8.237,3.016)--cycle;
\draw(-8.245,3.061)--(-8.233,3.06)--(-8.237,3.016);
\filldraw[fill opacity=0.8,fill=gray!20,draw=none](-8.248,3.004)--(-8.239,3.019)--(-8.237,3.016)--(-8.237,3.004)--cycle;
\draw(-8.237,3.016)--(-8.237,3.004)--(-8.248,3.004);
\filldraw[fill opacity=0.8,fill=gray!20,draw=none](-8.237,3.004)--(-8.236,3.023)--(-8.196,3.051)--(-8.158,3.042)--(-8.165,2.986)--cycle;
\draw(-8.196,3.051)--(-8.158,3.042)--(-8.165,2.986)--(-8.237,3.004)--(-8.236,3.023);
\filldraw[fill opacity=0.8,fill=gray!20](-8.25,2.948)--(-8.237,3.004)--(-8.165,2.986)--(-8.184,2.932)--cycle;
\filldraw[fill opacity=0.8,fill=gray!20](-8.269,2.897)--(-8.25,2.948)--(-8.184,2.932)--(-8.215,2.884)--cycle;
\filldraw[fill opacity=0.8,fill=gray!20,draw=none](-8.196,3.051)--(-8.161,3.065)--(-8.158,3.042)--cycle;
\draw(-8.161,3.065)--(-8.158,3.042)--(-8.196,3.051);
\filldraw[fill opacity=0.8,fill=gray!20](-8.184,2.932)--(-8.165,2.986)--(-8.143,2.962)--(-8.164,2.911)--cycle;
\filldraw[fill opacity=0.8,fill=gray!20](-8.215,2.884)--(-8.184,2.932)--(-8.164,2.911)--(-8.199,2.867)--cycle;
\filldraw[fill opacity=0.8,fill=gray!20,draw=none](-8.165,2.986)--(-8.158,3.042)--(-8.146,3.03)--(-8.14,2.981)--(-8.143,2.962)--cycle;
\draw(-8.14,2.981)--(-8.143,2.962)--(-8.165,2.986)--(-8.158,3.042)--(-8.146,3.03);
\filldraw[fill opacity=0.8,fill=gray!20,draw=none](-8.146,3.03)--(-8.158,3.042)--(-8.159,3.051)--(-8.155,3.05)--cycle;
\draw(-8.146,3.03)--(-8.158,3.042)--(-8.159,3.051);
\filldraw[fill opacity=0.8,fill=gray!20,draw=none](-8.164,3.074)--(-8.172,3.091)--(-8.149,3.091)--(-8.146,3.087)--cycle;
\draw(-8.172,3.091)--(-8.149,3.091)--(-8.146,3.087);
\filldraw[fill opacity=0.8,fill=gray!20,draw=none](-8.165,3.097)--(-8.149,3.091)--(-8.166,3.091)--cycle;
\draw(-8.165,3.097)--(-8.149,3.091)--(-8.166,3.091);
\filldraw[fill opacity=0.8,fill=gray!20,draw=none](-8.135,3.001)--(-8.164,3.074)--(-8.146,3.087)--(-8.129,3.07)--(-8.123,3.043)--cycle;
\draw(-8.146,3.087)--(-8.129,3.07)--(-8.123,3.043);
\filldraw[fill opacity=0.8,fill=gray!20,draw=none](-8.142,3.048)--(-8.159,3.051)--(-8.165,3.097)--(-8.163,3.095)--(-8.152,3.08)--cycle;
\draw(-8.159,3.051)--(-8.165,3.097)--(-8.163,3.095);
\filldraw[fill opacity=0.8,fill=gray!20,draw=none](-8.146,2.96)--(-8.167,2.915)--(-8.266,2.958)--(-8.248,3.004)--(-8.236,3.023)--(-8.196,3.051)--(-8.175,3.059)--(-8.155,3.05)--(-8.146,3.03)--(-8.14,2.981)--cycle;
\filldraw[fill opacity=0.8,fill=gray!20,draw=none](-8.173,2.941)--(-8.178,2.934)--(-8.308,2.99)--(-8.304,2.999)--(-8.298,3.007)--(-8.169,2.951)--cycle;
\draw(-8.298,3.007)--(-8.169,2.951);
\filldraw[fill opacity=0.8,fill=gray!20,draw=none](-8.164,3.074)--(-8.236,3.023)--(-8.203,3.078)--(-8.179,3.091)--(-8.172,3.091)--cycle;
\draw(-8.203,3.078)--(-8.179,3.091)--(-8.172,3.091);
\filldraw[fill opacity=0.8,fill=gray!20,draw=none](-8.252,3.057)--(-8.266,3.048)--(-8.276,3.069)--(-8.267,3.086)--cycle;
\filldraw[fill opacity=0.8,fill=gray!20,draw=none](-8.297,3.057)--(-8.283,3.05)--(-8.31,3.011)--cycle;
\draw(-8.297,3.057)--(-8.283,3.05);
\filldraw[fill opacity=0.8,fill=gray!20,draw=none](-8.296,3.083)--(-8.28,3.064)--(-8.292,3.065)--cycle;
\draw(-8.28,3.064)--(-8.292,3.065);
\filldraw[fill opacity=0.8,fill=gray!20,draw=none](-8.235,3.077)--(-8.237,3.115)--(-8.198,3.105)--cycle;
\draw(-8.235,3.077)--(-8.237,3.115)--(-8.198,3.105);
\filldraw[fill opacity=0.8,fill=gray!20,draw=none](-8.282,3.093)--(-8.279,3.091)--(-8.281,3.06)--(-8.292,3.065)--(-8.293,3.07)--cycle;
\draw(-8.281,3.06)--(-8.292,3.065);
\filldraw[fill opacity=0.8,fill=gray!20,draw=none](-8.286,3.052)--(-8.283,3.061)--(-8.281,3.06)--cycle;
\draw(-8.283,3.061)--(-8.281,3.06);
\filldraw[fill opacity=0.8,fill=gray!20,draw=none](-8.296,3.059)--(-8.283,3.083)--(-8.28,3.076)--(-8.282,3.05)--(-8.296,3.056)--cycle;
\draw(-8.282,3.05)--(-8.296,3.056);
\filldraw[fill opacity=0.8,fill=gray!20,draw=none](-8.24,2.982)--(-8.298,3.007)--(-8.286,3.052)--(-8.281,3.06)--(-8.24,3.042)--cycle;
\draw(-8.24,2.982)--(-8.298,3.007);
\draw(-8.281,3.06)--(-8.24,3.042);
\filldraw[fill opacity=0.8,fill=gray!20,draw=none](-8.296,3.059)--(-8.296,3.056)--(-8.297,3.057)--cycle;
\draw(-8.296,3.056)--(-8.297,3.057);
\filldraw[fill opacity=0.8,fill=gray!20,draw=none](-8.275,3.106)--(-8.274,3.107)--(-8.263,3.109)--(-8.209,3.074)--(-8.213,3.072)--(-8.273,3.098)--cycle;
\draw(-8.209,3.074)--(-8.213,3.072)--(-8.273,3.098);
\filldraw[fill opacity=0.8,fill=gray!20,draw=none](-8.236,3.067)--(-8.234,3.072)--(-8.228,3.078)--(-8.222,3.076)--cycle;
\draw(-8.228,3.078)--(-8.222,3.076);
\filldraw[fill opacity=0.8,fill=gray!20,draw=none](-8.234,3.072)--(-8.231,3.08)--(-8.228,3.078)--cycle;
\draw(-8.231,3.08)--(-8.228,3.078);
\filldraw[fill opacity=0.8,fill=gray!20,draw=none](-8.251,3.116)--(-8.286,3.118)--(-8.313,3.134)--(-8.323,3.148)--(-8.323,3.157)--cycle;
\draw(-8.251,3.116)--(-8.286,3.118);
\filldraw[fill opacity=0.8,fill=gray!20,draw=none](-8.251,3.116)--(-8.323,3.157)--(-8.323,3.168)--(-8.247,3.152)--(-8.237,3.115)--cycle;
\draw(-8.247,3.152)--(-8.237,3.115)--(-8.251,3.116);
\filldraw[fill opacity=0.8,fill=gray!20,draw=none](-8.263,3.109)--(-8.239,3.112)--(-8.193,3.083)--(-8.209,3.074)--cycle;
\draw(-8.193,3.083)--(-8.209,3.074);
\filldraw[fill opacity=0.8,fill=gray!20,draw=none](-8.196,3.051)--(-8.181,3.062)--(-8.175,3.059)--cycle;
\filldraw[fill opacity=0.8,fill=gray!20,draw=none](-8.175,3.059)--(-8.196,3.051)--(-8.233,3.06)--(-8.235,3.077)--(-8.225,3.085)--cycle;
\draw(-8.196,3.051)--(-8.233,3.06)--(-8.235,3.077);
\filldraw[fill opacity=0.8,fill=gray!20,draw=none](-8.234,3.072)--(-8.245,3.061)--(-8.252,3.057)--(-8.267,3.086)--(-8.263,3.093)--(-8.231,3.08)--cycle;
\draw(-8.263,3.093)--(-8.231,3.08);
\filldraw[fill opacity=0.8,fill=gray!20,draw=none](-8.281,3.06)--(-8.28,3.076)--(-8.276,3.069)--cycle;
\filldraw[fill opacity=0.8,fill=gray!20,draw=none](-8.276,3.069)--(-8.28,3.076)--(-8.279,3.091)--(-8.275,3.099)--(-8.263,3.093)--cycle;
\draw(-8.275,3.099)--(-8.263,3.093);
\filldraw[fill opacity=0.8,fill=gray!20,draw=none](-8.279,3.088)--(-8.248,3.058)--(-8.24,3.042)--(-8.281,3.06)--cycle;
\draw(-8.24,3.042)--(-8.281,3.06);
\filldraw[fill opacity=0.8,fill=gray!20,draw=none](-8.279,3.088)--(-8.279,3.091)--(-8.262,3.084)--(-8.248,3.058)--cycle;
\filldraw[fill opacity=0.8,fill=gray!20,draw=none](-8.252,3.064)--(-8.248,3.058)--(-8.258,3.037)--(-8.262,3.044)--(-8.262,3.074)--cycle;
\draw(-8.248,3.058)--(-8.258,3.037);
\filldraw[fill opacity=0.8,fill=gray!20,draw=none](-8.258,3.037)--(-8.262,3.027)--(-8.262,3.044)--cycle;
\draw(-8.258,3.037)--(-8.262,3.027);
\filldraw[fill opacity=0.8,fill=gray!20](-7.976,3.779)--(-7.975,3.835)--(-7.87,3.828)--(-7.882,3.772)--cycle;
\filldraw[fill opacity=0.8,fill=gray!20](-7.978,3.727)--(-7.976,3.779)--(-7.882,3.772)--(-7.902,3.721)--cycle;
\filldraw[fill opacity=0.8,fill=gray!20](-7.981,3.683)--(-7.978,3.727)--(-7.902,3.721)--(-7.927,3.679)--cycle;
\filldraw[fill opacity=0.8,fill=gray!20,draw=none](-8.236,3.023)--(-8.233,3.06)--(-8.196,3.051)--cycle;
\draw(-8.236,3.023)--(-8.233,3.06)--(-8.196,3.051);
\filldraw[fill opacity=0.8,fill=gray!20,draw=none](-8.236,3.067)--(-8.245,3.061)--(-8.234,3.072)--cycle;
\filldraw[fill opacity=0.8,fill=gray!20,draw=none](-8.245,3.061)--(-8.234,3.072)--(-8.233,3.06)--cycle;
\draw(-8.234,3.072)--(-8.233,3.06)--(-8.245,3.061);
\filldraw[fill opacity=0.8,fill=gray!20](-8.072,3.774)--(-8.082,3.83)--(-7.975,3.835)--(-7.976,3.779)--cycle;
\filldraw[fill opacity=0.8,fill=gray!20](-8.057,3.723)--(-8.072,3.774)--(-7.976,3.779)--(-7.978,3.727)--cycle;
\filldraw[fill opacity=0.8,fill=gray!20,draw=none](-8.241,3.075)--(-8.247,3.06)--(-8.252,3.064)--(-8.262,3.084)--(-8.262,3.095)--cycle;
\draw(-8.241,3.075)--(-8.247,3.06);
\filldraw[fill opacity=0.8,fill=gray!20,draw=none](-8.248,3.058)--(-8.226,3.036)--(-8.24,3.042)--cycle;
\draw(-8.226,3.036)--(-8.24,3.042);
\filldraw[fill opacity=0.8,fill=gray!20,draw=none](-8.224,3.03)--(-8.24,3.016)--(-8.24,3.042)--(-8.226,3.036)--cycle;
\draw(-8.24,3.042)--(-8.226,3.036);
\filldraw[fill opacity=0.8,fill=gray!20,draw=none](-8.224,3.03)--(-8.216,3.005)--(-8.227,2.976)--(-8.24,2.982)--(-8.24,3.016)--cycle;
\draw(-8.227,2.976)--(-8.24,2.982);
\filldraw[fill opacity=0.8,fill=gray!20,draw=none](-8.226,3.036)--(-8.224,3.03)--(-8.232,3.012)--(-8.258,3.037)--(-8.248,3.058)--cycle;
\draw(-8.224,3.03)--(-8.232,3.012);
\draw(-8.258,3.037)--(-8.248,3.058);
\filldraw[fill opacity=0.8,fill=gray!20,draw=none](-8.232,3.012)--(-8.246,2.982)--(-8.262,3.027)--(-8.258,3.037)--cycle;
\draw(-8.232,3.012)--(-8.246,2.982);
\draw(-8.262,3.027)--(-8.258,3.037);
\filldraw[fill opacity=0.8,fill=gray!20](-8.245,2.983)--(-8.573,3.126)--(-8.564,3.176)--(-8.236,3.033)--cycle;
\filldraw[fill opacity=0.8,fill=gray!20,draw=none](-8.323,2.988)--(-8.342,2.977)--(-8.342,2.985)--(-8.326,3.01)--(-8.322,3.01)--cycle;
\draw(-8.342,2.977)--(-8.342,2.985);
\draw(-8.326,3.01)--(-8.322,3.01);
\filldraw[fill opacity=0.8,fill=gray!20,draw=none](-8.322,3.01)--(-8.326,3.01)--(-8.321,3.016)--cycle;
\draw(-8.322,3.01)--(-8.326,3.01);
\filldraw[fill opacity=0.8,fill=gray!20,draw=none](-8.31,3.011)--(-8.313,3.002)--(-8.316,3.004)--cycle;
\draw(-8.313,3.002)--(-8.316,3.004);
\filldraw[fill opacity=0.8,fill=gray!20,draw=none](-8.246,2.984)--(-8.246,2.982)--(-8.249,2.973)--(-8.254,2.976)--cycle;
\draw(-8.246,2.982)--(-8.249,2.973)--(-8.254,2.976);
\filldraw[fill opacity=0.8,fill=gray!20,draw=none](-8.25,2.975)--(-8.577,3.118)--(-8.573,3.126)--(-8.245,2.983)--cycle;
\draw(-8.577,3.118)--(-8.573,3.126)--(-8.245,2.983)--(-8.25,2.975);
\filldraw[fill opacity=0.8,fill=gray!20,draw=none](-8.786,3.209)--(-8.765,3.245)--(-8.727,3.277)--(-8.682,3.292)--(-8.637,3.287)--(-8.598,3.263)--(-8.573,3.224)--(-8.564,3.176)--(-8.573,3.126)--(-8.577,3.118)--cycle;
\draw(-8.786,3.209)--(-8.765,3.245)--(-8.727,3.277)--(-8.682,3.292)--(-8.637,3.287)--(-8.598,3.263)--(-8.573,3.224)--(-8.564,3.176)--(-8.573,3.126)--(-8.577,3.118);
\filldraw[fill opacity=0.8,fill=gray!20,draw=none](-8.25,2.975)--(-8.577,3.118)--(-8.573,3.126)--(-8.245,2.983)--cycle;
\draw(-8.577,3.118)--(-8.573,3.126)--(-8.245,2.983)--(-8.25,2.975);
\filldraw[fill opacity=0.8,fill=gray!20](-8.245,2.983)--(-8.573,3.126)--(-8.564,3.176)--(-8.236,3.033)--cycle;
\filldraw[fill opacity=0.8,fill=gray!20](-8.682,2.985)--(-8.729,2.991)--(-8.721,2.997)--(-8.682,2.985)--cycle;
\filldraw[fill opacity=0.8,fill=gray!20,draw=none](-8.574,3.117)--(-8.583,3.075)--(-8.598,3.082)--(-8.577,3.118)--cycle;
\draw(-8.583,3.075)--(-8.598,3.082)--(-8.577,3.118);
\filldraw[fill opacity=0.8,fill=gray!20,draw=none](-8.574,3.117)--(-8.583,3.075)--(-8.598,3.082)--(-8.577,3.118)--cycle;
\draw(-8.583,3.075)--(-8.598,3.082)--(-8.577,3.118);
\filldraw[fill opacity=0.8,fill=gray!20,draw=none](-8.79,3.201)--(-8.786,3.209)--(-8.577,3.118)--(-8.598,3.082)--(-8.637,3.049)--(-8.682,3.035)--(-8.727,3.04)--(-8.765,3.063)--(-8.79,3.102)--(-8.799,3.15)--cycle;
\draw(-8.577,3.118)--(-8.598,3.082)--(-8.637,3.049)--(-8.682,3.035)--(-8.727,3.04)--(-8.765,3.063)--(-8.79,3.102)--(-8.799,3.15)--(-8.79,3.201)--(-8.786,3.209);
\filldraw[fill opacity=0.8,fill=gray!20,draw=none](-8.79,3.201)--(-8.786,3.209)--(-8.577,3.118)--(-8.598,3.082)--(-8.637,3.049)--(-8.682,3.035)--(-8.727,3.04)--(-8.765,3.063)--(-8.79,3.102)--(-8.799,3.15)--cycle;
\draw(-8.577,3.118)--(-8.598,3.082)--(-8.637,3.049)--(-8.682,3.035)--(-8.727,3.04)--(-8.765,3.063)--(-8.79,3.102)--(-8.799,3.15)--(-8.79,3.201)--(-8.786,3.209);
\filldraw[fill opacity=0.8,fill=gray!20](-8.279,4.175)--(-8.332,4.197)--(-8.314,4.209)--(-8.27,4.181)--cycle;
\filldraw[fill opacity=0.8,fill=gray!20,draw=none](-8.776,3.685)--(-8.792,3.693)--(-8.767,3.685)--cycle;
\filldraw[fill opacity=0.8,fill=gray!20,draw=none](-8.058,4.518)--(-8.044,4.498)--(-8.052,4.507)--(-8.084,4.548)--(-8.075,4.539)--cycle;
\draw(-8.044,4.498)--(-8.052,4.507)--(-8.084,4.548)--(-8.075,4.539);
\filldraw[fill opacity=0.8,fill=gray!20,draw=none](-8.041,4.495)--(-8.155,4.441)--(-8.187,4.478)--(-8.07,4.534)--cycle;
\draw(-8.041,4.495)--(-8.155,4.441)--(-8.187,4.478)--(-8.07,4.534);
\filldraw[fill opacity=0.8,fill=gray!20,draw=none](-8.058,4.518)--(-8.041,4.495)--(-8.037,4.491)--(-8.044,4.498)--cycle;
\draw(-8.037,4.491)--(-8.044,4.498);
\filldraw[fill opacity=0.8,fill=gray!20,draw=none](-8.019,4.443)--(-8.03,4.472)--(-8.012,4.437)--(-8.011,4.434)--cycle;
\draw(-8.012,4.437)--(-8.011,4.434)--(-8.019,4.443);
\filldraw[fill opacity=0.8,fill=gray!20,draw=none](-8.012,4.437)--(-8.018,4.449)--(-8.015,4.451)--cycle;
\draw(-8.018,4.449)--(-8.015,4.451);
\filldraw[fill opacity=0.8,fill=gray!20,draw=none](-8.018,4.449)--(-8.03,4.472)--(-8.037,4.491)--(-8.036,4.49)--cycle;
\draw(-8.037,4.491)--(-8.036,4.49);
\filldraw[fill opacity=0.8,fill=gray!20,draw=none](-8.036,4.49)--(-8.037,4.491)--(-8.041,4.495)--cycle;
\draw(-8.036,4.49)--(-8.037,4.491);
\filldraw[fill opacity=0.8,fill=gray!20,draw=none](-8.018,4.449)--(-8.134,4.394)--(-8.155,4.441)--(-8.039,4.496)--cycle;
\draw(-8.018,4.449)--(-8.134,4.394)--(-8.155,4.441)--(-8.039,4.496);
\filldraw[fill opacity=0.8,fill=gray!20,draw=none](-8.011,4.374)--(-8.003,4.379)--(-8.011,4.434)--(-8.011,4.434)--cycle;
\draw(-8.011,4.374)--(-8.003,4.379)--(-8.011,4.434)--(-8.011,4.434);
\filldraw[fill opacity=0.8,fill=gray!20,draw=none](-7.996,4.407)--(-8.125,4.346)--(-8.134,4.394)--(-8.018,4.449)--cycle;
\draw(-7.996,4.407)--(-8.125,4.346)--(-8.134,4.394)--(-8.018,4.449);
\filldraw[fill opacity=0.8,fill=gray!20](-8.682,2.985)--(-8.639,2.995)--(-8.634,2.99)--(-8.682,2.985)--cycle;
\filldraw[fill opacity=0.8,fill=gray!20](-8.222,4.167)--(-8.27,4.181)--(-8.248,4.186)--(-8.222,4.167)--cycle;
\filldraw[fill opacity=0.8,fill=gray!20,draw=none](-8.511,3.182)--(-8.5,3.16)--(-8.499,3.161)--(-8.505,3.207)--(-8.511,3.204)--cycle;
\draw(-8.5,3.16)--(-8.499,3.161)--(-8.505,3.207)--(-8.511,3.204);
\filldraw[fill opacity=0.8,fill=gray!20,draw=none](-8.488,3.173)--(-8.478,3.15)--(-8.462,3.16)--(-8.47,3.216)--(-8.488,3.204)--cycle;
\draw(-8.478,3.15)--(-8.462,3.16)--(-8.47,3.216)--(-8.488,3.204);
\filldraw[fill opacity=0.8,fill=gray!20,draw=none](-8.511,3.182)--(-8.511,3.153)--(-8.5,3.16)--cycle;
\draw(-8.511,3.153)--(-8.5,3.16);
\filldraw[fill opacity=0.8,fill=gray!20,draw=none](-8.529,3.229)--(-8.52,3.201)--(-8.518,3.199)--(-8.505,3.207)--(-8.508,3.215)--cycle;
\draw(-8.518,3.199)--(-8.505,3.207)--(-8.508,3.215);
\filldraw[fill opacity=0.8,fill=gray!20,draw=none](-8.511,3.182)--(-8.511,3.204)--(-8.518,3.199)--cycle;
\draw(-8.511,3.204)--(-8.518,3.199);
\filldraw[fill opacity=0.8,fill=gray!20,draw=none](-8.511,3.153)--(-8.511,3.182)--(-8.518,3.199)--(-8.519,3.198)--cycle;
\draw(-8.518,3.199)--(-8.519,3.198);
\filldraw[fill opacity=0.8,fill=gray!20,draw=none](-8.488,3.173)--(-8.489,3.143)--(-8.478,3.15)--cycle;
\draw(-8.489,3.143)--(-8.478,3.15);
\filldraw[fill opacity=0.8,fill=gray!20,draw=none](-8.52,3.201)--(-8.519,3.198)--(-8.518,3.199)--cycle;
\draw(-8.519,3.198)--(-8.518,3.199);
\filldraw[fill opacity=0.8,fill=gray!20,draw=none](-8.497,3.191)--(-8.488,3.173)--(-8.488,3.204)--(-8.498,3.198)--cycle;
\draw(-8.488,3.204)--(-8.498,3.198);
\filldraw[fill opacity=0.8,fill=gray!20,draw=none](-8.497,3.191)--(-8.489,3.143)--(-8.488,3.173)--cycle;
\filldraw[fill opacity=0.8,fill=gray!20,draw=none](-8.302,3.069)--(-8.305,3.07)--(-8.315,3.124)--(-8.302,3.119)--cycle;
\draw(-8.302,3.069)--(-8.305,3.07)--(-8.315,3.124)--(-8.302,3.119);
\filldraw[fill opacity=0.8,fill=gray!20,draw=none](-8.327,3.117)--(-8.313,3.096)--(-8.318,3.069)--cycle;
\filldraw[fill opacity=0.8,fill=gray!20,draw=none](-8.292,3.065)--(-8.302,3.069)--(-8.302,3.119)--cycle;
\draw(-8.292,3.065)--(-8.302,3.069);
\filldraw[fill opacity=0.8,fill=gray!20,draw=none](-8.327,3.117)--(-8.332,3.143)--(-8.325,3.139)--(-8.315,3.124)--(-8.312,3.106)--(-8.313,3.096)--cycle;
\draw(-8.325,3.139)--(-8.315,3.124)--(-8.312,3.106);
\filldraw[fill opacity=0.8,fill=gray!20,draw=none](-8.302,3.091)--(-8.312,3.066)--(-8.321,3.115)--cycle;
\filldraw[fill opacity=0.8,fill=gray!20,draw=none](-8.296,3.083)--(-8.292,3.065)--(-8.312,3.066)--(-8.302,3.091)--cycle;
\draw(-8.292,3.065)--(-8.312,3.066);
\filldraw[fill opacity=0.8,fill=gray!20,draw=none](-8.282,3.093)--(-8.293,3.07)--(-8.299,3.1)--cycle;
\filldraw[fill opacity=0.8,fill=gray!20,draw=none](-8.296,3.083)--(-8.302,3.091)--(-8.299,3.1)--cycle;
\filldraw[fill opacity=0.8,fill=gray!20,draw=none](-8.295,3.11)--(-8.302,3.091)--(-8.321,3.115)--(-8.322,3.121)--(-8.303,3.119)--cycle;
\draw(-8.322,3.121)--(-8.303,3.119);
\filldraw[fill opacity=0.8,fill=gray!20,draw=none](-8.295,3.11)--(-8.303,3.119)--(-8.292,3.119)--cycle;
\draw(-8.303,3.119)--(-8.292,3.119);
\filldraw[fill opacity=0.8,fill=gray!20,draw=none](-8.279,3.091)--(-8.299,3.1)--(-8.302,3.119)--(-8.278,3.108)--cycle;
\draw(-8.302,3.119)--(-8.278,3.108);
\filldraw[fill opacity=0.8,fill=gray!20,draw=none](-8.293,3.102)--(-8.283,3.083)--(-8.296,3.059)--cycle;
\filldraw[fill opacity=0.8,fill=gray!20,draw=none](-8.28,3.076)--(-8.283,3.083)--(-8.279,3.091)--cycle;
\filldraw[fill opacity=0.8,fill=gray!20,draw=none](-8.293,3.102)--(-8.293,3.106)--(-8.275,3.099)--(-8.283,3.083)--cycle;
\draw(-8.293,3.106)--(-8.275,3.099);
\filldraw[fill opacity=0.8,fill=gray!20,draw=none](-8.208,3.092)--(-8.201,3.101)--(-8.179,3.091)--(-8.193,3.083)--cycle;
\draw(-8.201,3.101)--(-8.179,3.091)--(-8.193,3.083);
\filldraw[fill opacity=0.8,fill=gray!20,draw=none](-8.275,3.106)--(-8.273,3.098)--(-8.278,3.1)--cycle;
\draw(-8.273,3.098)--(-8.278,3.1);
\filldraw[fill opacity=0.8,fill=gray!20,draw=none](-8.262,3.084)--(-8.279,3.091)--(-8.278,3.1)--(-8.275,3.106)--(-8.274,3.106)--cycle;
\draw(-8.275,3.106)--(-8.274,3.106);
\filldraw[fill opacity=0.8,fill=gray!20](-8.036,3.68)--(-8.057,3.723)--(-7.978,3.727)--(-7.981,3.683)--cycle;
\filldraw[fill opacity=0.8,fill=gray!20,draw=none](-8.303,3.119)--(-8.315,3.124)--(-8.344,3.167)--(-8.333,3.163)--cycle;
\draw(-8.303,3.119)--(-8.315,3.124)--(-8.344,3.167)--(-8.333,3.163);
\filldraw[fill opacity=0.8,fill=gray!20,draw=none](-8.286,3.118)--(-8.322,3.121)--(-8.323,3.139)--cycle;
\draw(-8.286,3.118)--(-8.322,3.121);
\filldraw[fill opacity=0.8,fill=gray!20,draw=none](-8.278,3.1)--(-8.278,3.108)--(-8.275,3.106)--cycle;
\draw(-8.278,3.108)--(-8.275,3.106);
\filldraw[fill opacity=0.8,fill=gray!20,draw=none](-8.277,3.107)--(-8.275,3.107)--(-8.278,3.1)--(-8.289,3.105)--cycle;
\draw(-8.278,3.1)--(-8.289,3.105);
\filldraw[fill opacity=0.8,fill=gray!20,draw=none](-8.277,3.107)--(-8.289,3.105)--(-8.296,3.108)--cycle;
\draw(-8.289,3.105)--(-8.296,3.108);
\filldraw[fill opacity=0.8,fill=gray!20,draw=none](-8.276,3.107)--(-8.283,3.093)--(-8.289,3.101)--(-8.292,3.108)--cycle;
\draw(-8.276,3.107)--(-8.283,3.093);
\filldraw[fill opacity=0.8,fill=gray!20,draw=none](-8.283,3.093)--(-8.285,3.088)--(-8.289,3.101)--cycle;
\draw(-8.283,3.093)--(-8.285,3.088);
\filldraw[fill opacity=0.8,fill=gray!20,draw=none](-8.305,3.14)--(-8.303,3.139)--(-8.275,3.107)--(-8.275,3.106)--(-8.284,3.11)--cycle;
\draw(-8.275,3.106)--(-8.284,3.11);
\filldraw[fill opacity=0.8,fill=gray!20,draw=none](-8.275,3.107)--(-8.262,3.095)--(-8.262,3.074)--(-8.283,3.093)--(-8.276,3.107)--cycle;
\draw(-8.283,3.093)--(-8.276,3.107);
\filldraw[fill opacity=0.8,fill=gray!20,draw=none](-8.262,3.074)--(-8.262,3.044)--(-8.285,3.088)--(-8.283,3.093)--cycle;
\draw(-8.285,3.088)--(-8.283,3.093);
\filldraw[fill opacity=0.8,fill=gray!20,draw=none](-8.237,3.115)--(-8.248,3.155)--(-8.165,3.098)--(-8.165,3.097)--cycle;
\draw(-8.165,3.098)--(-8.165,3.097)--(-8.237,3.115)--(-8.248,3.155);
\filldraw[fill opacity=0.8,fill=gray!20,draw=none](-8.208,3.092)--(-8.239,3.112)--(-8.23,3.114)--(-8.201,3.101)--cycle;
\draw(-8.23,3.114)--(-8.201,3.101);
\filldraw[fill opacity=0.8,fill=gray!20,draw=none](-8.165,3.097)--(-8.166,3.091)--(-8.179,3.091)--(-8.203,3.102)--cycle;
\draw(-8.166,3.091)--(-8.179,3.091)--(-8.203,3.102);
\filldraw[fill opacity=0.8,fill=gray!20,draw=none](-8.181,3.062)--(-8.164,3.074)--(-8.155,3.05)--cycle;
\filldraw[fill opacity=0.8,fill=gray!20,draw=none](-8.175,3.059)--(-8.225,3.085)--(-8.198,3.105)--(-8.165,3.097)--(-8.161,3.065)--cycle;
\draw(-8.198,3.105)--(-8.165,3.097)--(-8.161,3.065);
\filldraw[fill opacity=0.8,fill=gray!20,draw=none](-8.262,3.109)--(-8.262,3.095)--(-8.275,3.107)--cycle;
\filldraw[fill opacity=0.8,fill=gray!20,draw=none](-8.227,3.106)--(-8.241,3.075)--(-8.262,3.095)--(-8.263,3.125)--cycle;
\draw(-8.227,3.106)--(-8.241,3.075);
\filldraw[fill opacity=0.8,fill=gray!20,draw=none](-8.248,3.155)--(-8.25,3.162)--(-8.242,3.16)--(-8.168,3.105)--(-8.165,3.098)--cycle;
\draw(-8.248,3.155)--(-8.25,3.162)--(-8.242,3.16);
\draw(-8.168,3.105)--(-8.165,3.098);
\filldraw[fill opacity=0.8,fill=gray!20,draw=none](-8.17,3.1)--(-8.165,3.097)--(-8.203,3.102)--(-8.23,3.114)--cycle;
\draw(-8.17,3.1)--(-8.165,3.097);
\draw(-8.203,3.102)--(-8.23,3.114);
\filldraw[fill opacity=0.8,fill=gray!20,draw=none](-8.272,3.157)--(-8.278,3.164)--(-8.25,3.162)--(-8.247,3.152)--cycle;
\draw(-8.278,3.164)--(-8.25,3.162)--(-8.247,3.152);
\filldraw[fill opacity=0.8,fill=gray!20,draw=none](-8.278,3.164)--(-8.293,3.174)--(-8.251,3.166)--(-8.25,3.162)--cycle;
\draw(-8.251,3.166)--(-8.25,3.162)--(-8.278,3.164);
\filldraw[fill opacity=0.8,fill=gray!20,draw=none](-8.246,3.161)--(-8.25,3.162)--(-8.251,3.166)--cycle;
\draw(-8.246,3.161)--(-8.25,3.162)--(-8.251,3.166);
\filldraw[fill opacity=0.8,fill=gray!20,draw=none](-7.99,3.835)--(-7.974,3.877)--(-7.975,3.835)--cycle;
\draw(-7.974,3.877)--(-7.975,3.835)--(-7.99,3.835);
\filldraw[fill opacity=0.8,fill=gray!20](-7.975,3.835)--(-7.974,3.892)--(-7.866,3.885)--(-7.87,3.828)--cycle;
\filldraw[fill opacity=0.8,fill=gray!20,draw=none](-8.205,3.155)--(-8.21,3.143)--(-8.25,3.165)--cycle;
\draw(-8.205,3.155)--(-8.21,3.143);
\filldraw[fill opacity=0.8,fill=gray!20,draw=none](-8.189,3.108)--(-8.199,3.107)--(-8.224,3.112)--cycle;
\filldraw[fill opacity=0.8,fill=gray!20,draw=none](-8.189,3.108)--(-8.17,3.1)--(-8.199,3.107)--cycle;
\draw(-8.189,3.108)--(-8.17,3.1);
\filldraw[fill opacity=0.8,fill=gray!20,draw=none](-8.189,3.108)--(-8.223,3.033)--(-8.248,3.058)--(-8.224,3.112)--cycle;
\draw(-8.189,3.108)--(-8.223,3.033);
\draw(-8.248,3.058)--(-8.224,3.112);
\filldraw[fill opacity=0.8,fill=gray!20](-7.882,3.772)--(-7.87,3.828)--(-7.798,3.81)--(-7.817,3.756)--cycle;
\filldraw[fill opacity=0.8,fill=gray!20](-7.902,3.721)--(-7.882,3.772)--(-7.817,3.756)--(-7.848,3.708)--cycle;
\filldraw[fill opacity=0.8,fill=gray!20](-7.927,3.679)--(-7.902,3.721)--(-7.848,3.708)--(-7.889,3.67)--cycle;
\filldraw[fill opacity=0.8,fill=gray!20,draw=none](-8.242,3.16)--(-8.246,3.161)--(-8.251,3.166)--cycle;
\draw(-8.242,3.16)--(-8.246,3.161);
\filldraw[fill opacity=0.8,fill=gray!20,draw=none](-8.21,3.143)--(-8.227,3.106)--(-8.263,3.125)--(-8.263,3.138)--(-8.25,3.165)--cycle;
\draw(-8.21,3.143)--(-8.227,3.106);
\draw(-8.263,3.138)--(-8.25,3.165);
\filldraw[fill opacity=0.8,fill=gray!20,draw=none](-8.212,3.138)--(-8.179,3.132)--(-8.168,3.105)--cycle;
\draw(-8.179,3.132)--(-8.168,3.105);
\filldraw[fill opacity=0.8,fill=gray!20,draw=none](-8.188,3.073)--(-8.195,3.056)--(-8.212,3.041)--(-8.219,3.041)--(-8.203,3.078)--cycle;
\draw(-8.188,3.073)--(-8.195,3.056);
\draw(-8.219,3.041)--(-8.203,3.078);
\filldraw[fill opacity=0.8,fill=gray!20,draw=none](-8.198,3.056)--(-8.262,3.084)--(-8.274,3.106)--(-8.193,3.071)--cycle;
\draw(-8.274,3.106)--(-8.193,3.071);
\filldraw[fill opacity=0.8,fill=gray!20,draw=none](-8.252,3.064)--(-8.247,3.06)--(-8.248,3.058)--cycle;
\draw(-8.247,3.06)--(-8.248,3.058);
\filldraw[fill opacity=0.8,fill=gray!20,draw=none](-8.212,3.041)--(-8.22,3.033)--(-8.223,3.033)--(-8.219,3.041)--cycle;
\draw(-8.223,3.033)--(-8.219,3.041);
\filldraw[fill opacity=0.8,fill=gray!20,draw=none](-8.195,3.056)--(-8.203,3.038)--(-8.22,3.033)--cycle;
\draw(-8.195,3.056)--(-8.203,3.038);
\filldraw[fill opacity=0.8,fill=gray!20,draw=none](-8.248,3.058)--(-8.262,3.084)--(-8.198,3.056)--(-8.207,3.027)--(-8.226,3.036)--cycle;
\draw(-8.207,3.027)--(-8.226,3.036);
\filldraw[fill opacity=0.8,fill=gray!20,draw=none](-8.252,3.064)--(-8.262,3.074)--(-8.262,3.084)--cycle;
\filldraw[fill opacity=0.8,fill=gray!20](-8.236,3.033)--(-8.564,3.176)--(-8.573,3.224)--(-8.245,3.082)--cycle;
\filldraw[fill opacity=0.8,fill=gray!20](-8.236,3.033)--(-8.564,3.176)--(-8.573,3.224)--(-8.245,3.082)--cycle;
\filldraw[fill opacity=0.8,fill=gray!20](-8.222,4.167)--(-8.191,4.185)--(-8.172,4.18)--(-8.222,4.167)--cycle;
\filldraw[fill opacity=0.8,fill=gray!20](-8.676,3.338)--(-8.679,3.346)--(-8.656,3.344)--(-8.631,3.335)--cycle;
\filldraw[fill opacity=0.8,fill=gray!20](-8.723,3.336)--(-8.703,3.345)--(-8.679,3.346)--(-8.676,3.338)--cycle;
\filldraw[fill opacity=0.8,fill=gray!20](-8.773,3.362)--(-8.729,3.377)--(-8.707,3.381)--(-8.731,3.37)--cycle;
\filldraw[fill opacity=0.8,fill=gray!20,draw=none](-8.877,1.27)--(-8.85,1.325)--(-8.819,1.322)--(-8.831,1.267)--cycle;
\draw(-8.85,1.325)--(-8.819,1.322)--(-8.831,1.267)--(-8.877,1.27);
\filldraw[fill opacity=0.8,fill=gray!20,draw=none](-8.85,1.325)--(-8.841,1.381)--(-8.815,1.379)--(-8.819,1.322)--cycle;
\draw(-8.841,1.381)--(-8.815,1.379)--(-8.819,1.322)--(-8.85,1.325);
\filldraw[fill opacity=0.8,fill=gray!20,draw=none](-8.919,1.221)--(-8.877,1.27)--(-8.831,1.267)--(-8.85,1.216)--cycle;
\draw(-8.877,1.27)--(-8.831,1.267)--(-8.85,1.216)--(-8.919,1.221);
\filldraw[fill opacity=0.8,fill=gray!20,draw=none](-8.841,1.381)--(-8.85,1.436)--(-8.819,1.433)--(-8.815,1.379)--cycle;
\draw(-8.85,1.436)--(-8.819,1.433)--(-8.815,1.379)--(-8.841,1.381);
\filldraw[fill opacity=0.8,fill=gray!20](-8.819,1.322)--(-8.815,1.379)--(-8.74,1.361)--(-8.746,1.305)--cycle;
\filldraw[fill opacity=0.8,fill=gray!20](-8.831,1.267)--(-8.819,1.322)--(-8.746,1.305)--(-8.766,1.251)--cycle;
\filldraw[fill opacity=0.8,fill=gray!20,draw=none](-8.929,1.178)--(-8.927,1.215)--(-8.919,1.221)--(-8.85,1.216)--(-8.876,1.174)--cycle;
\draw(-8.919,1.221)--(-8.85,1.216)--(-8.876,1.174)--(-8.929,1.178)--(-8.927,1.215);
\filldraw[fill opacity=0.8,fill=gray!20,draw=none](-8.985,1.175)--(-8.994,1.193)--(-8.927,1.215)--(-8.929,1.178)--cycle;
\draw(-8.927,1.215)--(-8.929,1.178)--(-8.985,1.175)--(-8.994,1.193);
\filldraw[fill opacity=0.8,fill=gray!20](-8.815,1.379)--(-8.819,1.433)--(-8.746,1.415)--(-8.74,1.361)--cycle;
\filldraw[fill opacity=0.8,fill=gray!20](-8.85,1.216)--(-8.831,1.267)--(-8.766,1.251)--(-8.797,1.203)--cycle;
\filldraw[fill opacity=0.8,fill=gray!20,draw=none](-8.85,1.436)--(-8.877,1.484)--(-8.831,1.481)--(-8.819,1.433)--cycle;
\draw(-8.877,1.484)--(-8.831,1.481)--(-8.819,1.433)--(-8.85,1.436);
\filldraw[fill opacity=0.8,fill=gray!20](-8.876,1.174)--(-8.85,1.216)--(-8.797,1.203)--(-8.838,1.164)--cycle;
\filldraw[fill opacity=0.8,fill=gray!20](-8.819,1.433)--(-8.831,1.481)--(-8.766,1.465)--(-8.746,1.415)--cycle;
\filldraw[fill opacity=0.8,fill=gray!20,draw=none](-9.028,1.167)--(-9.053,1.193)--(-9.057,1.201)--(-8.994,1.193)--(-8.985,1.175)--cycle;
\draw(-8.994,1.193)--(-8.985,1.175)--(-9.028,1.167)--(-9.053,1.193);
\filldraw[fill opacity=0.8,fill=gray!20](-8.961,1.143)--(-8.985,1.175)--(-8.929,1.178)--(-8.933,1.145)--cycle;
\filldraw[fill opacity=0.8,fill=gray!20](-8.933,1.145)--(-8.929,1.178)--(-8.876,1.174)--(-8.905,1.143)--cycle;
\filldraw[fill opacity=0.8,fill=gray!20,draw=none](-8.877,1.484)--(-8.919,1.524)--(-8.85,1.519)--(-8.831,1.481)--cycle;
\draw(-8.919,1.524)--(-8.85,1.519)--(-8.831,1.481)--(-8.877,1.484);
\filldraw[fill opacity=0.8,fill=gray!20,draw=none](-8.983,1.139)--(-9.008,1.154)--(-9.016,1.169)--(-8.985,1.175)--(-8.961,1.143)--cycle;
\draw(-9.016,1.169)--(-8.985,1.175)--(-8.961,1.143)--(-8.983,1.139)--(-9.008,1.154);
\filldraw[fill opacity=0.8,fill=gray!20](-8.831,1.481)--(-8.85,1.519)--(-8.797,1.506)--(-8.766,1.465)--cycle;
\filldraw[fill opacity=0.8,fill=gray!20](-8.905,1.143)--(-8.876,1.174)--(-8.838,1.164)--(-8.885,1.138)--cycle;
\filldraw[fill opacity=0.8,fill=gray!20,draw=none](-9.072,1.204)--(-9.079,1.215)--(-9.068,1.205)--(-9.071,1.203)--cycle;
\draw(-9.068,1.205)--(-9.071,1.203);
\filldraw[fill opacity=0.8,fill=gray!20,draw=none](-9.072,1.204)--(-9.096,1.231)--(-9.079,1.215)--cycle;
\filldraw[fill opacity=0.8,fill=gray!20,draw=none](-9.096,1.231)--(-8.985,1.356)--(-8.96,1.33)--(-9.072,1.203)--cycle;
\draw(-8.985,1.356)--(-8.96,1.33)--(-9.072,1.203);
\filldraw[fill opacity=0.8,fill=gray!20,draw=none](-9.053,1.193)--(-9.06,1.201)--(-9.057,1.201)--cycle;
\draw(-9.053,1.193)--(-9.06,1.201);
\filldraw[fill opacity=0.8,fill=gray!20,draw=none](-9.031,1.165)--(-9.053,1.193)--(-9.028,1.167)--cycle;
\draw(-9.053,1.193)--(-9.028,1.167)--(-9.031,1.165);
\filldraw[fill opacity=0.8,fill=gray!20,draw=none](-9.008,1.154)--(-9.028,1.167)--(-9.016,1.169)--cycle;
\draw(-9.008,1.154)--(-9.028,1.167)--(-9.016,1.169);
\filldraw[fill opacity=0.8,fill=gray!20,draw=none](-8.998,1.143)--(-9.031,1.165)--(-9.028,1.167)--(-9.008,1.154)--cycle;
\draw(-9.031,1.165)--(-9.028,1.167)--(-9.008,1.154);
\filldraw[fill opacity=0.8,fill=gray!20,draw=none](-8.988,1.136)--(-8.998,1.143)--(-9.008,1.154)--(-8.983,1.139)--cycle;
\draw(-9.008,1.154)--(-8.983,1.139)--(-8.988,1.136);
\filldraw[fill opacity=0.8,fill=gray!20,draw=none](-8.884,1.27)--(-8.998,1.143)--(-9.036,1.168)--(-8.921,1.297)--cycle;
\draw(-9.036,1.168)--(-8.921,1.297)--(-8.884,1.27)--(-8.998,1.143);
\filldraw[fill opacity=0.8,fill=gray!20,draw=none](-8.884,1.27)--(-8.998,1.143)--(-9.036,1.168)--(-8.921,1.297)--cycle;
\draw(-9.036,1.168)--(-8.921,1.297)--(-8.884,1.27)--(-8.998,1.143);
\filldraw[fill opacity=0.8,fill=gray!20,draw=none](-8.884,1.27)--(-9.101,1.026)--(-9.103,1.029)--(-9.064,1.135)--(-8.921,1.297)--cycle;
\draw(-9.064,1.135)--(-8.921,1.297)--(-8.884,1.27)--(-9.101,1.026);
\filldraw[fill opacity=0.8,fill=gray!20,draw=none](-9.044,1.178)--(-9.046,1.185)--(-9.034,1.169)--cycle;
\filldraw[fill opacity=0.8,fill=gray!20,draw=none](-8.921,1.297)--(-9.034,1.169)--(-9.044,1.178)--(-8.984,1.303)--(-8.96,1.33)--cycle;
\draw(-8.984,1.303)--(-8.96,1.33)--(-8.921,1.297)--(-9.034,1.169);
\filldraw[fill opacity=0.8,fill=gray!20,draw=none](-9.046,1.185)--(-9.044,1.178)--(-9.072,1.203)--(-9.068,1.205)--(-9.06,1.201)--(-9.053,1.193)--cycle;
\draw(-9.072,1.203)--(-9.068,1.205);
\draw(-9.06,1.201)--(-9.053,1.193);
\filldraw[fill opacity=0.8,fill=gray!20,draw=none](-8.921,1.297)--(-9.034,1.169)--(-9.072,1.203)--(-8.96,1.33)--cycle;
\draw(-9.072,1.203)--(-8.96,1.33)--(-8.921,1.297)--(-9.034,1.169);
\filldraw[fill opacity=0.8,fill=gray!20,draw=none](-8.921,1.297)--(-9.034,1.169)--(-9.072,1.203)--(-8.96,1.33)--cycle;
\draw(-9.072,1.203)--(-8.96,1.33)--(-8.921,1.297)--(-9.034,1.169);
\filldraw[fill opacity=0.8,fill=gray!20,draw=none](-9.096,1.231)--(-8.985,1.356)--(-8.96,1.33)--(-9.072,1.203)--cycle;
\draw(-8.985,1.356)--(-8.96,1.33)--(-9.072,1.203);
\filldraw[fill opacity=0.8,fill=gray!20,draw=none](-9.103,1.223)--(-9.096,1.231)--(-9.072,1.203)--cycle;
\filldraw[fill opacity=0.8,fill=gray!20,draw=none](-9.103,1.223)--(-8.985,1.356)--(-8.96,1.33)--(-9.072,1.203)--cycle;
\draw(-8.985,1.356)--(-8.96,1.33)--(-9.072,1.203);
\filldraw[fill opacity=0.8,fill=gray!20,draw=none](-8.887,1.323)--(-8.87,1.305)--(-8.851,1.283)--(-8.838,1.261)--(-8.839,1.251)--(-8.856,1.254)--(-8.884,1.27)--(-8.921,1.297)--(-8.96,1.33)--(-8.985,1.356)--(-9.002,1.373)--(-9.021,1.396)--(-9.035,1.418)--(-9.033,1.428)--(-9.016,1.425)--(-8.988,1.408)--(-8.951,1.381)--(-8.912,1.348)--cycle;
\draw(-8.87,1.305)--(-8.851,1.283)--(-8.838,1.261)--(-8.839,1.251)--(-8.856,1.254)--(-8.884,1.27)--(-8.921,1.297)--(-8.96,1.33)--(-8.985,1.356);
\draw(-9.002,1.373)--(-9.021,1.396)--(-9.035,1.418)--(-9.033,1.428)--(-9.016,1.425)--(-8.988,1.408)--(-8.951,1.381)--(-8.912,1.348)--(-8.887,1.323);
\filldraw[fill opacity=0.8,fill=gray!20,draw=none](-8.887,1.323)--(-8.87,1.305)--(-8.851,1.283)--(-8.838,1.261)--(-8.839,1.251)--(-8.856,1.254)--(-8.884,1.27)--(-8.921,1.297)--(-8.96,1.33)--(-8.985,1.356)--(-9.002,1.373)--(-9.021,1.396)--(-9.035,1.418)--(-9.033,1.428)--(-9.016,1.425)--(-8.988,1.408)--(-8.951,1.381)--(-8.912,1.348)--cycle;
\draw(-8.87,1.305)--(-8.851,1.283)--(-8.838,1.261)--(-8.839,1.251)--(-8.856,1.254)--(-8.884,1.27)--(-8.921,1.297)--(-8.96,1.33)--(-8.985,1.356);
\draw(-9.002,1.373)--(-9.021,1.396)--(-9.035,1.418)--(-9.033,1.428)--(-9.016,1.425)--(-8.988,1.408)--(-8.951,1.381)--(-8.912,1.348)--(-8.887,1.323);
\filldraw[fill opacity=0.8,fill=gray!20,draw=none](-8.887,1.323)--(-8.87,1.305)--(-8.851,1.283)--(-8.838,1.261)--(-8.839,1.251)--(-8.856,1.254)--(-8.884,1.27)--(-8.921,1.297)--(-8.96,1.33)--(-8.985,1.356)--(-9.002,1.373)--(-9.021,1.396)--(-9.035,1.418)--(-9.033,1.428)--(-9.016,1.425)--(-8.988,1.408)--(-8.951,1.381)--(-8.912,1.348)--cycle;
\draw(-8.87,1.305)--(-8.851,1.283)--(-8.838,1.261)--(-8.839,1.251)--(-8.856,1.254)--(-8.884,1.27)--(-8.921,1.297)--(-8.96,1.33)--(-8.985,1.356);
\draw(-9.002,1.373)--(-9.021,1.396)--(-9.035,1.418)--(-9.033,1.428)--(-9.016,1.425)--(-8.988,1.408)--(-8.951,1.381)--(-8.912,1.348)--(-8.887,1.323);
\filldraw[fill opacity=0.8,fill=gray!20,draw=none](-9.044,1.178)--(-9.072,1.203)--(-8.984,1.303)--cycle;
\draw(-9.072,1.203)--(-8.984,1.303);
\filldraw[fill opacity=0.8,fill=gray!20,draw=none](-8.919,1.524)--(-8.927,1.528)--(-8.929,1.549)--(-8.876,1.545)--(-8.85,1.519)--cycle;
\draw(-8.927,1.528)--(-8.929,1.549)--(-8.876,1.545)--(-8.85,1.519)--(-8.919,1.524);
\filldraw[fill opacity=0.8,fill=gray!20,draw=none](-8.994,1.536)--(-8.985,1.546)--(-8.929,1.549)--(-8.927,1.528)--cycle;
\draw(-8.994,1.536)--(-8.985,1.546)--(-8.929,1.549)--(-8.927,1.528);
\filldraw[fill opacity=0.8,fill=gray!20,draw=none](-9.154,1.03)--(-9.155,.995)--(-9.156,.986)--(-9.167,.991)--(-9.159,1.036)--cycle;
\draw(-9.156,.986)--(-9.167,.991);
\filldraw[fill opacity=0.8,fill=gray!20,draw=none](-9.178,.949)--(-9.186,.937)--(-9.196,.941)--(-9.167,.991)--(-9.163,.989)--cycle;
\draw(-9.186,.937)--(-9.196,.941);
\draw(-9.167,.991)--(-9.163,.989);
\filldraw[fill opacity=0.8,fill=gray!20](-8.766,1.251)--(-8.746,1.305)--(-8.724,1.281)--(-8.746,1.23)--cycle;
\filldraw[fill opacity=0.8,fill=gray!20](-8.797,1.203)--(-8.766,1.251)--(-8.746,1.23)--(-8.781,1.186)--cycle;
\filldraw[fill opacity=0.8,fill=gray!20,draw=none](-9.125,1.27)--(-9.113,1.249)--(-9.13,1.272)--cycle;
\filldraw[fill opacity=0.8,fill=gray!20,draw=none](-9.125,1.27)--(-9.13,1.272)--(-9.143,1.29)--(-9.137,1.293)--cycle;
\draw(-9.143,1.29)--(-9.137,1.293);
\filldraw[fill opacity=0.8,fill=gray!20,draw=none](-9.002,1.373)--(-9.113,1.249)--(-9.13,1.272)--(-9.021,1.396)--cycle;
\draw(-9.13,1.272)--(-9.021,1.396)--(-9.002,1.373);
\filldraw[fill opacity=0.8,fill=gray!20,draw=none](-9.002,1.373)--(-9.188,1.164)--(-9.224,1.167)--(-9.021,1.396)--cycle;
\draw(-9.224,1.167)--(-9.021,1.396)--(-9.002,1.373);
\filldraw[fill opacity=0.8,fill=gray!20,draw=none](-9.002,1.373)--(-9.188,1.164)--(-9.224,1.167)--(-9.021,1.396)--cycle;
\draw(-9.224,1.167)--(-9.021,1.396)--(-9.002,1.373);
\filldraw[fill opacity=0.8,fill=gray!20](-8.746,1.305)--(-8.74,1.361)--(-8.717,1.336)--(-8.724,1.281)--cycle;
\filldraw[fill opacity=0.8,fill=gray!20,draw=none](-9.223,.909)--(-9.205,.933)--(-9.196,.941)--(-9.183,.936)--cycle;
\draw(-9.196,.941)--(-9.183,.936);
\filldraw[fill opacity=0.8,fill=gray!20](-8.838,1.164)--(-8.797,1.203)--(-8.781,1.186)--(-8.826,1.152)--cycle;
\filldraw[fill opacity=0.8,fill=gray!20](-8.936,1.125)--(-8.933,1.145)--(-8.905,1.143)--(-8.936,1.125)--cycle;
\filldraw[fill opacity=0.8,fill=gray!20](-8.936,1.125)--(-8.961,1.143)--(-8.933,1.145)--(-8.936,1.125)--cycle;
\filldraw[fill opacity=0.8,fill=gray!20,draw=none](-9.152,1.026)--(-9.159,1.036)--(-9.157,1.047)--(-9.146,1.042)--cycle;
\draw(-9.157,1.047)--(-9.146,1.042);
\filldraw[fill opacity=0.8,fill=gray!20,draw=none](-9.223,.909)--(-9.232,.903)--(-9.238,.906)--(-9.205,.933)--cycle;
\draw(-9.232,.903)--(-9.238,.906);
\filldraw[fill opacity=0.8,fill=gray!20,draw=none](-9.072,1.204)--(-9.071,1.203)--(-9.072,1.203)--cycle;
\draw(-9.071,1.203)--(-9.072,1.203);
\filldraw[fill opacity=0.8,fill=gray!20,draw=none](-9.178,.949)--(-9.163,.989)--(-9.155,.985)--cycle;
\draw(-9.163,.989)--(-9.155,.985);
\filldraw[fill opacity=0.8,fill=gray!20,draw=none](-9.193,.971)--(-9.202,.949)--(-9.205,.95)--cycle;
\draw(-9.202,.949)--(-9.205,.95);
\filldraw[fill opacity=0.8,fill=gray!20,draw=none](-9.222,.936)--(-9.217,.941)--(-9.205,.95)--(-9.202,.949)--cycle;
\draw(-9.205,.95)--(-9.202,.949);
\filldraw[fill opacity=0.8,fill=gray!20,draw=none](-9.222,.936)--(-9.227,.932)--(-9.217,.941)--cycle;
\filldraw[fill opacity=0.8,fill=gray!20,draw=none](-9.222,.936)--(-9.222,.938)--(-9.21,.952)--(-9.199,.951)--cycle;
\draw(-9.222,.936)--(-9.222,.938);
\draw(-9.21,.952)--(-9.199,.951);
\filldraw[fill opacity=0.8,fill=gray!20,draw=none](-9.193,.971)--(-9.179,.995)--(-9.175,.993)--(-9.202,.949)--cycle;
\draw(-9.179,.995)--(-9.175,.993);
\filldraw[fill opacity=0.8,fill=gray!20,draw=none](-9.21,.952)--(-9.188,.997)--(-9.133,.993)--(-9.143,.947)--cycle;
\draw(-9.188,.997)--(-9.133,.993)--(-9.143,.947)--(-9.21,.952);
\filldraw[fill opacity=0.8,fill=gray!20,draw=none](-9.105,.902)--(-9.186,.937)--(-9.155,.985)--(-9.106,.964)--cycle;
\draw(-9.105,.902)--(-9.186,.937);
\draw(-9.155,.985)--(-9.106,.964);
\filldraw[fill opacity=0.8,fill=gray!20,draw=none](-9.194,.907)--(-9.223,.909)--(-9.222,.938)--(-9.21,.952)--(-9.178,.949)--cycle;
\draw(-9.194,.907)--(-9.223,.909)--(-9.222,.938);
\draw(-9.21,.952)--(-9.178,.949);
\filldraw[fill opacity=0.8,fill=gray!20,draw=none](-9.174,1.025)--(-9.174,.996)--(-9.175,.993)--(-9.179,.995)--cycle;
\draw(-9.175,.993)--(-9.179,.995);
\filldraw[fill opacity=0.8,fill=gray!20,draw=none](-9.034,1.169)--(-9.045,1.158)--(-9.072,1.203)--cycle;
\draw(-9.034,1.169)--(-9.045,1.158);
\filldraw[fill opacity=0.8,fill=gray!20,draw=none](-9.034,1.169)--(-9.128,1.064)--(-9.136,1.068)--(-9.141,1.073)--(-9.158,1.107)--(-9.072,1.203)--cycle;
\draw(-9.034,1.169)--(-9.128,1.064);
\draw(-9.158,1.107)--(-9.072,1.203);
\filldraw[fill opacity=0.8,fill=gray!20,draw=none](-9.045,1.158)--(-9.128,1.064)--(-9.136,1.068)--(-9.139,1.071)--(-9.139,1.128)--(-9.072,1.203)--cycle;
\draw(-9.045,1.158)--(-9.128,1.064);
\draw(-9.139,1.128)--(-9.072,1.203);
\filldraw[fill opacity=0.8,fill=gray!20,draw=none](-9.034,1.169)--(-9.128,1.064)--(-9.136,1.068)--(-9.141,1.073)--(-9.158,1.107)--(-9.072,1.203)--cycle;
\draw(-9.034,1.169)--(-9.128,1.064);
\draw(-9.158,1.107)--(-9.072,1.203);
\filldraw[fill opacity=0.8,fill=gray!20,draw=none](-9.103,1.223)--(-9.096,1.231)--(-9.072,1.203)--cycle;
\filldraw[fill opacity=0.8,fill=gray!20,draw=none](-9.175,1.142)--(-9.103,1.223)--(-9.072,1.203)--(-9.164,1.1)--cycle;
\draw(-9.072,1.203)--(-9.164,1.1);
\filldraw[fill opacity=0.8,fill=gray!20,draw=none](-9.175,1.142)--(-9.103,1.223)--(-9.072,1.203)--(-9.164,1.1)--cycle;
\draw(-9.072,1.203)--(-9.164,1.1);
\filldraw[fill opacity=0.8,fill=gray!20,draw=none](-9.175,1.142)--(-9.103,1.223)--(-9.072,1.203)--(-9.164,1.1)--cycle;
\draw(-9.072,1.203)--(-9.164,1.1);
\filldraw[fill opacity=0.8,fill=gray!20,draw=none](-9.096,1.231)--(-9.072,1.204)--(-9.072,1.203)--(-9.091,1.19)--(-9.126,1.235)--(-9.111,1.245)--cycle;
\draw(-9.072,1.203)--(-9.091,1.19)--(-9.126,1.235)--(-9.111,1.245);
\filldraw[fill opacity=0.8,fill=gray!20,draw=none](-8.998,1.143)--(-9.101,1.026)--(-9.103,1.029)--(-9.064,1.135)--(-9.036,1.168)--cycle;
\draw(-8.998,1.143)--(-9.101,1.026);
\draw(-9.064,1.135)--(-9.036,1.168);
\filldraw[fill opacity=0.8,fill=gray!20,draw=none](-8.998,1.143)--(-9.101,1.026)--(-9.103,1.029)--(-9.064,1.135)--(-9.036,1.168)--cycle;
\draw(-8.998,1.143)--(-9.101,1.026);
\draw(-9.064,1.135)--(-9.036,1.168);
\filldraw[fill opacity=0.8,fill=gray!20,draw=none](-9.044,1.178)--(-9.038,1.16)--(-9.046,1.155)--(-9.091,1.19)--(-9.072,1.203)--cycle;
\draw(-9.038,1.16)--(-9.046,1.155)--(-9.091,1.19)--(-9.072,1.203);
\filldraw[fill opacity=0.8,fill=gray!20,draw=none](-8.936,1.125)--(-8.975,1.137)--(-8.976,1.141)--(-8.961,1.143)--(-8.936,1.125)--cycle;
\draw(-8.976,1.141)--(-8.961,1.143)--(-8.936,1.125)--(-8.936,1.125)--(-8.975,1.137);
\filldraw[fill opacity=0.8,fill=gray!20,draw=none](-9.154,1.045)--(-9.157,1.047)--(-9.159,1.056)--cycle;
\draw(-9.154,1.045)--(-9.157,1.047);
\filldraw[fill opacity=0.8,fill=gray!20,draw=none](-9.175,.949)--(-9.21,.952)--(-9.188,.997)--(-9.174,.996)--cycle;
\draw(-9.175,.949)--(-9.21,.952);
\draw(-9.188,.997)--(-9.174,.996);
\filldraw[fill opacity=0.8,fill=gray!20,draw=none](-9.155,.995)--(-9.155,.985)--(-9.156,.986)--cycle;
\draw(-9.155,.985)--(-9.156,.986);
\filldraw[fill opacity=0.8,fill=gray!20,draw=none](-9.175,.949)--(-9.174,.996)--(-9.133,.993)--(-9.143,.947)--cycle;
\draw(-9.174,.996)--(-9.133,.993)--(-9.143,.947)--(-9.175,.949);
\filldraw[fill opacity=0.8,fill=gray!20,draw=none](-9.223,.909)--(-9.222,.936)--(-9.199,.951)--(-9.143,.947)--(-9.159,.904)--cycle;
\draw(-9.199,.951)--(-9.143,.947)--(-9.159,.904)--(-9.223,.909)--(-9.222,.936);
\filldraw[fill opacity=0.8,fill=gray!20,draw=none](-8.975,1.137)--(-8.983,1.139)--(-8.976,1.141)--cycle;
\draw(-8.975,1.137)--(-8.983,1.139)--(-8.976,1.141);
\filldraw[fill opacity=0.8,fill=gray!20,draw=none](-8.973,1.13)--(-8.985,1.132)--(-8.99,1.135)--(-8.983,1.139)--(-8.975,1.137)--cycle;
\draw(-8.973,1.13)--(-8.985,1.132);
\draw(-8.99,1.135)--(-8.983,1.139)--(-8.975,1.137);
\filldraw[fill opacity=0.8,fill=gray!20,draw=none](-8.936,1.125)--(-8.973,1.13)--(-8.975,1.137)--(-8.936,1.125)--cycle;
\draw(-8.975,1.137)--(-8.936,1.125)--(-8.936,1.125)--(-8.973,1.13);
\filldraw[fill opacity=0.8,fill=gray!20,draw=none](-8.998,1.143)--(-8.988,1.136)--(-8.99,1.135)--cycle;
\draw(-8.988,1.136)--(-8.99,1.135);
\filldraw[fill opacity=0.8,fill=gray!20,draw=none](-8.969,1.126)--(-8.985,1.132)--(-8.973,1.13)--cycle;
\draw(-8.985,1.132)--(-8.973,1.13);
\filldraw[fill opacity=0.8,fill=gray!20,draw=none](-8.936,1.125)--(-8.969,1.126)--(-8.973,1.13)--(-8.936,1.125)--cycle;
\draw(-8.973,1.13)--(-8.936,1.125)--(-8.936,1.125)--(-8.969,1.126);
\filldraw[fill opacity=0.8,fill=gray!20,draw=none](-8.856,1.254)--(-8.969,1.126)--(-8.985,1.132)--(-8.99,1.135)--(-8.998,1.143)--(-8.884,1.27)--cycle;
\draw(-8.998,1.143)--(-8.884,1.27)--(-8.856,1.254)--(-8.969,1.126);
\filldraw[fill opacity=0.8,fill=gray!20,draw=none](-8.856,1.254)--(-8.969,1.126)--(-8.985,1.132)--(-8.99,1.135)--(-8.998,1.143)--(-8.884,1.27)--cycle;
\draw(-8.998,1.143)--(-8.884,1.27)--(-8.856,1.254)--(-8.969,1.126);
\filldraw[fill opacity=0.8,fill=gray!20,draw=none](-8.856,1.254)--(-8.915,1.187)--(-8.992,1.136)--(-8.998,1.143)--(-8.884,1.27)--cycle;
\draw(-8.998,1.143)--(-8.884,1.27)--(-8.856,1.254)--(-8.915,1.187);
\filldraw[fill opacity=0.8,fill=gray!20,draw=none](-9.154,1.065)--(-9.146,1.042)--(-9.154,1.045)--(-9.159,1.056)--(-9.164,1.08)--cycle;
\draw(-9.146,1.042)--(-9.154,1.045);
\filldraw[fill opacity=0.8,fill=gray!20,draw=none](-9.174,1.02)--(-9.174,1.025)--(-9.17,1.045)--(-9.166,1.043)--cycle;
\draw(-9.17,1.045)--(-9.166,1.043);
\filldraw[fill opacity=0.8,fill=gray!20](-8.74,1.361)--(-8.746,1.415)--(-8.724,1.392)--(-8.717,1.336)--cycle;
\filldraw[fill opacity=0.8,fill=gray!20,draw=none](-9.044,1.178)--(-9.034,1.169)--(-9.031,1.165)--(-9.038,1.16)--cycle;
\draw(-9.031,1.165)--(-9.038,1.16);
\filldraw[fill opacity=0.8,fill=gray!20,draw=none](-9.174,1.02)--(-9.166,1.043)--(-9.174,.996)--cycle;
\filldraw[fill opacity=0.8,fill=gray!20,draw=none](-9.155,.995)--(-9.154,1.03)--(-9.15,1.023)--cycle;
\filldraw[fill opacity=0.8,fill=gray!20,draw=none](-9.149,.994)--(-9.188,.997)--(-9.18,1.044)--(-9.166,1.043)--cycle;
\draw(-9.149,.994)--(-9.188,.997);
\draw(-9.18,1.044)--(-9.166,1.043);
\filldraw[fill opacity=0.8,fill=gray!20,draw=none](-9.149,.994)--(-9.188,.997)--(-9.18,1.044)--(-9.166,1.043)--cycle;
\draw(-9.149,.994)--(-9.188,.997);
\draw(-9.18,1.044)--(-9.166,1.043);
\filldraw[fill opacity=0.8,fill=gray!20,draw=none](-9.159,1.08)--(-9.154,1.065)--(-9.164,1.08)--(-9.164,1.082)--cycle;
\filldraw[fill opacity=0.8,fill=gray!20,draw=none](-9.174,1.065)--(-9.174,1.084)--(-9.171,1.07)--(-9.166,1.043)--(-9.17,1.045)--cycle;
\draw(-9.166,1.043)--(-9.17,1.045);
\filldraw[fill opacity=0.8,fill=gray!20](-8.85,1.519)--(-8.876,1.545)--(-8.838,1.536)--(-8.797,1.506)--cycle;
\filldraw[fill opacity=0.8,fill=gray!20](-8.936,1.125)--(-8.905,1.143)--(-8.885,1.138)--(-8.936,1.125)--cycle;
\filldraw[fill opacity=0.8,fill=gray!20,draw=none](-9.15,1.023)--(-9.152,1.026)--(-9.146,1.042)--cycle;
\filldraw[fill opacity=0.8,fill=gray!20,draw=none](-9.149,.994)--(-9.166,1.043)--(-9.13,1.04)--(-9.133,.993)--cycle;
\draw(-9.166,1.043)--(-9.13,1.04)--(-9.133,.993)--(-9.149,.994);
\filldraw[fill opacity=0.8,fill=gray!20,draw=none](-9.149,.994)--(-9.166,1.043)--(-9.13,1.04)--(-9.133,.993)--cycle;
\draw(-9.166,1.043)--(-9.13,1.04)--(-9.133,.993)--(-9.149,.994);
\filldraw[fill opacity=0.8,fill=gray!20,draw=none](-9.166,1.043)--(-9.157,1.039)--(-9.155,1.033)--(-9.155,.985)--(-9.175,.993)--cycle;
\draw(-9.166,1.043)--(-9.157,1.039);
\draw(-9.155,.985)--(-9.175,.993);
\filldraw[fill opacity=0.8,fill=gray!20,draw=none](-9.155,.995)--(-9.15,1.023)--(-9.139,1.009)--(-9.139,.979)--(-9.155,.985)--cycle;
\draw(-9.139,.979)--(-9.155,.985);
\filldraw[fill opacity=0.8,fill=gray!20,draw=none](-9.194,.907)--(-9.178,.949)--(-9.143,.947)--(-9.159,.904)--cycle;
\draw(-9.178,.949)--(-9.143,.947)--(-9.159,.904)--(-9.194,.907);
\filldraw[fill opacity=0.8,fill=gray!20,draw=none](-9.175,.993)--(-9.148,.981)--(-9.148,.973)--(-9.163,.932)--(-9.202,.949)--cycle;
\draw(-9.175,.993)--(-9.148,.981);
\draw(-9.163,.932)--(-9.202,.949);
\filldraw[fill opacity=0.8,fill=gray!20,draw=none](-9.227,.932)--(-9.222,.936)--(-9.24,.917)--(-9.243,.918)--cycle;
\draw(-9.24,.917)--(-9.243,.918);
\filldraw[fill opacity=0.8,fill=gray!20,draw=none](-9.243,.918)--(-9.242,.918)--(-9.287,.903)--(-9.288,.904)--cycle;
\draw(-9.243,.918)--(-9.242,.918);
\draw(-9.287,.903)--(-9.288,.904);
\filldraw[fill opacity=0.8,fill=gray!20,draw=none](-9.279,.885)--(-9.251,.902)--(-9.238,.906)--(-9.228,.901)--cycle;
\draw(-9.238,.906)--(-9.228,.901);
\filldraw[fill opacity=0.8,fill=gray!20,draw=none](-9.279,.885)--(-9.288,.889)--(-9.251,.902)--cycle;
\draw(-9.279,.885)--(-9.288,.889);
\filldraw[fill opacity=0.8,fill=gray!20,draw=none](-9.276,.907)--(-9.222,.922)--(-9.223,.909)--cycle;
\draw(-9.222,.922)--(-9.223,.909)--(-9.276,.907);
\filldraw[fill opacity=0.8,fill=gray!20,draw=none](-9.276,.907)--(-9.222,.922)--(-9.223,.909)--cycle;
\draw(-9.222,.922)--(-9.223,.909)--(-9.276,.907);
\filldraw[fill opacity=0.8,fill=gray!20,draw=none](-9.228,.901)--(-9.232,.903)--(-9.223,.909)--cycle;
\draw(-9.228,.901)--(-9.232,.903);
\filldraw[fill opacity=0.8,fill=gray!20,draw=none](-9.287,.903)--(-9.288,.906)--(-9.223,.909)--(-9.224,.892)--cycle;
\draw(-9.287,.903)--(-9.288,.906)--(-9.223,.909)--(-9.224,.892);
\filldraw[fill opacity=0.8,fill=gray!20](-9.225,.872)--(-9.223,.909)--(-9.159,.904)--(-9.18,.869)--cycle;
\filldraw[fill opacity=0.8,fill=gray!20](-9.225,.872)--(-9.223,.909)--(-9.159,.904)--(-9.18,.869)--cycle;
\filldraw[fill opacity=0.8,fill=gray!20,draw=none](-9.287,.903)--(-9.288,.906)--(-9.223,.909)--(-9.224,.892)--cycle;
\draw(-9.287,.903)--(-9.288,.906)--(-9.223,.909)--(-9.224,.892);
\filldraw[fill opacity=0.8,fill=gray!20,draw=none](-9.137,.899)--(-9.159,.904)--(-9.143,.947)--(-9.105,.937)--cycle;
\draw(-9.137,.899)--(-9.159,.904)--(-9.143,.947)--(-9.105,.937);
\filldraw[fill opacity=0.8,fill=gray!20,draw=none](-9.137,.899)--(-9.105,.937)--(-9.089,.933)--(-9.115,.894)--cycle;
\draw(-9.105,.937)--(-9.089,.933)--(-9.115,.894)--(-9.137,.899);
\filldraw[fill opacity=0.8,fill=gray!20,draw=none](-9.137,.899)--(-9.159,.904)--(-9.143,.947)--(-9.105,.937)--cycle;
\draw(-9.137,.899)--(-9.159,.904)--(-9.143,.947)--(-9.105,.937);
\filldraw[fill opacity=0.8,fill=gray!20](-9.18,.869)--(-9.159,.904)--(-9.115,.894)--(-9.149,.861)--cycle;
\filldraw[fill opacity=0.8,fill=gray!20](-9.18,.869)--(-9.159,.904)--(-9.115,.894)--(-9.149,.861)--cycle;
\filldraw[fill opacity=0.8,fill=gray!20,draw=none](-9.168,.875)--(-9.228,.901)--(-9.223,.909)--(-9.183,.936)--(-9.132,.914)--cycle;
\draw(-9.168,.875)--(-9.228,.901);
\draw(-9.183,.936)--(-9.132,.914);
\filldraw[fill opacity=0.8,fill=gray!20,draw=none](-9.279,.885)--(-9.287,.903)--(-9.224,.892)--(-9.225,.876)--cycle;
\draw(-9.279,.885)--(-9.287,.903);
\draw(-9.224,.892)--(-9.225,.876);
\filldraw[fill opacity=0.8,fill=gray!20,draw=none](-9.279,.885)--(-9.287,.903)--(-9.224,.892)--(-9.225,.876)--cycle;
\draw(-9.279,.885)--(-9.287,.903);
\draw(-9.224,.892)--(-9.225,.876);
\filldraw[fill opacity=0.8,fill=gray!20,draw=none](-9.272,.87)--(-9.279,.885)--(-9.225,.876)--(-9.225,.872)--cycle;
\draw(-9.225,.876)--(-9.225,.872)--(-9.272,.87)--(-9.279,.885);
\filldraw[fill opacity=0.8,fill=gray!20,draw=none](-9.272,.87)--(-9.279,.885)--(-9.225,.876)--(-9.225,.872)--cycle;
\draw(-9.225,.876)--(-9.225,.872)--(-9.272,.87)--(-9.279,.885);
\filldraw[fill opacity=0.8,fill=gray!20](-9.228,.845)--(-9.225,.872)--(-9.18,.869)--(-9.205,.843)--cycle;
\filldraw[fill opacity=0.8,fill=gray!20](-9.252,.844)--(-9.272,.87)--(-9.225,.872)--(-9.228,.845)--cycle;
\filldraw[fill opacity=0.8,fill=gray!20](-9.228,.845)--(-9.225,.872)--(-9.18,.869)--(-9.205,.843)--cycle;
\filldraw[fill opacity=0.8,fill=gray!20](-9.252,.844)--(-9.272,.87)--(-9.225,.872)--(-9.228,.845)--cycle;
\filldraw[fill opacity=0.8,fill=gray!20](-9.205,.843)--(-9.18,.869)--(-9.149,.861)--(-9.188,.839)--cycle;
\filldraw[fill opacity=0.8,fill=gray!20](-9.205,.843)--(-9.18,.869)--(-9.149,.861)--(-9.188,.839)--cycle;
\filldraw[fill opacity=0.8,fill=gray!20](-9.149,.861)--(-9.115,.894)--(-9.101,.879)--(-9.139,.851)--cycle;
\filldraw[fill opacity=0.8,fill=gray!20](-9.149,.861)--(-9.115,.894)--(-9.101,.879)--(-9.139,.851)--cycle;
\filldraw[fill opacity=0.8,fill=gray!20](-9.188,.839)--(-9.149,.861)--(-9.139,.851)--(-9.183,.834)--cycle;
\filldraw[fill opacity=0.8,fill=gray!20](-9.188,.839)--(-9.149,.861)--(-9.139,.851)--(-9.183,.834)--cycle;
\filldraw[fill opacity=0.8,fill=gray!20,draw=none](-9.146,.849)--(-9.139,.851)--(-9.148,.846)--cycle;
\draw(-9.146,.849)--(-9.139,.851)--(-9.148,.846);
\filldraw[fill opacity=0.8,fill=gray!20,draw=none](-9.146,.849)--(-9.139,.851)--(-9.148,.846)--cycle;
\draw(-9.146,.849)--(-9.139,.851)--(-9.148,.846);
\filldraw[fill opacity=0.8,fill=gray!20,draw=none](-9.183,.843)--(-9.279,.885)--(-9.228,.901)--(-9.133,.86)--cycle;
\draw(-9.183,.843)--(-9.279,.885);
\draw(-9.228,.901)--(-9.133,.86);
\filldraw[fill opacity=0.8,fill=gray!20,draw=none](-9.242,.918)--(-9.186,.893)--(-9.231,.879)--(-9.287,.903)--cycle;
\draw(-9.242,.918)--(-9.186,.893)--(-9.231,.879)--(-9.287,.903);
\filldraw[fill opacity=0.8,fill=gray!20,draw=none](-9.276,.907)--(-9.288,.906)--(-9.222,.938)--(-9.222,.922)--cycle;
\draw(-9.276,.907)--(-9.288,.906);
\draw(-9.222,.938)--(-9.222,.922);
\filldraw[fill opacity=0.8,fill=gray!20,draw=none](-9.276,.907)--(-9.288,.906)--(-9.222,.938)--(-9.222,.922)--cycle;
\draw(-9.276,.907)--(-9.288,.906);
\draw(-9.222,.938)--(-9.222,.922);
\filldraw[fill opacity=0.8,fill=gray!20,draw=none](-9.206,.946)--(-9.202,.949)--(-9.147,.925)--(-9.152,.923)--cycle;
\draw(-9.202,.949)--(-9.147,.925);
\filldraw[fill opacity=0.8,fill=gray!20,draw=none](-8.987,1.13)--(-9.005,1.135)--(-8.998,1.143)--cycle;
\draw(-9.005,1.135)--(-8.998,1.143);
\filldraw[fill opacity=0.8,fill=gray!20,draw=none](-8.987,1.13)--(-9.001,1.138)--(-8.998,1.143)--cycle;
\draw(-9.001,1.138)--(-8.998,1.143);
\filldraw[fill opacity=0.8,fill=gray!20,draw=none](-8.992,1.136)--(-9.019,1.119)--(-8.998,1.143)--cycle;
\draw(-9.019,1.119)--(-8.998,1.143);
\filldraw[fill opacity=0.8,fill=gray!20,draw=none](-8.987,1.13)--(-9.005,1.135)--(-9.001,1.138)--cycle;
\draw(-9.005,1.135)--(-9.001,1.138);
\filldraw[fill opacity=0.8,fill=gray!20,draw=none](-8.977,1.128)--(-8.97,1.125)--(-8.976,1.118)--(-8.987,1.13)--cycle;
\draw(-8.97,1.125)--(-8.976,1.118);
\filldraw[fill opacity=0.8,fill=gray!20,draw=none](-8.997,1.133)--(-8.915,1.187)--(-8.969,1.126)--cycle;
\draw(-8.915,1.187)--(-8.969,1.126);
\filldraw[fill opacity=0.8,fill=gray!20,draw=none](-8.998,1.143)--(-8.99,1.135)--(-8.993,1.133)--(-9.046,1.155)--(-9.031,1.165)--cycle;
\draw(-8.99,1.135)--(-8.993,1.133)--(-9.046,1.155)--(-9.031,1.165);
\filldraw[fill opacity=0.8,fill=gray!20,draw=none](-9.222,.936)--(-9.206,.946)--(-9.152,.923)--(-9.204,.901)--(-9.24,.917)--cycle;
\draw(-9.204,.901)--(-9.24,.917);
\filldraw[fill opacity=0.8,fill=gray!20](-8.885,1.138)--(-8.838,1.164)--(-8.826,1.152)--(-8.879,1.132)--cycle;
\filldraw[fill opacity=0.8,fill=gray!20,draw=none](-9.154,1.065)--(-9.149,1.057)--(-9.146,1.042)--cycle;
\filldraw[fill opacity=0.8,fill=gray!20,draw=none](-9.149,1.057)--(-9.146,1.042)--(-9.18,1.044)--(-9.183,1.059)--cycle;
\draw(-9.146,1.042)--(-9.18,1.044);
\filldraw[fill opacity=0.8,fill=gray!20,draw=none](-9.149,1.057)--(-9.146,1.042)--(-9.18,1.044)--(-9.183,1.059)--cycle;
\draw(-9.146,1.042)--(-9.18,1.044);
\filldraw[fill opacity=0.8,fill=gray!20,draw=none](-9.139,1.009)--(-9.15,1.023)--(-9.146,1.042)--(-9.139,1.039)--cycle;
\draw(-9.146,1.042)--(-9.139,1.039);
\filldraw[fill opacity=0.8,fill=gray!20,draw=none](-9.155,1.078)--(-9.164,1.082)--(-9.167,1.1)--(-9.164,1.099)--cycle;
\draw(-9.167,1.1)--(-9.164,1.099);
\filldraw[fill opacity=0.8,fill=gray!20,draw=none](-9.174,1.065)--(-9.179,1.093)--(-9.177,1.093)--(-9.175,1.087)--(-9.174,1.084)--cycle;
\draw(-9.179,1.093)--(-9.177,1.093);
\filldraw[fill opacity=0.8,fill=gray!20,draw=none](-9.094,1.054)--(-9.131,1.056)--(-9.133,1.058)--(-9.064,1.135)--cycle;
\draw(-9.133,1.058)--(-9.064,1.135);
\filldraw[fill opacity=0.8,fill=gray!20,draw=none](-9.146,1.042)--(-9.149,1.057)--(-9.135,1.056)--(-9.131,1.054)--(-9.13,1.04)--cycle;
\draw(-9.131,1.054)--(-9.13,1.04)--(-9.146,1.042);
\filldraw[fill opacity=0.8,fill=gray!20,draw=none](-9.094,1.054)--(-9.131,1.056)--(-9.133,1.058)--(-9.064,1.135)--cycle;
\draw(-9.133,1.058)--(-9.064,1.135);
\filldraw[fill opacity=0.8,fill=gray!20,draw=none](-9.094,1.054)--(-9.131,1.056)--(-9.133,1.058)--(-9.064,1.135)--cycle;
\draw(-9.133,1.058)--(-9.064,1.135);
\filldraw[fill opacity=0.8,fill=gray!20,draw=none](-9.151,1.067)--(-9.149,1.057)--(-9.154,1.065)--(-9.159,1.08)--(-9.155,1.078)--cycle;
\filldraw[fill opacity=0.8,fill=gray!20,draw=none](-9.177,1.085)--(-9.166,1.058)--(-9.183,1.059)--(-9.188,1.09)--(-9.183,1.089)--cycle;
\draw(-9.188,1.09)--(-9.183,1.089);
\filldraw[fill opacity=0.8,fill=gray!20,draw=none](-9.177,1.085)--(-9.166,1.058)--(-9.183,1.059)--(-9.188,1.09)--(-9.183,1.089)--cycle;
\draw(-9.188,1.09)--(-9.183,1.089);
\filldraw[fill opacity=0.8,fill=gray!20,draw=none](-9.175,1.087)--(-9.173,1.082)--(-9.171,1.07)--cycle;
\filldraw[fill opacity=0.8,fill=gray!20,draw=none](-9.177,1.085)--(-9.151,1.067)--(-9.149,1.057)--(-9.166,1.058)--cycle;
\filldraw[fill opacity=0.8,fill=gray!20,draw=none](-9.177,1.085)--(-9.151,1.067)--(-9.149,1.057)--(-9.166,1.058)--cycle;
\filldraw[fill opacity=0.8,fill=gray!20,draw=none](-9.173,1.082)--(-9.155,1.039)--(-9.166,1.043)--cycle;
\draw(-9.155,1.039)--(-9.166,1.043);
\filldraw[fill opacity=0.8,fill=gray!20,draw=none](-9.113,1.249)--(-9.188,1.164)--(-9.224,1.167)--(-9.13,1.272)--cycle;
\draw(-9.224,1.167)--(-9.13,1.272);
\filldraw[fill opacity=0.8,fill=gray!20,draw=none](-9.138,1.295)--(-9.137,1.293)--(-9.148,1.287)--(-9.148,1.287)--cycle;
\draw(-9.137,1.293)--(-9.148,1.287)--(-9.148,1.287);
\filldraw[fill opacity=0.8,fill=gray!20,draw=none](-9.13,1.272)--(-9.145,1.279)--(-9.148,1.287)--(-9.143,1.29)--cycle;
\draw(-9.145,1.279)--(-9.148,1.287)--(-9.143,1.29);
\filldraw[fill opacity=0.8,fill=gray!20,draw=none](-9.138,1.295)--(-9.148,1.287)--(-9.155,1.342)--(-9.147,1.348)--cycle;
\draw(-9.148,1.287)--(-9.155,1.342)--(-9.147,1.348);
\filldraw[fill opacity=0.8,fill=gray!20,draw=none](-9.021,1.396)--(-9.13,1.272)--(-9.145,1.279)--(-9.146,1.285)--(-9.146,1.292)--(-9.035,1.418)--cycle;
\draw(-9.146,1.292)--(-9.035,1.418)--(-9.021,1.396)--(-9.13,1.272);
\filldraw[fill opacity=0.8,fill=gray!20,draw=none](-9.021,1.396)--(-9.13,1.272)--(-9.145,1.279)--(-9.146,1.285)--(-9.146,1.292)--(-9.035,1.418)--cycle;
\draw(-9.146,1.292)--(-9.035,1.418)--(-9.021,1.396)--(-9.13,1.272);
\filldraw[fill opacity=0.8,fill=gray!20,draw=none](-9.021,1.396)--(-9.13,1.272)--(-9.145,1.279)--(-9.146,1.285)--(-9.146,1.292)--(-9.035,1.418)--cycle;
\draw(-9.146,1.292)--(-9.035,1.418)--(-9.021,1.396)--(-9.13,1.272);
\filldraw[fill opacity=0.8,fill=gray!20,draw=none](-9.164,1.099)--(-9.167,1.1)--(-9.173,1.109)--cycle;
\draw(-9.164,1.099)--(-9.167,1.1);
\filldraw[fill opacity=0.8,fill=gray!20,draw=none](-9.146,1.042)--(-9.151,1.067)--(-9.131,1.054)--(-9.13,1.04)--cycle;
\draw(-9.131,1.054)--(-9.13,1.04)--(-9.146,1.042);
\filldraw[fill opacity=0.8,fill=gray!20,draw=none](-9.141,1.073)--(-9.164,1.1)--(-9.158,1.107)--cycle;
\draw(-9.164,1.1)--(-9.158,1.107);
\filldraw[fill opacity=0.8,fill=gray!20,draw=none](-9.141,1.073)--(-9.164,1.1)--(-9.158,1.107)--cycle;
\draw(-9.164,1.1)--(-9.158,1.107);
\filldraw[fill opacity=0.8,fill=gray!20,draw=none](-9.155,1.076)--(-9.168,1.082)--(-9.174,1.089)--(-9.164,1.1)--cycle;
\draw(-9.174,1.089)--(-9.164,1.1);
\filldraw[fill opacity=0.8,fill=gray!20,draw=none](-9.155,1.076)--(-9.168,1.082)--(-9.174,1.089)--(-9.164,1.1)--cycle;
\draw(-9.174,1.089)--(-9.164,1.1);
\filldraw[fill opacity=0.8,fill=gray!20,draw=none](-9.139,1.069)--(-9.168,1.082)--(-9.174,1.089)--(-9.139,1.128)--cycle;
\draw(-9.174,1.089)--(-9.139,1.128);
\filldraw[fill opacity=0.8,fill=gray!20](-8.746,1.415)--(-8.766,1.465)--(-8.746,1.444)--(-8.724,1.392)--cycle;
\filldraw[fill opacity=0.8,fill=gray!20,draw=none](-9.145,1.279)--(-9.13,1.272)--(-9.14,1.262)--cycle;
\draw(-9.13,1.272)--(-9.14,1.262);
\filldraw[fill opacity=0.8,fill=gray!20,draw=none](-9.145,1.279)--(-9.13,1.272)--(-9.14,1.262)--cycle;
\draw(-9.13,1.272)--(-9.14,1.262);
\filldraw[fill opacity=0.8,fill=gray!20,draw=none](-9.145,1.279)--(-9.13,1.272)--(-9.14,1.262)--cycle;
\draw(-9.13,1.272)--(-9.14,1.262);
\filldraw[fill opacity=0.8,fill=gray!20,draw=none](-9.13,1.272)--(-9.113,1.249)--(-9.111,1.245)--(-9.126,1.235)--(-9.145,1.279)--cycle;
\draw(-9.111,1.245)--(-9.126,1.235)--(-9.145,1.279);
\filldraw[fill opacity=0.8,fill=gray!20,draw=none](-9.196,1.118)--(-9.177,1.092)--(-9.179,1.093)--cycle;
\draw(-9.177,1.092)--(-9.179,1.093);
\filldraw[fill opacity=0.8,fill=gray!20,draw=none](-9.164,1.099)--(-9.173,1.109)--(-9.186,1.129)--cycle;
\filldraw[fill opacity=0.8,fill=gray!20,draw=none](-9.149,1.057)--(-9.143,1.047)--(-9.139,1.039)--(-9.146,1.042)--cycle;
\draw(-9.139,1.039)--(-9.146,1.042);
\filldraw[fill opacity=0.8,fill=gray!20,draw=none](-9.115,.989)--(-9.133,.993)--(-9.13,1.032)--(-9.103,1.024)--(-9.089,1.003)--cycle;
\draw(-9.115,.989)--(-9.133,.993)--(-9.13,1.032);
\filldraw[fill opacity=0.8,fill=gray!20,draw=none](-9.115,.989)--(-9.133,.993)--(-9.13,1.032)--(-9.103,1.024)--(-9.089,1.003)--cycle;
\draw(-9.115,.989)--(-9.133,.993)--(-9.13,1.032);
\filldraw[fill opacity=0.8,fill=gray!20,draw=none](-9.138,.973)--(-9.133,.993)--(-9.115,.989)--cycle;
\draw(-9.138,.973)--(-9.133,.993)--(-9.115,.989);
\filldraw[fill opacity=0.8,fill=gray!20,draw=none](-9.138,.973)--(-9.133,.993)--(-9.115,.989)--cycle;
\draw(-9.138,.973)--(-9.133,.993)--(-9.115,.989);
\filldraw[fill opacity=0.8,fill=gray!20,draw=none](-9.186,1.129)--(-9.175,1.142)--(-9.164,1.1)--(-9.175,1.087)--cycle;
\draw(-9.164,1.1)--(-9.175,1.087);
\filldraw[fill opacity=0.8,fill=gray!20,draw=none](-9.186,1.129)--(-9.175,1.142)--(-9.164,1.1)--(-9.175,1.087)--cycle;
\draw(-9.164,1.1)--(-9.175,1.087);
\filldraw[fill opacity=0.8,fill=gray!20,draw=none](-9.186,1.129)--(-9.175,1.142)--(-9.164,1.1)--(-9.175,1.087)--cycle;
\draw(-9.164,1.1)--(-9.175,1.087);
\filldraw[fill opacity=0.8,fill=gray!20,draw=none](-9.136,1.068)--(-9.155,1.076)--(-9.164,1.1)--cycle;
\filldraw[fill opacity=0.8,fill=gray!20,draw=none](-9.136,1.068)--(-9.155,1.076)--(-9.164,1.1)--cycle;
\filldraw[fill opacity=0.8,fill=gray!20,draw=none](-9.178,1.128)--(-9.163,1.098)--(-9.164,1.099)--(-9.186,1.129)--(-9.196,1.143)--(-9.186,1.139)--cycle;
\draw(-9.163,1.098)--(-9.164,1.099);
\draw(-9.196,1.143)--(-9.186,1.139);
\filldraw[fill opacity=0.8,fill=gray!20,draw=none](-9.155,1.087)--(-9.155,1.078)--(-9.155,1.078)--(-9.164,1.099)--(-9.156,1.095)--cycle;
\draw(-9.164,1.099)--(-9.156,1.095);
\filldraw[fill opacity=0.8,fill=gray!20,draw=none](-9.105,.937)--(-9.143,.947)--(-9.138,.973)--(-9.115,.989)--(-9.073,.978)--cycle;
\draw(-9.105,.937)--(-9.143,.947)--(-9.138,.973);
\draw(-9.115,.989)--(-9.073,.978);
\filldraw[fill opacity=0.8,fill=gray!20,draw=none](-9.105,.937)--(-9.143,.947)--(-9.138,.973)--(-9.115,.989)--(-9.073,.978)--cycle;
\draw(-9.105,.937)--(-9.143,.947)--(-9.138,.973);
\draw(-9.115,.989)--(-9.073,.978);
\filldraw[fill opacity=0.8,fill=gray!20,draw=none](-9.183,1.089)--(-9.188,1.09)--(-9.189,1.092)--cycle;
\draw(-9.183,1.089)--(-9.188,1.09);
\filldraw[fill opacity=0.8,fill=gray!20,draw=none](-9.183,1.089)--(-9.188,1.09)--(-9.189,1.092)--cycle;
\draw(-9.183,1.089)--(-9.188,1.09);
\filldraw[fill opacity=0.8,fill=gray!20,draw=none](-9.06,1.513)--(-9.028,1.538)--(-8.985,1.546)--(-8.994,1.536)--cycle;
\draw(-9.06,1.513)--(-9.028,1.538)--(-8.985,1.546)--(-8.994,1.536);
\filldraw[fill opacity=0.8,fill=gray!20,draw=none](-9.161,1.074)--(-9.177,1.085)--(-9.174,1.089)--cycle;
\draw(-9.177,1.085)--(-9.174,1.089);
\filldraw[fill opacity=0.8,fill=gray!20,draw=none](-9.161,1.074)--(-9.177,1.085)--(-9.174,1.089)--cycle;
\draw(-9.177,1.085)--(-9.174,1.089);
\filldraw[fill opacity=0.8,fill=gray!20,draw=none](-9.161,1.074)--(-9.177,1.085)--(-9.174,1.089)--cycle;
\draw(-9.177,1.085)--(-9.174,1.089);
\filldraw[fill opacity=0.8,fill=gray!20,draw=none](-9.174,1.085)--(-9.177,1.093)--(-9.175,1.091)--(-9.174,1.089)--cycle;
\draw(-9.177,1.093)--(-9.175,1.091);
\filldraw[fill opacity=0.8,fill=gray!20,draw=none](-9.178,1.128)--(-9.155,1.095)--(-9.163,1.098)--cycle;
\draw(-9.155,1.095)--(-9.163,1.098);
\filldraw[fill opacity=0.8,fill=gray!20,draw=none](-9.179,1.089)--(-9.183,1.089)--(-9.189,1.092)--(-9.201,1.113)--cycle;
\draw(-9.179,1.089)--(-9.183,1.089);
\filldraw[fill opacity=0.8,fill=gray!20,draw=none](-9.179,1.089)--(-9.193,1.121)--(-9.186,1.129)--(-9.175,1.087)--(-9.176,1.086)--cycle;
\draw(-9.175,1.087)--(-9.176,1.086);
\filldraw[fill opacity=0.8,fill=gray!20,draw=none](-9.179,1.089)--(-9.183,1.089)--(-9.189,1.092)--(-9.201,1.113)--cycle;
\draw(-9.179,1.089)--(-9.183,1.089);
\filldraw[fill opacity=0.8,fill=gray!20,draw=none](-9.115,.989)--(-9.089,1.003)--(-9.073,.978)--cycle;
\draw(-9.073,.978)--(-9.115,.989);
\filldraw[fill opacity=0.8,fill=gray!20,draw=none](-9.115,.989)--(-9.089,1.003)--(-9.073,.978)--cycle;
\draw(-9.073,.978)--(-9.115,.989);
\filldraw[fill opacity=0.8,fill=gray!20,draw=none](-8.987,1.13)--(-8.972,1.122)--(-9.068,1.014)--(-9.075,1.055)--(-9.005,1.135)--cycle;
\draw(-8.972,1.122)--(-9.068,1.014);
\draw(-9.075,1.055)--(-9.005,1.135);
\filldraw[fill opacity=0.8,fill=gray!20,draw=none](-8.987,1.13)--(-8.976,1.118)--(-9.068,1.014)--(-9.075,1.055)--(-9.005,1.135)--cycle;
\draw(-8.976,1.118)--(-9.068,1.014);
\draw(-9.075,1.055)--(-9.005,1.135);
\filldraw[fill opacity=0.8,fill=gray!20,draw=none](-9.103,1.024)--(-9.13,1.032)--(-9.13,1.04)--(-9.11,1.036)--cycle;
\draw(-9.13,1.032)--(-9.13,1.04)--(-9.11,1.036);
\filldraw[fill opacity=0.8,fill=gray!20,draw=none](-9.103,1.024)--(-9.13,1.032)--(-9.13,1.04)--(-9.11,1.036)--cycle;
\draw(-9.13,1.032)--(-9.13,1.04)--(-9.11,1.036);
\filldraw[fill opacity=0.8,fill=gray!20,draw=none](-9.179,1.089)--(-9.193,1.121)--(-9.186,1.129)--(-9.175,1.087)--(-9.176,1.086)--cycle;
\draw(-9.175,1.087)--(-9.176,1.086);
\filldraw[fill opacity=0.8,fill=gray!20,draw=none](-9.177,1.085)--(-9.183,1.089)--(-9.179,1.089)--cycle;
\draw(-9.183,1.089)--(-9.179,1.089);
\filldraw[fill opacity=0.8,fill=gray!20,draw=none](-9.177,1.085)--(-9.183,1.089)--(-9.179,1.089)--cycle;
\draw(-9.183,1.089)--(-9.179,1.089);
\filldraw[fill opacity=0.8,fill=gray!20,draw=none](-9.201,1.113)--(-9.186,1.129)--(-9.175,1.087)--(-9.176,1.086)--cycle;
\draw(-9.175,1.087)--(-9.176,1.086);
\filldraw[fill opacity=0.8,fill=gray!20,draw=none](-9.135,1.056)--(-9.149,1.057)--(-9.151,1.067)--cycle;
\filldraw[fill opacity=0.8,fill=gray!20,draw=none](-9.151,1.067)--(-9.143,1.047)--(-9.149,1.057)--cycle;
\filldraw[fill opacity=0.8,fill=gray!20,draw=none](-9.201,1.113)--(-9.176,1.086)--(-9.254,.998)--(-9.28,1.023)--cycle;
\draw(-9.176,1.086)--(-9.254,.998)--(-9.28,1.023);
\filldraw[fill opacity=0.8,fill=gray!20,draw=none](-9.201,1.113)--(-9.176,1.086)--(-9.254,.998)--(-9.28,1.023)--cycle;
\draw(-9.176,1.086)--(-9.254,.998)--(-9.28,1.023);
\filldraw[fill opacity=0.8,fill=gray!20,draw=none](-9.201,1.113)--(-9.176,1.086)--(-9.254,.998)--(-9.28,1.023)--cycle;
\draw(-9.176,1.086)--(-9.254,.998)--(-9.28,1.023);
\filldraw[fill opacity=0.8,fill=gray!20,draw=none](-9.174,1.085)--(-9.174,1.089)--(-9.173,1.082)--cycle;
\filldraw[fill opacity=0.8,fill=gray!20,draw=none](-9.157,1.039)--(-9.155,1.039)--(-9.155,1.033)--cycle;
\draw(-9.157,1.039)--(-9.155,1.039);
\filldraw[fill opacity=0.8,fill=gray!20,draw=none](-9.161,1.074)--(-9.148,1.059)--(-9.143,1.047)--(-9.215,.965)--(-9.254,.998)--(-9.177,1.085)--cycle;
\draw(-9.143,1.047)--(-9.215,.965)--(-9.254,.998)--(-9.177,1.085);
\filldraw[fill opacity=0.8,fill=gray!20,draw=none](-9.161,1.074)--(-9.148,1.059)--(-9.143,1.047)--(-9.215,.965)--(-9.254,.998)--(-9.177,1.085)--cycle;
\draw(-9.143,1.047)--(-9.215,.965)--(-9.254,.998)--(-9.177,1.085);
\filldraw[fill opacity=0.8,fill=gray!20,draw=none](-9.161,1.074)--(-9.148,1.059)--(-9.143,1.047)--(-9.215,.965)--(-9.254,.998)--(-9.177,1.085)--cycle;
\draw(-9.143,1.047)--(-9.215,.965)--(-9.254,.998)--(-9.177,1.085);
\filldraw[fill opacity=0.8,fill=gray!20,draw=none](-9.155,1.076)--(-9.148,1.059)--(-9.168,1.082)--cycle;
\filldraw[fill opacity=0.8,fill=gray!20,draw=none](-9.155,1.076)--(-9.148,1.059)--(-9.168,1.082)--cycle;
\filldraw[fill opacity=0.8,fill=gray!20,draw=none](-9.139,1.069)--(-9.139,1.051)--(-9.14,1.05)--(-9.168,1.082)--cycle;
\draw(-9.139,1.051)--(-9.14,1.05);
\filldraw[fill opacity=0.8,fill=gray!20,draw=none](-9.161,1.074)--(-9.177,1.085)--(-9.179,1.089)--(-9.173,1.088)--cycle;
\draw(-9.179,1.089)--(-9.173,1.088);
\filldraw[fill opacity=0.8,fill=gray!20,draw=none](-9.161,1.074)--(-9.177,1.085)--(-9.179,1.089)--(-9.173,1.088)--cycle;
\draw(-9.179,1.089)--(-9.173,1.088);
\filldraw[fill opacity=0.8,fill=gray!20,draw=none](-9.123,1.039)--(-9.13,1.04)--(-9.131,1.054)--cycle;
\draw(-9.123,1.039)--(-9.13,1.04)--(-9.131,1.054);
\filldraw[fill opacity=0.8,fill=gray!20,draw=none](-9.179,1.089)--(-9.201,1.113)--(-9.193,1.121)--cycle;
\filldraw[fill opacity=0.8,fill=gray!20,draw=none](-9.179,1.089)--(-9.201,1.113)--(-9.193,1.121)--cycle;
\filldraw[fill opacity=0.8,fill=gray!20,draw=none](-9.194,1.116)--(-9.196,1.118)--(-9.205,1.132)--(-9.202,1.131)--cycle;
\draw(-9.205,1.132)--(-9.202,1.131);
\filldraw[fill opacity=0.8,fill=gray!20,draw=none](-9.194,1.116)--(-9.202,1.131)--(-9.175,1.091)--(-9.177,1.092)--cycle;
\draw(-9.175,1.091)--(-9.177,1.092);
\filldraw[fill opacity=0.8,fill=gray!20,draw=none](-9.155,1.087)--(-9.179,1.089)--(-9.201,1.113)--(-9.21,1.13)--(-9.155,1.126)--cycle;
\draw(-9.155,1.087)--(-9.179,1.089);
\draw(-9.21,1.13)--(-9.155,1.126);
\filldraw[fill opacity=0.8,fill=gray!20,draw=none](-9.052,1.05)--(-9.155,1.095)--(-9.186,1.139)--(-9.08,1.093)--cycle;
\draw(-9.052,1.05)--(-9.155,1.095);
\draw(-9.186,1.139)--(-9.08,1.093);
\filldraw[fill opacity=0.8,fill=gray!20,draw=none](-9.178,1.128)--(-9.21,1.13)--(-9.222,1.141)--(-9.223,1.162)--(-9.194,1.159)--cycle;
\draw(-9.178,1.128)--(-9.21,1.13);
\draw(-9.222,1.141)--(-9.223,1.162)--(-9.194,1.159);
\filldraw[fill opacity=0.8,fill=gray!20,draw=none](-9.123,1.039)--(-9.13,1.04)--(-9.131,1.054)--cycle;
\draw(-9.123,1.039)--(-9.13,1.04)--(-9.131,1.054);
\filldraw[fill opacity=0.8,fill=gray!20,draw=none](-9.136,1.068)--(-9.13,1.061)--(-9.135,1.056)--(-9.151,1.067)--(-9.155,1.076)--cycle;
\draw(-9.13,1.061)--(-9.135,1.056);
\filldraw[fill opacity=0.8,fill=gray!20,draw=none](-9.136,1.068)--(-9.13,1.061)--(-9.135,1.056)--(-9.151,1.067)--(-9.155,1.076)--cycle;
\draw(-9.13,1.061)--(-9.135,1.056);
\filldraw[fill opacity=0.8,fill=gray!20,draw=none](-9.13,1.061)--(-9.135,1.056)--(-9.139,1.059)--(-9.139,1.071)--cycle;
\draw(-9.13,1.061)--(-9.135,1.056);
\filldraw[fill opacity=0.8,fill=gray!20,draw=none](-9.128,1.064)--(-9.13,1.061)--(-9.136,1.068)--cycle;
\draw(-9.128,1.064)--(-9.13,1.061);
\filldraw[fill opacity=0.8,fill=gray!20,draw=none](-9.128,1.064)--(-9.13,1.061)--(-9.136,1.068)--cycle;
\draw(-9.128,1.064)--(-9.13,1.061);
\filldraw[fill opacity=0.8,fill=gray!20,draw=none](-9.128,1.064)--(-9.13,1.061)--(-9.136,1.068)--cycle;
\draw(-9.128,1.064)--(-9.13,1.061);
\filldraw[fill opacity=0.8,fill=gray!20,draw=none](-9.131,1.056)--(-9.135,1.056)--(-9.133,1.058)--cycle;
\draw(-9.135,1.056)--(-9.133,1.058);
\filldraw[fill opacity=0.8,fill=gray!20,draw=none](-9.131,1.056)--(-9.135,1.056)--(-9.133,1.058)--cycle;
\draw(-9.135,1.056)--(-9.133,1.058);
\filldraw[fill opacity=0.8,fill=gray!20,draw=none](-9.131,1.056)--(-9.135,1.056)--(-9.133,1.058)--cycle;
\draw(-9.135,1.056)--(-9.133,1.058);
\filldraw[fill opacity=0.8,fill=gray!20,draw=none](-9.151,1.067)--(-9.155,1.087)--(-9.133,1.086)--(-9.131,1.054)--cycle;
\draw(-9.155,1.087)--(-9.133,1.086)--(-9.131,1.054);
\filldraw[fill opacity=0.8,fill=gray!20,draw=none](-9.135,1.056)--(-9.139,1.051)--(-9.139,1.059)--cycle;
\draw(-9.135,1.056)--(-9.139,1.051);
\filldraw[fill opacity=0.8,fill=gray!20,draw=none](-9.135,1.056)--(-9.143,1.047)--(-9.151,1.067)--cycle;
\draw(-9.135,1.056)--(-9.143,1.047);
\filldraw[fill opacity=0.8,fill=gray!20,draw=none](-9.135,1.056)--(-9.143,1.047)--(-9.151,1.067)--cycle;
\draw(-9.135,1.056)--(-9.143,1.047);
\filldraw[fill opacity=0.8,fill=gray!20,draw=none](-9.135,1.056)--(-9.151,1.067)--(-9.155,1.087)--(-9.133,1.086)--(-9.131,1.056)--cycle;
\draw(-9.155,1.087)--(-9.133,1.086)--(-9.131,1.056);
\filldraw[fill opacity=0.8,fill=gray!20,draw=none](-9.148,1.059)--(-9.14,1.05)--(-9.143,1.047)--cycle;
\draw(-9.14,1.05)--(-9.143,1.047);
\filldraw[fill opacity=0.8,fill=gray!20,draw=none](-9.135,1.056)--(-9.131,1.056)--(-9.131,1.054)--cycle;
\draw(-9.131,1.056)--(-9.131,1.054);
\filldraw[fill opacity=0.8,fill=gray!20,draw=none](-9.11,1.036)--(-9.123,1.039)--(-9.131,1.054)--(-9.131,1.056)--cycle;
\draw(-9.11,1.036)--(-9.123,1.039);
\draw(-9.131,1.054)--(-9.131,1.056);
\filldraw[fill opacity=0.8,fill=gray!20,draw=none](-9.11,1.036)--(-9.123,1.039)--(-9.131,1.054)--(-9.131,1.056)--cycle;
\draw(-9.11,1.036)--(-9.123,1.039);
\draw(-9.131,1.054)--(-9.131,1.056);
\filldraw[fill opacity=0.8,fill=gray!20,draw=none](-9.143,1.047)--(-9.137,1.038)--(-9.139,1.039)--cycle;
\draw(-9.137,1.038)--(-9.139,1.039);
\filldraw[fill opacity=0.8,fill=gray!20,draw=none](-9.131,1.056)--(-9.103,1.029)--(-9.107,1.019)--(-9.129,.995)--(-9.139,1.051)--(-9.135,1.056)--cycle;
\draw(-9.107,1.019)--(-9.129,.995);
\draw(-9.139,1.051)--(-9.135,1.056);
\filldraw[fill opacity=0.8,fill=gray!20,draw=none](-9.094,1.054)--(-9.107,1.019)--(-9.129,.995)--(-9.139,1.051)--(-9.135,1.056)--cycle;
\draw(-9.107,1.019)--(-9.129,.995);
\draw(-9.139,1.051)--(-9.135,1.056);
\filldraw[fill opacity=0.8,fill=gray!20,draw=none](-9.094,1.054)--(-9.107,1.019)--(-9.129,.995)--(-9.139,1.051)--(-9.135,1.056)--cycle;
\draw(-9.107,1.019)--(-9.129,.995);
\draw(-9.139,1.051)--(-9.135,1.056);
\filldraw[fill opacity=0.8,fill=gray!20,draw=none](-9.131,1.056)--(-9.094,1.054)--(-9.103,1.029)--cycle;
\filldraw[fill opacity=0.8,fill=gray!20,draw=none](-9.068,1.014)--(-9.103,1.024)--(-9.075,1.055)--cycle;
\draw(-9.103,1.024)--(-9.075,1.055);
\filldraw[fill opacity=0.8,fill=gray!20,draw=none](-8.997,1.095)--(-9.068,1.014)--(-9.075,1.055)--(-9.071,1.06)--cycle;
\draw(-8.997,1.095)--(-9.068,1.014);
\draw(-9.075,1.055)--(-9.071,1.06);
\filldraw[fill opacity=0.8,fill=gray!20,draw=none](-9.068,1.014)--(-9.103,1.024)--(-9.075,1.055)--cycle;
\draw(-9.103,1.024)--(-9.075,1.055);
\filldraw[fill opacity=0.8,fill=gray!20,draw=none](-9.068,1.014)--(-9.103,1.024)--(-9.075,1.055)--cycle;
\draw(-9.103,1.024)--(-9.075,1.055);
\filldraw[fill opacity=0.8,fill=gray!20,draw=none](-9.11,1.036)--(-9.131,1.056)--(-9.133,1.086)--(-9.073,1.071)--(-9.067,1.025)--cycle;
\draw(-9.131,1.056)--(-9.133,1.086)--(-9.073,1.071)--(-9.067,1.025)--(-9.11,1.036);
\filldraw[fill opacity=0.8,fill=gray!20,draw=none](-9.11,1.036)--(-9.131,1.056)--(-9.133,1.086)--(-9.073,1.071)--(-9.067,1.025)--cycle;
\draw(-9.131,1.056)--(-9.133,1.086)--(-9.073,1.071)--(-9.067,1.025)--(-9.11,1.036);
\filldraw[fill opacity=0.8,fill=gray!20,draw=none](-9.103,1.024)--(-9.107,1.019)--(-9.104,1.027)--cycle;
\draw(-9.103,1.024)--(-9.107,1.019);
\filldraw[fill opacity=0.8,fill=gray!20,draw=none](-9.103,1.024)--(-9.107,1.019)--(-9.104,1.027)--cycle;
\draw(-9.103,1.024)--(-9.107,1.019);
\filldraw[fill opacity=0.8,fill=gray!20,draw=none](-9.103,1.024)--(-9.107,1.019)--(-9.104,1.027)--cycle;
\draw(-9.103,1.024)--(-9.107,1.019);
\filldraw[fill opacity=0.8,fill=gray!20,draw=none](-9.089,1.003)--(-9.103,1.024)--(-9.068,1.014)--cycle;
\filldraw[fill opacity=0.8,fill=gray!20,draw=none](-9.089,1.003)--(-9.103,1.024)--(-9.068,1.014)--cycle;
\filldraw[fill opacity=0.8,fill=gray!20,draw=none](-9.089,1.003)--(-9.068,1.014)--(-9.073,.978)--cycle;
\draw(-9.068,1.014)--(-9.073,.978);
\filldraw[fill opacity=0.8,fill=gray!20,draw=none](-9.089,1.003)--(-9.068,1.014)--(-9.073,.978)--cycle;
\draw(-9.068,1.014)--(-9.073,.978);
\filldraw[fill opacity=0.8,fill=gray!20,draw=none](-9.068,1.014)--(-9.109,.968)--(-9.129,.995)--(-9.103,1.024)--cycle;
\draw(-9.068,1.014)--(-9.109,.968);
\draw(-9.129,.995)--(-9.103,1.024);
\filldraw[fill opacity=0.8,fill=gray!20,draw=none](-9.109,.968)--(-9.113,.967)--(-9.139,.979)--(-9.139,1.009)--cycle;
\draw(-9.113,.967)--(-9.139,.979);
\filldraw[fill opacity=0.8,fill=gray!20,draw=none](-9.068,1.014)--(-9.109,.968)--(-9.129,.995)--(-9.103,1.024)--cycle;
\draw(-9.068,1.014)--(-9.109,.968);
\draw(-9.129,.995)--(-9.103,1.024);
\filldraw[fill opacity=0.8,fill=gray!20,draw=none](-9.068,1.014)--(-9.109,.968)--(-9.129,.995)--(-9.103,1.024)--cycle;
\draw(-9.068,1.014)--(-9.109,.968);
\draw(-9.129,.995)--(-9.103,1.024);
\filldraw[fill opacity=0.8,fill=gray!20,draw=none](-9.073,.978)--(-9.068,1.014)--(-9.058,1.004)--cycle;
\draw(-9.073,.978)--(-9.068,1.014);
\filldraw[fill opacity=0.8,fill=gray!20,draw=none](-9.073,.978)--(-9.068,1.014)--(-9.058,1.004)--cycle;
\draw(-9.073,.978)--(-9.068,1.014);
\filldraw[fill opacity=0.8,fill=gray!20,draw=none](-9.105,.937)--(-9.073,.978)--(-9.089,.933)--cycle;
\draw(-9.073,.978)--(-9.089,.933)--(-9.105,.937);
\filldraw[fill opacity=0.8,fill=gray!20,draw=none](-9.105,.937)--(-9.073,.978)--(-9.089,.933)--cycle;
\draw(-9.073,.978)--(-9.089,.933)--(-9.105,.937);
\filldraw[fill opacity=0.8,fill=gray!20](-9.089,.933)--(-9.073,.978)--(-9.054,.959)--(-9.072,.916)--cycle;
\filldraw[fill opacity=0.8,fill=gray!20](-9.089,.933)--(-9.073,.978)--(-9.054,.959)--(-9.072,.916)--cycle;
\filldraw[fill opacity=0.8,fill=gray!20,draw=none](-9.058,1.004)--(-9.084,.975)--(-9.104,.974)--(-9.068,1.014)--cycle;
\draw(-9.058,1.004)--(-9.084,.975);
\draw(-9.104,.974)--(-9.068,1.014);
\filldraw[fill opacity=0.8,fill=gray!20,draw=none](-9.084,.975)--(-9.095,.959)--(-9.113,.967)--cycle;
\draw(-9.095,.959)--(-9.113,.967);
\filldraw[fill opacity=0.8,fill=gray!20,draw=none](-9.137,.899)--(-9.105,.937)--(-9.089,.933)--(-9.115,.894)--cycle;
\draw(-9.105,.937)--(-9.089,.933)--(-9.115,.894)--(-9.137,.899);
\filldraw[fill opacity=0.8,fill=gray!20](-9.115,.894)--(-9.089,.933)--(-9.072,.916)--(-9.101,.879)--cycle;
\filldraw[fill opacity=0.8,fill=gray!20](-9.115,.894)--(-9.089,.933)--(-9.072,.916)--(-9.101,.879)--cycle;
\filldraw[fill opacity=0.8,fill=gray!20,draw=none](-9.058,1.004)--(-9.062,.999)--(-9.102,.974)--(-9.104,.974)--(-9.068,1.014)--cycle;
\draw(-9.058,1.004)--(-9.062,.999);
\draw(-9.104,.974)--(-9.068,1.014);
\filldraw[fill opacity=0.8,fill=gray!20,draw=none](-9.106,.945)--(-9.106,.964)--(-9.095,.959)--cycle;
\draw(-9.106,.964)--(-9.095,.959);
\filldraw[fill opacity=0.8,fill=gray!20,draw=none](-9.084,.975)--(-9.109,.968)--(-9.104,.974)--cycle;
\draw(-9.109,.968)--(-9.104,.974);
\filldraw[fill opacity=0.8,fill=gray!20,draw=none](-9.058,1.004)--(-9.084,.975)--(-9.104,.974)--(-9.068,1.014)--cycle;
\draw(-9.058,1.004)--(-9.084,.975);
\draw(-9.104,.974)--(-9.068,1.014);
\filldraw[fill opacity=0.8,fill=gray!20,draw=none](-9.102,.974)--(-9.062,.999)--(-9.084,.975)--cycle;
\draw(-9.062,.999)--(-9.084,.975);
\filldraw[fill opacity=0.8,fill=gray!20,draw=none](-9.074,.989)--(-9.139,1.018)--(-9.139,1.039)--(-9.063,1.005)--cycle;
\draw(-9.139,1.039)--(-9.063,1.005);
\filldraw[fill opacity=0.8,fill=gray!20,draw=none](-9.074,.989)--(-9.084,.975)--(-9.129,.995)--(-9.139,1.009)--(-9.139,1.018)--cycle;
\filldraw[fill opacity=0.8,fill=gray!20,draw=none](-9.129,.995)--(-9.154,1.035)--(-9.139,1.051)--cycle;
\draw(-9.154,1.035)--(-9.139,1.051);
\filldraw[fill opacity=0.8,fill=gray!20,draw=none](-9.129,.995)--(-9.154,1.035)--(-9.139,1.051)--cycle;
\draw(-9.154,1.035)--(-9.139,1.051);
\filldraw[fill opacity=0.8,fill=gray!20,draw=none](-9.133,1.015)--(-9.136,1.006)--(-9.154,1.035)--(-9.139,1.051)--cycle;
\draw(-9.154,1.035)--(-9.139,1.051);
\filldraw[fill opacity=0.8,fill=gray!20,draw=none](-9.101,1.026)--(-9.103,1.024)--(-9.104,1.027)--(-9.103,1.029)--cycle;
\draw(-9.101,1.026)--(-9.103,1.024);
\filldraw[fill opacity=0.8,fill=gray!20,draw=none](-9.101,1.026)--(-9.103,1.024)--(-9.104,1.027)--(-9.103,1.029)--cycle;
\draw(-9.101,1.026)--(-9.103,1.024);
\filldraw[fill opacity=0.8,fill=gray!20,draw=none](-9.101,1.026)--(-9.103,1.024)--(-9.104,1.027)--(-9.103,1.029)--cycle;
\draw(-9.101,1.026)--(-9.103,1.024);
\filldraw[fill opacity=0.8,fill=gray!20,draw=none](-9.103,1.024)--(-9.11,1.036)--(-9.067,1.025)--(-9.068,1.014)--cycle;
\draw(-9.11,1.036)--(-9.067,1.025)--(-9.068,1.014);
\filldraw[fill opacity=0.8,fill=gray!20,draw=none](-9.103,1.024)--(-9.11,1.036)--(-9.067,1.025)--(-9.068,1.014)--cycle;
\draw(-9.11,1.036)--(-9.067,1.025)--(-9.068,1.014);
\filldraw[fill opacity=0.8,fill=gray!20,draw=none](-8.961,1.121)--(-8.957,1.118)--(-9.058,1.004)--(-9.068,1.014)--(-8.976,1.118)--cycle;
\draw(-8.957,1.118)--(-9.058,1.004);
\draw(-9.068,1.014)--(-8.976,1.118);
\filldraw[fill opacity=0.8,fill=gray!20,draw=none](-8.961,1.121)--(-8.957,1.118)--(-9.058,1.004)--(-9.068,1.014)--(-8.976,1.118)--cycle;
\draw(-8.957,1.118)--(-9.058,1.004);
\draw(-9.068,1.014)--(-8.976,1.118);
\filldraw[fill opacity=0.8,fill=gray!20,draw=none](-8.961,1.121)--(-8.957,1.118)--(-9.058,1.004)--(-9.068,1.014)--(-8.976,1.118)--cycle;
\draw(-8.957,1.118)--(-9.058,1.004);
\draw(-9.068,1.014)--(-8.976,1.118);
\filldraw[fill opacity=0.8,fill=gray!20,draw=none](-9.058,1.004)--(-9.068,1.014)--(-9.067,1.025)--(-9.054,1.011)--cycle;
\draw(-9.068,1.014)--(-9.067,1.025)--(-9.054,1.011);
\filldraw[fill opacity=0.8,fill=gray!20,draw=none](-9.058,1.004)--(-9.068,1.014)--(-9.067,1.025)--(-9.054,1.011)--cycle;
\draw(-9.068,1.014)--(-9.067,1.025)--(-9.054,1.011);
\filldraw[fill opacity=0.8,fill=gray!20,draw=none](-8.969,1.126)--(-8.97,1.125)--(-8.977,1.128)--cycle;
\draw(-8.969,1.126)--(-8.97,1.125);
\filldraw[fill opacity=0.8,fill=gray!20,draw=none](-8.969,1.126)--(-8.97,1.125)--(-8.977,1.128)--cycle;
\draw(-8.969,1.126)--(-8.97,1.125);
\filldraw[fill opacity=0.8,fill=gray!20,draw=none](-8.997,1.133)--(-8.969,1.126)--(-8.997,1.095)--(-9.071,1.06)--(-9.019,1.119)--cycle;
\draw(-8.969,1.126)--(-8.997,1.095);
\draw(-9.071,1.06)--(-9.019,1.119);
\filldraw[fill opacity=0.8,fill=gray!20,draw=none](-9.054,1.011)--(-9.067,1.025)--(-9.073,1.071)--(-9.054,1.051)--(-9.05,1.021)--cycle;
\draw(-9.054,1.011)--(-9.067,1.025)--(-9.073,1.071)--(-9.054,1.051)--(-9.05,1.021);
\filldraw[fill opacity=0.8,fill=gray!20,draw=none](-9.054,1.011)--(-9.067,1.025)--(-9.073,1.071)--(-9.054,1.051)--(-9.05,1.021)--cycle;
\draw(-9.054,1.011)--(-9.067,1.025)--(-9.073,1.071)--(-9.054,1.051)--(-9.05,1.021);
\filldraw[fill opacity=0.8,fill=gray!20,draw=none](-9.143,1.047)--(-9.155,1.078)--(-9.06,1.036)--(-9.058,1.027)--(-9.059,1.011)--(-9.063,1.005)--(-9.137,1.038)--cycle;
\draw(-9.063,1.005)--(-9.137,1.038);
\filldraw[fill opacity=0.8,fill=gray!20,draw=none](-9.155,1.087)--(-9.153,1.077)--(-9.155,1.078)--cycle;
\filldraw[fill opacity=0.8,fill=gray!20,draw=none](-9.151,1.067)--(-9.161,1.074)--(-9.173,1.088)--(-9.155,1.087)--cycle;
\draw(-9.173,1.088)--(-9.155,1.087);
\filldraw[fill opacity=0.8,fill=gray!20,draw=none](-9.151,1.067)--(-9.161,1.074)--(-9.173,1.088)--(-9.155,1.087)--cycle;
\draw(-9.173,1.088)--(-9.155,1.087);
\filldraw[fill opacity=0.8,fill=gray!20,draw=none](-9.155,1.087)--(-9.156,1.095)--(-9.155,1.095)--cycle;
\draw(-9.156,1.095)--(-9.155,1.095);
\filldraw[fill opacity=0.8,fill=gray!20,draw=none](-9.155,1.104)--(-9.155,1.087)--(-9.174,1.089)--(-9.175,1.115)--cycle;
\draw(-9.155,1.087)--(-9.174,1.089);
\filldraw[fill opacity=0.8,fill=gray!20,draw=none](-9.155,1.087)--(-9.155,1.104)--(-9.135,1.092)--(-9.133,1.086)--cycle;
\draw(-9.135,1.092)--(-9.133,1.086)--(-9.155,1.087);
\filldraw[fill opacity=0.8,fill=gray!20,draw=none](-9.155,1.087)--(-9.155,1.104)--(-9.135,1.092)--(-9.133,1.086)--cycle;
\draw(-9.135,1.092)--(-9.133,1.086)--(-9.155,1.087);
\filldraw[fill opacity=0.8,fill=gray!20,draw=none](-9.061,1.046)--(-9.06,1.036)--(-9.153,1.077)--(-9.155,1.087)--(-9.155,1.095)--(-9.064,1.055)--cycle;
\draw(-9.155,1.095)--(-9.064,1.055);
\filldraw[fill opacity=0.8,fill=gray!20,draw=none](-9.173,1.082)--(-9.174,1.089)--(-9.175,1.091)--(-9.122,1.068)--(-9.113,1.02)--(-9.155,1.039)--cycle;
\draw(-9.175,1.091)--(-9.122,1.068)--(-9.113,1.02)--(-9.155,1.039);
\filldraw[fill opacity=0.8,fill=gray!20,draw=none](-9.175,1.115)--(-9.174,1.089)--(-9.179,1.089)--(-9.201,1.113)--(-9.21,1.13)--(-9.199,1.129)--cycle;
\draw(-9.174,1.089)--(-9.179,1.089);
\draw(-9.21,1.13)--(-9.199,1.129);
\filldraw[fill opacity=0.8,fill=gray!20](-8.936,1.125)--(-8.885,1.138)--(-8.879,1.132)--(-8.936,1.125)--cycle;
\filldraw[fill opacity=0.8,fill=gray!20,draw=none](-9.174,1.089)--(-9.175,1.091)--(-9.175,1.091)--cycle;
\draw(-9.175,1.091)--(-9.175,1.091);
\filldraw[fill opacity=0.8,fill=gray!20,draw=none](-9.188,1.164)--(-9.202,1.148)--(-9.227,1.164)--(-9.224,1.167)--cycle;
\draw(-9.227,1.164)--(-9.224,1.167);
\filldraw[fill opacity=0.8,fill=gray!20,draw=none](-9.188,1.164)--(-9.202,1.148)--(-9.227,1.164)--(-9.224,1.167)--cycle;
\draw(-9.227,1.164)--(-9.224,1.167);
\filldraw[fill opacity=0.8,fill=gray!20,draw=none](-9.188,1.164)--(-9.202,1.148)--(-9.227,1.164)--(-9.224,1.167)--cycle;
\draw(-9.227,1.164)--(-9.224,1.167);
\filldraw[fill opacity=0.8,fill=gray!20](-9.133,1.086)--(-9.143,1.125)--(-9.089,1.112)--(-9.073,1.071)--cycle;
\filldraw[fill opacity=0.8,fill=gray!20](-9.133,1.086)--(-9.143,1.125)--(-9.089,1.112)--(-9.073,1.071)--cycle;
\filldraw[fill opacity=0.8,fill=gray!20,draw=none](-9.145,1.279)--(-9.14,1.262)--(-9.224,1.167)--(-9.232,1.167)--(-9.247,1.178)--(-9.154,1.283)--cycle;
\draw(-9.14,1.262)--(-9.224,1.167);
\draw(-9.247,1.178)--(-9.154,1.283);
\filldraw[fill opacity=0.8,fill=gray!20,draw=none](-9.145,1.279)--(-9.14,1.262)--(-9.224,1.167)--(-9.232,1.167)--(-9.247,1.178)--(-9.154,1.283)--cycle;
\draw(-9.14,1.262)--(-9.224,1.167);
\draw(-9.247,1.178)--(-9.154,1.283);
\filldraw[fill opacity=0.8,fill=gray!20,draw=none](-9.223,1.162)--(-9.183,1.138)--(-9.196,1.143)--(-9.205,1.149)--cycle;
\draw(-9.183,1.138)--(-9.196,1.143);
\filldraw[fill opacity=0.8,fill=gray!20,draw=none](-9.22,1.142)--(-9.214,1.134)--(-9.222,1.141)--(-9.222,1.143)--cycle;
\draw(-9.222,1.141)--(-9.222,1.143);
\filldraw[fill opacity=0.8,fill=gray!20,draw=none](-9.235,1.145)--(-9.236,1.152)--(-9.222,1.143)--(-9.222,1.141)--cycle;
\draw(-9.222,1.143)--(-9.222,1.141);
\filldraw[fill opacity=0.8,fill=gray!20,draw=none](-9.235,1.145)--(-9.264,1.152)--(-9.26,1.157)--(-9.24,1.155)--(-9.236,1.152)--cycle;
\filldraw[fill opacity=0.8,fill=gray!20,draw=none](-9.235,1.145)--(-9.236,1.152)--(-9.222,1.143)--(-9.222,1.141)--cycle;
\draw(-9.222,1.143)--(-9.222,1.141);
\filldraw[fill opacity=0.8,fill=gray!20,draw=none](-9.235,1.145)--(-9.264,1.152)--(-9.26,1.157)--(-9.24,1.155)--(-9.236,1.152)--cycle;
\filldraw[fill opacity=0.8,fill=gray!20,draw=none](-9.212,1.137)--(-9.297,1.041)--(-9.316,1.064)--(-9.237,1.153)--cycle;
\draw(-9.297,1.041)--(-9.316,1.064)--(-9.237,1.153);
\filldraw[fill opacity=0.8,fill=gray!20,draw=none](-9.212,1.137)--(-9.297,1.041)--(-9.316,1.064)--(-9.237,1.153)--cycle;
\draw(-9.297,1.041)--(-9.316,1.064)--(-9.237,1.153);
\filldraw[fill opacity=0.8,fill=gray!20,draw=none](-9.212,1.137)--(-9.297,1.041)--(-9.316,1.064)--(-9.237,1.153)--cycle;
\draw(-9.297,1.041)--(-9.316,1.064)--(-9.237,1.153);
\filldraw[fill opacity=0.8,fill=gray!20,draw=none](-9.22,1.142)--(-9.199,1.129)--(-9.21,1.13)--(-9.214,1.134)--cycle;
\draw(-9.199,1.129)--(-9.21,1.13);
\filldraw[fill opacity=0.8,fill=gray!20,draw=none](-9.155,1.104)--(-9.155,1.122)--(-9.14,1.113)--(-9.135,1.092)--cycle;
\draw(-9.14,1.113)--(-9.135,1.092);
\filldraw[fill opacity=0.8,fill=gray!20,draw=none](-9.155,1.104)--(-9.155,1.122)--(-9.14,1.113)--(-9.135,1.092)--cycle;
\draw(-9.14,1.113)--(-9.135,1.092);
\filldraw[fill opacity=0.8,fill=gray!20,draw=none](-9.155,1.122)--(-9.155,1.104)--(-9.175,1.115)--(-9.175,1.128)--(-9.164,1.127)--cycle;
\draw(-9.175,1.128)--(-9.164,1.127);
\filldraw[fill opacity=0.8,fill=gray!20,draw=none](-9.175,1.115)--(-9.199,1.129)--(-9.175,1.128)--cycle;
\draw(-9.199,1.129)--(-9.175,1.128);
\filldraw[fill opacity=0.8,fill=gray!20,draw=none](-9.202,1.131)--(-9.147,1.107)--(-9.122,1.068)--(-9.175,1.091)--cycle;
\draw(-9.202,1.131)--(-9.147,1.107)--(-9.122,1.068)--(-9.175,1.091);
\filldraw[fill opacity=0.8,fill=gray!20,draw=none](-9.328,.891)--(-9.338,.895)--(-9.288,.889)--(-9.279,.885)--cycle;
\draw(-9.328,.891)--(-9.338,.895);
\draw(-9.288,.889)--(-9.279,.885);
\filldraw[fill opacity=0.8,fill=gray!20](-8.929,1.549)--(-8.933,1.559)--(-8.905,1.557)--(-8.876,1.545)--cycle;
\filldraw[fill opacity=0.8,fill=gray!20](-8.985,1.546)--(-8.961,1.557)--(-8.933,1.559)--(-8.929,1.549)--cycle;
\filldraw[fill opacity=0.8,fill=gray!20,draw=none](-9.288,.904)--(-9.288,.903)--(-9.298,.904)--(-9.332,.908)--(-9.333,.909)--cycle;
\draw(-9.288,.904)--(-9.288,.903);
\draw(-9.332,.908)--(-9.333,.909);
\filldraw[fill opacity=0.8,fill=gray!20,draw=none](-9.298,.904)--(-9.33,.907)--(-9.353,.921)--(-9.355,.924)--(-9.288,.906)--cycle;
\draw(-9.353,.921)--(-9.355,.924);
\draw(-9.288,.906)--(-9.298,.904);
\filldraw[fill opacity=0.8,fill=gray!20,draw=none](-9.288,.903)--(-9.287,.903)--(-9.298,.904)--cycle;
\draw(-9.288,.903)--(-9.287,.903);
\filldraw[fill opacity=0.8,fill=gray!20,draw=none](-9.319,.9)--(-9.288,.906)--(-9.273,.874)--cycle;
\draw(-9.319,.9)--(-9.288,.906)--(-9.273,.874);
\filldraw[fill opacity=0.8,fill=gray!20,draw=none](-9.319,.9)--(-9.288,.906)--(-9.273,.874)--cycle;
\draw(-9.319,.9)--(-9.288,.906)--(-9.273,.874);
\filldraw[fill opacity=0.8,fill=gray!20,draw=none](-9.298,.904)--(-9.33,.907)--(-9.353,.921)--(-9.355,.924)--(-9.288,.906)--cycle;
\draw(-9.353,.921)--(-9.355,.924);
\draw(-9.288,.906)--(-9.298,.904);
\filldraw[fill opacity=0.8,fill=gray!20,draw=none](-9.155,1.033)--(-9.146,.98)--(-9.155,.985)--cycle;
\draw(-9.146,.98)--(-9.155,.985);
\filldraw[fill opacity=0.8,fill=gray!20,draw=none](-9.148,.973)--(-9.148,.981)--(-9.146,.98)--cycle;
\draw(-9.148,.981)--(-9.146,.98);
\filldraw[fill opacity=0.8,fill=gray!20,draw=none](-9.136,1.006)--(-9.148,.973)--(-9.179,.938)--(-9.215,.965)--(-9.154,1.035)--cycle;
\draw(-9.148,.973)--(-9.179,.938)--(-9.215,.965)--(-9.154,1.035);
\filldraw[fill opacity=0.8,fill=gray!20,draw=none](-9.129,.995)--(-9.179,.938)--(-9.215,.965)--(-9.154,1.035)--cycle;
\draw(-9.129,.995)--(-9.179,.938)--(-9.215,.965)--(-9.154,1.035);
\filldraw[fill opacity=0.8,fill=gray!20,draw=none](-9.129,.995)--(-9.179,.938)--(-9.215,.965)--(-9.154,1.035)--cycle;
\draw(-9.129,.995)--(-9.179,.938)--(-9.215,.965)--(-9.154,1.035);
\filldraw[fill opacity=0.8,fill=gray!20,draw=none](-9.133,1.015)--(-9.129,.995)--(-9.136,1.006)--cycle;
\filldraw[fill opacity=0.8,fill=gray!20,draw=none](-9.136,1.006)--(-9.129,.995)--(-9.148,.973)--cycle;
\draw(-9.129,.995)--(-9.148,.973);
\filldraw[fill opacity=0.8,fill=gray!20,draw=none](-9.109,.968)--(-9.125,.95)--(-9.145,.977)--(-9.129,.995)--cycle;
\draw(-9.109,.968)--(-9.125,.95);
\draw(-9.145,.977)--(-9.129,.995);
\filldraw[fill opacity=0.8,fill=gray!20,draw=none](-9.109,.968)--(-9.125,.95)--(-9.145,.977)--(-9.129,.995)--cycle;
\draw(-9.109,.968)--(-9.125,.95);
\draw(-9.145,.977)--(-9.129,.995);
\filldraw[fill opacity=0.8,fill=gray!20,draw=none](-9.109,.968)--(-9.125,.95)--(-9.145,.977)--(-9.129,.995)--cycle;
\draw(-9.109,.968)--(-9.125,.95);
\draw(-9.145,.977)--(-9.129,.995);
\filldraw[fill opacity=0.8,fill=gray!20,draw=none](-9.084,.975)--(-9.1,.957)--(-9.125,.95)--(-9.104,.974)--cycle;
\draw(-9.084,.975)--(-9.1,.957);
\draw(-9.125,.95)--(-9.104,.974);
\filldraw[fill opacity=0.8,fill=gray!20,draw=none](-9.102,.974)--(-9.106,.971)--(-9.104,.974)--cycle;
\draw(-9.106,.971)--(-9.104,.974);
\filldraw[fill opacity=0.8,fill=gray!20,draw=none](-9.102,.974)--(-9.084,.975)--(-9.1,.957)--(-9.125,.95)--(-9.106,.971)--cycle;
\draw(-9.084,.975)--(-9.1,.957);
\draw(-9.125,.95)--(-9.106,.971);
\filldraw[fill opacity=0.8,fill=gray!20,draw=none](-9.084,.975)--(-9.109,.968)--(-9.129,.995)--cycle;
\filldraw[fill opacity=0.8,fill=gray!20,draw=none](-9.155,1.033)--(-9.155,1.039)--(-9.113,1.02)--(-9.122,.97)--(-9.146,.98)--cycle;
\draw(-9.155,1.039)--(-9.113,1.02)--(-9.122,.97)--(-9.146,.98);
\filldraw[fill opacity=0.8,fill=gray!20,draw=none](-9.33,.907)--(-9.298,.904)--(-9.319,.9)--cycle;
\draw(-9.298,.904)--(-9.319,.9);
\filldraw[fill opacity=0.8,fill=gray!20,draw=none](-9.33,.907)--(-9.298,.904)--(-9.319,.9)--cycle;
\draw(-9.298,.904)--(-9.319,.9);
\filldraw[fill opacity=0.8,fill=gray!20,draw=none](-9.298,.904)--(-9.287,.903)--(-9.231,.879)--(-9.276,.883)--(-9.33,.907)--cycle;
\draw(-9.287,.903)--(-9.231,.879)--(-9.276,.883)--(-9.33,.907);
\filldraw[fill opacity=0.8,fill=gray!20,draw=none](-8.936,1.125)--(-8.956,1.123)--(-8.969,1.126)--(-8.936,1.125)--cycle;
\draw(-8.969,1.126)--(-8.936,1.125)--(-8.936,1.125)--(-8.956,1.123);
\filldraw[fill opacity=0.8,fill=gray!20,draw=none](-8.963,1.122)--(-8.961,1.121)--(-8.976,1.118)--(-8.97,1.125)--cycle;
\draw(-8.976,1.118)--(-8.97,1.125);
\filldraw[fill opacity=0.8,fill=gray!20,draw=none](-8.963,1.122)--(-8.961,1.121)--(-8.976,1.118)--(-8.97,1.125)--cycle;
\draw(-8.976,1.118)--(-8.97,1.125);
\filldraw[fill opacity=0.8,fill=gray!20,draw=none](-8.963,1.122)--(-8.961,1.121)--(-8.976,1.118)--(-8.97,1.125)--cycle;
\draw(-8.976,1.118)--(-8.97,1.125);
\filldraw[fill opacity=0.8,fill=gray!20,draw=none](-8.952,1.121)--(-8.967,1.122)--(-8.944,1.124)--cycle;
\draw(-8.952,1.121)--(-8.967,1.122)--(-8.944,1.124);
\filldraw[fill opacity=0.8,fill=gray!20,draw=none](-8.97,1.125)--(-8.972,1.122)--(-8.987,1.13)--(-8.99,1.135)--cycle;
\draw(-8.97,1.125)--(-8.972,1.122);
\filldraw[fill opacity=0.8,fill=gray!20,draw=none](-8.963,1.122)--(-8.97,1.125)--(-8.969,1.126)--cycle;
\draw(-8.97,1.125)--(-8.969,1.126);
\filldraw[fill opacity=0.8,fill=gray!20,draw=none](-8.963,1.122)--(-8.97,1.125)--(-8.969,1.126)--cycle;
\draw(-8.97,1.125)--(-8.969,1.126);
\filldraw[fill opacity=0.8,fill=gray!20,draw=none](-8.963,1.122)--(-8.97,1.125)--(-8.969,1.126)--cycle;
\draw(-8.97,1.125)--(-8.969,1.126);
\filldraw[fill opacity=0.8,fill=gray!20,draw=none](-8.956,1.123)--(-8.967,1.122)--(-8.987,1.127)--(-8.969,1.126)--cycle;
\draw(-8.956,1.123)--(-8.967,1.122)--(-8.987,1.127)--(-8.969,1.126);
\filldraw[fill opacity=0.8,fill=gray!20,draw=none](-8.839,1.251)--(-8.952,1.123)--(-8.96,1.121)--(-8.963,1.122)--(-8.969,1.126)--(-8.856,1.254)--cycle;
\draw(-8.969,1.126)--(-8.856,1.254)--(-8.839,1.251)--(-8.952,1.123);
\filldraw[fill opacity=0.8,fill=gray!20,draw=none](-8.839,1.251)--(-8.952,1.123)--(-8.96,1.121)--(-8.963,1.122)--(-8.969,1.126)--(-8.856,1.254)--cycle;
\draw(-8.969,1.126)--(-8.856,1.254)--(-8.839,1.251)--(-8.952,1.123);
\filldraw[fill opacity=0.8,fill=gray!20,draw=none](-8.839,1.251)--(-8.952,1.123)--(-8.96,1.121)--(-8.963,1.122)--(-8.969,1.126)--(-8.856,1.254)--cycle;
\draw(-8.969,1.126)--(-8.856,1.254)--(-8.839,1.251)--(-8.952,1.123);
\filldraw[fill opacity=0.8,fill=gray!20](-8.766,1.465)--(-8.797,1.506)--(-8.781,1.489)--(-8.746,1.444)--cycle;
\filldraw[fill opacity=0.8,fill=gray!20,draw=none](-9.298,.904)--(-9.33,.907)--(-9.332,.908)--cycle;
\draw(-9.33,.907)--(-9.332,.908);
\filldraw[fill opacity=0.8,fill=gray!20,draw=none](-8.977,1.128)--(-8.987,1.13)--(-8.99,1.135)--cycle;
\filldraw[fill opacity=0.8,fill=gray!20,draw=none](-8.985,1.132)--(-8.993,1.133)--(-8.99,1.135)--cycle;
\draw(-8.985,1.132)--(-8.993,1.133)--(-8.99,1.135);
\filldraw[fill opacity=0.8,fill=gray!20,draw=none](-9.212,1.137)--(-9.22,1.142)--(-9.202,1.131)--(-9.205,1.132)--cycle;
\draw(-9.202,1.131)--(-9.205,1.132);
\filldraw[fill opacity=0.8,fill=gray!20](-9.028,1.538)--(-8.983,1.553)--(-8.961,1.557)--(-8.985,1.546)--cycle;
\filldraw[fill opacity=0.8,fill=gray!20,draw=none](-9.209,1.152)--(-9.202,1.148)--(-9.211,1.138)--(-9.222,1.145)--(-9.228,1.154)--cycle;
\filldraw[fill opacity=0.8,fill=gray!20,draw=none](-9.209,1.152)--(-9.202,1.148)--(-9.211,1.138)--(-9.222,1.145)--(-9.228,1.154)--cycle;
\filldraw[fill opacity=0.8,fill=gray!20,draw=none](-9.209,1.152)--(-9.202,1.148)--(-9.211,1.138)--(-9.222,1.145)--(-9.228,1.154)--cycle;
\filldraw[fill opacity=0.8,fill=gray!20,draw=none](-9.125,.95)--(-9.147,.925)--(-9.148,.973)--(-9.145,.977)--cycle;
\draw(-9.125,.95)--(-9.147,.925);
\draw(-9.148,.973)--(-9.145,.977);
\filldraw[fill opacity=0.8,fill=gray!20,draw=none](-9.148,.973)--(-9.147,.925)--(-9.163,.932)--cycle;
\draw(-9.147,.925)--(-9.163,.932);
\filldraw[fill opacity=0.8,fill=gray!20,draw=none](-9.125,.95)--(-9.147,.926)--(-9.153,.923)--(-9.179,.938)--(-9.145,.977)--cycle;
\draw(-9.125,.95)--(-9.147,.926);
\draw(-9.153,.923)--(-9.179,.938)--(-9.145,.977);
\filldraw[fill opacity=0.8,fill=gray!20,draw=none](-9.147,.926)--(-9.15,.922)--(-9.153,.923)--cycle;
\draw(-9.147,.926)--(-9.15,.922)--(-9.153,.923);
\filldraw[fill opacity=0.8,fill=gray!20,draw=none](-9.125,.95)--(-9.15,.922)--(-9.179,.938)--(-9.145,.977)--cycle;
\draw(-9.125,.95)--(-9.15,.922)--(-9.179,.938)--(-9.145,.977);
\filldraw[fill opacity=0.8,fill=gray!20,draw=none](-9.148,.973)--(-9.146,.98)--(-9.122,.97)--(-9.147,.925)--cycle;
\draw(-9.146,.98)--(-9.122,.97)--(-9.147,.925);
\filldraw[fill opacity=0.8,fill=gray!20,draw=none](-9.147,.925)--(-9.15,.922)--(-9.179,.938)--(-9.148,.973)--cycle;
\draw(-9.147,.925)--(-9.15,.922)--(-9.179,.938)--(-9.148,.973);
\filldraw[fill opacity=0.8,fill=gray!20,draw=none](-9.22,1.142)--(-9.227,1.146)--(-9.243,1.156)--(-9.24,1.154)--cycle;
\draw(-9.243,1.156)--(-9.24,1.154);
\filldraw[fill opacity=0.8,fill=gray!20,draw=none](-9.26,1.157)--(-9.264,1.152)--(-9.288,1.159)--(-9.276,1.159)--cycle;
\draw(-9.288,1.159)--(-9.276,1.159);
\filldraw[fill opacity=0.8,fill=gray!20,draw=none](-9.26,1.157)--(-9.264,1.152)--(-9.288,1.159)--(-9.276,1.159)--cycle;
\draw(-9.288,1.159)--(-9.276,1.159);
\filldraw[fill opacity=0.8,fill=gray!20,draw=none](-9.239,1.154)--(-9.237,1.153)--(-9.316,1.064)--(-9.329,1.086)--(-9.269,1.154)--cycle;
\draw(-9.237,1.153)--(-9.316,1.064)--(-9.329,1.086)--(-9.269,1.154);
\filldraw[fill opacity=0.8,fill=gray!20,draw=none](-9.239,1.154)--(-9.237,1.153)--(-9.316,1.064)--(-9.329,1.086)--(-9.269,1.154)--cycle;
\draw(-9.237,1.153)--(-9.316,1.064)--(-9.329,1.086)--(-9.269,1.154);
\filldraw[fill opacity=0.8,fill=gray!20,draw=none](-9.239,1.154)--(-9.237,1.153)--(-9.316,1.064)--(-9.329,1.086)--(-9.269,1.154)--cycle;
\draw(-9.237,1.153)--(-9.316,1.064)--(-9.329,1.086)--(-9.269,1.154);
\filldraw[fill opacity=0.8,fill=gray!20,draw=none](-9.1,.957)--(-9.103,.953)--(-9.147,.925)--(-9.125,.95)--cycle;
\draw(-9.1,.957)--(-9.103,.953);
\draw(-9.147,.925)--(-9.125,.95);
\filldraw[fill opacity=0.8,fill=gray!20,draw=none](-9.1,.957)--(-9.103,.953)--(-9.147,.925)--(-9.125,.95)--cycle;
\draw(-9.1,.957)--(-9.103,.953);
\draw(-9.147,.925)--(-9.125,.95);
\filldraw[fill opacity=0.8,fill=gray!20,draw=none](-9.1,.957)--(-9.103,.953)--(-9.147,.925)--(-9.125,.95)--cycle;
\draw(-9.1,.957)--(-9.103,.953);
\draw(-9.147,.925)--(-9.125,.95);
\filldraw[fill opacity=0.8,fill=gray!20,draw=none](-9.084,.975)--(-9.1,.957)--(-9.125,.95)--(-9.109,.968)--cycle;
\draw(-9.084,.975)--(-9.1,.957);
\draw(-9.125,.95)--(-9.109,.968);
\filldraw[fill opacity=0.8,fill=gray!20,draw=none](-9.131,.918)--(-9.147,.925)--(-9.122,.97)--(-9.098,.959)--cycle;
\draw(-9.131,.918)--(-9.147,.925)--(-9.122,.97)--(-9.098,.959);
\filldraw[fill opacity=0.8,fill=gray!20,draw=none](-9.27,1.153)--(-9.269,1.154)--(-9.329,1.086)--(-9.327,1.096)--(-9.295,1.132)--cycle;
\draw(-9.269,1.154)--(-9.329,1.086)--(-9.327,1.096)--(-9.295,1.132);
\filldraw[fill opacity=0.8,fill=gray!20,draw=none](-9.27,1.153)--(-9.269,1.154)--(-9.329,1.086)--(-9.327,1.096)--(-9.295,1.132)--cycle;
\draw(-9.269,1.154)--(-9.329,1.086)--(-9.327,1.096)--(-9.295,1.132);
\filldraw[fill opacity=0.8,fill=gray!20,draw=none](-9.27,1.153)--(-9.269,1.154)--(-9.329,1.086)--(-9.327,1.096)--(-9.295,1.132)--cycle;
\draw(-9.269,1.154)--(-9.329,1.086)--(-9.327,1.096)--(-9.295,1.132);
\filldraw[fill opacity=0.8,fill=gray!20,draw=none](-9.311,1.092)--(-9.312,1.093)--(-9.327,1.096)--(-9.329,1.086)--(-9.316,1.064)--(-9.297,1.041)--(-9.28,1.023)--(-9.254,.998)--(-9.215,.965)--(-9.179,.938)--(-9.15,.922)--cycle;
\draw(-9.312,1.093)--(-9.327,1.096)--(-9.329,1.086)--(-9.316,1.064)--(-9.297,1.041);
\draw(-9.28,1.023)--(-9.254,.998)--(-9.215,.965)--(-9.179,.938)--(-9.15,.922);
\filldraw[fill opacity=0.8,fill=gray!20,draw=none](-9.311,1.092)--(-9.312,1.093)--(-9.327,1.096)--(-9.329,1.086)--(-9.316,1.064)--(-9.297,1.041)--(-9.28,1.023)--(-9.254,.998)--(-9.215,.965)--(-9.179,.938)--(-9.153,.923)--(-9.151,.922)--cycle;
\draw(-9.312,1.093)--(-9.327,1.096)--(-9.329,1.086)--(-9.316,1.064)--(-9.297,1.041);
\draw(-9.28,1.023)--(-9.254,.998)--(-9.215,.965)--(-9.179,.938)--(-9.153,.923);
\filldraw[fill opacity=0.8,fill=gray!20,draw=none](-9.311,1.092)--(-9.312,1.093)--(-9.327,1.096)--(-9.329,1.086)--(-9.316,1.064)--(-9.297,1.041)--(-9.28,1.023)--(-9.254,.998)--(-9.215,.965)--(-9.179,.938)--(-9.15,.922)--cycle;
\draw(-9.312,1.093)--(-9.327,1.096)--(-9.329,1.086)--(-9.316,1.064)--(-9.297,1.041);
\draw(-9.28,1.023)--(-9.254,.998)--(-9.215,.965)--(-9.179,.938)--(-9.15,.922);
\filldraw[fill opacity=0.8,fill=gray!20,draw=none](-9.223,1.162)--(-9.205,1.149)--(-9.222,1.16)--(-9.232,1.167)--cycle;
\filldraw[fill opacity=0.8,fill=gray!20,draw=none](-9.145,1.279)--(-9.14,1.262)--(-9.224,1.167)--(-9.232,1.167)--(-9.247,1.178)--(-9.154,1.283)--cycle;
\draw(-9.14,1.262)--(-9.224,1.167);
\draw(-9.247,1.178)--(-9.154,1.283);
\filldraw[fill opacity=0.8,fill=gray!20,draw=none](-9.211,1.138)--(-9.212,1.137)--(-9.22,1.142)--(-9.222,1.145)--cycle;
\filldraw[fill opacity=0.8,fill=gray!20,draw=none](-9.211,1.138)--(-9.212,1.137)--(-9.22,1.142)--(-9.222,1.145)--cycle;
\filldraw[fill opacity=0.8,fill=gray!20,draw=none](-9.211,1.138)--(-9.212,1.137)--(-9.22,1.142)--(-9.222,1.145)--cycle;
\filldraw[fill opacity=0.8,fill=gray!20,draw=none](-9.212,1.137)--(-9.227,1.146)--(-9.22,1.142)--cycle;
\filldraw[fill opacity=0.8,fill=gray!20,draw=none](-9.209,1.152)--(-9.228,1.154)--(-9.232,1.159)--(-9.227,1.164)--cycle;
\draw(-9.232,1.159)--(-9.227,1.164);
\filldraw[fill opacity=0.8,fill=gray!20,draw=none](-9.209,1.152)--(-9.228,1.154)--(-9.232,1.159)--(-9.227,1.164)--cycle;
\draw(-9.232,1.159)--(-9.227,1.164);
\filldraw[fill opacity=0.8,fill=gray!20,draw=none](-9.209,1.152)--(-9.228,1.154)--(-9.232,1.159)--(-9.227,1.164)--cycle;
\draw(-9.232,1.159)--(-9.227,1.164);
\filldraw[fill opacity=0.8,fill=gray!20,draw=none](-9.164,1.127)--(-9.199,1.129)--(-9.222,1.143)--(-9.223,1.162)--cycle;
\draw(-9.164,1.127)--(-9.199,1.129);
\draw(-9.222,1.143)--(-9.223,1.162);
\filldraw[fill opacity=0.8,fill=gray!20,draw=none](-9.236,1.152)--(-9.236,1.154)--(-9.222,1.145)--(-9.222,1.143)--cycle;
\draw(-9.222,1.145)--(-9.222,1.143);
\filldraw[fill opacity=0.8,fill=gray!20,draw=none](-9.236,1.152)--(-9.236,1.154)--(-9.222,1.145)--(-9.222,1.143)--cycle;
\draw(-9.222,1.145)--(-9.222,1.143);
\filldraw[fill opacity=0.8,fill=gray!20,draw=none](-9.222,1.145)--(-9.22,1.142)--(-9.237,1.153)--(-9.236,1.154)--cycle;
\draw(-9.237,1.153)--(-9.236,1.154);
\filldraw[fill opacity=0.8,fill=gray!20,draw=none](-9.228,1.154)--(-9.22,1.142)--(-9.237,1.153)--(-9.235,1.154)--cycle;
\draw(-9.237,1.153)--(-9.235,1.154);
\filldraw[fill opacity=0.8,fill=gray!20,draw=none](-9.228,1.154)--(-9.22,1.142)--(-9.237,1.153)--(-9.235,1.154)--cycle;
\draw(-9.237,1.153)--(-9.235,1.154);
\filldraw[fill opacity=0.8,fill=gray!20,draw=none](-9.228,1.154)--(-9.222,1.145)--(-9.236,1.154)--(-9.235,1.154)--cycle;
\draw(-9.236,1.154)--(-9.235,1.154);
\filldraw[fill opacity=0.8,fill=gray!20,draw=none](-9.222,1.16)--(-9.238,1.17)--(-9.232,1.167)--cycle;
\draw(-9.238,1.17)--(-9.232,1.167);
\filldraw[fill opacity=0.8,fill=gray!20,draw=none](-9.228,1.154)--(-9.235,1.154)--(-9.232,1.159)--cycle;
\draw(-9.235,1.154)--(-9.232,1.159);
\filldraw[fill opacity=0.8,fill=gray!20,draw=none](-9.228,1.154)--(-9.235,1.154)--(-9.232,1.159)--cycle;
\draw(-9.235,1.154)--(-9.232,1.159);
\filldraw[fill opacity=0.8,fill=gray!20,draw=none](-9.228,1.154)--(-9.235,1.154)--(-9.232,1.159)--cycle;
\draw(-9.235,1.154)--(-9.232,1.159);
\filldraw[fill opacity=0.8,fill=gray!20,draw=none](-9.236,1.154)--(-9.236,1.154)--(-9.222,1.152)--(-9.222,1.145)--cycle;
\draw(-9.222,1.152)--(-9.222,1.145);
\filldraw[fill opacity=0.8,fill=gray!20,draw=none](-9.236,1.154)--(-9.236,1.154)--(-9.222,1.152)--(-9.222,1.145)--cycle;
\draw(-9.222,1.152)--(-9.222,1.145);
\filldraw[fill opacity=0.8,fill=gray!20,draw=none](-9.22,1.142)--(-9.237,1.153)--(-9.239,1.154)--(-9.186,1.131)--(-9.147,1.107)--(-9.202,1.131)--cycle;
\draw(-9.239,1.154)--(-9.186,1.131)--(-9.147,1.107)--(-9.202,1.131);
\filldraw[fill opacity=0.8,fill=gray!20,draw=none](-8.969,1.126)--(-8.987,1.127)--(-8.993,1.133)--(-8.985,1.132)--cycle;
\draw(-8.969,1.126)--(-8.987,1.127)--(-8.993,1.133)--(-8.985,1.132);
\filldraw[fill opacity=0.8,fill=gray!20](-8.987,1.127)--(-9.034,1.143)--(-9.046,1.155)--(-8.993,1.133)--cycle;
\filldraw[fill opacity=0.8,fill=gray!20,draw=none](-8.969,1.126)--(-8.977,1.128)--(-8.985,1.132)--cycle;
\filldraw[fill opacity=0.8,fill=gray!20,draw=none](-8.969,1.126)--(-8.977,1.128)--(-8.985,1.132)--cycle;
\filldraw[fill opacity=0.8,fill=gray!20,draw=none](-9.288,.867)--(-9.339,.896)--(-9.319,.9)--(-9.273,.874)--(-9.272,.87)--cycle;
\draw(-9.339,.896)--(-9.319,.9);
\draw(-9.273,.874)--(-9.272,.87)--(-9.288,.867);
\filldraw[fill opacity=0.8,fill=gray!20,draw=none](-9.288,.867)--(-9.339,.896)--(-9.319,.9)--(-9.273,.874)--(-9.272,.87)--cycle;
\draw(-9.339,.896)--(-9.319,.9);
\draw(-9.273,.874)--(-9.272,.87)--(-9.288,.867);
\filldraw[fill opacity=0.8,fill=gray!20](-9.27,.84)--(-9.307,.863)--(-9.272,.87)--(-9.252,.844)--cycle;
\filldraw[fill opacity=0.8,fill=gray!20](-9.27,.84)--(-9.307,.863)--(-9.272,.87)--(-9.252,.844)--cycle;
\filldraw[fill opacity=0.8,fill=gray!20,draw=none](-9.233,.849)--(-9.328,.891)--(-9.279,.885)--(-9.183,.843)--cycle;
\draw(-9.233,.849)--(-9.328,.891);
\draw(-9.279,.885)--(-9.183,.843);
\filldraw[fill opacity=0.8,fill=gray!20](-8.876,1.545)--(-8.905,1.557)--(-8.885,1.552)--(-8.838,1.536)--cycle;
\filldraw[fill opacity=0.8,fill=gray!20](-9.034,1.143)--(-9.075,1.173)--(-9.091,1.19)--(-9.046,1.155)--cycle;
\filldraw[fill opacity=0.8,fill=gray!20,draw=none](-8.936,1.125)--(-8.939,1.12)--(-8.952,1.121)--(-8.944,1.124)--(-8.936,1.125)--cycle;
\draw(-8.944,1.124)--(-8.936,1.125)--(-8.936,1.125)--(-8.939,1.12)--(-8.952,1.121);
\filldraw[fill opacity=0.8,fill=gray!20](-8.936,1.125)--(-8.879,1.132)--(-8.889,1.125)--(-8.936,1.125)--cycle;
\filldraw[fill opacity=0.8,fill=gray!20](-8.936,1.125)--(-8.911,1.121)--(-8.939,1.12)--(-8.936,1.125)--cycle;
\filldraw[fill opacity=0.8,fill=gray!20](-8.936,1.125)--(-8.889,1.125)--(-8.911,1.121)--(-8.936,1.125)--cycle;
\filldraw[fill opacity=0.8,fill=gray!20,draw=none](-9.068,1.508)--(-9.091,1.493)--(-9.046,1.526)--(-9.028,1.538)--(-9.06,1.513)--cycle;
\draw(-9.068,1.508)--(-9.091,1.493)--(-9.046,1.526)--(-9.028,1.538)--(-9.06,1.513);
\filldraw[fill opacity=0.8,fill=gray!20,draw=none](-9.178,1.128)--(-9.194,1.159)--(-9.157,1.152)--(-9.143,1.125)--cycle;
\draw(-9.157,1.152)--(-9.143,1.125)--(-9.178,1.128);
\filldraw[fill opacity=0.8,fill=gray!20](-8.879,1.132)--(-8.826,1.152)--(-8.844,1.141)--(-8.889,1.125)--cycle;
\filldraw[fill opacity=0.8,fill=gray!20](-9.075,1.173)--(-9.106,1.214)--(-9.126,1.235)--(-9.091,1.19)--cycle;
\filldraw[fill opacity=0.8,fill=gray!20,draw=none](-9.111,1.459)--(-9.126,1.449)--(-9.091,1.493)--(-9.068,1.508)--cycle;
\draw(-9.111,1.459)--(-9.126,1.449)--(-9.091,1.493)--(-9.068,1.508);
\filldraw[fill opacity=0.8,fill=gray!20,draw=none](-9.33,.907)--(-9.319,.9)--(-9.339,.896)--(-9.346,.909)--cycle;
\draw(-9.319,.9)--(-9.339,.896)--(-9.346,.909);
\filldraw[fill opacity=0.8,fill=gray!20,draw=none](-9.33,.907)--(-9.319,.9)--(-9.339,.896)--(-9.346,.909)--cycle;
\draw(-9.319,.9)--(-9.339,.896)--(-9.346,.909);
\filldraw[fill opacity=0.8,fill=gray!20,draw=none](-9.368,.916)--(-9.381,.921)--(-9.338,.895)--(-9.332,.892)--cycle;
\draw(-9.368,.916)--(-9.381,.921);
\draw(-9.338,.895)--(-9.332,.892);
\filldraw[fill opacity=0.8,fill=gray!20,draw=none](-9.349,.919)--(-9.368,.931)--(-9.372,.932)--cycle;
\draw(-9.368,.931)--(-9.372,.932);
\filldraw[fill opacity=0.8,fill=gray!20,draw=none](-9.34,.895)--(-9.37,.933)--(-9.367,.934)--(-9.355,.924)--(-9.339,.896)--cycle;
\draw(-9.37,.933)--(-9.367,.934);
\draw(-9.355,.924)--(-9.339,.896)--(-9.34,.895);
\filldraw[fill opacity=0.8,fill=gray!20,draw=none](-9.34,.895)--(-9.37,.933)--(-9.367,.934)--(-9.355,.924)--(-9.339,.896)--cycle;
\draw(-9.37,.933)--(-9.367,.934);
\draw(-9.355,.924)--(-9.339,.896)--(-9.34,.895);
\filldraw[fill opacity=0.8,fill=gray!20,draw=none](-9.349,.919)--(-9.333,.909)--(-9.276,.883)--(-9.314,.907)--(-9.368,.931)--cycle;
\draw(-9.333,.909)--(-9.276,.883)--(-9.314,.907)--(-9.368,.931);
\filldraw[fill opacity=0.8,fill=gray!20,draw=none](-9.155,1.122)--(-9.155,1.126)--(-9.143,1.125)--(-9.14,1.113)--cycle;
\draw(-9.155,1.126)--(-9.143,1.125)--(-9.14,1.113);
\filldraw[fill opacity=0.8,fill=gray!20,draw=none](-9.155,1.122)--(-9.155,1.126)--(-9.143,1.125)--(-9.14,1.113)--cycle;
\draw(-9.155,1.126)--(-9.143,1.125)--(-9.14,1.113);
\filldraw[fill opacity=0.8,fill=gray!20,draw=none](-9.155,1.122)--(-9.164,1.127)--(-9.155,1.126)--cycle;
\draw(-9.164,1.127)--(-9.155,1.126);
\filldraw[fill opacity=0.8,fill=gray!20,draw=none](-9.164,1.127)--(-9.223,1.162)--(-9.195,1.16)--(-9.157,1.152)--(-9.143,1.125)--cycle;
\draw(-9.223,1.162)--(-9.195,1.16);
\draw(-9.157,1.152)--(-9.143,1.125)--(-9.164,1.127);
\filldraw[fill opacity=0.8,fill=gray!20,draw=none](-9.284,1.16)--(-9.287,1.16)--(-9.242,1.155)--(-9.243,1.156)--cycle;
\draw(-9.242,1.155)--(-9.243,1.156);
\filldraw[fill opacity=0.8,fill=gray!20](-9.143,1.125)--(-9.159,1.157)--(-9.115,1.146)--(-9.089,1.112)--cycle;
\filldraw[fill opacity=0.8,fill=gray!20](-9.143,1.125)--(-9.159,1.157)--(-9.115,1.146)--(-9.089,1.112)--cycle;
\filldraw[fill opacity=0.8,fill=gray!20,draw=none](-9.224,1.167)--(-9.227,1.164)--(-9.232,1.167)--cycle;
\draw(-9.224,1.167)--(-9.227,1.164);
\filldraw[fill opacity=0.8,fill=gray!20,draw=none](-9.224,1.167)--(-9.227,1.164)--(-9.232,1.167)--cycle;
\draw(-9.224,1.167)--(-9.227,1.164);
\filldraw[fill opacity=0.8,fill=gray!20,draw=none](-9.224,1.167)--(-9.227,1.164)--(-9.232,1.167)--cycle;
\draw(-9.224,1.167)--(-9.227,1.164);
\filldraw[fill opacity=0.8,fill=gray!20,draw=none](-9.035,1.418)--(-9.146,1.292)--(-9.14,1.307)--(-9.033,1.428)--cycle;
\draw(-9.14,1.307)--(-9.033,1.428)--(-9.035,1.418)--(-9.146,1.292);
\filldraw[fill opacity=0.8,fill=gray!20,draw=none](-9.035,1.418)--(-9.146,1.292)--(-9.14,1.307)--(-9.033,1.428)--cycle;
\draw(-9.14,1.307)--(-9.033,1.428)--(-9.035,1.418)--(-9.146,1.292);
\filldraw[fill opacity=0.8,fill=gray!20,draw=none](-9.035,1.418)--(-9.146,1.292)--(-9.14,1.307)--(-9.033,1.428)--cycle;
\draw(-9.14,1.307)--(-9.033,1.428)--(-9.035,1.418)--(-9.146,1.292);
\filldraw[fill opacity=0.8,fill=gray!20,draw=none](-9.147,1.348)--(-9.155,1.342)--(-9.148,1.398)--(-9.137,1.404)--cycle;
\draw(-9.147,1.348)--(-9.155,1.342)--(-9.148,1.398)--(-9.137,1.404);
\filldraw[fill opacity=0.8,fill=gray!20](-9.106,1.214)--(-9.126,1.263)--(-9.148,1.287)--(-9.126,1.235)--cycle;
\filldraw[fill opacity=0.8,fill=gray!20,draw=none](-8.826,1.152)--(-8.781,1.186)--(-8.804,1.171)--(-8.812,1.165)--(-8.844,1.141)--cycle;
\draw(-8.812,1.165)--(-8.844,1.141)--(-8.826,1.152)--(-8.781,1.186)--(-8.804,1.171);
\filldraw[fill opacity=0.8,fill=gray!20,draw=none](-9.33,.907)--(-9.346,.909)--(-9.353,.921)--cycle;
\draw(-9.346,.909)--(-9.353,.921);
\filldraw[fill opacity=0.8,fill=gray!20,draw=none](-9.33,.907)--(-9.346,.909)--(-9.353,.921)--cycle;
\draw(-9.346,.909)--(-9.353,.921);
\filldraw[fill opacity=0.8,fill=gray!20](-8.797,1.506)--(-8.838,1.536)--(-8.826,1.523)--(-8.781,1.489)--cycle;
\filldraw[fill opacity=0.8,fill=gray!20,draw=none](-9.236,1.161)--(-9.237,1.167)--(-9.232,1.167)--(-9.227,1.164)--(-9.233,1.157)--cycle;
\draw(-9.227,1.164)--(-9.233,1.157);
\filldraw[fill opacity=0.8,fill=gray!20,draw=none](-9.236,1.161)--(-9.237,1.167)--(-9.232,1.167)--(-9.227,1.164)--(-9.233,1.157)--cycle;
\draw(-9.227,1.164)--(-9.233,1.157);
\filldraw[fill opacity=0.8,fill=gray!20,draw=none](-9.254,1.168)--(-9.241,1.167)--(-9.232,1.167)--(-9.238,1.17)--(-9.247,1.171)--cycle;
\draw(-9.232,1.167)--(-9.238,1.17);
\filldraw[fill opacity=0.8,fill=gray!20,draw=none](-9.236,1.161)--(-9.233,1.157)--(-9.236,1.154)--cycle;
\draw(-9.233,1.157)--(-9.236,1.154);
\filldraw[fill opacity=0.8,fill=gray!20,draw=none](-9.236,1.161)--(-9.237,1.167)--(-9.232,1.167)--(-9.227,1.164)--(-9.233,1.157)--cycle;
\draw(-9.227,1.164)--(-9.233,1.157);
\filldraw[fill opacity=0.8,fill=gray!20,draw=none](-9.236,1.161)--(-9.233,1.157)--(-9.236,1.154)--cycle;
\draw(-9.233,1.157)--(-9.236,1.154);
\filldraw[fill opacity=0.8,fill=gray!20,draw=none](-9.236,1.161)--(-9.233,1.157)--(-9.236,1.154)--cycle;
\draw(-9.233,1.157)--(-9.236,1.154);
\filldraw[fill opacity=0.8,fill=gray!20,draw=none](-9.237,1.154)--(-9.247,1.161)--(-9.223,1.162)--(-9.222,1.152)--cycle;
\draw(-9.247,1.161)--(-9.223,1.162)--(-9.222,1.152);
\filldraw[fill opacity=0.8,fill=gray!20,draw=none](-9.237,1.154)--(-9.247,1.161)--(-9.223,1.162)--(-9.222,1.152)--cycle;
\draw(-9.247,1.161)--(-9.223,1.162)--(-9.222,1.152);
\filldraw[fill opacity=0.8,fill=gray!20,draw=none](-9.236,1.152)--(-9.24,1.155)--(-9.237,1.154)--(-9.236,1.154)--cycle;
\filldraw[fill opacity=0.8,fill=gray!20,draw=none](-9.236,1.152)--(-9.24,1.155)--(-9.237,1.154)--(-9.236,1.154)--cycle;
\filldraw[fill opacity=0.8,fill=gray!20,draw=none](-9.236,1.154)--(-9.237,1.154)--(-9.236,1.154)--cycle;
\filldraw[fill opacity=0.8,fill=gray!20,draw=none](-9.236,1.154)--(-9.237,1.154)--(-9.236,1.154)--cycle;
\filldraw[fill opacity=0.8,fill=gray!20,draw=none](-9.24,1.155)--(-9.249,1.16)--(-9.247,1.161)--(-9.237,1.154)--cycle;
\draw(-9.249,1.16)--(-9.247,1.161);
\filldraw[fill opacity=0.8,fill=gray!20,draw=none](-9.24,1.155)--(-9.249,1.16)--(-9.247,1.161)--(-9.237,1.154)--cycle;
\draw(-9.249,1.16)--(-9.247,1.161);
\filldraw[fill opacity=0.8,fill=gray!20,draw=none](-9.254,1.168)--(-9.247,1.171)--(-9.251,1.171)--(-9.279,1.171)--cycle;
\filldraw[fill opacity=0.8,fill=gray!20,draw=none](-9.251,1.171)--(-9.287,1.175)--(-9.286,1.174)--(-9.279,1.171)--cycle;
\draw(-9.286,1.174)--(-9.279,1.171);
\filldraw[fill opacity=0.8,fill=gray!20,draw=none](-9.223,1.162)--(-9.232,1.167)--(-9.228,1.165)--cycle;
\draw(-9.232,1.167)--(-9.228,1.165);
\filldraw[fill opacity=0.8,fill=gray!20,draw=none](-9.241,1.167)--(-9.237,1.167)--(-9.236,1.161)--cycle;
\filldraw[fill opacity=0.8,fill=gray!20,draw=none](-9.237,1.167)--(-9.236,1.161)--(-9.241,1.167)--cycle;
\filldraw[fill opacity=0.8,fill=gray!20,draw=none](-9.237,1.167)--(-9.236,1.161)--(-9.241,1.167)--cycle;
\filldraw[fill opacity=0.8,fill=gray!20,draw=none](-9.236,1.161)--(-9.241,1.167)--(-9.223,1.165)--(-9.223,1.162)--cycle;
\draw(-9.223,1.165)--(-9.223,1.162)--(-9.236,1.161);
\filldraw[fill opacity=0.8,fill=gray!20,draw=none](-9.237,1.167)--(-9.237,1.171)--(-9.232,1.167)--cycle;
\filldraw[fill opacity=0.8,fill=gray!20,draw=none](-9.237,1.167)--(-9.237,1.171)--(-9.232,1.167)--cycle;
\filldraw[fill opacity=0.8,fill=gray!20,draw=none](-9.237,1.167)--(-9.237,1.171)--(-9.232,1.167)--cycle;
\filldraw[fill opacity=0.8,fill=gray!20,draw=none](-9.241,1.167)--(-9.228,1.165)--(-9.232,1.167)--cycle;
\draw(-9.228,1.165)--(-9.232,1.167);
\filldraw[fill opacity=0.8,fill=gray!20,draw=none](-9.284,1.16)--(-9.288,1.161)--(-9.287,1.16)--cycle;
\draw(-9.288,1.161)--(-9.287,1.16);
\filldraw[fill opacity=0.8,fill=gray!20,draw=none](-9.242,1.168)--(-9.237,1.171)--(-9.237,1.167)--(-9.241,1.167)--cycle;
\filldraw[fill opacity=0.8,fill=gray!20,draw=none](-9.237,1.171)--(-9.242,1.168)--(-9.249,1.176)--(-9.247,1.178)--cycle;
\draw(-9.249,1.176)--(-9.247,1.178);
\filldraw[fill opacity=0.8,fill=gray!20,draw=none](-9.237,1.171)--(-9.237,1.167)--(-9.241,1.167)--(-9.249,1.176)--(-9.247,1.178)--cycle;
\draw(-9.249,1.176)--(-9.247,1.178);
\filldraw[fill opacity=0.8,fill=gray!20,draw=none](-9.237,1.171)--(-9.237,1.167)--(-9.241,1.167)--(-9.249,1.176)--(-9.247,1.178)--cycle;
\draw(-9.249,1.176)--(-9.247,1.178);
\filldraw[fill opacity=0.8,fill=gray!20,draw=none](-9.236,1.161)--(-9.242,1.168)--(-9.224,1.171)--(-9.223,1.162)--cycle;
\draw(-9.224,1.171)--(-9.223,1.162)--(-9.236,1.161);
\filldraw[fill opacity=0.8,fill=gray!20](-9.223,1.162)--(-9.225,1.182)--(-9.18,1.178)--(-9.159,1.157)--cycle;
\filldraw[fill opacity=0.8,fill=gray!20](-9.223,1.162)--(-9.225,1.182)--(-9.18,1.178)--(-9.159,1.157)--cycle;
\filldraw[fill opacity=0.8,fill=gray!20,draw=none](-9.061,1.058)--(-9.073,1.071)--(-9.089,1.112)--(-9.072,1.094)--(-9.057,1.058)--cycle;
\draw(-9.061,1.058)--(-9.073,1.071)--(-9.089,1.112)--(-9.072,1.094)--(-9.057,1.058);
\filldraw[fill opacity=0.8,fill=gray!20,draw=none](-9.061,1.058)--(-9.073,1.071)--(-9.089,1.112)--(-9.072,1.094)--(-9.057,1.058)--cycle;
\draw(-9.061,1.058)--(-9.073,1.071)--(-9.089,1.112)--(-9.072,1.094)--(-9.057,1.058);
\filldraw[fill opacity=0.8,fill=gray!20,draw=none](-9.123,1.118)--(-9.093,1.099)--(-9.183,1.138)--(-9.223,1.162)--(-9.228,1.165)--(-9.133,1.124)--cycle;
\draw(-9.093,1.099)--(-9.183,1.138);
\draw(-9.228,1.165)--(-9.133,1.124);
\filldraw[fill opacity=0.8,fill=gray!20,draw=none](-9.137,1.404)--(-9.148,1.398)--(-9.126,1.449)--(-9.111,1.459)--cycle;
\draw(-9.137,1.404)--(-9.148,1.398)--(-9.126,1.449)--(-9.111,1.459);
\filldraw[fill opacity=0.8,fill=gray!20,draw=none](-9.246,1.167)--(-9.241,1.167)--(-9.236,1.161)--(-9.288,1.159)--(-9.287,1.16)--cycle;
\draw(-9.236,1.161)--(-9.288,1.159)--(-9.287,1.16);
\filldraw[fill opacity=0.8,fill=gray!20,draw=none](-9.242,1.168)--(-9.236,1.161)--(-9.288,1.159)--(-9.287,1.16)--cycle;
\draw(-9.236,1.161)--(-9.288,1.159)--(-9.287,1.16);
\filldraw[fill opacity=0.8,fill=gray!20,draw=none](-9.287,1.16)--(-9.231,1.136)--(-9.186,1.131)--(-9.242,1.155)--cycle;
\draw(-9.287,1.16)--(-9.231,1.136)--(-9.186,1.131)--(-9.242,1.155);
\filldraw[fill opacity=0.8,fill=gray!20,draw=none](-9.34,.895)--(-9.348,.89)--(-9.377,.927)--(-9.37,.933)--cycle;
\draw(-9.34,.895)--(-9.348,.89);
\draw(-9.377,.927)--(-9.37,.933);
\filldraw[fill opacity=0.8,fill=gray!20,draw=none](-9.34,.895)--(-9.348,.89)--(-9.377,.927)--(-9.37,.933)--cycle;
\draw(-9.34,.895)--(-9.348,.89);
\draw(-9.377,.927)--(-9.37,.933);
\filldraw[fill opacity=0.8,fill=gray!20,draw=none](-9.238,.853)--(-9.278,.876)--(-9.368,.916)--(-9.332,.892)--(-9.233,.849)--cycle;
\draw(-9.278,.876)--(-9.368,.916);
\draw(-9.332,.892)--(-9.233,.849);
\filldraw[fill opacity=0.8,fill=gray!20,draw=none](-9.241,1.167)--(-9.236,1.161)--(-9.236,1.154)--(-9.237,1.153)--(-9.256,1.167)--cycle;
\draw(-9.236,1.154)--(-9.237,1.153);
\filldraw[fill opacity=0.8,fill=gray!20,draw=none](-9.241,1.167)--(-9.236,1.161)--(-9.236,1.154)--(-9.237,1.153)--(-9.256,1.167)--cycle;
\draw(-9.236,1.154)--(-9.237,1.153);
\filldraw[fill opacity=0.8,fill=gray!20,draw=none](-9.241,1.167)--(-9.236,1.161)--(-9.236,1.154)--(-9.237,1.153)--(-9.256,1.167)--cycle;
\draw(-9.236,1.154)--(-9.237,1.153);
\filldraw[fill opacity=0.8,fill=gray!20,draw=none](-9.237,1.153)--(-9.24,1.154)--(-9.239,1.154)--cycle;
\draw(-9.24,1.154)--(-9.239,1.154);
\filldraw[fill opacity=0.8,fill=gray!20,draw=none](-9.146,1.285)--(-9.145,1.279)--(-9.146,1.28)--cycle;
\filldraw[fill opacity=0.8,fill=gray!20,draw=none](-9.146,1.285)--(-9.145,1.279)--(-9.146,1.28)--cycle;
\filldraw[fill opacity=0.8,fill=gray!20,draw=none](-9.146,1.285)--(-9.145,1.279)--(-9.146,1.28)--cycle;
\filldraw[fill opacity=0.8,fill=gray!20,draw=none](-9.26,1.157)--(-9.258,1.16)--(-9.249,1.16)--(-9.24,1.155)--cycle;
\draw(-9.258,1.16)--(-9.249,1.16);
\filldraw[fill opacity=0.8,fill=gray!20,draw=none](-9.26,1.157)--(-9.258,1.16)--(-9.249,1.16)--(-9.24,1.155)--cycle;
\draw(-9.258,1.16)--(-9.249,1.16);
\filldraw[fill opacity=0.8,fill=gray!20,draw=none](-9.126,1.263)--(-9.13,1.297)--(-9.152,1.319)--(-9.148,1.287)--cycle;
\draw(-9.152,1.319)--(-9.148,1.287)--(-9.126,1.263)--(-9.13,1.297);
\filldraw[fill opacity=0.8,fill=gray!20,draw=none](-8.804,1.171)--(-8.781,1.186)--(-8.746,1.23)--(-8.761,1.22)--cycle;
\draw(-8.804,1.171)--(-8.781,1.186)--(-8.746,1.23)--(-8.761,1.22);
\filldraw[fill opacity=0.8,fill=gray!20,draw=none](-9.288,.867)--(-9.307,.863)--(-9.339,.896)--cycle;
\draw(-9.288,.867)--(-9.307,.863)--(-9.339,.896);
\filldraw[fill opacity=0.8,fill=gray!20,draw=none](-9.288,.867)--(-9.307,.863)--(-9.339,.896)--cycle;
\draw(-9.288,.867)--(-9.307,.863)--(-9.339,.896);
\filldraw[fill opacity=0.8,fill=gray!20,draw=none](-9.146,1.285)--(-9.146,1.28)--(-9.154,1.283)--(-9.148,1.29)--cycle;
\draw(-9.154,1.283)--(-9.148,1.29);
\filldraw[fill opacity=0.8,fill=gray!20,draw=none](-9.146,1.285)--(-9.146,1.28)--(-9.154,1.283)--(-9.148,1.29)--cycle;
\draw(-9.154,1.283)--(-9.148,1.29);
\filldraw[fill opacity=0.8,fill=gray!20,draw=none](-9.146,1.285)--(-9.146,1.28)--(-9.154,1.283)--(-9.148,1.29)--cycle;
\draw(-9.154,1.283)--(-9.148,1.29);
\filldraw[fill opacity=0.8,fill=gray!20,draw=none](-9.26,1.157)--(-9.276,1.159)--(-9.258,1.16)--cycle;
\draw(-9.276,1.159)--(-9.258,1.16);
\filldraw[fill opacity=0.8,fill=gray!20,draw=none](-9.26,1.157)--(-9.276,1.159)--(-9.258,1.16)--cycle;
\draw(-9.276,1.159)--(-9.258,1.16);
\filldraw[fill opacity=0.8,fill=gray!20,draw=none](-9.254,1.168)--(-9.255,1.166)--(-9.287,1.16)--(-9.279,1.171)--cycle;
\draw(-9.287,1.16)--(-9.279,1.171);
\filldraw[fill opacity=0.8,fill=gray!20,draw=none](-9.249,1.176)--(-9.255,1.166)--(-9.287,1.16)--(-9.279,1.171)--cycle;
\draw(-9.287,1.16)--(-9.279,1.171);
\filldraw[fill opacity=0.8,fill=gray!20,draw=none](-9.256,1.167)--(-9.239,1.154)--(-9.269,1.154)--(-9.257,1.167)--cycle;
\draw(-9.269,1.154)--(-9.257,1.167);
\filldraw[fill opacity=0.8,fill=gray!20,draw=none](-9.256,1.167)--(-9.239,1.154)--(-9.269,1.154)--(-9.257,1.167)--cycle;
\draw(-9.269,1.154)--(-9.257,1.167);
\filldraw[fill opacity=0.8,fill=gray!20,draw=none](-9.256,1.167)--(-9.239,1.154)--(-9.269,1.154)--(-9.257,1.167)--cycle;
\draw(-9.269,1.154)--(-9.257,1.167);
\filldraw[fill opacity=0.8,fill=gray!20](-8.967,1.122)--(-8.997,1.134)--(-9.034,1.143)--(-8.987,1.127)--cycle;
\filldraw[fill opacity=0.8,fill=gray!20](-9.322,.854)--(-9.36,.883)--(-9.339,.896)--(-9.307,.863)--cycle;
\filldraw[fill opacity=0.8,fill=gray!20](-9.322,.854)--(-9.36,.883)--(-9.339,.896)--(-9.307,.863)--cycle;
\filldraw[fill opacity=0.8,fill=gray!20,draw=none](-9.146,1.285)--(-9.148,1.29)--(-9.146,1.292)--cycle;
\draw(-9.148,1.29)--(-9.146,1.292);
\filldraw[fill opacity=0.8,fill=gray!20,draw=none](-9.146,1.285)--(-9.148,1.29)--(-9.146,1.292)--cycle;
\draw(-9.148,1.29)--(-9.146,1.292);
\filldraw[fill opacity=0.8,fill=gray!20,draw=none](-9.146,1.285)--(-9.148,1.29)--(-9.146,1.292)--cycle;
\draw(-9.148,1.29)--(-9.146,1.292);
\filldraw[fill opacity=0.8,fill=gray!20,draw=none](-9.146,1.292)--(-9.236,1.191)--(-9.224,1.212)--(-9.14,1.307)--cycle;
\draw(-9.146,1.292)--(-9.236,1.191);
\draw(-9.224,1.212)--(-9.14,1.307);
\filldraw[fill opacity=0.8,fill=gray!20,draw=none](-9.14,1.307)--(-9.135,1.32)--(-9.155,1.342)--(-9.152,1.319)--cycle;
\draw(-9.135,1.32)--(-9.155,1.342)--(-9.152,1.319);
\filldraw[fill opacity=0.8,fill=gray!20,draw=none](-9.367,.934)--(-9.37,.933)--(-9.392,.969)--(-9.393,.972)--(-9.389,.975)--cycle;
\draw(-9.367,.934)--(-9.37,.933);
\draw(-9.393,.972)--(-9.389,.975);
\filldraw[fill opacity=0.8,fill=gray!20,draw=none](-9.367,.934)--(-9.37,.933)--(-9.392,.969)--(-9.393,.972)--(-9.389,.975)--cycle;
\draw(-9.367,.934)--(-9.37,.933);
\draw(-9.393,.972)--(-9.389,.975);
\filldraw[fill opacity=0.8,fill=gray!20](-9.046,1.526)--(-8.993,1.547)--(-8.983,1.553)--(-9.028,1.538)--cycle;
\filldraw[fill opacity=0.8,fill=gray!20](-8.889,1.125)--(-8.844,1.141)--(-8.887,1.132)--(-8.911,1.121)--cycle;
\filldraw[fill opacity=0.8,fill=gray!20,draw=none](-9.385,.953)--(-9.372,.932)--(-9.314,.907)--(-9.339,.946)--(-9.393,.969)--cycle;
\draw(-9.372,.932)--(-9.314,.907)--(-9.339,.946)--(-9.393,.969);
\filldraw[fill opacity=0.8,fill=gray!20,draw=none](-9.146,1.292)--(-9.246,1.179)--(-9.246,1.181)--(-9.235,1.2)--(-9.14,1.307)--cycle;
\draw(-9.146,1.292)--(-9.246,1.179);
\draw(-9.235,1.2)--(-9.14,1.307);
\filldraw[fill opacity=0.8,fill=gray!20,draw=none](-9.146,1.292)--(-9.246,1.179)--(-9.246,1.181)--(-9.235,1.2)--(-9.14,1.307)--cycle;
\draw(-9.146,1.292)--(-9.246,1.179);
\draw(-9.235,1.2)--(-9.14,1.307);
\filldraw[fill opacity=0.8,fill=gray!20,draw=none](-9.254,1.168)--(-9.246,1.167)--(-9.255,1.166)--cycle;
\filldraw[fill opacity=0.8,fill=gray!20,draw=none](-9.249,1.176)--(-9.242,1.168)--(-9.255,1.166)--cycle;
\filldraw[fill opacity=0.8,fill=gray!20,draw=none](-9.236,1.191)--(-9.246,1.179)--(-9.246,1.181)--(-9.235,1.2)--(-9.224,1.212)--cycle;
\draw(-9.236,1.191)--(-9.246,1.179);
\draw(-9.235,1.2)--(-9.224,1.212);
\filldraw[fill opacity=0.8,fill=gray!20,draw=none](-9.242,1.168)--(-9.249,1.176)--(-9.246,1.181)--(-9.225,1.182)--(-9.224,1.171)--cycle;
\draw(-9.246,1.181)--(-9.225,1.182)--(-9.224,1.171);
\filldraw[fill opacity=0.8,fill=gray!20,draw=none](-9.241,1.167)--(-9.249,1.176)--(-9.246,1.181)--(-9.225,1.182)--(-9.223,1.165)--cycle;
\draw(-9.246,1.181)--(-9.225,1.182)--(-9.223,1.165);
\filldraw[fill opacity=0.8,fill=gray!20,draw=none](-9.397,.959)--(-9.409,.964)--(-9.381,.921)--(-9.371,.917)--cycle;
\draw(-9.397,.959)--(-9.409,.964);
\draw(-9.381,.921)--(-9.371,.917);
\filldraw[fill opacity=0.8,fill=gray!20](-9.132,1.318)--(-9.126,1.374)--(-9.148,1.398)--(-9.155,1.342)--cycle;
\filldraw[fill opacity=0.8,fill=gray!20,draw=none](-8.761,1.22)--(-8.746,1.23)--(-8.724,1.281)--(-8.735,1.274)--cycle;
\draw(-8.761,1.22)--(-8.746,1.23)--(-8.724,1.281)--(-8.735,1.274);
\filldraw[fill opacity=0.8,fill=gray!20,draw=none](-9.195,1.16)--(-9.159,1.157)--(-9.157,1.152)--cycle;
\draw(-9.195,1.16)--(-9.159,1.157)--(-9.157,1.152);
\filldraw[fill opacity=0.8,fill=gray!20,draw=none](-9.194,1.159)--(-9.159,1.157)--(-9.157,1.152)--cycle;
\draw(-9.194,1.159)--(-9.159,1.157)--(-9.157,1.152);
\filldraw[fill opacity=0.8,fill=gray!20,draw=none](-8.955,1.12)--(-8.957,1.118)--(-8.963,1.122)--cycle;
\draw(-8.955,1.12)--(-8.957,1.118);
\filldraw[fill opacity=0.8,fill=gray!20,draw=none](-8.955,1.12)--(-8.957,1.118)--(-8.963,1.122)--cycle;
\draw(-8.955,1.12)--(-8.957,1.118);
\filldraw[fill opacity=0.8,fill=gray!20,draw=none](-8.955,1.12)--(-8.957,1.118)--(-8.963,1.122)--cycle;
\draw(-8.955,1.12)--(-8.957,1.118);
\filldraw[fill opacity=0.8,fill=gray!20,draw=none](-8.838,1.261)--(-8.951,1.133)--(-8.952,1.13)--(-8.952,1.123)--(-8.839,1.251)--cycle;
\draw(-8.952,1.123)--(-8.839,1.251)--(-8.838,1.261)--(-8.951,1.133);
\filldraw[fill opacity=0.8,fill=gray!20,draw=none](-8.838,1.261)--(-8.951,1.133)--(-8.952,1.13)--(-8.952,1.123)--(-8.839,1.251)--cycle;
\draw(-8.952,1.123)--(-8.839,1.251)--(-8.838,1.261)--(-8.951,1.133);
\filldraw[fill opacity=0.8,fill=gray!20,draw=none](-8.838,1.261)--(-8.951,1.133)--(-8.952,1.13)--(-8.952,1.123)--(-8.839,1.251)--cycle;
\draw(-8.952,1.123)--(-8.839,1.251)--(-8.838,1.261)--(-8.951,1.133);
\filldraw[fill opacity=0.8,fill=gray!20,draw=none](-9.151,1.126)--(-9.146,1.125)--(-9.133,1.124)--(-9.228,1.165)--(-9.246,1.167)--cycle;
\draw(-9.133,1.124)--(-9.228,1.165);
\filldraw[fill opacity=0.8,fill=gray!20,draw=none](-9.073,.978)--(-9.058,1.004)--(-9.049,.996)--(-9.054,.959)--cycle;
\draw(-9.049,.996)--(-9.054,.959)--(-9.073,.978);
\filldraw[fill opacity=0.8,fill=gray!20,draw=none](-9.073,.978)--(-9.058,1.004)--(-9.049,.996)--(-9.054,.959)--cycle;
\draw(-9.049,.996)--(-9.054,.959)--(-9.073,.978);
\filldraw[fill opacity=0.8,fill=gray!20](-8.933,1.559)--(-8.936,1.554)--(-8.936,1.554)--(-8.905,1.557)--cycle;
\filldraw[fill opacity=0.8,fill=gray!20](-8.961,1.557)--(-8.936,1.554)--(-8.936,1.554)--(-8.933,1.559)--cycle;
\filldraw[fill opacity=0.8,fill=gray!20,draw=none](-9.385,.953)--(-9.393,.969)--(-9.397,.971)--cycle;
\draw(-9.393,.969)--(-9.397,.971);
\filldraw[fill opacity=0.8,fill=gray!20,draw=none](-9.392,.969)--(-9.37,.933)--(-9.378,.927)--cycle;
\draw(-9.37,.933)--(-9.378,.927);
\filldraw[fill opacity=0.8,fill=gray!20,draw=none](-9.392,.969)--(-9.37,.933)--(-9.378,.927)--cycle;
\draw(-9.37,.933)--(-9.378,.927);
\filldraw[fill opacity=0.8,fill=gray!20,draw=none](-9.152,.923)--(-9.151,.922)--(-9.186,.893)--(-9.204,.901)--cycle;
\draw(-9.151,.922)--(-9.186,.893)--(-9.204,.901);
\filldraw[fill opacity=0.8,fill=gray!20,draw=none](-9.298,1.157)--(-9.287,1.16)--(-9.288,1.159)--cycle;
\draw(-9.287,1.16)--(-9.288,1.159)--(-9.298,1.157);
\filldraw[fill opacity=0.8,fill=gray!20,draw=none](-9.355,1.126)--(-9.353,1.129)--(-9.33,1.145)--(-9.298,1.157)--(-9.288,1.159)--cycle;
\draw(-9.355,1.126)--(-9.353,1.129);
\draw(-9.298,1.157)--(-9.288,1.159);
\filldraw[fill opacity=0.8,fill=gray!20,draw=none](-9.355,1.126)--(-9.353,1.129)--(-9.33,1.145)--(-9.298,1.157)--(-9.288,1.159)--cycle;
\draw(-9.355,1.126)--(-9.353,1.129);
\draw(-9.298,1.157)--(-9.288,1.159);
\filldraw[fill opacity=0.8,fill=gray!20,draw=none](-9.298,1.157)--(-9.287,1.16)--(-9.288,1.159)--cycle;
\draw(-9.287,1.16)--(-9.288,1.159)--(-9.298,1.157);
\filldraw[fill opacity=0.8,fill=gray!20](-9.231,.829)--(-9.252,.844)--(-9.228,.845)--(-9.231,.829)--cycle;
\filldraw[fill opacity=0.8,fill=gray!20](-9.231,.829)--(-9.228,.845)--(-9.205,.843)--(-9.231,.829)--cycle;
\filldraw[fill opacity=0.8,fill=gray!20](-9.231,.829)--(-9.228,.845)--(-9.205,.843)--(-9.231,.829)--cycle;
\filldraw[fill opacity=0.8,fill=gray!20](-9.231,.829)--(-9.252,.844)--(-9.228,.845)--(-9.231,.829)--cycle;
\filldraw[fill opacity=0.8,fill=gray!20,draw=none](-9.305,1.156)--(-9.295,1.132)--(-9.27,1.153)--(-9.288,1.161)--cycle;
\draw(-9.27,1.153)--(-9.288,1.161);
\filldraw[fill opacity=0.8,fill=gray!20](-9.278,.835)--(-9.322,.854)--(-9.307,.863)--(-9.27,.84)--cycle;
\filldraw[fill opacity=0.8,fill=gray!20](-9.278,.835)--(-9.322,.854)--(-9.307,.863)--(-9.27,.84)--cycle;
\filldraw[fill opacity=0.8,fill=gray!20](-8.838,1.536)--(-8.885,1.552)--(-8.879,1.546)--(-8.826,1.523)--cycle;
\filldraw[fill opacity=0.8,fill=gray!20,draw=none](-9.287,1.175)--(-9.288,1.175)--(-9.286,1.174)--cycle;
\draw(-9.288,1.175)--(-9.286,1.174);
\filldraw[fill opacity=0.8,fill=gray!20,draw=none](-9.298,1.157)--(-9.339,1.149)--(-9.328,1.154)--(-9.279,1.171)--(-9.287,1.16)--cycle;
\draw(-9.298,1.157)--(-9.339,1.149);
\draw(-9.279,1.171)--(-9.287,1.16);
\filldraw[fill opacity=0.8,fill=gray!20,draw=none](-9.298,1.157)--(-9.339,1.149)--(-9.328,1.154)--(-9.279,1.171)--(-9.287,1.16)--cycle;
\draw(-9.298,1.157)--(-9.339,1.149);
\draw(-9.279,1.171)--(-9.287,1.16);
\filldraw[fill opacity=0.8,fill=gray!20,draw=none](-9.242,1.168)--(-9.241,1.167)--(-9.244,1.167)--cycle;
\filldraw[fill opacity=0.8,fill=gray!20,draw=none](-9.241,1.167)--(-9.279,1.171)--(-9.274,1.176)--(-9.268,1.18)--(-9.253,1.18)--cycle;
\draw(-9.279,1.171)--(-9.274,1.176);
\draw(-9.268,1.18)--(-9.253,1.18);
\filldraw[fill opacity=0.8,fill=gray!20,draw=none](-9.254,1.168)--(-9.279,1.171)--(-9.263,1.164)--cycle;
\draw(-9.279,1.171)--(-9.263,1.164);
\filldraw[fill opacity=0.8,fill=gray!20,draw=none](-9.249,1.176)--(-9.267,1.156)--(-9.27,1.153)--(-9.281,1.149)--(-9.278,1.152)--cycle;
\draw(-9.249,1.176)--(-9.267,1.156);
\draw(-9.281,1.149)--(-9.278,1.152);
\filldraw[fill opacity=0.8,fill=gray!20,draw=none](-9.263,1.164)--(-9.257,1.167)--(-9.269,1.154)--(-9.281,1.149)--(-9.278,1.152)--cycle;
\draw(-9.257,1.167)--(-9.269,1.154);
\draw(-9.281,1.149)--(-9.278,1.152);
\filldraw[fill opacity=0.8,fill=gray!20,draw=none](-9.263,1.164)--(-9.257,1.167)--(-9.269,1.154)--(-9.281,1.149)--(-9.278,1.152)--cycle;
\draw(-9.257,1.167)--(-9.269,1.154);
\draw(-9.281,1.149)--(-9.278,1.152);
\filldraw[fill opacity=0.8,fill=gray!20,draw=none](-9.174,1.13)--(-9.151,1.126)--(-9.246,1.167)--(-9.254,1.168)--(-9.263,1.164)--(-9.198,1.136)--cycle;
\draw(-9.263,1.164)--(-9.198,1.136);
\filldraw[fill opacity=0.8,fill=gray!20,draw=none](-9.242,1.168)--(-9.244,1.167)--(-9.256,1.167)--(-9.257,1.167)--(-9.249,1.176)--cycle;
\draw(-9.257,1.167)--(-9.249,1.176);
\filldraw[fill opacity=0.8,fill=gray!20,draw=none](-9.241,1.167)--(-9.256,1.167)--(-9.257,1.167)--(-9.249,1.176)--cycle;
\draw(-9.257,1.167)--(-9.249,1.176);
\filldraw[fill opacity=0.8,fill=gray!20,draw=none](-9.241,1.167)--(-9.256,1.167)--(-9.257,1.167)--(-9.249,1.176)--cycle;
\draw(-9.257,1.167)--(-9.249,1.176);
\filldraw[fill opacity=0.8,fill=gray!20,draw=none](-9.285,1.172)--(-9.286,1.174)--(-9.288,1.175)--(-9.338,1.159)--(-9.33,1.155)--cycle;
\draw(-9.286,1.174)--(-9.288,1.175);
\draw(-9.338,1.159)--(-9.33,1.155);
\filldraw[fill opacity=0.8,fill=gray!20](-8.983,1.553)--(-8.936,1.554)--(-8.936,1.554)--(-8.961,1.557)--cycle;
\filldraw[fill opacity=0.8,fill=gray!20](-9.231,.829)--(-9.27,.84)--(-9.252,.844)--(-9.231,.829)--cycle;
\filldraw[fill opacity=0.8,fill=gray!20](-9.231,.829)--(-9.27,.84)--(-9.252,.844)--(-9.231,.829)--cycle;
\filldraw[fill opacity=0.8,fill=gray!20](-9.159,1.157)--(-9.18,1.178)--(-9.149,1.171)--(-9.115,1.146)--cycle;
\filldraw[fill opacity=0.8,fill=gray!20](-9.159,1.157)--(-9.18,1.178)--(-9.149,1.171)--(-9.115,1.146)--cycle;
\filldraw[fill opacity=0.8,fill=gray!20](-9.231,.829)--(-9.205,.843)--(-9.188,.839)--(-9.231,.829)--cycle;
\filldraw[fill opacity=0.8,fill=gray!20](-9.231,.829)--(-9.205,.843)--(-9.188,.839)--(-9.231,.829)--cycle;
\filldraw[fill opacity=0.8,fill=gray!20,draw=none](-9.304,1.154)--(-9.303,1.156)--(-9.298,1.157)--cycle;
\draw(-9.303,1.156)--(-9.298,1.157);
\filldraw[fill opacity=0.8,fill=gray!20,draw=none](-9.304,1.154)--(-9.303,1.156)--(-9.298,1.157)--cycle;
\draw(-9.303,1.156)--(-9.298,1.157);
\filldraw[fill opacity=0.8,fill=gray!20,draw=none](-9.295,1.132)--(-9.305,1.156)--(-9.333,1.146)--(-9.297,1.13)--cycle;
\draw(-9.333,1.146)--(-9.297,1.13);
\filldraw[fill opacity=0.8,fill=gray!20,draw=none](-8.939,1.12)--(-8.942,1.127)--(-8.95,1.13)--(-8.997,1.134)--(-8.967,1.122)--cycle;
\draw(-8.95,1.13)--(-8.997,1.134)--(-8.967,1.122)--(-8.939,1.12)--(-8.942,1.127);
\filldraw[fill opacity=0.8,fill=gray!20](-9.126,1.374)--(-9.106,1.428)--(-9.126,1.449)--(-9.148,1.398)--cycle;
\filldraw[fill opacity=0.8,fill=gray!20,draw=none](-8.735,1.274)--(-8.724,1.281)--(-8.717,1.336)--(-8.725,1.331)--cycle;
\draw(-8.735,1.274)--(-8.724,1.281)--(-8.717,1.336)--(-8.725,1.331);
\filldraw[fill opacity=0.8,fill=gray!20](-8.905,1.557)--(-8.936,1.554)--(-8.936,1.554)--(-8.885,1.552)--cycle;
\filldraw[fill opacity=0.8,fill=gray!20,draw=none](-9.283,1.173)--(-9.286,1.174)--(-9.285,1.172)--cycle;
\draw(-9.283,1.173)--(-9.286,1.174);
\filldraw[fill opacity=0.8,fill=gray!20,draw=none](-8.952,1.123)--(-8.955,1.12)--(-8.96,1.121)--cycle;
\draw(-8.952,1.123)--(-8.955,1.12);
\filldraw[fill opacity=0.8,fill=gray!20,draw=none](-8.952,1.123)--(-8.955,1.12)--(-8.96,1.121)--cycle;
\draw(-8.952,1.123)--(-8.955,1.12);
\filldraw[fill opacity=0.8,fill=gray!20,draw=none](-8.952,1.123)--(-8.955,1.12)--(-8.96,1.121)--cycle;
\draw(-8.952,1.123)--(-8.955,1.12);
\filldraw[fill opacity=0.8,fill=gray!20,draw=none](-9.389,.975)--(-9.393,.972)--(-9.401,1.017)--(-9.396,1.021)--cycle;
\draw(-9.389,.975)--(-9.393,.972);
\draw(-9.401,1.017)--(-9.396,1.021);
\filldraw[fill opacity=0.8,fill=gray!20,draw=none](-9.389,.975)--(-9.393,.972)--(-9.401,1.017)--(-9.396,1.021)--cycle;
\draw(-9.389,.975)--(-9.393,.972);
\draw(-9.401,1.017)--(-9.396,1.021);
\filldraw[fill opacity=0.8,fill=gray!20](-8.911,1.121)--(-8.887,1.132)--(-8.943,1.13)--(-8.939,1.12)--cycle;
\filldraw[fill opacity=0.8,fill=gray!20,draw=none](-9.304,1.154)--(-9.33,1.145)--(-9.319,1.153)--(-9.303,1.156)--cycle;
\draw(-9.319,1.153)--(-9.303,1.156);
\filldraw[fill opacity=0.8,fill=gray!20,draw=none](-9.304,1.154)--(-9.33,1.145)--(-9.319,1.153)--(-9.303,1.156)--cycle;
\draw(-9.319,1.153)--(-9.303,1.156);
\filldraw[fill opacity=0.8,fill=gray!20,draw=none](-9.401,.992)--(-9.397,.971)--(-9.339,.946)--(-9.348,.994)--(-9.401,1.017)--cycle;
\draw(-9.397,.971)--(-9.339,.946)--(-9.348,.994)--(-9.401,1.017);
\filldraw[fill opacity=0.8,fill=gray!20](-8.997,1.134)--(-9.022,1.159)--(-9.075,1.173)--(-9.034,1.143)--cycle;
\filldraw[fill opacity=0.8,fill=gray!20,draw=none](-9.392,.969)--(-9.378,.927)--(-9.385,.923)--(-9.4,.968)--(-9.394,.972)--cycle;
\draw(-9.378,.927)--(-9.385,.923);
\draw(-9.4,.968)--(-9.394,.972);
\filldraw[fill opacity=0.8,fill=gray!20,draw=none](-9.392,.969)--(-9.378,.927)--(-9.385,.923)--(-9.4,.968)--(-9.394,.972)--cycle;
\draw(-9.378,.927)--(-9.385,.923);
\draw(-9.4,.968)--(-9.394,.972);
\filldraw[fill opacity=0.8,fill=gray!20,draw=none](-9.352,.939)--(-9.397,.959)--(-9.371,.917)--(-9.323,.896)--cycle;
\draw(-9.352,.939)--(-9.397,.959);
\draw(-9.371,.917)--(-9.323,.896);
\filldraw[fill opacity=0.8,fill=gray!20,draw=none](-9.348,.89)--(-9.36,.883)--(-9.389,.92)--(-9.377,.927)--cycle;
\draw(-9.348,.89)--(-9.36,.883)--(-9.389,.92)--(-9.377,.927);
\filldraw[fill opacity=0.8,fill=gray!20,draw=none](-9.348,.89)--(-9.36,.883)--(-9.389,.92)--(-9.377,.927)--cycle;
\draw(-9.348,.89)--(-9.36,.883)--(-9.389,.92)--(-9.377,.927);
\filldraw[fill opacity=0.8,fill=gray!20,draw=none](-9.263,1.164)--(-9.283,1.173)--(-9.285,1.172)--(-9.278,1.152)--cycle;
\draw(-9.263,1.164)--(-9.283,1.173);
\filldraw[fill opacity=0.8,fill=gray!20,draw=none](-9.28,1.159)--(-9.285,1.172)--(-9.299,1.167)--cycle;
\filldraw[fill opacity=0.8,fill=gray!20,draw=none](-9.401,.992)--(-9.401,1.017)--(-9.406,1.019)--cycle;
\draw(-9.401,1.017)--(-9.406,1.019);
\filldraw[fill opacity=0.8,fill=gray!20,draw=none](-9.392,.969)--(-9.394,.972)--(-9.393,.972)--cycle;
\draw(-9.394,.972)--(-9.393,.972);
\filldraw[fill opacity=0.8,fill=gray!20,draw=none](-9.392,.969)--(-9.394,.972)--(-9.393,.972)--cycle;
\draw(-9.394,.972)--(-9.393,.972);
\filldraw[fill opacity=0.8,fill=gray!20,draw=none](-9.408,1.013)--(-9.419,1.018)--(-9.409,.964)--(-9.407,.964)--cycle;
\draw(-9.408,1.013)--(-9.419,1.018);
\draw(-9.409,.964)--(-9.407,.964);
\filldraw[fill opacity=0.8,fill=gray!20,draw=none](-9.033,1.428)--(-9.162,1.283)--(-9.13,1.297)--(-9.016,1.425)--cycle;
\draw(-9.13,1.297)--(-9.016,1.425)--(-9.033,1.428)--(-9.162,1.283);
\filldraw[fill opacity=0.8,fill=gray!20,draw=none](-9.033,1.428)--(-9.162,1.283)--(-9.13,1.297)--(-9.016,1.425)--cycle;
\draw(-9.13,1.297)--(-9.016,1.425)--(-9.033,1.428)--(-9.162,1.283);
\filldraw[fill opacity=0.8,fill=gray!20,draw=none](-9.162,1.283)--(-9.235,1.2)--(-9.193,1.226)--(-9.13,1.297)--cycle;
\draw(-9.162,1.283)--(-9.235,1.2);
\draw(-9.193,1.226)--(-9.13,1.297);
\filldraw[fill opacity=0.8,fill=gray!20,draw=none](-9.162,1.283)--(-9.235,1.2)--(-9.193,1.226)--(-9.13,1.297)--cycle;
\draw(-9.162,1.283)--(-9.235,1.2);
\draw(-9.193,1.226)--(-9.13,1.297);
\filldraw[fill opacity=0.8,fill=gray!20,draw=none](-9.033,1.428)--(-9.162,1.283)--(-9.13,1.297)--(-9.016,1.425)--cycle;
\draw(-9.13,1.297)--(-9.016,1.425)--(-9.033,1.428)--(-9.162,1.283);
\filldraw[fill opacity=0.8,fill=gray!20,draw=none](-9.152,.923)--(-9.147,.925)--(-9.151,.922)--cycle;
\draw(-9.147,.925)--(-9.151,.922);
\filldraw[fill opacity=0.8,fill=gray!20,draw=none](-9.311,1.092)--(-9.276,1.121)--(-9.231,1.136)--(-9.186,1.131)--(-9.147,1.107)--(-9.122,1.068)--(-9.113,1.02)--(-9.122,.97)--(-9.147,.925)--(-9.151,.922)--cycle;
\draw(-9.311,1.092)--(-9.276,1.121)--(-9.231,1.136)--(-9.186,1.131)--(-9.147,1.107)--(-9.122,1.068)--(-9.113,1.02)--(-9.122,.97)--(-9.147,.925)--(-9.151,.922);
\filldraw[fill opacity=0.8,fill=gray!20,draw=none](-9.122,1.068)--(-9.147,1.107)--(-9.186,1.131)--(-9.231,1.136)--(-9.276,1.121)--(-9.311,1.092)--(-9.151,.922)--(-9.147,.925)--(-9.122,.97)--(-9.113,1.02)--cycle;
\draw(-9.151,.922)--(-9.147,.925)--(-9.122,.97)--(-9.113,1.02)--(-9.122,1.068)--(-9.147,1.107)--(-9.186,1.131)--(-9.231,1.136)--(-9.276,1.121)--(-9.311,1.092);
\filldraw[fill opacity=0.8,fill=gray!20,draw=none](-9.072,.996)--(-9.098,.959)--(-9.122,.97)--(-9.113,1.02)--(-9.07,1.001)--cycle;
\draw(-9.098,.959)--(-9.122,.97)--(-9.113,1.02)--(-9.07,1.001);
\filldraw[fill opacity=0.8,fill=gray!20,draw=none](-9.349,1.133)--(-9.372,1.114)--(-9.368,1.113)--cycle;
\draw(-9.372,1.114)--(-9.368,1.113);
\filldraw[fill opacity=0.8,fill=gray!20,draw=none](-9.367,1.113)--(-9.37,1.111)--(-9.344,1.143)--(-9.339,1.149)--(-9.355,1.126)--cycle;
\draw(-9.367,1.113)--(-9.37,1.111);
\draw(-9.339,1.149)--(-9.355,1.126);
\filldraw[fill opacity=0.8,fill=gray!20,draw=none](-9.367,1.113)--(-9.37,1.111)--(-9.344,1.143)--(-9.339,1.149)--(-9.355,1.126)--cycle;
\draw(-9.367,1.113)--(-9.37,1.111);
\draw(-9.339,1.149)--(-9.355,1.126);
\filldraw[fill opacity=0.8,fill=gray!20,draw=none](-9.389,1.068)--(-9.393,1.065)--(-9.392,1.067)--(-9.37,1.111)--(-9.367,1.113)--cycle;
\draw(-9.389,1.068)--(-9.393,1.065);
\draw(-9.37,1.111)--(-9.367,1.113);
\filldraw[fill opacity=0.8,fill=gray!20,draw=none](-9.389,1.068)--(-9.393,1.065)--(-9.392,1.067)--(-9.37,1.111)--(-9.367,1.113)--cycle;
\draw(-9.389,1.068)--(-9.393,1.065);
\draw(-9.37,1.111)--(-9.367,1.113);
\filldraw[fill opacity=0.8,fill=gray!20,draw=none](-9.249,1.176)--(-9.279,1.171)--(-9.272,1.18)--(-9.253,1.18)--cycle;
\draw(-9.279,1.171)--(-9.272,1.18)--(-9.253,1.18);
\filldraw[fill opacity=0.8,fill=gray!20,draw=none](-9.246,1.179)--(-9.249,1.176)--(-9.246,1.181)--cycle;
\draw(-9.246,1.179)--(-9.249,1.176);
\filldraw[fill opacity=0.8,fill=gray!20,draw=none](-9.246,1.179)--(-9.249,1.176)--(-9.246,1.181)--cycle;
\draw(-9.246,1.179)--(-9.249,1.176);
\filldraw[fill opacity=0.8,fill=gray!20,draw=none](-9.246,1.179)--(-9.247,1.178)--(-9.247,1.179)--(-9.246,1.181)--cycle;
\draw(-9.246,1.179)--(-9.247,1.178);
\filldraw[fill opacity=0.8,fill=gray!20,draw=none](-9.247,1.178)--(-9.249,1.176)--(-9.247,1.179)--cycle;
\draw(-9.247,1.178)--(-9.249,1.176);
\filldraw[fill opacity=0.8,fill=gray!20,draw=none](-9.256,1.167)--(-9.257,1.167)--(-9.257,1.167)--cycle;
\draw(-9.257,1.167)--(-9.257,1.167);
\filldraw[fill opacity=0.8,fill=gray!20,draw=none](-9.256,1.167)--(-9.257,1.167)--(-9.257,1.167)--cycle;
\draw(-9.257,1.167)--(-9.257,1.167);
\filldraw[fill opacity=0.8,fill=gray!20,draw=none](-9.256,1.167)--(-9.257,1.167)--(-9.257,1.167)--cycle;
\draw(-9.257,1.167)--(-9.257,1.167);
\filldraw[fill opacity=0.8,fill=gray!20,draw=none](-9.249,1.176)--(-9.253,1.18)--(-9.246,1.181)--cycle;
\draw(-9.253,1.18)--(-9.246,1.181);
\filldraw[fill opacity=0.8,fill=gray!20,draw=none](-9.249,1.176)--(-9.253,1.18)--(-9.246,1.181)--cycle;
\draw(-9.253,1.18)--(-9.246,1.181);
\filldraw[fill opacity=0.8,fill=gray!20,draw=none](-9.349,1.133)--(-9.368,1.113)--(-9.33,1.096)--(-9.297,1.13)--(-9.333,1.146)--cycle;
\draw(-9.368,1.113)--(-9.33,1.096);
\draw(-9.297,1.13)--(-9.333,1.146);
\filldraw[fill opacity=0.8,fill=gray!20](-9.231,.829)--(-9.278,.835)--(-9.27,.84)--(-9.231,.829)--cycle;
\filldraw[fill opacity=0.8,fill=gray!20](-9.231,.829)--(-9.278,.835)--(-9.27,.84)--(-9.231,.829)--cycle;
\filldraw[fill opacity=0.8,fill=gray!20](-9.106,1.428)--(-9.075,1.476)--(-9.091,1.493)--(-9.126,1.449)--cycle;
\filldraw[fill opacity=0.8,fill=gray!20,draw=none](-8.725,1.331)--(-8.717,1.336)--(-8.724,1.392)--(-8.735,1.385)--cycle;
\draw(-8.725,1.331)--(-8.717,1.336)--(-8.724,1.392)--(-8.735,1.385);
\filldraw[fill opacity=0.8,fill=gray!20,draw=none](-9.153,.923)--(-9.15,.922)--(-9.149,.922)--cycle;
\draw(-9.153,.923)--(-9.15,.922)--(-9.149,.922);
\filldraw[fill opacity=0.8,fill=gray!20,draw=none](-9.408,1.013)--(-9.407,.964)--(-9.398,.96)--cycle;
\draw(-9.407,.964)--(-9.398,.96);
\filldraw[fill opacity=0.8,fill=gray!20,draw=none](-9.33,1.145)--(-9.346,1.139)--(-9.339,1.149)--(-9.319,1.153)--cycle;
\draw(-9.346,1.139)--(-9.339,1.149)--(-9.319,1.153);
\filldraw[fill opacity=0.8,fill=gray!20,draw=none](-9.33,1.145)--(-9.346,1.139)--(-9.339,1.149)--(-9.319,1.153)--cycle;
\draw(-9.346,1.139)--(-9.339,1.149)--(-9.319,1.153);
\filldraw[fill opacity=0.8,fill=gray!20,draw=none](-9.27,1.153)--(-9.295,1.132)--(-9.281,1.149)--cycle;
\draw(-9.295,1.132)--(-9.281,1.149);
\filldraw[fill opacity=0.8,fill=gray!20,draw=none](-9.27,1.153)--(-9.295,1.132)--(-9.281,1.149)--cycle;
\draw(-9.295,1.132)--(-9.281,1.149);
\filldraw[fill opacity=0.8,fill=gray!20,draw=none](-9.27,1.153)--(-9.295,1.132)--(-9.281,1.149)--cycle;
\draw(-9.295,1.132)--(-9.281,1.149);
\filldraw[fill opacity=0.8,fill=gray!20,draw=none](-9.295,1.132)--(-9.295,1.129)--(-9.276,1.121)--(-9.231,1.136)--(-9.27,1.153)--cycle;
\draw(-9.295,1.129)--(-9.276,1.121)--(-9.231,1.136)--(-9.27,1.153);
\filldraw[fill opacity=0.8,fill=gray!20,draw=none](-9.278,1.152)--(-9.28,1.159)--(-9.299,1.167)--(-9.33,1.155)--(-9.293,1.139)--cycle;
\draw(-9.33,1.155)--(-9.293,1.139);
\filldraw[fill opacity=0.8,fill=gray!20,draw=none](-9.328,1.154)--(-9.288,1.176)--(-9.272,1.18)--(-9.279,1.171)--cycle;
\draw(-9.288,1.176)--(-9.272,1.18)--(-9.279,1.171);
\filldraw[fill opacity=0.8,fill=gray!20,draw=none](-9.328,1.154)--(-9.288,1.176)--(-9.272,1.18)--(-9.279,1.171)--cycle;
\draw(-9.288,1.176)--(-9.272,1.18)--(-9.279,1.171);
\filldraw[fill opacity=0.8,fill=gray!20,draw=none](-9.084,.908)--(-9.093,.897)--(-9.105,.902)--(-9.106,.945)--(-9.095,.959)--(-9.064,.946)--cycle;
\draw(-9.093,.897)--(-9.105,.902);
\draw(-9.095,.959)--(-9.064,.946);
\filldraw[fill opacity=0.8,fill=gray!20,draw=none](-9.096,.899)--(-9.149,.922)--(-9.147,.925)--(-9.093,.902)--cycle;
\draw(-9.147,.925)--(-9.093,.902);
\filldraw[fill opacity=0.8,fill=gray!20,draw=none](-9.149,.922)--(-9.15,.922)--(-9.147,.925)--cycle;
\draw(-9.149,.922)--(-9.15,.922)--(-9.147,.925);
\filldraw[fill opacity=0.8,fill=gray!20,draw=none](-9.103,.953)--(-9.134,.919)--(-9.149,.922)--(-9.147,.925)--cycle;
\draw(-9.103,.953)--(-9.134,.919)--(-9.149,.922);
\filldraw[fill opacity=0.8,fill=gray!20,draw=none](-9.103,.953)--(-9.134,.919)--(-9.15,.922)--(-9.147,.925)--cycle;
\draw(-9.103,.953)--(-9.134,.919)--(-9.15,.922)--(-9.147,.925);
\filldraw[fill opacity=0.8,fill=gray!20,draw=none](-9.103,.953)--(-9.134,.919)--(-9.15,.922)--(-9.147,.925)--cycle;
\draw(-9.103,.953)--(-9.134,.919)--(-9.15,.922)--(-9.147,.925);
\filldraw[fill opacity=0.8,fill=gray!20,draw=none](-9.149,.922)--(-9.151,.922)--(-9.147,.925)--cycle;
\draw(-9.151,.922)--(-9.147,.925);
\filldraw[fill opacity=0.8,fill=gray!20](-8.993,1.547)--(-8.936,1.554)--(-8.936,1.554)--(-8.983,1.553)--cycle;
\filldraw[fill opacity=0.8,fill=gray!20,draw=none](-9.393,.972)--(-9.407,.963)--(-9.408,.972)--(-9.408,1.013)--(-9.401,1.017)--cycle;
\draw(-9.393,.972)--(-9.407,.963)--(-9.408,.972);
\draw(-9.408,1.013)--(-9.401,1.017);
\filldraw[fill opacity=0.8,fill=gray!20,draw=none](-9.393,.972)--(-9.407,.963)--(-9.408,.972)--(-9.408,1.013)--(-9.401,1.017)--cycle;
\draw(-9.393,.972)--(-9.407,.963)--(-9.408,.972);
\draw(-9.408,1.013)--(-9.401,1.017);
\filldraw[fill opacity=0.8,fill=gray!20,draw=none](-9.181,.991)--(-9.207,1.016)--(-9.246,1.049)--(-9.282,1.076)--(-9.309,1.091)--(-9.311,1.092)--(-9.15,.922)--(-9.134,.919)--(-9.132,.929)--(-9.145,.951)--(-9.165,.973)--cycle;
\draw(-9.181,.991)--(-9.207,1.016)--(-9.246,1.049)--(-9.282,1.076)--(-9.309,1.091);
\draw(-9.15,.922)--(-9.134,.919)--(-9.132,.929)--(-9.145,.951)--(-9.165,.973);
\filldraw[fill opacity=0.8,fill=gray!20,draw=none](-9.181,.991)--(-9.207,1.016)--(-9.246,1.049)--(-9.282,1.076)--(-9.309,1.091)--(-9.311,1.092)--(-9.15,.922)--(-9.134,.919)--(-9.132,.929)--(-9.145,.951)--(-9.165,.973)--cycle;
\draw(-9.181,.991)--(-9.207,1.016)--(-9.246,1.049)--(-9.282,1.076)--(-9.309,1.091);
\draw(-9.15,.922)--(-9.134,.919)--(-9.132,.929)--(-9.145,.951)--(-9.165,.973);
\filldraw[fill opacity=0.8,fill=gray!20,draw=none](-9.311,1.092)--(-9.314,1.089)--(-9.339,1.044)--(-9.348,.994)--(-9.339,.946)--(-9.314,.907)--(-9.276,.883)--(-9.231,.879)--(-9.186,.893)--(-9.151,.922)--cycle;
\draw(-9.311,1.092)--(-9.314,1.089)--(-9.339,1.044)--(-9.348,.994)--(-9.339,.946)--(-9.314,.907)--(-9.276,.883)--(-9.231,.879)--(-9.186,.893)--(-9.151,.922);
\filldraw[fill opacity=0.8,fill=gray!20,draw=none](-9.181,.991)--(-9.207,1.016)--(-9.246,1.049)--(-9.282,1.076)--(-9.309,1.091)--(-9.311,1.092)--(-9.151,.922)--(-9.149,.922)--(-9.134,.919)--(-9.132,.929)--(-9.145,.951)--(-9.165,.973)--cycle;
\draw(-9.181,.991)--(-9.207,1.016)--(-9.246,1.049)--(-9.282,1.076)--(-9.309,1.091);
\draw(-9.149,.922)--(-9.134,.919)--(-9.132,.929)--(-9.145,.951)--(-9.165,.973);
\filldraw[fill opacity=0.8,fill=gray!20,draw=none](-9.339,1.044)--(-9.314,1.089)--(-9.311,1.092)--(-9.151,.922)--(-9.186,.893)--(-9.231,.879)--(-9.276,.883)--(-9.314,.907)--(-9.339,.946)--(-9.348,.994)--cycle;
\draw(-9.151,.922)--(-9.186,.893)--(-9.231,.879)--(-9.276,.883)--(-9.314,.907)--(-9.339,.946)--(-9.348,.994)--(-9.339,1.044)--(-9.314,1.089)--(-9.311,1.092);
\filldraw[fill opacity=0.8,fill=gray!20,draw=none](-9.149,.922)--(-9.167,.885)--(-9.186,.893)--(-9.151,.922)--cycle;
\draw(-9.167,.885)--(-9.186,.893)--(-9.151,.922);
\filldraw[fill opacity=0.8,fill=gray!20,draw=none](-9.396,1.021)--(-9.401,1.017)--(-9.393,1.065)--(-9.389,1.068)--cycle;
\draw(-9.396,1.021)--(-9.401,1.017);
\draw(-9.393,1.065)--(-9.389,1.068);
\filldraw[fill opacity=0.8,fill=gray!20,draw=none](-9.396,1.021)--(-9.401,1.017)--(-9.393,1.065)--(-9.389,1.068)--cycle;
\draw(-9.396,1.021)--(-9.401,1.017);
\draw(-9.393,1.065)--(-9.389,1.068);
\filldraw[fill opacity=0.8,fill=gray!20,draw=none](-9.245,1.188)--(-9.246,1.181)--(-9.268,1.18)--cycle;
\draw(-9.246,1.181)--(-9.268,1.18);
\filldraw[fill opacity=0.8,fill=gray!20,draw=none](-9.245,1.188)--(-9.246,1.181)--(-9.268,1.18)--cycle;
\draw(-9.246,1.181)--(-9.268,1.18);
\filldraw[fill opacity=0.8,fill=gray!20,draw=none](-9.246,1.181)--(-9.249,1.176)--(-9.278,1.152)--(-9.245,1.188)--cycle;
\draw(-9.278,1.152)--(-9.245,1.188);
\filldraw[fill opacity=0.8,fill=gray!20,draw=none](-9.246,1.181)--(-9.249,1.176)--(-9.257,1.167)--(-9.269,1.162)--(-9.245,1.188)--cycle;
\draw(-9.249,1.176)--(-9.257,1.167);
\draw(-9.269,1.162)--(-9.245,1.188);
\filldraw[fill opacity=0.8,fill=gray!20,draw=none](-9.247,1.179)--(-9.249,1.176)--(-9.257,1.167)--(-9.269,1.162)--(-9.25,1.183)--cycle;
\draw(-9.249,1.176)--(-9.257,1.167);
\draw(-9.269,1.162)--(-9.25,1.183);
\filldraw[fill opacity=0.8,fill=gray!20](-9.231,.829)--(-9.188,.839)--(-9.183,.834)--(-9.231,.829)--cycle;
\filldraw[fill opacity=0.8,fill=gray!20](-9.231,.829)--(-9.188,.839)--(-9.183,.834)--(-9.231,.829)--cycle;
\filldraw[fill opacity=0.8,fill=gray!20,draw=none](-8.844,1.141)--(-8.812,1.165)--(-8.878,1.143)--(-8.887,1.132)--cycle;
\draw(-8.878,1.143)--(-8.887,1.132)--(-8.844,1.141)--(-8.812,1.165);
\filldraw[fill opacity=0.8,fill=gray!20,draw=none](-8.952,1.13)--(-8.961,1.113)--(-8.952,1.123)--cycle;
\draw(-8.961,1.113)--(-8.952,1.123);
\filldraw[fill opacity=0.8,fill=gray!20,draw=none](-8.952,1.13)--(-8.961,1.113)--(-8.952,1.123)--cycle;
\draw(-8.961,1.113)--(-8.952,1.123);
\filldraw[fill opacity=0.8,fill=gray!20,draw=none](-8.952,1.13)--(-8.961,1.113)--(-8.952,1.123)--cycle;
\draw(-8.961,1.113)--(-8.952,1.123);
\filldraw[fill opacity=0.8,fill=gray!20,draw=none](-9.353,1.129)--(-9.346,1.139)--(-9.33,1.145)--cycle;
\draw(-9.353,1.129)--(-9.346,1.139);
\filldraw[fill opacity=0.8,fill=gray!20,draw=none](-9.353,1.129)--(-9.346,1.139)--(-9.33,1.145)--cycle;
\draw(-9.353,1.129)--(-9.346,1.139);
\filldraw[fill opacity=0.8,fill=gray!20,draw=none](-9.314,1.13)--(-9.293,1.139)--(-9.338,1.159)--(-9.348,1.151)--(-9.351,1.146)--cycle;
\draw(-9.293,1.139)--(-9.338,1.159);
\filldraw[fill opacity=0.8,fill=gray!20,draw=none](-9.068,1.042)--(-9.067,1.036)--(-9.07,1.001)--(-9.113,1.02)--(-9.122,1.068)--(-9.069,1.045)--cycle;
\draw(-9.07,1.001)--(-9.113,1.02)--(-9.122,1.068)--(-9.069,1.045);
\filldraw[fill opacity=0.8,fill=gray!20,draw=none](-8.95,1.13)--(-8.963,1.155)--(-9.022,1.159)--(-8.997,1.134)--cycle;
\draw(-8.963,1.155)--(-9.022,1.159)--(-8.997,1.134)--(-8.95,1.13);
\filldraw[fill opacity=0.8,fill=gray!20,draw=none](-8.952,1.13)--(-8.952,1.132)--(-8.984,1.095)--(-8.972,1.1)--(-8.961,1.113)--cycle;
\draw(-8.952,1.132)--(-8.984,1.095);
\draw(-8.972,1.1)--(-8.961,1.113);
\filldraw[fill opacity=0.8,fill=gray!20,draw=none](-8.984,1.095)--(-9.031,1.043)--(-9.028,1.038)--(-8.972,1.1)--cycle;
\draw(-8.984,1.095)--(-9.031,1.043);
\draw(-9.028,1.038)--(-8.972,1.1);
\filldraw[fill opacity=0.8,fill=gray!20,draw=none](-8.952,1.13)--(-8.952,1.132)--(-8.984,1.095)--(-8.972,1.1)--(-8.961,1.113)--cycle;
\draw(-8.952,1.132)--(-8.984,1.095);
\draw(-8.972,1.1)--(-8.961,1.113);
\filldraw[fill opacity=0.8,fill=gray!20,draw=none](-8.984,1.095)--(-9.031,1.043)--(-9.028,1.038)--(-8.972,1.1)--cycle;
\draw(-8.984,1.095)--(-9.031,1.043);
\draw(-9.028,1.038)--(-8.972,1.1);
\filldraw[fill opacity=0.8,fill=gray!20,draw=none](-8.952,1.13)--(-8.952,1.132)--(-8.984,1.095)--(-8.972,1.1)--(-8.961,1.113)--cycle;
\draw(-8.952,1.132)--(-8.984,1.095);
\draw(-8.972,1.1)--(-8.961,1.113);
\filldraw[fill opacity=0.8,fill=gray!20,draw=none](-9.123,.865)--(-9.133,.86)--(-9.168,.875)--(-9.132,.914)--(-9.093,.897)--cycle;
\draw(-9.133,.86)--(-9.168,.875);
\draw(-9.132,.914)--(-9.093,.897);
\filldraw[fill opacity=0.8,fill=gray!20,draw=none](-9.096,.899)--(-9.108,.886)--(-9.131,.87)--(-9.167,.885)--(-9.149,.922)--cycle;
\draw(-9.131,.87)--(-9.167,.885);
\filldraw[fill opacity=0.8,fill=gray!20,draw=none](-9.351,1.146)--(-9.348,1.151)--(-9.353,1.147)--cycle;
\filldraw[fill opacity=0.8,fill=gray!20](-8.885,1.552)--(-8.936,1.554)--(-8.936,1.554)--(-8.879,1.546)--cycle;
\filldraw[fill opacity=0.8,fill=gray!20,draw=none](-9.406,1.019)--(-9.348,.994)--(-9.339,1.044)--(-9.397,1.07)--cycle;
\draw(-9.406,1.019)--(-9.348,.994)--(-9.339,1.044)--(-9.397,1.07);
\filldraw[fill opacity=0.8,fill=gray!20,draw=none](-9.267,1.156)--(-9.269,1.154)--(-9.27,1.153)--cycle;
\draw(-9.267,1.156)--(-9.269,1.154);
\filldraw[fill opacity=0.8,fill=gray!20,draw=none](-9.246,1.181)--(-9.245,1.188)--(-9.235,1.2)--cycle;
\draw(-9.245,1.188)--(-9.235,1.2);
\filldraw[fill opacity=0.8,fill=gray!20,draw=none](-9.247,1.179)--(-9.25,1.183)--(-9.235,1.2)--cycle;
\draw(-9.25,1.183)--(-9.235,1.2);
\filldraw[fill opacity=0.8,fill=gray!20,draw=none](-9.246,1.181)--(-9.245,1.188)--(-9.235,1.2)--cycle;
\draw(-9.245,1.188)--(-9.235,1.2);
\filldraw[fill opacity=0.8,fill=gray!20,draw=none](-9.246,1.181)--(-9.245,1.188)--(-9.242,1.189)--(-9.228,1.19)--(-9.225,1.182)--cycle;
\draw(-9.242,1.189)--(-9.228,1.19)--(-9.225,1.182)--(-9.246,1.181);
\filldraw[fill opacity=0.8,fill=gray!20,draw=none](-9.246,1.181)--(-9.245,1.188)--(-9.242,1.189)--(-9.228,1.19)--(-9.225,1.182)--cycle;
\draw(-9.242,1.189)--(-9.228,1.19)--(-9.225,1.182)--(-9.246,1.181);
\filldraw[fill opacity=0.8,fill=gray!20](-9.225,1.182)--(-9.228,1.19)--(-9.205,1.188)--(-9.18,1.178)--cycle;
\filldraw[fill opacity=0.8,fill=gray!20](-9.225,1.182)--(-9.228,1.19)--(-9.205,1.188)--(-9.18,1.178)--cycle;
\filldraw[fill opacity=0.8,fill=gray!20,draw=none](-9.351,1.146)--(-9.353,1.147)--(-9.381,1.123)--(-9.371,1.119)--cycle;
\draw(-9.381,1.123)--(-9.371,1.119);
\filldraw[fill opacity=0.8,fill=gray!20,draw=none](-8.984,1.095)--(-9.031,1.043)--(-9.028,1.038)--(-8.972,1.1)--cycle;
\draw(-8.984,1.095)--(-9.031,1.043);
\draw(-9.028,1.038)--(-8.972,1.1);
\filldraw[fill opacity=0.8,fill=gray!20,draw=none](-9.274,1.176)--(-9.272,1.18)--(-9.268,1.18)--cycle;
\draw(-9.274,1.176)--(-9.272,1.18)--(-9.268,1.18);
\filldraw[fill opacity=0.8,fill=gray!20](-9.075,1.476)--(-9.034,1.514)--(-9.046,1.526)--(-9.091,1.493)--cycle;
\filldraw[fill opacity=0.8,fill=gray!20,draw=none](-8.735,1.385)--(-8.724,1.392)--(-8.746,1.444)--(-8.761,1.434)--cycle;
\draw(-8.735,1.385)--(-8.724,1.392)--(-8.746,1.444)--(-8.761,1.434);
\filldraw[fill opacity=0.8,fill=gray!20,draw=none](-9.397,1.07)--(-9.392,1.067)--(-9.369,1.113)--(-9.372,1.114)--cycle;
\draw(-9.397,1.07)--(-9.392,1.067);
\draw(-9.369,1.113)--(-9.372,1.114);
\filldraw[fill opacity=0.8,fill=gray!20,draw=none](-9.344,1.143)--(-9.351,1.146)--(-9.371,1.119)--(-9.368,1.118)--cycle;
\draw(-9.371,1.119)--(-9.368,1.118);
\filldraw[fill opacity=0.8,fill=gray!20](-9.089,1.112)--(-9.115,1.146)--(-9.101,1.132)--(-9.072,1.094)--cycle;
\filldraw[fill opacity=0.8,fill=gray!20](-9.089,1.112)--(-9.115,1.146)--(-9.101,1.132)--(-9.072,1.094)--cycle;
\filldraw[fill opacity=0.8,fill=gray!20,draw=none](-9.4,1.06)--(-9.398,1.069)--(-9.409,1.074)--(-9.419,1.018)--(-9.416,1.017)--cycle;
\draw(-9.398,1.069)--(-9.409,1.074);
\draw(-9.419,1.018)--(-9.416,1.017);
\filldraw[fill opacity=0.8,fill=gray!20,draw=none](-9.164,.857)--(-9.174,.854)--(-9.231,.879)--(-9.186,.893)--(-9.131,.87)--cycle;
\draw(-9.174,.854)--(-9.231,.879)--(-9.186,.893)--(-9.131,.87);
\filldraw[fill opacity=0.8,fill=gray!20,draw=none](-9.245,1.188)--(-9.268,1.18)--(-9.272,1.18)--(-9.252,1.189)--(-9.245,1.189)--cycle;
\draw(-9.268,1.18)--(-9.272,1.18)--(-9.252,1.189)--(-9.245,1.189);
\filldraw[fill opacity=0.8,fill=gray!20,draw=none](-9.245,1.188)--(-9.268,1.18)--(-9.272,1.18)--(-9.252,1.189)--(-9.245,1.189)--cycle;
\draw(-9.268,1.18)--(-9.272,1.18)--(-9.252,1.189)--(-9.245,1.189);
\filldraw[fill opacity=0.8,fill=gray!20,draw=none](-9.392,1.067)--(-9.339,1.044)--(-9.314,1.089)--(-9.369,1.113)--cycle;
\draw(-9.392,1.067)--(-9.339,1.044)--(-9.314,1.089)--(-9.369,1.113);
\filldraw[fill opacity=0.8,fill=gray!20,draw=none](-9.4,1.06)--(-9.416,1.017)--(-9.408,1.013)--cycle;
\draw(-9.416,1.017)--(-9.408,1.013);
\filldraw[fill opacity=0.8,fill=gray!20,draw=none](-9.401,1.017)--(-9.413,1.01)--(-9.407,1.056)--(-9.393,1.065)--cycle;
\draw(-9.401,1.017)--(-9.413,1.01)--(-9.407,1.056)--(-9.393,1.065);
\filldraw[fill opacity=0.8,fill=gray!20,draw=none](-9.401,1.017)--(-9.413,1.01)--(-9.407,1.056)--(-9.393,1.065)--cycle;
\draw(-9.401,1.017)--(-9.413,1.01)--(-9.407,1.056)--(-9.393,1.065);
\filldraw[fill opacity=0.8,fill=gray!20,draw=none](-9.392,1.067)--(-9.38,1.101)--(-9.377,1.106)--(-9.37,1.111)--cycle;
\draw(-9.377,1.106)--(-9.37,1.111);
\filldraw[fill opacity=0.8,fill=gray!20,draw=none](-9.392,1.067)--(-9.38,1.101)--(-9.377,1.106)--(-9.37,1.111)--cycle;
\draw(-9.377,1.106)--(-9.37,1.111);
\filldraw[fill opacity=0.8,fill=gray!20,draw=none](-9.37,1.111)--(-9.377,1.106)--(-9.368,1.118)--(-9.344,1.143)--cycle;
\draw(-9.37,1.111)--(-9.377,1.106);
\filldraw[fill opacity=0.8,fill=gray!20,draw=none](-9.37,1.111)--(-9.377,1.106)--(-9.368,1.118)--(-9.344,1.143)--cycle;
\draw(-9.37,1.111)--(-9.377,1.106);
\filldraw[fill opacity=0.8,fill=gray!20](-9.307,1.173)--(-9.27,1.185)--(-9.252,1.189)--(-9.272,1.18)--cycle;
\filldraw[fill opacity=0.8,fill=gray!20](-9.307,1.173)--(-9.27,1.185)--(-9.252,1.189)--(-9.272,1.18)--cycle;
\filldraw[fill opacity=0.8,fill=gray!20,draw=none](-9.377,1.106)--(-9.371,1.119)--(-9.381,1.123)--(-9.409,1.074)--(-9.405,1.072)--cycle;
\draw(-9.371,1.119)--(-9.381,1.123);
\draw(-9.409,1.074)--(-9.405,1.072);
\filldraw[fill opacity=0.8,fill=gray!20,draw=none](-9.26,.885)--(-9.287,.923)--(-9.339,.946)--(-9.314,.907)--(-9.26,.883)--cycle;
\draw(-9.287,.923)--(-9.339,.946)--(-9.314,.907)--(-9.26,.883);
\filldraw[fill opacity=0.8,fill=gray!20,draw=none](-9.287,.926)--(-9.296,.971)--(-9.348,.994)--(-9.339,.946)--(-9.287,.923)--cycle;
\draw(-9.296,.971)--(-9.348,.994)--(-9.339,.946)--(-9.287,.923);
\filldraw[fill opacity=0.8,fill=gray!20,draw=none](-9.306,.927)--(-9.312,.954)--(-9.405,.995)--(-9.398,.96)--(-9.306,.92)--cycle;
\draw(-9.398,.96)--(-9.306,.92);
\filldraw[fill opacity=0.8,fill=gray!20](-9.022,1.159)--(-9.041,1.198)--(-9.106,1.214)--(-9.075,1.173)--cycle;
\filldraw[fill opacity=0.8,fill=gray!20,draw=none](-9.084,.975)--(-9.057,.982)--(-9.061,.954)--(-9.064,.946)--(-9.095,.959)--cycle;
\draw(-9.064,.946)--(-9.095,.959);
\filldraw[fill opacity=0.8,fill=gray!20,draw=none](-9.069,1.045)--(-9.122,1.068)--(-9.147,1.107)--(-9.093,1.083)--cycle;
\draw(-9.069,1.045)--(-9.122,1.068)--(-9.147,1.107)--(-9.093,1.083);
\filldraw[fill opacity=0.8,fill=gray!20,draw=none](-9.385,.923)--(-9.389,.92)--(-9.407,.963)--(-9.4,.968)--cycle;
\draw(-9.385,.923)--(-9.389,.92)--(-9.407,.963)--(-9.4,.968);
\filldraw[fill opacity=0.8,fill=gray!20,draw=none](-9.385,.923)--(-9.389,.92)--(-9.407,.963)--(-9.4,.968)--cycle;
\draw(-9.385,.923)--(-9.389,.92)--(-9.407,.963)--(-9.4,.968);
\filldraw[fill opacity=0.8,fill=gray!20](-9.034,1.514)--(-8.987,1.541)--(-8.993,1.547)--(-9.046,1.526)--cycle;
\filldraw[fill opacity=0.8,fill=gray!20,draw=none](-8.761,1.434)--(-8.746,1.444)--(-8.781,1.489)--(-8.804,1.474)--cycle;
\draw(-8.761,1.434)--(-8.746,1.444)--(-8.781,1.489)--(-8.804,1.474);
\filldraw[fill opacity=0.8,fill=gray!20,draw=none](-9.377,1.106)--(-9.368,1.118)--(-9.371,1.119)--cycle;
\draw(-9.368,1.118)--(-9.371,1.119);
\filldraw[fill opacity=0.8,fill=gray!20,draw=none](-9.382,1.103)--(-9.389,1.099)--(-9.36,1.135)--(-9.339,1.149)--cycle;
\draw(-9.382,1.103)--(-9.389,1.099)--(-9.36,1.135)--(-9.339,1.149);
\filldraw[fill opacity=0.8,fill=gray!20,draw=none](-9.382,1.103)--(-9.389,1.099)--(-9.36,1.135)--(-9.339,1.149)--cycle;
\draw(-9.382,1.103)--(-9.389,1.099)--(-9.36,1.135)--(-9.339,1.149);
\filldraw[fill opacity=0.8,fill=gray!20,draw=none](-9.339,1.149)--(-9.333,1.153)--(-9.328,1.154)--cycle;
\draw(-9.339,1.149)--(-9.333,1.153);
\filldraw[fill opacity=0.8,fill=gray!20,draw=none](-9.339,1.149)--(-9.333,1.153)--(-9.328,1.154)--cycle;
\draw(-9.339,1.149)--(-9.333,1.153);
\filldraw[fill opacity=0.8,fill=gray!20](-9.36,1.135)--(-9.322,1.163)--(-9.307,1.173)--(-9.339,1.149)--cycle;
\filldraw[fill opacity=0.8,fill=gray!20](-9.36,1.135)--(-9.322,1.163)--(-9.307,1.173)--(-9.339,1.149)--cycle;
\filldraw[fill opacity=0.8,fill=gray!20](-9.231,.829)--(-9.273,.83)--(-9.278,.835)--(-9.231,.829)--cycle;
\filldraw[fill opacity=0.8,fill=gray!20](-9.273,.83)--(-9.312,.844)--(-9.322,.854)--(-9.278,.835)--cycle;
\filldraw[fill opacity=0.8,fill=gray!20](-9.231,.829)--(-9.273,.83)--(-9.278,.835)--(-9.231,.829)--cycle;
\filldraw[fill opacity=0.8,fill=gray!20](-9.273,.83)--(-9.312,.844)--(-9.322,.854)--(-9.278,.835)--cycle;
\filldraw[fill opacity=0.8,fill=gray!20,draw=none](-9.397,1.068)--(-9.38,1.101)--(-9.393,1.065)--(-9.4,1.06)--cycle;
\draw(-9.393,1.065)--(-9.4,1.06);
\filldraw[fill opacity=0.8,fill=gray!20,draw=none](-9.397,1.068)--(-9.38,1.101)--(-9.393,1.065)--(-9.4,1.06)--cycle;
\draw(-9.393,1.065)--(-9.4,1.06);
\filldraw[fill opacity=0.8,fill=gray!20](-9.312,.844)--(-9.346,.868)--(-9.36,.883)--(-9.322,.854)--cycle;
\filldraw[fill opacity=0.8,fill=gray!20](-9.312,.844)--(-9.346,.868)--(-9.36,.883)--(-9.322,.854)--cycle;
\filldraw[fill opacity=0.8,fill=gray!20](-9.231,.829)--(-9.257,.826)--(-9.273,.83)--(-9.231,.829)--cycle;
\filldraw[fill opacity=0.8,fill=gray!20](-9.18,1.178)--(-9.205,1.188)--(-9.188,1.184)--(-9.149,1.171)--cycle;
\filldraw[fill opacity=0.8,fill=gray!20](-9.18,1.178)--(-9.205,1.188)--(-9.188,1.184)--(-9.149,1.171)--cycle;
\filldraw[fill opacity=0.8,fill=gray!20](-9.231,.829)--(-9.257,.826)--(-9.273,.83)--(-9.231,.829)--cycle;
\filldraw[fill opacity=0.8,fill=gray!20](-9.231,.829)--(-9.183,.834)--(-9.191,.829)--(-9.231,.829)--cycle;
\filldraw[fill opacity=0.8,fill=gray!20](-9.231,.829)--(-9.191,.829)--(-9.209,.825)--(-9.231,.829)--cycle;
\filldraw[fill opacity=0.8,fill=gray!20](-9.231,.829)--(-9.209,.825)--(-9.233,.824)--(-9.231,.829)--cycle;
\filldraw[fill opacity=0.8,fill=gray!20](-9.231,.829)--(-9.233,.824)--(-9.257,.826)--(-9.231,.829)--cycle;
\filldraw[fill opacity=0.8,fill=gray!20](-9.231,.829)--(-9.183,.834)--(-9.191,.829)--(-9.231,.829)--cycle;
\filldraw[fill opacity=0.8,fill=gray!20](-9.231,.829)--(-9.191,.829)--(-9.209,.825)--(-9.231,.829)--cycle;
\filldraw[fill opacity=0.8,fill=gray!20](-9.231,.829)--(-9.209,.825)--(-9.233,.824)--(-9.231,.829)--cycle;
\filldraw[fill opacity=0.8,fill=gray!20](-9.231,.829)--(-9.233,.824)--(-9.257,.826)--(-9.231,.829)--cycle;
\filldraw[fill opacity=0.8,fill=gray!20,draw=none](-9.328,1.154)--(-9.333,1.153)--(-9.307,1.173)--(-9.288,1.176)--cycle;
\draw(-9.333,1.153)--(-9.307,1.173)--(-9.288,1.176);
\filldraw[fill opacity=0.8,fill=gray!20,draw=none](-9.328,1.154)--(-9.333,1.153)--(-9.307,1.173)--(-9.288,1.176)--cycle;
\draw(-9.333,1.153)--(-9.307,1.173)--(-9.288,1.176);
\filldraw[fill opacity=0.8,fill=gray!20,draw=none](-9.031,1.043)--(-9.051,1.02)--(-9.032,1.033)--(-9.028,1.038)--cycle;
\draw(-9.031,1.043)--(-9.051,1.02);
\draw(-9.032,1.033)--(-9.028,1.038);
\filldraw[fill opacity=0.8,fill=gray!20,draw=none](-9.031,1.043)--(-9.056,1.014)--(-9.061,1)--(-9.028,1.038)--cycle;
\draw(-9.031,1.043)--(-9.056,1.014);
\draw(-9.061,1)--(-9.028,1.038);
\filldraw[fill opacity=0.8,fill=gray!20,draw=none](-9.031,1.043)--(-9.056,1.014)--(-9.061,1)--(-9.028,1.038)--cycle;
\draw(-9.031,1.043)--(-9.056,1.014);
\draw(-9.061,1)--(-9.028,1.038);
\filldraw[fill opacity=0.8,fill=gray!20,draw=none](-9.051,1.02)--(-9.056,1.014)--(-9.061,1)--(-9.032,1.033)--cycle;
\draw(-9.051,1.02)--(-9.056,1.014);
\draw(-9.061,1)--(-9.032,1.033);
\filldraw[fill opacity=0.8,fill=gray!20,draw=none](-8.851,1.283)--(-8.963,1.156)--(-8.961,1.121)--(-8.838,1.261)--cycle;
\draw(-8.961,1.121)--(-8.838,1.261)--(-8.851,1.283)--(-8.963,1.156);
\filldraw[fill opacity=0.8,fill=gray!20,draw=none](-8.851,1.283)--(-8.963,1.156)--(-8.961,1.121)--(-8.838,1.261)--cycle;
\draw(-8.961,1.121)--(-8.838,1.261)--(-8.851,1.283)--(-8.963,1.156);
\filldraw[fill opacity=0.8,fill=gray!20,draw=none](-8.851,1.283)--(-8.963,1.156)--(-8.961,1.121)--(-8.838,1.261)--cycle;
\draw(-8.961,1.121)--(-8.838,1.261)--(-8.851,1.283)--(-8.963,1.156);
\filldraw[fill opacity=0.8,fill=gray!20,draw=none](-9.14,1.307)--(-9.13,1.297)--(-9.132,1.318)--(-9.135,1.32)--cycle;
\draw(-9.13,1.297)--(-9.132,1.318)--(-9.135,1.32);
\filldraw[fill opacity=0.8,fill=gray!20,draw=none](-9.314,1.13)--(-9.344,1.143)--(-9.368,1.118)--(-9.356,1.112)--cycle;
\draw(-9.368,1.118)--(-9.356,1.112);
\filldraw[fill opacity=0.8,fill=gray!20](-9.346,.868)--(-9.372,.902)--(-9.389,.92)--(-9.36,.883)--cycle;
\filldraw[fill opacity=0.8,fill=gray!20](-9.346,.868)--(-9.372,.902)--(-9.389,.92)--(-9.36,.883)--cycle;
\filldraw[fill opacity=0.8,fill=gray!20,draw=none](-9.187,.831)--(-9.183,.834)--(-9.146,.849)--(-9.148,.846)--(-9.154,.841)--(-9.185,.831)--cycle;
\draw(-9.187,.831)--(-9.183,.834)--(-9.146,.849);
\draw(-9.148,.846)--(-9.154,.841)--(-9.185,.831);
\filldraw[fill opacity=0.8,fill=gray!20,draw=none](-9.187,.831)--(-9.185,.831)--(-9.191,.829)--cycle;
\draw(-9.185,.831)--(-9.191,.829)--(-9.187,.831);
\filldraw[fill opacity=0.8,fill=gray!20,draw=none](-9.187,.831)--(-9.183,.834)--(-9.146,.849)--(-9.148,.846)--(-9.154,.841)--(-9.185,.831)--cycle;
\draw(-9.187,.831)--(-9.183,.834)--(-9.146,.849);
\draw(-9.148,.846)--(-9.154,.841)--(-9.185,.831);
\filldraw[fill opacity=0.8,fill=gray!20](-8.987,1.541)--(-8.936,1.554)--(-8.936,1.554)--(-8.993,1.547)--cycle;
\filldraw[fill opacity=0.8,fill=gray!20,draw=none](-8.804,1.474)--(-8.781,1.489)--(-8.826,1.523)--(-8.844,1.512)--(-8.812,1.478)--cycle;
\draw(-8.804,1.474)--(-8.781,1.489)--(-8.826,1.523)--(-8.844,1.512)--(-8.812,1.478);
\filldraw[fill opacity=0.8,fill=gray!20,draw=none](-9.312,.954)--(-9.315,.973)--(-9.408,1.013)--(-9.405,.995)--cycle;
\draw(-9.315,.973)--(-9.408,1.013);
\filldraw[fill opacity=0.8,fill=gray!20,draw=none](-9.231,1.15)--(-9.263,1.164)--(-9.278,1.152)--(-9.272,1.137)--cycle;
\draw(-9.231,1.15)--(-9.263,1.164);
\filldraw[fill opacity=0.8,fill=gray!20,draw=none](-9.356,1.112)--(-9.368,1.118)--(-9.405,1.072)--(-9.355,1.05)--cycle;
\draw(-9.356,1.112)--(-9.368,1.118);
\draw(-9.405,1.072)--(-9.355,1.05);
\filldraw[fill opacity=0.8,fill=gray!20,draw=none](-9.058,1.004)--(-9.054,1.011)--(-9.048,1.005)--(-9.049,.996)--cycle;
\draw(-9.054,1.011)--(-9.048,1.005)--(-9.049,.996);
\filldraw[fill opacity=0.8,fill=gray!20,draw=none](-9.058,1.004)--(-9.054,1.011)--(-9.048,1.005)--(-9.049,.996)--cycle;
\draw(-9.054,1.011)--(-9.048,1.005)--(-9.049,.996);
\filldraw[fill opacity=0.8,fill=gray!20,draw=none](-9.162,1.283)--(-9.235,1.2)--(-9.193,1.226)--(-9.13,1.297)--cycle;
\draw(-9.162,1.283)--(-9.235,1.2);
\draw(-9.193,1.226)--(-9.13,1.297);
\filldraw[fill opacity=0.8,fill=gray!20,draw=none](-9.263,1.164)--(-9.278,1.152)--(-9.269,1.162)--cycle;
\draw(-9.278,1.152)--(-9.269,1.162);
\filldraw[fill opacity=0.8,fill=gray!20,draw=none](-9.263,1.164)--(-9.278,1.152)--(-9.269,1.162)--cycle;
\draw(-9.278,1.152)--(-9.269,1.162);
\filldraw[fill opacity=0.8,fill=gray!20,draw=none](-9.056,1.014)--(-9.07,.998)--(-9.084,.975)--(-9.061,1)--cycle;
\draw(-9.056,1.014)--(-9.07,.998);
\draw(-9.084,.975)--(-9.061,1);
\filldraw[fill opacity=0.8,fill=gray!20,draw=none](-9.056,1.014)--(-9.07,.998)--(-9.084,.975)--(-9.061,1)--cycle;
\draw(-9.056,1.014)--(-9.07,.998);
\draw(-9.084,.975)--(-9.061,1);
\filldraw[fill opacity=0.8,fill=gray!20,draw=none](-9.056,1.014)--(-9.07,.998)--(-9.084,.975)--(-9.061,1)--cycle;
\draw(-9.056,1.014)--(-9.07,.998);
\draw(-9.084,.975)--(-9.061,1);
\filldraw[fill opacity=0.8,fill=gray!20,draw=none](-9.092,.903)--(-9.093,.902)--(-9.131,.918)--(-9.098,.959)--(-9.069,.947)--cycle;
\draw(-9.093,.902)--(-9.131,.918);
\draw(-9.098,.959)--(-9.069,.947);
\filldraw[fill opacity=0.8,fill=gray!20,draw=none](-8.942,1.127)--(-8.943,1.13)--(-8.95,1.13)--cycle;
\draw(-8.942,1.127)--(-8.943,1.13)--(-8.95,1.13);
\filldraw[fill opacity=0.8,fill=gray!20](-9.372,.902)--(-9.389,.944)--(-9.407,.963)--(-9.389,.92)--cycle;
\filldraw[fill opacity=0.8,fill=gray!20](-9.372,.902)--(-9.389,.944)--(-9.407,.963)--(-9.389,.92)--cycle;
\filldraw[fill opacity=0.8,fill=gray!20,draw=none](-9.139,.851)--(-9.128,.86)--(-9.137,.854)--(-9.154,.841)--cycle;
\draw(-9.137,.854)--(-9.154,.841)--(-9.139,.851)--(-9.128,.86);
\filldraw[fill opacity=0.8,fill=gray!20,draw=none](-9.139,.851)--(-9.128,.86)--(-9.137,.854)--(-9.154,.841)--cycle;
\draw(-9.137,.854)--(-9.154,.841)--(-9.139,.851)--(-9.128,.86);
\filldraw[fill opacity=0.8,fill=gray!20,draw=none](-9.185,.855)--(-9.221,.86)--(-9.276,.883)--(-9.231,.879)--(-9.174,.854)--cycle;
\draw(-9.221,.86)--(-9.276,.883)--(-9.231,.879)--(-9.174,.854);
\filldraw[fill opacity=0.8,fill=gray!20](-8.967,1.536)--(-8.936,1.554)--(-8.936,1.554)--(-8.987,1.541)--cycle;
\filldraw[fill opacity=0.8,fill=gray!20](-8.939,1.534)--(-8.936,1.554)--(-8.936,1.554)--(-8.967,1.536)--cycle;
\filldraw[fill opacity=0.8,fill=gray!20](-8.826,1.523)--(-8.879,1.546)--(-8.889,1.539)--(-8.844,1.512)--cycle;
\filldraw[fill opacity=0.8,fill=gray!20](-8.879,1.546)--(-8.936,1.554)--(-8.936,1.554)--(-8.889,1.539)--cycle;
\filldraw[fill opacity=0.8,fill=gray!20](-8.889,1.539)--(-8.936,1.554)--(-8.936,1.554)--(-8.911,1.535)--cycle;
\filldraw[fill opacity=0.8,fill=gray!20](-8.911,1.535)--(-8.936,1.554)--(-8.936,1.554)--(-8.939,1.534)--cycle;
\filldraw[fill opacity=0.8,fill=gray!20](-9.115,1.146)--(-9.149,1.171)--(-9.139,1.161)--(-9.101,1.132)--cycle;
\filldraw[fill opacity=0.8,fill=gray!20](-9.115,1.146)--(-9.149,1.171)--(-9.139,1.161)--(-9.101,1.132)--cycle;
\filldraw[fill opacity=0.8,fill=gray!20,draw=none](-9.128,.86)--(-9.101,.879)--(-9.123,.865)--(-9.137,.854)--cycle;
\draw(-9.128,.86)--(-9.101,.879)--(-9.123,.865)--(-9.137,.854);
\filldraw[fill opacity=0.8,fill=gray!20,draw=none](-9.128,.86)--(-9.101,.879)--(-9.123,.865)--(-9.137,.854)--cycle;
\draw(-9.128,.86)--(-9.101,.879)--(-9.123,.865)--(-9.137,.854);
\filldraw[fill opacity=0.8,fill=gray!20,draw=none](-9.408,.972)--(-9.413,1.01)--(-9.408,1.013)--cycle;
\draw(-9.408,.972)--(-9.413,1.01)--(-9.408,1.013);
\filldraw[fill opacity=0.8,fill=gray!20,draw=none](-9.408,.972)--(-9.413,1.01)--(-9.408,1.013)--cycle;
\draw(-9.408,.972)--(-9.413,1.01)--(-9.408,1.013);
\filldraw[fill opacity=0.8,fill=gray!20,draw=none](-9.287,1.018)--(-9.287,1.021)--(-9.339,1.044)--(-9.348,.994)--(-9.296,.971)--cycle;
\draw(-9.287,1.021)--(-9.339,1.044)--(-9.348,.994)--(-9.296,.971);
\filldraw[fill opacity=0.8,fill=gray!20,draw=none](-9.306,1.02)--(-9.306,1.029)--(-9.397,1.068)--(-9.4,1.06)--(-9.408,1.013)--(-9.315,.973)--cycle;
\draw(-9.306,1.029)--(-9.397,1.068);
\draw(-9.408,1.013)--(-9.315,.973);
\filldraw[fill opacity=0.8,fill=gray!20,draw=none](-9.057,.982)--(-9.084,.975)--(-9.063,1.005)--(-9.053,1.001)--cycle;
\draw(-9.063,1.005)--(-9.053,1.001);
\filldraw[fill opacity=0.8,fill=gray!20,draw=none](-9.07,.998)--(-9.132,.929)--(-9.134,.919)--(-9.084,.975)--cycle;
\draw(-9.07,.998)--(-9.132,.929)--(-9.134,.919)--(-9.084,.975);
\filldraw[fill opacity=0.8,fill=gray!20,draw=none](-9.07,.998)--(-9.132,.929)--(-9.134,.919)--(-9.084,.975)--cycle;
\draw(-9.07,.998)--(-9.132,.929)--(-9.134,.919)--(-9.084,.975);
\filldraw[fill opacity=0.8,fill=gray!20,draw=none](-9.07,.998)--(-9.132,.929)--(-9.134,.919)--(-9.084,.975)--cycle;
\draw(-9.07,.998)--(-9.132,.929)--(-9.134,.919)--(-9.084,.975);
\filldraw[fill opacity=0.8,fill=gray!20,draw=none](-9.187,.831)--(-9.185,.831)--(-9.191,.829)--cycle;
\draw(-9.185,.831)--(-9.191,.829)--(-9.187,.831);
\filldraw[fill opacity=0.8,fill=gray!20,draw=none](-9.38,1.101)--(-9.378,1.105)--(-9.377,1.106)--cycle;
\draw(-9.378,1.105)--(-9.377,1.106);
\filldraw[fill opacity=0.8,fill=gray!20,draw=none](-9.38,1.101)--(-9.378,1.105)--(-9.377,1.106)--cycle;
\draw(-9.378,1.105)--(-9.377,1.106);
\filldraw[fill opacity=0.8,fill=gray!20,draw=none](-9.377,1.106)--(-9.382,1.103)--(-9.368,1.118)--cycle;
\draw(-9.377,1.106)--(-9.382,1.103);
\filldraw[fill opacity=0.8,fill=gray!20,draw=none](-9.377,1.106)--(-9.382,1.103)--(-9.368,1.118)--cycle;
\draw(-9.377,1.106)--(-9.382,1.103);
\filldraw[fill opacity=0.8,fill=gray!20,draw=none](-8.887,1.132)--(-8.878,1.143)--(-8.945,1.151)--(-8.943,1.13)--cycle;
\draw(-8.945,1.151)--(-8.943,1.13)--(-8.887,1.132)--(-8.878,1.143);
\filldraw[fill opacity=0.8,fill=gray!20,draw=none](-9.397,1.068)--(-9.398,1.069)--(-9.4,1.06)--cycle;
\draw(-9.397,1.068)--(-9.398,1.069);
\filldraw[fill opacity=0.8,fill=gray!20,draw=none](-9.38,1.101)--(-9.402,1.059)--(-9.407,1.056)--(-9.389,1.099)--(-9.378,1.105)--cycle;
\draw(-9.402,1.059)--(-9.407,1.056)--(-9.389,1.099)--(-9.378,1.105);
\filldraw[fill opacity=0.8,fill=gray!20,draw=none](-9.38,1.101)--(-9.402,1.059)--(-9.407,1.056)--(-9.389,1.099)--(-9.378,1.105)--cycle;
\draw(-9.402,1.059)--(-9.407,1.056)--(-9.389,1.099)--(-9.378,1.105);
\filldraw[fill opacity=0.8,fill=gray!20,draw=none](-9.397,1.068)--(-9.4,1.06)--(-9.402,1.059)--cycle;
\draw(-9.4,1.06)--(-9.402,1.059);
\filldraw[fill opacity=0.8,fill=gray!20,draw=none](-9.397,1.068)--(-9.4,1.06)--(-9.402,1.059)--cycle;
\draw(-9.4,1.06)--(-9.402,1.059);
\filldraw[fill opacity=0.8,fill=gray!20](-9.389,.944)--(-9.394,.989)--(-9.413,1.01)--(-9.407,.963)--cycle;
\filldraw[fill opacity=0.8,fill=gray!20](-9.389,.944)--(-9.394,.989)--(-9.413,1.01)--(-9.407,.963)--cycle;
\filldraw[fill opacity=0.8,fill=gray!20,draw=none](-9.101,.879)--(-9.072,.916)--(-9.079,.911)--(-9.123,.865)--cycle;
\draw(-9.123,.865)--(-9.101,.879)--(-9.072,.916)--(-9.079,.911);
\filldraw[fill opacity=0.8,fill=gray!20,draw=none](-9.101,.879)--(-9.072,.916)--(-9.079,.911)--(-9.123,.865)--cycle;
\draw(-9.123,.865)--(-9.101,.879)--(-9.072,.916)--(-9.079,.911);
\filldraw[fill opacity=0.8,fill=gray!20,draw=none](-9.128,1.105)--(-9.106,1.092)--(-9.093,1.083)--(-9.147,1.107)--(-9.186,1.131)--(-9.132,1.108)--cycle;
\draw(-9.093,1.083)--(-9.147,1.107)--(-9.186,1.131)--(-9.132,1.108);
\filldraw[fill opacity=0.8,fill=gray!20](-9.257,.826)--(-9.281,.836)--(-9.312,.844)--(-9.273,.83)--cycle;
\filldraw[fill opacity=0.8,fill=gray!20](-9.257,.826)--(-9.281,.836)--(-9.312,.844)--(-9.273,.83)--cycle;
\filldraw[fill opacity=0.8,fill=gray!20,draw=none](-9.068,.95)--(-9.069,.947)--(-9.098,.959)--(-9.069,1.001)--(-9.06,.997)--cycle;
\draw(-9.069,.947)--(-9.098,.959);
\draw(-9.069,1.001)--(-9.06,.997);
\filldraw[fill opacity=0.8,fill=gray!20,draw=none](-9.041,1.198)--(-9.051,1.238)--(-9.058,1.246)--(-9.126,1.263)--(-9.106,1.214)--cycle;
\draw(-9.058,1.246)--(-9.126,1.263)--(-9.106,1.214)--(-9.041,1.198)--(-9.051,1.238);
\filldraw[fill opacity=0.8,fill=gray!20,draw=none](-9.054,1.011)--(-9.05,1.021)--(-9.048,1.005)--cycle;
\draw(-9.05,1.021)--(-9.048,1.005)--(-9.054,1.011);
\filldraw[fill opacity=0.8,fill=gray!20,draw=none](-9.054,1.011)--(-9.05,1.021)--(-9.048,1.005)--cycle;
\draw(-9.05,1.021)--(-9.048,1.005)--(-9.054,1.011);
\filldraw[fill opacity=0.8,fill=gray!20,draw=none](-9.016,1.425)--(-9.13,1.297)--(-9.095,1.287)--(-8.988,1.408)--cycle;
\draw(-9.095,1.287)--(-8.988,1.408)--(-9.016,1.425)--(-9.13,1.297);
\filldraw[fill opacity=0.8,fill=gray!20,draw=none](-9.228,1.19)--(-9.23,1.186)--(-9.229,1.186)--(-9.205,1.188)--cycle;
\draw(-9.229,1.186)--(-9.205,1.188)--(-9.228,1.19)--(-9.23,1.186);
\filldraw[fill opacity=0.8,fill=gray!20,draw=none](-9.228,1.19)--(-9.23,1.186)--(-9.229,1.186)--(-9.205,1.188)--cycle;
\draw(-9.229,1.186)--(-9.205,1.188)--(-9.228,1.19)--(-9.23,1.186);
\filldraw[fill opacity=0.8,fill=gray!20](-9.149,1.171)--(-9.188,1.184)--(-9.183,1.179)--(-9.139,1.161)--cycle;
\filldraw[fill opacity=0.8,fill=gray!20](-9.149,1.171)--(-9.188,1.184)--(-9.183,1.179)--(-9.139,1.161)--cycle;
\filldraw[fill opacity=0.8,fill=gray!20,draw=none](-9.205,1.188)--(-9.229,1.186)--(-9.228,1.186)--(-9.188,1.184)--cycle;
\draw(-9.228,1.186)--(-9.188,1.184)--(-9.205,1.188)--(-9.229,1.186);
\filldraw[fill opacity=0.8,fill=gray!20,draw=none](-9.205,1.188)--(-9.229,1.186)--(-9.228,1.186)--(-9.188,1.184)--cycle;
\draw(-9.228,1.186)--(-9.188,1.184)--(-9.205,1.188)--(-9.229,1.186);
\filldraw[fill opacity=0.8,fill=gray!20,draw=none](-9.016,1.425)--(-9.13,1.297)--(-9.095,1.287)--(-8.988,1.408)--cycle;
\draw(-9.095,1.287)--(-8.988,1.408)--(-9.016,1.425)--(-9.13,1.297);
\filldraw[fill opacity=0.8,fill=gray!20,draw=none](-9.016,1.425)--(-9.13,1.297)--(-9.095,1.287)--(-8.988,1.408)--cycle;
\draw(-9.095,1.287)--(-8.988,1.408)--(-9.016,1.425)--(-9.13,1.297);
\filldraw[fill opacity=0.8,fill=gray!20](-9.322,1.163)--(-9.278,1.18)--(-9.27,1.185)--(-9.307,1.173)--cycle;
\filldraw[fill opacity=0.8,fill=gray!20](-9.322,1.163)--(-9.278,1.18)--(-9.27,1.185)--(-9.307,1.173)--cycle;
\filldraw[fill opacity=0.8,fill=gray!20,draw=none](-8.951,1.133)--(-8.952,1.132)--(-8.952,1.13)--cycle;
\draw(-8.951,1.133)--(-8.952,1.132);
\filldraw[fill opacity=0.8,fill=gray!20,draw=none](-8.951,1.133)--(-8.952,1.132)--(-8.952,1.13)--cycle;
\draw(-8.951,1.133)--(-8.952,1.132);
\filldraw[fill opacity=0.8,fill=gray!20,draw=none](-8.951,1.133)--(-8.952,1.132)--(-8.952,1.13)--cycle;
\draw(-8.951,1.133)--(-8.952,1.132);
\filldraw[fill opacity=0.8,fill=gray!20,draw=none](-8.95,1.13)--(-8.943,1.13)--(-8.945,1.151)--(-8.953,1.154)--(-8.963,1.155)--cycle;
\draw(-8.95,1.13)--(-8.943,1.13)--(-8.945,1.151);
\draw(-8.953,1.154)--(-8.963,1.155);
\filldraw[fill opacity=0.8,fill=gray!20,draw=none](-9.191,.829)--(-9.185,.831)--(-9.186,.835)--(-9.19,.835)--(-9.209,.825)--cycle;
\draw(-9.186,.835)--(-9.19,.835)--(-9.209,.825)--(-9.191,.829)--(-9.185,.831);
\filldraw[fill opacity=0.8,fill=gray!20,draw=none](-9.191,.829)--(-9.185,.831)--(-9.186,.835)--(-9.19,.835)--(-9.209,.825)--cycle;
\draw(-9.186,.835)--(-9.19,.835)--(-9.209,.825)--(-9.191,.829)--(-9.185,.831);
\filldraw[fill opacity=0.8,fill=gray!20,draw=none](-9.242,1.189)--(-9.23,1.186)--(-9.228,1.19)--cycle;
\draw(-9.23,1.186)--(-9.228,1.19)--(-9.242,1.189);
\filldraw[fill opacity=0.8,fill=gray!20,draw=none](-9.244,1.19)--(-9.278,1.152)--(-9.272,1.137)--(-9.228,1.186)--cycle;
\draw(-9.244,1.19)--(-9.278,1.152);
\draw(-9.272,1.137)--(-9.228,1.186);
\filldraw[fill opacity=0.8,fill=gray!20,draw=none](-9.244,1.19)--(-9.278,1.152)--(-9.272,1.137)--(-9.228,1.186)--cycle;
\draw(-9.244,1.19)--(-9.278,1.152);
\draw(-9.272,1.137)--(-9.228,1.186);
\filldraw[fill opacity=0.8,fill=gray!20,draw=none](-9.244,1.19)--(-9.245,1.188)--(-9.242,1.189)--cycle;
\draw(-9.244,1.19)--(-9.245,1.188);
\filldraw[fill opacity=0.8,fill=gray!20,draw=none](-9.242,1.189)--(-9.23,1.186)--(-9.228,1.19)--cycle;
\draw(-9.23,1.186)--(-9.228,1.19)--(-9.242,1.189);
\filldraw[fill opacity=0.8,fill=gray!20,draw=none](-9.235,1.2)--(-9.244,1.19)--(-9.228,1.186)--(-9.193,1.226)--cycle;
\draw(-9.235,1.2)--(-9.244,1.19);
\draw(-9.228,1.186)--(-9.193,1.226);
\filldraw[fill opacity=0.8,fill=gray!20,draw=none](-9.314,1.089)--(-9.276,1.121)--(-9.295,1.129)--cycle;
\draw(-9.314,1.089)--(-9.276,1.121)--(-9.295,1.129);
\filldraw[fill opacity=0.8,fill=gray!20,draw=none](-9.238,1.191)--(-9.245,1.188)--(-9.327,1.096)--(-9.311,1.092)--(-9.256,1.154)--cycle;
\draw(-9.245,1.188)--(-9.327,1.096)--(-9.311,1.092)--(-9.256,1.154);
\filldraw[fill opacity=0.8,fill=gray!20,draw=none](-9.233,1.201)--(-9.235,1.2)--(-9.244,1.19)--(-9.239,1.189)--cycle;
\draw(-9.235,1.2)--(-9.244,1.19);
\filldraw[fill opacity=0.8,fill=gray!20,draw=none](-9.235,1.2)--(-9.244,1.19)--(-9.242,1.189)--(-9.217,1.199)--(-9.193,1.226)--cycle;
\draw(-9.235,1.2)--(-9.244,1.19);
\draw(-9.217,1.199)--(-9.193,1.226);
\filldraw[fill opacity=0.8,fill=gray!20,draw=none](-9.061,1.058)--(-9.057,1.058)--(-9.054,1.051)--cycle;
\draw(-9.057,1.058)--(-9.054,1.051)--(-9.061,1.058);
\filldraw[fill opacity=0.8,fill=gray!20,draw=none](-9.061,1.058)--(-9.057,1.058)--(-9.054,1.051)--cycle;
\draw(-9.057,1.058)--(-9.054,1.051)--(-9.061,1.058);
\filldraw[fill opacity=0.8,fill=gray!20,draw=none](-9.233,1.201)--(-9.239,1.189)--(-9.228,1.186)--(-9.193,1.226)--cycle;
\draw(-9.228,1.186)--(-9.193,1.226);
\filldraw[fill opacity=0.8,fill=gray!20](-9.394,.989)--(-9.389,1.036)--(-9.407,1.056)--(-9.413,1.01)--cycle;
\filldraw[fill opacity=0.8,fill=gray!20](-9.394,.989)--(-9.389,1.036)--(-9.407,1.056)--(-9.413,1.01)--cycle;
\filldraw[fill opacity=0.8,fill=gray!20,draw=none](-9.084,.908)--(-9.072,.916)--(-9.054,.959)--(-9.06,.955)--cycle;
\draw(-9.084,.908)--(-9.072,.916)--(-9.054,.959)--(-9.06,.955);
\filldraw[fill opacity=0.8,fill=gray!20,draw=none](-9.084,.908)--(-9.072,.916)--(-9.054,.959)--(-9.06,.955)--cycle;
\draw(-9.084,.908)--(-9.072,.916)--(-9.054,.959)--(-9.06,.955);
\filldraw[fill opacity=0.8,fill=gray!20,draw=none](-9.245,1.188)--(-9.245,1.189)--(-9.242,1.189)--cycle;
\draw(-9.245,1.189)--(-9.242,1.189);
\filldraw[fill opacity=0.8,fill=gray!20,draw=none](-9.245,1.188)--(-9.245,1.189)--(-9.242,1.189)--cycle;
\draw(-9.245,1.189)--(-9.242,1.189);
\filldraw[fill opacity=0.8,fill=gray!20,draw=none](-9.185,.831)--(-9.154,.841)--(-9.186,.835)--cycle;
\draw(-9.185,.831)--(-9.154,.841)--(-9.186,.835);
\filldraw[fill opacity=0.8,fill=gray!20,draw=none](-9.185,.831)--(-9.154,.841)--(-9.186,.835)--cycle;
\draw(-9.185,.831)--(-9.154,.841)--(-9.186,.835);
\filldraw[fill opacity=0.8,fill=gray!20,draw=none](-8.963,1.156)--(-9.018,1.095)--(-8.984,1.095)--(-8.961,1.121)--cycle;
\draw(-8.963,1.156)--(-9.018,1.095);
\draw(-8.984,1.095)--(-8.961,1.121);
\filldraw[fill opacity=0.8,fill=gray!20,draw=none](-8.963,1.156)--(-9.018,1.095)--(-8.984,1.095)--(-8.961,1.121)--cycle;
\draw(-8.963,1.156)--(-9.018,1.095);
\draw(-8.984,1.095)--(-8.961,1.121);
\filldraw[fill opacity=0.8,fill=gray!20,draw=none](-8.963,1.156)--(-9.018,1.095)--(-8.984,1.095)--(-8.961,1.121)--cycle;
\draw(-8.963,1.156)--(-9.018,1.095);
\draw(-8.984,1.095)--(-8.961,1.121);
\filldraw[fill opacity=0.8,fill=gray!20](-8.997,1.505)--(-8.967,1.536)--(-8.987,1.541)--(-9.034,1.514)--cycle;
\filldraw[fill opacity=0.8,fill=gray!20,draw=none](-9.234,1.192)--(-9.238,1.191)--(-9.256,1.154)--(-9.228,1.186)--cycle;
\draw(-9.256,1.154)--(-9.228,1.186);
\filldraw[fill opacity=0.8,fill=gray!20,draw=none](-9.242,1.189)--(-9.252,1.189)--(-9.231,1.186)--(-9.231,1.186)--(-9.23,1.186)--cycle;
\draw(-9.242,1.189)--(-9.252,1.189)--(-9.231,1.186)--(-9.231,1.186)--(-9.23,1.186);
\filldraw[fill opacity=0.8,fill=gray!20,draw=none](-9.242,1.189)--(-9.252,1.189)--(-9.231,1.186)--(-9.231,1.186)--(-9.23,1.186)--cycle;
\draw(-9.242,1.189)--(-9.252,1.189)--(-9.231,1.186)--(-9.231,1.186)--(-9.23,1.186);
\filldraw[fill opacity=0.8,fill=gray!20,draw=none](-9.239,.871)--(-9.26,.883)--(-9.314,.907)--(-9.276,.883)--(-9.221,.86)--cycle;
\draw(-9.26,.883)--(-9.314,.907)--(-9.276,.883)--(-9.221,.86);
\filldraw[fill opacity=0.8,fill=gray!20](-9.27,1.185)--(-9.231,1.186)--(-9.231,1.186)--(-9.252,1.189)--cycle;
\filldraw[fill opacity=0.8,fill=gray!20](-9.27,1.185)--(-9.231,1.186)--(-9.231,1.186)--(-9.252,1.189)--cycle;
\filldraw[fill opacity=0.8,fill=gray!20,draw=none](-9.278,1.152)--(-9.327,1.096)--(-9.311,1.092)--(-9.272,1.137)--cycle;
\draw(-9.278,1.152)--(-9.327,1.096)--(-9.311,1.092)--(-9.272,1.137);
\filldraw[fill opacity=0.8,fill=gray!20,draw=none](-9.278,1.152)--(-9.327,1.096)--(-9.311,1.092)--(-9.272,1.137)--cycle;
\draw(-9.278,1.152)--(-9.327,1.096)--(-9.311,1.092)--(-9.272,1.137);
\filldraw[fill opacity=0.8,fill=gray!20,draw=none](-9.272,1.137)--(-9.278,1.152)--(-9.293,1.139)--cycle;
\filldraw[fill opacity=0.8,fill=gray!20](-9.389,1.036)--(-9.372,1.081)--(-9.389,1.099)--(-9.407,1.056)--cycle;
\filldraw[fill opacity=0.8,fill=gray!20](-9.389,1.036)--(-9.372,1.081)--(-9.389,1.099)--(-9.407,1.056)--cycle;
\filldraw[fill opacity=0.8,fill=gray!20](-9.233,.824)--(-9.236,.833)--(-9.281,.836)--(-9.257,.826)--cycle;
\filldraw[fill opacity=0.8,fill=gray!20](-9.233,.824)--(-9.236,.833)--(-9.281,.836)--(-9.257,.826)--cycle;
\filldraw[fill opacity=0.8,fill=gray!20,draw=none](-9.061,.954)--(-9.054,.959)--(-9.053,.967)--(-9.053,1.001)--cycle;
\draw(-9.061,.954)--(-9.054,.959)--(-9.053,.967);
\filldraw[fill opacity=0.8,fill=gray!20,draw=none](-9.061,.954)--(-9.054,.959)--(-9.053,.967)--(-9.053,1.001)--cycle;
\draw(-9.061,.954)--(-9.054,.959)--(-9.053,.967);
\filldraw[fill opacity=0.8,fill=gray!20](-8.844,1.512)--(-8.889,1.539)--(-8.911,1.535)--(-8.887,1.503)--cycle;
\filldraw[fill opacity=0.8,fill=gray!20,draw=none](-9.018,1.095)--(-9.051,1.057)--(-9.031,1.043)--(-8.984,1.095)--cycle;
\draw(-9.018,1.095)--(-9.051,1.057);
\draw(-9.031,1.043)--(-8.984,1.095);
\filldraw[fill opacity=0.8,fill=gray!20,draw=none](-9.018,1.095)--(-9.051,1.057)--(-9.031,1.043)--(-8.984,1.095)--cycle;
\draw(-9.018,1.095)--(-9.051,1.057);
\draw(-9.031,1.043)--(-8.984,1.095);
\filldraw[fill opacity=0.8,fill=gray!20,draw=none](-9.018,1.095)--(-9.051,1.057)--(-9.031,1.043)--(-8.984,1.095)--cycle;
\draw(-9.018,1.095)--(-9.051,1.057);
\draw(-9.031,1.043)--(-8.984,1.095);
\filldraw[fill opacity=0.8,fill=gray!20,draw=none](-9.053,.967)--(-9.048,1.005)--(-9.053,1.001)--cycle;
\draw(-9.053,.967)--(-9.048,1.005)--(-9.053,1.001);
\filldraw[fill opacity=0.8,fill=gray!20,draw=none](-9.053,.967)--(-9.048,1.005)--(-9.053,1.001)--cycle;
\draw(-9.053,.967)--(-9.048,1.005)--(-9.053,1.001);
\filldraw[fill opacity=0.8,fill=gray!20](-9.209,.825)--(-9.19,.835)--(-9.236,.833)--(-9.233,.824)--cycle;
\filldraw[fill opacity=0.8,fill=gray!20](-9.209,.825)--(-9.19,.835)--(-9.236,.833)--(-9.233,.824)--cycle;
\filldraw[fill opacity=0.8,fill=gray!20,draw=none](-9.295,1.132)--(-9.297,1.13)--(-9.295,1.129)--cycle;
\draw(-9.297,1.13)--(-9.295,1.129);
\filldraw[fill opacity=0.8,fill=gray!20,draw=none](-9.058,1.246)--(-9.063,1.256)--(-9.125,1.316)--(-9.132,1.318)--(-9.126,1.263)--cycle;
\draw(-9.125,1.316)--(-9.132,1.318)--(-9.126,1.263)--(-9.058,1.246);
\filldraw[fill opacity=0.8,fill=gray!20,draw=none](-9.33,1.096)--(-9.314,1.089)--(-9.295,1.129)--(-9.297,1.13)--cycle;
\draw(-9.33,1.096)--(-9.314,1.089);
\draw(-9.295,1.129)--(-9.297,1.13);
\filldraw[fill opacity=0.8,fill=gray!20](-9.188,1.184)--(-9.231,1.186)--(-9.231,1.186)--(-9.183,1.179)--cycle;
\filldraw[fill opacity=0.8,fill=gray!20](-9.188,1.184)--(-9.231,1.186)--(-9.231,1.186)--(-9.183,1.179)--cycle;
\filldraw[fill opacity=0.8,fill=gray!20,draw=none](-9.3,1.107)--(-9.334,1.121)--(-9.356,1.112)--(-9.317,1.095)--cycle;
\draw(-9.356,1.112)--(-9.317,1.095);
\filldraw[fill opacity=0.8,fill=gray!20,draw=none](-9.33,1.06)--(-9.317,1.095)--(-9.356,1.112)--(-9.355,1.071)--cycle;
\draw(-9.317,1.095)--(-9.356,1.112);
\filldraw[fill opacity=0.8,fill=gray!20,draw=none](-9.186,1.117)--(-9.132,1.108)--(-9.186,1.131)--(-9.231,1.136)--(-9.191,1.118)--cycle;
\draw(-9.132,1.108)--(-9.186,1.131)--(-9.231,1.136)--(-9.191,1.118);
\filldraw[fill opacity=0.8,fill=gray!20](-9.281,.836)--(-9.302,.857)--(-9.346,.868)--(-9.312,.844)--cycle;
\filldraw[fill opacity=0.8,fill=gray!20](-9.281,.836)--(-9.302,.857)--(-9.346,.868)--(-9.312,.844)--cycle;
\filldraw[fill opacity=0.8,fill=gray!20,draw=none](-8.87,1.305)--(-8.981,1.18)--(-8.963,1.156)--(-8.851,1.283)--cycle;
\draw(-8.963,1.156)--(-8.851,1.283)--(-8.87,1.305);
\filldraw[fill opacity=0.8,fill=gray!20,draw=none](-8.87,1.305)--(-8.981,1.18)--(-8.963,1.156)--(-8.851,1.283)--cycle;
\draw(-8.963,1.156)--(-8.851,1.283)--(-8.87,1.305);
\filldraw[fill opacity=0.8,fill=gray!20,draw=none](-8.87,1.305)--(-8.981,1.18)--(-8.963,1.156)--(-8.851,1.283)--cycle;
\draw(-8.963,1.156)--(-8.851,1.283)--(-8.87,1.305);
\filldraw[fill opacity=0.8,fill=gray!20,draw=none](-9.234,1.192)--(-9.228,1.186)--(-9.217,1.199)--cycle;
\draw(-9.228,1.186)--(-9.217,1.199);
\filldraw[fill opacity=0.8,fill=gray!20,draw=none](-9.059,1.011)--(-9.06,1.004)--(-9.063,1.005)--cycle;
\draw(-9.06,1.004)--(-9.063,1.005);
\filldraw[fill opacity=0.8,fill=gray!20,draw=none](-9.117,.872)--(-9.079,.911)--(-9.092,.903)--cycle;
\draw(-9.079,.911)--(-9.092,.903);
\filldraw[fill opacity=0.8,fill=gray!20,draw=none](-9.117,.872)--(-9.079,.911)--(-9.092,.903)--cycle;
\draw(-9.079,.911)--(-9.092,.903);
\filldraw[fill opacity=0.8,fill=gray!20](-9.372,1.081)--(-9.346,1.121)--(-9.36,1.135)--(-9.389,1.099)--cycle;
\filldraw[fill opacity=0.8,fill=gray!20](-9.372,1.081)--(-9.346,1.121)--(-9.36,1.135)--(-9.389,1.099)--cycle;
\filldraw[fill opacity=0.8,fill=gray!20,draw=none](-9.068,1.038)--(-9.145,.951)--(-9.132,.929)--(-9.051,1.02)--cycle;
\draw(-9.068,1.038)--(-9.145,.951)--(-9.132,.929)--(-9.051,1.02);
\filldraw[fill opacity=0.8,fill=gray!20,draw=none](-9.068,1.038)--(-9.145,.951)--(-9.132,.929)--(-9.051,1.02)--cycle;
\draw(-9.068,1.038)--(-9.145,.951)--(-9.132,.929)--(-9.051,1.02);
\filldraw[fill opacity=0.8,fill=gray!20,draw=none](-9.068,1.038)--(-9.145,.951)--(-9.132,.929)--(-9.051,1.02)--cycle;
\draw(-9.068,1.038)--(-9.145,.951)--(-9.132,.929)--(-9.051,1.02);
\filldraw[fill opacity=0.8,fill=gray!20,draw=none](-9.059,1.011)--(-9.056,1.016)--(-9.053,1.001)--(-9.06,1.004)--cycle;
\draw(-9.053,1.001)--(-9.06,1.004);
\filldraw[fill opacity=0.8,fill=gray!20,draw=none](-9.051,1.02)--(-9.056,1.016)--(-9.053,1.001)--(-9.048,1.005)--(-9.05,1.018)--cycle;
\draw(-9.053,1.001)--(-9.048,1.005)--(-9.05,1.018);
\filldraw[fill opacity=0.8,fill=gray!20,draw=none](-9.051,1.02)--(-9.056,1.016)--(-9.053,1.001)--(-9.048,1.005)--(-9.05,1.018)--cycle;
\draw(-9.053,1.001)--(-9.048,1.005)--(-9.05,1.018);
\filldraw[fill opacity=0.8,fill=gray!20,draw=none](-9.056,1.016)--(-9.05,1.021)--(-9.054,1.051)--(-9.061,1.046)--cycle;
\draw(-9.05,1.021)--(-9.054,1.051)--(-9.061,1.046);
\filldraw[fill opacity=0.8,fill=gray!20,draw=none](-9.056,1.016)--(-9.05,1.021)--(-9.054,1.051)--(-9.061,1.046)--cycle;
\draw(-9.05,1.021)--(-9.054,1.051)--(-9.061,1.046);
\filldraw[fill opacity=0.8,fill=gray!20,draw=none](-9.061,1.046)--(-9.054,1.051)--(-9.072,1.094)--(-9.076,1.092)--cycle;
\draw(-9.061,1.046)--(-9.054,1.051)--(-9.072,1.094)--(-9.076,1.092);
\filldraw[fill opacity=0.8,fill=gray!20,draw=none](-9.061,1.046)--(-9.054,1.051)--(-9.072,1.094)--(-9.076,1.092)--cycle;
\draw(-9.061,1.046)--(-9.054,1.051)--(-9.072,1.094)--(-9.076,1.092);
\filldraw[fill opacity=0.8,fill=gray!20,draw=none](-9.051,1.057)--(-9.055,1.052)--(-9.057,1.034)--(-9.044,1.028)--(-9.031,1.043)--cycle;
\draw(-9.051,1.057)--(-9.055,1.052);
\draw(-9.044,1.028)--(-9.031,1.043);
\filldraw[fill opacity=0.8,fill=gray!20,draw=none](-9.051,1.057)--(-9.055,1.052)--(-9.057,1.034)--(-9.044,1.028)--(-9.031,1.043)--cycle;
\draw(-9.051,1.057)--(-9.055,1.052);
\draw(-9.044,1.028)--(-9.031,1.043);
\filldraw[fill opacity=0.8,fill=gray!20,draw=none](-9.051,1.057)--(-9.055,1.052)--(-9.057,1.034)--(-9.044,1.028)--(-9.031,1.043)--cycle;
\draw(-9.051,1.057)--(-9.055,1.052);
\draw(-9.044,1.028)--(-9.031,1.043);
\filldraw[fill opacity=0.8,fill=gray!20](-9.022,1.463)--(-8.997,1.505)--(-9.034,1.514)--(-9.075,1.476)--cycle;
\filldraw[fill opacity=0.8,fill=gray!20,draw=none](-9.191,1.118)--(-9.231,1.136)--(-9.276,1.121)--(-9.254,1.112)--cycle;
\draw(-9.191,1.118)--(-9.231,1.136)--(-9.276,1.121)--(-9.254,1.112);
\filldraw[fill opacity=0.8,fill=gray!20,draw=none](-9.257,1.113)--(-9.276,1.121)--(-9.314,1.089)--cycle;
\draw(-9.257,1.113)--(-9.276,1.121)--(-9.314,1.089);
\filldraw[fill opacity=0.8,fill=gray!20,draw=none](-9.272,1.137)--(-9.29,1.116)--(-9.257,1.113)--(-9.245,1.118)--(-9.233,1.132)--cycle;
\draw(-9.272,1.137)--(-9.29,1.116);
\draw(-9.245,1.118)--(-9.233,1.132);
\filldraw[fill opacity=0.8,fill=gray!20,draw=none](-9.272,1.137)--(-9.29,1.116)--(-9.25,1.112)--(-9.233,1.132)--cycle;
\draw(-9.272,1.137)--(-9.29,1.116);
\draw(-9.25,1.112)--(-9.233,1.132);
\filldraw[fill opacity=0.8,fill=gray!20,draw=none](-9.272,1.137)--(-9.29,1.116)--(-9.25,1.112)--(-9.233,1.132)--cycle;
\draw(-9.272,1.137)--(-9.29,1.116);
\draw(-9.25,1.112)--(-9.233,1.132);
\filldraw[fill opacity=0.8,fill=gray!20,draw=none](-9.29,1.116)--(-9.311,1.092)--(-9.282,1.076)--(-9.25,1.112)--cycle;
\draw(-9.29,1.116)--(-9.311,1.092)--(-9.282,1.076)--(-9.25,1.112);
\filldraw[fill opacity=0.8,fill=gray!20,draw=none](-9.29,1.116)--(-9.311,1.092)--(-9.282,1.076)--(-9.25,1.112)--cycle;
\draw(-9.29,1.116)--(-9.311,1.092)--(-9.282,1.076)--(-9.25,1.112);
\filldraw[fill opacity=0.8,fill=gray!20,draw=none](-9.29,1.116)--(-9.311,1.092)--(-9.282,1.076)--(-9.25,1.112)--cycle;
\draw(-9.29,1.116)--(-9.311,1.092)--(-9.282,1.076)--(-9.25,1.112);
\filldraw[fill opacity=0.8,fill=gray!20,draw=none](-9.3,1.107)--(-9.285,1.117)--(-9.314,1.13)--(-9.334,1.121)--cycle;
\filldraw[fill opacity=0.8,fill=gray!20,draw=none](-9.13,1.297)--(-9.136,1.289)--(-9.101,1.28)--(-9.095,1.287)--cycle;
\draw(-9.13,1.297)--(-9.136,1.289);
\draw(-9.101,1.28)--(-9.095,1.287);
\filldraw[fill opacity=0.8,fill=gray!20,draw=none](-9.13,1.297)--(-9.228,1.186)--(-9.19,1.181)--(-9.095,1.287)--cycle;
\draw(-9.13,1.297)--(-9.228,1.186);
\draw(-9.19,1.181)--(-9.095,1.287);
\filldraw[fill opacity=0.8,fill=gray!20,draw=none](-9.13,1.297)--(-9.228,1.186)--(-9.19,1.181)--(-9.095,1.287)--cycle;
\draw(-9.13,1.297)--(-9.228,1.186);
\draw(-9.19,1.181)--(-9.095,1.287);
\filldraw[fill opacity=0.8,fill=gray!20](-9.278,1.18)--(-9.231,1.186)--(-9.231,1.186)--(-9.27,1.185)--cycle;
\filldraw[fill opacity=0.8,fill=gray!20](-9.278,1.18)--(-9.231,1.186)--(-9.231,1.186)--(-9.27,1.185)--cycle;
\filldraw[fill opacity=0.8,fill=gray!20,draw=none](-9.051,1.02)--(-9.05,1.018)--(-9.05,1.021)--cycle;
\draw(-9.05,1.018)--(-9.05,1.021);
\filldraw[fill opacity=0.8,fill=gray!20,draw=none](-9.051,1.02)--(-9.05,1.018)--(-9.05,1.021)--cycle;
\draw(-9.05,1.018)--(-9.05,1.021);
\filldraw[fill opacity=0.8,fill=gray!20,draw=none](-9.154,.841)--(-9.151,.844)--(-9.183,.843)--(-9.19,.835)--cycle;
\draw(-9.183,.843)--(-9.19,.835)--(-9.154,.841)--(-9.151,.844);
\filldraw[fill opacity=0.8,fill=gray!20,draw=none](-9.154,.841)--(-9.151,.844)--(-9.183,.843)--(-9.19,.835)--cycle;
\draw(-9.183,.843)--(-9.19,.835)--(-9.154,.841)--(-9.151,.844);
\filldraw[fill opacity=0.8,fill=gray!20](-9.183,1.179)--(-9.231,1.186)--(-9.231,1.186)--(-9.191,1.174)--cycle;
\filldraw[fill opacity=0.8,fill=gray!20](-9.183,1.179)--(-9.231,1.186)--(-9.231,1.186)--(-9.191,1.174)--cycle;
\filldraw[fill opacity=0.8,fill=gray!20,draw=none](-9.202,1.172)--(-9.191,1.174)--(-9.224,1.184)--cycle;
\draw(-9.202,1.172)--(-9.191,1.174)--(-9.224,1.184);
\filldraw[fill opacity=0.8,fill=gray!20,draw=none](-9.202,1.172)--(-9.191,1.174)--(-9.224,1.184)--cycle;
\draw(-9.202,1.172)--(-9.191,1.174)--(-9.224,1.184);
\filldraw[fill opacity=0.8,fill=gray!20,draw=none](-9.136,1.289)--(-9.228,1.186)--(-9.196,1.174)--(-9.101,1.28)--cycle;
\draw(-9.136,1.289)--(-9.228,1.186);
\draw(-9.196,1.174)--(-9.101,1.28);
\filldraw[fill opacity=0.8,fill=gray!20,draw=none](-9.092,.903)--(-9.084,.908)--(-9.06,.955)--(-9.067,.95)--cycle;
\draw(-9.092,.903)--(-9.084,.908);
\draw(-9.06,.955)--(-9.067,.95);
\filldraw[fill opacity=0.8,fill=gray!20,draw=none](-9.095,.901)--(-9.092,.903)--(-9.067,.95)--(-9.072,.947)--cycle;
\draw(-9.095,.901)--(-9.092,.903);
\draw(-9.067,.95)--(-9.072,.947);
\filldraw[fill opacity=0.8,fill=gray!20,draw=none](-9.092,.903)--(-9.084,.908)--(-9.06,.955)--(-9.067,.95)--cycle;
\draw(-9.092,.903)--(-9.084,.908);
\draw(-9.06,.955)--(-9.067,.95);
\filldraw[fill opacity=0.8,fill=gray!20,draw=none](-9.057,1.034)--(-9.058,1.027)--(-9.051,1.02)--(-9.044,1.028)--cycle;
\draw(-9.051,1.02)--(-9.044,1.028);
\filldraw[fill opacity=0.8,fill=gray!20,draw=none](-9.057,1.034)--(-9.058,1.027)--(-9.051,1.02)--(-9.044,1.028)--cycle;
\draw(-9.051,1.02)--(-9.044,1.028);
\filldraw[fill opacity=0.8,fill=gray!20,draw=none](-9.057,1.034)--(-9.058,1.027)--(-9.051,1.02)--(-9.044,1.028)--cycle;
\draw(-9.051,1.02)--(-9.044,1.028);
\filldraw[fill opacity=0.8,fill=gray!20](-8.943,1.501)--(-8.939,1.534)--(-8.967,1.536)--(-8.997,1.505)--cycle;
\filldraw[fill opacity=0.8,fill=gray!20,draw=none](-8.963,1.155)--(-8.981,1.18)--(-8.995,1.194)--(-9.041,1.198)--(-9.022,1.159)--cycle;
\draw(-8.995,1.194)--(-9.041,1.198)--(-9.022,1.159)--(-8.963,1.155);
\filldraw[fill opacity=0.8,fill=gray!20,draw=none](-9.151,.844)--(-9.137,.854)--(-9.174,.854)--(-9.183,.843)--cycle;
\draw(-9.151,.844)--(-9.137,.854);
\draw(-9.174,.854)--(-9.183,.843);
\filldraw[fill opacity=0.8,fill=gray!20,draw=none](-9.151,.844)--(-9.137,.854)--(-9.174,.854)--(-9.183,.843)--cycle;
\draw(-9.151,.844)--(-9.137,.854);
\draw(-9.174,.854)--(-9.183,.843);
\filldraw[fill opacity=0.8,fill=gray!20,draw=none](-9.084,.908)--(-9.064,.946)--(-9.057,.943)--cycle;
\draw(-9.064,.946)--(-9.057,.943);
\filldraw[fill opacity=0.8,fill=gray!20,draw=none](-9.05,.984)--(-9.044,.986)--(-9.052,.94)--(-9.064,.946)--cycle;
\draw(-9.052,.94)--(-9.064,.946);
\filldraw[fill opacity=0.8,fill=gray!20,draw=none](-9.125,1.316)--(-9.131,1.326)--(-9.132,1.318)--cycle;
\draw(-9.131,1.326)--(-9.132,1.318)--(-9.125,1.316);
\filldraw[fill opacity=0.8,fill=gray!20](-8.887,1.503)--(-8.911,1.535)--(-8.939,1.534)--(-8.943,1.501)--cycle;
\filldraw[fill opacity=0.8,fill=gray!20,draw=none](-9.057,1.035)--(-9.049,1.031)--(-9.048,1.03)--(-9.059,1.011)--cycle;
\filldraw[fill opacity=0.8,fill=gray!20,draw=none](-9.123,.865)--(-9.093,.897)--(-9.085,.893)--cycle;
\draw(-9.093,.897)--(-9.085,.893);
\filldraw[fill opacity=0.8,fill=gray!20,draw=none](-9.08,.891)--(-9.093,.897)--(-9.057,.943)--(-9.052,.94)--cycle;
\draw(-9.08,.891)--(-9.093,.897);
\draw(-9.057,.943)--(-9.052,.94);
\filldraw[fill opacity=0.8,fill=gray!20,draw=none](-8.988,1.408)--(-9.187,1.183)--(-9.036,1.285)--(-8.951,1.381)--cycle;
\draw(-9.036,1.285)--(-8.951,1.381)--(-8.988,1.408)--(-9.187,1.183);
\filldraw[fill opacity=0.8,fill=gray!20,draw=none](-8.988,1.408)--(-9.187,1.183)--(-9.036,1.285)--(-8.951,1.381)--cycle;
\draw(-9.036,1.285)--(-8.951,1.381)--(-8.988,1.408)--(-9.187,1.183);
\filldraw[fill opacity=0.8,fill=gray!20,draw=none](-8.988,1.408)--(-9.187,1.183)--(-9.036,1.285)--(-8.951,1.381)--cycle;
\draw(-9.036,1.285)--(-8.951,1.381)--(-8.988,1.408)--(-9.187,1.183);
\filldraw[fill opacity=0.8,fill=gray!20,draw=none](-9.125,1.316)--(-9.057,1.299)--(-9.053,1.356)--(-9.126,1.374)--(-9.131,1.326)--cycle;
\draw(-9.125,1.316)--(-9.057,1.299)--(-9.053,1.356)--(-9.126,1.374)--(-9.131,1.326);
\filldraw[fill opacity=0.8,fill=gray!20](-9.346,1.121)--(-9.312,1.153)--(-9.322,1.163)--(-9.36,1.135)--cycle;
\filldraw[fill opacity=0.8,fill=gray!20](-9.346,1.121)--(-9.312,1.153)--(-9.322,1.163)--(-9.36,1.135)--cycle;
\filldraw[fill opacity=0.8,fill=gray!20,draw=none](-9.057,.982)--(-9.05,.984)--(-9.061,.954)--cycle;
\filldraw[fill opacity=0.8,fill=gray!20,draw=none](-9.068,.95)--(-9.061,.954)--(-9.06,.964)--(-9.06,.997)--cycle;
\draw(-9.068,.95)--(-9.061,.954);
\filldraw[fill opacity=0.8,fill=gray!20,draw=none](-9.068,.95)--(-9.061,.954)--(-9.06,.964)--(-9.06,.997)--cycle;
\draw(-9.068,.95)--(-9.061,.954);
\filldraw[fill opacity=0.8,fill=gray!20,draw=none](-9.254,1.112)--(-9.257,1.113)--(-9.314,1.089)--(-9.298,1.082)--cycle;
\draw(-9.254,1.112)--(-9.257,1.113);
\draw(-9.314,1.089)--(-9.298,1.082);
\filldraw[fill opacity=0.8,fill=gray!20,draw=none](-9.314,1.13)--(-9.285,1.117)--(-9.268,1.128)--(-9.293,1.139)--cycle;
\draw(-9.268,1.128)--(-9.293,1.139);
\filldraw[fill opacity=0.8,fill=gray!20,draw=none](-9.06,.964)--(-9.053,1.001)--(-9.06,.997)--cycle;
\draw(-9.053,1.001)--(-9.06,.997);
\filldraw[fill opacity=0.8,fill=gray!20,draw=none](-9.06,.964)--(-9.053,1.001)--(-9.06,.997)--cycle;
\draw(-9.053,1.001)--(-9.06,.997);
\filldraw[fill opacity=0.8,fill=gray!20,draw=none](-9.283,.886)--(-9.306,.92)--(-9.352,.939)--(-9.323,.896)--(-9.278,.876)--cycle;
\draw(-9.306,.92)--(-9.352,.939);
\draw(-9.323,.896)--(-9.278,.876);
\filldraw[fill opacity=0.8,fill=gray!20,draw=none](-9.072,.996)--(-9.07,1.001)--(-9.069,1.001)--cycle;
\draw(-9.07,1.001)--(-9.069,1.001);
\filldraw[fill opacity=0.8,fill=gray!20,draw=none](-8.812,1.478)--(-8.844,1.512)--(-8.887,1.503)--(-8.878,1.485)--cycle;
\draw(-8.812,1.478)--(-8.844,1.512)--(-8.887,1.503)--(-8.878,1.485);
\filldraw[fill opacity=0.8,fill=gray!20,draw=none](-9.044,.986)--(-9.057,.982)--(-9.053,1.001)--(-9.042,.996)--cycle;
\draw(-9.053,1.001)--(-9.042,.996);
\filldraw[fill opacity=0.8,fill=gray!20](-9.041,1.412)--(-9.022,1.463)--(-9.075,1.476)--(-9.106,1.428)--cycle;
\filldraw[fill opacity=0.8,fill=gray!20,draw=none](-9.063,1.012)--(-9.069,1.001)--(-9.07,1.001)--(-9.067,1.036)--cycle;
\draw(-9.069,1.001)--(-9.07,1.001);
\filldraw[fill opacity=0.8,fill=gray!20,draw=none](-9.056,1.016)--(-9.048,1.03)--(-9.042,.996)--(-9.053,1.001)--cycle;
\draw(-9.042,.996)--(-9.053,1.001);
\filldraw[fill opacity=0.8,fill=gray!20,draw=none](-9.056,1.016)--(-9.063,1.012)--(-9.06,.997)--(-9.053,1.001)--cycle;
\draw(-9.06,.997)--(-9.053,1.001);
\filldraw[fill opacity=0.8,fill=gray!20,draw=none](-9.056,1.016)--(-9.063,1.012)--(-9.06,.997)--(-9.053,1.001)--cycle;
\draw(-9.06,.997)--(-9.053,1.001);
\filldraw[fill opacity=0.8,fill=gray!20,draw=none](-9.198,1.136)--(-9.231,1.15)--(-9.272,1.137)--(-9.269,1.128)--cycle;
\draw(-9.198,1.136)--(-9.231,1.15);
\filldraw[fill opacity=0.8,fill=gray!20](-9.139,1.161)--(-9.183,1.179)--(-9.191,1.174)--(-9.154,1.151)--cycle;
\filldraw[fill opacity=0.8,fill=gray!20](-9.139,1.161)--(-9.183,1.179)--(-9.191,1.174)--(-9.154,1.151)--cycle;
\filldraw[fill opacity=0.8,fill=gray!20,draw=none](-9.095,1.287)--(-9.076,1.282)--(-9.089,1.307)--(-9.125,1.316)--cycle;
\draw(-9.089,1.307)--(-9.125,1.316);
\filldraw[fill opacity=0.8,fill=gray!20,draw=none](-9.063,1.012)--(-9.06,.997)--(-9.069,1.001)--cycle;
\draw(-9.06,.997)--(-9.069,1.001);
\filldraw[fill opacity=0.8,fill=gray!20](-9.302,.857)--(-9.318,.889)--(-9.372,.902)--(-9.346,.868)--cycle;
\filldraw[fill opacity=0.8,fill=gray!20](-9.302,.857)--(-9.318,.889)--(-9.372,.902)--(-9.346,.868)--cycle;
\filldraw[fill opacity=0.8,fill=gray!20](-9.053,1.356)--(-9.041,1.412)--(-9.106,1.428)--(-9.126,1.374)--cycle;
\filldraw[fill opacity=0.8,fill=gray!20,draw=none](-9.137,.854)--(-9.123,.865)--(-9.173,.856)--(-9.174,.854)--cycle;
\draw(-9.137,.854)--(-9.123,.865)--(-9.173,.856)--(-9.174,.854);
\filldraw[fill opacity=0.8,fill=gray!20,draw=none](-9.137,.854)--(-9.123,.865)--(-9.173,.856)--(-9.174,.854)--cycle;
\draw(-9.137,.854)--(-9.123,.865)--(-9.173,.856)--(-9.174,.854);
\filldraw[fill opacity=0.8,fill=gray!20](-9.312,1.153)--(-9.273,1.175)--(-9.278,1.18)--(-9.322,1.163)--cycle;
\filldraw[fill opacity=0.8,fill=gray!20](-9.312,1.153)--(-9.273,1.175)--(-9.278,1.18)--(-9.322,1.163)--cycle;
\filldraw[fill opacity=0.8,fill=gray!20,draw=none](-9.092,1.082)--(-9.072,1.094)--(-9.101,1.132)--(-9.121,1.119)--cycle;
\draw(-9.092,1.082)--(-9.072,1.094)--(-9.101,1.132)--(-9.121,1.119);
\filldraw[fill opacity=0.8,fill=gray!20,draw=none](-9.092,1.082)--(-9.072,1.094)--(-9.101,1.132)--(-9.121,1.119)--cycle;
\draw(-9.092,1.082)--(-9.072,1.094)--(-9.101,1.132)--(-9.121,1.119);
\filldraw[fill opacity=0.8,fill=gray!20,draw=none](-9.272,1.137)--(-9.293,1.139)--(-9.269,1.128)--cycle;
\draw(-9.293,1.139)--(-9.269,1.128);
\filldraw[fill opacity=0.8,fill=gray!20,draw=none](-9.062,1.013)--(-9.056,1.016)--(-9.057,1.02)--(-9.067,1.042)--(-9.069,1.041)--cycle;
\draw(-9.067,1.042)--(-9.069,1.041);
\filldraw[fill opacity=0.8,fill=gray!20,draw=none](-9.062,1.013)--(-9.056,1.016)--(-9.057,1.02)--(-9.067,1.042)--(-9.069,1.041)--cycle;
\draw(-9.067,1.042)--(-9.069,1.041);
\filldraw[fill opacity=0.8,fill=gray!20,draw=none](-9.23,1.186)--(-9.231,1.186)--(-9.231,1.186)--(-9.229,1.186)--cycle;
\draw(-9.23,1.186)--(-9.231,1.186)--(-9.231,1.186)--(-9.229,1.186);
\filldraw[fill opacity=0.8,fill=gray!20,draw=none](-9.23,1.186)--(-9.231,1.186)--(-9.231,1.186)--(-9.229,1.186)--cycle;
\draw(-9.23,1.186)--(-9.231,1.186)--(-9.231,1.186)--(-9.229,1.186);
\filldraw[fill opacity=0.8,fill=gray!20,draw=none](-9.057,1.02)--(-9.061,1.046)--(-9.067,1.042)--cycle;
\draw(-9.061,1.046)--(-9.067,1.042);
\filldraw[fill opacity=0.8,fill=gray!20,draw=none](-9.057,1.02)--(-9.061,1.046)--(-9.067,1.042)--cycle;
\draw(-9.061,1.046)--(-9.067,1.042);
\filldraw[fill opacity=0.8,fill=gray!20,draw=none](-8.963,1.155)--(-8.953,1.154)--(-8.981,1.18)--cycle;
\draw(-8.963,1.155)--(-8.953,1.154);
\filldraw[fill opacity=0.8,fill=gray!20,draw=none](-9.06,1.036)--(-9.057,1.035)--(-9.058,1.027)--cycle;
\filldraw[fill opacity=0.8,fill=gray!20,draw=none](-9.067,1.038)--(-9.067,1.036)--(-9.068,1.042)--cycle;
\filldraw[fill opacity=0.8,fill=gray!20,draw=none](-9.057,1.034)--(-9.068,1.039)--(-9.068,1.038)--(-9.058,1.027)--cycle;
\draw(-9.068,1.039)--(-9.068,1.038);
\filldraw[fill opacity=0.8,fill=gray!20,draw=none](-9.057,1.034)--(-9.067,1.038)--(-9.067,1.036)--(-9.058,1.027)--cycle;
\filldraw[fill opacity=0.8,fill=gray!20,draw=none](-9.057,1.034)--(-9.068,1.039)--(-9.068,1.038)--(-9.058,1.027)--cycle;
\draw(-9.068,1.039)--(-9.068,1.038);
\filldraw[fill opacity=0.8,fill=gray!20](-9.273,1.175)--(-9.231,1.186)--(-9.231,1.186)--(-9.278,1.18)--cycle;
\filldraw[fill opacity=0.8,fill=gray!20](-9.273,1.175)--(-9.231,1.186)--(-9.231,1.186)--(-9.278,1.18)--cycle;
\filldraw[fill opacity=0.8,fill=gray!20](-9.101,1.132)--(-9.139,1.161)--(-9.154,1.151)--(-9.123,1.118)--cycle;
\filldraw[fill opacity=0.8,fill=gray!20](-9.101,1.132)--(-9.139,1.161)--(-9.154,1.151)--(-9.123,1.118)--cycle;
\filldraw[fill opacity=0.8,fill=gray!20,draw=none](-9.057,1.035)--(-9.06,1.036)--(-9.063,1.055)--(-9.055,1.051)--cycle;
\draw(-9.063,1.055)--(-9.055,1.051);
\filldraw[fill opacity=0.8,fill=gray!20,draw=none](-9.068,1.066)--(-9.076,1.092)--(-9.091,1.082)--cycle;
\draw(-9.076,1.092)--(-9.091,1.082);
\filldraw[fill opacity=0.8,fill=gray!20,draw=none](-9.068,1.066)--(-9.076,1.092)--(-9.091,1.082)--cycle;
\draw(-9.076,1.092)--(-9.091,1.082);
\filldraw[fill opacity=0.8,fill=gray!20,draw=none](-8.981,1.18)--(-9.071,1.079)--(-9.063,1.065)--(-9.051,1.057)--(-8.963,1.156)--cycle;
\draw(-9.051,1.057)--(-8.963,1.156);
\filldraw[fill opacity=0.8,fill=gray!20,draw=none](-8.981,1.18)--(-9.06,1.091)--(-9.055,1.052)--(-8.963,1.156)--cycle;
\draw(-9.055,1.052)--(-8.963,1.156);
\filldraw[fill opacity=0.8,fill=gray!20,draw=none](-8.981,1.18)--(-9.06,1.091)--(-9.055,1.052)--(-8.963,1.156)--cycle;
\draw(-9.055,1.052)--(-8.963,1.156);
\filldraw[fill opacity=0.8,fill=gray!20,draw=none](-9.123,.855)--(-9.131,.859)--(-9.085,.893)--(-9.08,.891)--cycle;
\draw(-9.123,.855)--(-9.131,.859);
\draw(-9.085,.893)--(-9.08,.891);
\filldraw[fill opacity=0.8,fill=gray!20,draw=none](-9.229,1.186)--(-9.231,1.186)--(-9.231,1.186)--(-9.228,1.186)--cycle;
\draw(-9.229,1.186)--(-9.231,1.186)--(-9.231,1.186)--(-9.228,1.186);
\filldraw[fill opacity=0.8,fill=gray!20,draw=none](-9.229,1.186)--(-9.231,1.186)--(-9.231,1.186)--(-9.228,1.186)--cycle;
\draw(-9.229,1.186)--(-9.231,1.186)--(-9.231,1.186)--(-9.228,1.186);
\filldraw[fill opacity=0.8,fill=gray!20,draw=none](-9.117,.872)--(-9.093,.902)--(-9.108,.886)--(-9.123,.865)--cycle;
\draw(-9.108,.886)--(-9.123,.865);
\filldraw[fill opacity=0.8,fill=gray!20,draw=none](-9.117,.872)--(-9.093,.902)--(-9.108,.886)--(-9.123,.865)--cycle;
\draw(-9.108,.886)--(-9.123,.865);
\filldraw[fill opacity=0.8,fill=gray!20,draw=none](-9.095,.901)--(-9.092,.903)--(-9.067,.95)--(-9.072,.947)--cycle;
\draw(-9.095,.901)--(-9.092,.903);
\draw(-9.067,.95)--(-9.072,.947);
\filldraw[fill opacity=0.8,fill=gray!20,draw=none](-9.203,1.174)--(-9.228,1.186)--(-9.272,1.137)--(-9.233,1.132)--(-9.2,1.168)--cycle;
\draw(-9.228,1.186)--(-9.272,1.137);
\draw(-9.233,1.132)--(-9.2,1.168);
\filldraw[fill opacity=0.8,fill=gray!20,draw=none](-9.203,1.174)--(-9.228,1.186)--(-9.272,1.137)--(-9.233,1.132)--(-9.2,1.168)--cycle;
\draw(-9.228,1.186)--(-9.272,1.137);
\draw(-9.233,1.132)--(-9.2,1.168);
\filldraw[fill opacity=0.8,fill=gray!20,draw=none](-9.2,1.168)--(-9.17,1.148)--(-9.154,1.151)--(-9.191,1.174)--(-9.202,1.172)--cycle;
\draw(-9.17,1.148)--(-9.154,1.151)--(-9.191,1.174)--(-9.202,1.172);
\filldraw[fill opacity=0.8,fill=gray!20,draw=none](-9.208,1.183)--(-9.228,1.186)--(-9.203,1.174)--cycle;
\filldraw[fill opacity=0.8,fill=gray!20,draw=none](-9.208,1.183)--(-9.228,1.186)--(-9.203,1.174)--cycle;
\filldraw[fill opacity=0.8,fill=gray!20,draw=none](-9.205,1.177)--(-9.228,1.186)--(-9.272,1.137)--(-9.233,1.132)--(-9.2,1.168)--cycle;
\draw(-9.228,1.186)--(-9.272,1.137);
\draw(-9.233,1.132)--(-9.2,1.168);
\filldraw[fill opacity=0.8,fill=gray!20,draw=none](-9.059,.975)--(-9.064,.945)--(-9.069,.947)--cycle;
\draw(-9.064,.945)--(-9.069,.947);
\filldraw[fill opacity=0.8,fill=gray!20,draw=none](-9.078,.921)--(-9.092,.903)--(-9.069,.947)--(-9.064,.945)--cycle;
\draw(-9.069,.947)--(-9.064,.945);
\filldraw[fill opacity=0.8,fill=gray!20,draw=none](-9.123,.865)--(-9.131,.859)--(-9.133,.86)--cycle;
\draw(-9.131,.859)--(-9.133,.86);
\filldraw[fill opacity=0.8,fill=gray!20,draw=none](-9.149,.847)--(-9.173,.839)--(-9.183,.843)--(-9.133,.86)--(-9.131,.859)--cycle;
\draw(-9.173,.839)--(-9.183,.843);
\draw(-9.133,.86)--(-9.131,.859);
\filldraw[fill opacity=0.8,fill=gray!20](-9.236,.833)--(-9.238,.853)--(-9.302,.857)--(-9.281,.836)--cycle;
\filldraw[fill opacity=0.8,fill=gray!20](-9.236,.833)--(-9.238,.853)--(-9.302,.857)--(-9.281,.836)--cycle;
\filldraw[fill opacity=0.8,fill=gray!20](-9.257,1.171)--(-9.231,1.186)--(-9.231,1.186)--(-9.273,1.175)--cycle;
\filldraw[fill opacity=0.8,fill=gray!20](-9.257,1.171)--(-9.231,1.186)--(-9.231,1.186)--(-9.273,1.175)--cycle;
\filldraw[fill opacity=0.8,fill=gray!20,draw=none](-9.205,1.171)--(-9.202,1.172)--(-9.224,1.184)--(-9.231,1.186)--(-9.231,1.186)--(-9.214,1.173)--cycle;
\draw(-9.205,1.171)--(-9.202,1.172);
\draw(-9.224,1.184)--(-9.231,1.186)--(-9.231,1.186)--(-9.214,1.173);
\filldraw[fill opacity=0.8,fill=gray!20,draw=none](-9.205,1.171)--(-9.202,1.172)--(-9.224,1.184)--(-9.231,1.186)--(-9.231,1.186)--(-9.214,1.173)--cycle;
\draw(-9.205,1.171)--(-9.202,1.172);
\draw(-9.224,1.184)--(-9.231,1.186)--(-9.231,1.186)--(-9.214,1.173);
\filldraw[fill opacity=0.8,fill=gray!20](-9.209,1.17)--(-9.231,1.186)--(-9.231,1.186)--(-9.233,1.169)--cycle;
\filldraw[fill opacity=0.8,fill=gray!20](-9.233,1.169)--(-9.231,1.186)--(-9.231,1.186)--(-9.257,1.171)--cycle;
\filldraw[fill opacity=0.8,fill=gray!20](-9.209,1.17)--(-9.231,1.186)--(-9.231,1.186)--(-9.233,1.169)--cycle;
\filldraw[fill opacity=0.8,fill=gray!20](-9.233,1.169)--(-9.231,1.186)--(-9.231,1.186)--(-9.257,1.171)--cycle;
\filldraw[fill opacity=0.8,fill=gray!20,draw=none](-9.059,.975)--(-9.068,.95)--(-9.06,.997)--(-9.055,.995)--cycle;
\draw(-9.06,.997)--(-9.055,.995);
\filldraw[fill opacity=0.8,fill=gray!20,draw=none](-9.072,.947)--(-9.068,.95)--(-9.06,.997)--(-9.065,.994)--cycle;
\draw(-9.072,.947)--(-9.068,.95);
\draw(-9.06,.997)--(-9.065,.994);
\filldraw[fill opacity=0.8,fill=gray!20,draw=none](-9.072,.947)--(-9.068,.95)--(-9.06,.997)--(-9.065,.994)--cycle;
\draw(-9.072,.947)--(-9.068,.95);
\draw(-9.06,.997)--(-9.065,.994);
\filldraw[fill opacity=0.8,fill=gray!20,draw=none](-9.143,.861)--(-9.123,.865)--(-9.108,.886)--cycle;
\draw(-9.143,.861)--(-9.123,.865)--(-9.108,.886);
\filldraw[fill opacity=0.8,fill=gray!20,draw=none](-9.143,.861)--(-9.123,.865)--(-9.108,.886)--cycle;
\draw(-9.143,.861)--(-9.123,.865)--(-9.108,.886);
\filldraw[fill opacity=0.8,fill=gray!20,draw=none](-9.085,1.062)--(-9.069,1.041)--(-9.061,1.046)--(-9.068,1.066)--(-9.091,1.082)--(-9.095,1.08)--cycle;
\draw(-9.069,1.041)--(-9.061,1.046);
\draw(-9.091,1.082)--(-9.095,1.08);
\filldraw[fill opacity=0.8,fill=gray!20,draw=none](-9.085,1.062)--(-9.069,1.041)--(-9.061,1.046)--(-9.068,1.066)--(-9.091,1.082)--(-9.095,1.08)--cycle;
\draw(-9.069,1.041)--(-9.061,1.046);
\draw(-9.091,1.082)--(-9.095,1.08);
\filldraw[fill opacity=0.8,fill=gray!20,draw=none](-9.061,1.046)--(-9.064,1.055)--(-9.063,1.055)--cycle;
\draw(-9.064,1.055)--(-9.063,1.055);
\filldraw[fill opacity=0.8,fill=gray!20,draw=none](-9.066,1.04)--(-9.067,1.038)--(-9.068,1.042)--(-9.068,1.045)--(-9.067,1.044)--cycle;
\draw(-9.068,1.045)--(-9.067,1.044);
\filldraw[fill opacity=0.8,fill=gray!20,draw=none](-9.055,1.052)--(-9.068,1.039)--(-9.057,1.034)--cycle;
\draw(-9.055,1.052)--(-9.068,1.039);
\filldraw[fill opacity=0.8,fill=gray!20,draw=none](-9.055,1.052)--(-9.068,1.039)--(-9.057,1.034)--cycle;
\draw(-9.055,1.052)--(-9.068,1.039);
\filldraw[fill opacity=0.8,fill=gray!20,draw=none](-9.055,1.052)--(-9.068,1.039)--(-9.057,1.034)--cycle;
\draw(-9.055,1.052)--(-9.068,1.039);
\filldraw[fill opacity=0.8,fill=gray!20,draw=none](-8.997,1.198)--(-8.887,1.323)--(-8.912,1.348)--(-9.022,1.225)--cycle;
\draw(-8.887,1.323)--(-8.912,1.348)--(-9.022,1.225);
\filldraw[fill opacity=0.8,fill=gray!20,draw=none](-8.997,1.198)--(-8.887,1.323)--(-8.912,1.348)--(-9.022,1.225)--cycle;
\draw(-8.887,1.323)--(-8.912,1.348)--(-9.022,1.225);
\filldraw[fill opacity=0.8,fill=gray!20,draw=none](-8.997,1.198)--(-8.887,1.323)--(-8.912,1.348)--(-9.022,1.225)--cycle;
\draw(-8.887,1.323)--(-8.912,1.348)--(-9.022,1.225);
\filldraw[fill opacity=0.8,fill=gray!20,draw=none](-9.149,.847)--(-9.131,.859)--(-9.123,.855)--cycle;
\draw(-9.131,.859)--(-9.123,.855);
\filldraw[fill opacity=0.8,fill=gray!20,draw=none](-9.049,1.031)--(-9.057,1.035)--(-9.055,1.051)--(-9.052,1.05)--cycle;
\draw(-9.055,1.051)--(-9.052,1.05);
\filldraw[fill opacity=0.8,fill=gray!20,draw=none](-9.298,1.082)--(-9.314,1.089)--(-9.339,1.044)--(-9.316,1.034)--cycle;
\draw(-9.298,1.082)--(-9.314,1.089)--(-9.339,1.044)--(-9.316,1.034);
\filldraw[fill opacity=0.8,fill=gray!20,draw=none](-9.33,1.06)--(-9.355,1.071)--(-9.355,1.05)--(-9.337,1.042)--cycle;
\draw(-9.355,1.05)--(-9.337,1.042);
\filldraw[fill opacity=0.8,fill=gray!20,draw=none](-9.078,.921)--(-9.09,.9)--(-9.093,.902)--cycle;
\draw(-9.09,.9)--(-9.093,.902);
\filldraw[fill opacity=0.8,fill=gray!20,draw=none](-9.093,.897)--(-9.096,.899)--(-9.093,.902)--(-9.09,.9)--cycle;
\draw(-9.093,.902)--(-9.09,.9);
\filldraw[fill opacity=0.8,fill=gray!20,draw=none](-9.093,.902)--(-9.092,.903)--(-9.095,.901)--(-9.106,.888)--(-9.108,.886)--cycle;
\draw(-9.092,.903)--(-9.095,.901);
\draw(-9.106,.888)--(-9.108,.886);
\filldraw[fill opacity=0.8,fill=gray!20,draw=none](-9.093,.902)--(-9.092,.903)--(-9.095,.901)--(-9.106,.888)--(-9.108,.886)--cycle;
\draw(-9.092,.903)--(-9.095,.901);
\draw(-9.106,.888)--(-9.108,.886);
\filldraw[fill opacity=0.8,fill=gray!20](-9.19,.835)--(-9.173,.856)--(-9.238,.853)--(-9.236,.833)--cycle;
\filldraw[fill opacity=0.8,fill=gray!20](-9.19,.835)--(-9.173,.856)--(-9.238,.853)--(-9.236,.833)--cycle;
\filldraw[fill opacity=0.8,fill=gray!20,draw=none](-9.063,1.012)--(-9.059,1.018)--(-9.055,.995)--(-9.06,.997)--cycle;
\draw(-9.055,.995)--(-9.06,.997);
\filldraw[fill opacity=0.8,fill=gray!20,draw=none](-9.063,1.012)--(-9.067,1.009)--(-9.065,.994)--(-9.06,.997)--cycle;
\draw(-9.065,.994)--(-9.06,.997);
\filldraw[fill opacity=0.8,fill=gray!20,draw=none](-9.063,1.012)--(-9.067,1.009)--(-9.065,.994)--(-9.06,.997)--cycle;
\draw(-9.065,.994)--(-9.06,.997);
\filldraw[fill opacity=0.8,fill=gray!20,draw=none](-9.096,.899)--(-9.093,.897)--(-9.112,.882)--cycle;
\filldraw[fill opacity=0.8,fill=gray!20,draw=none](-8.951,1.381)--(-9.07,1.247)--(-9.022,1.225)--(-8.912,1.348)--cycle;
\draw(-9.022,1.225)--(-8.912,1.348)--(-8.951,1.381)--(-9.07,1.247);
\filldraw[fill opacity=0.8,fill=gray!20,draw=none](-8.951,1.381)--(-9.07,1.247)--(-9.022,1.225)--(-8.912,1.348)--cycle;
\draw(-9.022,1.225)--(-8.912,1.348)--(-8.951,1.381)--(-9.07,1.247);
\filldraw[fill opacity=0.8,fill=gray!20,draw=none](-8.951,1.381)--(-9.07,1.247)--(-9.022,1.225)--(-8.912,1.348)--cycle;
\draw(-9.022,1.225)--(-8.912,1.348)--(-8.951,1.381)--(-9.07,1.247);
\filldraw[fill opacity=0.8,fill=gray!20,draw=none](-9.08,1.068)--(-9.067,1.044)--(-9.069,1.045)--(-9.093,1.083)--(-9.09,1.082)--cycle;
\draw(-9.067,1.044)--(-9.069,1.045);
\draw(-9.093,1.083)--(-9.09,1.082);
\filldraw[fill opacity=0.8,fill=gray!20,draw=none](-9.069,1.041)--(-9.085,1.062)--(-9.165,.973)--(-9.145,.951)--(-9.068,1.038)--cycle;
\draw(-9.165,.973)--(-9.145,.951)--(-9.068,1.038);
\filldraw[fill opacity=0.8,fill=gray!20,draw=none](-9.069,1.041)--(-9.085,1.062)--(-9.165,.973)--(-9.145,.951)--(-9.068,1.038)--cycle;
\draw(-9.165,.973)--(-9.145,.951)--(-9.068,1.038);
\filldraw[fill opacity=0.8,fill=gray!20,draw=none](-9.069,1.041)--(-9.074,1.056)--(-9.08,1.068)--(-9.085,1.062)--cycle;
\filldraw[fill opacity=0.8,fill=gray!20,draw=none](-9.069,1.041)--(-9.074,1.056)--(-9.08,1.068)--(-9.085,1.062)--cycle;
\filldraw[fill opacity=0.8,fill=gray!20,draw=none](-9.069,1.041)--(-9.085,1.062)--(-9.165,.973)--(-9.145,.951)--(-9.068,1.038)--cycle;
\draw(-9.165,.973)--(-9.145,.951)--(-9.068,1.038);
\filldraw[fill opacity=0.8,fill=gray!20,draw=none](-9.067,1.009)--(-9.062,1.013)--(-9.069,1.041)--(-9.072,1.039)--cycle;
\draw(-9.069,1.041)--(-9.072,1.039);
\filldraw[fill opacity=0.8,fill=gray!20,draw=none](-9.067,1.009)--(-9.062,1.013)--(-9.069,1.041)--(-9.072,1.039)--cycle;
\draw(-9.069,1.041)--(-9.072,1.039);
\filldraw[fill opacity=0.8,fill=gray!20](-9.318,.889)--(-9.328,.929)--(-9.389,.944)--(-9.372,.902)--cycle;
\filldraw[fill opacity=0.8,fill=gray!20](-9.318,.889)--(-9.328,.929)--(-9.389,.944)--(-9.372,.902)--cycle;
\filldraw[fill opacity=0.8,fill=gray!20,draw=none](-9.068,1.042)--(-9.069,1.045)--(-9.068,1.045)--cycle;
\draw(-9.069,1.045)--(-9.068,1.045);
\filldraw[fill opacity=0.8,fill=gray!20,draw=none](-9.063,1.065)--(-9.056,1.052)--(-9.051,1.057)--cycle;
\draw(-9.056,1.052)--(-9.051,1.057);
\filldraw[fill opacity=0.8,fill=gray!20,draw=none](-9.063,1.065)--(-9.076,1.073)--(-9.078,1.071)--(-9.074,1.056)--(-9.065,1.041)--(-9.056,1.052)--cycle;
\draw(-9.065,1.041)--(-9.056,1.052);
\filldraw[fill opacity=0.8,fill=gray!20,draw=none](-9.074,1.056)--(-9.069,1.041)--(-9.068,1.039)--(-9.065,1.041)--cycle;
\draw(-9.068,1.039)--(-9.065,1.041);
\filldraw[fill opacity=0.8,fill=gray!20,draw=none](-9.071,1.079)--(-9.076,1.073)--(-9.063,1.065)--cycle;
\filldraw[fill opacity=0.8,fill=gray!20,draw=none](-9.071,1.079)--(-9.078,1.071)--(-9.074,1.056)--(-9.065,1.041)--(-9.055,1.052)--cycle;
\draw(-9.065,1.041)--(-9.055,1.052);
\filldraw[fill opacity=0.8,fill=gray!20,draw=none](-9.071,1.079)--(-9.078,1.071)--(-9.074,1.056)--(-9.065,1.041)--(-9.055,1.052)--cycle;
\draw(-9.065,1.041)--(-9.055,1.052);
\filldraw[fill opacity=0.8,fill=gray!20,draw=none](-9.074,1.056)--(-9.068,1.038)--(-9.065,1.041)--cycle;
\draw(-9.068,1.038)--(-9.065,1.041);
\filldraw[fill opacity=0.8,fill=gray!20,draw=none](-9.074,1.056)--(-9.068,1.038)--(-9.065,1.041)--cycle;
\draw(-9.068,1.038)--(-9.065,1.041);
\filldraw[fill opacity=0.8,fill=gray!20,draw=none](-9.059,1.018)--(-9.063,1.012)--(-9.065,1.024)--(-9.066,1.04)--(-9.066,1.044)--(-9.064,1.043)--cycle;
\draw(-9.066,1.044)--(-9.064,1.043);
\filldraw[fill opacity=0.8,fill=gray!20,draw=none](-9.17,1.148)--(-9.154,1.151)--(-9.191,1.174)--(-9.205,1.171)--cycle;
\draw(-9.17,1.148)--(-9.154,1.151)--(-9.191,1.174)--(-9.205,1.171);
\filldraw[fill opacity=0.8,fill=gray!20,draw=none](-9.091,1.248)--(-9.07,1.247)--(-9.036,1.285)--cycle;
\draw(-9.07,1.247)--(-9.036,1.285);
\filldraw[fill opacity=0.8,fill=gray!20,draw=none](-9.091,1.248)--(-9.07,1.247)--(-9.036,1.285)--cycle;
\draw(-9.07,1.247)--(-9.036,1.285);
\filldraw[fill opacity=0.8,fill=gray!20,draw=none](-9.091,1.248)--(-9.07,1.247)--(-9.036,1.285)--cycle;
\draw(-9.07,1.247)--(-9.036,1.285);
\filldraw[fill opacity=0.8,fill=gray!20,draw=none](-9.095,1.287)--(-9.063,1.256)--(-9.076,1.282)--cycle;
\filldraw[fill opacity=0.8,fill=gray!20,draw=none](-8.995,1.194)--(-8.997,1.198)--(-9.039,1.244)--(-9.053,1.245)--(-9.041,1.198)--cycle;
\draw(-9.039,1.244)--(-9.053,1.245)--(-9.041,1.198)--(-8.995,1.194);
\filldraw[fill opacity=0.8,fill=gray!20,draw=none](-8.953,1.458)--(-8.945,1.463)--(-8.943,1.501)--(-8.997,1.505)--(-9.022,1.463)--cycle;
\draw(-8.945,1.463)--(-8.943,1.501)--(-8.997,1.505)--(-9.022,1.463)--(-8.953,1.458);
\filldraw[fill opacity=0.8,fill=gray!20,draw=none](-9.065,1.024)--(-9.067,1.036)--(-9.066,1.04)--cycle;
\filldraw[fill opacity=0.8,fill=gray!20,draw=none](-9.108,.886)--(-9.112,.882)--(-9.123,.872)--(-9.129,.869)--(-9.131,.87)--cycle;
\draw(-9.129,.869)--(-9.131,.87);
\filldraw[fill opacity=0.8,fill=gray!20,draw=none](-9.164,.857)--(-9.131,.87)--(-9.108,.886)--(-9.106,.888)--(-9.173,.856)--cycle;
\draw(-9.108,.886)--(-9.106,.888);
\draw(-9.173,.856)--(-9.164,.857);
\filldraw[fill opacity=0.8,fill=gray!20,draw=none](-9.164,.857)--(-9.131,.87)--(-9.108,.886)--(-9.106,.888)--(-9.173,.856)--cycle;
\draw(-9.108,.886)--(-9.106,.888);
\draw(-9.173,.856)--(-9.164,.857);
\filldraw[fill opacity=0.8,fill=gray!20,draw=none](-9.311,1.092)--(-9.311,1.092)--(-9.312,1.093)--cycle;
\draw(-9.311,1.092)--(-9.312,1.093);
\filldraw[fill opacity=0.8,fill=gray!20,draw=none](-9.311,1.092)--(-9.311,1.092)--(-9.312,1.093)--cycle;
\draw(-9.311,1.092)--(-9.312,1.093);
\filldraw[fill opacity=0.8,fill=gray!20,draw=none](-9.311,1.092)--(-9.311,1.092)--(-9.312,1.093)--cycle;
\draw(-9.311,1.092)--(-9.312,1.093);
\filldraw[fill opacity=0.8,fill=gray!20,draw=none](-9.208,1.183)--(-9.203,1.174)--(-9.198,1.171)--(-9.19,1.181)--cycle;
\draw(-9.198,1.171)--(-9.19,1.181);
\filldraw[fill opacity=0.8,fill=gray!20,draw=none](-9.208,1.183)--(-9.203,1.174)--(-9.198,1.171)--(-9.19,1.181)--cycle;
\draw(-9.198,1.171)--(-9.19,1.181);
\filldraw[fill opacity=0.8,fill=gray!20,draw=none](-9.164,.857)--(-9.143,.861)--(-9.131,.87)--cycle;
\draw(-9.164,.857)--(-9.143,.861);
\filldraw[fill opacity=0.8,fill=gray!20,draw=none](-9.164,.857)--(-9.143,.861)--(-9.131,.87)--cycle;
\draw(-9.164,.857)--(-9.143,.861);
\filldraw[fill opacity=0.8,fill=gray!20,draw=none](-9.164,.857)--(-9.131,.87)--(-9.129,.869)--cycle;
\draw(-9.131,.87)--(-9.129,.869);
\filldraw[fill opacity=0.8,fill=gray!20,draw=none](-9.067,1.038)--(-9.068,1.039)--(-9.068,1.038)--(-9.067,1.036)--cycle;
\draw(-9.068,1.039)--(-9.068,1.038);
\filldraw[fill opacity=0.8,fill=gray!20](-9.281,1.145)--(-9.257,1.171)--(-9.273,1.175)--(-9.312,1.153)--cycle;
\filldraw[fill opacity=0.8,fill=gray!20](-9.281,1.145)--(-9.257,1.171)--(-9.273,1.175)--(-9.312,1.153)--cycle;
\filldraw[fill opacity=0.8,fill=gray!20,draw=none](-9.06,1.091)--(-9.071,1.079)--(-9.055,1.052)--cycle;
\filldraw[fill opacity=0.8,fill=gray!20,draw=none](-9.06,1.091)--(-9.071,1.079)--(-9.055,1.052)--cycle;
\filldraw[fill opacity=0.8,fill=gray!20,draw=none](-9.309,1.091)--(-9.311,1.092)--(-9.311,1.092)--cycle;
\draw(-9.309,1.091)--(-9.311,1.092);
\filldraw[fill opacity=0.8,fill=gray!20,draw=none](-9.309,1.091)--(-9.311,1.092)--(-9.311,1.092)--cycle;
\draw(-9.309,1.091)--(-9.311,1.092);
\filldraw[fill opacity=0.8,fill=gray!20,draw=none](-9.309,1.091)--(-9.311,1.092)--(-9.311,1.092)--cycle;
\draw(-9.309,1.091)--(-9.311,1.092);
\filldraw[fill opacity=0.8,fill=gray!20,draw=none](-8.878,1.485)--(-8.887,1.503)--(-8.943,1.501)--(-8.945,1.463)--cycle;
\draw(-8.878,1.485)--(-8.887,1.503)--(-8.943,1.501)--(-8.945,1.463);
\filldraw[fill opacity=0.8,fill=gray!20,draw=none](-9.173,.853)--(-9.174,.854)--(-9.129,.869)--(-9.128,.868)--cycle;
\draw(-9.173,.853)--(-9.174,.854);
\draw(-9.129,.869)--(-9.128,.868);
\filldraw[fill opacity=0.8,fill=gray!20,draw=none](-9.123,.872)--(-9.128,.868)--(-9.129,.869)--cycle;
\draw(-9.128,.868)--(-9.129,.869);
\filldraw[fill opacity=0.8,fill=gray!20,draw=none](-9.205,.843)--(-9.231,.848)--(-9.233,.849)--(-9.183,.843)--cycle;
\draw(-9.231,.848)--(-9.233,.849);
\filldraw[fill opacity=0.8,fill=gray!20,draw=none](-9.21,.843)--(-9.183,.843)--(-9.173,.839)--cycle;
\draw(-9.183,.843)--(-9.173,.839);
\filldraw[fill opacity=0.8,fill=gray!20,draw=none](-9.066,1.04)--(-9.067,1.044)--(-9.066,1.044)--cycle;
\draw(-9.067,1.044)--(-9.066,1.044);
\filldraw[fill opacity=0.8,fill=gray!20,draw=none](-9.069,1.041)--(-9.068,1.038)--(-9.068,1.039)--cycle;
\draw(-9.068,1.038)--(-9.068,1.039);
\filldraw[fill opacity=0.8,fill=gray!20,draw=none](-9.08,1.068)--(-9.064,1.043)--(-9.067,1.044)--cycle;
\draw(-9.064,1.043)--(-9.067,1.044);
\filldraw[fill opacity=0.8,fill=gray!20,draw=none](-9.106,1.092)--(-9.093,1.083)--(-9.121,1.119)--(-9.123,1.118)--(-9.109,1.096)--cycle;
\draw(-9.121,1.119)--(-9.123,1.118)--(-9.109,1.096);
\filldraw[fill opacity=0.8,fill=gray!20,draw=none](-9.106,1.092)--(-9.093,1.083)--(-9.121,1.119)--(-9.123,1.118)--(-9.109,1.096)--cycle;
\draw(-9.121,1.119)--(-9.123,1.118)--(-9.109,1.096);
\filldraw[fill opacity=0.8,fill=gray!20,draw=none](-9.092,1.1)--(-9.087,1.097)--(-9.08,1.093)--(-9.093,1.099)--(-9.123,1.118)--cycle;
\draw(-9.08,1.093)--(-9.093,1.099);
\filldraw[fill opacity=0.8,fill=gray!20,draw=none](-9.069,1.041)--(-9.074,1.056)--(-9.08,1.068)--(-9.085,1.062)--cycle;
\filldraw[fill opacity=0.8,fill=gray!20,draw=none](-9.085,1.062)--(-9.074,1.041)--(-9.071,1.04)--(-9.069,1.041)--cycle;
\draw(-9.071,1.04)--(-9.069,1.041);
\filldraw[fill opacity=0.8,fill=gray!20,draw=none](-9.085,1.062)--(-9.074,1.041)--(-9.071,1.04)--(-9.069,1.041)--cycle;
\draw(-9.071,1.04)--(-9.069,1.041);
\filldraw[fill opacity=0.8,fill=gray!20,draw=none](-9.205,1.177)--(-9.2,1.168)--(-9.196,1.174)--cycle;
\draw(-9.2,1.168)--(-9.196,1.174);
\filldraw[fill opacity=0.8,fill=gray!20,draw=none](-9.063,1.256)--(-9.053,1.246)--(-9.057,1.299)--(-9.089,1.307)--cycle;
\draw(-9.053,1.246)--(-9.057,1.299)--(-9.089,1.307);
\filldraw[fill opacity=0.8,fill=gray!20](-9.328,.929)--(-9.331,.974)--(-9.394,.989)--(-9.389,.944)--cycle;
\filldraw[fill opacity=0.8,fill=gray!20](-9.328,.929)--(-9.331,.974)--(-9.394,.989)--(-9.389,.944)--cycle;
\filldraw[fill opacity=0.8,fill=gray!20,draw=none](-9.074,1.041)--(-9.072,1.039)--(-9.071,1.04)--cycle;
\draw(-9.072,1.039)--(-9.071,1.04);
\filldraw[fill opacity=0.8,fill=gray!20,draw=none](-9.074,1.041)--(-9.072,1.039)--(-9.071,1.04)--cycle;
\draw(-9.072,1.039)--(-9.071,1.04);
\filldraw[fill opacity=0.8,fill=gray!20,draw=none](-9.203,1.174)--(-9.2,1.168)--(-9.198,1.171)--cycle;
\draw(-9.2,1.168)--(-9.198,1.171);
\filldraw[fill opacity=0.8,fill=gray!20,draw=none](-9.203,1.174)--(-9.2,1.168)--(-9.198,1.171)--cycle;
\draw(-9.2,1.168)--(-9.198,1.171);
\filldraw[fill opacity=0.8,fill=gray!20,draw=none](-9.106,1.092)--(-9.099,1.084)--(-9.095,1.08)--(-9.092,1.082)--(-9.093,1.083)--cycle;
\draw(-9.095,1.08)--(-9.092,1.082);
\filldraw[fill opacity=0.8,fill=gray!20,draw=none](-9.106,1.092)--(-9.099,1.084)--(-9.095,1.08)--(-9.092,1.082)--(-9.093,1.083)--cycle;
\draw(-9.095,1.08)--(-9.092,1.082);
\filldraw[fill opacity=0.8,fill=gray!20,draw=none](-9.087,1.097)--(-8.997,1.198)--(-9.022,1.225)--(-9.118,1.116)--cycle;
\draw(-9.022,1.225)--(-9.118,1.116);
\filldraw[fill opacity=0.8,fill=gray!20,draw=none](-9.087,1.097)--(-8.997,1.198)--(-9.022,1.225)--(-9.118,1.116)--cycle;
\draw(-9.022,1.225)--(-9.118,1.116);
\filldraw[fill opacity=0.8,fill=gray!20,draw=none](-9.087,1.097)--(-8.997,1.198)--(-9.022,1.225)--(-9.118,1.116)--cycle;
\draw(-9.022,1.225)--(-9.118,1.116);
\filldraw[fill opacity=0.8,fill=gray!20,draw=none](-8.997,1.198)--(-9.022,1.243)--(-9.039,1.244)--cycle;
\draw(-9.022,1.243)--(-9.039,1.244);
\filldraw[fill opacity=0.8,fill=gray!20,draw=none](-9.166,1.14)--(-9.146,1.125)--(-9.128,1.123)--(-9.154,1.151)--(-9.17,1.148)--cycle;
\draw(-9.128,1.123)--(-9.154,1.151)--(-9.17,1.148);
\filldraw[fill opacity=0.8,fill=gray!20,draw=none](-9.166,1.14)--(-9.146,1.125)--(-9.128,1.123)--(-9.154,1.151)--(-9.17,1.148)--cycle;
\draw(-9.128,1.123)--(-9.154,1.151)--(-9.17,1.148);
\filldraw[fill opacity=0.8,fill=gray!20,draw=none](-9.091,1.248)--(-9.187,1.183)--(-9.19,1.181)--(-9.17,1.148)--(-9.163,1.143)--(-9.07,1.247)--cycle;
\draw(-9.187,1.183)--(-9.19,1.181);
\draw(-9.163,1.143)--(-9.07,1.247);
\filldraw[fill opacity=0.8,fill=gray!20,draw=none](-9.091,1.248)--(-9.187,1.183)--(-9.19,1.181)--(-9.17,1.148)--(-9.163,1.143)--(-9.07,1.247)--cycle;
\draw(-9.187,1.183)--(-9.19,1.181);
\draw(-9.163,1.143)--(-9.07,1.247);
\filldraw[fill opacity=0.8,fill=gray!20,draw=none](-9.091,1.248)--(-9.187,1.183)--(-9.19,1.181)--(-9.17,1.148)--(-9.163,1.143)--(-9.07,1.247)--cycle;
\draw(-9.187,1.183)--(-9.19,1.181);
\draw(-9.163,1.143)--(-9.07,1.247);
\filldraw[fill opacity=0.8,fill=gray!20,draw=none](-9.205,1.171)--(-9.214,1.173)--(-9.209,1.17)--cycle;
\draw(-9.214,1.173)--(-9.209,1.17)--(-9.205,1.171);
\filldraw[fill opacity=0.8,fill=gray!20,draw=none](-9.205,1.171)--(-9.214,1.173)--(-9.209,1.17)--cycle;
\draw(-9.214,1.173)--(-9.209,1.17)--(-9.205,1.171);
\filldraw[fill opacity=0.8,fill=gray!20,draw=none](-9.233,1.132)--(-9.282,1.076)--(-9.246,1.049)--(-9.166,1.139)--cycle;
\draw(-9.233,1.132)--(-9.282,1.076)--(-9.246,1.049)--(-9.166,1.139);
\filldraw[fill opacity=0.8,fill=gray!20,draw=none](-9.2,1.168)--(-9.282,1.076)--(-9.246,1.049)--(-9.163,1.143)--cycle;
\draw(-9.2,1.168)--(-9.282,1.076)--(-9.246,1.049)--(-9.163,1.143);
\filldraw[fill opacity=0.8,fill=gray!20,draw=none](-9.2,1.168)--(-9.282,1.076)--(-9.246,1.049)--(-9.163,1.143)--cycle;
\draw(-9.2,1.168)--(-9.282,1.076)--(-9.246,1.049)--(-9.163,1.143);
\filldraw[fill opacity=0.8,fill=gray!20,draw=none](-9.2,1.168)--(-9.233,1.132)--(-9.166,1.139)--(-9.163,1.143)--cycle;
\draw(-9.2,1.168)--(-9.233,1.132);
\draw(-9.166,1.139)--(-9.163,1.143);
\filldraw[fill opacity=0.8,fill=gray!20,draw=none](-9.187,1.144)--(-9.202,1.172)--(-9.209,1.17)--(-9.19,1.144)--cycle;
\draw(-9.202,1.172)--(-9.209,1.17)--(-9.19,1.144)--(-9.187,1.144);
\filldraw[fill opacity=0.8,fill=gray!20,draw=none](-9.074,1.056)--(-9.078,1.071)--(-9.08,1.068)--cycle;
\filldraw[fill opacity=0.8,fill=gray!20,draw=none](-9.074,1.056)--(-9.078,1.071)--(-9.08,1.068)--cycle;
\filldraw[fill opacity=0.8,fill=gray!20,draw=none](-9.074,1.056)--(-9.078,1.071)--(-9.08,1.068)--cycle;
\filldraw[fill opacity=0.8,fill=gray!20](-9.302,1.11)--(-9.281,1.145)--(-9.312,1.153)--(-9.346,1.121)--cycle;
\filldraw[fill opacity=0.8,fill=gray!20](-9.302,1.11)--(-9.281,1.145)--(-9.312,1.153)--(-9.346,1.121)--cycle;
\filldraw[fill opacity=0.8,fill=gray!20,draw=none](-9.19,1.181)--(-9.2,1.168)--(-9.17,1.148)--cycle;
\draw(-9.19,1.181)--(-9.2,1.168);
\filldraw[fill opacity=0.8,fill=gray!20,draw=none](-9.19,1.181)--(-9.2,1.168)--(-9.17,1.148)--cycle;
\draw(-9.19,1.181)--(-9.2,1.168);
\filldraw[fill opacity=0.8,fill=gray!20,draw=none](-9.19,1.181)--(-9.2,1.168)--(-9.17,1.148)--cycle;
\draw(-9.19,1.181)--(-9.2,1.168);
\filldraw[fill opacity=0.8,fill=gray!20,draw=none](-9.17,1.148)--(-9.205,1.171)--(-9.209,1.17)--(-9.19,1.144)--cycle;
\draw(-9.205,1.171)--(-9.209,1.17)--(-9.19,1.144)--(-9.17,1.148);
\filldraw[fill opacity=0.8,fill=gray!20,draw=none](-9.058,1.246)--(-9.053,1.245)--(-9.053,1.246)--(-9.063,1.256)--cycle;
\draw(-9.058,1.246)--(-9.053,1.245)--(-9.053,1.246);
\filldraw[fill opacity=0.8,fill=gray!20](-9.236,1.142)--(-9.233,1.169)--(-9.257,1.171)--(-9.281,1.145)--cycle;
\filldraw[fill opacity=0.8,fill=gray!20](-9.236,1.142)--(-9.233,1.169)--(-9.257,1.171)--(-9.281,1.145)--cycle;
\filldraw[fill opacity=0.8,fill=gray!20,draw=none](-9.238,.853)--(-9.297,.887)--(-9.318,.889)--(-9.302,.857)--cycle;
\draw(-9.297,.887)--(-9.318,.889)--(-9.302,.857)--(-9.238,.853);
\filldraw[fill opacity=0.8,fill=gray!20,draw=none](-9.238,.853)--(-9.297,.887)--(-9.318,.889)--(-9.302,.857)--cycle;
\draw(-9.297,.887)--(-9.318,.889)--(-9.302,.857)--(-9.238,.853);
\filldraw[fill opacity=0.8,fill=gray!20,draw=none](-9.193,1.134)--(-9.198,1.136)--(-9.269,1.128)--(-9.252,1.121)--cycle;
\draw(-9.193,1.134)--(-9.198,1.136);
\draw(-9.269,1.128)--(-9.252,1.121);
\filldraw[fill opacity=0.8,fill=gray!20,draw=none](-9.257,1.113)--(-9.25,1.112)--(-9.245,1.118)--cycle;
\draw(-9.25,1.112)--(-9.245,1.118);
\filldraw[fill opacity=0.8,fill=gray!20,draw=none](-9.285,1.117)--(-9.268,1.109)--(-9.252,1.121)--(-9.268,1.128)--cycle;
\draw(-9.252,1.121)--(-9.268,1.128);
\filldraw[fill opacity=0.8,fill=gray!20,draw=none](-9.051,1.238)--(-9.053,1.245)--(-9.058,1.246)--cycle;
\draw(-9.051,1.238)--(-9.053,1.245)--(-9.058,1.246);
\filldraw[fill opacity=0.8,fill=gray!20,draw=none](-9.146,1.125)--(-9.133,1.116)--(-9.123,1.118)--(-9.128,1.123)--cycle;
\draw(-9.133,1.116)--(-9.123,1.118)--(-9.128,1.123);
\filldraw[fill opacity=0.8,fill=gray!20,draw=none](-9.146,1.125)--(-9.133,1.116)--(-9.123,1.118)--(-9.128,1.123)--cycle;
\draw(-9.133,1.116)--(-9.123,1.118)--(-9.128,1.123);
\filldraw[fill opacity=0.8,fill=gray!20,draw=none](-9.07,1.247)--(-9.164,1.142)--(-9.118,1.116)--(-9.022,1.225)--cycle;
\draw(-9.07,1.247)--(-9.164,1.142);
\draw(-9.118,1.116)--(-9.022,1.225);
\filldraw[fill opacity=0.8,fill=gray!20,draw=none](-9.07,1.247)--(-9.164,1.142)--(-9.118,1.116)--(-9.022,1.225)--cycle;
\draw(-9.07,1.247)--(-9.164,1.142);
\draw(-9.118,1.116)--(-9.022,1.225);
\filldraw[fill opacity=0.8,fill=gray!20,draw=none](-9.07,1.247)--(-9.164,1.142)--(-9.118,1.116)--(-9.022,1.225)--cycle;
\draw(-9.07,1.247)--(-9.164,1.142);
\draw(-9.118,1.116)--(-9.022,1.225);
\filldraw[fill opacity=0.8,fill=gray!20,draw=none](-9.2,1.168)--(-9.191,1.152)--(-9.175,1.147)--(-9.17,1.148)--cycle;
\draw(-9.175,1.147)--(-9.17,1.148);
\filldraw[fill opacity=0.8,fill=gray!20,draw=none](-9.239,1.079)--(-9.221,1.097)--(-9.254,1.112)--(-9.298,1.082)--(-9.26,1.065)--cycle;
\draw(-9.221,1.097)--(-9.254,1.112);
\draw(-9.298,1.082)--(-9.26,1.065);
\filldraw[fill opacity=0.8,fill=gray!20,draw=none](-9.246,1.1)--(-9.285,1.117)--(-9.317,1.095)--(-9.278,1.078)--cycle;
\draw(-9.317,1.095)--(-9.278,1.078);
\filldraw[fill opacity=0.8,fill=gray!20,draw=none](-9.33,1.06)--(-9.313,1.053)--(-9.297,1.087)--(-9.317,1.095)--cycle;
\draw(-9.297,1.087)--(-9.317,1.095);
\filldraw[fill opacity=0.8,fill=gray!20](-9.331,.974)--(-9.328,1.021)--(-9.389,1.036)--(-9.394,.989)--cycle;
\filldraw[fill opacity=0.8,fill=gray!20](-9.331,.974)--(-9.328,1.021)--(-9.389,1.036)--(-9.394,.989)--cycle;
\filldraw[fill opacity=0.8,fill=gray!20](-9.19,1.144)--(-9.209,1.17)--(-9.233,1.169)--(-9.236,1.142)--cycle;
\filldraw[fill opacity=0.8,fill=gray!20](-9.19,1.144)--(-9.209,1.17)--(-9.233,1.169)--(-9.236,1.142)--cycle;
\filldraw[fill opacity=0.8,fill=gray!20,draw=none](-9.175,.854)--(-9.185,.855)--(-9.174,.854)--cycle;
\filldraw[fill opacity=0.8,fill=gray!20,draw=none](-9.177,.854)--(-9.174,.854)--(-9.173,.853)--cycle;
\draw(-9.174,.854)--(-9.173,.853);
\filldraw[fill opacity=0.8,fill=gray!20,draw=none](-9.212,.854)--(-9.173,.856)--(-9.24,.873)--(-9.239,.871)--cycle;
\draw(-9.212,.854)--(-9.173,.856);
\draw(-9.24,.873)--(-9.239,.871);
\filldraw[fill opacity=0.8,fill=gray!20,draw=none](-9.212,.854)--(-9.173,.856)--(-9.24,.873)--(-9.239,.871)--cycle;
\draw(-9.212,.854)--(-9.173,.856);
\draw(-9.24,.873)--(-9.239,.871);
\filldraw[fill opacity=0.8,fill=gray!20,draw=none](-9.175,.854)--(-9.177,.854)--(-9.218,.858)--(-9.219,.859)--(-9.185,.855)--cycle;
\draw(-9.218,.858)--(-9.219,.859);
\filldraw[fill opacity=0.8,fill=gray!20,draw=none](-9.022,1.243)--(-9.031,1.297)--(-9.057,1.299)--(-9.053,1.245)--cycle;
\draw(-9.031,1.297)--(-9.057,1.299)--(-9.053,1.245)--(-9.022,1.243);
\filldraw[fill opacity=0.8,fill=gray!20,draw=none](-8.995,1.408)--(-8.953,1.458)--(-9.022,1.463)--(-9.041,1.412)--cycle;
\draw(-8.953,1.458)--(-9.022,1.463)--(-9.041,1.412)--(-8.995,1.408);
\filldraw[fill opacity=0.8,fill=gray!20](-9.318,1.068)--(-9.302,1.11)--(-9.346,1.121)--(-9.372,1.081)--cycle;
\filldraw[fill opacity=0.8,fill=gray!20](-9.318,1.068)--(-9.302,1.11)--(-9.346,1.121)--(-9.372,1.081)--cycle;
\filldraw[fill opacity=0.8,fill=gray!20,draw=none](-9.106,1.092)--(-9.097,1.086)--(-9.09,1.082)--(-9.093,1.083)--cycle;
\draw(-9.09,1.082)--(-9.093,1.083);
\filldraw[fill opacity=0.8,fill=gray!20,draw=none](-9.185,.855)--(-9.219,.859)--(-9.221,.86)--cycle;
\draw(-9.219,.859)--(-9.221,.86);
\filldraw[fill opacity=0.8,fill=gray!20,draw=none](-9.313,1.053)--(-9.33,1.06)--(-9.337,1.042)--(-9.322,1.036)--cycle;
\draw(-9.337,1.042)--(-9.322,1.036);
\filldraw[fill opacity=0.8,fill=gray!20](-9.328,1.021)--(-9.318,1.068)--(-9.372,1.081)--(-9.389,1.036)--cycle;
\filldraw[fill opacity=0.8,fill=gray!20](-9.328,1.021)--(-9.318,1.068)--(-9.372,1.081)--(-9.389,1.036)--cycle;
\filldraw[fill opacity=0.8,fill=gray!20,draw=none](-9.205,.843)--(-9.21,.843)--(-9.223,.845)--(-9.231,.848)--cycle;
\draw(-9.223,.845)--(-9.231,.848);
\filldraw[fill opacity=0.8,fill=gray!20,draw=none](-9.212,.854)--(-9.239,.871)--(-9.238,.853)--cycle;
\draw(-9.239,.871)--(-9.238,.853)--(-9.212,.854);
\filldraw[fill opacity=0.8,fill=gray!20,draw=none](-9.212,.854)--(-9.239,.871)--(-9.238,.853)--cycle;
\draw(-9.239,.871)--(-9.238,.853)--(-9.212,.854);
\filldraw[fill opacity=0.8,fill=gray!20,draw=none](-9.238,.853)--(-9.233,.849)--(-9.229,.847)--cycle;
\draw(-9.233,.849)--(-9.229,.847);
\filldraw[fill opacity=0.8,fill=gray!20,draw=none](-9.19,1.118)--(-9.191,1.118)--(-9.254,1.112)--(-9.243,1.107)--cycle;
\draw(-9.19,1.118)--(-9.191,1.118);
\draw(-9.254,1.112)--(-9.243,1.107);
\filldraw[fill opacity=0.8,fill=gray!20,draw=none](-9.26,1.063)--(-9.26,1.065)--(-9.298,1.082)--(-9.316,1.034)--(-9.287,1.021)--cycle;
\draw(-9.26,1.065)--(-9.298,1.082);
\draw(-9.316,1.034)--(-9.287,1.021);
\filldraw[fill opacity=0.8,fill=gray!20,draw=none](-9.127,1.106)--(-9.109,1.096)--(-9.123,1.118)--(-9.133,1.116)--cycle;
\draw(-9.109,1.096)--(-9.123,1.118)--(-9.133,1.116);
\filldraw[fill opacity=0.8,fill=gray!20,draw=none](-9.127,1.106)--(-9.109,1.096)--(-9.123,1.118)--(-9.133,1.116)--cycle;
\draw(-9.109,1.096)--(-9.123,1.118)--(-9.133,1.116);
\filldraw[fill opacity=0.8,fill=gray!20,draw=none](-9.096,1.088)--(-9.087,1.097)--(-9.118,1.116)--(-9.127,1.106)--cycle;
\draw(-9.118,1.116)--(-9.127,1.106);
\filldraw[fill opacity=0.8,fill=gray!20,draw=none](-9.096,1.088)--(-9.087,1.097)--(-9.118,1.116)--(-9.127,1.106)--cycle;
\draw(-9.118,1.116)--(-9.127,1.106);
\filldraw[fill opacity=0.8,fill=gray!20,draw=none](-9.096,1.088)--(-9.087,1.097)--(-9.118,1.116)--(-9.127,1.106)--cycle;
\draw(-9.118,1.116)--(-9.127,1.106);
\filldraw[fill opacity=0.8,fill=gray!20,draw=none](-9.087,1.097)--(-9.127,1.121)--(-9.123,1.119)--cycle;
\draw(-9.127,1.121)--(-9.123,1.119);
\filldraw[fill opacity=0.8,fill=gray!20,draw=none](-9.238,.853)--(-9.239,.871)--(-9.262,.885)--(-9.297,.887)--cycle;
\draw(-9.238,.853)--(-9.239,.871);
\draw(-9.262,.885)--(-9.297,.887);
\filldraw[fill opacity=0.8,fill=gray!20,draw=none](-9.238,.853)--(-9.239,.871)--(-9.262,.885)--(-9.297,.887)--cycle;
\draw(-9.238,.853)--(-9.239,.871);
\draw(-9.262,.885)--(-9.297,.887);
\filldraw[fill opacity=0.8,fill=gray!20,draw=none](-9.256,.865)--(-9.238,.853)--(-9.229,.847)--(-9.223,.845)--cycle;
\draw(-9.229,.847)--(-9.223,.845);
\filldraw[fill opacity=0.8,fill=gray!20,draw=none](-9.174,1.13)--(-9.198,1.136)--(-9.193,1.134)--cycle;
\draw(-9.198,1.136)--(-9.193,1.134);
\filldraw[fill opacity=0.8,fill=gray!20,draw=none](-9.118,1.116)--(-9.092,1.1)--(-9.123,1.118)--(-9.129,1.122)--cycle;
\filldraw[fill opacity=0.8,fill=gray!20,draw=none](-9.097,1.086)--(-9.096,1.088)--(-9.127,1.106)--(-9.128,1.105)--cycle;
\draw(-9.127,1.106)--(-9.128,1.105);
\filldraw[fill opacity=0.8,fill=gray!20,draw=none](-9.097,1.086)--(-9.096,1.088)--(-9.127,1.106)--(-9.128,1.105)--cycle;
\draw(-9.127,1.106)--(-9.128,1.105);
\filldraw[fill opacity=0.8,fill=gray!20,draw=none](-9.097,1.086)--(-9.096,1.088)--(-9.127,1.106)--(-9.128,1.105)--cycle;
\draw(-9.127,1.106)--(-9.128,1.105);
\filldraw[fill opacity=0.8,fill=gray!20,draw=none](-9.106,1.092)--(-9.112,1.096)--(-9.097,1.086)--cycle;
\filldraw[fill opacity=0.8,fill=gray!20,draw=none](-9.099,1.084)--(-9.097,1.086)--(-9.106,1.092)--cycle;
\filldraw[fill opacity=0.8,fill=gray!20,draw=none](-9.099,1.084)--(-9.097,1.086)--(-9.106,1.092)--cycle;
\filldraw[fill opacity=0.8,fill=gray!20,draw=none](-9.099,1.084)--(-9.097,1.086)--(-9.106,1.092)--cycle;
\filldraw[fill opacity=0.8,fill=gray!20,draw=none](-9.099,1.084)--(-9.106,1.092)--(-9.128,1.105)--(-9.207,1.016)--(-9.181,.991)--cycle;
\draw(-9.128,1.105)--(-9.207,1.016)--(-9.181,.991);
\filldraw[fill opacity=0.8,fill=gray!20,draw=none](-9.099,1.084)--(-9.106,1.092)--(-9.128,1.105)--(-9.207,1.016)--(-9.181,.991)--cycle;
\draw(-9.128,1.105)--(-9.207,1.016)--(-9.181,.991);
\filldraw[fill opacity=0.8,fill=gray!20,draw=none](-9.099,1.084)--(-9.106,1.092)--(-9.128,1.105)--(-9.207,1.016)--(-9.181,.991)--cycle;
\draw(-9.128,1.105)--(-9.207,1.016)--(-9.181,.991);
\filldraw[fill opacity=0.8,fill=gray!20,draw=none](-9.099,1.084)--(-9.106,1.092)--(-9.108,1.093)--(-9.106,1.091)--cycle;
\draw(-9.108,1.093)--(-9.106,1.091);
\filldraw[fill opacity=0.8,fill=gray!20,draw=none](-9.099,1.084)--(-9.106,1.092)--(-9.108,1.093)--(-9.106,1.091)--cycle;
\draw(-9.108,1.093)--(-9.106,1.091);
\filldraw[fill opacity=0.8,fill=gray!20,draw=none](-9.166,1.14)--(-9.17,1.148)--(-9.175,1.147)--cycle;
\draw(-9.17,1.148)--(-9.175,1.147);
\filldraw[fill opacity=0.8,fill=gray!20,draw=none](-9.166,1.14)--(-9.17,1.148)--(-9.175,1.147)--cycle;
\draw(-9.17,1.148)--(-9.175,1.147);
\filldraw[fill opacity=0.8,fill=gray!20,draw=none](-9.238,.853)--(-9.256,.865)--(-9.266,.871)--(-9.278,.876)--cycle;
\draw(-9.266,.871)--(-9.278,.876);
\filldraw[fill opacity=0.8,fill=gray!20,draw=none](-9.123,1.118)--(-9.133,1.124)--(-9.129,1.122)--cycle;
\draw(-9.133,1.124)--(-9.129,1.122);
\filldraw[fill opacity=0.8,fill=gray!20,draw=none](-9.146,1.125)--(-9.129,1.122)--(-9.133,1.124)--cycle;
\draw(-9.129,1.122)--(-9.133,1.124);
\filldraw[fill opacity=0.8,fill=gray!20,draw=none](-9.116,1.099)--(-9.112,1.096)--(-9.106,1.092)--(-9.128,1.105)--cycle;
\filldraw[fill opacity=0.8,fill=gray!20,draw=none](-9.106,1.092)--(-9.109,1.096)--(-9.108,1.093)--cycle;
\draw(-9.109,1.096)--(-9.108,1.093);
\filldraw[fill opacity=0.8,fill=gray!20,draw=none](-9.106,1.092)--(-9.109,1.096)--(-9.108,1.093)--cycle;
\draw(-9.109,1.096)--(-9.108,1.093);
\filldraw[fill opacity=0.8,fill=gray!20,draw=none](-9.191,1.152)--(-9.187,1.144)--(-9.175,1.147)--cycle;
\draw(-9.187,1.144)--(-9.175,1.147);
\filldraw[fill opacity=0.8,fill=gray!20,draw=none](-9.234,.868)--(-9.239,.871)--(-9.221,.86)--(-9.218,.858)--cycle;
\draw(-9.221,.86)--(-9.218,.858);
\filldraw[fill opacity=0.8,fill=gray!20,draw=none](-9.126,1.104)--(-9.108,1.093)--(-9.109,1.096)--(-9.127,1.106)--cycle;
\draw(-9.108,1.093)--(-9.109,1.096);
\filldraw[fill opacity=0.8,fill=gray!20,draw=none](-9.126,1.104)--(-9.108,1.093)--(-9.109,1.096)--(-9.127,1.106)--cycle;
\draw(-9.108,1.093)--(-9.109,1.096);
\filldraw[fill opacity=0.8,fill=gray!20,draw=none](-9.128,1.107)--(-9.141,1.129)--(-9.164,1.142)--(-9.186,1.117)--cycle;
\draw(-9.164,1.142)--(-9.186,1.117);
\filldraw[fill opacity=0.8,fill=gray!20,draw=none](-9.128,1.107)--(-9.141,1.129)--(-9.164,1.142)--(-9.186,1.117)--cycle;
\draw(-9.164,1.142)--(-9.186,1.117);
\filldraw[fill opacity=0.8,fill=gray!20,draw=none](-9.128,1.107)--(-9.141,1.129)--(-9.164,1.142)--(-9.186,1.117)--cycle;
\draw(-9.164,1.142)--(-9.186,1.117);
\filldraw[fill opacity=0.8,fill=gray!20,draw=none](-9.158,1.126)--(-9.174,1.13)--(-9.193,1.134)--(-9.183,1.129)--cycle;
\draw(-9.193,1.134)--(-9.183,1.129);
\filldraw[fill opacity=0.8,fill=gray!20,draw=none](-9.166,1.14)--(-9.175,1.147)--(-9.19,1.144)--(-9.183,1.129)--cycle;
\draw(-9.175,1.147)--(-9.19,1.144)--(-9.183,1.129);
\filldraw[fill opacity=0.8,fill=gray!20,draw=none](-9.166,1.14)--(-9.175,1.147)--(-9.19,1.144)--(-9.183,1.129)--cycle;
\draw(-9.175,1.147)--(-9.19,1.144)--(-9.183,1.129);
\filldraw[fill opacity=0.8,fill=gray!20,draw=none](-9.031,1.297)--(-9.022,1.354)--(-9.053,1.356)--(-9.057,1.299)--cycle;
\draw(-9.022,1.354)--(-9.053,1.356)--(-9.057,1.299)--(-9.031,1.297);
\filldraw[fill opacity=0.8,fill=gray!20,draw=none](-9.022,1.354)--(-8.995,1.408)--(-9.041,1.412)--(-9.053,1.356)--cycle;
\draw(-8.995,1.408)--(-9.041,1.412)--(-9.053,1.356)--(-9.022,1.354);
\filldraw[fill opacity=0.8,fill=gray!20,draw=none](-9.128,1.107)--(-9.186,1.117)--(-9.246,1.049)--(-9.207,1.016)--(-9.127,1.106)--cycle;
\draw(-9.186,1.117)--(-9.246,1.049)--(-9.207,1.016)--(-9.127,1.106);
\filldraw[fill opacity=0.8,fill=gray!20,draw=none](-9.128,1.107)--(-9.186,1.117)--(-9.246,1.049)--(-9.207,1.016)--(-9.127,1.106)--cycle;
\draw(-9.186,1.117)--(-9.246,1.049)--(-9.207,1.016)--(-9.127,1.106);
\filldraw[fill opacity=0.8,fill=gray!20,draw=none](-9.128,1.107)--(-9.186,1.117)--(-9.246,1.049)--(-9.207,1.016)--(-9.127,1.106)--cycle;
\draw(-9.186,1.117)--(-9.246,1.049)--(-9.207,1.016)--(-9.127,1.106);
\filldraw[fill opacity=0.8,fill=gray!20,draw=none](-9.106,1.091)--(-9.108,1.093)--(-9.131,1.107)--(-9.164,1.11)--(-9.173,1.108)--cycle;
\draw(-9.106,1.091)--(-9.108,1.093);
\draw(-9.164,1.11)--(-9.173,1.108);
\filldraw[fill opacity=0.8,fill=gray!20,draw=none](-9.106,1.091)--(-9.108,1.093)--(-9.131,1.107)--(-9.164,1.11)--(-9.173,1.108)--cycle;
\draw(-9.106,1.091)--(-9.108,1.093);
\draw(-9.164,1.11)--(-9.173,1.108);
\filldraw[fill opacity=0.8,fill=gray!20,draw=none](-9.174,1.13)--(-9.158,1.126)--(-9.15,1.126)--(-9.151,1.126)--cycle;
\filldraw[fill opacity=0.8,fill=gray!20,draw=none](-9.158,1.126)--(-9.166,1.14)--(-9.183,1.129)--cycle;
\filldraw[fill opacity=0.8,fill=gray!20,draw=none](-9.146,1.125)--(-9.166,1.14)--(-9.158,1.126)--cycle;
\filldraw[fill opacity=0.8,fill=gray!20,draw=none](-9.146,1.125)--(-9.166,1.14)--(-9.158,1.126)--cycle;
\filldraw[fill opacity=0.8,fill=gray!20,draw=none](-9.168,1.11)--(-9.144,1.107)--(-9.191,1.118)--(-9.174,1.111)--cycle;
\draw(-9.191,1.118)--(-9.174,1.111);
\filldraw[fill opacity=0.8,fill=gray!20,draw=none](-9.186,1.117)--(-9.144,1.107)--(-9.128,1.106)--(-9.132,1.108)--cycle;
\draw(-9.128,1.106)--(-9.132,1.108);
\filldraw[fill opacity=0.8,fill=gray!20,draw=none](-9.164,1.11)--(-9.15,1.113)--(-9.158,1.126)--(-9.183,1.129)--(-9.174,1.111)--cycle;
\draw(-9.164,1.11)--(-9.15,1.113);
\draw(-9.183,1.129)--(-9.174,1.111);
\filldraw[fill opacity=0.8,fill=gray!20,draw=none](-9.164,1.11)--(-9.15,1.113)--(-9.166,1.14)--(-9.183,1.129)--(-9.174,1.111)--cycle;
\draw(-9.164,1.11)--(-9.15,1.113);
\draw(-9.183,1.129)--(-9.174,1.111);
\filldraw[fill opacity=0.8,fill=gray!20,draw=none](-9.141,1.129)--(-9.127,1.106)--(-9.118,1.116)--cycle;
\draw(-9.127,1.106)--(-9.118,1.116);
\filldraw[fill opacity=0.8,fill=gray!20,draw=none](-9.141,1.129)--(-9.127,1.106)--(-9.118,1.116)--cycle;
\draw(-9.127,1.106)--(-9.118,1.116);
\filldraw[fill opacity=0.8,fill=gray!20,draw=none](-9.141,1.129)--(-9.127,1.106)--(-9.118,1.116)--cycle;
\draw(-9.127,1.106)--(-9.118,1.116);
\filldraw[fill opacity=0.8,fill=gray!20,draw=none](-9.118,1.116)--(-9.129,1.122)--(-9.127,1.121)--cycle;
\draw(-9.129,1.122)--(-9.127,1.121);
\filldraw[fill opacity=0.8,fill=gray!20,draw=none](-9.183,1.129)--(-9.193,1.134)--(-9.252,1.121)--(-9.233,1.113)--cycle;
\draw(-9.183,1.129)--(-9.193,1.134);
\draw(-9.252,1.121)--(-9.233,1.113);
\filldraw[fill opacity=0.8,fill=gray!20,draw=none](-9.146,1.125)--(-9.158,1.126)--(-9.15,1.113)--(-9.133,1.116)--cycle;
\draw(-9.15,1.113)--(-9.133,1.116);
\filldraw[fill opacity=0.8,fill=gray!20,draw=none](-9.146,1.125)--(-9.158,1.126)--(-9.15,1.113)--(-9.133,1.116)--cycle;
\draw(-9.15,1.113)--(-9.133,1.116);
\filldraw[fill opacity=0.8,fill=gray!20,draw=none](-9.151,1.126)--(-9.145,1.123)--(-9.132,1.12)--(-9.123,1.119)--(-9.129,1.122)--cycle;
\draw(-9.123,1.119)--(-9.129,1.122);
\filldraw[fill opacity=0.8,fill=gray!20,draw=none](-9.127,1.106)--(-9.133,1.116)--(-9.141,1.114)--cycle;
\draw(-9.133,1.116)--(-9.141,1.114);
\filldraw[fill opacity=0.8,fill=gray!20,draw=none](-9.127,1.106)--(-9.133,1.116)--(-9.141,1.114)--cycle;
\draw(-9.133,1.116)--(-9.141,1.114);
\filldraw[fill opacity=0.8,fill=gray!20,draw=none](-9.158,1.126)--(-9.145,1.123)--(-9.15,1.126)--cycle;
\filldraw[fill opacity=0.8,fill=gray!20,draw=none](-9.185,1.108)--(-9.174,1.111)--(-9.19,1.118)--(-9.243,1.107)--(-9.221,1.097)--cycle;
\draw(-9.174,1.111)--(-9.19,1.118);
\draw(-9.243,1.107)--(-9.221,1.097);
\filldraw[fill opacity=0.8,fill=gray!20,draw=none](-9.128,1.105)--(-9.132,1.108)--(-9.131,1.107)--cycle;
\draw(-9.132,1.108)--(-9.131,1.107);
\filldraw[fill opacity=0.8,fill=gray!20,draw=none](-9.126,1.104)--(-9.127,1.106)--(-9.141,1.114)--(-9.143,1.114)--cycle;
\draw(-9.141,1.114)--(-9.143,1.114);
\filldraw[fill opacity=0.8,fill=gray!20,draw=none](-9.126,1.104)--(-9.127,1.106)--(-9.141,1.114)--(-9.143,1.114)--cycle;
\draw(-9.141,1.114)--(-9.143,1.114);
\filldraw[fill opacity=0.8,fill=gray!20,draw=none](-9.116,1.099)--(-9.128,1.105)--(-9.131,1.107)--(-9.128,1.106)--cycle;
\draw(-9.131,1.107)--(-9.128,1.106);
\filldraw[fill opacity=0.8,fill=gray!20,draw=none](-9.283,.886)--(-9.312,.928)--(-9.328,.929)--(-9.318,.889)--cycle;
\draw(-9.312,.928)--(-9.328,.929)--(-9.318,.889)--(-9.283,.886);
\filldraw[fill opacity=0.8,fill=gray!20,draw=none](-9.268,1.109)--(-9.246,1.1)--(-9.238,1.105)--(-9.233,1.113)--(-9.252,1.121)--cycle;
\draw(-9.233,1.113)--(-9.252,1.121);
\filldraw[fill opacity=0.8,fill=gray!20](-9.238,1.105)--(-9.236,1.142)--(-9.281,1.145)--(-9.302,1.11)--cycle;
\filldraw[fill opacity=0.8,fill=gray!20,draw=none](-9.283,.886)--(-9.312,.928)--(-9.328,.929)--(-9.318,.889)--cycle;
\draw(-9.312,.928)--(-9.328,.929)--(-9.318,.889)--(-9.283,.886);
\filldraw[fill opacity=0.8,fill=gray!20](-9.238,1.105)--(-9.236,1.142)--(-9.281,1.145)--(-9.302,1.11)--cycle;
\filldraw[fill opacity=0.8,fill=gray!20,draw=none](-9.283,.886)--(-9.278,.876)--(-9.276,.875)--cycle;
\draw(-9.278,.876)--(-9.276,.875);
\filldraw[fill opacity=0.8,fill=gray!20,draw=none](-9.131,1.107)--(-9.143,1.114)--(-9.164,1.11)--cycle;
\draw(-9.143,1.114)--(-9.164,1.11);
\filldraw[fill opacity=0.8,fill=gray!20,draw=none](-9.131,1.107)--(-9.143,1.114)--(-9.164,1.11)--cycle;
\draw(-9.143,1.114)--(-9.164,1.11);
\filldraw[fill opacity=0.8,fill=gray!20,draw=none](-9.172,1.125)--(-9.14,1.121)--(-9.145,1.123)--(-9.158,1.126)--(-9.183,1.129)--(-9.175,1.126)--cycle;
\draw(-9.183,1.129)--(-9.175,1.126);
\filldraw[fill opacity=0.8,fill=gray!20](-9.173,1.108)--(-9.19,1.144)--(-9.236,1.142)--(-9.238,1.105)--cycle;
\filldraw[fill opacity=0.8,fill=gray!20](-9.173,1.108)--(-9.19,1.144)--(-9.236,1.142)--(-9.238,1.105)--cycle;
\filldraw[fill opacity=0.8,fill=gray!20,draw=none](-9.283,.886)--(-9.26,.885)--(-9.289,.926)--(-9.312,.928)--cycle;
\draw(-9.283,.886)--(-9.26,.885);
\draw(-9.289,.926)--(-9.312,.928);
\filldraw[fill opacity=0.8,fill=gray!20,draw=none](-9.283,.886)--(-9.26,.885)--(-9.289,.926)--(-9.312,.928)--cycle;
\draw(-9.283,.886)--(-9.26,.885);
\draw(-9.289,.926)--(-9.312,.928);
\filldraw[fill opacity=0.8,fill=gray!20,draw=none](-9.294,.914)--(-9.299,.916)--(-9.283,.886)--(-9.276,.875)--(-9.266,.871)--cycle;
\draw(-9.294,.914)--(-9.299,.916);
\draw(-9.276,.875)--(-9.266,.871);
\filldraw[fill opacity=0.8,fill=gray!20,draw=none](-9.145,1.123)--(-9.14,1.121)--(-9.132,1.12)--cycle;
\filldraw[fill opacity=0.8,fill=gray!20,draw=none](-9.283,1.065)--(-9.278,1.078)--(-9.297,1.087)--(-9.322,1.036)--(-9.306,1.029)--cycle;
\draw(-9.278,1.078)--(-9.297,1.087);
\draw(-9.322,1.036)--(-9.306,1.029);
\filldraw[fill opacity=0.8,fill=gray!20,draw=none](-9.234,.868)--(-9.256,.882)--(-9.26,.883)--cycle;
\draw(-9.256,.882)--(-9.26,.883);
\filldraw[fill opacity=0.8,fill=gray!20,draw=none](-9.239,.871)--(-9.24,.873)--(-9.251,.884)--(-9.262,.885)--cycle;
\draw(-9.239,.871)--(-9.24,.873);
\draw(-9.251,.884)--(-9.262,.885);
\filldraw[fill opacity=0.8,fill=gray!20,draw=none](-9.239,.871)--(-9.24,.873)--(-9.251,.884)--(-9.262,.885)--cycle;
\draw(-9.239,.871)--(-9.24,.873);
\draw(-9.251,.884)--(-9.262,.885);
\filldraw[fill opacity=0.8,fill=gray!20,draw=none](-9.283,.886)--(-9.299,.916)--(-9.306,.92)--cycle;
\draw(-9.299,.916)--(-9.306,.92);
\filldraw[fill opacity=0.8,fill=gray!20,draw=none](-9.173,1.125)--(-9.183,1.129)--(-9.233,1.113)--(-9.223,1.108)--cycle;
\draw(-9.173,1.125)--(-9.183,1.129);
\draw(-9.233,1.113)--(-9.223,1.108);
\filldraw[fill opacity=0.8,fill=gray!20,draw=none](-9.26,.885)--(-9.251,.884)--(-9.273,.925)--(-9.289,.926)--cycle;
\draw(-9.26,.885)--(-9.251,.884);
\draw(-9.273,.925)--(-9.289,.926);
\filldraw[fill opacity=0.8,fill=gray!20,draw=none](-9.26,.885)--(-9.251,.884)--(-9.273,.925)--(-9.289,.926)--cycle;
\draw(-9.26,.885)--(-9.251,.884);
\draw(-9.273,.925)--(-9.289,.926);
\filldraw[fill opacity=0.8,fill=gray!20,draw=none](-9.268,.9)--(-9.26,.883)--(-9.256,.882)--cycle;
\draw(-9.26,.883)--(-9.256,.882);
\filldraw[fill opacity=0.8,fill=gray!20,draw=none](-9.26,.885)--(-9.268,.9)--(-9.282,.921)--(-9.287,.923)--cycle;
\draw(-9.282,.921)--(-9.287,.923);
\filldraw[fill opacity=0.8,fill=gray!20,draw=none](-9.164,1.11)--(-9.174,1.111)--(-9.173,1.108)--cycle;
\draw(-9.174,1.111)--(-9.173,1.108)--(-9.164,1.11);
\filldraw[fill opacity=0.8,fill=gray!20,draw=none](-9.164,1.11)--(-9.174,1.111)--(-9.173,1.108)--cycle;
\draw(-9.174,1.111)--(-9.173,1.108)--(-9.164,1.11);
\filldraw[fill opacity=0.8,fill=gray!20,draw=none](-9.297,1.066)--(-9.238,1.105)--(-9.302,1.11)--(-9.318,1.068)--cycle;
\draw(-9.238,1.105)--(-9.302,1.11)--(-9.318,1.068)--(-9.297,1.066);
\filldraw[fill opacity=0.8,fill=gray!20,draw=none](-9.297,1.066)--(-9.238,1.105)--(-9.302,1.11)--(-9.318,1.068)--cycle;
\draw(-9.238,1.105)--(-9.302,1.11)--(-9.318,1.068)--(-9.297,1.066);
\filldraw[fill opacity=0.8,fill=gray!20,draw=none](-9.306,.927)--(-9.315,.973)--(-9.331,.974)--(-9.328,.929)--cycle;
\draw(-9.315,.973)--(-9.331,.974)--(-9.328,.929)--(-9.306,.927);
\filldraw[fill opacity=0.8,fill=gray!20,draw=none](-9.306,.927)--(-9.315,.973)--(-9.331,.974)--(-9.328,.929)--cycle;
\draw(-9.315,.973)--(-9.331,.974)--(-9.328,.929)--(-9.306,.927);
\filldraw[fill opacity=0.8,fill=gray!20,draw=none](-9.306,.927)--(-9.306,.92)--(-9.305,.919)--cycle;
\draw(-9.306,.92)--(-9.305,.919);
\filldraw[fill opacity=0.8,fill=gray!20,draw=none](-9.306,.927)--(-9.287,.926)--(-9.296,.971)--(-9.315,.973)--cycle;
\draw(-9.306,.927)--(-9.287,.926);
\draw(-9.296,.971)--(-9.315,.973);
\filldraw[fill opacity=0.8,fill=gray!20,draw=none](-9.306,.927)--(-9.287,.926)--(-9.296,.971)--(-9.315,.973)--cycle;
\draw(-9.306,.927)--(-9.287,.926);
\draw(-9.296,.971)--(-9.315,.973);
\filldraw[fill opacity=0.8,fill=gray!20,draw=none](-9.296,.926)--(-9.3,.949)--(-9.307,.952)--(-9.306,.927)--(-9.305,.919)--(-9.299,.916)--cycle;
\draw(-9.305,.919)--(-9.299,.916);
\filldraw[fill opacity=0.8,fill=gray!20,draw=none](-9.168,1.11)--(-9.174,1.111)--(-9.173,1.111)--cycle;
\draw(-9.174,1.111)--(-9.173,1.111);
\filldraw[fill opacity=0.8,fill=gray!20,draw=none](-9.174,1.111)--(-9.174,1.111)--(-9.185,1.108)--cycle;
\draw(-9.174,1.111)--(-9.174,1.111);
\filldraw[fill opacity=0.8,fill=gray!20,draw=none](-9.173,1.108)--(-9.212,1.106)--(-9.239,1.079)--(-9.24,1.076)--cycle;
\draw(-9.173,1.108)--(-9.212,1.106);
\draw(-9.239,1.079)--(-9.24,1.076);
\filldraw[fill opacity=0.8,fill=gray!20,draw=none](-9.173,1.108)--(-9.212,1.106)--(-9.239,1.079)--(-9.24,1.076)--cycle;
\draw(-9.173,1.108)--(-9.212,1.106);
\draw(-9.239,1.079)--(-9.24,1.076);
\filldraw[fill opacity=0.8,fill=gray!20,draw=none](-9.173,1.111)--(-9.174,1.111)--(-9.185,1.108)--(-9.219,1.096)--(-9.218,1.096)--cycle;
\draw(-9.173,1.111)--(-9.174,1.111);
\draw(-9.219,1.096)--(-9.218,1.096);
\filldraw[fill opacity=0.8,fill=gray!20,draw=none](-9.172,1.125)--(-9.175,1.126)--(-9.173,1.125)--cycle;
\draw(-9.175,1.126)--(-9.173,1.125);
\filldraw[fill opacity=0.8,fill=gray!20,draw=none](-9.306,.927)--(-9.307,.952)--(-9.312,.954)--cycle;
\filldraw[fill opacity=0.8,fill=gray!20,draw=none](-9.185,1.108)--(-9.221,1.097)--(-9.219,1.096)--cycle;
\draw(-9.221,1.097)--(-9.219,1.096);
\filldraw[fill opacity=0.8,fill=gray!20,draw=none](-9.287,.926)--(-9.273,.925)--(-9.281,.97)--(-9.296,.971)--cycle;
\draw(-9.287,.926)--(-9.273,.925);
\draw(-9.281,.97)--(-9.296,.971);
\filldraw[fill opacity=0.8,fill=gray!20,draw=none](-9.287,.926)--(-9.273,.925)--(-9.281,.97)--(-9.296,.971)--cycle;
\draw(-9.287,.926)--(-9.273,.925);
\draw(-9.281,.97)--(-9.296,.971);
\filldraw[fill opacity=0.8,fill=gray!20,draw=none](-9.287,.95)--(-9.287,.923)--(-9.282,.921)--cycle;
\draw(-9.287,.923)--(-9.282,.921);
\filldraw[fill opacity=0.8,fill=gray!20,draw=none](-9.287,.926)--(-9.287,.95)--(-9.291,.969)--(-9.296,.971)--cycle;
\draw(-9.291,.969)--(-9.296,.971);
\filldraw[fill opacity=0.8,fill=gray!20,draw=none](-9.212,1.106)--(-9.238,1.105)--(-9.239,1.079)--cycle;
\draw(-9.212,1.106)--(-9.238,1.105)--(-9.239,1.079);
\filldraw[fill opacity=0.8,fill=gray!20,draw=none](-9.212,1.106)--(-9.238,1.105)--(-9.239,1.079)--cycle;
\draw(-9.212,1.106)--(-9.238,1.105)--(-9.239,1.079);
\filldraw[fill opacity=0.8,fill=gray!20,draw=none](-9.238,1.105)--(-9.229,1.111)--(-9.233,1.113)--cycle;
\draw(-9.229,1.111)--(-9.233,1.113);
\filldraw[fill opacity=0.8,fill=gray!20,draw=none](-9.312,.954)--(-9.3,.949)--(-9.304,.968)--(-9.315,.973)--cycle;
\draw(-9.304,.968)--(-9.315,.973);
\filldraw[fill opacity=0.8,fill=gray!20,draw=none](-9.306,1.02)--(-9.306,1.067)--(-9.318,1.068)--(-9.328,1.021)--cycle;
\draw(-9.306,1.067)--(-9.318,1.068)--(-9.328,1.021)--(-9.306,1.02);
\filldraw[fill opacity=0.8,fill=gray!20,draw=none](-9.281,.97)--(-9.273,1.017)--(-9.328,1.021)--(-9.331,.974)--cycle;
\draw(-9.273,1.017)--(-9.328,1.021)--(-9.331,.974)--(-9.281,.97);
\filldraw[fill opacity=0.8,fill=gray!20,draw=none](-9.281,.97)--(-9.273,1.017)--(-9.328,1.021)--(-9.331,.974)--cycle;
\draw(-9.273,1.017)--(-9.328,1.021)--(-9.331,.974)--(-9.281,.97);
\filldraw[fill opacity=0.8,fill=gray!20,draw=none](-9.306,1.02)--(-9.306,1.067)--(-9.318,1.068)--(-9.328,1.021)--cycle;
\draw(-9.306,1.067)--(-9.318,1.068)--(-9.328,1.021)--(-9.306,1.02);
\filldraw[fill opacity=0.8,fill=gray!20,draw=none](-9.296,.926)--(-9.299,.916)--(-9.294,.914)--cycle;
\draw(-9.299,.916)--(-9.294,.914);
\filldraw[fill opacity=0.8,fill=gray!20,draw=none](-9.306,1.029)--(-9.283,1.065)--(-9.306,1.067)--cycle;
\draw(-9.283,1.065)--(-9.306,1.067);
\filldraw[fill opacity=0.8,fill=gray!20,draw=none](-9.306,1.029)--(-9.283,1.065)--(-9.306,1.067)--cycle;
\draw(-9.283,1.065)--(-9.306,1.067);
\filldraw[fill opacity=0.8,fill=gray!20,draw=none](-9.297,1.066)--(-9.262,1.064)--(-9.239,1.079)--(-9.238,1.105)--cycle;
\draw(-9.297,1.066)--(-9.262,1.064);
\draw(-9.239,1.079)--(-9.238,1.105);
\filldraw[fill opacity=0.8,fill=gray!20,draw=none](-9.297,1.066)--(-9.262,1.064)--(-9.239,1.079)--(-9.238,1.105)--cycle;
\draw(-9.297,1.066)--(-9.262,1.064);
\draw(-9.239,1.079)--(-9.238,1.105);
\filldraw[fill opacity=0.8,fill=gray!20,draw=none](-9.238,1.105)--(-9.243,1.099)--(-9.238,1.096)--(-9.223,1.108)--(-9.229,1.111)--cycle;
\draw(-9.223,1.108)--(-9.229,1.111);
\filldraw[fill opacity=0.8,fill=gray!20,draw=none](-9.246,1.1)--(-9.243,1.099)--(-9.238,1.105)--cycle;
\filldraw[fill opacity=0.8,fill=gray!20,draw=none](-9.307,.995)--(-9.306,1.02)--(-9.315,.973)--cycle;
\filldraw[fill opacity=0.8,fill=gray!20,draw=none](-9.286,1.061)--(-9.306,1.029)--(-9.306,1.02)--(-9.287,1.018)--cycle;
\draw(-9.306,1.02)--(-9.287,1.018);
\filldraw[fill opacity=0.8,fill=gray!20,draw=none](-9.307,.995)--(-9.315,.973)--(-9.307,.969)--cycle;
\draw(-9.315,.973)--(-9.307,.969);
\filldraw[fill opacity=0.8,fill=gray!20,draw=none](-9.238,1.096)--(-9.246,1.1)--(-9.278,1.078)--(-9.266,1.073)--cycle;
\draw(-9.278,1.078)--(-9.266,1.073);
\filldraw[fill opacity=0.8,fill=gray!20,draw=none](-9.239,1.079)--(-9.234,1.082)--(-9.218,1.096)--(-9.221,1.097)--cycle;
\draw(-9.218,1.096)--(-9.221,1.097);
\filldraw[fill opacity=0.8,fill=gray!20,draw=none](-9.286,1.061)--(-9.306,1.029)--(-9.306,1.02)--(-9.287,1.018)--cycle;
\draw(-9.306,1.02)--(-9.287,1.018);
\filldraw[fill opacity=0.8,fill=gray!20,draw=none](-9.276,1.077)--(-9.278,1.078)--(-9.283,1.065)--cycle;
\draw(-9.276,1.077)--(-9.278,1.078);
\filldraw[fill opacity=0.8,fill=gray!20,draw=none](-9.286,1.061)--(-9.286,1.051)--(-9.262,1.064)--(-9.283,1.065)--cycle;
\draw(-9.262,1.064)--(-9.283,1.065);
\filldraw[fill opacity=0.8,fill=gray!20,draw=none](-9.292,1.044)--(-9.296,1.045)--(-9.306,1.029)--(-9.299,1.026)--cycle;
\draw(-9.306,1.029)--(-9.299,1.026);
\filldraw[fill opacity=0.8,fill=gray!20,draw=none](-9.306,1.02)--(-9.305,1.028)--(-9.306,1.029)--cycle;
\draw(-9.305,1.028)--(-9.306,1.029);
\filldraw[fill opacity=0.8,fill=gray!20,draw=none](-9.286,1.061)--(-9.286,1.051)--(-9.262,1.064)--(-9.283,1.065)--cycle;
\draw(-9.262,1.064)--(-9.283,1.065);
\filldraw[fill opacity=0.8,fill=gray!20,draw=none](-9.292,1.044)--(-9.283,1.065)--(-9.296,1.045)--cycle;
\filldraw[fill opacity=0.8,fill=gray!20,draw=none](-9.286,1.051)--(-9.287,1.021)--(-9.26,1.063)--(-9.262,1.064)--cycle;
\draw(-9.26,1.063)--(-9.262,1.064);
\filldraw[fill opacity=0.8,fill=gray!20,draw=none](-9.286,1.051)--(-9.287,1.021)--(-9.26,1.063)--(-9.262,1.064)--cycle;
\draw(-9.26,1.063)--(-9.262,1.064);
\filldraw[fill opacity=0.8,fill=gray!20,draw=none](-9.284,1.04)--(-9.266,1.073)--(-9.276,1.077)--(-9.283,1.065)--(-9.292,1.044)--cycle;
\draw(-9.266,1.073)--(-9.276,1.077);
\filldraw[fill opacity=0.8,fill=gray!20,draw=none](-9.287,1.021)--(-9.287,1.018)--(-9.273,1.017)--(-9.251,1.063)--(-9.26,1.063)--cycle;
\draw(-9.287,1.018)--(-9.273,1.017);
\draw(-9.251,1.063)--(-9.26,1.063);
\filldraw[fill opacity=0.8,fill=gray!20,draw=none](-9.287,1.021)--(-9.287,1.018)--(-9.273,1.017)--(-9.251,1.063)--(-9.26,1.063)--cycle;
\draw(-9.287,1.018)--(-9.273,1.017);
\draw(-9.251,1.063)--(-9.26,1.063);
\filldraw[fill opacity=0.8,fill=gray!20,draw=none](-9.286,.997)--(-9.282,1.019)--(-9.286,1.021)--(-9.296,.971)--cycle;
\draw(-9.282,1.019)--(-9.286,1.021);
\filldraw[fill opacity=0.8,fill=gray!20,draw=none](-9.286,.997)--(-9.296,.971)--(-9.291,.969)--cycle;
\draw(-9.296,.971)--(-9.291,.969);
\filldraw[fill opacity=0.8,fill=gray!20,draw=none](-9.307,.995)--(-9.296,1.024)--(-9.305,1.028)--(-9.306,1.02)--cycle;
\draw(-9.296,1.024)--(-9.305,1.028);
\filldraw[fill opacity=0.8,fill=gray!20,draw=none](-9.294,1.023)--(-9.296,1.024)--(-9.307,.995)--(-9.307,.969)--(-9.304,.968)--cycle;
\draw(-9.294,1.023)--(-9.296,1.024);
\draw(-9.307,.969)--(-9.304,.968);
\filldraw[fill opacity=0.8,fill=gray!20,draw=none](-9.262,1.064)--(-9.251,1.063)--(-9.24,1.076)--(-9.239,1.079)--cycle;
\draw(-9.262,1.064)--(-9.251,1.063);
\draw(-9.24,1.076)--(-9.239,1.079);
\filldraw[fill opacity=0.8,fill=gray!20,draw=none](-9.262,1.064)--(-9.251,1.063)--(-9.24,1.076)--(-9.239,1.079)--cycle;
\draw(-9.262,1.064)--(-9.251,1.063);
\draw(-9.24,1.076)--(-9.239,1.079);
\filldraw[fill opacity=0.8,fill=gray!20,draw=none](-9.234,1.082)--(-9.26,1.065)--(-9.256,1.064)--cycle;
\draw(-9.26,1.065)--(-9.256,1.064);
\filldraw[fill opacity=0.8,fill=gray!20,draw=none](-9.268,1.043)--(-9.256,1.064)--(-9.26,1.065)--cycle;
\draw(-9.256,1.064)--(-9.26,1.065);
\filldraw[fill opacity=0.8,fill=gray!20,draw=none](-9.268,1.043)--(-9.26,1.063)--(-9.287,1.021)--(-9.282,1.019)--cycle;
\draw(-9.287,1.021)--(-9.282,1.019);
\filldraw[fill opacity=0.8,fill=gray!20,draw=none](-9.287,1.018)--(-9.286,1.021)--(-9.287,1.021)--cycle;
\draw(-9.286,1.021)--(-9.287,1.021);
\filldraw[fill opacity=0.8,fill=gray!20,draw=none](-9.284,1.04)--(-9.292,1.044)--(-9.299,1.026)--(-9.294,1.023)--cycle;
\draw(-9.299,1.026)--(-9.294,1.023);
\filldraw[fill opacity=0.5,fill=gray!20](-7.696,1.143)--(-7.746,.721)--(-7.894,.316)--(-8.129,-.044)--(-8.434,-.333)--(-8.791,-.533)--(-9.173,-.629)--(-9.555,-.616)--(-9.911,-.494)--(-10.217,-.271)--(-10.452,.037)--(-10.599,.409)--(-10.65,.821)--(-10.599,1.243)--(-10.452,1.648)--(-10.217,2.007)--(-9.911,2.297)--(-9.555,2.497)--(-9.173,2.593)--(-8.791,2.58)--(-8.434,2.458)--(-8.129,2.235)--(-7.894,1.927)--(-7.746,1.555)--cycle;
\filldraw[fill opacity=0.5,fill=gray!20](-7.746,.721)--(-7.573,.645)--(-7.721,.24)--(-7.894,.316)--cycle;
\filldraw[fill opacity=0.5,fill=gray!20](-7.894,.316)--(-7.721,.24)--(-7.956,-.119)--(-8.129,-.044)--cycle;
\filldraw[fill opacity=0.8,fill=gray!20,draw=none](-8.862,3.045)--(-8.878,3.03)--(-8.868,3.054)--cycle;
\draw(-8.878,3.03)--(-8.868,3.054);
\filldraw[fill opacity=0.8,fill=gray!20](-8.172,4.18)--(-8.124,4.207)--(-8.113,4.194)--(-8.166,4.174)--cycle;
\filldraw[fill opacity=0.5,fill=gray!20](-7.819,-.28)--(-7.835,-.296)--(-8.207,-.648)--(-8.18,-.621)--cycle;
\filldraw[fill opacity=0.8,fill=gray!20](-8.682,2.949)--(-8.732,2.951)--(-8.738,2.957)--(-8.682,2.949)--cycle;
\filldraw[fill opacity=0.8,fill=gray!20](-8.732,2.951)--(-8.78,2.967)--(-8.791,2.979)--(-8.738,2.957)--cycle;
\filldraw[fill opacity=0.8,fill=gray!20](-8.837,3.317)--(-8.791,3.35)--(-8.773,3.362)--(-8.811,3.334)--cycle;
\filldraw[fill opacity=0.8,fill=gray!20](-8.54,3.268)--(-8.566,3.302)--(-8.552,3.288)--(-8.523,3.251)--cycle;
\filldraw[fill opacity=0.8,fill=gray!20,draw=none](-8.638,2.988)--(-8.634,2.99)--(-8.618,2.997)--(-8.611,2.998)--(-8.614,2.995)--(-8.636,2.987)--cycle;
\draw(-8.638,2.988)--(-8.634,2.99)--(-8.618,2.997);
\draw(-8.614,2.995)--(-8.636,2.987);
\filldraw[fill opacity=0.8,fill=gray!20](-8.682,2.949)--(-8.656,2.945)--(-8.685,2.944)--(-8.682,2.949)--cycle;
\filldraw[fill opacity=0.8,fill=gray!20](-8.682,2.985)--(-8.642,2.985)--(-8.66,2.982)--(-8.682,2.985)--cycle;
\filldraw[fill opacity=0.8,fill=gray!20](-8.682,2.985)--(-8.634,2.99)--(-8.642,2.985)--(-8.682,2.985)--cycle;
\filldraw[fill opacity=0.8,fill=gray!20,draw=none](-8.638,2.988)--(-8.636,2.987)--(-8.642,2.985)--cycle;
\draw(-8.636,2.987)--(-8.642,2.985)--(-8.638,2.988);
\filldraw[fill opacity=0.8,fill=gray!20,draw=none](-8.631,3.003)--(-8.614,3)--(-8.609,3.013)--(-8.63,3.012)--(-8.636,3.008)--cycle;
\draw(-8.614,3)--(-8.609,3.013);
\filldraw[fill opacity=0.8,fill=gray!20,draw=none](-8.614,3)--(-8.612,3.01)--(-8.625,3.01)--(-8.631,3.003)--cycle;
\draw(-8.625,3.01)--(-8.631,3.003);
\filldraw[fill opacity=0.8,fill=gray!20,draw=none](-8.609,3.013)--(-8.602,3.029)--(-8.63,3.012)--cycle;
\draw(-8.609,3.013)--(-8.602,3.029);
\filldraw[fill opacity=0.8,fill=gray!20,draw=none](-8.612,3.01)--(-8.612,3.014)--(-8.615,3.014)--(-8.625,3.01)--cycle;
\draw(-8.612,3.014)--(-8.615,3.014);
\filldraw[fill opacity=0.8,fill=gray!20,draw=none](-8.611,3.015)--(-8.615,3.014)--(-8.612,3.014)--cycle;
\draw(-8.615,3.014)--(-8.612,3.014);
\filldraw[fill opacity=0.8,fill=gray!20,draw=none](-8.617,3.007)--(-8.682,3.035)--(-8.637,3.049)--(-8.599,3.033)--cycle;
\draw(-8.617,3.007)--(-8.682,3.035)--(-8.637,3.049)--(-8.599,3.033);
\filldraw[fill opacity=0.8,fill=gray!20,draw=none](-8.614,2.995)--(-8.611,2.998)--(-8.602,3.019)--(-8.609,3.013)--(-8.616,2.997)--cycle;
\draw(-8.611,2.998)--(-8.602,3.019);
\draw(-8.609,3.013)--(-8.616,2.997);
\filldraw[fill opacity=0.8,fill=gray!20,draw=none](-8.602,3.019)--(-8.589,3.048)--(-8.602,3.029)--(-8.609,3.013)--cycle;
\draw(-8.602,3.019)--(-8.589,3.048);
\draw(-8.602,3.029)--(-8.609,3.013);
\filldraw[fill opacity=0.8,fill=gray!20,draw=none](-8.637,2.947)--(-8.634,2.949)--(-8.633,2.948)--(-8.635,2.944)--cycle;
\draw(-8.633,2.948)--(-8.635,2.944);
\filldraw[fill opacity=0.8,fill=gray!20,draw=none](-8.633,2.95)--(-8.631,2.951)--(-8.634,2.95)--cycle;
\draw(-8.631,2.951)--(-8.634,2.95)--(-8.633,2.95);
\filldraw[fill opacity=0.8,fill=gray!20,draw=none](-8.62,2.977)--(-8.611,2.998)--(-8.619,2.99)--(-8.622,2.982)--cycle;
\draw(-8.62,2.977)--(-8.611,2.998);
\draw(-8.619,2.99)--(-8.622,2.982);
\filldraw[fill opacity=0.8,fill=gray!20,draw=none](-8.611,2.998)--(-8.602,3)--(-8.605,2.998)--(-8.614,2.995)--cycle;
\draw(-8.602,3)--(-8.605,2.998)--(-8.614,2.995);
\filldraw[fill opacity=0.8,fill=gray!20,draw=none](-8.612,3.01)--(-8.615,2.996)--(-8.605,2.998)--(-8.588,3.01)--cycle;
\draw(-8.615,2.996)--(-8.605,2.998)--(-8.588,3.01);
\filldraw[fill opacity=0.8,fill=gray!20,draw=none](-8.632,2.951)--(-8.629,2.953)--(-8.625,2.956)--(-8.601,2.965)--(-8.592,2.964)--(-8.631,2.951)--cycle;
\draw(-8.629,2.953)--(-8.625,2.956)--(-8.601,2.965);
\draw(-8.592,2.964)--(-8.631,2.951);
\filldraw[fill opacity=0.8,fill=gray!20,draw=none](-8.632,2.951)--(-8.62,2.977)--(-8.622,2.982)--(-8.634,2.954)--cycle;
\draw(-8.632,2.951)--(-8.62,2.977);
\draw(-8.622,2.982)--(-8.634,2.954);
\filldraw[fill opacity=0.8,fill=gray!20,draw=none](-8.592,2.964)--(-8.59,2.965)--(-8.552,2.993)--(-8.612,2.982)--cycle;
\draw(-8.592,2.964)--(-8.59,2.965)--(-8.552,2.993)--(-8.612,2.982);
\filldraw[fill opacity=0.8,fill=gray!20,draw=none](-8.612,3.01)--(-8.588,3.01)--(-8.574,3.022)--(-8.612,3.014)--cycle;
\draw(-8.588,3.01)--(-8.574,3.022)--(-8.612,3.014);
\filldraw[fill opacity=0.8,fill=gray!20,draw=none](-8.582,3.026)--(-8.594,3.018)--(-8.574,3.022)--(-8.566,3.032)--cycle;
\draw(-8.594,3.018)--(-8.574,3.022)--(-8.566,3.032);
\filldraw[fill opacity=0.8,fill=gray!20,draw=none](-8.552,2.993)--(-8.547,3)--(-8.585,3.001)--(-8.612,2.982)--cycle;
\draw(-8.612,2.982)--(-8.552,2.993)--(-8.547,3);
\filldraw[fill opacity=0.8,fill=gray!20,draw=none](-8.547,3)--(-8.525,3.03)--(-8.562,3.017)--(-8.585,3.001)--cycle;
\draw(-8.547,3)--(-8.525,3.03);
\filldraw[fill opacity=0.8,fill=gray!20,draw=none](-8.582,3.026)--(-8.611,3.015)--(-8.612,3.014)--(-8.594,3.018)--cycle;
\draw(-8.612,3.014)--(-8.594,3.018);
\filldraw[fill opacity=0.8,fill=gray!20,draw=none](-8.585,3.001)--(-8.562,3.017)--(-8.603,3.001)--(-8.603,3.001)--cycle;
\filldraw[fill opacity=0.8,fill=gray!20,draw=none](-8.585,3.001)--(-8.603,3.001)--(-8.605,2.998)--(-8.612,2.982)--cycle;
\draw(-8.605,2.998)--(-8.612,2.982);
\filldraw[fill opacity=0.8,fill=gray!20,draw=none](-8.556,3.008)--(-8.518,2.963)--(-8.586,2.993)--cycle;
\draw(-8.518,2.963)--(-8.586,2.993);
\filldraw[fill opacity=0.8,fill=gray!20,draw=none](-8.518,2.963)--(-8.682,3.035)--(-8.637,3.049)--(-8.564,3.018)--cycle;
\draw(-8.518,2.963)--(-8.682,3.035)--(-8.637,3.049)--(-8.564,3.018);
\filldraw[fill opacity=0.8,fill=gray!20](-8.78,2.967)--(-8.82,2.997)--(-8.837,3.014)--(-8.791,2.979)--cycle;
\filldraw[fill opacity=0.8,fill=gray!20](-8.682,2.949)--(-8.713,2.946)--(-8.732,2.951)--(-8.682,2.949)--cycle;
\filldraw[fill opacity=0.8,fill=gray!20](-8.682,2.949)--(-8.685,2.944)--(-8.713,2.946)--(-8.682,2.949)--cycle;
\filldraw[fill opacity=0.8,fill=gray!20](-8.758,3.329)--(-8.721,3.342)--(-8.703,3.345)--(-8.723,3.336)--cycle;
\filldraw[fill opacity=0.8,fill=gray!20,draw=none](-8.07,4.533)--(-8.084,4.548)--(-8.118,4.573)--(-8.12,4.573)--(-8.115,4.568)--cycle;
\draw(-8.07,4.533)--(-8.084,4.548)--(-8.118,4.573);
\draw(-8.12,4.573)--(-8.115,4.568);
\filldraw[fill opacity=0.8,fill=gray!20,draw=none](-8.058,4.518)--(-8.075,4.539)--(-8.07,4.533)--cycle;
\draw(-8.075,4.539)--(-8.07,4.533);
\filldraw[fill opacity=0.8,fill=gray!20,draw=none](-8.07,4.533)--(-8.115,4.568)--(-8.113,4.566)--(-8.084,4.543)--cycle;
\draw(-8.115,4.568)--(-8.113,4.566)--(-8.084,4.543);
\filldraw[fill opacity=0.8,fill=gray!20,draw=none](-8.109,4.556)--(-8.084,4.543)--(-8.098,4.554)--cycle;
\draw(-8.084,4.543)--(-8.098,4.554);
\filldraw[fill opacity=0.8,fill=gray!20,draw=none](-8.109,4.556)--(-8.098,4.554)--(-8.113,4.566)--(-8.119,4.562)--cycle;
\draw(-8.098,4.554)--(-8.113,4.566)--(-8.119,4.562);
\filldraw[fill opacity=0.8,fill=gray!20,draw=none](-8.037,4.55)--(-8.187,4.478)--(-8.225,4.501)--(-8.068,4.576)--cycle;
\draw(-8.037,4.55)--(-8.187,4.478)--(-8.225,4.501)--(-8.068,4.576);
\filldraw[fill opacity=0.8,fill=gray!20](-8.82,2.997)--(-8.852,3.038)--(-8.871,3.059)--(-8.837,3.014)--cycle;
\filldraw[fill opacity=0.8,fill=gray!20,draw=none](-8.529,3.229)--(-8.508,3.215)--(-8.523,3.251)--(-8.534,3.244)--cycle;
\draw(-8.508,3.215)--(-8.523,3.251)--(-8.534,3.244);
\filldraw[fill opacity=0.8,fill=gray!20,draw=none](-8.505,3.219)--(-8.486,3.205)--(-8.47,3.216)--(-8.487,3.257)--(-8.516,3.252)--cycle;
\draw(-8.486,3.205)--(-8.47,3.216)--(-8.487,3.257);
\filldraw[fill opacity=0.8,fill=gray!20,draw=none](-8.529,3.229)--(-8.534,3.244)--(-8.542,3.238)--cycle;
\draw(-8.534,3.244)--(-8.542,3.238);
\filldraw[fill opacity=0.8,fill=gray!20,draw=none](-8.52,3.201)--(-8.529,3.229)--(-8.542,3.238)--(-8.543,3.238)--cycle;
\draw(-8.542,3.238)--(-8.543,3.238);
\filldraw[fill opacity=0.8,fill=gray!20,draw=none](-8.505,3.219)--(-8.498,3.198)--(-8.486,3.205)--cycle;
\draw(-8.498,3.198)--(-8.486,3.205);
\filldraw[fill opacity=0.8,fill=gray!20,draw=none](-8.497,3.191)--(-8.498,3.198)--(-8.5,3.197)--cycle;
\draw(-8.498,3.198)--(-8.5,3.197);
\filldraw[fill opacity=0.8,fill=gray!20,draw=none](-8.505,3.219)--(-8.516,3.252)--(-8.523,3.248)--(-8.516,3.226)--cycle;
\draw(-8.516,3.252)--(-8.523,3.248)--(-8.516,3.226);
\filldraw[fill opacity=0.8,fill=gray!20,draw=none](-8.5,3.197)--(-8.498,3.198)--(-8.505,3.219)--(-8.516,3.226)--(-8.513,3.219)--cycle;
\draw(-8.5,3.197)--(-8.498,3.198);
\draw(-8.516,3.226)--(-8.513,3.219);
\filldraw[fill opacity=0.8,fill=gray!20,draw=none](-8.521,3.23)--(-8.513,3.219)--(-8.516,3.226)--cycle;
\draw(-8.513,3.219)--(-8.516,3.226);
\filldraw[fill opacity=0.8,fill=gray!20,draw=none](-8.344,3.143)--(-8.346,3.152)--(-8.332,3.143)--(-8.327,3.117)--cycle;
\filldraw[fill opacity=0.8,fill=gray!20,draw=none](-8.332,3.143)--(-8.334,3.152)--(-8.325,3.139)--cycle;
\draw(-8.334,3.152)--(-8.325,3.139);
\filldraw[fill opacity=0.8,fill=gray!20,draw=none](-8.353,3.156)--(-8.367,3.177)--(-8.358,3.176)--(-8.344,3.167)--(-8.334,3.152)--(-8.332,3.143)--cycle;
\draw(-8.358,3.176)--(-8.344,3.167)--(-8.334,3.152);
\filldraw[fill opacity=0.8,fill=gray!20,draw=none](-8.321,3.115)--(-8.326,3.121)--(-8.322,3.121)--cycle;
\draw(-8.326,3.121)--(-8.322,3.121);
\filldraw[fill opacity=0.8,fill=gray!20,draw=none](-8.323,3.139)--(-8.322,3.121)--(-8.326,3.121)--(-8.342,3.144)--(-8.342,3.151)--cycle;
\draw(-8.322,3.121)--(-8.326,3.121);
\draw(-8.342,3.144)--(-8.342,3.151);
\filldraw[fill opacity=0.8,fill=gray!20,draw=none](-8.344,3.143)--(-8.353,3.156)--(-8.346,3.152)--cycle;
\filldraw[fill opacity=0.8,fill=gray!20,draw=none](-8.305,3.14)--(-8.284,3.11)--(-8.303,3.119)--(-8.323,3.148)--cycle;
\draw(-8.284,3.11)--(-8.303,3.119);
\filldraw[fill opacity=0.8,fill=gray!20,draw=none](-8.348,3.154)--(-8.342,3.151)--(-8.342,3.144)--cycle;
\draw(-8.342,3.151)--(-8.342,3.144);
\filldraw[fill opacity=0.8,fill=gray!20,draw=none](-8.313,3.134)--(-8.323,3.139)--(-8.323,3.148)--cycle;
\filldraw[fill opacity=0.8,fill=gray!20,draw=none](-8.344,3.169)--(-8.337,3.164)--(-8.344,3.167)--(-8.386,3.194)--(-8.378,3.19)--cycle;
\draw(-8.337,3.164)--(-8.344,3.167)--(-8.386,3.194)--(-8.378,3.19);
\filldraw[fill opacity=0.8,fill=gray!20,draw=none](-8.323,3.157)--(-8.323,3.139)--(-8.342,3.151)--(-8.343,3.168)--cycle;
\draw(-8.342,3.151)--(-8.343,3.168);
\filldraw[fill opacity=0.8,fill=gray!20,draw=none](-8.296,3.108)--(-8.293,3.106)--(-8.293,3.102)--cycle;
\draw(-8.296,3.108)--(-8.293,3.106);
\filldraw[fill opacity=0.8,fill=gray!20,draw=none](-8.323,3.157)--(-8.343,3.168)--(-8.343,3.169)--(-8.323,3.168)--cycle;
\draw(-8.343,3.168)--(-8.343,3.169)--(-8.323,3.168);
\filldraw[fill opacity=0.8,fill=gray!20,draw=none](-8.278,3.164)--(-8.343,3.169)--(-8.344,3.185)--(-8.293,3.174)--cycle;
\draw(-8.278,3.164)--(-8.343,3.169)--(-8.344,3.185);
\filldraw[fill opacity=0.8,fill=gray!20,draw=none](-8.272,3.157)--(-8.323,3.168)--(-8.278,3.164)--cycle;
\draw(-8.323,3.168)--(-8.278,3.164);
\filldraw[fill opacity=0.8,fill=gray!20,draw=none](-8.303,3.139)--(-8.323,3.148)--(-8.333,3.163)--(-8.318,3.156)--cycle;
\draw(-8.333,3.163)--(-8.318,3.156);
\filldraw[fill opacity=0.8,fill=gray!20,draw=none](-8.303,3.139)--(-8.299,3.129)--(-8.318,3.14)--(-8.313,3.151)--cycle;
\draw(-8.318,3.14)--(-8.313,3.151);
\filldraw[fill opacity=0.8,fill=gray!20,draw=none](-8.344,3.169)--(-8.336,3.164)--(-8.337,3.164)--cycle;
\draw(-8.336,3.164)--(-8.337,3.164);
\filldraw[fill opacity=0.8,fill=gray!20,draw=none](-8.367,3.177)--(-8.371,3.185)--(-8.358,3.176)--cycle;
\draw(-8.371,3.185)--(-8.358,3.176);
\filldraw[fill opacity=0.8,fill=gray!20,draw=none](-8.344,3.169)--(-8.378,3.19)--(-8.374,3.188)--cycle;
\draw(-8.378,3.19)--(-8.374,3.188);
\filldraw[fill opacity=0.8,fill=gray!20,draw=none](-8.356,3.168)--(-8.364,3.187)--(-8.344,3.185)--(-8.343,3.169)--cycle;
\draw(-8.344,3.185)--(-8.343,3.169)--(-8.356,3.168);
\filldraw[fill opacity=0.8,fill=gray!20,draw=none](-8.26,3.168)--(-8.263,3.138)--(-8.267,3.127)--(-8.307,3.15)--(-8.31,3.158)--(-8.302,3.176)--cycle;
\draw(-8.263,3.138)--(-8.267,3.127);
\draw(-8.31,3.158)--(-8.302,3.176);
\filldraw[fill opacity=0.8,fill=gray!20,draw=none](-8.357,3.178)--(-8.318,3.156)--(-8.336,3.164)--(-8.344,3.169)--(-8.361,3.18)--cycle;
\draw(-8.318,3.156)--(-8.336,3.164);
\filldraw[fill opacity=0.8,fill=gray!20,draw=none](-8.333,3.165)--(-8.317,3.156)--(-8.313,3.151)--(-8.318,3.14)--(-8.326,3.145)--(-8.342,3.167)--cycle;
\draw(-8.313,3.151)--(-8.318,3.14);
\filldraw[fill opacity=0.8,fill=gray!20,draw=none](-8.299,3.129)--(-8.289,3.101)--(-8.32,3.136)--(-8.318,3.14)--cycle;
\draw(-8.32,3.136)--(-8.318,3.14);
\filldraw[fill opacity=0.8,fill=gray!20,draw=none](-8.318,3.14)--(-8.32,3.136)--(-8.326,3.145)--cycle;
\draw(-8.318,3.14)--(-8.32,3.136);
\filldraw[fill opacity=0.8,fill=gray!20,draw=none](-8.267,3.127)--(-8.273,3.114)--(-8.289,3.123)--(-8.303,3.139)--(-8.307,3.15)--cycle;
\draw(-8.267,3.127)--(-8.273,3.114);
\filldraw[fill opacity=0.8,fill=gray!20,draw=none](-8.289,3.123)--(-8.299,3.129)--(-8.303,3.139)--cycle;
\filldraw[fill opacity=0.8,fill=gray!20,draw=none](-8.289,3.123)--(-8.273,3.114)--(-8.276,3.108)--cycle;
\draw(-8.273,3.114)--(-8.276,3.108);
\filldraw[fill opacity=0.8,fill=gray!20,draw=none](-8.303,3.139)--(-8.281,3.129)--(-8.275,3.107)--cycle;
\filldraw[fill opacity=0.8,fill=gray!20,draw=none](-8.289,3.123)--(-8.276,3.108)--(-8.276,3.107)--(-8.292,3.108)--(-8.299,3.129)--cycle;
\draw(-8.276,3.108)--(-8.276,3.107);
\filldraw[fill opacity=0.8,fill=gray!20,draw=none](-8.275,3.107)--(-8.276,3.107)--(-8.276,3.108)--cycle;
\draw(-8.276,3.107)--(-8.276,3.108);
\filldraw[fill opacity=0.8,fill=gray!20,draw=none](-8.303,3.134)--(-8.278,3.118)--(-8.275,3.107)--(-8.296,3.108)--cycle;
\filldraw[fill opacity=0.8,fill=gray!20,draw=none](-8.263,3.125)--(-8.262,3.109)--(-8.275,3.107)--(-8.276,3.108)--(-8.267,3.127)--cycle;
\draw(-8.276,3.108)--(-8.267,3.127);
\filldraw[fill opacity=0.8,fill=gray!20,draw=none](-8.278,3.118)--(-8.271,3.114)--(-8.275,3.107)--cycle;
\filldraw[fill opacity=0.8,fill=gray!20,draw=none](-8.274,3.107)--(-8.271,3.114)--(-8.263,3.109)--cycle;
\filldraw[fill opacity=0.8,fill=gray!20,draw=none](-8.263,3.125)--(-8.267,3.127)--(-8.263,3.138)--cycle;
\draw(-8.267,3.127)--(-8.263,3.138);
\filldraw[fill opacity=0.8,fill=gray!20,draw=none](-8.263,3.109)--(-8.271,3.114)--(-8.264,3.129)--(-8.239,3.112)--cycle;
\filldraw[fill opacity=0.8,fill=gray!20,draw=none](-8.271,3.114)--(-8.278,3.118)--(-8.283,3.137)--(-8.264,3.129)--cycle;
\draw(-8.283,3.137)--(-8.264,3.129);
\filldraw[fill opacity=0.8,fill=gray!20,draw=none](-8.179,3.093)--(-8.188,3.073)--(-8.203,3.078)--(-8.189,3.108)--cycle;
\draw(-8.179,3.093)--(-8.188,3.073);
\draw(-8.203,3.078)--(-8.189,3.108);
\filldraw[fill opacity=0.8,fill=gray!20,draw=none](-8.281,3.129)--(-8.223,3.104)--(-8.193,3.071)--(-8.193,3.071)--(-8.275,3.106)--cycle;
\draw(-8.193,3.071)--(-8.275,3.106);
\filldraw[fill opacity=0.8,fill=gray!20,draw=none](-8.303,3.139)--(-8.313,3.151)--(-8.31,3.158)--cycle;
\draw(-8.313,3.151)--(-8.31,3.158);
\filldraw[fill opacity=0.8,fill=gray!20,draw=none](-8.317,3.156)--(-8.312,3.153)--(-8.313,3.151)--cycle;
\draw(-8.312,3.153)--(-8.313,3.151);
\filldraw[fill opacity=0.8,fill=gray!20,draw=none](-8.239,3.112)--(-8.264,3.129)--(-8.23,3.114)--cycle;
\draw(-8.264,3.129)--(-8.23,3.114);
\filldraw[fill opacity=0.8,fill=gray!20,draw=none](-8.291,3.134)--(-8.303,3.139)--(-8.313,3.151)--(-8.317,3.156)--(-8.279,3.139)--cycle;
\draw(-8.317,3.156)--(-8.279,3.139);
\filldraw[fill opacity=0.8,fill=gray!20,draw=none](-8.189,3.108)--(-8.224,3.112)--(-8.216,3.131)--cycle;
\draw(-8.224,3.112)--(-8.216,3.131);
\filldraw[fill opacity=0.8,fill=gray!20,draw=none](-8.221,3.122)--(-8.189,3.108)--(-8.224,3.112)--(-8.23,3.114)--(-8.248,3.122)--cycle;
\draw(-8.221,3.122)--(-8.189,3.108);
\draw(-8.23,3.114)--(-8.248,3.122);
\filldraw[fill opacity=0.8,fill=gray!20,draw=none](-8.25,3.166)--(-8.263,3.138)--(-8.26,3.168)--cycle;
\draw(-8.25,3.166)--(-8.263,3.138);
\filldraw[fill opacity=0.8,fill=gray!20,draw=none](-8.296,3.152)--(-8.292,3.153)--(-8.221,3.122)--(-8.248,3.122)--(-8.265,3.129)--cycle;
\draw(-8.292,3.153)--(-8.221,3.122);
\draw(-8.248,3.122)--(-8.265,3.129);
\filldraw[fill opacity=0.8,fill=gray!20,draw=none](-8.317,3.156)--(-8.323,3.164)--(-8.311,3.162)--(-8.31,3.158)--(-8.312,3.153)--cycle;
\draw(-8.31,3.158)--(-8.312,3.153);
\filldraw[fill opacity=0.8,fill=gray!20,draw=none](-8.314,3.177)--(-8.311,3.162)--(-8.333,3.165)--(-8.348,3.174)--(-8.353,3.181)--(-8.353,3.182)--cycle;
\draw(-8.353,3.181)--(-8.353,3.182);
\filldraw[fill opacity=0.8,fill=gray!20,draw=none](-8.333,3.165)--(-8.323,3.164)--(-8.317,3.156)--cycle;
\filldraw[fill opacity=0.8,fill=gray!20,draw=none](-8.348,3.174)--(-8.296,3.152)--(-8.279,3.139)--(-8.317,3.156)--cycle;
\draw(-8.279,3.139)--(-8.317,3.156);
\filldraw[fill opacity=0.8,fill=gray!20,draw=none](-8.296,3.152)--(-8.265,3.129)--(-8.306,3.147)--cycle;
\draw(-8.265,3.129)--(-8.306,3.147);
\filldraw[fill opacity=0.8,fill=gray!20,draw=none](-8.233,3.115)--(-8.223,3.104)--(-8.291,3.134)--(-8.279,3.139)--(-8.245,3.124)--cycle;
\draw(-8.279,3.139)--(-8.245,3.124);
\filldraw[fill opacity=0.8,fill=gray!20,draw=none](-8.278,3.118)--(-8.303,3.134)--(-8.306,3.147)--(-8.283,3.137)--cycle;
\draw(-8.306,3.147)--(-8.283,3.137);
\filldraw[fill opacity=0.8,fill=gray!20](-8.245,3.082)--(-8.573,3.224)--(-8.598,3.263)--(-8.271,3.12)--cycle;
\filldraw[fill opacity=0.8,fill=gray!20](-8.245,3.082)--(-8.573,3.224)--(-8.598,3.263)--(-8.271,3.12)--cycle;
\filldraw[fill opacity=0.8,fill=gray!20](-8.724,2.986)--(-8.763,3)--(-8.773,3.01)--(-8.729,2.991)--cycle;
\filldraw[fill opacity=0.8,fill=gray!20](-8.682,2.985)--(-8.724,2.986)--(-8.729,2.991)--(-8.682,2.985)--cycle;
\filldraw[fill opacity=0.8,fill=gray!20](-8.811,3.291)--(-8.773,3.319)--(-8.758,3.329)--(-8.79,3.305)--cycle;
\filldraw[fill opacity=0.8,fill=gray!20](-8.412,4.491)--(-8.377,4.535)--(-8.352,4.552)--(-8.381,4.511)--cycle;
\filldraw[fill opacity=0.8,fill=gray!20](-8.763,3)--(-8.797,3.024)--(-8.811,3.039)--(-8.773,3.01)--cycle;
\filldraw[fill opacity=0.8,fill=gray!20](-8.682,2.985)--(-8.708,2.982)--(-8.724,2.986)--(-8.682,2.985)--cycle;
\filldraw[fill opacity=0.8,fill=gray!20](-8.631,3.335)--(-8.656,3.344)--(-8.639,3.34)--(-8.6,3.327)--cycle;
\filldraw[fill opacity=0.8,fill=gray!20](-8.682,2.985)--(-8.684,2.981)--(-8.708,2.982)--(-8.682,2.985)--cycle;
\filldraw[fill opacity=0.8,fill=gray!20](-8.682,2.985)--(-8.66,2.982)--(-8.684,2.981)--(-8.682,2.985)--cycle;
\filldraw[fill opacity=0.8,fill=gray!20](-8.222,4.167)--(-8.279,4.175)--(-8.27,4.181)--(-8.222,4.167)--cycle;
\filldraw[fill opacity=0.5,fill=gray!20](-7.696,1.143)--(-7.523,1.068)--(-7.573,.645)--(-7.746,.721)--cycle;
\filldraw[fill opacity=0.5,fill=gray!20](-7.925,2.574)--(-7.997,2.591)--(-7.706,2.209)--(-7.632,2.189)--cycle;
\filldraw[fill opacity=0.8,fill=gray!20,draw=none](-8.883,3.03)--(-8.88,3.035)--(-8.868,3.054)--(-8.869,3.051)--cycle;
\draw(-8.883,3.03)--(-8.88,3.035);
\draw(-8.868,3.054)--(-8.869,3.051);
\filldraw[fill opacity=0.8,fill=gray!20,draw=none](-8.868,3.055)--(-8.868,3.063)--(-8.86,3.081)--(-8.849,3.098)--(-8.849,3.096)--(-8.868,3.054)--cycle;
\draw(-8.868,3.063)--(-8.86,3.081);
\draw(-8.849,3.096)--(-8.868,3.054);
\filldraw[fill opacity=0.8,fill=gray!20,draw=none](-8.868,3.055)--(-8.868,3.063)--(-8.879,3.076)--(-8.871,3.059)--cycle;
\draw(-8.879,3.076)--(-8.871,3.059)--(-8.868,3.055);
\filldraw[fill opacity=0.8,fill=gray!20,draw=none](-8.88,3.035)--(-8.87,3.059)--(-8.868,3.054)--cycle;
\draw(-8.88,3.035)--(-8.87,3.059);
\filldraw[fill opacity=0.8,fill=gray!20,draw=none](-8.849,3.107)--(-8.857,3.117)--(-8.849,3.098)--cycle;
\draw(-8.857,3.117)--(-8.849,3.098);
\filldraw[fill opacity=0.8,fill=gray!20,draw=none](-8.86,3.081)--(-8.802,3.216)--(-8.798,3.214)--(-8.846,3.103)--cycle;
\draw(-8.86,3.081)--(-8.802,3.216)--(-8.798,3.214)--(-8.846,3.103);
\filldraw[fill opacity=0.8,fill=gray!20](-8.797,3.024)--(-8.823,3.059)--(-8.84,3.076)--(-8.811,3.039)--cycle;
\filldraw[fill opacity=0.8,fill=gray!20](-8.222,4.167)--(-8.172,4.18)--(-8.166,4.174)--(-8.222,4.167)--cycle;
\filldraw[fill opacity=0.8,fill=gray!20,draw=none](-8.868,3.063)--(-8.867,3.077)--(-8.871,3.087)--(-8.893,3.111)--(-8.879,3.076)--cycle;
\draw(-8.867,3.077)--(-8.871,3.087)--(-8.893,3.111)--(-8.879,3.076);
\filldraw[fill opacity=0.8,fill=gray!20,draw=none](-8.837,3.094)--(-8.787,3.209)--(-8.802,3.216)--(-8.849,3.107)--cycle;
\draw(-8.837,3.094)--(-8.787,3.209)--(-8.802,3.216)--(-8.849,3.107);
\filldraw[fill opacity=0.8,fill=gray!20,draw=none](-8.823,3.059)--(-8.84,3.1)--(-8.849,3.109)--(-8.849,3.098)--(-8.84,3.076)--cycle;
\draw(-8.849,3.098)--(-8.84,3.076)--(-8.823,3.059)--(-8.84,3.1)--(-8.849,3.109);
\filldraw[fill opacity=0.8,fill=gray!20,draw=none](-8.636,2.987)--(-8.614,2.995)--(-8.615,2.996)--(-8.637,2.992)--cycle;
\draw(-8.636,2.987)--(-8.614,2.995);
\draw(-8.615,2.996)--(-8.637,2.992);
\filldraw[fill opacity=0.8,fill=gray!20,draw=none](-8.614,2.995)--(-8.605,2.998)--(-8.615,2.996)--cycle;
\draw(-8.614,2.995)--(-8.605,2.998)--(-8.615,2.996);
\filldraw[fill opacity=0.8,fill=gray!20,draw=none](-8.614,2.995)--(-8.616,2.997)--(-8.619,2.99)--cycle;
\draw(-8.616,2.997)--(-8.619,2.99);
\filldraw[fill opacity=0.8,fill=gray!20,draw=none](-8.637,2.992)--(-8.615,2.996)--(-8.614,3)--(-8.631,3.003)--(-8.639,2.993)--cycle;
\draw(-8.637,2.992)--(-8.615,2.996);
\draw(-8.631,3.003)--(-8.639,2.993);
\filldraw[fill opacity=0.8,fill=gray!20,draw=none](-8.642,2.985)--(-8.636,2.987)--(-8.637,2.992)--(-8.641,2.991)--(-8.66,2.982)--cycle;
\draw(-8.637,2.992)--(-8.641,2.991)--(-8.66,2.982)--(-8.642,2.985)--(-8.636,2.987);
\filldraw[fill opacity=0.8,fill=gray!20,draw=none](-8.622,2.982)--(-8.616,2.997)--(-8.638,2.994)--cycle;
\draw(-8.622,2.982)--(-8.616,2.997);
\filldraw[fill opacity=0.8,fill=gray!20,draw=none](-8.625,2.996)--(-8.616,2.997)--(-8.614,3)--(-8.631,3.003)--cycle;
\draw(-8.616,2.997)--(-8.614,3);
\filldraw[fill opacity=0.8,fill=gray!20,draw=none](-8.639,2.993)--(-8.625,3.01)--(-8.656,3.005)--cycle;
\draw(-8.639,2.993)--(-8.625,3.01);
\filldraw[fill opacity=0.8,fill=gray!20,draw=none](-8.638,2.994)--(-8.625,2.996)--(-8.631,3.003)--(-8.638,3.004)--(-8.641,2.997)--cycle;
\draw(-8.638,3.004)--(-8.641,2.997);
\filldraw[fill opacity=0.8,fill=gray!20,draw=none](-8.631,3.003)--(-8.636,3.008)--(-8.636,3.008)--(-8.638,3.004)--cycle;
\draw(-8.636,3.008)--(-8.638,3.004);
\filldraw[fill opacity=0.8,fill=gray!20,draw=none](-8.636,3.008)--(-8.63,3.012)--(-8.635,3.012)--(-8.636,3.009)--cycle;
\draw(-8.635,3.012)--(-8.636,3.009);
\filldraw[fill opacity=0.8,fill=gray!20,draw=none](-8.656,3.005)--(-8.635,3.008)--(-8.645,3.011)--(-8.663,3.01)--cycle;
\draw(-8.645,3.011)--(-8.663,3.01);
\filldraw[fill opacity=0.8,fill=gray!20,draw=none](-8.636,3.008)--(-8.636,3.009)--(-8.636,3.008)--cycle;
\draw(-8.636,3.009)--(-8.636,3.008);
\filldraw[fill opacity=0.8,fill=gray!20,draw=none](-8.639,3.007)--(-8.636,3.008)--(-8.635,3.012)--(-8.647,3.013)--cycle;
\draw(-8.636,3.008)--(-8.635,3.012);
\filldraw[fill opacity=0.8,fill=gray!20,draw=none](-8.635,3.008)--(-8.625,3.01)--(-8.636,3.011)--(-8.645,3.011)--cycle;
\draw(-8.636,3.011)--(-8.645,3.011);
\filldraw[fill opacity=0.8,fill=gray!20,draw=none](-8.641,2.997)--(-8.636,3.008)--(-8.65,3.003)--cycle;
\draw(-8.641,2.997)--(-8.636,3.008);
\filldraw[fill opacity=0.8,fill=gray!20,draw=none](-8.663,3.01)--(-8.645,3.011)--(-8.647,3.013)--(-8.672,3.016)--cycle;
\draw(-8.663,3.01)--(-8.645,3.011);
\filldraw[fill opacity=0.8,fill=gray!20,draw=none](-8.65,3.003)--(-8.639,3.007)--(-8.647,3.013)--(-8.669,3.014)--cycle;
\filldraw[fill opacity=0.8,fill=gray!20,draw=none](-8.645,3.011)--(-8.636,3.011)--(-8.647,3.013)--cycle;
\draw(-8.645,3.011)--(-8.636,3.011);
\filldraw[fill opacity=0.8,fill=gray!20,draw=none](-8.612,2.982)--(-8.604,3.001)--(-8.634,2.995)--cycle;
\draw(-8.612,2.982)--(-8.604,3.001);
\filldraw[fill opacity=0.8,fill=gray!20,draw=none](-8.603,3.001)--(-8.604,3.001)--(-8.605,2.998)--cycle;
\draw(-8.604,3.001)--(-8.605,2.998);
\filldraw[fill opacity=0.8,fill=gray!20,draw=none](-8.634,2.995)--(-8.604,3.001)--(-8.652,3.007)--cycle;
\filldraw[fill opacity=0.8,fill=gray!20,draw=none](-8.461,2.92)--(-8.436,2.928)--(-8.403,2.924)--(-8.436,2.913)--(-8.453,2.915)--cycle;
\draw(-8.403,2.924)--(-8.436,2.913)--(-8.453,2.915);
\filldraw[fill opacity=0.8,fill=gray!20,draw=none](-8.461,2.92)--(-8.481,2.932)--(-8.436,2.928)--cycle;
\filldraw[fill opacity=0.8,fill=gray!20,draw=none](-8.427,2.909)--(-8.436,2.913)--(-8.386,2.93)--(-8.374,2.924)--cycle;
\draw(-8.427,2.909)--(-8.436,2.913)--(-8.386,2.93)--(-8.374,2.924);
\filldraw[fill opacity=0.8,fill=gray!20,draw=none](-8.449,2.913)--(-8.456,2.916)--(-8.436,2.913)--(-8.427,2.909)--cycle;
\draw(-8.456,2.916)--(-8.436,2.913)--(-8.427,2.909);
\filldraw[fill opacity=0.8,fill=gray!20,draw=none](-8.361,2.865)--(-8.424,2.899)--(-8.378,2.901)--(-8.345,2.895)--(-8.348,2.86)--cycle;
\draw(-8.424,2.899)--(-8.378,2.901);
\draw(-8.345,2.895)--(-8.348,2.86);
\filldraw[fill opacity=0.8,fill=gray!20,draw=none](-8.378,2.901)--(-8.424,2.899)--(-8.416,2.907)--cycle;
\draw(-8.378,2.901)--(-8.424,2.899);
\filldraw[fill opacity=0.8,fill=gray!20,draw=none](-8.416,2.907)--(-8.424,2.899)--(-8.427,2.909)--cycle;
\draw(-8.424,2.899)--(-8.427,2.909);
\filldraw[fill opacity=0.8,fill=gray!20,draw=none](-8.434,2.905)--(-8.443,2.911)--(-8.427,2.909)--cycle;
\filldraw[fill opacity=0.8,fill=gray!20,draw=none](-8.434,2.905)--(-8.427,2.909)--(-8.424,2.899)--cycle;
\draw(-8.427,2.909)--(-8.424,2.899);
\filldraw[fill opacity=0.8,fill=gray!20,draw=none](-8.442,2.911)--(-8.449,2.913)--(-8.427,2.909)--cycle;
\filldraw[fill opacity=0.8,fill=gray!20,draw=none](-8.442,2.911)--(-8.47,2.929)--(-8.432,2.926)--(-8.427,2.909)--cycle;
\draw(-8.432,2.926)--(-8.427,2.909);
\filldraw[fill opacity=0.8,fill=gray!20,draw=none](-8.442,2.911)--(-8.443,2.911)--(-8.474,2.929)--(-8.47,2.929)--cycle;
\filldraw[fill opacity=0.8,fill=gray!20,draw=none](-8.451,2.919)--(-8.727,3.04)--(-8.682,3.035)--(-8.518,2.963)--cycle;
\draw(-8.451,2.919)--(-8.727,3.04)--(-8.682,3.035)--(-8.518,2.963);
\filldraw[fill opacity=0.8,fill=gray!20,draw=none](-8.451,2.919)--(-8.727,3.04)--(-8.682,3.035)--(-8.518,2.963)--cycle;
\draw(-8.451,2.919)--(-8.727,3.04)--(-8.682,3.035)--(-8.518,2.963);
\filldraw[fill opacity=0.8,fill=gray!20](-8.216,4.591)--(-8.219,4.601)--(-8.191,4.599)--(-8.162,4.587)--cycle;
\filldraw[fill opacity=0.8,fill=gray!20](-8.272,4.588)--(-8.248,4.6)--(-8.219,4.601)--(-8.216,4.591)--cycle;
\filldraw[fill opacity=0.8,fill=gray!20](-8.566,3.302)--(-8.6,3.327)--(-8.59,3.317)--(-8.552,3.288)--cycle;
\filldraw[fill opacity=0.8,fill=gray!20,draw=none](-8.53,3.007)--(-8.527,3.01)--(-8.534,3.005)--cycle;
\draw(-8.53,3.007)--(-8.527,3.01)--(-8.534,3.005);
\filldraw[fill opacity=0.8,fill=gray!20](-8.871,3.087)--(-8.878,3.142)--(-8.901,3.166)--(-8.893,3.111)--cycle;
\filldraw[fill opacity=0.8,fill=gray!20,draw=none](-8.771,3.685)--(-8.776,3.685)--(-8.639,4.05)--cycle;
\draw(-8.776,3.685)--(-8.639,4.05)--(-8.771,3.685);
\filldraw[fill opacity=0.8,fill=gray!20](-8.713,2.946)--(-8.742,2.958)--(-8.78,2.967)--(-8.732,2.951)--cycle;
\filldraw[fill opacity=0.8,fill=gray!20,draw=none](-8.849,3.107)--(-8.849,3.109)--(-8.858,3.119)--(-8.857,3.117)--cycle;
\draw(-8.849,3.109)--(-8.858,3.119)--(-8.857,3.117);
\filldraw[fill opacity=0.8,fill=gray!20,draw=none](-8.849,3.109)--(-8.844,3.131)--(-8.845,3.145)--(-8.864,3.166)--(-8.858,3.119)--cycle;
\draw(-8.844,3.131)--(-8.845,3.145)--(-8.864,3.166)--(-8.858,3.119)--(-8.849,3.109);
\filldraw[fill opacity=0.8,fill=gray!20,draw=none](-8.562,3.262)--(-8.544,3.24)--(-8.542,3.238)--(-8.523,3.251)--(-8.535,3.265)--cycle;
\draw(-8.542,3.238)--(-8.523,3.251)--(-8.535,3.265);
\filldraw[fill opacity=0.8,fill=gray!20,draw=none](-8.544,3.24)--(-8.562,3.262)--(-8.566,3.262)--(-8.559,3.249)--cycle;
\draw(-8.566,3.262)--(-8.559,3.249);
\filldraw[fill opacity=0.8,fill=gray!20,draw=none](-8.582,3.263)--(-8.559,3.249)--(-8.566,3.262)--cycle;
\draw(-8.559,3.249)--(-8.566,3.262);
\filldraw[fill opacity=0.8,fill=gray!20,draw=none](-8.521,3.23)--(-8.516,3.226)--(-8.523,3.248)--(-8.533,3.246)--cycle;
\draw(-8.516,3.226)--(-8.523,3.248)--(-8.533,3.246);
\filldraw[fill opacity=0.8,fill=gray!20,draw=none](-8.521,3.23)--(-8.533,3.246)--(-8.544,3.244)--cycle;
\draw(-8.533,3.246)--(-8.544,3.244);
\filldraw[fill opacity=0.8,fill=gray!20,draw=none](-8.562,3.254)--(-8.544,3.244)--(-8.533,3.246)--(-8.539,3.252)--cycle;
\draw(-8.544,3.244)--(-8.533,3.246);
\filldraw[fill opacity=0.8,fill=gray!20,draw=none](-8.49,3.216)--(-8.598,3.263)--(-8.637,3.287)--(-8.477,3.217)--cycle;
\draw(-8.49,3.216)--(-8.598,3.263)--(-8.637,3.287)--(-8.477,3.217);
\filldraw[fill opacity=0.8,fill=gray!20,draw=none](-8.49,3.216)--(-8.598,3.263)--(-8.637,3.287)--(-8.477,3.217)--cycle;
\draw(-8.49,3.216)--(-8.598,3.263)--(-8.637,3.287)--(-8.477,3.217);
\filldraw[fill opacity=0.8,fill=gray!20,draw=none](-8.856,3.05)--(-8.837,3.094)--(-8.849,3.107)--(-8.868,3.063)--cycle;
\draw(-8.856,3.05)--(-8.837,3.094);
\draw(-8.849,3.107)--(-8.868,3.063);
\filldraw[fill opacity=0.8,fill=gray!20,draw=none](-8.868,3.055)--(-8.852,3.038)--(-8.867,3.077)--cycle;
\draw(-8.868,3.055)--(-8.852,3.038)--(-8.867,3.077);
\filldraw[fill opacity=0.8,fill=gray!20,draw=none](-8.868,3.055)--(-8.87,3.059)--(-8.868,3.063)--cycle;
\draw(-8.87,3.059)--(-8.868,3.063);
\filldraw[fill opacity=0.8,fill=gray!20,draw=none](-8.123,4.559)--(-8.113,4.566)--(-8.125,4.571)--(-8.144,4.562)--(-8.143,4.561)--cycle;
\draw(-8.123,4.559)--(-8.113,4.566)--(-8.125,4.571);
\draw(-8.144,4.562)--(-8.143,4.561);
\filldraw[fill opacity=0.8,fill=gray!20,draw=none](-8.109,4.556)--(-8.119,4.562)--(-8.123,4.559)--cycle;
\draw(-8.119,4.562)--(-8.123,4.559);
\filldraw[fill opacity=0.8,fill=gray!20,draw=none](-8.105,4.558)--(-8.225,4.501)--(-8.262,4.506)--(-8.125,4.571)--cycle;
\draw(-8.105,4.558)--(-8.225,4.501)--(-8.262,4.506)--(-8.125,4.571);
\filldraw[fill opacity=0.8,fill=gray!20](-8.708,2.982)--(-8.732,2.992)--(-8.763,3)--(-8.724,2.986)--cycle;
\filldraw[fill opacity=0.8,fill=gray!20](-8.791,3.35)--(-8.738,3.371)--(-8.729,3.377)--(-8.773,3.362)--cycle;
\filldraw[fill opacity=0.8,fill=gray!20](-8.314,4.58)--(-8.27,4.595)--(-8.248,4.6)--(-8.272,4.588)--cycle;
\filldraw[fill opacity=0.8,fill=gray!20](-8.66,2.982)--(-8.641,2.991)--(-8.687,2.989)--(-8.684,2.981)--cycle;
\filldraw[fill opacity=0.8,fill=gray!20,draw=none](-8.637,2.992)--(-8.639,2.993)--(-8.641,2.991)--cycle;
\draw(-8.639,2.993)--(-8.641,2.991)--(-8.637,2.992);
\filldraw[fill opacity=0.8,fill=gray!20,draw=none](-8.641,2.991)--(-8.639,2.993)--(-8.656,3.005)--(-8.688,3)--(-8.687,2.989)--cycle;
\draw(-8.688,3)--(-8.687,2.989)--(-8.641,2.991)--(-8.639,2.993);
\filldraw[fill opacity=0.8,fill=gray!20,draw=none](-8.646,2.956)--(-8.638,2.954)--(-8.634,2.954)--(-8.622,2.982)--(-8.638,2.994)--(-8.643,2.994)--(-8.655,2.966)--cycle;
\draw(-8.634,2.954)--(-8.622,2.982);
\draw(-8.643,2.994)--(-8.655,2.966);
\filldraw[fill opacity=0.8,fill=gray!20,draw=none](-8.634,2.95)--(-8.631,2.951)--(-8.609,2.961)--(-8.633,2.956)--(-8.656,2.945)--cycle;
\draw(-8.609,2.961)--(-8.633,2.956)--(-8.656,2.945)--(-8.634,2.95)--(-8.631,2.951);
\filldraw[fill opacity=0.8,fill=gray!20,draw=none](-8.635,2.943)--(-8.632,2.951)--(-8.639,2.943)--(-8.668,2.877)--cycle;
\draw(-8.635,2.943)--(-8.632,2.951);
\draw(-8.639,2.943)--(-8.668,2.877);
\filldraw[fill opacity=0.8,fill=gray!20,draw=none](-8.632,2.951)--(-8.634,2.954)--(-8.639,2.943)--cycle;
\draw(-8.634,2.954)--(-8.639,2.943);
\filldraw[fill opacity=0.8,fill=gray!20,draw=none](-8.589,3.048)--(-8.562,3.111)--(-8.565,3.113)--(-8.602,3.029)--cycle;
\draw(-8.589,3.048)--(-8.562,3.111)--(-8.565,3.113)--(-8.602,3.029);
\filldraw[fill opacity=0.8,fill=gray!20,draw=none](-8.582,3.026)--(-8.566,3.032)--(-8.559,3.042)--cycle;
\draw(-8.566,3.032)--(-8.559,3.042);
\filldraw[fill opacity=0.8,fill=gray!20,draw=none](-8.581,3.068)--(-8.569,3.032)--(-8.576,3.023)--(-8.599,3.033)--cycle;
\draw(-8.576,3.023)--(-8.599,3.033);
\filldraw[fill opacity=0.8,fill=gray!20,draw=none](-8.581,3.068)--(-8.569,3.032)--(-8.576,3.023)--(-8.599,3.033)--cycle;
\draw(-8.576,3.023)--(-8.599,3.033);
\filldraw[fill opacity=0.8,fill=gray!20](-8.773,3.319)--(-8.729,3.336)--(-8.721,3.342)--(-8.758,3.329)--cycle;
\filldraw[fill opacity=0.5,fill=gray!20](-8.129,-.044)--(-7.956,-.119)--(-8.261,-.409)--(-8.434,-.333)--cycle;
\filldraw[fill opacity=0.8,fill=gray!20,draw=none](-8.631,2.951)--(-8.59,2.965)--(-8.609,2.961)--cycle;
\draw(-8.631,2.951)--(-8.59,2.965)--(-8.609,2.961);
\filldraw[fill opacity=0.8,fill=gray!20](-8.878,3.142)--(-8.871,3.198)--(-8.893,3.222)--(-8.901,3.166)--cycle;
\filldraw[fill opacity=0.8,fill=gray!20](-8.273,4.169)--(-8.321,4.185)--(-8.332,4.197)--(-8.279,4.175)--cycle;
\filldraw[fill opacity=0.8,fill=gray!20](-8.222,4.167)--(-8.273,4.169)--(-8.279,4.175)--(-8.222,4.167)--cycle;
\filldraw[fill opacity=0.8,fill=gray!20](-8.377,4.535)--(-8.332,4.568)--(-8.314,4.58)--(-8.352,4.552)--cycle;
\filldraw[fill opacity=0.8,fill=gray!20](-8.162,4.587)--(-8.191,4.599)--(-8.172,4.594)--(-8.124,4.578)--cycle;
\filldraw[fill opacity=0.8,fill=gray!20](-8.321,4.185)--(-8.361,4.215)--(-8.377,4.232)--(-8.332,4.197)--cycle;
\filldraw[fill opacity=0.8,fill=gray!20](-8.222,4.167)--(-8.254,4.164)--(-8.273,4.169)--(-8.222,4.167)--cycle;
\filldraw[fill opacity=0.8,fill=gray!20](-8.222,4.167)--(-8.226,4.162)--(-8.254,4.164)--(-8.222,4.167)--cycle;
\filldraw[fill opacity=0.8,fill=gray!20](-8.222,4.167)--(-8.197,4.163)--(-8.226,4.162)--(-8.222,4.167)--cycle;
\filldraw[fill opacity=0.8,fill=gray!20](-8.222,4.167)--(-8.175,4.168)--(-8.197,4.163)--(-8.222,4.167)--cycle;
\filldraw[fill opacity=0.8,fill=gray!20](-8.222,4.167)--(-8.166,4.174)--(-8.175,4.168)--(-8.222,4.167)--cycle;
\filldraw[fill opacity=0.8,fill=gray!20](-8.845,3.145)--(-8.84,3.192)--(-8.858,3.212)--(-8.864,3.166)--cycle;
\filldraw[fill opacity=0.8,fill=gray!20,draw=none](-8.776,3.685)--(-8.788,3.685)--(-8.792,3.687)--(-8.826,3.703)--(-8.792,3.693)--cycle;
\draw(-8.788,3.685)--(-8.792,3.687)--(-8.826,3.703);
\filldraw[fill opacity=0.8,fill=gray!20](-8.707,3.381)--(-8.682,3.378)--(-8.682,3.378)--(-8.678,3.383)--cycle;
\filldraw[fill opacity=0.8,fill=gray!20](-8.703,3.345)--(-8.682,3.342)--(-8.682,3.342)--(-8.679,3.346)--cycle;
\filldraw[fill opacity=0.8,fill=gray!20](-8.679,3.346)--(-8.682,3.342)--(-8.682,3.342)--(-8.656,3.344)--cycle;
\filldraw[fill opacity=0.8,fill=gray!20](-8.361,4.215)--(-8.393,4.256)--(-8.412,4.277)--(-8.377,4.232)--cycle;
\filldraw[fill opacity=0.8,fill=gray!20](-8.166,4.174)--(-8.113,4.194)--(-8.131,4.183)--(-8.175,4.168)--cycle;
\filldraw[fill opacity=0.8,fill=gray!20](-8.6,3.327)--(-8.639,3.34)--(-8.634,3.335)--(-8.59,3.317)--cycle;
\filldraw[fill opacity=0.8,fill=gray!20,draw=none](-8.63,3.012)--(-8.602,3.029)--(-8.565,3.113)--(-8.587,3.122)--(-8.635,3.012)--cycle;
\draw(-8.602,3.029)--(-8.565,3.113)--(-8.587,3.122)--(-8.635,3.012);
\filldraw[fill opacity=0.8,fill=gray!20,draw=none](-8.647,3.013)--(-8.635,3.012)--(-8.587,3.122)--(-8.623,3.138)--(-8.671,3.027)--cycle;
\draw(-8.635,3.012)--(-8.587,3.122)--(-8.623,3.138)--(-8.671,3.027);
\filldraw[fill opacity=0.8,fill=gray!20,draw=none](-8.672,3.016)--(-8.636,3.011)--(-8.624,3.012)--(-8.618,3.025)--(-8.69,3.028)--(-8.69,3.028)--cycle;
\draw(-8.636,3.011)--(-8.624,3.012)--(-8.618,3.025);
\draw(-8.69,3.028)--(-8.69,3.028);
\filldraw[fill opacity=0.8,fill=gray!20,draw=none](-8.669,3.014)--(-8.647,3.013)--(-8.671,3.027)--(-8.675,3.018)--cycle;
\draw(-8.671,3.027)--(-8.675,3.018);
\filldraw[fill opacity=0.8,fill=gray!20,draw=none](-8.71,3.041)--(-8.675,3.018)--(-8.623,3.138)--(-8.668,3.157)--(-8.715,3.047)--cycle;
\draw(-8.675,3.018)--(-8.623,3.138)--(-8.668,3.157)--(-8.715,3.047);
\filldraw[fill opacity=0.8,fill=gray!20,draw=none](-8.618,3.025)--(-8.611,3.043)--(-8.691,3.04)--(-8.69,3.028)--cycle;
\draw(-8.618,3.025)--(-8.611,3.043)--(-8.691,3.04)--(-8.69,3.028);
\filldraw[fill opacity=0.8,fill=gray!20,draw=none](-8.711,3.04)--(-8.69,3.028)--(-8.69,3.028)--(-8.71,3.041)--(-8.711,3.041)--cycle;
\draw(-8.69,3.028)--(-8.69,3.028);
\draw(-8.71,3.041)--(-8.711,3.041);
\filldraw[fill opacity=0.8,fill=gray!20,draw=none](-8.71,3.041)--(-8.707,3.038)--(-8.675,3.018)--cycle;
\filldraw[fill opacity=0.8,fill=gray!20,draw=none](-8.69,3.028)--(-8.691,3.04)--(-8.71,3.041)--cycle;
\draw(-8.69,3.028)--(-8.691,3.04)--(-8.71,3.041);
\filldraw[fill opacity=0.8,fill=gray!20,draw=none](-8.556,3.008)--(-8.586,2.993)--(-8.617,3.007)--(-8.599,3.033)--(-8.564,3.018)--cycle;
\draw(-8.586,2.993)--(-8.617,3.007);
\draw(-8.599,3.033)--(-8.564,3.018);
\filldraw[fill opacity=0.8,fill=gray!20,draw=none](-8.615,3.014)--(-8.582,3.026)--(-8.559,3.042)--(-8.549,3.055)--(-8.611,3.043)--(-8.624,3.012)--cycle;
\draw(-8.559,3.042)--(-8.549,3.055)--(-8.611,3.043)--(-8.624,3.012)--(-8.615,3.014);
\filldraw[fill opacity=0.8,fill=gray!20,draw=none](-8.615,3.014)--(-8.624,3.012)--(-8.625,3.01)--cycle;
\draw(-8.615,3.014)--(-8.624,3.012)--(-8.625,3.01);
\filldraw[fill opacity=0.8,fill=gray!20,draw=none](-8.625,3.01)--(-8.624,3.012)--(-8.636,3.011)--cycle;
\draw(-8.625,3.01)--(-8.624,3.012)--(-8.636,3.011);
\filldraw[fill opacity=0.8,fill=gray!20,draw=none](-8.652,3.007)--(-8.604,3.001)--(-8.597,3.019)--(-8.666,3.016)--cycle;
\draw(-8.604,3.001)--(-8.597,3.019)--(-8.666,3.016);
\filldraw[fill opacity=0.8,fill=gray!20](-8.549,3.055)--(-8.534,3.096)--(-8.603,3.083)--(-8.611,3.043)--cycle;
\filldraw[fill opacity=0.8,fill=gray!20](-8.611,3.043)--(-8.603,3.083)--(-8.692,3.079)--(-8.691,3.04)--cycle;
\filldraw[fill opacity=0.8,fill=gray!20,draw=none](-8.747,3.083)--(-8.715,3.047)--(-8.668,3.157)--(-8.715,3.178)--(-8.751,3.094)--cycle;
\draw(-8.715,3.047)--(-8.668,3.157)--(-8.715,3.178)--(-8.751,3.094);
\filldraw[fill opacity=0.8,fill=gray!20,draw=none](-8.738,3.079)--(-8.716,3.047)--(-8.698,3.04)--(-8.691,3.04)--(-8.692,3.079)--(-8.738,3.082)--cycle;
\draw(-8.698,3.04)--(-8.691,3.04)--(-8.692,3.079)--(-8.738,3.082);
\filldraw[fill opacity=0.8,fill=gray!20,draw=none](-8.688,3.03)--(-8.666,3.016)--(-8.619,3.018)--(-8.62,3.065)--(-8.694,3.062)--(-8.694,3.04)--cycle;
\draw(-8.666,3.016)--(-8.619,3.018);
\draw(-8.62,3.065)--(-8.694,3.062)--(-8.694,3.04);
\filldraw[fill opacity=0.8,fill=gray!20,draw=none](-8.469,2.931)--(-8.518,2.963)--(-8.436,2.928)--cycle;
\draw(-8.518,2.963)--(-8.436,2.928);
\filldraw[fill opacity=0.8,fill=gray!20,draw=none](-8.469,2.931)--(-8.518,2.963)--(-8.436,2.928)--cycle;
\draw(-8.518,2.963)--(-8.436,2.928);
\filldraw[fill opacity=0.8,fill=gray!20,draw=none](-8.543,3.059)--(-8.52,3.103)--(-8.519,3.106)--(-8.534,3.096)--(-8.549,3.055)--cycle;
\draw(-8.519,3.106)--(-8.534,3.096)--(-8.549,3.055)--(-8.543,3.059);
\filldraw[fill opacity=0.8,fill=gray!20,draw=none](-8.519,3.106)--(-8.511,3.153)--(-8.529,3.141)--(-8.534,3.096)--cycle;
\draw(-8.511,3.153)--(-8.529,3.141)--(-8.534,3.096)--(-8.519,3.106);
\filldraw[fill opacity=0.8,fill=gray!20,draw=none](-8.567,3.027)--(-8.569,3.032)--(-8.544,3.058)--(-8.532,3.053)--cycle;
\draw(-8.544,3.058)--(-8.532,3.053);
\filldraw[fill opacity=0.8,fill=gray!20,draw=none](-8.516,3.038)--(-8.505,3.067)--(-8.516,3.053)--(-8.523,3.033)--cycle;
\draw(-8.516,3.053)--(-8.523,3.033)--(-8.516,3.038);
\filldraw[fill opacity=0.8,fill=gray!20,draw=none](-8.517,3.092)--(-8.497,3.083)--(-8.521,3.048)--(-8.544,3.058)--cycle;
\draw(-8.521,3.048)--(-8.544,3.058);
\filldraw[fill opacity=0.8,fill=gray!20,draw=none](-8.517,3.092)--(-8.497,3.083)--(-8.521,3.048)--(-8.544,3.058)--cycle;
\draw(-8.521,3.048)--(-8.544,3.058);
\filldraw[fill opacity=0.8,fill=gray!20,draw=none](-8.516,3.038)--(-8.523,3.033)--(-8.525,3.03)--cycle;
\draw(-8.516,3.038)--(-8.523,3.033)--(-8.525,3.03);
\filldraw[fill opacity=0.8,fill=gray!20,draw=none](-8.525,3.03)--(-8.523,3.033)--(-8.544,3.029)--(-8.562,3.017)--cycle;
\draw(-8.525,3.03)--(-8.523,3.033)--(-8.544,3.029);
\filldraw[fill opacity=0.8,fill=gray!20,draw=none](-8.56,3.016)--(-8.564,3.018)--(-8.569,3.032)--(-8.544,3.058)--(-8.521,3.048)--cycle;
\draw(-8.56,3.016)--(-8.564,3.018);
\draw(-8.544,3.058)--(-8.521,3.048);
\filldraw[fill opacity=0.8,fill=gray!20,draw=none](-8.511,3.153)--(-8.519,3.198)--(-8.534,3.188)--(-8.529,3.141)--cycle;
\draw(-8.519,3.198)--(-8.534,3.188)--(-8.529,3.141)--(-8.511,3.153);
\filldraw[fill opacity=0.8,fill=gray!20,draw=none](-8.56,3.016)--(-8.564,3.018)--(-8.567,3.027)--(-8.532,3.053)--(-8.521,3.048)--cycle;
\draw(-8.56,3.016)--(-8.564,3.018);
\draw(-8.532,3.053)--(-8.521,3.048);
\filldraw[fill opacity=0.8,fill=gray!20,draw=none](-8.559,3.042)--(-8.569,3.032)--(-8.569,3.035)--cycle;
\filldraw[fill opacity=0.8,fill=gray!20,draw=none](-8.559,3.042)--(-8.569,3.032)--(-8.569,3.035)--cycle;
\filldraw[fill opacity=0.8,fill=gray!20,draw=none](-8.569,3.032)--(-8.564,3.018)--(-8.576,3.023)--cycle;
\draw(-8.564,3.018)--(-8.576,3.023);
\filldraw[fill opacity=0.8,fill=gray!20,draw=none](-8.569,3.032)--(-8.564,3.018)--(-8.576,3.023)--cycle;
\draw(-8.564,3.018)--(-8.576,3.023);
\filldraw[fill opacity=0.8,fill=gray!20](-8.523,3.033)--(-8.505,3.082)--(-8.587,3.066)--(-8.597,3.019)--cycle;
\filldraw[fill opacity=0.8,fill=gray!20,draw=none](-8.5,3.086)--(-8.497,3.093)--(-8.489,3.143)--(-8.498,3.137)--(-8.505,3.082)--cycle;
\draw(-8.489,3.143)--(-8.498,3.137)--(-8.505,3.082)--(-8.5,3.086);
\filldraw[fill opacity=0.8,fill=gray!20](-8.534,3.096)--(-8.529,3.141)--(-8.6,3.128)--(-8.603,3.083)--cycle;
\filldraw[fill opacity=0.8,fill=gray!20,draw=none](-8.5,3.086)--(-8.505,3.082)--(-8.513,3.059)--cycle;
\draw(-8.5,3.086)--(-8.505,3.082)--(-8.513,3.059);
\filldraw[fill opacity=0.8,fill=gray!20,draw=none](-8.538,3.006)--(-8.56,3.016)--(-8.521,3.048)--(-8.496,3.037)--cycle;
\draw(-8.538,3.006)--(-8.56,3.016);
\draw(-8.521,3.048)--(-8.496,3.037);
\filldraw[fill opacity=0.8,fill=gray!20,draw=none](-8.538,3.006)--(-8.56,3.016)--(-8.521,3.048)--(-8.496,3.037)--cycle;
\draw(-8.538,3.006)--(-8.56,3.016);
\draw(-8.521,3.048)--(-8.496,3.037);
\filldraw[fill opacity=0.8,fill=gray!20,draw=none](-8.497,3.083)--(-8.489,3.079)--(-8.516,3.046)--(-8.521,3.048)--cycle;
\draw(-8.516,3.046)--(-8.521,3.048);
\filldraw[fill opacity=0.8,fill=gray!20,draw=none](-8.497,3.083)--(-8.331,3.01)--(-8.348,2.972)--(-8.521,3.048)--cycle;
\draw(-8.348,2.972)--(-8.521,3.048);
\filldraw[fill opacity=0.8,fill=gray!20,draw=none](-8.489,3.143)--(-8.497,3.191)--(-8.5,3.197)--(-8.505,3.193)--(-8.498,3.137)--cycle;
\draw(-8.5,3.197)--(-8.505,3.193)--(-8.498,3.137)--(-8.489,3.143);
\filldraw[fill opacity=0.8,fill=gray!20,draw=none](-8.563,3.016)--(-8.564,3.018)--(-8.562,3.017)--cycle;
\draw(-8.564,3.018)--(-8.562,3.017);
\filldraw[fill opacity=0.8,fill=gray!20,draw=none](-8.563,3.016)--(-8.564,3.018)--(-8.562,3.017)--cycle;
\draw(-8.564,3.018)--(-8.562,3.017);
\filldraw[fill opacity=0.8,fill=gray!20,draw=none](-8.562,3.017)--(-8.544,3.029)--(-8.589,3.021)--(-8.603,3.001)--cycle;
\draw(-8.544,3.029)--(-8.589,3.021);
\filldraw[fill opacity=0.8,fill=gray!20,draw=none](-8.436,2.928)--(-8.518,2.963)--(-8.563,3.016)--(-8.562,3.017)--(-8.391,2.942)--cycle;
\draw(-8.436,2.928)--(-8.518,2.963);
\draw(-8.562,3.017)--(-8.391,2.942);
\filldraw[fill opacity=0.8,fill=gray!20,draw=none](-8.436,2.928)--(-8.518,2.963)--(-8.563,3.016)--(-8.562,3.017)--(-8.391,2.942)--cycle;
\draw(-8.436,2.928)--(-8.518,2.963);
\draw(-8.562,3.017)--(-8.391,2.942);
\filldraw[fill opacity=0.8,fill=gray!20,draw=none](-8.489,3.079)--(-8.331,3.01)--(-8.348,2.972)--(-8.516,3.046)--cycle;
\draw(-8.348,2.972)--(-8.516,3.046);
\filldraw[fill opacity=0.8,fill=gray!20](-8.505,3.082)--(-8.498,3.137)--(-8.583,3.12)--(-8.587,3.066)--cycle;
\filldraw[fill opacity=0.8,fill=gray!20](-8.529,3.141)--(-8.534,3.188)--(-8.603,3.175)--(-8.6,3.128)--cycle;
\filldraw[fill opacity=0.8,fill=gray!20](-8.498,3.137)--(-8.505,3.193)--(-8.587,3.177)--(-8.583,3.12)--cycle;
\filldraw[fill opacity=0.8,fill=gray!20,draw=none](-8.491,2.986)--(-8.538,3.006)--(-8.496,3.037)--(-8.43,3.008)--cycle;
\draw(-8.491,2.986)--(-8.538,3.006);
\draw(-8.496,3.037)--(-8.43,3.008);
\filldraw[fill opacity=0.8,fill=gray!20,draw=none](-8.353,2.974)--(-8.371,2.942)--(-8.386,2.93)--(-8.403,2.924)--(-8.412,2.925)--cycle;
\draw(-8.371,2.942)--(-8.386,2.93)--(-8.403,2.924);
\filldraw[fill opacity=0.8,fill=gray!20,draw=none](-8.386,2.94)--(-8.538,3.006)--(-8.496,3.037)--(-8.348,2.972)--cycle;
\draw(-8.386,2.94)--(-8.538,3.006);
\draw(-8.496,3.037)--(-8.348,2.972);
\filldraw[fill opacity=0.8,fill=gray!20,draw=none](-8.391,3.18)--(-8.367,3.177)--(-8.353,3.156)--cycle;
\filldraw[fill opacity=0.8,fill=gray!20,draw=none](-8.385,3.176)--(-8.408,3.18)--(-8.49,3.216)--(-8.477,3.217)--(-8.391,3.18)--cycle;
\draw(-8.408,3.18)--(-8.49,3.216);
\draw(-8.477,3.217)--(-8.391,3.18);
\filldraw[fill opacity=0.8,fill=gray!20,draw=none](-8.385,3.176)--(-8.408,3.18)--(-8.49,3.216)--(-8.477,3.217)--(-8.391,3.18)--cycle;
\draw(-8.408,3.18)--(-8.49,3.216);
\draw(-8.477,3.217)--(-8.391,3.18);
\filldraw[fill opacity=0.8,fill=gray!20,draw=none](-8.386,2.94)--(-8.491,2.986)--(-8.43,3.008)--(-8.348,2.972)--cycle;
\draw(-8.386,2.94)--(-8.491,2.986);
\draw(-8.43,3.008)--(-8.348,2.972);
\filldraw[fill opacity=0.8,fill=gray!20,draw=none](-8.412,2.952)--(-8.439,2.95)--(-8.449,3.006)--(-8.342,3.011)--(-8.342,3.009)--cycle;
\draw(-8.412,2.952)--(-8.439,2.95)--(-8.449,3.006)--(-8.342,3.011)--(-8.342,3.009);
\filldraw[fill opacity=0.8,fill=gray!20,draw=none](-8.341,3.01)--(-8.327,3.021)--(-8.327,3.019)--(-8.344,2.989)--cycle;
\filldraw[fill opacity=0.8,fill=gray!20,draw=none](-8.341,3.011)--(-8.342,3.011)--(-8.341,3.068)--(-8.332,3.068)--cycle;
\draw(-8.341,3.011)--(-8.342,3.011)--(-8.341,3.068)--(-8.332,3.068);
\filldraw[fill opacity=0.8,fill=gray!20](-8.449,3.006)--(-8.452,3.063)--(-8.341,3.068)--(-8.342,3.011)--cycle;
\filldraw[fill opacity=0.8,fill=gray!20,draw=none](-8.332,3.068)--(-8.341,3.068)--(-8.342,3.122)--(-8.341,3.122)--cycle;
\draw(-8.332,3.068)--(-8.341,3.068)--(-8.342,3.122)--(-8.341,3.122);
\filldraw[fill opacity=0.8,fill=gray!20](-8.452,3.063)--(-8.449,3.117)--(-8.342,3.122)--(-8.341,3.068)--cycle;
\filldraw[fill opacity=0.8,fill=gray!20,draw=none](-8.374,2.924)--(-8.378,2.926)--(-8.344,2.955)--cycle;
\draw(-8.374,2.924)--(-8.378,2.926);
\filldraw[fill opacity=0.8,fill=gray!20,draw=none](-8.402,2.921)--(-8.384,2.939)--(-8.343,2.951)--(-8.345,2.911)--cycle;
\draw(-8.343,2.951)--(-8.345,2.911);
\filldraw[fill opacity=0.8,fill=gray!20,draw=none](-8.378,2.901)--(-8.416,2.907)--(-8.402,2.921)--(-8.345,2.911)--(-8.345,2.903)--cycle;
\draw(-8.345,2.911)--(-8.345,2.903)--(-8.378,2.901);
\filldraw[fill opacity=0.8,fill=gray!20,draw=none](-8.402,2.921)--(-8.416,2.907)--(-8.427,2.909)--(-8.432,2.926)--cycle;
\draw(-8.427,2.909)--(-8.432,2.926);
\filldraw[fill opacity=0.8,fill=gray!20,draw=none](-8.361,2.911)--(-8.392,2.894)--(-8.427,2.909)--(-8.374,2.924)--(-8.356,2.916)--cycle;
\draw(-8.392,2.894)--(-8.427,2.909);
\draw(-8.374,2.924)--(-8.356,2.916);
\filldraw[fill opacity=0.8,fill=gray!20,draw=none](-8.341,3.01)--(-8.344,2.989)--(-8.353,2.974)--(-8.412,2.925)--(-8.441,2.928)--cycle;
\filldraw[fill opacity=0.8,fill=gray!20,draw=none](-8.46,2.946)--(-8.531,2.989)--(-8.531,2.991)--(-8.449,3.006)--(-8.439,2.95)--cycle;
\draw(-8.531,2.989)--(-8.531,2.991)--(-8.449,3.006)--(-8.439,2.95)--(-8.46,2.946);
\filldraw[fill opacity=0.8,fill=gray!20](-8.531,2.991)--(-8.538,3.047)--(-8.452,3.063)--(-8.449,3.006)--cycle;
\filldraw[fill opacity=0.8,fill=gray!20,draw=none](-8.436,2.94)--(-8.439,2.95)--(-8.412,2.952)--cycle;
\draw(-8.436,2.94)--(-8.439,2.95)--(-8.412,2.952);
\filldraw[fill opacity=0.8,fill=gray!20,draw=none](-8.37,3.168)--(-8.351,3.155)--(-8.408,3.18)--(-8.383,3.176)--cycle;
\draw(-8.351,3.155)--(-8.408,3.18);
\filldraw[fill opacity=0.8,fill=gray!20,draw=none](-8.348,3.154)--(-8.356,3.168)--(-8.343,3.169)--(-8.342,3.151)--cycle;
\draw(-8.356,3.168)--(-8.343,3.169)--(-8.342,3.151);
\filldraw[fill opacity=0.8,fill=gray!20,draw=none](-8.37,3.168)--(-8.356,3.168)--(-8.348,3.154)--cycle;
\draw(-8.37,3.168)--(-8.356,3.168);
\filldraw[fill opacity=0.8,fill=gray!20,draw=none](-8.351,3.155)--(-8.408,3.18)--(-8.382,3.176)--cycle;
\draw(-8.351,3.155)--(-8.408,3.18);
\filldraw[fill opacity=0.8,fill=gray!20,draw=none](-8.449,3.117)--(-8.439,3.165)--(-8.412,3.166)--(-8.342,3.124)--(-8.342,3.122)--cycle;
\draw(-8.342,3.124)--(-8.342,3.122)--(-8.449,3.117)--(-8.439,3.165)--(-8.412,3.166);
\filldraw[fill opacity=0.8,fill=gray!20](-8.538,3.047)--(-8.531,3.101)--(-8.449,3.117)--(-8.452,3.063)--cycle;
\filldraw[fill opacity=0.8,fill=gray!20,draw=none](-8.46,2.946)--(-8.439,2.95)--(-8.436,2.94)--cycle;
\draw(-8.46,2.946)--(-8.439,2.95)--(-8.436,2.94);
\filldraw[fill opacity=0.8,fill=gray!20,draw=none](-8.531,3.101)--(-8.531,3.103)--(-8.46,3.161)--(-8.439,3.165)--(-8.449,3.117)--cycle;
\draw(-8.46,3.161)--(-8.439,3.165)--(-8.449,3.117)--(-8.531,3.101)--(-8.531,3.103);
\filldraw[fill opacity=0.8,fill=gray!20,draw=none](-8.412,3.166)--(-8.439,3.165)--(-8.436,3.172)--cycle;
\draw(-8.412,3.166)--(-8.439,3.165)--(-8.436,3.172);
\filldraw[fill opacity=0.8,fill=gray!20,draw=none](-8.46,3.161)--(-8.436,3.172)--(-8.439,3.165)--cycle;
\draw(-8.436,3.172)--(-8.439,3.165)--(-8.46,3.161);
\filldraw[fill opacity=0.8,fill=gray!20,draw=none](-8.532,2.99)--(-8.541,3.045)--(-8.538,3.047)--(-8.531,2.991)--cycle;
\draw(-8.541,3.045)--(-8.538,3.047)--(-8.531,2.991)--(-8.532,2.99);
\filldraw[fill opacity=0.8,fill=gray!20,draw=none](-8.541,3.045)--(-8.532,3.101)--(-8.531,3.101)--(-8.538,3.047)--cycle;
\draw(-8.532,3.101)--(-8.531,3.101)--(-8.538,3.047)--(-8.541,3.045);
\filldraw[fill opacity=0.8,fill=gray!20,draw=none](-8.531,2.989)--(-8.532,2.99)--(-8.541,3.045)--(-8.531,3.103)--(-8.46,3.161)--(-8.436,3.172)--(-8.412,3.166)--(-8.341,3.123)--(-8.332,3.068)--(-8.341,3.01)--(-8.412,2.952)--(-8.443,2.936)--cycle;
\filldraw[fill opacity=0.8,fill=gray!20,draw=none](-8.402,2.921)--(-8.432,2.926)--(-8.384,2.939)--cycle;
\filldraw[fill opacity=0.8,fill=gray!20,draw=none](-8.409,2.916)--(-8.436,2.928)--(-8.391,2.942)--(-8.367,2.932)--cycle;
\draw(-8.409,2.916)--(-8.436,2.928);
\draw(-8.391,2.942)--(-8.367,2.932);
\filldraw[fill opacity=0.8,fill=gray!20,draw=none](-8.313,2.941)--(-8.316,2.936)--(-8.344,2.933)--(-8.343,2.951)--(-8.322,2.953)--(-8.313,2.953)--cycle;
\draw(-8.344,2.933)--(-8.343,2.951);
\draw(-8.322,2.953)--(-8.313,2.953);
\filldraw[fill opacity=0.8,fill=gray!20,draw=none](-8.336,2.943)--(-8.361,2.919)--(-8.374,2.924)--(-8.349,2.949)--cycle;
\draw(-8.361,2.919)--(-8.374,2.924);
\filldraw[fill opacity=0.8,fill=gray!20,draw=none](-8.38,2.919)--(-8.396,2.91)--(-8.436,2.928)--(-8.391,2.942)--(-8.367,2.932)--cycle;
\draw(-8.396,2.91)--(-8.436,2.928);
\draw(-8.391,2.942)--(-8.367,2.932);
\filldraw[fill opacity=0.8,fill=gray!20,draw=none](-8.412,2.952)--(-8.441,2.928)--(-8.457,2.93)--cycle;
\filldraw[fill opacity=0.8,fill=gray!20,draw=none](-8.457,2.93)--(-8.436,2.928)--(-8.432,2.926)--cycle;
\draw(-8.436,2.928)--(-8.432,2.926);
\filldraw[fill opacity=0.8,fill=gray!20,draw=none](-8.457,2.93)--(-8.436,2.928)--(-8.432,2.926)--cycle;
\draw(-8.436,2.928)--(-8.432,2.926);
\filldraw[fill opacity=0.8,fill=gray!20,draw=none](-8.46,2.946)--(-8.443,2.936)--(-8.457,2.93)--(-8.481,2.932)--(-8.519,2.956)--(-8.525,2.964)--cycle;
\filldraw[fill opacity=0.8,fill=gray!20,draw=none](-8.378,2.901)--(-8.345,2.903)--(-8.345,2.895)--cycle;
\draw(-8.378,2.901)--(-8.345,2.903)--(-8.345,2.895);
\filldraw[fill opacity=0.8,fill=gray!20,draw=none](-8.427,2.909)--(-8.451,2.919)--(-8.469,2.931)--(-8.457,2.93)--(-8.432,2.926)--(-8.391,2.908)--cycle;
\draw(-8.427,2.909)--(-8.451,2.919);
\draw(-8.432,2.926)--(-8.391,2.908);
\filldraw[fill opacity=0.8,fill=gray!20,draw=none](-8.427,2.909)--(-8.451,2.919)--(-8.469,2.931)--(-8.457,2.93)--(-8.432,2.926)--(-8.391,2.908)--cycle;
\draw(-8.427,2.909)--(-8.451,2.919);
\draw(-8.432,2.926)--(-8.391,2.908);
\filldraw[fill opacity=0.8,fill=gray!20,draw=none](-8.474,2.929)--(-8.492,2.94)--(-8.46,2.946)--(-8.436,2.94)--(-8.432,2.926)--cycle;
\draw(-8.492,2.94)--(-8.46,2.946);
\draw(-8.436,2.94)--(-8.432,2.926);
\filldraw[fill opacity=0.8,fill=gray!20,draw=none](-8.412,2.952)--(-8.342,3.009)--(-8.343,2.955)--cycle;
\draw(-8.342,3.009)--(-8.343,2.955)--(-8.412,2.952);
\filldraw[fill opacity=0.8,fill=gray!20,draw=none](-8.384,2.939)--(-8.37,2.954)--(-8.343,2.955)--(-8.343,2.951)--cycle;
\draw(-8.37,2.954)--(-8.343,2.955)--(-8.343,2.951);
\filldraw[fill opacity=0.8,fill=gray!20,draw=none](-8.531,2.989)--(-8.46,2.946)--(-8.516,2.962)--cycle;
\filldraw[fill opacity=0.8,fill=gray!20,draw=none](-8.545,2.997)--(-8.531,2.989)--(-8.516,2.962)--(-8.525,2.964)--(-8.545,2.995)--cycle;
\filldraw[fill opacity=0.8,fill=gray!20,draw=none](-8.46,2.946)--(-8.492,2.94)--(-8.515,2.954)--(-8.523,2.965)--(-8.531,2.989)--cycle;
\draw(-8.46,2.946)--(-8.492,2.94);
\draw(-8.523,2.965)--(-8.531,2.989);
\filldraw[fill opacity=0.8,fill=gray!20,draw=none](-8.335,2.943)--(-8.349,2.949)--(-8.344,2.955)--(-8.336,2.962)--(-8.325,2.957)--cycle;
\draw(-8.336,2.962)--(-8.325,2.957);
\filldraw[fill opacity=0.8,fill=gray!20,draw=none](-8.323,2.988)--(-8.323,2.966)--(-8.343,2.956)--(-8.342,2.977)--cycle;
\draw(-8.343,2.956)--(-8.342,2.977);
\filldraw[fill opacity=0.8,fill=gray!20,draw=none](-8.361,2.929)--(-8.386,2.94)--(-8.348,2.972)--(-8.336,2.967)--cycle;
\draw(-8.361,2.929)--(-8.386,2.94);
\draw(-8.348,2.972)--(-8.336,2.967);
\filldraw[fill opacity=0.8,fill=gray!20,draw=none](-8.361,2.929)--(-8.386,2.94)--(-8.348,2.972)--(-8.336,2.967)--cycle;
\draw(-8.361,2.929)--(-8.386,2.94);
\draw(-8.348,2.972)--(-8.336,2.967);
\filldraw[fill opacity=0.8,fill=gray!20,draw=none](-8.384,2.939)--(-8.432,2.926)--(-8.436,2.94)--(-8.412,2.952)--(-8.37,2.954)--cycle;
\draw(-8.432,2.926)--(-8.436,2.94);
\draw(-8.412,2.952)--(-8.37,2.954);
\filldraw[fill opacity=0.8,fill=gray!20,draw=none](-8.394,2.894)--(-8.442,2.911)--(-8.427,2.909)--(-8.392,2.894)--cycle;
\draw(-8.427,2.909)--(-8.392,2.894);
\filldraw[fill opacity=0.8,fill=gray!20,draw=none](-8.38,2.919)--(-8.396,2.91)--(-8.409,2.916)--(-8.367,2.932)--cycle;
\draw(-8.396,2.91)--(-8.409,2.916);
\filldraw[fill opacity=0.8,fill=gray!20,draw=none](-8.313,2.941)--(-8.291,2.932)--(-8.296,2.882)--(-8.313,2.889)--cycle;
\draw(-8.313,2.941)--(-8.291,2.932)--(-8.296,2.882)--(-8.313,2.889);
\filldraw[fill opacity=0.8,fill=gray!20,draw=none](-8.279,2.898)--(-8.281,2.921)--(-8.27,2.949)--cycle;
\filldraw[fill opacity=0.8,fill=gray!20,draw=none](-8.279,2.898)--(-8.281,2.888)--(-8.288,2.903)--(-8.281,2.921)--cycle;
\filldraw[fill opacity=0.8,fill=gray!20,draw=none](-8.281,2.888)--(-8.288,2.847)--(-8.296,2.882)--(-8.295,2.886)--(-8.288,2.903)--cycle;
\draw(-8.288,2.847)--(-8.296,2.882)--(-8.295,2.886);
\filldraw[fill opacity=0.8,fill=gray!20,draw=none](-8.279,2.898)--(-8.345,2.903)--(-8.344,2.933)--(-8.271,2.941)--cycle;
\draw(-8.279,2.898)--(-8.345,2.903)--(-8.344,2.933);
\filldraw[fill opacity=0.8,fill=gray!20,draw=none](-8.313,2.941)--(-8.313,2.936)--(-8.316,2.936)--cycle;
\filldraw[fill opacity=0.8,fill=gray!20,draw=none](-8.316,2.891)--(-8.327,2.857)--(-8.34,2.858)--(-8.348,2.86)--(-8.345,2.903)--(-8.319,2.901)--cycle;
\draw(-8.327,2.857)--(-8.34,2.858);
\draw(-8.348,2.86)--(-8.345,2.903)--(-8.319,2.901);
\filldraw[fill opacity=0.8,fill=gray!20,draw=none](-8.316,2.891)--(-8.319,2.901)--(-8.313,2.9)--cycle;
\draw(-8.319,2.901)--(-8.313,2.9);
\filldraw[fill opacity=0.8,fill=gray!20,draw=none](-8.331,2.949)--(-8.313,2.941)--(-8.313,2.9)--(-8.316,2.891)--(-8.357,2.909)--cycle;
\draw(-8.331,2.949)--(-8.313,2.941);
\draw(-8.316,2.891)--(-8.357,2.909);
\filldraw[fill opacity=0.8,fill=gray!20,draw=none](-8.313,2.941)--(-8.313,2.953)--(-8.307,2.952)--cycle;
\draw(-8.313,2.953)--(-8.307,2.952);
\filldraw[fill opacity=0.8,fill=gray!20,draw=none](-8.301,2.978)--(-8.309,2.94)--(-8.331,2.949)--cycle;
\draw(-8.309,2.94)--(-8.331,2.949);
\filldraw[fill opacity=0.8,fill=gray!20,draw=none](-8.313,2.936)--(-8.313,2.941)--(-8.307,2.952)--(-8.25,2.948)--(-8.251,2.943)--cycle;
\draw(-8.307,2.952)--(-8.25,2.948)--(-8.251,2.943);
\filldraw[fill opacity=0.8,fill=gray!20,draw=none](-8.279,2.898)--(-8.271,2.941)--(-8.251,2.943)--(-8.269,2.897)--cycle;
\draw(-8.251,2.943)--(-8.269,2.897)--(-8.279,2.898);
\filldraw[fill opacity=0.8,fill=gray!20,draw=none](-8.391,2.908)--(-8.396,2.91)--(-8.38,2.919)--cycle;
\draw(-8.391,2.908)--(-8.396,2.91);
\filldraw[fill opacity=0.8,fill=gray!20,draw=none](-8.391,2.908)--(-8.396,2.91)--(-8.38,2.919)--cycle;
\draw(-8.391,2.908)--(-8.396,2.91);
\filldraw[fill opacity=0.8,fill=gray!20,draw=none](-8.401,2.897)--(-8.419,2.904)--(-8.47,2.934)--(-8.505,2.974)--(-8.38,2.919)--cycle;
\draw(-8.505,2.974)--(-8.38,2.919);
\filldraw[fill opacity=0.8,fill=gray!20,draw=none](-8.303,2.852)--(-8.304,2.856)--(-8.302,2.885)--(-8.296,2.882)--(-8.295,2.879)--cycle;
\draw(-8.302,2.885)--(-8.296,2.882)--(-8.295,2.879);
\filldraw[fill opacity=0.8,fill=gray!20,draw=none](-8.303,2.852)--(-8.295,2.879)--(-8.287,2.844)--(-8.302,2.85)--cycle;
\draw(-8.295,2.879)--(-8.287,2.844)--(-8.302,2.85);
\filldraw[fill opacity=0.8,fill=gray!20,draw=none](-8.304,2.856)--(-8.316,2.891)--(-8.302,2.885)--cycle;
\draw(-8.316,2.891)--(-8.302,2.885);
\filldraw[fill opacity=0.8,fill=gray!20,draw=none](-8.313,2.9)--(-8.313,2.889)--(-8.316,2.891)--cycle;
\draw(-8.313,2.889)--(-8.316,2.891);
\filldraw[fill opacity=0.8,fill=gray!20,draw=none](-8.279,2.898)--(-8.269,2.897)--(-8.277,2.883)--cycle;
\draw(-8.279,2.898)--(-8.269,2.897)--(-8.277,2.883);
\filldraw[fill opacity=0.8,fill=gray!20,draw=none](-8.277,2.883)--(-8.272,2.828)--(-8.287,2.844)--(-8.288,2.847)--(-8.281,2.888)--cycle;
\draw(-8.272,2.828)--(-8.287,2.844)--(-8.288,2.847);
\filldraw[fill opacity=0.8,fill=gray!20,draw=none](-8.257,2.846)--(-8.275,2.862)--(-8.277,2.883)--cycle;
\filldraw[fill opacity=0.8,fill=gray!20,draw=none](-8.277,2.883)--(-8.269,2.897)--(-8.215,2.884)--(-8.256,2.846)--cycle;
\draw(-8.277,2.883)--(-8.269,2.897)--(-8.215,2.884)--(-8.256,2.846);
\filldraw[fill opacity=0.8,fill=gray!20,draw=none](-8.382,2.89)--(-8.389,2.893)--(-8.39,2.895)--(-8.361,2.911)--cycle;
\draw(-8.382,2.89)--(-8.389,2.893);
\filldraw[fill opacity=0.8,fill=gray!20,draw=none](-8.375,2.89)--(-8.37,2.914)--(-8.357,2.909)--(-8.361,2.889)--cycle;
\draw(-8.37,2.914)--(-8.357,2.909);
\filldraw[fill opacity=0.8,fill=gray!20,draw=none](-8.389,2.893)--(-8.392,2.894)--(-8.39,2.895)--cycle;
\draw(-8.389,2.893)--(-8.392,2.894);
\filldraw[fill opacity=0.8,fill=gray!20,draw=none](-8.401,2.897)--(-8.38,2.919)--(-8.37,2.914)--(-8.375,2.89)--(-8.38,2.89)--cycle;
\draw(-8.38,2.919)--(-8.37,2.914);
\filldraw[fill opacity=0.8,fill=gray!20,draw=none](-8.277,2.883)--(-8.281,2.888)--(-8.279,2.898)--cycle;
\filldraw[fill opacity=0.8,fill=gray!20,draw=none](-8.304,2.856)--(-8.316,2.891)--(-8.313,2.9)--(-8.279,2.898)--(-8.277,2.883)--(-8.294,2.855)--cycle;
\draw(-8.313,2.9)--(-8.279,2.898);
\draw(-8.277,2.883)--(-8.294,2.855)--(-8.304,2.856);
\filldraw[fill opacity=0.8,fill=gray!20,draw=none](-8.304,2.856)--(-8.31,2.856)--(-8.324,2.866)--(-8.316,2.891)--cycle;
\draw(-8.304,2.856)--(-8.31,2.856);
\filldraw[fill opacity=0.8,fill=gray!20,draw=none](-8.31,2.856)--(-8.327,2.857)--(-8.324,2.866)--cycle;
\draw(-8.31,2.856)--(-8.327,2.857);
\filldraw[fill opacity=0.5,fill=gray!20](-8.377,2.868)--(-8.456,2.859)--(-8.083,2.588)--(-7.997,2.591)--cycle;
\filldraw[fill opacity=0.8,fill=gray!20,draw=none](-8.363,2.878)--(-8.36,2.89)--(-8.311,2.854)--(-8.356,2.874)--cycle;
\draw(-8.311,2.854)--(-8.356,2.874);
\filldraw[fill opacity=0.8,fill=gray!20,draw=none](-8.375,2.89)--(-8.361,2.889)--(-8.363,2.877)--(-8.376,2.882)--cycle;
\draw(-8.363,2.877)--(-8.376,2.882);
\filldraw[fill opacity=0.8,fill=gray!20,draw=none](-8.375,2.89)--(-8.376,2.882)--(-8.394,2.89)--cycle;
\draw(-8.376,2.882)--(-8.394,2.89);
\filldraw[fill opacity=0.8,fill=gray!20,draw=none](-8.346,2.88)--(-8.382,2.89)--(-8.373,2.899)--cycle;
\filldraw[fill opacity=0.8,fill=gray!20,draw=none](-8.315,2.856)--(-8.31,2.856)--(-8.3,2.848)--(-8.301,2.847)--cycle;
\draw(-8.315,2.856)--(-8.31,2.856);
\draw(-8.3,2.848)--(-8.301,2.847);
\filldraw[fill opacity=0.8,fill=gray!20,draw=none](-8.363,2.893)--(-8.399,2.897)--(-8.427,2.909)--(-8.391,2.908)--(-8.376,2.901)--cycle;
\draw(-8.363,2.893)--(-8.399,2.897)--(-8.427,2.909);
\draw(-8.391,2.908)--(-8.376,2.901);
\filldraw[fill opacity=0.8,fill=gray!20,draw=none](-8.363,2.893)--(-8.399,2.897)--(-8.427,2.909)--(-8.409,2.908)--cycle;
\draw(-8.363,2.893)--(-8.399,2.897)--(-8.427,2.909);
\filldraw[fill opacity=0.8,fill=gray!20](-8.438,2.92)--(-8.765,3.063)--(-8.727,3.04)--(-8.399,2.897)--cycle;
\filldraw[fill opacity=0.8,fill=gray!20](-8.438,2.92)--(-8.765,3.063)--(-8.727,3.04)--(-8.399,2.897)--cycle;
\filldraw[fill opacity=0.8,fill=gray!20](-8.729,3.377)--(-8.682,3.378)--(-8.682,3.378)--(-8.707,3.381)--cycle;
\filldraw[fill opacity=0.8,fill=gray!20](-8.721,3.342)--(-8.682,3.342)--(-8.682,3.342)--(-8.703,3.345)--cycle;
\filldraw[fill opacity=0.8,fill=gray!20](-8.871,3.198)--(-8.852,3.252)--(-8.871,3.273)--(-8.893,3.222)--cycle;
\filldraw[fill opacity=0.8,fill=gray!20](-8.84,3.192)--(-8.823,3.237)--(-8.84,3.255)--(-8.858,3.212)--cycle;
\filldraw[fill opacity=0.8,fill=gray!20](-8.656,3.344)--(-8.682,3.342)--(-8.682,3.342)--(-8.639,3.34)--cycle;
\filldraw[fill opacity=0.8,fill=gray!20](-8.684,2.981)--(-8.687,2.989)--(-8.732,2.992)--(-8.708,2.982)--cycle;
\filldraw[fill opacity=0.8,fill=gray!20](-8.685,2.944)--(-8.688,2.954)--(-8.742,2.958)--(-8.713,2.946)--cycle;
\filldraw[fill opacity=0.8,fill=gray!20](-8.393,4.256)--(-8.412,4.305)--(-8.434,4.329)--(-8.412,4.277)--cycle;
\filldraw[fill opacity=0.8,fill=gray!20](-8.113,4.194)--(-8.067,4.228)--(-8.093,4.211)--(-8.131,4.183)--cycle;
\filldraw[fill opacity=0.5,fill=gray!20](-7.369,1.603)--(-7.369,1.65)--(-7.308,1.149)--(-7.31,1.118)--cycle;
\filldraw[fill opacity=0.8,fill=gray!20](-8.656,2.945)--(-8.633,2.956)--(-8.688,2.954)--(-8.685,2.944)--cycle;
\filldraw[fill opacity=0.8,fill=gray!20,draw=none](-8.519,3.198)--(-8.52,3.201)--(-8.543,3.238)--(-8.549,3.234)--(-8.534,3.188)--cycle;
\draw(-8.543,3.238)--(-8.549,3.234)--(-8.534,3.188)--(-8.519,3.198);
\filldraw[fill opacity=0.8,fill=gray!20,draw=none](-8.544,3.24)--(-8.543,3.238)--(-8.542,3.238)--cycle;
\draw(-8.543,3.238)--(-8.542,3.238);
\filldraw[fill opacity=0.8,fill=gray!20,draw=none](-8.543,3.238)--(-8.544,3.24)--(-8.559,3.249)--(-8.549,3.234)--cycle;
\draw(-8.559,3.249)--(-8.549,3.234)--(-8.543,3.238);
\filldraw[fill opacity=0.8,fill=gray!20,draw=none](-8.549,3.234)--(-8.559,3.249)--(-8.582,3.263)--(-8.615,3.266)--(-8.624,3.264)--(-8.611,3.222)--cycle;
\draw(-8.615,3.266)--(-8.624,3.264)--(-8.611,3.222)--(-8.549,3.234)--(-8.559,3.249);
\filldraw[fill opacity=0.8,fill=gray!20](-8.534,3.188)--(-8.549,3.234)--(-8.611,3.222)--(-8.603,3.175)--cycle;
\filldraw[fill opacity=0.8,fill=gray!20,draw=none](-8.505,3.193)--(-8.513,3.219)--(-8.521,3.23)--(-8.544,3.244)--(-8.597,3.233)--(-8.587,3.177)--cycle;
\draw(-8.544,3.244)--(-8.597,3.233)--(-8.587,3.177)--(-8.505,3.193)--(-8.513,3.219);
\filldraw[fill opacity=0.8,fill=gray!20,draw=none](-8.615,3.266)--(-8.625,3.267)--(-8.624,3.264)--cycle;
\draw(-8.625,3.267)--(-8.624,3.264)--(-8.615,3.266);
\filldraw[fill opacity=0.8,fill=gray!20,draw=none](-8.544,3.244)--(-8.562,3.254)--(-8.604,3.258)--(-8.597,3.233)--cycle;
\draw(-8.604,3.258)--(-8.597,3.233)--(-8.544,3.244);
\filldraw[fill opacity=0.8,fill=gray!20,draw=none](-8.342,3.124)--(-8.363,3.162)--(-8.353,3.156)--(-8.327,3.117)--(-8.327,3.115)--cycle;
\filldraw[fill opacity=0.8,fill=gray!20,draw=none](-8.342,3.124)--(-8.441,3.183)--(-8.436,3.185)--(-8.391,3.18)--(-8.363,3.162)--cycle;
\filldraw[fill opacity=0.8,fill=gray!20,draw=none](-8.412,3.166)--(-8.37,3.168)--(-8.348,3.154)--(-8.342,3.144)--(-8.342,3.124)--cycle;
\draw(-8.412,3.166)--(-8.37,3.168);
\draw(-8.342,3.144)--(-8.342,3.124);
\filldraw[fill opacity=0.8,fill=gray!20,draw=none](-8.384,3.177)--(-8.358,3.174)--(-8.356,3.168)--(-8.37,3.168)--cycle;
\draw(-8.356,3.168)--(-8.37,3.168);
\filldraw[fill opacity=0.8,fill=gray!20,draw=none](-8.36,3.165)--(-8.334,3.148)--(-8.351,3.155)--(-8.382,3.176)--(-8.381,3.175)--(-8.369,3.17)--cycle;
\draw(-8.334,3.148)--(-8.351,3.155);
\draw(-8.381,3.175)--(-8.369,3.17);
\filldraw[fill opacity=0.8,fill=gray!20,draw=none](-8.37,3.168)--(-8.348,3.154)--(-8.351,3.155)--cycle;
\draw(-8.348,3.154)--(-8.351,3.155);
\filldraw[fill opacity=0.8,fill=gray!20,draw=none](-8.36,3.165)--(-8.326,3.145)--(-8.334,3.148)--cycle;
\draw(-8.326,3.145)--(-8.334,3.148);
\filldraw[fill opacity=0.8,fill=gray!20,draw=none](-8.36,3.165)--(-8.334,3.148)--(-8.348,3.154)--(-8.383,3.176)--(-8.381,3.175)--(-8.369,3.17)--cycle;
\draw(-8.334,3.148)--(-8.348,3.154);
\draw(-8.381,3.175)--(-8.369,3.17);
\filldraw[fill opacity=0.8,fill=gray!20,draw=none](-8.457,3.178)--(-8.441,3.183)--(-8.412,3.166)--cycle;
\filldraw[fill opacity=0.8,fill=gray!20,draw=none](-8.384,3.177)--(-8.37,3.168)--(-8.412,3.166)--(-8.436,3.172)--(-8.432,3.183)--cycle;
\draw(-8.37,3.168)--(-8.412,3.166);
\draw(-8.436,3.172)--(-8.432,3.183);
\filldraw[fill opacity=0.8,fill=gray!20,draw=none](-8.36,3.165)--(-8.326,3.145)--(-8.334,3.148)--cycle;
\draw(-8.326,3.145)--(-8.334,3.148);
\filldraw[fill opacity=0.8,fill=gray!20,draw=none](-8.341,3.123)--(-8.327,3.115)--(-8.325,3.106)--(-8.327,3.097)--cycle;
\filldraw[fill opacity=0.8,fill=gray!20,draw=none](-8.326,3.121)--(-8.342,3.122)--(-8.342,3.144)--cycle;
\draw(-8.326,3.121)--(-8.342,3.122)--(-8.342,3.144);
\filldraw[fill opacity=0.8,fill=gray!20,draw=none](-8.341,3.123)--(-8.327,3.097)--(-8.332,3.068)--cycle;
\filldraw[fill opacity=0.8,fill=gray!20,draw=none](-8.322,3.05)--(-8.327,3.021)--(-8.341,3.01)--cycle;
\filldraw[fill opacity=0.8,fill=gray!20,draw=none](-8.315,2.974)--(-8.323,2.956)--(-8.333,2.961)--(-8.323,2.977)--cycle;
\draw(-8.323,2.956)--(-8.333,2.961);
\filldraw[fill opacity=0.8,fill=gray!20,draw=none](-8.315,2.974)--(-8.314,2.984)--(-8.308,2.99)--cycle;
\filldraw[fill opacity=0.8,fill=gray!20,draw=none](-8.222,2.878)--(-8.215,2.884)--(-8.199,2.867)--(-8.203,2.864)--cycle;
\draw(-8.222,2.878)--(-8.215,2.884)--(-8.199,2.867)--(-8.203,2.864);
\filldraw[fill opacity=0.8,fill=gray!20,draw=none](-8.256,2.846)--(-8.222,2.878)--(-8.203,2.864)--(-8.215,2.855)--cycle;
\draw(-8.256,2.846)--(-8.222,2.878);
\draw(-8.203,2.864)--(-8.215,2.855);
\filldraw[fill opacity=0.8,fill=gray!20,draw=none](-8.331,3.01)--(-8.316,3.004)--(-8.333,2.966)--(-8.348,2.972)--cycle;
\draw(-8.333,2.966)--(-8.348,2.972);
\filldraw[fill opacity=0.8,fill=gray!20,draw=none](-8.331,3.01)--(-8.316,3.004)--(-8.333,2.966)--(-8.348,2.972)--cycle;
\draw(-8.333,2.966)--(-8.348,2.972);
\filldraw[fill opacity=0.8,fill=gray!20,draw=none](-8.342,2.985)--(-8.342,3.011)--(-8.326,3.01)--cycle;
\draw(-8.342,2.985)--(-8.342,3.011)--(-8.326,3.01);
\filldraw[fill opacity=0.8,fill=gray!20,draw=none](-8.325,3.106)--(-8.318,3.069)--(-8.322,3.05)--(-8.341,3.01)--cycle;
\filldraw[fill opacity=0.8,fill=gray!20,draw=none](-8.321,3.115)--(-8.312,3.066)--(-8.332,3.068)--(-8.341,3.122)--(-8.326,3.121)--cycle;
\draw(-8.312,3.066)--(-8.332,3.068);
\draw(-8.341,3.122)--(-8.326,3.121);
\filldraw[fill opacity=0.8,fill=gray!20,draw=none](-8.342,3.167)--(-8.322,3.138)--(-8.383,3.113)--(-8.359,3.169)--cycle;
\draw(-8.383,3.113)--(-8.359,3.169);
\filldraw[fill opacity=0.8,fill=gray!20,draw=none](-8.321,3.016)--(-8.326,3.01)--(-8.341,3.011)--(-8.332,3.068)--(-8.312,3.066)--cycle;
\draw(-8.326,3.01)--(-8.341,3.011);
\draw(-8.332,3.068)--(-8.312,3.066);
\filldraw[fill opacity=0.8,fill=gray!20,draw=none](-8.378,3.092)--(-8.297,3.057)--(-8.31,3.011)--(-8.316,3.004)--(-8.361,3.023)--cycle;
\draw(-8.378,3.092)--(-8.297,3.057);
\draw(-8.316,3.004)--(-8.361,3.023);
\filldraw[fill opacity=0.8,fill=gray!20,draw=none](-8.322,3.138)--(-8.32,3.136)--(-8.337,3.096)--(-8.388,3.102)--(-8.383,3.113)--cycle;
\draw(-8.32,3.136)--(-8.337,3.096);
\draw(-8.388,3.102)--(-8.383,3.113);
\filldraw[fill opacity=0.8,fill=gray!20,draw=none](-8.296,3.108)--(-8.337,3.096)--(-8.32,3.136)--cycle;
\draw(-8.337,3.096)--(-8.32,3.136);
\filldraw[fill opacity=0.8,fill=gray!20,draw=none](-8.367,3.169)--(-8.359,3.169)--(-8.36,3.165)--cycle;
\draw(-8.359,3.169)--(-8.36,3.165);
\filldraw[fill opacity=0.8,fill=gray!20,draw=none](-8.337,3.156)--(-8.322,3.154)--(-8.306,3.147)--(-8.303,3.134)--cycle;
\draw(-8.322,3.154)--(-8.306,3.147);
\filldraw[fill opacity=0.8,fill=gray!20,draw=none](-8.303,3.134)--(-8.326,3.145)--(-8.369,3.17)--(-8.337,3.156)--cycle;
\draw(-8.303,3.134)--(-8.326,3.145);
\draw(-8.369,3.17)--(-8.337,3.156);
\filldraw[fill opacity=0.8,fill=gray!20,draw=none](-8.311,3.138)--(-8.326,3.145)--(-8.36,3.165)--(-8.367,3.169)--(-8.347,3.161)--cycle;
\draw(-8.311,3.138)--(-8.326,3.145);
\draw(-8.367,3.169)--(-8.347,3.161);
\filldraw[fill opacity=0.8,fill=gray!20,draw=none](-8.367,3.169)--(-8.36,3.165)--(-8.388,3.102)--(-8.418,3.131)--(-8.4,3.172)--cycle;
\draw(-8.36,3.165)--(-8.388,3.102);
\draw(-8.418,3.131)--(-8.4,3.172);
\filldraw[fill opacity=0.8,fill=gray!20,draw=none](-8.303,3.134)--(-8.311,3.138)--(-8.347,3.161)--(-8.337,3.156)--cycle;
\draw(-8.303,3.134)--(-8.311,3.138);
\draw(-8.347,3.161)--(-8.337,3.156);
\filldraw[fill opacity=0.8,fill=gray!20,draw=none](-8.421,3.163)--(-8.406,3.172)--(-8.4,3.172)--(-8.418,3.131)--cycle;
\draw(-8.4,3.172)--(-8.418,3.131);
\filldraw[fill opacity=0.8,fill=gray!20,draw=none](-8.391,3.165)--(-8.337,3.156)--(-8.303,3.134)--(-8.296,3.108)--(-8.421,3.163)--cycle;
\draw(-8.296,3.108)--(-8.421,3.163);
\filldraw[fill opacity=0.8,fill=gray!20,draw=none](-8.36,3.165)--(-8.369,3.17)--(-8.367,3.169)--cycle;
\draw(-8.369,3.17)--(-8.367,3.169);
\filldraw[fill opacity=0.8,fill=gray!20,draw=none](-8.271,3.12)--(-8.303,3.134)--(-8.337,3.156)--(-8.309,3.144)--cycle;
\draw(-8.337,3.156)--(-8.309,3.144)--(-8.271,3.12)--(-8.303,3.134);
\filldraw[fill opacity=0.8,fill=gray!20,draw=none](-8.271,3.12)--(-8.303,3.134)--(-8.337,3.156)--(-8.309,3.144)--cycle;
\draw(-8.337,3.156)--(-8.309,3.144)--(-8.271,3.12)--(-8.303,3.134);
\filldraw[fill opacity=0.8,fill=gray!20](-8.309,3.144)--(-8.637,3.287)--(-8.682,3.292)--(-8.354,3.149)--cycle;
\filldraw[fill opacity=0.8,fill=gray!20](-8.309,3.144)--(-8.637,3.287)--(-8.682,3.292)--(-8.354,3.149)--cycle;
\filldraw[fill opacity=0.8,fill=gray!20](-8.732,2.992)--(-8.753,3.013)--(-8.797,3.024)--(-8.763,3)--cycle;
\filldraw[fill opacity=0.5,fill=gray!20](-9.862,-.797)--(-9.891,-.69)--(-10.292,-.553)--(-10.281,-.653)--cycle;
\filldraw[fill opacity=0.5,fill=gray!20](-9.43,-.892)--(-9.891,-.69)--(-10.292,-.553)--(-9.831,-.754)--cycle;
\filldraw[fill opacity=0.8,fill=gray!20](-8.742,2.958)--(-8.767,2.984)--(-8.82,2.997)--(-8.78,2.967)--cycle;
\filldraw[fill opacity=0.8,fill=gray!20](-8.823,3.237)--(-8.797,3.277)--(-8.811,3.291)--(-8.84,3.255)--cycle;
\filldraw[fill opacity=0.8,fill=gray!20](-8.412,4.305)--(-8.419,4.36)--(-8.442,4.384)--(-8.434,4.329)--cycle;
\filldraw[fill opacity=0.8,fill=gray!20](-8.067,4.228)--(-8.033,4.272)--(-8.064,4.251)--(-8.093,4.211)--cycle;
\filldraw[fill opacity=0.5,fill=gray!20](-7.549,2.104)--(-7.579,2.153)--(-7.396,1.691)--(-7.369,1.65)--cycle;
\filldraw[fill opacity=0.8,fill=gray!20](-8.852,3.252)--(-8.82,3.3)--(-8.837,3.317)--(-8.871,3.273)--cycle;
\filldraw[fill opacity=0.8,fill=gray!20](-8.729,3.336)--(-8.682,3.342)--(-8.682,3.342)--(-8.721,3.342)--cycle;
\filldraw[fill opacity=0.8,fill=gray!20,draw=none](-8.886,2.981)--(-8.856,3.05)--(-8.868,3.063)--(-8.883,3.03)--cycle;
\draw(-8.886,2.981)--(-8.856,3.05);
\draw(-8.868,3.063)--(-8.883,3.03);
\filldraw[fill opacity=0.8,fill=gray!20](-8.254,4.164)--(-8.283,4.176)--(-8.321,4.185)--(-8.273,4.169)--cycle;
\filldraw[fill opacity=0.8,fill=gray!20](-8.738,3.371)--(-8.682,3.378)--(-8.682,3.378)--(-8.729,3.377)--cycle;
\filldraw[fill opacity=0.8,fill=gray!20,draw=none](-8.118,4.573)--(-8.124,4.578)--(-8.12,4.573)--cycle;
\draw(-8.118,4.573)--(-8.124,4.578)--(-8.12,4.573);
\filldraw[fill opacity=0.8,fill=gray!20,draw=none](-8.12,4.573)--(-8.124,4.578)--(-8.172,4.594)--(-8.166,4.588)--(-8.125,4.571)--cycle;
\draw(-8.12,4.573)--(-8.124,4.578)--(-8.172,4.594)--(-8.166,4.588)--(-8.125,4.571);
\filldraw[fill opacity=0.8,fill=gray!20,draw=none](-7.771,4.739)--(-7.791,4.742)--(-7.781,4.753)--(-7.715,4.765)--cycle;
\draw(-7.781,4.753)--(-7.715,4.765);
\filldraw[fill opacity=0.8,fill=gray!20,draw=none](-7.781,4.753)--(-7.72,4.81)--(-7.705,4.813)--(-7.715,4.765)--cycle;
\draw(-7.72,4.81)--(-7.705,4.813)--(-7.715,4.765)--(-7.781,4.753);
\filldraw[fill opacity=0.8,fill=gray!20](-7.66,4.793)--(-8.262,4.506)--(-8.293,4.491)--(-7.691,4.778)--cycle;
\filldraw[fill opacity=0.8,fill=gray!20](-8.639,3.34)--(-8.682,3.342)--(-8.682,3.342)--(-8.634,3.335)--cycle;
\filldraw[fill opacity=0.8,fill=gray!20,draw=none](-8.592,2.964)--(-8.612,2.982)--(-8.633,2.956)--cycle;
\draw(-8.612,2.982)--(-8.633,2.956)--(-8.592,2.964);
\filldraw[fill opacity=0.8,fill=gray!20](-8.332,4.568)--(-8.279,4.589)--(-8.27,4.595)--(-8.314,4.58)--cycle;
\filldraw[fill opacity=0.8,fill=gray!20,draw=none](-8.603,3.001)--(-8.603,3.001)--(-8.604,3.001)--cycle;
\filldraw[fill opacity=0.8,fill=gray!20,draw=none](-8.603,3.001)--(-8.589,3.021)--(-8.597,3.019)--(-8.604,3.001)--cycle;
\draw(-8.589,3.021)--(-8.597,3.019)--(-8.604,3.001);
\filldraw[fill opacity=0.8,fill=gray!20,draw=none](-8.71,3.039)--(-8.707,3.038)--(-8.715,3.047)--cycle;
\filldraw[fill opacity=0.8,fill=gray!20,draw=none](-8.716,3.047)--(-8.711,3.041)--(-8.698,3.04)--cycle;
\draw(-8.711,3.041)--(-8.698,3.04);
\filldraw[fill opacity=0.8,fill=gray!20,draw=none](-8.619,3.018)--(-8.597,3.019)--(-8.587,3.066)--(-8.62,3.065)--cycle;
\draw(-8.619,3.018)--(-8.597,3.019)--(-8.587,3.066)--(-8.62,3.065);
\filldraw[fill opacity=0.8,fill=gray!20,draw=none](-8.694,3.04)--(-8.694,3.062)--(-8.71,3.063)--cycle;
\draw(-8.694,3.04)--(-8.694,3.062)--(-8.71,3.063);
\filldraw[fill opacity=0.8,fill=gray!20,draw=none](-8.774,3.14)--(-8.751,3.094)--(-8.715,3.178)--(-8.757,3.196)--(-8.774,3.155)--cycle;
\draw(-8.751,3.094)--(-8.715,3.178)--(-8.757,3.196)--(-8.774,3.155);
\filldraw[fill opacity=0.8,fill=gray!20,draw=none](-8.738,3.082)--(-8.692,3.079)--(-8.693,3.123)--(-8.747,3.127)--cycle;
\draw(-8.738,3.082)--(-8.692,3.079)--(-8.693,3.123)--(-8.747,3.127);
\filldraw[fill opacity=0.8,fill=gray!20,draw=none](-8.715,3.069)--(-8.71,3.063)--(-8.694,3.062)--(-8.695,3.115)--(-8.724,3.118)--cycle;
\draw(-8.71,3.063)--(-8.694,3.062)--(-8.695,3.115)--(-8.724,3.118);
\filldraw[fill opacity=0.8,fill=gray!20,draw=none](-8.167,2.915)--(-8.171,2.906)--(-8.203,2.865)--(-8.204,2.863)--(-8.279,2.896)--(-8.279,2.898)--(-8.27,2.949)--(-8.266,2.958)--cycle;
\filldraw[fill opacity=0.8,fill=gray!20,draw=none](-8.204,2.863)--(-8.214,2.856)--(-8.257,2.846)--(-8.277,2.883)--(-8.279,2.896)--cycle;
\filldraw[fill opacity=0.8,fill=gray!20,draw=none](-8.354,2.926)--(-8.336,2.943)--(-8.218,2.892)--(-8.256,2.873)--(-8.356,2.916)--cycle;
\draw(-8.256,2.873)--(-8.356,2.916);
\filldraw[fill opacity=0.8,fill=gray!20,draw=none](-8.354,2.926)--(-8.356,2.916)--(-8.361,2.919)--cycle;
\draw(-8.356,2.916)--(-8.361,2.919);
\filldraw[fill opacity=0.8,fill=gray!20,draw=none](-8.38,2.919)--(-8.367,2.932)--(-8.361,2.929)--cycle;
\draw(-8.367,2.932)--(-8.361,2.929);
\filldraw[fill opacity=0.8,fill=gray!20,draw=none](-8.38,2.919)--(-8.367,2.932)--(-8.361,2.929)--cycle;
\draw(-8.367,2.932)--(-8.361,2.929);
\filldraw[fill opacity=0.8,fill=gray!20,draw=none](-8.361,2.911)--(-8.356,2.916)--(-8.353,2.915)--cycle;
\draw(-8.356,2.916)--(-8.353,2.915);
\filldraw[fill opacity=0.8,fill=gray!20,draw=none](-8.511,2.98)--(-8.532,3.034)--(-8.532,3.037)--(-8.331,2.949)--(-8.357,2.909)--(-8.505,2.974)--cycle;
\draw(-8.532,3.037)--(-8.331,2.949);
\draw(-8.357,2.909)--(-8.505,2.974);
\filldraw[fill opacity=0.8,fill=gray!20,draw=none](-8.531,2.989)--(-8.545,2.997)--(-8.55,3.025)--cycle;
\filldraw[fill opacity=0.8,fill=gray!20,draw=none](-8.536,2.987)--(-8.531,2.991)--(-8.523,2.965)--cycle;
\draw(-8.536,2.987)--(-8.531,2.991)--(-8.523,2.965);
\filldraw[fill opacity=0.8,fill=gray!20,draw=none](-8.532,2.99)--(-8.55,3.025)--(-8.554,3.043)--(-8.547,3.082)--cycle;
\filldraw[fill opacity=0.8,fill=gray!20,draw=none](-8.532,2.99)--(-8.536,2.987)--(-8.539,2.992)--(-8.547,3.04)--(-8.541,3.045)--cycle;
\draw(-8.532,2.99)--(-8.536,2.987);
\draw(-8.547,3.04)--(-8.541,3.045);
\filldraw[fill opacity=0.8,fill=gray!20](-8.463,2.959)--(-8.79,3.102)--(-8.765,3.063)--(-8.438,2.92)--cycle;
\filldraw[fill opacity=0.8,fill=gray!20](-8.463,2.959)--(-8.79,3.102)--(-8.765,3.063)--(-8.438,2.92)--cycle;
\filldraw[fill opacity=0.8,fill=gray!20](-8.797,3.277)--(-8.763,3.309)--(-8.773,3.319)--(-8.811,3.291)--cycle;
\filldraw[fill opacity=0.8,fill=gray!20](-8.175,4.168)--(-8.131,4.183)--(-8.173,4.174)--(-8.197,4.163)--cycle;
\filldraw[fill opacity=0.8,fill=gray!20](-8.419,4.36)--(-8.412,4.416)--(-8.434,4.44)--(-8.442,4.384)--cycle;
\filldraw[fill opacity=0.8,fill=gray!20](-8.033,4.272)--(-8.011,4.323)--(-8.045,4.3)--(-8.064,4.251)--cycle;
\filldraw[fill opacity=0.8,fill=gray!20,draw=none](-8.849,3.109)--(-8.84,3.1)--(-8.844,3.131)--cycle;
\draw(-8.849,3.109)--(-8.84,3.1)--(-8.844,3.131);
\filldraw[fill opacity=0.8,fill=gray!20](-8.82,3.3)--(-8.78,3.338)--(-8.791,3.35)--(-8.837,3.317)--cycle;
\filldraw[fill opacity=0.8,fill=gray!20](-8.219,4.601)--(-8.222,4.596)--(-8.222,4.596)--(-8.191,4.599)--cycle;
\filldraw[fill opacity=0.8,fill=gray!20](-8.248,4.6)--(-8.222,4.596)--(-8.222,4.596)--(-8.219,4.601)--cycle;
\filldraw[fill opacity=0.8,fill=gray!20](-8.753,3.013)--(-8.769,3.045)--(-8.823,3.059)--(-8.797,3.024)--cycle;
\filldraw[fill opacity=0.8,fill=gray!20,draw=none](-8.5,3.197)--(-8.513,3.219)--(-8.505,3.193)--cycle;
\draw(-8.513,3.219)--(-8.505,3.193)--(-8.5,3.197);
\filldraw[fill opacity=0.8,fill=gray!20,draw=none](-8.636,3.264)--(-8.624,3.264)--(-8.625,3.267)--cycle;
\draw(-8.636,3.264)--(-8.624,3.264)--(-8.625,3.267);
\filldraw[fill opacity=0.8,fill=gray!20,draw=none](-8.611,3.222)--(-8.624,3.264)--(-8.636,3.264)--(-8.672,3.253)--(-8.69,3.235)--(-8.691,3.218)--cycle;
\draw(-8.69,3.235)--(-8.691,3.218)--(-8.611,3.222)--(-8.624,3.264)--(-8.636,3.264);
\filldraw[fill opacity=0.8,fill=gray!20,draw=none](-8.666,3.23)--(-8.597,3.233)--(-8.604,3.258)--(-8.652,3.245)--cycle;
\draw(-8.666,3.23)--(-8.597,3.233)--(-8.604,3.258);
\filldraw[fill opacity=0.8,fill=gray!20](-8.603,3.175)--(-8.611,3.222)--(-8.691,3.218)--(-8.692,3.171)--cycle;
\filldraw[fill opacity=0.8,fill=gray!20,draw=none](-8.62,3.176)--(-8.587,3.177)--(-8.597,3.233)--(-8.619,3.232)--cycle;
\draw(-8.62,3.176)--(-8.587,3.177)--(-8.597,3.233)--(-8.619,3.232);
\filldraw[fill opacity=0.8,fill=gray!20,draw=none](-8.738,3.178)--(-8.738,3.174)--(-8.692,3.171)--(-8.691,3.218)--(-8.711,3.22)--cycle;
\draw(-8.738,3.174)--(-8.692,3.171)--(-8.691,3.218)--(-8.711,3.22);
\filldraw[fill opacity=0.8,fill=gray!20,draw=none](-8.62,3.176)--(-8.619,3.232)--(-8.666,3.23)--(-8.688,3.212)--(-8.694,3.199)--(-8.694,3.173)--cycle;
\draw(-8.619,3.232)--(-8.666,3.23);
\draw(-8.694,3.199)--(-8.694,3.173)--(-8.62,3.176);
\filldraw[fill opacity=0.8,fill=gray!20,draw=none](-8.421,3.163)--(-8.296,3.108)--(-8.293,3.102)--(-8.296,3.059)--(-8.297,3.057)--(-8.416,3.109)--cycle;
\draw(-8.421,3.163)--(-8.296,3.108);
\draw(-8.297,3.057)--(-8.416,3.109);
\filldraw[fill opacity=0.8,fill=gray!20,draw=none](-8.46,3.161)--(-8.443,3.174)--(-8.436,3.172)--cycle;
\filldraw[fill opacity=0.8,fill=gray!20,draw=none](-8.46,3.161)--(-8.516,3.133)--(-8.509,3.146)--(-8.481,3.17)--(-8.457,3.178)--(-8.443,3.174)--cycle;
\filldraw[fill opacity=0.8,fill=gray!20,draw=none](-8.46,3.161)--(-8.492,3.154)--(-8.474,3.167)--(-8.432,3.183)--(-8.436,3.172)--cycle;
\draw(-8.46,3.161)--(-8.492,3.154);
\draw(-8.432,3.183)--(-8.436,3.172);
\filldraw[fill opacity=0.8,fill=gray!20,draw=none](-8.451,3.157)--(-8.421,3.163)--(-8.416,3.109)--(-8.486,3.139)--cycle;
\draw(-8.416,3.109)--(-8.486,3.139);
\filldraw[fill opacity=0.8,fill=gray!20,draw=none](-8.499,3.129)--(-8.486,3.139)--(-8.378,3.092)--(-8.361,3.023)--(-8.524,3.095)--cycle;
\draw(-8.486,3.139)--(-8.378,3.092);
\draw(-8.361,3.023)--(-8.524,3.095);
\filldraw[fill opacity=0.8,fill=gray!20,draw=none](-8.421,3.163)--(-8.418,3.131)--(-8.449,3.062)--(-8.47,3.072)--(-8.433,3.155)--cycle;
\draw(-8.418,3.131)--(-8.449,3.062)--(-8.47,3.072)--(-8.433,3.155);
\filldraw[fill opacity=0.8,fill=gray!20,draw=none](-8.436,3.15)--(-8.433,3.155)--(-8.47,3.072)--(-8.474,3.074)--(-8.455,3.117)--cycle;
\draw(-8.433,3.155)--(-8.47,3.072)--(-8.474,3.074)--(-8.455,3.117);
\filldraw[fill opacity=0.8,fill=gray!20](-8.354,3.149)--(-8.682,3.292)--(-8.727,3.277)--(-8.399,3.134)--cycle;
\filldraw[fill opacity=0.8,fill=gray!20](-8.354,3.149)--(-8.682,3.292)--(-8.727,3.277)--(-8.399,3.134)--cycle;
\filldraw[fill opacity=0.8,fill=gray!20](-8.763,3.309)--(-8.724,3.331)--(-8.729,3.336)--(-8.773,3.319)--cycle;
\filldraw[fill opacity=0.5,fill=gray!20](-7.746,1.555)--(-7.573,1.479)--(-7.523,1.068)--(-7.696,1.143)--cycle;
\filldraw[fill opacity=0.8,fill=gray!20,draw=none](-8.562,3.262)--(-8.535,3.265)--(-8.552,3.288)--(-8.572,3.275)--cycle;
\draw(-8.535,3.265)--(-8.552,3.288)--(-8.572,3.275);
\filldraw[fill opacity=0.8,fill=gray!20,draw=none](-8.487,3.257)--(-8.492,3.268)--(-8.516,3.252)--cycle;
\draw(-8.487,3.257)--(-8.492,3.268)--(-8.516,3.252);
\filldraw[fill opacity=0.8,fill=gray!20](-8.27,4.595)--(-8.222,4.596)--(-8.222,4.596)--(-8.248,4.6)--cycle;
\filldraw[fill opacity=0.8,fill=gray!20,draw=none](-8.764,3.133)--(-8.768,3.137)--(-8.799,3.15)--(-8.79,3.102)--(-8.747,3.083)--cycle;
\draw(-8.768,3.137)--(-8.799,3.15)--(-8.79,3.102)--(-8.747,3.083);
\filldraw[fill opacity=0.8,fill=gray!20,draw=none](-8.764,3.133)--(-8.768,3.137)--(-8.799,3.15)--(-8.79,3.102)--(-8.747,3.083)--cycle;
\draw(-8.768,3.137)--(-8.799,3.15)--(-8.79,3.102)--(-8.747,3.083);
\filldraw[fill opacity=0.8,fill=gray!20,draw=none](-8.769,3.045)--(-8.774,3.066)--(-8.804,3.088)--(-8.837,3.094)--(-8.823,3.059)--cycle;
\draw(-8.837,3.094)--(-8.823,3.059)--(-8.769,3.045)--(-8.774,3.066);
\filldraw[fill opacity=0.8,fill=gray!20,draw=none](-8.774,3.066)--(-8.763,3.067)--(-8.753,3.09)--(-8.788,3.123)--(-8.804,3.088)--cycle;
\draw(-8.763,3.067)--(-8.753,3.09);
\draw(-8.788,3.123)--(-8.804,3.088);
\filldraw[fill opacity=0.8,fill=gray!20,draw=none](-8.713,3.041)--(-8.744,3.059)--(-8.776,3.072)--(-8.769,3.045)--cycle;
\draw(-8.776,3.072)--(-8.769,3.045)--(-8.713,3.041);
\filldraw[fill opacity=0.8,fill=gray!20,draw=none](-8.797,3.024)--(-8.784,3.017)--(-8.766,3.059)--(-8.774,3.066)--(-8.815,3.061)--(-8.823,3.044)--cycle;
\draw(-8.784,3.017)--(-8.766,3.059);
\draw(-8.815,3.061)--(-8.823,3.044);
\filldraw[fill opacity=0.8,fill=gray!20](-8.767,2.984)--(-8.786,3.022)--(-8.852,3.038)--(-8.82,2.997)--cycle;
\filldraw[fill opacity=0.8,fill=gray!20](-8.78,3.338)--(-8.732,3.365)--(-8.738,3.371)--(-8.791,3.35)--cycle;
\filldraw[fill opacity=0.8,fill=gray!20](-8.492,3.268)--(-8.527,3.313)--(-8.552,3.296)--(-8.523,3.248)--cycle;
\filldraw[fill opacity=0.8,fill=gray!20,draw=none](-8.649,2.92)--(-8.639,2.944)--(-8.682,2.904)--(-8.683,2.902)--cycle;
\draw(-8.649,2.92)--(-8.639,2.944);
\draw(-8.682,2.904)--(-8.683,2.902);
\filldraw[fill opacity=0.8,fill=gray!20,draw=none](-8.639,2.944)--(-8.635,2.953)--(-8.638,2.954)--(-8.661,2.951)--(-8.682,2.904)--cycle;
\draw(-8.639,2.944)--(-8.635,2.953);
\draw(-8.661,2.951)--(-8.682,2.904);
\filldraw[fill opacity=0.8,fill=gray!20,draw=none](-8.635,2.953)--(-8.634,2.954)--(-8.638,2.954)--cycle;
\draw(-8.635,2.953)--(-8.634,2.954);
\filldraw[fill opacity=0.8,fill=gray!20](-8.724,3.331)--(-8.682,3.342)--(-8.682,3.342)--(-8.729,3.336)--cycle;
\filldraw[fill opacity=0.8,fill=gray!20](-8.552,3.288)--(-8.59,3.317)--(-8.605,3.307)--(-8.574,3.274)--cycle;
\filldraw[fill opacity=0.8,fill=gray!20](-8.412,4.416)--(-8.393,4.47)--(-8.412,4.491)--(-8.434,4.44)--cycle;
\filldraw[fill opacity=0.8,fill=gray!20](-8.226,4.162)--(-8.229,4.172)--(-8.283,4.176)--(-8.254,4.164)--cycle;
\filldraw[fill opacity=0.8,fill=gray!20](-8.191,4.599)--(-8.222,4.596)--(-8.222,4.596)--(-8.172,4.594)--cycle;
\filldraw[fill opacity=0.8,fill=gray!20](-8.609,4.036)--(-8.639,4.05)--(-8.86,3.464)--cycle;
\filldraw[fill opacity=0.8,fill=gray!20](-8.86,3.464)--(-8.83,3.45)--(-8.609,4.036)--cycle;
\filldraw[fill opacity=0.8,fill=gray!20](-8.708,3.327)--(-8.682,3.342)--(-8.682,3.342)--(-8.724,3.331)--cycle;
\filldraw[fill opacity=0.8,fill=gray!20](-8.687,2.989)--(-8.689,3.009)--(-8.753,3.013)--(-8.732,2.992)--cycle;
\filldraw[fill opacity=0.8,fill=gray!20](-8.684,3.325)--(-8.682,3.342)--(-8.682,3.342)--(-8.708,3.327)--cycle;
\filldraw[fill opacity=0.8,fill=gray!20](-8.66,3.327)--(-8.682,3.342)--(-8.682,3.342)--(-8.684,3.325)--cycle;
\filldraw[fill opacity=0.8,fill=gray!20](-8.642,3.33)--(-8.682,3.342)--(-8.682,3.342)--(-8.66,3.327)--cycle;
\filldraw[fill opacity=0.8,fill=gray!20](-8.634,3.335)--(-8.682,3.342)--(-8.682,3.342)--(-8.642,3.33)--cycle;
\filldraw[fill opacity=0.8,fill=gray!20](-8.59,3.317)--(-8.634,3.335)--(-8.642,3.33)--(-8.605,3.307)--cycle;
\filldraw[fill opacity=0.8,fill=gray!20](-8.197,4.163)--(-8.173,4.174)--(-8.229,4.172)--(-8.226,4.162)--cycle;
\filldraw[fill opacity=0.8,fill=gray!20,draw=none](-8.646,2.956)--(-8.655,2.966)--(-8.691,2.976)--(-8.688,2.954)--cycle;
\draw(-8.691,2.976)--(-8.688,2.954)--(-8.646,2.956);
\filldraw[fill opacity=0.8,fill=gray!20,draw=none](-8.638,2.954)--(-8.658,2.959)--(-8.661,2.951)--cycle;
\draw(-8.658,2.959)--(-8.661,2.951);
\filldraw[fill opacity=0.5,fill=gray!20](-7.395,.612)--(-7.369,.62)--(-7.543,.143)--(-7.561,.157)--cycle;
\filldraw[fill opacity=0.5,fill=gray!20](-7.561,.157)--(-7.543,.143)--(-7.819,-.28)--(-7.825,-.247)--cycle;
\filldraw[fill opacity=0.8,fill=gray!20](-8.732,3.365)--(-8.682,3.378)--(-8.682,3.378)--(-8.738,3.371)--cycle;
\filldraw[fill opacity=0.8,fill=gray!20](-8.527,3.313)--(-8.572,3.347)--(-8.59,3.336)--(-8.552,3.296)--cycle;
\filldraw[fill opacity=0.8,fill=gray!20](-8.283,4.176)--(-8.308,4.202)--(-8.361,4.215)--(-8.321,4.185)--cycle;
\filldraw[fill opacity=0.8,fill=gray!20,draw=none](-8.722,3.661)--(-8.83,3.45)--(-8.766,3.423)--(-8.682,3.644)--cycle;
\draw(-8.722,3.661)--(-8.83,3.45)--(-8.766,3.423)--(-8.682,3.644);
\filldraw[fill opacity=0.8,fill=gray!20,draw=none](-8.722,3.661)--(-8.747,3.671)--(-8.83,3.45)--cycle;
\draw(-8.747,3.671)--(-8.83,3.45)--(-8.722,3.661);
\filldraw[fill opacity=0.8,fill=gray!20,draw=none](-8.709,3.653)--(-8.715,3.655)--(-8.788,3.685)--(-8.776,3.685)--cycle;
\draw(-8.709,3.653)--(-8.715,3.655)--(-8.788,3.685);
\filldraw[fill opacity=0.8,fill=gray!20,draw=none](-8.041,4.495)--(-8.058,4.518)--(-8.07,4.533)--cycle;
\filldraw[fill opacity=0.8,fill=gray!20](-8.713,3.36)--(-8.682,3.378)--(-8.682,3.378)--(-8.732,3.365)--cycle;
\filldraw[fill opacity=0.8,fill=gray!20,draw=none](-8.691,2.976)--(-8.655,2.966)--(-8.641,2.997)--(-8.65,3.003)--(-8.688,2.988)--(-8.693,2.977)--cycle;
\draw(-8.655,2.966)--(-8.641,2.997);
\draw(-8.688,2.988)--(-8.693,2.977);
\filldraw[fill opacity=0.8,fill=gray!20](-8.688,2.954)--(-8.691,2.978)--(-8.767,2.984)--(-8.742,2.958)--cycle;
\filldraw[fill opacity=0.8,fill=gray!20](-8.685,3.358)--(-8.682,3.378)--(-8.682,3.378)--(-8.713,3.36)--cycle;
\filldraw[fill opacity=0.8,fill=gray!20](-8.656,3.359)--(-8.682,3.378)--(-8.682,3.378)--(-8.685,3.358)--cycle;
\filldraw[fill opacity=0.8,fill=gray!20](-8.634,3.363)--(-8.682,3.378)--(-8.682,3.378)--(-8.656,3.359)--cycle;
\filldraw[fill opacity=0.8,fill=gray!20,draw=none](-8.711,3.221)--(-8.711,3.22)--(-8.691,3.218)--(-8.69,3.235)--cycle;
\draw(-8.711,3.22)--(-8.691,3.218)--(-8.69,3.235);
\filldraw[fill opacity=0.8,fill=gray!20](-8.603,3.083)--(-8.6,3.128)--(-8.693,3.123)--(-8.692,3.079)--cycle;
\filldraw[fill opacity=0.8,fill=gray!20](-8.6,3.128)--(-8.603,3.175)--(-8.692,3.171)--(-8.693,3.123)--cycle;
\filldraw[fill opacity=0.8,fill=gray!20,draw=none](-8.626,3.119)--(-8.583,3.12)--(-8.587,3.177)--(-8.625,3.176)--cycle;
\draw(-8.626,3.119)--(-8.583,3.12)--(-8.587,3.177)--(-8.625,3.176);
\filldraw[fill opacity=0.8,fill=gray!20,draw=none](-8.516,3.133)--(-8.46,3.161)--(-8.531,3.103)--cycle;
\filldraw[fill opacity=0.8,fill=gray!20,draw=none](-8.516,3.133)--(-8.531,3.103)--(-8.545,3.091)--(-8.545,3.094)--(-8.525,3.129)--cycle;
\filldraw[fill opacity=0.8,fill=gray!20,draw=none](-8.531,3.103)--(-8.523,3.124)--(-8.515,3.136)--(-8.492,3.154)--(-8.46,3.161)--cycle;
\draw(-8.531,3.103)--(-8.523,3.124);
\draw(-8.492,3.154)--(-8.46,3.161);
\filldraw[fill opacity=0.8,fill=gray!20,draw=none](-8.388,3.102)--(-8.413,3.046)--(-8.449,3.062)--(-8.418,3.131)--cycle;
\draw(-8.388,3.102)--(-8.413,3.046)--(-8.449,3.062)--(-8.418,3.131);
\filldraw[fill opacity=0.8,fill=gray!20,draw=none](-8.516,3.133)--(-8.525,3.129)--(-8.519,3.138)--(-8.509,3.146)--cycle;
\filldraw[fill opacity=0.8,fill=gray!20,draw=none](-8.452,3.109)--(-8.455,3.117)--(-8.474,3.074)--(-8.462,3.069)--cycle;
\draw(-8.455,3.117)--(-8.474,3.074)--(-8.462,3.069);
\filldraw[fill opacity=0.8,fill=gray!20](-8.399,3.134)--(-8.727,3.277)--(-8.765,3.245)--(-8.438,3.102)--cycle;
\filldraw[fill opacity=0.8,fill=gray!20](-8.399,3.134)--(-8.727,3.277)--(-8.765,3.245)--(-8.438,3.102)--cycle;
\filldraw[fill opacity=0.8,fill=gray!20](-8.393,4.47)--(-8.361,4.518)--(-8.377,4.535)--(-8.412,4.491)--cycle;
\filldraw[fill opacity=0.8,fill=gray!20,draw=none](-9.277,-1.022)--(-9.277,-1.138)--(-9.22,-1.094)--(-9.22,-1.02)--cycle;
\draw(-9.277,-1.022)--(-9.277,-1.138);
\draw(-9.22,-1.094)--(-9.22,-1.02);
\filldraw[fill opacity=0.8,fill=gray!20,draw=none](-9.285,-1.117)--(-9.277,-1.138)--(-9.22,-1.094)--cycle;
\filldraw[fill opacity=0.8,fill=gray!20,draw=none](-9.285,-1.117)--(-9.22,-1.094)--(-9.291,-1.101)--cycle;
\draw(-9.22,-1.094)--(-9.291,-1.101);
\filldraw[fill opacity=0.8,fill=gray!20,draw=none](-9.274,-1.109)--(-9.433,-1.039)--(-9.405,-1.083)--(-9.277,-1.138)--cycle;
\draw(-9.274,-1.109)--(-9.433,-1.039)--(-9.405,-1.083)--(-9.277,-1.138);
\filldraw[fill opacity=0.8,fill=gray!20](-8.827,3.704)--(-8.792,3.687)--(-8.571,4.273)--cycle;
\filldraw[fill opacity=0.8,fill=gray!20](-8.571,4.273)--(-8.606,4.29)--(-8.827,3.704)--cycle;
\filldraw[fill opacity=0.8,fill=gray!20](-8.279,4.589)--(-8.222,4.596)--(-8.222,4.596)--(-8.27,4.595)--cycle;
\filldraw[fill opacity=0.8,fill=gray!20,draw=none](-8.12,4.573)--(-8.125,4.571)--(-8.113,4.566)--cycle;
\draw(-8.125,4.571)--(-8.113,4.566)--(-8.12,4.573);
\filldraw[fill opacity=0.8,fill=gray!20,draw=none](-8.036,4.49)--(-8.041,4.495)--(-8.07,4.533)--(-8.067,4.531)--(-8.033,4.486)--cycle;
\draw(-8.07,4.533)--(-8.067,4.531)--(-8.033,4.486)--(-8.036,4.49);
\filldraw[fill opacity=0.8,fill=gray!20,draw=none](-9.341,-1.118)--(-9.341,-1.165)--(-9.369,-1.176)--(-9.369,-1.098)--cycle;
\draw(-9.341,-1.118)--(-9.341,-1.165)--(-9.369,-1.176)--(-9.369,-1.098);
\filldraw[fill opacity=0.8,fill=gray!20,draw=none](-8.646,2.956)--(-8.633,2.956)--(-8.612,2.982)--(-8.666,2.979)--cycle;
\draw(-8.646,2.956)--(-8.633,2.956)--(-8.612,2.982)--(-8.666,2.979);
\filldraw[fill opacity=0.8,fill=gray!20,draw=none](-8.016,4.464)--(-8.014,4.451)--(-8.018,4.449)--(-8.036,4.49)--cycle;
\draw(-8.014,4.451)--(-8.018,4.449);
\filldraw[fill opacity=0.8,fill=gray!20,draw=none](-8.018,4.449)--(-8.036,4.49)--(-8.033,4.486)--(-8.012,4.437)--cycle;
\draw(-8.036,4.49)--(-8.033,4.486)--(-8.012,4.437);
\filldraw[fill opacity=0.8,fill=gray!20](-8.131,4.183)--(-8.093,4.211)--(-8.153,4.2)--(-8.173,4.174)--cycle;
\filldraw[fill opacity=0.8,fill=gray!20,draw=none](-8.016,4.464)--(-8.036,4.49)--(-8.039,4.496)--(-8.021,4.505)--cycle;
\draw(-8.039,4.496)--(-8.021,4.505);
\filldraw[fill opacity=0.8,fill=gray!20,draw=none](-8.067,4.531)--(-8.084,4.543)--(-8.109,4.556)--(-8.123,4.559)--(-8.131,4.554)--(-8.093,4.514)--cycle;
\draw(-8.123,4.559)--(-8.131,4.554)--(-8.093,4.514)--(-8.067,4.531)--(-8.084,4.543);
\filldraw[fill opacity=0.8,fill=gray!20,draw=none](-8.092,4.564)--(-8.105,4.558)--(-8.125,4.571)--(-8.104,4.581)--cycle;
\draw(-8.092,4.564)--(-8.105,4.558);
\draw(-8.125,4.571)--(-8.104,4.581);
\filldraw[fill opacity=0.8,fill=gray!20,draw=none](-8.123,4.559)--(-8.143,4.561)--(-8.131,4.554)--cycle;
\draw(-8.143,4.561)--(-8.131,4.554)--(-8.123,4.559);
\filldraw[fill opacity=0.8,fill=gray!20,draw=none](-8.145,4.561)--(-8.143,4.561)--(-8.144,4.562)--cycle;
\draw(-8.143,4.561)--(-8.144,4.562);
\filldraw[fill opacity=0.8,fill=gray!20,draw=none](-8.131,4.554)--(-8.143,4.561)--(-8.145,4.561)--(-8.175,4.548)--(-8.173,4.546)--cycle;
\draw(-8.175,4.548)--(-8.173,4.546)--(-8.131,4.554)--(-8.143,4.561);
\filldraw[fill opacity=0.8,fill=gray!20](-8.093,4.514)--(-8.131,4.554)--(-8.173,4.546)--(-8.153,4.503)--cycle;
\filldraw[fill opacity=0.8,fill=gray!20,draw=none](-8.176,4.545)--(-8.173,4.546)--(-8.175,4.548)--cycle;
\draw(-8.176,4.545)--(-8.173,4.546)--(-8.175,4.548);
\filldraw[fill opacity=0.8,fill=gray!20,draw=none](-8.199,4.515)--(-8.199,4.501)--(-8.153,4.503)--(-8.173,4.546)--(-8.176,4.545)--cycle;
\draw(-8.199,4.501)--(-8.153,4.503)--(-8.173,4.546)--(-8.176,4.545);
\filldraw[fill opacity=0.8,fill=gray!20](-7.515,4.596)--(-7.503,4.652)--(-7.43,4.634)--(-7.45,4.58)--cycle;
\filldraw[fill opacity=0.8,fill=gray!20,draw=none](-7.576,4.66)--(-7.577,4.659)--(-7.578,4.661)--cycle;
\draw(-7.576,4.66)--(-7.577,4.659);
\filldraw[fill opacity=0.8,fill=gray!20,draw=none](-7.549,4.606)--(-7.574,4.624)--(-7.577,4.665)--(-7.574,4.671)--(-7.527,4.652)--(-7.544,4.609)--cycle;
\draw(-7.577,4.665)--(-7.574,4.671)--(-7.527,4.652)--(-7.544,4.609);
\filldraw[fill opacity=0.8,fill=gray!20,draw=none](-7.584,4.675)--(-7.578,4.665)--(-7.581,4.666)--(-7.586,4.675)--cycle;
\filldraw[fill opacity=0.8,fill=gray!20,draw=none](-7.568,4.619)--(-7.581,4.629)--(-7.577,4.659)--(-7.573,4.661)--cycle;
\draw(-7.577,4.659)--(-7.573,4.661);
\filldraw[fill opacity=0.8,fill=gray!20,draw=none](-7.574,4.624)--(-7.589,4.634)--(-7.577,4.665)--cycle;
\draw(-7.589,4.634)--(-7.577,4.665);
\filldraw[fill opacity=0.8,fill=gray!20,draw=none](-7.578,4.665)--(-7.576,4.66)--(-7.578,4.661)--(-7.581,4.666)--cycle;
\filldraw[fill opacity=0.8,fill=gray!20,draw=none](-7.584,4.675)--(-7.578,4.665)--(-7.591,4.635)--(-7.64,4.666)--(-7.629,4.692)--cycle;
\draw(-7.578,4.665)--(-7.591,4.635);
\draw(-7.64,4.666)--(-7.629,4.692);
\filldraw[fill opacity=0.8,fill=gray!20,draw=none](-7.584,4.675)--(-7.575,4.671)--(-7.578,4.665)--cycle;
\draw(-7.575,4.671)--(-7.578,4.665);
\filldraw[fill opacity=0.8,fill=gray!20](-7.499,4.709)--(-7.503,4.763)--(-7.43,4.745)--(-7.424,4.69)--cycle;
\filldraw[fill opacity=0.8,fill=gray!20,draw=none](-7.476,4.772)--(-7.444,4.748)--(-7.503,4.763)--(-7.507,4.778)--cycle;
\draw(-7.444,4.748)--(-7.503,4.763)--(-7.507,4.778);
\filldraw[fill opacity=0.8,fill=gray!20,draw=none](-7.476,4.772)--(-7.507,4.778)--(-7.513,4.799)--cycle;
\draw(-7.507,4.778)--(-7.513,4.799);
\filldraw[fill opacity=0.8,fill=gray!20,draw=none](-7.518,4.803)--(-7.52,4.799)--(-7.555,4.813)--cycle;
\draw(-7.518,4.803)--(-7.52,4.799);
\filldraw[fill opacity=0.8,fill=gray!20,draw=none](-7.504,4.817)--(-7.49,4.802)--(-7.513,4.799)--(-7.518,4.803)--(-7.51,4.82)--cycle;
\draw(-7.518,4.803)--(-7.51,4.82)--(-7.504,4.817);
\filldraw[fill opacity=0.8,fill=gray!20,draw=none](-7.513,4.799)--(-7.52,4.799)--(-7.518,4.803)--cycle;
\draw(-7.52,4.799)--(-7.518,4.803);
\filldraw[fill opacity=0.8,fill=gray!20,draw=none](-7.555,4.813)--(-7.515,4.81)--(-7.512,4.795)--cycle;
\draw(-7.555,4.813)--(-7.515,4.81)--(-7.512,4.795);
\filldraw[fill opacity=0.8,fill=gray!20,draw=none](-7.578,4.665)--(-7.573,4.661)--(-7.576,4.66)--cycle;
\draw(-7.573,4.661)--(-7.576,4.66);
\filldraw[fill opacity=0.8,fill=gray!20,draw=none](-7.528,4.652)--(-7.547,4.61)--(-7.552,4.607)--(-7.591,4.635)--(-7.575,4.671)--cycle;
\draw(-7.528,4.652)--(-7.547,4.61);
\draw(-7.591,4.635)--(-7.575,4.671);
\filldraw[fill opacity=0.8,fill=gray!20,draw=none](-7.578,4.665)--(-7.584,4.675)--(-7.563,4.666)--(-7.573,4.661)--cycle;
\draw(-7.563,4.666)--(-7.573,4.661);
\filldraw[fill opacity=0.8,fill=gray!20,draw=none](-7.544,4.609)--(-7.527,4.652)--(-7.508,4.644)--cycle;
\draw(-7.544,4.609)--(-7.527,4.652)--(-7.508,4.644);
\filldraw[fill opacity=0.8,fill=gray!20](-7.503,4.652)--(-7.499,4.709)--(-7.424,4.69)--(-7.43,4.634)--cycle;
\filldraw[fill opacity=0.8,fill=gray!20,draw=none](-7.509,4.644)--(-7.547,4.61)--(-7.528,4.652)--cycle;
\draw(-7.547,4.61)--(-7.528,4.652);
\filldraw[fill opacity=0.8,fill=gray!20,draw=none](-7.526,4.651)--(-7.525,4.647)--(-7.547,4.622)--(-7.562,4.614)--(-7.568,4.619)--(-7.573,4.661)--(-7.563,4.666)--cycle;
\draw(-7.526,4.651)--(-7.525,4.647);
\draw(-7.547,4.622)--(-7.562,4.614);
\draw(-7.573,4.661)--(-7.563,4.666);
\filldraw[fill opacity=0.8,fill=gray!20,draw=none](-7.531,4.59)--(-7.546,4.604)--(-7.544,4.609)--(-7.508,4.644)--(-7.493,4.637)--(-7.508,4.6)--cycle;
\draw(-7.546,4.604)--(-7.544,4.609);
\draw(-7.508,4.644)--(-7.493,4.637)--(-7.508,4.6);
\filldraw[fill opacity=0.8,fill=gray!20,draw=none](-7.549,4.606)--(-7.544,4.609)--(-7.546,4.604)--cycle;
\draw(-7.544,4.609)--(-7.546,4.604);
\filldraw[fill opacity=0.8,fill=gray!20,draw=none](-7.509,4.644)--(-7.494,4.638)--(-7.51,4.601)--(-7.534,4.592)--(-7.549,4.605)--(-7.547,4.61)--cycle;
\draw(-7.494,4.638)--(-7.51,4.601);
\draw(-7.549,4.605)--(-7.547,4.61);
\filldraw[fill opacity=0.8,fill=gray!20,draw=none](-7.525,4.647)--(-7.523,4.633)--(-7.547,4.622)--cycle;
\draw(-7.525,4.647)--(-7.523,4.633)--(-7.547,4.622);
\filldraw[fill opacity=0.8,fill=gray!20,draw=none](-7.775,4.548)--(-7.789,4.547)--(-7.784,4.56)--cycle;
\draw(-7.789,4.547)--(-7.784,4.56);
\filldraw[fill opacity=0.8,fill=gray!20,draw=none](-7.789,4.57)--(-7.799,4.547)--(-7.822,4.557)--(-7.813,4.579)--cycle;
\draw(-7.789,4.57)--(-7.799,4.547)--(-7.822,4.557)--(-7.813,4.579);
\filldraw[fill opacity=0.8,fill=gray!20,draw=none](-7.796,4.567)--(-7.801,4.57)--(-7.797,4.573)--(-7.79,4.572)--(-7.781,4.566)--cycle;
\draw(-7.801,4.57)--(-7.797,4.573);
\filldraw[fill opacity=0.8,fill=gray!20,draw=none](-7.612,4.865)--(-7.62,4.848)--(-7.662,4.859)--(-7.652,4.882)--cycle;
\draw(-7.662,4.859)--(-7.652,4.882)--(-7.612,4.865)--(-7.62,4.848);
\filldraw[fill opacity=0.8,fill=gray!20,draw=none](-7.704,4.763)--(-7.719,4.764)--(-7.713,4.767)--cycle;
\draw(-7.719,4.764)--(-7.713,4.767);
\filldraw[fill opacity=0.8,fill=gray!20,draw=none](-7.79,4.572)--(-7.794,4.575)--(-7.793,4.575)--(-7.788,4.572)--cycle;
\draw(-7.794,4.575)--(-7.793,4.575);
\filldraw[fill opacity=0.8,fill=gray!20,draw=none](-7.797,4.573)--(-7.794,4.575)--(-7.79,4.572)--cycle;
\draw(-7.797,4.573)--(-7.794,4.575);
\filldraw[fill opacity=0.8,fill=gray!20,draw=none](-7.713,4.767)--(-7.704,4.763)--(-7.788,4.572)--(-7.797,4.573)--(-7.813,4.579)--(-7.735,4.757)--cycle;
\draw(-7.704,4.763)--(-7.788,4.572);
\draw(-7.813,4.579)--(-7.735,4.757);
\filldraw[fill opacity=0.8,fill=gray!20,draw=none](-7.796,4.567)--(-7.735,4.732)--(-7.719,4.726)--(-7.781,4.566)--cycle;
\draw(-7.735,4.732)--(-7.719,4.726)--(-7.781,4.566);
\filldraw[fill opacity=0.8,fill=gray!20,draw=none](-8.011,4.374)--(-8.011,4.434)--(-8.045,4.411)--(-8.039,4.355)--cycle;
\draw(-8.011,4.434)--(-8.045,4.411)--(-8.039,4.355)--(-8.011,4.374);
\filldraw[fill opacity=0.8,fill=gray!20,draw=none](-8.013,4.385)--(-7.998,4.392)--(-7.989,4.371)--(-8.018,4.357)--cycle;
\draw(-7.989,4.371)--(-8.018,4.357);
\filldraw[fill opacity=0.8,fill=gray!20,draw=none](-8.038,4.326)--(-8.021,4.356)--(-8.016,4.37)--(-8.039,4.355)--(-8.043,4.324)--cycle;
\draw(-8.016,4.37)--(-8.039,4.355)--(-8.043,4.324);
\filldraw[fill opacity=0.8,fill=gray!20,draw=none](-7.937,4.374)--(-8.038,4.326)--(-8.021,4.356)--(-7.952,4.388)--cycle;
\draw(-7.937,4.374)--(-8.038,4.326);
\draw(-8.021,4.356)--(-7.952,4.388);
\filldraw[fill opacity=0.8,fill=gray!20,draw=none](-8.107,4.316)--(-8.071,4.311)--(-8.043,4.324)--(-8.039,4.355)--(-8.124,4.339)--(-8.125,4.333)--cycle;
\draw(-8.043,4.324)--(-8.039,4.355)--(-8.124,4.339)--(-8.125,4.333);
\filldraw[fill opacity=0.8,fill=gray!20,draw=none](-7.908,4.411)--(-8.069,4.334)--(-8.049,4.321)--(-7.837,4.422)--cycle;
\draw(-7.908,4.411)--(-8.069,4.334);
\draw(-8.049,4.321)--(-7.837,4.422);
\filldraw[fill opacity=0.8,fill=gray!20,draw=none](-7.746,4.524)--(-7.754,4.527)--(-7.781,4.539)--(-7.754,4.527)--(-7.751,4.526)--cycle;
\draw(-7.754,4.527)--(-7.751,4.526);
\filldraw[fill opacity=0.8,fill=gray!20,draw=none](-7.754,4.527)--(-7.774,4.48)--(-7.787,4.532)--(-7.757,4.529)--cycle;
\filldraw[fill opacity=0.8,fill=gray!20,draw=none](-7.754,4.527)--(-7.757,4.529)--(-7.754,4.528)--cycle;
\filldraw[fill opacity=0.8,fill=gray!20,draw=none](-7.787,4.542)--(-7.757,4.529)--(-7.795,4.532)--(-7.791,4.541)--cycle;
\draw(-7.795,4.532)--(-7.791,4.541);
\filldraw[fill opacity=0.8,fill=gray!20,draw=none](-7.814,4.554)--(-7.808,4.568)--(-7.781,4.566)--(-7.79,4.543)--cycle;
\draw(-7.814,4.554)--(-7.808,4.568);
\draw(-7.781,4.566)--(-7.79,4.543);
\filldraw[fill opacity=0.8,fill=gray!20,draw=none](-7.64,4.801)--(-7.658,4.76)--(-7.704,4.763)--(-7.7,4.773)--cycle;
\draw(-7.64,4.801)--(-7.658,4.76);
\draw(-7.704,4.763)--(-7.7,4.773);
\filldraw[fill opacity=0.8,fill=gray!20,draw=none](-7.827,4.68)--(-7.822,4.694)--(-7.805,4.724)--(-7.8,4.726)--(-7.803,4.695)--cycle;
\draw(-7.8,4.726)--(-7.803,4.695)--(-7.827,4.68);
\filldraw[fill opacity=0.8,fill=gray!20](-7.779,4.799)--(-7.75,4.839)--(-7.69,4.85)--(-7.705,4.813)--cycle;
\filldraw[fill opacity=0.8,fill=gray!20,draw=none](-7.656,4.852)--(-7.69,4.85)--(-7.673,4.871)--cycle;
\draw(-7.656,4.852)--(-7.69,4.85)--(-7.673,4.871);
\filldraw[fill opacity=0.8,fill=gray!20,draw=none](-7.655,4.876)--(-7.658,4.869)--(-7.685,4.873)--cycle;
\draw(-7.655,4.876)--(-7.658,4.869);
\filldraw[fill opacity=0.8,fill=gray!20,draw=none](-7.685,4.873)--(-7.669,4.876)--(-7.69,4.85)--cycle;
\draw(-7.685,4.873)--(-7.669,4.876)--(-7.69,4.85);
\filldraw[fill opacity=0.8,fill=gray!20,draw=none](-7.75,4.839)--(-7.712,4.867)--(-7.685,4.873)--(-7.69,4.85)--cycle;
\draw(-7.69,4.85)--(-7.75,4.839)--(-7.712,4.867)--(-7.685,4.873);
\filldraw[fill opacity=0.8,fill=gray!20,draw=none](-7.658,4.869)--(-7.7,4.773)--(-7.735,4.757)--(-7.685,4.873)--cycle;
\draw(-7.658,4.869)--(-7.7,4.773);
\draw(-7.735,4.757)--(-7.685,4.873);
\filldraw[fill opacity=0.8,fill=gray!20,draw=none](-7.656,4.852)--(-7.673,4.871)--(-7.669,4.876)--(-7.64,4.877)--(-7.612,4.865)--(-7.611,4.854)--cycle;
\draw(-7.673,4.871)--(-7.669,4.876)--(-7.64,4.877);
\draw(-7.612,4.865)--(-7.611,4.854)--(-7.656,4.852);
\filldraw[fill opacity=0.8,fill=gray!20,draw=none](-7.62,4.848)--(-7.64,4.801)--(-7.7,4.773)--(-7.662,4.859)--cycle;
\draw(-7.62,4.848)--(-7.64,4.801);
\draw(-7.7,4.773)--(-7.662,4.859);
\filldraw[fill opacity=0.8,fill=gray!20,draw=none](-7.638,4.853)--(-7.611,4.854)--(-7.61,4.835)--cycle;
\draw(-7.638,4.853)--(-7.611,4.854)--(-7.61,4.835);
\filldraw[fill opacity=0.8,fill=gray!20,draw=none](-7.564,4.84)--(-7.597,4.765)--(-7.658,4.76)--(-7.622,4.843)--cycle;
\draw(-7.564,4.84)--(-7.597,4.765);
\draw(-7.658,4.76)--(-7.622,4.843);
\filldraw[fill opacity=0.8,fill=gray!20,draw=none](-7.713,4.767)--(-7.7,4.773)--(-7.704,4.763)--cycle;
\draw(-7.7,4.773)--(-7.704,4.763);
\filldraw[fill opacity=0.8,fill=gray!20,draw=none](-7.623,4.788)--(-7.679,4.761)--(-7.704,4.763)--(-7.713,4.767)--(-7.66,4.793)--cycle;
\draw(-7.713,4.767)--(-7.66,4.793)--(-7.623,4.788)--(-7.679,4.761);
\filldraw[fill opacity=0.8,fill=gray!20,draw=none](-7.813,4.579)--(-7.815,4.576)--(-7.817,4.58)--cycle;
\draw(-7.815,4.576)--(-7.817,4.58);
\filldraw[fill opacity=0.8,fill=gray!20,draw=none](-7.753,4.727)--(-7.737,4.756)--(-7.735,4.757)--(-7.813,4.579)--(-7.815,4.576)--(-7.815,4.579)--(-7.803,4.611)--(-7.785,4.655)--(-7.762,4.707)--cycle;
\draw(-7.735,4.757)--(-7.813,4.579);
\draw(-7.785,4.655)--(-7.762,4.707);
\filldraw[fill opacity=0.8,fill=gray!20,draw=none](-7.709,4.722)--(-7.735,4.732)--(-7.743,4.735)--(-7.746,4.735)--cycle;
\draw(-7.735,4.732)--(-7.743,4.735)--(-7.746,4.735);
\filldraw[fill opacity=0.8,fill=gray!20,draw=none](-7.804,4.576)--(-7.797,4.573)--(-7.804,4.568)--(-7.808,4.569)--cycle;
\draw(-7.797,4.573)--(-7.804,4.568);
\filldraw[fill opacity=0.8,fill=gray!20,draw=none](-7.807,4.56)--(-7.81,4.564)--(-7.801,4.57)--cycle;
\draw(-7.807,4.56)--(-7.81,4.564)--(-7.801,4.57);
\filldraw[fill opacity=0.8,fill=gray!20,draw=none](-7.813,4.579)--(-7.804,4.576)--(-7.808,4.569)--(-7.813,4.571)--(-7.815,4.576)--cycle;
\draw(-7.813,4.571)--(-7.815,4.576);
\filldraw[fill opacity=0.8,fill=gray!20,draw=none](-7.768,4.643)--(-7.802,4.568)--(-7.808,4.568)--(-7.743,4.735)--(-7.735,4.732)--cycle;
\draw(-7.808,4.568)--(-7.743,4.735)--(-7.735,4.732);
\filldraw[fill opacity=0.8,fill=gray!20,draw=none](-7.808,4.569)--(-7.81,4.565)--(-7.813,4.571)--cycle;
\draw(-7.81,4.565)--(-7.813,4.571);
\filldraw[fill opacity=0.8,fill=gray!20,draw=none](-7.81,4.565)--(-7.81,4.569)--(-7.795,4.615)--(-7.792,4.622)--(-7.752,4.725)--(-7.751,4.728)--(-7.747,4.735)--(-7.743,4.735)--(-7.808,4.569)--cycle;
\draw(-7.792,4.622)--(-7.752,4.725);
\draw(-7.747,4.735)--(-7.743,4.735)--(-7.808,4.569);
\filldraw[fill opacity=0.8,fill=gray!20,draw=none](-7.732,4.73)--(-7.746,4.735)--(-7.747,4.735)--(-7.742,4.733)--(-7.737,4.731)--cycle;
\draw(-7.746,4.735)--(-7.747,4.735);
\filldraw[fill opacity=0.8,fill=gray!20](-7.691,4.778)--(-8.293,4.491)--(-8.313,4.46)--(-7.711,4.747)--cycle;
\filldraw[fill opacity=0.8,fill=gray!20,draw=none](-8.646,2.956)--(-8.655,2.966)--(-8.658,2.959)--cycle;
\draw(-8.655,2.966)--(-8.658,2.959);
\filldraw[fill opacity=0.8,fill=gray!20,draw=none](-8.069,4.334)--(-8.22,4.262)--(-8.183,4.257)--(-8.049,4.321)--cycle;
\draw(-8.069,4.334)--(-8.22,4.262)--(-8.183,4.257)--(-8.049,4.321);
\filldraw[fill opacity=0.8,fill=gray!20,draw=none](-8.774,3.155)--(-8.757,3.196)--(-8.783,3.207)--(-8.79,3.202)--cycle;
\draw(-8.774,3.155)--(-8.757,3.196)--(-8.783,3.207);
\filldraw[fill opacity=0.8,fill=gray!20,draw=none](-8.747,3.127)--(-8.693,3.123)--(-8.692,3.171)--(-8.738,3.174)--cycle;
\draw(-8.747,3.127)--(-8.693,3.123)--(-8.692,3.171)--(-8.738,3.174);
\filldraw[fill opacity=0.8,fill=gray!20](-8.587,3.066)--(-8.583,3.12)--(-8.695,3.115)--(-8.694,3.062)--cycle;
\filldraw[fill opacity=0.8,fill=gray!20,draw=none](-8.626,3.119)--(-8.625,3.176)--(-8.694,3.173)--(-8.695,3.115)--cycle;
\draw(-8.625,3.176)--(-8.694,3.173)--(-8.695,3.115)--(-8.626,3.119);
\filldraw[fill opacity=0.8,fill=gray!20,draw=none](-8.715,3.168)--(-8.724,3.118)--(-8.695,3.115)--(-8.694,3.173)--(-8.71,3.174)--cycle;
\draw(-8.724,3.118)--(-8.695,3.115)--(-8.694,3.173)--(-8.71,3.174);
\filldraw[fill opacity=0.8,fill=gray!20,draw=none](-8.528,3.086)--(-8.524,3.095)--(-8.399,3.04)--(-8.467,3.009)--(-8.532,3.037)--cycle;
\draw(-8.524,3.095)--(-8.399,3.04);
\draw(-8.467,3.009)--(-8.532,3.037);
\filldraw[fill opacity=0.8,fill=gray!20,draw=none](-8.541,3.045)--(-8.547,3.082)--(-8.545,3.091)--(-8.531,3.103)--cycle;
\filldraw[fill opacity=0.8,fill=gray!20,draw=none](-8.541,3.045)--(-8.547,3.04)--(-8.539,3.091)--(-8.536,3.098)--(-8.532,3.101)--cycle;
\draw(-8.541,3.045)--(-8.547,3.04);
\draw(-8.536,3.098)--(-8.532,3.101);
\filldraw[fill opacity=0.8,fill=gray!20](-8.463,3.058)--(-8.79,3.201)--(-8.799,3.15)--(-8.472,3.008)--cycle;
\filldraw[fill opacity=0.8,fill=gray!20](-8.463,3.058)--(-8.79,3.201)--(-8.799,3.15)--(-8.472,3.008)--cycle;
\filldraw[fill opacity=0.8,fill=gray!20,draw=none](-7.992,4.409)--(-7.996,4.407)--(-8.012,4.437)--(-8.015,4.451)--(-7.989,4.463)--cycle;
\draw(-7.992,4.409)--(-7.996,4.407);
\draw(-8.015,4.451)--(-7.989,4.463);
\filldraw[fill opacity=0.8,fill=gray!20](-8.172,4.594)--(-8.222,4.596)--(-8.222,4.596)--(-8.166,4.588)--cycle;
\filldraw[fill opacity=0.8,fill=gray!20,draw=none](-8.823,3.044)--(-8.817,3.057)--(-8.837,3.094)--(-8.856,3.05)--cycle;
\draw(-8.823,3.044)--(-8.817,3.057);
\draw(-8.837,3.094)--(-8.856,3.05);
\filldraw[fill opacity=0.8,fill=gray!20,draw=none](-8.817,3.057)--(-8.774,3.155)--(-8.79,3.202)--(-8.837,3.094)--cycle;
\draw(-8.817,3.057)--(-8.774,3.155);
\draw(-8.79,3.202)--(-8.837,3.094);
\filldraw[fill opacity=0.8,fill=gray!20,draw=none](-8.774,3.066)--(-8.804,3.088)--(-8.815,3.061)--cycle;
\draw(-8.804,3.088)--(-8.815,3.061);
\filldraw[fill opacity=0.8,fill=gray!20](-8.786,3.022)--(-8.798,3.069)--(-8.871,3.087)--(-8.852,3.038)--cycle;
\filldraw[fill opacity=0.8,fill=gray!20](-8.732,3.301)--(-8.708,3.327)--(-8.724,3.331)--(-8.763,3.309)--cycle;
\filldraw[fill opacity=0.5,fill=gray!20](-7.338,1.088)--(-7.31,1.118)--(-7.369,.62)--(-7.395,.612)--cycle;
\filldraw[fill opacity=0.8,fill=gray!20,draw=none](-8.71,3.174)--(-8.694,3.173)--(-8.694,3.199)--cycle;
\draw(-8.71,3.174)--(-8.694,3.173)--(-8.694,3.199);
\filldraw[fill opacity=0.8,fill=gray!20,draw=none](-8.21,2.896)--(-8.218,2.892)--(-8.335,2.943)--(-8.325,2.957)--(-8.2,2.903)--cycle;
\draw(-8.325,2.957)--(-8.2,2.903);
\filldraw[fill opacity=0.8,fill=gray!20,draw=none](-8.333,2.966)--(-8.314,2.984)--(-8.316,2.963)--(-8.331,2.949)--(-8.339,2.953)--cycle;
\draw(-8.331,2.949)--(-8.339,2.953);
\filldraw[fill opacity=0.8,fill=gray!20,draw=none](-8.178,2.934)--(-8.2,2.903)--(-8.323,2.956)--(-8.308,2.99)--cycle;
\draw(-8.2,2.903)--(-8.323,2.956);
\filldraw[fill opacity=0.8,fill=gray!20,draw=none](-8.316,3.004)--(-8.313,3.002)--(-8.314,2.984)--(-8.333,2.966)--cycle;
\draw(-8.316,3.004)--(-8.313,3.002);
\filldraw[fill opacity=0.8,fill=gray!20,draw=none](-8.251,2.974)--(-8.254,2.976)--(-8.249,2.973)--cycle;
\draw(-8.254,2.976)--(-8.249,2.973);
\filldraw[fill opacity=0.8,fill=gray!20,draw=none](-8.316,3.004)--(-8.25,2.975)--(-8.271,2.939)--(-8.333,2.966)--cycle;
\draw(-8.25,2.975)--(-8.271,2.939)--(-8.333,2.966);
\filldraw[fill opacity=0.8,fill=gray!20,draw=none](-8.458,3.067)--(-8.462,3.069)--(-8.474,3.074)--(-8.47,3.072)--(-8.449,3.062)--(-8.413,3.046)--(-8.369,3.026)--(-8.322,3.005)--(-8.28,2.987)--(-8.254,2.976)--(-8.251,2.974)--cycle;
\draw(-8.462,3.069)--(-8.474,3.074)--(-8.47,3.072)--(-8.449,3.062)--(-8.413,3.046)--(-8.369,3.026)--(-8.322,3.005)--(-8.28,2.987)--(-8.254,2.976);
\filldraw[fill opacity=0.8,fill=gray!20,draw=none](-8.536,3.098)--(-8.523,3.124)--(-8.531,3.101)--cycle;
\draw(-8.523,3.124)--(-8.531,3.101)--(-8.536,3.098);
\filldraw[fill opacity=0.8,fill=gray!20,draw=none](-8.459,3.066)--(-8.438,3.102)--(-8.765,3.245)--(-8.786,3.209)--cycle;
\draw(-8.459,3.066)--(-8.438,3.102)--(-8.765,3.245)--(-8.786,3.209);
\filldraw[fill opacity=0.8,fill=gray!20,draw=none](-8.459,3.066)--(-8.438,3.102)--(-8.765,3.245)--(-8.786,3.209)--cycle;
\draw(-8.459,3.066)--(-8.438,3.102)--(-8.765,3.245)--(-8.786,3.209);
\filldraw[fill opacity=0.5,fill=gray!20](-8.18,-.621)--(-8.207,-.648)--(-8.641,-.892)--(-8.599,-.856)--cycle;
\filldraw[fill opacity=0.8,fill=gray!20](-8.361,4.518)--(-8.321,4.556)--(-8.332,4.568)--(-8.377,4.535)--cycle;
\filldraw[fill opacity=0.8,fill=gray!20,draw=none](-8.011,4.434)--(-8.011,4.434)--(-8.012,4.437)--cycle;
\draw(-8.011,4.434)--(-8.011,4.434)--(-8.012,4.437);
\filldraw[fill opacity=0.8,fill=gray!20,draw=none](-8.011,4.434)--(-8.012,4.437)--(-8.033,4.486)--(-8.064,4.466)--(-8.045,4.411)--cycle;
\draw(-8.012,4.437)--(-8.033,4.486)--(-8.064,4.466)--(-8.045,4.411)--(-8.011,4.434);
\filldraw[fill opacity=0.8,fill=gray!20,draw=none](-8.459,3.066)--(-8.786,3.209)--(-8.79,3.201)--(-8.463,3.058)--cycle;
\draw(-8.786,3.209)--(-8.79,3.201)--(-8.463,3.058)--(-8.459,3.066);
\filldraw[fill opacity=0.8,fill=gray!20,draw=none](-8.459,3.066)--(-8.786,3.209)--(-8.79,3.201)--(-8.463,3.058)--cycle;
\draw(-8.786,3.209)--(-8.79,3.201)--(-8.463,3.058)--(-8.459,3.066);
\filldraw[fill opacity=0.8,fill=gray!20](-8.605,3.307)--(-8.642,3.33)--(-8.66,3.327)--(-8.641,3.3)--cycle;
\filldraw[fill opacity=0.8,fill=gray!20,draw=none](-8.904,2.898)--(-8.907,2.934)--(-8.912,2.922)--cycle;
\draw(-8.907,2.934)--(-8.912,2.922);
\filldraw[fill opacity=0.8,fill=gray!20,draw=none](-8.901,2.864)--(-8.836,3.013)--(-8.856,3.05)--(-8.907,2.934)--cycle;
\draw(-8.901,2.864)--(-8.836,3.013);
\draw(-8.856,3.05)--(-8.907,2.934);
\filldraw[fill opacity=0.8,fill=gray!20,draw=none](-8.836,3.013)--(-8.823,3.044)--(-8.856,3.05)--cycle;
\draw(-8.836,3.013)--(-8.823,3.044);
\filldraw[fill opacity=0.8,fill=gray!20,draw=none](-8.779,3.083)--(-8.779,3.085)--(-8.84,3.1)--(-8.837,3.094)--cycle;
\draw(-8.779,3.083)--(-8.779,3.085)--(-8.84,3.1)--(-8.837,3.094);
\filldraw[fill opacity=0.8,fill=gray!20,draw=none](-8.783,3.207)--(-8.787,3.209)--(-8.79,3.202)--cycle;
\draw(-8.783,3.207)--(-8.787,3.209)--(-8.79,3.202);
\filldraw[fill opacity=0.8,fill=gray!20,draw=none](-8.774,3.11)--(-8.774,3.14)--(-8.778,3.146)--(-8.788,3.123)--cycle;
\draw(-8.778,3.146)--(-8.788,3.123);
\filldraw[fill opacity=0.8,fill=gray!20](-8.779,3.085)--(-8.782,3.13)--(-8.845,3.145)--(-8.84,3.1)--cycle;
\filldraw[fill opacity=0.8,fill=gray!20,draw=none](-7.998,4.392)--(-8.013,4.385)--(-8.011,4.4)--(-8.003,4.404)--cycle;
\draw(-8.011,4.4)--(-8.003,4.404);
\filldraw[fill opacity=0.8,fill=gray!20,draw=none](-8.07,4.533)--(-8.084,4.543)--(-8.067,4.531)--cycle;
\draw(-8.084,4.543)--(-8.067,4.531)--(-8.07,4.533);
\filldraw[fill opacity=0.8,fill=gray!20](-8.308,4.202)--(-8.327,4.24)--(-8.393,4.256)--(-8.361,4.215)--cycle;
\filldraw[fill opacity=0.8,fill=gray!20](-8.742,3.329)--(-8.713,3.36)--(-8.732,3.365)--(-8.78,3.338)--cycle;
\filldraw[fill opacity=0.8,fill=gray!20](-8.321,4.556)--(-8.273,4.583)--(-8.279,4.589)--(-8.332,4.568)--cycle;
\filldraw[fill opacity=0.8,fill=gray!20](-8.033,4.486)--(-8.067,4.531)--(-8.093,4.514)--(-8.064,4.466)--cycle;
\filldraw[fill opacity=0.8,fill=gray!20,draw=none](-8.638,2.994)--(-8.641,2.997)--(-8.643,2.994)--cycle;
\draw(-8.641,2.997)--(-8.643,2.994);
\filldraw[fill opacity=0.8,fill=gray!20](-8.753,3.266)--(-8.732,3.301)--(-8.763,3.309)--(-8.797,3.277)--cycle;
\filldraw[fill opacity=0.8,fill=gray!20](-8.59,3.336)--(-8.634,3.363)--(-8.656,3.359)--(-8.633,3.327)--cycle;
\filldraw[fill opacity=0.8,fill=gray!20,draw=none](-8.689,3.009)--(-8.69,3.018)--(-8.763,3.033)--(-8.753,3.013)--cycle;
\draw(-8.763,3.033)--(-8.753,3.013)--(-8.689,3.009)--(-8.69,3.018);
\filldraw[fill opacity=0.8,fill=gray!20](-8.687,3.298)--(-8.684,3.325)--(-8.708,3.327)--(-8.732,3.301)--cycle;
\filldraw[fill opacity=0.8,fill=gray!20,draw=none](-8.656,3.005)--(-8.663,3.01)--(-8.689,3.009)--(-8.688,3)--cycle;
\draw(-8.663,3.01)--(-8.689,3.009)--(-8.688,3);
\filldraw[fill opacity=0.5,fill=gray!20](-8.434,-.333)--(-8.261,-.409)--(-8.618,-.608)--(-8.791,-.533)--cycle;
\filldraw[fill opacity=0.8,fill=gray!20](-8.782,3.13)--(-8.779,3.177)--(-8.84,3.192)--(-8.845,3.145)--cycle;
\filldraw[fill opacity=0.8,fill=gray!20](-8.641,3.3)--(-8.66,3.327)--(-8.684,3.325)--(-8.687,3.298)--cycle;
\filldraw[fill opacity=0.8,fill=gray!20](-8.273,4.583)--(-8.222,4.596)--(-8.222,4.596)--(-8.279,4.589)--cycle;
\filldraw[fill opacity=0.8,fill=gray!20,draw=none](-8.774,3.066)--(-8.779,3.083)--(-8.804,3.088)--cycle;
\draw(-8.774,3.066)--(-8.779,3.083);
\filldraw[fill opacity=0.8,fill=gray!20](-8.798,3.069)--(-8.803,3.123)--(-8.878,3.142)--(-8.871,3.087)--cycle;
\filldraw[fill opacity=0.8,fill=gray!20,draw=none](-8.709,2.842)--(-8.66,2.955)--(-8.693,2.977)--(-8.76,2.823)--cycle;
\draw(-8.709,2.842)--(-8.66,2.955);
\draw(-8.693,2.977)--(-8.76,2.823);
\filldraw[fill opacity=0.8,fill=gray!20,draw=none](-8.691,2.976)--(-8.66,2.955)--(-8.655,2.966)--cycle;
\draw(-8.66,2.955)--(-8.655,2.966);
\filldraw[fill opacity=0.8,fill=gray!20,draw=none](-8.655,2.966)--(-8.666,2.979)--(-8.691,2.978)--(-8.691,2.976)--cycle;
\draw(-8.666,2.979)--(-8.691,2.978)--(-8.691,2.976);
\filldraw[fill opacity=0.5,fill=gray!20](-7.87,2.536)--(-7.925,2.574)--(-7.632,2.189)--(-7.579,2.153)--cycle;
\filldraw[fill opacity=0.8,fill=gray!20,draw=none](-8.594,3.27)--(-8.574,3.274)--(-8.605,3.307)--(-8.641,3.3)--(-8.639,3.296)--cycle;
\draw(-8.594,3.27)--(-8.574,3.274)--(-8.605,3.307)--(-8.641,3.3)--(-8.639,3.296);
\filldraw[fill opacity=0.8,fill=gray!20](-8.769,3.224)--(-8.753,3.266)--(-8.797,3.277)--(-8.823,3.237)--cycle;
\filldraw[fill opacity=0.8,fill=gray!20](-8.229,4.172)--(-8.232,4.196)--(-8.308,4.202)--(-8.283,4.176)--cycle;
\filldraw[fill opacity=0.8,fill=gray!20](-8.254,4.578)--(-8.222,4.596)--(-8.222,4.596)--(-8.273,4.583)--cycle;
\filldraw[fill opacity=0.8,fill=gray!20](-8.175,4.582)--(-8.222,4.596)--(-8.222,4.596)--(-8.197,4.577)--cycle;
\filldraw[fill opacity=0.8,fill=gray!20](-8.166,4.588)--(-8.222,4.596)--(-8.222,4.596)--(-8.175,4.582)--cycle;
\filldraw[fill opacity=0.8,fill=gray!20](-8.197,4.577)--(-8.222,4.596)--(-8.222,4.596)--(-8.226,4.576)--cycle;
\filldraw[fill opacity=0.8,fill=gray!20](-8.226,4.576)--(-8.222,4.596)--(-8.222,4.596)--(-8.254,4.578)--cycle;
\filldraw[fill opacity=0.8,fill=gray!20,draw=none](-8.125,4.571)--(-8.166,4.588)--(-8.175,4.582)--(-8.144,4.562)--cycle;
\draw(-8.125,4.571)--(-8.166,4.588)--(-8.175,4.582)--(-8.144,4.562);
\filldraw[fill opacity=0.8,fill=gray!20](-8.093,4.211)--(-8.064,4.251)--(-8.137,4.237)--(-8.153,4.2)--cycle;
\filldraw[fill opacity=0.5,fill=gray!20,draw=none](-9.005,-.98)--(-9.004,-.98)--(-8.693,-.902)--(-8.698,-.9)--cycle;
\draw(-9.004,-.98)--(-8.693,-.902)--(-8.698,-.9);
\filldraw[fill opacity=0.8,fill=gray!20](-8.779,3.177)--(-8.769,3.224)--(-8.823,3.237)--(-8.84,3.192)--cycle;
\filldraw[fill opacity=0.8,fill=gray!20](-8.767,3.287)--(-8.742,3.329)--(-8.78,3.338)--(-8.82,3.3)--cycle;
\filldraw[fill opacity=0.8,fill=gray!20,draw=none](-8.562,3.262)--(-8.572,3.275)--(-8.574,3.274)--(-8.566,3.262)--cycle;
\draw(-8.572,3.275)--(-8.574,3.274)--(-8.566,3.262);
\filldraw[fill opacity=0.5,fill=gray!20](-7.825,-.247)--(-7.819,-.28)--(-8.18,-.621)--(-8.169,-.573)--cycle;
\filldraw[fill opacity=0.8,fill=gray!20,draw=none](-8.663,3.01)--(-8.672,3.016)--(-8.69,3.018)--(-8.689,3.009)--cycle;
\draw(-8.69,3.018)--(-8.689,3.009)--(-8.663,3.01);
\filldraw[fill opacity=0.8,fill=gray!20,draw=none](-8.65,3.003)--(-8.669,3.014)--(-8.676,3.015)--(-8.688,2.988)--cycle;
\draw(-8.676,3.015)--(-8.688,2.988);
\filldraw[fill opacity=0.8,fill=gray!20](-8.064,4.466)--(-8.093,4.514)--(-8.153,4.503)--(-8.137,4.451)--cycle;
\filldraw[fill opacity=0.8,fill=gray!20,draw=none](-8.206,4.471)--(-8.2,4.449)--(-8.137,4.451)--(-8.153,4.503)--(-8.199,4.501)--cycle;
\draw(-8.2,4.449)--(-8.137,4.451)--(-8.153,4.503)--(-8.199,4.501);
\filldraw[fill opacity=0.8,fill=gray!20,draw=none](-7.723,4.724)--(-7.671,4.705)--(-7.526,4.651)--(-7.563,4.666)--(-7.584,4.675)--(-7.629,4.692)--(-7.645,4.699)--(-7.709,4.722)--(-7.732,4.73)--(-7.737,4.73)--cycle;
\draw(-7.629,4.692)--(-7.645,4.699);
\filldraw[fill opacity=0.8,fill=gray!20,draw=none](-7.645,4.699)--(-7.678,4.711)--(-7.719,4.726)--(-7.735,4.732)--cycle;
\draw(-7.645,4.699)--(-7.678,4.711)--(-7.719,4.726)--(-7.735,4.732);
\filldraw[fill opacity=0.8,fill=gray!20,draw=none](-7.585,4.659)--(-7.594,4.635)--(-7.639,4.663)--(-7.633,4.677)--cycle;
\draw(-7.639,4.663)--(-7.633,4.677)--(-7.585,4.659)--(-7.594,4.635);
\filldraw[fill opacity=0.8,fill=gray!20,draw=none](-7.633,4.677)--(-7.639,4.663)--(-7.682,4.686)--(-7.679,4.694)--cycle;
\draw(-7.682,4.686)--(-7.679,4.694)--(-7.633,4.677)--(-7.639,4.663);
\filldraw[fill opacity=0.8,fill=gray!20,draw=none](-7.543,4.643)--(-7.55,4.622)--(-7.564,4.613)--(-7.594,4.635)--(-7.585,4.659)--cycle;
\draw(-7.594,4.635)--(-7.585,4.659)--(-7.543,4.643)--(-7.55,4.622);
\filldraw[fill opacity=0.8,fill=gray!20,draw=none](-7.831,4.616)--(-7.832,4.65)--(-7.827,4.68)--(-7.803,4.695)--(-7.797,4.639)--cycle;
\draw(-7.827,4.68)--(-7.803,4.695)--(-7.797,4.639)--(-7.831,4.616);
\filldraw[fill opacity=0.8,fill=gray!20,draw=none](-7.639,4.663)--(-7.634,4.674)--(-7.588,4.653)--(-7.596,4.636)--cycle;
\draw(-7.639,4.663)--(-7.634,4.674)--(-7.588,4.653)--(-7.596,4.636);
\filldraw[fill opacity=0.8,fill=gray!20,draw=none](-7.682,4.687)--(-7.679,4.694)--(-7.634,4.674)--(-7.639,4.663)--cycle;
\draw(-7.682,4.687)--(-7.679,4.694)--(-7.634,4.674)--(-7.639,4.663);
\filldraw[fill opacity=0.8,fill=gray!20,draw=none](-7.728,4.574)--(-7.732,4.575)--(-7.719,4.604)--(-7.716,4.603)--cycle;
\draw(-7.732,4.575)--(-7.719,4.604);
\filldraw[fill opacity=0.8,fill=gray!20,draw=none](-7.716,4.603)--(-7.719,4.604)--(-7.682,4.687)--cycle;
\draw(-7.719,4.604)--(-7.682,4.687);
\filldraw[fill opacity=0.8,fill=gray!20,draw=none](-7.751,4.535)--(-7.783,4.558)--(-7.768,4.556)--(-7.758,4.549)--(-7.75,4.536)--cycle;
\draw(-7.758,4.549)--(-7.75,4.536)--(-7.751,4.535);
\filldraw[fill opacity=0.8,fill=gray!20,draw=none](-7.783,4.558)--(-7.796,4.567)--(-7.781,4.566)--(-7.768,4.556)--cycle;
\filldraw[fill opacity=0.8,fill=gray!20,draw=none](-7.783,4.558)--(-7.715,4.71)--(-7.679,4.694)--(-7.75,4.534)--cycle;
\draw(-7.783,4.558)--(-7.715,4.71)--(-7.679,4.694)--(-7.75,4.534);
\filldraw[fill opacity=0.8,fill=gray!20,draw=none](-7.679,4.694)--(-7.728,4.574)--(-7.755,4.603)--(-7.716,4.707)--cycle;
\draw(-7.755,4.603)--(-7.716,4.707)--(-7.679,4.694);
\filldraw[fill opacity=0.8,fill=gray!20,draw=none](-7.82,4.467)--(-7.814,4.482)--(-7.794,4.465)--cycle;
\draw(-7.82,4.467)--(-7.814,4.482);
\filldraw[fill opacity=0.8,fill=gray!20,draw=none](-7.789,4.543)--(-7.783,4.558)--(-7.75,4.534)--(-7.754,4.527)--cycle;
\draw(-7.789,4.543)--(-7.783,4.558);
\draw(-7.75,4.534)--(-7.754,4.527);
\filldraw[fill opacity=0.8,fill=gray!20,draw=none](-7.783,4.558)--(-7.807,4.56)--(-7.801,4.57)--cycle;
\filldraw[fill opacity=0.8,fill=gray!20,draw=none](-7.768,4.643)--(-7.796,4.567)--(-7.802,4.568)--cycle;
\filldraw[fill opacity=0.8,fill=gray!20,draw=none](-7.797,4.546)--(-7.807,4.56)--(-7.736,4.72)--(-7.715,4.71)--(-7.789,4.543)--cycle;
\draw(-7.807,4.56)--(-7.736,4.72)--(-7.715,4.71)--(-7.789,4.543);
\filldraw[fill opacity=0.8,fill=gray!20,draw=none](-7.783,4.558)--(-7.751,4.535)--(-7.759,4.53)--(-7.796,4.546)--(-7.807,4.56)--cycle;
\draw(-7.751,4.535)--(-7.759,4.53);
\draw(-7.796,4.546)--(-7.807,4.56);
\filldraw[fill opacity=0.8,fill=gray!20](-7.716,4.707)--(-7.775,4.551)--(-7.797,4.558)--(-7.738,4.714)--cycle;
\filldraw[fill opacity=0.8,fill=gray!20,draw=none](-7.808,4.569)--(-7.804,4.568)--(-7.81,4.564)--(-7.81,4.565)--cycle;
\draw(-7.804,4.568)--(-7.81,4.564)--(-7.81,4.565);
\filldraw[fill opacity=0.8,fill=gray!20,draw=none](-7.808,4.561)--(-7.808,4.563)--(-7.789,4.612)--(-7.74,4.722)--(-7.736,4.72)--(-7.807,4.56)--cycle;
\draw(-7.789,4.612)--(-7.74,4.722)--(-7.736,4.72)--(-7.807,4.56);
\filldraw[fill opacity=0.8,fill=gray!20,draw=none](-7.738,4.714)--(-7.797,4.558)--(-7.8,4.558)--(-7.787,4.597)--(-7.743,4.714)--cycle;
\draw(-7.787,4.597)--(-7.743,4.714)--(-7.738,4.714)--(-7.797,4.558)--(-7.8,4.558);
\filldraw[fill opacity=0.8,fill=gray!20,draw=none](-7.62,4.669)--(-7.679,4.694)--(-7.716,4.707)--(-7.738,4.714)--(-7.743,4.714)--(-7.734,4.71)--cycle;
\draw(-7.679,4.694)--(-7.716,4.707)--(-7.738,4.714)--(-7.743,4.714)--(-7.734,4.71);
\filldraw[fill opacity=0.8,fill=gray!20,draw=none](-7.62,4.669)--(-7.679,4.694)--(-7.715,4.71)--(-7.736,4.72)--(-7.74,4.722)--(-7.73,4.718)--cycle;
\draw(-7.679,4.694)--(-7.715,4.71)--(-7.736,4.72)--(-7.74,4.722)--(-7.73,4.718);
\filldraw[fill opacity=0.8,fill=gray!20,draw=none](-7.584,4.675)--(-7.586,4.675)--(-7.627,4.692)--(-7.629,4.692)--cycle;
\draw(-7.586,4.675)--(-7.627,4.692)--(-7.629,4.692);
\filldraw[fill opacity=0.8,fill=gray!20,draw=none](-7.551,4.796)--(-7.567,4.814)--(-7.555,4.813)--(-7.52,4.799)--cycle;
\draw(-7.567,4.814)--(-7.555,4.813);
\filldraw[fill opacity=0.8,fill=gray!20,draw=none](-7.555,4.813)--(-7.574,4.818)--(-7.573,4.82)--cycle;
\draw(-7.574,4.818)--(-7.573,4.82);
\filldraw[fill opacity=0.8,fill=gray!20,draw=none](-7.567,4.814)--(-7.61,4.842)--(-7.611,4.854)--(-7.583,4.852)--(-7.523,4.826)--(-7.515,4.81)--cycle;
\draw(-7.61,4.842)--(-7.611,4.854)--(-7.583,4.852);
\draw(-7.523,4.826)--(-7.515,4.81)--(-7.567,4.814);
\filldraw[fill opacity=0.8,fill=gray!20,draw=none](-7.561,4.815)--(-7.571,4.707)--(-7.611,4.723)--(-7.613,4.727)--(-7.574,4.818)--cycle;
\draw(-7.613,4.727)--(-7.574,4.818);
\filldraw[fill opacity=0.8,fill=gray!20,draw=none](-7.597,4.765)--(-7.612,4.729)--(-7.663,4.748)--(-7.658,4.76)--cycle;
\draw(-7.597,4.765)--(-7.612,4.729);
\draw(-7.663,4.748)--(-7.658,4.76);
\filldraw[fill opacity=0.8,fill=gray!20,draw=none](-7.64,4.739)--(-7.612,4.729)--(-7.619,4.715)--cycle;
\draw(-7.612,4.729)--(-7.619,4.715);
\filldraw[fill opacity=0.8,fill=gray!20,draw=none](-7.611,4.723)--(-7.614,4.725)--(-7.613,4.727)--cycle;
\draw(-7.614,4.725)--(-7.613,4.727);
\filldraw[fill opacity=0.8,fill=gray!20,draw=none](-7.611,4.723)--(-7.601,4.705)--(-7.619,4.715)--(-7.614,4.725)--cycle;
\draw(-7.619,4.715)--(-7.614,4.725);
\filldraw[fill opacity=0.8,fill=gray!20,draw=none](-7.584,4.675)--(-7.611,4.723)--(-7.571,4.707)--(-7.575,4.673)--(-7.575,4.671)--cycle;
\draw(-7.575,4.673)--(-7.575,4.671);
\filldraw[fill opacity=0.8,fill=gray!20,draw=none](-7.561,4.815)--(-7.555,4.813)--(-7.52,4.799)--(-7.575,4.673)--cycle;
\draw(-7.52,4.799)--(-7.575,4.673);
\filldraw[fill opacity=0.8,fill=gray!20,draw=none](-7.553,4.728)--(-7.601,4.705)--(-7.619,4.715)--(-7.64,4.739)--(-7.585,4.765)--cycle;
\draw(-7.64,4.739)--(-7.585,4.765)--(-7.553,4.728)--(-7.601,4.705);
\filldraw[fill opacity=0.8,fill=gray!20,draw=none](-7.513,4.799)--(-7.476,4.772)--(-7.528,4.652)--(-7.575,4.671)--(-7.52,4.799)--cycle;
\draw(-7.476,4.772)--(-7.528,4.652);
\draw(-7.575,4.671)--(-7.52,4.799);
\filldraw[fill opacity=0.8,fill=gray!20,draw=none](-7.531,4.681)--(-7.563,4.666)--(-7.586,4.675)--(-7.601,4.705)--(-7.553,4.728)--cycle;
\draw(-7.601,4.705)--(-7.553,4.728)--(-7.531,4.681)--(-7.563,4.666);
\filldraw[fill opacity=0.8,fill=gray!20,draw=none](-7.601,4.705)--(-7.584,4.675)--(-7.629,4.692)--(-7.619,4.715)--cycle;
\draw(-7.629,4.692)--(-7.619,4.715);
\filldraw[fill opacity=0.8,fill=gray!20,draw=none](-7.563,4.666)--(-7.576,4.671)--(-7.586,4.675)--cycle;
\draw(-7.576,4.671)--(-7.586,4.675);
\filldraw[fill opacity=0.8,fill=gray!20,draw=none](-7.574,4.671)--(-7.525,4.651)--(-7.527,4.652)--(-7.574,4.671)--(-7.576,4.671)--cycle;
\draw(-7.525,4.651)--(-7.527,4.652)--(-7.574,4.671)--(-7.576,4.671);
\filldraw[fill opacity=0.8,fill=gray!20,draw=none](-7.566,4.614)--(-7.596,4.636)--(-7.588,4.653)--(-7.546,4.635)--(-7.551,4.622)--cycle;
\draw(-7.596,4.636)--(-7.588,4.653)--(-7.546,4.635)--(-7.551,4.622);
\filldraw[fill opacity=0.8,fill=gray!20,draw=none](-7.559,4.611)--(-7.562,4.614)--(-7.547,4.622)--cycle;
\draw(-7.562,4.614)--(-7.547,4.622);
\filldraw[fill opacity=0.8,fill=gray!20,draw=none](-7.552,4.607)--(-7.547,4.61)--(-7.549,4.605)--cycle;
\draw(-7.547,4.61)--(-7.549,4.605);
\filldraw[fill opacity=0.8,fill=gray!20,draw=none](-7.524,4.624)--(-7.554,4.606)--(-7.559,4.611)--(-7.547,4.622)--(-7.523,4.633)--cycle;
\draw(-7.547,4.622)--(-7.523,4.633)--(-7.524,4.624);
\filldraw[fill opacity=0.8,fill=gray!20,draw=none](-7.564,4.613)--(-7.55,4.622)--(-7.556,4.608)--cycle;
\draw(-7.55,4.622)--(-7.556,4.608);
\filldraw[fill opacity=0.8,fill=gray!20,draw=none](-7.566,4.614)--(-7.551,4.622)--(-7.558,4.608)--cycle;
\draw(-7.551,4.622)--(-7.558,4.608);
\filldraw[fill opacity=0.8,fill=gray!20,draw=none](-7.534,4.639)--(-7.525,4.635)--(-7.523,4.605)--(-7.544,4.597)--(-7.556,4.608)--(-7.55,4.622)--cycle;
\draw(-7.534,4.639)--(-7.525,4.635);
\draw(-7.556,4.608)--(-7.55,4.622);
\filldraw[fill opacity=0.8,fill=gray!20,draw=none](-7.524,4.624)--(-7.549,4.601)--(-7.554,4.606)--cycle;
\filldraw[fill opacity=0.8,fill=gray!20,draw=none](-7.525,4.622)--(-7.549,4.601)--(-7.558,4.608)--(-7.551,4.622)--(-7.54,4.633)--(-7.525,4.626)--cycle;
\draw(-7.558,4.608)--(-7.551,4.622);
\draw(-7.54,4.633)--(-7.525,4.626);
\filldraw[fill opacity=0.8,fill=gray!20,draw=none](-7.525,4.606)--(-7.546,4.598)--(-7.549,4.601)--(-7.525,4.622)--cycle;
\filldraw[fill opacity=0.8,fill=gray!20,draw=none](-7.524,4.624)--(-7.527,4.607)--(-7.546,4.598)--(-7.549,4.601)--cycle;
\draw(-7.524,4.624)--(-7.527,4.607);
\filldraw[fill opacity=0.8,fill=gray!20,draw=none](-7.481,4.566)--(-7.515,4.596)--(-7.45,4.58)--(-7.454,4.573)--cycle;
\draw(-7.515,4.596)--(-7.45,4.58)--(-7.454,4.573);
\filldraw[fill opacity=0.8,fill=gray!20,draw=none](-7.531,4.59)--(-7.508,4.6)--(-7.516,4.578)--cycle;
\draw(-7.508,4.6)--(-7.516,4.578);
\filldraw[fill opacity=0.8,fill=gray!20,draw=none](-7.534,4.592)--(-7.51,4.601)--(-7.52,4.58)--cycle;
\draw(-7.51,4.601)--(-7.52,4.58);
\filldraw[fill opacity=0.8,fill=gray!20,draw=none](-7.544,4.597)--(-7.521,4.607)--(-7.529,4.584)--cycle;
\draw(-7.521,4.607)--(-7.529,4.584);
\filldraw[fill opacity=0.8,fill=gray!20,draw=none](-7.546,4.598)--(-7.522,4.607)--(-7.531,4.585)--cycle;
\draw(-7.522,4.607)--(-7.531,4.585);
\filldraw[fill opacity=0.8,fill=gray!20,draw=none](-7.527,4.603)--(-7.537,4.593)--(-7.54,4.592)--cycle;
\draw(-7.527,4.603)--(-7.537,4.593)--(-7.54,4.592);
\filldraw[fill opacity=0.8,fill=gray!20,draw=none](-7.54,4.592)--(-7.537,4.593)--(-7.537,4.589)--cycle;
\draw(-7.54,4.592)--(-7.537,4.593)--(-7.537,4.589);
\filldraw[fill opacity=0.8,fill=gray!20,draw=none](-7.536,4.588)--(-7.527,4.607)--(-7.529,4.59)--(-7.534,4.588)--cycle;
\draw(-7.527,4.607)--(-7.529,4.59)--(-7.534,4.588);
\filldraw[fill opacity=0.8,fill=gray!20,draw=none](-7.531,4.588)--(-7.534,4.588)--(-7.529,4.59)--cycle;
\draw(-7.534,4.588)--(-7.529,4.59)--(-7.531,4.588);
\filldraw[fill opacity=0.8,fill=gray!20,draw=none](-7.527,4.603)--(-7.53,4.588)--(-7.537,4.588)--(-7.537,4.593)--cycle;
\draw(-7.537,4.588)--(-7.537,4.593)--(-7.527,4.603);
\filldraw[fill opacity=0.8,fill=gray!20,draw=none](-7.536,4.588)--(-7.546,4.598)--(-7.527,4.607)--(-7.534,4.588)--cycle;
\filldraw[fill opacity=0.8,fill=gray!20,draw=none](-7.73,4.718)--(-7.62,4.669)--(-7.527,4.603)--(-7.54,4.592)--(-7.742,4.697)--(-7.74,4.701)--(-7.734,4.711)--cycle;
\draw(-7.742,4.697)--(-7.74,4.701);
\filldraw[fill opacity=0.8,fill=gray!20,draw=none](-7.629,4.638)--(-7.635,4.631)--(-7.747,4.683)--(-7.742,4.697)--cycle;
\draw(-7.747,4.683)--(-7.742,4.697);
\filldraw[fill opacity=0.8,fill=gray!20,draw=none](-7.62,4.669)--(-7.73,4.718)--(-7.728,4.717)--(-7.724,4.715)--(-7.719,4.713)--(-7.716,4.712)--cycle;
\draw(-7.73,4.718)--(-7.728,4.717);
\draw(-7.719,4.713)--(-7.716,4.712);
\filldraw[fill opacity=0.8,fill=gray!20,draw=none](-7.62,4.669)--(-7.734,4.71)--(-7.728,4.707)--(-7.716,4.702)--cycle;
\draw(-7.734,4.71)--(-7.728,4.707)--(-7.716,4.702);
\filldraw[fill opacity=0.8,fill=gray!20,draw=none](-7.534,4.639)--(-7.55,4.622)--(-7.543,4.643)--cycle;
\draw(-7.55,4.622)--(-7.543,4.643)--(-7.534,4.639);
\filldraw[fill opacity=0.8,fill=gray!20,draw=none](-7.62,4.669)--(-7.525,4.635)--(-7.543,4.643)--(-7.585,4.659)--(-7.633,4.677)--(-7.679,4.694)--cycle;
\draw(-7.525,4.635)--(-7.543,4.643)--(-7.585,4.659)--(-7.633,4.677)--(-7.679,4.694);
\filldraw[fill opacity=0.8,fill=gray!20,draw=none](-7.671,4.705)--(-7.62,4.669)--(-7.73,4.718)--(-7.723,4.724)--cycle;
\draw(-7.73,4.718)--(-7.723,4.724);
\filldraw[fill opacity=0.8,fill=gray!20,draw=none](-7.737,4.537)--(-7.738,4.538)--(-7.692,4.521)--(-7.687,4.519)--cycle;
\draw(-7.738,4.538)--(-7.692,4.521)--(-7.687,4.519);
\filldraw[fill opacity=0.8,fill=gray!20,draw=none](-7.682,4.496)--(-7.682,4.496)--(-7.698,4.489)--cycle;
\draw(-7.682,4.496)--(-7.698,4.489);
\filldraw[fill opacity=0.8,fill=gray!20,draw=none](-7.737,4.537)--(-7.72,4.573)--(-7.682,4.567)--(-7.701,4.524)--cycle;
\draw(-7.682,4.567)--(-7.701,4.524);
\filldraw[fill opacity=0.8,fill=gray!20,draw=none](-7.732,4.548)--(-7.712,4.581)--(-7.682,4.567)--cycle;
\draw(-7.732,4.548)--(-7.712,4.581)--(-7.682,4.567);
\filldraw[fill opacity=0.8,fill=gray!20,draw=none](-7.724,4.476)--(-7.722,4.477)--(-7.723,4.475)--cycle;
\draw(-7.722,4.477)--(-7.723,4.475);
\filldraw[fill opacity=0.8,fill=gray!20,draw=none](-7.682,4.496)--(-7.699,4.488)--(-7.712,4.496)--(-7.692,4.5)--cycle;
\draw(-7.699,4.488)--(-7.712,4.496)--(-7.692,4.5);
\filldraw[fill opacity=0.8,fill=gray!20,draw=none](-7.692,4.5)--(-7.712,4.496)--(-7.733,4.518)--cycle;
\draw(-7.692,4.5)--(-7.712,4.496)--(-7.733,4.518);
\filldraw[fill opacity=0.8,fill=gray!20,draw=none](-7.738,4.52)--(-7.74,4.526)--(-7.733,4.518)--cycle;
\draw(-7.74,4.526)--(-7.733,4.518);
\filldraw[fill opacity=0.8,fill=gray!20,draw=none](-7.72,4.491)--(-7.728,4.492)--(-7.738,4.52)--(-7.733,4.518)--(-7.712,4.496)--cycle;
\draw(-7.733,4.518)--(-7.712,4.496)--(-7.72,4.491);
\filldraw[fill opacity=0.8,fill=gray!20,draw=none](-7.685,4.515)--(-7.676,4.52)--(-7.69,4.604)--(-7.712,4.581)--(-7.738,4.538)--cycle;
\draw(-7.69,4.604)--(-7.712,4.581)--(-7.738,4.538);
\filldraw[fill opacity=0.8,fill=gray!20,draw=none](-7.72,4.573)--(-7.728,4.574)--(-7.716,4.603)--(-7.707,4.6)--cycle;
\filldraw[fill opacity=0.8,fill=gray!20,draw=none](-7.685,4.515)--(-7.738,4.538)--(-7.741,4.529)--(-7.728,4.492)--cycle;
\draw(-7.738,4.538)--(-7.741,4.529);
\filldraw[fill opacity=0.8,fill=gray!20,draw=none](-7.741,4.533)--(-7.741,4.531)--(-7.743,4.531)--(-7.746,4.532)--(-7.749,4.535)--cycle;
\draw(-7.746,4.532)--(-7.749,4.535);
\filldraw[fill opacity=0.8,fill=gray!20,draw=none](-7.74,4.53)--(-7.743,4.531)--(-7.741,4.531)--cycle;
\filldraw[fill opacity=0.8,fill=gray!20,draw=none](-7.737,4.537)--(-7.742,4.528)--(-7.746,4.541)--cycle;
\filldraw[fill opacity=0.8,fill=gray!20,draw=none](-7.752,4.534)--(-7.75,4.536)--(-7.745,4.531)--cycle;
\draw(-7.752,4.534)--(-7.75,4.536)--(-7.745,4.531);
\filldraw[fill opacity=0.8,fill=gray!20,draw=none](-7.745,4.532)--(-7.75,4.535)--(-7.747,4.542)--(-7.744,4.534)--cycle;
\draw(-7.75,4.535)--(-7.747,4.542);
\filldraw[fill opacity=0.8,fill=gray!20,draw=none](-7.683,4.577)--(-7.678,4.52)--(-7.717,4.542)--(-7.722,4.601)--cycle;
\draw(-7.717,4.542)--(-7.722,4.601)--(-7.683,4.577)--(-7.678,4.52);
\filldraw[fill opacity=0.8,fill=gray!20,draw=none](-7.676,4.52)--(-7.655,4.531)--(-7.621,4.551)--(-7.612,4.621)--(-7.622,4.631)--(-7.649,4.631)--(-7.68,4.614)--(-7.69,4.604)--cycle;
\draw(-7.612,4.621)--(-7.622,4.631)--(-7.649,4.631)--(-7.68,4.614)--(-7.69,4.604);
\filldraw[fill opacity=0.8,fill=gray!20,draw=none](-7.635,4.631)--(-7.686,4.579)--(-7.722,4.601)--(-7.746,4.635)--(-7.751,4.674)--(-7.747,4.683)--cycle;
\draw(-7.686,4.579)--(-7.722,4.601)--(-7.746,4.635)--(-7.751,4.674)--(-7.747,4.683);
\filldraw[fill opacity=0.8,fill=gray!20,draw=none](-7.759,4.684)--(-7.752,4.725)--(-7.792,4.622)--cycle;
\draw(-7.752,4.725)--(-7.792,4.622);
\filldraw[fill opacity=0.8,fill=gray!20,draw=none](-7.74,4.701)--(-7.743,4.714)--(-7.758,4.673)--cycle;
\draw(-7.743,4.714)--(-7.758,4.673);
\filldraw[fill opacity=0.8,fill=gray!20,draw=none](-7.728,4.574)--(-7.725,4.571)--(-7.738,4.538)--(-7.742,4.539)--cycle;
\draw(-7.725,4.571)--(-7.738,4.538)--(-7.742,4.539);
\filldraw[fill opacity=0.8,fill=gray!20,draw=none](-7.737,4.537)--(-7.746,4.541)--(-7.747,4.542)--(-7.732,4.575)--(-7.72,4.573)--cycle;
\draw(-7.747,4.542)--(-7.732,4.575);
\filldraw[fill opacity=0.8,fill=gray!20,draw=none](-7.728,4.574)--(-7.742,4.539)--(-7.775,4.551)--(-7.755,4.603)--cycle;
\draw(-7.742,4.539)--(-7.775,4.551)--(-7.755,4.603);
\filldraw[fill opacity=0.8,fill=gray!20,draw=none](-7.746,4.635)--(-7.742,4.583)--(-7.748,4.637)--(-7.751,4.674)--cycle;
\draw(-7.748,4.637)--(-7.751,4.674)--(-7.746,4.635)--(-7.742,4.583);
\filldraw[fill opacity=0.8,fill=gray!20,draw=none](-7.74,4.701)--(-7.734,4.71)--(-7.743,4.714)--cycle;
\draw(-7.734,4.71)--(-7.743,4.714);
\filldraw[fill opacity=0.8,fill=gray!20,draw=none](-7.74,4.701)--(-7.751,4.674)--(-7.748,4.637)--(-7.737,4.685)--cycle;
\draw(-7.74,4.701)--(-7.751,4.674)--(-7.748,4.637);
\filldraw[fill opacity=0.8,fill=gray!20,draw=none](-7.752,4.683)--(-7.758,4.673)--(-7.78,4.615)--(-7.78,4.608)--(-7.765,4.629)--(-7.75,4.659)--cycle;
\draw(-7.758,4.673)--(-7.78,4.615);
\filldraw[fill opacity=0.8,fill=gray!20,draw=none](-7.739,4.705)--(-7.73,4.718)--(-7.74,4.722)--(-7.743,4.715)--cycle;
\draw(-7.73,4.718)--(-7.74,4.722)--(-7.743,4.715);
\filldraw[fill opacity=0.8,fill=gray!20,draw=none](-7.74,4.701)--(-7.752,4.683)--(-7.75,4.659)--(-7.737,4.685)--cycle;
\filldraw[fill opacity=0.8,fill=gray!20,draw=none](-7.739,4.705)--(-7.743,4.715)--(-7.761,4.674)--cycle;
\draw(-7.743,4.715)--(-7.761,4.674);
\filldraw[fill opacity=0.8,fill=gray!20,draw=none](-7.734,4.71)--(-7.74,4.701)--(-7.737,4.685)--(-7.728,4.707)--cycle;
\draw(-7.737,4.685)--(-7.728,4.707)--(-7.734,4.71);
\filldraw[fill opacity=0.8,fill=gray!20,draw=none](-7.739,4.705)--(-7.74,4.701)--(-7.737,4.685)--(-7.735,4.693)--cycle;
\draw(-7.739,4.705)--(-7.74,4.701);
\filldraw[fill opacity=0.8,fill=gray!20,draw=none](-7.781,4.622)--(-7.735,4.693)--(-7.739,4.705)--(-7.761,4.674)--(-7.789,4.612)--cycle;
\draw(-7.761,4.674)--(-7.789,4.612);
\filldraw[fill opacity=0.8,fill=gray!20,draw=none](-7.769,4.679)--(-7.762,4.707)--(-7.785,4.655)--cycle;
\draw(-7.762,4.707)--(-7.785,4.655);
\filldraw[fill opacity=0.8,fill=gray!20,draw=none](-7.723,4.724)--(-7.737,4.73)--(-7.737,4.731)--(-7.724,4.725)--cycle;
\filldraw[fill opacity=0.8,fill=gray!20,draw=none](-7.671,4.705)--(-7.723,4.724)--(-7.719,4.723)--(-7.715,4.721)--cycle;
\filldraw[fill opacity=0.8,fill=gray!20,draw=none](-7.671,4.705)--(-7.723,4.724)--(-7.713,4.734)--cycle;
\draw(-7.723,4.724)--(-7.713,4.734);
\filldraw[fill opacity=0.8,fill=gray!20,draw=none](-7.74,4.701)--(-7.736,4.712)--(-7.73,4.718)--cycle;
\draw(-7.74,4.701)--(-7.736,4.712)--(-7.73,4.718);
\filldraw[fill opacity=0.8,fill=gray!20,draw=none](-7.739,4.705)--(-7.735,4.693)--(-7.731,4.701)--(-7.728,4.717)--(-7.73,4.718)--cycle;
\draw(-7.735,4.693)--(-7.731,4.701);
\draw(-7.728,4.717)--(-7.73,4.718);
\filldraw[fill opacity=0.8,fill=gray!20,draw=none](-7.739,4.705)--(-7.735,4.693)--(-7.734,4.696)--(-7.736,4.712)--cycle;
\draw(-7.734,4.696)--(-7.736,4.712)--(-7.739,4.705);
\filldraw[fill opacity=0.8,fill=gray!20,draw=none](-7.73,4.718)--(-7.734,4.711)--(-7.73,4.718)--cycle;
\filldraw[fill opacity=0.8,fill=gray!20,draw=none](-7.731,4.701)--(-7.724,4.715)--(-7.73,4.718)--(-7.736,4.712)--(-7.735,4.698)--cycle;
\draw(-7.73,4.718)--(-7.736,4.712)--(-7.735,4.698);
\filldraw[fill opacity=0.8,fill=gray!20,draw=none](-7.724,4.715)--(-7.719,4.723)--(-7.723,4.724)--(-7.73,4.718)--cycle;
\draw(-7.723,4.724)--(-7.73,4.718);
\filldraw[fill opacity=0.8,fill=gray!20,draw=none](-7.728,4.717)--(-7.727,4.716)--(-7.724,4.715)--cycle;
\draw(-7.728,4.717)--(-7.727,4.716);
\filldraw[fill opacity=0.8,fill=gray!20,draw=none](-7.731,4.701)--(-7.727,4.716)--(-7.728,4.717)--cycle;
\draw(-7.727,4.716)--(-7.728,4.717);
\filldraw[fill opacity=0.8,fill=gray!20,draw=none](-7.719,4.723)--(-7.713,4.734)--(-7.723,4.724)--cycle;
\draw(-7.713,4.734)--(-7.723,4.724);
\filldraw[fill opacity=0.8,fill=gray!20,draw=none](-7.723,4.724)--(-7.724,4.725)--(-7.719,4.723)--cycle;
\filldraw[fill opacity=0.8,fill=gray!20,draw=none](-7.731,4.701)--(-7.735,4.698)--(-7.734,4.696)--cycle;
\draw(-7.735,4.698)--(-7.734,4.696);
\filldraw[fill opacity=0.8,fill=gray!20,draw=none](-7.716,4.702)--(-7.728,4.707)--(-7.737,4.685)--(-7.715,4.69)--cycle;
\draw(-7.716,4.702)--(-7.728,4.707)--(-7.737,4.685);
\filldraw[fill opacity=0.8,fill=gray!20,draw=none](-7.716,4.695)--(-7.716,4.712)--(-7.719,4.713)--(-7.731,4.701)--(-7.735,4.693)--cycle;
\draw(-7.716,4.712)--(-7.719,4.713);
\draw(-7.731,4.701)--(-7.735,4.693);
\filldraw[fill opacity=0.8,fill=gray!20](-7.711,4.747)--(-8.313,4.46)--(-8.32,4.417)--(-7.717,4.704)--cycle;
\filldraw[fill opacity=0.8,fill=gray!20](-8.173,4.174)--(-8.153,4.2)--(-8.232,4.196)--(-8.229,4.172)--cycle;
\filldraw[fill opacity=0.8,fill=gray!20](-8.039,4.355)--(-8.045,4.411)--(-8.128,4.395)--(-8.124,4.339)--cycle;
\filldraw[fill opacity=0.8,fill=gray!20,draw=none](-8.125,4.333)--(-8.124,4.339)--(-8.138,4.338)--cycle;
\draw(-8.125,4.333)--(-8.124,4.339)--(-8.138,4.338);
\filldraw[fill opacity=0.8,fill=gray!20,draw=none](-8.179,4.375)--(-8.143,4.339)--(-8.138,4.338)--(-8.124,4.339)--(-8.128,4.395)--(-8.184,4.393)--cycle;
\draw(-8.138,4.338)--(-8.124,4.339)--(-8.128,4.395)--(-8.184,4.393);
\filldraw[fill opacity=0.8,fill=gray!20,draw=none](-7.968,4.423)--(-8.257,4.285)--(-8.22,4.262)--(-7.958,4.387)--cycle;
\draw(-7.968,4.423)--(-8.257,4.285)--(-8.22,4.262)--(-7.958,4.387);
\filldraw[fill opacity=0.8,fill=gray!20](-8.688,3.325)--(-8.685,3.358)--(-8.713,3.36)--(-8.742,3.329)--cycle;
\filldraw[fill opacity=0.8,fill=gray!20,draw=none](-8.695,2.978)--(-8.693,2.977)--(-8.688,2.988)--(-8.703,2.984)--cycle;
\draw(-8.693,2.977)--(-8.688,2.988);
\filldraw[fill opacity=0.8,fill=gray!20,draw=none](-8.703,2.984)--(-8.688,2.988)--(-8.676,3.015)--(-8.726,3.022)--(-8.734,3.004)--cycle;
\draw(-8.688,2.988)--(-8.676,3.015);
\draw(-8.726,3.022)--(-8.734,3.004);
\filldraw[fill opacity=0.8,fill=gray!20,draw=none](-8.734,3.004)--(-8.726,3.022)--(-8.778,3.031)--(-8.784,3.017)--cycle;
\draw(-8.734,3.004)--(-8.726,3.022);
\draw(-8.778,3.031)--(-8.784,3.017);
\filldraw[fill opacity=0.8,fill=gray!20](-8.691,2.978)--(-8.693,3.015)--(-8.786,3.022)--(-8.767,2.984)--cycle;
\filldraw[fill opacity=0.5,fill=gray!20](-9.82,-.885)--(-9.862,-.797)--(-10.281,-.653)--(-10.254,-.736)--cycle;
\filldraw[fill opacity=0.8,fill=gray!20](-8.803,3.123)--(-8.798,3.18)--(-8.871,3.198)--(-8.878,3.142)--cycle;
\filldraw[fill opacity=0.8,fill=gray!20](-8.633,3.327)--(-8.656,3.359)--(-8.685,3.358)--(-8.688,3.325)--cycle;
\filldraw[fill opacity=0.8,fill=gray!20](-8.327,4.24)--(-8.339,4.287)--(-8.412,4.305)--(-8.393,4.256)--cycle;
\filldraw[fill opacity=0.8,fill=gray!20,draw=none](-8.69,3.018)--(-8.69,3.028)--(-8.713,3.041)--(-8.769,3.045)--(-8.763,3.033)--cycle;
\draw(-8.69,3.018)--(-8.69,3.028);
\draw(-8.713,3.041)--(-8.769,3.045)--(-8.763,3.033);
\filldraw[fill opacity=0.8,fill=gray!20,draw=none](-9.043,-.99)--(-9.034,-1.021)--(-9.001,-1.018)--(-9.005,-.98)--cycle;
\draw(-9.034,-1.021)--(-9.001,-1.018)--(-9.005,-.98);
\filldraw[fill opacity=0.8,fill=gray!20,draw=none](-9.035,-1.017)--(-9.034,-1.021)--(-9.039,-1.02)--cycle;
\draw(-9.034,-1.021)--(-9.039,-1.02);
\filldraw[fill opacity=0.8,fill=gray!20,draw=none](-9.155,-1.022)--(-9.165,-1.036)--(-9.12,-1.083)--(-9.12,-1.024)--cycle;
\draw(-9.12,-1.083)--(-9.12,-1.024);
\filldraw[fill opacity=0.8,fill=gray!20,draw=none](-9.043,-.99)--(-9.166,-1.021)--(-9.064,-1.025)--(-9.034,-1.021)--cycle;
\draw(-9.064,-1.025)--(-9.034,-1.021);
\filldraw[fill opacity=0.8,fill=gray!20,draw=none](-9.043,-.99)--(-9.037,-.989)--(-9.035,-1.017)--(-9.036,-1.016)--cycle;
\filldraw[fill opacity=0.8,fill=gray!20,draw=none](-9.092,-1.118)--(-9.085,-1.085)--(-9.085,-1.06)--cycle;
\draw(-9.085,-1.085)--(-9.085,-1.06);
\filldraw[fill opacity=0.8,fill=gray!20,draw=none](-9.037,-.989)--(-9.026,-.987)--(-9.022,-1.006)--(-9.035,-1.017)--cycle;
\draw(-9.026,-.987)--(-9.022,-1.006);
\filldraw[fill opacity=0.8,fill=gray!20,draw=none](-9.092,-1.118)--(-9.085,-1.137)--(-9.085,-1.085)--cycle;
\draw(-9.085,-1.137)--(-9.085,-1.085);
\filldraw[fill opacity=0.8,fill=gray!20,draw=none](-9.067,-1.093)--(-9.092,-1.118)--(-9.085,-1.06)--cycle;
\filldraw[fill opacity=0.8,fill=gray!20,draw=none](-9.067,-1.093)--(-9.085,-1.06)--(-9.049,-1.075)--cycle;
\draw(-9.085,-1.06)--(-9.049,-1.075);
\filldraw[fill opacity=0.8,fill=gray!20,draw=none](-9.097,-1.08)--(-9.049,-1.075)--(-9.011,-1.019)--(-9.098,-1.028)--cycle;
\draw(-9.097,-1.08)--(-9.049,-1.075);
\draw(-9.011,-1.019)--(-9.098,-1.028);
\filldraw[fill opacity=0.8,fill=gray!20,draw=none](-9.022,-1.006)--(-9.021,-1.014)--(-9.022,-1.024)--(-9.035,-1.017)--cycle;
\draw(-9.022,-1.006)--(-9.021,-1.014);
\filldraw[fill opacity=0.8,fill=gray!20,draw=none](-9.166,-1.021)--(-9.178,-1.024)--(-9.165,-1.036)--(-9.064,-1.025)--cycle;
\draw(-9.165,-1.036)--(-9.064,-1.025);
\filldraw[fill opacity=0.8,fill=gray!20,draw=none](-9.407,-1.076)--(-9.394,-1.069)--(-9.369,-1.098)--cycle;
\filldraw[fill opacity=0.8,fill=gray!20,draw=none](-9.277,-1.138)--(-9.092,-1.118)--(-9.097,-1.08)--(-9.22,-1.094)--cycle;
\draw(-9.277,-1.138)--(-9.092,-1.118);
\draw(-9.097,-1.08)--(-9.22,-1.094);
\filldraw[fill opacity=0.8,fill=gray!20,draw=none](-9.152,-1.018)--(-9.165,-1.021)--(-9.165,-1.036)--cycle;
\draw(-9.165,-1.021)--(-9.165,-1.036);
\filldraw[fill opacity=0.8,fill=gray!20,draw=none](-9.149,-1.018)--(-9.079,-1.02)--(-9.021,-1.014)--(-9.026,-.987)--cycle;
\draw(-9.021,-1.014)--(-9.026,-.987);
\filldraw[fill opacity=0.5,fill=gray!20](-8.641,-.892)--(-8.693,-.902)--(-9.167,-1.022)--(-9.106,-1.009)--cycle;
\filldraw[fill opacity=0.8,fill=gray!20,draw=none](-8.582,3.263)--(-8.566,3.262)--(-8.574,3.274)--(-8.594,3.27)--cycle;
\draw(-8.566,3.262)--(-8.574,3.274)--(-8.594,3.27);
\filldraw[fill opacity=0.8,fill=gray!20](-8.552,3.296)--(-8.59,3.336)--(-8.633,3.327)--(-8.612,3.285)--cycle;
\filldraw[fill opacity=0.8,fill=gray!20,draw=none](-8.612,2.982)--(-8.634,2.995)--(-8.691,2.984)--(-8.691,2.978)--cycle;
\draw(-8.691,2.984)--(-8.691,2.978)--(-8.612,2.982);
\filldraw[fill opacity=0.8,fill=gray!20](-8.786,3.236)--(-8.767,3.287)--(-8.82,3.3)--(-8.852,3.252)--cycle;
\filldraw[fill opacity=0.8,fill=gray!20,draw=none](-8.672,3.016)--(-8.69,3.028)--(-8.69,3.018)--cycle;
\draw(-8.69,3.028)--(-8.69,3.018);
\filldraw[fill opacity=0.8,fill=gray!20,draw=none](-9.369,-1.098)--(-9.369,-1.176)--(-9.377,-1.188)--(-9.377,-1.141)--cycle;
\draw(-9.369,-1.098)--(-9.369,-1.176)--(-9.377,-1.188)--(-9.377,-1.141);
\filldraw[fill opacity=0.8,fill=gray!20,draw=none](-8.846,2.876)--(-8.785,3.015)--(-8.797,3.024)--(-8.825,3.039)--(-8.878,2.918)--cycle;
\draw(-8.846,2.876)--(-8.785,3.015);
\draw(-8.825,3.039)--(-8.878,2.918);
\filldraw[fill opacity=0.8,fill=gray!20,draw=none](-8.797,3.024)--(-8.823,3.044)--(-8.825,3.039)--cycle;
\draw(-8.823,3.044)--(-8.825,3.039);
\filldraw[fill opacity=0.8,fill=gray!20,draw=none](-8.774,3.14)--(-8.774,3.155)--(-8.778,3.146)--cycle;
\draw(-8.774,3.155)--(-8.778,3.146);
\filldraw[fill opacity=0.8,fill=gray!20](-8.798,3.18)--(-8.786,3.236)--(-8.852,3.252)--(-8.871,3.198)--cycle;
\filldraw[fill opacity=0.8,fill=gray!20,draw=none](-8.774,3.11)--(-8.753,3.09)--(-8.751,3.094)--(-8.774,3.14)--cycle;
\draw(-8.753,3.09)--(-8.751,3.094);
\filldraw[fill opacity=0.8,fill=gray!20](-8.283,4.547)--(-8.254,4.578)--(-8.273,4.583)--(-8.321,4.556)--cycle;
\filldraw[fill opacity=0.8,fill=gray!20,draw=none](-8.669,3.014)--(-8.675,3.018)--(-8.676,3.015)--cycle;
\draw(-8.675,3.018)--(-8.676,3.015);
\filldraw[fill opacity=0.8,fill=gray!20](-8.689,3.261)--(-8.687,3.298)--(-8.732,3.301)--(-8.753,3.266)--cycle;
\filldraw[fill opacity=0.8,fill=gray!20,draw=none](-8.582,3.263)--(-8.594,3.27)--(-8.615,3.266)--cycle;
\draw(-8.594,3.27)--(-8.615,3.266);
\filldraw[fill opacity=0.8,fill=gray!20,draw=none](-8.145,4.561)--(-8.144,4.562)--(-8.175,4.582)--(-8.197,4.577)--(-8.185,4.561)--cycle;
\draw(-8.144,4.562)--(-8.175,4.582)--(-8.197,4.577)--(-8.185,4.561);
\filldraw[fill opacity=0.8,fill=gray!20,draw=none](-8.639,3.296)--(-8.641,3.3)--(-8.645,3.3)--cycle;
\draw(-8.639,3.296)--(-8.641,3.3)--(-8.645,3.3);
\filldraw[fill opacity=0.8,fill=gray!20,draw=none](-8.787,2.761)--(-8.693,2.977)--(-8.695,2.978)--(-8.707,2.983)--(-8.748,2.971)--(-8.843,2.753)--cycle;
\draw(-8.787,2.761)--(-8.693,2.977);
\draw(-8.748,2.971)--(-8.843,2.753);
\filldraw[fill opacity=0.8,fill=gray!20,draw=none](-8.693,2.977)--(-8.693,2.977)--(-8.695,2.978)--cycle;
\draw(-8.693,2.977)--(-8.693,2.977);
\filldraw[fill opacity=0.8,fill=gray!20,draw=none](-8.694,3.017)--(-8.676,3.015)--(-8.675,3.018)--(-8.71,3.039)--cycle;
\draw(-8.676,3.015)--(-8.675,3.018);
\filldraw[fill opacity=0.8,fill=gray!20,draw=none](-8.615,3.266)--(-8.594,3.27)--(-8.639,3.296)--(-8.625,3.267)--cycle;
\draw(-8.615,3.266)--(-8.594,3.27);
\draw(-8.639,3.296)--(-8.625,3.267);
\filldraw[fill opacity=0.8,fill=gray!20,draw=none](-8.625,3.267)--(-8.639,3.296)--(-8.645,3.3)--(-8.687,3.298)--(-8.688,3.278)--cycle;
\draw(-8.625,3.267)--(-8.639,3.296);
\draw(-8.645,3.3)--(-8.687,3.298)--(-8.688,3.278);
\filldraw[fill opacity=0.8,fill=gray!20](-8.064,4.251)--(-8.045,4.3)--(-8.128,4.285)--(-8.137,4.237)--cycle;
\filldraw[fill opacity=0.8,fill=gray!20,draw=none](-8.695,2.978)--(-8.703,2.984)--(-8.707,2.983)--cycle;
\filldraw[fill opacity=0.8,fill=gray!20,draw=none](-9.46,-.996)--(-9.451,-.995)--(-9.412,-1.049)--(-9.455,-1.053)--cycle;
\draw(-9.412,-1.049)--(-9.455,-1.053)--(-9.46,-.996)--(-9.451,-.995);
\filldraw[fill opacity=0.8,fill=gray!20,draw=none](-9.394,-1.069)--(-9.407,-1.076)--(-9.447,-1.052)--(-9.412,-1.049)--cycle;
\draw(-9.447,-1.052)--(-9.412,-1.049);
\filldraw[fill opacity=0.8,fill=gray!20](-8.339,4.287)--(-8.343,4.341)--(-8.419,4.36)--(-8.412,4.305)--cycle;
\filldraw[fill opacity=0.8,fill=gray!20,draw=none](-8.533,3.246)--(-8.523,3.248)--(-8.552,3.296)--(-8.572,3.292)--cycle;
\draw(-8.533,3.246)--(-8.523,3.248)--(-8.552,3.296)--(-8.572,3.292);
\filldraw[fill opacity=0.8,fill=gray!20,draw=none](-8.472,3.008)--(-8.768,3.137)--(-8.689,3.058)--(-8.463,2.959)--cycle;
\draw(-8.689,3.058)--(-8.463,2.959)--(-8.472,3.008)--(-8.768,3.137);
\filldraw[fill opacity=0.8,fill=gray!20,draw=none](-8.472,3.008)--(-8.768,3.137)--(-8.689,3.058)--(-8.463,2.959)--cycle;
\draw(-8.689,3.058)--(-8.463,2.959)--(-8.472,3.008)--(-8.768,3.137);
\filldraw[fill opacity=0.8,fill=gray!20,draw=none](-8.199,4.422)--(-8.184,4.393)--(-8.128,4.395)--(-8.137,4.451)--(-8.2,4.449)--cycle;
\draw(-8.184,4.393)--(-8.128,4.395)--(-8.137,4.451)--(-8.2,4.449);
\filldraw[fill opacity=0.8,fill=gray!20,draw=none](-7.815,4.579)--(-7.824,4.601)--(-7.831,4.613)--(-7.815,4.576)--cycle;
\draw(-7.831,4.613)--(-7.815,4.576);
\filldraw[fill opacity=0.8,fill=gray!20](-7.717,4.704)--(-8.32,4.417)--(-8.311,4.369)--(-7.709,4.656)--cycle;
\filldraw[fill opacity=0.8,fill=gray!20,draw=none](-8.707,2.983)--(-8.703,2.984)--(-8.734,3.004)--(-8.738,2.994)--cycle;
\draw(-8.734,3.004)--(-8.738,2.994);
\filldraw[fill opacity=0.8,fill=gray!20](-8.045,4.411)--(-8.064,4.466)--(-8.137,4.451)--(-8.128,4.395)--cycle;
\filldraw[fill opacity=0.8,fill=gray!20,draw=none](-7.984,4.468)--(-8.29,4.322)--(-8.257,4.285)--(-7.968,4.423)--cycle;
\draw(-7.984,4.468)--(-8.29,4.322)--(-8.257,4.285)--(-7.968,4.423);
\filldraw[fill opacity=0.8,fill=gray!20,draw=none](-8.785,3.015)--(-8.784,3.017)--(-8.797,3.024)--cycle;
\draw(-8.785,3.015)--(-8.784,3.017);
\filldraw[fill opacity=0.8,fill=gray!20,draw=none](-9.436,-1.114)--(-9.441,-1.106)--(-9.369,-1.098)--(-9.306,-1.134)--cycle;
\draw(-9.436,-1.114)--(-9.441,-1.106)--(-9.369,-1.098);
\filldraw[fill opacity=0.8,fill=gray!20,draw=none](-9.455,-1.053)--(-9.447,-1.052)--(-9.369,-1.098)--(-9.441,-1.106)--cycle;
\draw(-9.369,-1.098)--(-9.441,-1.106)--(-9.455,-1.053)--(-9.447,-1.052);
\filldraw[fill opacity=0.8,fill=gray!20,draw=none](-9.436,-1.114)--(-9.306,-1.134)--(-9.419,-1.146)--cycle;
\draw(-9.306,-1.134)--(-9.419,-1.146)--(-9.436,-1.114);
\filldraw[fill opacity=0.8,fill=gray!20,draw=none](-9.369,-1.098)--(-9.377,-1.141)--(-9.377,-1.09)--cycle;
\draw(-9.377,-1.141)--(-9.377,-1.09);
\filldraw[fill opacity=0.8,fill=gray!20,draw=none](-9.369,-1.098)--(-9.377,-1.09)--(-9.377,-1.064)--cycle;
\draw(-9.377,-1.09)--(-9.377,-1.064);
\filldraw[fill opacity=0.8,fill=gray!20,draw=none](-8.707,2.983)--(-8.738,2.994)--(-8.748,2.971)--cycle;
\draw(-8.738,2.994)--(-8.748,2.971);
\filldraw[fill opacity=0.8,fill=gray!20,draw=none](-8.694,3.017)--(-8.71,3.039)--(-8.717,3.044)--(-8.726,3.022)--cycle;
\draw(-8.717,3.044)--(-8.726,3.022);
\filldraw[fill opacity=0.8,fill=gray!20,draw=none](-8.774,3.066)--(-8.766,3.059)--(-8.763,3.067)--cycle;
\draw(-8.766,3.059)--(-8.763,3.067);
\filldraw[fill opacity=0.8,fill=gray!20,draw=none](-8.634,2.995)--(-8.652,3.007)--(-8.693,3.012)--(-8.691,2.984)--cycle;
\draw(-8.693,3.012)--(-8.691,2.984);
\filldraw[fill opacity=0.8,fill=gray!20,draw=none](-8.764,3.133)--(-8.747,3.083)--(-8.689,3.058)--cycle;
\draw(-8.747,3.083)--(-8.689,3.058);
\filldraw[fill opacity=0.8,fill=gray!20,draw=none](-8.764,3.133)--(-8.747,3.083)--(-8.689,3.058)--cycle;
\draw(-8.747,3.083)--(-8.689,3.058);
\filldraw[fill opacity=0.8,fill=gray!20,draw=none](-8.726,3.022)--(-8.717,3.044)--(-8.762,3.069)--(-8.778,3.031)--cycle;
\draw(-8.726,3.022)--(-8.717,3.044);
\draw(-8.762,3.069)--(-8.778,3.031);
\filldraw[fill opacity=0.8,fill=gray!20,draw=none](-8.744,3.059)--(-8.777,3.078)--(-8.776,3.072)--cycle;
\draw(-8.777,3.078)--(-8.776,3.072);
\filldraw[fill opacity=0.8,fill=gray!20,draw=none](-8.711,3.04)--(-8.711,3.041)--(-8.713,3.041)--cycle;
\draw(-8.711,3.041)--(-8.713,3.041);
\filldraw[fill opacity=0.8,fill=gray!20](-8.308,4.505)--(-8.283,4.547)--(-8.321,4.556)--(-8.361,4.518)--cycle;
\filldraw[fill opacity=0.8,fill=gray!20,draw=none](-8.803,2.846)--(-8.743,2.984)--(-8.784,3.017)--(-8.846,2.876)--cycle;
\draw(-8.803,2.846)--(-8.743,2.984);
\draw(-8.784,3.017)--(-8.846,2.876);
\filldraw[fill opacity=0.8,fill=gray!20,draw=none](-8.041,4.303)--(-8.045,4.307)--(-8.045,4.3)--cycle;
\draw(-8.045,4.307)--(-8.045,4.3)--(-8.041,4.303);
\filldraw[fill opacity=0.8,fill=gray!20,draw=none](-8.743,2.984)--(-8.734,3.004)--(-8.784,3.017)--cycle;
\draw(-8.743,2.984)--(-8.734,3.004);
\filldraw[fill opacity=0.8,fill=gray!20,draw=none](-8.737,3.056)--(-8.737,3.055)--(-8.713,3.041)--(-8.711,3.041)--(-8.716,3.047)--cycle;
\draw(-8.713,3.041)--(-8.711,3.041);
\filldraw[fill opacity=0.8,fill=gray!20,draw=none](-8.717,3.044)--(-8.715,3.047)--(-8.753,3.09)--(-8.762,3.069)--cycle;
\draw(-8.717,3.044)--(-8.715,3.047);
\draw(-8.753,3.09)--(-8.762,3.069);
\filldraw[fill opacity=0.8,fill=gray!20,draw=none](-8.747,3.083)--(-8.751,3.094)--(-8.753,3.09)--cycle;
\draw(-8.751,3.094)--(-8.753,3.09);
\filldraw[fill opacity=0.8,fill=gray!20,draw=none](-9.377,-1.141)--(-9.306,-1.134)--(-9.266,-1.154)--(-9.362,-1.164)--cycle;
\draw(-9.377,-1.141)--(-9.306,-1.134);
\draw(-9.266,-1.154)--(-9.362,-1.164);
\filldraw[fill opacity=0.8,fill=gray!20,draw=none](-9.377,-1.163)--(-9.377,-1.179)--(-9.363,-1.192)--(-9.362,-1.19)--(-9.362,-1.164)--cycle;
\draw(-9.377,-1.163)--(-9.377,-1.179);
\draw(-9.362,-1.19)--(-9.362,-1.164);
\filldraw[fill opacity=0.8,fill=gray!20,draw=none](-9.377,-1.141)--(-9.377,-1.163)--(-9.362,-1.164)--cycle;
\draw(-9.377,-1.141)--(-9.377,-1.163);
\filldraw[fill opacity=0.8,fill=gray!20,draw=none](-9.377,-1.09)--(-9.377,-1.141)--(-9.362,-1.164)--(-9.362,-1.094)--cycle;
\draw(-9.377,-1.09)--(-9.377,-1.141);
\draw(-9.362,-1.164)--(-9.362,-1.094);
\filldraw[fill opacity=0.8,fill=gray!20,draw=none](-9.377,-1.141)--(-9.362,-1.164)--(-9.385,-1.167)--cycle;
\draw(-9.362,-1.164)--(-9.385,-1.167);
\filldraw[fill opacity=0.8,fill=gray!20,draw=none](-9.362,-1.164)--(-9.266,-1.154)--(-9.256,-1.156)--(-9.327,-1.163)--cycle;
\draw(-9.362,-1.164)--(-9.266,-1.154);
\draw(-9.256,-1.156)--(-9.327,-1.163);
\filldraw[fill opacity=0.8,fill=gray!20,draw=none](-9.362,-1.136)--(-9.362,-1.19)--(-9.327,-1.163)--(-9.327,-1.148)--cycle;
\draw(-9.362,-1.136)--(-9.362,-1.19);
\draw(-9.327,-1.163)--(-9.327,-1.148);
\filldraw[fill opacity=0.8,fill=gray!20,draw=none](-9.394,-1.168)--(-9.362,-1.164)--(-9.327,-1.163)--(-9.369,-1.168)--cycle;
\draw(-9.327,-1.163)--(-9.369,-1.168)--(-9.394,-1.168)--(-9.362,-1.164);
\filldraw[fill opacity=0.8,fill=gray!20,draw=none](-9.362,-1.094)--(-9.362,-1.136)--(-9.327,-1.148)--(-9.327,-1.112)--cycle;
\draw(-9.362,-1.094)--(-9.362,-1.136);
\draw(-9.327,-1.148)--(-9.327,-1.112);
\filldraw[fill opacity=0.8,fill=gray!20,draw=none](-9.327,-1.086)--(-9.327,-1.163)--(-9.277,-1.138)--cycle;
\draw(-9.327,-1.086)--(-9.327,-1.163);
\filldraw[fill opacity=0.8,fill=gray!20](-9.369,-1.168)--(-9.042,-1.132)--(-9.021,-1.111)--(-9.348,-1.146)--cycle;
\filldraw[fill opacity=0.8,fill=gray!20,draw=none](-8.744,3.059)--(-8.737,3.055)--(-8.737,3.056)--cycle;
\filldraw[fill opacity=0.8,fill=gray!20,draw=none](-8.71,3.039)--(-8.715,3.047)--(-8.717,3.044)--cycle;
\draw(-8.715,3.047)--(-8.717,3.044);
\filldraw[fill opacity=0.8,fill=gray!20,draw=none](-8.744,3.059)--(-8.737,3.056)--(-8.738,3.079)--(-8.74,3.082)--(-8.779,3.085)--(-8.777,3.078)--cycle;
\draw(-8.74,3.082)--(-8.779,3.085)--(-8.777,3.078);
\filldraw[fill opacity=0.8,fill=gray!20,draw=none](-8.737,3.056)--(-8.716,3.047)--(-8.738,3.079)--cycle;
\filldraw[fill opacity=0.8,fill=gray!20](-8.229,4.543)--(-8.226,4.576)--(-8.254,4.578)--(-8.283,4.547)--cycle;
\filldraw[fill opacity=0.8,fill=gray!20](-8.232,4.196)--(-8.234,4.233)--(-8.327,4.24)--(-8.308,4.202)--cycle;
\filldraw[fill opacity=0.8,fill=gray!20,draw=none](-8.714,3.016)--(-8.789,3.032)--(-8.786,3.022)--cycle;
\draw(-8.789,3.032)--(-8.786,3.022)--(-8.714,3.016);
\filldraw[fill opacity=0.8,fill=gray!20](-8.691,3.281)--(-8.688,3.325)--(-8.742,3.329)--(-8.767,3.287)--cycle;
\filldraw[fill opacity=0.8,fill=gray!20,draw=none](-8.038,4.326)--(-8.043,4.324)--(-8.044,4.316)--cycle;
\draw(-8.043,4.324)--(-8.044,4.316);
\filldraw[fill opacity=0.8,fill=gray!20,draw=none](-8.049,4.321)--(-8.055,4.318)--(-8.038,4.326)--(-8.031,4.329)--cycle;
\draw(-8.049,4.321)--(-8.055,4.318);
\draw(-8.038,4.326)--(-8.031,4.329);
\filldraw[fill opacity=0.8,fill=gray!20,draw=none](-8.145,4.561)--(-8.185,4.561)--(-8.175,4.548)--cycle;
\draw(-8.185,4.561)--(-8.175,4.548);
\filldraw[fill opacity=0.5,fill=gray!20](-9.031,2.906)--(-9.079,2.706)--(-8.723,2.584)--(-8.631,2.768)--cycle;
\filldraw[fill opacity=0.5,fill=gray!20](-8.57,2.704)--(-9.031,2.906)--(-8.631,2.768)--(-8.169,2.567)--cycle;
\filldraw[fill opacity=0.8,fill=gray!20](-8.343,4.341)--(-8.339,4.398)--(-8.412,4.416)--(-8.419,4.36)--cycle;
\filldraw[fill opacity=0.5,fill=gray!20](-8.961,2.969)--(-9.031,2.906)--(-8.631,2.768)--(-8.541,2.825)--cycle;
\filldraw[fill opacity=0.8,fill=gray!20,draw=none](-8.185,4.561)--(-8.197,4.577)--(-8.226,4.576)--(-8.228,4.557)--cycle;
\draw(-8.185,4.561)--(-8.197,4.577)--(-8.226,4.576)--(-8.228,4.557);
\filldraw[fill opacity=0.8,fill=gray!20,draw=none](-8.005,4.515)--(-8.311,4.369)--(-8.29,4.322)--(-7.984,4.468)--cycle;
\draw(-8.005,4.515)--(-8.311,4.369)--(-8.29,4.322)--(-7.984,4.468);
\filldraw[fill opacity=0.8,fill=gray!20,draw=none](-8.713,3.22)--(-8.711,3.221)--(-8.695,3.262)--(-8.753,3.266)--(-8.769,3.224)--cycle;
\draw(-8.695,3.262)--(-8.753,3.266)--(-8.769,3.224)--(-8.713,3.22);
\filldraw[fill opacity=0.8,fill=gray!20,draw=none](-8.74,3.082)--(-8.778,3.13)--(-8.782,3.13)--(-8.779,3.085)--cycle;
\draw(-8.778,3.13)--(-8.782,3.13)--(-8.779,3.085)--(-8.74,3.082);
\filldraw[fill opacity=0.8,fill=gray!20,draw=none](-8.562,3.254)--(-8.539,3.252)--(-8.572,3.292)--(-8.612,3.285)--cycle;
\draw(-8.572,3.292)--(-8.612,3.285);
\filldraw[fill opacity=0.8,fill=gray!20,draw=none](-9.419,-1.146)--(-9.377,-1.141)--(-9.385,-1.167)--(-9.394,-1.168)--cycle;
\draw(-9.385,-1.167)--(-9.394,-1.168)--(-9.419,-1.146)--(-9.377,-1.141);
\filldraw[fill opacity=0.8,fill=gray!20,draw=none](-9.377,-1.064)--(-9.377,-1.09)--(-9.362,-1.094)--(-9.362,-1.06)--cycle;
\draw(-9.377,-1.064)--(-9.377,-1.09);
\draw(-9.362,-1.094)--(-9.362,-1.06);
\filldraw[fill opacity=0.8,fill=gray!20,draw=none](-9.362,-1.06)--(-9.362,-1.094)--(-9.327,-1.112)--(-9.327,-1.086)--cycle;
\draw(-9.362,-1.06)--(-9.362,-1.094);
\draw(-9.327,-1.112)--(-9.327,-1.086);
\filldraw[fill opacity=0.8,fill=gray!20,draw=none](-9.333,-1.101)--(-9.455,-1.048)--(-9.455,-1.053)--(-9.441,-1.106)--(-9.419,-1.146)--(-9.394,-1.168)--(-9.369,-1.168)--(-9.348,-1.146)--(-9.334,-1.106)--cycle;
\draw(-9.455,-1.048)--(-9.455,-1.053)--(-9.441,-1.106)--(-9.419,-1.146)--(-9.394,-1.168)--(-9.369,-1.168)--(-9.348,-1.146)--(-9.334,-1.106)--(-9.333,-1.101);
\filldraw[fill opacity=0.8,fill=gray!20,draw=none](-9.333,-1.101)--(-9.333,-1.095)--(-9.424,-.997)--(-9.458,-.982)--(-9.46,-.996)--(-9.455,-1.048)--cycle;
\draw(-9.333,-1.101)--(-9.333,-1.095);
\draw(-9.458,-.982)--(-9.46,-.996)--(-9.455,-1.048);
\filldraw[fill opacity=0.8,fill=gray!20,draw=none](-9.092,-1.18)--(-9.092,-1.196)--(-9.085,-1.183)--(-9.085,-1.159)--cycle;
\draw(-9.092,-1.18)--(-9.092,-1.196)--(-9.085,-1.183)--(-9.085,-1.159);
\filldraw[fill opacity=0.8,fill=gray!20,draw=none](-8.612,3.285)--(-8.633,3.327)--(-8.685,3.325)--cycle;
\draw(-8.612,3.285)--(-8.633,3.327)--(-8.685,3.325);
\filldraw[fill opacity=0.8,fill=gray!20,draw=none](-8.714,3.016)--(-8.693,3.015)--(-8.694,3.033)--(-8.75,3.066)--(-8.798,3.069)--(-8.789,3.032)--cycle;
\draw(-8.714,3.016)--(-8.693,3.015)--(-8.694,3.033);
\draw(-8.75,3.066)--(-8.798,3.069)--(-8.789,3.032);
\filldraw[fill opacity=0.8,fill=gray!20](-8.153,4.2)--(-8.137,4.237)--(-8.234,4.233)--(-8.232,4.196)--cycle;
\filldraw[fill opacity=0.8,fill=gray!20,draw=none](-8.085,4.293)--(-8.045,4.3)--(-8.045,4.307)--(-8.107,4.316)--cycle;
\draw(-8.085,4.293)--(-8.045,4.3)--(-8.045,4.307);
\filldraw[fill opacity=0.8,fill=gray!20](-8.327,4.454)--(-8.308,4.505)--(-8.361,4.518)--(-8.393,4.47)--cycle;
\filldraw[fill opacity=0.5,fill=gray!20](-9.079,2.706)--(-8.906,2.631)--(-8.55,2.508)--(-8.723,2.584)--cycle;
\filldraw[fill opacity=0.8,fill=gray!20,draw=none](-8.636,3.264)--(-8.625,3.267)--(-8.688,3.278)--(-8.689,3.261)--cycle;
\draw(-8.688,3.278)--(-8.689,3.261)--(-8.636,3.264);
\filldraw[fill opacity=0.8,fill=gray!20,draw=none](-8.071,4.311)--(-8.045,4.307)--(-8.043,4.324)--cycle;
\draw(-8.045,4.307)--(-8.043,4.324);
\filldraw[fill opacity=0.5,fill=gray!20](-7.395,1.551)--(-7.369,1.603)--(-7.31,1.118)--(-7.338,1.088)--cycle;
\filldraw[fill opacity=0.8,fill=gray!20,draw=none](-8.652,3.007)--(-8.666,3.016)--(-8.693,3.015)--(-8.693,3.012)--cycle;
\draw(-8.666,3.016)--(-8.693,3.015)--(-8.693,3.012);
\filldraw[fill opacity=0.8,fill=gray!20](-8.339,4.398)--(-8.327,4.454)--(-8.393,4.47)--(-8.412,4.416)--cycle;
\filldraw[fill opacity=0.8,fill=gray!20,draw=none](-9.092,-1.118)--(-9.092,-1.18)--(-9.085,-1.159)--(-9.085,-1.137)--cycle;
\draw(-9.092,-1.118)--(-9.092,-1.18);
\draw(-9.085,-1.159)--(-9.085,-1.137);
\filldraw[fill opacity=0.8,fill=gray!20,draw=none](-8.672,3.253)--(-8.636,3.264)--(-8.663,3.263)--cycle;
\draw(-8.636,3.264)--(-8.663,3.263);
\filldraw[fill opacity=0.8,fill=gray!20,draw=none](-8.738,3.079)--(-8.738,3.082)--(-8.74,3.082)--cycle;
\draw(-8.738,3.082)--(-8.74,3.082);
\filldraw[fill opacity=0.8,fill=gray!20,draw=none](-9.424,-.997)--(-9.456,-.962)--(-9.458,-.982)--cycle;
\draw(-9.456,-.962)--(-9.458,-.982);
\filldraw[fill opacity=0.8,fill=gray!20,draw=none](-8.043,4.324)--(-8.055,4.318)--(-8.071,4.311)--cycle;
\draw(-8.055,4.318)--(-8.071,4.311);
\filldraw[fill opacity=0.5,fill=gray!20](-7.543,2.042)--(-7.549,2.104)--(-7.369,1.65)--(-7.369,1.603)--cycle;
\filldraw[fill opacity=0.8,fill=gray!20,draw=none](-8.74,3.082)--(-8.738,3.082)--(-8.747,3.127)--(-8.778,3.13)--cycle;
\draw(-8.74,3.082)--(-8.738,3.082);
\draw(-8.747,3.127)--(-8.778,3.13);
\filldraw[fill opacity=0.8,fill=gray!20,draw=none](-8.649,3.631)--(-8.766,3.423)--(-8.677,3.388)--(-8.594,3.609)--cycle;
\draw(-8.649,3.631)--(-8.766,3.423)--(-8.677,3.388)--(-8.594,3.609);
\filldraw[fill opacity=0.8,fill=gray!20,draw=none](-8.649,3.631)--(-8.682,3.644)--(-8.766,3.423)--cycle;
\draw(-8.682,3.644)--(-8.766,3.423)--(-8.649,3.631);
\filldraw[fill opacity=0.8,fill=gray!20,draw=none](-8.602,3.611)--(-8.608,3.613)--(-8.709,3.653)--(-8.747,3.671)--cycle;
\draw(-8.602,3.611)--(-8.608,3.613)--(-8.709,3.653);
\filldraw[fill opacity=0.8,fill=gray!20,draw=none](-8.722,3.661)--(-8.545,4.009)--(-8.609,4.036)--(-8.747,3.671)--cycle;
\draw(-8.722,3.661)--(-8.545,4.009)--(-8.609,4.036)--(-8.747,3.671);
\filldraw[fill opacity=0.5,fill=gray!20](-7.894,1.927)--(-7.721,1.852)--(-7.573,1.479)--(-7.746,1.555)--cycle;
\filldraw[fill opacity=0.8,fill=gray!20,draw=none](-8.612,3.285)--(-8.685,3.325)--(-8.688,3.325)--(-8.691,3.281)--cycle;
\draw(-8.685,3.325)--(-8.688,3.325)--(-8.691,3.281)--(-8.612,3.285);
\filldraw[fill opacity=0.8,fill=gray!20,draw=none](-8.737,3.207)--(-8.737,3.221)--(-8.769,3.224)--(-8.777,3.185)--cycle;
\draw(-8.737,3.221)--(-8.769,3.224)--(-8.777,3.185);
\filldraw[fill opacity=0.8,fill=gray!20,draw=none](-8.747,3.127)--(-8.763,3.176)--(-8.779,3.177)--(-8.782,3.13)--cycle;
\draw(-8.763,3.176)--(-8.779,3.177)--(-8.782,3.13)--(-8.747,3.127);
\filldraw[fill opacity=0.8,fill=gray!20,draw=none](-8.672,3.253)--(-8.663,3.263)--(-8.689,3.261)--(-8.69,3.248)--cycle;
\draw(-8.663,3.263)--(-8.689,3.261)--(-8.69,3.248);
\filldraw[fill opacity=0.8,fill=gray!20,draw=none](-8.562,3.254)--(-8.612,3.285)--(-8.604,3.258)--cycle;
\draw(-8.612,3.285)--(-8.604,3.258);
\filldraw[fill opacity=0.8,fill=gray!20,draw=none](-9.426,-.994)--(-9.417,-.996)--(-9.417,-.987)--(-9.456,-.962)--cycle;
\filldraw[fill opacity=0.5,fill=gray!20,draw=none](-9.567,-.974)--(-9.443,-.97)--(-9.458,-.961)--(-9.604,-.956)--cycle;
\draw(-9.458,-.961)--(-9.604,-.956);
\filldraw[fill opacity=0.8,fill=gray!20,draw=none](-8.711,3.221)--(-8.69,3.235)--(-8.689,3.261)--(-8.695,3.262)--cycle;
\draw(-8.69,3.235)--(-8.689,3.261)--(-8.695,3.262);
\filldraw[fill opacity=0.5,fill=gray!20,draw=none](-9.682,-.914)--(-9.732,-.888)--(-9.82,-.885)--(-9.81,-.898)--cycle;
\draw(-9.732,-.888)--(-9.82,-.885)--(-9.81,-.898);
\filldraw[fill opacity=0.8,fill=gray!20,draw=none](-8.74,3.174)--(-8.738,3.178)--(-8.737,3.207)--(-8.777,3.185)--(-8.779,3.177)--cycle;
\draw(-8.777,3.185)--(-8.779,3.177)--(-8.74,3.174);
\filldraw[fill opacity=0.8,fill=gray!20,draw=none](-8.18,4.545)--(-8.175,4.548)--(-8.185,4.561)--(-8.228,4.557)--(-8.229,4.543)--cycle;
\draw(-8.175,4.548)--(-8.185,4.561);
\draw(-8.228,4.557)--(-8.229,4.543)--(-8.18,4.545);
\filldraw[fill opacity=0.8,fill=gray!20,draw=none](-8.672,3.253)--(-8.69,3.248)--(-8.69,3.235)--cycle;
\draw(-8.69,3.248)--(-8.69,3.235);
\filldraw[fill opacity=0.8,fill=gray!20,draw=none](-8.666,3.016)--(-8.694,3.033)--(-8.693,3.015)--cycle;
\draw(-8.694,3.033)--(-8.693,3.015)--(-8.666,3.016);
\filldraw[fill opacity=0.8,fill=gray!20,draw=none](-9.377,-1.179)--(-9.377,-1.188)--(-9.366,-1.197)--(-9.363,-1.192)--cycle;
\draw(-9.377,-1.179)--(-9.377,-1.188)--(-9.366,-1.197);
\filldraw[fill opacity=0.8,fill=gray!20,draw=none](-8.737,3.207)--(-8.713,3.22)--(-8.737,3.221)--cycle;
\draw(-8.713,3.22)--(-8.737,3.221);
\filldraw[fill opacity=0.8,fill=gray!20,draw=none](-8.747,3.127)--(-8.738,3.174)--(-8.763,3.176)--cycle;
\draw(-8.738,3.174)--(-8.763,3.176);
\filldraw[fill opacity=0.8,fill=gray!20,draw=none](-8.714,3.063)--(-8.715,3.069)--(-8.756,3.12)--(-8.803,3.123)--(-8.798,3.069)--cycle;
\draw(-8.756,3.12)--(-8.803,3.123)--(-8.798,3.069)--(-8.714,3.063);
\filldraw[fill opacity=0.8,fill=gray!20,draw=none](-8.714,3.231)--(-8.693,3.233)--(-8.691,3.281)--(-8.767,3.287)--(-8.786,3.236)--cycle;
\draw(-8.693,3.233)--(-8.691,3.281)--(-8.767,3.287)--(-8.786,3.236)--(-8.714,3.231);
\filldraw[fill opacity=0.8,fill=gray!20,draw=none](-8.738,3.178)--(-8.711,3.22)--(-8.713,3.22)--(-8.737,3.207)--cycle;
\draw(-8.711,3.22)--(-8.713,3.22);
\filldraw[fill opacity=0.8,fill=gray!20,draw=none](-8.74,3.174)--(-8.738,3.174)--(-8.738,3.178)--cycle;
\draw(-8.74,3.174)--(-8.738,3.174);
\filldraw[fill opacity=0.8,fill=gray!20,draw=none](-8.711,3.221)--(-8.713,3.22)--(-8.711,3.22)--cycle;
\draw(-8.713,3.22)--(-8.711,3.22);
\filldraw[fill opacity=0.8,fill=gray!20](-8.234,4.233)--(-8.235,4.28)--(-8.339,4.287)--(-8.327,4.24)--cycle;
\filldraw[fill opacity=0.8,fill=gray!20,draw=none](-8.272,4.502)--(-8.23,4.522)--(-8.229,4.543)--(-8.283,4.547)--(-8.308,4.505)--cycle;
\draw(-8.23,4.522)--(-8.229,4.543)--(-8.283,4.547)--(-8.308,4.505)--(-8.272,4.502);
\filldraw[fill opacity=0.8,fill=gray!20,draw=none](-8.604,3.258)--(-8.612,3.285)--(-8.691,3.281)--(-8.691,3.272)--cycle;
\draw(-8.604,3.258)--(-8.612,3.285)--(-8.691,3.281)--(-8.691,3.272);
\filldraw[fill opacity=0.8,fill=gray!20,draw=none](-8.688,3.03)--(-8.694,3.04)--(-8.694,3.033)--cycle;
\draw(-8.694,3.04)--(-8.694,3.033);
\filldraw[fill opacity=0.8,fill=gray!20,draw=none](-8.18,4.545)--(-8.176,4.545)--(-8.175,4.548)--cycle;
\draw(-8.18,4.545)--(-8.176,4.545);
\filldraw[fill opacity=0.8,fill=gray!20,draw=none](-8.085,4.293)--(-8.107,4.316)--(-8.126,4.318)--(-8.128,4.285)--cycle;
\draw(-8.126,4.318)--(-8.128,4.285)--(-8.085,4.293);
\filldraw[fill opacity=0.8,fill=gray!20,draw=none](-8.722,3.661)--(-8.682,3.644)--(-8.545,4.009)--cycle;
\draw(-8.682,3.644)--(-8.545,4.009)--(-8.722,3.661);
\filldraw[fill opacity=0.8,fill=gray!20](-8.792,3.687)--(-8.715,3.655)--(-8.494,4.241)--cycle;
\filldraw[fill opacity=0.8,fill=gray!20](-8.494,4.241)--(-8.571,4.273)--(-8.792,3.687)--cycle;
\filldraw[fill opacity=0.8,fill=gray!20,draw=none](-8.694,3.033)--(-8.694,3.04)--(-8.71,3.063)--(-8.75,3.066)--cycle;
\draw(-8.694,3.033)--(-8.694,3.04);
\draw(-8.71,3.063)--(-8.75,3.066);
\filldraw[fill opacity=0.8,fill=gray!20,draw=none](-8.199,4.515)--(-8.176,4.545)--(-8.18,4.545)--(-8.199,4.536)--cycle;
\draw(-8.176,4.545)--(-8.18,4.545);
\filldraw[fill opacity=0.8,fill=gray!20,draw=none](-9.092,-1.118)--(-9.021,-1.111)--(-9.006,-1.07)--(-9.097,-1.08)--cycle;
\draw(-9.092,-1.118)--(-9.021,-1.111)--(-9.006,-1.07)--(-9.097,-1.08);
\filldraw[fill opacity=0.8,fill=gray!20,draw=none](-8.18,4.545)--(-8.229,4.543)--(-8.23,4.522)--cycle;
\draw(-8.18,4.545)--(-8.229,4.543)--(-8.23,4.522);
\filldraw[fill opacity=0.8,fill=gray!20,draw=none](-9.401,-1)--(-9.401,-1.007)--(-9.365,-1.023)--(-9.332,-1.016)--cycle;
\filldraw[fill opacity=0.5,fill=gray!20,draw=none](-9.4,-1.001)--(-9.394,-1.001)--(-9.443,-.97)--(-9.455,-.97)--cycle;
\draw(-9.4,-1.001)--(-9.394,-1.001);
\filldraw[fill opacity=0.8,fill=gray!20,draw=none](-8.756,3.12)--(-8.734,3.146)--(-8.744,3.176)--(-8.798,3.18)--(-8.803,3.123)--cycle;
\draw(-8.744,3.176)--(-8.798,3.18)--(-8.803,3.123)--(-8.756,3.12);
\filldraw[fill opacity=0.8,fill=gray!20,draw=none](-8.75,3.177)--(-8.713,3.196)--(-8.713,3.23)--(-8.786,3.236)--(-8.798,3.18)--cycle;
\draw(-8.713,3.23)--(-8.786,3.236)--(-8.798,3.18)--(-8.75,3.177);
\filldraw[fill opacity=0.8,fill=gray!20](-8.137,4.237)--(-8.128,4.285)--(-8.235,4.28)--(-8.234,4.233)--cycle;
\filldraw[fill opacity=0.8,fill=gray!20,draw=none](-8.652,3.245)--(-8.604,3.258)--(-8.634,3.263)--cycle;
\filldraw[fill opacity=0.8,fill=gray!20,draw=none](-9.102,-1.147)--(-9.12,-1.198)--(-9.12,-1.207)--(-9.092,-1.196)--(-9.092,-1.14)--cycle;
\draw(-9.12,-1.198)--(-9.12,-1.207)--(-9.092,-1.196)--(-9.092,-1.14);
\filldraw[fill opacity=0.8,fill=gray!20,draw=none](-9.12,-1.16)--(-9.12,-1.198)--(-9.102,-1.147)--cycle;
\draw(-9.12,-1.16)--(-9.12,-1.198);
\filldraw[fill opacity=0.8,fill=gray!20,draw=none](-9.165,-1.204)--(-9.165,-1.214)--(-9.12,-1.207)--(-9.12,-1.16)--cycle;
\draw(-9.165,-1.204)--(-9.165,-1.214)--(-9.12,-1.207)--(-9.12,-1.16);
\filldraw[fill opacity=0.8,fill=gray!20,draw=none](-9.165,-1.179)--(-9.141,-1.18)--(-9.12,-1.16)--cycle;
\filldraw[fill opacity=0.8,fill=gray!20,draw=none](-9.165,-1.179)--(-9.165,-1.204)--(-9.141,-1.18)--cycle;
\draw(-9.165,-1.179)--(-9.165,-1.204);
\filldraw[fill opacity=0.8,fill=gray!20,draw=none](-9.149,-1.186)--(-9.213,-1.158)--(-9.155,-1.145)--(-9.099,-1.169)--cycle;
\draw(-9.155,-1.145)--(-9.099,-1.169)--(-9.149,-1.186)--(-9.213,-1.158);
\filldraw[fill opacity=0.8,fill=gray!20,draw=none](-9.102,-1.147)--(-9.092,-1.14)--(-9.092,-1.118)--cycle;
\draw(-9.092,-1.14)--(-9.092,-1.118);
\filldraw[fill opacity=0.8,fill=gray!20,draw=none](-9.108,-1.143)--(-9.1,-1.141)--(-9.092,-1.118)--cycle;
\filldraw[fill opacity=0.8,fill=gray!20,draw=none](-9.108,-1.143)--(-9.12,-1.16)--(-9.102,-1.147)--(-9.1,-1.141)--cycle;
\filldraw[fill opacity=0.8,fill=gray!20,draw=none](-9.099,-1.169)--(-9.155,-1.145)--(-9.092,-1.118)--(-9.056,-1.134)--cycle;
\draw(-9.092,-1.118)--(-9.056,-1.134)--(-9.099,-1.169)--(-9.155,-1.145);
\filldraw[fill opacity=0.8,fill=gray!20,draw=none](-8.652,3.245)--(-8.634,3.263)--(-8.691,3.272)--(-8.693,3.233)--cycle;
\draw(-8.691,3.272)--(-8.693,3.233);
\filldraw[fill opacity=0.8,fill=gray!20,draw=none](-8.458,3.558)--(-8.577,3.35)--(-8.481,3.315)--(-8.398,3.536)--cycle;
\draw(-8.458,3.558)--(-8.577,3.35)--(-8.481,3.315)--(-8.398,3.536);
\filldraw[fill opacity=0.8,fill=gray!20,draw=none](-8.458,3.558)--(-8.494,3.571)--(-8.577,3.35)--cycle;
\draw(-8.494,3.571)--(-8.577,3.35)--(-8.458,3.558);
\filldraw[fill opacity=0.8,fill=gray!20,draw=none](-8.321,3.509)--(-8.379,3.528)--(-8.488,3.567)--(-8.495,3.57)--cycle;
\draw(-8.379,3.528)--(-8.488,3.567)--(-8.495,3.57);
\filldraw[fill opacity=0.5,fill=gray!20](-7.573,.645)--(-7.395,.612)--(-7.561,.157)--(-7.721,.24)--cycle;
\filldraw[fill opacity=0.5,fill=gray!20](-7.721,.24)--(-7.561,.157)--(-7.825,-.247)--(-7.956,-.119)--cycle;
\filldraw[fill opacity=0.8,fill=gray!20,draw=none](-8.21,4.5)--(-8.199,4.515)--(-8.199,4.536)--(-8.23,4.522)--(-8.232,4.499)--cycle;
\draw(-8.23,4.522)--(-8.232,4.499)--(-8.21,4.5);
\filldraw[fill opacity=0.8,fill=gray!20,draw=none](-9.426,-.994)--(-9.417,-1.005)--(-9.417,-.996)--cycle;
\filldraw[fill opacity=0.5,fill=gray!20,draw=none](-9.4,-1.001)--(-9.455,-.97)--(-9.726,-.979)--(-9.708,-.991)--cycle;
\draw(-9.726,-.979)--(-9.708,-.991)--(-9.4,-1.001);
\filldraw[fill opacity=0.8,fill=gray!20,draw=none](-8.714,3.063)--(-8.71,3.063)--(-8.715,3.069)--cycle;
\draw(-8.714,3.063)--(-8.71,3.063);
\filldraw[fill opacity=0.8,fill=gray!20,draw=none](-9.22,-1.193)--(-9.22,-1.208)--(-9.165,-1.207)--(-9.165,-1.179)--cycle;
\draw(-9.22,-1.193)--(-9.22,-1.208);
\draw(-9.165,-1.207)--(-9.165,-1.179);
\filldraw[fill opacity=0.8,fill=gray!20,draw=none](-9.22,-1.171)--(-9.195,-1.186)--(-9.165,-1.179)--cycle;
\filldraw[fill opacity=0.8,fill=gray!20,draw=none](-9.199,-1.18)--(-9.256,-1.156)--(-9.213,-1.158)--(-9.149,-1.186)--cycle;
\draw(-9.213,-1.158)--(-9.149,-1.186)--(-9.199,-1.18)--(-9.256,-1.156);
\filldraw[fill opacity=0.8,fill=gray!20,draw=none](-8.715,3.069)--(-8.724,3.118)--(-8.756,3.12)--cycle;
\draw(-8.724,3.118)--(-8.756,3.12);
\filldraw[fill opacity=0.8,fill=gray!20,draw=none](-8.734,3.146)--(-8.756,3.12)--(-8.724,3.118)--cycle;
\draw(-8.756,3.12)--(-8.724,3.118);
\filldraw[fill opacity=0.8,fill=gray!20,draw=none](-8.666,3.23)--(-8.652,3.245)--(-8.693,3.233)--(-8.693,3.229)--cycle;
\draw(-8.693,3.233)--(-8.693,3.229)--(-8.666,3.23);
\filldraw[fill opacity=0.8,fill=gray!20,draw=none](-9.363,-1.192)--(-9.366,-1.197)--(-9.362,-1.2)--(-9.362,-1.193)--cycle;
\draw(-9.366,-1.197)--(-9.362,-1.2)--(-9.362,-1.193);
\filldraw[fill opacity=0.8,fill=gray!20,draw=none](-8.272,4.502)--(-8.308,4.505)--(-8.317,4.481)--cycle;
\draw(-8.272,4.502)--(-8.308,4.505)--(-8.317,4.481);
\filldraw[fill opacity=0.8,fill=gray!20](-8.235,4.28)--(-8.236,4.334)--(-8.343,4.341)--(-8.339,4.287)--cycle;
\filldraw[fill opacity=0.5,fill=gray!20,draw=none](-8.33,2.85)--(-8.361,2.864)--(-8.377,2.868)--(-7.997,2.591)--(-7.932,2.575)--cycle;
\draw(-8.361,2.864)--(-8.377,2.868)--(-7.997,2.591)--(-7.932,2.575);
\filldraw[fill opacity=0.5,fill=gray!20](-8.169,-.573)--(-8.18,-.621)--(-8.599,-.856)--(-8.57,-.798)--cycle;
\filldraw[fill opacity=0.5,fill=gray!20](-9.769,-.95)--(-9.82,-.885)--(-10.254,-.736)--(-10.211,-.798)--cycle;
\filldraw[fill opacity=0.5,fill=gray!20](-7.523,1.068)--(-7.338,1.088)--(-7.395,.612)--(-7.573,.645)--cycle;
\filldraw[fill opacity=0.8,fill=gray!20,draw=none](-8.713,3.196)--(-8.75,3.177)--(-8.714,3.174)--cycle;
\draw(-8.75,3.177)--(-8.714,3.174);
\filldraw[fill opacity=0.8,fill=gray!20,draw=none](-8.714,3.231)--(-8.693,3.229)--(-8.693,3.233)--cycle;
\draw(-8.714,3.231)--(-8.693,3.229)--(-8.693,3.233);
\filldraw[fill opacity=0.8,fill=gray!20,draw=none](-8.713,3.196)--(-8.694,3.207)--(-8.693,3.229)--(-8.713,3.23)--cycle;
\draw(-8.694,3.207)--(-8.693,3.229)--(-8.713,3.23);
\filldraw[fill opacity=0.8,fill=gray!20,draw=none](-8.734,3.146)--(-8.715,3.168)--(-8.714,3.174)--(-8.744,3.176)--cycle;
\draw(-8.714,3.174)--(-8.744,3.176);
\filldraw[fill opacity=0.8,fill=gray!20,draw=none](-8.107,4.316)--(-8.125,4.333)--(-8.126,4.318)--cycle;
\draw(-8.125,4.333)--(-8.126,4.318);
\filldraw[fill opacity=0.8,fill=gray!20,draw=none](-8.272,4.502)--(-8.232,4.499)--(-8.23,4.522)--cycle;
\draw(-8.272,4.502)--(-8.232,4.499)--(-8.23,4.522);
\filldraw[fill opacity=0.8,fill=gray!20,draw=none](-8.734,3.146)--(-8.724,3.118)--(-8.715,3.168)--cycle;
\filldraw[fill opacity=0.8,fill=gray!20,draw=none](-8.688,3.212)--(-8.666,3.23)--(-8.68,3.23)--cycle;
\draw(-8.666,3.23)--(-8.68,3.23);
\filldraw[fill opacity=0.8,fill=gray!20,draw=none](-8.199,4.515)--(-8.21,4.5)--(-8.199,4.501)--cycle;
\draw(-8.21,4.5)--(-8.199,4.501);
\filldraw[fill opacity=0.8,fill=gray!20,draw=none](-8.26,4.449)--(-8.233,4.474)--(-8.232,4.499)--(-8.272,4.502)--(-8.317,4.481)--(-8.327,4.454)--cycle;
\draw(-8.233,4.474)--(-8.232,4.499)--(-8.272,4.502);
\draw(-8.317,4.481)--(-8.327,4.454)--(-8.26,4.449);
\filldraw[fill opacity=0.8,fill=gray!20,draw=none](-8.688,3.212)--(-8.68,3.23)--(-8.693,3.229)--(-8.694,3.207)--cycle;
\draw(-8.68,3.23)--(-8.693,3.229)--(-8.694,3.207);
\filldraw[fill opacity=0.8,fill=gray!20,draw=none](-8.713,3.196)--(-8.714,3.174)--(-8.71,3.174)--(-8.694,3.199)--(-8.694,3.207)--cycle;
\draw(-8.714,3.174)--(-8.71,3.174);
\draw(-8.694,3.199)--(-8.694,3.207);
\filldraw[fill opacity=0.8,fill=gray!20,draw=none](-9.363,-1.192)--(-9.362,-1.193)--(-9.362,-1.19)--cycle;
\draw(-9.362,-1.193)--(-9.362,-1.19);
\filldraw[fill opacity=0.8,fill=gray!20,draw=none](-9.12,-1.083)--(-9.12,-1.108)--(-9.092,-1.118)--cycle;
\draw(-9.12,-1.083)--(-9.12,-1.108);
\filldraw[fill opacity=0.8,fill=gray!20,draw=none](-9.12,-1.108)--(-9.12,-1.145)--(-9.108,-1.143)--(-9.092,-1.118)--cycle;
\draw(-9.12,-1.108)--(-9.12,-1.145);
\filldraw[fill opacity=0.8,fill=gray!20,draw=none](-9.067,-1.093)--(-9.063,-1.101)--(-9.092,-1.118)--cycle;
\filldraw[fill opacity=0.8,fill=gray!20,draw=none](-9.063,-1.101)--(-9.05,-1.124)--(-9.056,-1.134)--(-9.092,-1.118)--cycle;
\draw(-9.05,-1.124)--(-9.056,-1.134)--(-9.092,-1.118);
\filldraw[fill opacity=0.8,fill=gray!20,draw=none](-8.128,4.285)--(-8.126,4.314)--(-8.128,4.318)--(-8.235,4.305)--(-8.235,4.28)--cycle;
\draw(-8.235,4.305)--(-8.235,4.28)--(-8.128,4.285)--(-8.126,4.314);
\filldraw[fill opacity=0.5,fill=gray!20](-7.835,2.479)--(-7.87,2.536)--(-7.579,2.153)--(-7.549,2.104)--cycle;
\filldraw[fill opacity=0.8,fill=gray!20,draw=none](-8.688,3.212)--(-8.694,3.207)--(-8.694,3.199)--cycle;
\draw(-8.694,3.207)--(-8.694,3.199);
\filldraw[fill opacity=0.8,fill=gray!20,draw=none](-9.362,-1.19)--(-9.362,-1.2)--(-9.327,-1.21)--(-9.327,-1.163)--cycle;
\draw(-9.362,-1.19)--(-9.362,-1.2)--(-9.327,-1.21)--(-9.327,-1.163);
\filldraw[fill opacity=0.8,fill=gray!20,draw=none](-8.267,3.522)--(-8.281,3.493)--(-8.278,3.492)--cycle;
\draw(-8.267,3.522)--(-8.281,3.493)--(-8.278,3.492);
\filldraw[fill opacity=0.8,fill=gray!20,draw=none](-8.715,3.168)--(-8.71,3.174)--(-8.714,3.174)--cycle;
\draw(-8.71,3.174)--(-8.714,3.174);
\filldraw[fill opacity=0.8,fill=gray!20,draw=none](-9.014,-1.071)--(-9.006,-1.07)--(-9.006,-1.065)--cycle;
\draw(-9.014,-1.071)--(-9.006,-1.07)--(-9.006,-1.065);
\filldraw[fill opacity=0.8,fill=gray!20,draw=none](-9.049,-1.075)--(-9.014,-1.071)--(-9.006,-1.065)--(-9.001,-1.018)--(-9.011,-1.019)--cycle;
\draw(-9.049,-1.075)--(-9.014,-1.071);
\draw(-9.006,-1.065)--(-9.001,-1.018)--(-9.011,-1.019);
\filldraw[fill opacity=0.5,fill=gray!20](-8.791,-.533)--(-8.618,-.608)--(-9,-.705)--(-9.173,-.629)--cycle;
\filldraw[fill opacity=0.8,fill=gray!20,draw=none](-8.206,4.471)--(-8.199,4.501)--(-8.214,4.5)--cycle;
\draw(-8.199,4.501)--(-8.214,4.5);
\filldraw[fill opacity=0.8,fill=gray!20,draw=none](-8.237,4.391)--(-8.235,4.394)--(-8.234,4.447)--(-8.327,4.454)--(-8.339,4.398)--cycle;
\draw(-8.235,4.394)--(-8.234,4.447)--(-8.327,4.454)--(-8.339,4.398)--(-8.237,4.391);
\filldraw[fill opacity=0.8,fill=gray!20](-8.236,4.334)--(-8.235,4.391)--(-8.339,4.398)--(-8.343,4.341)--cycle;
\filldraw[fill opacity=0.8,fill=gray!20](-8.374,3.525)--(-8.281,3.493)--(-8.06,4.079)--cycle;
\filldraw[fill opacity=0.8,fill=gray!20,draw=none](-8.212,4.448)--(-8.206,4.471)--(-8.214,4.5)--(-8.232,4.499)--(-8.234,4.447)--cycle;
\draw(-8.214,4.5)--(-8.232,4.499)--(-8.234,4.447)--(-8.212,4.448);
\filldraw[fill opacity=0.8,fill=gray!20,draw=none](-8.126,4.314)--(-8.126,4.318)--(-8.128,4.318)--cycle;
\draw(-8.126,4.314)--(-8.126,4.318);
\filldraw[fill opacity=0.8,fill=gray!20,draw=none](-8.128,4.318)--(-8.126,4.318)--(-8.125,4.333)--(-8.138,4.338)--(-8.141,4.338)--cycle;
\draw(-8.126,4.318)--(-8.125,4.333);
\draw(-8.138,4.338)--(-8.141,4.338);
\filldraw[fill opacity=0.8,fill=gray!20,draw=none](-8.128,4.318)--(-8.141,4.338)--(-8.236,4.334)--(-8.235,4.305)--cycle;
\draw(-8.141,4.338)--(-8.236,4.334)--(-8.235,4.305);
\filldraw[fill opacity=0.8,fill=gray!20,draw=none](-9.063,-1.101)--(-9.067,-1.093)--(-9.049,-1.075)--(-9.032,-1.083)--cycle;
\draw(-9.049,-1.075)--(-9.032,-1.083);
\filldraw[fill opacity=0.8,fill=gray!20,draw=none](-9.417,-1)--(-9.417,-1.005)--(-9.333,-1.095)--(-9.329,-1.054)--(-9.33,-1.038)--cycle;
\draw(-9.333,-1.095)--(-9.329,-1.054)--(-9.33,-1.038);
\filldraw[fill opacity=0.5,fill=gray!20](-8.889,3.007)--(-8.961,2.969)--(-8.541,2.825)--(-8.456,2.859)--cycle;
\filldraw[fill opacity=0.5,fill=gray!20,draw=none](-9.567,-.974)--(-9.604,-.956)--(-9.769,-.95)--(-9.726,-.979)--cycle;
\draw(-9.604,-.956)--(-9.769,-.95)--(-9.726,-.979);
\filldraw[fill opacity=0.5,fill=gray!20](-7.956,-.119)--(-7.825,-.247)--(-8.169,-.573)--(-8.261,-.409)--cycle;
\filldraw[fill opacity=0.8,fill=gray!20,draw=none](-8.649,3.631)--(-8.456,3.974)--(-8.545,4.009)--(-8.682,3.644)--cycle;
\draw(-8.649,3.631)--(-8.456,3.974)--(-8.545,4.009)--(-8.682,3.644);
\filldraw[fill opacity=0.8,fill=gray!20,draw=none](-8.556,3.595)--(-8.594,3.609)--(-8.677,3.388)--cycle;
\draw(-8.594,3.609)--(-8.677,3.388)--(-8.556,3.595);
\filldraw[fill opacity=0.8,fill=gray!20,draw=none](-8.556,3.595)--(-8.677,3.388)--(-8.577,3.35)--(-8.494,3.571)--cycle;
\draw(-8.556,3.595)--(-8.677,3.388)--(-8.577,3.35)--(-8.494,3.571);
\filldraw[fill opacity=0.8,fill=gray!20,draw=none](-8.489,3.568)--(-8.495,3.57)--(-8.602,3.611)--(-8.682,3.644)--cycle;
\draw(-8.495,3.57)--(-8.602,3.611);
\filldraw[fill opacity=0.5,fill=gray!20](-8.599,-.856)--(-8.641,-.892)--(-9.106,-1.009)--(-9.05,-.97)--cycle;
\filldraw[fill opacity=0.8,fill=gray!20,draw=none](-8.206,4.471)--(-8.212,4.448)--(-8.2,4.449)--cycle;
\draw(-8.212,4.448)--(-8.2,4.449);
\filldraw[fill opacity=0.8,fill=gray!20,draw=none](-9.22,-1.119)--(-9.22,-1.171)--(-9.165,-1.179)--(-9.165,-1.109)--cycle;
\draw(-9.22,-1.119)--(-9.22,-1.171);
\draw(-9.165,-1.179)--(-9.165,-1.109);
\filldraw[fill opacity=0.8,fill=gray!20,draw=none](-9.277,-1.138)--(-9.22,-1.171)--(-9.22,-1.119)--cycle;
\draw(-9.22,-1.171)--(-9.22,-1.119);
\filldraw[fill opacity=0.8,fill=gray!20,draw=none](-9.241,-1.154)--(-9.277,-1.138)--(-9.256,-1.156)--(-9.199,-1.18)--cycle;
\draw(-9.256,-1.156)--(-9.199,-1.18)--(-9.241,-1.154)--(-9.277,-1.138);
\filldraw[fill opacity=0.8,fill=gray!20,draw=none](-8.141,4.338)--(-8.138,4.338)--(-8.143,4.339)--cycle;
\draw(-8.141,4.338)--(-8.138,4.338);
\filldraw[fill opacity=0.8,fill=gray!20,draw=none](-8.168,4.337)--(-8.141,4.338)--(-8.143,4.339)--(-8.171,4.348)--cycle;
\draw(-8.168,4.337)--(-8.141,4.338);
\filldraw[fill opacity=0.8,fill=gray!20,draw=none](-8.171,4.348)--(-8.143,4.339)--(-8.179,4.375)--cycle;
\filldraw[fill opacity=0.8,fill=gray!20,draw=none](-8.26,4.449)--(-8.234,4.447)--(-8.233,4.474)--cycle;
\draw(-8.26,4.449)--(-8.234,4.447)--(-8.233,4.474);
\filldraw[fill opacity=0.8,fill=gray!20,draw=none](-9.049,-1.075)--(-9.034,-1.021)--(-9.022,-1.027)--cycle;
\draw(-9.034,-1.021)--(-9.022,-1.027);
\filldraw[fill opacity=0.8,fill=gray!20,draw=none](-9.028,-1.084)--(-9.049,-1.075)--(-9.022,-1.027)--(-9.018,-1.029)--cycle;
\draw(-9.022,-1.027)--(-9.018,-1.029)--(-9.028,-1.084)--(-9.049,-1.075);
\filldraw[fill opacity=0.8,fill=gray!20,draw=none](-9.108,-1.143)--(-9.12,-1.145)--(-9.12,-1.16)--cycle;
\draw(-9.12,-1.145)--(-9.12,-1.16);
\filldraw[fill opacity=0.8,fill=gray!20](-8.06,4.079)--(-8.153,4.112)--(-8.374,3.525)--cycle;
\filldraw[fill opacity=0.5,fill=gray!20](-10.2,-.368)--(-10.027,-.444)--(-10.333,-.221)--(-10.506,-.146)--cycle;
\filldraw[fill opacity=0.5,fill=gray!20](-10.292,-.553)--(-10.2,-.368)--(-10.506,-.146)--(-10.636,-.302)--cycle;
\filldraw[fill opacity=0.8,fill=gray!20,draw=none](-8.168,4.337)--(-8.171,4.348)--(-8.235,4.367)--(-8.236,4.334)--cycle;
\draw(-8.235,4.367)--(-8.236,4.334)--(-8.168,4.337);
\filldraw[fill opacity=0.8,fill=gray!20,draw=none](-9.05,-1.124)--(-9.063,-1.101)--(-9.032,-1.083)--(-9.028,-1.084)--cycle;
\draw(-9.032,-1.083)--(-9.028,-1.084)--(-9.05,-1.124);
\filldraw[fill opacity=0.8,fill=gray!20,draw=none](-8.171,4.348)--(-8.179,4.375)--(-8.196,4.392)--(-8.235,4.391)--(-8.235,4.367)--cycle;
\draw(-8.196,4.392)--(-8.235,4.391)--(-8.235,4.367);
\filldraw[fill opacity=0.8,fill=gray!20,draw=none](-8.199,4.422)--(-8.2,4.449)--(-8.214,4.448)--cycle;
\draw(-8.2,4.449)--(-8.214,4.448);
\filldraw[fill opacity=0.5,fill=gray!20](-9.831,-.754)--(-10.292,-.553)--(-10.636,-.302)--(-10.175,-.504)--cycle;
\filldraw[fill opacity=0.5,fill=gray!20](-10.281,-.653)--(-10.292,-.553)--(-10.636,-.302)--(-10.642,-.391)--cycle;
\filldraw[fill opacity=0.8,fill=gray!20,draw=none](-8.198,4.392)--(-8.199,4.422)--(-8.214,4.448)--(-8.234,4.447)--(-8.235,4.391)--cycle;
\draw(-8.214,4.448)--(-8.234,4.447)--(-8.235,4.391)--(-8.198,4.392);
\filldraw[fill opacity=0.8,fill=gray!20,draw=none](-8.179,4.375)--(-8.184,4.393)--(-8.196,4.392)--cycle;
\draw(-8.184,4.393)--(-8.196,4.392);
\filldraw[fill opacity=0.8,fill=gray!20,draw=none](-8.199,4.422)--(-8.198,4.394)--(-8.196,4.392)--(-8.184,4.393)--cycle;
\draw(-8.196,4.392)--(-8.184,4.393);
\filldraw[fill opacity=0.8,fill=gray!20,draw=none](-8.649,3.631)--(-8.594,3.609)--(-8.456,3.974)--cycle;
\draw(-8.594,3.609)--(-8.456,3.974)--(-8.649,3.631);
\filldraw[fill opacity=0.8,fill=gray!20,draw=none](-8.198,4.394)--(-8.198,4.392)--(-8.196,4.392)--cycle;
\draw(-8.198,4.392)--(-8.196,4.392);
\filldraw[fill opacity=0.8,fill=gray!20,draw=none](-9.035,-1.017)--(-9.018,-1.027)--(-9.018,-1.029)--(-9.034,-1.021)--cycle;
\draw(-9.018,-1.027)--(-9.018,-1.029)--(-9.034,-1.021);
\filldraw[fill opacity=0.8,fill=gray!20,draw=none](-9.165,-1.109)--(-9.165,-1.179)--(-9.12,-1.16)--(-9.12,-1.108)--cycle;
\draw(-9.165,-1.109)--(-9.165,-1.179);
\draw(-9.12,-1.16)--(-9.12,-1.108);
\filldraw[fill opacity=0.8,fill=gray!20,draw=none](-9.036,-1.016)--(-9.035,-1.017)--(-9.035,-1.017)--cycle;
\filldraw[fill opacity=0.8,fill=gray!20,draw=none](-8.398,3.536)--(-8.489,3.568)--(-8.546,3.59)--cycle;
\filldraw[fill opacity=0.8,fill=gray!20,draw=none](-8.458,3.558)--(-8.398,3.536)--(-8.26,3.902)--cycle;
\draw(-8.398,3.536)--(-8.26,3.902)--(-8.458,3.558);
\filldraw[fill opacity=0.8,fill=gray!20,draw=none](-9.327,-1.163)--(-9.327,-1.21)--(-9.277,-1.216)--(-9.277,-1.138)--cycle;
\draw(-9.327,-1.163)--(-9.327,-1.21)--(-9.277,-1.216)--(-9.277,-1.138);
\filldraw[fill opacity=0.8,fill=gray!20,draw=none](-8.237,4.391)--(-8.235,4.391)--(-8.235,4.394)--cycle;
\draw(-8.237,4.391)--(-8.235,4.391)--(-8.235,4.394);
\filldraw[fill opacity=0.8,fill=gray!20,draw=none](-9.22,-1.035)--(-9.22,-1.094)--(-9.165,-1.036)--(-9.165,-1.021)--cycle;
\draw(-9.22,-1.035)--(-9.22,-1.094);
\draw(-9.165,-1.036)--(-9.165,-1.021);
\filldraw[fill opacity=0.8,fill=gray!20,draw=none](-9.291,-1.101)--(-9.097,-1.08)--(-9.098,-1.028)--(-9.32,-1.053)--cycle;
\draw(-9.291,-1.101)--(-9.097,-1.08);
\draw(-9.098,-1.028)--(-9.32,-1.053);
\filldraw[fill opacity=0.8,fill=gray!20,draw=none](-9.165,-1.074)--(-9.165,-1.109)--(-9.12,-1.108)--(-9.12,-1.083)--cycle;
\draw(-9.165,-1.074)--(-9.165,-1.109);
\draw(-9.12,-1.108)--(-9.12,-1.083);
\filldraw[fill opacity=0.8,fill=gray!20,draw=none](-8.458,3.558)--(-8.26,3.902)--(-8.356,3.937)--(-8.494,3.571)--cycle;
\draw(-8.458,3.558)--(-8.26,3.902)--(-8.356,3.937)--(-8.494,3.571);
\filldraw[fill opacity=0.8,fill=gray!20,draw=none](-9.165,-1.036)--(-9.165,-1.074)--(-9.12,-1.083)--cycle;
\draw(-9.165,-1.036)--(-9.165,-1.074);
\filldraw[fill opacity=0.8,fill=gray!20](-8.387,4.199)--(-8.494,4.241)--(-8.715,3.655)--cycle;
\filldraw[fill opacity=0.8,fill=gray!20](-8.715,3.655)--(-8.608,3.613)--(-8.387,4.199)--cycle;
\filldraw[fill opacity=0.5,fill=gray!20,draw=none](-9.277,-1.015)--(-9.327,-1.009)--(-9.308,-1.012)--cycle;
\filldraw[fill opacity=0.8,fill=gray!20,draw=none](-9.349,-1.005)--(-9.322,-1.011)--(-9.315,-.982)--(-9.327,-.977)--cycle;
\draw(-9.315,-.982)--(-9.327,-.977);
\filldraw[fill opacity=0.5,fill=gray!20,draw=none](-9.347,-1.015)--(-9.332,-1.016)--(-9.394,-1.001)--(-9.4,-1.001)--cycle;
\draw(-9.394,-1.001)--(-9.4,-1.001);
\filldraw[fill opacity=0.8,fill=gray!20,draw=none](-9.22,-1.208)--(-9.22,-1.218)--(-9.165,-1.214)--(-9.165,-1.207)--cycle;
\draw(-9.22,-1.208)--(-9.22,-1.218)--(-9.165,-1.214)--(-9.165,-1.207);
\filldraw[fill opacity=0.5,fill=gray!20,draw=none](-9.346,-1.015)--(-9.4,-1.001)--(-9.708,-.991)--(-9.642,-1.005)--cycle;
\draw(-9.4,-1.001)--(-9.708,-.991)--(-9.642,-1.005)--(-9.346,-1.015);
\filldraw[fill opacity=0.8,fill=gray!20,draw=none](-9.22,-1.094)--(-9.22,-1.119)--(-9.165,-1.109)--(-9.165,-1.036)--cycle;
\draw(-9.22,-1.094)--(-9.22,-1.119);
\draw(-9.165,-1.109)--(-9.165,-1.036);
\filldraw[fill opacity=0.8,fill=gray!20,draw=none](-9.277,-1.138)--(-9.22,-1.119)--(-9.22,-1.094)--cycle;
\draw(-9.22,-1.119)--(-9.22,-1.094);
\filldraw[fill opacity=0.8,fill=gray!20,draw=none](-9.271,-1.105)--(-9.27,-1.111)--(-9.241,-1.154)--(-9.199,-1.18)--(-9.149,-1.186)--(-9.099,-1.169)--(-9.056,-1.134)--(-9.028,-1.084)--(-9.027,-1.079)--cycle;
\draw(-9.271,-1.105)--(-9.27,-1.111)--(-9.241,-1.154)--(-9.199,-1.18)--(-9.149,-1.186)--(-9.099,-1.169)--(-9.056,-1.134)--(-9.028,-1.084)--(-9.027,-1.079);
\filldraw[fill opacity=0.8,fill=gray!20,draw=none](-9.021,-1.014)--(-9.018,-1.027)--(-9.022,-1.024)--cycle;
\draw(-9.021,-1.014)--(-9.018,-1.027);
\filldraw[fill opacity=0.8,fill=gray!20,draw=none](-9.178,-1.024)--(-9.267,-1.047)--(-9.165,-1.036)--cycle;
\draw(-9.267,-1.047)--(-9.165,-1.036);
\filldraw[fill opacity=0.8,fill=gray!20,draw=none](-9.218,-1.035)--(-9.279,-1.051)--(-9.28,-1.057)--(-9.271,-1.105)--(-9.027,-1.079)--(-9.018,-1.029)--(-9.021,-1.014)--cycle;
\draw(-9.279,-1.051)--(-9.28,-1.057)--(-9.271,-1.105);
\draw(-9.027,-1.079)--(-9.018,-1.029)--(-9.021,-1.014);
\filldraw[fill opacity=0.8,fill=gray!20,draw=none](-8.494,3.571)--(-8.546,3.59)--(-8.594,3.609)--cycle;
\filldraw[fill opacity=0.8,fill=gray!20,draw=none](-8.556,3.595)--(-8.356,3.937)--(-8.456,3.974)--(-8.594,3.609)--cycle;
\draw(-8.556,3.595)--(-8.356,3.937)--(-8.456,3.974)--(-8.594,3.609);
\filldraw[fill opacity=0.5,fill=gray!20](-7.573,1.479)--(-7.395,1.551)--(-7.338,1.088)--(-7.523,1.068)--cycle;
\filldraw[fill opacity=0.8,fill=gray!20,draw=none](-8.556,3.595)--(-8.494,3.571)--(-8.356,3.937)--cycle;
\draw(-8.494,3.571)--(-8.356,3.937)--(-8.556,3.595);
\filldraw[fill opacity=0.5,fill=gray!20](-7.561,1.97)--(-7.543,2.042)--(-7.369,1.603)--(-7.395,1.551)--cycle;
\filldraw[fill opacity=0.8,fill=gray!20,draw=none](-9.277,-1.138)--(-9.277,-1.216)--(-9.22,-1.218)--(-9.22,-1.209)--cycle;
\draw(-9.277,-1.138)--(-9.277,-1.216)--(-9.22,-1.218)--(-9.22,-1.209);
\filldraw[fill opacity=0.8,fill=gray!20,draw=none](-9.293,-1.017)--(-9.343,-1.026)--(-9.344,-1.029)--(-9.296,-1.05)--cycle;
\draw(-9.344,-1.029)--(-9.296,-1.05);
\filldraw[fill opacity=0.8,fill=gray!20,draw=none](-9.22,-1.171)--(-9.22,-1.193)--(-9.195,-1.186)--cycle;
\draw(-9.22,-1.171)--(-9.22,-1.193);
\filldraw[fill opacity=0.8,fill=gray!20,draw=none](-9.256,-1.164)--(-9.22,-1.209)--(-9.22,-1.171)--cycle;
\draw(-9.22,-1.209)--(-9.22,-1.171);
\filldraw[fill opacity=0.8,fill=gray!20,draw=none](-9.277,-1.138)--(-9.256,-1.164)--(-9.22,-1.171)--cycle;
\filldraw[fill opacity=0.8,fill=gray!20,draw=none](-9.27,-1.111)--(-9.274,-1.109)--(-9.277,-1.138)--(-9.241,-1.154)--cycle;
\draw(-9.277,-1.138)--(-9.241,-1.154)--(-9.27,-1.111)--(-9.274,-1.109);
\filldraw[fill opacity=0.8,fill=gray!20,draw=none](-9.348,-1.146)--(-9.277,-1.138)--(-9.291,-1.101)--(-9.334,-1.106)--cycle;
\draw(-9.291,-1.101)--(-9.334,-1.106)--(-9.348,-1.146)--(-9.277,-1.138);
\filldraw[fill opacity=0.8,fill=gray!20](-8.488,3.567)--(-8.374,3.525)--(-8.153,4.112)--cycle;
\filldraw[fill opacity=0.5,fill=gray!20](-8.129,2.235)--(-7.956,2.16)--(-7.721,1.852)--(-7.894,1.927)--cycle;
\filldraw[fill opacity=0.8,fill=gray!20,draw=none](-9.244,-.919)--(-9.277,-.936)--(-9.277,-1.015)--(-9.22,-1.019)--(-9.22,-.984)--cycle;
\draw(-9.277,-.936)--(-9.277,-1.015);
\draw(-9.22,-1.019)--(-9.22,-.984);
\filldraw[fill opacity=0.8,fill=gray!20,draw=none](-9.277,-1.015)--(-9.277,-1.022)--(-9.22,-1.02)--(-9.22,-1.019)--cycle;
\draw(-9.277,-1.015)--(-9.277,-1.022);
\draw(-9.22,-1.02)--(-9.22,-1.019);
\filldraw[fill opacity=0.5,fill=gray!20,draw=none](-9.322,-1.011)--(-9.293,-1.017)--(-9.274,-1.018)--(-9.308,-1.012)--cycle;
\filldraw[fill opacity=0.5,fill=gray!20,draw=none](-9.277,-1.015)--(-9.308,-1.012)--(-9.274,-1.018)--(-9.242,-1.019)--(-9.237,-1.019)--cycle;
\draw(-9.242,-1.019)--(-9.237,-1.019);
\filldraw[fill opacity=0.8,fill=gray!20,draw=none](-9.322,-1.011)--(-9.293,-1.017)--(-9.291,-.992)--(-9.315,-.982)--cycle;
\draw(-9.291,-.992)--(-9.315,-.982);
\filldraw[fill opacity=0.5,fill=gray!20,draw=none](-9.25,-1.018)--(-9.237,-1.019)--(-9.189,-1.021)--cycle;
\draw(-9.237,-1.019)--(-9.189,-1.021);
\filldraw[fill opacity=0.5,fill=gray!20,draw=none](-9.277,-1.015)--(-9.25,-1.018)--(-9.22,-1.019)--cycle;
\filldraw[fill opacity=0.8,fill=gray!20,draw=none](-9.303,-1.017)--(-9.207,-1.02)--(-9.22,-.984)--(-9.291,-.992)--cycle;
\draw(-9.22,-.984)--(-9.291,-.992);
\filldraw[fill opacity=0.8,fill=gray!20,draw=none](-9.25,-.977)--(-9.22,-.984)--(-9.277,-.936)--cycle;
\filldraw[fill opacity=0.8,fill=gray!20,draw=none](-9.291,-.992)--(-9.244,-.987)--(-9.277,-.936)--cycle;
\draw(-9.291,-.992)--(-9.244,-.987);
\filldraw[fill opacity=0.8,fill=gray!20,draw=none](-9.247,-.962)--(-9.327,-.977)--(-9.27,-1.001)--cycle;
\draw(-9.327,-.977)--(-9.27,-1.001)--(-9.247,-.962);
\filldraw[fill opacity=0.8,fill=gray!20,draw=none](-9.285,-.969)--(-9.277,-.936)--(-9.327,-.977)--cycle;
\filldraw[fill opacity=0.5,fill=gray!20,draw=none](-8.262,2.819)--(-8.33,2.85)--(-7.932,2.575)--(-7.925,2.574)--cycle;
\draw(-7.932,2.575)--(-7.925,2.574)--(-8.262,2.819);
\filldraw[fill opacity=0.5,fill=gray!20](-9.708,-.991)--(-9.769,-.95)--(-10.211,-.798)--(-10.154,-.838)--cycle;
\filldraw[fill opacity=0.8,fill=gray!20,draw=none](-9.207,-1.02)--(-9.181,-1.021)--(-9.22,-.984)--cycle;
\filldraw[fill opacity=0.8,fill=gray!20,draw=none](-9.25,-.977)--(-9.244,-.987)--(-9.22,-.984)--cycle;
\draw(-9.244,-.987)--(-9.22,-.984);
\filldraw[fill opacity=0.8,fill=gray!20,draw=none](-9.244,-.919)--(-9.22,-.984)--(-9.22,-.907)--cycle;
\draw(-9.22,-.984)--(-9.22,-.907);
\filldraw[fill opacity=0.8,fill=gray!20,draw=none](-9.272,-1.014)--(-9.163,-1.017)--(-9.163,-.905)--(-9.199,-.916)--(-9.241,-.952)--(-9.27,-1.001)--cycle;
\draw(-9.163,-.905)--(-9.199,-.916)--(-9.241,-.952)--(-9.27,-1.001)--(-9.272,-1.014);
\filldraw[fill opacity=0.8,fill=gray!20,draw=none](-9.332,-1.016)--(-9.303,-1.017)--(-9.291,-.992)--(-9.334,-.997)--cycle;
\draw(-9.291,-.992)--(-9.334,-.997)--(-9.332,-1.016);
\filldraw[fill opacity=0.8,fill=gray!20,draw=none](-9.293,-1.017)--(-9.272,-1.014)--(-9.27,-1.001)--(-9.291,-.992)--cycle;
\draw(-9.272,-1.014)--(-9.27,-1.001)--(-9.291,-.992);
\filldraw[fill opacity=0.5,fill=gray!20,draw=none](-9.27,-1.003)--(-9.106,-1.009)--(-9.167,-1.022)--(-9.331,-1.016)--cycle;
\draw(-9.27,-1.003)--(-9.106,-1.009)--(-9.167,-1.022)--(-9.331,-1.016);
\filldraw[fill opacity=0.8,fill=gray!20,draw=none](-9.181,-1.021)--(-9.22,-1.02)--(-9.22,-1.035)--(-9.178,-1.024)--cycle;
\draw(-9.22,-1.02)--(-9.22,-1.035);
\filldraw[fill opacity=0.8,fill=gray!20,draw=none](-9.207,-1.02)--(-9.204,-1.031)--(-9.168,-1.022)--cycle;
\filldraw[fill opacity=0.8,fill=gray!20,draw=none](-9.149,-1.018)--(-9.218,-1.035)--(-9.079,-1.02)--cycle;
\filldraw[fill opacity=0.8,fill=gray!20,draw=none](-9.372,-1.007)--(-9.332,-1.016)--(-9.334,-.997)--(-9.348,-.944)--(-9.369,-.904)--(-9.372,-.902)--cycle;
\draw(-9.332,-1.016)--(-9.334,-.997)--(-9.348,-.944)--(-9.369,-.904)--(-9.372,-.902);
\filldraw[fill opacity=0.8,fill=gray!20,draw=none](-9.181,-1.021)--(-9.178,-1.024)--(-9.165,-1.021)--cycle;
\filldraw[fill opacity=0.8,fill=gray!20](-8.153,4.112)--(-8.267,4.154)--(-8.488,3.567)--cycle;
\filldraw[fill opacity=0.8,fill=gray!20,draw=none](-8.287,2.837)--(-8.291,2.845)--(-8.287,2.844)--(-8.266,2.822)--cycle;
\draw(-8.291,2.845)--(-8.287,2.844)--(-8.266,2.822);
\filldraw[fill opacity=0.8,fill=gray!20,draw=none](-8.258,2.822)--(-8.266,2.822)--(-8.272,2.828)--(-8.274,2.853)--cycle;
\draw(-8.258,2.822)--(-8.266,2.822)--(-8.272,2.828);
\filldraw[fill opacity=0.8,fill=gray!20,draw=none](-8.287,2.837)--(-8.297,2.844)--(-8.303,2.85)--(-8.291,2.845)--cycle;
\draw(-8.303,2.85)--(-8.291,2.845);
\filldraw[fill opacity=0.8,fill=gray!20,draw=none](-8.304,2.856)--(-8.302,2.85)--(-8.304,2.851)--cycle;
\draw(-8.302,2.85)--(-8.304,2.851);
\filldraw[fill opacity=0.8,fill=gray!20,draw=none](-8.31,2.856)--(-8.294,2.855)--(-8.3,2.848)--cycle;
\draw(-8.31,2.856)--(-8.294,2.855)--(-8.3,2.848);
\filldraw[fill opacity=0.8,fill=gray!20,draw=none](-8.243,2.834)--(-8.258,2.822)--(-8.274,2.853)--(-8.275,2.862)--cycle;
\filldraw[fill opacity=0.8,fill=gray!20,draw=none](-8.301,2.847)--(-8.294,2.855)--(-8.256,2.846)--(-8.28,2.832)--cycle;
\draw(-8.301,2.847)--(-8.294,2.855)--(-8.256,2.846)--(-8.28,2.832);
\filldraw[fill opacity=0.8,fill=gray!20,draw=none](-8.258,2.829)--(-8.272,2.826)--(-8.266,2.822)--(-8.259,2.822)--(-8.253,2.827)--cycle;
\draw(-8.266,2.822)--(-8.259,2.822);
\filldraw[fill opacity=0.5,fill=gray!20](-8.25,2.813)--(-8.308,2.852)--(-7.925,2.574)--(-7.87,2.536)--cycle;
\filldraw[fill opacity=0.8,fill=gray!20,draw=none](-9.233,-.945)--(-9.25,-.904)--(-9.277,-.936)--(-9.241,-.952)--cycle;
\draw(-9.277,-.936)--(-9.241,-.952)--(-9.233,-.945);
\filldraw[fill opacity=0.8,fill=gray!20,draw=none](-9.318,-.92)--(-9.306,-.915)--(-9.327,-.899)--(-9.362,-.903)--cycle;
\draw(-9.327,-.899)--(-9.362,-.903);
\filldraw[fill opacity=0.8,fill=gray!20,draw=none](-9.365,-1.023)--(-9.33,-1.038)--(-9.332,-1.016)--cycle;
\draw(-9.33,-1.038)--(-9.332,-1.016);
\filldraw[fill opacity=0.8,fill=gray!20](-8.608,3.613)--(-8.488,3.567)--(-8.267,4.154)--cycle;
\filldraw[fill opacity=0.8,fill=gray!20](-8.267,4.154)--(-8.387,4.199)--(-8.608,3.613)--cycle;
\filldraw[fill opacity=0.5,fill=gray!20](-8.261,-.409)--(-8.169,-.573)--(-8.57,-.798)--(-8.618,-.608)--cycle;
\filldraw[fill opacity=0.8,fill=gray!20,draw=none](-9.332,-1.016)--(-9.329,-1.054)--(-9.267,-1.047)--(-9.204,-1.031)--(-9.207,-1.02)--cycle;
\draw(-9.332,-1.016)--(-9.329,-1.054)--(-9.267,-1.047);
\filldraw[fill opacity=0.8,fill=gray!20,draw=none](-8.481,2.932)--(-8.453,2.915)--(-8.486,2.919)--(-8.514,2.936)--cycle;
\draw(-8.453,2.915)--(-8.486,2.919)--(-8.514,2.936);
\filldraw[fill opacity=0.8,fill=gray!20,draw=none](-8.442,2.911)--(-8.474,2.913)--(-8.486,2.919)--(-8.456,2.916)--cycle;
\draw(-8.474,2.913)--(-8.486,2.919)--(-8.456,2.916);
\filldraw[fill opacity=0.8,fill=gray!20,draw=none](-8.453,2.894)--(-8.454,2.893)--(-8.496,2.907)--(-8.496,2.908)--cycle;
\draw(-8.453,2.894)--(-8.454,2.893);
\draw(-8.496,2.907)--(-8.496,2.908);
\filldraw[fill opacity=0.8,fill=gray!20,draw=none](-8.504,2.911)--(-8.496,2.908)--(-8.496,2.907)--cycle;
\draw(-8.496,2.908)--(-8.496,2.907);
\filldraw[fill opacity=0.5,fill=gray!20](-8.819,3.019)--(-8.889,3.007)--(-8.456,2.859)--(-8.377,2.868)--cycle;
\filldraw[fill opacity=0.5,fill=gray!20](-7.819,2.405)--(-7.835,2.479)--(-7.549,2.104)--(-7.543,2.042)--cycle;
\filldraw[fill opacity=0.5,fill=gray!20](-10.254,-.736)--(-10.281,-.653)--(-10.642,-.391)--(-10.626,-.465)--cycle;
\filldraw[fill opacity=0.8,fill=gray!20,draw=none](-9.352,-.936)--(-9.318,-.92)--(-9.362,-.903)--(-9.369,-.904)--cycle;
\draw(-9.362,-.903)--(-9.369,-.904)--(-9.352,-.936);
\filldraw[fill opacity=0.8,fill=gray!20,draw=none](-9.244,-.919)--(-9.249,-.903)--(-9.277,-.936)--cycle;
\filldraw[fill opacity=0.8,fill=gray!20,draw=none](-9.334,-.997)--(-9.291,-.992)--(-9.277,-.936)--(-9.348,-.944)--cycle;
\draw(-9.277,-.936)--(-9.348,-.944)--(-9.334,-.997)--(-9.291,-.992);
\filldraw[fill opacity=0.8,fill=gray!20,draw=none](-9.285,-.969)--(-9.247,-.962)--(-9.241,-.952)--(-9.277,-.936)--cycle;
\draw(-9.247,-.962)--(-9.241,-.952)--(-9.277,-.936);
\filldraw[fill opacity=0.5,fill=gray!20](-9.461,2.719)--(-9.288,2.644)--(-8.906,2.631)--(-9.079,2.706)--cycle;
\filldraw[fill opacity=0.5,fill=gray!20](-9.461,2.921)--(-9.461,2.719)--(-9.079,2.706)--(-9.031,2.906)--cycle;
\filldraw[fill opacity=0.8,fill=gray!20,draw=none](-9.284,-1.055)--(-9.296,-1.05)--(-9.291,-1.101)--cycle;
\draw(-9.284,-1.055)--(-9.296,-1.05);
\filldraw[fill opacity=0.8,fill=gray!20,draw=none](-9.28,-1.057)--(-9.284,-1.055)--(-9.291,-1.101)--(-9.27,-1.111)--cycle;
\draw(-9.291,-1.101)--(-9.27,-1.111)--(-9.28,-1.057)--(-9.284,-1.055);
\filldraw[fill opacity=0.8,fill=gray!20,draw=none](-9.334,-1.106)--(-9.291,-1.101)--(-9.32,-1.053)--(-9.329,-1.054)--cycle;
\draw(-9.32,-1.053)--(-9.329,-1.054)--(-9.334,-1.106)--(-9.291,-1.101);
\filldraw[fill opacity=0.8,fill=gray!20,draw=none](-9.163,-1.017)--(-9.163,-1.021)--(-9.149,-1.018)--cycle;
\filldraw[fill opacity=0.5,fill=gray!20,draw=none](-9.274,-1.018)--(-9.273,-1.018)--(-9.242,-1.019)--cycle;
\draw(-9.273,-1.018)--(-9.242,-1.019);
\filldraw[fill opacity=0.8,fill=gray!20,draw=none](-9.244,-.919)--(-9.233,-.945)--(-9.199,-.916)--(-9.22,-.907)--cycle;
\draw(-9.233,-.945)--(-9.199,-.916)--(-9.22,-.907);
\filldraw[fill opacity=0.8,fill=gray!20,draw=none](-9.272,-1.014)--(-9.277,-1.042)--(-9.218,-1.035)--(-9.163,-1.021)--(-9.163,-1.017)--cycle;
\draw(-9.272,-1.014)--(-9.277,-1.042);
\filldraw[fill opacity=0.8,fill=gray!20,draw=none](-9.19,-.903)--(-9.22,-.907)--(-9.21,-.911)--cycle;
\draw(-9.22,-.907)--(-9.21,-.911);
\filldraw[fill opacity=0.5,fill=gray!20](-8.57,-.798)--(-8.599,-.856)--(-9.05,-.97)--(-9,-.906)--cycle;
\filldraw[fill opacity=0.5,fill=gray!20,draw=none](-9.347,-1.015)--(-9.346,-1.015)--(-9.273,-1.018)--(-9.274,-1.018)--cycle;
\draw(-9.346,-1.015)--(-9.273,-1.018);
\filldraw[fill opacity=0.5,fill=gray!20](-9,2.719)--(-9.461,2.921)--(-9.031,2.906)--(-8.57,2.704)--cycle;
\filldraw[fill opacity=0.5,fill=gray!20](-9.411,2.985)--(-9.461,2.921)--(-9.031,2.906)--(-8.961,2.969)--cycle;
\filldraw[fill opacity=0.8,fill=gray!20,draw=none](-9.293,-1.017)--(-9.273,-1.022)--(-9.272,-1.014)--cycle;
\draw(-9.273,-1.022)--(-9.272,-1.014);
\filldraw[fill opacity=0.8,fill=gray!20,draw=none](-9.287,-1.018)--(-9.293,-1.017)--(-9.296,-1.05)--cycle;
\filldraw[fill opacity=0.8,fill=gray!20,draw=none](-9.273,-1.022)--(-9.287,-1.018)--(-9.296,-1.05)--(-9.28,-1.057)--cycle;
\draw(-9.296,-1.05)--(-9.28,-1.057)--(-9.273,-1.022);
\filldraw[fill opacity=0.8,fill=gray!20,draw=none](-9.277,-1.042)--(-9.279,-1.051)--(-9.218,-1.035)--cycle;
\draw(-9.277,-1.042)--(-9.279,-1.051);
\filldraw[fill opacity=0.5,fill=gray!20](-9.173,-.629)--(-9,-.705)--(-9.382,-.692)--(-9.555,-.616)--cycle;
\filldraw[fill opacity=0.5,fill=gray!20,draw=none](-9.27,-1.003)--(-9.331,-1.016)--(-9.642,-1.005)--(-9.572,-.993)--cycle;
\draw(-9.331,-1.016)--(-9.642,-1.005)--(-9.572,-.993)--(-9.27,-1.003);
\filldraw[fill opacity=0.8,fill=gray!20,draw=none](-7.754,4.503)--(-7.761,4.51)--(-7.754,4.527)--(-7.744,4.523)--cycle;
\draw(-7.761,4.51)--(-7.754,4.527);
\filldraw[fill opacity=0.8,fill=gray!20,draw=none](-7.787,4.542)--(-7.791,4.541)--(-7.79,4.543)--cycle;
\draw(-7.791,4.541)--(-7.79,4.543);
\filldraw[fill opacity=0.8,fill=gray!20,draw=none](-7.807,4.471)--(-7.814,4.489)--(-7.789,4.543)--(-7.754,4.527)--(-7.779,4.469)--cycle;
\draw(-7.814,4.489)--(-7.789,4.543);
\draw(-7.754,4.527)--(-7.779,4.469);
\filldraw[fill opacity=0.8,fill=gray!20,draw=none](-7.728,4.338)--(-7.773,4.356)--(-7.76,4.351)--(-7.707,4.33)--(-7.66,4.311)--(-7.657,4.31)--cycle;
\draw(-7.773,4.356)--(-7.76,4.351)--(-7.707,4.33)--(-7.66,4.311)--(-7.657,4.31);
\filldraw[fill opacity=0.8,fill=gray!20,draw=none](-7.707,4.33)--(-7.7,4.349)--(-7.654,4.337)--(-7.652,4.331)--(-7.66,4.311)--cycle;
\draw(-7.652,4.331)--(-7.66,4.311)--(-7.707,4.33)--(-7.7,4.349);
\filldraw[fill opacity=0.8,fill=gray!20,draw=none](-7.76,4.351)--(-7.754,4.368)--(-7.7,4.349)--(-7.707,4.33)--cycle;
\draw(-7.7,4.349)--(-7.707,4.33)--(-7.76,4.351)--(-7.754,4.368);
\filldraw[fill opacity=0.8,fill=gray!20,draw=none](-7.728,4.338)--(-7.815,4.372)--(-7.811,4.37)--(-7.773,4.356)--cycle;
\draw(-7.815,4.372)--(-7.811,4.37)--(-7.773,4.356);
\filldraw[fill opacity=0.8,fill=gray!20,draw=none](-7.799,4.304)--(-7.766,4.378)--(-7.718,4.361)--(-7.745,4.299)--cycle;
\draw(-7.799,4.304)--(-7.766,4.378);
\draw(-7.718,4.361)--(-7.745,4.299);
\filldraw[fill opacity=0.8,fill=gray!20,draw=none](-7.933,4.014)--(-7.963,4.048)--(-7.944,4.099)--(-7.822,4.374)--(-7.776,4.356)--(-7.929,4.013)--cycle;
\draw(-7.944,4.099)--(-7.822,4.374);
\draw(-7.776,4.356)--(-7.929,4.013);
\filldraw[fill opacity=0.8,fill=gray!20,draw=none](-7.654,4.337)--(-7.65,4.336)--(-7.652,4.331)--cycle;
\draw(-7.65,4.336)--(-7.652,4.331);
\filldraw[fill opacity=0.8,fill=gray!20,draw=none](-7.652,4.307)--(-7.66,4.311)--(-7.652,4.331)--cycle;
\draw(-7.652,4.307)--(-7.66,4.311)--(-7.652,4.331);
\filldraw[fill opacity=0.8,fill=gray!20,draw=none](-7.652,4.307)--(-7.652,4.331)--(-7.65,4.336)--(-7.631,4.334)--(-7.621,4.309)--(-7.626,4.297)--cycle;
\draw(-7.652,4.331)--(-7.65,4.336);
\draw(-7.621,4.309)--(-7.626,4.297)--(-7.652,4.307);
\filldraw[fill opacity=0.8,fill=gray!20,draw=none](-7.686,4.321)--(-7.702,4.328)--(-7.657,4.31)--(-7.626,4.297)--(-7.622,4.295)--cycle;
\draw(-7.657,4.31)--(-7.626,4.297)--(-7.622,4.295);
\filldraw[fill opacity=0.8,fill=gray!20,draw=none](-7.673,4.349)--(-7.665,4.368)--(-7.657,4.347)--cycle;
\draw(-7.673,4.349)--(-7.665,4.368);
\filldraw[fill opacity=0.8,fill=gray!20,draw=none](-7.639,4.345)--(-7.737,4.394)--(-7.755,4.412)--(-7.665,4.368)--cycle;
\draw(-7.639,4.345)--(-7.737,4.394)--(-7.755,4.412)--(-7.665,4.368);
\filldraw[fill opacity=0.8,fill=gray!20,draw=none](-7.729,4.492)--(-7.728,4.492)--(-7.741,4.529)--(-7.746,4.516)--cycle;
\draw(-7.741,4.529)--(-7.746,4.516);
\filldraw[fill opacity=0.8,fill=gray!20,draw=none](-7.767,4.476)--(-7.774,4.481)--(-7.761,4.51)--(-7.754,4.503)--cycle;
\draw(-7.774,4.481)--(-7.761,4.51);
\filldraw[fill opacity=0.8,fill=gray!20,draw=none](-7.729,4.492)--(-7.746,4.516)--(-7.756,4.491)--(-7.757,4.476)--cycle;
\draw(-7.746,4.516)--(-7.756,4.491)--(-7.757,4.476);
\filldraw[fill opacity=0.8,fill=gray!20,draw=none](-7.728,4.492)--(-7.734,4.494)--(-7.759,4.507)--(-7.775,4.519)--(-7.759,4.53)--(-7.738,4.52)--cycle;
\draw(-7.759,4.507)--(-7.775,4.519)--(-7.759,4.53);
\filldraw[fill opacity=0.8,fill=gray!20,draw=none](-7.82,4.505)--(-7.822,4.534)--(-7.814,4.554)--(-7.803,4.549)--cycle;
\draw(-7.822,4.534)--(-7.814,4.554);
\filldraw[fill opacity=0.8,fill=gray!20,draw=none](-7.797,4.546)--(-7.789,4.543)--(-7.791,4.538)--cycle;
\draw(-7.789,4.543)--(-7.791,4.538);
\filldraw[fill opacity=0.8,fill=gray!20,draw=none](-7.759,4.53)--(-7.775,4.519)--(-7.796,4.546)--cycle;
\draw(-7.759,4.53)--(-7.775,4.519)--(-7.796,4.546);
\filldraw[fill opacity=0.8,fill=gray!20,draw=none](-7.814,4.496)--(-7.815,4.542)--(-7.811,4.552)--(-7.797,4.546)--(-7.791,4.538)--(-7.812,4.492)--cycle;
\draw(-7.815,4.542)--(-7.811,4.552);
\draw(-7.791,4.538)--(-7.812,4.492);
\filldraw[fill opacity=0.8,fill=gray!20,draw=none](-7.729,4.492)--(-7.735,4.489)--(-7.698,4.489)--cycle;
\filldraw[fill opacity=0.8,fill=gray!20,draw=none](-7.726,4.487)--(-7.729,4.492)--(-7.757,4.476)--(-7.761,4.447)--(-7.755,4.412)--(-7.737,4.394)--(-7.71,4.393)--(-7.694,4.401)--cycle;
\draw(-7.757,4.476)--(-7.761,4.447)--(-7.755,4.412)--(-7.737,4.394)--(-7.71,4.393)--(-7.694,4.401);
\filldraw[fill opacity=0.8,fill=gray!20,draw=none](-7.746,4.524)--(-7.744,4.523)--(-7.754,4.527)--cycle;
\filldraw[fill opacity=0.8,fill=gray!20,draw=none](-7.744,4.523)--(-7.754,4.527)--(-7.751,4.533)--(-7.743,4.531)--(-7.742,4.528)--cycle;
\draw(-7.754,4.527)--(-7.751,4.533);
\filldraw[fill opacity=0.8,fill=gray!20,draw=none](-7.74,4.53)--(-7.74,4.526)--(-7.745,4.531)--(-7.743,4.531)--cycle;
\draw(-7.74,4.526)--(-7.745,4.531);
\filldraw[fill opacity=0.8,fill=gray!20,draw=none](-7.781,4.539)--(-7.744,4.523)--(-7.744,4.523)--(-7.765,4.532)--(-7.787,4.542)--cycle;
\filldraw[fill opacity=0.8,fill=gray!20,draw=none](-7.77,4.512)--(-7.75,4.486)--(-7.734,4.494)--cycle;
\draw(-7.75,4.486)--(-7.734,4.494);
\filldraw[fill opacity=0.8,fill=gray!20,draw=none](-7.773,4.517)--(-7.775,4.519)--(-7.759,4.507)--cycle;
\draw(-7.773,4.517)--(-7.775,4.519)--(-7.759,4.507);
\filldraw[fill opacity=0.8,fill=gray!20,draw=none](-7.751,4.526)--(-7.772,4.516)--(-7.77,4.512)--(-7.734,4.494)--(-7.707,4.507)--cycle;
\draw(-7.751,4.526)--(-7.772,4.516);
\draw(-7.734,4.494)--(-7.707,4.507);
\filldraw[fill opacity=0.8,fill=gray!20,draw=none](-7.749,4.498)--(-7.773,4.517)--(-7.759,4.507)--(-7.745,4.496)--cycle;
\draw(-7.759,4.507)--(-7.745,4.496);
\filldraw[fill opacity=0.8,fill=gray!20,draw=none](-7.818,4.547)--(-7.815,4.554)--(-7.814,4.554)--(-7.822,4.534)--cycle;
\draw(-7.814,4.554)--(-7.822,4.534);
\filldraw[fill opacity=0.8,fill=gray!20,draw=none](-7.773,4.517)--(-7.795,4.545)--(-7.796,4.546)--(-7.775,4.519)--cycle;
\draw(-7.796,4.546)--(-7.775,4.519)--(-7.773,4.517);
\filldraw[fill opacity=0.8,fill=gray!20,draw=none](-7.813,4.549)--(-7.812,4.552)--(-7.811,4.552)--(-7.815,4.542)--cycle;
\draw(-7.811,4.552)--(-7.815,4.542);
\filldraw[fill opacity=0.8,fill=gray!20,draw=none](-7.809,4.551)--(-7.984,4.468)--(-7.968,4.423)--(-7.751,4.526)--cycle;
\draw(-7.809,4.551)--(-7.984,4.468);
\draw(-7.968,4.423)--(-7.751,4.526);
\filldraw[fill opacity=0.5,fill=gray!20](-7.721,1.852)--(-7.561,1.97)--(-7.395,1.551)--(-7.573,1.479)--cycle;
\filldraw[fill opacity=0.8,fill=gray!20](-7.781,.345)--(-7.78,.392)--(-7.693,.386)--(-7.703,.34)--cycle;
\filldraw[fill opacity=0.8,fill=gray!20](-7.869,.388)--(-7.872,.436)--(-7.779,.44)--(-7.78,.392)--cycle;
\filldraw[fill opacity=0.8,fill=gray!20](-7.78,.392)--(-7.779,.44)--(-7.69,.433)--(-7.693,.386)--cycle;
\filldraw[fill opacity=0.8,fill=gray!20](-7.861,.342)--(-7.869,.388)--(-7.78,.392)--(-7.781,.345)--cycle;
\filldraw[fill opacity=0.8,fill=gray!20](-7.872,.436)--(-7.869,.481)--(-7.78,.485)--(-7.779,.44)--cycle;
\filldraw[fill opacity=0.8,fill=gray!20](-7.779,.44)--(-7.78,.485)--(-7.693,.478)--(-7.69,.433)--cycle;
\filldraw[fill opacity=0.8,fill=gray!20](-7.783,.302)--(-7.781,.345)--(-7.703,.34)--(-7.719,.297)--cycle;
\filldraw[fill opacity=0.8,fill=gray!20](-7.848,.299)--(-7.861,.342)--(-7.781,.345)--(-7.783,.302)--cycle;
\filldraw[fill opacity=0.8,fill=gray!20](-7.05,.434)--(-7.046,.488)--(-6.939,.493)--(-6.938,.439)--cycle;
\filldraw[fill opacity=0.8,fill=gray!20](-6.938,.439)--(-6.939,.493)--(-6.835,.485)--(-6.831,.431)--cycle;
\filldraw[fill opacity=0.8,fill=gray!20,draw=none](-7.835,4.594)--(-7.835,4.596)--(-8.005,4.515)--(-7.984,4.468)--(-7.833,4.54)--cycle;
\draw(-7.835,4.596)--(-8.005,4.515);
\draw(-7.984,4.468)--(-7.833,4.54);
\filldraw[fill opacity=0.5,fill=gray!20](-9.642,-1.005)--(-9.708,-.991)--(-10.154,-.838)--(-10.084,-.854)--cycle;
\filldraw[fill opacity=0.8,fill=gray!20](-8.146,3.76)--(-8.164,3.815)--(-8.082,3.83)--(-8.072,3.774)--cycle;
\filldraw[fill opacity=0.8,fill=gray!20](-8.164,3.815)--(-8.17,3.871)--(-8.085,3.887)--(-8.082,3.83)--cycle;
\filldraw[fill opacity=0.8,fill=gray!20](-7.923,.33)--(-7.938,.375)--(-7.869,.388)--(-7.861,.342)--cycle;
\filldraw[fill opacity=0.8,fill=gray!20](-7.938,.375)--(-7.943,.422)--(-7.872,.436)--(-7.869,.388)--cycle;
\filldraw[fill opacity=0.8,fill=gray!20](-7.78,.485)--(-7.781,.524)--(-7.703,.518)--(-7.693,.478)--cycle;
\filldraw[fill opacity=0.8,fill=gray!20](-7.869,.481)--(-7.861,.52)--(-7.781,.524)--(-7.78,.485)--cycle;
\filldraw[fill opacity=0.8,fill=gray!20](-7.975,3.946)--(-7.976,3.993)--(-7.882,3.986)--(-7.87,3.939)--cycle;
\filldraw[fill opacity=0.5,fill=gray!20](-9.05,-.97)--(-9.106,-1.009)--(-9.572,-.993)--(-9.5,-.955)--cycle;
\filldraw[fill opacity=0.5,fill=gray!20](-10.506,-.146)--(-10.333,-.221)--(-10.567,.087)--(-10.74,.163)--cycle;
\filldraw[fill opacity=0.5,fill=gray!20](-10.636,-.302)--(-10.506,-.146)--(-10.74,.163)--(-10.9,.044)--cycle;
\filldraw[fill opacity=0.8,fill=gray!20](-7.943,.422)--(-7.938,.467)--(-7.869,.481)--(-7.872,.436)--cycle;
\filldraw[fill opacity=0.8,fill=gray!20](-8.17,3.871)--(-8.164,3.925)--(-8.082,3.941)--(-8.085,3.887)--cycle;
\filldraw[fill opacity=0.8,fill=gray!20](-7.898,.289)--(-7.923,.33)--(-7.861,.342)--(-7.848,.299)--cycle;
\filldraw[fill opacity=0.5,fill=gray!20](-8.207,2.75)--(-8.25,2.813)--(-7.87,2.536)--(-7.835,2.479)--cycle;
\filldraw[fill opacity=0.8,fill=gray!20,draw=none](-7.124,.352)--(-7.138,.323)--(-7.646,.328)--(-7.665,.376)--(-7.127,.37)--cycle;
\draw(-7.138,.323)--(-7.646,.328);
\draw(-7.665,.376)--(-7.127,.37);
\filldraw[fill opacity=0.8,fill=gray!20,draw=none](-7.118,.37)--(-7.647,.376)--(-7.657,.402)--(-7.648,.427)--(-7.136,.422)--cycle;
\draw(-7.118,.37)--(-7.647,.376);
\draw(-7.648,.427)--(-7.136,.422);
\filldraw[fill opacity=0.8,fill=gray!20,draw=none](-7.119,.363)--(-7.124,.352)--(-7.127,.37)--(-7.118,.37)--cycle;
\draw(-7.127,.37)--(-7.118,.37);
\filldraw[fill opacity=0.8,fill=gray!20,draw=none](-7.119,.363)--(-7.118,.37)--(-7.115,.37)--cycle;
\draw(-7.118,.37)--(-7.115,.37);
\filldraw[fill opacity=0.8,fill=gray!20,draw=none](-7.115,.37)--(-7.118,.37)--(-7.125,.392)--(-7.114,.422)--cycle;
\draw(-7.115,.37)--(-7.118,.37);
\filldraw[fill opacity=0.8,fill=gray!20,draw=none](-7.125,.392)--(-7.136,.422)--(-7.114,.422)--cycle;
\draw(-7.136,.422)--(-7.114,.422);
\filldraw[fill opacity=0.8,fill=gray!20](-7.129,.361)--(-7.135,.417)--(-7.05,.434)--(-7.046,.377)--cycle;
\filldraw[fill opacity=0.8,fill=gray!20,draw=none](-7.124,.352)--(-7.119,.363)--(-7.122,.34)--cycle;
\filldraw[fill opacity=0.8,fill=gray!20,draw=none](-7.124,.352)--(-7.122,.34)--(-7.124,.322)--(-7.138,.323)--cycle;
\draw(-7.124,.322)--(-7.138,.323);
\filldraw[fill opacity=0.8,fill=gray!20](-7.11,.307)--(-7.129,.361)--(-7.046,.377)--(-7.037,.321)--cycle;
\filldraw[fill opacity=0.8,fill=gray!20](-8.117,3.712)--(-8.146,3.76)--(-8.072,3.774)--(-8.057,3.723)--cycle;
\filldraw[fill opacity=0.8,fill=gray!20,draw=none](-8.316,3.003)--(-8.316,3.004)--(-8.25,2.975)--(-8.251,2.974)--cycle;
\draw(-8.25,2.975)--(-8.251,2.974);
\filldraw[fill opacity=0.8,fill=gray!20,draw=none](-8.333,2.966)--(-8.339,2.953)--(-8.344,2.955)--cycle;
\draw(-8.339,2.953)--(-8.344,2.955);
\filldraw[fill opacity=0.8,fill=gray!20,draw=none](-8.309,2.907)--(-8.361,2.929)--(-8.336,2.967)--(-8.271,2.939)--cycle;
\draw(-8.336,2.967)--(-8.271,2.939)--(-8.309,2.907)--(-8.361,2.929);
\filldraw[fill opacity=0.8,fill=gray!20,draw=none](-8.309,2.907)--(-8.361,2.929)--(-8.336,2.967)--(-8.271,2.939)--cycle;
\draw(-8.336,2.967)--(-8.271,2.939)--(-8.309,2.907)--(-8.361,2.929);
\filldraw[fill opacity=0.8,fill=gray!20,draw=none](-8.399,3.04)--(-8.316,3.004)--(-8.333,2.966)--(-8.344,2.955)--(-8.467,3.009)--cycle;
\draw(-8.399,3.04)--(-8.316,3.004);
\draw(-8.344,2.955)--(-8.467,3.009);
\filldraw[fill opacity=0.8,fill=gray!20,draw=none](-8.316,3.003)--(-8.251,2.974)--(-8.271,2.939)--(-8.333,2.966)--cycle;
\draw(-8.251,2.974)--(-8.271,2.939)--(-8.333,2.966);
\filldraw[fill opacity=0.8,fill=gray!20](-6.939,.493)--(-6.94,.54)--(-6.847,.533)--(-6.835,.485)--cycle;
\filldraw[fill opacity=0.8,fill=gray!20](-7.046,.488)--(-7.037,.535)--(-6.94,.54)--(-6.939,.493)--cycle;
\filldraw[fill opacity=0.8,fill=gray!20,draw=none](-7.647,.376)--(-7.665,.376)--(-7.665,.379)--(-7.657,.402)--cycle;
\draw(-7.647,.376)--(-7.665,.376);
\filldraw[fill opacity=0.8,fill=gray!20,draw=none](-7.657,.402)--(-7.665,.379)--(-7.666,.427)--cycle;
\filldraw[fill opacity=0.8,fill=gray!20,draw=none](-7.657,.402)--(-7.666,.427)--(-7.648,.427)--cycle;
\draw(-7.666,.427)--(-7.648,.427);
\filldraw[fill opacity=0.8,fill=gray!20](-7.693,.386)--(-7.69,.433)--(-7.627,.418)--(-7.632,.371)--cycle;
\filldraw[fill opacity=0.8,fill=gray!20,draw=none](-7.656,.352)--(-7.646,.328)--(-7.664,.328)--cycle;
\draw(-7.646,.328)--(-7.664,.328);
\filldraw[fill opacity=0.8,fill=gray!20](-7.785,.265)--(-7.783,.302)--(-7.719,.297)--(-7.74,.262)--cycle;
\filldraw[fill opacity=0.8,fill=gray!20](-7.831,.263)--(-7.848,.299)--(-7.783,.302)--(-7.785,.265)--cycle;
\filldraw[fill opacity=0.8,fill=gray!20,draw=none](-7.719,.297)--(-7.703,.34)--(-7.665,.33)--(-7.66,.309)--(-7.675,.286)--cycle;
\draw(-7.66,.309)--(-7.675,.286)--(-7.719,.297)--(-7.703,.34)--(-7.665,.33);
\filldraw[fill opacity=0.8,fill=gray!20](-7.74,.262)--(-7.719,.297)--(-7.675,.286)--(-7.709,.254)--cycle;
\filldraw[fill opacity=0.8,fill=gray!20,draw=none](-7.666,.291)--(-7.667,.298)--(-7.66,.309)--cycle;
\draw(-7.667,.298)--(-7.66,.309);
\filldraw[fill opacity=0.8,fill=gray!20,draw=none](-7.666,.291)--(-7.67,.281)--(-7.675,.286)--(-7.667,.298)--cycle;
\draw(-7.67,.281)--(-7.675,.286)--(-7.667,.298);
\filldraw[fill opacity=0.8,fill=gray!20,draw=none](-7.657,.317)--(-7.666,.291)--(-7.788,.292)--(-7.786,.329)--(-7.664,.328)--cycle;
\draw(-7.666,.291)--(-7.788,.292)--(-7.786,.329)--(-7.664,.328);
\filldraw[fill opacity=0.8,fill=gray!20,draw=none](-7.153,.285)--(-7.64,.291)--(-7.664,.328)--(-7.124,.322)--cycle;
\draw(-7.153,.285)--(-7.64,.291);
\draw(-7.664,.328)--(-7.124,.322);
\filldraw[fill opacity=0.8,fill=gray!20,draw=none](-7.657,.317)--(-7.64,.291)--(-7.666,.291)--cycle;
\draw(-7.64,.291)--(-7.666,.291);
\filldraw[fill opacity=0.8,fill=gray!20,draw=none](-7.665,.33)--(-7.649,.326)--(-7.66,.309)--cycle;
\draw(-7.665,.33)--(-7.649,.326)--(-7.66,.309);
\filldraw[fill opacity=0.8,fill=gray!20,draw=none](-7.656,.352)--(-7.664,.328)--(-7.667,.376)--(-7.665,.376)--cycle;
\draw(-7.667,.376)--(-7.665,.376);
\filldraw[fill opacity=0.8,fill=gray!20,draw=none](-7.665,.376)--(-7.667,.376)--(-7.665,.379)--cycle;
\draw(-7.665,.376)--(-7.667,.376);
\filldraw[fill opacity=0.8,fill=gray!20](-7.703,.34)--(-7.693,.386)--(-7.632,.371)--(-7.649,.326)--cycle;
\filldraw[fill opacity=0.8,fill=gray!20](-7.87,3.828)--(-7.866,3.885)--(-7.791,3.866)--(-7.798,3.81)--cycle;
\filldraw[fill opacity=0.8,fill=gray!20,draw=none](-7.114,.422)--(-7.648,.427)--(-7.657,.45)--(-7.647,.474)--(-7.134,.469)--cycle;
\draw(-7.114,.422)--(-7.648,.427);
\draw(-7.647,.474)--(-7.134,.469);
\filldraw[fill opacity=0.8,fill=gray!20,draw=none](-7.124,.482)--(-7.127,.469)--(-7.665,.474)--(-7.646,.51)--(-7.138,.504)--cycle;
\draw(-7.127,.469)--(-7.665,.474);
\draw(-7.646,.51)--(-7.138,.504);
\filldraw[fill opacity=0.8,fill=gray!20,draw=none](-7.114,.422)--(-7.134,.469)--(-7.115,.469)--cycle;
\draw(-7.134,.469)--(-7.115,.469);
\filldraw[fill opacity=0.8,fill=gray!20,draw=none](-7.124,.482)--(-7.119,.474)--(-7.118,.469)--(-7.127,.469)--cycle;
\draw(-7.118,.469)--(-7.127,.469);
\filldraw[fill opacity=0.8,fill=gray!20,draw=none](-7.119,.474)--(-7.115,.469)--(-7.118,.469)--cycle;
\draw(-7.115,.469)--(-7.118,.469);
\filldraw[fill opacity=0.8,fill=gray!20](-7.135,.417)--(-7.129,.472)--(-7.046,.488)--(-7.05,.434)--cycle;
\filldraw[fill opacity=0.8,fill=gray!20](-7.081,.258)--(-7.11,.307)--(-7.037,.321)--(-7.021,.27)--cycle;
\filldraw[fill opacity=0.8,fill=gray!20,draw=none](-7.657,.45)--(-7.666,.427)--(-7.665,.472)--cycle;
\filldraw[fill opacity=0.8,fill=gray!20,draw=none](-7.648,.427)--(-7.666,.427)--(-7.657,.45)--cycle;
\draw(-7.648,.427)--(-7.666,.427);
\filldraw[fill opacity=0.8,fill=gray!20,draw=none](-7.657,.45)--(-7.665,.472)--(-7.665,.474)--(-7.647,.474)--cycle;
\draw(-7.665,.474)--(-7.647,.474);
\filldraw[fill opacity=0.8,fill=gray!20](-7.69,.433)--(-7.693,.478)--(-7.632,.463)--(-7.627,.418)--cycle;
\filldraw[fill opacity=0.5,fill=gray!20](-10.211,-.798)--(-10.254,-.736)--(-10.626,-.465)--(-10.591,-.522)--cycle;
\filldraw[fill opacity=0.8,fill=gray!20,draw=none](-8.427,2.885)--(-8.444,2.895)--(-8.424,2.899)--(-8.416,2.882)--cycle;
\draw(-8.444,2.895)--(-8.424,2.899)--(-8.416,2.882);
\filldraw[fill opacity=0.8,fill=gray!20,draw=none](-8.444,2.895)--(-8.505,2.932)--(-8.474,2.929)--(-8.424,2.899)--cycle;
\draw(-8.424,2.899)--(-8.444,2.895);
\filldraw[fill opacity=0.8,fill=gray!20,draw=none](-8.398,2.885)--(-8.379,2.875)--(-8.416,2.882)--(-8.423,2.898)--cycle;
\draw(-8.416,2.882)--(-8.423,2.898);
\filldraw[fill opacity=0.8,fill=gray!20,draw=none](-8.398,2.885)--(-8.423,2.898)--(-8.424,2.899)--cycle;
\draw(-8.423,2.898)--(-8.424,2.899);
\filldraw[fill opacity=0.8,fill=gray!20,draw=none](-8.442,2.911)--(-8.394,2.894)--(-8.411,2.896)--(-8.451,2.903)--(-8.474,2.913)--cycle;
\draw(-8.451,2.903)--(-8.474,2.913);
\filldraw[fill opacity=0.8,fill=gray!20,draw=none](-8.519,2.956)--(-8.506,2.935)--(-8.514,2.936)--(-8.529,2.945)--(-8.547,2.973)--cycle;
\draw(-8.514,2.936)--(-8.529,2.945)--(-8.547,2.973);
\filldraw[fill opacity=0.8,fill=gray!20,draw=none](-8.379,2.875)--(-8.361,2.865)--(-8.415,2.881)--(-8.416,2.882)--cycle;
\draw(-8.415,2.881)--(-8.416,2.882);
\filldraw[fill opacity=0.8,fill=gray!20,draw=none](-8.437,2.888)--(-8.416,2.882)--(-8.415,2.881)--cycle;
\draw(-8.416,2.882)--(-8.415,2.881);
\filldraw[fill opacity=0.8,fill=gray!20,draw=none](-8.427,2.885)--(-8.437,2.888)--(-8.453,2.894)--(-8.444,2.895)--cycle;
\draw(-8.453,2.894)--(-8.444,2.895);
\filldraw[fill opacity=0.8,fill=gray!20,draw=none](-8.513,2.937)--(-8.495,2.924)--(-8.486,2.919)--(-8.474,2.913)--cycle;
\draw(-8.495,2.924)--(-8.486,2.919)--(-8.474,2.913);
\filldraw[fill opacity=0.8,fill=gray!20,draw=none](-8.513,2.937)--(-8.522,2.942)--(-8.529,2.945)--(-8.495,2.924)--cycle;
\draw(-8.522,2.942)--(-8.529,2.945)--(-8.495,2.924);
\filldraw[fill opacity=0.8,fill=gray!20,draw=none](-8.444,2.895)--(-8.453,2.894)--(-8.496,2.908)--(-8.511,2.933)--(-8.505,2.932)--cycle;
\draw(-8.444,2.895)--(-8.453,2.894);
\draw(-8.496,2.908)--(-8.511,2.933);
\filldraw[fill opacity=0.8,fill=gray!20,draw=none](-8.34,2.858)--(-8.315,2.856)--(-8.301,2.847)--cycle;
\draw(-8.34,2.858)--(-8.315,2.856);
\filldraw[fill opacity=0.8,fill=gray!20,draw=none](-8.519,2.956)--(-8.481,2.932)--(-8.506,2.935)--cycle;
\filldraw[fill opacity=0.8,fill=gray!20,draw=none](-8.474,2.929)--(-8.511,2.933)--(-8.513,2.936)--(-8.492,2.94)--cycle;
\draw(-8.511,2.933)--(-8.513,2.936)--(-8.492,2.94);
\filldraw[fill opacity=0.8,fill=gray!20,draw=none](-8.492,2.94)--(-8.513,2.936)--(-8.52,2.958)--cycle;
\draw(-8.492,2.94)--(-8.513,2.936)--(-8.52,2.958);
\filldraw[fill opacity=0.8,fill=gray!20,draw=none](-8.545,2.995)--(-8.519,2.956)--(-8.54,2.969)--cycle;
\filldraw[fill opacity=0.8,fill=gray!20,draw=none](-8.518,2.941)--(-8.521,2.942)--(-8.512,2.936)--cycle;
\draw(-8.518,2.941)--(-8.521,2.942);
\filldraw[fill opacity=0.8,fill=gray!20,draw=none](-8.521,2.942)--(-8.522,2.942)--(-8.512,2.936)--cycle;
\draw(-8.521,2.942)--(-8.522,2.942);
\filldraw[fill opacity=0.8,fill=gray!20,draw=none](-8.515,2.935)--(-8.536,2.968)--(-8.52,2.958)--(-8.513,2.936)--cycle;
\draw(-8.52,2.958)--(-8.513,2.936)--(-8.515,2.935);
\filldraw[fill opacity=0.8,fill=gray!20,draw=none](-8.504,2.911)--(-8.531,2.925)--(-8.513,2.936)--(-8.496,2.908)--cycle;
\draw(-8.531,2.925)--(-8.513,2.936)--(-8.496,2.908);
\filldraw[fill opacity=0.8,fill=gray!20,draw=none](-8.356,2.874)--(-8.311,2.854)--(-8.302,2.848)--(-8.326,2.856)--cycle;
\draw(-8.356,2.874)--(-8.311,2.854);
\filldraw[fill opacity=0.8,fill=gray!20,draw=none](-8.36,2.89)--(-8.357,2.909)--(-8.316,2.891)--(-8.304,2.856)--(-8.304,2.851)--(-8.311,2.854)--cycle;
\draw(-8.357,2.909)--(-8.316,2.891);
\draw(-8.304,2.851)--(-8.311,2.854);
\filldraw[fill opacity=0.8,fill=gray!20,draw=none](-8.294,2.855)--(-8.277,2.883)--(-8.256,2.846)--cycle;
\draw(-8.256,2.846)--(-8.294,2.855)--(-8.277,2.883);
\filldraw[fill opacity=0.8,fill=gray!20,draw=none](-8.515,2.954)--(-8.52,2.958)--(-8.523,2.965)--cycle;
\draw(-8.52,2.958)--(-8.523,2.965);
\filldraw[fill opacity=0.8,fill=gray!20,draw=none](-8.545,2.995)--(-8.54,2.969)--(-8.547,2.973)--(-8.557,2.988)--(-8.564,3.024)--cycle;
\draw(-8.547,2.973)--(-8.557,2.988)--(-8.564,3.024);
\filldraw[fill opacity=0.8,fill=gray!20,draw=none](-8.545,2.983)--(-8.557,2.988)--(-8.529,2.945)--(-8.526,2.944)--cycle;
\draw(-8.545,2.983)--(-8.557,2.988)--(-8.529,2.945)--(-8.526,2.944);
\filldraw[fill opacity=0.8,fill=gray!20,draw=none](-8.545,2.983)--(-8.526,2.944)--(-8.518,2.941)--cycle;
\draw(-8.526,2.944)--(-8.518,2.941);
\filldraw[fill opacity=0.8,fill=gray!20,draw=none](-8.523,2.949)--(-8.515,2.935)--(-8.52,2.932)--(-8.545,2.928)--(-8.55,2.93)--(-8.56,2.953)--cycle;
\draw(-8.515,2.935)--(-8.52,2.932);
\draw(-8.55,2.93)--(-8.56,2.953);
\filldraw[fill opacity=0.8,fill=gray!20,draw=none](-8.533,2.923)--(-8.531,2.925)--(-8.504,2.911)--cycle;
\draw(-8.533,2.923)--(-8.531,2.925);
\filldraw[fill opacity=0.8,fill=gray!20,draw=none](-8.537,2.929)--(-8.52,2.932)--(-8.531,2.925)--cycle;
\draw(-8.52,2.932)--(-8.531,2.925);
\filldraw[fill opacity=0.8,fill=gray!20,draw=none](-8.394,2.89)--(-8.356,2.874)--(-8.326,2.856)--(-8.403,2.884)--cycle;
\draw(-8.394,2.89)--(-8.356,2.874);
\filldraw[fill opacity=0.8,fill=gray!20,draw=none](-8.363,2.878)--(-8.356,2.874)--(-8.363,2.877)--cycle;
\draw(-8.356,2.874)--(-8.363,2.877);
\filldraw[fill opacity=0.8,fill=gray!20,draw=none](-8.545,2.928)--(-8.537,2.929)--(-8.531,2.925)--(-8.533,2.923)--cycle;
\draw(-8.531,2.925)--(-8.533,2.923);
\filldraw[fill opacity=0.8,fill=gray!20,draw=none](-8.394,2.882)--(-8.415,2.888)--(-8.422,2.891)--(-8.395,2.89)--cycle;
\draw(-8.415,2.888)--(-8.422,2.891);
\filldraw[fill opacity=0.8,fill=gray!20,draw=none](-8.422,2.891)--(-8.438,2.898)--(-8.394,2.894)--(-8.388,2.892)--(-8.382,2.89)--cycle;
\draw(-8.422,2.891)--(-8.438,2.898);
\draw(-8.388,2.892)--(-8.382,2.89);
\filldraw[fill opacity=0.8,fill=gray!20,draw=none](-8.411,2.896)--(-8.438,2.898)--(-8.451,2.903)--cycle;
\draw(-8.438,2.898)--(-8.451,2.903);
\filldraw[fill opacity=0.8,fill=gray!20,draw=none](-8.414,2.888)--(-8.424,2.894)--(-8.451,2.908)--(-8.422,2.891)--(-8.415,2.888)--cycle;
\draw(-8.422,2.891)--(-8.415,2.888);
\filldraw[fill opacity=0.8,fill=gray!20,draw=none](-8.523,2.949)--(-8.56,2.953)--(-8.566,2.968)--(-8.55,2.978)--(-8.536,2.968)--cycle;
\draw(-8.56,2.953)--(-8.566,2.968)--(-8.55,2.978);
\filldraw[fill opacity=0.8,fill=gray!20,draw=none](-8.539,2.932)--(-8.554,2.947)--(-8.554,2.94)--(-8.55,2.93)--cycle;
\draw(-8.554,2.94)--(-8.55,2.93);
\filldraw[fill opacity=0.5,fill=gray!20](-8.753,3.005)--(-8.819,3.019)--(-8.377,2.868)--(-8.308,2.852)--cycle;
\filldraw[fill opacity=0.8,fill=gray!20](-7.866,3.885)--(-7.87,3.939)--(-7.798,3.921)--(-7.791,3.866)--cycle;
\filldraw[fill opacity=0.8,fill=gray!20,draw=none](-6.768,.344)--(-6.774,.359)--(-6.775,.367)--(-6.763,.366)--cycle;
\draw(-6.775,.367)--(-6.763,.366);
\filldraw[fill opacity=0.8,fill=gray!20,draw=none](-6.768,.344)--(-6.763,.366)--(-6.335,.362)--(-6.301,.327)--(-6.316,.314)--(-6.758,.318)--cycle;
\draw(-6.763,.366)--(-6.335,.362);
\draw(-6.316,.314)--(-6.758,.318);
\filldraw[fill opacity=0.8,fill=gray!20,draw=none](-6.773,.327)--(-6.774,.359)--(-6.768,.344)--cycle;
\filldraw[fill opacity=0.8,fill=gray!20,draw=none](-6.773,.327)--(-6.768,.344)--(-6.758,.318)--(-6.772,.319)--cycle;
\draw(-6.758,.318)--(-6.772,.319);
\filldraw[fill opacity=0.8,fill=gray!20](-6.847,.319)--(-6.835,.374)--(-6.762,.356)--(-6.782,.303)--cycle;
\filldraw[fill opacity=0.8,fill=gray!20,draw=none](-6.778,.418)--(-6.767,.388)--(-6.775,.367)--(-6.777,.367)--cycle;
\draw(-6.775,.367)--(-6.777,.367);
\filldraw[fill opacity=0.8,fill=gray!20,draw=none](-6.774,.359)--(-6.777,.367)--(-6.775,.367)--cycle;
\draw(-6.777,.367)--(-6.775,.367);
\filldraw[fill opacity=0.8,fill=gray!20,draw=none](-6.767,.388)--(-6.759,.366)--(-6.775,.367)--cycle;
\draw(-6.759,.366)--(-6.775,.367);
\filldraw[fill opacity=0.8,fill=gray!20,draw=none](-6.767,.388)--(-6.756,.418)--(-6.519,.415)--(-6.262,.367)--(-6.264,.361)--(-6.759,.366)--cycle;
\draw(-6.756,.418)--(-6.519,.415);
\draw(-6.264,.361)--(-6.759,.366);
\filldraw[fill opacity=0.8,fill=gray!20,draw=none](-6.778,.418)--(-6.756,.418)--(-6.767,.388)--cycle;
\draw(-6.778,.418)--(-6.756,.418);
\filldraw[fill opacity=0.8,fill=gray!20](-6.835,.374)--(-6.831,.431)--(-6.755,.413)--(-6.762,.356)--cycle;
\filldraw[fill opacity=0.8,fill=gray!20](-6.945,.229)--(-6.942,.273)--(-6.866,.268)--(-6.891,.226)--cycle;
\filldraw[fill opacity=0.8,fill=gray!20](-7.001,.227)--(-7.021,.27)--(-6.942,.273)--(-6.945,.229)--cycle;
\filldraw[fill opacity=0.8,fill=gray!20](-7.938,.467)--(-7.923,.508)--(-7.861,.52)--(-7.869,.481)--cycle;
\filldraw[fill opacity=0.8,fill=gray!20](-7.867,.256)--(-7.898,.289)--(-7.848,.299)--(-7.831,.263)--cycle;
\filldraw[fill opacity=0.8,fill=gray!20](-8.164,3.925)--(-8.146,3.975)--(-8.072,3.989)--(-8.082,3.941)--cycle;
\filldraw[fill opacity=0.8,fill=gray!20,draw=none](-6.777,.465)--(-6.759,.465)--(-6.778,.418)--cycle;
\draw(-6.777,.465)--(-6.759,.465);
\filldraw[fill opacity=0.8,fill=gray!20,draw=none](-6.245,.412)--(-6.261,.452)--(-6.231,.453)--cycle;
\filldraw[fill opacity=0.8,fill=gray!20,draw=none](-6.253,.452)--(-6.261,.452)--(-6.264,.46)--cycle;
\filldraw[fill opacity=0.8,fill=gray!20,draw=none](-6.759,.465)--(-6.231,.459)--(-6.27,.413)--(-6.778,.418)--cycle;
\draw(-6.759,.465)--(-6.231,.459);
\draw(-6.27,.413)--(-6.778,.418);
\filldraw[fill opacity=0.8,fill=gray!20,draw=none](-6.774,.47)--(-6.775,.465)--(-6.777,.465)--cycle;
\draw(-6.775,.465)--(-6.777,.465);
\filldraw[fill opacity=0.8,fill=gray!20,draw=none](-6.774,.47)--(-6.768,.481)--(-6.763,.465)--(-6.775,.465)--cycle;
\draw(-6.763,.465)--(-6.775,.465);
\filldraw[fill opacity=0.8,fill=gray!20,draw=none](-6.768,.481)--(-6.758,.5)--(-6.316,.496)--(-6.301,.485)--(-6.335,.46)--(-6.763,.465)--cycle;
\draw(-6.758,.5)--(-6.316,.496);
\draw(-6.335,.46)--(-6.763,.465);
\filldraw[fill opacity=0.8,fill=gray!20](-6.831,.431)--(-6.835,.485)--(-6.762,.467)--(-6.755,.413)--cycle;
\filldraw[fill opacity=0.8,fill=gray!20](-6.866,.268)--(-6.847,.319)--(-6.782,.303)--(-6.813,.255)--cycle;
\filldraw[fill opacity=0.8,fill=gray!20,draw=none](-8.113,3.708)--(-8.071,3.695)--(-8.075,3.684)--cycle;
\draw(-8.071,3.695)--(-8.075,3.684);
\filldraw[fill opacity=0.8,fill=gray!20,draw=none](-8.047,3.678)--(-8.074,3.688)--(-8.071,3.695)--cycle;
\draw(-8.074,3.688)--(-8.071,3.695);
\filldraw[fill opacity=0.8,fill=gray!20,draw=none](-8.047,3.678)--(-8.028,3.665)--(-8.041,3.636)--(-8.076,3.683)--(-8.074,3.688)--cycle;
\draw(-8.028,3.665)--(-8.041,3.636);
\draw(-8.076,3.683)--(-8.074,3.688);
\filldraw[fill opacity=0.8,fill=gray!20,draw=none](-8.113,3.708)--(-8.075,3.684)--(-8.248,3.297)--(-8.22,3.479)--(-8.118,3.709)--cycle;
\draw(-8.075,3.684)--(-8.248,3.297);
\draw(-8.22,3.479)--(-8.118,3.709);
\filldraw[fill opacity=0.8,fill=gray!20,draw=none](-8.041,3.636)--(-8.25,3.166)--(-8.26,3.168)--(-8.248,3.297)--(-8.076,3.683)--cycle;
\draw(-8.041,3.636)--(-8.25,3.166);
\draw(-8.248,3.297)--(-8.076,3.683);
\filldraw[fill opacity=0.8,fill=gray!20](-8.079,3.672)--(-8.117,3.712)--(-8.057,3.723)--(-8.036,3.68)--cycle;
\filldraw[fill opacity=0.8,fill=gray!20,draw=none](-6.078,.568)--(-6.062,.566)--(-6.014,.567)--cycle;
\filldraw[fill opacity=0.8,fill=gray!20,draw=none](-6.099,.458)--(-6.08,.45)--(-5.819,.841)--cycle;
\draw(-6.099,.458)--(-6.08,.45)--(-5.819,.841);
\filldraw[fill opacity=0.5,fill=gray!20](-7.825,2.317)--(-7.819,2.405)--(-7.543,2.042)--(-7.561,1.97)--cycle;
\filldraw[fill opacity=0.8,fill=gray!20](-7.781,.524)--(-7.783,.554)--(-7.719,.55)--(-7.703,.518)--cycle;
\filldraw[fill opacity=0.8,fill=gray!20](-7.861,.52)--(-7.848,.552)--(-7.783,.554)--(-7.781,.524)--cycle;
\filldraw[fill opacity=0.8,fill=gray!20,draw=none](-7.665,.472)--(-7.667,.474)--(-7.665,.474)--cycle;
\draw(-7.667,.474)--(-7.665,.474);
\filldraw[fill opacity=0.8,fill=gray!20,draw=none](-7.656,.492)--(-7.664,.51)--(-7.646,.51)--cycle;
\draw(-7.664,.51)--(-7.646,.51);
\filldraw[fill opacity=0.8,fill=gray!20,draw=none](-7.656,.492)--(-7.665,.474)--(-7.667,.474)--(-7.664,.51)--cycle;
\draw(-7.665,.474)--(-7.667,.474);
\filldraw[fill opacity=0.8,fill=gray!20](-7.693,.478)--(-7.703,.518)--(-7.649,.505)--(-7.632,.463)--cycle;
\filldraw[fill opacity=0.8,fill=gray!20,draw=none](-7.664,.51)--(-7.786,.511)--(-7.788,.53)--(-7.64,.528)--cycle;
\draw(-7.664,.51)--(-7.786,.511)--(-7.788,.53)--(-7.64,.528);
\filldraw[fill opacity=0.8,fill=gray!20,draw=none](-7.665,.509)--(-7.66,.519)--(-7.649,.505)--cycle;
\draw(-7.66,.519)--(-7.649,.505)--(-7.665,.509);
\filldraw[fill opacity=0.8,fill=gray!20,draw=none](-7.122,.491)--(-7.119,.474)--(-7.124,.482)--cycle;
\filldraw[fill opacity=0.8,fill=gray!20,draw=none](-7.122,.491)--(-7.124,.482)--(-7.138,.504)--(-7.124,.504)--cycle;
\draw(-7.138,.504)--(-7.124,.504);
\filldraw[fill opacity=0.8,fill=gray!20](-7.129,.472)--(-7.11,.521)--(-7.037,.535)--(-7.046,.488)--cycle;
\filldraw[fill opacity=0.8,fill=gray!20,draw=none](-8.436,3.185)--(-8.403,3.195)--(-8.386,3.194)--(-8.371,3.185)--(-8.367,3.177)--cycle;
\draw(-8.403,3.195)--(-8.386,3.194)--(-8.371,3.185);
\filldraw[fill opacity=0.8,fill=gray!20,draw=none](-7.665,.379)--(-7.667,.376)--(-7.784,.377)--(-7.784,.429)--(-7.666,.427)--cycle;
\draw(-7.667,.376)--(-7.784,.377)--(-7.784,.429)--(-7.666,.427);
\filldraw[fill opacity=0.8,fill=gray!20,draw=none](-7.664,.328)--(-7.786,.329)--(-7.784,.377)--(-7.667,.376)--cycle;
\draw(-7.664,.328)--(-7.786,.329)--(-7.784,.377)--(-7.667,.376);
\filldraw[fill opacity=0.8,fill=gray!20,draw=none](-7.795,.29)--(-7.815,.278)--(-7.842,.279)--(-7.86,.298)--(-7.868,.332)--(-7.864,.377)--(-7.849,.426)--(-7.825,.47)--(-7.796,.504)--(-7.786,.51)--cycle;
\draw(-7.795,.29)--(-7.815,.278)--(-7.842,.279)--(-7.86,.298)--(-7.868,.332)--(-7.864,.377)--(-7.849,.426)--(-7.825,.47)--(-7.796,.504)--(-7.786,.51);
\filldraw[fill opacity=0.8,fill=gray!20,draw=none](-7.923,.508)--(-7.901,.538)--(-7.894,.543)--(-7.848,.552)--(-7.861,.52)--cycle;
\draw(-7.894,.543)--(-7.848,.552)--(-7.861,.52)--(-7.923,.508)--(-7.901,.538);
\filldraw[fill opacity=0.8,fill=gray!20](-7.848,.552)--(-7.831,.573)--(-7.785,.575)--(-7.783,.554)--cycle;
\filldraw[fill opacity=0.8,fill=gray!20,draw=none](-7.865,.548)--(-7.875,.56)--(-7.867,.566)--(-7.831,.573)--(-7.848,.552)--cycle;
\draw(-7.875,.56)--(-7.867,.566)--(-7.831,.573)--(-7.848,.552)--(-7.865,.548);
\filldraw[fill opacity=0.8,fill=gray!20,draw=none](-7.865,.548)--(-7.898,.542)--(-7.875,.56)--cycle;
\draw(-7.865,.548)--(-7.898,.542)--(-7.875,.56);
\filldraw[fill opacity=0.8,fill=gray!20,draw=none](-7.787,.53)--(-7.786,.51)--(-7.796,.504)--(-7.86,.531)--cycle;
\draw(-7.786,.51)--(-7.796,.504)--(-7.86,.531);
\filldraw[fill opacity=0.8,fill=gray!20](-7.796,.423)--(-7.795,.471)--(-7.793,.508)--(-7.79,.529)--(-7.788,.53)--(-7.786,.511)--(-7.784,.476)--(-7.784,.429)--(-7.784,.377)--(-7.786,.329)--(-7.788,.292)--(-7.79,.271)--(-7.793,.27)--(-7.795,.289)--(-7.796,.324)--(-7.797,.371)--cycle;
\filldraw[fill opacity=0.8,fill=gray!20](-7.043,.219)--(-7.081,.258)--(-7.021,.27)--(-7.001,.227)--cycle;
\filldraw[fill opacity=0.8,fill=gray!20](-8.072,3.989)--(-8.057,4.026)--(-7.978,4.03)--(-7.976,3.993)--cycle;
\filldraw[fill opacity=0.8,fill=gray!20,draw=none](-7.925,4.022)--(-7.799,4.304)--(-7.745,4.299)--(-7.883,3.989)--cycle;
\draw(-7.925,4.022)--(-7.799,4.304);
\draw(-7.745,4.299)--(-7.883,3.989);
\filldraw[fill opacity=0.8,fill=gray!20,draw=none](-7.933,4.003)--(-7.925,4.022)--(-7.883,3.989)--cycle;
\draw(-7.933,4.003)--(-7.925,4.022);
\filldraw[fill opacity=0.8,fill=gray!20,draw=none](-7.933,4.014)--(-7.929,4.013)--(-7.93,4.011)--cycle;
\draw(-7.929,4.013)--(-7.93,4.011);
\filldraw[fill opacity=0.8,fill=gray!20,draw=none](-7.97,4.027)--(-7.97,4.028)--(-7.933,4.014)--(-7.93,4.011)--(-7.933,4.003)--cycle;
\draw(-7.93,4.011)--(-7.933,4.003);
\filldraw[fill opacity=0.8,fill=gray!20,draw=none](-7.97,4.028)--(-7.963,4.048)--(-7.933,4.014)--cycle;
\filldraw[fill opacity=0.8,fill=gray!20,draw=none](-7.97,4.028)--(-7.97,4.027)--(-7.975,4.03)--cycle;
\filldraw[fill opacity=0.8,fill=gray!20,draw=none](-7.97,4.028)--(-7.975,4.03)--(-7.944,4.099)--cycle;
\draw(-7.975,4.03)--(-7.944,4.099);
\filldraw[fill opacity=0.8,fill=gray!20](-7.976,3.993)--(-7.978,4.03)--(-7.902,4.024)--(-7.882,3.986)--cycle;
\filldraw[fill opacity=0.5,fill=gray!20](-10.175,-.504)--(-10.636,-.302)--(-10.9,.044)--(-10.439,-.157)--cycle;
\filldraw[fill opacity=0.5,fill=gray!20](-10.642,-.391)--(-10.636,-.302)--(-10.9,.044)--(-10.918,-.028)--cycle;
\filldraw[fill opacity=0.8,fill=gray!20,draw=none](-7.665,.472)--(-7.666,.427)--(-7.784,.429)--(-7.784,.476)--(-7.667,.474)--cycle;
\draw(-7.666,.427)--(-7.784,.429)--(-7.784,.476)--(-7.667,.474);
\filldraw[fill opacity=0.8,fill=gray!20,draw=none](-7.875,3.984)--(-7.882,3.989)--(-7.833,3.991)--(-7.843,3.967)--cycle;
\draw(-7.833,3.991)--(-7.843,3.967);
\filldraw[fill opacity=0.8,fill=gray!20,draw=none](-7.875,3.984)--(-7.843,3.967)--(-7.845,3.962)--cycle;
\draw(-7.843,3.967)--(-7.845,3.962);
\filldraw[fill opacity=0.8,fill=gray!20,draw=none](-7.832,3.992)--(-7.686,4.321)--(-7.679,4.318)--(-7.68,4.252)--(-7.812,3.956)--cycle;
\draw(-7.832,3.992)--(-7.686,4.321);
\draw(-7.68,4.252)--(-7.812,3.956);
\filldraw[fill opacity=0.8,fill=gray!20,draw=none](-7.845,3.962)--(-7.832,3.992)--(-7.812,3.956)--cycle;
\draw(-7.845,3.962)--(-7.832,3.992);
\filldraw[fill opacity=0.8,fill=gray!20](-7.87,3.939)--(-7.882,3.986)--(-7.817,3.97)--(-7.798,3.921)--cycle;
\filldraw[fill opacity=0.5,fill=gray!20](-9.355,3.023)--(-9.411,2.985)--(-8.961,2.969)--(-8.889,3.007)--cycle;
\filldraw[fill opacity=0.8,fill=gray!20](-7.788,.238)--(-7.785,.265)--(-7.74,.262)--(-7.764,.236)--cycle;
\filldraw[fill opacity=0.8,fill=gray!20](-7.812,.237)--(-7.831,.263)--(-7.785,.265)--(-7.788,.238)--cycle;
\filldraw[fill opacity=0.8,fill=gray!20,draw=none](-8.385,3.176)--(-8.391,3.18)--(-8.381,3.175)--cycle;
\draw(-8.391,3.18)--(-8.381,3.175);
\filldraw[fill opacity=0.8,fill=gray!20,draw=none](-8.385,3.176)--(-8.391,3.18)--(-8.381,3.175)--cycle;
\draw(-8.391,3.18)--(-8.381,3.175);
\filldraw[fill opacity=0.8,fill=gray!20](-6.94,.54)--(-6.942,.576)--(-6.866,.571)--(-6.847,.533)--cycle;
\filldraw[fill opacity=0.8,fill=gray!20](-7.037,.535)--(-7.021,.573)--(-6.942,.576)--(-6.94,.54)--cycle;
\filldraw[fill opacity=0.8,fill=gray!20](-6.891,.226)--(-6.866,.268)--(-6.813,.255)--(-6.853,.216)--cycle;
\filldraw[fill opacity=0.8,fill=gray!20,draw=none](-6.773,.494)--(-6.768,.481)--(-6.774,.47)--cycle;
\filldraw[fill opacity=0.8,fill=gray!20](-6.835,.485)--(-6.847,.533)--(-6.782,.517)--(-6.762,.467)--cycle;
\filldraw[fill opacity=0.8,fill=gray!20,draw=none](-8.349,2.859)--(-8.358,2.864)--(-8.348,2.86)--(-8.348,2.859)--cycle;
\draw(-8.348,2.86)--(-8.348,2.859)--(-8.349,2.859);
\filldraw[fill opacity=0.8,fill=gray!20,draw=none](-8.34,2.858)--(-8.348,2.859)--(-8.348,2.86)--cycle;
\draw(-8.34,2.858)--(-8.348,2.859)--(-8.348,2.86);
\filldraw[fill opacity=0.8,fill=gray!20,draw=none](-8.348,2.858)--(-8.348,2.859)--(-8.34,2.858)--(-8.301,2.847)--(-8.307,2.841)--cycle;
\draw(-8.348,2.858)--(-8.348,2.859)--(-8.34,2.858);
\draw(-8.301,2.847)--(-8.307,2.841);
\filldraw[fill opacity=0.5,fill=gray!20,draw=none](-8.33,2.85)--(-8.345,2.861)--(-8.361,2.864)--cycle;
\draw(-8.345,2.861)--(-8.361,2.864);
\filldraw[fill opacity=0.8,fill=gray!20,draw=none](-8.446,2.848)--(-8.484,2.888)--(-8.454,2.893)--(-8.437,2.888)--(-8.415,2.881)--(-8.403,2.856)--cycle;
\draw(-8.415,2.881)--(-8.403,2.856)--(-8.446,2.848)--(-8.484,2.888)--(-8.454,2.893);
\filldraw[fill opacity=0.8,fill=gray!20,draw=none](-8.361,2.865)--(-8.351,2.859)--(-8.403,2.856)--(-8.415,2.881)--cycle;
\draw(-8.351,2.859)--(-8.403,2.856)--(-8.415,2.881);
\filldraw[fill opacity=0.8,fill=gray!20,draw=none](-8.364,3.187)--(-8.358,3.174)--(-8.384,3.177)--(-8.409,3.193)--cycle;
\filldraw[fill opacity=0.8,fill=gray!20,draw=none](-8.436,3.185)--(-8.469,3.188)--(-8.436,3.199)--(-8.403,3.195)--cycle;
\draw(-8.469,3.188)--(-8.436,3.199)--(-8.403,3.195);
\filldraw[fill opacity=0.8,fill=gray!20,draw=none](-8.354,3.178)--(-8.349,3.175)--(-8.361,3.18)--(-8.374,3.188)--(-8.364,3.184)--cycle;
\draw(-8.374,3.188)--(-8.364,3.184);
\filldraw[fill opacity=0.8,fill=gray!20,draw=none](-8.409,3.193)--(-8.384,3.177)--(-8.432,3.183)--(-8.427,3.195)--cycle;
\draw(-8.432,3.183)--(-8.427,3.195);
\filldraw[fill opacity=0.8,fill=gray!20,draw=none](-8.357,3.178)--(-8.348,3.174)--(-8.317,3.156)--(-8.318,3.156)--cycle;
\draw(-8.317,3.156)--(-8.318,3.156);
\filldraw[fill opacity=0.8,fill=gray!20,draw=none](-8.313,3.151)--(-8.318,3.156)--(-8.317,3.156)--cycle;
\draw(-8.318,3.156)--(-8.317,3.156);
\filldraw[fill opacity=0.8,fill=gray!20,draw=none](-8.302,3.176)--(-8.31,3.158)--(-8.314,3.177)--cycle;
\draw(-8.302,3.176)--(-8.31,3.158);
\filldraw[fill opacity=0.8,fill=gray!20,draw=none](-8.333,3.165)--(-8.342,3.167)--(-8.348,3.174)--cycle;
\filldraw[fill opacity=0.8,fill=gray!20,draw=none](-8.354,3.178)--(-8.348,3.174)--(-8.349,3.175)--cycle;
\filldraw[fill opacity=0.8,fill=gray!20,draw=none](-8.342,3.167)--(-8.359,3.169)--(-8.353,3.181)--cycle;
\draw(-8.359,3.169)--(-8.353,3.181);
\filldraw[fill opacity=0.8,fill=gray!20,draw=none](-8.354,3.178)--(-8.364,3.184)--(-8.362,3.183)--cycle;
\draw(-8.364,3.184)--(-8.362,3.183);
\filldraw[fill opacity=0.8,fill=gray!20,draw=none](-8.354,3.183)--(-8.353,3.181)--(-8.359,3.169)--(-8.4,3.172)--(-8.394,3.185)--cycle;
\draw(-8.353,3.181)--(-8.359,3.169);
\draw(-8.4,3.172)--(-8.394,3.185);
\filldraw[fill opacity=0.8,fill=gray!20,draw=none](-8.296,3.152)--(-8.348,3.174)--(-8.354,3.178)--(-8.362,3.183)--(-8.305,3.158)--cycle;
\draw(-8.362,3.183)--(-8.305,3.158);
\filldraw[fill opacity=0.8,fill=gray!20,draw=none](-8.404,3.185)--(-8.394,3.185)--(-8.398,3.176)--(-8.406,3.172)--(-8.425,3.171)--(-8.426,3.172)--cycle;
\draw(-8.394,3.185)--(-8.398,3.176);
\filldraw[fill opacity=0.8,fill=gray!20,draw=none](-8.303,3.158)--(-8.292,3.153)--(-8.306,3.147)--(-8.365,3.173)--cycle;
\draw(-8.303,3.158)--(-8.292,3.153);
\draw(-8.306,3.147)--(-8.365,3.173);
\filldraw[fill opacity=0.8,fill=gray!20,draw=none](-8.296,3.152)--(-8.283,3.146)--(-8.245,3.124)--(-8.279,3.139)--cycle;
\draw(-8.245,3.124)--(-8.279,3.139);
\filldraw[fill opacity=0.8,fill=gray!20,draw=none](-8.283,3.146)--(-8.296,3.152)--(-8.305,3.158)--(-8.303,3.158)--cycle;
\draw(-8.305,3.158)--(-8.303,3.158);
\filldraw[fill opacity=0.8,fill=gray!20,draw=none](-8.315,3.151)--(-8.18,3.092)--(-8.17,3.1)--(-8.303,3.158)--cycle;
\draw(-8.315,3.151)--(-8.18,3.092);
\draw(-8.17,3.1)--(-8.303,3.158);
\filldraw[fill opacity=0.8,fill=gray!20,draw=none](-8.354,3.168)--(-8.322,3.154)--(-8.391,3.165)--cycle;
\draw(-8.354,3.168)--(-8.322,3.154);
\filldraw[fill opacity=0.8,fill=gray!20,draw=none](-8.481,3.17)--(-8.461,3.187)--(-8.436,3.185)--cycle;
\filldraw[fill opacity=0.8,fill=gray!20,draw=none](-8.398,3.176)--(-8.4,3.172)--(-8.406,3.172)--cycle;
\draw(-8.398,3.176)--(-8.4,3.172);
\filldraw[fill opacity=0.8,fill=gray!20,draw=none](-8.431,3.167)--(-8.421,3.163)--(-8.451,3.157)--cycle;
\draw(-8.431,3.167)--(-8.421,3.163);
\filldraw[fill opacity=0.8,fill=gray!20,draw=none](-8.425,3.171)--(-8.406,3.172)--(-8.421,3.163)--cycle;
\filldraw[fill opacity=0.8,fill=gray!20,draw=none](-8.425,3.172)--(-8.382,3.176)--(-8.365,3.173)--(-8.354,3.168)--(-8.391,3.165)--(-8.425,3.171)--cycle;
\draw(-8.365,3.173)--(-8.354,3.168);
\filldraw[fill opacity=0.8,fill=gray!20,draw=none](-8.146,3.03)--(-8.155,3.05)--(-8.142,3.048)--(-8.139,3.039)--(-8.137,3.03)--(-8.135,3.018)--cycle;
\draw(-8.137,3.03)--(-8.135,3.018)--(-8.146,3.03);
\filldraw[fill opacity=0.8,fill=gray!20,draw=none](-8.269,2.87)--(-8.278,2.869)--(-8.365,2.907)--(-8.361,2.911)--(-8.353,2.915)--(-8.256,2.873)--cycle;
\draw(-8.353,2.915)--(-8.256,2.873);
\filldraw[fill opacity=0.8,fill=gray!20,draw=none](-8.278,2.869)--(-8.301,2.868)--(-8.346,2.88)--(-8.373,2.899)--(-8.365,2.907)--cycle;
\filldraw[fill opacity=0.8,fill=gray!20,draw=none](-8.354,2.892)--(-8.391,2.908)--(-8.38,2.919)--(-8.361,2.929)--(-8.309,2.907)--cycle;
\draw(-8.361,2.929)--(-8.309,2.907)--(-8.354,2.892)--(-8.391,2.908);
\filldraw[fill opacity=0.8,fill=gray!20,draw=none](-8.354,2.892)--(-8.391,2.908)--(-8.38,2.919)--(-8.361,2.929)--(-8.309,2.907)--cycle;
\draw(-8.361,2.929)--(-8.309,2.907)--(-8.354,2.892)--(-8.391,2.908);
\filldraw[fill opacity=0.8,fill=gray!20,draw=none](-8.296,3.108)--(-8.289,3.101)--(-8.285,3.088)--(-8.322,3.005)--(-8.369,3.026)--(-8.337,3.096)--cycle;
\draw(-8.285,3.088)--(-8.322,3.005)--(-8.369,3.026)--(-8.337,3.096);
\filldraw[fill opacity=0.8,fill=gray!20,draw=none](-8.337,3.096)--(-8.369,3.026)--(-8.413,3.046)--(-8.388,3.102)--cycle;
\draw(-8.337,3.096)--(-8.369,3.026)--(-8.413,3.046)--(-8.388,3.102);
\filldraw[fill opacity=0.8,fill=gray!20,draw=none](-8.262,3.044)--(-8.262,3.027)--(-8.28,2.987)--(-8.322,3.005)--(-8.285,3.088)--cycle;
\draw(-8.262,3.027)--(-8.28,2.987)--(-8.322,3.005)--(-8.285,3.088);
\filldraw[fill opacity=0.8,fill=gray!20,draw=none](-8.246,2.984)--(-8.254,2.976)--(-8.28,2.987)--(-8.262,3.027)--cycle;
\draw(-8.254,2.976)--(-8.28,2.987)--(-8.262,3.027);
\filldraw[fill opacity=0.8,fill=gray!20,draw=none](-8.363,2.893)--(-8.409,2.908)--(-8.391,2.908)--(-8.376,2.901)--cycle;
\draw(-8.391,2.908)--(-8.376,2.901);
\filldraw[fill opacity=0.8,fill=gray!20,draw=none](-8.362,2.893)--(-8.363,2.893)--(-8.376,2.901)--(-8.371,2.899)--cycle;
\draw(-8.362,2.893)--(-8.363,2.893);
\draw(-8.376,2.901)--(-8.371,2.899);
\filldraw[fill opacity=0.8,fill=gray!20,draw=none](-8.362,2.893)--(-8.363,2.893)--(-8.376,2.901)--(-8.371,2.899)--cycle;
\draw(-8.362,2.893)--(-8.363,2.893);
\draw(-8.376,2.901)--(-8.371,2.899);
\filldraw[fill opacity=0.8,fill=gray!20,draw=none](-8.362,2.893)--(-8.371,2.899)--(-8.354,2.892)--cycle;
\draw(-8.371,2.899)--(-8.354,2.892)--(-8.362,2.893);
\filldraw[fill opacity=0.8,fill=gray!20,draw=none](-8.362,2.893)--(-8.371,2.899)--(-8.354,2.892)--cycle;
\draw(-8.371,2.899)--(-8.354,2.892)--(-8.362,2.893);
\filldraw[fill opacity=0.8,fill=gray!20,draw=none](-8.251,2.974)--(-8.245,2.983)--(-8.236,3.033)--(-8.245,3.082)--(-8.271,3.12)--(-8.309,3.144)--(-8.354,3.149)--(-8.399,3.134)--(-8.438,3.102)--(-8.458,3.067)--cycle;
\draw(-8.251,2.974)--(-8.245,2.983)--(-8.236,3.033)--(-8.245,3.082)--(-8.271,3.12)--(-8.309,3.144)--(-8.354,3.149)--(-8.399,3.134)--(-8.438,3.102)--(-8.458,3.067);
\filldraw[fill opacity=0.8,fill=gray!20,draw=none](-8.251,2.974)--(-8.245,2.983)--(-8.236,3.033)--(-8.245,3.082)--(-8.271,3.12)--(-8.309,3.144)--(-8.354,3.149)--(-8.399,3.134)--(-8.438,3.102)--(-8.458,3.067)--cycle;
\draw(-8.251,2.974)--(-8.245,2.983)--(-8.236,3.033)--(-8.245,3.082)--(-8.271,3.12)--(-8.309,3.144)--(-8.354,3.149)--(-8.399,3.134)--(-8.438,3.102)--(-8.458,3.067);
\filldraw[fill opacity=0.8,fill=gray!20,draw=none](-8.472,3.008)--(-8.463,2.959)--(-8.438,2.92)--(-8.399,2.897)--(-8.354,2.892)--(-8.309,2.907)--(-8.271,2.939)--(-8.251,2.974)--(-8.458,3.067)--(-8.463,3.058)--cycle;
\draw(-8.458,3.067)--(-8.463,3.058)--(-8.472,3.008)--(-8.463,2.959)--(-8.438,2.92)--(-8.399,2.897)--(-8.354,2.892)--(-8.309,2.907)--(-8.271,2.939)--(-8.251,2.974);
\filldraw[fill opacity=0.8,fill=gray!20,draw=none](-8.472,3.008)--(-8.463,2.959)--(-8.438,2.92)--(-8.399,2.897)--(-8.354,2.892)--(-8.309,2.907)--(-8.271,2.939)--(-8.251,2.974)--(-8.458,3.067)--(-8.463,3.058)--cycle;
\draw(-8.458,3.067)--(-8.463,3.058)--(-8.472,3.008)--(-8.463,2.959)--(-8.438,2.92)--(-8.399,2.897)--(-8.354,2.892)--(-8.309,2.907)--(-8.271,2.939)--(-8.251,2.974);
\filldraw[fill opacity=0.8,fill=gray!20,draw=none](-8.236,3.033)--(-8.245,2.983)--(-8.251,2.974)--(-8.458,3.067)--(-8.438,3.102)--(-8.399,3.134)--(-8.354,3.149)--(-8.309,3.144)--(-8.271,3.12)--(-8.245,3.082)--cycle;
\draw(-8.458,3.067)--(-8.438,3.102)--(-8.399,3.134)--(-8.354,3.149)--(-8.309,3.144)--(-8.271,3.12)--(-8.245,3.082)--(-8.236,3.033)--(-8.245,2.983)--(-8.251,2.974);
\filldraw[fill opacity=0.8,fill=gray!20,draw=none](-8.251,2.974)--(-8.271,2.939)--(-8.309,2.907)--(-8.354,2.892)--(-8.399,2.897)--(-8.438,2.92)--(-8.463,2.959)--(-8.472,3.008)--(-8.463,3.058)--(-8.458,3.067)--cycle;
\draw(-8.251,2.974)--(-8.271,2.939)--(-8.309,2.907)--(-8.354,2.892)--(-8.399,2.897)--(-8.438,2.92)--(-8.463,2.959)--(-8.472,3.008)--(-8.463,3.058)--(-8.458,3.067);
\filldraw[fill opacity=0.8,fill=gray!20,draw=none](-8.236,3.033)--(-8.245,2.983)--(-8.251,2.974)--(-8.458,3.067)--(-8.438,3.102)--(-8.399,3.134)--(-8.354,3.149)--(-8.309,3.144)--(-8.271,3.12)--(-8.245,3.082)--cycle;
\draw(-8.458,3.067)--(-8.438,3.102)--(-8.399,3.134)--(-8.354,3.149)--(-8.309,3.144)--(-8.271,3.12)--(-8.245,3.082)--(-8.236,3.033)--(-8.245,2.983)--(-8.251,2.974);
\filldraw[fill opacity=0.8,fill=gray!20,draw=none](-8.237,3.127)--(-8.231,3.123)--(-8.283,3.146)--(-8.303,3.158)--(-8.246,3.133)--cycle;
\draw(-8.303,3.158)--(-8.246,3.133);
\filldraw[fill opacity=0.8,fill=gray!20,draw=none](-8.283,3.146)--(-8.231,3.123)--(-8.207,3.108)--(-8.245,3.124)--cycle;
\draw(-8.207,3.108)--(-8.245,3.124);
\filldraw[fill opacity=0.8,fill=gray!20,draw=none](-8.282,3.132)--(-8.239,3.107)--(-8.256,3.125)--(-8.271,3.132)--cycle;
\draw(-8.256,3.125)--(-8.271,3.132);
\filldraw[fill opacity=0.8,fill=gray!20,draw=none](-8.271,3.12)--(-8.239,3.107)--(-8.203,3.067)--(-8.201,3.062)--(-8.245,3.082)--cycle;
\draw(-8.201,3.062)--(-8.245,3.082)--(-8.271,3.12)--(-8.239,3.107);
\filldraw[fill opacity=0.8,fill=gray!20,draw=none](-8.135,3.001)--(-8.123,3.043)--(-8.12,3.031)--(-8.125,2.982)--(-8.126,2.979)--cycle;
\draw(-8.123,3.043)--(-8.12,3.031)--(-8.125,2.982)--(-8.126,2.979);
\filldraw[fill opacity=0.8,fill=gray!20,draw=none](-8.131,3.042)--(-8.121,3.036)--(-8.129,3.07)--(-8.151,3.079)--cycle;
\draw(-8.121,3.036)--(-8.129,3.07)--(-8.151,3.079);
\filldraw[fill opacity=0.8,fill=gray!20,draw=none](-8.142,3.048)--(-8.131,3.042)--(-8.151,3.079)--(-8.152,3.08)--cycle;
\draw(-8.151,3.079)--(-8.152,3.08);
\filldraw[fill opacity=0.8,fill=gray!20,draw=none](-8.271,3.12)--(-8.239,3.107)--(-8.203,3.067)--(-8.201,3.062)--(-8.245,3.082)--cycle;
\draw(-8.201,3.062)--(-8.245,3.082)--(-8.271,3.12)--(-8.239,3.107);
\filldraw[fill opacity=0.8,fill=gray!20,draw=none](-8.151,3.079)--(-8.129,3.07)--(-8.149,3.091)--(-8.165,3.098)--cycle;
\draw(-8.151,3.079)--(-8.129,3.07)--(-8.149,3.091)--(-8.165,3.098);
\filldraw[fill opacity=0.8,fill=gray!20,draw=none](-8.163,3.095)--(-8.143,3.073)--(-8.141,3.064)--cycle;
\draw(-8.163,3.095)--(-8.143,3.073)--(-8.141,3.064);
\filldraw[fill opacity=0.8,fill=gray!20,draw=none](-8.309,3.144)--(-8.282,3.132)--(-8.239,3.107)--(-8.271,3.12)--cycle;
\draw(-8.239,3.107)--(-8.271,3.12)--(-8.309,3.144)--(-8.282,3.132);
\filldraw[fill opacity=0.8,fill=gray!20,draw=none](-8.208,3.088)--(-8.189,3.077)--(-8.18,3.092)--(-8.233,3.115)--cycle;
\draw(-8.18,3.092)--(-8.233,3.115);
\filldraw[fill opacity=0.8,fill=gray!20,draw=none](-8.18,3.092)--(-8.151,3.079)--(-8.165,3.098)--(-8.17,3.1)--cycle;
\draw(-8.18,3.092)--(-8.151,3.079);
\draw(-8.165,3.098)--(-8.17,3.1);
\filldraw[fill opacity=0.8,fill=gray!20,draw=none](-8.239,3.107)--(-8.208,3.088)--(-8.233,3.115)--(-8.256,3.125)--cycle;
\draw(-8.233,3.115)--(-8.256,3.125);
\filldraw[fill opacity=0.8,fill=gray!20,draw=none](-8.233,3.115)--(-8.213,3.1)--(-8.223,3.104)--cycle;
\filldraw[fill opacity=0.8,fill=gray!20,draw=none](-8.309,3.144)--(-8.282,3.132)--(-8.239,3.107)--(-8.271,3.12)--cycle;
\draw(-8.239,3.107)--(-8.271,3.12)--(-8.309,3.144)--(-8.282,3.132);
\filldraw[fill opacity=0.8,fill=gray!20,draw=none](-8.282,3.132)--(-8.271,3.132)--(-8.307,3.147)--cycle;
\draw(-8.271,3.132)--(-8.307,3.147);
\filldraw[fill opacity=0.8,fill=gray!20,draw=none](-8.354,3.149)--(-8.318,3.133)--(-8.282,3.132)--(-8.309,3.144)--cycle;
\draw(-8.282,3.132)--(-8.309,3.144)--(-8.354,3.149)--(-8.318,3.133);
\filldraw[fill opacity=0.8,fill=gray!20,draw=none](-8.204,3.103)--(-8.201,3.095)--(-8.213,3.1)--(-8.245,3.124)--(-8.207,3.108)--cycle;
\draw(-8.245,3.124)--(-8.207,3.108);
\filldraw[fill opacity=0.8,fill=gray!20,draw=none](-8.203,3.067)--(-8.193,3.071)--(-8.189,3.077)--(-8.239,3.107)--cycle;
\filldraw[fill opacity=0.8,fill=gray!20,draw=none](-8.193,3.072)--(-8.193,3.071)--(-8.223,3.104)--(-8.213,3.1)--(-8.199,3.089)--cycle;
\filldraw[fill opacity=0.8,fill=gray!20,draw=none](-8.282,3.132)--(-8.251,3.119)--(-8.242,3.114)--(-8.216,3.096)--(-8.239,3.107)--cycle;
\draw(-8.282,3.132)--(-8.251,3.119);
\draw(-8.216,3.096)--(-8.239,3.107);
\filldraw[fill opacity=0.8,fill=gray!20,draw=none](-8.282,3.132)--(-8.251,3.119)--(-8.242,3.114)--(-8.216,3.096)--(-8.239,3.107)--cycle;
\draw(-8.282,3.132)--(-8.251,3.119);
\draw(-8.216,3.096)--(-8.239,3.107);
\filldraw[fill opacity=0.8,fill=gray!20,draw=none](-8.354,3.149)--(-8.318,3.133)--(-8.282,3.132)--(-8.309,3.144)--cycle;
\draw(-8.282,3.132)--(-8.309,3.144)--(-8.354,3.149)--(-8.318,3.133);
\filldraw[fill opacity=0.8,fill=gray!20,draw=none](-8.318,3.133)--(-8.282,3.132)--(-8.305,3.146)--cycle;
\filldraw[fill opacity=0.8,fill=gray!20,draw=none](-8.226,3.036)--(-8.223,3.033)--(-8.224,3.03)--cycle;
\draw(-8.223,3.033)--(-8.224,3.03);
\filldraw[fill opacity=0.8,fill=gray!20,draw=none](-8.22,3.033)--(-8.224,3.03)--(-8.223,3.033)--cycle;
\draw(-8.224,3.03)--(-8.223,3.033);
\filldraw[fill opacity=0.8,fill=gray!20,draw=none](-8.224,3.03)--(-8.226,3.036)--(-8.22,3.033)--cycle;
\draw(-8.226,3.036)--(-8.22,3.033);
\filldraw[fill opacity=0.8,fill=gray!20,draw=none](-8.216,3.005)--(-8.224,3.03)--(-8.22,3.033)--(-8.207,3.027)--cycle;
\draw(-8.22,3.033)--(-8.207,3.027);
\filldraw[fill opacity=0.8,fill=gray!20,draw=none](-8.22,3.033)--(-8.203,3.038)--(-8.213,3.017)--(-8.232,3.012)--(-8.224,3.03)--cycle;
\draw(-8.203,3.038)--(-8.213,3.017);
\draw(-8.232,3.012)--(-8.224,3.03);
\filldraw[fill opacity=0.8,fill=gray!20,draw=none](-8.318,3.133)--(-8.329,3.122)--(-8.203,3.067)--(-8.239,3.107)--(-8.282,3.132)--cycle;
\draw(-8.329,3.122)--(-8.203,3.067);
\filldraw[fill opacity=0.8,fill=gray!20,draw=none](-8.399,3.134)--(-8.362,3.118)--(-8.35,3.119)--(-8.328,3.122)--(-8.312,3.131)--(-8.354,3.149)--cycle;
\draw(-8.312,3.131)--(-8.354,3.149)--(-8.399,3.134)--(-8.362,3.118);
\filldraw[fill opacity=0.8,fill=gray!20,draw=none](-8.245,3.082)--(-8.201,3.062)--(-8.202,3.052)--(-8.211,3.022)--(-8.236,3.033)--cycle;
\draw(-8.211,3.022)--(-8.236,3.033)--(-8.245,3.082)--(-8.201,3.062);
\filldraw[fill opacity=0.8,fill=gray!20,draw=none](-8.399,3.134)--(-8.362,3.118)--(-8.35,3.119)--(-8.328,3.122)--(-8.312,3.131)--(-8.354,3.149)--cycle;
\draw(-8.312,3.131)--(-8.354,3.149)--(-8.399,3.134)--(-8.362,3.118);
\filldraw[fill opacity=0.8,fill=gray!20,draw=none](-8.326,3.151)--(-8.342,3.135)--(-8.34,3.134)--(-8.318,3.133)--(-8.305,3.146)--(-8.307,3.147)--(-8.315,3.151)--cycle;
\draw(-8.307,3.147)--(-8.315,3.151);
\filldraw[fill opacity=0.8,fill=gray!20,draw=none](-8.179,3.132)--(-8.189,3.108)--(-8.216,3.131)--(-8.212,3.138)--cycle;
\draw(-8.179,3.132)--(-8.189,3.108);
\draw(-8.216,3.131)--(-8.212,3.138);
\filldraw[fill opacity=0.8,fill=gray!20,draw=none](-8.165,2.999)--(-8.169,2.951)--(-8.201,2.965)--(-8.216,3.005)--(-8.207,3.027)--(-8.166,3.01)--cycle;
\draw(-8.169,2.951)--(-8.201,2.965);
\draw(-8.207,3.027)--(-8.166,3.01);
\filldraw[fill opacity=0.8,fill=gray!20,draw=none](-8.245,3.082)--(-8.201,3.062)--(-8.202,3.052)--(-8.211,3.022)--(-8.236,3.033)--cycle;
\draw(-8.211,3.022)--(-8.236,3.033)--(-8.245,3.082)--(-8.201,3.062);
\filldraw[fill opacity=0.8,fill=gray!20,draw=none](-8.213,3.017)--(-8.217,3.008)--(-8.244,2.983)--(-8.244,2.986)--(-8.232,3.012)--cycle;
\draw(-8.213,3.017)--(-8.217,3.008);
\draw(-8.244,2.986)--(-8.232,3.012);
\filldraw[fill opacity=0.8,fill=gray!20,draw=none](-8.244,2.983)--(-8.246,2.982)--(-8.244,2.986)--cycle;
\draw(-8.246,2.982)--(-8.244,2.986);
\filldraw[fill opacity=0.8,fill=gray!20,draw=none](-8.216,3.005)--(-8.201,2.965)--(-8.227,2.976)--cycle;
\draw(-8.201,2.965)--(-8.227,2.976);
\filldraw[fill opacity=0.8,fill=gray!20,draw=none](-8.14,2.981)--(-8.146,3.03)--(-8.135,3.001)--cycle;
\filldraw[fill opacity=0.8,fill=gray!20,draw=none](-8.146,3.03)--(-8.135,3.018)--(-8.14,2.981)--cycle;
\draw(-8.146,3.03)--(-8.135,3.018)--(-8.14,2.981);
\filldraw[fill opacity=0.8,fill=gray!20,draw=none](-8.199,3.089)--(-8.213,3.1)--(-8.201,3.095)--cycle;
\filldraw[fill opacity=0.8,fill=gray!20,draw=none](-8.22,3.13)--(-8.235,3.131)--(-8.235,3.13)--(-8.212,3.091)--(-8.209,3.092)--(-8.204,3.103)--cycle;
\draw(-8.235,3.131)--(-8.235,3.13);
\draw(-8.209,3.092)--(-8.204,3.103);
\filldraw[fill opacity=0.8,fill=gray!20,draw=none](-8.211,3.088)--(-8.203,3.068)--(-8.239,3.107)--(-8.216,3.096)--cycle;
\draw(-8.239,3.107)--(-8.216,3.096);
\filldraw[fill opacity=0.8,fill=gray!20,draw=none](-8.211,3.088)--(-8.203,3.068)--(-8.239,3.107)--(-8.216,3.096)--cycle;
\draw(-8.239,3.107)--(-8.216,3.096);
\filldraw[fill opacity=0.8,fill=gray!20,draw=none](-8.236,3.033)--(-8.211,3.022)--(-8.229,2.976)--(-8.245,2.983)--cycle;
\draw(-8.229,2.976)--(-8.245,2.983)--(-8.236,3.033)--(-8.211,3.022);
\filldraw[fill opacity=0.8,fill=gray!20,draw=none](-8.236,3.033)--(-8.211,3.022)--(-8.229,2.976)--(-8.245,2.983)--cycle;
\draw(-8.229,2.976)--(-8.245,2.983)--(-8.236,3.033)--(-8.211,3.022);
\filldraw[fill opacity=0.8,fill=gray!20,draw=none](-8.309,3.084)--(-8.27,3.051)--(-8.228,3.049)--(-8.204,3.068)--(-8.328,3.122)--cycle;
\draw(-8.204,3.068)--(-8.328,3.122);
\filldraw[fill opacity=0.8,fill=gray!20,draw=none](-8.227,3.116)--(-8.235,3.13)--(-8.241,3.117)--cycle;
\draw(-8.235,3.13)--(-8.241,3.117);
\filldraw[fill opacity=0.8,fill=gray!20,draw=none](-8.203,3.129)--(-8.22,3.13)--(-8.204,3.103)--(-8.201,3.111)--cycle;
\draw(-8.204,3.103)--(-8.201,3.111);
\filldraw[fill opacity=0.8,fill=gray!20,draw=none](-8.23,3.123)--(-8.231,3.123)--(-8.237,3.127)--cycle;
\filldraw[fill opacity=0.8,fill=gray!20,draw=none](-8.227,3.116)--(-8.241,3.117)--(-8.265,3.062)--(-8.212,3.091)--cycle;
\draw(-8.241,3.117)--(-8.265,3.062);
\filldraw[fill opacity=0.8,fill=gray!20,draw=none](-8.282,3.138)--(-8.251,3.119)--(-8.241,3.117)--(-8.235,3.131)--cycle;
\draw(-8.241,3.117)--(-8.235,3.131);
\filldraw[fill opacity=0.8,fill=gray!20,draw=none](-8.218,3.118)--(-8.23,3.123)--(-8.237,3.127)--(-8.244,3.132)--(-8.235,3.128)--cycle;
\draw(-8.244,3.132)--(-8.235,3.128);
\filldraw[fill opacity=0.8,fill=gray!20,draw=none](-8.237,3.127)--(-8.246,3.133)--(-8.244,3.132)--cycle;
\draw(-8.246,3.133)--(-8.244,3.132);
\filldraw[fill opacity=0.8,fill=gray!20,draw=none](-8.318,3.133)--(-8.303,3.127)--(-8.291,3.124)--(-8.251,3.119)--(-8.282,3.132)--cycle;
\draw(-8.318,3.133)--(-8.303,3.127);
\draw(-8.251,3.119)--(-8.282,3.132);
\filldraw[fill opacity=0.8,fill=gray!20,draw=none](-8.318,3.133)--(-8.303,3.127)--(-8.291,3.124)--(-8.251,3.119)--(-8.282,3.132)--cycle;
\draw(-8.318,3.133)--(-8.303,3.127);
\draw(-8.251,3.119)--(-8.282,3.132);
\filldraw[fill opacity=0.8,fill=gray!20,draw=none](-8.282,3.138)--(-8.285,3.138)--(-8.29,3.128)--(-8.288,3.124)--(-8.251,3.119)--cycle;
\draw(-8.285,3.138)--(-8.29,3.128);
\filldraw[fill opacity=0.8,fill=gray!20,draw=none](-7.979,3.65)--(-7.984,3.65)--(-7.981,3.683)--(-7.927,3.679)--(-7.956,3.648)--cycle;
\draw(-7.984,3.65)--(-7.981,3.683)--(-7.927,3.679)--(-7.956,3.648)--(-7.979,3.65);
\filldraw[fill opacity=0.8,fill=gray!20](-7.956,3.648)--(-7.927,3.679)--(-7.889,3.67)--(-7.936,3.643)--cycle;
\filldraw[fill opacity=0.8,fill=gray!20,draw=none](-7.975,3.642)--(-7.958,3.627)--(-8.179,3.132)--(-8.199,3.168)--(-7.987,3.644)--cycle;
\draw(-7.958,3.627)--(-8.179,3.132);
\draw(-8.199,3.168)--(-7.987,3.644);
\filldraw[fill opacity=0.8,fill=gray!20,draw=none](-8.179,3.132)--(-8.212,3.138)--(-8.199,3.168)--cycle;
\draw(-8.212,3.138)--(-8.199,3.168);
\filldraw[fill opacity=0.8,fill=gray!20,draw=none](-8.212,3.138)--(-8.242,3.16)--(-8.184,3.146)--(-8.179,3.132)--cycle;
\draw(-8.242,3.16)--(-8.184,3.146)--(-8.179,3.132);
\filldraw[fill opacity=0.8,fill=gray!20,draw=none](-7.956,3.594)--(-8.154,3.149)--(-8.179,3.132)--(-7.958,3.627)--cycle;
\draw(-7.956,3.594)--(-8.154,3.149);
\draw(-8.179,3.132)--(-7.958,3.627);
\filldraw[fill opacity=0.8,fill=gray!20,draw=none](-8.168,3.105)--(-8.179,3.132)--(-8.167,3.119)--cycle;
\draw(-8.168,3.105)--(-8.179,3.132);
\filldraw[fill opacity=0.8,fill=gray!20,draw=none](-8.154,3.149)--(-8.179,3.093)--(-8.189,3.108)--(-8.179,3.132)--cycle;
\draw(-8.154,3.149)--(-8.179,3.093);
\draw(-8.189,3.108)--(-8.179,3.132);
\filldraw[fill opacity=0.8,fill=gray!20,draw=none](-8.167,3.119)--(-8.179,3.132)--(-8.184,3.146)--(-8.167,3.128)--cycle;
\draw(-8.179,3.132)--(-8.184,3.146)--(-8.167,3.128);
\filldraw[fill opacity=0.8,fill=gray!20,draw=none](-8.244,2.983)--(-8.247,2.972)--(-8.249,2.973)--(-8.246,2.982)--cycle;
\draw(-8.247,2.972)--(-8.249,2.973)--(-8.246,2.982);
\filldraw[fill opacity=0.8,fill=gray!20,draw=none](-8.251,2.974)--(-8.249,2.973)--(-8.247,2.972)--cycle;
\draw(-8.249,2.973)--(-8.247,2.972);
\filldraw[fill opacity=0.8,fill=gray!20,draw=none](-8.245,2.983)--(-8.257,2.932)--(-8.271,2.939)--cycle;
\draw(-8.257,2.932)--(-8.271,2.939)--(-8.245,2.983);
\filldraw[fill opacity=0.8,fill=gray!20,draw=none](-8.245,2.983)--(-8.181,2.955)--(-8.185,2.946)--(-8.209,2.912)--(-8.257,2.932)--cycle;
\draw(-8.245,2.983)--(-8.181,2.955);
\draw(-8.209,2.912)--(-8.257,2.932);
\filldraw[fill opacity=0.8,fill=gray!20,draw=none](-8.203,2.865)--(-8.199,2.867)--(-8.164,2.911)--(-8.171,2.906)--cycle;
\draw(-8.203,2.865)--(-8.199,2.867)--(-8.164,2.911)--(-8.171,2.906);
\filldraw[fill opacity=0.8,fill=gray!20,draw=none](-8.159,2.898)--(-8.17,2.879)--(-8.184,2.863)--(-8.196,2.86)--(-8.204,2.863)--cycle;
\draw(-8.159,2.898)--(-8.17,2.879)--(-8.184,2.863);
\filldraw[fill opacity=0.8,fill=gray!20,draw=none](-8.245,2.983)--(-8.181,2.955)--(-8.185,2.946)--(-8.209,2.912)--(-8.271,2.939)--cycle;
\draw(-8.209,2.912)--(-8.271,2.939)--(-8.245,2.983)--(-8.181,2.955);
\filldraw[fill opacity=0.8,fill=gray!20,draw=none](-8.217,3.008)--(-8.235,2.967)--(-8.247,2.972)--(-8.244,2.983)--cycle;
\draw(-8.217,3.008)--(-8.235,2.967)--(-8.247,2.972);
\filldraw[fill opacity=0.8,fill=gray!20,draw=none](-7.987,3.645)--(-8.2,3.167)--(-8.25,3.165)--(-8.026,3.67)--cycle;
\draw(-7.987,3.645)--(-8.2,3.167);
\draw(-8.25,3.165)--(-8.026,3.67);
\filldraw[fill opacity=0.8,fill=gray!20,draw=none](-8.167,3.128)--(-8.184,3.146)--(-8.215,3.187)--(-8.199,3.17)--(-8.167,3.129)--cycle;
\draw(-8.167,3.128)--(-8.184,3.146)--(-8.215,3.187)--(-8.199,3.17)--(-8.167,3.129);
\filldraw[fill opacity=0.8,fill=gray!20,draw=none](-8.184,3.131)--(-8.201,3.105)--(-8.201,3.095)--(-8.193,3.072)--(-8.167,3.129)--cycle;
\draw(-8.193,3.072)--(-8.167,3.129);
\filldraw[fill opacity=0.8,fill=gray!20,draw=none](-8.201,3.105)--(-8.209,3.092)--(-8.26,2.979)--(-8.238,2.969)--(-8.202,3.052)--cycle;
\draw(-8.209,3.092)--(-8.26,2.979)--(-8.238,2.969)--(-8.202,3.052);
\filldraw[fill opacity=0.8,fill=gray!20,draw=none](-8.184,3.131)--(-8.192,3.132)--(-8.209,3.092)--cycle;
\draw(-8.192,3.132)--(-8.209,3.092);
\filldraw[fill opacity=0.8,fill=gray!20,draw=none](-8.188,3.097)--(-8.19,3.09)--(-8.201,3.095)--(-8.206,3.107)--(-8.19,3.1)--cycle;
\draw(-8.206,3.107)--(-8.19,3.1);
\filldraw[fill opacity=0.8,fill=gray!20,draw=none](-8.203,3.129)--(-8.201,3.111)--(-8.193,3.128)--cycle;
\draw(-8.201,3.111)--(-8.193,3.128);
\filldraw[fill opacity=0.8,fill=gray!20,draw=none](-8.204,3.103)--(-8.207,3.108)--(-8.206,3.107)--cycle;
\draw(-8.207,3.108)--(-8.206,3.107);
\filldraw[fill opacity=0.8,fill=gray!20,draw=none](-8.231,3.123)--(-8.222,3.12)--(-8.206,3.107)--(-8.207,3.108)--cycle;
\draw(-8.206,3.107)--(-8.207,3.108);
\filldraw[fill opacity=0.8,fill=gray!20,draw=none](-8.196,3.105)--(-8.196,3.105)--(-8.205,3.121)--(-8.209,3.125)--(-8.223,3.127)--(-8.227,3.123)--cycle;
\draw(-8.196,3.105)--(-8.196,3.105)--(-8.205,3.121);
\filldraw[fill opacity=0.8,fill=gray!20,draw=none](-8.318,3.133)--(-8.34,3.134)--(-8.329,3.122)--cycle;
\filldraw[fill opacity=0.8,fill=gray!20,draw=none](-8.328,3.122)--(-8.303,3.127)--(-8.312,3.131)--cycle;
\draw(-8.303,3.127)--(-8.312,3.131);
\filldraw[fill opacity=0.8,fill=gray!20,draw=none](-8.328,3.122)--(-8.303,3.127)--(-8.312,3.131)--cycle;
\draw(-8.303,3.127)--(-8.312,3.131);
\filldraw[fill opacity=0.8,fill=gray!20,draw=none](-8.29,3.129)--(-8.289,3.139)--(-8.305,3.142)--(-8.326,3.138)--(-8.334,3.134)--(-8.328,3.122)--cycle;
\filldraw[fill opacity=0.8,fill=gray!20,draw=none](-8.289,3.139)--(-8.29,3.128)--(-8.285,3.138)--cycle;
\draw(-8.29,3.128)--(-8.285,3.138);
\filldraw[fill opacity=0.8,fill=gray!20,draw=none](-8.251,3.119)--(-8.242,3.114)--(-8.241,3.117)--cycle;
\draw(-8.242,3.114)--(-8.241,3.117);
\filldraw[fill opacity=0.8,fill=gray!20,draw=none](-8.249,3.118)--(-8.237,3.112)--(-8.214,3.095)--(-8.216,3.096)--cycle;
\draw(-8.249,3.118)--(-8.237,3.112);
\draw(-8.214,3.095)--(-8.216,3.096);
\filldraw[fill opacity=0.8,fill=gray!20,draw=none](-8.242,3.114)--(-8.22,3.1)--(-8.214,3.095)--(-8.216,3.096)--cycle;
\draw(-8.214,3.095)--(-8.216,3.096);
\filldraw[fill opacity=0.8,fill=gray!20,draw=none](-8.242,3.114)--(-8.249,3.118)--(-8.237,3.112)--(-8.22,3.1)--cycle;
\draw(-8.249,3.118)--(-8.237,3.112);
\filldraw[fill opacity=0.8,fill=gray!20,draw=none](-8.251,3.119)--(-8.249,3.118)--(-8.242,3.114)--cycle;
\draw(-8.251,3.119)--(-8.249,3.118);
\filldraw[fill opacity=0.8,fill=gray!20,draw=none](-8.251,3.119)--(-8.249,3.118)--(-8.242,3.114)--cycle;
\draw(-8.251,3.119)--(-8.249,3.118);
\filldraw[fill opacity=0.8,fill=gray!20,draw=none](-8.251,3.119)--(-8.288,3.124)--(-8.265,3.062)--(-8.242,3.114)--cycle;
\draw(-8.265,3.062)--(-8.242,3.114);
\filldraw[fill opacity=0.8,fill=gray!20,draw=none](-8.29,3.128)--(-8.309,3.084)--(-8.27,3.051)--(-8.265,3.062)--cycle;
\draw(-8.29,3.128)--(-8.309,3.084);
\draw(-8.27,3.051)--(-8.265,3.062);
\filldraw[fill opacity=0.8,fill=gray!20,draw=none](-8.201,3.095)--(-8.202,3.052)--(-8.193,3.072)--cycle;
\draw(-8.202,3.052)--(-8.193,3.072);
\filldraw[fill opacity=0.8,fill=gray!20,draw=none](-8.193,3.072)--(-8.199,3.089)--(-8.191,3.083)--cycle;
\filldraw[fill opacity=0.8,fill=gray!20,draw=none](-8.203,3.067)--(-8.197,3.065)--(-8.193,3.071)--cycle;
\draw(-8.203,3.067)--(-8.197,3.065);
\filldraw[fill opacity=0.8,fill=gray!20,draw=none](-8.201,3.062)--(-8.199,3.061)--(-8.202,3.052)--cycle;
\draw(-8.201,3.062)--(-8.199,3.061);
\filldraw[fill opacity=0.8,fill=gray!20,draw=none](-8.201,3.062)--(-8.199,3.061)--(-8.202,3.052)--cycle;
\draw(-8.201,3.062)--(-8.199,3.061);
\filldraw[fill opacity=0.8,fill=gray!20,draw=none](-8.203,3.067)--(-8.199,3.062)--(-8.199,3.061)--(-8.201,3.062)--cycle;
\draw(-8.199,3.061)--(-8.201,3.062);
\filldraw[fill opacity=0.8,fill=gray!20,draw=none](-8.203,3.067)--(-8.199,3.062)--(-8.199,3.061)--(-8.201,3.062)--cycle;
\draw(-8.199,3.061)--(-8.201,3.062);
\filldraw[fill opacity=0.8,fill=gray!20,draw=none](-8.27,3.051)--(-8.295,2.995)--(-8.26,2.979)--(-8.228,3.049)--cycle;
\draw(-8.27,3.051)--(-8.295,2.995)--(-8.26,2.979)--(-8.228,3.049);
\filldraw[fill opacity=0.8,fill=gray!20,draw=none](-8.247,3.035)--(-8.219,3.023)--(-8.197,3.065)--(-8.204,3.068)--cycle;
\draw(-8.247,3.035)--(-8.219,3.023);
\draw(-8.197,3.065)--(-8.204,3.068);
\filldraw[fill opacity=0.8,fill=gray!20,draw=none](-8.191,3.083)--(-8.199,3.089)--(-8.201,3.095)--(-8.19,3.09)--cycle;
\filldraw[fill opacity=0.8,fill=gray!20,draw=none](-8.291,3.124)--(-8.263,3.119)--(-8.249,3.118)--(-8.251,3.119)--cycle;
\draw(-8.249,3.118)--(-8.251,3.119);
\filldraw[fill opacity=0.8,fill=gray!20,draw=none](-8.291,3.124)--(-8.263,3.119)--(-8.249,3.118)--(-8.251,3.119)--cycle;
\draw(-8.249,3.118)--(-8.251,3.119);
\filldraw[fill opacity=0.8,fill=gray!20,draw=none](-8.198,3.066)--(-8.198,3.062)--(-8.203,3.068)--(-8.211,3.088)--cycle;
\filldraw[fill opacity=0.8,fill=gray!20,draw=none](-8.203,3.068)--(-8.211,3.088)--(-8.198,3.066)--(-8.198,3.062)--cycle;
\filldraw[fill opacity=0.8,fill=gray!20,draw=none](-8.265,3.062)--(-8.27,3.051)--(-8.228,3.049)--(-8.209,3.092)--cycle;
\draw(-8.265,3.062)--(-8.27,3.051);
\draw(-8.228,3.049)--(-8.209,3.092);
\filldraw[fill opacity=0.8,fill=gray!20,draw=none](-8.198,3.066)--(-8.211,3.088)--(-8.214,3.095)--(-8.194,3.087)--cycle;
\draw(-8.214,3.095)--(-8.194,3.087);
\filldraw[fill opacity=0.8,fill=gray!20,draw=none](-8.211,3.088)--(-8.214,3.095)--(-8.194,3.087)--(-8.198,3.066)--cycle;
\draw(-8.214,3.095)--(-8.194,3.087);
\filldraw[fill opacity=0.8,fill=gray!20,draw=none](-8.211,3.088)--(-8.216,3.096)--(-8.214,3.095)--cycle;
\draw(-8.216,3.096)--(-8.214,3.095);
\filldraw[fill opacity=0.8,fill=gray!20,draw=none](-8.211,3.088)--(-8.216,3.096)--(-8.214,3.095)--cycle;
\draw(-8.216,3.096)--(-8.214,3.095);
\filldraw[fill opacity=0.8,fill=gray!20,draw=none](-8.239,3.114)--(-8.225,3.125)--(-8.226,3.127)--(-8.266,3.13)--cycle;
\filldraw[fill opacity=0.8,fill=gray!20,draw=none](-8.208,3.113)--(-8.218,3.118)--(-8.235,3.128)--(-8.222,3.122)--cycle;
\draw(-8.235,3.128)--(-8.222,3.122)--(-8.208,3.113);
\filldraw[fill opacity=0.8,fill=gray!20,draw=none](-8.29,3.121)--(-8.277,3.115)--(-8.234,3.111)--(-8.249,3.118)--cycle;
\draw(-8.29,3.121)--(-8.277,3.115);
\draw(-8.234,3.111)--(-8.249,3.118);
\filldraw[fill opacity=0.8,fill=gray!20,draw=none](-8.29,3.121)--(-8.277,3.115)--(-8.242,3.112)--(-8.237,3.112)--(-8.249,3.118)--cycle;
\draw(-8.29,3.121)--(-8.277,3.115);
\draw(-8.237,3.112)--(-8.249,3.118);
\filldraw[fill opacity=0.8,fill=gray!20,draw=none](-8.29,3.129)--(-8.328,3.122)--(-8.303,3.098)--(-8.29,3.128)--cycle;
\draw(-8.303,3.098)--(-8.29,3.128);
\filldraw[fill opacity=0.8,fill=gray!20,draw=none](-8.291,3.124)--(-8.301,3.126)--(-8.29,3.121)--(-8.263,3.119)--cycle;
\draw(-8.301,3.126)--(-8.29,3.121);
\filldraw[fill opacity=0.8,fill=gray!20,draw=none](-8.291,3.124)--(-8.301,3.126)--(-8.29,3.121)--(-8.263,3.119)--cycle;
\draw(-8.301,3.126)--(-8.29,3.121);
\filldraw[fill opacity=0.8,fill=gray!20,draw=none](-8.242,3.112)--(-8.239,3.114)--(-8.266,3.13)--(-8.282,3.132)--(-8.277,3.115)--cycle;
\draw(-8.282,3.132)--(-8.277,3.115);
\filldraw[fill opacity=0.8,fill=gray!20](-8.222,3.122)--(-8.386,3.194)--(-8.436,3.199)--(-8.272,3.128)--cycle;
\filldraw[fill opacity=0.8,fill=gray!20,draw=none](-6.123,.471)--(-6.123,.429)--(-6.178,.451)--(-6.178,.479)--cycle;
\draw(-6.123,.471)--(-6.123,.429);
\draw(-6.178,.451)--(-6.178,.479);
\filldraw[fill opacity=0.8,fill=gray!20,draw=none](-6.176,.439)--(-6.142,.412)--(-6.157,.39)--(-6.17,.412)--cycle;
\draw(-6.142,.412)--(-6.157,.39);
\filldraw[fill opacity=0.8,fill=gray!20,draw=none](-6.176,.394)--(-6.123,.378)--(-6.123,.37)--(-6.178,.359)--(-6.178,.39)--cycle;
\draw(-6.123,.378)--(-6.123,.37);
\draw(-6.178,.359)--(-6.178,.39);
\filldraw[fill opacity=0.8,fill=gray!20,draw=none](-6.107,.373)--(-6.113,.364)--(-6.148,.385)--cycle;
\draw(-6.107,.373)--(-6.113,.364);
\filldraw[fill opacity=0.8,fill=gray!20,draw=none](-6.067,.434)--(-6.067,.357)--(-6.111,.367)--(-6.123,.378)--(-6.123,.429)--cycle;
\draw(-6.067,.434)--(-6.067,.357);
\draw(-6.123,.378)--(-6.123,.429);
\filldraw[fill opacity=0.8,fill=gray!20,draw=none](-6.176,.439)--(-6.17,.412)--(-6.183,.432)--(-6.177,.44)--cycle;
\draw(-6.183,.432)--(-6.177,.44);
\filldraw[fill opacity=0.8,fill=gray!20,draw=none](-6.177,.44)--(-6.183,.432)--(-6.185,.459)--cycle;
\draw(-6.177,.44)--(-6.183,.432);
\filldraw[fill opacity=0.8,fill=gray!20,draw=none](-6.231,.459)--(-6.197,.459)--(-6.165,.412)--(-6.27,.413)--cycle;
\draw(-6.231,.459)--(-6.197,.459);
\draw(-6.165,.412)--(-6.27,.413);
\filldraw[fill opacity=0.8,fill=gray!20,draw=none](-6.144,.366)--(-6.136,.354)--(-6.178,.342)--(-6.178,.359)--cycle;
\draw(-6.178,.342)--(-6.178,.359);
\filldraw[fill opacity=0.8,fill=gray!20,draw=none](-6.149,.386)--(-6.113,.364)--(-6.138,.327)--cycle;
\draw(-6.113,.364)--(-6.138,.327);
\filldraw[fill opacity=0.8,fill=gray!20,draw=none](-6.262,.367)--(-6.519,.415)--(-6.245,.412)--cycle;
\draw(-6.519,.415)--(-6.245,.412);
\filldraw[fill opacity=0.8,fill=gray!20,draw=none](-6.335,.362)--(-6.264,.361)--(-6.301,.327)--cycle;
\draw(-6.335,.362)--(-6.264,.361);
\filldraw[fill opacity=0.8,fill=gray!20,draw=none](-6.317,.452)--(-6.261,.452)--(-6.233,.379)--(-6.313,.346)--(-6.316,.35)--(-6.326,.404)--cycle;
\draw(-6.313,.346)--(-6.316,.35)--(-6.326,.404)--(-6.317,.452);
\filldraw[fill opacity=0.8,fill=gray!20,draw=none](-6.253,.371)--(-6.301,.327)--(-6.313,.346)--cycle;
\draw(-6.301,.327)--(-6.313,.346);
\filldraw[fill opacity=0.8,fill=gray!20,draw=none](-6.17,.412)--(-6.157,.39)--(-6.158,.388)--(-6.169,.406)--cycle;
\draw(-6.157,.39)--(-6.158,.388);
\filldraw[fill opacity=0.8,fill=gray!20,draw=none](-6.262,.367)--(-6.245,.412)--(-6.165,.412)--(-6.197,.36)--(-6.227,.361)--cycle;
\draw(-6.245,.412)--(-6.165,.412);
\draw(-6.197,.36)--(-6.227,.361);
\filldraw[fill opacity=0.8,fill=gray!20,draw=none](-6.17,.412)--(-6.169,.406)--(-6.184,.43)--(-6.183,.432)--cycle;
\draw(-6.184,.43)--(-6.183,.432);
\filldraw[fill opacity=0.8,fill=gray!20,draw=none](-6.185,.36)--(-6.178,.39)--(-6.178,.37)--cycle;
\draw(-6.178,.39)--(-6.178,.37);
\filldraw[fill opacity=0.8,fill=gray!20,draw=none](-6.176,.394)--(-6.144,.36)--(-6.197,.36)--cycle;
\draw(-6.144,.36)--(-6.197,.36);
\filldraw[fill opacity=0.8,fill=gray!20,draw=none](-6.194,.348)--(-6.227,.361)--(-6.185,.36)--cycle;
\draw(-6.227,.361)--(-6.185,.36);
\filldraw[fill opacity=0.8,fill=gray!20,draw=none](-6.178,.395)--(-6.178,.39)--(-6.187,.353)--(-6.223,.328)--(-6.223,.376)--cycle;
\draw(-6.178,.395)--(-6.178,.39);
\draw(-6.223,.328)--(-6.223,.376);
\filldraw[fill opacity=0.8,fill=gray!20,draw=none](-6.194,.348)--(-6.185,.36)--(-6.107,.359)--(-6.108,.316)--cycle;
\draw(-6.185,.36)--(-6.107,.359)--(-6.108,.316);
\filldraw[fill opacity=0.8,fill=gray!20,draw=none](-6.149,.386)--(-6.138,.327)--(-6.146,.314)--(-6.186,.346)--(-6.157,.39)--cycle;
\draw(-6.138,.327)--(-6.146,.314)--(-6.186,.346)--(-6.157,.39);
\filldraw[fill opacity=0.8,fill=gray!20,draw=none](-6.123,.429)--(-6.123,.399)--(-6.149,.386)--(-6.178,.395)--(-6.178,.451)--cycle;
\draw(-6.123,.429)--(-6.123,.399);
\draw(-6.178,.395)--(-6.178,.451);
\filldraw[fill opacity=0.8,fill=gray!20,draw=none](-6.185,.459)--(-6.178,.451)--(-6.178,.431)--cycle;
\draw(-6.178,.451)--(-6.178,.431);
\filldraw[fill opacity=0.8,fill=gray!20,draw=none](-6.178,.498)--(-6.181,.491)--(-6.183,.489)--cycle;
\draw(-6.181,.491)--(-6.183,.489);
\filldraw[fill opacity=0.8,fill=gray!20,draw=none](-6.185,.459)--(-6.184,.481)--(-6.178,.479)--(-6.178,.451)--cycle;
\draw(-6.178,.479)--(-6.178,.451);
\filldraw[fill opacity=0.8,fill=gray!20,draw=none](-6.183,.439)--(-6.197,.459)--(-6.185,.459)--cycle;
\draw(-6.197,.459)--(-6.185,.459);
\filldraw[fill opacity=0.8,fill=gray!20,draw=none](-6.178,.421)--(-6.178,.395)--(-6.223,.376)--(-6.223,.438)--cycle;
\draw(-6.178,.421)--(-6.178,.395);
\draw(-6.223,.376)--(-6.223,.438);
\filldraw[fill opacity=0.8,fill=gray!20,draw=none](-6.215,.478)--(-6.209,.484)--(-6.194,.467)--(-6.195,.459)--(-6.206,.459)--cycle;
\draw(-6.195,.459)--(-6.206,.459);
\filldraw[fill opacity=0.8,fill=gray!20,draw=none](-6.194,.467)--(-6.185,.459)--(-6.195,.459)--cycle;
\draw(-6.185,.459)--(-6.195,.459);
\filldraw[fill opacity=0.8,fill=gray!20,draw=none](-6.194,.467)--(-6.185,.458)--(-6.188,.425)--(-6.211,.433)--cycle;
\filldraw[fill opacity=0.8,fill=gray!20,draw=none](-6.209,.484)--(-6.205,.489)--(-6.192,.484)--(-6.185,.458)--cycle;
\filldraw[fill opacity=0.8,fill=gray!20,draw=none](-6.194,.467)--(-6.192,.475)--(-6.185,.459)--cycle;
\filldraw[fill opacity=0.8,fill=gray!20,draw=none](-6.215,.478)--(-6.206,.459)--(-6.231,.459)--cycle;
\draw(-6.206,.459)--(-6.231,.459);
\filldraw[fill opacity=0.8,fill=gray!20,draw=none](-6.209,.484)--(-6.194,.467)--(-6.211,.433)--(-6.223,.438)--(-6.223,.464)--cycle;
\draw(-6.223,.438)--(-6.223,.464);
\filldraw[fill opacity=0.8,fill=gray!20,draw=none](-6.205,.489)--(-6.197,.501)--(-6.192,.484)--cycle;
\filldraw[fill opacity=0.8,fill=gray!20,draw=none](-6.194,.467)--(-6.209,.484)--(-6.201,.493)--(-6.192,.475)--cycle;
\filldraw[fill opacity=0.8,fill=gray!20,draw=none](-6.262,.367)--(-6.227,.361)--(-6.264,.361)--cycle;
\draw(-6.227,.361)--(-6.264,.361);
\filldraw[fill opacity=0.8,fill=gray!20,draw=none](-6.301,.485)--(-6.289,.477)--(-6.325,.46)--(-6.335,.46)--cycle;
\draw(-6.325,.46)--(-6.335,.46);
\filldraw[fill opacity=0.8,fill=gray!20,draw=none](-6.289,.477)--(-6.264,.46)--(-6.325,.46)--cycle;
\draw(-6.264,.46)--(-6.325,.46);
\filldraw[fill opacity=0.8,fill=gray!20,draw=none](-6.274,.485)--(-6.261,.452)--(-6.317,.452)--(-6.316,.459)--(-6.297,.491)--cycle;
\draw(-6.317,.452)--(-6.316,.459)--(-6.297,.491);
\filldraw[fill opacity=0.8,fill=gray!20,draw=none](-6.289,.477)--(-6.253,.495)--(-6.245,.495)--(-6.241,.459)--(-6.264,.46)--cycle;
\draw(-6.253,.495)--(-6.245,.495);
\draw(-6.241,.459)--(-6.264,.46);
\filldraw[fill opacity=0.8,fill=gray!20,draw=none](-6.245,.495)--(-6.22,.495)--(-6.209,.484)--(-6.231,.459)--(-6.241,.459)--cycle;
\draw(-6.245,.495)--(-6.22,.495);
\draw(-6.231,.459)--(-6.241,.459);
\filldraw[fill opacity=0.8,fill=gray!20,draw=none](-6.223,.464)--(-6.223,.376)--(-6.252,.325)--(-6.252,.421)--cycle;
\draw(-6.223,.464)--(-6.223,.376);
\draw(-6.252,.325)--(-6.252,.421);
\filldraw[fill opacity=0.8,fill=gray!20,draw=none](-6.185,.458)--(-6.188,.425)--(-6.216,.382)--(-6.229,.419)--(-6.192,.475)--cycle;
\draw(-6.188,.425)--(-6.216,.382)--(-6.229,.419)--(-6.192,.475);
\filldraw[fill opacity=0.8,fill=gray!20,draw=none](-6.184,.481)--(-6.181,.523)--(-6.178,.526)--(-6.178,.479)--cycle;
\draw(-6.178,.526)--(-6.178,.479);
\filldraw[fill opacity=0.8,fill=gray!20,draw=none](-6.185,.459)--(-6.197,.501)--(-6.181,.523)--cycle;
\filldraw[fill opacity=0.8,fill=gray!20,draw=none](-6.156,.55)--(-6.178,.498)--(-6.183,.489)--(-6.187,.482)--(-6.17,.534)--(-6.159,.549)--cycle;
\draw(-6.183,.489)--(-6.187,.482);
\draw(-6.17,.534)--(-6.159,.549);
\filldraw[fill opacity=0.8,fill=gray!20,draw=none](-6.102,.537)--(-6.123,.546)--(-6.156,.55)--(-6.159,.549)--(-6.159,.547)--cycle;
\draw(-6.102,.537)--(-6.123,.546);
\filldraw[fill opacity=0.8,fill=gray!20,draw=none](-6.185,.458)--(-6.192,.475)--(-6.187,.482)--cycle;
\draw(-6.192,.475)--(-6.187,.482);
\filldraw[fill opacity=0.8,fill=gray!20,draw=none](-6.185,.459)--(-6.187,.482)--(-6.183,.489)--cycle;
\draw(-6.187,.482)--(-6.183,.489);
\filldraw[fill opacity=0.8,fill=gray!20,draw=none](-6.192,.475)--(-6.201,.493)--(-6.2,.494)--(-6.189,.494)--cycle;
\draw(-6.2,.494)--(-6.189,.494);
\filldraw[fill opacity=0.8,fill=gray!20,draw=none](-6.19,.503)--(-6.189,.494)--(-6.2,.494)--cycle;
\draw(-6.189,.494)--(-6.2,.494);
\filldraw[fill opacity=0.8,fill=gray!20,draw=none](-6.187,.484)--(-6.187,.482)--(-6.229,.419)--(-6.225,.45)--(-6.19,.503)--cycle;
\draw(-6.187,.482)--(-6.229,.419)--(-6.225,.45)--(-6.19,.503);
\filldraw[fill opacity=0.8,fill=gray!20,draw=none](-6.176,.514)--(-6.187,.484)--(-6.19,.503)--(-6.183,.514)--cycle;
\draw(-6.19,.503)--(-6.183,.514);
\filldraw[fill opacity=0.8,fill=gray!20,draw=none](-6.192,.475)--(-6.189,.494)--(-6.108,.493)--(-6.107,.458)--(-6.185,.459)--cycle;
\draw(-6.189,.494)--(-6.108,.493)--(-6.107,.458)--(-6.185,.459);
\filldraw[fill opacity=0.8,fill=gray!20,draw=none](-6.19,.503)--(-6.179,.513)--(-6.11,.512)--(-6.108,.493)--(-6.189,.494)--cycle;
\draw(-6.179,.513)--(-6.11,.512)--(-6.108,.493)--(-6.189,.494);
\filldraw[fill opacity=0.8,fill=gray!20,draw=none](-6.082,.528)--(-6.102,.537)--(-6.159,.547)--(-6.158,.545)--(-6.142,.538)--cycle;
\draw(-6.082,.528)--(-6.102,.537);
\draw(-6.158,.545)--(-6.142,.538);
\filldraw[fill opacity=0.8,fill=gray!20,draw=none](-6.123,.282)--(-6.123,.26)--(-6.178,.268)--cycle;
\draw(-6.123,.282)--(-6.123,.26);
\filldraw[fill opacity=0.8,fill=gray!20,draw=none](-6.177,.512)--(-6.159,.511)--(-6.113,.511)--(-6.11,.512)--(-6.179,.513)--cycle;
\draw(-6.159,.511)--(-6.113,.511)--(-6.11,.512)--(-6.179,.513);
\filldraw[fill opacity=0.8,fill=gray!20,draw=none](-6.164,.531)--(-6.143,.518)--(-6.114,.52)--(-6.113,.525)--(-6.158,.545)--cycle;
\draw(-6.113,.525)--(-6.158,.545);
\filldraw[fill opacity=0.8,fill=gray!20,draw=none](-6.155,.5)--(-6.126,.49)--(-6.115,.49)--(-6.113,.511)--(-6.159,.511)--cycle;
\draw(-6.126,.49)--(-6.115,.49)--(-6.113,.511)--(-6.159,.511);
\filldraw[fill opacity=0.8,fill=gray!20,draw=none](-6.038,.72)--(-6.204,.471)--(-6.169,.478)--(-5.953,.802)--cycle;
\draw(-6.038,.72)--(-6.204,.471)--(-6.169,.478)--(-5.953,.802);
\filldraw[fill opacity=0.8,fill=gray!20](-7.482,4.808)--(-7.442,4.79)--(-7.418,4.78)--(-7.414,4.778)--(-7.43,4.785)--(-7.464,4.8)--(-7.51,4.82)--(-7.563,4.843)--(-7.612,4.865)--(-7.652,4.882)--(-7.676,4.893)--(-7.68,4.894)--(-7.664,4.887)--(-7.63,4.873)--(-7.583,4.852)--(-7.531,4.829)--cycle;
\filldraw[fill opacity=0.8,fill=gray!20,draw=none](-8.03,3.672)--(-8.026,3.67)--(-8.028,3.665)--(-8.041,3.674)--cycle;
\draw(-8.026,3.67)--(-8.028,3.665);
\filldraw[fill opacity=0.8,fill=gray!20,draw=none](-8.03,3.672)--(-8.025,3.671)--(-8.026,3.67)--cycle;
\draw(-8.025,3.671)--(-8.026,3.67);
\filldraw[fill opacity=0.8,fill=gray!20,draw=none](-7.984,3.651)--(-7.987,3.645)--(-8.026,3.67)--(-8.025,3.671)--cycle;
\draw(-7.984,3.651)--(-7.987,3.645);
\draw(-8.026,3.67)--(-8.025,3.671);
\filldraw[fill opacity=0.8,fill=gray!20](-8.013,3.649)--(-8.036,3.68)--(-7.981,3.683)--(-7.984,3.65)--cycle;
\filldraw[fill opacity=0.5,fill=gray!20](-8.618,-.608)--(-8.57,-.798)--(-9,-.906)--(-9,-.705)--cycle;
\filldraw[fill opacity=0.8,fill=gray!20,draw=none](-7.951,.417)--(-7.947,.461)--(-7.938,.467)--(-7.943,.422)--cycle;
\draw(-7.947,.461)--(-7.938,.467)--(-7.943,.422)--(-7.951,.417);
\filldraw[fill opacity=0.8,fill=gray!20,draw=none](-7.958,.387)--(-7.96,.411)--(-7.943,.422)--(-7.942,.41)--cycle;
\draw(-7.96,.411)--(-7.943,.422)--(-7.942,.41);
\filldraw[fill opacity=0.8,fill=gray!20,draw=none](-7.947,.461)--(-7.945,.466)--(-7.925,.502)--(-7.938,.467)--cycle;
\draw(-7.925,.502)--(-7.938,.467)--(-7.947,.461);
\filldraw[fill opacity=0.8,fill=gray!20,draw=none](-7.949,.438)--(-7.951,.417)--(-7.96,.411)--(-7.959,.417)--cycle;
\draw(-7.951,.417)--(-7.96,.411);
\filldraw[fill opacity=0.8,fill=gray!20,draw=none](-7.949,.438)--(-7.959,.417)--(-7.948,.461)--(-7.947,.461)--cycle;
\draw(-7.948,.461)--(-7.947,.461);
\filldraw[fill opacity=0.8,fill=gray!20,draw=none](-7.945,.466)--(-7.947,.461)--(-7.948,.461)--cycle;
\draw(-7.947,.461)--(-7.948,.461);
\filldraw[fill opacity=0.8,fill=gray!20,draw=none](-7.945,.466)--(-7.929,.504)--(-7.923,.508)--(-7.925,.502)--cycle;
\draw(-7.929,.504)--(-7.923,.508)--(-7.925,.502);
\filldraw[fill opacity=0.8,fill=gray!20,draw=none](-7.927,.505)--(-7.923,.511)--(-7.901,.538)--(-7.923,.508)--cycle;
\draw(-7.901,.538)--(-7.923,.508)--(-7.927,.505);
\filldraw[fill opacity=0.8,fill=gray!20,draw=none](-7.927,.505)--(-7.929,.504)--(-7.923,.511)--cycle;
\draw(-7.927,.505)--(-7.929,.504);
\filldraw[fill opacity=0.8,fill=gray!20](-7.869,1.103)--(-7.872,1.15)--(-7.779,1.154)--(-7.78,1.107)--cycle;
\filldraw[fill opacity=0.8,fill=gray!20](-7.872,1.15)--(-7.869,1.195)--(-7.78,1.199)--(-7.779,1.154)--cycle;
\filldraw[fill opacity=0.8,fill=gray!20](-7.938,1.089)--(-7.943,1.136)--(-7.872,1.15)--(-7.869,1.103)--cycle;
\filldraw[fill opacity=0.8,fill=gray!20](-7.943,1.136)--(-7.938,1.182)--(-7.869,1.195)--(-7.872,1.15)--cycle;
\filldraw[fill opacity=0.8,fill=gray!20,draw=none](-9.056,.881)--(-9.08,.891)--(-9.052,.94)--(-9.028,.93)--cycle;
\draw(-9.052,.94)--(-9.028,.93)--(-9.056,.881)--(-9.08,.891);
\filldraw[fill opacity=0.8,fill=gray!20,draw=none](-9.028,.93)--(-9.052,.94)--(-9.042,.996)--(-9.018,.986)--cycle;
\draw(-9.042,.996)--(-9.018,.986)--(-9.028,.93)--(-9.052,.94);
\filldraw[fill opacity=0.8,fill=gray!20,draw=none](-9.099,.845)--(-9.123,.855)--(-9.08,.891)--(-9.056,.881)--cycle;
\draw(-9.08,.891)--(-9.056,.881)--(-9.099,.845)--(-9.123,.855);
\filldraw[fill opacity=0.8,fill=gray!20,draw=none](-9.04,.934)--(-9.064,.945)--(-9.055,.995)--(-9.031,.984)--cycle;
\draw(-9.04,.934)--(-9.064,.945);
\draw(-9.055,.995)--(-9.031,.984);
\filldraw[fill opacity=0.8,fill=gray!20,draw=none](-9.065,.89)--(-9.09,.9)--(-9.064,.945)--(-9.04,.934)--cycle;
\draw(-9.065,.89)--(-9.09,.9);
\draw(-9.064,.945)--(-9.04,.934);
\filldraw[fill opacity=0.8,fill=gray!20,draw=none](-9.018,.986)--(-9.042,.996)--(-9.052,1.05)--(-9.028,1.039)--cycle;
\draw(-9.052,1.05)--(-9.028,1.039)--(-9.018,.986)--(-9.042,.996);
\filldraw[fill opacity=0.8,fill=gray!20,draw=none](-9.149,.829)--(-9.173,.839)--(-9.123,.855)--(-9.099,.845)--cycle;
\draw(-9.123,.855)--(-9.099,.845)--(-9.149,.829)--(-9.173,.839);
\filldraw[fill opacity=0.8,fill=gray!20,draw=none](-9.104,.858)--(-9.128,.868)--(-9.09,.9)--(-9.065,.89)--cycle;
\draw(-9.104,.858)--(-9.128,.868);
\draw(-9.09,.9)--(-9.065,.89);
\filldraw[fill opacity=0.8,fill=gray!20,draw=none](-9.072,.947)--(-9.065,.994)--(-9.078,.985)--(-9.083,.94)--cycle;
\draw(-9.065,.994)--(-9.078,.985)--(-9.083,.94)--(-9.072,.947);
\filldraw[fill opacity=0.8,fill=gray!20,draw=none](-9.072,.947)--(-9.065,.994)--(-9.078,.985)--(-9.083,.94)--cycle;
\draw(-9.065,.994)--(-9.078,.985)--(-9.083,.94)--(-9.072,.947);
\filldraw[fill opacity=0.8,fill=gray!20,draw=none](-9.095,.901)--(-9.072,.947)--(-9.083,.94)--(-9.098,.899)--cycle;
\draw(-9.072,.947)--(-9.083,.94)--(-9.098,.899)--(-9.095,.901);
\filldraw[fill opacity=0.8,fill=gray!20,draw=none](-9.095,.901)--(-9.072,.947)--(-9.083,.94)--(-9.098,.899)--cycle;
\draw(-9.072,.947)--(-9.083,.94)--(-9.098,.899)--(-9.095,.901);
\filldraw[fill opacity=0.8,fill=gray!20,draw=none](-9.095,.901)--(-9.098,.899)--(-9.106,.888)--cycle;
\draw(-9.095,.901)--(-9.098,.899)--(-9.106,.888);
\filldraw[fill opacity=0.8,fill=gray!20,draw=none](-9.095,.901)--(-9.098,.899)--(-9.106,.888)--cycle;
\draw(-9.095,.901)--(-9.098,.899)--(-9.106,.888);
\filldraw[fill opacity=0.8,fill=gray!20,draw=none](-9.031,.984)--(-9.055,.995)--(-9.064,1.043)--(-9.04,1.033)--cycle;
\draw(-9.031,.984)--(-9.055,.995);
\draw(-9.064,1.043)--(-9.04,1.033);
\filldraw[fill opacity=0.8,fill=gray!20,draw=none](-9.106,.888)--(-9.098,.899)--(-9.16,.887)--(-9.173,.856)--cycle;
\draw(-9.106,.888)--(-9.098,.899)--(-9.16,.887)--(-9.173,.856);
\filldraw[fill opacity=0.8,fill=gray!20,draw=none](-9.106,.888)--(-9.098,.899)--(-9.16,.887)--(-9.173,.856)--cycle;
\draw(-9.106,.888)--(-9.098,.899)--(-9.16,.887)--(-9.173,.856);
\filldraw[fill opacity=0.8,fill=gray!20,draw=none](-8.981,1.18)--(-8.87,1.305)--(-8.877,1.314)--(-8.987,1.19)--cycle;
\draw(-8.87,1.305)--(-8.877,1.314)--(-8.987,1.19);
\filldraw[fill opacity=0.8,fill=gray!20,draw=none](-8.981,1.18)--(-8.87,1.305)--(-8.877,1.314)--(-8.987,1.19)--cycle;
\draw(-8.87,1.305)--(-8.877,1.314)--(-8.987,1.19);
\filldraw[fill opacity=0.8,fill=gray!20,draw=none](-8.981,1.18)--(-8.87,1.305)--(-8.877,1.314)--(-8.987,1.19)--cycle;
\draw(-8.87,1.305)--(-8.877,1.314)--(-8.987,1.19);
\filldraw[fill opacity=0.8,fill=gray!20,draw=none](-8.953,1.154)--(-8.945,1.154)--(-8.946,1.163)--(-8.994,1.194)--(-8.995,1.194)--cycle;
\draw(-8.953,1.154)--(-8.945,1.154)--(-8.946,1.163);
\draw(-8.994,1.194)--(-8.995,1.194);
\filldraw[fill opacity=0.8,fill=gray!20,draw=none](-8.985,1.175)--(-8.981,1.18)--(-8.987,1.19)--(-8.988,1.188)--cycle;
\draw(-8.987,1.19)--(-8.988,1.188);
\filldraw[fill opacity=0.8,fill=gray!20,draw=none](-9.065,.994)--(-9.072,1.039)--(-9.083,1.032)--(-9.078,.985)--cycle;
\draw(-9.072,1.039)--(-9.083,1.032)--(-9.078,.985)--(-9.065,.994);
\filldraw[fill opacity=0.8,fill=gray!20,draw=none](-9.065,.994)--(-9.072,1.039)--(-9.083,1.032)--(-9.078,.985)--cycle;
\draw(-9.072,1.039)--(-9.083,1.032)--(-9.078,.985)--(-9.065,.994);
\filldraw[fill opacity=0.8,fill=gray!20,draw=none](-9.149,.843)--(-9.173,.853)--(-9.128,.868)--(-9.104,.858)--cycle;
\draw(-9.149,.843)--(-9.173,.853);
\draw(-9.128,.868)--(-9.104,.858);
\filldraw[fill opacity=0.8,fill=gray!20](-9.098,.899)--(-9.083,.94)--(-9.152,.926)--(-9.16,.887)--cycle;
\filldraw[fill opacity=0.8,fill=gray!20](-9.098,.899)--(-9.083,.94)--(-9.152,.926)--(-9.16,.887)--cycle;
\filldraw[fill opacity=0.8,fill=gray!20,draw=none](-9.037,.915)--(-9.028,.93)--(-9.025,.944)--(-9.061,.959)--cycle;
\draw(-9.037,.915)--(-9.028,.93)--(-9.025,.944);
\filldraw[fill opacity=0.8,fill=gray!20](-9.083,.94)--(-9.078,.985)--(-9.149,.971)--(-9.152,.926)--cycle;
\filldraw[fill opacity=0.8,fill=gray!20](-9.083,.94)--(-9.078,.985)--(-9.149,.971)--(-9.152,.926)--cycle;
\filldraw[fill opacity=0.8,fill=gray!20,draw=none](-9.085,1.062)--(-9.092,1.072)--(-9.172,.981)--(-9.165,.973)--cycle;
\draw(-9.092,1.072)--(-9.172,.981)--(-9.165,.973);
\filldraw[fill opacity=0.8,fill=gray!20,draw=none](-9.085,1.062)--(-9.092,1.072)--(-9.172,.981)--(-9.165,.973)--cycle;
\draw(-9.092,1.072)--(-9.172,.981)--(-9.165,.973);
\filldraw[fill opacity=0.8,fill=gray!20,draw=none](-9.085,1.062)--(-9.092,1.072)--(-9.172,.981)--(-9.165,.973)--cycle;
\draw(-9.092,1.072)--(-9.172,.981)--(-9.165,.973);
\filldraw[fill opacity=0.8,fill=gray!20,draw=none](-9.104,.858)--(-9.071,.868)--(-9.056,.881)--(-9.038,.913)--cycle;
\draw(-9.071,.868)--(-9.056,.881)--(-9.038,.913);
\filldraw[fill opacity=0.8,fill=gray!20,draw=none](-9.104,.858)--(-9.124,.84)--(-9.116,.839)--(-9.099,.845)--(-9.071,.868)--cycle;
\draw(-9.116,.839)--(-9.099,.845)--(-9.071,.868);
\filldraw[fill opacity=0.8,fill=gray!20,draw=none](-8.949,.79)--(-9.104,.858)--(-9.065,.89)--(-8.899,.817)--cycle;
\draw(-8.949,.79)--(-9.104,.858);
\draw(-9.065,.89)--(-8.899,.817);
\filldraw[fill opacity=0.8,fill=gray!20,draw=none](-8.993,.858)--(-9.006,.863)--(-9.013,.922)--(-8.941,.891)--cycle;
\draw(-8.993,.858)--(-9.006,.863);
\draw(-9.013,.922)--(-8.941,.891);
\filldraw[fill opacity=0.8,fill=gray!20,draw=none](-8.953,.841)--(-8.993,.858)--(-8.941,.891)--(-8.893,.87)--cycle;
\draw(-8.953,.841)--(-8.993,.858);
\draw(-8.941,.891)--(-8.893,.87);
\filldraw[fill opacity=0.8,fill=gray!20,draw=none](-8.885,.843)--(-8.905,.82)--(-8.953,.841)--(-8.893,.87)--(-8.876,.863)--cycle;
\draw(-8.905,.82)--(-8.953,.841);
\draw(-8.893,.87)--(-8.876,.863);
\filldraw[fill opacity=0.8,fill=gray!20,draw=none](-8.885,.843)--(-8.891,.829)--(-8.902,.818)--(-8.905,.82)--cycle;
\draw(-8.902,.818)--(-8.905,.82);
\filldraw[fill opacity=0.8,fill=gray!20,draw=none](-8.919,.797)--(-8.935,.784)--(-8.949,.79)--(-8.901,.815)--cycle;
\draw(-8.935,.784)--(-8.949,.79);
\filldraw[fill opacity=0.8,fill=gray!20,draw=none](-8.875,.854)--(-8.885,.843)--(-8.876,.863)--(-8.871,.86)--cycle;
\draw(-8.876,.863)--(-8.871,.86);
\filldraw[fill opacity=0.8,fill=gray!20,draw=none](-8.897,.816)--(-8.902,.818)--(-8.891,.829)--cycle;
\draw(-8.897,.816)--(-8.902,.818);
\filldraw[fill opacity=0.8,fill=gray!20,draw=none](-8.936,.853)--(-8.891,.855)--(-8.891,.816)--cycle;
\draw(-8.936,.853)--(-8.891,.855)--(-8.891,.816);
\filldraw[fill opacity=0.8,fill=gray!20,draw=none](-9.066,1.084)--(-8.981,1.18)--(-8.987,1.19)--(-9.075,1.09)--cycle;
\draw(-8.987,1.19)--(-9.075,1.09);
\filldraw[fill opacity=0.8,fill=gray!20,draw=none](-9.066,1.084)--(-8.981,1.18)--(-8.987,1.19)--(-9.075,1.09)--cycle;
\draw(-8.987,1.19)--(-9.075,1.09);
\filldraw[fill opacity=0.8,fill=gray!20,draw=none](-9.066,1.084)--(-8.985,1.175)--(-8.988,1.188)--(-9.075,1.09)--cycle;
\draw(-8.988,1.188)--(-9.075,1.09);
\filldraw[fill opacity=0.8,fill=gray!20,draw=none](-9.077,1.089)--(-9.071,1.079)--(-9.08,1.093)--(-9.08,1.093)--cycle;
\draw(-9.08,1.093)--(-9.08,1.093);
\filldraw[fill opacity=0.8,fill=gray!20,draw=none](-9.08,1.068)--(-9.076,1.073)--(-9.085,1.079)--(-9.086,1.078)--cycle;
\draw(-9.085,1.079)--(-9.086,1.078);
\filldraw[fill opacity=0.8,fill=gray!20,draw=none](-9.08,1.068)--(-9.076,1.073)--(-9.085,1.079)--(-9.086,1.078)--cycle;
\draw(-9.085,1.079)--(-9.086,1.078);
\filldraw[fill opacity=0.8,fill=gray!20,draw=none](-9.08,1.068)--(-9.076,1.073)--(-9.085,1.079)--(-9.086,1.078)--cycle;
\draw(-9.085,1.079)--(-9.086,1.078);
\filldraw[fill opacity=0.8,fill=gray!20,draw=none](-9.059,1.041)--(-9.064,1.043)--(-9.08,1.068)--(-9.086,1.078)--cycle;
\draw(-9.059,1.041)--(-9.064,1.043);
\filldraw[fill opacity=0.8,fill=gray!20,draw=none](-9.095,1.079)--(-9.099,1.084)--(-9.181,.991)--(-9.172,.981)--(-9.092,1.072)--cycle;
\draw(-9.181,.991)--(-9.172,.981)--(-9.092,1.072);
\filldraw[fill opacity=0.8,fill=gray!20,draw=none](-9.071,1.079)--(-9.066,1.084)--(-9.075,1.09)--(-9.077,1.089)--cycle;
\draw(-9.075,1.09)--(-9.077,1.089);
\filldraw[fill opacity=0.8,fill=gray!20,draw=none](-9.076,1.073)--(-9.066,1.084)--(-9.075,1.09)--(-9.085,1.079)--cycle;
\draw(-9.075,1.09)--(-9.085,1.079);
\filldraw[fill opacity=0.8,fill=gray!20,draw=none](-9.071,1.079)--(-9.066,1.084)--(-9.075,1.09)--(-9.077,1.089)--cycle;
\draw(-9.075,1.09)--(-9.077,1.089);
\filldraw[fill opacity=0.8,fill=gray!20,draw=none](-9.08,1.093)--(-9.077,1.089)--(-9.075,1.09)--cycle;
\draw(-9.077,1.089)--(-9.075,1.09);
\filldraw[fill opacity=0.8,fill=gray!20,draw=none](-9.08,1.093)--(-9.077,1.089)--(-9.075,1.09)--cycle;
\draw(-9.077,1.089)--(-9.075,1.09);
\filldraw[fill opacity=0.8,fill=gray!20,draw=none](-9.08,1.093)--(-9.077,1.089)--(-9.075,1.09)--cycle;
\draw(-9.077,1.089)--(-9.075,1.09);
\filldraw[fill opacity=0.8,fill=gray!20,draw=none](-9.028,1.039)--(-9.046,1.047)--(-9.077,1.089)--(-9.079,1.092)--(-9.056,1.083)--cycle;
\draw(-9.079,1.092)--(-9.056,1.083)--(-9.028,1.039)--(-9.046,1.047);
\filldraw[fill opacity=0.8,fill=gray!20,draw=none](-9.04,1.033)--(-9.024,.949)--(-9.018,.986)--(-9.021,1.004)--cycle;
\draw(-9.024,.949)--(-9.018,.986)--(-9.021,1.004);
\filldraw[fill opacity=0.8,fill=gray!20,draw=none](-8.865,.898)--(-8.869,.891)--(-8.893,.87)--(-9.013,.922)--(-9.031,.982)--(-9.031,.984)--(-8.861,.91)--cycle;
\draw(-8.893,.87)--(-9.013,.922);
\draw(-9.031,.984)--(-8.861,.91);
\filldraw[fill opacity=0.8,fill=gray!20,draw=none](-8.869,.891)--(-8.883,.866)--(-8.893,.87)--cycle;
\draw(-8.883,.866)--(-8.893,.87);
\filldraw[fill opacity=0.8,fill=gray!20,draw=none](-8.87,.86)--(-8.883,.866)--(-8.862,.904)--cycle;
\draw(-8.87,.86)--(-8.883,.866);
\filldraw[fill opacity=0.8,fill=gray!20,draw=none](-8.885,.843)--(-8.875,.854)--(-8.891,.829)--cycle;
\filldraw[fill opacity=0.8,fill=gray!20,draw=none](-8.847,.812)--(-8.86,.796)--(-8.892,.799)--(-8.891,.855)--(-8.858,.853)--cycle;
\draw(-8.86,.796)--(-8.892,.799)--(-8.891,.855)--(-8.858,.853);
\filldraw[fill opacity=0.8,fill=gray!20,draw=none](-8.87,.86)--(-8.871,.854)--(-8.875,.854)--cycle;
\draw(-8.871,.854)--(-8.875,.854);
\filldraw[fill opacity=0.8,fill=gray!20,draw=none](-8.826,.863)--(-8.844,.852)--(-8.871,.854)--(-8.87,.86)--(-8.83,.908)--cycle;
\draw(-8.844,.852)--(-8.871,.854);
\filldraw[fill opacity=0.8,fill=gray!20,draw=none](-8.868,.851)--(-8.891,.829)--(-8.871,.86)--(-8.866,.858)--cycle;
\draw(-8.871,.86)--(-8.866,.858);
\filldraw[fill opacity=0.8,fill=gray!20,draw=none](-8.876,.89)--(-8.868,.873)--(-8.87,.86)--(-8.875,.854)--(-8.891,.855)--(-8.89,.877)--cycle;
\draw(-8.875,.854)--(-8.891,.855)--(-8.89,.877);
\filldraw[fill opacity=0.8,fill=gray!20](-8.998,.85)--(-9.001,.907)--(-8.89,.912)--(-8.891,.855)--cycle;
\filldraw[fill opacity=0.8,fill=gray!20,draw=none](-8.872,.895)--(-8.89,.877)--(-8.89,.893)--cycle;
\draw(-8.89,.877)--(-8.89,.893);
\filldraw[fill opacity=0.8,fill=gray!20,draw=none](-8.87,.86)--(-8.861,.91)--(-8.83,.908)--cycle;
\draw(-8.861,.91)--(-8.83,.908);
\filldraw[fill opacity=0.8,fill=gray!20,draw=none](-8.865,.898)--(-8.861,.91)--(-8.859,.909)--cycle;
\draw(-8.861,.91)--(-8.859,.909);
\filldraw[fill opacity=0.8,fill=gray!20,draw=none](-8.863,.896)--(-8.89,.893)--(-8.89,.912)--(-8.861,.91)--cycle;
\draw(-8.89,.893)--(-8.89,.912)--(-8.861,.91);
\filldraw[fill opacity=0.8,fill=gray!20,draw=none](-9.199,.834)--(-9.223,.845)--(-9.173,.839)--(-9.149,.829)--cycle;
\draw(-9.173,.839)--(-9.149,.829)--(-9.199,.834)--(-9.223,.845);
\filldraw[fill opacity=0.8,fill=gray!20,draw=none](-9.149,.843)--(-9.124,.84)--(-9.104,.858)--cycle;
\filldraw[fill opacity=0.8,fill=gray!20,draw=none](-9.124,.84)--(-9.132,.834)--(-9.116,.839)--cycle;
\draw(-9.132,.834)--(-9.116,.839);
\filldraw[fill opacity=0.8,fill=gray!20,draw=none](-9.149,.843)--(-9.173,.835)--(-9.166,.83)--(-9.149,.829)--(-9.132,.834)--(-9.124,.84)--cycle;
\draw(-9.166,.83)--(-9.149,.829)--(-9.132,.834);
\filldraw[fill opacity=0.8,fill=gray!20,draw=none](-8.981,.77)--(-9.149,.843)--(-9.104,.858)--(-8.945,.788)--cycle;
\draw(-8.981,.77)--(-9.149,.843);
\draw(-9.104,.858)--(-8.945,.788);
\filldraw[fill opacity=0.8,fill=gray!20,draw=none](-8.919,.797)--(-8.901,.815)--(-8.899,.817)--(-8.897,.816)--cycle;
\draw(-8.899,.817)--(-8.897,.816);
\filldraw[fill opacity=0.8,fill=gray!20,draw=none](-8.988,.794)--(-8.998,.85)--(-8.936,.853)--(-8.891,.816)--(-8.892,.799)--cycle;
\draw(-8.891,.816)--(-8.892,.799)--(-8.988,.794)--(-8.998,.85)--(-8.936,.853);
\filldraw[fill opacity=0.8,fill=gray!20,draw=none](-8.888,.812)--(-8.897,.816)--(-8.891,.829)--(-8.862,.856)--(-8.859,.855)--cycle;
\draw(-8.888,.812)--(-8.897,.816);
\draw(-8.862,.856)--(-8.859,.855);
\filldraw[fill opacity=0.8,fill=gray!20,draw=none](-9.045,1.059)--(-9.04,1.033)--(-9.021,1.004)--(-9.028,1.039)--(-9.038,1.054)--cycle;
\draw(-9.021,1.004)--(-9.028,1.039)--(-9.038,1.054);
\filldraw[fill opacity=0.8,fill=gray!20,draw=none](-8.899,.927)--(-9.031,.984)--(-9.04,1.033)--(-8.873,.96)--cycle;
\draw(-8.899,.927)--(-9.031,.984);
\draw(-9.04,1.033)--(-8.873,.96);
\filldraw[fill opacity=0.8,fill=gray!20,draw=none](-8.87,.958)--(-8.828,.947)--(-8.856,.91)--(-8.861,.91)--cycle;
\draw(-8.856,.91)--(-8.861,.91);
\filldraw[fill opacity=0.8,fill=gray!20,draw=none](-8.854,.943)--(-8.859,.909)--(-8.899,.927)--(-8.873,.96)--(-8.857,.953)--cycle;
\draw(-8.859,.909)--(-8.899,.927);
\draw(-8.873,.96)--(-8.857,.953);
\filldraw[fill opacity=0.8,fill=gray!20,draw=none](-8.87,.958)--(-8.861,.91)--(-8.89,.912)--(-8.891,.964)--cycle;
\draw(-8.861,.91)--(-8.89,.912)--(-8.891,.964);
\filldraw[fill opacity=0.8,fill=gray!20,draw=none](-8.967,.909)--(-9.001,.917)--(-8.998,.961)--(-8.891,.966)--(-8.89,.912)--cycle;
\draw(-9.001,.917)--(-8.998,.961)--(-8.891,.966)--(-8.89,.912)--(-8.967,.909);
\filldraw[fill opacity=0.8,fill=gray!20,draw=none](-8.848,.881)--(-8.862,.856)--(-8.87,.86)--(-8.862,.904)--(-8.859,.909)--(-8.851,.906)--cycle;
\draw(-8.862,.856)--(-8.87,.86);
\draw(-8.859,.909)--(-8.851,.906);
\filldraw[fill opacity=0.8,fill=gray!20,draw=none](-8.876,.89)--(-8.872,.895)--(-8.863,.896)--(-8.868,.873)--cycle;
\filldraw[fill opacity=0.8,fill=gray!20,draw=none](-8.829,.834)--(-8.847,.812)--(-8.858,.853)--(-8.825,.85)--cycle;
\draw(-8.858,.853)--(-8.825,.85);
\filldraw[fill opacity=0.8,fill=gray!20,draw=none](-8.868,.851)--(-8.866,.858)--(-8.862,.856)--cycle;
\draw(-8.866,.858)--(-8.862,.856);
\filldraw[fill opacity=0.8,fill=gray!20,draw=none](-8.826,.863)--(-8.825,.85)--(-8.844,.852)--cycle;
\draw(-8.825,.85)--(-8.844,.852);
\filldraw[fill opacity=0.8,fill=gray!20,draw=none](-8.829,.834)--(-8.825,.85)--(-8.816,.85)--cycle;
\draw(-8.825,.85)--(-8.816,.85);
\filldraw[fill opacity=0.8,fill=gray!20,draw=none](-8.806,.876)--(-8.804,.874)--(-8.816,.85)--(-8.825,.85)--(-8.826,.863)--cycle;
\draw(-8.816,.85)--(-8.825,.85);
\filldraw[fill opacity=0.8,fill=gray!20,draw=none](-8.806,.876)--(-8.826,.863)--(-8.83,.908)--(-8.829,.908)--cycle;
\draw(-8.83,.908)--(-8.829,.908);
\filldraw[fill opacity=0.8,fill=gray!20,draw=none](-8.848,.881)--(-8.847,.872)--(-8.853,.852)--(-8.862,.856)--cycle;
\draw(-8.853,.852)--(-8.862,.856);
\filldraw[fill opacity=0.8,fill=gray!20,draw=none](-8.804,.877)--(-8.806,.876)--(-8.829,.908)--(-8.807,.906)--cycle;
\draw(-8.829,.908)--(-8.807,.906);
\filldraw[fill opacity=0.8,fill=gray!20,draw=none](-8.857,.846)--(-8.883,.81)--(-8.888,.812)--(-8.859,.855)--(-8.854,.853)--cycle;
\draw(-8.883,.81)--(-8.888,.812);
\draw(-8.859,.855)--(-8.854,.853);
\filldraw[fill opacity=0.8,fill=gray!20,draw=none](-8.828,.947)--(-8.825,.946)--(-8.815,.923)--(-8.816,.907)--(-8.856,.91)--cycle;
\draw(-8.816,.907)--(-8.856,.91);
\filldraw[fill opacity=0.8,fill=gray!20,draw=none](-8.804,.877)--(-8.806,.901)--(-8.804,.901)--(-8.793,.876)--(-8.796,.875)--cycle;
\draw(-8.793,.876)--(-8.796,.875);
\filldraw[fill opacity=0.8,fill=gray!20,draw=none](-8.804,.876)--(-8.804,.877)--(-8.796,.875)--cycle;
\filldraw[fill opacity=0.8,fill=gray!20,draw=none](-8.806,.876)--(-8.804,.877)--(-8.804,.876)--(-8.804,.874)--cycle;
\filldraw[fill opacity=0.8,fill=gray!20,draw=none](-8.804,.876)--(-8.845,.879)--(-8.844,.888)--(-8.804,.877)--cycle;
\filldraw[fill opacity=0.8,fill=gray!20,draw=none](-8.804,.877)--(-8.803,.878)--(-8.804,.876)--cycle;
\filldraw[fill opacity=0.8,fill=gray!20,draw=none](-8.804,.877)--(-8.839,.887)--(-8.843,.902)--(-8.842,.913)--(-8.806,.901)--cycle;
\filldraw[fill opacity=0.8,fill=gray!20,draw=none](-8.804,.901)--(-8.807,.906)--(-8.809,.924)--(-8.784,.93)--(-8.699,.897)--(-8.756,.884)--cycle;
\draw(-8.809,.924)--(-8.784,.93);
\draw(-8.699,.897)--(-8.756,.884);
\filldraw[fill opacity=0.8,fill=gray!20,draw=none](-8.804,.901)--(-8.756,.884)--(-8.793,.876)--cycle;
\draw(-8.756,.884)--(-8.793,.876);
\filldraw[fill opacity=0.8,fill=gray!20,draw=none](-8.804,.901)--(-8.806,.901)--(-8.807,.906)--cycle;
\filldraw[fill opacity=0.8,fill=gray!20,draw=none](-8.803,.878)--(-8.804,.877)--(-8.807,.906)--(-8.79,.905)--cycle;
\draw(-8.807,.906)--(-8.79,.905);
\filldraw[fill opacity=0.8,fill=gray!20,draw=none](-8.844,.888)--(-8.848,.881)--(-8.851,.906)--(-8.843,.902)--cycle;
\draw(-8.851,.906)--(-8.843,.902);
\filldraw[fill opacity=0.8,fill=gray!20,draw=none](-8.857,.846)--(-8.854,.853)--(-8.853,.852)--cycle;
\draw(-8.854,.853)--(-8.853,.852);
\filldraw[fill opacity=0.8,fill=gray!20,draw=none](-8.848,.881)--(-8.844,.888)--(-8.845,.879)--(-8.847,.872)--cycle;
\filldraw[fill opacity=0.8,fill=gray!20,draw=none](-8.847,.915)--(-8.851,.906)--(-8.859,.909)--(-8.853,.951)--(-8.85,.95)--cycle;
\draw(-8.851,.906)--(-8.859,.909);
\draw(-8.853,.951)--(-8.85,.95);
\filldraw[fill opacity=0.8,fill=gray!20,draw=none](-8.847,.915)--(-8.846,.903)--(-8.851,.906)--cycle;
\draw(-8.846,.903)--(-8.851,.906);
\filldraw[fill opacity=0.8,fill=gray!20,draw=none](-8.843,.902)--(-8.839,.887)--(-8.844,.888)--cycle;
\filldraw[fill opacity=0.8,fill=gray!20,draw=none](-8.847,.915)--(-8.847,.915)--(-8.843,.902)--(-8.846,.903)--cycle;
\draw(-8.843,.902)--(-8.846,.903);
\filldraw[fill opacity=0.8,fill=gray!20,draw=none](-8.843,.902)--(-8.844,.888)--(-8.902,.903)--(-8.848,.916)--(-8.847,.915)--cycle;
\draw(-8.902,.903)--(-8.848,.916);
\filldraw[fill opacity=0.8,fill=gray!20,draw=none](-8.841,.894)--(-8.844,.888)--(-8.843,.902)--(-8.839,.901)--cycle;
\draw(-8.843,.902)--(-8.839,.901);
\filldraw[fill opacity=0.8,fill=gray!20,draw=none](-8.845,.879)--(-8.962,.89)--(-8.902,.903)--(-8.844,.888)--cycle;
\draw(-8.962,.89)--(-8.902,.903);
\filldraw[fill opacity=0.8,fill=gray!20,draw=none](-8.844,.888)--(-8.841,.894)--(-8.845,.879)--cycle;
\filldraw[fill opacity=0.8,fill=gray!20,draw=none](-8.828,.947)--(-8.87,.958)--(-8.875,.965)--(-8.817,.961)--cycle;
\draw(-8.875,.965)--(-8.817,.961);
\filldraw[fill opacity=0.8,fill=gray!20,draw=none](-8.893,.939)--(-8.848,.916)--(-8.925,.898)--cycle;
\draw(-8.848,.916)--(-8.925,.898);
\filldraw[fill opacity=0.8,fill=gray!20,draw=none](-8.897,.942)--(-8.893,.939)--(-8.925,.898)--(-8.962,.89)--cycle;
\draw(-8.925,.898)--(-8.962,.89);
\filldraw[fill opacity=0.5,fill=gray!20](-8.434,2.458)--(-8.261,2.383)--(-7.956,2.16)--(-8.129,2.235)--cycle;
\filldraw[fill opacity=0.8,fill=gray!20,draw=none](-8.962,.89)--(-8.948,.922)--(-8.897,.917)--cycle;
\draw(-8.962,.89)--(-8.948,.922);
\filldraw[fill opacity=0.5,fill=gray!20](-8.18,2.667)--(-8.207,2.75)--(-7.835,2.479)--(-7.819,2.405)--cycle;
\filldraw[fill opacity=0.8,fill=gray!20,draw=none](-8.735,1.274)--(-8.725,1.331)--(-8.753,1.313)--(-8.759,1.258)--cycle;
\draw(-8.725,1.331)--(-8.753,1.313)--(-8.759,1.258)--(-8.735,1.274);
\filldraw[fill opacity=0.8,fill=gray!20,draw=none](-8.761,1.22)--(-8.735,1.274)--(-8.759,1.258)--(-8.777,1.209)--cycle;
\draw(-8.735,1.274)--(-8.759,1.258)--(-8.777,1.209)--(-8.761,1.22);
\filldraw[fill opacity=0.8,fill=gray!20,draw=none](-8.853,.951)--(-8.87,.958)--(-8.857,.963)--cycle;
\draw(-8.853,.951)--(-8.87,.958);
\filldraw[fill opacity=0.8,fill=gray!20,draw=none](-8.854,.943)--(-8.857,.953)--(-8.853,.951)--cycle;
\draw(-8.857,.953)--(-8.853,.951);
\filldraw[fill opacity=0.8,fill=gray!20](-7.869,1.195)--(-7.861,1.235)--(-7.781,1.238)--(-7.78,1.199)--cycle;
\filldraw[fill opacity=0.8,fill=gray!20](-7.938,1.182)--(-7.923,1.223)--(-7.861,1.235)--(-7.869,1.195)--cycle;
\filldraw[fill opacity=0.8,fill=gray!20,draw=none](-9.046,1.047)--(-9.052,1.05)--(-9.071,1.079)--(-9.077,1.089)--cycle;
\draw(-9.046,1.047)--(-9.052,1.05);
\filldraw[fill opacity=0.8,fill=gray!20,draw=none](-8.85,.95)--(-8.853,.951)--(-8.856,.96)--cycle;
\draw(-8.85,.95)--(-8.853,.951);
\filldraw[fill opacity=0.8,fill=gray!20,draw=none](-8.893,.939)--(-8.881,.955)--(-8.856,.96)--(-8.85,.95)--(-8.847,.916)--(-8.848,.916)--cycle;
\draw(-8.881,.955)--(-8.856,.96);
\draw(-8.847,.916)--(-8.848,.916);
\filldraw[fill opacity=0.8,fill=gray!20,draw=none](-8.893,.939)--(-8.897,.942)--(-8.881,.955)--cycle;
\filldraw[fill opacity=0.8,fill=gray!20,draw=none](-8.924,.92)--(-8.897,.942)--(-8.888,.939)--(-8.897,.917)--cycle;
\draw(-8.888,.939)--(-8.897,.917);
\filldraw[fill opacity=0.8,fill=gray!20,draw=none](-8.804,1.171)--(-8.761,1.22)--(-8.777,1.209)--(-8.806,1.169)--cycle;
\draw(-8.761,1.22)--(-8.777,1.209)--(-8.806,1.169)--(-8.804,1.171);
\filldraw[fill opacity=0.8,fill=gray!20,draw=none](-8.725,1.331)--(-8.735,1.385)--(-8.759,1.369)--(-8.753,1.313)--cycle;
\draw(-8.735,1.385)--(-8.759,1.369)--(-8.753,1.313)--(-8.725,1.331);
\filldraw[fill opacity=0.8,fill=gray!20,draw=none](-8.804,1.171)--(-8.806,1.169)--(-8.812,1.165)--cycle;
\draw(-8.804,1.171)--(-8.806,1.169)--(-8.812,1.165);
\filldraw[fill opacity=0.8,fill=gray!20,draw=none](-8.812,1.165)--(-8.806,1.169)--(-8.867,1.157)--(-8.878,1.143)--cycle;
\draw(-8.812,1.165)--(-8.806,1.169)--(-8.867,1.157)--(-8.878,1.143);
\filldraw[fill opacity=0.8,fill=gray!20](-8.806,1.169)--(-8.777,1.209)--(-8.851,1.195)--(-8.867,1.157)--cycle;
\filldraw[fill opacity=0.8,fill=gray!20,draw=none](-8.735,1.385)--(-8.761,1.434)--(-8.777,1.424)--(-8.759,1.369)--cycle;
\draw(-8.761,1.434)--(-8.777,1.424)--(-8.759,1.369)--(-8.735,1.385);
\filldraw[fill opacity=0.8,fill=gray!20,draw=none](-9.076,1.073)--(-9.071,1.079)--(-9.077,1.089)--(-9.085,1.079)--cycle;
\draw(-9.077,1.089)--(-9.085,1.079);
\filldraw[fill opacity=0.8,fill=gray!20,draw=none](-9.076,1.073)--(-9.071,1.079)--(-9.077,1.089)--(-9.085,1.079)--cycle;
\draw(-9.077,1.089)--(-9.085,1.079);
\filldraw[fill opacity=0.8,fill=gray!20,draw=none](-9.089,1.082)--(-9.086,1.078)--(-9.085,1.079)--cycle;
\draw(-9.086,1.078)--(-9.085,1.079);
\filldraw[fill opacity=0.8,fill=gray!20,draw=none](-9.089,1.082)--(-9.086,1.078)--(-9.085,1.079)--cycle;
\draw(-9.086,1.078)--(-9.085,1.079);
\filldraw[fill opacity=0.8,fill=gray!20,draw=none](-9.089,1.082)--(-9.086,1.078)--(-9.085,1.079)--cycle;
\draw(-9.086,1.078)--(-9.085,1.079);
\filldraw[fill opacity=0.8,fill=gray!20,draw=none](-8.891,.988)--(-8.876,.961)--(-9.007,1.018)--(-8.93,1.012)--(-8.899,.999)--cycle;
\draw(-8.876,.961)--(-9.007,1.018);
\draw(-8.93,1.012)--(-8.899,.999);
\filldraw[fill opacity=0.8,fill=gray!20,draw=none](-8.891,.988)--(-8.873,.962)--(-8.873,.96)--(-8.876,.961)--cycle;
\draw(-8.873,.96)--(-8.876,.961);
\filldraw[fill opacity=0.8,fill=gray!20,draw=none](-8.871,.959)--(-8.873,.96)--(-8.873,.962)--cycle;
\draw(-8.871,.959)--(-8.873,.96);
\filldraw[fill opacity=0.8,fill=gray!20,draw=none](-8.87,.958)--(-8.891,.964)--(-8.891,.966)--(-8.875,.965)--cycle;
\draw(-8.891,.964)--(-8.891,.966)--(-8.875,.965);
\filldraw[fill opacity=0.8,fill=gray!20,draw=none](-8.854,.973)--(-8.869,.964)--(-8.891,.966)--(-8.891,.994)--cycle;
\draw(-8.869,.964)--(-8.891,.966)--(-8.891,.994);
\filldraw[fill opacity=0.8,fill=gray!20,draw=none](-9.08,1.093)--(-9.087,1.097)--(-9.094,1.089)--(-9.089,1.082)--(-9.085,1.079)--(-9.077,1.089)--cycle;
\draw(-9.085,1.079)--(-9.077,1.089);
\filldraw[fill opacity=0.8,fill=gray!20,draw=none](-9.08,1.093)--(-9.087,1.097)--(-9.094,1.089)--(-9.089,1.082)--(-9.085,1.079)--(-9.077,1.089)--cycle;
\draw(-9.085,1.079)--(-9.077,1.089);
\filldraw[fill opacity=0.8,fill=gray!20,draw=none](-9.08,1.093)--(-9.087,1.097)--(-9.094,1.089)--(-9.089,1.082)--(-9.085,1.079)--(-9.077,1.089)--cycle;
\draw(-9.085,1.079)--(-9.077,1.089);
\filldraw[fill opacity=0.8,fill=gray!20,draw=none](-9.095,1.079)--(-9.099,1.084)--(-9.181,.991)--(-9.172,.981)--(-9.092,1.072)--cycle;
\draw(-9.181,.991)--(-9.172,.981)--(-9.092,1.072);
\filldraw[fill opacity=0.8,fill=gray!20,draw=none](-9.089,1.082)--(-9.09,1.082)--(-9.097,1.086)--cycle;
\draw(-9.089,1.082)--(-9.09,1.082);
\filldraw[fill opacity=0.8,fill=gray!20,draw=none](-9.094,1.089)--(-9.098,1.085)--(-9.095,1.079)--(-9.09,1.073)--(-9.086,1.078)--cycle;
\draw(-9.09,1.073)--(-9.086,1.078);
\filldraw[fill opacity=0.8,fill=gray!20,draw=none](-9.077,1.036)--(-9.072,1.039)--(-9.095,1.08)--(-9.095,1.079)--cycle;
\draw(-9.077,1.036)--(-9.072,1.039);
\draw(-9.095,1.08)--(-9.095,1.079);
\filldraw[fill opacity=0.8,fill=gray!20,draw=none](-9.077,1.036)--(-9.072,1.039)--(-9.095,1.08)--(-9.095,1.079)--cycle;
\draw(-9.077,1.036)--(-9.072,1.039);
\draw(-9.095,1.08)--(-9.095,1.079);
\filldraw[fill opacity=0.8,fill=gray!20,draw=none](-9.094,1.089)--(-9.098,1.085)--(-9.092,1.072)--(-9.086,1.078)--cycle;
\draw(-9.092,1.072)--(-9.086,1.078);
\filldraw[fill opacity=0.8,fill=gray!20,draw=none](-9.094,1.089)--(-9.098,1.085)--(-9.092,1.072)--(-9.086,1.078)--cycle;
\draw(-9.092,1.072)--(-9.086,1.078);
\filldraw[fill opacity=0.8,fill=gray!20,draw=none](-8.899,.999)--(-8.93,1.012)--(-8.917,1.009)--cycle;
\draw(-8.899,.999)--(-8.93,1.012);
\filldraw[fill opacity=0.8,fill=gray!20,draw=none](-8.897,.998)--(-8.891,.994)--(-8.891,.988)--cycle;
\draw(-8.891,.994)--(-8.891,.988);
\filldraw[fill opacity=0.8,fill=gray!20,draw=none](-8.819,.994)--(-8.854,.973)--(-8.891,.994)--(-8.892,1.013)--(-8.871,1.011)--(-8.828,1.002)--(-8.82,.998)--cycle;
\draw(-8.891,.994)--(-8.892,1.013)--(-8.871,1.011);
\filldraw[fill opacity=0.8,fill=gray!20,draw=none](-8.869,.981)--(-8.873,.962)--(-8.899,.999)--(-8.877,.989)--cycle;
\draw(-8.899,.999)--(-8.877,.989);
\filldraw[fill opacity=0.8,fill=gray!20,draw=none](-8.896,.966)--(-8.974,.997)--(-8.914,1.008)--(-8.897,.998)--(-8.891,.988)--(-8.891,.966)--cycle;
\draw(-8.891,.988)--(-8.891,.966)--(-8.896,.966);
\filldraw[fill opacity=0.8,fill=gray!20,draw=none](-8.896,.966)--(-8.998,.961)--(-8.989,1.003)--cycle;
\draw(-8.896,.966)--(-8.998,.961)--(-8.989,1.003);
\filldraw[fill opacity=0.8,fill=gray!20,draw=none](-8.907,.944)--(-8.881,.955)--(-8.888,.939)--cycle;
\draw(-8.881,.955)--(-8.888,.939);
\filldraw[fill opacity=0.8,fill=gray!20,draw=none](-8.907,.944)--(-8.935,.952)--(-8.93,.963)--(-8.889,.975)--(-8.874,.97)--(-8.881,.955)--cycle;
\draw(-8.935,.952)--(-8.93,.963);
\draw(-8.874,.97)--(-8.881,.955);
\filldraw[fill opacity=0.8,fill=gray!20,draw=none](-8.924,.92)--(-8.948,.922)--(-8.945,.929)--(-8.907,.944)--(-8.897,.942)--cycle;
\draw(-8.948,.922)--(-8.945,.929);
\filldraw[fill opacity=0.8,fill=gray!20,draw=none](-8.945,.929)--(-8.935,.952)--(-8.907,.944)--cycle;
\draw(-8.945,.929)--(-8.935,.952);
\filldraw[fill opacity=0.8,fill=gray!20,draw=none](-8.878,1.143)--(-8.867,1.157)--(-8.945,1.154)--(-8.945,1.151)--cycle;
\draw(-8.878,1.143)--(-8.867,1.157)--(-8.945,1.154)--(-8.945,1.151);
\filldraw[fill opacity=0.8,fill=gray!20](-8.777,1.209)--(-8.759,1.258)--(-8.841,1.242)--(-8.851,1.195)--cycle;
\filldraw[fill opacity=0.8,fill=gray!20,draw=none](-8.761,1.434)--(-8.804,1.474)--(-8.806,1.472)--(-8.777,1.424)--cycle;
\draw(-8.804,1.474)--(-8.806,1.472)--(-8.777,1.424)--(-8.761,1.434);
\filldraw[fill opacity=0.8,fill=gray!20](-8.759,1.258)--(-8.753,1.313)--(-8.838,1.296)--(-8.841,1.242)--cycle;
\filldraw[fill opacity=0.8,fill=gray!20](-8.867,1.157)--(-8.851,1.195)--(-8.947,1.191)--(-8.945,1.154)--cycle;
\filldraw[fill opacity=0.8,fill=gray!20](-8.753,1.313)--(-8.759,1.369)--(-8.841,1.353)--(-8.838,1.296)--cycle;
\filldraw[fill opacity=0.8,fill=gray!20](-8.759,1.369)--(-8.777,1.424)--(-8.851,1.409)--(-8.841,1.353)--cycle;
\filldraw[fill opacity=0.8,fill=gray!20,draw=none](-8.887,1.323)--(-8.997,1.198)--(-8.994,1.194)--(-8.987,1.19)--(-8.877,1.314)--cycle;
\draw(-8.987,1.19)--(-8.877,1.314)--(-8.887,1.323);
\filldraw[fill opacity=0.8,fill=gray!20,draw=none](-8.887,1.323)--(-8.997,1.198)--(-8.994,1.194)--(-8.987,1.19)--(-8.877,1.314)--cycle;
\draw(-8.987,1.19)--(-8.877,1.314)--(-8.887,1.323);
\filldraw[fill opacity=0.8,fill=gray!20,draw=none](-8.887,1.323)--(-8.997,1.198)--(-8.994,1.194)--(-8.987,1.19)--(-8.877,1.314)--cycle;
\draw(-8.987,1.19)--(-8.877,1.314)--(-8.887,1.323);
\filldraw[fill opacity=0.8,fill=gray!20](-8.851,1.195)--(-8.841,1.242)--(-8.949,1.238)--(-8.947,1.191)--cycle;
\filldraw[fill opacity=0.8,fill=gray!20,draw=none](-8.998,1.197)--(-9.085,1.1)--(-9.08,1.093)--(-9.075,1.09)--(-8.987,1.19)--cycle;
\draw(-9.075,1.09)--(-8.987,1.19);
\filldraw[fill opacity=0.8,fill=gray!20,draw=none](-8.998,1.197)--(-9.085,1.1)--(-9.08,1.093)--(-9.075,1.09)--(-8.987,1.19)--cycle;
\draw(-9.075,1.09)--(-8.987,1.19);
\filldraw[fill opacity=0.8,fill=gray!20,draw=none](-8.998,1.197)--(-9.085,1.1)--(-9.08,1.093)--(-9.075,1.09)--(-8.987,1.19)--cycle;
\draw(-9.075,1.09)--(-8.987,1.19);
\filldraw[fill opacity=0.8,fill=gray!20,draw=none](-9.08,1.093)--(-9.085,1.1)--(-9.087,1.097)--cycle;
\filldraw[fill opacity=0.8,fill=gray!20,draw=none](-9.08,1.093)--(-9.085,1.1)--(-9.087,1.097)--cycle;
\filldraw[fill opacity=0.8,fill=gray!20,draw=none](-9.08,1.093)--(-9.085,1.1)--(-9.087,1.097)--cycle;
\filldraw[fill opacity=0.8,fill=gray!20,draw=none](-9.056,1.083)--(-9.08,1.093)--(-9.123,1.119)--(-9.099,1.109)--cycle;
\draw(-9.123,1.119)--(-9.099,1.109)--(-9.056,1.083)--(-9.08,1.093);
\filldraw[fill opacity=0.8,fill=gray!20,draw=none](-9.045,1.059)--(-9.038,1.054)--(-9.046,1.068)--cycle;
\draw(-9.038,1.054)--(-9.046,1.068);
\filldraw[fill opacity=0.8,fill=gray!20,draw=none](-9.065,1.071)--(-9.04,1.033)--(-9.045,1.059)--cycle;
\filldraw[fill opacity=0.8,fill=gray!20,draw=none](-9.079,1.092)--(-9.065,1.071)--(-9.045,1.059)--(-9.046,1.068)--(-9.056,1.083)--(-9.071,1.092)--cycle;
\draw(-9.046,1.068)--(-9.056,1.083)--(-9.071,1.092);
\filldraw[fill opacity=0.8,fill=gray!20,draw=none](-9.007,1.018)--(-9.04,1.033)--(-9.065,1.071)--(-8.93,1.012)--cycle;
\draw(-9.007,1.018)--(-9.04,1.033);
\draw(-9.065,1.071)--(-8.93,1.012);
\filldraw[fill opacity=0.8,fill=gray!20,draw=none](-9.104,1.095)--(-9.065,1.071)--(-9.079,1.092)--cycle;
\filldraw[fill opacity=0.8,fill=gray!20,draw=none](-8.967,.909)--(-9.001,.907)--(-9.001,.917)--cycle;
\draw(-8.967,.909)--(-9.001,.907)--(-9.001,.917);
\filldraw[fill opacity=0.8,fill=gray!20,draw=none](-8.998,.906)--(-8.96,.993)--(-8.912,1.005)--(-8.962,.89)--cycle;
\draw(-8.912,1.005)--(-8.962,.89)--(-8.998,.906)--(-8.96,.993);
\filldraw[fill opacity=0.8,fill=gray!20,draw=none](-8.962,.936)--(-8.91,.948)--(-8.897,.942)--(-8.962,.89)--cycle;
\draw(-8.962,.89)--(-8.962,.936)--(-8.91,.948);
\filldraw[fill opacity=0.8,fill=gray!20](-8.938,.934)--(-7.825,.47)--(-7.849,.426)--(-8.962,.89)--cycle;
\filldraw[fill opacity=0.8,fill=gray!20,draw=none](-8.981,.77)--(-8.945,.788)--(-8.933,.783)--cycle;
\draw(-8.945,.788)--(-8.933,.783);
\filldraw[fill opacity=0.8,fill=gray!20,draw=none](-8.933,.783)--(-8.935,.784)--(-8.919,.797)--cycle;
\draw(-8.933,.783)--(-8.935,.784);
\filldraw[fill opacity=0.8,fill=gray!20,draw=none](-8.978,.795)--(-8.892,.799)--(-8.894,.748)--cycle;
\draw(-8.978,.795)--(-8.892,.799)--(-8.894,.748);
\filldraw[fill opacity=0.8,fill=gray!20,draw=none](-8.902,.78)--(-8.909,.773)--(-8.933,.783)--(-8.919,.797)--(-8.897,.816)--(-8.895,.815)--cycle;
\draw(-8.909,.773)--(-8.933,.783);
\draw(-8.897,.816)--(-8.895,.815);
\filldraw[fill opacity=0.8,fill=gray!20,draw=none](-8.855,.761)--(-8.865,.744)--(-8.894,.747)--(-8.892,.799)--(-8.877,.797)--cycle;
\draw(-8.865,.744)--(-8.894,.747)--(-8.892,.799)--(-8.877,.797);
\filldraw[fill opacity=0.8,fill=gray!20,draw=none](-8.88,.801)--(-8.902,.78)--(-8.895,.815)--(-8.876,.807)--cycle;
\draw(-8.895,.815)--(-8.876,.807);
\filldraw[fill opacity=0.8,fill=gray!20,draw=none](-8.842,.785)--(-8.855,.761)--(-8.877,.797)--(-8.842,.795)--cycle;
\draw(-8.877,.797)--(-8.842,.795);
\filldraw[fill opacity=0.8,fill=gray!20,draw=none](-8.829,.834)--(-8.837,.795)--(-8.86,.796)--cycle;
\draw(-8.837,.795)--(-8.86,.796);
\filldraw[fill opacity=0.8,fill=gray!20,draw=none](-8.842,.785)--(-8.842,.795)--(-8.837,.795)--cycle;
\draw(-8.842,.795)--(-8.837,.795);
\filldraw[fill opacity=0.8,fill=gray!20,draw=none](-8.857,.846)--(-8.875,.806)--(-8.883,.81)--cycle;
\draw(-8.875,.806)--(-8.883,.81);
\filldraw[fill opacity=0.8,fill=gray!20,draw=none](-8.88,.801)--(-8.876,.807)--(-8.875,.806)--cycle;
\draw(-8.876,.807)--(-8.875,.806);
\filldraw[fill opacity=0.8,fill=gray!20](-8.962,.89)--(-7.849,.426)--(-7.864,.377)--(-8.977,.842)--cycle;
\filldraw[fill opacity=0.8,fill=gray!20,draw=none](-8.349,2.859)--(-8.351,2.859)--(-8.361,2.865)--(-8.358,2.864)--cycle;
\draw(-8.349,2.859)--(-8.351,2.859);
\filldraw[fill opacity=0.8,fill=gray!20](-7.83,.233)--(-7.867,.256)--(-7.831,.263)--(-7.812,.237)--cycle;
\filldraw[fill opacity=0.8,fill=gray!20,draw=none](-8.437,2.888)--(-8.454,2.893)--(-8.453,2.894)--cycle;
\draw(-8.454,2.893)--(-8.453,2.894);
\filldraw[fill opacity=0.8,fill=gray!20,draw=none](-8.349,2.859)--(-8.348,2.859)--(-8.348,2.858)--cycle;
\draw(-8.349,2.859)--(-8.348,2.859)--(-8.348,2.858);
\filldraw[fill opacity=0.8,fill=gray!20,draw=none](-8.38,2.825)--(-8.403,2.856)--(-8.349,2.859)--(-8.348,2.858)--(-8.351,2.826)--cycle;
\draw(-8.348,2.858)--(-8.351,2.826)--(-8.38,2.825)--(-8.403,2.856)--(-8.349,2.859);
\filldraw[fill opacity=0.8,fill=gray!20,draw=none](-7.574,4.671)--(-7.526,4.651)--(-7.513,4.646)--(-7.525,4.651)--cycle;
\draw(-7.513,4.646)--(-7.525,4.651);
\filldraw[fill opacity=0.8,fill=gray!20,draw=none](-8.289,2.835)--(-8.307,2.841)--(-8.301,2.847)--(-8.297,2.844)--cycle;
\draw(-8.307,2.841)--(-8.301,2.847);
\filldraw[fill opacity=0.5,fill=gray!20,draw=none](-8.319,2.845)--(-8.289,2.835)--(-8.326,2.856)--(-8.345,2.861)--(-8.33,2.85)--cycle;
\draw(-8.326,2.856)--(-8.345,2.861);
\filldraw[fill opacity=0.5,fill=gray!20,draw=none](-8.319,2.845)--(-8.262,2.819)--(-8.267,2.823)--(-8.289,2.835)--cycle;
\draw(-8.262,2.819)--(-8.267,2.823);
\filldraw[fill opacity=0.8,fill=gray!20,draw=none](-8.351,2.826)--(-8.348,2.858)--(-8.307,2.841)--(-8.323,2.824)--cycle;
\draw(-8.307,2.841)--(-8.323,2.824)--(-8.351,2.826)--(-8.348,2.858);
\filldraw[fill opacity=0.8,fill=gray!20](-8.146,3.975)--(-8.117,4.015)--(-8.057,4.026)--(-8.072,3.989)--cycle;
\filldraw[fill opacity=0.8,fill=gray!20](-7.764,.236)--(-7.74,.262)--(-7.709,.254)--(-7.748,.232)--cycle;
\filldraw[fill opacity=0.8,fill=gray!20,draw=none](-7.665,.509)--(-7.703,.518)--(-7.719,.55)--(-7.675,.539)--(-7.66,.519)--cycle;
\draw(-7.665,.509)--(-7.703,.518)--(-7.719,.55)--(-7.675,.539)--(-7.66,.519);
\filldraw[fill opacity=0.8,fill=gray!20,draw=none](-7.49,4.802)--(-7.472,4.781)--(-7.476,4.772)--(-7.513,4.799)--cycle;
\draw(-7.472,4.781)--(-7.476,4.772);
\filldraw[fill opacity=0.8,fill=gray!20,draw=none](-7.504,4.817)--(-7.464,4.8)--(-7.472,4.781)--cycle;
\draw(-7.504,4.817)--(-7.464,4.8)--(-7.472,4.781);
\filldraw[fill opacity=0.8,fill=gray!20,draw=none](-7.456,4.797)--(-7.439,4.765)--(-7.476,4.772)--(-7.464,4.8)--cycle;
\draw(-7.476,4.772)--(-7.464,4.8)--(-7.456,4.797);
\filldraw[fill opacity=0.8,fill=gray!20,draw=none](-7.456,4.797)--(-7.43,4.785)--(-7.439,4.765)--cycle;
\draw(-7.456,4.797)--(-7.43,4.785)--(-7.439,4.765);
\filldraw[fill opacity=0.8,fill=gray!20,draw=none](-7.444,4.748)--(-7.513,4.799)--(-7.515,4.81)--(-7.45,4.794)--(-7.43,4.745)--cycle;
\draw(-7.513,4.799)--(-7.515,4.81)--(-7.45,4.794)--(-7.43,4.745)--(-7.444,4.748);
\filldraw[fill opacity=0.8,fill=gray!20](-6.948,.196)--(-6.945,.229)--(-6.891,.226)--(-6.92,.194)--cycle;
\filldraw[fill opacity=0.8,fill=gray!20](-6.977,.195)--(-7.001,.227)--(-6.945,.229)--(-6.948,.196)--cycle;
\filldraw[fill opacity=0.8,fill=gray!20,draw=none](-7.667,.474)--(-7.784,.476)--(-7.786,.511)--(-7.664,.51)--cycle;
\draw(-7.667,.474)--(-7.784,.476)--(-7.786,.511)--(-7.664,.51);
\filldraw[fill opacity=0.8,fill=gray!20,draw=none](-7.679,4.761)--(-7.658,4.76)--(-7.663,4.748)--cycle;
\draw(-7.658,4.76)--(-7.663,4.748);
\filldraw[fill opacity=0.8,fill=gray!20,draw=none](-7.585,4.765)--(-7.64,4.739)--(-7.663,4.748)--(-7.679,4.761)--(-7.623,4.788)--cycle;
\draw(-7.679,4.761)--(-7.623,4.788)--(-7.585,4.765)--(-7.64,4.739);
\filldraw[fill opacity=0.8,fill=gray!20,draw=none](-8.454,2.893)--(-8.484,2.888)--(-8.496,2.907)--cycle;
\draw(-8.454,2.893)--(-8.484,2.888)--(-8.496,2.907);
\filldraw[fill opacity=0.8,fill=gray!20,draw=none](-7.956,.363)--(-7.958,.387)--(-7.942,.41)--(-7.938,.375)--cycle;
\draw(-7.942,.41)--(-7.938,.375)--(-7.956,.363);
\filldraw[fill opacity=0.8,fill=gray!20,draw=none](-7.964,.349)--(-7.967,.356)--(-7.938,.375)--(-7.928,.347)--cycle;
\draw(-7.964,.349)--(-7.967,.356)--(-7.938,.375)--(-7.928,.347);
\filldraw[fill opacity=0.8,fill=gray!20,draw=none](-7.958,.387)--(-7.956,.363)--(-7.967,.356)--(-7.968,.368)--(-7.967,.374)--cycle;
\draw(-7.956,.363)--(-7.967,.356)--(-7.968,.368);
\filldraw[fill opacity=0.8,fill=gray!20,draw=none](-7.958,.387)--(-7.967,.374)--(-7.962,.41)--(-7.96,.411)--cycle;
\draw(-7.962,.41)--(-7.96,.411);
\filldraw[fill opacity=0.8,fill=gray!20,draw=none](-7.959,.417)--(-7.96,.411)--(-7.962,.41)--cycle;
\draw(-7.96,.411)--(-7.962,.41);
\filldraw[fill opacity=0.8,fill=gray!20,draw=none](-8.981,.77)--(-8.988,.794)--(-8.978,.795)--(-8.912,.758)--cycle;
\draw(-8.981,.77)--(-8.988,.794)--(-8.978,.795);
\filldraw[fill opacity=0.8,fill=gray!20,draw=none](-8.973,.743)--(-8.981,.77)--(-8.912,.758)--(-8.894,.748)--(-8.894,.747)--cycle;
\draw(-8.894,.748)--(-8.894,.747)--(-8.973,.743)--(-8.981,.77);
\filldraw[fill opacity=0.8,fill=gray!20,draw=none](-8.908,.766)--(-8.928,.756)--(-8.968,.773)--(-8.933,.783)--(-8.903,.77)--cycle;
\draw(-8.933,.783)--(-8.903,.77);
\filldraw[fill opacity=0.8,fill=gray!20,draw=none](-8.842,.759)--(-8.853,.758)--(-8.855,.761)--(-8.842,.785)--cycle;
\filldraw[fill opacity=0.8,fill=gray!20,draw=none](-8.843,.739)--(-8.851,.717)--(-8.892,.746)--(-8.845,.743)--cycle;
\draw(-8.892,.746)--(-8.845,.743);
\filldraw[fill opacity=0.8,fill=gray!20,draw=none](-8.853,.758)--(-8.858,.757)--(-8.855,.761)--cycle;
\filldraw[fill opacity=0.8,fill=gray!20,draw=none](-8.853,.758)--(-8.845,.743)--(-8.865,.744)--(-8.858,.757)--cycle;
\draw(-8.845,.743)--(-8.865,.744);
\filldraw[fill opacity=0.8,fill=gray!20,draw=none](-8.842,.759)--(-8.842,.743)--(-8.845,.743)--(-8.853,.758)--cycle;
\draw(-8.842,.743)--(-8.845,.743);
\filldraw[fill opacity=0.8,fill=gray!20,draw=none](-8.843,.739)--(-8.845,.743)--(-8.842,.743)--cycle;
\draw(-8.845,.743)--(-8.842,.743);
\filldraw[fill opacity=0.8,fill=gray!20,draw=none](-8.901,.769)--(-8.909,.773)--(-8.88,.801)--cycle;
\draw(-8.901,.769)--(-8.909,.773);
\filldraw[fill opacity=0.8,fill=gray!20,draw=none](-8.908,.766)--(-8.903,.77)--(-8.901,.769)--cycle;
\draw(-8.903,.77)--(-8.901,.769);
\filldraw[fill opacity=0.8,fill=gray!20](-8.977,.842)--(-7.864,.377)--(-7.868,.332)--(-8.981,.797)--cycle;
\filldraw[fill opacity=0.8,fill=gray!20,draw=none](-8.364,3.187)--(-8.371,3.205)--(-8.345,3.206)--(-8.344,3.185)--cycle;
\draw(-8.371,3.205)--(-8.345,3.206)--(-8.344,3.185);
\filldraw[fill opacity=0.8,fill=gray!20,draw=none](-8.248,3.297)--(-8.296,3.189)--(-8.319,3.197)--(-8.328,3.237)--(-8.22,3.479)--cycle;
\draw(-8.248,3.297)--(-8.296,3.189);
\draw(-8.328,3.237)--(-8.22,3.479);
\filldraw[fill opacity=0.8,fill=gray!20,draw=none](-8.26,3.168)--(-8.301,3.179)--(-8.248,3.297)--cycle;
\draw(-8.301,3.179)--(-8.248,3.297);
\filldraw[fill opacity=0.8,fill=gray!20,draw=none](-8.296,3.189)--(-8.301,3.179)--(-8.317,3.19)--(-8.319,3.197)--cycle;
\draw(-8.296,3.189)--(-8.301,3.179);
\filldraw[fill opacity=0.8,fill=gray!20,draw=none](-8.319,3.197)--(-8.317,3.19)--(-8.342,3.206)--cycle;
\filldraw[fill opacity=0.8,fill=gray!20,draw=none](-8.319,3.197)--(-8.342,3.206)--(-8.328,3.237)--cycle;
\draw(-8.342,3.206)--(-8.328,3.237);
\filldraw[fill opacity=0.8,fill=gray!20,draw=none](-8.344,3.185)--(-8.345,3.206)--(-8.269,3.2)--(-8.251,3.166)--cycle;
\draw(-8.344,3.185)--(-8.345,3.206)--(-8.269,3.2)--(-8.251,3.166);
\filldraw[fill opacity=0.8,fill=gray!20,draw=none](-7.923,.511)--(-7.904,.538)--(-7.898,.542)--(-7.901,.538)--cycle;
\draw(-7.904,.538)--(-7.898,.542)--(-7.901,.538);
\filldraw[fill opacity=0.8,fill=gray!20,draw=none](-7.901,.538)--(-7.898,.542)--(-7.894,.543)--cycle;
\draw(-7.901,.538)--(-7.898,.542)--(-7.894,.543);
\filldraw[fill opacity=0.8,fill=gray!20,draw=none](-7.904,.538)--(-7.873,.561)--(-7.875,.56)--(-7.898,.542)--cycle;
\draw(-7.875,.56)--(-7.898,.542)--(-7.904,.538);
\filldraw[fill opacity=0.8,fill=gray!20](-7.861,1.056)--(-7.869,1.103)--(-7.78,1.107)--(-7.781,1.06)--cycle;
\filldraw[fill opacity=0.8,fill=gray!20](-7.848,1.013)--(-7.861,1.056)--(-7.781,1.06)--(-7.783,1.016)--cycle;
\filldraw[fill opacity=0.8,fill=gray!20](-7.923,1.044)--(-7.938,1.089)--(-7.869,1.103)--(-7.861,1.056)--cycle;
\filldraw[fill opacity=0.8,fill=gray!20,draw=none](-8.735,.879)--(-8.663,.884)--(-8.611,.866)--(-8.645,.858)--cycle;
\draw(-8.611,.866)--(-8.645,.858);
\filldraw[fill opacity=0.8,fill=gray!20,draw=none](-8.663,.884)--(-8.735,.879)--(-8.756,.884)--(-8.699,.897)--cycle;
\draw(-8.756,.884)--(-8.699,.897);
\filldraw[fill opacity=0.8,fill=gray!20,draw=none](-8.796,.875)--(-8.735,.879)--(-8.645,.858)--cycle;
\filldraw[fill opacity=0.8,fill=gray!20,draw=none](-8.735,.879)--(-8.796,.875)--(-8.756,.884)--cycle;
\draw(-8.796,.875)--(-8.756,.884);
\filldraw[fill opacity=0.8,fill=gray!20,draw=none](-8.812,.949)--(-8.783,.93)--(-8.815,.923)--cycle;
\draw(-8.783,.93)--(-8.815,.923);
\filldraw[fill opacity=0.8,fill=gray!20,draw=none](-8.819,.994)--(-8.81,.96)--(-8.869,.964)--cycle;
\draw(-8.81,.96)--(-8.869,.964);
\filldraw[fill opacity=0.8,fill=gray!20,draw=none](-8.812,.949)--(-8.809,.971)--(-7.965,1.16)--(-7.956,1.128)--(-7.957,1.116)--(-8.783,.93)--cycle;
\draw(-8.809,.971)--(-7.965,1.16);
\draw(-7.957,1.116)--(-8.783,.93);
\filldraw[fill opacity=0.8,fill=gray!20,draw=none](-8.807,.906)--(-8.815,.923)--(-8.809,.924)--cycle;
\draw(-8.815,.923)--(-8.809,.924);
\filldraw[fill opacity=0.8,fill=gray!20,draw=none](-8.812,.943)--(-8.788,.936)--(-8.79,.905)--(-8.816,.907)--cycle;
\draw(-8.79,.905)--(-8.816,.907);
\filldraw[fill opacity=0.8,fill=gray!20,draw=none](-8.803,.878)--(-8.79,.905)--(-8.782,.904)--(-8.783,.891)--cycle;
\draw(-8.79,.905)--(-8.782,.904)--(-8.783,.891);
\filldraw[fill opacity=0.8,fill=gray!20,draw=none](-8.807,.906)--(-8.806,.901)--(-8.847,.915)--(-8.847,.916)--(-8.815,.923)--cycle;
\draw(-8.847,.916)--(-8.815,.923);
\filldraw[fill opacity=0.8,fill=gray!20,draw=none](-8.825,.946)--(-8.812,.943)--(-8.815,.923)--cycle;
\filldraw[fill opacity=0.8,fill=gray!20,draw=none](-8.788,.936)--(-8.828,.947)--(-8.817,.961)--(-8.786,.958)--cycle;
\draw(-8.817,.961)--(-8.786,.958);
\filldraw[fill opacity=0.8,fill=gray!20,draw=none](-8.851,.961)--(-8.837,.965)--(-8.812,.949)--(-8.815,.923)--(-8.847,.916)--cycle;
\draw(-8.851,.961)--(-8.837,.965);
\draw(-8.815,.923)--(-8.847,.916);
\filldraw[fill opacity=0.8,fill=gray!20,draw=none](-8.79,.905)--(-8.788,.936)--(-8.785,.935)--(-8.782,.904)--cycle;
\draw(-8.785,.935)--(-8.782,.904)--(-8.79,.905);
\filldraw[fill opacity=0.8,fill=gray!20,draw=none](-8.783,.891)--(-8.782,.904)--(-8.775,.903)--cycle;
\draw(-8.783,.891)--(-8.782,.904)--(-8.775,.903);
\filldraw[fill opacity=0.8,fill=gray!20,draw=none](-8.772,.911)--(-8.775,.903)--(-8.782,.904)--cycle;
\draw(-8.775,.903)--(-8.782,.904);
\filldraw[fill opacity=0.8,fill=gray!20,draw=none](-8.784,.93)--(-7.964,1.114)--(-7.951,1.081)--(-7.949,1.065)--(-8.699,.897)--cycle;
\draw(-8.784,.93)--(-7.964,1.114);
\draw(-7.949,1.065)--(-8.699,.897);
\filldraw[fill opacity=0.8,fill=gray!20,draw=none](-8.764,.933)--(-8.772,.911)--(-8.782,.904)--(-8.785,.935)--cycle;
\draw(-8.782,.904)--(-8.785,.935);
\filldraw[fill opacity=0.8,fill=gray!20,draw=none](-8.843,.902)--(-8.847,.915)--(-8.842,.913)--cycle;
\filldraw[fill opacity=0.8,fill=gray!20,draw=none](-8.847,.915)--(-8.835,.937)--(-8.835,.936)--(-8.838,.9)--(-8.843,.902)--cycle;
\draw(-8.838,.9)--(-8.843,.902);
\filldraw[fill opacity=0.8,fill=gray!20,draw=none](-8.835,.937)--(-8.847,.915)--(-8.85,.95)--(-8.838,.944)--cycle;
\draw(-8.85,.95)--(-8.838,.944);
\filldraw[fill opacity=0.8,fill=gray!20,draw=none](-8.847,.915)--(-8.848,.916)--(-8.847,.916)--cycle;
\draw(-8.848,.916)--(-8.847,.916);
\filldraw[fill opacity=0.8,fill=gray!20,draw=none](-8.841,.894)--(-8.839,.901)--(-8.838,.9)--cycle;
\draw(-8.839,.901)--(-8.838,.9);
\filldraw[fill opacity=0.8,fill=gray!20,draw=none](-8.859,.969)--(-8.857,.963)--(-8.87,.958)--(-8.871,.959)--(-8.873,.962)--(-8.869,.981)--cycle;
\draw(-8.87,.958)--(-8.871,.959);
\filldraw[fill opacity=0.8,fill=gray!20,draw=none](-8.859,.969)--(-8.856,.96)--(-8.881,.955)--cycle;
\draw(-8.856,.96)--(-8.881,.955);
\filldraw[fill opacity=0.8,fill=gray!20,draw=none](-8.835,.936)--(-8.897,.917)--(-8.877,.939)--(-8.835,.937)--cycle;
\filldraw[fill opacity=0.8,fill=gray!20,draw=none](-8.877,.939)--(-8.897,.917)--(-8.888,.939)--cycle;
\draw(-8.897,.917)--(-8.888,.939);
\filldraw[fill opacity=0.8,fill=gray!20,draw=none](-8.859,.969)--(-8.851,.974)--(-8.837,.965)--(-8.856,.96)--cycle;
\draw(-8.837,.965)--(-8.856,.96);
\filldraw[fill opacity=0.8,fill=gray!20,draw=none](-8.813,.97)--(-8.82,.998)--(-8.804,.99)--cycle;
\filldraw[fill opacity=0.8,fill=gray!20,draw=none](-8.804,.99)--(-8.816,.996)--(-8.799,1.006)--(-8.798,1.004)--cycle;
\draw(-8.799,1.006)--(-8.798,1.004);
\filldraw[fill opacity=0.8,fill=gray!20,draw=none](-8.816,.996)--(-8.84,1.009)--(-8.799,1.006)--cycle;
\draw(-8.84,1.009)--(-8.799,1.006);
\filldraw[fill opacity=0.8,fill=gray!20,draw=none](-8.856,.96)--(-8.851,.961)--(-8.85,.95)--cycle;
\draw(-8.856,.96)--(-8.851,.961);
\filldraw[fill opacity=0.8,fill=gray!20,draw=none](-8.838,.944)--(-8.85,.95)--(-8.856,.96)--(-8.857,.963)--(-8.855,.964)--cycle;
\draw(-8.838,.944)--(-8.85,.95);
\filldraw[fill opacity=0.8,fill=gray!20,draw=none](-8.851,.974)--(-8.815,.998)--(-8.801,.997)--(-8.813,.97)--(-8.837,.965)--cycle;
\draw(-8.813,.97)--(-8.837,.965);
\filldraw[fill opacity=0.8,fill=gray!20,draw=none](-8.855,.964)--(-8.857,.963)--(-8.859,.969)--cycle;
\filldraw[fill opacity=0.8,fill=gray!20,draw=none](-8.863,.966)--(-8.855,.964)--(-8.854,.963)--(-8.877,.939)--(-8.888,.939)--(-8.881,.955)--cycle;
\draw(-8.888,.939)--(-8.881,.955);
\filldraw[fill opacity=0.8,fill=gray!20,draw=none](-8.914,1.008)--(-8.892,1.013)--(-8.891,.994)--cycle;
\draw(-8.892,1.013)--(-8.891,.994);
\filldraw[fill opacity=0.8,fill=gray!20,draw=none](-8.875,.988)--(-8.899,.999)--(-8.917,1.009)--(-8.905,1.006)--cycle;
\draw(-8.875,.988)--(-8.899,.999);
\filldraw[fill opacity=0.8,fill=gray!20,draw=none](-8.869,.981)--(-8.877,.989)--(-8.875,.988)--cycle;
\draw(-8.877,.989)--(-8.875,.988);
\filldraw[fill opacity=0.8,fill=gray!20,draw=none](-8.863,.966)--(-8.881,.955)--(-8.874,.97)--cycle;
\draw(-8.881,.955)--(-8.874,.97);
\filldraw[fill opacity=0.8,fill=gray!20,draw=none](-8.825,.997)--(-8.823,.996)--(-8.854,.964)--(-8.874,.97)--(-8.869,.981)--cycle;
\draw(-8.874,.97)--(-8.869,.981);
\filldraw[fill opacity=0.5,fill=gray!20](-8.169,2.567)--(-8.18,2.667)--(-7.819,2.405)--(-7.825,2.317)--cycle;
\filldraw[fill opacity=0.8,fill=gray!20,draw=none](-8.889,.975)--(-8.869,.981)--(-8.874,.97)--cycle;
\draw(-8.869,.981)--(-8.874,.97);
\filldraw[fill opacity=0.8,fill=gray!20,draw=none](-8.889,.975)--(-8.92,.987)--(-8.912,1.005)--(-8.905,1.006)--(-8.868,.984)--(-8.869,.981)--cycle;
\draw(-8.92,.987)--(-8.912,1.005);
\draw(-8.868,.984)--(-8.869,.981);
\filldraw[fill opacity=0.8,fill=gray!20,draw=none](-8.869,.981)--(-8.875,.988)--(-8.869,.985)--cycle;
\draw(-8.875,.988)--(-8.869,.985);
\filldraw[fill opacity=0.8,fill=gray!20,draw=none](-8.831,1.001)--(-8.825,.997)--(-8.869,.981)--(-8.858,1.006)--cycle;
\draw(-8.869,.981)--(-8.858,1.006);
\filldraw[fill opacity=0.8,fill=gray!20,draw=none](-8.905,1.006)--(-8.888,1.01)--(-8.858,1.006)--(-8.868,.984)--cycle;
\draw(-8.858,1.006)--(-8.868,.984);
\filldraw[fill opacity=0.8,fill=gray!20,draw=none](-8.859,.969)--(-8.869,.981)--(-8.869,.985)--(-8.865,.984)--cycle;
\draw(-8.869,.985)--(-8.865,.984);
\filldraw[fill opacity=0.8,fill=gray!20,draw=none](-8.869,.985)--(-8.875,.988)--(-8.905,1.006)--(-8.894,1.004)--(-8.885,1)--cycle;
\draw(-8.869,.985)--(-8.875,.988);
\draw(-8.894,1.004)--(-8.885,1);
\filldraw[fill opacity=0.8,fill=gray!20,draw=none](-8.93,.963)--(-8.92,.987)--(-8.889,.975)--cycle;
\draw(-8.93,.963)--(-8.92,.987);
\filldraw[fill opacity=0.8,fill=gray!20,draw=none](-8.954,.972)--(-8.874,.99)--(-8.865,.984)--(-8.859,.969)--(-8.881,.955)--(-8.962,.936)--cycle;
\draw(-8.881,.955)--(-8.962,.936)--(-8.954,.972)--(-8.874,.99);
\filldraw[fill opacity=0.8,fill=gray!20,draw=none](-8.897,.942)--(-8.91,.948)--(-8.881,.955)--cycle;
\draw(-8.91,.948)--(-8.881,.955);
\filldraw[fill opacity=0.8,fill=gray!20,draw=none](-8.835,.937)--(-8.877,.939)--(-8.854,.964)--(-8.844,.961)--cycle;
\filldraw[fill opacity=0.8,fill=gray!20,draw=none](-8.835,.936)--(-8.835,.937)--(-8.833,.937)--cycle;
\filldraw[fill opacity=0.8,fill=gray!20,draw=none](-8.835,.937)--(-8.835,.938)--(-8.835,.936)--cycle;
\filldraw[fill opacity=0.8,fill=gray!20](-8.909,.968)--(-7.796,.504)--(-7.825,.47)--(-8.938,.934)--cycle;
\filldraw[fill opacity=0.8,fill=gray!20,draw=none](-7.949,.313)--(-7.964,.349)--(-7.928,.347)--(-7.923,.33)--cycle;
\draw(-7.928,.347)--(-7.923,.33)--(-7.949,.313)--(-7.964,.349);
\filldraw[fill opacity=0.8,fill=gray!20](-7.92,.275)--(-7.949,.313)--(-7.923,.33)--(-7.898,.289)--cycle;
\filldraw[fill opacity=0.8,fill=gray!20,draw=none](-8.03,3.672)--(-8.041,3.674)--(-8.047,3.678)--cycle;
\filldraw[fill opacity=0.8,fill=gray!20](-8.035,3.644)--(-8.079,3.672)--(-8.036,3.68)--(-8.013,3.649)--cycle;
\filldraw[fill opacity=0.8,fill=gray!20,draw=none](-8.47,2.919)--(-8.518,2.941)--(-8.512,2.936)--(-8.474,2.913)--(-8.422,2.891)--cycle;
\draw(-8.47,2.919)--(-8.518,2.941);
\draw(-8.474,2.913)--(-8.422,2.891);
\filldraw[fill opacity=0.8,fill=gray!20,draw=none](-7.846,4.383)--(-7.815,4.372)--(-7.776,4.356)--cycle;
\draw(-7.846,4.383)--(-7.815,4.372);
\filldraw[fill opacity=0.8,fill=gray!20](-7.11,.521)--(-7.081,.561)--(-7.021,.573)--(-7.037,.535)--cycle;
\filldraw[fill opacity=0.8,fill=gray!20,draw=none](-7.515,4.81)--(-7.523,4.826)--(-7.45,4.794)--cycle;
\draw(-7.45,4.794)--(-7.515,4.81)--(-7.523,4.826);
\filldraw[fill opacity=0.8,fill=gray!20,draw=none](-7.439,4.765)--(-7.479,4.672)--(-7.509,4.644)--(-7.528,4.652)--(-7.476,4.772)--cycle;
\draw(-7.439,4.765)--(-7.479,4.672);
\draw(-7.528,4.652)--(-7.476,4.772);
\filldraw[fill opacity=0.8,fill=gray!20,draw=none](-7.526,4.651)--(-7.563,4.666)--(-7.531,4.681)--cycle;
\draw(-7.563,4.666)--(-7.531,4.681)--(-7.526,4.651);
\filldraw[fill opacity=0.8,fill=gray!20,draw=none](-7.69,4.7)--(-7.524,4.625)--(-7.523,4.633)--(-7.525,4.647)--(-7.54,4.656)--(-7.715,4.721)--(-7.716,4.715)--cycle;
\draw(-7.524,4.625)--(-7.523,4.633)--(-7.525,4.647);
\draw(-7.715,4.721)--(-7.716,4.715);
\filldraw[fill opacity=0.8,fill=gray!20,draw=none](-7.525,4.635)--(-7.512,4.63)--(-7.521,4.607)--(-7.523,4.605)--cycle;
\draw(-7.525,4.635)--(-7.512,4.63)--(-7.521,4.607);
\filldraw[fill opacity=0.8,fill=gray!20,draw=none](-7.508,4.6)--(-7.493,4.637)--(-7.477,4.63)--(-7.478,4.628)--cycle;
\draw(-7.508,4.6)--(-7.493,4.637)--(-7.477,4.63)--(-7.478,4.628);
\filldraw[fill opacity=0.8,fill=gray!20,draw=none](-7.478,4.63)--(-7.479,4.629)--(-7.51,4.601)--(-7.494,4.638)--cycle;
\draw(-7.478,4.63)--(-7.479,4.629);
\draw(-7.51,4.601)--(-7.494,4.638);
\filldraw[fill opacity=0.8,fill=gray!20,draw=none](-7.516,4.578)--(-7.508,4.6)--(-7.478,4.628)--(-7.496,4.582)--cycle;
\draw(-7.516,4.578)--(-7.508,4.6);
\draw(-7.478,4.628)--(-7.496,4.582);
\filldraw[fill opacity=0.8,fill=gray!20,draw=none](-7.479,4.629)--(-7.499,4.583)--(-7.52,4.58)--(-7.51,4.601)--cycle;
\draw(-7.479,4.629)--(-7.499,4.583);
\draw(-7.52,4.58)--(-7.51,4.601);
\filldraw[fill opacity=0.8,fill=gray!20,draw=none](-7.512,4.618)--(-7.516,4.611)--(-7.521,4.607)--(-7.512,4.63)--cycle;
\draw(-7.521,4.607)--(-7.512,4.63);
\filldraw[fill opacity=0.8,fill=gray!20,draw=none](-7.506,4.627)--(-7.512,4.618)--(-7.512,4.63)--cycle;
\draw(-7.512,4.63)--(-7.506,4.627);
\filldraw[fill opacity=0.8,fill=gray!20,draw=none](-7.512,4.618)--(-7.511,4.615)--(-7.516,4.611)--cycle;
\filldraw[fill opacity=0.8,fill=gray!20,draw=none](-7.507,4.618)--(-7.513,4.614)--(-7.513,4.617)--(-7.51,4.619)--cycle;
\draw(-7.513,4.617)--(-7.51,4.619);
\filldraw[fill opacity=0.8,fill=gray!20,draw=none](-7.525,4.647)--(-7.531,4.681)--(-7.553,4.728)--(-7.585,4.765)--(-7.623,4.788)--(-7.66,4.793)--(-7.691,4.778)--(-7.705,4.756)--cycle;
\draw(-7.525,4.647)--(-7.531,4.681)--(-7.553,4.728)--(-7.585,4.765)--(-7.623,4.788)--(-7.66,4.793)--(-7.691,4.778)--(-7.705,4.756);
\filldraw[fill opacity=0.8,fill=gray!20,draw=none](-7.504,4.626)--(-7.504,4.625)--(-7.51,4.619)--(-7.62,4.669)--cycle;
\draw(-7.504,4.626)--(-7.504,4.625)--(-7.51,4.619);
\filldraw[fill opacity=0.8,fill=gray!20,draw=none](-7.507,4.618)--(-7.51,4.619)--(-7.504,4.625)--(-7.504,4.62)--cycle;
\draw(-7.51,4.619)--(-7.504,4.625)--(-7.504,4.62);
\filldraw[fill opacity=0.8,fill=gray!20,draw=none](-7.499,4.639)--(-7.487,4.633)--(-7.486,4.623)--(-7.499,4.588)--(-7.501,4.587)--(-7.504,4.625)--cycle;
\draw(-7.487,4.633)--(-7.486,4.623);
\draw(-7.501,4.587)--(-7.504,4.625)--(-7.499,4.639);
\filldraw[fill opacity=0.8,fill=gray!20,draw=none](-7.506,4.627)--(-7.502,4.625)--(-7.511,4.615)--(-7.512,4.618)--cycle;
\draw(-7.506,4.627)--(-7.502,4.625);
\filldraw[fill opacity=0.8,fill=gray!20,draw=none](-7.511,4.615)--(-7.512,4.588)--(-7.529,4.584)--(-7.521,4.607)--cycle;
\draw(-7.529,4.584)--(-7.521,4.607);
\filldraw[fill opacity=0.8,fill=gray!20,draw=none](-7.513,4.614)--(-7.507,4.618)--(-7.504,4.617)--(-7.503,4.607)--(-7.511,4.588)--(-7.519,4.588)--cycle;
\draw(-7.504,4.617)--(-7.503,4.607);
\filldraw[fill opacity=0.8,fill=gray!20,draw=none](-7.507,4.618)--(-7.504,4.62)--(-7.504,4.617)--cycle;
\draw(-7.504,4.62)--(-7.504,4.617);
\filldraw[fill opacity=0.8,fill=gray!20,draw=none](-7.511,4.615)--(-7.502,4.625)--(-7.498,4.623)--(-7.511,4.588)--(-7.512,4.588)--cycle;
\draw(-7.502,4.625)--(-7.498,4.623)--(-7.511,4.588);
\filldraw[fill opacity=0.8,fill=gray!20,draw=none](-7.526,4.627)--(-7.504,4.617)--(-7.501,4.615)--(-7.513,4.621)--(-7.522,4.624)--(-7.546,4.635)--(-7.588,4.653)--(-7.634,4.674)--(-7.679,4.694)--(-7.561,4.643)--cycle;
\draw(-7.522,4.624)--(-7.546,4.635)--(-7.588,4.653)--(-7.634,4.674)--(-7.679,4.694);
\draw(-7.561,4.643)--(-7.526,4.627)--(-7.504,4.617)--(-7.501,4.615)--(-7.513,4.621);
\filldraw[fill opacity=0.8,fill=gray!20,draw=none](-7.8,4.726)--(-7.798,4.743)--(-7.771,4.739)--cycle;
\draw(-7.8,4.726)--(-7.798,4.743);
\filldraw[fill opacity=0.8,fill=gray!20](-8.402,2.82)--(-8.446,2.848)--(-8.403,2.856)--(-8.38,2.825)--cycle;
\filldraw[fill opacity=0.8,fill=gray!20,draw=none](-7.686,4.321)--(-7.728,4.338)--(-7.702,4.328)--cycle;
\filldraw[fill opacity=0.8,fill=gray!20,draw=none](-7.745,4.299)--(-7.718,4.361)--(-7.683,4.352)--(-7.679,4.335)--(-7.692,4.306)--cycle;
\draw(-7.745,4.299)--(-7.718,4.361);
\draw(-7.679,4.335)--(-7.692,4.306);
\filldraw[fill opacity=0.8,fill=gray!20,draw=none](-7.979,4.036)--(-7.929,4.229)--(-7.859,4.388)--(-7.822,4.374)--(-7.975,4.03)--cycle;
\draw(-7.929,4.229)--(-7.859,4.388);
\draw(-7.822,4.374)--(-7.975,4.03);
\filldraw[fill opacity=0.8,fill=gray!20,draw=none](-7.679,4.318)--(-7.655,4.308)--(-7.68,4.252)--cycle;
\draw(-7.655,4.308)--(-7.68,4.252);
\filldraw[fill opacity=0.8,fill=gray!20,draw=none](-7.876,4.393)--(-7.859,4.388)--(-7.929,4.229)--cycle;
\draw(-7.859,4.388)--(-7.929,4.229);
\filldraw[fill opacity=0.8,fill=gray!20,draw=none](-8.025,4.055)--(-8.03,4.061)--(-7.881,4.395)--(-7.876,4.393)--(-7.929,4.229)--(-8.006,4.057)--cycle;
\draw(-8.03,4.061)--(-7.881,4.395);
\draw(-7.929,4.229)--(-8.006,4.057);
\filldraw[fill opacity=0.8,fill=gray!20,draw=none](-7.812,3.956)--(-7.655,4.308)--(-7.648,4.304)--(-7.647,4.286)--(-7.787,3.973)--cycle;
\draw(-7.812,3.956)--(-7.655,4.308);
\draw(-7.647,4.286)--(-7.787,3.973);
\filldraw[fill opacity=0.8,fill=gray!20,draw=none](-7.644,4.302)--(-7.648,4.297)--(-7.648,4.304)--cycle;
\filldraw[fill opacity=0.8,fill=gray!20,draw=none](-7.648,4.297)--(-7.644,4.302)--(-7.641,4.301)--(-7.647,4.286)--cycle;
\draw(-7.641,4.301)--(-7.647,4.286);
\filldraw[fill opacity=0.8,fill=gray!20,draw=none](-7.812,3.956)--(-7.787,3.973)--(-7.8,3.943)--cycle;
\draw(-7.787,3.973)--(-7.8,3.943);
\filldraw[fill opacity=0.8,fill=gray!20,draw=none](-7.8,3.954)--(-7.787,3.972)--(-7.641,4.301)--(-7.645,4.301)--(-7.8,3.954)--cycle;
\draw(-7.787,3.972)--(-7.641,4.301);
\draw(-7.645,4.301)--(-7.8,3.954);
\filldraw[fill opacity=0.8,fill=gray!20,draw=none](-7.913,3.811)--(-7.845,3.962)--(-7.812,3.956)--(-7.882,3.797)--cycle;
\draw(-7.812,3.956)--(-7.882,3.797)--(-7.913,3.811)--(-7.845,3.962);
\filldraw[fill opacity=0.8,fill=gray!20,draw=none](-7.698,4.183)--(-7.645,4.301)--(-7.667,4.309)--(-7.703,4.228)--cycle;
\draw(-7.698,4.183)--(-7.645,4.301);
\draw(-7.667,4.309)--(-7.703,4.228);
\filldraw[fill opacity=0.8,fill=gray!20,draw=none](-7.68,4.313)--(-7.731,4.333)--(-7.784,4.353)--(-7.831,4.372)--(-7.865,4.387)--(-7.881,4.394)--(-7.877,4.394)--(-7.852,4.385)--(-7.846,4.383)--(-7.776,4.356)--(-7.728,4.338)--(-7.622,4.295)--(-7.61,4.289)--(-7.615,4.29)--(-7.639,4.298)--cycle;
\draw(-7.622,4.295)--(-7.61,4.289)--(-7.615,4.29)--(-7.639,4.298)--(-7.68,4.313)--(-7.731,4.333)--(-7.784,4.353)--(-7.831,4.372)--(-7.865,4.387)--(-7.881,4.394)--(-7.877,4.394)--(-7.852,4.385)--(-7.846,4.383);
\filldraw[fill opacity=0.8,fill=gray!20,draw=none](-7.781,4.753)--(-7.797,4.75)--(-7.779,4.799)--(-7.72,4.81)--cycle;
\draw(-7.781,4.753)--(-7.797,4.75)--(-7.779,4.799)--(-7.72,4.81);
\filldraw[fill opacity=0.8,fill=gray!20,draw=none](-8.002,3.85)--(-7.933,4.003)--(-7.883,3.989)--(-7.955,3.829)--cycle;
\draw(-7.883,3.989)--(-7.955,3.829)--(-8.002,3.85)--(-7.933,4.003);
\filldraw[fill opacity=0.8,fill=gray!20,draw=none](-8.029,3.863)--(-7.97,4.027)--(-7.933,4.003)--(-8.002,3.85)--cycle;
\draw(-7.933,4.003)--(-8.002,3.85)--(-8.029,3.863);
\filldraw[fill opacity=0.8,fill=gray!20,draw=none](-8.364,3.187)--(-8.409,3.193)--(-8.424,3.202)--(-8.371,3.205)--cycle;
\draw(-8.424,3.202)--(-8.371,3.205);
\filldraw[fill opacity=0.8,fill=gray!20,draw=none](-7.883,3.989)--(-7.745,4.299)--(-7.692,4.306)--(-7.833,3.991)--cycle;
\draw(-7.883,3.989)--(-7.745,4.299);
\draw(-7.692,4.306)--(-7.833,3.991);
\filldraw[fill opacity=0.8,fill=gray!20,draw=none](-7.883,3.989)--(-7.882,3.989)--(-7.875,3.984)--cycle;
\filldraw[fill opacity=0.8,fill=gray!20](-7.882,3.986)--(-7.902,4.024)--(-7.848,4.011)--(-7.817,3.97)--cycle;
\filldraw[fill opacity=0.8,fill=gray!20,draw=none](-7.955,3.829)--(-7.883,3.989)--(-7.875,3.984)--(-7.845,3.962)--(-7.913,3.811)--cycle;
\draw(-7.845,3.962)--(-7.913,3.811)--(-7.955,3.829)--(-7.883,3.989);
\filldraw[fill opacity=0.8,fill=gray!20,draw=none](-7.97,4.027)--(-8.029,3.863)--(-8.046,3.87)--(-7.975,4.03)--cycle;
\draw(-8.029,3.863)--(-8.046,3.87)--(-7.975,4.03);
\filldraw[fill opacity=0.8,fill=gray!20](-7.928,3.819)--(-7.893,3.803)--(-7.871,3.793)--(-7.868,3.791)--(-7.882,3.797)--(-7.913,3.811)--(-7.955,3.829)--(-8.002,3.85)--(-8.046,3.87)--(-8.082,3.886)--(-8.103,3.896)--(-8.107,3.898)--(-8.092,3.892)--(-8.062,3.878)--(-8.02,3.86)--(-7.973,3.839)--cycle;
\filldraw[fill opacity=0.8,fill=gray!20](-7.928,3.819)--(-7.973,3.839)--(-8.02,3.86)--(-8.062,3.878)--(-8.092,3.892)--(-8.107,3.898)--(-8.103,3.896)--(-8.082,3.886)--(-8.046,3.87)--(-8.002,3.85)--(-7.955,3.829)--(-7.913,3.811)--(-7.882,3.797)--(-7.868,3.791)--(-7.871,3.793)--(-7.893,3.803)--cycle;
\filldraw[fill opacity=0.8,fill=gray!20,draw=none](-7.955,3.829)--(-8.025,3.671)--(-8.03,3.672)--(-8.047,3.678)--(-8.071,3.695)--(-8.002,3.85)--cycle;
\draw(-8.071,3.695)--(-8.002,3.85)--(-7.955,3.829)--(-8.025,3.671);
\filldraw[fill opacity=0.8,fill=gray!20,draw=none](-8.118,3.711)--(-8.117,3.712)--(-8.113,3.708)--cycle;
\draw(-8.118,3.711)--(-8.117,3.712)--(-8.113,3.708);
\filldraw[fill opacity=0.8,fill=gray!20,draw=none](-8.002,3.85)--(-8.071,3.695)--(-8.118,3.709)--(-8.046,3.87)--cycle;
\draw(-8.118,3.709)--(-8.046,3.87)--(-8.002,3.85)--(-8.071,3.695);
\filldraw[fill opacity=0.8,fill=gray!20,draw=none](-6.946,.368)--(-7.115,.37)--(-7.114,.422)--(-6.945,.42)--cycle;
\draw(-7.114,.422)--(-6.945,.42)--(-6.946,.368)--(-7.115,.37);
\filldraw[fill opacity=0.8,fill=gray!20,draw=none](-7.118,.322)--(-7.122,.34)--(-7.118,.33)--cycle;
\draw(-7.122,.34)--(-7.118,.33);
\filldraw[fill opacity=0.8,fill=gray!20,draw=none](-7.115,.304)--(-7.118,.322)--(-7.118,.33)--(-7.11,.307)--cycle;
\draw(-7.118,.33)--(-7.11,.307)--(-7.115,.304);
\filldraw[fill opacity=0.8,fill=gray!20,draw=none](-6.947,.32)--(-7.124,.322)--(-7.119,.363)--(-7.115,.37)--(-6.946,.368)--cycle;
\draw(-7.115,.37)--(-6.946,.368)--(-6.947,.32)--(-7.124,.322);
\filldraw[fill opacity=0.8,fill=gray!20](-7.882,.247)--(-7.92,.275)--(-7.898,.289)--(-7.867,.256)--cycle;
\filldraw[fill opacity=0.8,fill=gray!20](-7.783,.554)--(-7.785,.575)--(-7.74,.571)--(-7.719,.55)--cycle;
\filldraw[fill opacity=0.8,fill=gray!20](-6.999,.191)--(-7.043,.219)--(-7.001,.227)--(-6.977,.195)--cycle;
\filldraw[fill opacity=0.8,fill=gray!20,draw=none](-7.481,4.566)--(-7.454,4.573)--(-7.467,4.553)--cycle;
\draw(-7.454,4.573)--(-7.467,4.553);
\filldraw[fill opacity=0.8,fill=gray!20,draw=none](-6.945,.42)--(-7.114,.422)--(-7.115,.469)--(-6.946,.467)--cycle;
\draw(-7.115,.469)--(-6.946,.467)--(-6.945,.42)--(-7.114,.422);
\filldraw[fill opacity=0.8,fill=gray!20,draw=none](-8.26,3.168)--(-8.302,3.176)--(-8.301,3.179)--cycle;
\draw(-8.302,3.176)--(-8.301,3.179);
\filldraw[fill opacity=0.8,fill=gray!20,draw=none](-8.301,3.179)--(-8.302,3.176)--(-8.314,3.177)--(-8.317,3.19)--cycle;
\draw(-8.301,3.179)--(-8.302,3.176);
\filldraw[fill opacity=0.8,fill=gray!20,draw=none](-7.913,3.811)--(-7.984,3.651)--(-8.025,3.671)--(-7.955,3.829)--cycle;
\draw(-8.025,3.671)--(-7.955,3.829)--(-7.913,3.811)--(-7.984,3.651);
\filldraw[fill opacity=0.8,fill=gray!20](-8.199,3.792)--(-8.206,3.847)--(-8.17,3.871)--(-8.164,3.815)--cycle;
\filldraw[fill opacity=0.8,fill=gray!20](-8.177,3.74)--(-8.199,3.792)--(-8.164,3.815)--(-8.146,3.76)--cycle;
\filldraw[fill opacity=0.8,fill=gray!20,draw=none](-8.117,3.71)--(-8.22,3.479)--(-8.183,3.658)--(-8.15,3.733)--cycle;
\draw(-8.117,3.71)--(-8.22,3.479);
\draw(-8.183,3.658)--(-8.15,3.733);
\filldraw[fill opacity=0.8,fill=gray!20,draw=none](-8.15,3.733)--(-8.183,3.658)--(-8.163,3.735)--cycle;
\draw(-8.15,3.733)--(-8.183,3.658);
\filldraw[fill opacity=0.8,fill=gray!20,draw=none](-8.163,3.735)--(-8.166,3.724)--(-8.174,3.736)--cycle;
\filldraw[fill opacity=0.8,fill=gray!20](-8.142,3.695)--(-8.177,3.74)--(-8.146,3.76)--(-8.117,3.712)--cycle;
\filldraw[fill opacity=0.8,fill=gray!20,draw=none](-6.159,.551)--(-6.159,.549)--(-6.156,.55)--cycle;
\filldraw[fill opacity=0.8,fill=gray!20,draw=none](-6.156,.55)--(-6.159,.549)--(-6.15,.563)--cycle;
\draw(-6.159,.549)--(-6.15,.563);
\filldraw[fill opacity=0.8,fill=gray!20,draw=none](-6.176,.514)--(-6.183,.514)--(-6.17,.534)--cycle;
\draw(-6.183,.514)--(-6.17,.534);
\filldraw[fill opacity=0.8,fill=gray!20,draw=none](-6.105,.631)--(-6.225,.45)--(-6.204,.471)--(-6.038,.72)--cycle;
\draw(-6.105,.631)--(-6.225,.45)--(-6.204,.471)--(-6.038,.72);
\filldraw[fill opacity=0.8,fill=gray!20,draw=none](-6.047,.211)--(-6.047,.175)--(-6.002,.183)--(-6.002,.208)--cycle;
\draw(-6.047,.211)--(-6.047,.175)--(-6.002,.183)--(-6.002,.208);
\filldraw[fill opacity=0.8,fill=gray!20,draw=none](-6.047,.211)--(-6.002,.208)--(-6.002,.229)--cycle;
\draw(-6.002,.208)--(-6.002,.229);
\filldraw[fill opacity=0.8,fill=gray!20,draw=none](-6.036,.206)--(-6.178,.268)--(-6.123,.282)--(-6.002,.229)--cycle;
\draw(-6.036,.206)--(-6.178,.268);
\draw(-6.123,.282)--(-6.002,.229);
\filldraw[fill opacity=0.8,fill=gray!20,draw=none](-8.297,2.844)--(-8.311,2.854)--(-8.303,2.85)--cycle;
\draw(-8.311,2.854)--(-8.303,2.85);
\filldraw[fill opacity=0.8,fill=gray!20,draw=none](-6.163,.233)--(-6.178,.232)--(-6.178,.242)--cycle;
\draw(-6.163,.233)--(-6.178,.232)--(-6.178,.242);
\filldraw[fill opacity=0.8,fill=gray!20,draw=none](-8.458,3.183)--(-8.427,3.195)--(-8.432,3.183)--cycle;
\draw(-8.427,3.195)--(-8.432,3.183);
\filldraw[fill opacity=0.8,fill=gray!20,draw=none](-8.474,3.167)--(-8.451,3.183)--(-8.432,3.183)--cycle;
\filldraw[fill opacity=0.8,fill=gray!20,draw=none](-8.481,3.17)--(-8.514,3.159)--(-8.486,3.183)--(-8.469,3.188)--(-8.461,3.187)--cycle;
\draw(-8.514,3.159)--(-8.486,3.183)--(-8.469,3.188);
\filldraw[fill opacity=0.8,fill=gray!20,draw=none](-8.519,3.138)--(-8.506,3.162)--(-8.481,3.17)--cycle;
\filldraw[fill opacity=0.8,fill=gray!20,draw=none](-8.474,3.167)--(-8.505,3.155)--(-8.474,3.177)--(-8.458,3.183)--(-8.451,3.183)--cycle;
\filldraw[fill opacity=0.8,fill=gray!20,draw=none](-8.431,3.172)--(-8.391,3.165)--(-8.421,3.163)--(-8.44,3.171)--cycle;
\draw(-8.421,3.163)--(-8.44,3.171);
\filldraw[fill opacity=0.8,fill=gray!20,draw=none](-8.425,3.171)--(-8.421,3.163)--(-8.433,3.155)--(-8.426,3.17)--cycle;
\draw(-8.433,3.155)--(-8.426,3.17);
\filldraw[fill opacity=0.8,fill=gray!20,draw=none](-8.453,3.168)--(-8.44,3.171)--(-8.431,3.167)--(-8.451,3.157)--(-8.491,3.149)--cycle;
\draw(-8.44,3.171)--(-8.431,3.167);
\filldraw[fill opacity=0.8,fill=gray!20,draw=none](-8.425,3.172)--(-8.425,3.171)--(-8.431,3.172)--cycle;
\filldraw[fill opacity=0.8,fill=gray!20,draw=none](-8.425,3.171)--(-8.426,3.17)--(-8.426,3.172)--cycle;
\draw(-8.426,3.17)--(-8.426,3.172);
\filldraw[fill opacity=0.8,fill=gray!20,draw=none](-8.426,3.172)--(-8.426,3.17)--(-8.427,3.17)--cycle;
\draw(-8.426,3.172)--(-8.426,3.17);
\filldraw[fill opacity=0.8,fill=gray!20,draw=none](-8.434,3.16)--(-8.427,3.17)--(-8.426,3.17)--(-8.43,3.162)--(-8.436,3.15)--(-8.446,3.136)--(-8.443,3.144)--cycle;
\draw(-8.426,3.17)--(-8.43,3.162);
\draw(-8.446,3.136)--(-8.443,3.144);
\filldraw[fill opacity=0.8,fill=gray!20,draw=none](-8.43,3.162)--(-8.433,3.155)--(-8.436,3.15)--cycle;
\draw(-8.43,3.162)--(-8.433,3.155);
\filldraw[fill opacity=0.8,fill=gray!20,draw=none](-8.41,3.097)--(-8.429,3.054)--(-8.387,3.036)--(-8.347,3.125)--cycle;
\draw(-8.41,3.097)--(-8.429,3.054)--(-8.387,3.036)--(-8.347,3.125);
\filldraw[fill opacity=0.8,fill=gray!20,draw=none](-8.438,3.102)--(-8.423,3.096)--(-8.389,3.099)--(-8.36,3.117)--(-8.399,3.134)--cycle;
\draw(-8.36,3.117)--(-8.399,3.134)--(-8.438,3.102)--(-8.423,3.096);
\filldraw[fill opacity=0.8,fill=gray!20,draw=none](-8.309,3.084)--(-8.34,3.015)--(-8.295,2.995)--(-8.27,3.051)--cycle;
\draw(-8.309,3.084)--(-8.34,3.015)--(-8.295,2.995)--(-8.27,3.051);
\filldraw[fill opacity=0.8,fill=gray!20,draw=none](-8.438,3.102)--(-8.423,3.096)--(-8.389,3.099)--(-8.36,3.117)--(-8.399,3.134)--cycle;
\draw(-8.36,3.117)--(-8.399,3.134)--(-8.438,3.102)--(-8.423,3.096);
\filldraw[fill opacity=0.8,fill=gray!20,draw=none](-8.375,3.134)--(-8.398,3.124)--(-8.401,3.117)--(-8.362,3.118)--(-8.347,3.125)--(-8.342,3.135)--cycle;
\draw(-8.398,3.124)--(-8.401,3.117);
\draw(-8.347,3.125)--(-8.342,3.135);
\filldraw[fill opacity=0.8,fill=gray!20,draw=none](-8.328,3.122)--(-8.338,3.131)--(-8.347,3.125)--(-8.387,3.036)--(-8.34,3.015)--(-8.309,3.084)--cycle;
\draw(-8.347,3.125)--(-8.387,3.036)--(-8.34,3.015)--(-8.309,3.084);
\filldraw[fill opacity=0.8,fill=gray!20,draw=none](-8.35,3.119)--(-8.36,3.117)--(-8.348,3.112)--(-8.334,3.119)--cycle;
\draw(-8.36,3.117)--(-8.348,3.112);
\filldraw[fill opacity=0.8,fill=gray!20,draw=none](-8.35,3.119)--(-8.36,3.117)--(-8.348,3.112)--(-8.334,3.119)--cycle;
\draw(-8.36,3.117)--(-8.348,3.112);
\filldraw[fill opacity=0.8,fill=gray!20,draw=none](-8.326,3.138)--(-8.342,3.135)--(-8.347,3.125)--cycle;
\draw(-8.342,3.135)--(-8.347,3.125);
\filldraw[fill opacity=0.8,fill=gray!20,draw=none](-8.35,3.119)--(-8.334,3.119)--(-8.328,3.122)--cycle;
\filldraw[fill opacity=0.8,fill=gray!20,draw=none](-8.35,3.119)--(-8.334,3.119)--(-8.328,3.122)--cycle;
\filldraw[fill opacity=0.8,fill=gray!20,draw=none](-8.342,3.135)--(-8.341,3.134)--(-8.34,3.134)--cycle;
\filldraw[fill opacity=0.8,fill=gray!20,draw=none](-8.342,3.135)--(-8.344,3.134)--(-8.341,3.134)--cycle;
\filldraw[fill opacity=0.8,fill=gray!20,draw=none](-8.34,3.134)--(-8.341,3.134)--(-8.335,3.125)--(-8.329,3.122)--cycle;
\draw(-8.335,3.125)--(-8.329,3.122);
\filldraw[fill opacity=0.8,fill=gray!20,draw=none](-8.334,3.134)--(-8.338,3.131)--(-8.328,3.122)--cycle;
\filldraw[fill opacity=0.8,fill=gray!20,draw=none](-8.341,3.134)--(-8.344,3.134)--(-8.347,3.13)--(-8.335,3.125)--cycle;
\draw(-8.347,3.13)--(-8.335,3.125);
\filldraw[fill opacity=0.8,fill=gray!20,draw=none](-8.328,3.122)--(-8.309,3.084)--(-8.303,3.098)--cycle;
\draw(-8.309,3.084)--(-8.303,3.098);
\filldraw[fill opacity=0.8,fill=gray!20,draw=none](-8.33,3.102)--(-8.309,3.084)--(-8.328,3.122)--(-8.335,3.125)--cycle;
\draw(-8.328,3.122)--(-8.335,3.125);
\filldraw[fill opacity=0.8,fill=gray!20,draw=none](-8.344,3.114)--(-8.33,3.102)--(-8.334,3.119)--cycle;
\filldraw[fill opacity=0.8,fill=gray!20,draw=none](-8.328,3.122)--(-8.344,3.114)--(-8.301,3.126)--(-8.303,3.127)--cycle;
\draw(-8.301,3.126)--(-8.303,3.127);
\filldraw[fill opacity=0.8,fill=gray!20,draw=none](-8.328,3.122)--(-8.344,3.114)--(-8.301,3.126)--(-8.303,3.127)--cycle;
\draw(-8.301,3.126)--(-8.303,3.127);
\filldraw[fill opacity=0.8,fill=gray!20,draw=none](-8.303,3.127)--(-8.301,3.126)--(-8.291,3.124)--cycle;
\draw(-8.303,3.127)--(-8.301,3.126);
\filldraw[fill opacity=0.8,fill=gray!20,draw=none](-8.303,3.127)--(-8.301,3.126)--(-8.291,3.124)--cycle;
\draw(-8.303,3.127)--(-8.301,3.126);
\filldraw[fill opacity=0.8,fill=gray!20,draw=none](-8.346,3.115)--(-8.344,3.114)--(-8.334,3.119)--(-8.335,3.125)--(-8.338,3.126)--cycle;
\draw(-8.335,3.125)--(-8.338,3.126);
\filldraw[fill opacity=0.8,fill=gray!20,draw=none](-8.309,3.084)--(-8.344,3.114)--(-8.348,3.112)--(-8.365,3.086)--(-8.295,3.056)--cycle;
\draw(-8.365,3.086)--(-8.295,3.056);
\filldraw[fill opacity=0.8,fill=gray!20,draw=none](-8.261,3.098)--(-8.242,3.112)--(-8.277,3.115)--(-8.272,3.101)--cycle;
\draw(-8.277,3.115)--(-8.272,3.101);
\filldraw[fill opacity=0.8,fill=gray!20,draw=none](-8.344,3.114)--(-8.348,3.112)--(-8.324,3.102)--(-8.277,3.115)--(-8.301,3.126)--cycle;
\draw(-8.348,3.112)--(-8.324,3.102);
\draw(-8.277,3.115)--(-8.301,3.126);
\filldraw[fill opacity=0.8,fill=gray!20,draw=none](-8.344,3.114)--(-8.348,3.112)--(-8.332,3.105)--(-8.29,3.121)--(-8.301,3.126)--cycle;
\draw(-8.348,3.112)--(-8.332,3.105);
\draw(-8.29,3.121)--(-8.301,3.126);
\filldraw[fill opacity=0.8,fill=gray!20,draw=none](-8.35,3.119)--(-8.362,3.118)--(-8.36,3.117)--cycle;
\draw(-8.362,3.118)--(-8.36,3.117);
\filldraw[fill opacity=0.8,fill=gray!20,draw=none](-8.35,3.119)--(-8.362,3.118)--(-8.36,3.117)--cycle;
\draw(-8.362,3.118)--(-8.36,3.117);
\filldraw[fill opacity=0.8,fill=gray!20,draw=none](-8.389,3.099)--(-8.355,3.101)--(-8.348,3.112)--(-8.36,3.117)--cycle;
\draw(-8.348,3.112)--(-8.36,3.117);
\filldraw[fill opacity=0.8,fill=gray!20,draw=none](-8.389,3.099)--(-8.355,3.101)--(-8.348,3.112)--(-8.36,3.117)--cycle;
\draw(-8.348,3.112)--(-8.36,3.117);
\filldraw[fill opacity=0.8,fill=gray!20,draw=none](-8.355,3.126)--(-8.355,3.101)--(-8.338,3.126)--(-8.347,3.13)--cycle;
\draw(-8.338,3.126)--(-8.347,3.13);
\filldraw[fill opacity=0.8,fill=gray!20,draw=none](-8.332,3.105)--(-8.324,3.102)--(-8.277,3.115)--(-8.29,3.121)--cycle;
\draw(-8.332,3.105)--(-8.324,3.102);
\draw(-8.277,3.115)--(-8.29,3.121);
\filldraw[fill opacity=0.8,fill=gray!20,draw=none](-8.324,3.102)--(-8.277,3.115)--(-8.282,3.132)--(-8.3,3.127)--cycle;
\draw(-8.277,3.115)--(-8.282,3.132);
\filldraw[fill opacity=0.8,fill=gray!20,draw=none](-8.211,3.022)--(-8.177,3.007)--(-8.177,3.004)--(-8.181,2.955)--(-8.229,2.976)--cycle;
\draw(-8.211,3.022)--(-8.177,3.007);
\draw(-8.181,2.955)--(-8.229,2.976);
\filldraw[fill opacity=0.8,fill=gray!20,draw=none](-8.211,3.022)--(-8.176,3.007)--(-8.181,2.955)--(-8.229,2.976)--cycle;
\draw(-8.211,3.022)--(-8.176,3.007);
\draw(-8.181,2.955)--(-8.229,2.976);
\filldraw[fill opacity=0.8,fill=gray!20,draw=none](-8.285,3.051)--(-8.247,3.035)--(-8.228,3.049)--cycle;
\draw(-8.285,3.051)--(-8.247,3.035);
\filldraw[fill opacity=0.8,fill=gray!20,draw=none](-8.322,3.104)--(-8.3,3.127)--(-8.322,3.12)--cycle;
\filldraw[fill opacity=0.8,fill=gray!20,draw=none](-8.371,3.028)--(-8.243,2.972)--(-8.219,3.023)--(-8.355,3.082)--cycle;
\draw(-8.371,3.028)--(-8.243,2.972);
\draw(-8.219,3.023)--(-8.355,3.082);
\filldraw[fill opacity=0.8,fill=gray!20,draw=none](-8.376,3.075)--(-8.393,3.037)--(-8.371,3.028)--(-8.355,3.082)--(-8.365,3.086)--cycle;
\draw(-8.393,3.037)--(-8.371,3.028);
\draw(-8.355,3.082)--(-8.365,3.086);
\filldraw[fill opacity=0.8,fill=gray!20,draw=none](-8.324,3.102)--(-8.338,3.087)--(-8.296,3.089)--(-8.272,3.101)--(-8.277,3.115)--cycle;
\draw(-8.338,3.087)--(-8.296,3.089);
\draw(-8.272,3.101)--(-8.277,3.115);
\filldraw[fill opacity=0.8,fill=gray!20,draw=none](-8.376,3.075)--(-8.361,3.069)--(-8.323,3.101)--(-8.342,3.109)--cycle;
\draw(-8.376,3.075)--(-8.361,3.069);
\draw(-8.323,3.101)--(-8.342,3.109);
\filldraw[fill opacity=0.8,fill=gray!20,draw=none](-8.349,3.102)--(-8.329,3.103)--(-8.342,3.109)--cycle;
\draw(-8.329,3.103)--(-8.342,3.109);
\filldraw[fill opacity=0.8,fill=gray!20,draw=none](-8.349,3.102)--(-8.365,3.086)--(-8.373,3.074)--(-8.361,3.069)--(-8.323,3.101)--(-8.329,3.103)--cycle;
\draw(-8.373,3.074)--(-8.361,3.069);
\draw(-8.323,3.101)--(-8.329,3.103);
\filldraw[fill opacity=0.8,fill=gray!20,draw=none](-8.348,3.104)--(-8.352,3.087)--(-8.338,3.087)--(-8.322,3.104)--(-8.322,3.12)--(-8.335,3.117)--cycle;
\draw(-8.352,3.087)--(-8.338,3.087);
\filldraw[fill opacity=0.8,fill=gray!20](-8.272,3.128)--(-8.436,3.199)--(-8.486,3.183)--(-8.323,3.111)--cycle;
\filldraw[fill opacity=0.8,fill=gray!20,draw=none](-8.317,3.19)--(-8.314,3.177)--(-8.353,3.182)--(-8.342,3.206)--cycle;
\draw(-8.353,3.182)--(-8.342,3.206);
\filldraw[fill opacity=0.8,fill=gray!20,draw=none](-8.2,3.15)--(-8.242,3.16)--(-8.251,3.166)--cycle;
\draw(-8.2,3.15)--(-8.242,3.16);
\filldraw[fill opacity=0.8,fill=gray!20,draw=none](-8.2,3.167)--(-8.205,3.155)--(-8.25,3.165)--cycle;
\draw(-8.2,3.167)--(-8.205,3.155);
\filldraw[fill opacity=0.8,fill=gray!20,draw=none](-8.2,3.15)--(-8.251,3.166)--(-8.269,3.2)--(-8.215,3.187)--(-8.184,3.146)--cycle;
\draw(-8.251,3.166)--(-8.269,3.2)--(-8.215,3.187)--(-8.184,3.146)--(-8.2,3.15);
\filldraw[fill opacity=0.8,fill=gray!20](-6.92,.194)--(-6.891,.226)--(-6.853,.216)--(-6.901,.19)--cycle;
\filldraw[fill opacity=0.8,fill=gray!20](-6.847,.533)--(-6.866,.571)--(-6.813,.558)--(-6.782,.517)--cycle;
\filldraw[fill opacity=0.8,fill=gray!20,draw=none](-7.737,4.537)--(-7.742,4.539)--(-7.738,4.538)--cycle;
\draw(-7.742,4.539)--(-7.738,4.538);
\filldraw[fill opacity=0.8,fill=gray!20,draw=none](-7.773,4.468)--(-7.77,4.47)--(-7.779,4.469)--cycle;
\filldraw[fill opacity=0.8,fill=gray!20,draw=none](-8.55,2.978)--(-8.536,2.987)--(-8.523,2.965)--(-8.52,2.958)--cycle;
\draw(-8.55,2.978)--(-8.536,2.987);
\draw(-8.523,2.965)--(-8.52,2.958);
\filldraw[fill opacity=0.8,fill=gray!20,draw=none](-6.945,.42)--(-6.778,.418)--(-6.777,.367)--(-6.946,.368)--cycle;
\draw(-6.777,.367)--(-6.946,.368)--(-6.945,.42)--(-6.778,.418);
\filldraw[fill opacity=0.8,fill=gray!20,draw=none](-6.775,.319)--(-6.779,.3)--(-6.782,.303)--(-6.774,.322)--cycle;
\draw(-6.779,.3)--(-6.782,.303)--(-6.774,.322);
\filldraw[fill opacity=0.8,fill=gray!20,draw=none](-6.775,.319)--(-6.774,.322)--(-6.773,.327)--cycle;
\draw(-6.774,.322)--(-6.773,.327);
\filldraw[fill opacity=0.8,fill=gray!20,draw=none](-6.946,.368)--(-6.777,.367)--(-6.774,.359)--(-6.772,.319)--(-6.947,.32)--cycle;
\draw(-6.772,.319)--(-6.947,.32)--(-6.946,.368)--(-6.777,.367);
\filldraw[fill opacity=0.8,fill=gray!20,draw=none](-7.12,.301)--(-7.121,.285)--(-7.153,.285)--(-7.124,.322)--cycle;
\draw(-7.121,.285)--(-7.153,.285);
\filldraw[fill opacity=0.8,fill=gray!20,draw=none](-7.116,.285)--(-7.121,.285)--(-7.12,.301)--cycle;
\draw(-7.116,.285)--(-7.121,.285);
\filldraw[fill opacity=0.8,fill=gray!20,draw=none](-7.125,.265)--(-7.142,.285)--(-7.116,.285)--cycle;
\draw(-7.142,.285)--(-7.116,.285);
\filldraw[fill opacity=0.8,fill=gray!20,draw=none](-7.125,.265)--(-7.141,.286)--(-7.11,.307)--(-7.108,.304)--cycle;
\draw(-7.125,.265)--(-7.141,.286)--(-7.11,.307)--(-7.108,.304);
\filldraw[fill opacity=0.8,fill=gray!20,draw=none](-7.124,.264)--(-7.125,.265)--(-7.108,.304)--(-7.086,.266)--cycle;
\draw(-7.124,.264)--(-7.125,.265);
\draw(-7.108,.304)--(-7.086,.266);
\filldraw[fill opacity=0.8,fill=gray!20,draw=none](-7.118,.322)--(-7.115,.304)--(-7.12,.301)--cycle;
\draw(-7.115,.304)--(-7.12,.301);
\filldraw[fill opacity=0.8,fill=gray!20,draw=none](-6.949,.283)--(-7.116,.285)--(-7.12,.301)--(-7.118,.322)--(-6.947,.32)--cycle;
\draw(-7.118,.322)--(-6.947,.32)--(-6.949,.283)--(-7.116,.285);
\filldraw[fill opacity=0.8,fill=gray!20,draw=none](-7.124,.504)--(-7.653,.51)--(-7.657,.515)--(-7.64,.528)--(-7.153,.523)--cycle;
\draw(-7.124,.504)--(-7.653,.51);
\draw(-7.64,.528)--(-7.153,.523);
\filldraw[fill opacity=0.8,fill=gray!20,draw=none](-7.122,.491)--(-7.124,.504)--(-7.118,.504)--cycle;
\draw(-7.124,.504)--(-7.118,.504);
\filldraw[fill opacity=0.8,fill=gray!20,draw=none](-7.12,.515)--(-7.11,.521)--(-7.118,.5)--cycle;
\draw(-7.12,.515)--(-7.11,.521)--(-7.118,.5);
\filldraw[fill opacity=0.8,fill=gray!20,draw=none](-7.124,.504)--(-7.153,.523)--(-7.116,.523)--cycle;
\draw(-7.153,.523)--(-7.116,.523);
\filldraw[fill opacity=0.8,fill=gray!20,draw=none](-7.141,.501)--(-7.125,.522)--(-7.108,.524)--(-7.11,.521)--cycle;
\draw(-7.108,.524)--(-7.11,.521)--(-7.141,.501)--(-7.125,.522);
\filldraw[fill opacity=0.8,fill=gray!20,draw=none](-6.947,.502)--(-7.124,.504)--(-7.116,.523)--(-6.949,.521)--cycle;
\draw(-7.116,.523)--(-6.949,.521)--(-6.947,.502)--(-7.124,.504);
\filldraw[fill opacity=0.8,fill=gray!20,draw=none](-7.163,.449)--(-7.141,.501)--(-7.12,.515)--(-7.118,.5)--(-7.129,.472)--cycle;
\draw(-7.118,.5)--(-7.129,.472)--(-7.163,.449)--(-7.141,.501)--(-7.12,.515);
\filldraw[fill opacity=0.8,fill=gray!20,draw=none](-6.946,.467)--(-7.115,.469)--(-7.119,.474)--(-7.122,.491)--(-7.118,.504)--(-6.947,.502)--cycle;
\draw(-7.118,.504)--(-6.947,.502)--(-6.946,.467)--(-7.115,.469);
\filldraw[fill opacity=0.8,fill=gray!20,draw=none](-7.107,.242)--(-7.124,.264)--(-7.086,.266)--(-7.081,.258)--cycle;
\draw(-7.086,.266)--(-7.081,.258)--(-7.107,.242)--(-7.124,.264);
\filldraw[fill opacity=0.8,fill=gray!20,draw=none](-6.952,.263)--(-7.124,.264)--(-7.125,.265)--(-7.116,.285)--(-6.949,.283)--cycle;
\draw(-7.116,.285)--(-6.949,.283)--(-6.952,.263)--(-7.124,.264);
\filldraw[fill opacity=0.8,fill=gray!20,draw=none](-6.954,.261)--(-7.121,.263)--(-7.124,.264)--(-6.952,.263)--cycle;
\draw(-7.124,.264)--(-6.952,.263)--(-6.954,.261)--(-7.121,.263);
\filldraw[fill opacity=0.8,fill=gray!20](-6.958,.362)--(-6.958,.414)--(-6.956,.462)--(-6.954,.499)--(-6.952,.52)--(-6.949,.521)--(-6.947,.502)--(-6.946,.467)--(-6.945,.42)--(-6.946,.368)--(-6.947,.32)--(-6.949,.283)--(-6.952,.263)--(-6.954,.261)--(-6.956,.28)--(-6.958,.315)--cycle;
\filldraw[fill opacity=0.8,fill=gray!20](-6.958,.362)--(-6.958,.315)--(-6.956,.28)--(-6.954,.261)--(-6.952,.263)--(-6.949,.283)--(-6.947,.32)--(-6.946,.368)--(-6.945,.42)--(-6.946,.467)--(-6.947,.502)--(-6.949,.521)--(-6.952,.52)--(-6.954,.499)--(-6.956,.462)--(-6.958,.414)--cycle;
\filldraw[fill opacity=0.8,fill=gray!20,draw=none](-7.664,.275)--(-7.666,.291)--(-7.662,.291)--cycle;
\draw(-7.666,.291)--(-7.662,.291);
\filldraw[fill opacity=0.8,fill=gray!20,draw=none](-7.663,.27)--(-7.664,.275)--(-7.662,.291)--(-7.634,.291)--cycle;
\draw(-7.662,.291)--(-7.634,.291);
\filldraw[fill opacity=0.8,fill=gray!20,draw=none](-7.67,.281)--(-7.66,.309)--(-7.649,.326)--(-7.632,.309)--(-7.661,.272)--cycle;
\draw(-7.66,.309)--(-7.649,.326)--(-7.632,.309)--(-7.661,.272)--(-7.67,.281);
\filldraw[fill opacity=0.8,fill=gray!20](-7.649,.326)--(-7.632,.371)--(-7.614,.351)--(-7.632,.309)--cycle;
\filldraw[fill opacity=0.8,fill=gray!20,draw=none](-7.71,4.489)--(-7.701,4.487)--(-7.698,4.489)--cycle;
\draw(-7.701,4.487)--(-7.698,4.489);
\filldraw[fill opacity=0.8,fill=gray!20,draw=none](-7.726,4.487)--(-7.694,4.401)--(-7.678,4.41)--(-7.675,4.413)--cycle;
\draw(-7.694,4.401)--(-7.678,4.41)--(-7.675,4.413);
\filldraw[fill opacity=0.8,fill=gray!20](-7.677,4.462)--(-7.73,4.484)--(-7.712,4.496)--(-7.668,4.468)--cycle;
\filldraw[fill opacity=0.8,fill=gray!20,draw=none](-7.704,4.505)--(-7.729,4.492)--(-7.698,4.489)--(-7.682,4.496)--cycle;
\draw(-7.698,4.489)--(-7.682,4.496);
\filldraw[fill opacity=0.8,fill=gray!20,draw=none](-7.72,4.491)--(-7.726,4.487)--(-7.728,4.492)--cycle;
\draw(-7.72,4.491)--(-7.726,4.487);
\filldraw[fill opacity=0.8,fill=gray!20,draw=none](-7.676,4.52)--(-7.685,4.515)--(-7.66,4.507)--(-7.644,4.514)--cycle;
\draw(-7.66,4.507)--(-7.644,4.514);
\filldraw[fill opacity=0.8,fill=gray!20,draw=none](-7.675,4.521)--(-7.676,4.52)--(-7.644,4.514)--(-7.629,4.521)--cycle;
\draw(-7.644,4.514)--(-7.629,4.521);
\filldraw[fill opacity=0.8,fill=gray!20,draw=none](-7.629,4.521)--(-7.628,4.508)--(-7.675,4.52)--(-7.675,4.521)--cycle;
\draw(-7.629,4.521)--(-7.628,4.508);
\filldraw[fill opacity=0.8,fill=gray!20,draw=none](-7.684,4.514)--(-7.736,4.533)--(-7.797,4.557)--(-7.8,4.558)--(-7.797,4.558)--(-7.775,4.551)--(-7.742,4.539)--(-7.737,4.537)--cycle;
\draw(-7.8,4.558)--(-7.797,4.558)--(-7.775,4.551)--(-7.742,4.539);
\filldraw[fill opacity=0.8,fill=gray!20,draw=none](-7.583,4.852)--(-7.611,4.854)--(-7.612,4.865)--cycle;
\draw(-7.583,4.852)--(-7.611,4.854)--(-7.612,4.865);
\filldraw[fill opacity=0.8,fill=gray!20,draw=none](-6.946,.467)--(-6.777,.465)--(-6.778,.418)--(-6.945,.42)--cycle;
\draw(-6.778,.418)--(-6.945,.42)--(-6.946,.467)--(-6.777,.465);
\filldraw[fill opacity=0.8,fill=gray!20,draw=none](-7.743,4.531)--(-7.745,4.531)--(-7.746,4.532)--cycle;
\draw(-7.745,4.531)--(-7.746,4.532);
\filldraw[fill opacity=0.8,fill=gray!20,draw=none](-7.745,4.532)--(-7.744,4.534)--(-7.743,4.531)--cycle;
\filldraw[fill opacity=0.8,fill=gray!20,draw=none](-6.178,.242)--(-6.178,.232)--(-6.223,.225)--(-6.223,.271)--cycle;
\draw(-6.178,.242)--(-6.178,.232)--(-6.223,.225)--(-6.223,.271);
\filldraw[fill opacity=0.8,fill=gray!20,draw=none](-7.64,4.478)--(-7.823,4.557)--(-7.822,4.557)--(-7.804,4.549)--(-7.744,4.523)--cycle;
\draw(-7.823,4.557)--(-7.822,4.557)--(-7.804,4.549);
\filldraw[fill opacity=0.8,fill=gray!20,draw=none](-7.757,4.476)--(-7.768,4.477)--(-7.787,4.468)--(-7.77,4.47)--cycle;
\draw(-7.768,4.477)--(-7.787,4.468);
\filldraw[fill opacity=0.8,fill=gray!20,draw=none](-7.773,4.463)--(-7.781,4.466)--(-7.779,4.469)--(-7.77,4.47)--cycle;
\draw(-7.781,4.466)--(-7.779,4.469);
\filldraw[fill opacity=0.8,fill=gray!20,draw=none](-7.784,4.442)--(-7.79,4.446)--(-7.781,4.466)--(-7.773,4.463)--cycle;
\draw(-7.79,4.446)--(-7.781,4.466);
\filldraw[fill opacity=0.8,fill=gray!20](-7.632,.371)--(-7.627,.418)--(-7.608,.398)--(-7.614,.351)--cycle;
\filldraw[fill opacity=0.8,fill=gray!20,draw=none](-7.818,4.372)--(-7.822,4.374)--(-7.79,4.446)--(-7.784,4.442)--cycle;
\draw(-7.822,4.374)--(-7.79,4.446);
\filldraw[fill opacity=0.8,fill=gray!20,draw=none](-7.745,4.532)--(-7.751,4.533)--(-7.75,4.535)--cycle;
\draw(-7.751,4.533)--(-7.75,4.535);
\filldraw[fill opacity=0.8,fill=gray!20,draw=none](-7.791,4.742)--(-7.798,4.743)--(-7.797,4.75)--(-7.781,4.753)--cycle;
\draw(-7.798,4.743)--(-7.797,4.75)--(-7.781,4.753);
\filldraw[fill opacity=0.8,fill=gray!20,draw=none](-7.738,4.52)--(-7.759,4.53)--(-7.752,4.534)--(-7.745,4.531)--(-7.74,4.526)--cycle;
\draw(-7.759,4.53)--(-7.752,4.534);
\draw(-7.745,4.531)--(-7.74,4.526);
\filldraw[fill opacity=0.8,fill=gray!20,draw=none](-7.12,.301)--(-7.124,.322)--(-7.118,.322)--cycle;
\draw(-7.124,.322)--(-7.118,.322);
\filldraw[fill opacity=0.8,fill=gray!20,draw=none](-7.118,.322)--(-7.12,.301)--(-7.141,.286)--(-7.163,.338)--(-7.129,.361)--(-7.122,.34)--cycle;
\draw(-7.12,.301)--(-7.141,.286)--(-7.163,.338)--(-7.129,.361)--(-7.122,.34);
\filldraw[fill opacity=0.8,fill=gray!20](-7.163,.338)--(-7.171,.394)--(-7.135,.417)--(-7.129,.361)--cycle;
\filldraw[fill opacity=0.8,fill=gray!20,draw=none](-7.523,4.826)--(-7.535,4.848)--(-7.481,4.835)--(-7.45,4.794)--cycle;
\draw(-7.523,4.826)--(-7.535,4.848)--(-7.481,4.835)--(-7.45,4.794);
\filldraw[fill opacity=0.8,fill=gray!20,draw=none](-7.583,4.852)--(-7.535,4.848)--(-7.523,4.826)--cycle;
\draw(-7.583,4.852)--(-7.535,4.848)--(-7.523,4.826);
\filldraw[fill opacity=0.8,fill=gray!20,draw=none](-6.178,.268)--(-6.178,.242)--(-6.223,.271)--cycle;
\draw(-6.178,.268)--(-6.178,.242);
\filldraw[fill opacity=0.8,fill=gray!20](-8.206,3.847)--(-8.199,3.903)--(-8.164,3.925)--(-8.17,3.871)--cycle;
\filldraw[fill opacity=0.8,fill=gray!20,draw=none](-7.807,4.471)--(-7.779,4.469)--(-7.793,4.438)--cycle;
\draw(-7.779,4.469)--(-7.793,4.438);
\filldraw[fill opacity=0.8,fill=gray!20,draw=none](-7.812,4.459)--(-7.816,4.467)--(-7.809,4.466)--cycle;
\filldraw[fill opacity=0.8,fill=gray!20,draw=none](-7.877,4.394)--(-7.843,4.479)--(-7.82,4.467)--(-7.852,4.385)--cycle;
\draw(-7.82,4.467)--(-7.852,4.385)--(-7.877,4.394)--(-7.843,4.479);
\filldraw[fill opacity=0.8,fill=gray!20,draw=none](-7.843,4.479)--(-7.822,4.534)--(-7.818,4.472)--(-7.82,4.467)--cycle;
\draw(-7.843,4.479)--(-7.822,4.534);
\draw(-7.818,4.472)--(-7.82,4.467);
\filldraw[fill opacity=0.8,fill=gray!20,draw=none](-7.859,4.388)--(-7.814,4.489)--(-7.793,4.438)--(-7.822,4.374)--cycle;
\draw(-7.859,4.388)--(-7.814,4.489);
\draw(-7.793,4.438)--(-7.822,4.374);
\filldraw[fill opacity=0.8,fill=gray!20,draw=none](-6.186,.255)--(-6.223,.271)--(-6.178,.268)--(-6.151,.256)--cycle;
\draw(-6.186,.255)--(-6.223,.271);
\draw(-6.178,.268)--(-6.151,.256);
\filldraw[fill opacity=0.8,fill=gray!20,draw=none](-6.241,.283)--(-6.271,.3)--(-6.287,.313)--(-6.222,.313)--cycle;
\draw(-6.287,.313)--(-6.222,.313);
\filldraw[fill opacity=0.8,fill=gray!20,draw=none](-6.241,.283)--(-6.223,.31)--(-6.22,.313)--(-6.148,.312)--(-6.138,.308)--(-6.166,.275)--(-6.226,.276)--cycle;
\draw(-6.22,.313)--(-6.148,.312);
\draw(-6.166,.275)--(-6.226,.276);
\filldraw[fill opacity=0.8,fill=gray!20,draw=none](-6.223,.328)--(-6.223,.31)--(-6.241,.283)--(-6.247,.288)--cycle;
\draw(-6.223,.328)--(-6.223,.31);
\filldraw[fill opacity=0.8,fill=gray!20,draw=none](-6.223,.376)--(-6.223,.328)--(-6.247,.288)--(-6.252,.291)--(-6.252,.325)--cycle;
\draw(-6.223,.376)--(-6.223,.328);
\draw(-6.252,.291)--(-6.252,.325);
\filldraw[fill opacity=0.8,fill=gray!20,draw=none](-6.152,.243)--(-6.168,.247)--(-6.186,.255)--(-6.151,.256)--(-6.106,.236)--cycle;
\draw(-6.168,.247)--(-6.186,.255);
\draw(-6.151,.256)--(-6.106,.236);
\filldraw[fill opacity=0.8,fill=gray!20,draw=none](-6.226,.276)--(-6.11,.274)--(-6.113,.254)--(-6.183,.254)--cycle;
\draw(-6.226,.276)--(-6.11,.274)--(-6.113,.254)--(-6.183,.254);
\filldraw[fill opacity=0.8,fill=gray!20,draw=none](-6.183,.254)--(-6.113,.254)--(-6.115,.252)--(-6.174,.253)--cycle;
\draw(-6.183,.254)--(-6.113,.254)--(-6.115,.252)--(-6.174,.253);
\filldraw[fill opacity=0.8,fill=gray!20,draw=none](-6.152,.243)--(-6.106,.236)--(-6.075,.223)--cycle;
\draw(-6.106,.236)--(-6.075,.223);
\filldraw[fill opacity=0.8,fill=gray!20,draw=none](-6.103,.197)--(-6.103,.172)--(-6.047,.175)--(-6.047,.211)--cycle;
\draw(-6.103,.197)--(-6.103,.172)--(-6.047,.175)--(-6.047,.211);
\filldraw[fill opacity=0.8,fill=gray!20,draw=none](-6.103,.218)--(-6.168,.247)--(-6.075,.223)--(-6.047,.211)--cycle;
\draw(-6.103,.218)--(-6.168,.247);
\draw(-6.075,.223)--(-6.047,.211);
\filldraw[fill opacity=0.8,fill=gray!20,draw=none](-6.192,.265)--(-6.174,.253)--(-6.115,.252)--(-6.117,.271)--(-6.192,.272)--cycle;
\draw(-6.174,.253)--(-6.115,.252)--(-6.117,.271)--(-6.192,.272);
\filldraw[fill opacity=0.8,fill=gray!20,draw=none](-6.03,.341)--(-6.022,.348)--(-6.021,.34)--cycle;
\filldraw[fill opacity=0.8,fill=gray!20,draw=none](-6.022,.348)--(-6.021,.34)--(-6.057,.286)--(-6.1,.294)--(-6.061,.353)--cycle;
\draw(-6.021,.34)--(-6.057,.286)--(-6.1,.294)--(-6.061,.353);
\filldraw[fill opacity=0.8,fill=gray!20,draw=none](-6.123,.399)--(-6.123,.378)--(-6.149,.386)--cycle;
\draw(-6.123,.399)--(-6.123,.378);
\filldraw[fill opacity=0.8,fill=gray!20,draw=none](-6.169,.406)--(-6.158,.388)--(-6.163,.38)--cycle;
\draw(-6.158,.388)--(-6.163,.38);
\filldraw[fill opacity=0.8,fill=gray!20,draw=none](-6.176,.394)--(-6.178,.39)--(-6.178,.395)--cycle;
\draw(-6.178,.39)--(-6.178,.395);
\filldraw[fill opacity=0.8,fill=gray!20,draw=none](-6.176,.394)--(-6.165,.412)--(-6.106,.411)--(-6.107,.359)--(-6.144,.36)--cycle;
\draw(-6.165,.412)--(-6.106,.411)--(-6.107,.359)--(-6.144,.36);
\filldraw[fill opacity=0.8,fill=gray!20,draw=none](-6.178,.421)--(-6.169,.406)--(-6.163,.38)--(-6.186,.346)--(-6.216,.382)--(-6.188,.425)--cycle;
\draw(-6.163,.38)--(-6.186,.346)--(-6.216,.382)--(-6.188,.425);
\filldraw[fill opacity=0.8,fill=gray!20,draw=none](-6.178,.431)--(-6.185,.459)--(-6.107,.458)--(-6.106,.411)--(-6.165,.412)--cycle;
\draw(-6.185,.459)--(-6.107,.458)--(-6.106,.411)--(-6.165,.412);
\filldraw[fill opacity=0.8,fill=gray!20,draw=none](-6.138,.308)--(-6.114,.3)--(-6.149,.275)--(-6.166,.275)--cycle;
\draw(-6.149,.275)--(-6.166,.275);
\filldraw[fill opacity=0.8,fill=gray!20,draw=none](-6.178,.421)--(-6.188,.425)--(-6.184,.43)--cycle;
\draw(-6.188,.425)--(-6.184,.43);
\filldraw[fill opacity=0.8,fill=gray!20,draw=none](-6.178,.431)--(-6.183,.439)--(-6.185,.459)--cycle;
\filldraw[fill opacity=0.8,fill=gray!20,draw=none](-6.185,.458)--(-6.183,.432)--(-6.188,.425)--cycle;
\draw(-6.183,.432)--(-6.188,.425);
\filldraw[fill opacity=0.8,fill=gray!20,draw=none](-6.185,.458)--(-6.178,.431)--(-6.178,.421)--(-6.188,.425)--cycle;
\draw(-6.178,.431)--(-6.178,.421);
\filldraw[fill opacity=0.8,fill=gray!20,draw=none](-6.109,.298)--(-6.109,.298)--(-6.1,.294)--(-6.093,.292)--cycle;
\draw(-6.109,.298)--(-6.1,.294)--(-6.093,.292);
\filldraw[fill opacity=0.8,fill=gray!20,draw=none](-6.109,.298)--(-6.109,.303)--(-6.108,.316)--(-6.107,.359)--(-6.106,.411)--(-6.107,.458)--(-6.108,.493)--(-6.11,.512)--(-6.113,.511)--(-6.115,.49)--(-6.117,.466)--cycle;
\draw(-6.109,.298)--(-6.109,.303);
\draw(-6.108,.316)--(-6.107,.359)--(-6.106,.411)--(-6.107,.458)--(-6.108,.493)--(-6.11,.512)--(-6.113,.511)--(-6.115,.49)--(-6.117,.466);
\filldraw[fill opacity=0.8,fill=gray!20,draw=none](-6.039,.419)--(-6.08,.45)--(-6.117,.467)--(-6.109,.298)--(-6.093,.292)--(-6.057,.286)--(-6.022,.293)--(-6,.314)--(-5.996,.345)--(-6.01,.382)--cycle;
\draw(-6.093,.292)--(-6.057,.286)--(-6.022,.293)--(-6,.314)--(-5.996,.345)--(-6.01,.382)--(-6.039,.419)--(-6.08,.45)--(-6.117,.467);
\filldraw[fill opacity=0.8,fill=gray!20,draw=none](-6.103,.423)--(-6.103,.218)--(-6.077,.203)--(-6.047,.211)--(-6.047,.452)--cycle;
\draw(-6.103,.423)--(-6.103,.218);
\draw(-6.047,.211)--(-6.047,.452);
\filldraw[fill opacity=0.8,fill=gray!20,draw=none](-6.103,.218)--(-6.047,.211)--(-6.036,.206)--cycle;
\draw(-6.047,.211)--(-6.036,.206);
\filldraw[fill opacity=0.8,fill=gray!20,draw=none](-6.109,.298)--(-6.114,.3)--(-6.109,.298)--cycle;
\draw(-6.114,.3)--(-6.109,.298);
\filldraw[fill opacity=0.8,fill=gray!20,draw=none](-6.109,.303)--(-6.11,.274)--(-6.149,.275)--cycle;
\draw(-6.109,.303)--(-6.11,.274)--(-6.149,.275);
\filldraw[fill opacity=0.8,fill=gray!20,draw=none](-6.109,.298)--(-6.117,.466)--(-6.133,.472)--(-6.169,.478)--(-6.204,.471)--(-6.225,.45)--(-6.229,.419)--(-6.216,.382)--(-6.186,.346)--(-6.146,.314)--(-6.114,.3)--cycle;
\draw(-6.133,.472)--(-6.169,.478)--(-6.204,.471)--(-6.225,.45)--(-6.229,.419)--(-6.216,.382)--(-6.186,.346)--(-6.146,.314)--(-6.114,.3);
\filldraw[fill opacity=0.8,fill=gray!20,draw=none](-6.119,.306)--(-6.117,.271)--(-6.115,.252)--(-6.113,.254)--(-6.11,.274)--(-6.109,.298)--(-6.117,.466)--(-6.117,.453)--(-6.119,.405)--(-6.119,.353)--cycle;
\draw(-6.117,.466)--(-6.117,.453)--(-6.119,.405)--(-6.119,.353)--(-6.119,.306)--(-6.117,.271)--(-6.115,.252)--(-6.113,.254)--(-6.11,.274)--(-6.109,.298);
\filldraw[fill opacity=0.8,fill=gray!20,draw=none](-6.206,.301)--(-6.192,.272)--(-6.117,.271)--(-6.119,.306)--(-6.196,.307)--cycle;
\draw(-6.192,.272)--(-6.117,.271)--(-6.119,.306)--(-6.196,.307);
\filldraw[fill opacity=0.8,fill=gray!20,draw=none](-6.772,.319)--(-6.316,.314)--(-6.394,.277)--(-6.749,.281)--cycle;
\draw(-6.772,.319)--(-6.316,.314);
\draw(-6.394,.277)--(-6.749,.281);
\filldraw[fill opacity=0.8,fill=gray!20,draw=none](-6.223,.31)--(-6.222,.313)--(-6.22,.313)--cycle;
\draw(-6.222,.313)--(-6.22,.313);
\filldraw[fill opacity=0.8,fill=gray!20,draw=none](-6.194,.348)--(-6.22,.313)--(-6.223,.31)--(-6.223,.328)--cycle;
\draw(-6.223,.31)--(-6.223,.328);
\filldraw[fill opacity=0.8,fill=gray!20,draw=none](-6.196,.307)--(-6.119,.306)--(-6.119,.353)--(-6.177,.354)--cycle;
\draw(-6.196,.307)--(-6.119,.306)--(-6.119,.353)--(-6.177,.354);
\filldraw[fill opacity=0.8,fill=gray!20,draw=none](-6.159,.331)--(-6.159,.251)--(-6.103,.218)--(-6.103,.336)--cycle;
\draw(-6.159,.331)--(-6.159,.251);
\draw(-6.103,.218)--(-6.103,.336);
\filldraw[fill opacity=0.8,fill=gray!20,draw=none](-6.076,.208)--(-6.085,.211)--(-6.103,.218)--(-6.036,.206)--(-6.031,.204)--cycle;
\draw(-6.085,.211)--(-6.103,.218);
\draw(-6.036,.206)--(-6.031,.204)--(-6.076,.208);
\filldraw[fill opacity=0.8,fill=gray!20](-6.124,.235)--(-6.287,.307)--(-6.245,.28)--(-6.081,.209)--cycle;
\filldraw[fill opacity=0.8,fill=gray!20,draw=none](-7.978,4.032)--(-7.976,4.031)--(-7.975,4.03)--cycle;
\filldraw[fill opacity=0.8,fill=gray!20,draw=none](-7.978,4.032)--(-7.98,4.033)--(-7.979,4.036)--(-7.976,4.031)--cycle;
\filldraw[fill opacity=0.8,fill=gray!20,draw=none](-8.008,4.052)--(-7.929,4.229)--(-7.98,4.033)--cycle;
\draw(-8.008,4.052)--(-7.929,4.229);
\filldraw[fill opacity=0.8,fill=gray!20](-7.978,4.03)--(-7.981,4.054)--(-7.927,4.05)--(-7.902,4.024)--cycle;
\filldraw[fill opacity=0.8,fill=gray!20,draw=none](-7.978,4.032)--(-7.98,4.033)--(-7.98,4.033)--cycle;
\filldraw[fill opacity=0.8,fill=gray!20,draw=none](-8.013,4.042)--(-8.008,4.052)--(-7.98,4.033)--(-7.98,4.033)--cycle;
\draw(-8.013,4.042)--(-8.008,4.052);
\filldraw[fill opacity=0.8,fill=gray!20,draw=none](-8.022,4.052)--(-8.025,4.055)--(-8.006,4.057)--(-8.01,4.049)--cycle;
\draw(-8.006,4.057)--(-8.01,4.049);
\filldraw[fill opacity=0.8,fill=gray!20,draw=none](-8.022,4.052)--(-8.01,4.049)--(-8.013,4.042)--cycle;
\draw(-8.01,4.049)--(-8.013,4.042);
\filldraw[fill opacity=0.8,fill=gray!20](-8.057,4.026)--(-8.036,4.052)--(-7.981,4.054)--(-7.978,4.03)--cycle;
\filldraw[fill opacity=0.8,fill=gray!20,draw=none](-7.664,.275)--(-7.663,.27)--(-7.665,.27)--cycle;
\draw(-7.663,.27)--(-7.665,.27);
\filldraw[fill opacity=0.8,fill=gray!20](-7.709,.254)--(-7.675,.286)--(-7.661,.272)--(-7.699,.244)--cycle;
\filldraw[fill opacity=0.8,fill=gray!20](-7.79,.221)--(-7.812,.237)--(-7.788,.238)--(-7.79,.221)--cycle;
\filldraw[fill opacity=0.8,fill=gray!20](-7.79,.221)--(-7.788,.238)--(-7.764,.236)--(-7.79,.221)--cycle;
\filldraw[fill opacity=0.8,fill=gray!20,draw=none](-8.509,2.871)--(-8.544,2.916)--(-8.533,2.923)--(-8.504,2.911)--(-8.496,2.907)--(-8.484,2.888)--cycle;
\draw(-8.496,2.907)--(-8.484,2.888)--(-8.509,2.871)--(-8.544,2.916)--(-8.533,2.923);
\filldraw[fill opacity=0.8,fill=gray!20](-7.838,.228)--(-7.882,.247)--(-7.867,.256)--(-7.83,.233)--cycle;
\filldraw[fill opacity=0.8,fill=gray!20,draw=none](-8.097,3.66)--(-8.142,3.695)--(-8.118,3.711)--(-8.113,3.708)--(-8.079,3.672)--cycle;
\draw(-8.113,3.708)--(-8.079,3.672)--(-8.097,3.66)--(-8.142,3.695)--(-8.118,3.711);
\filldraw[fill opacity=0.8,fill=gray!20,draw=none](-7.98,4.033)--(-7.978,4.032)--(-7.975,4.03)--(-7.988,4.001)--cycle;
\draw(-7.975,4.03)--(-7.988,4.001);
\filldraw[fill opacity=0.8,fill=gray!20,draw=none](-8.082,3.886)--(-8.013,4.042)--(-7.98,4.033)--(-7.988,4.001)--(-8.046,3.87)--cycle;
\draw(-7.988,4.001)--(-8.046,3.87)--(-8.082,3.886)--(-8.013,4.042);
\filldraw[fill opacity=0.8,fill=gray!20,draw=none](-8.046,3.87)--(-8.117,3.71)--(-8.15,3.733)--(-8.082,3.886)--cycle;
\draw(-8.15,3.733)--(-8.082,3.886)--(-8.046,3.87)--(-8.117,3.71);
\filldraw[fill opacity=0.8,fill=gray!20,draw=none](-7.945,.309)--(-7.943,.315)--(-7.955,.337)--(-7.964,.349)--(-7.949,.313)--cycle;
\draw(-7.964,.349)--(-7.949,.313)--(-7.945,.309);
\filldraw[fill opacity=0.8,fill=gray!20,draw=none](-7.955,.337)--(-7.962,.351)--(-7.967,.356)--(-7.964,.349)--cycle;
\draw(-7.962,.351)--(-7.967,.356)--(-7.964,.349);
\filldraw[fill opacity=0.8,fill=gray!20,draw=none](-7.967,.374)--(-7.968,.368)--(-7.969,.372)--cycle;
\draw(-7.968,.368)--(-7.969,.372);
\filldraw[fill opacity=0.8,fill=gray!20,draw=none](-7.965,.355)--(-7.968,.368)--(-7.967,.356)--cycle;
\draw(-7.968,.368)--(-7.967,.356)--(-7.965,.355);
\filldraw[fill opacity=0.8,fill=gray!20,draw=none](-7.965,.355)--(-7.962,.351)--(-7.969,.372)--(-7.968,.368)--cycle;
\draw(-7.965,.355)--(-7.962,.351);
\draw(-7.969,.372)--(-7.968,.368);
\filldraw[fill opacity=0.8,fill=gray!20,draw=none](-9.194,.848)--(-9.173,.835)--(-9.149,.843)--cycle;
\filldraw[fill opacity=0.8,fill=gray!20,draw=none](-8.851,.717)--(-8.857,.7)--(-8.897,.703)--(-8.894,.747)--(-8.892,.746)--cycle;
\draw(-8.857,.7)--(-8.897,.703)--(-8.894,.747)--(-8.892,.746);
\filldraw[fill opacity=0.8,fill=gray!20,draw=none](-8.9,.702)--(-8.973,.743)--(-8.894,.747)--(-8.897,.703)--cycle;
\draw(-8.973,.743)--(-8.894,.747)--(-8.897,.703)--(-8.9,.702);
\filldraw[fill opacity=0.8,fill=gray!20,draw=none](-8.928,.756)--(-8.937,.751)--(-8.981,.77)--(-8.968,.773)--cycle;
\draw(-8.937,.751)--(-8.981,.77);
\filldraw[fill opacity=0.8,fill=gray!20,draw=none](-9.023,.773)--(-9.194,.848)--(-9.149,.843)--(-8.981,.77)--cycle;
\draw(-9.023,.773)--(-9.194,.848);
\draw(-9.149,.843)--(-8.981,.77);
\filldraw[fill opacity=0.8,fill=gray!20,draw=none](-9.041,.784)--(-8.988,.794)--(-8.973,.743)--cycle;
\draw(-9.041,.784)--(-8.988,.794)--(-8.973,.743);
\filldraw[fill opacity=0.8,fill=gray!20,draw=none](-8.937,.747)--(-8.964,.748)--(-9.023,.773)--(-8.981,.77)--(-8.933,.749)--cycle;
\draw(-8.964,.748)--(-9.023,.773);
\draw(-8.981,.77)--(-8.933,.749);
\filldraw[fill opacity=0.8,fill=gray!20,draw=none](-8.851,.717)--(-8.843,.739)--(-8.834,.713)--(-8.838,.707)--cycle;
\draw(-8.834,.713)--(-8.838,.707);
\filldraw[fill opacity=0.8,fill=gray!20,draw=none](-8.857,.7)--(-8.851,.717)--(-8.838,.707)--(-8.843,.699)--cycle;
\draw(-8.838,.707)--(-8.843,.699)--(-8.857,.7);
\filldraw[fill opacity=0.8,fill=gray!20,draw=none](-8.834,.704)--(-8.838,.707)--(-8.834,.713)--cycle;
\draw(-8.838,.707)--(-8.834,.713);
\filldraw[fill opacity=0.8,fill=gray!20,draw=none](-8.834,.704)--(-8.833,.696)--(-8.843,.699)--(-8.838,.707)--cycle;
\draw(-8.833,.696)--(-8.843,.699)--(-8.838,.707);
\filldraw[fill opacity=0.8,fill=gray!20,draw=none](-8.921,.753)--(-8.928,.756)--(-8.908,.766)--cycle;
\filldraw[fill opacity=0.8,fill=gray!20,draw=none](-8.921,.753)--(-8.928,.746)--(-8.937,.751)--(-8.928,.756)--cycle;
\draw(-8.928,.746)--(-8.937,.751);
\filldraw[fill opacity=0.8,fill=gray!20,draw=none](-8.937,.747)--(-8.933,.749)--(-8.928,.746)--cycle;
\draw(-8.933,.749)--(-8.928,.746);
\filldraw[fill opacity=0.8,fill=gray!20](-8.981,.797)--(-7.868,.332)--(-7.86,.298)--(-8.973,.762)--cycle;
\filldraw[fill opacity=0.8,fill=gray!20,draw=none](-8.554,3.043)--(-8.545,2.995)--(-8.559,3.016)--cycle;
\filldraw[fill opacity=0.8,fill=gray!20](-8.464,2.836)--(-8.509,2.871)--(-8.484,2.888)--(-8.446,2.848)--cycle;
\filldraw[fill opacity=0.8,fill=gray!20,draw=none](-8.328,3.237)--(-8.342,3.207)--(-8.343,3.207)--(-8.369,3.242)--(-8.338,3.311)--cycle;
\draw(-8.328,3.237)--(-8.342,3.207);
\draw(-8.369,3.242)--(-8.338,3.311);
\filldraw[fill opacity=0.8,fill=gray!20,draw=none](-8.342,3.207)--(-8.342,3.206)--(-8.343,3.207)--cycle;
\draw(-8.342,3.207)--(-8.342,3.206);
\filldraw[fill opacity=0.8,fill=gray!20,draw=none](-8.345,3.208)--(-8.343,3.207)--(-8.342,3.206)--cycle;
\filldraw[fill opacity=0.8,fill=gray!20,draw=none](-8.343,3.207)--(-8.345,3.208)--(-8.375,3.228)--(-8.369,3.242)--cycle;
\draw(-8.375,3.228)--(-8.369,3.242);
\filldraw[fill opacity=0.8,fill=gray!20](-8.345,3.206)--(-8.348,3.23)--(-8.294,3.226)--(-8.269,3.2)--cycle;
\filldraw[fill opacity=0.8,fill=gray!20,draw=none](-8.345,3.208)--(-8.373,3.216)--(-8.377,3.223)--(-8.375,3.228)--cycle;
\draw(-8.377,3.223)--(-8.375,3.228);
\filldraw[fill opacity=0.8,fill=gray!20,draw=none](-8.373,3.216)--(-8.38,3.218)--(-8.377,3.223)--cycle;
\draw(-8.38,3.218)--(-8.377,3.223);
\filldraw[fill opacity=0.8,fill=gray!20,draw=none](-8.373,3.233)--(-8.377,3.225)--(-8.378,3.225)--(-8.381,3.232)--cycle;
\draw(-8.373,3.233)--(-8.377,3.225);
\filldraw[fill opacity=0.8,fill=gray!20,draw=none](-8.377,3.225)--(-8.377,3.223)--(-8.378,3.225)--cycle;
\draw(-8.377,3.225)--(-8.377,3.223);
\filldraw[fill opacity=0.8,fill=gray!20,draw=none](-8.378,3.225)--(-8.377,3.223)--(-8.38,3.218)--(-8.389,3.228)--cycle;
\draw(-8.377,3.223)--(-8.38,3.218);
\filldraw[fill opacity=0.8,fill=gray!20,draw=none](-8.381,3.232)--(-8.378,3.225)--(-8.389,3.228)--(-8.392,3.231)--cycle;
\filldraw[fill opacity=0.8,fill=gray!20](-8.424,3.202)--(-8.403,3.227)--(-8.348,3.23)--(-8.345,3.206)--cycle;
\filldraw[fill opacity=0.8,fill=gray!20,draw=none](-7.808,4.499)--(-7.816,4.495)--(-7.785,4.469)--(-7.768,4.477)--cycle;
\draw(-7.808,4.499)--(-7.816,4.495);
\draw(-7.785,4.469)--(-7.768,4.477);
\filldraw[fill opacity=0.8,fill=gray!20,draw=none](-8.22,3.479)--(-8.328,3.237)--(-8.338,3.311)--(-8.183,3.658)--cycle;
\draw(-8.22,3.479)--(-8.328,3.237);
\draw(-8.338,3.311)--(-8.183,3.658);
\filldraw[fill opacity=0.8,fill=gray!20,draw=none](-8.464,2.917)--(-8.476,2.926)--(-8.394,2.89)--(-8.403,2.884)--(-8.414,2.888)--cycle;
\draw(-8.476,2.926)--(-8.394,2.89);
\filldraw[fill opacity=0.8,fill=gray!20,draw=none](-8.47,2.934)--(-8.402,2.894)--(-8.452,2.916)--cycle;
\draw(-8.402,2.894)--(-8.452,2.916);
\filldraw[fill opacity=0.8,fill=gray!20,draw=none](-8.251,2.974)--(-8.271,2.939)--(-8.309,2.907)--(-8.354,2.892)--(-8.399,2.897)--(-8.438,2.92)--(-8.463,2.959)--(-8.472,3.008)--(-8.463,3.058)--(-8.458,3.067)--cycle;
\draw(-8.251,2.974)--(-8.271,2.939)--(-8.309,2.907)--(-8.354,2.892)--(-8.399,2.897)--(-8.438,2.92)--(-8.463,2.959)--(-8.472,3.008)--(-8.463,3.058)--(-8.458,3.067);
\filldraw[fill opacity=0.8,fill=gray!20,draw=none](-8.47,2.934)--(-8.501,2.953)--(-8.517,2.971)--(-8.523,2.981)--(-8.505,2.974)--cycle;
\draw(-8.523,2.981)--(-8.505,2.974);
\filldraw[fill opacity=0.8,fill=gray!20,draw=none](-8.501,2.953)--(-8.47,2.934)--(-8.452,2.916)--(-8.476,2.926)--cycle;
\draw(-8.452,2.916)--(-8.476,2.926);
\filldraw[fill opacity=0.8,fill=gray!20,draw=none](-8.258,2.829)--(-8.276,2.83)--(-8.272,2.826)--cycle;
\filldraw[fill opacity=0.8,fill=gray!20,draw=none](-8.419,2.904)--(-8.38,2.89)--(-8.394,2.89)--(-8.402,2.894)--cycle;
\draw(-8.394,2.89)--(-8.402,2.894);
\filldraw[fill opacity=0.8,fill=gray!20,draw=none](-8.301,2.868)--(-8.326,2.865)--(-8.382,2.89)--(-8.382,2.89)--cycle;
\draw(-8.326,2.865)--(-8.382,2.89);
\filldraw[fill opacity=0.8,fill=gray!20,draw=none](-8.214,2.856)--(-8.243,2.834)--(-8.257,2.846)--cycle;
\filldraw[fill opacity=0.8,fill=gray!20,draw=none](-8.269,2.825)--(-8.28,2.832)--(-8.256,2.846)--(-8.245,2.833)--(-8.265,2.826)--cycle;
\draw(-8.28,2.832)--(-8.256,2.846)--(-8.245,2.833)--(-8.265,2.826);
\filldraw[fill opacity=0.8,fill=gray!20,draw=none](-8.256,2.846)--(-8.215,2.855)--(-8.245,2.833)--cycle;
\draw(-8.215,2.855)--(-8.245,2.833)--(-8.256,2.846);
\filldraw[fill opacity=0.8,fill=gray!20,draw=none](-8.243,2.834)--(-8.232,2.824)--(-8.237,2.822)--(-8.258,2.822)--cycle;
\draw(-8.232,2.824)--(-8.237,2.822)--(-8.258,2.822);
\filldraw[fill opacity=0.8,fill=gray!20,draw=none](-8.214,2.856)--(-8.184,2.863)--(-8.203,2.842)--(-8.232,2.824)--(-8.243,2.834)--cycle;
\draw(-8.184,2.863)--(-8.203,2.842)--(-8.232,2.824);
\filldraw[fill opacity=0.8,fill=gray!20,draw=none](-8.354,2.892)--(-8.318,2.876)--(-8.372,2.885)--(-8.399,2.897)--cycle;
\draw(-8.372,2.885)--(-8.399,2.897)--(-8.354,2.892)--(-8.318,2.876);
\filldraw[fill opacity=0.8,fill=gray!20,draw=none](-8.354,2.892)--(-8.318,2.876)--(-8.372,2.885)--(-8.399,2.897)--cycle;
\draw(-8.372,2.885)--(-8.399,2.897)--(-8.354,2.892)--(-8.318,2.876);
\filldraw[fill opacity=0.8,fill=gray!20,draw=none](-8.399,2.897)--(-8.372,2.885)--(-8.406,2.907)--(-8.438,2.92)--cycle;
\draw(-8.406,2.907)--(-8.438,2.92)--(-8.399,2.897)--(-8.372,2.885);
\filldraw[fill opacity=0.8,fill=gray!20,draw=none](-8.399,2.897)--(-8.372,2.885)--(-8.406,2.907)--(-8.438,2.92)--cycle;
\draw(-8.406,2.907)--(-8.438,2.92)--(-8.399,2.897)--(-8.372,2.885);
\filldraw[fill opacity=0.8,fill=gray!20,draw=none](-8.344,2.868)--(-8.351,2.87)--(-8.394,2.882)--(-8.395,2.89)--(-8.382,2.89)--(-8.326,2.865)--cycle;
\draw(-8.382,2.89)--(-8.326,2.865);
\filldraw[fill opacity=0.8,fill=gray!20,draw=none](-8.424,2.894)--(-8.464,2.917)--(-8.47,2.919)--(-8.451,2.908)--cycle;
\draw(-8.464,2.917)--(-8.47,2.919);
\filldraw[fill opacity=0.8,fill=gray!20,draw=none](-8.43,2.902)--(-8.412,2.889)--(-8.351,2.87)--(-8.344,2.868)--(-8.425,2.904)--cycle;
\draw(-8.344,2.868)--(-8.425,2.904);
\filldraw[fill opacity=0.8,fill=gray!20,draw=none](-8.438,2.92)--(-8.406,2.907)--(-8.413,2.933)--(-8.416,2.939)--(-8.463,2.959)--cycle;
\draw(-8.416,2.939)--(-8.463,2.959)--(-8.438,2.92)--(-8.406,2.907);
\filldraw[fill opacity=0.8,fill=gray!20,draw=none](-8.438,2.92)--(-8.406,2.907)--(-8.413,2.933)--(-8.416,2.939)--(-8.463,2.959)--cycle;
\draw(-8.416,2.939)--(-8.463,2.959)--(-8.438,2.92)--(-8.406,2.907);
\filldraw[fill opacity=0.8,fill=gray!20,draw=none](-8.434,2.934)--(-8.545,2.983)--(-8.518,2.941)--(-8.424,2.899)--cycle;
\draw(-8.434,2.934)--(-8.545,2.983);
\draw(-8.518,2.941)--(-8.424,2.899);
\filldraw[fill opacity=0.8,fill=gray!20,draw=none](-8.313,2.835)--(-8.307,2.841)--(-8.289,2.835)--(-8.284,2.83)--cycle;
\draw(-8.313,2.835)--(-8.307,2.841);
\filldraw[fill opacity=0.8,fill=gray!20,draw=none](-8.405,2.883)--(-8.415,2.888)--(-8.403,2.884)--cycle;
\draw(-8.405,2.883)--(-8.415,2.888);
\filldraw[fill opacity=0.8,fill=gray!20,draw=none](-8.414,2.888)--(-8.415,2.888)--(-8.405,2.883)--cycle;
\draw(-8.415,2.888)--(-8.405,2.883);
\filldraw[fill opacity=0.8,fill=gray!20,draw=none](-8.302,2.848)--(-8.266,2.822)--(-8.435,2.896)--cycle;
\draw(-8.266,2.822)--(-8.435,2.896);
\filldraw[fill opacity=0.8,fill=gray!20,draw=none](-8.409,3.193)--(-8.427,3.195)--(-8.424,3.202)--cycle;
\draw(-8.427,3.195)--(-8.424,3.202);
\filldraw[fill opacity=0.8,fill=gray!20,draw=none](-7.834,4.711)--(-7.832,4.727)--(-7.797,4.75)--(-7.8,4.726)--cycle;
\draw(-7.834,4.711)--(-7.832,4.727)--(-7.797,4.75)--(-7.8,4.726);
\filldraw[fill opacity=0.8,fill=gray!20](-7.79,.221)--(-7.83,.233)--(-7.812,.237)--(-7.79,.221)--cycle;
\filldraw[fill opacity=0.8,fill=gray!20](-7.627,.418)--(-7.632,.463)--(-7.614,.444)--(-7.608,.398)--cycle;
\filldraw[fill opacity=0.8,fill=gray!20,draw=none](-7.583,4.852)--(-7.612,4.865)--(-7.614,4.878)--(-7.56,4.874)--(-7.535,4.848)--cycle;
\draw(-7.612,4.865)--(-7.614,4.878)--(-7.56,4.874)--(-7.535,4.848)--(-7.583,4.852);
\filldraw[fill opacity=0.8,fill=gray!20,draw=none](-8.345,3.208)--(-8.342,3.206)--(-8.353,3.181)--(-8.373,3.216)--cycle;
\draw(-8.342,3.206)--(-8.353,3.181);
\filldraw[fill opacity=0.8,fill=gray!20,draw=none](-8.373,3.216)--(-8.354,3.183)--(-8.394,3.185)--(-8.38,3.218)--cycle;
\draw(-8.394,3.185)--(-8.38,3.218);
\filldraw[fill opacity=0.8,fill=gray!20,draw=none](-7.683,4.352)--(-7.673,4.349)--(-7.679,4.335)--cycle;
\draw(-7.673,4.349)--(-7.679,4.335);
\filldraw[fill opacity=0.8,fill=gray!20](-7.171,.394)--(-7.163,.449)--(-7.129,.472)--(-7.135,.417)--cycle;
\filldraw[fill opacity=0.8,fill=gray!20,draw=none](-7.454,4.573)--(-7.45,4.58)--(-7.448,4.578)--cycle;
\draw(-7.454,4.573)--(-7.45,4.58)--(-7.448,4.578);
\filldraw[fill opacity=0.8,fill=gray!20](-7.45,4.58)--(-7.43,4.634)--(-7.408,4.61)--(-7.43,4.559)--cycle;
\filldraw[fill opacity=0.8,fill=gray!20](-7.061,.207)--(-7.107,.242)--(-7.081,.258)--(-7.043,.219)--cycle;
\filldraw[fill opacity=0.8,fill=gray!20](-7.021,.573)--(-7.001,.598)--(-6.945,.601)--(-6.942,.576)--cycle;
\filldraw[fill opacity=0.8,fill=gray!20](-6.942,.576)--(-6.945,.601)--(-6.891,.597)--(-6.866,.571)--cycle;
\filldraw[fill opacity=0.8,fill=gray!20,draw=none](-7.467,4.553)--(-7.454,4.573)--(-7.448,4.578)--(-7.43,4.559)--cycle;
\draw(-7.467,4.553)--(-7.454,4.573);
\draw(-7.448,4.578)--(-7.43,4.559);
\filldraw[fill opacity=0.8,fill=gray!20](-7.79,.221)--(-7.764,.236)--(-7.748,.232)--(-7.79,.221)--cycle;
\filldraw[fill opacity=0.8,fill=gray!20](-7.719,.55)--(-7.74,.571)--(-7.709,.564)--(-7.675,.539)--cycle;
\filldraw[fill opacity=0.8,fill=gray!20,draw=none](-8.561,3.079)--(-8.554,3.043)--(-8.559,3.016)--(-8.564,3.024)--(-8.567,3.042)--cycle;
\draw(-8.564,3.024)--(-8.567,3.042)--(-8.561,3.079);
\filldraw[fill opacity=0.8,fill=gray!20,draw=none](-8.564,3.04)--(-8.567,3.042)--(-8.557,2.988)--(-8.545,2.983)--cycle;
\draw(-8.564,3.04)--(-8.567,3.042)--(-8.557,2.988)--(-8.545,2.983);
\filldraw[fill opacity=0.8,fill=gray!20](-7.43,4.634)--(-7.424,4.69)--(-7.401,4.666)--(-7.408,4.61)--cycle;
\filldraw[fill opacity=0.8,fill=gray!20,draw=none](-8.289,2.835)--(-8.297,2.844)--(-8.28,2.832)--cycle;
\filldraw[fill opacity=0.8,fill=gray!20,draw=none](-7.664,.275)--(-7.665,.27)--(-7.79,.271)--(-7.788,.292)--(-7.666,.291)--cycle;
\draw(-7.665,.27)--(-7.79,.271)--(-7.788,.292)--(-7.666,.291);
\filldraw[fill opacity=0.8,fill=gray!20,draw=none](-7.459,4.539)--(-7.467,4.553)--(-7.457,4.555)--cycle;
\filldraw[fill opacity=0.8,fill=gray!20,draw=none](-8.539,2.992)--(-8.536,2.987)--(-8.538,2.986)--cycle;
\draw(-8.536,2.987)--(-8.538,2.986);
\filldraw[fill opacity=0.8,fill=gray!20,draw=none](-7.765,4.532)--(-7.804,4.549)--(-7.799,4.547)--(-7.79,4.543)--cycle;
\draw(-7.804,4.549)--(-7.799,4.547)--(-7.79,4.543);
\filldraw[fill opacity=0.8,fill=gray!20,draw=none](-7.551,4.622)--(-7.546,4.635)--(-7.54,4.633)--cycle;
\draw(-7.551,4.622)--(-7.546,4.635)--(-7.54,4.633);
\filldraw[fill opacity=0.8,fill=gray!20](-7.748,.232)--(-7.709,.254)--(-7.699,.244)--(-7.743,.227)--cycle;
\filldraw[fill opacity=0.8,fill=gray!20](-7.848,3.708)--(-7.817,3.756)--(-7.797,3.735)--(-7.832,3.691)--cycle;
\filldraw[fill opacity=0.8,fill=gray!20](-7.817,3.756)--(-7.798,3.81)--(-7.776,3.786)--(-7.797,3.735)--cycle;
\filldraw[fill opacity=0.8,fill=gray!20,draw=none](-7.679,4.318)--(-7.679,4.335)--(-7.673,4.349)--(-7.657,4.347)--(-7.648,4.323)--(-7.655,4.308)--cycle;
\draw(-7.679,4.335)--(-7.673,4.349);
\draw(-7.648,4.323)--(-7.655,4.308);
\filldraw[fill opacity=0.8,fill=gray!20,draw=none](-7.686,4.321)--(-7.679,4.335)--(-7.679,4.318)--cycle;
\draw(-7.686,4.321)--(-7.679,4.335);
\filldraw[fill opacity=0.8,fill=gray!20,draw=none](-8.458,3.183)--(-8.489,3.184)--(-8.484,3.191)--(-8.458,3.196)--(-8.427,3.195)--cycle;
\draw(-8.489,3.184)--(-8.484,3.191)--(-8.458,3.196);
\filldraw[fill opacity=0.8,fill=gray!20,draw=none](-8.458,3.196)--(-8.424,3.202)--(-8.427,3.195)--cycle;
\draw(-8.458,3.196)--(-8.424,3.202)--(-8.427,3.195);
\filldraw[fill opacity=0.8,fill=gray!20,draw=none](-7.704,4.505)--(-7.707,4.507)--(-7.768,4.477)--(-7.757,4.476)--cycle;
\draw(-7.707,4.507)--(-7.768,4.477);
\filldraw[fill opacity=0.8,fill=gray!20](-7.832,4.727)--(-7.81,4.778)--(-7.779,4.799)--(-7.797,4.75)--cycle;
\filldraw[fill opacity=0.8,fill=gray!20](-7.798,3.81)--(-7.791,3.866)--(-7.768,3.842)--(-7.776,3.786)--cycle;
\filldraw[fill opacity=0.8,fill=gray!20,draw=none](-6.782,.282)--(-6.784,.299)--(-6.782,.303)--(-6.776,.297)--cycle;
\draw(-6.784,.299)--(-6.782,.303)--(-6.776,.297);
\filldraw[fill opacity=0.8,fill=gray!20,draw=none](-6.782,.282)--(-6.789,.261)--(-6.807,.263)--(-6.784,.299)--cycle;
\draw(-6.807,.263)--(-6.784,.299);
\filldraw[fill opacity=0.8,fill=gray!20,draw=none](-6.776,.297)--(-6.779,.3)--(-6.775,.319)--cycle;
\draw(-6.776,.297)--(-6.779,.3);
\filldraw[fill opacity=0.8,fill=gray!20,draw=none](-6.947,.32)--(-6.772,.319)--(-6.77,.315)--(-6.782,.282)--(-6.949,.283)--cycle;
\draw(-6.782,.282)--(-6.949,.283)--(-6.947,.32)--(-6.772,.319);
\filldraw[fill opacity=0.8,fill=gray!20,draw=none](-8.554,3.036)--(-8.564,3.04)--(-8.545,2.983)--cycle;
\draw(-8.554,3.036)--(-8.564,3.04);
\filldraw[fill opacity=0.8,fill=gray!20,draw=none](-8.582,3.059)--(-8.567,3.052)--(-8.566,3.05)--(-8.57,3.026)--(-8.573,3.018)--cycle;
\draw(-8.582,3.059)--(-8.567,3.052);
\filldraw[fill opacity=0.8,fill=gray!20,draw=none](-8.573,3.018)--(-8.57,3.026)--(-8.572,3.011)--cycle;
\filldraw[fill opacity=0.8,fill=gray!20,draw=none](-8.539,2.992)--(-8.538,2.986)--(-8.55,2.978)--(-8.569,2.993)--(-8.573,3.023)--(-8.558,3.034)--cycle;
\draw(-8.538,2.986)--(-8.55,2.978);
\draw(-8.569,2.993)--(-8.573,3.023)--(-8.558,3.034);
\filldraw[fill opacity=0.8,fill=gray!20,draw=none](-7.979,3.65)--(-7.984,3.65)--(-7.984,3.65)--cycle;
\draw(-7.979,3.65)--(-7.984,3.65)--(-7.984,3.65);
\filldraw[fill opacity=0.8,fill=gray!20,draw=none](-7.984,3.65)--(-7.956,3.648)--(-7.966,3.642)--cycle;
\draw(-7.984,3.65)--(-7.956,3.648)--(-7.966,3.642);
\filldraw[fill opacity=0.8,fill=gray!20,draw=none](-7.959,3.646)--(-7.956,3.648)--(-7.936,3.643)--(-7.953,3.639)--cycle;
\draw(-7.959,3.646)--(-7.956,3.648)--(-7.936,3.643)--(-7.953,3.639);
\filldraw[fill opacity=0.8,fill=gray!20,draw=none](-7.966,3.642)--(-7.959,3.646)--(-7.953,3.639)--cycle;
\draw(-7.966,3.642)--(-7.959,3.646);
\filldraw[fill opacity=0.8,fill=gray!20,draw=none](-7.882,3.797)--(-7.958,3.627)--(-7.984,3.651)--(-7.913,3.811)--cycle;
\draw(-7.984,3.651)--(-7.913,3.811)--(-7.882,3.797)--(-7.958,3.627);
\filldraw[fill opacity=0.8,fill=gray!20,draw=none](-7.979,3.646)--(-7.985,3.648)--(-7.984,3.651)--cycle;
\draw(-7.985,3.648)--(-7.984,3.651);
\filldraw[fill opacity=0.8,fill=gray!20,draw=none](-7.979,3.646)--(-7.975,3.642)--(-7.987,3.644)--(-7.985,3.648)--cycle;
\draw(-7.987,3.644)--(-7.985,3.648);
\filldraw[fill opacity=0.8,fill=gray!20,draw=none](-8.394,2.894)--(-8.392,2.894)--(-8.388,2.892)--cycle;
\draw(-8.392,2.894)--(-8.388,2.892);
\filldraw[fill opacity=0.8,fill=gray!20,draw=none](-8.323,2.824)--(-8.313,2.835)--(-8.284,2.83)--(-8.303,2.819)--cycle;
\draw(-8.284,2.83)--(-8.303,2.819)--(-8.323,2.824)--(-8.313,2.835);
\filldraw[fill opacity=0.8,fill=gray!20,draw=none](-8.519,3.138)--(-8.547,3.115)--(-8.529,3.147)--(-8.514,3.159)--(-8.506,3.162)--cycle;
\draw(-8.547,3.115)--(-8.529,3.147)--(-8.514,3.159);
\filldraw[fill opacity=0.8,fill=gray!20,draw=none](-8.492,3.154)--(-8.512,3.151)--(-8.505,3.155)--(-8.474,3.167)--cycle;
\draw(-8.492,3.154)--(-8.512,3.151);
\filldraw[fill opacity=0.8,fill=gray!20,draw=none](-8.515,3.136)--(-8.503,3.152)--(-8.492,3.154)--cycle;
\draw(-8.503,3.152)--(-8.492,3.154);
\filldraw[fill opacity=0.8,fill=gray!20,draw=none](-8.515,3.136)--(-8.52,3.131)--(-8.513,3.15)--(-8.512,3.151)--(-8.503,3.152)--cycle;
\draw(-8.52,3.131)--(-8.513,3.15);
\draw(-8.512,3.151)--(-8.503,3.152);
\filldraw[fill opacity=0.8,fill=gray!20,draw=none](-8.545,3.094)--(-8.54,3.121)--(-8.519,3.138)--cycle;
\filldraw[fill opacity=0.8,fill=gray!20,draw=none](-8.531,3.116)--(-8.527,3.125)--(-8.518,3.143)--(-8.513,3.15)--(-8.52,3.131)--cycle;
\draw(-8.513,3.15)--(-8.52,3.131);
\filldraw[fill opacity=0.8,fill=gray!20,draw=none](-8.491,3.141)--(-8.486,3.139)--(-8.499,3.129)--cycle;
\draw(-8.491,3.141)--(-8.486,3.139);
\filldraw[fill opacity=0.8,fill=gray!20,draw=none](-8.491,3.149)--(-8.451,3.157)--(-8.486,3.139)--(-8.499,3.145)--cycle;
\draw(-8.486,3.139)--(-8.499,3.145);
\filldraw[fill opacity=0.8,fill=gray!20,draw=none](-8.545,3.094)--(-8.559,3.069)--(-8.561,3.079)--(-8.557,3.098)--(-8.547,3.115)--(-8.54,3.121)--cycle;
\draw(-8.561,3.079)--(-8.557,3.098)--(-8.547,3.115);
\filldraw[fill opacity=0.8,fill=gray!20,draw=none](-8.508,3.138)--(-8.499,3.145)--(-8.491,3.141)--(-8.499,3.129)--(-8.53,3.107)--cycle;
\draw(-8.499,3.145)--(-8.491,3.141);
\filldraw[fill opacity=0.8,fill=gray!20,draw=none](-8.436,3.15)--(-8.455,3.117)--(-8.446,3.136)--cycle;
\draw(-8.455,3.117)--(-8.446,3.136);
\filldraw[fill opacity=0.8,fill=gray!20,draw=none](-8.443,3.144)--(-8.446,3.136)--(-8.437,3.126)--cycle;
\draw(-8.443,3.144)--(-8.446,3.136);
\filldraw[fill opacity=0.8,fill=gray!20,draw=none](-8.437,3.126)--(-8.446,3.136)--(-8.452,3.109)--(-8.447,3.094)--(-8.435,3.121)--cycle;
\draw(-8.447,3.094)--(-8.435,3.121);
\filldraw[fill opacity=0.8,fill=gray!20,draw=none](-8.446,3.136)--(-8.455,3.117)--(-8.452,3.109)--cycle;
\draw(-8.446,3.136)--(-8.455,3.117);
\filldraw[fill opacity=0.8,fill=gray!20,draw=none](-8.34,3.015)--(-8.387,3.036)--(-8.429,3.054)--(-8.454,3.065)--(-8.458,3.067)--(-8.251,2.974)--(-8.247,2.972)--(-8.235,2.967)--(-8.238,2.969)--(-8.26,2.979)--(-8.295,2.995)--cycle;
\draw(-8.247,2.972)--(-8.235,2.967)--(-8.238,2.969)--(-8.26,2.979)--(-8.295,2.995)--(-8.34,3.015)--(-8.387,3.036)--(-8.429,3.054)--(-8.454,3.065);
\filldraw[fill opacity=0.8,fill=gray!20](-7.936,3.643)--(-7.889,3.67)--(-7.878,3.658)--(-7.931,3.637)--cycle;
\filldraw[fill opacity=0.8,fill=gray!20,draw=none](-7.953,3.639)--(-7.936,3.643)--(-7.931,3.637)--(-7.935,3.636)--cycle;
\draw(-7.953,3.639)--(-7.936,3.643)--(-7.931,3.637)--(-7.935,3.636);
\filldraw[fill opacity=0.8,fill=gray!20,draw=none](-8.454,3.065)--(-8.429,3.054)--(-8.4,3.119)--cycle;
\draw(-8.454,3.065)--(-8.429,3.054)--(-8.4,3.119);
\filldraw[fill opacity=0.8,fill=gray!20,draw=none](-8.463,3.058)--(-8.399,3.03)--(-8.368,3.072)--(-8.438,3.102)--cycle;
\draw(-8.368,3.072)--(-8.438,3.102)--(-8.463,3.058)--(-8.399,3.03);
\filldraw[fill opacity=0.8,fill=gray!20,draw=none](-8.463,3.058)--(-8.399,3.03)--(-8.368,3.072)--(-8.438,3.102)--cycle;
\draw(-8.368,3.072)--(-8.438,3.102)--(-8.463,3.058)--(-8.399,3.03);
\filldraw[fill opacity=0.8,fill=gray!20,draw=none](-8.435,3.121)--(-8.459,3.068)--(-8.454,3.065)--(-8.4,3.119)--(-8.398,3.124)--cycle;
\draw(-8.435,3.121)--(-8.459,3.068)--(-8.454,3.065);
\draw(-8.4,3.119)--(-8.398,3.124);
\filldraw[fill opacity=0.8,fill=gray!20,draw=none](-8.423,3.096)--(-8.376,3.075)--(-8.349,3.102)--cycle;
\draw(-8.423,3.096)--(-8.376,3.075);
\filldraw[fill opacity=0.8,fill=gray!20,draw=none](-8.423,3.096)--(-8.376,3.075)--(-8.365,3.086)--(-8.355,3.101)--cycle;
\draw(-8.423,3.096)--(-8.376,3.075);
\filldraw[fill opacity=0.8,fill=gray!20,draw=none](-8.401,3.117)--(-8.41,3.097)--(-8.362,3.118)--cycle;
\draw(-8.401,3.117)--(-8.41,3.097);
\filldraw[fill opacity=0.8,fill=gray!20,draw=none](-8.344,3.114)--(-8.346,3.115)--(-8.348,3.112)--cycle;
\filldraw[fill opacity=0.8,fill=gray!20,draw=none](-8.355,3.126)--(-8.356,3.126)--(-8.378,3.092)--(-8.365,3.086)--(-8.355,3.101)--cycle;
\draw(-8.378,3.092)--(-8.365,3.086);
\filldraw[fill opacity=0.8,fill=gray!20,draw=none](-8.365,3.086)--(-8.342,3.109)--(-8.348,3.112)--cycle;
\draw(-8.342,3.109)--(-8.348,3.112);
\filldraw[fill opacity=0.8,fill=gray!20,draw=none](-8.355,3.101)--(-8.349,3.102)--(-8.342,3.109)--(-8.348,3.112)--cycle;
\draw(-8.342,3.109)--(-8.348,3.112);
\filldraw[fill opacity=0.8,fill=gray!20,draw=none](-8.376,3.075)--(-8.373,3.074)--(-8.365,3.086)--cycle;
\draw(-8.376,3.075)--(-8.373,3.074);
\filldraw[fill opacity=0.8,fill=gray!20,draw=none](-8.376,3.075)--(-8.365,3.086)--(-8.37,3.088)--cycle;
\draw(-8.365,3.086)--(-8.37,3.088);
\filldraw[fill opacity=0.8,fill=gray!20,draw=none](-8.386,3.085)--(-8.405,3.042)--(-8.393,3.037)--(-8.37,3.088)--(-8.378,3.092)--cycle;
\draw(-8.405,3.042)--(-8.393,3.037);
\draw(-8.37,3.088)--(-8.378,3.092);
\filldraw[fill opacity=0.8,fill=gray!20,draw=none](-8.309,3.084)--(-8.295,3.056)--(-8.285,3.051)--(-8.27,3.051)--cycle;
\draw(-8.295,3.056)--(-8.285,3.051);
\filldraw[fill opacity=0.8,fill=gray!20,draw=none](-8.348,3.104)--(-8.365,3.086)--(-8.352,3.087)--cycle;
\draw(-8.365,3.086)--(-8.352,3.087);
\filldraw[fill opacity=0.8,fill=gray!20,draw=none](-8.258,2.829)--(-8.253,2.827)--(-8.252,2.828)--(-8.253,2.829)--cycle;
\draw(-8.252,2.828)--(-8.253,2.829);
\filldraw[fill opacity=0.8,fill=gray!20,draw=none](-8.171,2.906)--(-8.184,2.879)--(-8.203,2.865)--cycle;
\filldraw[fill opacity=0.8,fill=gray!20,draw=none](-8.271,2.939)--(-8.209,2.912)--(-8.223,2.902)--(-8.258,2.884)--(-8.309,2.907)--cycle;
\draw(-8.258,2.884)--(-8.309,2.907)--(-8.271,2.939)--(-8.209,2.912);
\filldraw[fill opacity=0.8,fill=gray!20,draw=none](-8.271,2.939)--(-8.209,2.912)--(-8.223,2.902)--(-8.258,2.884)--(-8.309,2.907)--cycle;
\draw(-8.258,2.884)--(-8.309,2.907)--(-8.271,2.939)--(-8.209,2.912);
\filldraw[fill opacity=0.8,fill=gray!20,draw=none](-8.411,2.984)--(-8.265,2.92)--(-8.243,2.972)--(-8.396,3.039)--cycle;
\draw(-8.411,2.984)--(-8.265,2.92);
\draw(-8.243,2.972)--(-8.396,3.039);
\filldraw[fill opacity=0.8,fill=gray!20,draw=none](-8.296,3.089)--(-8.352,3.087)--(-8.366,3.056)--(-8.367,3.032)--cycle;
\draw(-8.296,3.089)--(-8.352,3.087);
\draw(-8.366,3.056)--(-8.367,3.032);
\filldraw[fill opacity=0.8,fill=gray!20,draw=none](-8.399,3.03)--(-8.388,3.025)--(-8.382,3.031)--(-8.359,3.068)--(-8.368,3.072)--cycle;
\draw(-8.399,3.03)--(-8.388,3.025);
\draw(-8.359,3.068)--(-8.368,3.072);
\filldraw[fill opacity=0.8,fill=gray!20,draw=none](-8.399,3.03)--(-8.388,3.025)--(-8.382,3.031)--(-8.359,3.068)--(-8.368,3.072)--cycle;
\draw(-8.399,3.03)--(-8.388,3.025);
\draw(-8.359,3.068)--(-8.368,3.072);
\filldraw[fill opacity=0.8,fill=gray!20,draw=none](-8.352,3.087)--(-8.365,3.086)--(-8.366,3.085)--(-8.366,3.056)--cycle;
\draw(-8.352,3.087)--(-8.365,3.086);
\draw(-8.366,3.085)--(-8.366,3.056);
\filldraw[fill opacity=0.8,fill=gray!20,draw=none](-8.375,3.08)--(-8.386,3.064)--(-8.386,3.031)--(-8.384,3.031)--(-8.382,3.032)--(-8.366,3.056)--(-8.366,3.085)--cycle;
\draw(-8.386,3.031)--(-8.384,3.031);
\draw(-8.366,3.056)--(-8.366,3.085);
\filldraw[fill opacity=0.8,fill=gray!20](-8.323,3.111)--(-8.486,3.183)--(-8.529,3.147)--(-8.365,3.076)--cycle;
\filldraw[fill opacity=0.8,fill=gray!20,draw=none](-8.539,2.992)--(-8.558,3.034)--(-8.547,3.04)--cycle;
\draw(-8.558,3.034)--(-8.547,3.04);
\filldraw[fill opacity=0.8,fill=gray!20](-8.117,4.015)--(-8.079,4.043)--(-8.036,4.052)--(-8.057,4.026)--cycle;
\filldraw[fill opacity=0.8,fill=gray!20](-8.199,3.903)--(-8.177,3.954)--(-8.146,3.975)--(-8.164,3.925)--cycle;
\filldraw[fill opacity=0.8,fill=gray!20](-7.889,3.67)--(-7.848,3.708)--(-7.832,3.691)--(-7.878,3.658)--cycle;
\filldraw[fill opacity=0.8,fill=gray!20,draw=none](-7.987,3.63)--(-7.984,3.65)--(-7.966,3.642)--(-7.987,3.63)--cycle;
\draw(-7.966,3.642)--(-7.987,3.63)--(-7.987,3.63)--(-7.984,3.65);
\filldraw[fill opacity=0.8,fill=gray!20](-7.987,3.63)--(-8.013,3.649)--(-7.984,3.65)--(-7.987,3.63)--cycle;
\filldraw[fill opacity=0.8,fill=gray!20,draw=none](-7.526,4.651)--(-7.484,4.633)--(-7.493,4.637)--(-7.513,4.646)--cycle;
\draw(-7.484,4.633)--(-7.493,4.637)--(-7.513,4.646);
\filldraw[fill opacity=0.8,fill=gray!20,draw=none](-7.125,.265)--(-7.125,.264)--(-7.663,.27)--(-7.634,.291)--(-7.142,.285)--cycle;
\draw(-7.125,.264)--(-7.663,.27);
\draw(-7.634,.291)--(-7.142,.285);
\filldraw[fill opacity=0.8,fill=gray!20,draw=none](-7.788,.531)--(-7.766,.522)--(-7.786,.51)--cycle;
\draw(-7.788,.531)--(-7.766,.522)--(-7.786,.51);
\filldraw[fill opacity=0.8,fill=gray!20,draw=none](-7.756,.33)--(-7.785,.296)--(-7.795,.29)--(-7.786,.51)--(-7.766,.522)--(-7.739,.521)--(-7.721,.502)--(-7.713,.468)--(-7.716,.423)--(-7.731,.374)--cycle;
\draw(-7.786,.51)--(-7.766,.522)--(-7.739,.521)--(-7.721,.502)--(-7.713,.468)--(-7.716,.423)--(-7.731,.374)--(-7.756,.33)--(-7.785,.296)--(-7.795,.29);
\filldraw[fill opacity=0.8,fill=gray!20](-8.044,3.638)--(-8.097,3.66)--(-8.079,3.672)--(-8.035,3.644)--cycle;
\filldraw[fill opacity=0.8,fill=gray!20,draw=none](-6.775,.5)--(-6.773,.494)--(-6.774,.499)--cycle;
\draw(-6.773,.494)--(-6.774,.499);
\filldraw[fill opacity=0.8,fill=gray!20,draw=none](-6.775,.5)--(-6.774,.499)--(-6.782,.517)--(-6.779,.514)--cycle;
\draw(-6.774,.499)--(-6.782,.517)--(-6.779,.514);
\filldraw[fill opacity=0.8,fill=gray!20,draw=none](-6.947,.502)--(-6.772,.5)--(-6.774,.47)--(-6.777,.465)--(-6.946,.467)--cycle;
\draw(-6.777,.465)--(-6.946,.467)--(-6.947,.502)--(-6.772,.5);
\filldraw[fill opacity=0.8,fill=gray!20](-7.424,4.69)--(-7.43,4.745)--(-7.408,4.721)--(-7.401,4.666)--cycle;
\filldraw[fill opacity=0.8,fill=gray!20](-7.632,.463)--(-7.649,.505)--(-7.632,.487)--(-7.614,.444)--cycle;
\filldraw[fill opacity=0.8,fill=gray!20,draw=none](-6.159,.551)--(-6.178,.526)--(-6.178,.554)--cycle;
\draw(-6.178,.526)--(-6.178,.554);
\filldraw[fill opacity=0.8,fill=gray!20,draw=none](-6.159,.549)--(-6.16,.557)--(-6.195,.561)--(-6.165,.548)--cycle;
\draw(-6.16,.557)--(-6.195,.561)--(-6.165,.548);
\filldraw[fill opacity=0.8,fill=gray!20,draw=none](-6.776,.297)--(-6.77,.315)--(-6.749,.281)--(-6.778,.281)--cycle;
\draw(-6.749,.281)--(-6.778,.281);
\filldraw[fill opacity=0.8,fill=gray!20,draw=none](-6.776,.297)--(-6.775,.319)--(-6.773,.327)--(-6.762,.356)--(-6.74,.333)--(-6.762,.281)--cycle;
\draw(-6.773,.327)--(-6.762,.356)--(-6.74,.333)--(-6.762,.281)--(-6.776,.297);
\filldraw[fill opacity=0.8,fill=gray!20,draw=none](-8.545,3.094)--(-8.554,3.043)--(-8.559,3.069)--cycle;
\filldraw[fill opacity=0.8,fill=gray!20,draw=none](-7.64,4.877)--(-7.614,4.878)--(-7.612,4.865)--cycle;
\draw(-7.64,4.877)--(-7.614,4.878)--(-7.612,4.865);
\filldraw[fill opacity=0.8,fill=gray!20](-8.484,3.191)--(-8.446,3.219)--(-8.403,3.227)--(-8.424,3.202)--cycle;
\filldraw[fill opacity=0.8,fill=gray!20](-8.354,2.806)--(-8.351,2.826)--(-8.323,2.824)--(-8.354,2.806)--cycle;
\filldraw[fill opacity=0.8,fill=gray!20](-8.354,2.806)--(-8.38,2.825)--(-8.351,2.826)--(-8.354,2.806)--cycle;
\filldraw[fill opacity=0.8,fill=gray!20,draw=none](-7.822,4.694)--(-7.814,4.72)--(-7.805,4.724)--cycle;
\filldraw[fill opacity=0.8,fill=gray!20](-6.762,.356)--(-6.755,.413)--(-6.732,.388)--(-6.74,.333)--cycle;
\filldraw[fill opacity=0.8,fill=gray!20,draw=none](-7.653,.51)--(-7.664,.51)--(-7.657,.515)--cycle;
\draw(-7.653,.51)--(-7.664,.51);
\filldraw[fill opacity=0.8,fill=gray!20,draw=none](-7.77,4.512)--(-7.775,4.515)--(-7.808,4.499)--(-7.768,4.477)--(-7.75,4.486)--cycle;
\draw(-7.775,4.515)--(-7.808,4.499);
\draw(-7.768,4.477)--(-7.75,4.486);
\filldraw[fill opacity=0.8,fill=gray!20](-8.411,2.814)--(-8.464,2.836)--(-8.446,2.848)--(-8.402,2.82)--cycle;
\filldraw[fill opacity=0.8,fill=gray!20](-7.79,.221)--(-7.838,.228)--(-7.83,.233)--(-7.79,.221)--cycle;
\filldraw[fill opacity=0.8,fill=gray!20](-7.535,4.848)--(-7.56,4.874)--(-7.522,4.865)--(-7.481,4.835)--cycle;
\filldraw[fill opacity=0.8,fill=gray!20](-7.987,3.63)--(-8.035,3.644)--(-8.013,3.649)--(-7.987,3.63)--cycle;
\filldraw[fill opacity=0.8,fill=gray!20](-7.791,3.866)--(-7.798,3.921)--(-7.776,3.897)--(-7.768,3.842)--cycle;
\filldraw[fill opacity=0.8,fill=gray!20,draw=none](-6.789,.261)--(-6.797,.238)--(-6.813,.255)--(-6.807,.263)--cycle;
\draw(-6.797,.238)--(-6.813,.255)--(-6.807,.263);
\filldraw[fill opacity=0.8,fill=gray!20,draw=none](-7.509,4.644)--(-7.479,4.672)--(-7.494,4.638)--cycle;
\draw(-7.479,4.672)--(-7.494,4.638);
\filldraw[fill opacity=0.8,fill=gray!20,draw=none](-7.652,4.882)--(-7.655,4.876)--(-7.685,4.873)--(-7.676,4.893)--cycle;
\draw(-7.685,4.873)--(-7.676,4.893)--(-7.652,4.882)--(-7.655,4.876);
\filldraw[fill opacity=0.8,fill=gray!20](-6.853,.216)--(-6.813,.255)--(-6.797,.238)--(-6.842,.204)--cycle;
\filldraw[fill opacity=0.8,fill=gray!20](-7.081,.561)--(-7.043,.59)--(-7.001,.598)--(-7.021,.573)--cycle;
\filldraw[fill opacity=0.8,fill=gray!20](-6.952,.177)--(-6.977,.195)--(-6.948,.196)--(-6.952,.177)--cycle;
\filldraw[fill opacity=0.8,fill=gray!20](-6.952,.177)--(-6.948,.196)--(-6.92,.194)--(-6.952,.177)--cycle;
\filldraw[fill opacity=0.8,fill=gray!20](-7.008,.185)--(-7.061,.207)--(-7.043,.219)--(-6.999,.191)--cycle;
\filldraw[fill opacity=0.8,fill=gray!20](-7.902,4.024)--(-7.927,4.05)--(-7.889,4.041)--(-7.848,4.011)--cycle;
\filldraw[fill opacity=0.8,fill=gray!20](-8.354,2.806)--(-8.402,2.82)--(-8.38,2.825)--(-8.354,2.806)--cycle;
\filldraw[fill opacity=0.8,fill=gray!20,draw=none](-8.553,3.096)--(-8.557,3.098)--(-8.567,3.042)--(-8.554,3.036)--cycle;
\draw(-8.553,3.096)--(-8.557,3.098)--(-8.567,3.042)--(-8.554,3.036);
\filldraw[fill opacity=0.8,fill=gray!20](-7.79,.221)--(-7.748,.232)--(-7.743,.227)--(-7.79,.221)--cycle;
\filldraw[fill opacity=0.8,fill=gray!20,draw=none](-7.62,4.669)--(-7.561,4.643)--(-7.524,4.63)--(-7.502,4.623)--(-7.498,4.623)--(-7.512,4.63)--(-7.525,4.635)--cycle;
\draw(-7.561,4.643)--(-7.524,4.63)--(-7.502,4.623)--(-7.498,4.623)--(-7.512,4.63)--(-7.525,4.635);
\filldraw[fill opacity=0.8,fill=gray!20,draw=none](-7.499,4.639)--(-7.504,4.626)--(-7.62,4.669)--(-7.671,4.705)--cycle;
\draw(-7.499,4.639)--(-7.504,4.626);
\filldraw[fill opacity=0.8,fill=gray!20,draw=none](-7.506,4.639)--(-7.482,4.63)--(-7.477,4.63)--(-7.484,4.633)--(-7.526,4.651)--(-7.715,4.721)--(-7.687,4.709)--(-7.651,4.694)--(-7.598,4.673)--(-7.547,4.654)--cycle;
\draw(-7.687,4.709)--(-7.651,4.694)--(-7.598,4.673)--(-7.547,4.654)--(-7.506,4.639)--(-7.482,4.63)--(-7.477,4.63)--(-7.484,4.633);
\filldraw[fill opacity=0.5,fill=gray!20,draw=none](-8.289,2.835)--(-8.28,2.832)--(-8.308,2.852)--(-8.326,2.856)--cycle;
\draw(-8.28,2.832)--(-8.308,2.852)--(-8.326,2.856);
\filldraw[fill opacity=0.8,fill=gray!20,draw=none](-6.252,.493)--(-6.252,.291)--(-6.259,.325)--(-6.259,.435)--cycle;
\draw(-6.252,.493)--(-6.252,.291);
\draw(-6.259,.325)--(-6.259,.435);
\filldraw[fill opacity=0.8,fill=gray!20,draw=none](-6.274,.485)--(-6.297,.491)--(-6.287,.509)--(-6.284,.512)--cycle;
\draw(-6.297,.491)--(-6.287,.509)--(-6.284,.512);
\filldraw[fill opacity=0.8,fill=gray!20,draw=none](-6.196,.406)--(-6.206,.415)--(-6.209,.413)--(-6.209,.303)--(-6.206,.301)--(-6.166,.326)--cycle;
\draw(-6.209,.413)--(-6.209,.303);
\filldraw[fill opacity=0.8,fill=gray!20,draw=none](-6.207,.303)--(-6.206,.301)--(-6.196,.307)--(-6.207,.307)--cycle;
\draw(-6.196,.307)--(-6.207,.307);
\filldraw[fill opacity=0.8,fill=gray!20,draw=none](-6.206,.301)--(-6.159,.251)--(-6.159,.331)--cycle;
\draw(-6.159,.251)--(-6.159,.331);
\filldraw[fill opacity=0.8,fill=gray!20](-6.152,.279)--(-6.316,.35)--(-6.287,.307)--(-6.124,.235)--cycle;
\filldraw[fill opacity=0.8,fill=gray!20,draw=none](-7.945,.466)--(-7.948,.461)--(-7.967,.449)--(-7.949,.491)--(-7.929,.504)--cycle;
\draw(-7.948,.461)--(-7.967,.449)--(-7.949,.491)--(-7.929,.504);
\filldraw[fill opacity=0.8,fill=gray!20](-8.269,3.2)--(-8.294,3.226)--(-8.256,3.217)--(-8.215,3.187)--cycle;
\filldraw[fill opacity=0.8,fill=gray!20](-8.354,2.806)--(-8.323,2.824)--(-8.303,2.819)--(-8.354,2.806)--cycle;
\filldraw[fill opacity=0.8,fill=gray!20](-6.952,.177)--(-6.999,.191)--(-6.977,.195)--(-6.952,.177)--cycle;
\filldraw[fill opacity=0.8,fill=gray!20](-6.755,.413)--(-6.762,.467)--(-6.74,.444)--(-6.732,.388)--cycle;
\filldraw[fill opacity=0.8,fill=gray!20,draw=none](-8.536,3.098)--(-8.538,3.097)--(-8.531,3.116)--(-8.52,3.131)--(-8.523,3.124)--cycle;
\draw(-8.536,3.098)--(-8.538,3.097);
\draw(-8.52,3.131)--(-8.523,3.124);
\filldraw[fill opacity=0.8,fill=gray!20,draw=none](-8.523,3.124)--(-8.52,3.131)--(-8.515,3.136)--cycle;
\draw(-8.523,3.124)--(-8.52,3.131);
\filldraw[fill opacity=0.8,fill=gray!20,draw=none](-8.539,3.091)--(-8.547,3.04)--(-8.548,3.065)--cycle;
\filldraw[fill opacity=0.8,fill=gray!20,draw=none](-8.539,3.091)--(-8.538,3.097)--(-8.536,3.098)--cycle;
\draw(-8.538,3.097)--(-8.536,3.098);
\filldraw[fill opacity=0.8,fill=gray!20,draw=none](-8.539,3.091)--(-8.548,3.065)--(-8.548,3.073)--(-8.544,3.093)--(-8.538,3.097)--cycle;
\draw(-8.544,3.093)--(-8.538,3.097);
\filldraw[fill opacity=0.8,fill=gray!20,draw=none](-8.538,3.097)--(-8.544,3.093)--(-8.536,3.109)--(-8.531,3.116)--cycle;
\draw(-8.538,3.097)--(-8.544,3.093);
\filldraw[fill opacity=0.8,fill=gray!20,draw=none](-8.527,3.125)--(-8.531,3.116)--(-8.536,3.109)--cycle;
\filldraw[fill opacity=0.8,fill=gray!20,draw=none](-8.53,3.107)--(-8.499,3.129)--(-8.524,3.095)--(-8.536,3.1)--cycle;
\draw(-8.524,3.095)--(-8.536,3.1);
\filldraw[fill opacity=0.8,fill=gray!20,draw=none](-8.528,3.086)--(-8.527,3.096)--(-8.524,3.095)--cycle;
\draw(-8.527,3.096)--(-8.524,3.095);
\filldraw[fill opacity=0.8,fill=gray!20,draw=none](-8.54,3.09)--(-8.536,3.1)--(-8.527,3.096)--(-8.528,3.086)--(-8.543,3.052)--cycle;
\draw(-8.536,3.1)--(-8.527,3.096);
\filldraw[fill opacity=0.8,fill=gray!20,draw=none](-8.543,3.052)--(-8.528,3.086)--(-8.532,3.037)--(-8.544,3.042)--cycle;
\draw(-8.532,3.037)--(-8.544,3.042);
\filldraw[fill opacity=0.8,fill=gray!20,draw=none](-8.452,3.109)--(-8.462,3.069)--(-8.459,3.068)--(-8.447,3.094)--cycle;
\draw(-8.462,3.069)--(-8.459,3.068)--(-8.447,3.094);
\filldraw[fill opacity=0.8,fill=gray!20,draw=none](-8.454,3.065)--(-8.459,3.068)--(-8.462,3.069)--cycle;
\draw(-8.454,3.065)--(-8.459,3.068)--(-8.462,3.069);
\filldraw[fill opacity=0.8,fill=gray!20,draw=none](-8.472,3.008)--(-8.413,2.982)--(-8.389,3.025)--(-8.463,3.058)--cycle;
\draw(-8.389,3.025)--(-8.463,3.058)--(-8.472,3.008)--(-8.413,2.982);
\filldraw[fill opacity=0.8,fill=gray!20,draw=none](-8.472,3.008)--(-8.413,2.982)--(-8.389,3.025)--(-8.463,3.058)--cycle;
\draw(-8.389,3.025)--(-8.463,3.058)--(-8.472,3.008)--(-8.413,2.982);
\filldraw[fill opacity=0.8,fill=gray!20,draw=none](-8.411,3.034)--(-8.423,2.989)--(-8.411,2.984)--(-8.396,3.039)--(-8.405,3.042)--cycle;
\draw(-8.423,2.989)--(-8.411,2.984);
\draw(-8.396,3.039)--(-8.405,3.042);
\filldraw[fill opacity=0.8,fill=gray!20,draw=none](-8.409,3.015)--(-8.384,3.031)--(-8.405,3.033)--cycle;
\draw(-8.384,3.031)--(-8.405,3.033);
\filldraw[fill opacity=0.8,fill=gray!20,draw=none](-8.386,3.064)--(-8.406,3.033)--(-8.386,3.031)--cycle;
\draw(-8.406,3.033)--(-8.386,3.031);
\filldraw[fill opacity=0.8,fill=gray!20](-8.365,3.076)--(-8.529,3.147)--(-8.557,3.098)--(-8.393,3.026)--cycle;
\filldraw[fill opacity=0.8,fill=gray!20,draw=none](-7.657,4.347)--(-7.639,4.345)--(-7.648,4.323)--cycle;
\draw(-7.639,4.345)--(-7.648,4.323);
\filldraw[fill opacity=0.8,fill=gray!20,draw=none](-7.959,.417)--(-7.962,.41)--(-7.973,.402)--(-7.967,.449)--(-7.948,.461)--cycle;
\draw(-7.962,.41)--(-7.973,.402)--(-7.967,.449)--(-7.948,.461);
\filldraw[fill opacity=0.8,fill=gray!20,draw=none](-8.474,3.177)--(-8.465,3.184)--(-8.458,3.183)--cycle;
\filldraw[fill opacity=0.5,fill=gray!20](-9.891,2.812)--(-9.844,2.623)--(-9.461,2.719)--(-9.461,2.921)--cycle;
\filldraw[fill opacity=0.5,fill=gray!20](-9.844,2.623)--(-9.671,2.547)--(-9.288,2.644)--(-9.461,2.719)--cycle;
\filldraw[fill opacity=0.8,fill=gray!20](-7.785,.575)--(-7.788,.583)--(-7.764,.581)--(-7.74,.571)--cycle;
\filldraw[fill opacity=0.8,fill=gray!20](-7.831,.573)--(-7.812,.582)--(-7.788,.583)--(-7.785,.575)--cycle;
\filldraw[fill opacity=0.5,fill=gray!20,draw=none](-8.289,2.835)--(-8.267,2.823)--(-8.28,2.832)--cycle;
\draw(-8.267,2.823)--(-8.28,2.832);
\filldraw[fill opacity=0.8,fill=gray!20,draw=none](-8.289,2.835)--(-8.28,2.832)--(-8.284,2.83)--cycle;
\draw(-8.28,2.832)--(-8.284,2.83);
\filldraw[fill opacity=0.8,fill=gray!20,draw=none](-6.316,.314)--(-6.297,.314)--(-6.271,.3)--(-6.25,.281)--(-6.25,.276)--(-6.394,.277)--cycle;
\draw(-6.316,.314)--(-6.297,.314);
\draw(-6.25,.276)--(-6.394,.277);
\filldraw[fill opacity=0.8,fill=gray!20,draw=none](-6.754,.281)--(-6.25,.276)--(-6.252,.271)--(-6.274,.255)--(-6.778,.261)--cycle;
\draw(-6.754,.281)--(-6.25,.276);
\draw(-6.274,.255)--(-6.778,.261);
\filldraw[fill opacity=0.8,fill=gray!20,draw=none](-6.25,.281)--(-6.271,.3)--(-6.249,.288)--cycle;
\filldraw[fill opacity=0.8,fill=gray!20,draw=none](-6.245,.277)--(-6.25,.281)--(-6.249,.288)--(-6.241,.283)--cycle;
\filldraw[fill opacity=0.8,fill=gray!20,draw=none](-6.25,.281)--(-6.245,.277)--(-6.246,.276)--(-6.25,.276)--cycle;
\draw(-6.246,.276)--(-6.25,.276);
\filldraw[fill opacity=0.8,fill=gray!20,draw=none](-6.25,.276)--(-6.246,.276)--(-6.252,.271)--cycle;
\draw(-6.25,.276)--(-6.246,.276);
\filldraw[fill opacity=0.8,fill=gray!20,draw=none](-6.223,.271)--(-6.223,.225)--(-6.252,.214)--(-6.252,.291)--cycle;
\draw(-6.223,.271)--(-6.223,.225)--(-6.252,.214)--(-6.252,.291);
\filldraw[fill opacity=0.8,fill=gray!20,draw=none](-8.548,3.073)--(-8.547,3.04)--(-8.554,3.036)--cycle;
\draw(-8.547,3.04)--(-8.554,3.036);
\filldraw[fill opacity=0.8,fill=gray!20,draw=none](-8.284,2.83)--(-8.28,2.832)--(-8.276,2.83)--cycle;
\draw(-8.284,2.83)--(-8.28,2.832);
\filldraw[fill opacity=0.5,fill=gray!20,draw=none](-8.394,2.88)--(-8.394,2.882)--(-8.308,2.852)--(-8.293,2.842)--cycle;
\draw(-8.394,2.882)--(-8.308,2.852)--(-8.293,2.842);
\filldraw[fill opacity=0.8,fill=gray!20,draw=none](-8.394,2.88)--(-8.404,2.883)--(-8.405,2.883)--(-8.403,2.884)--(-8.394,2.882)--cycle;
\draw(-8.404,2.883)--(-8.405,2.883);
\filldraw[fill opacity=0.8,fill=gray!20,draw=none](-8.394,2.88)--(-8.394,2.882)--(-8.351,2.87)--cycle;
\filldraw[fill opacity=0.5,fill=gray!20,draw=none](-8.394,2.88)--(-8.407,2.886)--(-8.413,2.889)--(-8.394,2.882)--cycle;
\draw(-8.413,2.889)--(-8.394,2.882);
\filldraw[fill opacity=0.8,fill=gray!20,draw=none](-8.414,2.888)--(-8.412,2.889)--(-8.43,2.902)--(-8.464,2.917)--cycle;
\draw(-8.43,2.902)--(-8.464,2.917);
\filldraw[fill opacity=0.8,fill=gray!20,draw=none](-8.464,2.917)--(-8.414,2.888)--(-8.435,2.896)--(-8.439,2.898)--cycle;
\draw(-8.435,2.896)--(-8.439,2.898);
\filldraw[fill opacity=0.8,fill=gray!20,draw=none](-8.407,2.886)--(-8.413,2.889)--(-8.414,2.888)--(-8.405,2.883)--(-8.404,2.883)--cycle;
\draw(-8.405,2.883)--(-8.404,2.883);
\filldraw[fill opacity=0.8,fill=gray!20,draw=none](-8.407,2.886)--(-8.412,2.889)--(-8.413,2.889)--cycle;
\filldraw[fill opacity=0.5,fill=gray!20,draw=none](-8.407,2.886)--(-8.435,2.896)--(-8.413,2.889)--cycle;
\draw(-8.435,2.896)--(-8.413,2.889);
\filldraw[fill opacity=0.8,fill=gray!20,draw=none](-8.435,2.896)--(-8.266,2.822)--(-8.306,2.852)--(-8.34,2.867)--cycle;
\draw(-8.435,2.896)--(-8.266,2.822);
\draw(-8.306,2.852)--(-8.34,2.867);
\filldraw[fill opacity=0.8,fill=gray!20](-6.866,.571)--(-6.891,.597)--(-6.853,.587)--(-6.813,.558)--cycle;
\filldraw[fill opacity=0.8,fill=gray!20](-6.952,.177)--(-6.92,.194)--(-6.901,.19)--(-6.952,.177)--cycle;
\filldraw[fill opacity=0.8,fill=gray!20,draw=none](-8.532,3.034)--(-8.529,3.003)--(-8.535,3.011)--(-8.543,3.031)--(-8.544,3.042)--(-8.533,3.037)--cycle;
\draw(-8.544,3.042)--(-8.533,3.037);
\filldraw[fill opacity=0.8,fill=gray!20,draw=none](-8.532,3.034)--(-8.511,2.98)--(-8.529,3.003)--cycle;
\filldraw[fill opacity=0.8,fill=gray!20,draw=none](-8.511,2.98)--(-8.505,2.974)--(-8.509,2.975)--cycle;
\draw(-8.505,2.974)--(-8.509,2.975);
\filldraw[fill opacity=0.8,fill=gray!20,draw=none](-8.527,2.992)--(-8.529,3.003)--(-8.511,2.98)--(-8.509,2.975)--(-8.523,2.981)--cycle;
\draw(-8.509,2.975)--(-8.523,2.981);
\filldraw[fill opacity=0.8,fill=gray!20,draw=none](-8.535,3.011)--(-8.529,3.003)--(-8.527,2.992)--cycle;
\filldraw[fill opacity=0.8,fill=gray!20,draw=none](-8.223,2.85)--(-8.248,2.831)--(-8.245,2.833)--(-8.199,2.867)--(-8.214,2.857)--cycle;
\draw(-8.248,2.831)--(-8.245,2.833)--(-8.199,2.867)--(-8.214,2.857);
\filldraw[fill opacity=0.8,fill=gray!20,draw=none](-8.204,2.863)--(-8.196,2.86)--(-8.214,2.856)--cycle;
\filldraw[fill opacity=0.8,fill=gray!20,draw=none](-8.309,2.907)--(-8.258,2.884)--(-8.288,2.878)--(-8.318,2.876)--(-8.354,2.892)--cycle;
\draw(-8.318,2.876)--(-8.354,2.892)--(-8.309,2.907)--(-8.258,2.884);
\filldraw[fill opacity=0.8,fill=gray!20,draw=none](-8.309,2.907)--(-8.258,2.884)--(-8.288,2.878)--(-8.318,2.876)--(-8.354,2.892)--cycle;
\draw(-8.318,2.876)--(-8.354,2.892)--(-8.309,2.907)--(-8.258,2.884);
\filldraw[fill opacity=0.8,fill=gray!20,draw=none](-8.402,2.894)--(-8.354,2.873)--(-8.288,2.878)--(-8.413,2.933)--cycle;
\draw(-8.402,2.894)--(-8.354,2.873);
\draw(-8.288,2.878)--(-8.413,2.933);
\filldraw[fill opacity=0.8,fill=gray!20,draw=none](-8.254,2.829)--(-8.237,2.822)--(-8.203,2.842)--(-8.209,2.844)--cycle;
\draw(-8.254,2.829)--(-8.237,2.822)--(-8.203,2.842)--(-8.209,2.844);
\filldraw[fill opacity=0.8,fill=gray!20,draw=none](-8.463,2.959)--(-8.416,2.939)--(-8.413,2.982)--(-8.472,3.008)--cycle;
\draw(-8.413,2.982)--(-8.472,3.008)--(-8.463,2.959)--(-8.416,2.939);
\filldraw[fill opacity=0.8,fill=gray!20,draw=none](-8.257,2.829)--(-8.245,2.833)--(-8.25,2.83)--cycle;
\draw(-8.257,2.829)--(-8.245,2.833)--(-8.25,2.83);
\filldraw[fill opacity=0.8,fill=gray!20,draw=none](-8.463,2.959)--(-8.416,2.939)--(-8.413,2.982)--(-8.472,3.008)--cycle;
\draw(-8.413,2.982)--(-8.472,3.008)--(-8.463,2.959)--(-8.416,2.939);
\filldraw[fill opacity=0.8,fill=gray!20,draw=none](-8.428,2.981)--(-8.429,2.965)--(-8.413,2.933)--(-8.288,2.878)--(-8.205,2.894)--(-8.423,2.989)--cycle;
\draw(-8.413,2.933)--(-8.288,2.878);
\draw(-8.205,2.894)--(-8.423,2.989);
\filldraw[fill opacity=0.8,fill=gray!20,draw=none](-8.445,2.912)--(-8.402,2.894)--(-8.413,2.933)--(-8.431,2.941)--cycle;
\draw(-8.445,2.912)--(-8.402,2.894);
\draw(-8.413,2.933)--(-8.431,2.941);
\filldraw[fill opacity=0.8,fill=gray!20,draw=none](-8.344,2.868)--(-8.254,2.829)--(-8.209,2.844)--(-8.217,2.848)--cycle;
\draw(-8.344,2.868)--(-8.254,2.829);
\draw(-8.209,2.844)--(-8.217,2.848);
\filldraw[fill opacity=0.8,fill=gray!20,draw=none](-8.25,2.83)--(-8.248,2.831)--(-8.225,2.849)--(-8.231,2.846)--(-8.242,2.837)--cycle;
\draw(-8.25,2.83)--(-8.248,2.831);
\draw(-8.231,2.846)--(-8.242,2.837);
\filldraw[fill opacity=0.8,fill=gray!20,draw=none](-8.25,2.83)--(-8.242,2.837)--(-8.263,2.822)--cycle;
\draw(-8.242,2.837)--(-8.263,2.822)--(-8.25,2.83);
\filldraw[fill opacity=0.8,fill=gray!20,draw=none](-8.223,2.85)--(-8.203,2.842)--(-8.17,2.879)--(-8.179,2.883)--cycle;
\draw(-8.223,2.85)--(-8.203,2.842)--(-8.17,2.879)--(-8.179,2.883);
\filldraw[fill opacity=0.8,fill=gray!20,draw=none](-8.354,2.873)--(-8.344,2.868)--(-8.217,2.848)--(-8.288,2.878)--cycle;
\draw(-8.354,2.873)--(-8.344,2.868);
\draw(-8.217,2.848)--(-8.288,2.878);
\filldraw[fill opacity=0.8,fill=gray!20,draw=none](-8.225,2.849)--(-8.216,2.856)--(-8.225,2.85)--(-8.231,2.846)--cycle;
\draw(-8.216,2.856)--(-8.225,2.85)--(-8.231,2.846);
\filldraw[fill opacity=0.8,fill=gray!20,draw=none](-8.223,2.85)--(-8.214,2.857)--(-8.216,2.856)--cycle;
\draw(-8.214,2.857)--(-8.216,2.856);
\filldraw[fill opacity=0.8,fill=gray!20,draw=none](-8.214,2.857)--(-8.203,2.865)--(-8.171,2.906)--(-8.186,2.897)--cycle;
\draw(-8.214,2.857)--(-8.203,2.865);
\draw(-8.171,2.906)--(-8.186,2.897);
\filldraw[fill opacity=0.8,fill=gray!20,draw=none](-8.413,2.933)--(-8.406,2.907)--(-8.398,2.903)--cycle;
\draw(-8.406,2.907)--(-8.398,2.903);
\filldraw[fill opacity=0.8,fill=gray!20,draw=none](-8.406,2.907)--(-8.398,2.903)--(-8.413,2.933)--cycle;
\draw(-8.406,2.907)--(-8.398,2.903);
\filldraw[fill opacity=0.8,fill=gray!20,draw=none](-8.269,2.87)--(-8.223,2.85)--(-8.192,2.874)--(-8.189,2.886)--cycle;
\draw(-8.269,2.87)--(-8.223,2.85);
\filldraw[fill opacity=0.8,fill=gray!20,draw=none](-8.276,2.83)--(-8.253,2.829)--(-8.306,2.852)--cycle;
\draw(-8.253,2.829)--(-8.306,2.852);
\filldraw[fill opacity=0.8,fill=gray!20,draw=none](-8.269,2.825)--(-8.265,2.826)--(-8.268,2.824)--cycle;
\draw(-8.265,2.826)--(-8.268,2.824);
\filldraw[fill opacity=0.8,fill=gray!20,draw=none](-8.268,2.824)--(-8.257,2.829)--(-8.25,2.83)--(-8.263,2.822)--(-8.263,2.821)--cycle;
\draw(-8.268,2.824)--(-8.257,2.829);
\draw(-8.25,2.83)--(-8.263,2.822)--(-8.263,2.821);
\filldraw[fill opacity=0.8,fill=gray!20,draw=none](-8.264,2.822)--(-8.263,2.822)--(-8.225,2.85)--(-8.285,2.839)--(-8.285,2.838)--cycle;
\draw(-8.264,2.822)--(-8.263,2.822)--(-8.225,2.85)--(-8.285,2.839)--(-8.285,2.838);
\filldraw[fill opacity=0.8,fill=gray!20,draw=none](-8.288,2.878)--(-8.269,2.87)--(-8.189,2.886)--(-8.188,2.887)--(-8.205,2.894)--cycle;
\draw(-8.288,2.878)--(-8.269,2.87);
\draw(-8.188,2.887)--(-8.205,2.894);
\filldraw[fill opacity=0.8,fill=gray!20,draw=none](-8.413,2.933)--(-8.414,2.938)--(-8.416,2.939)--cycle;
\draw(-8.414,2.938)--(-8.416,2.939);
\filldraw[fill opacity=0.8,fill=gray!20,draw=none](-8.398,2.903)--(-8.383,2.897)--(-8.407,2.935)--(-8.416,2.939)--cycle;
\draw(-8.398,2.903)--(-8.383,2.897);
\draw(-8.407,2.935)--(-8.416,2.939);
\filldraw[fill opacity=0.8,fill=gray!20,draw=none](-8.301,2.868)--(-8.258,2.855)--(-8.282,2.846)--(-8.326,2.865)--cycle;
\draw(-8.282,2.846)--(-8.326,2.865);
\filldraw[fill opacity=0.8,fill=gray!20,draw=none](-8.278,2.869)--(-8.251,2.858)--(-8.258,2.855)--(-8.301,2.868)--cycle;
\filldraw[fill opacity=0.8,fill=gray!20,draw=none](-8.413,2.933)--(-8.398,2.903)--(-8.361,2.887)--(-8.366,2.897)--(-8.382,2.92)--(-8.388,2.926)--(-8.414,2.938)--cycle;
\draw(-8.398,2.903)--(-8.361,2.887);
\draw(-8.388,2.926)--(-8.414,2.938);
\filldraw[fill opacity=0.8,fill=gray!20,draw=none](-8.221,2.853)--(-8.214,2.857)--(-8.192,2.888)--cycle;
\draw(-8.221,2.853)--(-8.214,2.857);
\filldraw[fill opacity=0.8,fill=gray!20,draw=none](-8.383,2.897)--(-8.361,2.887)--(-8.366,2.897)--(-8.382,2.92)--(-8.388,2.926)--(-8.407,2.935)--cycle;
\draw(-8.383,2.897)--(-8.361,2.887);
\draw(-8.388,2.926)--(-8.407,2.935);
\filldraw[fill opacity=0.8,fill=gray!20,draw=none](-8.429,2.965)--(-8.431,2.941)--(-8.413,2.933)--cycle;
\draw(-8.431,2.941)--(-8.413,2.933);
\filldraw[fill opacity=0.8,fill=gray!20,draw=none](-8.416,2.939)--(-8.388,2.926)--(-8.397,2.975)--(-8.413,2.982)--cycle;
\draw(-8.416,2.939)--(-8.388,2.926);
\draw(-8.397,2.975)--(-8.413,2.982);
\filldraw[fill opacity=0.8,fill=gray!20,draw=none](-8.416,2.939)--(-8.414,2.938)--(-8.397,2.973)--(-8.397,2.975)--(-8.413,2.982)--cycle;
\draw(-8.416,2.939)--(-8.414,2.938);
\draw(-8.397,2.975)--(-8.413,2.982);
\filldraw[fill opacity=0.8,fill=gray!20,draw=none](-8.251,2.858)--(-8.278,2.869)--(-8.253,2.872)--(-8.235,2.864)--cycle;
\draw(-8.253,2.872)--(-8.235,2.864);
\filldraw[fill opacity=0.8,fill=gray!20,draw=none](-8.21,2.896)--(-8.192,2.888)--(-8.184,2.898)--(-8.173,2.922)--cycle;
\draw(-8.21,2.896)--(-8.192,2.888);
\filldraw[fill opacity=0.8,fill=gray!20,draw=none](-8.221,2.853)--(-8.192,2.888)--(-8.186,2.897)--(-8.189,2.895)--(-8.198,2.888)--(-8.225,2.85)--cycle;
\draw(-8.186,2.897)--(-8.189,2.895);
\draw(-8.198,2.888)--(-8.225,2.85)--(-8.221,2.853);
\filldraw[fill opacity=0.8,fill=gray!20,draw=none](-8.397,2.973)--(-8.414,2.938)--(-8.388,2.926)--cycle;
\draw(-8.414,2.938)--(-8.388,2.926);
\filldraw[fill opacity=0.8,fill=gray!20,draw=none](-8.318,2.876)--(-8.277,2.858)--(-8.324,2.864)--(-8.372,2.885)--cycle;
\draw(-8.318,2.876)--(-8.277,2.858);
\draw(-8.324,2.864)--(-8.372,2.885);
\filldraw[fill opacity=0.8,fill=gray!20,draw=none](-8.318,2.876)--(-8.277,2.858)--(-8.324,2.864)--(-8.372,2.885)--cycle;
\draw(-8.318,2.876)--(-8.277,2.858);
\draw(-8.324,2.864)--(-8.372,2.885);
\filldraw[fill opacity=0.8,fill=gray!20,draw=none](-8.395,2.907)--(-8.366,2.89)--(-8.366,2.897)--(-8.382,2.92)--(-8.406,2.922)--cycle;
\draw(-8.366,2.89)--(-8.366,2.897);
\draw(-8.382,2.92)--(-8.406,2.922);
\filldraw[fill opacity=0.8,fill=gray!20,draw=none](-8.413,2.958)--(-8.406,2.922)--(-8.382,2.92)--cycle;
\draw(-8.406,2.922)--(-8.382,2.92);
\filldraw[fill opacity=0.8,fill=gray!20,draw=none](-8.406,2.95)--(-8.382,2.92)--(-8.368,2.919)--(-8.377,2.973)--(-8.397,2.975)--cycle;
\draw(-8.382,2.92)--(-8.368,2.919);
\draw(-8.377,2.973)--(-8.397,2.975);
\filldraw[fill opacity=0.8,fill=gray!20,draw=none](-8.413,2.958)--(-8.406,2.95)--(-8.397,2.975)--(-8.417,2.976)--cycle;
\draw(-8.397,2.975)--(-8.417,2.976);
\filldraw[fill opacity=0.8,fill=gray!20,draw=none](-8.403,2.97)--(-8.554,3.036)--(-8.545,2.983)--(-8.393,2.917)--cycle;
\draw(-8.545,2.983)--(-8.393,2.917)--(-8.403,2.97)--(-8.554,3.036);
\filldraw[fill opacity=0.8,fill=gray!20,draw=none](-7.814,4.496)--(-7.812,4.492)--(-7.814,4.489)--cycle;
\draw(-7.812,4.492)--(-7.814,4.489);
\filldraw[fill opacity=0.8,fill=gray!20,draw=none](-7.797,4.546)--(-7.811,4.552)--(-7.807,4.56)--cycle;
\draw(-7.811,4.552)--(-7.807,4.56);
\filldraw[fill opacity=0.8,fill=gray!20,draw=none](-7.877,4.394)--(-7.868,4.418)--(-7.851,4.462)--(-7.85,4.462)--(-7.877,4.394)--cycle;
\draw(-7.85,4.462)--(-7.877,4.394)--(-7.877,4.394);
\filldraw[fill opacity=0.8,fill=gray!20,draw=none](-7.876,4.393)--(-7.86,4.441)--(-7.824,4.523)--(-7.814,4.496)--(-7.814,4.489)--(-7.859,4.388)--cycle;
\draw(-7.86,4.441)--(-7.824,4.523);
\draw(-7.814,4.489)--(-7.859,4.388);
\filldraw[fill opacity=0.8,fill=gray!20,draw=none](-7.417,4.78)--(-7.414,4.778)--(-7.426,4.75)--(-7.439,4.765)--cycle;
\draw(-7.417,4.78)--(-7.414,4.778)--(-7.426,4.75);
\filldraw[fill opacity=0.8,fill=gray!20,draw=none](-7.417,4.78)--(-7.439,4.765)--(-7.43,4.785)--cycle;
\draw(-7.439,4.765)--(-7.43,4.785)--(-7.417,4.78);
\filldraw[fill opacity=0.8,fill=gray!20,draw=none](-7.419,4.771)--(-7.425,4.761)--(-7.426,4.75)--(-7.422,4.759)--cycle;
\draw(-7.426,4.75)--(-7.422,4.759);
\filldraw[fill opacity=0.8,fill=gray!20](-7.43,4.745)--(-7.45,4.794)--(-7.43,4.773)--(-7.408,4.721)--cycle;
\filldraw[fill opacity=0.8,fill=gray!20,draw=none](-7.62,4.294)--(-7.626,4.297)--(-7.621,4.309)--cycle;
\draw(-7.62,4.294)--(-7.626,4.297)--(-7.621,4.309);
\filldraw[fill opacity=0.8,fill=gray!20,draw=none](-7.578,4.327)--(-7.594,4.335)--(-7.612,4.332)--cycle;
\draw(-7.578,4.327)--(-7.594,4.335);
\filldraw[fill opacity=0.8,fill=gray!20,draw=none](-7.614,4.291)--(-7.62,4.294)--(-7.621,4.309)--(-7.591,4.338)--(-7.602,4.309)--cycle;
\draw(-7.614,4.291)--(-7.62,4.294);
\draw(-7.591,4.338)--(-7.602,4.309);
\filldraw[fill opacity=0.8,fill=gray!20,draw=none](-7.594,4.335)--(-7.621,4.309)--(-7.612,4.332)--cycle;
\draw(-7.621,4.309)--(-7.612,4.332);
\filldraw[fill opacity=0.8,fill=gray!20,draw=none](-7.594,4.335)--(-7.71,4.393)--(-7.737,4.394)--(-7.612,4.332)--cycle;
\draw(-7.594,4.335)--(-7.71,4.393)--(-7.737,4.394)--(-7.612,4.332);
\filldraw[fill opacity=0.8,fill=gray!20,draw=none](-8.022,4.052)--(-8.036,4.052)--(-8.03,4.054)--cycle;
\draw(-8.022,4.052)--(-8.036,4.052)--(-8.03,4.054);
\filldraw[fill opacity=0.8,fill=gray!20,draw=none](-8.033,4.054)--(-8.03,4.054)--(-8.032,4.053)--cycle;
\draw(-8.03,4.054)--(-8.032,4.053);
\filldraw[fill opacity=0.8,fill=gray!20,draw=none](-8.103,3.896)--(-8.03,4.061)--(-8.013,4.042)--(-8.082,3.886)--cycle;
\draw(-8.013,4.042)--(-8.082,3.886)--(-8.103,3.896)--(-8.03,4.061);
\filldraw[fill opacity=0.8,fill=gray!20](-6.901,.19)--(-6.853,.216)--(-6.842,.204)--(-6.895,.183)--cycle;
\filldraw[fill opacity=0.8,fill=gray!20](-7.81,4.778)--(-7.775,4.822)--(-7.75,4.839)--(-7.779,4.799)--cycle;
\filldraw[fill opacity=0.8,fill=gray!20](-7.649,.505)--(-7.675,.539)--(-7.661,.525)--(-7.632,.487)--cycle;
\filldraw[fill opacity=0.8,fill=gray!20,draw=none](-7.459,4.539)--(-7.457,4.555)--(-7.43,4.559)--(-7.453,4.53)--cycle;
\draw(-7.43,4.559)--(-7.453,4.53);
\filldraw[fill opacity=0.8,fill=gray!20,draw=none](-8.082,3.886)--(-8.15,3.733)--(-8.163,3.735)--(-8.143,3.806)--(-8.103,3.896)--cycle;
\draw(-8.143,3.806)--(-8.103,3.896)--(-8.082,3.886)--(-8.15,3.733);
\filldraw[fill opacity=0.8,fill=gray!20,draw=none](-6.192,.548)--(-6.178,.554)--(-6.178,.526)--(-6.205,.489)--(-6.223,.496)--(-6.223,.513)--cycle;
\draw(-6.178,.554)--(-6.178,.526);
\draw(-6.223,.496)--(-6.223,.513);
\filldraw[fill opacity=0.8,fill=gray!20,draw=none](-6.164,.531)--(-6.158,.545)--(-6.195,.561)--(-6.207,.557)--cycle;
\draw(-6.158,.545)--(-6.195,.561)--(-6.207,.557);
\filldraw[fill opacity=0.8,fill=gray!20,draw=none](-8.166,3.724)--(-8.183,3.658)--(-8.338,3.311)--(-8.337,3.37)--(-8.174,3.736)--cycle;
\draw(-8.183,3.658)--(-8.338,3.311);
\draw(-8.337,3.37)--(-8.174,3.736);
\filldraw[fill opacity=0.8,fill=gray!20,draw=none](-7.69,4.7)--(-7.527,4.607)--(-7.524,4.625)--cycle;
\draw(-7.527,4.607)--(-7.524,4.625);
\filldraw[fill opacity=0.8,fill=gray!20,draw=none](-7.8,3.954)--(-7.8,3.952)--(-7.798,3.948)--(-7.787,3.972)--cycle;
\draw(-7.798,3.948)--(-7.787,3.972);
\filldraw[fill opacity=0.8,fill=gray!20,draw=none](-7.8,3.952)--(-7.8,3.943)--(-7.798,3.948)--cycle;
\draw(-7.8,3.943)--(-7.798,3.948);
\filldraw[fill opacity=0.8,fill=gray!20](-7.798,3.921)--(-7.817,3.97)--(-7.797,3.949)--(-7.776,3.897)--cycle;
\filldraw[fill opacity=0.8,fill=gray!20,draw=none](-7.525,4.622)--(-7.525,4.626)--(-7.522,4.624)--cycle;
\draw(-7.525,4.626)--(-7.522,4.624);
\filldraw[fill opacity=0.8,fill=gray!20,draw=none](-7.923,.511)--(-7.929,.504)--(-7.949,.491)--(-7.92,.528)--(-7.904,.538)--cycle;
\draw(-7.929,.504)--(-7.949,.491)--(-7.92,.528)--(-7.904,.538);
\filldraw[fill opacity=0.8,fill=gray!20,draw=none](-7.54,4.656)--(-7.705,4.756)--(-7.711,4.747)--(-7.715,4.721)--cycle;
\draw(-7.705,4.756)--(-7.711,4.747)--(-7.715,4.721);
\filldraw[fill opacity=0.8,fill=gray!20,draw=none](-8.338,3.311)--(-8.373,3.233)--(-8.381,3.232)--(-8.389,3.255)--(-8.337,3.37)--cycle;
\draw(-8.338,3.311)--(-8.373,3.233);
\draw(-8.389,3.255)--(-8.337,3.37);
\filldraw[fill opacity=0.8,fill=gray!20,draw=none](-7.987,3.63)--(-7.966,3.642)--(-7.953,3.639)--(-7.987,3.63)--cycle;
\draw(-7.953,3.639)--(-7.987,3.63)--(-7.987,3.63)--(-7.966,3.642);
\filldraw[fill opacity=0.8,fill=gray!20,draw=none](-8.474,3.177)--(-8.501,3.167)--(-8.489,3.184)--(-8.465,3.184)--cycle;
\draw(-8.501,3.167)--(-8.489,3.184);
\filldraw[fill opacity=0.8,fill=gray!20,draw=none](-8.389,3.228)--(-8.403,3.227)--(-8.397,3.23)--cycle;
\draw(-8.389,3.228)--(-8.403,3.227)--(-8.397,3.23);
\filldraw[fill opacity=0.8,fill=gray!20,draw=none](-8.4,3.23)--(-8.397,3.23)--(-8.399,3.229)--cycle;
\draw(-8.397,3.23)--(-8.399,3.229);
\filldraw[fill opacity=0.8,fill=gray!20,draw=none](-8.38,3.218)--(-8.391,3.193)--(-8.404,3.185)--(-8.42,3.184)--(-8.397,3.237)--cycle;
\draw(-8.38,3.218)--(-8.391,3.193);
\draw(-8.42,3.184)--(-8.397,3.237);
\filldraw[fill opacity=0.8,fill=gray!20,draw=none](-8.391,3.193)--(-8.394,3.185)--(-8.404,3.185)--cycle;
\draw(-8.391,3.193)--(-8.394,3.185);
\filldraw[fill opacity=0.8,fill=gray!20,draw=none](-6.301,.327)--(-6.287,.313)--(-6.316,.314)--cycle;
\draw(-6.287,.313)--(-6.316,.314);
\filldraw[fill opacity=0.8,fill=gray!20](-8.177,3.954)--(-8.142,3.998)--(-8.117,4.015)--(-8.146,3.975)--cycle;
\filldraw[fill opacity=0.8,fill=gray!20,draw=none](-7.525,4.606)--(-7.525,4.622)--(-7.522,4.624)--(-7.515,4.622)--(-7.522,4.607)--cycle;
\draw(-7.522,4.624)--(-7.515,4.622)--(-7.522,4.607);
\filldraw[fill opacity=0.8,fill=gray!20,draw=none](-8.505,3.155)--(-8.511,3.153)--(-8.501,3.167)--(-8.474,3.177)--cycle;
\draw(-8.511,3.153)--(-8.501,3.167);
\filldraw[fill opacity=0.8,fill=gray!20,draw=none](-7.824,4.523)--(-7.815,4.542)--(-7.814,4.496)--cycle;
\draw(-7.824,4.523)--(-7.815,4.542);
\filldraw[fill opacity=0.8,fill=gray!20,draw=none](-7.867,.566)--(-7.84,.575)--(-7.826,.575)--(-7.831,.573)--cycle;
\draw(-7.826,.575)--(-7.831,.573)--(-7.867,.566)--(-7.84,.575);
\filldraw[fill opacity=0.8,fill=gray!20,draw=none](-8.532,3.034)--(-8.533,3.037)--(-8.532,3.037)--cycle;
\draw(-8.533,3.037)--(-8.532,3.037);
\filldraw[fill opacity=0.8,fill=gray!20,draw=none](-8.413,2.982)--(-8.397,2.975)--(-8.388,3.025)--(-8.389,3.025)--cycle;
\draw(-8.413,2.982)--(-8.397,2.975);
\draw(-8.388,3.025)--(-8.389,3.025);
\filldraw[fill opacity=0.8,fill=gray!20,draw=none](-7.942,3.63)--(-7.94,3.628)--(-7.956,3.594)--(-7.958,3.622)--cycle;
\draw(-7.94,3.628)--(-7.956,3.594);
\filldraw[fill opacity=0.8,fill=gray!20,draw=none](-8.413,2.982)--(-8.397,2.975)--(-8.388,3.025)--(-8.389,3.025)--cycle;
\draw(-8.413,2.982)--(-8.397,2.975);
\draw(-8.388,3.025)--(-8.389,3.025);
\filldraw[fill opacity=0.8,fill=gray!20,draw=none](-8.409,3.015)--(-8.417,2.976)--(-8.377,2.973)--(-8.368,3.03)--(-8.384,3.031)--cycle;
\draw(-8.417,2.976)--(-8.377,2.973);
\draw(-8.368,3.03)--(-8.384,3.031);
\filldraw[fill opacity=0.8,fill=gray!20,draw=none](-8.393,3.026)--(-8.553,3.096)--(-8.554,3.036)--(-8.403,2.97)--cycle;
\draw(-8.554,3.036)--(-8.403,2.97)--(-8.393,3.026)--(-8.553,3.096);
\filldraw[fill opacity=0.8,fill=gray!20,draw=none](-7.955,.337)--(-7.943,.315)--(-7.942,.321)--cycle;
\filldraw[fill opacity=0.8,fill=gray!20,draw=none](-8.897,.7)--(-8.897,.703)--(-8.843,.699)--(-8.851,.69)--cycle;
\draw(-8.897,.7)--(-8.897,.703)--(-8.843,.699)--(-8.851,.69);
\filldraw[fill opacity=0.8,fill=gray!20,draw=none](-8.851,.69)--(-8.843,.699)--(-8.823,.694)--(-8.814,.684)--cycle;
\draw(-8.851,.69)--(-8.843,.699)--(-8.823,.694);
\filldraw[fill opacity=0.8,fill=gray!20,draw=none](-8.823,.694)--(-8.833,.696)--(-8.834,.704)--cycle;
\draw(-8.823,.694)--(-8.833,.696);
\filldraw[fill opacity=0.8,fill=gray!20,draw=none](-8.948,.741)--(-8.964,.748)--(-8.937,.747)--cycle;
\draw(-8.948,.741)--(-8.964,.748);
\filldraw[fill opacity=0.8,fill=gray!20](-8.973,.762)--(-7.86,.298)--(-7.842,.279)--(-8.954,.743)--cycle;
\filldraw[fill opacity=0.8,fill=gray!20,draw=none](-7.516,4.578)--(-7.496,4.582)--(-7.504,4.561)--cycle;
\draw(-7.496,4.582)--(-7.504,4.561);
\filldraw[fill opacity=0.8,fill=gray!20](-7.987,3.63)--(-8.044,3.638)--(-8.035,3.644)--(-7.987,3.63)--cycle;
\filldraw[fill opacity=0.8,fill=gray!20,draw=none](-7.513,4.621)--(-7.515,4.622)--(-7.522,4.624)--cycle;
\draw(-7.513,4.621)--(-7.515,4.622)--(-7.522,4.624);
\filldraw[fill opacity=0.8,fill=gray!20,draw=none](-8.165,3.097)--(-8.168,3.105)--(-8.167,3.119)--(-8.156,3.106)--(-8.143,3.073)--cycle;
\draw(-8.156,3.106)--(-8.143,3.073)--(-8.165,3.097)--(-8.168,3.105);
\filldraw[fill opacity=0.8,fill=gray!20,draw=none](-7.967,.374)--(-7.969,.372)--(-7.973,.402)--(-7.962,.41)--cycle;
\draw(-7.969,.372)--(-7.973,.402)--(-7.962,.41);
\filldraw[fill opacity=0.8,fill=gray!20,draw=none](-7.746,.529)--(-7.779,.53)--(-7.757,.529)--cycle;
\draw(-7.746,.529)--(-7.779,.53);
\filldraw[fill opacity=0.8,fill=gray!20,draw=none](-7.873,.561)--(-7.869,.564)--(-7.867,.566)--(-7.875,.56)--cycle;
\draw(-7.869,.564)--(-7.867,.566)--(-7.875,.56);
\filldraw[fill opacity=0.8,fill=gray!20,draw=none](-7.84,.575)--(-7.83,.578)--(-7.812,.582)--(-7.826,.575)--cycle;
\draw(-7.84,.575)--(-7.83,.578)--(-7.812,.582)--(-7.826,.575);
\filldraw[fill opacity=0.8,fill=gray!20,draw=none](-7.872,.562)--(-7.865,.564)--(-7.843,.571)--(-7.84,.575)--(-7.867,.566)--cycle;
\draw(-7.84,.575)--(-7.867,.566)--(-7.872,.562);
\filldraw[fill opacity=0.8,fill=gray!20,draw=none](-7.843,.571)--(-7.865,.564)--(-7.86,.564)--(-7.843,.571)--cycle;
\draw(-7.86,.564)--(-7.843,.571);
\filldraw[fill opacity=0.8,fill=gray!20,draw=none](-7.84,.564)--(-7.843,.571)--(-7.86,.564)--cycle;
\draw(-7.843,.571)--(-7.86,.564);
\filldraw[fill opacity=0.8,fill=gray!20,draw=none](-7.866,.574)--(-7.739,.521)--(-7.766,.522)--(-7.874,.567)--cycle;
\draw(-7.866,.574)--(-7.739,.521)--(-7.766,.522)--(-7.874,.567);
\filldraw[fill opacity=0.8,fill=gray!20,draw=none](-8.512,3.151)--(-8.513,3.15)--(-8.511,3.153)--(-8.505,3.155)--cycle;
\draw(-8.512,3.151)--(-8.513,3.15)--(-8.511,3.153);
\filldraw[fill opacity=0.8,fill=gray!20,draw=none](-6.192,.548)--(-6.207,.557)--(-6.245,.544)--(-6.223,.535)--cycle;
\draw(-6.207,.557)--(-6.245,.544)--(-6.223,.535);
\filldraw[fill opacity=0.8,fill=gray!20,draw=none](-7.499,4.583)--(-7.508,4.563)--(-7.52,4.58)--cycle;
\draw(-7.499,4.583)--(-7.508,4.563);
\filldraw[fill opacity=0.8,fill=gray!20,draw=none](-7.426,4.75)--(-7.478,4.63)--(-7.494,4.638)--(-7.439,4.765)--cycle;
\draw(-7.426,4.75)--(-7.478,4.63);
\draw(-7.494,4.638)--(-7.439,4.765);
\filldraw[fill opacity=0.8,fill=gray!20,draw=none](-8.513,3.15)--(-8.513,3.15)--(-8.512,3.151)--cycle;
\draw(-8.513,3.15)--(-8.513,3.15)--(-8.512,3.151);
\filldraw[fill opacity=0.8,fill=gray!20,draw=none](-6.776,.297)--(-6.778,.281)--(-6.782,.282)--cycle;
\draw(-6.778,.281)--(-6.782,.282);
\filldraw[fill opacity=0.8,fill=gray!20,draw=none](-6.782,.282)--(-6.754,.281)--(-6.778,.261)--(-6.779,.261)--cycle;
\draw(-6.782,.282)--(-6.754,.281);
\draw(-6.778,.261)--(-6.779,.261);
\filldraw[fill opacity=0.8,fill=gray!20,draw=none](-6.797,.238)--(-6.776,.297)--(-6.762,.281)--(-6.797,.238)--cycle;
\draw(-6.776,.297)--(-6.762,.281)--(-6.797,.238)--(-6.797,.238);
\filldraw[fill opacity=0.8,fill=gray!20,draw=none](-8.284,2.83)--(-8.276,2.83)--(-8.268,2.824)--(-8.279,2.82)--cycle;
\draw(-8.268,2.824)--(-8.279,2.82);
\filldraw[fill opacity=0.8,fill=gray!20,draw=none](-8.303,2.819)--(-8.284,2.83)--(-8.279,2.82)--(-8.298,2.813)--cycle;
\draw(-8.279,2.82)--(-8.298,2.813)--(-8.303,2.819)--(-8.284,2.83);
\filldraw[fill opacity=0.8,fill=gray!20,draw=none](-8.518,3.143)--(-8.515,3.149)--(-8.513,3.15)--(-8.513,3.15)--cycle;
\draw(-8.515,3.149)--(-8.513,3.15)--(-8.513,3.15);
\filldraw[fill opacity=0.8,fill=gray!20,draw=none](-6.773,.494)--(-6.772,.5)--(-6.758,.5)--(-6.768,.481)--cycle;
\draw(-6.772,.5)--(-6.758,.5);
\filldraw[fill opacity=0.5,fill=gray!20](-9.572,-.993)--(-9.642,-1.005)--(-10.084,-.854)--(-10.005,-.844)--cycle;
\filldraw[fill opacity=0.8,fill=gray!20,draw=none](-8.52,3.146)--(-8.511,3.153)--(-8.513,3.15)--cycle;
\draw(-8.511,3.153)--(-8.513,3.15)--(-8.52,3.146);
\filldraw[fill opacity=0.8,fill=gray!20,draw=none](-7.865,.564)--(-7.872,.562)--(-7.875,.56)--cycle;
\draw(-7.872,.562)--(-7.875,.56);
\filldraw[fill opacity=0.8,fill=gray!20,draw=none](-7.869,.565)--(-7.788,.531)--(-7.787,.53)--(-7.86,.531)--(-7.895,.545)--cycle;
\draw(-7.869,.565)--(-7.788,.531);
\draw(-7.86,.531)--(-7.895,.545);
\filldraw[fill opacity=0.8,fill=gray!20](-7.79,.221)--(-7.833,.223)--(-7.838,.228)--(-7.79,.221)--cycle;
\filldraw[fill opacity=0.8,fill=gray!20](-7.833,.223)--(-7.872,.236)--(-7.882,.247)--(-7.838,.228)--cycle;
\filldraw[fill opacity=0.8,fill=gray!20](-8.354,2.806)--(-8.411,2.814)--(-8.402,2.82)--(-8.354,2.806)--cycle;
\filldraw[fill opacity=0.8,fill=gray!20,draw=none](-6.271,.3)--(-6.297,.314)--(-6.287,.313)--cycle;
\draw(-6.297,.314)--(-6.287,.313);
\filldraw[fill opacity=0.8,fill=gray!20,draw=none](-6.776,.511)--(-6.77,.502)--(-6.772,.5)--(-6.775,.5)--cycle;
\draw(-6.772,.5)--(-6.775,.5);
\filldraw[fill opacity=0.8,fill=gray!20,draw=none](-6.192,.548)--(-6.223,.513)--(-6.223,.535)--cycle;
\draw(-6.223,.513)--(-6.223,.535);
\filldraw[fill opacity=0.8,fill=gray!20,draw=none](-6.77,.502)--(-6.769,.5)--(-6.772,.5)--cycle;
\draw(-6.769,.5)--(-6.772,.5);
\filldraw[fill opacity=0.8,fill=gray!20,draw=none](-6.241,.509)--(-6.238,.514)--(-6.28,.515)--(-6.287,.509)--(-6.271,.502)--cycle;
\draw(-6.28,.515)--(-6.287,.509)--(-6.271,.502);
\filldraw[fill opacity=0.8,fill=gray!20,draw=none](-6.287,.495)--(-6.276,.504)--(-6.287,.509)--(-6.297,.491)--cycle;
\draw(-6.276,.504)--(-6.287,.509)--(-6.297,.491);
\filldraw[fill opacity=0.8,fill=gray!20,draw=none](-6.287,.495)--(-6.297,.491)--(-6.301,.485)--cycle;
\draw(-6.297,.491)--(-6.301,.485);
\filldraw[fill opacity=0.8,fill=gray!20,draw=none](-6.316,.496)--(-6.253,.495)--(-6.289,.477)--cycle;
\draw(-6.316,.496)--(-6.253,.495);
\filldraw[fill opacity=0.8,fill=gray!20,draw=none](-6.238,.514)--(-6.223,.535)--(-6.245,.544)--(-6.28,.515)--cycle;
\draw(-6.223,.535)--(-6.245,.544)--(-6.28,.515);
\filldraw[fill opacity=0.8,fill=gray!20,draw=none](-6.271,.502)--(-6.276,.504)--(-6.287,.495)--cycle;
\draw(-6.271,.502)--(-6.276,.504);
\filldraw[fill opacity=0.8,fill=gray!20,draw=none](-6.77,.502)--(-6.749,.519)--(-6.246,.513)--(-6.245,.513)--(-6.271,.502)--(-6.297,.495)--(-6.769,.5)--cycle;
\draw(-6.749,.519)--(-6.246,.513);
\draw(-6.297,.495)--(-6.769,.5);
\filldraw[fill opacity=0.8,fill=gray!20,draw=none](-6.776,.511)--(-6.778,.519)--(-6.749,.519)--(-6.77,.502)--cycle;
\draw(-6.778,.519)--(-6.749,.519);
\filldraw[fill opacity=0.8,fill=gray!20,draw=none](-6.762,.467)--(-6.773,.494)--(-6.779,.514)--(-6.762,.496)--(-6.74,.444)--cycle;
\draw(-6.779,.514)--(-6.762,.496)--(-6.74,.444)--(-6.762,.467)--(-6.773,.494);
\filldraw[fill opacity=0.8,fill=gray!20](-7.79,.221)--(-7.751,.222)--(-7.769,.218)--(-7.79,.221)--cycle;
\filldraw[fill opacity=0.8,fill=gray!20](-7.79,.221)--(-7.743,.227)--(-7.751,.222)--(-7.79,.221)--cycle;
\filldraw[fill opacity=0.8,fill=gray!20](-7.79,.221)--(-7.769,.218)--(-7.793,.217)--(-7.79,.221)--cycle;
\filldraw[fill opacity=0.8,fill=gray!20](-7.79,.221)--(-7.793,.217)--(-7.816,.219)--(-7.79,.221)--cycle;
\filldraw[fill opacity=0.8,fill=gray!20](-7.872,.236)--(-7.906,.261)--(-7.92,.275)--(-7.882,.247)--cycle;
\filldraw[fill opacity=0.8,fill=gray!20](-7.74,.571)--(-7.764,.581)--(-7.748,.577)--(-7.709,.564)--cycle;
\filldraw[fill opacity=0.8,fill=gray!20](-7.79,.221)--(-7.816,.219)--(-7.833,.223)--(-7.79,.221)--cycle;
\filldraw[fill opacity=0.8,fill=gray!20,draw=none](-8.259,2.822)--(-8.237,2.822)--(-8.252,2.828)--cycle;
\draw(-8.259,2.822)--(-8.237,2.822)--(-8.252,2.828);
\filldraw[fill opacity=0.8,fill=gray!20,draw=none](-8.513,3.152)--(-8.503,3.164)--(-8.511,3.153)--cycle;
\draw(-8.503,3.164)--(-8.511,3.153);
\filldraw[fill opacity=0.8,fill=gray!20,draw=none](-7.729,4.492)--(-7.726,4.487)--(-7.728,4.492)--cycle;
\filldraw[fill opacity=0.8,fill=gray!20,draw=none](-7.728,4.492)--(-7.726,4.487)--(-7.73,4.484)--(-7.745,4.496)--cycle;
\draw(-7.726,4.487)--(-7.73,4.484)--(-7.745,4.496);
\filldraw[fill opacity=0.8,fill=gray!20,draw=none](-7.669,4.49)--(-7.704,4.505)--(-7.728,4.492)--(-7.726,4.487)--(-7.675,4.413)--(-7.657,4.433)--cycle;
\draw(-7.675,4.413)--(-7.657,4.433);
\filldraw[fill opacity=0.8,fill=gray!20](-6.952,.177)--(-7.008,.185)--(-6.999,.191)--(-6.952,.177)--cycle;
\filldraw[fill opacity=0.8,fill=gray!20,draw=none](-7.822,4.694)--(-7.833,4.676)--(-7.839,4.671)--(-7.834,4.711)--(-7.814,4.72)--cycle;
\draw(-7.833,4.676)--(-7.839,4.671)--(-7.834,4.711);
\filldraw[fill opacity=0.8,fill=gray!20,draw=none](-7.505,4.563)--(-7.504,4.561)--(-7.505,4.558)--cycle;
\draw(-7.504,4.561)--(-7.505,4.558);
\filldraw[fill opacity=0.8,fill=gray!20,draw=none](-7.509,4.564)--(-7.508,4.563)--(-7.509,4.56)--cycle;
\draw(-7.508,4.563)--(-7.509,4.56);
\filldraw[fill opacity=0.8,fill=gray!20,draw=none](-7.505,4.558)--(-7.504,4.561)--(-7.507,4.558)--(-7.517,4.538)--cycle;
\draw(-7.505,4.558)--(-7.504,4.561);
\filldraw[fill opacity=0.8,fill=gray!20,draw=none](-7.512,4.584)--(-7.512,4.588)--(-7.511,4.588)--cycle;
\filldraw[fill opacity=0.8,fill=gray!20,draw=none](-7.511,4.588)--(-7.518,4.569)--(-7.529,4.584)--cycle;
\draw(-7.511,4.588)--(-7.518,4.569);
\filldraw[fill opacity=0.8,fill=gray!20,draw=none](-7.64,4.877)--(-7.654,4.883)--(-7.646,4.887)--(-7.617,4.888)--(-7.614,4.878)--cycle;
\draw(-7.654,4.883)--(-7.646,4.887)--(-7.617,4.888)--(-7.614,4.878)--(-7.64,4.877);
\filldraw[fill opacity=0.8,fill=gray!20](-7.614,4.878)--(-7.617,4.888)--(-7.589,4.886)--(-7.56,4.874)--cycle;
\filldraw[fill opacity=0.8,fill=gray!20,draw=none](-6.949,.521)--(-7.142,.523)--(-7.124,.521)--(-6.952,.52)--cycle;
\draw(-7.124,.521)--(-6.952,.52)--(-6.949,.521)--(-7.142,.523);
\filldraw[fill opacity=0.8,fill=gray!20,draw=none](-7.125,.522)--(-7.107,.545)--(-7.081,.561)--(-7.108,.524)--cycle;
\draw(-7.125,.522)--(-7.107,.545)--(-7.081,.561)--(-7.108,.524);
\filldraw[fill opacity=0.5,fill=gray!20](-9,-.906)--(-9.05,-.97)--(-9.5,-.955)--(-9.43,-.892)--cycle;
\filldraw[fill opacity=0.8,fill=gray!20,draw=none](-8.518,3.143)--(-8.524,3.135)--(-8.52,3.146)--(-8.515,3.149)--cycle;
\draw(-8.52,3.146)--(-8.515,3.149);
\filldraw[fill opacity=0.8,fill=gray!20,draw=none](-7.531,4.588)--(-7.529,4.59)--(-7.527,4.607)--(-7.716,4.715)--(-7.717,4.704)--(-7.714,4.685)--cycle;
\draw(-7.531,4.588)--(-7.529,4.59)--(-7.527,4.607);
\draw(-7.716,4.715)--(-7.717,4.704)--(-7.714,4.685);
\filldraw[fill opacity=0.8,fill=gray!20,draw=none](-6.949,.283)--(-6.782,.282)--(-6.779,.261)--(-6.952,.263)--cycle;
\draw(-6.779,.261)--(-6.952,.263)--(-6.949,.283)--(-6.782,.282);
\filldraw[fill opacity=0.8,fill=gray!20](-7.743,.227)--(-7.699,.244)--(-7.714,.234)--(-7.751,.222)--cycle;
\filldraw[fill opacity=0.8,fill=gray!20](-7.906,.261)--(-7.932,.295)--(-7.949,.313)--(-7.92,.275)--cycle;
\filldraw[fill opacity=0.8,fill=gray!20,draw=none](-8.55,2.978)--(-8.566,2.968)--(-8.569,2.993)--cycle;
\draw(-8.55,2.978)--(-8.566,2.968)--(-8.569,2.993);
\filldraw[fill opacity=0.8,fill=gray!20,draw=none](-8.539,3.162)--(-8.52,3.154)--(-8.525,3.143)--(-8.554,3.108)--(-8.566,3.113)--cycle;
\draw(-8.554,3.108)--(-8.566,3.113)--(-8.539,3.162)--(-8.52,3.154);
\filldraw[fill opacity=0.8,fill=gray!20,draw=none](-8.52,3.154)--(-8.517,3.153)--(-8.525,3.143)--cycle;
\draw(-8.52,3.154)--(-8.517,3.153);
\filldraw[fill opacity=0.8,fill=gray!20,draw=none](-8.514,3.156)--(-8.517,3.153)--(-8.522,3.155)--cycle;
\draw(-8.517,3.153)--(-8.522,3.155);
\filldraw[fill opacity=0.8,fill=gray!20,draw=none](-8.495,3.184)--(-8.514,3.156)--(-8.522,3.155)--(-8.529,3.158)--cycle;
\draw(-8.522,3.155)--(-8.529,3.158);
\filldraw[fill opacity=0.8,fill=gray!20,draw=none](-8.506,3.199)--(-8.489,3.192)--(-8.495,3.184)--(-8.529,3.158)--(-8.539,3.162)--cycle;
\draw(-8.529,3.158)--(-8.539,3.162)--(-8.506,3.199)--(-8.489,3.192);
\filldraw[fill opacity=0.8,fill=gray!20,draw=none](-8.554,3.113)--(-8.537,3.134)--(-8.508,3.196)--(-8.539,3.162)--(-8.541,3.158)--cycle;
\draw(-8.508,3.196)--(-8.539,3.162)--(-8.541,3.158);
\filldraw[fill opacity=0.8,fill=gray!20,draw=none](-8.404,3.185)--(-8.426,3.172)--(-8.42,3.184)--cycle;
\draw(-8.426,3.172)--(-8.42,3.184);
\filldraw[fill opacity=0.8,fill=gray!20,draw=none](-8.524,3.162)--(-8.537,3.134)--(-8.506,3.176)--cycle;
\filldraw[fill opacity=0.8,fill=gray!20,draw=none](-8.513,3.152)--(-8.52,3.146)--(-8.544,3.13)--(-8.509,3.174)--(-8.484,3.191)--(-8.503,3.164)--cycle;
\draw(-8.52,3.146)--(-8.544,3.13)--(-8.509,3.174)--(-8.484,3.191)--(-8.503,3.164);
\filldraw[fill opacity=0.8,fill=gray!20,draw=none](-7.511,4.62)--(-7.51,4.619)--(-7.513,4.617)--cycle;
\draw(-7.51,4.619)--(-7.513,4.617);
\filldraw[fill opacity=0.8,fill=gray!20,draw=none](-7.522,4.607)--(-7.515,4.622)--(-7.508,4.619)--cycle;
\draw(-7.522,4.607)--(-7.515,4.622)--(-7.508,4.619);
\filldraw[fill opacity=0.8,fill=gray!20,draw=none](-7.531,4.585)--(-7.522,4.607)--(-7.508,4.619)--(-7.501,4.615)--(-7.513,4.589)--cycle;
\draw(-7.531,4.585)--(-7.522,4.607);
\draw(-7.508,4.619)--(-7.501,4.615)--(-7.513,4.589);
\filldraw[fill opacity=0.8,fill=gray!20,draw=none](-8.527,3.125)--(-8.524,3.135)--(-8.518,3.143)--cycle;
\filldraw[fill opacity=0.8,fill=gray!20,draw=none](-7.512,4.584)--(-7.519,4.568)--(-7.52,4.57)--(-7.521,4.578)--(-7.519,4.588)--(-7.512,4.588)--cycle;
\filldraw[fill opacity=0.8,fill=gray!20,draw=none](-7.521,4.578)--(-7.52,4.57)--(-7.522,4.572)--cycle;
\filldraw[fill opacity=0.8,fill=gray!20,draw=none](-7.531,4.585)--(-7.513,4.589)--(-7.521,4.57)--cycle;
\draw(-7.513,4.589)--(-7.521,4.57);
\filldraw[fill opacity=0.8,fill=gray!20](-8.354,2.806)--(-8.303,2.819)--(-8.298,2.813)--(-8.354,2.806)--cycle;
\filldraw[fill opacity=0.8,fill=gray!20,draw=none](-7.648,4.323)--(-7.639,4.345)--(-7.62,4.348)--cycle;
\draw(-7.648,4.323)--(-7.639,4.345);
\filldraw[fill opacity=0.8,fill=gray!20,draw=none](-7.644,4.302)--(-7.648,4.304)--(-7.648,4.323)--(-7.62,4.348)--(-7.634,4.315)--cycle;
\draw(-7.62,4.348)--(-7.634,4.315);
\filldraw[fill opacity=0.8,fill=gray!20,draw=none](-7.655,4.308)--(-7.648,4.323)--(-7.648,4.304)--cycle;
\draw(-7.655,4.308)--(-7.648,4.323);
\filldraw[fill opacity=0.8,fill=gray!20,draw=none](-8.566,3.113)--(-8.546,3.104)--(-8.549,3.09)--(-8.57,3.053)--(-8.584,3.059)--cycle;
\draw(-8.57,3.053)--(-8.584,3.059)--(-8.566,3.113)--(-8.546,3.104);
\filldraw[fill opacity=0.8,fill=gray!20,draw=none](-8.549,3.09)--(-8.559,3.049)--(-8.57,3.053)--cycle;
\draw(-8.559,3.049)--(-8.57,3.053);
\filldraw[fill opacity=0.8,fill=gray!20,draw=none](-8.548,3.073)--(-8.554,3.036)--(-8.554,3.086)--(-8.548,3.091)--cycle;
\draw(-8.554,3.086)--(-8.548,3.091);
\filldraw[fill opacity=0.8,fill=gray!20,draw=none](-8.561,3.044)--(-8.567,3.052)--(-8.559,3.049)--cycle;
\draw(-8.567,3.052)--(-8.559,3.049);
\filldraw[fill opacity=0.8,fill=gray!20,draw=none](-8.57,3.026)--(-8.566,3.05)--(-8.561,3.044)--cycle;
\filldraw[fill opacity=0.8,fill=gray!20,draw=none](-8.584,3.059)--(-8.582,3.059)--(-8.573,3.018)--(-8.579,3.006)--(-8.588,3.01)--cycle;
\draw(-8.579,3.006)--(-8.588,3.01)--(-8.584,3.059)--(-8.582,3.059);
\filldraw[fill opacity=0.8,fill=gray!20,draw=none](-8.569,3.066)--(-8.563,3.081)--(-8.554,3.113)--(-8.572,3.088)--cycle;
\filldraw[fill opacity=0.8,fill=gray!20,draw=none](-8.569,3.06)--(-8.563,3.081)--(-8.569,3.066)--cycle;
\filldraw[fill opacity=0.8,fill=gray!20,draw=none](-8.554,3.036)--(-8.573,3.023)--(-8.566,3.079)--(-8.554,3.086)--cycle;
\draw(-8.554,3.036)--(-8.573,3.023)--(-8.566,3.079)--(-8.554,3.086);
\filldraw[fill opacity=0.8,fill=gray!20,draw=none](-7.527,4.603)--(-7.518,4.612)--(-7.513,4.614)--(-7.519,4.588)--(-7.53,4.588)--cycle;
\draw(-7.527,4.603)--(-7.518,4.612);
\filldraw[fill opacity=0.8,fill=gray!20,draw=none](-7.511,4.62)--(-7.513,4.617)--(-7.527,4.603)--(-7.62,4.669)--cycle;
\draw(-7.513,4.617)--(-7.527,4.603);
\filldraw[fill opacity=0.8,fill=gray!20,draw=none](-6.259,.435)--(-6.259,.325)--(-6.244,.368)--cycle;
\draw(-6.259,.435)--(-6.259,.325);
\filldraw[fill opacity=0.8,fill=gray!20,draw=none](-6.244,.368)--(-6.209,.303)--(-6.209,.413)--cycle;
\draw(-6.209,.303)--(-6.209,.413);
\filldraw[fill opacity=0.8,fill=gray!20,draw=none](-6.207,.307)--(-6.203,.307)--(-6.195,.354)--(-6.213,.354)--cycle;
\draw(-6.207,.307)--(-6.203,.307);
\draw(-6.195,.354)--(-6.213,.354);
\filldraw[fill opacity=0.8,fill=gray!20,draw=none](-6.203,.307)--(-6.196,.307)--(-6.177,.354)--(-6.195,.354)--cycle;
\draw(-6.203,.307)--(-6.196,.307);
\draw(-6.177,.354)--(-6.195,.354);
\filldraw[fill opacity=0.8,fill=gray!20,draw=none](-6.177,.354)--(-6.119,.353)--(-6.119,.405)--(-6.196,.406)--cycle;
\draw(-6.177,.354)--(-6.119,.353)--(-6.119,.405)--(-6.196,.406);
\filldraw[fill opacity=0.8,fill=gray!20,draw=none](-6.196,.406)--(-6.166,.326)--(-6.159,.331)--(-6.159,.373)--cycle;
\draw(-6.159,.331)--(-6.159,.373);
\filldraw[fill opacity=0.8,fill=gray!20](-6.162,.332)--(-6.326,.404)--(-6.316,.35)--(-6.152,.279)--cycle;
\filldraw[fill opacity=0.8,fill=gray!20,draw=none](-8.022,4.052)--(-8.03,4.054)--(-8.013,4.063)--(-7.984,4.064)--(-7.981,4.054)--cycle;
\draw(-8.03,4.054)--(-8.013,4.063)--(-7.984,4.064)--(-7.981,4.054)--(-8.022,4.052);
\filldraw[fill opacity=0.8,fill=gray!20](-7.981,4.054)--(-7.984,4.064)--(-7.956,4.062)--(-7.927,4.05)--cycle;
\filldraw[fill opacity=0.8,fill=gray!20,draw=none](-8.534,3.12)--(-8.524,3.135)--(-8.527,3.125)--(-8.536,3.109)--(-8.541,3.102)--cycle;
\filldraw[fill opacity=0.8,fill=gray!20,draw=none](-8.381,3.232)--(-8.392,3.231)--(-8.397,3.237)--(-8.389,3.255)--cycle;
\draw(-8.397,3.237)--(-8.389,3.255);
\filldraw[fill opacity=0.8,fill=gray!20,draw=none](-7.536,4.588)--(-7.534,4.588)--(-7.535,4.587)--cycle;
\draw(-7.534,4.588)--(-7.535,4.587);
\filldraw[fill opacity=0.8,fill=gray!20,draw=none](-7.521,4.578)--(-7.522,4.572)--(-7.536,4.588)--(-7.522,4.588)--cycle;
\filldraw[fill opacity=0.8,fill=gray!20,draw=none](-7.882,3.797)--(-7.812,3.956)--(-7.8,3.943)--(-7.868,3.791)--cycle;
\draw(-7.8,3.943)--(-7.868,3.791)--(-7.882,3.797)--(-7.812,3.956);
\filldraw[fill opacity=0.8,fill=gray!20,draw=none](-7.948,3.635)--(-7.954,3.637)--(-7.953,3.639)--cycle;
\draw(-7.954,3.637)--(-7.953,3.639);
\filldraw[fill opacity=0.8,fill=gray!20,draw=none](-7.948,3.635)--(-7.942,3.63)--(-7.958,3.622)--(-7.958,3.627)--(-7.954,3.637)--cycle;
\draw(-7.958,3.627)--(-7.954,3.637);
\filldraw[fill opacity=0.8,fill=gray!20,draw=none](-7.987,3.63)--(-7.953,3.639)--(-7.935,3.636)--(-7.987,3.63)--cycle;
\draw(-7.935,3.636)--(-7.987,3.63)--(-7.987,3.63)--(-7.953,3.639);
\filldraw[fill opacity=0.8,fill=gray!20,draw=none](-7.948,3.635)--(-7.931,3.637)--(-7.94,3.631)--cycle;
\draw(-7.948,3.635)--(-7.931,3.637)--(-7.94,3.631);
\filldraw[fill opacity=0.8,fill=gray!20,draw=none](-7.868,3.791)--(-7.94,3.628)--(-7.953,3.639)--(-7.882,3.797)--cycle;
\draw(-7.953,3.639)--(-7.882,3.797)--(-7.868,3.791)--(-7.94,3.628);
\filldraw[fill opacity=0.8,fill=gray!20,draw=none](-7.64,4.877)--(-7.669,4.876)--(-7.654,4.883)--cycle;
\draw(-7.64,4.877)--(-7.669,4.876)--(-7.654,4.883);
\filldraw[fill opacity=0.8,fill=gray!20](-6.952,.177)--(-6.901,.19)--(-6.895,.183)--(-6.952,.177)--cycle;
\filldraw[fill opacity=0.8,fill=gray!20,draw=none](-7.531,4.588)--(-7.533,4.584)--(-7.535,4.587)--(-7.534,4.588)--cycle;
\draw(-7.531,4.588)--(-7.533,4.584);
\draw(-7.535,4.587)--(-7.534,4.588);
\filldraw[fill opacity=0.8,fill=gray!20,draw=none](-6.782,.519)--(-6.776,.511)--(-6.782,.517)--(-6.784,.52)--cycle;
\draw(-6.776,.511)--(-6.782,.517)--(-6.784,.52);
\filldraw[fill opacity=0.8,fill=gray!20,draw=none](-6.949,.521)--(-6.778,.519)--(-6.775,.5)--(-6.947,.502)--cycle;
\draw(-6.775,.5)--(-6.947,.502)--(-6.949,.521)--(-6.778,.519);
\filldraw[fill opacity=0.8,fill=gray!20,draw=none](-7.734,4.494)--(-7.745,4.496)--(-7.759,4.507)--cycle;
\draw(-7.745,4.496)--(-7.759,4.507);
\filldraw[fill opacity=0.8,fill=gray!20,draw=none](-8.544,3.093)--(-8.541,3.102)--(-8.536,3.109)--cycle;
\filldraw[fill opacity=0.8,fill=gray!20,draw=none](-7.45,4.794)--(-7.445,4.792)--(-7.43,4.773)--cycle;
\draw(-7.445,4.792)--(-7.43,4.773)--(-7.45,4.794);
\filldraw[fill opacity=0.8,fill=gray!20,draw=none](-7.45,4.794)--(-7.481,4.835)--(-7.465,4.818)--(-7.445,4.792)--cycle;
\draw(-7.45,4.794)--(-7.481,4.835)--(-7.465,4.818)--(-7.445,4.792);
\filldraw[fill opacity=0.8,fill=gray!20,draw=none](-8.545,2.928)--(-8.533,2.923)--(-8.544,2.916)--(-8.549,2.927)--cycle;
\draw(-8.533,2.923)--(-8.544,2.916)--(-8.549,2.927);
\filldraw[fill opacity=0.8,fill=gray!20,draw=none](-8.548,3.073)--(-8.548,3.082)--(-8.544,3.093)--cycle;
\filldraw[fill opacity=0.8,fill=gray!20](-7.699,.244)--(-7.661,.272)--(-7.682,.258)--(-7.714,.234)--cycle;
\filldraw[fill opacity=0.8,fill=gray!20,draw=none](-8.389,3.228)--(-8.397,3.23)--(-8.38,3.239)--(-8.351,3.24)--(-8.348,3.23)--cycle;
\draw(-8.397,3.23)--(-8.38,3.239)--(-8.351,3.24)--(-8.348,3.23)--(-8.389,3.228);
\filldraw[fill opacity=0.8,fill=gray!20](-8.348,3.23)--(-8.351,3.24)--(-8.323,3.238)--(-8.294,3.226)--cycle;
\filldraw[fill opacity=0.8,fill=gray!20,draw=none](-7.667,.269)--(-7.793,.27)--(-7.79,.271)--(-7.665,.27)--cycle;
\draw(-7.667,.269)--(-7.793,.27)--(-7.79,.271)--(-7.665,.27);
\filldraw[fill opacity=0.5,fill=gray!20](-10.154,-.838)--(-10.211,-.798)--(-10.591,-.522)--(-10.536,-.56)--cycle;
\filldraw[fill opacity=0.8,fill=gray!20](-7.675,.539)--(-7.709,.564)--(-7.699,.553)--(-7.661,.525)--cycle;
\filldraw[fill opacity=0.8,fill=gray!20,draw=none](-8.524,3.135)--(-8.534,3.12)--(-8.526,3.142)--(-8.52,3.146)--cycle;
\draw(-8.526,3.142)--(-8.52,3.146);
\filldraw[fill opacity=0.8,fill=gray!20,draw=none](-8.544,3.093)--(-8.55,3.089)--(-8.541,3.102)--cycle;
\draw(-8.544,3.093)--(-8.55,3.089);
\filldraw[fill opacity=0.8,fill=gray!20,draw=none](-8.548,3.082)--(-8.548,3.091)--(-8.544,3.093)--cycle;
\draw(-8.548,3.091)--(-8.544,3.093);
\filldraw[fill opacity=0.8,fill=gray!20,draw=none](-7.827,4.68)--(-7.833,4.676)--(-7.822,4.694)--cycle;
\draw(-7.827,4.68)--(-7.833,4.676);
\filldraw[fill opacity=0.8,fill=gray!20,draw=none](-7.513,4.614)--(-7.518,4.612)--(-7.513,4.617)--cycle;
\draw(-7.518,4.612)--(-7.513,4.617);
\filldraw[fill opacity=0.8,fill=gray!20](-7.817,3.97)--(-7.848,4.011)--(-7.832,3.994)--(-7.797,3.949)--cycle;
\filldraw[fill opacity=0.8,fill=gray!20](-6.945,.601)--(-6.948,.61)--(-6.92,.608)--(-6.891,.597)--cycle;
\filldraw[fill opacity=0.8,fill=gray!20](-7.001,.598)--(-6.977,.609)--(-6.948,.61)--(-6.945,.601)--cycle;
\filldraw[fill opacity=0.8,fill=gray!20,draw=none](-7.681,4.882)--(-7.685,4.873)--(-7.687,4.876)--cycle;
\draw(-7.681,4.882)--(-7.685,4.873);
\filldraw[fill opacity=0.8,fill=gray!20,draw=none](-7.712,4.867)--(-7.668,4.882)--(-7.657,4.884)--(-7.654,4.883)--(-7.669,4.876)--cycle;
\draw(-7.654,4.883)--(-7.669,4.876)--(-7.712,4.867)--(-7.668,4.882)--(-7.657,4.884);
\filldraw[fill opacity=0.8,fill=gray!20,draw=none](-7.904,.538)--(-7.92,.528)--(-7.882,.556)--(-7.869,.564)--cycle;
\draw(-7.904,.538)--(-7.92,.528)--(-7.882,.556)--(-7.869,.564);
\filldraw[fill opacity=0.8,fill=gray!20,draw=none](-7.459,4.539)--(-7.453,4.53)--(-7.461,4.52)--cycle;
\draw(-7.453,4.53)--(-7.461,4.52);
\filldraw[fill opacity=0.8,fill=gray!20,draw=none](-8.163,3.735)--(-8.174,3.736)--(-8.143,3.806)--cycle;
\draw(-8.174,3.736)--(-8.143,3.806);
\filldraw[fill opacity=0.8,fill=gray!20,draw=none](-8.549,3.09)--(-8.546,3.104)--(-8.542,3.102)--cycle;
\draw(-8.546,3.104)--(-8.542,3.102);
\filldraw[fill opacity=0.8,fill=gray!20,draw=none](-8.535,3.119)--(-8.542,3.102)--(-8.552,3.107)--cycle;
\draw(-8.542,3.102)--(-8.552,3.107);
\filldraw[fill opacity=0.8,fill=gray!20,draw=none](-8.534,3.12)--(-8.541,3.102)--(-8.55,3.089)--(-8.561,3.082)--cycle;
\draw(-8.55,3.089)--(-8.561,3.082);
\filldraw[fill opacity=0.8,fill=gray!20,draw=none](-7.533,4.584)--(-7.531,4.588)--(-7.714,4.685)--(-7.711,4.668)--cycle;
\draw(-7.533,4.584)--(-7.531,4.588);
\draw(-7.714,4.685)--(-7.711,4.668);
\filldraw[fill opacity=0.8,fill=gray!20](-7.781,1.06)--(-7.78,1.107)--(-7.693,1.1)--(-7.703,1.054)--cycle;
\filldraw[fill opacity=0.8,fill=gray!20](-7.78,1.107)--(-7.779,1.154)--(-7.69,1.148)--(-7.693,1.1)--cycle;
\filldraw[fill opacity=0.8,fill=gray!20,draw=none](-7.641,.269)--(-7.667,.269)--(-7.665,.27)--cycle;
\draw(-7.641,.269)--(-7.667,.269);
\filldraw[fill opacity=0.8,fill=gray!20](-7.05,.195)--(-7.091,.224)--(-7.107,.242)--(-7.061,.207)--cycle;
\filldraw[fill opacity=0.8,fill=gray!20,draw=none](-7.106,.241)--(-7.127,.271)--(-7.141,.286)--(-7.107,.242)--cycle;
\draw(-7.127,.271)--(-7.141,.286)--(-7.107,.242)--(-7.106,.241);
\filldraw[fill opacity=0.8,fill=gray!20,draw=none](-7.121,.263)--(-7.641,.269)--(-7.665,.27)--(-7.124,.264)--cycle;
\draw(-7.121,.263)--(-7.641,.269);
\draw(-7.665,.27)--(-7.124,.264);
\filldraw[fill opacity=0.8,fill=gray!20,draw=none](-7.779,.53)--(-7.788,.53)--(-7.79,.529)--(-7.744,.528)--cycle;
\draw(-7.779,.53)--(-7.788,.53)--(-7.79,.529)--(-7.744,.528);
\filldraw[fill opacity=0.8,fill=gray!20,draw=none](-6.145,.591)--(-6.127,.592)--(-6.159,.551)--(-6.178,.554)--(-6.178,.563)--cycle;
\draw(-6.145,.591)--(-6.127,.592);
\draw(-6.178,.554)--(-6.178,.563);
\filldraw[fill opacity=0.8,fill=gray!20,draw=none](-8.079,4.043)--(-8.055,4.052)--(-8.033,4.054)--(-8.032,4.053)--(-8.036,4.052)--cycle;
\draw(-8.032,4.053)--(-8.036,4.052)--(-8.079,4.043)--(-8.055,4.052);
\filldraw[fill opacity=0.8,fill=gray!20,draw=none](-6.129,.583)--(-6.127,.592)--(-6.145,.591)--cycle;
\draw(-6.127,.592)--(-6.145,.591);
\filldraw[fill opacity=0.8,fill=gray!20,draw=none](-7.832,4.65)--(-7.832,4.676)--(-7.827,4.68)--cycle;
\draw(-7.832,4.676)--(-7.827,4.68);
\filldraw[fill opacity=0.8,fill=gray!20,draw=none](-7.884,.571)--(-7.869,.565)--(-7.895,.545)--(-7.902,.548)--cycle;
\draw(-7.884,.571)--(-7.869,.565);
\draw(-7.895,.545)--(-7.902,.548);
\filldraw[fill opacity=0.8,fill=gray!20](-7.775,4.822)--(-7.73,4.856)--(-7.712,4.867)--(-7.75,4.839)--cycle;
\filldraw[fill opacity=0.8,fill=gray!20,draw=none](-8.539,3.101)--(-8.536,3.1)--(-8.54,3.09)--cycle;
\draw(-8.539,3.101)--(-8.536,3.1);
\filldraw[fill opacity=0.8,fill=gray!20,draw=none](-8.53,3.107)--(-8.536,3.1)--(-8.539,3.101)--cycle;
\draw(-8.536,3.1)--(-8.539,3.101);
\filldraw[fill opacity=0.8,fill=gray!20,draw=none](-7.667,.28)--(-7.673,.287)--(-7.795,.289)--(-7.793,.27)--(-7.671,.269)--cycle;
\draw(-7.673,.287)--(-7.795,.289)--(-7.793,.27)--(-7.671,.269);
\filldraw[fill opacity=0.8,fill=gray!20,draw=none](-7.665,.27)--(-7.661,.272)--(-7.632,.309)--(-7.658,.292)--(-7.675,.268)--cycle;
\draw(-7.665,.27)--(-7.661,.272)--(-7.632,.309)--(-7.658,.292)--(-7.675,.268);
\filldraw[fill opacity=0.8,fill=gray!20,draw=none](-7.666,.528)--(-7.746,.529)--(-7.757,.529)--(-7.744,.528)--(-7.663,.527)--cycle;
\draw(-7.666,.528)--(-7.746,.529);
\draw(-7.744,.528)--(-7.663,.527);
\filldraw[fill opacity=0.8,fill=gray!20,draw=none](-7.697,4.519)--(-7.736,4.533)--(-7.751,4.526)--(-7.707,4.507)--(-7.692,4.513)--cycle;
\draw(-7.736,4.533)--(-7.751,4.526);
\draw(-7.707,4.507)--(-7.692,4.513);
\filldraw[fill opacity=0.8,fill=gray!20,draw=none](-7.742,4.583)--(-7.802,4.555)--(-7.795,4.545)--(-7.751,4.526)--(-7.735,4.533)--cycle;
\draw(-7.742,4.583)--(-7.802,4.555);
\draw(-7.751,4.526)--(-7.735,4.533);
\filldraw[fill opacity=0.8,fill=gray!20,draw=none](-7.684,4.514)--(-7.673,4.509)--(-7.71,4.524)--(-7.736,4.533)--cycle;
\filldraw[fill opacity=0.8,fill=gray!20,draw=none](-7.669,4.49)--(-7.673,4.509)--(-7.685,4.515)--(-7.704,4.505)--cycle;
\filldraw[fill opacity=0.8,fill=gray!20,draw=none](-7.685,4.515)--(-7.688,4.516)--(-7.707,4.507)--(-7.704,4.505)--cycle;
\draw(-7.688,4.516)--(-7.707,4.507);
\filldraw[fill opacity=0.8,fill=gray!20,draw=none](-7.764,4.544)--(-7.797,4.557)--(-7.71,4.524)--cycle;
\filldraw[fill opacity=0.8,fill=gray!20,draw=none](-7.717,4.542)--(-7.736,4.533)--(-7.697,4.519)--cycle;
\draw(-7.717,4.542)--(-7.736,4.533);
\filldraw[fill opacity=0.8,fill=gray!20,draw=none](-7.742,4.583)--(-7.735,4.533)--(-7.717,4.542)--cycle;
\draw(-7.735,4.533)--(-7.717,4.542);
\filldraw[fill opacity=0.8,fill=gray!20,draw=none](-7.662,4.506)--(-7.663,4.506)--(-7.673,4.509)--(-7.684,4.514)--cycle;
\filldraw[fill opacity=0.8,fill=gray!20,draw=none](-7.673,4.509)--(-7.675,4.521)--(-7.685,4.515)--cycle;
\filldraw[fill opacity=0.8,fill=gray!20,draw=none](-7.72,4.526)--(-7.764,4.544)--(-7.71,4.524)--(-7.703,4.521)--cycle;
\filldraw[fill opacity=0.8,fill=gray!20,draw=none](-7.816,4.554)--(-7.814,4.558)--(-7.808,4.569)--(-7.814,4.554)--cycle;
\draw(-7.808,4.569)--(-7.814,4.554);
\filldraw[fill opacity=0.8,fill=gray!20,draw=none](-7.697,4.519)--(-7.692,4.513)--(-7.688,4.516)--cycle;
\draw(-7.692,4.513)--(-7.688,4.516);
\filldraw[fill opacity=0.8,fill=gray!20,draw=none](-7.762,4.63)--(-7.765,4.629)--(-7.804,4.573)--(-7.808,4.563)--(-7.806,4.56)--(-7.802,4.555)--(-7.742,4.583)--cycle;
\draw(-7.762,4.63)--(-7.765,4.629);
\draw(-7.802,4.555)--(-7.742,4.583);
\filldraw[fill opacity=0.8,fill=gray!20,draw=none](-7.675,4.521)--(-7.678,4.52)--(-7.68,4.542)--cycle;
\draw(-7.678,4.52)--(-7.68,4.542);
\filldraw[fill opacity=0.8,fill=gray!20,draw=none](-7.736,4.532)--(-7.761,4.54)--(-7.777,4.549)--(-7.764,4.544)--(-7.736,4.532)--cycle;
\filldraw[fill opacity=0.8,fill=gray!20,draw=none](-7.732,4.53)--(-7.736,4.532)--(-7.736,4.532)--(-7.72,4.526)--cycle;
\filldraw[fill opacity=0.8,fill=gray!20,draw=none](-7.732,4.53)--(-7.72,4.526)--(-7.715,4.524)--cycle;
\filldraw[fill opacity=0.8,fill=gray!20,draw=none](-7.676,4.52)--(-7.678,4.52)--(-7.688,4.516)--(-7.685,4.515)--cycle;
\draw(-7.678,4.52)--(-7.688,4.516);
\filldraw[fill opacity=0.8,fill=gray!20,draw=none](-7.675,4.521)--(-7.672,4.504)--(-7.676,4.506)--(-7.678,4.52)--cycle;
\draw(-7.676,4.506)--(-7.678,4.52);
\filldraw[fill opacity=0.8,fill=gray!20,draw=none](-7.634,4.475)--(-7.628,4.472)--(-7.629,4.477)--cycle;
\draw(-7.628,4.472)--(-7.629,4.477);
\filldraw[fill opacity=0.8,fill=gray!20,draw=none](-7.621,4.487)--(-7.673,4.509)--(-7.669,4.49)--(-7.633,4.474)--(-7.625,4.48)--cycle;
\filldraw[fill opacity=0.8,fill=gray!20,draw=none](-7.677,4.511)--(-7.675,4.493)--(-7.714,4.51)--(-7.715,4.524)--cycle;
\draw(-7.677,4.511)--(-7.675,4.493);
\draw(-7.714,4.51)--(-7.715,4.524);
\filldraw[fill opacity=0.8,fill=gray!20,draw=none](-7.737,4.53)--(-7.736,4.532)--(-7.732,4.53)--(-7.715,4.524)--cycle;
\filldraw[fill opacity=0.8,fill=gray!20,draw=none](-7.736,4.532)--(-7.736,4.532)--(-7.732,4.53)--cycle;
\filldraw[fill opacity=0.8,fill=gray!20,draw=none](-7.722,4.601)--(-7.715,4.524)--(-7.737,4.53)--(-7.746,4.635)--cycle;
\draw(-7.737,4.53)--(-7.746,4.635)--(-7.722,4.601)--(-7.715,4.524);
\filldraw[fill opacity=0.8,fill=gray!20,draw=none](-8.535,3.119)--(-8.508,3.138)--(-8.53,3.107)--(-8.539,3.101)--(-8.542,3.102)--cycle;
\draw(-8.539,3.101)--(-8.542,3.102);
\filldraw[fill opacity=0.8,fill=gray!20,draw=none](-7.701,4.466)--(-7.718,4.472)--(-7.73,4.484)--(-7.722,4.481)--cycle;
\draw(-7.701,4.466)--(-7.718,4.472)--(-7.73,4.484)--(-7.722,4.481);
\filldraw[fill opacity=0.8,fill=gray!20](-7.56,4.874)--(-7.589,4.886)--(-7.569,4.881)--(-7.522,4.865)--cycle;
\filldraw[fill opacity=0.8,fill=gray!20,draw=none](-7.728,4.482)--(-7.749,4.498)--(-7.745,4.496)--(-7.73,4.484)--cycle;
\draw(-7.745,4.496)--(-7.73,4.484)--(-7.728,4.482);
\filldraw[fill opacity=0.8,fill=gray!20,draw=none](-7.818,4.572)--(-7.813,4.579)--(-7.822,4.557)--(-7.823,4.557)--cycle;
\draw(-7.813,4.579)--(-7.822,4.557)--(-7.823,4.557);
\filldraw[fill opacity=0.8,fill=gray!20,draw=none](-7.813,4.56)--(-7.835,4.594)--(-7.833,4.54)--(-7.818,4.547)--cycle;
\draw(-7.833,4.54)--(-7.818,4.547);
\filldraw[fill opacity=0.8,fill=gray!20,draw=none](-8.549,3.09)--(-8.542,3.102)--(-8.539,3.101)--(-8.54,3.09)--(-8.559,3.049)--(-8.559,3.049)--cycle;
\draw(-8.542,3.102)--(-8.539,3.101);
\draw(-8.559,3.049)--(-8.559,3.049);
\filldraw[fill opacity=0.8,fill=gray!20](-7.779,1.154)--(-7.78,1.199)--(-7.693,1.193)--(-7.69,1.148)--cycle;
\filldraw[fill opacity=0.8,fill=gray!20,draw=none](-8.54,3.09)--(-8.543,3.052)--(-8.547,3.043)--(-8.559,3.049)--cycle;
\draw(-8.547,3.043)--(-8.559,3.049);
\filldraw[fill opacity=0.8,fill=gray!20,draw=none](-7.851,4.462)--(-7.816,4.554)--(-7.818,4.547)--(-7.825,4.526)--(-7.85,4.462)--cycle;
\draw(-7.825,4.526)--(-7.85,4.462);
\filldraw[fill opacity=0.8,fill=gray!20,draw=none](-8.446,3.219)--(-8.422,3.227)--(-8.4,3.23)--(-8.399,3.229)--(-8.403,3.227)--cycle;
\draw(-8.399,3.229)--(-8.403,3.227)--(-8.446,3.219)--(-8.422,3.227);
\filldraw[fill opacity=0.8,fill=gray!20](-7.816,.219)--(-7.841,.229)--(-7.872,.236)--(-7.833,.223)--cycle;
\filldraw[fill opacity=0.5,fill=gray!20](-10.626,-.465)--(-10.642,-.391)--(-10.918,-.028)--(-10.912,-.09)--cycle;
\filldraw[fill opacity=0.8,fill=gray!20,draw=none](-7.962,.351)--(-7.948,.337)--(-7.954,.382)--(-7.973,.402)--(-7.969,.372)--cycle;
\draw(-7.962,.351)--(-7.948,.337)--(-7.954,.382)--(-7.973,.402)--(-7.969,.372);
\filldraw[fill opacity=0.8,fill=gray!20,draw=none](-6.782,.519)--(-6.784,.52)--(-6.813,.558)--(-6.797,.541)--cycle;
\draw(-6.784,.52)--(-6.813,.558)--(-6.797,.541);
\filldraw[fill opacity=0.8,fill=gray!20](-8.142,3.998)--(-8.097,4.032)--(-8.079,4.043)--(-8.117,4.015)--cycle;
\filldraw[fill opacity=0.8,fill=gray!20](-7.987,3.63)--(-8.038,3.632)--(-8.044,3.638)--(-7.987,3.63)--cycle;
\filldraw[fill opacity=0.8,fill=gray!20](-8.038,3.632)--(-8.085,3.648)--(-8.097,3.66)--(-8.044,3.638)--cycle;
\filldraw[fill opacity=0.8,fill=gray!20,draw=none](-8.543,3.052)--(-8.544,3.042)--(-8.547,3.043)--cycle;
\draw(-8.544,3.042)--(-8.547,3.043);
\filldraw[fill opacity=0.8,fill=gray!20,draw=none](-7.877,4.394)--(-7.881,4.395)--(-7.86,4.441)--(-7.868,4.418)--cycle;
\draw(-7.881,4.395)--(-7.86,4.441);
\filldraw[fill opacity=0.8,fill=gray!20,draw=none](-7.854,4.463)--(-7.851,4.462)--(-7.86,4.441)--(-7.984,4.164)--cycle;
\draw(-7.86,4.441)--(-7.984,4.164);
\filldraw[fill opacity=0.8,fill=gray!20,draw=none](-7.817,4.555)--(-7.877,4.394)--(-7.881,4.394)--(-7.818,4.555)--cycle;
\draw(-7.877,4.394)--(-7.881,4.394)--(-7.818,4.555);
\filldraw[fill opacity=0.8,fill=gray!20,draw=none](-6.145,.591)--(-6.178,.563)--(-6.178,.589)--cycle;
\draw(-6.178,.563)--(-6.178,.589)--(-6.145,.591);
\filldraw[fill opacity=0.8,fill=gray!20,draw=none](-6.178,.589)--(-6.178,.563)--(-6.187,.554)--(-6.223,.557)--(-6.223,.582)--cycle;
\draw(-6.223,.557)--(-6.223,.582)--(-6.178,.589)--(-6.178,.563);
\filldraw[fill opacity=0.8,fill=gray!20,draw=none](-6.223,.582)--(-6.223,.535)--(-6.252,.515)--(-6.252,.571)--cycle;
\draw(-6.252,.515)--(-6.252,.571)--(-6.223,.582)--(-6.223,.535);
\filldraw[fill opacity=0.8,fill=gray!20,draw=none](-6.252,.571)--(-6.252,.518)--(-6.259,.543)--(-6.259,.558)--cycle;
\draw(-6.259,.543)--(-6.259,.558)--(-6.252,.571)--(-6.252,.518);
\filldraw[fill opacity=0.8,fill=gray!20,draw=none](-6.134,.564)--(-6.129,.583)--(-6.145,.591)--(-6.178,.589)--(-6.223,.582)--(-6.252,.571)--(-6.258,.56)--cycle;
\draw(-6.145,.591)--(-6.178,.589)--(-6.223,.582)--(-6.252,.571)--(-6.258,.56);
\filldraw[fill opacity=0.8,fill=gray!20,draw=none](-7.818,4.547)--(-7.822,4.534)--(-7.825,4.526)--cycle;
\draw(-7.822,4.534)--(-7.825,4.526);
\filldraw[fill opacity=0.8,fill=gray!20,draw=none](-7.147,.523)--(-7.666,.528)--(-7.663,.527)--(-7.191,.522)--cycle;
\draw(-7.147,.523)--(-7.666,.528);
\draw(-7.663,.527)--(-7.191,.522);
\filldraw[fill opacity=0.8,fill=gray!20,draw=none](-8.502,3.146)--(-8.499,3.145)--(-8.508,3.138)--cycle;
\draw(-8.502,3.146)--(-8.499,3.145);
\filldraw[fill opacity=0.8,fill=gray!20,draw=none](-8.491,3.149)--(-8.499,3.145)--(-8.503,3.147)--cycle;
\draw(-8.499,3.145)--(-8.503,3.147);
\filldraw[fill opacity=0.8,fill=gray!20,draw=none](-8.543,3.031)--(-8.548,3.044)--(-8.544,3.042)--cycle;
\draw(-8.548,3.044)--(-8.544,3.042);
\filldraw[fill opacity=0.8,fill=gray!20,draw=none](-7.917,.585)--(-7.884,.571)--(-7.902,.548)--(-7.911,.552)--cycle;
\draw(-7.917,.585)--(-7.884,.571);
\draw(-7.902,.548)--(-7.911,.552);
\filldraw[fill opacity=0.8,fill=gray!20,draw=none](-6.231,.502)--(-6.222,.495)--(-6.253,.495)--cycle;
\draw(-6.222,.495)--(-6.253,.495);
\filldraw[fill opacity=0.8,fill=gray!20,draw=none](-6.231,.502)--(-6.223,.496)--(-6.223,.464)--(-6.252,.421)--(-6.252,.468)--cycle;
\draw(-6.223,.496)--(-6.223,.464);
\draw(-6.252,.421)--(-6.252,.468);
\filldraw[fill opacity=0.8,fill=gray!20,draw=none](-6.206,.415)--(-6.196,.406)--(-6.119,.405)--(-6.117,.453)--(-6.192,.454)--cycle;
\draw(-6.196,.406)--(-6.119,.405)--(-6.117,.453)--(-6.192,.454);
\filldraw[fill opacity=0.8,fill=gray!20,draw=none](-6.206,.415)--(-6.159,.373)--(-6.159,.453)--cycle;
\draw(-6.159,.373)--(-6.159,.453);
\filldraw[fill opacity=0.8,fill=gray!20,draw=none](-6.206,.415)--(-6.207,.412)--(-6.207,.406)--(-6.196,.406)--cycle;
\draw(-6.207,.406)--(-6.196,.406);
\filldraw[fill opacity=0.8,fill=gray!20,draw=none](-6.213,.354)--(-6.177,.354)--(-6.196,.406)--(-6.207,.406)--cycle;
\draw(-6.213,.354)--(-6.177,.354);
\draw(-6.196,.406)--(-6.207,.406);
\filldraw[fill opacity=0.8,fill=gray!20,draw=none](-6.159,.412)--(-6.159,.331)--(-6.103,.336)--(-6.103,.407)--cycle;
\draw(-6.159,.412)--(-6.159,.331);
\draw(-6.103,.336)--(-6.103,.407);
\filldraw[fill opacity=0.8,fill=gray!20](-6.152,.388)--(-6.316,.459)--(-6.326,.404)--(-6.162,.332)--cycle;
\filldraw[fill opacity=0.8,fill=gray!20,draw=none](-6.192,.548)--(-6.187,.554)--(-6.178,.554)--cycle;
\filldraw[fill opacity=0.8,fill=gray!20,draw=none](-6.187,.554)--(-6.178,.563)--(-6.178,.554)--cycle;
\draw(-6.178,.563)--(-6.178,.554);
\filldraw[fill opacity=0.8,fill=gray!20,draw=none](-7.877,4.394)--(-7.868,4.418)--(-7.876,4.393)--cycle;
\filldraw[fill opacity=0.8,fill=gray!20,draw=none](-8.525,3.143)--(-8.517,3.153)--(-8.502,3.146)--(-8.508,3.138)--(-8.535,3.119)--cycle;
\draw(-8.517,3.153)--(-8.502,3.146);
\filldraw[fill opacity=0.8,fill=gray!20,draw=none](-7.987,3.63)--(-7.948,3.635)--(-7.94,3.631)--(-7.987,3.63)--cycle;
\draw(-7.94,3.631)--(-7.987,3.63)--(-7.987,3.63)--(-7.948,3.635);
\filldraw[fill opacity=0.8,fill=gray!20](-7.987,3.63)--(-7.94,3.631)--(-7.962,3.626)--(-7.987,3.63)--cycle;
\filldraw[fill opacity=0.8,fill=gray!20](-7.987,3.63)--(-7.962,3.626)--(-7.991,3.625)--(-7.987,3.63)--cycle;
\filldraw[fill opacity=0.8,fill=gray!20](-7.987,3.63)--(-7.991,3.625)--(-8.019,3.627)--(-7.987,3.63)--cycle;
\filldraw[fill opacity=0.8,fill=gray!20](-7.987,3.63)--(-8.019,3.627)--(-8.038,3.632)--(-7.987,3.63)--cycle;
\filldraw[fill opacity=0.8,fill=gray!20](-8.085,3.648)--(-8.126,3.678)--(-8.142,3.695)--(-8.097,3.66)--cycle;
\filldraw[fill opacity=0.8,fill=gray!20](-7.927,4.05)--(-7.956,4.062)--(-7.936,4.057)--(-7.889,4.041)--cycle;
\filldraw[fill opacity=0.8,fill=gray!20](-7.783,1.016)--(-7.781,1.06)--(-7.703,1.054)--(-7.719,1.012)--cycle;
\filldraw[fill opacity=0.8,fill=gray!20](-7.043,.59)--(-6.999,.605)--(-6.977,.609)--(-7.001,.598)--cycle;
\filldraw[fill opacity=0.8,fill=gray!20,draw=none](-8.514,3.156)--(-8.453,3.168)--(-8.491,3.149)--(-8.503,3.147)--(-8.517,3.153)--cycle;
\draw(-8.503,3.147)--(-8.517,3.153);
\filldraw[fill opacity=0.8,fill=gray!20,draw=none](-8.543,3.031)--(-8.542,3.02)--(-8.561,3.044)--(-8.559,3.049)--(-8.548,3.044)--cycle;
\draw(-8.559,3.049)--(-8.548,3.044);
\filldraw[fill opacity=0.8,fill=gray!20,draw=none](-8.525,3.143)--(-8.535,3.119)--(-8.552,3.107)--(-8.554,3.108)--cycle;
\draw(-8.552,3.107)--(-8.554,3.108);
\filldraw[fill opacity=0.8,fill=gray!20,draw=none](-8.537,3.134)--(-8.554,3.113)--(-8.563,3.081)--cycle;
\filldraw[fill opacity=0.8,fill=gray!20,draw=none](-8.534,3.12)--(-8.561,3.082)--(-8.566,3.079)--(-8.544,3.13)--(-8.526,3.142)--cycle;
\draw(-8.561,3.082)--(-8.566,3.079)--(-8.544,3.13)--(-8.526,3.142);
\filldraw[fill opacity=0.5,fill=gray!20](-7.956,2.16)--(-7.825,2.317)--(-7.561,1.97)--(-7.721,1.852)--cycle;
\filldraw[fill opacity=0.5,fill=gray!20](-9.555,-.616)--(-9.382,-.692)--(-9.738,-.57)--(-9.911,-.494)--cycle;
\filldraw[fill opacity=0.8,fill=gray!20](-8.405,2.808)--(-8.452,2.824)--(-8.464,2.836)--(-8.411,2.814)--cycle;
\filldraw[fill opacity=0.8,fill=gray!20](-8.354,2.806)--(-8.405,2.808)--(-8.411,2.814)--(-8.354,2.806)--cycle;
\filldraw[fill opacity=0.8,fill=gray!20,draw=none](-8.502,3.179)--(-8.506,3.177)--(-8.464,3.207)--(-8.446,3.219)--(-8.484,3.191)--cycle;
\draw(-8.506,3.177)--(-8.464,3.207)--(-8.446,3.219)--(-8.484,3.191)--(-8.502,3.179);
\filldraw[fill opacity=0.8,fill=gray!20](-7.751,.222)--(-7.714,.234)--(-7.749,.227)--(-7.769,.218)--cycle;
\filldraw[fill opacity=0.8,fill=gray!20,draw=none](-8.543,3.031)--(-8.535,3.011)--(-8.542,3.02)--cycle;
\filldraw[fill opacity=0.8,fill=gray!20,draw=none](-5.967,.33)--(-5.972,.352)--(-5.965,.371)--cycle;
\filldraw[fill opacity=0.8,fill=gray!20,draw=none](-5.967,.376)--(-5.967,.213)--(-5.982,.272)--(-5.982,.303)--cycle;
\draw(-5.967,.376)--(-5.967,.213);
\draw(-5.982,.272)--(-5.982,.303);
\filldraw[fill opacity=0.8,fill=gray!20,draw=none](-7.945,.309)--(-7.934,.297)--(-7.943,.315)--cycle;
\draw(-7.945,.309)--(-7.934,.297);
\filldraw[fill opacity=0.8,fill=gray!20](-8.354,2.806)--(-8.358,2.801)--(-8.386,2.803)--(-8.354,2.806)--cycle;
\filldraw[fill opacity=0.8,fill=gray!20](-8.354,2.806)--(-8.386,2.803)--(-8.405,2.808)--(-8.354,2.806)--cycle;
\filldraw[fill opacity=0.8,fill=gray!20](-8.452,2.824)--(-8.493,2.854)--(-8.509,2.871)--(-8.464,2.836)--cycle;
\filldraw[fill opacity=0.8,fill=gray!20](-8.354,2.806)--(-8.298,2.813)--(-8.307,2.807)--(-8.354,2.806)--cycle;
\filldraw[fill opacity=0.8,fill=gray!20](-8.354,2.806)--(-8.307,2.807)--(-8.329,2.802)--(-8.354,2.806)--cycle;
\filldraw[fill opacity=0.8,fill=gray!20](-8.354,2.806)--(-8.329,2.802)--(-8.358,2.801)--(-8.354,2.806)--cycle;
\filldraw[fill opacity=0.8,fill=gray!20](-8.294,3.226)--(-8.323,3.238)--(-8.303,3.233)--(-8.256,3.217)--cycle;
\filldraw[fill opacity=0.8,fill=gray!20,draw=none](-7.676,4.893)--(-7.681,4.882)--(-7.687,4.876)--(-7.688,4.877)--(-7.68,4.894)--cycle;
\draw(-7.688,4.877)--(-7.68,4.894)--(-7.676,4.893)--(-7.681,4.882);
\filldraw[fill opacity=0.8,fill=gray!20,draw=none](-7.689,4.873)--(-7.687,4.876)--(-7.699,4.872)--cycle;
\draw(-7.687,4.876)--(-7.699,4.872);
\filldraw[fill opacity=0.8,fill=gray!20,draw=none](-7.685,4.873)--(-7.735,4.757)--(-7.741,4.755)--(-7.688,4.877)--cycle;
\draw(-7.685,4.873)--(-7.735,4.757);
\draw(-7.741,4.755)--(-7.688,4.877);
\filldraw[fill opacity=0.8,fill=gray!20,draw=none](-8.511,.857)--(-8.583,.858)--(-8.611,.866)--(-8.549,.88)--cycle;
\draw(-8.611,.866)--(-8.549,.88);
\filldraw[fill opacity=0.8,fill=gray!20,draw=none](-8.663,.884)--(-8.569,.89)--(-8.549,.88)--(-8.611,.866)--cycle;
\draw(-8.549,.88)--(-8.611,.866);
\filldraw[fill opacity=0.8,fill=gray!20,draw=none](-8.569,.89)--(-8.663,.884)--(-8.699,.897)--(-8.617,.915)--cycle;
\draw(-8.699,.897)--(-8.617,.915);
\filldraw[fill opacity=0.8,fill=gray!20,draw=none](-8.583,.858)--(-8.511,.857)--(-8.5,.85)--(-8.534,.843)--cycle;
\draw(-8.5,.85)--(-8.534,.843);
\filldraw[fill opacity=0.8,fill=gray!20,draw=none](-8.583,.858)--(-8.645,.858)--(-8.611,.866)--cycle;
\draw(-8.645,.858)--(-8.611,.866);
\filldraw[fill opacity=0.8,fill=gray!20,draw=none](-8.645,.858)--(-8.583,.858)--(-8.534,.843)--cycle;
\filldraw[fill opacity=0.8,fill=gray!20,draw=none](-8.812,.949)--(-8.837,.965)--(-8.809,.971)--cycle;
\draw(-8.837,.965)--(-8.809,.971);
\filldraw[fill opacity=0.8,fill=gray!20,draw=none](-8.788,.936)--(-8.786,.958)--(-8.785,.935)--cycle;
\draw(-8.786,.958)--(-8.785,.935);
\filldraw[fill opacity=0.8,fill=gray!20,draw=none](-8.764,.933)--(-8.785,.935)--(-8.786,.958)--(-8.758,.951)--cycle;
\draw(-8.785,.935)--(-8.786,.958)--(-8.758,.951);
\filldraw[fill opacity=0.8,fill=gray!20,draw=none](-8.746,.931)--(-8.772,.911)--(-8.764,.933)--cycle;
\filldraw[fill opacity=0.8,fill=gray!20,draw=none](-8.732,.942)--(-8.746,.931)--(-8.764,.933)--(-8.758,.951)--(-8.739,.947)--cycle;
\draw(-8.758,.951)--(-8.739,.947);
\filldraw[fill opacity=0.8,fill=gray!20,draw=none](-8.859,.969)--(-8.865,.984)--(-8.851,.974)--cycle;
\filldraw[fill opacity=0.8,fill=gray!20,draw=none](-8.826,1.008)--(-8.892,1.013)--(-8.843,1.019)--cycle;
\draw(-8.826,1.008)--(-8.892,1.013);
\filldraw[fill opacity=0.8,fill=gray!20,draw=none](-8.871,1.011)--(-8.84,1.009)--(-8.828,1.002)--cycle;
\draw(-8.871,1.011)--(-8.84,1.009);
\filldraw[fill opacity=0.8,fill=gray!20,draw=none](-8.826,1.008)--(-8.828,1.009)--(-8.801,1.01)--(-8.799,1.006)--cycle;
\draw(-8.801,1.01)--(-8.799,1.006)--(-8.826,1.008);
\filldraw[fill opacity=0.8,fill=gray!20,draw=none](-8.828,1.009)--(-8.843,1.019)--(-8.837,1.02)--(-8.801,1.01)--cycle;
\filldraw[fill opacity=0.8,fill=gray!20,draw=none](-8.836,1.02)--(-8.808,1.012)--(-8.82,.999)--(-8.825,.997)--(-8.855,1.014)--(-8.854,1.017)--cycle;
\draw(-8.855,1.014)--(-8.854,1.017);
\filldraw[fill opacity=0.8,fill=gray!20,draw=none](-8.854,.964)--(-8.823,.996)--(-8.81,.989)--(-8.825,.955)--cycle;
\draw(-8.81,.989)--(-8.825,.955);
\filldraw[fill opacity=0.8,fill=gray!20,draw=none](-8.82,.999)--(-8.823,.996)--(-8.825,.997)--cycle;
\filldraw[fill opacity=0.8,fill=gray!20,draw=none](-8.82,.995)--(-8.823,.996)--(-8.82,.999)--(-8.803,1.006)--cycle;
\filldraw[fill opacity=0.8,fill=gray!20,draw=none](-8.851,.974)--(-8.865,.984)--(-8.867,.991)--(-8.833,.999)--(-8.815,.998)--cycle;
\draw(-8.867,.991)--(-8.833,.999);
\filldraw[fill opacity=0.8,fill=gray!20,draw=none](-8.84,.97)--(-8.855,.964)--(-8.859,.969)--(-8.865,.984)--(-8.841,.973)--cycle;
\draw(-8.865,.984)--(-8.841,.973);
\filldraw[fill opacity=0.8,fill=gray!20,draw=none](-8.835,.937)--(-8.844,.961)--(-8.825,.955)--(-8.833,.937)--cycle;
\draw(-8.825,.955)--(-8.833,.937);
\filldraw[fill opacity=0.8,fill=gray!20,draw=none](-8.833,.942)--(-8.838,.944)--(-8.855,.964)--(-8.84,.97)--cycle;
\draw(-8.833,.942)--(-8.838,.944);
\filldraw[fill opacity=0.8,fill=gray!20,draw=none](-8.855,.964)--(-8.854,.964)--(-8.854,.963)--cycle;
\filldraw[fill opacity=0.8,fill=gray!20,draw=none](-8.835,.938)--(-8.835,.937)--(-8.838,.944)--(-8.835,.943)--cycle;
\draw(-8.838,.944)--(-8.835,.943);
\filldraw[fill opacity=0.8,fill=gray!20,draw=none](-8.833,.937)--(-8.808,.951)--(-8.779,.945)--cycle;
\filldraw[fill opacity=0.8,fill=gray!20,draw=none](-8.808,.951)--(-8.833,.937)--(-8.825,.955)--cycle;
\draw(-8.833,.937)--(-8.825,.955);
\filldraw[fill opacity=0.8,fill=gray!20,draw=none](-8.799,.997)--(-8.72,.991)--(-8.802,.973)--cycle;
\draw(-8.72,.991)--(-8.802,.973);
\filldraw[fill opacity=0.8,fill=gray!20,draw=none](-8.801,.997)--(-8.799,.997)--(-8.802,.973)--(-8.813,.97)--cycle;
\draw(-8.802,.973)--(-8.813,.97);
\filldraw[fill opacity=0.8,fill=gray!20,draw=none](-8.81,.96)--(-8.813,.97)--(-8.804,.99)--(-8.793,.983)--(-8.786,.958)--cycle;
\draw(-8.793,.983)--(-8.786,.958)--(-8.81,.96);
\filldraw[fill opacity=0.8,fill=gray!20,draw=none](-8.798,1.004)--(-8.799,1.006)--(-8.798,1.006)--cycle;
\draw(-8.798,1.004)--(-8.799,1.006)--(-8.798,1.006);
\filldraw[fill opacity=0.8,fill=gray!20,draw=none](-8.795,.996)--(-8.798,1.007)--(-7.96,1.195)--(-7.955,1.171)--(-7.957,1.162)--(-8.72,.991)--cycle;
\draw(-8.798,1.007)--(-7.96,1.195);
\draw(-7.957,1.162)--(-8.72,.991);
\filldraw[fill opacity=0.8,fill=gray!20,draw=none](-8.786,.958)--(-8.794,.99)--(-8.762,.984)--(-8.718,.951)--(-8.714,.94)--cycle;
\draw(-8.718,.951)--(-8.714,.94)--(-8.786,.958)--(-8.794,.99);
\filldraw[fill opacity=0.8,fill=gray!20,draw=none](-8.779,.969)--(-8.808,.951)--(-8.825,.955)--(-8.81,.989)--cycle;
\draw(-8.825,.955)--(-8.81,.989);
\filldraw[fill opacity=0.8,fill=gray!20,draw=none](-8.835,.938)--(-8.835,.943)--(-8.833,.942)--cycle;
\draw(-8.835,.943)--(-8.833,.942);
\filldraw[fill opacity=0.8,fill=gray!20,draw=none](-8.808,.951)--(-8.779,.969)--(-8.771,.963)--(-8.779,.945)--cycle;
\draw(-8.771,.963)--(-8.779,.945);
\filldraw[fill opacity=0.8,fill=gray!20,draw=none](-8.879,.986)--(-7.917,.585)--(-7.911,.552)--(-8.909,.968)--cycle;
\draw(-7.911,.552)--(-8.909,.968)--(-8.879,.986)--(-7.917,.585);
\filldraw[fill opacity=0.8,fill=gray!20,draw=none](-6.159,.549)--(-6.165,.548)--(-6.158,.545)--cycle;
\draw(-6.165,.548)--(-6.158,.545);
\filldraw[fill opacity=0.5,fill=gray!20](-9.294,3.036)--(-9.355,3.023)--(-8.889,3.007)--(-8.819,3.019)--cycle;
\filldraw[fill opacity=0.8,fill=gray!20,draw=none](-7.665,.27)--(-7.675,.268)--(-7.682,.258)--cycle;
\draw(-7.675,.268)--(-7.682,.258)--(-7.665,.27);
\filldraw[fill opacity=0.8,fill=gray!20](-7.632,.309)--(-7.614,.351)--(-7.643,.333)--(-7.658,.292)--cycle;
\filldraw[fill opacity=0.8,fill=gray!20](-7.954,.382)--(-7.948,.429)--(-7.967,.449)--(-7.973,.402)--cycle;
\filldraw[fill opacity=0.8,fill=gray!20,draw=none](-7.614,4.291)--(-7.602,4.309)--(-7.61,4.289)--cycle;
\draw(-7.602,4.309)--(-7.61,4.289)--(-7.614,4.291);
\filldraw[fill opacity=0.8,fill=gray!20,draw=none](-7.868,4.418)--(-7.817,4.555)--(-7.816,4.554)--cycle;
\filldraw[fill opacity=0.8,fill=gray!20,draw=none](-6.241,.509)--(-6.231,.502)--(-6.253,.495)--(-6.297,.495)--cycle;
\draw(-6.253,.495)--(-6.297,.495);
\filldraw[fill opacity=0.8,fill=gray!20,draw=none](-6.252,.493)--(-6.271,.502)--(-6.287,.495)--(-6.301,.485)--(-6.316,.459)--(-6.271,.44)--cycle;
\draw(-6.252,.493)--(-6.271,.502);
\draw(-6.301,.485)--(-6.316,.459)--(-6.271,.44);
\filldraw[fill opacity=0.8,fill=gray!20](-8.126,3.678)--(-8.157,3.719)--(-8.177,3.74)--(-8.142,3.695)--cycle;
\filldraw[fill opacity=0.8,fill=gray!20](-7.931,3.637)--(-7.878,3.658)--(-7.896,3.646)--(-7.94,3.631)--cycle;
\filldraw[fill opacity=0.5,fill=gray!20](-9.43,2.611)--(-9.891,2.812)--(-9.461,2.921)--(-9,2.719)--cycle;
\filldraw[fill opacity=0.5,fill=gray!20](-9.862,2.871)--(-9.891,2.812)--(-9.461,2.921)--(-9.411,2.985)--cycle;
\filldraw[fill opacity=0.8,fill=gray!20](-7.107,.545)--(-7.061,.578)--(-7.043,.59)--(-7.081,.561)--cycle;
\filldraw[fill opacity=0.8,fill=gray!20](-6.952,.177)--(-7.002,.179)--(-7.008,.185)--(-6.952,.177)--cycle;
\filldraw[fill opacity=0.8,fill=gray!20](-7.002,.179)--(-7.05,.195)--(-7.061,.207)--(-7.008,.185)--cycle;
\filldraw[fill opacity=0.8,fill=gray!20,draw=none](-7.772,4.516)--(-7.775,4.515)--(-7.77,4.512)--cycle;
\draw(-7.772,4.516)--(-7.775,4.515);
\filldraw[fill opacity=0.8,fill=gray!20,draw=none](-7.795,4.545)--(-7.802,4.555)--(-7.806,4.56)--(-7.808,4.561)--(-7.807,4.56)--(-7.796,4.546)--cycle;
\draw(-7.806,4.56)--(-7.808,4.561);
\draw(-7.807,4.56)--(-7.796,4.546);
\filldraw[fill opacity=0.8,fill=gray!20,draw=none](-6.245,.277)--(-6.241,.283)--(-6.226,.276)--(-6.243,.276)--cycle;
\draw(-6.226,.276)--(-6.243,.276);
\filldraw[fill opacity=0.8,fill=gray!20,draw=none](-6.187,.554)--(-6.192,.548)--(-6.223,.535)--(-6.223,.557)--cycle;
\draw(-6.223,.535)--(-6.223,.557);
\filldraw[fill opacity=0.8,fill=gray!20,draw=none](-8.568,2.993)--(-8.57,3.002)--(-8.571,3.006)--(-8.569,2.993)--cycle;
\draw(-8.571,3.006)--(-8.569,2.993);
\filldraw[fill opacity=0.8,fill=gray!20,draw=none](-8.57,3.002)--(-8.572,3.011)--(-8.571,3.006)--cycle;
\draw(-8.572,3.011)--(-8.571,3.006);
\filldraw[fill opacity=0.8,fill=gray!20,draw=none](-8.57,3.026)--(-8.561,3.044)--(-8.535,3.011)--(-8.527,2.992)--(-8.526,2.983)--(-8.573,3.003)--cycle;
\draw(-8.526,2.983)--(-8.573,3.003);
\filldraw[fill opacity=0.8,fill=gray!20,draw=none](-6.205,.489)--(-6.223,.464)--(-6.223,.496)--cycle;
\draw(-6.223,.464)--(-6.223,.496);
\filldraw[fill opacity=0.8,fill=gray!20,draw=none](-7.749,4.498)--(-7.728,4.482)--(-7.718,4.472)--(-7.759,4.502)--(-7.765,4.508)--cycle;
\draw(-7.728,4.482)--(-7.718,4.472)--(-7.759,4.502)--(-7.765,4.508);
\filldraw[fill opacity=0.8,fill=gray!20,draw=none](-7.521,4.578)--(-7.522,4.588)--(-7.519,4.588)--cycle;
\filldraw[fill opacity=0.8,fill=gray!20,draw=none](-8.035,4.058)--(-7.855,4.463)--(-7.854,4.463)--(-7.984,4.164)--(-8.033,4.054)--cycle;
\draw(-8.035,4.058)--(-7.855,4.463);
\draw(-7.984,4.164)--(-8.033,4.054);
\filldraw[fill opacity=0.8,fill=gray!20,draw=none](-8.055,4.052)--(-8.035,4.058)--(-8.013,4.063)--(-8.03,4.054)--cycle;
\draw(-8.055,4.052)--(-8.035,4.058)--(-8.013,4.063)--(-8.03,4.054);
\filldraw[fill opacity=0.8,fill=gray!20](-6.952,.177)--(-6.983,.174)--(-7.002,.179)--(-6.952,.177)--cycle;
\filldraw[fill opacity=0.8,fill=gray!20](-6.952,.177)--(-6.895,.183)--(-6.904,.177)--(-6.952,.177)--cycle;
\filldraw[fill opacity=0.8,fill=gray!20](-6.952,.177)--(-6.904,.177)--(-6.926,.173)--(-6.952,.177)--cycle;
\filldraw[fill opacity=0.8,fill=gray!20](-6.952,.177)--(-6.926,.173)--(-6.955,.172)--(-6.952,.177)--cycle;
\filldraw[fill opacity=0.8,fill=gray!20](-6.952,.177)--(-6.955,.172)--(-6.983,.174)--(-6.952,.177)--cycle;
\filldraw[fill opacity=0.8,fill=gray!20](-6.891,.597)--(-6.92,.608)--(-6.901,.604)--(-6.853,.587)--cycle;
\filldraw[fill opacity=0.8,fill=gray!20](-7.788,.583)--(-7.79,.579)--(-7.79,.579)--(-7.764,.581)--cycle;
\filldraw[fill opacity=0.8,fill=gray!20](-7.812,.582)--(-7.79,.579)--(-7.79,.579)--(-7.788,.583)--cycle;
\filldraw[fill opacity=0.8,fill=gray!20,draw=none](-6.22,.495)--(-6.2,.494)--(-6.209,.484)--cycle;
\draw(-6.22,.495)--(-6.2,.494);
\filldraw[fill opacity=0.8,fill=gray!20,draw=none](-7.808,4.561)--(-7.81,4.564)--(-7.807,4.56)--cycle;
\draw(-7.808,4.561)--(-7.81,4.564)--(-7.807,4.56);
\filldraw[fill opacity=0.8,fill=gray!20](-8.493,2.854)--(-8.524,2.895)--(-8.544,2.916)--(-8.509,2.871)--cycle;
\filldraw[fill opacity=0.8,fill=gray!20,draw=none](-8.298,2.813)--(-8.268,2.824)--(-8.263,2.821)--(-8.307,2.807)--cycle;
\draw(-8.263,2.821)--(-8.307,2.807)--(-8.298,2.813)--(-8.268,2.824);
\filldraw[fill opacity=0.8,fill=gray!20,draw=none](-7.749,4.498)--(-7.765,4.508)--(-7.773,4.517)--cycle;
\draw(-7.765,4.508)--(-7.773,4.517);
\filldraw[fill opacity=0.8,fill=gray!20,draw=none](-7.812,4.552)--(-7.808,4.562)--(-7.807,4.56)--(-7.811,4.552)--cycle;
\draw(-7.807,4.56)--(-7.811,4.552);
\filldraw[fill opacity=0.8,fill=gray!20,draw=none](-7.808,4.563)--(-7.789,4.612)--(-7.801,4.597)--(-7.81,4.569)--cycle;
\filldraw[fill opacity=0.8,fill=gray!20,draw=none](-7.808,4.563)--(-7.807,4.564)--(-7.801,4.584)--(-7.792,4.604)--(-7.789,4.612)--cycle;
\draw(-7.792,4.604)--(-7.789,4.612);
\filldraw[fill opacity=0.8,fill=gray!20,draw=none](-7.807,4.599)--(-7.815,4.579)--(-7.81,4.569)--(-7.795,4.615)--(-7.796,4.614)--cycle;
\draw(-7.795,4.615)--(-7.796,4.614);
\filldraw[fill opacity=0.8,fill=gray!20,draw=none](-7.81,4.569)--(-7.81,4.572)--(-7.808,4.58)--(-7.795,4.615)--cycle;
\filldraw[fill opacity=0.8,fill=gray!20,draw=none](-7.807,4.568)--(-7.801,4.584)--(-7.807,4.564)--cycle;
\filldraw[fill opacity=0.8,fill=gray!20,draw=none](-7.8,4.553)--(-7.81,4.573)--(-7.81,4.565)--(-7.81,4.564)--cycle;
\draw(-7.81,4.565)--(-7.81,4.564)--(-7.8,4.553);
\filldraw[fill opacity=0.8,fill=gray!20,draw=none](-7.81,4.556)--(-7.813,4.56)--(-7.818,4.547)--(-7.813,4.549)--cycle;
\draw(-7.818,4.547)--(-7.813,4.549);
\filldraw[fill opacity=0.8,fill=gray!20,draw=none](-7.851,4.462)--(-7.813,4.553)--(-7.812,4.552)--(-7.815,4.542)--(-7.851,4.462)--cycle;
\draw(-7.815,4.542)--(-7.851,4.462);
\filldraw[fill opacity=0.8,fill=gray!20,draw=none](-6.164,.531)--(-6.192,.548)--(-6.223,.535)--(-6.172,.513)--cycle;
\draw(-6.223,.535)--(-6.172,.513);
\filldraw[fill opacity=0.8,fill=gray!20,draw=none](-8.527,2.992)--(-8.523,2.981)--(-8.526,2.983)--cycle;
\draw(-8.523,2.981)--(-8.526,2.983);
\filldraw[fill opacity=0.8,fill=gray!20,draw=none](-7.832,4.65)--(-7.834,4.633)--(-7.839,4.671)--(-7.832,4.676)--cycle;
\draw(-7.834,4.633)--(-7.839,4.671)--(-7.832,4.676);
\filldraw[fill opacity=0.8,fill=gray!20,draw=none](-7.81,4.569)--(-7.815,4.579)--(-7.815,4.576)--(-7.81,4.565)--cycle;
\draw(-7.815,4.576)--(-7.81,4.565);
\filldraw[fill opacity=0.8,fill=gray!20,draw=none](-7.81,4.565)--(-7.813,4.56)--(-7.81,4.569)--cycle;
\filldraw[fill opacity=0.8,fill=gray!20,draw=none](-7.851,4.462)--(-7.851,4.462)--(-7.86,4.441)--cycle;
\draw(-7.851,4.462)--(-7.86,4.441);
\filldraw[fill opacity=0.8,fill=gray!20,draw=none](-8.545,2.928)--(-8.549,2.927)--(-8.55,2.93)--cycle;
\draw(-8.549,2.927)--(-8.55,2.93);
\filldraw[fill opacity=0.8,fill=gray!20,draw=none](-6.252,.271)--(-6.257,.255)--(-6.274,.255)--cycle;
\draw(-6.257,.255)--(-6.274,.255);
\filldraw[fill opacity=0.8,fill=gray!20,draw=none](-6.264,.255)--(-6.257,.255)--(-6.259,.255)--cycle;
\draw(-6.264,.255)--(-6.257,.255);
\filldraw[fill opacity=0.8,fill=gray!20,draw=none](-6.252,.291)--(-6.252,.214)--(-6.259,.201)--(-6.259,.3)--cycle;
\draw(-6.252,.291)--(-6.252,.214)--(-6.259,.201)--(-6.259,.3);
\filldraw[fill opacity=0.8,fill=gray!20,draw=none](-7.499,4.639)--(-7.489,4.663)--(-7.487,4.633)--cycle;
\draw(-7.499,4.639)--(-7.489,4.663)--(-7.487,4.633);
\filldraw[fill opacity=0.8,fill=gray!20,draw=none](-7.494,4.702)--(-7.489,4.663)--(-7.499,4.639)--(-7.671,4.705)--(-7.713,4.734)--(-7.701,4.745)--(-7.658,4.764)--(-7.607,4.77)--(-7.558,4.76)--(-7.518,4.736)--cycle;
\draw(-7.701,4.745)--(-7.658,4.764)--(-7.607,4.77)--(-7.558,4.76)--(-7.518,4.736)--(-7.494,4.702)--(-7.489,4.663)--(-7.499,4.639);
\filldraw[fill opacity=0.8,fill=gray!20](-7.709,.564)--(-7.748,.577)--(-7.743,.572)--(-7.699,.553)--cycle;
\filldraw[fill opacity=0.8,fill=gray!20](-7.481,4.835)--(-7.522,4.865)--(-7.511,4.853)--(-7.465,4.818)--cycle;
\filldraw[fill opacity=0.8,fill=gray!20,draw=none](-8.444,3.173)--(-8.44,3.171)--(-8.453,3.168)--cycle;
\draw(-8.444,3.173)--(-8.44,3.171);
\filldraw[fill opacity=0.8,fill=gray!20,draw=none](-8.431,3.172)--(-8.44,3.171)--(-8.448,3.174)--cycle;
\draw(-8.44,3.171)--(-8.448,3.174);
\filldraw[fill opacity=0.8,fill=gray!20,draw=none](-8.389,3.255)--(-8.4,3.23)--(-8.402,3.234)--(-8.386,3.271)--cycle;
\draw(-8.389,3.255)--(-8.4,3.23);
\draw(-8.402,3.234)--(-8.386,3.271);
\filldraw[fill opacity=0.8,fill=gray!20,draw=none](-8.422,3.227)--(-8.402,3.234)--(-8.38,3.239)--(-8.397,3.23)--cycle;
\draw(-8.422,3.227)--(-8.402,3.234)--(-8.38,3.239)--(-8.397,3.23);
\filldraw[fill opacity=0.8,fill=gray!20,draw=none](-8.517,2.971)--(-8.529,2.984)--(-8.523,2.981)--cycle;
\draw(-8.529,2.984)--(-8.523,2.981);
\filldraw[fill opacity=0.8,fill=gray!20,draw=none](-8.495,3.184)--(-8.485,3.191)--(-8.444,3.173)--(-8.453,3.168)--(-8.514,3.156)--cycle;
\draw(-8.485,3.191)--(-8.444,3.173);
\filldraw[fill opacity=0.8,fill=gray!20,draw=none](-7.795,4.545)--(-7.773,4.517)--(-7.759,4.502)--(-7.79,4.543)--cycle;
\draw(-7.773,4.517)--(-7.759,4.502)--(-7.79,4.543);
\filldraw[fill opacity=0.8,fill=gray!20,draw=none](-8.107,3.898)--(-8.035,4.058)--(-8.033,4.054)--(-8.103,3.896)--cycle;
\draw(-8.033,4.054)--(-8.103,3.896)--(-8.107,3.898)--(-8.035,4.058);
\filldraw[fill opacity=0.8,fill=gray!20,draw=none](-7.815,4.576)--(-7.818,4.572)--(-7.815,4.579)--cycle;
\filldraw[fill opacity=0.8,fill=gray!20,draw=none](-7.81,4.556)--(-7.813,4.549)--(-7.808,4.552)--cycle;
\draw(-7.813,4.549)--(-7.808,4.552);
\filldraw[fill opacity=0.8,fill=gray!20,draw=none](-7.807,4.551)--(-7.826,4.559)--(-7.823,4.557)--cycle;
\draw(-7.826,4.559)--(-7.823,4.557);
\filldraw[fill opacity=0.8,fill=gray!20,draw=none](-7.81,4.569)--(-7.824,4.601)--(-7.833,4.597)--(-7.833,4.592)--(-7.813,4.56)--cycle;
\draw(-7.824,4.601)--(-7.833,4.597);
\filldraw[fill opacity=0.8,fill=gray!20,draw=none](-7.818,4.572)--(-7.823,4.557)--(-7.824,4.558)--(-7.821,4.566)--cycle;
\draw(-7.823,4.557)--(-7.824,4.558);
\filldraw[fill opacity=0.8,fill=gray!20,draw=none](-6.245,.277)--(-6.243,.276)--(-6.246,.276)--cycle;
\draw(-6.243,.276)--(-6.246,.276);
\filldraw[fill opacity=0.8,fill=gray!20,draw=none](-7.818,4.547)--(-7.816,4.554)--(-7.815,4.554)--cycle;
\filldraw[fill opacity=0.8,fill=gray!20,draw=none](-8.175,3.737)--(-8.174,3.746)--(-8.185,3.759)--(-8.177,3.74)--cycle;
\draw(-8.185,3.759)--(-8.177,3.74)--(-8.175,3.737);
\filldraw[fill opacity=0.8,fill=gray!20,draw=none](-8.103,3.896)--(-8.174,3.736)--(-8.176,3.741)--(-8.107,3.898)--cycle;
\draw(-8.176,3.741)--(-8.107,3.898)--(-8.103,3.896)--(-8.174,3.736);
\filldraw[fill opacity=0.8,fill=gray!20](-6.895,.183)--(-6.842,.204)--(-6.86,.192)--(-6.904,.177)--cycle;
\filldraw[fill opacity=0.8,fill=gray!20](-7.83,.578)--(-7.79,.579)--(-7.79,.579)--(-7.812,.582)--cycle;
\filldraw[fill opacity=0.8,fill=gray!20,draw=none](-8.174,3.736)--(-8.187,3.717)--(-8.176,3.741)--cycle;
\draw(-8.187,3.717)--(-8.176,3.741);
\filldraw[fill opacity=0.8,fill=gray!20,draw=none](-8.174,3.736)--(-8.175,3.734)--(-8.189,3.713)--(-8.187,3.717)--cycle;
\draw(-8.174,3.736)--(-8.175,3.734);
\draw(-8.189,3.713)--(-8.187,3.717);
\filldraw[fill opacity=0.8,fill=gray!20,draw=none](-8.175,3.734)--(-8.389,3.255)--(-8.386,3.271)--(-8.189,3.713)--cycle;
\draw(-8.175,3.734)--(-8.389,3.255);
\draw(-8.386,3.271)--(-8.189,3.713);
\filldraw[fill opacity=0.8,fill=gray!20](-7.878,3.658)--(-7.832,3.691)--(-7.858,3.674)--(-7.896,3.646)--cycle;
\filldraw[fill opacity=0.8,fill=gray!20,draw=none](-7.831,4.616)--(-7.832,4.616)--(-7.834,4.633)--(-7.832,4.65)--cycle;
\draw(-7.831,4.616)--(-7.832,4.616)--(-7.834,4.633);
\filldraw[fill opacity=0.8,fill=gray!20,draw=none](-6.252,.271)--(-6.246,.276)--(-6.226,.276)--(-6.183,.254)--(-6.257,.255)--cycle;
\draw(-6.246,.276)--(-6.226,.276);
\draw(-6.183,.254)--(-6.257,.255);
\filldraw[fill opacity=0.8,fill=gray!20,draw=none](-8.554,3.113)--(-8.541,3.158)--(-8.566,3.113)--(-8.576,3.084)--cycle;
\draw(-8.541,3.158)--(-8.566,3.113)--(-8.576,3.084);
\filldraw[fill opacity=0.8,fill=gray!20](-7.78,1.199)--(-7.781,1.238)--(-7.703,1.232)--(-7.693,1.193)--cycle;
\filldraw[fill opacity=0.8,fill=gray!20,draw=none](-8.423,3.182)--(-8.42,3.184)--(-8.426,3.172)--cycle;
\draw(-8.42,3.184)--(-8.426,3.172);
\filldraw[fill opacity=0.8,fill=gray!20,draw=none](-8.483,3.192)--(-8.382,3.176)--(-8.431,3.172)--(-8.448,3.174)--(-8.485,3.191)--cycle;
\draw(-8.448,3.174)--(-8.485,3.191);
\filldraw[fill opacity=0.8,fill=gray!20](-7.848,4.011)--(-7.889,4.041)--(-7.878,4.029)--(-7.832,3.994)--cycle;
\filldraw[fill opacity=0.8,fill=gray!20,draw=none](-7.124,.264)--(-7.125,.264)--(-7.125,.265)--cycle;
\draw(-7.124,.264)--(-7.125,.264);
\filldraw[fill opacity=0.8,fill=gray!20,draw=none](-6.956,.28)--(-7.129,.282)--(-7.126,.263)--(-6.954,.261)--cycle;
\draw(-7.126,.263)--(-6.954,.261)--(-6.956,.28)--(-7.129,.282);
\filldraw[fill opacity=0.8,fill=gray!20,draw=none](-7.106,.241)--(-7.091,.224)--(-7.122,.265)--(-7.127,.271)--cycle;
\draw(-7.106,.241)--(-7.091,.224)--(-7.122,.265)--(-7.127,.271);
\filldraw[fill opacity=0.8,fill=gray!20,draw=none](-8.57,3.002)--(-8.568,2.993)--(-8.567,2.993)--cycle;
\filldraw[fill opacity=0.8,fill=gray!20,draw=none](-8.517,2.971)--(-8.509,2.958)--(-8.576,2.999)--(-8.573,3.003)--(-8.529,2.984)--cycle;
\draw(-8.573,3.003)--(-8.529,2.984);
\filldraw[fill opacity=0.8,fill=gray!20,draw=none](-7.955,.337)--(-7.942,.321)--(-7.948,.337)--(-7.962,.351)--cycle;
\draw(-7.942,.321)--(-7.948,.337)--(-7.962,.351);
\filldraw[fill opacity=0.8,fill=gray!20,draw=none](-7.91,.317)--(-7.904,.326)--(-7.948,.337)--(-7.942,.321)--cycle;
\draw(-7.904,.326)--(-7.948,.337)--(-7.942,.321);
\filldraw[fill opacity=0.8,fill=gray!20,draw=none](-8.823,.694)--(-8.834,.704)--(-8.834,.713)--(-8.818,.741)--(-8.764,.728)--(-8.805,.689)--cycle;
\draw(-8.834,.713)--(-8.818,.741)--(-8.764,.728)--(-8.805,.689)--(-8.823,.694);
\filldraw[fill opacity=0.8,fill=gray!20,draw=none](-8.823,.694)--(-8.805,.689)--(-8.814,.684)--cycle;
\draw(-8.823,.694)--(-8.805,.689)--(-8.814,.684);
\filldraw[fill opacity=0.8,fill=gray!20,draw=none](-8.788,.684)--(-8.8,.684)--(-8.805,.689)--(-8.764,.728)--(-8.748,.711)--(-8.778,.689)--cycle;
\draw(-8.8,.684)--(-8.805,.689)--(-8.764,.728)--(-8.748,.711)--(-8.778,.689);
\filldraw[fill opacity=0.8,fill=gray!20,draw=none](-8.814,.684)--(-8.805,.689)--(-8.8,.684)--cycle;
\draw(-8.814,.684)--(-8.805,.689)--(-8.8,.684);
\filldraw[fill opacity=0.8,fill=gray!20,draw=none](-8.948,.741)--(-8.937,.747)--(-8.928,.746)--(-8.903,.736)--cycle;
\draw(-8.928,.746)--(-8.903,.736)--(-8.948,.741);
\filldraw[fill opacity=0.8,fill=gray!20](-7.861,1.235)--(-7.848,1.266)--(-7.783,1.269)--(-7.781,1.238)--cycle;
\filldraw[fill opacity=0.8,fill=gray!20,draw=none](-9.095,1.079)--(-9.099,1.084)--(-9.181,.991)--(-9.172,.981)--(-9.092,1.072)--cycle;
\draw(-9.181,.991)--(-9.172,.981)--(-9.092,1.072);
\filldraw[fill opacity=0.8,fill=gray!20](-9.078,.985)--(-9.083,1.032)--(-9.152,1.019)--(-9.149,.971)--cycle;
\filldraw[fill opacity=0.8,fill=gray!20](-9.078,.985)--(-9.083,1.032)--(-9.152,1.019)--(-9.149,.971)--cycle;
\filldraw[fill opacity=0.8,fill=gray!20,draw=none](-9.181,.991)--(-9.165,.973)--(-9.172,.981)--cycle;
\draw(-9.165,.973)--(-9.172,.981)--(-9.181,.991);
\filldraw[fill opacity=0.8,fill=gray!20,draw=none](-9.181,.991)--(-9.165,.973)--(-9.172,.981)--cycle;
\draw(-9.165,.973)--(-9.172,.981)--(-9.181,.991);
\filldraw[fill opacity=0.8,fill=gray!20,draw=none](-9.181,.991)--(-9.165,.973)--(-9.172,.981)--cycle;
\draw(-9.165,.973)--(-9.172,.981)--(-9.181,.991);
\filldraw[fill opacity=0.8,fill=gray!20](-9.152,.926)--(-9.149,.971)--(-9.242,.967)--(-9.241,.922)--cycle;
\filldraw[fill opacity=0.8,fill=gray!20](-9.152,.926)--(-9.149,.971)--(-9.242,.967)--(-9.241,.922)--cycle;
\filldraw[fill opacity=0.8,fill=gray!20](-9.152,1.019)--(-9.16,1.065)--(-9.24,1.062)--(-9.241,1.015)--cycle;
\filldraw[fill opacity=0.8,fill=gray!20](-9.152,1.019)--(-9.16,1.065)--(-9.24,1.062)--(-9.241,1.015)--cycle;
\filldraw[fill opacity=0.8,fill=gray!20](-9.149,.971)--(-9.152,1.019)--(-9.241,1.015)--(-9.242,.967)--cycle;
\filldraw[fill opacity=0.8,fill=gray!20](-9.149,.971)--(-9.152,1.019)--(-9.241,1.015)--(-9.242,.967)--cycle;
\filldraw[fill opacity=0.8,fill=gray!20,draw=none](-9.281,.97)--(-9.242,.967)--(-9.241,1.015)--(-9.273,1.017)--cycle;
\draw(-9.281,.97)--(-9.242,.967)--(-9.241,1.015)--(-9.273,1.017);
\filldraw[fill opacity=0.8,fill=gray!20,draw=none](-9.281,.97)--(-9.242,.967)--(-9.241,1.015)--(-9.273,1.017)--cycle;
\draw(-9.281,.97)--(-9.242,.967)--(-9.241,1.015)--(-9.273,1.017);
\filldraw[fill opacity=0.8,fill=gray!20,draw=none](-9.273,1.017)--(-9.241,1.015)--(-9.24,1.062)--(-9.251,1.063)--cycle;
\draw(-9.273,1.017)--(-9.241,1.015)--(-9.24,1.062)--(-9.251,1.063);
\filldraw[fill opacity=0.8,fill=gray!20,draw=none](-9.273,1.017)--(-9.241,1.015)--(-9.24,1.062)--(-9.251,1.063)--cycle;
\draw(-9.273,1.017)--(-9.241,1.015)--(-9.24,1.062)--(-9.251,1.063);
\filldraw[fill opacity=0.8,fill=gray!20,draw=none](-9.258,1.009)--(-9.26,.994)--(-9.047,.905)--(-9.038,.913)--(-9.037,.915)--(-9.061,.959)--(-9.243,1.035)--cycle;
\draw(-9.038,.913)--(-9.037,.915);
\filldraw[fill opacity=0.8,fill=gray!20,draw=none](-9.006,.863)--(-9.019,.869)--(-9.045,.897)--(-9.044,.927)--(-9.04,.934)--(-9.013,.922)--cycle;
\draw(-9.006,.863)--(-9.019,.869);
\draw(-9.04,.934)--(-9.013,.922);
\filldraw[fill opacity=0.8,fill=gray!20,draw=none](-8.491,.841)--(-8.534,.843)--(-8.5,.85)--cycle;
\draw(-8.534,.843)--(-8.5,.85);
\filldraw[fill opacity=0.8,fill=gray!20,draw=none](-8.534,.843)--(-8.491,.841)--(-8.479,.83)--cycle;
\filldraw[fill opacity=0.8,fill=gray!20,draw=none](-8.479,.83)--(-7.866,.574)--(-7.874,.567)--(-8.504,.83)--cycle;
\draw(-8.479,.83)--(-7.866,.574);
\draw(-7.874,.567)--(-8.504,.83);
\filldraw[fill opacity=0.8,fill=gray!20,draw=none](-8.479,.83)--(-8.504,.83)--(-8.534,.843)--cycle;
\draw(-8.504,.83)--(-8.534,.843);
\filldraw[fill opacity=0.8,fill=gray!20,draw=none](-8.837,.795)--(-8.829,.834)--(-8.816,.85)--(-8.786,.847)--(-8.799,.792)--cycle;
\draw(-8.816,.85)--(-8.786,.847)--(-8.799,.792)--(-8.837,.795);
\filldraw[fill opacity=0.8,fill=gray!20,draw=none](-8.786,.849)--(-8.787,.874)--(-8.645,.858)--(-8.772,.83)--cycle;
\draw(-8.645,.858)--(-8.772,.83);
\filldraw[fill opacity=0.8,fill=gray!20,draw=none](-9.041,.784)--(-9.064,.798)--(-9.072,.809)--(-9.08,.834)--(-8.998,.85)--(-8.988,.794)--cycle;
\draw(-9.072,.809)--(-9.08,.834)--(-8.998,.85)--(-8.988,.794)--(-9.041,.784);
\filldraw[fill opacity=0.8,fill=gray!20,draw=none](-9.194,.848)--(-9.218,.858)--(-9.173,.853)--(-9.149,.843)--cycle;
\draw(-9.194,.848)--(-9.218,.858);
\draw(-9.173,.853)--(-9.149,.843);
\filldraw[fill opacity=0.8,fill=gray!20,draw=none](-9.173,.856)--(-9.16,.887)--(-9.24,.883)--(-9.24,.873)--cycle;
\draw(-9.173,.856)--(-9.16,.887)--(-9.24,.883)--(-9.24,.873);
\filldraw[fill opacity=0.8,fill=gray!20,draw=none](-9.173,.856)--(-9.16,.887)--(-9.24,.883)--(-9.24,.873)--cycle;
\draw(-9.173,.856)--(-9.16,.887)--(-9.24,.883)--(-9.24,.873);
\filldraw[fill opacity=0.8,fill=gray!20](-9.16,.887)--(-9.152,.926)--(-9.241,.922)--(-9.24,.883)--cycle;
\filldraw[fill opacity=0.8,fill=gray!20](-9.16,.887)--(-9.152,.926)--(-9.241,.922)--(-9.24,.883)--cycle;
\filldraw[fill opacity=0.8,fill=gray!20,draw=none](-9.24,.873)--(-9.24,.883)--(-9.251,.884)--cycle;
\draw(-9.24,.873)--(-9.24,.883)--(-9.251,.884);
\filldraw[fill opacity=0.8,fill=gray!20,draw=none](-9.24,.873)--(-9.24,.883)--(-9.251,.884)--cycle;
\draw(-9.24,.873)--(-9.24,.883)--(-9.251,.884);
\filldraw[fill opacity=0.8,fill=gray!20,draw=none](-9.251,.884)--(-9.24,.883)--(-9.241,.922)--(-9.273,.925)--cycle;
\draw(-9.251,.884)--(-9.24,.883)--(-9.241,.922)--(-9.273,.925);
\filldraw[fill opacity=0.8,fill=gray!20,draw=none](-9.251,.884)--(-9.24,.883)--(-9.241,.922)--(-9.273,.925)--cycle;
\draw(-9.251,.884)--(-9.24,.883)--(-9.241,.922)--(-9.273,.925);
\filldraw[fill opacity=0.8,fill=gray!20,draw=none](-9.273,.925)--(-9.241,.922)--(-9.242,.967)--(-9.281,.97)--cycle;
\draw(-9.273,.925)--(-9.241,.922)--(-9.242,.967)--(-9.281,.97);
\filldraw[fill opacity=0.8,fill=gray!20,draw=none](-9.273,.925)--(-9.241,.922)--(-9.242,.967)--(-9.281,.97)--cycle;
\draw(-9.273,.925)--(-9.241,.922)--(-9.242,.967)--(-9.281,.97);
\filldraw[fill opacity=0.8,fill=gray!20,draw=none](-9.26,.994)--(-9.267,.959)--(-9.258,.91)--(-9.232,.871)--(-9.194,.848)--(-9.149,.843)--(-9.104,.858)--(-9.047,.905)--cycle;
\filldraw[fill opacity=0.8,fill=gray!20,draw=none](-9.045,.897)--(-9.055,.908)--(-9.044,.927)--cycle;
\filldraw[fill opacity=0.8,fill=gray!20,draw=none](-9.019,.869)--(-9.065,.89)--(-9.055,.908)--cycle;
\draw(-9.019,.869)--(-9.065,.89);
\filldraw[fill opacity=0.8,fill=gray!20](-9.08,.834)--(-9.087,.891)--(-9.001,.907)--(-8.998,.85)--cycle;
\filldraw[fill opacity=0.8,fill=gray!20,draw=none](-8.786,.849)--(-8.786,.85)--(-8.794,.875)--(-8.787,.874)--cycle;
\filldraw[fill opacity=0.8,fill=gray!20,draw=none](-8.786,.85)--(-8.804,.873)--(-8.796,.875)--(-8.794,.875)--cycle;
\draw(-8.804,.873)--(-8.796,.875);
\filldraw[fill opacity=0.8,fill=gray!20,draw=none](-8.801,.849)--(-8.816,.85)--(-8.804,.876)--cycle;
\draw(-8.801,.849)--(-8.816,.85);
\filldraw[fill opacity=0.8,fill=gray!20,draw=none](-8.845,.879)--(-8.796,.875)--(-8.847,.864)--cycle;
\draw(-8.796,.875)--(-8.847,.864);
\filldraw[fill opacity=0.8,fill=gray!20](-9.083,1.032)--(-9.098,1.077)--(-9.16,1.065)--(-9.152,1.019)--cycle;
\filldraw[fill opacity=0.8,fill=gray!20](-9.083,1.032)--(-9.098,1.077)--(-9.16,1.065)--(-9.152,1.019)--cycle;
\filldraw[fill opacity=0.8,fill=gray!20,draw=none](-9.061,.959)--(-9.025,.944)--(-9.024,.949)--(-9.03,.982)--(-9.119,1.064)--cycle;
\draw(-9.025,.944)--(-9.024,.949);
\filldraw[fill opacity=0.8,fill=gray!20,draw=none](-9.013,.922)--(-9.04,.934)--(-9.031,.982)--cycle;
\draw(-9.013,.922)--(-9.04,.934);
\filldraw[fill opacity=0.8,fill=gray!20](-9.087,.891)--(-9.08,.945)--(-8.998,.961)--(-9.001,.907)--cycle;
\filldraw[fill opacity=0.8,fill=gray!20,draw=none](-8.842,.759)--(-8.842,.785)--(-8.837,.795)--(-8.799,.792)--(-8.809,.765)--cycle;
\draw(-8.837,.795)--(-8.799,.792)--(-8.809,.765);
\filldraw[fill opacity=0.8,fill=gray!20,draw=none](-8.801,.849)--(-8.804,.876)--(-8.803,.878)--(-8.783,.891)--(-8.786,.847)--cycle;
\draw(-8.783,.891)--(-8.786,.847)--(-8.801,.849);
\filldraw[fill opacity=0.8,fill=gray!20,draw=none](-8.778,.829)--(-8.645,.858)--(-8.534,.843)--(-8.733,.798)--cycle;
\draw(-8.778,.829)--(-8.645,.858);
\draw(-8.534,.843)--(-8.733,.798);
\filldraw[fill opacity=0.8,fill=gray!20,draw=none](-8.729,.799)--(-8.534,.843)--(-8.479,.83)--(-8.666,.788)--cycle;
\draw(-8.729,.799)--(-8.534,.843);
\draw(-8.479,.83)--(-8.666,.788);
\filldraw[fill opacity=0.8,fill=gray!20,draw=none](-8.784,.827)--(-8.789,.836)--(-8.786,.847)--(-8.785,.847)--cycle;
\draw(-8.789,.836)--(-8.786,.847)--(-8.785,.847);
\filldraw[fill opacity=0.8,fill=gray!20,draw=none](-8.784,.827)--(-8.782,.788)--(-8.799,.792)--(-8.789,.836)--cycle;
\draw(-8.782,.788)--(-8.799,.792)--(-8.789,.836);
\filldraw[fill opacity=0.8,fill=gray!20,draw=none](-8.845,.849)--(-8.841,.865)--(-8.804,.873)--(-8.786,.849)--(-8.784,.827)--(-8.842,.814)--cycle;
\draw(-8.841,.865)--(-8.804,.873);
\draw(-8.784,.827)--(-8.842,.814);
\filldraw[fill opacity=0.8,fill=gray!20,draw=none](-8.786,.849)--(-8.772,.83)--(-8.78,.828)--cycle;
\draw(-8.772,.83)--(-8.78,.828);
\filldraw[fill opacity=0.8,fill=gray!20,draw=none](-8.786,.847)--(-8.783,.891)--(-8.775,.903)--(-8.707,.886)--(-8.714,.83)--cycle;
\draw(-8.775,.903)--(-8.707,.886)--(-8.714,.83)--(-8.786,.847)--(-8.783,.891);
\filldraw[fill opacity=0.8,fill=gray!20,draw=none](-8.786,.849)--(-8.78,.828)--(-8.784,.827)--cycle;
\draw(-8.78,.828)--(-8.784,.827);
\filldraw[fill opacity=0.8,fill=gray!20,draw=none](-8.784,.827)--(-8.778,.829)--(-8.733,.798)--(-8.765,.791)--cycle;
\draw(-8.784,.827)--(-8.778,.829);
\draw(-8.733,.798)--(-8.765,.791);
\filldraw[fill opacity=0.8,fill=gray!20,draw=none](-8.737,.797)--(-8.729,.799)--(-8.666,.788)--(-8.735,.773)--cycle;
\draw(-8.737,.797)--(-8.729,.799);
\draw(-8.666,.788)--(-8.735,.773);
\filldraw[fill opacity=0.8,fill=gray!20,draw=none](-8.763,.791)--(-8.737,.797)--(-8.735,.773)--cycle;
\draw(-8.763,.791)--(-8.737,.797);
\filldraw[fill opacity=0.8,fill=gray!20,draw=none](-8.809,.765)--(-8.799,.792)--(-8.733,.776)--(-8.735,.773)--cycle;
\draw(-8.809,.765)--(-8.799,.792)--(-8.733,.776)--(-8.735,.773);
\filldraw[fill opacity=0.8,fill=gray!20,draw=none](-8.842,.743)--(-8.842,.759)--(-8.809,.765)--(-8.818,.741)--cycle;
\draw(-8.809,.765)--(-8.818,.741)--(-8.842,.743);
\filldraw[fill opacity=0.8,fill=gray!20,draw=none](-8.78,.732)--(-8.818,.741)--(-8.809,.765)--(-8.735,.773)--(-8.753,.746)--cycle;
\draw(-8.78,.732)--(-8.818,.741)--(-8.809,.765);
\draw(-8.735,.773)--(-8.753,.746);
\filldraw[fill opacity=0.8,fill=gray!20,draw=none](-8.743,.778)--(-8.782,.788)--(-8.782,.8)--cycle;
\draw(-8.743,.778)--(-8.782,.788);
\filldraw[fill opacity=0.8,fill=gray!20,draw=none](-8.783,.802)--(-8.785,.847)--(-8.714,.83)--(-8.727,.793)--cycle;
\draw(-8.785,.847)--(-8.714,.83)--(-8.727,.793);
\filldraw[fill opacity=0.8,fill=gray!20,draw=none](-8.87,.756)--(-8.836,.775)--(-8.763,.791)--(-8.735,.773)--(-8.867,.743)--cycle;
\draw(-8.836,.775)--(-8.763,.791);
\draw(-8.735,.773)--(-8.867,.743);
\filldraw[fill opacity=0.8,fill=gray!20,draw=none](-8.743,.778)--(-8.782,.8)--(-8.783,.802)--(-8.727,.793)--(-8.733,.776)--cycle;
\draw(-8.727,.793)--(-8.733,.776)--(-8.743,.778);
\filldraw[fill opacity=0.8,fill=gray!20,draw=none](-8.847,.813)--(-8.784,.827)--(-8.77,.8)--(-8.836,.775)--(-8.873,.766)--cycle;
\draw(-8.847,.813)--(-8.784,.827);
\draw(-8.836,.775)--(-8.873,.766);
\filldraw[fill opacity=0.8,fill=gray!20,draw=none](-8.845,.849)--(-8.842,.814)--(-8.856,.811)--cycle;
\draw(-8.842,.814)--(-8.856,.811);
\filldraw[fill opacity=0.8,fill=gray!20,draw=none](-8.856,.844)--(-8.855,.798)--(-8.875,.806)--(-8.858,.845)--cycle;
\draw(-8.855,.798)--(-8.875,.806);
\filldraw[fill opacity=0.8,fill=gray!20,draw=none](-8.77,.8)--(-8.765,.791)--(-8.836,.775)--cycle;
\draw(-8.765,.791)--(-8.836,.775);
\filldraw[fill opacity=0.8,fill=gray!20,draw=none](-8.87,.756)--(-8.873,.766)--(-8.836,.775)--cycle;
\draw(-8.873,.766)--(-8.836,.775);
\filldraw[fill opacity=0.8,fill=gray!20,draw=none](-8.858,.75)--(-8.901,.769)--(-8.88,.801)--(-8.875,.806)--(-8.82,.782)--cycle;
\draw(-8.875,.806)--(-8.82,.782)--(-8.858,.75)--(-8.901,.769);
\filldraw[fill opacity=0.8,fill=gray!20,draw=none](-8.856,.844)--(-8.858,.845)--(-8.857,.846)--(-8.856,.847)--cycle;
\filldraw[fill opacity=0.8,fill=gray!20,draw=none](-8.896,.742)--(-8.921,.753)--(-8.908,.766)--(-8.901,.769)--(-8.87,.756)--cycle;
\draw(-8.901,.769)--(-8.87,.756);
\filldraw[fill opacity=0.8,fill=gray!20,draw=none](-8.936,.793)--(-8.847,.813)--(-8.873,.766)--(-8.914,.757)--cycle;
\draw(-8.873,.766)--(-8.914,.757)--(-8.936,.793)--(-8.847,.813);
\filldraw[fill opacity=0.8,fill=gray!20,draw=none](-8.845,.849)--(-8.853,.852)--(-8.847,.872)--cycle;
\draw(-8.845,.849)--(-8.853,.852);
\filldraw[fill opacity=0.8,fill=gray!20,draw=none](-8.962,.89)--(-8.861,.881)--(-8.868,.859)--(-8.953,.84)--cycle;
\draw(-8.868,.859)--(-8.953,.84)--(-8.962,.89);
\filldraw[fill opacity=0.8,fill=gray!20,draw=none](-8.799,.819)--(-8.82,.782)--(-8.855,.798)--(-8.856,.844)--cycle;
\draw(-8.799,.819)--(-8.82,.782)--(-8.855,.798);
\filldraw[fill opacity=0.8,fill=gray!20,draw=none](-8.953,.84)--(-8.883,.855)--(-8.848,.84)--(-8.856,.811)--(-8.936,.793)--cycle;
\draw(-8.856,.811)--(-8.936,.793)--(-8.953,.84)--(-8.883,.855);
\filldraw[fill opacity=0.8,fill=gray!20,draw=none](-8.802,.82)--(-8.856,.844)--(-8.856,.847)--(-8.853,.852)--(-8.794,.827)--cycle;
\draw(-8.853,.852)--(-8.794,.827);
\filldraw[fill opacity=0.8,fill=gray!20,draw=none](-8.868,.849)--(-8.868,.859)--(-8.841,.865)--(-8.848,.84)--cycle;
\draw(-8.868,.859)--(-8.841,.865);
\filldraw[fill opacity=0.8,fill=gray!20,draw=none](-8.811,.873)--(-8.81,.834)--(-8.845,.849)--(-8.847,.872)--(-8.841,.894)--(-8.838,.9)--(-8.817,.891)--cycle;
\draw(-8.81,.834)--(-8.845,.849);
\draw(-8.838,.9)--(-8.817,.891);
\filldraw[fill opacity=0.8,fill=gray!20,draw=none](-8.861,.881)--(-8.845,.879)--(-8.847,.864)--(-8.868,.859)--cycle;
\draw(-8.847,.864)--(-8.868,.859);
\filldraw[fill opacity=0.8,fill=gray!20,draw=none](-8.868,.849)--(-8.883,.855)--(-8.868,.859)--cycle;
\draw(-8.883,.855)--(-8.868,.859);
\filldraw[fill opacity=0.8,fill=gray!20,draw=none](-8.73,.772)--(-8.734,.773)--(-8.479,.83)--(-8.489,.823)--(-8.718,.772)--cycle;
\draw(-8.734,.773)--(-8.479,.83);
\draw(-8.489,.823)--(-8.718,.772);
\filldraw[fill opacity=0.8,fill=gray!20,draw=none](-8.73,.772)--(-8.718,.772)--(-8.72,.771)--cycle;
\draw(-8.718,.772)--(-8.72,.771);
\filldraw[fill opacity=0.8,fill=gray!20,draw=none](-8.701,.791)--(-8.725,.767)--(-8.733,.776)--(-8.714,.83)--(-8.692,.806)--cycle;
\draw(-8.725,.767)--(-8.733,.776)--(-8.714,.83)--(-8.692,.806);
\filldraw[fill opacity=0.8,fill=gray!20,draw=none](-8.753,.746)--(-8.735,.773)--(-8.73,.772)--(-8.725,.767)--cycle;
\draw(-8.753,.746)--(-8.735,.773);
\draw(-8.73,.772)--(-8.725,.767);
\filldraw[fill opacity=0.8,fill=gray!20,draw=none](-8.891,.738)--(-8.735,.773)--(-8.73,.772)--(-8.72,.771)--(-8.869,.738)--cycle;
\draw(-8.72,.771)--(-8.869,.738)--(-8.891,.738)--(-8.735,.773);
\filldraw[fill opacity=0.8,fill=gray!20,draw=none](-8.896,.742)--(-8.914,.757)--(-8.873,.766)--(-8.87,.756)--cycle;
\draw(-8.896,.742)--(-8.914,.757)--(-8.873,.766);
\filldraw[fill opacity=0.8,fill=gray!20,draw=none](-8.896,.742)--(-8.87,.756)--(-8.867,.743)--(-8.891,.738)--cycle;
\draw(-8.867,.743)--(-8.891,.738)--(-8.896,.742);
\filldraw[fill opacity=0.8,fill=gray!20,draw=none](-8.891,.74)--(-8.896,.742)--(-8.87,.756)--(-8.858,.75)--cycle;
\draw(-8.87,.756)--(-8.858,.75)--(-8.891,.74);
\filldraw[fill opacity=0.8,fill=gray!20,draw=none](-8.891,.74)--(-8.858,.75)--(-8.82,.782)--(-8.799,.819)--(-8.903,.864)--cycle;
\draw(-8.891,.74)--(-8.858,.75)--(-8.82,.782)--(-8.799,.819);
\filldraw[fill opacity=0.8,fill=gray!20,draw=none](-8.844,.839)--(-8.962,.89)--(-8.953,.84)--(-8.936,.793)--(-8.914,.757)--(-8.896,.742)--(-8.887,.738)--(-8.869,.738)--(-8.853,.757)--(-8.844,.792)--cycle;
\draw(-8.962,.89)--(-8.953,.84)--(-8.936,.793)--(-8.914,.757)--(-8.896,.742);
\draw(-8.887,.738)--(-8.869,.738)--(-8.853,.757)--(-8.844,.792)--(-8.844,.839);
\filldraw[fill opacity=0.8,fill=gray!20,draw=none](-8.896,.742)--(-8.891,.738)--(-8.887,.738)--cycle;
\draw(-8.896,.742)--(-8.891,.738)--(-8.887,.738);
\filldraw[fill opacity=0.8,fill=gray!20,draw=none](-8.891,.74)--(-8.903,.736)--(-8.928,.746)--(-8.921,.753)--cycle;
\draw(-8.891,.74)--(-8.903,.736)--(-8.928,.746);
\filldraw[fill opacity=0.8,fill=gray!20,draw=none](-8.954,.744)--(-8.948,.741)--(-8.903,.736)--(-8.891,.74)--(-8.903,.864)--cycle;
\draw(-8.954,.744)--(-8.948,.741)--(-8.903,.736)--(-8.891,.74);
\filldraw[fill opacity=0.8,fill=gray!20,draw=none](-8.954,.744)--(-8.956,.745)--(-8.948,.741)--cycle;
\draw(-8.948,.741)--(-8.954,.744);
\filldraw[fill opacity=0.8,fill=gray!20](-8.954,.743)--(-7.842,.279)--(-7.815,.278)--(-8.928,.742)--cycle;
\filldraw[fill opacity=0.8,fill=gray!20,draw=none](-8.4,3.23)--(-8.42,3.184)--(-8.423,3.182)--(-8.418,3.2)--(-8.402,3.234)--cycle;
\draw(-8.4,3.23)--(-8.42,3.184);
\draw(-8.418,3.2)--(-8.402,3.234);
\filldraw[fill opacity=0.8,fill=gray!20](-7.614,.351)--(-7.608,.398)--(-7.638,.378)--(-7.643,.333)--cycle;
\filldraw[fill opacity=0.8,fill=gray!20](-7.948,.429)--(-7.932,.474)--(-7.949,.491)--(-7.967,.449)--cycle;
\filldraw[fill opacity=0.8,fill=gray!20](-7.793,.217)--(-7.796,.225)--(-7.841,.229)--(-7.816,.219)--cycle;
\filldraw[fill opacity=0.8,fill=gray!20](-7.764,.581)--(-7.79,.579)--(-7.79,.579)--(-7.748,.577)--cycle;
\filldraw[fill opacity=0.8,fill=gray!20,draw=none](-8.423,3.182)--(-8.426,3.172)--(-8.427,3.17)--(-8.433,3.165)--(-8.427,3.179)--cycle;
\draw(-8.433,3.165)--(-8.427,3.179);
\filldraw[fill opacity=0.8,fill=gray!20,draw=none](-7.835,4.594)--(-7.835,4.596)--(-7.835,4.596)--cycle;
\draw(-7.835,4.596)--(-7.835,4.596);
\filldraw[fill opacity=0.8,fill=gray!20,draw=none](-7.833,4.597)--(-7.835,4.596)--(-7.835,4.594)--(-7.833,4.592)--cycle;
\draw(-7.833,4.597)--(-7.835,4.596);
\filldraw[fill opacity=0.8,fill=gray!20,draw=none](-7.855,4.463)--(-7.815,4.554)--(-7.813,4.553)--(-7.851,4.462)--cycle;
\draw(-7.855,4.463)--(-7.815,4.554);
\filldraw[fill opacity=0.8,fill=gray!20,draw=none](-7.749,.514)--(-7.763,.528)--(-7.79,.529)--(-7.793,.508)--(-7.754,.507)--cycle;
\draw(-7.763,.528)--(-7.79,.529)--(-7.793,.508)--(-7.754,.507);
\filldraw[fill opacity=0.8,fill=gray!20,draw=none](-7.78,.527)--(-7.721,.502)--(-7.739,.521)--(-7.757,.529)--cycle;
\draw(-7.78,.527)--(-7.721,.502)--(-7.739,.521)--(-7.757,.529);
\filldraw[fill opacity=0.8,fill=gray!20,draw=none](-8.524,2.895)--(-8.539,2.932)--(-8.55,2.93)--(-8.544,2.916)--cycle;
\draw(-8.55,2.93)--(-8.544,2.916)--(-8.524,2.895)--(-8.539,2.932);
\filldraw[fill opacity=0.8,fill=gray!20,draw=none](-8.174,3.746)--(-8.174,3.76)--(-8.177,3.768)--(-8.199,3.792)--(-8.185,3.759)--cycle;
\draw(-8.174,3.76)--(-8.177,3.768)--(-8.199,3.792)--(-8.185,3.759);
\filldraw[fill opacity=0.8,fill=gray!20](-7.898,1.004)--(-7.923,1.044)--(-7.861,1.056)--(-7.848,1.013)--cycle;
\filldraw[fill opacity=0.8,fill=gray!20,draw=none](-7.813,4.549)--(-7.812,4.552)--(-7.812,4.552)--cycle;
\filldraw[fill opacity=0.8,fill=gray!20,draw=none](-7.74,4.755)--(-7.737,4.756)--(-7.753,4.727)--cycle;
\filldraw[fill opacity=0.8,fill=gray!20,draw=none](-8.517,2.971)--(-8.501,2.953)--(-8.509,2.958)--cycle;
\filldraw[fill opacity=0.8,fill=gray!20,draw=none](-7.831,4.613)--(-7.832,4.616)--(-7.831,4.616)--cycle;
\draw(-7.831,4.613)--(-7.832,4.616)--(-7.831,4.616);
\filldraw[fill opacity=0.8,fill=gray!20](-8.215,3.187)--(-8.256,3.217)--(-8.245,3.205)--(-8.199,3.17)--cycle;
\filldraw[fill opacity=0.8,fill=gray!20](-7.769,.218)--(-7.749,.227)--(-7.796,.225)--(-7.793,.217)--cycle;
\filldraw[fill opacity=0.8,fill=gray!20,draw=none](-7.61,4.289)--(-7.592,4.335)--(-7.592,4.347)--(-7.615,4.29)--cycle;
\draw(-7.592,4.347)--(-7.615,4.29)--(-7.61,4.289)--(-7.592,4.335);
\filldraw[fill opacity=0.8,fill=gray!20,draw=none](-8.569,3.066)--(-8.572,3.088)--(-8.576,3.084)--(-8.584,3.059)--(-8.586,3.031)--cycle;
\draw(-8.576,3.084)--(-8.584,3.059)--(-8.586,3.031);
\filldraw[fill opacity=0.8,fill=gray!20,draw=none](-7.815,4.579)--(-7.815,4.582)--(-7.803,4.611)--cycle;
\filldraw[fill opacity=0.8,fill=gray!20,draw=none](-7.815,4.579)--(-7.803,4.611)--(-7.824,4.601)--cycle;
\draw(-7.803,4.611)--(-7.824,4.601);
\filldraw[fill opacity=0.8,fill=gray!20,draw=none](-7.797,4.557)--(-7.8,4.558)--(-7.8,4.558)--cycle;
\draw(-7.8,4.558)--(-7.8,4.558);
\filldraw[fill opacity=0.8,fill=gray!20,draw=none](-7.8,4.558)--(-7.8,4.558)--(-7.793,4.581)--(-7.787,4.597)--cycle;
\draw(-7.8,4.558)--(-7.8,4.558);
\draw(-7.793,4.581)--(-7.787,4.597);
\filldraw[fill opacity=0.8,fill=gray!20,draw=none](-7.8,4.558)--(-7.802,4.558)--(-7.793,4.581)--cycle;
\draw(-7.8,4.558)--(-7.802,4.558)--(-7.793,4.581);
\filldraw[fill opacity=0.8,fill=gray!20,draw=none](-7.802,4.558)--(-7.8,4.558)--(-7.797,4.557)--(-7.764,4.544)--cycle;
\draw(-7.802,4.558)--(-7.8,4.558);
\filldraw[fill opacity=0.8,fill=gray!20,draw=none](-7.78,4.615)--(-7.802,4.558)--(-7.801,4.558)--(-7.779,4.601)--cycle;
\draw(-7.78,4.615)--(-7.802,4.558)--(-7.801,4.558);
\filldraw[fill opacity=0.8,fill=gray!20,draw=none](-7.765,4.629)--(-7.781,4.622)--(-7.789,4.612)--(-7.804,4.573)--cycle;
\draw(-7.765,4.629)--(-7.781,4.622);
\filldraw[fill opacity=0.8,fill=gray!20,draw=none](-7.81,4.573)--(-7.808,4.58)--(-7.81,4.572)--cycle;
\filldraw[fill opacity=0.8,fill=gray!20,draw=none](-7.815,4.582)--(-7.814,4.587)--(-7.785,4.655)--cycle;
\draw(-7.814,4.587)--(-7.785,4.655);
\filldraw[fill opacity=0.8,fill=gray!20,draw=none](-7.808,4.58)--(-7.81,4.573)--(-7.81,4.576)--(-7.805,4.59)--cycle;
\draw(-7.81,4.576)--(-7.805,4.59);
\filldraw[fill opacity=0.8,fill=gray!20,draw=none](-7.808,4.58)--(-7.805,4.59)--(-7.792,4.622)--cycle;
\draw(-7.805,4.59)--(-7.792,4.622);
\filldraw[fill opacity=0.8,fill=gray!20,draw=none](-7.807,4.57)--(-7.799,4.59)--(-7.801,4.584)--(-7.807,4.568)--cycle;
\draw(-7.807,4.57)--(-7.799,4.59);
\filldraw[fill opacity=0.8,fill=gray!20,draw=none](-7.761,4.54)--(-7.765,4.542)--(-7.787,4.551)--(-7.802,4.558)--(-7.777,4.549)--cycle;
\draw(-7.765,4.542)--(-7.787,4.551)--(-7.802,4.558);
\filldraw[fill opacity=0.8,fill=gray!20,draw=none](-7.789,4.612)--(-7.786,4.619)--(-7.795,4.615)--(-7.801,4.597)--cycle;
\draw(-7.786,4.619)--(-7.795,4.615);
\filldraw[fill opacity=0.8,fill=gray!20,draw=none](-7.807,4.572)--(-7.76,4.682)--(-7.759,4.684)--(-7.792,4.622)--(-7.81,4.576)--cycle;
\draw(-7.792,4.622)--(-7.81,4.576);
\filldraw[fill opacity=0.8,fill=gray!20,draw=none](-7.748,4.637)--(-7.762,4.63)--(-7.756,4.615)--cycle;
\draw(-7.748,4.637)--(-7.762,4.63);
\filldraw[fill opacity=0.8,fill=gray!20,draw=none](-7.765,4.629)--(-7.801,4.558)--(-7.787,4.551)--(-7.749,4.652)--cycle;
\draw(-7.801,4.558)--(-7.787,4.551)--(-7.749,4.652);
\filldraw[fill opacity=0.8,fill=gray!20,draw=none](-7.78,4.608)--(-7.779,4.601)--(-7.765,4.629)--cycle;
\filldraw[fill opacity=0.8,fill=gray!20,draw=none](-7.765,4.629)--(-7.749,4.652)--(-7.737,4.685)--cycle;
\draw(-7.749,4.652)--(-7.737,4.685);
\filldraw[fill opacity=0.8,fill=gray!20,draw=none](-7.796,4.557)--(-7.735,4.693)--(-7.792,4.604)--(-7.807,4.57)--cycle;
\draw(-7.796,4.557)--(-7.735,4.693);
\draw(-7.792,4.604)--(-7.807,4.57);
\filldraw[fill opacity=0.8,fill=gray!20,draw=none](-7.807,4.599)--(-7.796,4.614)--(-7.803,4.611)--cycle;
\draw(-7.796,4.614)--(-7.803,4.611);
\filldraw[fill opacity=0.8,fill=gray!20,draw=none](-7.799,4.59)--(-7.792,4.604)--(-7.801,4.584)--cycle;
\draw(-7.799,4.59)--(-7.792,4.604);
\filldraw[fill opacity=0.8,fill=gray!20,draw=none](-7.781,4.622)--(-7.789,4.612)--(-7.792,4.604)--cycle;
\draw(-7.789,4.612)--(-7.792,4.604);
\filldraw[fill opacity=0.8,fill=gray!20,draw=none](-7.781,4.622)--(-7.786,4.619)--(-7.789,4.612)--cycle;
\draw(-7.781,4.622)--(-7.786,4.619);
\filldraw[fill opacity=0.8,fill=gray!20,draw=none](-7.797,4.561)--(-7.767,4.638)--(-7.76,4.682)--(-7.807,4.572)--cycle;
\draw(-7.797,4.561)--(-7.767,4.638);
\filldraw[fill opacity=0.8,fill=gray!20,draw=none](-7.769,4.679)--(-7.785,4.655)--(-7.814,4.587)--(-7.802,4.572)--(-7.793,4.591)--cycle;
\draw(-7.785,4.655)--(-7.814,4.587);
\draw(-7.802,4.572)--(-7.793,4.591);
\filldraw[fill opacity=0.8,fill=gray!20,draw=none](-7.8,4.553)--(-7.79,4.543)--(-7.81,4.592)--(-7.832,4.616)--(-7.831,4.613)--cycle;
\draw(-7.8,4.553)--(-7.79,4.543)--(-7.81,4.592)--(-7.832,4.616)--(-7.831,4.613);
\filldraw[fill opacity=0.8,fill=gray!20,draw=none](-7.808,4.563)--(-7.812,4.552)--(-7.812,4.552)--(-7.807,4.564)--cycle;
\filldraw[fill opacity=0.8,fill=gray!20,draw=none](-6.241,.509)--(-6.238,.514)--(-6.231,.514)--(-6.223,.498)--(-6.223,.496)--cycle;
\draw(-6.223,.498)--(-6.223,.496);
\filldraw[fill opacity=0.8,fill=gray!20,draw=none](-6.231,.514)--(-6.223,.514)--(-6.223,.498)--cycle;
\draw(-6.223,.514)--(-6.223,.498);
\filldraw[fill opacity=0.8,fill=gray!20,draw=none](-6.231,.502)--(-6.194,.513)--(-6.179,.513)--(-6.2,.494)--(-6.222,.495)--cycle;
\draw(-6.194,.513)--(-6.179,.513);
\draw(-6.2,.494)--(-6.222,.495);
\filldraw[fill opacity=0.8,fill=gray!20,draw=none](-7.823,4.606)--(-7.834,4.633)--(-7.832,4.616)--cycle;
\draw(-7.834,4.633)--(-7.832,4.616)--(-7.823,4.606);
\filldraw[fill opacity=0.8,fill=gray!20,draw=none](-7.592,4.341)--(-7.592,4.335)--(-7.591,4.338)--(-7.585,4.364)--cycle;
\draw(-7.592,4.335)--(-7.591,4.338);
\filldraw[fill opacity=0.8,fill=gray!20,draw=none](-7.81,4.569)--(-7.81,4.573)--(-7.815,4.582)--(-7.815,4.579)--cycle;
\filldraw[fill opacity=0.8,fill=gray!20,draw=none](-7.817,4.555)--(-7.81,4.573)--(-7.81,4.572)--(-7.816,4.554)--cycle;
\filldraw[fill opacity=0.8,fill=gray!20,draw=none](-7.644,4.302)--(-7.634,4.315)--(-7.641,4.301)--cycle;
\draw(-7.634,4.315)--(-7.641,4.301);
\filldraw[fill opacity=0.8,fill=gray!20,draw=none](-7.503,4.607)--(-7.501,4.587)--(-7.511,4.588)--cycle;
\draw(-7.503,4.607)--(-7.501,4.587);
\filldraw[fill opacity=0.8,fill=gray!20](-6.842,.204)--(-6.797,.238)--(-6.822,.221)--(-6.86,.192)--cycle;
\filldraw[fill opacity=0.8,fill=gray!20](-7.122,.265)--(-7.141,.315)--(-7.163,.338)--(-7.141,.286)--cycle;
\filldraw[fill opacity=0.8,fill=gray!20,draw=none](-7.808,4.561)--(-7.808,4.562)--(-7.808,4.563)--cycle;
\filldraw[fill opacity=0.8,fill=gray!20,draw=none](-7.808,4.562)--(-7.811,4.566)--(-7.813,4.56)--(-7.81,4.556)--cycle;
\filldraw[fill opacity=0.8,fill=gray!20,draw=none](-7.808,4.562)--(-7.808,4.563)--(-7.81,4.569)--(-7.811,4.566)--cycle;
\filldraw[fill opacity=0.8,fill=gray!20,draw=none](-7.813,4.553)--(-7.807,4.568)--(-7.807,4.564)--(-7.812,4.552)--cycle;
\filldraw[fill opacity=0.8,fill=gray!20,draw=none](-7.81,4.569)--(-7.813,4.56)--(-7.814,4.558)--(-7.81,4.572)--cycle;
\filldraw[fill opacity=0.8,fill=gray!20,draw=none](-8.167,3.129)--(-8.238,2.969)--(-8.235,2.967)--(-8.154,3.147)--cycle;
\draw(-8.167,3.129)--(-8.238,2.969)--(-8.235,2.967)--(-8.154,3.147);
\filldraw[fill opacity=0.8,fill=gray!20,draw=none](-8.149,2.992)--(-8.154,2.972)--(-8.154,2.955)--(-8.143,2.962)--(-8.135,3.018)--(-8.139,3.015)--cycle;
\draw(-8.154,2.955)--(-8.143,2.962)--(-8.135,3.018)--(-8.139,3.015);
\filldraw[fill opacity=0.8,fill=gray!20,draw=none](-8.14,2.981)--(-8.135,3.001)--(-8.126,2.979)--(-8.136,2.949)--cycle;
\draw(-8.126,2.979)--(-8.136,2.949);
\filldraw[fill opacity=0.8,fill=gray!20,draw=none](-8.181,3.049)--(-8.161,2.997)--(-8.149,2.992)--(-8.139,3.015)--(-8.137,3.03)--(-8.139,3.039)--(-8.182,3.058)--cycle;
\draw(-8.161,2.997)--(-8.149,2.992);
\draw(-8.139,3.039)--(-8.182,3.058);
\filldraw[fill opacity=0.8,fill=gray!20,draw=none](-8.189,3.077)--(-8.142,3.048)--(-8.152,3.08)--(-8.18,3.092)--cycle;
\draw(-8.152,3.08)--(-8.18,3.092);
\filldraw[fill opacity=0.8,fill=gray!20,draw=none](-8.142,3.048)--(-8.152,3.08)--(-8.141,3.064)--(-8.139,3.048)--cycle;
\draw(-8.141,3.064)--(-8.139,3.048);
\filldraw[fill opacity=0.8,fill=gray!20,draw=none](-8.156,3.01)--(-8.155,3.005)--(-8.135,3.018)--(-8.137,3.035)--cycle;
\draw(-8.155,3.005)--(-8.135,3.018)--(-8.137,3.035);
\filldraw[fill opacity=0.8,fill=gray!20,draw=none](-8.142,3.048)--(-8.139,3.039)--(-8.128,3.034)--(-8.131,3.042)--cycle;
\draw(-8.139,3.039)--(-8.128,3.034);
\filldraw[fill opacity=0.8,fill=gray!20,draw=none](-8.142,3.048)--(-8.139,3.048)--(-8.137,3.035)--cycle;
\draw(-8.139,3.048)--(-8.137,3.035);
\filldraw[fill opacity=0.8,fill=gray!20,draw=none](-8.193,3.072)--(-8.185,3.06)--(-8.139,3.039)--(-8.142,3.048)--(-8.189,3.077)--cycle;
\draw(-8.185,3.06)--(-8.139,3.039);
\filldraw[fill opacity=0.8,fill=gray!20,draw=none](-8.147,3.022)--(-8.137,3.035)--(-8.143,3.073)--(-8.164,3.059)--cycle;
\draw(-8.137,3.035)--(-8.143,3.073)--(-8.164,3.059);
\filldraw[fill opacity=0.8,fill=gray!20,draw=none](-8.149,2.992)--(-8.143,2.989)--(-8.139,3.015)--cycle;
\draw(-8.149,2.992)--(-8.143,2.989);
\filldraw[fill opacity=0.8,fill=gray!20,draw=none](-8.149,2.992)--(-8.139,3.015)--(-8.145,3.011)--cycle;
\draw(-8.139,3.015)--(-8.145,3.011);
\filldraw[fill opacity=0.8,fill=gray!20,draw=none](-8.17,2.94)--(-8.166,2.938)--(-8.159,2.951)--(-8.149,2.992)--(-8.165,2.999)--cycle;
\draw(-8.17,2.94)--(-8.166,2.938);
\draw(-8.149,2.992)--(-8.165,2.999);
\filldraw[fill opacity=0.8,fill=gray!20,draw=none](-8.154,2.981)--(-8.149,2.992)--(-8.145,3.011)--(-8.155,3.005)--cycle;
\draw(-8.145,3.011)--(-8.155,3.005);
\filldraw[fill opacity=0.8,fill=gray!20,draw=none](-8.158,3.023)--(-8.153,3.013)--(-8.147,3.022)--(-8.164,3.059)--cycle;
\filldraw[fill opacity=0.8,fill=gray!20,draw=none](-8.158,3.023)--(-8.156,3.01)--(-8.153,3.013)--cycle;
\filldraw[fill opacity=0.8,fill=gray!20,draw=none](-8.198,3.056)--(-8.148,3.034)--(-8.142,2.999)--(-8.207,3.027)--cycle;
\draw(-8.148,3.034)--(-8.142,2.999)--(-8.207,3.027);
\filldraw[fill opacity=0.8,fill=gray!20,draw=none](-8.429,3.168)--(-8.427,3.17)--(-8.434,3.16)--cycle;
\filldraw[fill opacity=0.8,fill=gray!20,draw=none](-7.469,4.514)--(-7.461,4.52)--(-7.457,4.526)--cycle;
\draw(-7.461,4.52)--(-7.457,4.526);
\filldraw[fill opacity=0.8,fill=gray!20](-7.841,.229)--(-7.862,.25)--(-7.906,.261)--(-7.872,.236)--cycle;
\filldraw[fill opacity=0.8,fill=gray!20,draw=none](-7.648,1.042)--(-7.651,1.087)--(-7.637,1.086)--cycle;
\draw(-7.651,1.087)--(-7.637,1.086);
\filldraw[fill opacity=0.8,fill=gray!20,draw=none](-7.629,1.086)--(-7.65,1.087)--(-7.65,1.09)--(-7.64,1.113)--cycle;
\draw(-7.629,1.086)--(-7.65,1.087);
\filldraw[fill opacity=0.8,fill=gray!20,draw=none](-7.64,1.113)--(-7.65,1.09)--(-7.65,1.138)--cycle;
\filldraw[fill opacity=0.8,fill=gray!20,draw=none](-7.64,1.113)--(-7.65,1.138)--(-7.63,1.137)--cycle;
\draw(-7.65,1.138)--(-7.63,1.137);
\filldraw[fill opacity=0.8,fill=gray!20](-6.985,1.065)--(-6.988,1.121)--(-6.877,1.126)--(-6.878,1.069)--cycle;
\filldraw[fill opacity=0.8,fill=gray!20](-6.975,1.009)--(-6.985,1.065)--(-6.878,1.069)--(-6.879,1.013)--cycle;
\filldraw[fill opacity=0.8,fill=gray!20,draw=none](-7.04,1.084)--(-7.048,1.063)--(-7.629,1.086)--(-7.64,1.113)--(-7.63,1.137)--(-7.05,1.114)--cycle;
\draw(-7.048,1.063)--(-7.629,1.086);
\draw(-7.63,1.137)--(-7.05,1.114);
\filldraw[fill opacity=0.8,fill=gray!20](-7.693,1.1)--(-7.69,1.148)--(-7.627,1.132)--(-7.632,1.085)--cycle;
\filldraw[fill opacity=0.8,fill=gray!20,draw=none](-7.65,1.087)--(-7.651,1.087)--(-7.65,1.09)--cycle;
\draw(-7.65,1.087)--(-7.651,1.087);
\filldraw[fill opacity=0.8,fill=gray!20,draw=none](-7.035,1.055)--(-7.052,1.015)--(-7.648,1.038)--(-7.648,1.042)--(-7.637,1.086)--(-7.034,1.062)--cycle;
\draw(-7.052,1.015)--(-7.648,1.038);
\draw(-7.637,1.086)--(-7.034,1.062);
\filldraw[fill opacity=0.8,fill=gray!20](-7.703,1.054)--(-7.693,1.1)--(-7.632,1.085)--(-7.649,1.04)--cycle;
\filldraw[fill opacity=0.8,fill=gray!20](-7.785,.98)--(-7.783,1.016)--(-7.719,1.012)--(-7.74,.976)--cycle;
\filldraw[fill opacity=0.8,fill=gray!20](-7.831,.977)--(-7.848,1.013)--(-7.783,1.016)--(-7.785,.98)--cycle;
\filldraw[fill opacity=0.8,fill=gray!20,draw=none](-7.802,4.555)--(-7.809,4.551)--(-7.795,4.545)--cycle;
\draw(-7.802,4.555)--(-7.809,4.551);
\filldraw[fill opacity=0.8,fill=gray!20,draw=none](-6.231,.514)--(-6.236,.526)--(-6.223,.535)--(-6.223,.514)--cycle;
\draw(-6.223,.535)--(-6.223,.514);
\filldraw[fill opacity=0.8,fill=gray!20,draw=none](-6.246,.513)--(-6.25,.517)--(-6.236,.526)--(-6.231,.514)--cycle;
\filldraw[fill opacity=0.8,fill=gray!20,draw=none](-6.241,.509)--(-6.193,.522)--(-6.223,.535)--cycle;
\draw(-6.193,.522)--(-6.223,.535);
\filldraw[fill opacity=0.8,fill=gray!20,draw=none](-6.241,.509)--(-6.231,.502)--(-6.252,.468)--(-6.252,.493)--cycle;
\draw(-6.252,.468)--(-6.252,.493);
\filldraw[fill opacity=0.8,fill=gray!20,draw=none](-6.241,.509)--(-6.226,.513)--(-6.194,.513)--(-6.231,.502)--cycle;
\draw(-6.226,.513)--(-6.194,.513);
\filldraw[fill opacity=0.8,fill=gray!20,draw=none](-6.183,.511)--(-6.174,.511)--(-6.177,.512)--(-6.194,.513)--(-6.226,.513)--cycle;
\draw(-6.183,.511)--(-6.174,.511);
\draw(-6.194,.513)--(-6.226,.513);
\filldraw[fill opacity=0.8,fill=gray!20,draw=none](-6.177,.512)--(-6.179,.513)--(-6.194,.513)--cycle;
\draw(-6.179,.513)--(-6.194,.513);
\filldraw[fill opacity=0.8,fill=gray!20,draw=none](-6.179,.501)--(-6.186,.519)--(-6.193,.522)--(-6.271,.502)--(-6.199,.47)--cycle;
\draw(-6.186,.519)--(-6.193,.522);
\draw(-6.271,.502)--(-6.199,.47);
\filldraw[fill opacity=0.8,fill=gray!20,draw=none](-6.241,.509)--(-6.246,.513)--(-6.238,.514)--cycle;
\filldraw[fill opacity=0.8,fill=gray!20,draw=none](-6.241,.509)--(-6.246,.513)--(-6.226,.513)--cycle;
\draw(-6.246,.513)--(-6.226,.513);
\filldraw[fill opacity=0.8,fill=gray!20,draw=none](-6.779,.518)--(-6.778,.518)--(-6.754,.519)--(-6.782,.519)--cycle;
\draw(-6.779,.518)--(-6.778,.518);
\draw(-6.754,.519)--(-6.782,.519);
\filldraw[fill opacity=0.8,fill=gray!20,draw=none](-6.776,.511)--(-6.797,.541)--(-6.797,.541)--(-6.762,.496)--cycle;
\draw(-6.797,.541)--(-6.797,.541)--(-6.762,.496)--(-6.776,.511);
\filldraw[fill opacity=0.8,fill=gray!20,draw=none](-7.843,.571)--(-7.842,.571)--(-7.838,.573)--(-7.83,.578)--(-7.84,.575)--cycle;
\draw(-7.842,.571)--(-7.838,.573)--(-7.83,.578)--(-7.84,.575);
\filldraw[fill opacity=0.8,fill=gray!20,draw=none](-7.808,4.562)--(-7.81,4.556)--(-7.808,4.552)--(-7.802,4.555)--cycle;
\draw(-7.808,4.552)--(-7.802,4.555);
\filldraw[fill opacity=0.8,fill=gray!20,draw=none](-7.815,4.579)--(-7.818,4.572)--(-7.821,4.566)--(-7.815,4.582)--cycle;
\filldraw[fill opacity=0.8,fill=gray!20,draw=none](-8.553,2.954)--(-8.566,2.968)--(-8.554,2.94)--cycle;
\draw(-8.553,2.954)--(-8.566,2.968)--(-8.554,2.94);
\filldraw[fill opacity=0.8,fill=gray!20](-6.813,.558)--(-6.853,.587)--(-6.842,.575)--(-6.797,.541)--cycle;
\filldraw[fill opacity=0.8,fill=gray!20,draw=none](-8.556,2.961)--(-8.567,2.993)--(-8.569,2.993)--(-8.567,2.977)--cycle;
\draw(-8.569,2.993)--(-8.567,2.977);
\filldraw[fill opacity=0.8,fill=gray!20,draw=none](-8.577,2.998)--(-8.576,2.999)--(-8.501,2.953)--(-8.476,2.926)--(-8.558,2.962)--cycle;
\draw(-8.476,2.926)--(-8.558,2.962);
\filldraw[fill opacity=0.8,fill=gray!20,draw=none](-7.807,4.551)--(-7.64,4.478)--(-7.629,4.473)--(-7.69,4.499)--(-7.73,4.517)--(-7.776,4.537)--(-7.81,4.552)--(-7.827,4.559)--(-7.826,4.559)--cycle;
\draw(-7.69,4.499)--(-7.73,4.517)--(-7.776,4.537)--(-7.81,4.552)--(-7.827,4.559)--(-7.826,4.559);
\filldraw[fill opacity=0.8,fill=gray!20,draw=none](-7.483,4.503)--(-7.469,4.514)--(-7.457,4.526)--(-7.453,4.53)--(-7.471,4.525)--(-7.491,4.498)--cycle;
\draw(-7.457,4.526)--(-7.453,4.53);
\draw(-7.471,4.525)--(-7.491,4.498)--(-7.483,4.503);
\filldraw[fill opacity=0.8,fill=gray!20,draw=none](-8.307,2.807)--(-8.263,2.821)--(-8.264,2.822)--(-8.305,2.813)--(-8.329,2.802)--cycle;
\draw(-8.264,2.822)--(-8.305,2.813)--(-8.329,2.802)--(-8.307,2.807)--(-8.263,2.821);
\filldraw[fill opacity=0.8,fill=gray!20,draw=none](-8.393,2.878)--(-8.404,2.883)--(-8.394,2.88)--cycle;
\draw(-8.393,2.878)--(-8.404,2.883);
\filldraw[fill opacity=0.8,fill=gray!20,draw=none](-8.407,2.886)--(-8.404,2.883)--(-8.393,2.878)--cycle;
\draw(-8.404,2.883)--(-8.393,2.878);
\filldraw[fill opacity=0.8,fill=gray!20,draw=none](-8.264,2.822)--(-8.266,2.824)--(-8.293,2.828)--(-8.305,2.813)--cycle;
\draw(-8.293,2.828)--(-8.305,2.813)--(-8.264,2.822);
\filldraw[fill opacity=0.8,fill=gray!20,draw=none](-8.459,2.879)--(-8.462,2.891)--(-8.539,2.932)--(-8.524,2.895)--cycle;
\draw(-8.539,2.932)--(-8.524,2.895)--(-8.459,2.879)--(-8.462,2.891);
\filldraw[fill opacity=0.8,fill=gray!20,draw=none](-8.364,2.835)--(-8.364,2.85)--(-8.377,2.856)--(-8.427,2.876)--(-8.459,2.879)--(-8.44,2.841)--cycle;
\draw(-8.427,2.876)--(-8.459,2.879)--(-8.44,2.841)--(-8.364,2.835)--(-8.364,2.85);
\filldraw[fill opacity=0.8,fill=gray!20,draw=none](-8.427,2.876)--(-8.462,2.891)--(-8.459,2.879)--cycle;
\draw(-8.462,2.891)--(-8.459,2.879)--(-8.427,2.876);
\filldraw[fill opacity=0.5,fill=gray!20,draw=none](-8.693,2.964)--(-8.753,3.005)--(-8.435,2.896)--(-8.293,2.842)--(-8.25,2.813)--cycle;
\draw(-8.293,2.842)--(-8.25,2.813)--(-8.693,2.964)--(-8.753,3.005)--(-8.435,2.896);
\filldraw[fill opacity=0.8,fill=gray!20,draw=none](-7.823,4.606)--(-7.81,4.592)--(-7.817,4.647)--(-7.839,4.671)--(-7.834,4.633)--cycle;
\draw(-7.823,4.606)--(-7.81,4.592)--(-7.817,4.647)--(-7.839,4.671)--(-7.834,4.633);
\filldraw[fill opacity=0.8,fill=gray!20](-8.177,3.768)--(-8.184,3.823)--(-8.206,3.847)--(-8.199,3.792)--cycle;
\filldraw[fill opacity=0.8,fill=gray!20](-7.832,3.691)--(-7.797,3.735)--(-7.829,3.715)--(-7.858,3.674)--cycle;
\filldraw[fill opacity=0.8,fill=gray!20,draw=none](-7.142,.523)--(-7.147,.523)--(-7.191,.522)--(-7.124,.521)--cycle;
\draw(-7.142,.523)--(-7.147,.523);
\draw(-7.191,.522)--(-7.124,.521);
\filldraw[fill opacity=0.8,fill=gray!20,draw=none](-7.657,4.884)--(-7.646,4.887)--(-7.654,4.883)--cycle;
\draw(-7.657,4.884)--(-7.646,4.887)--(-7.654,4.883);
\filldraw[fill opacity=0.8,fill=gray!20,draw=none](-7.453,4.53)--(-7.43,4.559)--(-7.461,4.539)--(-7.471,4.525)--cycle;
\draw(-7.453,4.53)--(-7.43,4.559)--(-7.461,4.539)--(-7.471,4.525);
\filldraw[fill opacity=0.8,fill=gray!20,draw=none](-7.795,4.545)--(-7.79,4.543)--(-7.806,4.559)--cycle;
\draw(-7.79,4.543)--(-7.806,4.559);
\filldraw[fill opacity=0.8,fill=gray!20,draw=none](-8.107,3.898)--(-8.174,3.746)--(-8.163,3.733)--(-8.092,3.892)--cycle;
\draw(-8.163,3.733)--(-8.092,3.892)--(-8.107,3.898)--(-8.174,3.746);
\filldraw[fill opacity=0.8,fill=gray!20,draw=none](-8.175,3.737)--(-8.157,3.719)--(-8.174,3.76)--cycle;
\draw(-8.175,3.737)--(-8.157,3.719)--(-8.174,3.76);
\filldraw[fill opacity=0.8,fill=gray!20,draw=none](-7.511,4.56)--(-7.513,4.562)--(-7.512,4.584)--(-7.511,4.588)--(-7.501,4.587)--(-7.501,4.583)--cycle;
\draw(-7.501,4.587)--(-7.501,4.583);
\filldraw[fill opacity=0.8,fill=gray!20,draw=none](-7.499,4.587)--(-7.501,4.583)--(-7.509,4.562)--(-7.508,4.563)--(-7.499,4.583)--cycle;
\draw(-7.508,4.563)--(-7.499,4.583);
\filldraw[fill opacity=0.8,fill=gray!20,draw=none](-7.486,4.623)--(-7.499,4.588)--(-7.499,4.583)--(-7.484,4.617)--cycle;
\draw(-7.499,4.583)--(-7.484,4.617);
\filldraw[fill opacity=0.8,fill=gray!20,draw=none](-7.511,4.56)--(-7.512,4.559)--(-7.517,4.547)--(-7.509,4.56)--(-7.508,4.563)--cycle;
\draw(-7.509,4.56)--(-7.508,4.563);
\filldraw[fill opacity=0.8,fill=gray!20,draw=none](-7.501,4.583)--(-7.511,4.56)--(-7.509,4.562)--cycle;
\filldraw[fill opacity=0.8,fill=gray!20,draw=none](-7.509,4.557)--(-7.511,4.56)--(-7.501,4.583)--(-7.5,4.577)--cycle;
\draw(-7.501,4.583)--(-7.5,4.577);
\filldraw[fill opacity=0.8,fill=gray!20,draw=none](-7.499,4.588)--(-7.496,4.59)--(-7.5,4.577)--(-7.501,4.583)--cycle;
\draw(-7.5,4.577)--(-7.501,4.583);
\filldraw[fill opacity=0.8,fill=gray!20,draw=none](-7.486,4.623)--(-7.485,4.62)--(-7.496,4.59)--(-7.499,4.588)--cycle;
\draw(-7.486,4.623)--(-7.485,4.62);
\filldraw[fill opacity=0.8,fill=gray!20,draw=none](-7.516,4.538)--(-7.511,4.56)--(-7.509,4.557)--cycle;
\filldraw[fill opacity=0.8,fill=gray!20,draw=none](-7.504,4.561)--(-7.496,4.582)--(-7.496,4.594)--(-7.511,4.555)--cycle;
\draw(-7.504,4.561)--(-7.496,4.582);
\draw(-7.496,4.594)--(-7.511,4.555);
\filldraw[fill opacity=0.8,fill=gray!20,draw=none](-7.865,.564)--(-7.875,.56)--(-7.882,.556)--(-7.86,.564)--cycle;
\draw(-7.875,.56)--(-7.882,.556)--(-7.86,.564);
\filldraw[fill opacity=0.8,fill=gray!20,draw=none](-7.483,4.631)--(-7.486,4.623)--(-7.484,4.617)--(-7.478,4.63)--cycle;
\draw(-7.484,4.617)--(-7.478,4.63);
\filldraw[fill opacity=0.8,fill=gray!20,draw=none](-7.496,4.582)--(-7.477,4.63)--(-7.482,4.63)--(-7.496,4.594)--cycle;
\draw(-7.496,4.582)--(-7.477,4.63)--(-7.482,4.63)--(-7.496,4.594);
\filldraw[fill opacity=0.8,fill=gray!20,draw=none](-7.808,4.563)--(-7.808,4.562)--(-7.806,4.56)--cycle;
\filldraw[fill opacity=0.8,fill=gray!20](-7.932,.474)--(-7.906,.514)--(-7.92,.528)--(-7.949,.491)--cycle;
\filldraw[fill opacity=0.8,fill=gray!20](-7.608,.398)--(-7.614,.444)--(-7.643,.425)--(-7.638,.378)--cycle;
\filldraw[fill opacity=0.8,fill=gray!20,draw=none](-7.64,1.16)--(-7.65,1.138)--(-7.65,1.182)--cycle;
\filldraw[fill opacity=0.8,fill=gray!20](-6.988,1.121)--(-6.985,1.175)--(-6.878,1.18)--(-6.877,1.126)--cycle;
\filldraw[fill opacity=0.8,fill=gray!20,draw=none](-7.04,1.141)--(-7.05,1.114)--(-7.65,1.138)--(-7.629,1.184)--(-7.048,1.161)--cycle;
\draw(-7.05,1.114)--(-7.65,1.138);
\draw(-7.629,1.184)--(-7.048,1.161);
\filldraw[fill opacity=0.8,fill=gray!20,draw=none](-7.64,1.16)--(-7.65,1.182)--(-7.65,1.185)--(-7.629,1.184)--cycle;
\draw(-7.65,1.185)--(-7.629,1.184);
\filldraw[fill opacity=0.8,fill=gray!20](-7.69,1.148)--(-7.693,1.193)--(-7.632,1.178)--(-7.627,1.132)--cycle;
\filldraw[fill opacity=0.8,fill=gray!20](-7.719,1.012)--(-7.703,1.054)--(-7.649,1.04)--(-7.675,1.001)--cycle;
\filldraw[fill opacity=0.8,fill=gray!20,draw=none](-7.802,4.555)--(-7.806,4.559)--(-7.806,4.56)--cycle;
\draw(-7.806,4.559)--(-7.806,4.56);
\filldraw[fill opacity=0.8,fill=gray!20,draw=none](-7.815,4.579)--(-7.815,4.582)--(-7.824,4.601)--cycle;
\filldraw[fill opacity=0.8,fill=gray!20](-8.019,3.627)--(-8.048,3.639)--(-8.085,3.648)--(-8.038,3.632)--cycle;
\filldraw[fill opacity=0.8,fill=gray!20,draw=none](-6.796,.239)--(-6.778,.261)--(-6.787,.26)--cycle;
\draw(-6.796,.239)--(-6.778,.261);
\filldraw[fill opacity=0.8,fill=gray!20,draw=none](-6.952,.263)--(-6.779,.261)--(-6.761,.259)--(-6.954,.261)--cycle;
\draw(-6.761,.259)--(-6.954,.261)--(-6.952,.263)--(-6.779,.261);
\filldraw[fill opacity=0.8,fill=gray!20,draw=none](-7.73,4.856)--(-7.693,4.87)--(-7.689,4.873)--(-7.699,4.872)--(-7.712,4.867)--cycle;
\draw(-7.699,4.872)--(-7.712,4.867)--(-7.73,4.856)--(-7.693,4.87);
\filldraw[fill opacity=0.8,fill=gray!20,draw=none](-5.993,.67)--(-6.083,.534)--(-6.055,.518)--(-5.951,.66)--cycle;
\draw(-5.993,.67)--(-6.083,.534);
\filldraw[fill opacity=0.8,fill=gray!20,draw=none](-7.872,.546)--(-7.84,.564)--(-7.86,.564)--(-7.882,.556)--cycle;
\draw(-7.86,.564)--(-7.882,.556)--(-7.872,.546)--(-7.84,.564);
\filldraw[fill opacity=0.8,fill=gray!20,draw=none](-7.841,.538)--(-7.829,.551)--(-7.84,.564)--(-7.872,.546)--cycle;
\draw(-7.84,.564)--(-7.872,.546)--(-7.841,.538)--(-7.829,.551);
\filldraw[fill opacity=0.8,fill=gray!20,draw=none](-7.821,.537)--(-7.82,.544)--(-7.829,.551)--(-7.841,.538)--cycle;
\draw(-7.829,.551)--(-7.841,.538)--(-7.821,.537);
\filldraw[fill opacity=0.8,fill=gray!20,draw=none](-8.712,.927)--(-8.479,.83)--(-8.534,.843)--(-8.72,.92)--cycle;
\draw(-8.712,.927)--(-8.479,.83);
\draw(-8.534,.843)--(-8.72,.92);
\filldraw[fill opacity=0.8,fill=gray!20,draw=none](-8.712,.927)--(-8.72,.92)--(-8.746,.931)--cycle;
\draw(-8.72,.92)--(-8.746,.931);
\filldraw[fill opacity=0.8,fill=gray!20,draw=none](-8.775,.903)--(-8.772,.911)--(-8.732,.942)--(-8.712,.927)--(-8.707,.886)--cycle;
\draw(-8.712,.927)--(-8.707,.886)--(-8.775,.903);
\filldraw[fill opacity=0.8,fill=gray!20,draw=none](-8.745,.941)--(-8.712,.927)--(-8.746,.931)--(-8.763,.938)--cycle;
\draw(-8.745,.941)--(-8.712,.927);
\draw(-8.746,.931)--(-8.763,.938);
\filldraw[fill opacity=0.8,fill=gray!20,draw=none](-8.811,.888)--(-8.834,.933)--(-8.833,.937)--(-8.811,.94)--cycle;
\draw(-8.834,.933)--(-8.833,.937);
\filldraw[fill opacity=0.8,fill=gray!20,draw=none](-8.811,.888)--(-8.811,.94)--(-8.779,.945)--(-8.807,.881)--cycle;
\draw(-8.779,.945)--(-8.807,.881);
\filldraw[fill opacity=0.8,fill=gray!20,draw=none](-8.745,.941)--(-8.763,.938)--(-8.779,.945)--cycle;
\draw(-8.763,.938)--(-8.779,.945);
\filldraw[fill opacity=0.8,fill=gray!20,draw=none](-8.807,.881)--(-8.779,.945)--(-8.759,.953)--(-8.745,.941)--(-8.782,.856)--cycle;
\draw(-8.807,.881)--(-8.779,.945);
\draw(-8.745,.941)--(-8.782,.856);
\filldraw[fill opacity=0.8,fill=gray!20,draw=none](-8.759,.953)--(-8.779,.945)--(-8.771,.963)--cycle;
\draw(-8.779,.945)--(-8.771,.963);
\filldraw[fill opacity=0.8,fill=gray!20,draw=none](-8.802,.965)--(-8.745,.941)--(-8.779,.945)--(-8.822,.963)--cycle;
\draw(-8.802,.965)--(-8.745,.941);
\draw(-8.779,.945)--(-8.822,.963);
\filldraw[fill opacity=0.8,fill=gray!20,draw=none](-8.855,.986)--(-8.852,.986)--(-8.802,.965)--(-8.822,.963)--(-8.828,.965)--cycle;
\draw(-8.855,.986)--(-8.852,.986)--(-8.802,.965);
\draw(-8.822,.963)--(-8.828,.965);
\filldraw[fill opacity=0.8,fill=gray!20,draw=none](-8.937,.991)--(-8.898,1)--(-8.833,.999)--(-8.954,.972)--cycle;
\draw(-8.833,.999)--(-8.954,.972)--(-8.937,.991)--(-8.898,1);
\filldraw[fill opacity=0.8,fill=gray!20,draw=none](-8.874,.99)--(-8.867,.991)--(-8.865,.984)--cycle;
\draw(-8.874,.99)--(-8.867,.991);
\filldraw[fill opacity=0.8,fill=gray!20,draw=none](-8.855,.986)--(-8.837,.972)--(-8.869,.985)--(-8.882,.998)--cycle;
\draw(-8.837,.972)--(-8.869,.985);
\filldraw[fill opacity=0.8,fill=gray!20,draw=none](-8.84,.97)--(-8.841,.973)--(-8.837,.972)--cycle;
\draw(-8.841,.973)--(-8.837,.972);
\filldraw[fill opacity=0.8,fill=gray!20,draw=none](-8.855,.986)--(-8.828,.965)--(-8.879,.986)--cycle;
\draw(-8.828,.965)--(-8.879,.986)--(-8.855,.986);
\filldraw[fill opacity=0.8,fill=gray!20,draw=none](-8.794,.925)--(-8.833,.942)--(-8.84,.97)--(-8.837,.972)--(-8.82,.964)--cycle;
\draw(-8.837,.972)--(-8.82,.964)--(-8.794,.925)--(-8.833,.942);
\filldraw[fill opacity=0.8,fill=gray!20,draw=none](-8.686,.864)--(-8.707,.886)--(-8.712,.927)--(-8.701,.911)--(-8.689,.876)--cycle;
\draw(-8.686,.864)--(-8.707,.886)--(-8.712,.927);
\filldraw[fill opacity=0.8,fill=gray!20,draw=none](-8.798,.818)--(-8.745,.941)--(-8.734,.925)--(-8.783,.812)--cycle;
\draw(-8.734,.925)--(-8.783,.812)--(-8.798,.818)--(-8.745,.941);
\filldraw[fill opacity=0.8,fill=gray!20,draw=none](-8.82,.892)--(-8.838,.9)--(-8.835,.936)--cycle;
\draw(-8.82,.892)--(-8.838,.9);
\filldraw[fill opacity=0.8,fill=gray!20,draw=none](-8.917,.87)--(-8.897,.917)--(-8.833,.937)--(-8.87,.85)--cycle;
\draw(-8.833,.937)--(-8.87,.85)--(-8.917,.87)--(-8.897,.917);
\filldraw[fill opacity=0.8,fill=gray!20,draw=none](-8.962,.89)--(-8.897,.917)--(-8.917,.87)--cycle;
\draw(-8.897,.917)--(-8.917,.87)--(-8.962,.89);
\filldraw[fill opacity=0.8,fill=gray!20,draw=none](-8.785,.877)--(-8.82,.892)--(-8.835,.936)--(-8.835,.938)--(-8.833,.942)--(-8.794,.925)--cycle;
\draw(-8.833,.942)--(-8.794,.925)--(-8.785,.877)--(-8.82,.892);
\filldraw[fill opacity=0.8,fill=gray!20,draw=none](-8.811,.888)--(-8.807,.881)--(-8.811,.873)--cycle;
\draw(-8.807,.881)--(-8.811,.873);
\filldraw[fill opacity=0.8,fill=gray!20,draw=none](-8.811,.873)--(-8.794,.827)--(-8.81,.834)--cycle;
\draw(-8.794,.827)--(-8.81,.834);
\filldraw[fill opacity=0.8,fill=gray!20,draw=none](-8.828,.832)--(-8.807,.881)--(-8.782,.856)--(-8.798,.818)--cycle;
\draw(-8.782,.856)--(-8.798,.818)--(-8.828,.832)--(-8.807,.881);
\filldraw[fill opacity=0.8,fill=gray!20,draw=none](-8.794,.827)--(-8.817,.891)--(-8.785,.877)--cycle;
\draw(-8.817,.891)--(-8.785,.877)--(-8.794,.827);
\filldraw[fill opacity=0.8,fill=gray!20,draw=none](-8.87,.85)--(-8.834,.933)--(-8.811,.888)--(-8.811,.873)--(-8.828,.832)--cycle;
\draw(-8.811,.873)--(-8.828,.832)--(-8.87,.85)--(-8.834,.933);
\filldraw[fill opacity=0.8,fill=gray!20,draw=none](-8.799,.819)--(-8.802,.82)--(-8.794,.827)--cycle;
\draw(-8.794,.827)--(-8.799,.819);
\filldraw[fill opacity=0.8,fill=gray!20,draw=none](-8.853,.985)--(-8.82,.964)--(-8.837,.972)--(-8.855,.986)--cycle;
\draw(-8.853,.985)--(-8.82,.964)--(-8.837,.972);
\filldraw[fill opacity=0.8,fill=gray!20,draw=none](-8.903,.864)--(-8.799,.819)--(-8.794,.827)--(-8.785,.877)--(-8.794,.925)--(-8.82,.964)--(-8.853,.985)--cycle;
\draw(-8.799,.819)--(-8.794,.827)--(-8.785,.877)--(-8.794,.925)--(-8.82,.964)--(-8.853,.985);
\filldraw[fill opacity=0.8,fill=gray!20,draw=none](-8.833,.967)--(-7.78,.527)--(-7.757,.529)--(-8.852,.986)--cycle;
\draw(-7.757,.529)--(-8.852,.986)--(-8.833,.967)--(-7.78,.527);
\filldraw[fill opacity=0.8,fill=gray!20,draw=none](-8.554,2.955)--(-8.556,2.961)--(-8.567,2.977)--(-8.566,2.968)--cycle;
\draw(-8.567,2.977)--(-8.566,2.968)--(-8.554,2.955);
\filldraw[fill opacity=0.8,fill=gray!20,draw=none](-7.425,4.763)--(-7.483,4.631)--(-7.478,4.63)--(-7.426,4.75)--cycle;
\draw(-7.425,4.763)--(-7.483,4.631);
\draw(-7.478,4.63)--(-7.426,4.75);
\filldraw[fill opacity=0.8,fill=gray!20,draw=none](-7.417,4.78)--(-7.422,4.759)--(-7.414,4.778)--cycle;
\draw(-7.422,4.759)--(-7.414,4.778)--(-7.417,4.78);
\filldraw[fill opacity=0.8,fill=gray!20,draw=none](-8.421,3.197)--(-8.373,3.176)--(-8.382,3.176)--(-8.436,3.184)--cycle;
\draw(-8.421,3.197)--(-8.373,3.176);
\filldraw[fill opacity=0.8,fill=gray!20,draw=none](-8.423,3.182)--(-8.427,3.179)--(-8.418,3.2)--cycle;
\draw(-8.427,3.179)--(-8.418,3.2);
\filldraw[fill opacity=0.8,fill=gray!20](-8.402,3.234)--(-8.354,3.235)--(-8.354,3.235)--(-8.38,3.239)--cycle;
\filldraw[fill opacity=0.8,fill=gray!20,draw=none](-8.411,3.216)--(-8.419,3.197)--(-8.4,3.2)--cycle;
\draw(-8.411,3.216)--(-8.419,3.197);
\filldraw[fill opacity=0.8,fill=gray!20,draw=none](-8.419,3.208)--(-8.303,3.158)--(-8.365,3.173)--(-8.419,3.196)--cycle;
\draw(-8.419,3.208)--(-8.303,3.158);
\draw(-8.365,3.173)--(-8.419,3.196);
\filldraw[fill opacity=0.8,fill=gray!20,draw=none](-8.373,3.176)--(-8.365,3.173)--(-8.382,3.176)--cycle;
\draw(-8.373,3.176)--(-8.365,3.173);
\filldraw[fill opacity=0.8,fill=gray!20,draw=none](-7.52,4.566)--(-7.52,4.57)--(-7.519,4.568)--cycle;
\filldraw[fill opacity=0.8,fill=gray!20,draw=none](-7.513,4.589)--(-7.501,4.615)--(-7.504,4.617)--(-7.512,4.599)--cycle;
\draw(-7.513,4.589)--(-7.501,4.615)--(-7.504,4.617)--(-7.512,4.599);
\filldraw[fill opacity=0.8,fill=gray!20,draw=none](-7.521,4.57)--(-7.513,4.589)--(-7.512,4.599)--(-7.527,4.565)--cycle;
\draw(-7.521,4.57)--(-7.513,4.589);
\draw(-7.512,4.599)--(-7.527,4.565);
\filldraw[fill opacity=0.8,fill=gray!20,draw=none](-7.511,4.599)--(-7.524,4.564)--(-7.518,4.569)--(-7.511,4.588)--cycle;
\draw(-7.511,4.599)--(-7.524,4.564);
\draw(-7.518,4.569)--(-7.511,4.588);
\filldraw[fill opacity=0.8,fill=gray!20,draw=none](-7.522,4.566)--(-7.531,4.545)--(-7.518,4.569)--cycle;
\filldraw[fill opacity=0.8,fill=gray!20,draw=none](-6.252,.291)--(-6.259,.3)--(-6.259,.325)--cycle;
\draw(-6.259,.3)--(-6.259,.325);
\filldraw[fill opacity=0.8,fill=gray!20,draw=none](-7.512,4.584)--(-7.513,4.562)--(-7.519,4.568)--cycle;
\filldraw[fill opacity=0.8,fill=gray!20,draw=none](-7.502,4.623)--(-7.511,4.599)--(-7.511,4.588)--(-7.498,4.623)--cycle;
\draw(-7.511,4.588)--(-7.498,4.623)--(-7.502,4.623)--(-7.511,4.599);
\filldraw[fill opacity=0.8,fill=gray!20](-7.838,.573)--(-7.79,.579)--(-7.79,.579)--(-7.83,.578)--cycle;
\filldraw[fill opacity=0.8,fill=gray!20,draw=none](-8.464,2.917)--(-8.494,2.934)--(-8.476,2.926)--cycle;
\draw(-8.494,2.934)--(-8.476,2.926);
\filldraw[fill opacity=0.8,fill=gray!20,draw=none](-7.521,4.57)--(-7.525,4.567)--(-7.536,4.547)--cycle;
\filldraw[fill opacity=0.8,fill=gray!20,draw=none](-7.627,4.492)--(-7.663,4.506)--(-7.662,4.506)--cycle;
\filldraw[fill opacity=0.8,fill=gray!20,draw=none](-7.494,4.702)--(-7.488,4.631)--(-7.487,4.633)--(-7.489,4.663)--cycle;
\draw(-7.487,4.633)--(-7.489,4.663)--(-7.494,4.702)--(-7.488,4.631);
\filldraw[fill opacity=0.8,fill=gray!20,draw=none](-7.641,4.301)--(-7.62,4.348)--(-7.62,4.358)--(-7.645,4.301)--cycle;
\draw(-7.641,4.301)--(-7.62,4.348);
\draw(-7.62,4.358)--(-7.645,4.301);
\filldraw[fill opacity=0.8,fill=gray!20,draw=none](-7.62,4.348)--(-7.611,4.377)--(-7.62,4.358)--cycle;
\draw(-7.611,4.377)--(-7.62,4.358);
\filldraw[fill opacity=0.8,fill=gray!20](-7.714,.234)--(-7.682,.258)--(-7.733,.248)--(-7.749,.227)--cycle;
\filldraw[fill opacity=0.8,fill=gray!20](-8.386,2.803)--(-8.415,2.815)--(-8.452,2.824)--(-8.405,2.808)--cycle;
\filldraw[fill opacity=0.8,fill=gray!20](-8.097,4.032)--(-8.044,4.052)--(-8.035,4.058)--(-8.079,4.043)--cycle;
\filldraw[fill opacity=0.8,fill=gray!20,draw=none](-7.621,4.551)--(-7.655,4.572)--(-7.717,4.542)--(-7.697,4.519)--(-7.688,4.516)--(-7.655,4.531)--cycle;
\draw(-7.621,4.551)--(-7.655,4.572)--(-7.717,4.542);
\draw(-7.688,4.516)--(-7.655,4.531);
\filldraw[fill opacity=0.8,fill=gray!20](-7.141,.315)--(-7.148,.37)--(-7.171,.394)--(-7.163,.338)--cycle;
\filldraw[fill opacity=0.8,fill=gray!20,draw=none](-6.797,.238)--(-6.796,.239)--(-6.787,.26)--(-6.795,.259)--(-6.822,.221)--cycle;
\draw(-6.795,.259)--(-6.822,.221)--(-6.797,.238)--(-6.796,.239);
\filldraw[fill opacity=0.8,fill=gray!20,draw=none](-7.817,4.647)--(-7.81,4.703)--(-7.822,4.717)--(-7.834,4.711)--(-7.839,4.671)--cycle;
\draw(-7.834,4.711)--(-7.839,4.671)--(-7.817,4.647)--(-7.81,4.703)--(-7.822,4.717);
\filldraw[fill opacity=0.8,fill=gray!20,draw=none](-7.43,4.559)--(-7.443,4.588)--(-7.461,4.539)--cycle;
\draw(-7.443,4.588)--(-7.461,4.539)--(-7.43,4.559);
\filldraw[fill opacity=0.8,fill=gray!20,draw=none](-7.43,4.559)--(-7.408,4.61)--(-7.443,4.588)--cycle;
\draw(-7.43,4.559)--(-7.408,4.61)--(-7.443,4.588);
\filldraw[fill opacity=0.8,fill=gray!20,draw=none](-6.061,.64)--(-6.097,.586)--(-6.064,.562)--(-6.032,.611)--cycle;
\draw(-6.061,.64)--(-6.097,.586);
\draw(-6.064,.562)--(-6.032,.611);
\filldraw[fill opacity=0.8,fill=gray!20,draw=none](-7.868,3.791)--(-7.8,3.943)--(-7.8,3.952)--(-7.8,3.953)--(-7.871,3.793)--cycle;
\draw(-7.8,3.953)--(-7.871,3.793)--(-7.868,3.791)--(-7.8,3.943);
\filldraw[fill opacity=0.8,fill=gray!20,draw=none](-7.628,4.508)--(-7.627,4.492)--(-7.673,4.509)--(-7.675,4.52)--cycle;
\draw(-7.628,4.508)--(-7.627,4.492);
\filldraw[fill opacity=0.8,fill=gray!20,draw=none](-7.815,4.582)--(-7.824,4.558)--(-7.827,4.559)--(-7.814,4.587)--cycle;
\draw(-7.824,4.558)--(-7.827,4.559)--(-7.814,4.587);
\filldraw[fill opacity=0.8,fill=gray!20,draw=none](-7.871,3.793)--(-7.942,3.635)--(-7.939,3.631)--(-7.868,3.791)--cycle;
\draw(-7.939,3.631)--(-7.868,3.791)--(-7.871,3.793)--(-7.942,3.635);
\filldraw[fill opacity=0.8,fill=gray!20,draw=none](-7.94,3.631)--(-7.916,3.642)--(-7.938,3.638)--(-7.962,3.626)--cycle;
\draw(-7.916,3.642)--(-7.938,3.638)--(-7.962,3.626)--(-7.94,3.631);
\filldraw[fill opacity=0.8,fill=gray!20,draw=none](-7.94,3.631)--(-7.896,3.646)--(-7.916,3.642)--cycle;
\draw(-7.94,3.631)--(-7.896,3.646)--(-7.916,3.642);
\filldraw[fill opacity=0.8,fill=gray!20,draw=none](-6.952,.52)--(-6.779,.518)--(-6.782,.519)--(-6.949,.521)--cycle;
\draw(-6.782,.519)--(-6.949,.521)--(-6.952,.52)--(-6.779,.518);
\filldraw[fill opacity=0.8,fill=gray!20,draw=none](-7.516,4.538)--(-7.52,4.566)--(-7.519,4.568)--(-7.511,4.56)--cycle;
\filldraw[fill opacity=0.8,fill=gray!20](-7.748,.577)--(-7.79,.579)--(-7.79,.579)--(-7.743,.572)--cycle;
\filldraw[fill opacity=0.8,fill=gray!20,draw=none](-7.942,3.635)--(-7.957,3.6)--(-7.943,3.622)--(-7.939,3.631)--cycle;
\draw(-7.942,3.635)--(-7.957,3.6);
\draw(-7.943,3.622)--(-7.939,3.631);
\filldraw[fill opacity=0.8,fill=gray!20,draw=none](-7.957,3.6)--(-8.166,3.131)--(-8.152,3.152)--(-7.943,3.622)--cycle;
\draw(-7.957,3.6)--(-8.166,3.131);
\draw(-8.152,3.152)--(-7.943,3.622);
\filldraw[fill opacity=0.8,fill=gray!20,draw=none](-8.495,3.184)--(-8.489,3.192)--(-8.485,3.191)--cycle;
\draw(-8.489,3.192)--(-8.485,3.191);
\filldraw[fill opacity=0.8,fill=gray!20](-6.983,.174)--(-7.012,.186)--(-7.05,.195)--(-7.002,.179)--cycle;
\filldraw[fill opacity=0.8,fill=gray!20](-7.867,.971)--(-7.898,1.004)--(-7.848,1.013)--(-7.831,.977)--cycle;
\filldraw[fill opacity=0.8,fill=gray!20](-8.464,3.207)--(-8.411,3.228)--(-8.402,3.234)--(-8.446,3.219)--cycle;
\filldraw[fill opacity=0.8,fill=gray!20,draw=none](-6.245,.513)--(-6.241,.509)--(-6.271,.502)--cycle;
\filldraw[fill opacity=0.8,fill=gray!20,draw=none](-8.573,3.018)--(-8.572,3.011)--(-8.573,3.003)--(-8.579,3.006)--cycle;
\draw(-8.573,3.003)--(-8.579,3.006);
\filldraw[fill opacity=0.8,fill=gray!20,draw=none](-6.779,.261)--(-6.274,.255)--(-6.302,.254)--(-6.761,.259)--cycle;
\draw(-6.779,.261)--(-6.274,.255);
\draw(-6.302,.254)--(-6.761,.259);
\filldraw[fill opacity=0.8,fill=gray!20](-7.646,4.887)--(-7.62,4.883)--(-7.62,4.883)--(-7.617,4.888)--cycle;
\filldraw[fill opacity=0.8,fill=gray!20](-7.617,4.888)--(-7.62,4.883)--(-7.62,4.883)--(-7.589,4.886)--cycle;
\filldraw[fill opacity=0.8,fill=gray!20](-8.184,3.823)--(-8.177,3.879)--(-8.199,3.903)--(-8.206,3.847)--cycle;
\filldraw[fill opacity=0.8,fill=gray!20](-7.797,3.735)--(-7.776,3.786)--(-7.81,3.764)--(-7.829,3.715)--cycle;
\filldraw[fill opacity=0.8,fill=gray!20,draw=none](-8.166,3.131)--(-8.167,3.129)--(-8.154,3.147)--(-8.152,3.152)--cycle;
\draw(-8.166,3.131)--(-8.167,3.129);
\draw(-8.154,3.147)--(-8.152,3.152);
\filldraw[fill opacity=0.8,fill=gray!20,draw=none](-8.461,3.21)--(-8.454,3.212)--(-8.421,3.197)--(-8.436,3.184)--(-8.483,3.192)--cycle;
\draw(-8.454,3.212)--(-8.421,3.197);
\filldraw[fill opacity=0.8,fill=gray!20,draw=none](-8.167,3.119)--(-8.167,3.128)--(-8.164,3.125)--(-8.156,3.106)--cycle;
\draw(-8.167,3.128)--(-8.164,3.125)--(-8.156,3.106);
\filldraw[fill opacity=0.8,fill=gray!20,draw=none](-6.114,.52)--(-6.103,.521)--(-6.113,.525)--cycle;
\draw(-6.103,.521)--(-6.113,.525);
\filldraw[fill opacity=0.8,fill=gray!20,draw=none](-6.07,.567)--(-6.107,.571)--(-6.143,.518)--(-6.091,.522)--(-6.064,.562)--cycle;
\draw(-6.107,.571)--(-6.143,.518);
\draw(-6.091,.522)--(-6.064,.562);
\filldraw[fill opacity=0.8,fill=gray!20,draw=none](-6.134,.564)--(-6.095,.565)--(-6.129,.583)--cycle;
\filldraw[fill opacity=0.8,fill=gray!20](-7.614,.444)--(-7.632,.487)--(-7.658,.47)--(-7.643,.425)--cycle;
\filldraw[fill opacity=0.8,fill=gray!20](-7.906,.514)--(-7.872,.546)--(-7.882,.556)--(-7.92,.528)--cycle;
\filldraw[fill opacity=0.8,fill=gray!20,draw=none](-8.483,3.192)--(-8.485,3.191)--(-8.492,3.193)--cycle;
\draw(-8.485,3.191)--(-8.492,3.193);
\filldraw[fill opacity=0.8,fill=gray!20,draw=none](-8.429,3.168)--(-8.434,3.16)--(-8.438,3.153)--(-8.433,3.165)--cycle;
\draw(-8.438,3.153)--(-8.433,3.165);
\filldraw[fill opacity=0.8,fill=gray!20,draw=none](-8.556,2.961)--(-8.494,2.934)--(-8.464,2.917)--(-8.439,2.898)--(-8.544,2.944)--cycle;
\draw(-8.556,2.961)--(-8.494,2.934);
\draw(-8.439,2.898)--(-8.544,2.944);
\filldraw[fill opacity=0.8,fill=gray!20,draw=none](-8.472,3.219)--(-8.457,3.213)--(-8.483,3.192)--(-8.492,3.193)--(-8.506,3.199)--cycle;
\draw(-8.492,3.193)--(-8.506,3.199)--(-8.472,3.219)--(-8.457,3.213);
\filldraw[fill opacity=0.8,fill=gray!20,draw=none](-8.434,3.16)--(-8.443,3.144)--(-8.438,3.153)--cycle;
\draw(-8.443,3.144)--(-8.438,3.153);
\filldraw[fill opacity=0.8,fill=gray!20,draw=none](-8.419,3.197)--(-8.433,3.165)--(-8.42,3.156)--(-8.4,3.2)--cycle;
\draw(-8.419,3.197)--(-8.433,3.165);
\draw(-8.42,3.156)--(-8.4,3.2);
\filldraw[fill opacity=0.8,fill=gray!20,draw=none](-8.433,3.165)--(-8.438,3.153)--(-8.427,3.14)--(-8.42,3.156)--cycle;
\draw(-8.433,3.165)--(-8.438,3.153);
\draw(-8.427,3.14)--(-8.42,3.156);
\filldraw[fill opacity=0.8,fill=gray!20,draw=none](-8.438,3.153)--(-8.443,3.144)--(-8.437,3.126)--(-8.435,3.123)--(-8.427,3.14)--cycle;
\draw(-8.438,3.153)--(-8.443,3.144);
\draw(-8.435,3.123)--(-8.427,3.14);
\filldraw[fill opacity=0.8,fill=gray!20,draw=none](-8.437,3.126)--(-8.435,3.121)--(-8.435,3.123)--cycle;
\draw(-8.435,3.121)--(-8.435,3.123);
\filldraw[fill opacity=0.8,fill=gray!20,draw=none](-8.4,3.2)--(-8.427,3.14)--(-8.393,3.133)--(-8.374,3.178)--cycle;
\draw(-8.4,3.2)--(-8.427,3.14);
\draw(-8.393,3.133)--(-8.374,3.178);
\filldraw[fill opacity=0.8,fill=gray!20,draw=none](-8.427,3.14)--(-8.435,3.121)--(-8.398,3.124)--(-8.393,3.133)--cycle;
\draw(-8.427,3.14)--(-8.435,3.121);
\draw(-8.398,3.124)--(-8.393,3.133);
\filldraw[fill opacity=0.8,fill=gray!20,draw=none](-8.512,2.99)--(-8.473,3.018)--(-8.461,3.037)--(-8.439,3.092)--(-8.43,3.143)--(-8.431,3.158)--(-8.453,3.196)--(-8.494,3.186)--(-8.506,3.176)--(-8.537,3.134)--(-8.563,3.081)--(-8.569,3.06)--(-8.562,3.011)--(-8.548,2.976)--cycle;
\filldraw[fill opacity=0.8,fill=gray!20,draw=none](-8.57,3.002)--(-8.567,2.993)--(-8.55,2.99)--(-8.551,2.999)--(-8.573,3.023)--(-8.572,3.011)--cycle;
\draw(-8.55,2.99)--(-8.551,2.999)--(-8.573,3.023)--(-8.572,3.011);
\filldraw[fill opacity=0.8,fill=gray!20,draw=none](-6.954,.261)--(-6.787,.26)--(-6.784,.267)--(-6.785,.278)--(-6.956,.28)--cycle;
\draw(-6.785,.278)--(-6.956,.28)--(-6.954,.261)--(-6.787,.26);
\filldraw[fill opacity=0.8,fill=gray!20,draw=none](-6.778,.261)--(-6.762,.281)--(-6.793,.261)--(-6.795,.259)--cycle;
\draw(-6.778,.261)--(-6.762,.281)--(-6.793,.261)--(-6.795,.259);
\filldraw[fill opacity=0.8,fill=gray!20](-7.061,.578)--(-7.008,.599)--(-6.999,.605)--(-7.043,.59)--cycle;
\filldraw[fill opacity=0.8,fill=gray!20](-7.522,4.865)--(-7.569,4.881)--(-7.563,4.875)--(-7.511,4.853)--cycle;
\filldraw[fill opacity=0.8,fill=gray!20,draw=none](-7.676,4.52)--(-7.655,4.531)--(-7.678,4.52)--cycle;
\draw(-7.655,4.531)--(-7.678,4.52);
\filldraw[fill opacity=0.8,fill=gray!20](-7.984,4.064)--(-7.987,4.059)--(-7.987,4.059)--(-7.956,4.062)--cycle;
\filldraw[fill opacity=0.8,fill=gray!20](-8.013,4.063)--(-7.987,4.059)--(-7.987,4.059)--(-7.984,4.064)--cycle;
\filldraw[fill opacity=0.8,fill=gray!20](-8.551,2.999)--(-8.544,3.055)--(-8.566,3.079)--(-8.573,3.023)--cycle;
\filldraw[fill opacity=0.8,fill=gray!20,draw=none](-8.171,2.906)--(-8.164,2.911)--(-8.143,2.962)--(-8.146,2.96)--cycle;
\draw(-8.171,2.906)--(-8.164,2.911)--(-8.143,2.962)--(-8.146,2.96);
\filldraw[fill opacity=0.8,fill=gray!20,draw=none](-7.483,4.503)--(-7.474,4.509)--(-7.469,4.514)--cycle;
\draw(-7.483,4.503)--(-7.474,4.509);
\filldraw[fill opacity=0.8,fill=gray!20,draw=none](-8.576,2.999)--(-8.587,3.005)--(-8.588,3.01)--(-8.573,3.003)--cycle;
\draw(-8.587,3.005)--(-8.588,3.01)--(-8.573,3.003);
\filldraw[fill opacity=0.8,fill=gray!20,draw=none](-8.584,2.989)--(-8.587,3.005)--(-8.576,2.999)--cycle;
\draw(-8.584,2.989)--(-8.587,3.005);
\filldraw[fill opacity=0.8,fill=gray!20,draw=none](-8.569,3.06)--(-8.569,3.066)--(-8.586,3.031)--(-8.588,3.01)--(-8.586,2.998)--cycle;
\draw(-8.586,3.031)--(-8.588,3.01)--(-8.586,2.998);
\filldraw[fill opacity=0.8,fill=gray!20,draw=none](-7.621,4.551)--(-7.655,4.531)--(-7.618,4.549)--cycle;
\draw(-7.655,4.531)--(-7.618,4.549)--(-7.621,4.551);
\filldraw[fill opacity=0.8,fill=gray!20,draw=none](-7.603,4.534)--(-7.599,4.561)--(-7.675,4.521)--(-7.673,4.509)--(-7.62,4.486)--cycle;
\draw(-7.62,4.486)--(-7.603,4.534)--(-7.599,4.561);
\filldraw[fill opacity=0.8,fill=gray!20](-6.904,.177)--(-6.86,.192)--(-6.903,.184)--(-6.926,.173)--cycle;
\filldraw[fill opacity=0.8,fill=gray!20,draw=none](-7.81,4.573)--(-7.817,4.555)--(-7.818,4.555)--(-7.81,4.576)--cycle;
\draw(-7.818,4.555)--(-7.81,4.576);
\filldraw[fill opacity=0.8,fill=gray!20,draw=none](-7.822,4.717)--(-7.832,4.727)--(-7.834,4.711)--cycle;
\draw(-7.822,4.717)--(-7.832,4.727)--(-7.834,4.711);
\filldraw[fill opacity=0.8,fill=gray!20,draw=none](-7.678,4.52)--(-7.677,4.511)--(-7.715,4.524)--(-7.717,4.542)--cycle;
\draw(-7.678,4.52)--(-7.677,4.511);
\draw(-7.715,4.524)--(-7.717,4.542);
\filldraw[fill opacity=0.8,fill=gray!20,draw=none](-7.676,4.461)--(-7.681,4.463)--(-7.718,4.472)--(-7.701,4.466)--cycle;
\draw(-7.676,4.461)--(-7.681,4.463)--(-7.718,4.472)--(-7.701,4.466);
\filldraw[fill opacity=0.8,fill=gray!20](-7.781,1.238)--(-7.783,1.269)--(-7.719,1.264)--(-7.703,1.232)--cycle;
\filldraw[fill opacity=0.8,fill=gray!20,draw=none](-7.657,4.884)--(-7.653,4.883)--(-7.62,4.883)--(-7.62,4.883)--(-7.646,4.887)--cycle;
\draw(-7.653,4.883)--(-7.62,4.883)--(-7.62,4.883)--(-7.646,4.887)--(-7.657,4.884);
\filldraw[fill opacity=0.8,fill=gray!20,draw=none](-7.65,1.182)--(-7.651,1.185)--(-7.65,1.185)--cycle;
\draw(-7.651,1.185)--(-7.65,1.185);
\filldraw[fill opacity=0.8,fill=gray!20,draw=none](-7.641,1.198)--(-7.65,1.185)--(-7.651,1.185)--(-7.648,1.217)--cycle;
\draw(-7.65,1.185)--(-7.651,1.185);
\filldraw[fill opacity=0.8,fill=gray!20,draw=none](-7.637,1.184)--(-7.65,1.185)--(-7.641,1.198)--cycle;
\draw(-7.637,1.184)--(-7.65,1.185);
\filldraw[fill opacity=0.8,fill=gray!20,draw=none](-7.622,1.184)--(-7.637,1.184)--(-7.641,1.198)--(-7.637,1.204)--cycle;
\draw(-7.622,1.184)--(-7.637,1.184);
\filldraw[fill opacity=0.8,fill=gray!20,draw=none](-7.637,1.204)--(-7.641,1.198)--(-7.648,1.217)--(-7.648,1.218)--cycle;
\filldraw[fill opacity=0.8,fill=gray!20](-7.693,1.193)--(-7.703,1.232)--(-7.649,1.219)--(-7.632,1.178)--cycle;
\filldraw[fill opacity=0.8,fill=gray!20](-7.74,.976)--(-7.719,1.012)--(-7.675,1.001)--(-7.709,.969)--cycle;
\filldraw[fill opacity=0.8,fill=gray!20](-7.889,4.041)--(-7.936,4.057)--(-7.931,4.051)--(-7.878,4.029)--cycle;
\filldraw[fill opacity=0.8,fill=gray!20,draw=none](-6.778,.518)--(-6.274,.512)--(-6.252,.513)--(-6.25,.513)--(-6.754,.519)--cycle;
\draw(-6.778,.518)--(-6.274,.512);
\draw(-6.25,.513)--(-6.754,.519);
\filldraw[fill opacity=0.8,fill=gray!20](-7.148,.37)--(-7.141,.426)--(-7.163,.449)--(-7.171,.394)--cycle;
\filldraw[fill opacity=0.8,fill=gray!20,draw=none](-6.956,.28)--(-6.785,.278)--(-6.782,.291)--(-6.785,.308)--(-6.788,.314)--(-6.958,.315)--cycle;
\draw(-6.788,.314)--(-6.958,.315)--(-6.956,.28)--(-6.785,.278);
\filldraw[fill opacity=0.8,fill=gray!20](-6.762,.281)--(-6.74,.333)--(-6.775,.31)--(-6.793,.261)--cycle;
\filldraw[fill opacity=0.8,fill=gray!20](-8.351,3.24)--(-8.354,3.235)--(-8.354,3.235)--(-8.323,3.238)--cycle;
\filldraw[fill opacity=0.8,fill=gray!20](-8.38,3.239)--(-8.354,3.235)--(-8.354,3.235)--(-8.351,3.24)--cycle;
\filldraw[fill opacity=0.8,fill=gray!20,draw=none](-7.499,4.588)--(-7.501,4.583)--(-7.501,4.587)--cycle;
\draw(-7.501,4.583)--(-7.501,4.587);
\filldraw[fill opacity=0.8,fill=gray!20,draw=none](-7.943,.315)--(-7.934,.297)--(-7.932,.295)--(-7.942,.321)--cycle;
\draw(-7.934,.297)--(-7.932,.295)--(-7.942,.321);
\filldraw[fill opacity=0.8,fill=gray!20](-7.862,.25)--(-7.878,.282)--(-7.932,.295)--(-7.906,.261)--cycle;
\filldraw[fill opacity=0.8,fill=gray!20,draw=none](-7.673,.287)--(-7.667,.28)--(-7.658,.292)--(-7.673,.289)--cycle;
\draw(-7.667,.28)--(-7.658,.292)--(-7.673,.289);
\filldraw[fill opacity=0.8,fill=gray!20,draw=none](-7.644,.287)--(-7.663,.287)--(-7.671,.269)--(-7.667,.269)--cycle;
\draw(-7.644,.287)--(-7.663,.287);
\draw(-7.671,.269)--(-7.667,.269);
\filldraw[fill opacity=0.8,fill=gray!20,draw=none](-7.129,.282)--(-7.644,.287)--(-7.667,.269)--(-7.126,.263)--cycle;
\draw(-7.129,.282)--(-7.644,.287);
\draw(-7.667,.269)--(-7.126,.263);
\filldraw[fill opacity=0.8,fill=gray!20,draw=none](-8.185,3.06)--(-8.181,3.049)--(-8.198,3.056)--(-8.193,3.071)--(-8.192,3.07)--cycle;
\draw(-8.193,3.071)--(-8.192,3.07);
\filldraw[fill opacity=0.8,fill=gray!20,draw=none](-8.199,3.061)--(-8.198,3.061)--(-8.177,3.007)--(-8.211,3.022)--cycle;
\draw(-8.199,3.061)--(-8.198,3.061);
\draw(-8.177,3.007)--(-8.211,3.022);
\filldraw[fill opacity=0.8,fill=gray!20,draw=none](-8.181,3.049)--(-8.182,3.058)--(-8.185,3.06)--cycle;
\draw(-8.182,3.058)--(-8.185,3.06);
\filldraw[fill opacity=0.8,fill=gray!20,draw=none](-8.161,3.05)--(-8.175,3.046)--(-8.178,3.047)--(-8.192,3.07)--(-8.164,3.058)--cycle;
\draw(-8.192,3.07)--(-8.164,3.058);
\filldraw[fill opacity=0.8,fill=gray!20,draw=none](-8.198,3.062)--(-8.18,3.038)--(-8.181,3.049)--(-8.185,3.06)--(-8.197,3.065)--cycle;
\draw(-8.185,3.06)--(-8.197,3.065);
\filldraw[fill opacity=0.8,fill=gray!20,draw=none](-8.185,3.06)--(-8.178,3.047)--(-8.181,3.049)--cycle;
\filldraw[fill opacity=0.8,fill=gray!20,draw=none](-8.181,3.049)--(-8.177,3.007)--(-8.175,3.003)--(-8.161,2.997)--cycle;
\draw(-8.175,3.003)--(-8.161,2.997);
\filldraw[fill opacity=0.8,fill=gray!20,draw=none](-8.198,3.062)--(-8.177,3.007)--(-8.18,3.038)--cycle;
\filldraw[fill opacity=0.8,fill=gray!20,draw=none](-8.161,2.966)--(-8.166,3.01)--(-8.155,3.005)--cycle;
\draw(-8.166,3.01)--(-8.155,3.005);
\filldraw[fill opacity=0.8,fill=gray!20,draw=none](-8.148,3.034)--(-8.175,3.046)--(-8.152,3.053)--cycle;
\draw(-8.152,3.053)--(-8.148,3.034);
\filldraw[fill opacity=0.8,fill=gray!20,draw=none](-8.17,3.049)--(-8.161,3)--(-8.155,3.005)--(-8.158,3.023)--cycle;
\draw(-8.161,3)--(-8.155,3.005);
\filldraw[fill opacity=0.8,fill=gray!20,draw=none](-8.17,3.049)--(-8.158,3.023)--(-8.164,3.059)--(-8.171,3.055)--cycle;
\draw(-8.164,3.059)--(-8.171,3.055);
\filldraw[fill opacity=0.8,fill=gray!20,draw=none](-8.17,3.049)--(-8.169,3.048)--(-8.161,3.026)--(-8.161,3)--cycle;
\draw(-8.17,3.049)--(-8.169,3.048);
\filldraw[fill opacity=0.8,fill=gray!20,draw=none](-8.161,2.95)--(-8.159,2.951)--(-8.154,2.972)--(-8.155,3.005)--(-8.161,3)--cycle;
\draw(-8.161,2.95)--(-8.159,2.951);
\draw(-8.155,3.005)--(-8.161,3);
\filldraw[fill opacity=0.8,fill=gray!20,draw=none](-8.159,2.951)--(-8.154,2.955)--(-8.154,2.972)--cycle;
\draw(-8.159,2.951)--(-8.154,2.955);
\filldraw[fill opacity=0.8,fill=gray!20,draw=none](-8.159,2.951)--(-8.139,2.988)--(-8.149,2.992)--cycle;
\draw(-8.139,2.988)--(-8.149,2.992);
\filldraw[fill opacity=0.8,fill=gray!20,draw=none](-8.149,2.992)--(-8.154,2.981)--(-8.154,2.972)--cycle;
\filldraw[fill opacity=0.8,fill=gray!20,draw=none](-8.182,2.945)--(-8.17,2.94)--(-8.165,2.999)--(-8.177,3.004)--cycle;
\draw(-8.182,2.945)--(-8.17,2.94);
\draw(-8.165,2.999)--(-8.177,3.004);
\filldraw[fill opacity=0.8,fill=gray!20,draw=none](-8.165,2.999)--(-8.161,2.966)--(-8.164,2.949)--(-8.169,2.951)--cycle;
\draw(-8.164,2.949)--(-8.169,2.951);
\filldraw[fill opacity=0.8,fill=gray!20,draw=none](-8.173,2.941)--(-8.169,2.951)--(-8.167,2.95)--cycle;
\draw(-8.169,2.951)--(-8.167,2.95);
\filldraw[fill opacity=0.8,fill=gray!20,draw=none](-8.159,2.978)--(-8.155,3.005)--(-8.152,3.004)--cycle;
\draw(-8.155,3.005)--(-8.152,3.004);
\filldraw[fill opacity=0.8,fill=gray!20,draw=none](-8.151,2.983)--(-8.157,2.985)--(-8.152,3.004)--(-8.145,3.001)--cycle;
\draw(-8.152,3.004)--(-8.145,3.001);
\filldraw[fill opacity=0.8,fill=gray!20,draw=none](-8.153,2.909)--(-8.159,2.912)--(-8.14,2.981)--(-8.136,2.949)--(-8.142,2.928)--cycle;
\draw(-8.136,2.949)--(-8.142,2.928)--(-8.153,2.909);
\filldraw[fill opacity=0.8,fill=gray!20,draw=none](-8.181,3.086)--(-8.164,3.059)--(-8.143,3.073)--(-8.164,3.125)--(-8.189,3.109)--cycle;
\draw(-8.164,3.059)--(-8.143,3.073)--(-8.164,3.125)--(-8.189,3.109);
\filldraw[fill opacity=0.8,fill=gray!20,draw=none](-8.193,3.071)--(-8.191,3.083)--(-8.168,3.065)--(-8.164,3.058)--(-8.192,3.07)--cycle;
\draw(-8.164,3.058)--(-8.192,3.07);
\filldraw[fill opacity=0.8,fill=gray!20,draw=none](-8.164,3.046)--(-8.158,3.016)--(-8.148,2.998)--(-8.143,3.008)--(-8.145,3.017)--cycle;
\draw(-8.143,3.008)--(-8.145,3.017);
\filldraw[fill opacity=0.8,fill=gray!20,draw=none](-8.168,3.065)--(-8.153,3.053)--(-8.164,3.058)--cycle;
\draw(-8.153,3.053)--(-8.164,3.058);
\filldraw[fill opacity=0.8,fill=gray!20,draw=none](-8.152,3.052)--(-8.161,3.05)--(-8.164,3.058)--(-8.153,3.053)--cycle;
\draw(-8.164,3.058)--(-8.153,3.053);
\filldraw[fill opacity=0.8,fill=gray!20,draw=none](-8.168,3.065)--(-8.191,3.083)--(-8.19,3.09)--(-8.181,3.086)--cycle;
\filldraw[fill opacity=0.8,fill=gray!20,draw=none](-8.181,3.086)--(-8.171,3.055)--(-8.164,3.059)--cycle;
\draw(-8.171,3.055)--(-8.164,3.059);
\filldraw[fill opacity=0.8,fill=gray!20,draw=none](-8.168,3.065)--(-8.181,3.086)--(-8.17,3.081)--(-8.152,3.053)--(-8.153,3.053)--cycle;
\draw(-8.17,3.081)--(-8.152,3.053)--(-8.153,3.053);
\filldraw[fill opacity=0.8,fill=gray!20,draw=none](-8.164,3.046)--(-8.145,3.017)--(-8.152,3.053)--(-8.17,3.081)--cycle;
\draw(-8.145,3.017)--(-8.152,3.053)--(-8.17,3.081);
\filldraw[fill opacity=0.8,fill=gray!20,draw=none](-8.148,2.998)--(-8.153,3.007)--(-8.155,2.998)--(-8.153,2.989)--cycle;
\filldraw[fill opacity=0.8,fill=gray!20,draw=none](-8.153,3.007)--(-8.158,3.016)--(-8.155,2.998)--cycle;
\filldraw[fill opacity=0.8,fill=gray!20,draw=none](-8.162,3.035)--(-8.162,3.045)--(-8.143,3.037)--(-8.151,2.996)--(-8.155,2.998)--cycle;
\draw(-8.162,3.045)--(-8.143,3.037);
\draw(-8.151,2.996)--(-8.155,2.998);
\filldraw[fill opacity=0.8,fill=gray!20,draw=none](-8.169,3.048)--(-8.164,3.046)--(-8.162,3.035)--(-8.161,3.026)--cycle;
\draw(-8.169,3.048)--(-8.164,3.046);
\filldraw[fill opacity=0.8,fill=gray!20,draw=none](-8.126,2.982)--(-8.125,2.982)--(-8.12,3.031)--(-8.139,3.039)--cycle;
\draw(-8.126,2.982)--(-8.125,2.982)--(-8.12,3.031)--(-8.139,3.039);
\filldraw[fill opacity=0.8,fill=gray!20,draw=none](-8.139,3.039)--(-8.137,3.035)--(-8.137,3.03)--cycle;
\draw(-8.137,3.035)--(-8.137,3.03);
\filldraw[fill opacity=0.8,fill=gray!20,draw=none](-8.143,2.989)--(-8.126,2.982)--(-8.137,3.03)--cycle;
\draw(-8.143,2.989)--(-8.126,2.982);
\filldraw[fill opacity=0.8,fill=gray!20,draw=none](-8.164,2.944)--(-8.184,2.898)--(-8.171,2.906)--(-8.146,2.96)--(-8.159,2.952)--cycle;
\draw(-8.184,2.898)--(-8.171,2.906);
\draw(-8.146,2.96)--(-8.159,2.952);
\filldraw[fill opacity=0.8,fill=gray!20,draw=none](-8.146,2.96)--(-8.159,2.912)--(-8.167,2.915)--cycle;
\filldraw[fill opacity=0.8,fill=gray!20,draw=none](-8.166,2.938)--(-8.142,2.928)--(-8.125,2.982)--(-8.139,2.988)--cycle;
\draw(-8.166,2.938)--(-8.142,2.928)--(-8.125,2.982)--(-8.139,2.988);
\filldraw[fill opacity=0.8,fill=gray!20,draw=none](-8.223,2.902)--(-8.21,2.896)--(-8.173,2.922)--(-8.166,2.938)--(-8.17,2.94)--cycle;
\draw(-8.223,2.902)--(-8.21,2.896);
\draw(-8.166,2.938)--(-8.17,2.94);
\filldraw[fill opacity=0.8,fill=gray!20,draw=none](-8.199,3.061)--(-8.141,3.036)--(-8.204,3.019)--(-8.211,3.022)--cycle;
\draw(-8.199,3.061)--(-8.141,3.036);
\draw(-8.204,3.019)--(-8.211,3.022);
\filldraw[fill opacity=0.5,fill=gray!20](-10.9,.044)--(-10.74,.163)--(-10.888,.535)--(-11.066,.464)--cycle;
\filldraw[fill opacity=0.5,fill=gray!20](-10.74,.163)--(-10.567,.087)--(-10.715,.46)--(-10.888,.535)--cycle;
\filldraw[fill opacity=0.8,fill=gray!20](-7.408,4.61)--(-7.401,4.666)--(-7.437,4.642)--(-7.443,4.588)--cycle;
\filldraw[fill opacity=0.8,fill=gray!20,draw=none](-7.822,4.717)--(-7.801,4.727)--(-7.79,4.757)--(-7.81,4.778)--(-7.832,4.727)--cycle;
\draw(-7.801,4.727)--(-7.79,4.757)--(-7.81,4.778)--(-7.832,4.727)--(-7.822,4.717);
\filldraw[fill opacity=0.8,fill=gray!20](-7.589,4.886)--(-7.62,4.883)--(-7.62,4.883)--(-7.569,4.881)--cycle;
\filldraw[fill opacity=0.8,fill=gray!20,draw=none](-7.62,4.669)--(-7.716,4.712)--(-7.695,4.702)--(-7.653,4.684)--(-7.606,4.663)--(-7.561,4.643)--cycle;
\draw(-7.716,4.712)--(-7.695,4.702)--(-7.653,4.684)--(-7.606,4.663)--(-7.561,4.643);
\filldraw[fill opacity=0.8,fill=gray!20,draw=none](-7.561,4.643)--(-7.62,4.669)--(-7.716,4.702)--(-7.698,4.695)--(-7.655,4.678)--(-7.607,4.66)--cycle;
\draw(-7.716,4.702)--(-7.698,4.695)--(-7.655,4.678)--(-7.607,4.66)--(-7.561,4.643);
\filldraw[fill opacity=0.8,fill=gray!20,draw=none](-7.815,4.554)--(-7.807,4.57)--(-7.807,4.568)--(-7.813,4.553)--cycle;
\draw(-7.815,4.554)--(-7.807,4.57);
\filldraw[fill opacity=0.8,fill=gray!20](-8.035,4.058)--(-7.987,4.059)--(-7.987,4.059)--(-8.013,4.063)--cycle;
\filldraw[fill opacity=0.8,fill=gray!20,draw=none](-7.665,.522)--(-7.665,.527)--(-7.744,.528)--(-7.669,.512)--cycle;
\draw(-7.665,.527)--(-7.744,.528);
\filldraw[fill opacity=0.8,fill=gray!20](-7.862,.503)--(-7.841,.538)--(-7.872,.546)--(-7.906,.514)--cycle;
\filldraw[fill opacity=0.8,fill=gray!20,draw=none](-7.746,.482)--(-7.713,.468)--(-7.721,.502)--(-7.758,.518)--cycle;
\draw(-7.746,.482)--(-7.713,.468)--(-7.721,.502)--(-7.758,.518);
\filldraw[fill opacity=0.8,fill=gray!20,draw=none](-7.669,.512)--(-7.739,.527)--(-7.749,.514)--(-7.742,.507)--(-7.671,.506)--cycle;
\draw(-7.742,.507)--(-7.671,.506);
\filldraw[fill opacity=0.8,fill=gray!20,draw=none](-7.667,.485)--(-7.671,.506)--(-7.742,.507)--(-7.687,.469)--(-7.673,.469)--cycle;
\draw(-7.671,.506)--(-7.742,.507);
\draw(-7.687,.469)--(-7.673,.469);
\filldraw[fill opacity=0.8,fill=gray!20](-7.632,.487)--(-7.661,.525)--(-7.682,.511)--(-7.658,.47)--cycle;
\filldraw[fill opacity=0.8,fill=gray!20,draw=none](-5.974,.271)--(-6.003,.284)--(-5.978,.335)--(-5.967,.33)--cycle;
\draw(-5.974,.271)--(-6.003,.284);
\draw(-5.978,.335)--(-5.967,.33);
\filldraw[fill opacity=0.8,fill=gray!20,draw=none](-5.967,.33)--(-5.978,.335)--(-5.972,.352)--cycle;
\draw(-5.967,.33)--(-5.978,.335);
\filldraw[fill opacity=0.8,fill=gray!20,draw=none](-5.97,.384)--(-5.968,.203)--(-5.967,.206)--(-5.967,.376)--cycle;
\draw(-5.968,.203)--(-5.967,.206)--(-5.967,.376);
\filldraw[fill opacity=0.8,fill=gray!20,draw=none](-7.68,4.875)--(-7.689,4.873)--(-7.741,4.755)--(-7.717,4.766)--(-7.672,4.87)--cycle;
\draw(-7.689,4.873)--(-7.741,4.755);
\draw(-7.717,4.766)--(-7.672,4.87);
\filldraw[fill opacity=0.8,fill=gray!20,draw=none](-7.693,4.87)--(-7.677,4.876)--(-7.668,4.882)--(-7.687,4.876)--cycle;
\draw(-7.693,4.87)--(-7.677,4.876)--(-7.668,4.882)--(-7.687,4.876);
\filldraw[fill opacity=0.8,fill=gray!20](-8.256,3.217)--(-8.303,3.233)--(-8.298,3.227)--(-8.245,3.205)--cycle;
\filldraw[fill opacity=0.8,fill=gray!20,draw=none](-7.474,4.509)--(-7.491,4.498)--(-7.505,4.488)--cycle;
\draw(-7.474,4.509)--(-7.491,4.498)--(-7.505,4.488);
\filldraw[fill opacity=0.8,fill=gray!20,draw=none](-7.739,.527)--(-7.744,.528)--(-7.763,.528)--(-7.749,.514)--cycle;
\draw(-7.744,.528)--(-7.763,.528);
\filldraw[fill opacity=0.8,fill=gray!20,draw=none](-7.592,4.341)--(-7.585,4.364)--(-7.583,4.371)--(-7.592,4.347)--cycle;
\draw(-7.583,4.371)--(-7.592,4.347);
\filldraw[fill opacity=0.8,fill=gray!20](-6.977,.609)--(-6.952,.605)--(-6.952,.605)--(-6.948,.61)--cycle;
\filldraw[fill opacity=0.8,fill=gray!20](-6.948,.61)--(-6.952,.605)--(-6.952,.605)--(-6.92,.608)--cycle;
\filldraw[fill opacity=0.8,fill=gray!20,draw=none](-8.524,3.162)--(-8.506,3.176)--(-8.475,3.217)--(-8.506,3.199)--(-8.508,3.196)--cycle;
\draw(-8.475,3.217)--(-8.506,3.199)--(-8.508,3.196);
\filldraw[fill opacity=0.8,fill=gray!20,draw=none](-7.65,1.09)--(-7.651,1.087)--(-7.769,1.091)--(-7.767,1.143)--(-7.65,1.138)--cycle;
\draw(-7.651,1.087)--(-7.769,1.091)--(-7.767,1.143)--(-7.65,1.138);
\filldraw[fill opacity=0.8,fill=gray!20,draw=none](-7.65,1.038)--(-7.649,1.04)--(-7.648,1.04)--cycle;
\draw(-7.649,1.04)--(-7.648,1.04);
\filldraw[fill opacity=0.8,fill=gray!20,draw=none](-7.65,1.038)--(-7.673,.999)--(-7.675,1.001)--(-7.649,1.04)--cycle;
\draw(-7.673,.999)--(-7.675,1.001)--(-7.649,1.04);
\filldraw[fill opacity=0.8,fill=gray!20,draw=none](-7.648,1.038)--(-7.774,1.043)--(-7.769,1.091)--(-7.651,1.087)--cycle;
\draw(-7.648,1.038)--(-7.774,1.043)--(-7.769,1.091)--(-7.651,1.087);
\filldraw[fill opacity=0.8,fill=gray!20,draw=none](-7.786,.995)--(-7.801,1.007)--(-7.823,1.043)--(-7.84,1.09)--(-7.849,1.14)--(-7.85,1.186)--(-7.841,1.222)--(-7.824,1.241)--(-7.804,1.241)--(-7.794,1.234)--cycle;
\draw(-7.786,.995)--(-7.801,1.007)--(-7.823,1.043)--(-7.84,1.09)--(-7.849,1.14)--(-7.85,1.186)--(-7.841,1.222)--(-7.824,1.241)--(-7.804,1.241);
\filldraw[fill opacity=0.8,fill=gray!20,draw=none](-7.914,.993)--(-7.938,1.034)--(-7.923,1.044)--(-7.898,1.004)--cycle;
\draw(-7.938,1.034)--(-7.923,1.044)--(-7.898,1.004)--(-7.914,.993);
\filldraw[fill opacity=0.8,fill=gray!20,draw=none](-7.882,.961)--(-7.917,.988)--(-7.917,.992)--(-7.898,1.004)--(-7.867,.971)--cycle;
\draw(-7.917,.992)--(-7.898,1.004)--(-7.867,.971)--(-7.882,.961)--(-7.917,.988);
\filldraw[fill opacity=0.8,fill=gray!20,draw=none](-7.932,1.038)--(-7.951,1.081)--(-7.938,1.089)--(-7.923,1.044)--cycle;
\draw(-7.951,1.081)--(-7.938,1.089)--(-7.923,1.044)--(-7.932,1.038);
\filldraw[fill opacity=0.8,fill=gray!20,draw=none](-7.964,1.114)--(-7.953,1.117)--(-7.951,1.081)--cycle;
\draw(-7.964,1.114)--(-7.953,1.117);
\filldraw[fill opacity=0.8,fill=gray!20,draw=none](-7.956,1.128)--(-7.953,1.117)--(-7.957,1.116)--cycle;
\draw(-7.953,1.117)--(-7.957,1.116);
\filldraw[fill opacity=0.8,fill=gray!20](-7.967,1.07)--(-7.973,1.117)--(-7.943,1.136)--(-7.938,1.089)--cycle;
\filldraw[fill opacity=0.8,fill=gray!20,draw=none](-7.953,1.117)--(-7.849,1.14)--(-7.84,1.09)--(-7.949,1.065)--cycle;
\draw(-7.953,1.117)--(-7.849,1.14)--(-7.84,1.09)--(-7.949,1.065);
\filldraw[fill opacity=0.8,fill=gray!20,draw=none](-8.569,.89)--(-8.617,.915)--(-7.944,1.066)--(-7.935,1.018)--(-8.475,.897)--cycle;
\draw(-8.617,.915)--(-7.944,1.066);
\draw(-7.935,1.018)--(-8.475,.897);
\filldraw[fill opacity=0.8,fill=gray!20,draw=none](-7.932,1.038)--(-7.949,1.027)--(-7.967,1.07)--(-7.951,1.081)--cycle;
\draw(-7.932,1.038)--(-7.949,1.027)--(-7.967,1.07)--(-7.951,1.081);
\filldraw[fill opacity=0.8,fill=gray!20,draw=none](-7.944,1.066)--(-7.84,1.09)--(-7.823,1.043)--(-7.935,1.018)--cycle;
\draw(-7.944,1.066)--(-7.84,1.09)--(-7.823,1.043)--(-7.935,1.018);
\filldraw[fill opacity=0.8,fill=gray!20,draw=none](-7.929,1.019)--(-7.914,.993)--(-7.92,.99)--(-7.924,.995)--cycle;
\draw(-7.914,.993)--(-7.92,.99)--(-7.924,.995);
\filldraw[fill opacity=0.8,fill=gray!20,draw=none](-7.917,.988)--(-7.92,.99)--(-7.917,.992)--cycle;
\draw(-7.917,.988)--(-7.92,.99)--(-7.917,.992);
\filldraw[fill opacity=0.8,fill=gray!20,draw=none](-7.918,.988)--(-7.924,.995)--(-7.929,1.019)--(-7.823,1.043)--(-7.801,1.007)--(-7.911,.983)--cycle;
\draw(-7.929,1.019)--(-7.823,1.043)--(-7.801,1.007)--(-7.911,.983);
\filldraw[fill opacity=0.8,fill=gray!20,draw=none](-7.889,.963)--(-7.917,.988)--(-7.89,.968)--cycle;
\draw(-7.917,.988)--(-7.89,.968);
\filldraw[fill opacity=0.8,fill=gray!20,draw=none](-7.904,.976)--(-7.911,.983)--(-7.801,1.007)--(-7.788,.996)--cycle;
\draw(-7.911,.983)--(-7.801,1.007)--(-7.788,.996);
\filldraw[fill opacity=0.8,fill=gray!20](-7.812,.951)--(-7.831,.977)--(-7.785,.98)--(-7.788,.952)--cycle;
\filldraw[fill opacity=0.8,fill=gray!20](-7.83,.947)--(-7.867,.971)--(-7.831,.977)--(-7.812,.951)--cycle;
\filldraw[fill opacity=0.8,fill=gray!20](-7.838,.942)--(-7.882,.961)--(-7.867,.971)--(-7.83,.947)--cycle;
\filldraw[fill opacity=0.8,fill=gray!20,draw=none](-7.872,.951)--(-7.889,.963)--(-7.89,.968)--(-7.882,.961)--cycle;
\draw(-7.89,.968)--(-7.882,.961)--(-7.872,.951)--(-7.889,.963);
\filldraw[fill opacity=0.8,fill=gray!20,draw=none](-7.92,.974)--(-7.788,.996)--(-7.778,.988)--(-7.889,.963)--cycle;
\draw(-7.788,.996)--(-7.778,.988)--(-7.889,.963);
\filldraw[fill opacity=0.8,fill=gray!20](-7.812,1.137)--(-7.807,1.185)--(-7.799,1.222)--(-7.79,1.243)--(-7.781,1.244)--(-7.774,1.225)--(-7.769,1.19)--(-7.767,1.143)--(-7.769,1.091)--(-7.774,1.043)--(-7.781,1.006)--(-7.79,.986)--(-7.799,.985)--(-7.807,1.003)--(-7.812,1.039)--(-7.814,1.086)--cycle;
\filldraw[fill opacity=0.8,fill=gray!20,draw=none](-8.569,3.06)--(-8.574,3.04)--(-8.562,3.011)--cycle;
\filldraw[fill opacity=0.8,fill=gray!20,draw=none](-7.499,4.587)--(-7.499,4.595)--(-7.515,4.557)--(-7.511,4.56)--cycle;
\draw(-7.499,4.595)--(-7.515,4.557);
\filldraw[fill opacity=0.8,fill=gray!20,draw=none](-8.554,2.955)--(-8.544,2.944)--(-8.55,2.99)--(-8.567,2.993)--cycle;
\draw(-8.554,2.955)--(-8.544,2.944)--(-8.55,2.99);
\filldraw[fill opacity=0.8,fill=gray!20](-7.991,3.625)--(-7.994,3.635)--(-8.048,3.639)--(-8.019,3.627)--cycle;
\filldraw[fill opacity=0.8,fill=gray!20](-8.177,3.879)--(-8.157,3.933)--(-8.177,3.954)--(-8.199,3.903)--cycle;
\filldraw[fill opacity=0.8,fill=gray!20](-7.776,3.786)--(-7.768,3.842)--(-7.804,3.818)--(-7.81,3.764)--cycle;
\filldraw[fill opacity=0.8,fill=gray!20](-7.956,4.062)--(-7.987,4.059)--(-7.987,4.059)--(-7.936,4.057)--cycle;
\filldraw[fill opacity=0.8,fill=gray!20,draw=none](-8.131,3.042)--(-8.128,3.034)--(-8.12,3.031)--(-8.121,3.036)--cycle;
\draw(-8.128,3.034)--(-8.12,3.031)--(-8.121,3.036);
\filldraw[fill opacity=0.8,fill=gray!20,draw=none](-7.65,1.182)--(-7.65,1.138)--(-7.767,1.143)--(-7.769,1.19)--(-7.651,1.185)--cycle;
\draw(-7.65,1.138)--(-7.767,1.143)--(-7.769,1.19)--(-7.651,1.185);
\filldraw[fill opacity=0.8,fill=gray!20](-6.853,.587)--(-6.901,.604)--(-6.895,.597)--(-6.842,.575)--cycle;
\filldraw[fill opacity=0.8,fill=gray!20,draw=none](-7.742,4.733)--(-7.747,4.735)--(-7.748,4.735)--cycle;
\draw(-7.747,4.735)--(-7.748,4.735);
\filldraw[fill opacity=0.8,fill=gray!20,draw=none](-7.751,4.728)--(-7.748,4.735)--(-7.747,4.735)--cycle;
\draw(-7.751,4.728)--(-7.748,4.735)--(-7.747,4.735);
\filldraw[fill opacity=0.8,fill=gray!20,draw=none](-7.843,.571)--(-7.843,.571)--(-7.842,.571)--cycle;
\draw(-7.843,.571)--(-7.842,.571);
\filldraw[fill opacity=0.8,fill=gray!20,draw=none](-7.665,.522)--(-7.663,.526)--(-7.665,.527)--cycle;
\filldraw[fill opacity=0.8,fill=gray!20,draw=none](-7.127,.485)--(-7.121,.501)--(-7.124,.522)--(-7.141,.501)--cycle;
\draw(-7.124,.522)--(-7.141,.501)--(-7.127,.485);
\filldraw[fill opacity=0.8,fill=gray!20,draw=none](-6.952,.52)--(-7.665,.527)--(-7.641,.506)--(-6.954,.499)--cycle;
\draw(-7.641,.506)--(-6.954,.499)--(-6.952,.52)--(-7.665,.527);
\filldraw[fill opacity=0.8,fill=gray!20,draw=none](-7.84,.564)--(-7.833,.568)--(-7.838,.573)--(-7.843,.571)--cycle;
\draw(-7.84,.564)--(-7.833,.568)--(-7.838,.573)--(-7.843,.571);
\filldraw[fill opacity=0.8,fill=gray!20,draw=none](-7.675,4.493)--(-7.673,4.467)--(-7.697,4.48)--(-7.712,4.49)--(-7.714,4.51)--cycle;
\draw(-7.675,4.493)--(-7.673,4.467);
\draw(-7.712,4.49)--(-7.714,4.51);
\filldraw[fill opacity=0.8,fill=gray!20](-7.681,4.463)--(-7.706,4.489)--(-7.759,4.502)--(-7.718,4.472)--cycle;
\filldraw[fill opacity=0.8,fill=gray!20,draw=none](-7.517,4.538)--(-7.507,4.558)--(-7.511,4.555)--(-7.52,4.532)--cycle;
\draw(-7.511,4.555)--(-7.52,4.532);
\filldraw[fill opacity=0.8,fill=gray!20,draw=none](-7.657,4.884)--(-7.668,4.882)--(-7.653,4.883)--cycle;
\draw(-7.657,4.884)--(-7.668,4.882)--(-7.653,4.883);
\filldraw[fill opacity=0.8,fill=gray!20,draw=none](-7.74,4.755)--(-7.753,4.727)--(-7.755,4.724)--(-7.741,4.755)--cycle;
\draw(-7.755,4.724)--(-7.741,4.755);
\filldraw[fill opacity=0.8,fill=gray!20,draw=none](-7.665,.522)--(-7.661,.525)--(-7.664,.527)--cycle;
\draw(-7.665,.522)--(-7.661,.525)--(-7.664,.527);
\filldraw[fill opacity=0.8,fill=gray!20](-7.833,.568)--(-7.79,.579)--(-7.79,.579)--(-7.838,.573)--cycle;
\filldraw[fill opacity=0.8,fill=gray!20,draw=none](-7.893,3.803)--(-7.962,3.647)--(-7.952,3.637)--(-7.944,3.635)--(-7.942,3.635)--(-7.871,3.793)--cycle;
\draw(-7.942,3.635)--(-7.871,3.793)--(-7.893,3.803)--(-7.962,3.647);
\filldraw[fill opacity=0.8,fill=gray!20](-7.962,3.626)--(-7.938,3.638)--(-7.994,3.635)--(-7.991,3.625)--cycle;
\filldraw[fill opacity=0.8,fill=gray!20](-7.788,.952)--(-7.785,.98)--(-7.74,.976)--(-7.764,.95)--cycle;
\filldraw[fill opacity=0.8,fill=gray!20,draw=none](-7.665,.522)--(-7.664,.527)--(-7.699,.553)--(-7.714,.544)--(-7.682,.511)--cycle;
\draw(-7.664,.527)--(-7.699,.553)--(-7.714,.544)--(-7.682,.511)--(-7.665,.522);
\filldraw[fill opacity=0.8,fill=gray!20,draw=none](-7.663,4.506)--(-7.663,4.506)--(-7.667,4.507)--(-7.673,4.509)--cycle;
\filldraw[fill opacity=0.8,fill=gray!20](-8.544,3.055)--(-8.524,3.109)--(-8.544,3.13)--(-8.566,3.079)--cycle;
\filldraw[fill opacity=0.8,fill=gray!20](-8.358,2.801)--(-8.361,2.811)--(-8.415,2.815)--(-8.386,2.803)--cycle;
\filldraw[fill opacity=0.8,fill=gray!20](-8.323,3.238)--(-8.354,3.235)--(-8.354,3.235)--(-8.303,3.233)--cycle;
\filldraw[fill opacity=0.8,fill=gray!20,draw=none](-7.673,4.509)--(-7.667,4.507)--(-7.71,4.524)--cycle;
\filldraw[fill opacity=0.8,fill=gray!20](-6.999,.605)--(-6.952,.605)--(-6.952,.605)--(-6.977,.609)--cycle;
\filldraw[fill opacity=0.8,fill=gray!20,draw=none](-8.584,2.989)--(-8.577,2.998)--(-8.558,2.962)--(-8.58,2.971)--cycle;
\draw(-8.558,2.962)--(-8.58,2.971)--(-8.584,2.989);
\filldraw[fill opacity=0.8,fill=gray!20,draw=none](-7.483,4.631)--(-7.499,4.595)--(-7.499,4.588)--cycle;
\draw(-7.483,4.631)--(-7.499,4.595);
\filldraw[fill opacity=0.8,fill=gray!20,draw=none](-8.558,2.962)--(-8.556,2.961)--(-8.544,2.944)--cycle;
\draw(-8.558,2.962)--(-8.556,2.961);
\filldraw[fill opacity=0.8,fill=gray!20,draw=none](-8.545,2.938)--(-8.544,2.943)--(-8.544,2.944)--(-8.553,2.954)--(-8.554,2.947)--cycle;
\draw(-8.544,2.943)--(-8.544,2.944)--(-8.553,2.954);
\filldraw[fill opacity=0.8,fill=gray!20,draw=none](-8.58,2.971)--(-8.558,2.962)--(-8.544,2.944)--(-8.559,2.95)--cycle;
\draw(-8.544,2.944)--(-8.559,2.95)--(-8.58,2.971)--(-8.558,2.962);
\filldraw[fill opacity=0.8,fill=gray!20,draw=none](-8.562,3.011)--(-8.574,3.04)--(-8.586,2.998)--(-8.58,2.971)--(-8.559,2.95)--(-8.554,2.95)--cycle;
\draw(-8.586,2.998)--(-8.58,2.971)--(-8.559,2.95)--(-8.554,2.95);
\filldraw[fill opacity=0.8,fill=gray!20,draw=none](-7.82,4.502)--(-7.8,4.547)--(-7.815,4.554)--(-7.835,4.508)--cycle;
\draw(-7.82,4.502)--(-7.8,4.547);
\draw(-7.815,4.554)--(-7.835,4.508);
\filldraw[fill opacity=0.8,fill=gray!20,draw=none](-7.865,4.387)--(-7.802,4.548)--(-7.815,4.554)--(-7.854,4.463)--(-7.881,4.394)--cycle;
\draw(-7.854,4.463)--(-7.881,4.394)--(-7.865,4.387)--(-7.802,4.548);
\filldraw[fill opacity=0.8,fill=gray!20,draw=none](-8.25,2.845)--(-8.282,2.846)--(-8.285,2.839)--cycle;
\draw(-8.282,2.846)--(-8.285,2.839)--(-8.25,2.845);
\filldraw[fill opacity=0.8,fill=gray!20,draw=none](-8.311,2.837)--(-8.285,2.839)--(-8.282,2.846)--(-8.335,2.853)--cycle;
\draw(-8.311,2.837)--(-8.285,2.839)--(-8.282,2.846);
\filldraw[fill opacity=0.8,fill=gray!20,draw=none](-8.323,2.847)--(-8.393,2.878)--(-8.394,2.88)--(-8.351,2.87)--(-8.313,2.859)--(-8.272,2.842)--cycle;
\draw(-8.313,2.859)--(-8.272,2.842)--(-8.323,2.847)--(-8.393,2.878);
\filldraw[fill opacity=0.8,fill=gray!20,draw=none](-7.512,4.559)--(-7.511,4.56)--(-7.511,4.56)--cycle;
\filldraw[fill opacity=0.8,fill=gray!20,draw=none](-7.663,.526)--(-7.669,.512)--(-7.645,.506)--(-7.641,.506)--cycle;
\draw(-7.645,.506)--(-7.641,.506);
\filldraw[fill opacity=0.8,fill=gray!20,draw=none](-5.963,.37)--(-5.963,.371)--(-5.964,.369)--cycle;
\draw(-5.963,.371)--(-5.964,.369);
\filldraw[fill opacity=0.8,fill=gray!20,draw=none](-7.815,4.554)--(-7.818,4.555)--(-7.854,4.463)--cycle;
\draw(-7.818,4.555)--(-7.854,4.463);
\filldraw[fill opacity=0.8,fill=gray!20,draw=none](-7.521,4.54)--(-7.512,4.559)--(-7.511,4.56)--(-7.515,4.557)--(-7.525,4.534)--cycle;
\draw(-7.515,4.557)--(-7.525,4.534);
\filldraw[fill opacity=0.8,fill=gray!20](-8.048,3.639)--(-8.073,3.665)--(-8.126,3.678)--(-8.085,3.648)--cycle;
\filldraw[fill opacity=0.8,fill=gray!20,draw=none](-7.512,4.559)--(-7.521,4.54)--(-7.517,4.547)--cycle;
\filldraw[fill opacity=0.8,fill=gray!20,draw=none](-7.624,4.488)--(-7.72,4.526)--(-7.703,4.521)--(-7.667,4.507)--cycle;
\filldraw[fill opacity=0.8,fill=gray!20,draw=none](-7.419,4.771)--(-7.417,4.78)--(-7.418,4.78)--(-7.425,4.763)--(-7.425,4.761)--cycle;
\draw(-7.417,4.78)--(-7.418,4.78)--(-7.425,4.763);
\filldraw[fill opacity=0.8,fill=gray!20](-8.329,2.802)--(-8.305,2.813)--(-8.361,2.811)--(-8.358,2.801)--cycle;
\filldraw[fill opacity=0.8,fill=gray!20](-6.955,.172)--(-6.958,.182)--(-7.012,.186)--(-6.983,.174)--cycle;
\filldraw[fill opacity=0.8,fill=gray!20](-6.74,.333)--(-6.732,.388)--(-6.768,.365)--(-6.775,.31)--cycle;
\filldraw[fill opacity=0.8,fill=gray!20,draw=none](-6.954,.499)--(-7.121,.501)--(-7.127,.485)--(-7.129,.464)--(-6.956,.462)--cycle;
\draw(-7.129,.464)--(-6.956,.462)--(-6.954,.499)--(-7.121,.501);
\filldraw[fill opacity=0.8,fill=gray!20](-7.141,.426)--(-7.122,.48)--(-7.141,.501)--(-7.163,.449)--cycle;
\filldraw[fill opacity=0.8,fill=gray!20](-6.92,.608)--(-6.952,.605)--(-6.952,.605)--(-6.901,.604)--cycle;
\filldraw[fill opacity=0.8,fill=gray!20,draw=none](-8.502,3.179)--(-8.509,3.174)--(-8.506,3.177)--cycle;
\draw(-8.502,3.179)--(-8.509,3.174)--(-8.506,3.177);
\filldraw[fill opacity=0.8,fill=gray!20](-7.769,.563)--(-7.79,.579)--(-7.79,.579)--(-7.793,.562)--cycle;
\filldraw[fill opacity=0.8,fill=gray!20](-7.751,.567)--(-7.79,.579)--(-7.79,.579)--(-7.769,.563)--cycle;
\filldraw[fill opacity=0.8,fill=gray!20](-7.743,.572)--(-7.79,.579)--(-7.79,.579)--(-7.751,.567)--cycle;
\filldraw[fill opacity=0.8,fill=gray!20](-7.793,.562)--(-7.79,.579)--(-7.79,.579)--(-7.816,.564)--cycle;
\filldraw[fill opacity=0.8,fill=gray!20](-7.699,.553)--(-7.743,.572)--(-7.751,.567)--(-7.714,.544)--cycle;
\filldraw[fill opacity=0.8,fill=gray!20](-7.816,.564)--(-7.79,.579)--(-7.79,.579)--(-7.833,.568)--cycle;
\filldraw[fill opacity=0.8,fill=gray!20](-7.796,.225)--(-7.798,.246)--(-7.862,.25)--(-7.841,.229)--cycle;
\filldraw[fill opacity=0.8,fill=gray!20,draw=none](-7.509,4.557)--(-7.5,4.577)--(-7.497,4.544)--cycle;
\draw(-7.5,4.577)--(-7.497,4.544);
\filldraw[fill opacity=0.8,fill=gray!20,draw=none](-7.495,4.517)--(-7.504,4.497)--(-7.498,4.497)--(-7.491,4.498)--(-7.475,4.52)--(-7.477,4.536)--(-7.478,4.535)--cycle;
\draw(-7.498,4.497)--(-7.491,4.498)--(-7.475,4.52);
\draw(-7.477,4.536)--(-7.478,4.535);
\filldraw[fill opacity=0.8,fill=gray!20,draw=none](-7.495,4.517)--(-7.478,4.535)--(-7.487,4.534)--cycle;
\draw(-7.478,4.535)--(-7.487,4.534);
\filldraw[fill opacity=0.8,fill=gray!20,draw=none](-7.48,4.557)--(-7.487,4.534)--(-7.477,4.536)--cycle;
\draw(-7.487,4.534)--(-7.477,4.536);
\filldraw[fill opacity=0.8,fill=gray!20,draw=none](-7.496,4.59)--(-7.483,4.595)--(-7.475,4.511)--(-7.497,4.544)--(-7.5,4.577)--cycle;
\draw(-7.483,4.595)--(-7.475,4.511);
\draw(-7.497,4.544)--(-7.5,4.577);
\filldraw[fill opacity=0.8,fill=gray!20,draw=none](-8.452,3.212)--(-8.425,3.211)--(-8.419,3.208)--(-8.419,3.196)--(-8.454,3.212)--cycle;
\draw(-8.425,3.211)--(-8.419,3.208);
\draw(-8.419,3.196)--(-8.454,3.212);
\filldraw[fill opacity=0.8,fill=gray!20,draw=none](-8.415,3.186)--(-8.408,3.193)--(-8.425,3.211)--(-8.452,3.195)--cycle;
\draw(-8.425,3.211)--(-8.452,3.195)--(-8.415,3.186)--(-8.408,3.193);
\filldraw[fill opacity=0.8,fill=gray!20,draw=none](-8.369,3.174)--(-8.374,3.178)--(-8.382,3.159)--(-8.353,3.152)--cycle;
\draw(-8.374,3.178)--(-8.382,3.159);
\filldraw[fill opacity=0.8,fill=gray!20,draw=none](-8.405,3.185)--(-8.404,3.19)--(-8.408,3.193)--(-8.415,3.186)--cycle;
\draw(-8.408,3.193)--(-8.415,3.186)--(-8.405,3.185);
\filldraw[fill opacity=0.8,fill=gray!20,draw=none](-8.426,3.202)--(-8.422,3.197)--(-8.315,3.151)--(-8.303,3.158)--(-8.419,3.208)--cycle;
\draw(-8.426,3.202)--(-8.422,3.197)--(-8.315,3.151);
\draw(-8.303,3.158)--(-8.419,3.208);
\filldraw[fill opacity=0.8,fill=gray!20,draw=none](-7.815,4.554)--(-7.807,4.572)--(-7.81,4.576)--(-7.818,4.555)--cycle;
\draw(-7.81,4.576)--(-7.818,4.555);
\filldraw[fill opacity=0.8,fill=gray!20,draw=none](-6.252,.513)--(-6.246,.513)--(-6.25,.513)--cycle;
\draw(-6.246,.513)--(-6.25,.513);
\filldraw[fill opacity=0.8,fill=gray!20,draw=none](-6.25,.517)--(-6.241,.509)--(-6.252,.493)--(-6.252,.515)--cycle;
\draw(-6.252,.493)--(-6.252,.515);
\filldraw[fill opacity=0.8,fill=gray!20,draw=none](-7.673,.289)--(-7.678,.323)--(-7.756,.324)--(-7.794,.289)--(-7.673,.287)--cycle;
\draw(-7.678,.323)--(-7.756,.324);
\draw(-7.794,.289)--(-7.673,.287);
\filldraw[fill opacity=0.8,fill=gray!20,draw=none](-7.682,.258)--(-7.667,.28)--(-7.674,.289)--(-7.72,.28)--(-7.733,.248)--cycle;
\draw(-7.674,.289)--(-7.72,.28)--(-7.733,.248)--(-7.682,.258)--(-7.667,.28);
\filldraw[fill opacity=0.8,fill=gray!20](-7.923,1.223)--(-7.898,1.256)--(-7.848,1.266)--(-7.861,1.235)--cycle;
\filldraw[fill opacity=0.8,fill=gray!20,draw=none](-7.753,4.727)--(-7.762,4.707)--(-7.755,4.724)--cycle;
\draw(-7.762,4.707)--(-7.755,4.724);
\filldraw[fill opacity=0.8,fill=gray!20,draw=none](-8.171,2.906)--(-8.165,2.914)--(-8.153,2.909)--(-8.159,2.898)--(-8.184,2.879)--cycle;
\draw(-8.153,2.909)--(-8.159,2.898);
\filldraw[fill opacity=0.8,fill=gray!20](-7.79,4.757)--(-7.759,4.805)--(-7.775,4.822)--(-7.81,4.778)--cycle;
\filldraw[fill opacity=0.8,fill=gray!20](-7.401,4.666)--(-7.408,4.721)--(-7.443,4.699)--(-7.437,4.642)--cycle;
\filldraw[fill opacity=0.8,fill=gray!20,draw=none](-8.351,2.87)--(-8.32,2.862)--(-8.313,2.859)--cycle;
\draw(-8.32,2.862)--(-8.313,2.859);
\filldraw[fill opacity=0.8,fill=gray!20,draw=none](-8.351,2.87)--(-8.34,2.867)--(-8.344,2.868)--cycle;
\draw(-8.34,2.867)--(-8.344,2.868);
\filldraw[fill opacity=0.8,fill=gray!20,draw=none](-8.344,2.868)--(-8.326,2.865)--(-8.32,2.862)--cycle;
\draw(-8.326,2.865)--(-8.32,2.862);
\filldraw[fill opacity=0.8,fill=gray!20,draw=none](-8.218,2.892)--(-8.203,2.885)--(-8.235,2.864)--(-8.256,2.873)--cycle;
\draw(-8.235,2.864)--(-8.256,2.873);
\filldraw[fill opacity=0.8,fill=gray!20,draw=none](-8.269,2.87)--(-8.256,2.873)--(-8.253,2.872)--cycle;
\draw(-8.256,2.873)--(-8.253,2.872);
\filldraw[fill opacity=0.8,fill=gray!20,draw=none](-8.251,2.881)--(-8.234,2.874)--(-8.277,2.858)--(-8.318,2.876)--cycle;
\draw(-8.251,2.881)--(-8.234,2.874);
\draw(-8.277,2.858)--(-8.318,2.876);
\filldraw[fill opacity=0.8,fill=gray!20,draw=none](-8.251,2.881)--(-8.234,2.874)--(-8.277,2.858)--(-8.318,2.876)--cycle;
\draw(-8.251,2.881)--(-8.234,2.874);
\draw(-8.277,2.858)--(-8.318,2.876);
\filldraw[fill opacity=0.8,fill=gray!20,draw=none](-8.196,2.891)--(-8.203,2.885)--(-8.218,2.892)--(-8.198,2.902)--(-8.19,2.898)--cycle;
\draw(-8.198,2.902)--(-8.19,2.898);
\filldraw[fill opacity=0.8,fill=gray!20,draw=none](-8.372,2.885)--(-8.324,2.864)--(-8.338,2.873)--(-8.361,2.887)--(-8.406,2.907)--cycle;
\draw(-8.372,2.885)--(-8.324,2.864);
\draw(-8.361,2.887)--(-8.406,2.907);
\filldraw[fill opacity=0.8,fill=gray!20,draw=none](-8.306,2.852)--(-8.277,2.858)--(-8.272,2.868)--(-8.296,2.875)--(-8.338,2.873)--cycle;
\draw(-8.277,2.858)--(-8.272,2.868);
\draw(-8.296,2.875)--(-8.338,2.873);
\filldraw[fill opacity=0.8,fill=gray!20,draw=none](-8.277,2.858)--(-8.293,2.858)--(-8.318,2.861)--(-8.324,2.864)--cycle;
\draw(-8.318,2.861)--(-8.324,2.864);
\filldraw[fill opacity=0.8,fill=gray!20,draw=none](-8.277,2.858)--(-8.293,2.858)--(-8.318,2.861)--(-8.324,2.864)--cycle;
\draw(-8.318,2.861)--(-8.324,2.864);
\filldraw[fill opacity=0.8,fill=gray!20,draw=none](-8.372,2.885)--(-8.323,2.863)--(-8.361,2.887)--(-8.406,2.907)--cycle;
\draw(-8.372,2.885)--(-8.323,2.863);
\draw(-8.361,2.887)--(-8.406,2.907);
\filldraw[fill opacity=0.8,fill=gray!20,draw=none](-8.335,2.853)--(-8.317,2.85)--(-8.306,2.852)--(-8.338,2.873)--(-8.365,2.872)--cycle;
\draw(-8.338,2.873)--(-8.365,2.872);
\filldraw[fill opacity=0.8,fill=gray!20,draw=none](-8.258,2.884)--(-8.251,2.881)--(-8.288,2.878)--cycle;
\draw(-8.258,2.884)--(-8.251,2.881);
\filldraw[fill opacity=0.8,fill=gray!20,draw=none](-8.258,2.884)--(-8.251,2.881)--(-8.288,2.878)--cycle;
\draw(-8.258,2.884)--(-8.251,2.881);
\filldraw[fill opacity=0.8,fill=gray!20,draw=none](-8.265,2.92)--(-8.233,2.906)--(-8.207,2.956)--(-8.243,2.972)--cycle;
\draw(-8.265,2.92)--(-8.233,2.906);
\draw(-8.207,2.956)--(-8.243,2.972);
\filldraw[fill opacity=0.8,fill=gray!20,draw=none](-8.365,2.872)--(-8.338,2.873)--(-8.366,2.89)--(-8.366,2.873)--cycle;
\draw(-8.365,2.872)--(-8.338,2.873);
\draw(-8.366,2.89)--(-8.366,2.873);
\filldraw[fill opacity=0.8,fill=gray!20,draw=none](-8.375,2.878)--(-8.366,2.873)--(-8.366,2.882)--(-8.381,2.887)--cycle;
\draw(-8.366,2.873)--(-8.366,2.882);
\filldraw[fill opacity=0.8,fill=gray!20,draw=none](-8.365,2.874)--(-8.405,2.891)--(-8.412,2.889)--(-8.407,2.886)--(-8.393,2.878)--(-8.323,2.847)--cycle;
\draw(-8.393,2.878)--(-8.323,2.847)--(-8.365,2.874)--(-8.405,2.891);
\filldraw[fill opacity=0.8,fill=gray!20,draw=none](-7.672,4.504)--(-7.669,4.49)--(-7.675,4.493)--(-7.676,4.506)--cycle;
\draw(-7.675,4.493)--(-7.676,4.506);
\filldraw[fill opacity=0.8,fill=gray!20,draw=none](-7.742,4.733)--(-7.748,4.735)--(-7.748,4.735)--(-7.732,4.728)--(-7.726,4.725)--cycle;
\draw(-7.748,4.735)--(-7.748,4.735)--(-7.732,4.728)--(-7.726,4.725);
\filldraw[fill opacity=0.8,fill=gray!20,draw=none](-5.97,.384)--(-5.974,.394)--(-5.974,.193)--(-5.968,.203)--cycle;
\draw(-5.974,.394)--(-5.974,.193)--(-5.968,.203);
\filldraw[fill opacity=0.8,fill=gray!20,draw=none](-8.198,3.061)--(-8.17,3.049)--(-8.161,3)--(-8.177,3.007)--cycle;
\draw(-8.198,3.061)--(-8.17,3.049);
\draw(-8.161,3)--(-8.177,3.007);
\filldraw[fill opacity=0.8,fill=gray!20,draw=none](-8.164,2.944)--(-8.159,2.952)--(-8.162,2.95)--cycle;
\draw(-8.159,2.952)--(-8.162,2.95);
\filldraw[fill opacity=0.8,fill=gray!20,draw=none](-8.161,2.959)--(-8.162,2.95)--(-8.161,2.95)--(-8.161,2.959)--cycle;
\draw(-8.162,2.95)--(-8.161,2.95);
\filldraw[fill opacity=0.8,fill=gray!20,draw=none](-8.178,2.934)--(-8.173,2.931)--(-8.179,2.92)--(-8.198,2.902)--(-8.2,2.903)--cycle;
\draw(-8.198,2.902)--(-8.2,2.903);
\filldraw[fill opacity=0.8,fill=gray!20,draw=none](-8.209,2.912)--(-8.17,2.94)--(-8.185,2.946)--cycle;
\draw(-8.17,2.94)--(-8.185,2.946);
\filldraw[fill opacity=0.8,fill=gray!20,draw=none](-8.179,2.92)--(-8.19,2.898)--(-8.198,2.902)--cycle;
\draw(-8.19,2.898)--(-8.198,2.902);
\filldraw[fill opacity=0.8,fill=gray!20,draw=none](-8.173,2.941)--(-8.177,2.933)--(-8.178,2.934)--cycle;
\filldraw[fill opacity=0.8,fill=gray!20,draw=none](-8.173,2.931)--(-8.177,2.933)--(-8.173,2.941)--(-8.167,2.95)--(-8.164,2.949)--cycle;
\draw(-8.167,2.95)--(-8.164,2.949);
\filldraw[fill opacity=0.8,fill=gray!20,draw=none](-8.243,2.972)--(-8.182,2.945)--(-8.177,3.004)--(-8.219,3.023)--cycle;
\draw(-8.243,2.972)--(-8.182,2.945);
\draw(-8.177,3.004)--(-8.219,3.023);
\filldraw[fill opacity=0.8,fill=gray!20,draw=none](-8.145,2.98)--(-8.151,2.983)--(-8.145,3.001)--(-8.142,2.999)--cycle;
\draw(-8.145,3.001)--(-8.142,2.999)--(-8.145,2.98);
\filldraw[fill opacity=0.8,fill=gray!20,draw=none](-8.162,3.035)--(-8.155,2.998)--(-8.161,3)--cycle;
\draw(-8.155,2.998)--(-8.161,3);
\filldraw[fill opacity=0.8,fill=gray!20,draw=none](-8.178,2.924)--(-8.164,2.944)--(-8.162,2.95)--(-8.171,2.944)--cycle;
\draw(-8.162,2.95)--(-8.171,2.944);
\filldraw[fill opacity=0.8,fill=gray!20,draw=none](-8.164,2.948)--(-8.162,2.95)--(-8.161,2.959)--cycle;
\draw(-8.164,2.948)--(-8.162,2.95);
\filldraw[fill opacity=0.8,fill=gray!20,draw=none](-8.145,2.98)--(-8.148,2.962)--(-8.155,2.945)--(-8.164,2.949)--(-8.151,2.983)--cycle;
\draw(-8.145,2.98)--(-8.148,2.962);
\draw(-8.155,2.945)--(-8.164,2.949);
\filldraw[fill opacity=0.8,fill=gray!20,draw=none](-8.148,2.998)--(-8.153,2.989)--(-8.148,2.962)--(-8.143,2.989)--cycle;
\draw(-8.148,2.962)--(-8.143,2.989);
\filldraw[fill opacity=0.8,fill=gray!20,draw=none](-8.148,2.998)--(-8.143,2.989)--(-8.142,2.999)--(-8.143,3.008)--cycle;
\draw(-8.143,2.989)--(-8.142,2.999)--(-8.143,3.008);
\filldraw[fill opacity=0.8,fill=gray!20,draw=none](-7.953,2.921)--(-7.984,2.923)--(-7.994,2.972)--cycle;
\draw(-7.953,2.921)--(-7.984,2.923);
\filldraw[fill opacity=0.8,fill=gray!20,draw=none](-7.965,2.865)--(-7.984,2.923)--(-7.906,2.918)--(-7.91,2.861)--cycle;
\draw(-7.984,2.923)--(-7.906,2.918)--(-7.91,2.861)--(-7.965,2.865);
\filldraw[fill opacity=0.8,fill=gray!20,draw=none](-7.953,2.921)--(-7.994,2.972)--(-7.995,2.978)--(-7.91,2.972)--(-7.906,2.918)--cycle;
\draw(-7.995,2.978)--(-7.91,2.972)--(-7.906,2.918)--(-7.953,2.921);
\filldraw[fill opacity=0.8,fill=gray!20,draw=none](-7.93,2.911)--(-7.962,2.914)--(-7.971,2.959)--(-7.969,2.959)--cycle;
\draw(-7.93,2.911)--(-7.962,2.914);
\draw(-7.971,2.959)--(-7.969,2.959);
\filldraw[fill opacity=0.8,fill=gray!20,draw=none](-7.946,2.865)--(-7.962,2.914)--(-7.926,2.911)--(-7.93,2.864)--cycle;
\draw(-7.962,2.914)--(-7.926,2.911)--(-7.93,2.864)--(-7.946,2.865);
\filldraw[fill opacity=0.8,fill=gray!20,draw=none](-7.996,2.81)--(-7.995,2.845)--(-7.959,2.864)--(-7.91,2.861)--(-7.922,2.805)--cycle;
\draw(-7.959,2.864)--(-7.91,2.861)--(-7.922,2.805)--(-7.996,2.81);
\filldraw[fill opacity=0.8,fill=gray!20](-7.922,2.805)--(-7.91,2.861)--(-7.837,2.843)--(-7.857,2.789)--cycle;
\filldraw[fill opacity=0.8,fill=gray!20](-7.91,2.861)--(-7.906,2.918)--(-7.831,2.899)--(-7.837,2.843)--cycle;
\filldraw[fill opacity=0.8,fill=gray!20,draw=none](-7.907,2.858)--(-7.93,2.864)--(-7.926,2.911)--(-7.863,2.895)--(-7.868,2.86)--cycle;
\draw(-7.907,2.858)--(-7.93,2.864)--(-7.926,2.911)--(-7.863,2.895)--(-7.868,2.86);
\filldraw[fill opacity=0.8,fill=gray!20,draw=none](-7.93,2.911)--(-7.969,2.959)--(-7.93,2.956)--(-7.926,2.911)--cycle;
\draw(-7.969,2.959)--(-7.93,2.956)--(-7.926,2.911)--(-7.93,2.911);
\filldraw[fill opacity=0.8,fill=gray!20](-7.906,2.918)--(-7.91,2.972)--(-7.837,2.954)--(-7.831,2.899)--cycle;
\filldraw[fill opacity=0.8,fill=gray!20](-7.926,2.911)--(-7.93,2.956)--(-7.869,2.941)--(-7.863,2.895)--cycle;
\filldraw[fill opacity=0.8,fill=gray!20,draw=none](-7.969,2.959)--(-7.971,2.959)--(-7.971,2.962)--cycle;
\draw(-7.969,2.959)--(-7.971,2.959);
\filldraw[fill opacity=0.8,fill=gray!20,draw=none](-7.959,2.975)--(-7.995,2.978)--(-7.995,2.996)--cycle;
\draw(-7.959,2.975)--(-7.995,2.978);
\filldraw[fill opacity=0.8,fill=gray!20,draw=none](-7.959,2.975)--(-7.995,2.996)--(-7.996,3.025)--(-7.92,3.009)--(-7.91,2.972)--cycle;
\draw(-7.92,3.009)--(-7.91,2.972)--(-7.959,2.975);
\filldraw[fill opacity=0.8,fill=gray!20,draw=none](-7.969,2.959)--(-7.971,2.962)--(-7.971,2.986)--(-7.931,2.963)--(-7.93,2.956)--cycle;
\draw(-7.931,2.963)--(-7.93,2.956)--(-7.969,2.959);
\filldraw[fill opacity=0.8,fill=gray!20,draw=none](-8.141,3.036)--(-7.918,2.939)--(-7.909,2.89)--(-8.204,3.019)--cycle;
\draw(-8.141,3.036)--(-7.918,2.939)--(-7.909,2.89)--(-8.204,3.019);
\filldraw[fill opacity=0.8,fill=gray!20](-8.415,2.815)--(-8.44,2.841)--(-8.493,2.854)--(-8.452,2.824)--cycle;
\filldraw[fill opacity=0.8,fill=gray!20,draw=none](-8.167,2.915)--(-8.165,2.914)--(-8.171,2.906)--cycle;
\filldraw[fill opacity=0.8,fill=gray!20,draw=none](-7.904,.326)--(-7.889,.337)--(-7.891,.367)--(-7.954,.382)--(-7.948,.337)--cycle;
\draw(-7.889,.337)--(-7.891,.367)--(-7.954,.382)--(-7.948,.337)--(-7.904,.326);
\filldraw[fill opacity=0.8,fill=gray!20,draw=none](-7.884,.337)--(-7.857,.364)--(-7.891,.367)--(-7.889,.337)--cycle;
\draw(-7.857,.364)--(-7.891,.367)--(-7.889,.337);
\filldraw[fill opacity=0.8,fill=gray!20,draw=none](-8.843,.739)--(-8.842,.743)--(-8.818,.741)--(-8.834,.713)--cycle;
\draw(-8.842,.743)--(-8.818,.741)--(-8.834,.713);
\filldraw[fill opacity=0.8,fill=gray!20,draw=none](-8.73,.772)--(-8.735,.773)--(-8.734,.773)--cycle;
\draw(-8.735,.773)--(-8.734,.773);
\filldraw[fill opacity=0.8,fill=gray!20,draw=none](-8.735,.773)--(-8.733,.776)--(-8.73,.772)--cycle;
\draw(-8.735,.773)--(-8.733,.776)--(-8.73,.772);
\filldraw[fill opacity=0.8,fill=gray!20,draw=none](-8.869,.738)--(-8.688,.778)--(-8.694,.792)--(-8.853,.757)--cycle;
\draw(-8.694,.792)--(-8.853,.757)--(-8.869,.738)--(-8.688,.778);
\filldraw[fill opacity=0.8,fill=gray!20,draw=none](-8.76,.703)--(-8.778,.689)--(-8.748,.711)--(-8.756,.705)--cycle;
\draw(-8.778,.689)--(-8.748,.711)--(-8.756,.705);
\filldraw[fill opacity=0.8,fill=gray!20,draw=none](-8.928,.742)--(-7.815,.278)--(-7.795,.29)--(-7.796,.3)--(-8.898,.76)--cycle;
\draw(-7.796,.3)--(-8.898,.76)--(-8.928,.742)--(-7.815,.278)--(-7.795,.29);
\filldraw[fill opacity=0.8,fill=gray!20](-6.926,.173)--(-6.903,.184)--(-6.958,.182)--(-6.955,.172)--cycle;
\filldraw[fill opacity=0.8,fill=gray!20,draw=none](-6.274,.255)--(-6.264,.255)--(-6.259,.255)--(-6.263,.254)--(-6.302,.254)--cycle;
\draw(-6.274,.255)--(-6.264,.255);
\draw(-6.263,.254)--(-6.302,.254);
\filldraw[fill opacity=0.8,fill=gray!20,draw=none](-7.752,4.725)--(-7.751,4.728)--(-7.751,4.728)--cycle;
\draw(-7.752,4.725)--(-7.751,4.728);
\filldraw[fill opacity=0.8,fill=gray!20,draw=none](-7.687,4.609)--(-7.655,4.572)--(-7.621,4.551)--(-7.612,4.621)--(-7.711,4.668)--(-7.709,4.656)--cycle;
\draw(-7.711,4.668)--(-7.709,4.656)--(-7.687,4.609)--(-7.655,4.572)--(-7.621,4.551);
\filldraw[fill opacity=0.8,fill=gray!20,draw=none](-7.621,4.551)--(-7.599,4.564)--(-7.597,4.578)--(-7.603,4.612)--(-7.612,4.621)--cycle;
\draw(-7.599,4.564)--(-7.597,4.578)--(-7.603,4.612)--(-7.612,4.621);
\filldraw[fill opacity=0.8,fill=gray!20,draw=none](-6.257,.255)--(-6.183,.254)--(-6.174,.253)--(-6.263,.254)--cycle;
\draw(-6.257,.255)--(-6.183,.254);
\draw(-6.174,.253)--(-6.263,.254);
\filldraw[fill opacity=0.8,fill=gray!20,draw=none](-7.522,4.566)--(-7.524,4.564)--(-7.531,4.545)--cycle;
\draw(-7.524,4.564)--(-7.531,4.545);
\filldraw[fill opacity=0.8,fill=gray!20,draw=none](-7.677,4.876)--(-7.647,4.88)--(-7.653,4.883)--(-7.668,4.882)--cycle;
\draw(-7.653,4.883)--(-7.668,4.882)--(-7.677,4.876)--(-7.647,4.88);
\filldraw[fill opacity=0.8,fill=gray!20,draw=none](-7.956,1.128)--(-7.953,1.172)--(-7.938,1.182)--(-7.943,1.136)--cycle;
\draw(-7.953,1.172)--(-7.938,1.182)--(-7.943,1.136)--(-7.956,1.128);
\filldraw[fill opacity=0.8,fill=gray!20,draw=none](-7.965,1.16)--(-7.85,1.186)--(-7.849,1.14)--(-7.953,1.117)--cycle;
\draw(-7.965,1.16)--(-7.85,1.186)--(-7.849,1.14)--(-7.953,1.117);
\filldraw[fill opacity=0.8,fill=gray!20,draw=none](-7.759,4.684)--(-7.739,4.722)--(-7.751,4.728)--(-7.752,4.725)--cycle;
\draw(-7.751,4.728)--(-7.752,4.725);
\filldraw[fill opacity=0.8,fill=gray!20,draw=none](-7.735,4.721)--(-7.732,4.728)--(-7.748,4.735)--(-7.751,4.728)--cycle;
\draw(-7.735,4.721)--(-7.732,4.728)--(-7.748,4.735)--(-7.751,4.728);
\filldraw[fill opacity=0.8,fill=gray!20,draw=none](-7.8,4.547)--(-7.796,4.557)--(-7.807,4.57)--(-7.815,4.554)--cycle;
\draw(-7.8,4.547)--(-7.796,4.557);
\draw(-7.807,4.57)--(-7.815,4.554);
\filldraw[fill opacity=0.8,fill=gray!20,draw=none](-6.002,.229)--(-6.002,.183)--(-5.974,.193)--(-5.974,.271)--cycle;
\draw(-6.002,.229)--(-6.002,.183)--(-5.974,.193)--(-5.974,.271);
\filldraw[fill opacity=0.8,fill=gray!20,draw=none](-5.985,.222)--(-6.033,.242)--(-6.003,.284)--(-5.974,.271)--cycle;
\draw(-5.985,.222)--(-6.033,.242);
\draw(-6.003,.284)--(-5.974,.271);
\filldraw[fill opacity=0.8,fill=gray!20,draw=none](-8.013,4.068)--(-7.82,4.502)--(-7.835,4.508)--(-8.035,4.058)--cycle;
\draw(-8.013,4.068)--(-7.82,4.502);
\draw(-7.835,4.508)--(-8.035,4.058);
\filldraw[fill opacity=0.8,fill=gray!20](-7.749,.227)--(-7.733,.248)--(-7.798,.246)--(-7.796,.225)--cycle;
\filldraw[fill opacity=0.8,fill=gray!20](-8.157,3.933)--(-8.126,3.981)--(-8.142,3.998)--(-8.177,3.954)--cycle;
\filldraw[fill opacity=0.8,fill=gray!20](-7.768,3.842)--(-7.776,3.897)--(-7.81,3.874)--(-7.804,3.818)--cycle;
\filldraw[fill opacity=0.8,fill=gray!20,draw=none](-7.68,4.894)--(-7.689,4.873)--(-7.669,4.877)--(-7.664,4.887)--cycle;
\draw(-7.669,4.877)--(-7.664,4.887)--(-7.68,4.894)--(-7.689,4.873);
\filldraw[fill opacity=0.8,fill=gray!20,draw=none](-7.719,4.723)--(-7.7,4.714)--(-7.689,4.71)--(-7.687,4.709)--cycle;
\draw(-7.689,4.71)--(-7.687,4.709);
\filldraw[fill opacity=0.8,fill=gray!20,draw=none](-7.717,4.712)--(-7.713,4.734)--(-7.724,4.715)--cycle;
\filldraw[fill opacity=0.8,fill=gray!20,draw=none](-7.687,4.609)--(-7.742,4.583)--(-7.717,4.542)--(-7.655,4.572)--cycle;
\draw(-7.717,4.542)--(-7.655,4.572)--(-7.687,4.609)--(-7.742,4.583);
\filldraw[fill opacity=0.8,fill=gray!20,draw=none](-7.709,4.656)--(-7.748,4.637)--(-7.756,4.615)--(-7.742,4.583)--(-7.687,4.609)--cycle;
\draw(-7.742,4.583)--(-7.687,4.609)--(-7.709,4.656)--(-7.748,4.637);
\filldraw[fill opacity=0.8,fill=gray!20,draw=none](-7.742,4.583)--(-7.739,4.554)--(-7.748,4.637)--cycle;
\draw(-7.742,4.583)--(-7.739,4.554);
\filldraw[fill opacity=0.8,fill=gray!20,draw=none](-7.731,4.727)--(-7.735,4.729)--(-7.742,4.733)--(-7.734,4.729)--cycle;
\filldraw[fill opacity=0.8,fill=gray!20,draw=none](-7.731,4.727)--(-7.714,4.72)--(-7.719,4.723)--(-7.735,4.729)--cycle;
\filldraw[fill opacity=0.8,fill=gray!20,draw=none](-7.759,4.684)--(-7.767,4.638)--(-7.735,4.721)--(-7.739,4.722)--cycle;
\draw(-7.767,4.638)--(-7.735,4.721);
\filldraw[fill opacity=0.8,fill=gray!20,draw=none](-7.727,4.726)--(-7.731,4.727)--(-7.734,4.729)--(-7.729,4.726)--cycle;
\filldraw[fill opacity=0.8,fill=gray!20,draw=none](-7.741,4.755)--(-7.755,4.724)--(-7.738,4.717)--(-7.717,4.766)--cycle;
\draw(-7.741,4.755)--(-7.755,4.724);
\draw(-7.738,4.717)--(-7.717,4.766);
\filldraw[fill opacity=0.8,fill=gray!20,draw=none](-7.744,4.72)--(-7.755,4.724)--(-7.762,4.707)--(-7.769,4.679)--cycle;
\draw(-7.755,4.724)--(-7.762,4.707);
\filldraw[fill opacity=0.8,fill=gray!20,draw=none](-7.802,4.548)--(-7.797,4.561)--(-7.807,4.572)--(-7.815,4.554)--cycle;
\draw(-7.802,4.548)--(-7.797,4.561);
\filldraw[fill opacity=0.8,fill=gray!20,draw=none](-7.814,4.587)--(-7.827,4.559)--(-7.81,4.552)--(-7.802,4.572)--cycle;
\draw(-7.814,4.587)--(-7.827,4.559)--(-7.81,4.552)--(-7.802,4.572);
\filldraw[fill opacity=0.8,fill=gray!20,draw=none](-7.514,4.52)--(-7.516,4.538)--(-7.509,4.557)--(-7.501,4.549)--cycle;
\filldraw[fill opacity=0.8,fill=gray!20](-7.012,.186)--(-7.037,.211)--(-7.091,.224)--(-7.05,.195)--cycle;
\filldraw[fill opacity=0.8,fill=gray!20,draw=none](-7.525,4.567)--(-7.527,4.565)--(-7.536,4.547)--cycle;
\draw(-7.527,4.565)--(-7.536,4.547);
\filldraw[fill opacity=0.8,fill=gray!20,draw=none](-8.026,4.054)--(-7.987,4.059)--(-7.987,4.059)--(-8.035,4.058)--cycle;
\draw(-8.026,4.054)--(-7.987,4.059)--(-7.987,4.059)--(-8.035,4.058);
\filldraw[fill opacity=0.8,fill=gray!20,draw=none](-8.022,4.05)--(-7.987,4.059)--(-7.987,4.059)--(-8.026,4.054)--cycle;
\draw(-8.022,4.05)--(-7.987,4.059)--(-7.987,4.059)--(-8.026,4.054);
\filldraw[fill opacity=0.8,fill=gray!20,draw=none](-8.092,3.892)--(-8.013,4.068)--(-8.035,4.058)--(-8.107,3.898)--cycle;
\draw(-8.035,4.058)--(-8.107,3.898)--(-8.092,3.892)--(-8.013,4.068);
\filldraw[fill opacity=0.8,fill=gray!20,draw=none](-7.485,4.62)--(-7.483,4.595)--(-7.496,4.59)--cycle;
\draw(-7.485,4.62)--(-7.483,4.595);
\filldraw[fill opacity=0.8,fill=gray!20,draw=none](-7.527,4.565)--(-7.504,4.617)--(-7.526,4.627)--(-7.55,4.571)--cycle;
\draw(-7.527,4.565)--(-7.504,4.617)--(-7.526,4.627)--(-7.55,4.571);
\filldraw[fill opacity=0.8,fill=gray!20,draw=none](-7.524,4.63)--(-7.547,4.57)--(-7.524,4.564)--(-7.502,4.623)--cycle;
\draw(-7.524,4.564)--(-7.502,4.623)--(-7.524,4.63)--(-7.547,4.57);
\filldraw[fill opacity=0.8,fill=gray!20,draw=none](-7.536,4.547)--(-7.527,4.565)--(-7.55,4.571)--(-7.564,4.541)--cycle;
\draw(-7.536,4.547)--(-7.527,4.565);
\draw(-7.55,4.571)--(-7.564,4.541);
\filldraw[fill opacity=0.8,fill=gray!20,draw=none](-7.536,4.547)--(-7.564,4.541)--(-7.574,4.519)--cycle;
\draw(-7.564,4.541)--(-7.574,4.519);
\filldraw[fill opacity=0.8,fill=gray!20,draw=none](-7.547,4.57)--(-7.559,4.538)--(-7.531,4.545)--(-7.524,4.564)--cycle;
\draw(-7.547,4.57)--(-7.559,4.538);
\draw(-7.531,4.545)--(-7.524,4.564);
\filldraw[fill opacity=0.8,fill=gray!20,draw=none](-7.559,4.538)--(-7.567,4.516)--(-7.531,4.545)--cycle;
\draw(-7.559,4.538)--(-7.567,4.516);
\filldraw[fill opacity=0.8,fill=gray!20,draw=none](-7.487,4.632)--(-7.508,4.639)--(-7.556,4.527)--(-7.525,4.534)--(-7.486,4.624)--cycle;
\draw(-7.508,4.639)--(-7.556,4.527);
\draw(-7.525,4.534)--(-7.486,4.624);
\filldraw[fill opacity=0.8,fill=gray!20,draw=none](-7.511,4.555)--(-7.485,4.621)--(-7.487,4.632)--(-7.506,4.639)--(-7.536,4.562)--cycle;
\draw(-7.511,4.555)--(-7.485,4.621);
\draw(-7.487,4.632)--(-7.506,4.639)--(-7.536,4.562);
\filldraw[fill opacity=0.8,fill=gray!20,draw=none](-7.509,4.487)--(-7.505,4.488)--(-7.504,4.489)--cycle;
\draw(-7.505,4.488)--(-7.504,4.489);
\filldraw[fill opacity=0.8,fill=gray!20,draw=none](-7.512,4.498)--(-7.514,4.52)--(-7.501,4.549)--(-7.497,4.544)--(-7.495,4.517)--cycle;
\draw(-7.497,4.544)--(-7.495,4.517);
\filldraw[fill opacity=0.8,fill=gray!20,draw=none](-7.51,4.487)--(-7.509,4.487)--(-7.504,4.489)--(-7.491,4.498)--(-7.504,4.496)--cycle;
\draw(-7.504,4.489)--(-7.491,4.498)--(-7.504,4.496);
\filldraw[fill opacity=0.8,fill=gray!20,draw=none](-7.49,4.657)--(-7.498,4.662)--(-7.508,4.639)--(-7.487,4.632)--cycle;
\draw(-7.498,4.662)--(-7.508,4.639);
\filldraw[fill opacity=0.8,fill=gray!20,draw=none](-7.475,4.511)--(-7.471,4.466)--(-7.488,4.441)--(-7.497,4.544)--cycle;
\draw(-7.475,4.511)--(-7.471,4.466);
\draw(-7.488,4.441)--(-7.497,4.544);
\filldraw[fill opacity=0.8,fill=gray!20,draw=none](-7.526,4.495)--(-7.516,4.538)--(-7.514,4.52)--cycle;
\filldraw[fill opacity=0.8,fill=gray!20,draw=none](-7.517,4.538)--(-7.52,4.532)--(-7.523,4.524)--cycle;
\draw(-7.52,4.532)--(-7.523,4.524);
\filldraw[fill opacity=0.8,fill=gray!20,draw=none](-7.521,4.54)--(-7.525,4.534)--(-7.528,4.527)--cycle;
\draw(-7.525,4.534)--(-7.528,4.527);
\filldraw[fill opacity=0.8,fill=gray!20,draw=none](-7.538,4.532)--(-7.553,4.509)--(-7.528,4.527)--(-7.525,4.534)--cycle;
\draw(-7.528,4.527)--(-7.525,4.534);
\filldraw[fill opacity=0.8,fill=gray!20,draw=none](-7.546,4.506)--(-7.523,4.524)--(-7.52,4.532)--(-7.533,4.529)--cycle;
\draw(-7.523,4.524)--(-7.52,4.532);
\filldraw[fill opacity=0.8,fill=gray!20,draw=none](-7.512,4.498)--(-7.504,4.497)--(-7.495,4.517)--cycle;
\filldraw[fill opacity=0.8,fill=gray!20,draw=none](-7.526,4.495)--(-7.514,4.52)--(-7.512,4.498)--(-7.516,4.494)--(-7.527,4.484)--(-7.527,4.487)--cycle;
\draw(-7.527,4.484)--(-7.527,4.487);
\filldraw[fill opacity=0.8,fill=gray!20,draw=none](-7.512,4.498)--(-7.495,4.517)--(-7.487,4.534)--(-7.535,4.524)--(-7.546,4.499)--cycle;
\draw(-7.487,4.534)--(-7.535,4.524)--(-7.546,4.499);
\filldraw[fill opacity=0.8,fill=gray!20,draw=none](-7.526,4.495)--(-7.527,4.487)--(-7.528,4.49)--cycle;
\draw(-7.527,4.487)--(-7.528,4.49);
\filldraw[fill opacity=0.8,fill=gray!20,draw=none](-7.512,4.496)--(-7.512,4.498)--(-7.495,4.517)--(-7.494,4.51)--cycle;
\draw(-7.495,4.517)--(-7.494,4.51);
\filldraw[fill opacity=0.8,fill=gray!20,draw=none](-7.537,4.479)--(-7.509,4.487)--(-7.529,4.484)--cycle;
\filldraw[fill opacity=0.8,fill=gray!20,draw=none](-7.528,4.484)--(-7.51,4.487)--(-7.504,4.496)--(-7.516,4.494)--cycle;
\draw(-7.504,4.496)--(-7.516,4.494);
\filldraw[fill opacity=0.8,fill=gray!20,draw=none](-7.516,4.494)--(-7.498,4.497)--(-7.512,4.498)--cycle;
\draw(-7.516,4.494)--(-7.498,4.497);
\filldraw[fill opacity=0.8,fill=gray!20,draw=none](-7.505,4.432)--(-7.512,4.496)--(-7.494,4.51)--(-7.488,4.441)--cycle;
\draw(-7.494,4.51)--(-7.488,4.441);
\filldraw[fill opacity=0.8,fill=gray!20,draw=none](-7.538,4.532)--(-7.556,4.527)--(-7.57,4.495)--(-7.553,4.509)--cycle;
\draw(-7.556,4.527)--(-7.57,4.495);
\filldraw[fill opacity=0.8,fill=gray!20,draw=none](-7.577,4.477)--(-7.575,4.474)--(-7.553,4.509)--(-7.57,4.495)--(-7.575,4.484)--cycle;
\draw(-7.57,4.495)--(-7.575,4.484);
\filldraw[fill opacity=0.8,fill=gray!20,draw=none](-7.578,4.496)--(-7.575,4.484)--(-7.57,4.495)--cycle;
\draw(-7.575,4.484)--(-7.57,4.495);
\filldraw[fill opacity=0.8,fill=gray!20,draw=none](-7.616,4.497)--(-7.574,4.519)--(-7.604,4.522)--cycle;
\filldraw[fill opacity=0.8,fill=gray!20,draw=none](-7.623,4.494)--(-7.616,4.497)--(-7.604,4.522)--(-7.611,4.523)--cycle;
\filldraw[fill opacity=0.8,fill=gray!20,draw=none](-7.608,4.52)--(-7.62,4.486)--(-7.567,4.516)--cycle;
\draw(-7.608,4.52)--(-7.62,4.486);
\filldraw[fill opacity=0.8,fill=gray!20,draw=none](-7.604,4.522)--(-7.574,4.519)--(-7.526,4.627)--(-7.549,4.637)--cycle;
\draw(-7.574,4.519)--(-7.526,4.627)--(-7.549,4.637);
\filldraw[fill opacity=0.8,fill=gray!20,draw=none](-7.611,4.523)--(-7.604,4.522)--(-7.549,4.637)--(-7.561,4.643)--cycle;
\draw(-7.549,4.637)--(-7.561,4.643);
\filldraw[fill opacity=0.8,fill=gray!20,draw=none](-7.561,4.643)--(-7.608,4.52)--(-7.567,4.516)--(-7.524,4.63)--cycle;
\draw(-7.567,4.516)--(-7.524,4.63)--(-7.561,4.643)--(-7.608,4.52);
\filldraw[fill opacity=0.8,fill=gray!20,draw=none](-7.508,4.677)--(-7.583,4.497)--(-7.57,4.495)--(-7.498,4.662)--cycle;
\draw(-7.57,4.495)--(-7.498,4.662);
\filldraw[fill opacity=0.8,fill=gray!20,draw=none](-7.574,4.493)--(-7.564,4.492)--(-7.506,4.639)--(-7.508,4.64)--cycle;
\draw(-7.564,4.492)--(-7.506,4.639)--(-7.508,4.64);
\filldraw[fill opacity=0.8,fill=gray!20,draw=none](-7.655,4.531)--(-7.599,4.561)--(-7.599,4.564)--cycle;
\draw(-7.599,4.561)--(-7.599,4.564);
\filldraw[fill opacity=0.8,fill=gray!20,draw=none](-7.578,4.496)--(-7.583,4.497)--(-7.589,4.481)--(-7.578,4.478)--(-7.575,4.484)--cycle;
\draw(-7.578,4.478)--(-7.575,4.484);
\filldraw[fill opacity=0.8,fill=gray!20,draw=none](-7.579,4.493)--(-7.574,4.493)--(-7.508,4.64)--(-7.521,4.644)--cycle;
\draw(-7.508,4.64)--(-7.521,4.644);
\filldraw[fill opacity=0.8,fill=gray!20,draw=none](-7.52,4.532)--(-7.511,4.555)--(-7.536,4.562)--(-7.551,4.525)--cycle;
\draw(-7.52,4.532)--(-7.511,4.555);
\draw(-7.536,4.562)--(-7.551,4.525);
\filldraw[fill opacity=0.8,fill=gray!20,draw=none](-7.546,4.506)--(-7.533,4.529)--(-7.551,4.525)--(-7.564,4.492)--cycle;
\draw(-7.551,4.525)--(-7.564,4.492);
\filldraw[fill opacity=0.8,fill=gray!20,draw=none](-7.487,4.534)--(-7.48,4.557)--(-7.483,4.58)--(-7.525,4.572)--(-7.535,4.524)--cycle;
\draw(-7.483,4.58)--(-7.525,4.572)--(-7.535,4.524)--(-7.487,4.534);
\filldraw[fill opacity=0.8,fill=gray!20,draw=none](-7.508,4.677)--(-7.513,4.684)--(-7.532,4.691)--(-7.612,4.499)--(-7.583,4.497)--cycle;
\filldraw[fill opacity=0.8,fill=gray!20,draw=none](-7.483,4.58)--(-7.484,4.597)--(-7.508,4.627)--(-7.511,4.628)--(-7.522,4.626)--(-7.525,4.572)--cycle;
\draw(-7.511,4.628)--(-7.522,4.626)--(-7.525,4.572)--(-7.483,4.58);
\filldraw[fill opacity=0.8,fill=gray!20,draw=none](-7.518,4.736)--(-7.496,4.493)--(-7.475,4.493)--(-7.494,4.702)--cycle;
\draw(-7.475,4.493)--(-7.494,4.702)--(-7.518,4.736)--(-7.496,4.493);
\filldraw[fill opacity=0.8,fill=gray!20,draw=none](-8.044,4.052)--(-8.026,4.054)--(-8.035,4.058)--cycle;
\draw(-8.035,4.058)--(-8.044,4.052)--(-8.026,4.054);
\filldraw[fill opacity=0.8,fill=gray!20,draw=none](-8.174,3.746)--(-8.402,3.234)--(-8.38,3.244)--(-8.163,3.733)--cycle;
\draw(-8.174,3.746)--(-8.402,3.234);
\draw(-8.38,3.244)--(-8.163,3.733);
\filldraw[fill opacity=0.8,fill=gray!20,draw=none](-7.585,4.364)--(-7.58,4.379)--(-7.583,4.371)--cycle;
\draw(-7.58,4.379)--(-7.583,4.371);
\filldraw[fill opacity=0.8,fill=gray!20](-8.524,3.109)--(-8.493,3.157)--(-8.509,3.174)--(-8.544,3.13)--cycle;
\filldraw[fill opacity=0.8,fill=gray!20,draw=none](-7.89,.285)--(-7.942,.321)--(-7.932,.295)--cycle;
\draw(-7.942,.321)--(-7.932,.295)--(-7.89,.285);
\filldraw[fill opacity=0.8,fill=gray!20,draw=none](-8.43,2.902)--(-8.439,2.898)--(-8.435,2.896)--(-8.412,2.889)--cycle;
\draw(-8.439,2.898)--(-8.435,2.896);
\filldraw[fill opacity=0.8,fill=gray!20,draw=none](-6.259,.255)--(-6.259,.201)--(-6.244,.189)--(-6.244,.253)--cycle;
\draw(-6.259,.255)--(-6.259,.201)--(-6.244,.189)--(-6.244,.253);
\filldraw[fill opacity=0.8,fill=gray!20,draw=none](-7.663,1.002)--(-7.664,1.013)--(-7.65,1.038)--cycle;
\filldraw[fill opacity=0.8,fill=gray!20,draw=none](-7.663,1.002)--(-7.666,.992)--(-7.673,.999)--(-7.664,1.013)--cycle;
\draw(-7.666,.992)--(-7.673,.999);
\filldraw[fill opacity=0.8,fill=gray!20,draw=none](-7.049,.994)--(-7.051,.999)--(-7.006,1.06)--(-6.985,1.065)--(-6.975,1.009)--cycle;
\draw(-7.006,1.06)--(-6.985,1.065)--(-6.975,1.009)--(-7.049,.994)--(-7.051,.999);
\filldraw[fill opacity=0.8,fill=gray!20,draw=none](-6.965,.956)--(-7.027,.957)--(-7.044,.986)--(-7.015,1.001)--(-6.975,1.009)--(-6.96,.957)--cycle;
\draw(-7.027,.957)--(-7.044,.986);
\draw(-7.015,1.001)--(-6.975,1.009)--(-6.96,.957)--(-6.965,.956);
\filldraw[fill opacity=0.8,fill=gray!20,draw=none](-7.044,.986)--(-7.049,.994)--(-7.015,1.001)--cycle;
\draw(-7.044,.986)--(-7.049,.994)--(-7.015,1.001);
\filldraw[fill opacity=0.8,fill=gray!20,draw=none](-7.051,.993)--(-7.051,.999)--(-7.049,.994)--cycle;
\draw(-7.051,.999)--(-7.049,.994)--(-7.051,.993);
\filldraw[fill opacity=0.8,fill=gray!20,draw=none](-7.048,.977)--(-7.051,.993)--(-7.049,.994)--(-7.044,.986)--cycle;
\draw(-7.051,.993)--(-7.049,.994)--(-7.044,.986);
\filldraw[fill opacity=0.8,fill=gray!20,draw=none](-7.044,.957)--(-7.048,.977)--(-7.044,.986)--(-7.027,.957)--cycle;
\draw(-7.044,.986)--(-7.027,.957);
\filldraw[fill opacity=0.8,fill=gray!20](-6.881,.971)--(-7.781,1.006)--(-7.774,1.043)--(-6.874,1.008)--cycle;
\filldraw[fill opacity=0.8,fill=gray!20](-7.896,3.646)--(-7.858,3.674)--(-7.918,3.663)--(-7.938,3.638)--cycle;
\filldraw[fill opacity=0.8,fill=gray!20](-7.569,4.881)--(-7.62,4.883)--(-7.62,4.883)--(-7.563,4.875)--cycle;
\filldraw[fill opacity=0.8,fill=gray!20,draw=none](-7.488,4.631)--(-7.486,4.623)--(-7.487,4.633)--cycle;
\draw(-7.486,4.623)--(-7.487,4.633);
\filldraw[fill opacity=0.8,fill=gray!20](-7.764,.95)--(-7.74,.976)--(-7.709,.969)--(-7.748,.946)--cycle;
\filldraw[fill opacity=0.8,fill=gray!20](-7.703,1.232)--(-7.719,1.264)--(-7.675,1.253)--(-7.649,1.219)--cycle;
\filldraw[fill opacity=0.8,fill=gray!20,draw=none](-6.274,.512)--(-6.183,.511)--(-6.226,.513)--(-6.246,.513)--cycle;
\draw(-6.274,.512)--(-6.183,.511);
\draw(-6.226,.513)--(-6.246,.513);
\filldraw[fill opacity=0.8,fill=gray!20,draw=none](-8.394,3.23)--(-8.354,3.235)--(-8.354,3.235)--(-8.402,3.234)--cycle;
\draw(-8.394,3.23)--(-8.354,3.235)--(-8.354,3.235)--(-8.402,3.234);
\filldraw[fill opacity=0.8,fill=gray!20,draw=none](-8.389,3.226)--(-8.354,3.235)--(-8.354,3.235)--(-8.394,3.23)--cycle;
\draw(-8.389,3.226)--(-8.354,3.235)--(-8.354,3.235)--(-8.394,3.23);
\filldraw[fill opacity=0.8,fill=gray!20,draw=none](-8.402,3.234)--(-8.411,3.216)--(-8.4,3.2)--(-8.38,3.244)--cycle;
\draw(-8.402,3.234)--(-8.411,3.216);
\draw(-8.4,3.2)--(-8.38,3.244);
\filldraw[fill opacity=0.8,fill=gray!20,draw=none](-8.45,3.219)--(-8.457,3.213)--(-8.472,3.219)--cycle;
\draw(-8.457,3.213)--(-8.472,3.219)--(-8.45,3.219);
\filldraw[fill opacity=0.8,fill=gray!20,draw=none](-8.45,3.219)--(-8.472,3.219)--(-8.475,3.217)--(-8.506,3.176)--cycle;
\draw(-8.45,3.219)--(-8.472,3.219)--(-8.475,3.217);
\filldraw[fill opacity=0.8,fill=gray!20,draw=none](-8.412,2.889)--(-8.405,2.891)--(-8.43,2.902)--cycle;
\draw(-8.405,2.891)--(-8.43,2.902);
\filldraw[fill opacity=0.8,fill=gray!20,draw=none](-8.411,3.228)--(-8.394,3.23)--(-8.402,3.234)--cycle;
\draw(-8.402,3.234)--(-8.411,3.228)--(-8.394,3.23);
\filldraw[fill opacity=0.8,fill=gray!20,draw=none](-5.91,.305)--(-5.967,.33)--(-5.965,.371)--(-5.96,.387)--(-5.9,.361)--cycle;
\draw(-5.96,.387)--(-5.9,.361)--(-5.91,.305)--(-5.967,.33);
\filldraw[fill opacity=0.8,fill=gray!20,draw=none](-8.538,2.943)--(-8.544,2.944)--(-8.544,2.943)--cycle;
\draw(-8.538,2.943)--(-8.544,2.944)--(-8.544,2.943);
\filldraw[fill opacity=0.8,fill=gray!20,draw=none](-8.538,2.943)--(-8.506,2.939)--(-8.477,2.961)--(-8.474,2.969)--(-8.475,2.98)--(-8.512,2.99)--(-8.548,2.976)--(-8.544,2.944)--cycle;
\draw(-8.474,2.969)--(-8.475,2.98)--(-8.512,2.99);
\draw(-8.548,2.976)--(-8.544,2.944)--(-8.538,2.943);
\filldraw[fill opacity=0.8,fill=gray!20,draw=none](-8.559,2.95)--(-8.439,2.898)--(-8.425,2.904)--(-8.53,2.949)--cycle;
\draw(-8.425,2.904)--(-8.53,2.949)--(-8.559,2.95)--(-8.439,2.898);
\filldraw[fill opacity=0.8,fill=gray!20](-6.732,.388)--(-6.74,.444)--(-6.775,.421)--(-6.768,.365)--cycle;
\filldraw[fill opacity=0.8,fill=gray!20,draw=none](-8.263,2.821)--(-8.263,2.822)--(-8.264,2.822)--cycle;
\draw(-8.263,2.821)--(-8.263,2.822)--(-8.264,2.822);
\filldraw[fill opacity=0.8,fill=gray!20,draw=none](-7.444,4.765)--(-7.453,4.765)--(-7.492,4.675)--(-7.487,4.632)--(-7.483,4.631)--(-7.427,4.758)--cycle;
\draw(-7.453,4.765)--(-7.492,4.675);
\draw(-7.483,4.631)--(-7.427,4.758);
\filldraw[fill opacity=0.8,fill=gray!20,draw=none](-7.408,4.721)--(-7.425,4.761)--(-7.451,4.721)--(-7.443,4.699)--cycle;
\draw(-7.451,4.721)--(-7.443,4.699)--(-7.408,4.721)--(-7.425,4.761);
\filldraw[fill opacity=0.8,fill=gray!20](-7.759,4.805)--(-7.718,4.843)--(-7.73,4.856)--(-7.775,4.822)--cycle;
\filldraw[fill opacity=0.8,fill=gray!20,draw=none](-7.65,1.22)--(-7.648,1.219)--(-7.649,1.219)--cycle;
\draw(-7.648,1.219)--(-7.649,1.219);
\filldraw[fill opacity=0.8,fill=gray!20,draw=none](-7.664,1.24)--(-7.65,1.22)--(-7.649,1.219)--(-7.665,1.24)--cycle;
\draw(-7.649,1.219)--(-7.665,1.24);
\filldraw[fill opacity=0.8,fill=gray!20,draw=none](-7.651,1.185)--(-7.769,1.19)--(-7.774,1.225)--(-7.648,1.22)--cycle;
\draw(-7.651,1.185)--(-7.769,1.19)--(-7.774,1.225)--(-7.648,1.22);
\filldraw[fill opacity=0.8,fill=gray!20](-7.936,4.057)--(-7.987,4.059)--(-7.987,4.059)--(-7.931,4.051)--cycle;
\filldraw[fill opacity=0.8,fill=gray!20,draw=none](-7.647,4.88)--(-7.62,4.883)--(-7.62,4.883)--(-7.653,4.883)--cycle;
\draw(-7.647,4.88)--(-7.62,4.883)--(-7.62,4.883)--(-7.653,4.883);
\filldraw[fill opacity=0.8,fill=gray!20](-7.008,.599)--(-6.952,.605)--(-6.952,.605)--(-6.999,.605)--cycle;
\filldraw[fill opacity=0.8,fill=gray!20,draw=none](-7.8,3.952)--(-7.8,3.954)--(-7.8,3.953)--cycle;
\draw(-7.8,3.954)--(-7.8,3.953);
\filldraw[fill opacity=0.8,fill=gray!20,draw=none](-7.667,.28)--(-7.663,.287)--(-7.673,.287)--cycle;
\draw(-7.663,.287)--(-7.673,.287);
\filldraw[fill opacity=0.8,fill=gray!20,draw=none](-6.207,.413)--(-6.192,.454)--(-6.192,.467)--(-6.252,.493)--(-6.267,.45)--(-6.259,.435)--(-6.209,.413)--cycle;
\draw(-6.192,.467)--(-6.252,.493);
\draw(-6.259,.435)--(-6.209,.413);
\filldraw[fill opacity=0.8,fill=gray!20,draw=none](-6.259,.505)--(-6.257,.512)--(-6.264,.512)--cycle;
\draw(-6.257,.512)--(-6.264,.512);
\filldraw[fill opacity=0.8,fill=gray!20,draw=none](-6.252,.513)--(-6.252,.493)--(-6.259,.46)--(-6.259,.512)--cycle;
\draw(-6.252,.513)--(-6.252,.493);
\draw(-6.259,.46)--(-6.259,.512);
\filldraw[fill opacity=0.8,fill=gray!20,draw=none](-6.258,.467)--(-6.252,.493)--(-6.259,.435)--cycle;
\filldraw[fill opacity=0.8,fill=gray!20](-8.126,3.981)--(-8.085,4.019)--(-8.097,4.032)--(-8.142,3.998)--cycle;
\filldraw[fill opacity=0.8,fill=gray!20,draw=none](-7.871,3.793)--(-7.8,3.953)--(-7.825,3.956)--(-7.893,3.803)--cycle;
\draw(-7.825,3.956)--(-7.893,3.803)--(-7.871,3.793)--(-7.8,3.953);
\filldraw[fill opacity=0.8,fill=gray!20](-7.776,3.897)--(-7.797,3.949)--(-7.829,3.929)--(-7.81,3.874)--cycle;
\filldraw[fill opacity=0.8,fill=gray!20,draw=none](-5.933,.266)--(-5.938,.256)--(-5.974,.271)--(-5.967,.33)--cycle;
\draw(-5.933,.266)--(-5.938,.256)--(-5.974,.271);
\filldraw[fill opacity=0.8,fill=gray!20,draw=none](-5.933,.266)--(-5.967,.33)--(-5.91,.305)--cycle;
\draw(-5.967,.33)--(-5.91,.305)--(-5.933,.266);
\filldraw[fill opacity=0.8,fill=gray!20](-6.86,.192)--(-6.822,.221)--(-6.882,.209)--(-6.903,.184)--cycle;
\filldraw[fill opacity=0.8,fill=gray!20](-8.303,3.233)--(-8.354,3.235)--(-8.354,3.235)--(-8.298,3.227)--cycle;
\filldraw[fill opacity=0.8,fill=gray!20,draw=none](-7.669,4.462)--(-7.681,4.463)--(-7.676,4.461)--cycle;
\draw(-7.669,4.462)--(-7.681,4.463)--(-7.676,4.461);
\filldraw[fill opacity=0.8,fill=gray!20,draw=none](-7.669,4.462)--(-7.663,4.464)--(-7.689,4.471)--(-7.681,4.463)--cycle;
\draw(-7.689,4.471)--(-7.681,4.463)--(-7.669,4.462);
\filldraw[fill opacity=0.8,fill=gray!20,draw=none](-7.669,4.49)--(-7.657,4.433)--(-7.67,4.434)--(-7.675,4.493)--cycle;
\draw(-7.67,4.434)--(-7.675,4.493);
\filldraw[fill opacity=0.8,fill=gray!20,draw=none](-7.669,4.49)--(-7.662,4.456)--(-7.633,4.474)--cycle;
\filldraw[fill opacity=0.8,fill=gray!20,draw=none](-7.119,.484)--(-7.091,.527)--(-7.107,.545)--(-7.124,.522)--cycle;
\draw(-7.119,.484)--(-7.091,.527)--(-7.107,.545)--(-7.124,.522);
\filldraw[fill opacity=0.8,fill=gray!20,draw=none](-7.744,4.72)--(-7.769,4.679)--(-7.793,4.591)--(-7.738,4.717)--cycle;
\draw(-7.793,4.591)--(-7.738,4.717);
\filldraw[fill opacity=0.8,fill=gray!20,draw=none](-5.954,.384)--(-5.96,.387)--(-5.957,.405)--cycle;
\draw(-5.954,.384)--(-5.96,.387);
\filldraw[fill opacity=0.8,fill=gray!20,draw=none](-7.82,.56)--(-7.816,.564)--(-7.833,.568)--(-7.84,.564)--cycle;
\draw(-7.82,.56)--(-7.816,.564)--(-7.833,.568)--(-7.84,.564);
\filldraw[fill opacity=0.8,fill=gray!20,draw=none](-7.615,4.29)--(-7.592,4.347)--(-7.613,4.366)--(-7.639,4.298)--cycle;
\draw(-7.613,4.366)--(-7.639,4.298)--(-7.615,4.29)--(-7.592,4.347);
\filldraw[fill opacity=0.8,fill=gray!20,draw=none](-7.919,.305)--(-7.91,.317)--(-7.942,.321)--cycle;
\filldraw[fill opacity=0.8,fill=gray!20,draw=none](-8.167,3.128)--(-8.167,3.129)--(-8.164,3.125)--cycle;
\draw(-8.167,3.129)--(-8.164,3.125)--(-8.167,3.128);
\filldraw[fill opacity=0.8,fill=gray!20,draw=none](-6.07,.567)--(-6.097,.586)--(-6.107,.571)--cycle;
\draw(-6.097,.586)--(-6.107,.571);
\filldraw[fill opacity=0.8,fill=gray!20,draw=none](-8.493,3.157)--(-8.486,3.163)--(-8.506,3.177)--(-8.509,3.174)--cycle;
\draw(-8.506,3.177)--(-8.509,3.174)--(-8.493,3.157)--(-8.486,3.163);
\filldraw[fill opacity=0.8,fill=gray!20,draw=none](-7.592,4.347)--(-7.583,4.371)--(-7.598,4.403)--(-7.613,4.366)--cycle;
\draw(-7.592,4.347)--(-7.583,4.371);
\draw(-7.598,4.403)--(-7.613,4.366);
\filldraw[fill opacity=0.8,fill=gray!20](-6.901,.604)--(-6.952,.605)--(-6.952,.605)--(-6.895,.597)--cycle;
\filldraw[fill opacity=0.8,fill=gray!20,draw=none](-7.481,4.786)--(-7.491,4.786)--(-7.502,4.76)--(-7.492,4.675)--(-7.453,4.765)--cycle;
\draw(-7.491,4.786)--(-7.502,4.76);
\draw(-7.492,4.675)--(-7.453,4.765);
\filldraw[fill opacity=0.8,fill=gray!20,draw=none](-7.425,4.761)--(-7.43,4.773)--(-7.461,4.753)--(-7.451,4.721)--cycle;
\draw(-7.425,4.761)--(-7.43,4.773)--(-7.461,4.753)--(-7.451,4.721);
\filldraw[fill opacity=0.8,fill=gray!20,draw=none](-8.461,3.21)--(-8.457,3.213)--(-8.454,3.212)--cycle;
\draw(-8.457,3.213)--(-8.454,3.212);
\filldraw[fill opacity=0.8,fill=gray!20,draw=none](-6.252,.518)--(-6.252,.513)--(-6.259,.512)--(-6.259,.543)--cycle;
\draw(-6.252,.518)--(-6.252,.513);
\draw(-6.259,.512)--(-6.259,.543);
\filldraw[fill opacity=0.8,fill=gray!20,draw=none](-7.583,4.411)--(-7.589,4.384)--(-7.583,4.371)--(-7.58,4.379)--cycle;
\draw(-7.583,4.371)--(-7.58,4.379);
\filldraw[fill opacity=0.8,fill=gray!20,draw=none](-7.488,4.631)--(-7.488,4.631)--(-7.488,4.63)--(-7.485,4.62)--(-7.486,4.623)--cycle;
\draw(-7.488,4.631)--(-7.488,4.63);
\draw(-7.485,4.62)--(-7.486,4.623);
\filldraw[fill opacity=0.8,fill=gray!20,draw=none](-6.244,.263)--(-6.25,.254)--(-6.192,.253)--(-6.192,.265)--(-6.201,.272)--(-6.227,.272)--cycle;
\draw(-6.25,.254)--(-6.192,.253);
\draw(-6.201,.272)--(-6.227,.272);
\filldraw[fill opacity=0.8,fill=gray!20,draw=none](-6.259,.325)--(-6.259,.255)--(-6.244,.253)--(-6.244,.329)--cycle;
\draw(-6.259,.325)--(-6.259,.255);
\draw(-6.244,.253)--(-6.244,.329);
\filldraw[fill opacity=0.8,fill=gray!20](-7.783,1.269)--(-7.785,1.289)--(-7.74,1.286)--(-7.719,1.264)--cycle;
\filldraw[fill opacity=0.8,fill=gray!20](-7.848,1.266)--(-7.831,1.287)--(-7.785,1.289)--(-7.783,1.269)--cycle;
\filldraw[fill opacity=0.8,fill=gray!20,draw=none](-8.486,3.163)--(-8.452,3.195)--(-8.464,3.207)--(-8.506,3.177)--cycle;
\draw(-8.486,3.163)--(-8.452,3.195)--(-8.464,3.207)--(-8.506,3.177);
\filldraw[fill opacity=0.8,fill=gray!20,draw=none](-7.749,.514)--(-7.754,.507)--(-7.742,.507)--cycle;
\draw(-7.754,.507)--(-7.742,.507);
\filldraw[fill opacity=0.8,fill=gray!20,draw=none](-7.829,.551)--(-7.82,.56)--(-7.84,.564)--cycle;
\draw(-7.829,.551)--(-7.82,.56);
\filldraw[fill opacity=0.8,fill=gray!20,draw=none](-7.713,4.49)--(-7.73,4.517)--(-7.79,4.543)--(-7.759,4.502)--cycle;
\draw(-7.79,4.543)--(-7.759,4.502)--(-7.713,4.49);
\filldraw[fill opacity=0.8,fill=gray!20,draw=none](-7.43,4.773)--(-7.445,4.792)--(-7.481,4.808)--(-7.491,4.801)--(-7.461,4.753)--cycle;
\draw(-7.481,4.808)--(-7.491,4.801)--(-7.461,4.753)--(-7.43,4.773)--(-7.445,4.792);
\filldraw[fill opacity=0.8,fill=gray!20](-7.718,4.843)--(-7.671,4.87)--(-7.677,4.876)--(-7.73,4.856)--cycle;
\filldraw[fill opacity=0.8,fill=gray!20,draw=none](-6.174,.511)--(-6.159,.511)--(-6.177,.512)--cycle;
\draw(-6.174,.511)--(-6.159,.511);
\filldraw[fill opacity=0.8,fill=gray!20,draw=none](-6.176,.507)--(-6.155,.5)--(-6.159,.511)--(-6.174,.511)--cycle;
\draw(-6.159,.511)--(-6.174,.511);
\filldraw[fill opacity=0.8,fill=gray!20,draw=none](-6.114,.52)--(-6.143,.518)--(-6.155,.5)--(-6.126,.49)--cycle;
\draw(-6.143,.518)--(-6.155,.5);
\filldraw[fill opacity=0.8,fill=gray!20,draw=none](-6.126,.49)--(-6.155,.5)--(-6.169,.478)--(-6.133,.472)--cycle;
\draw(-6.155,.5)--(-6.169,.478)--(-6.133,.472);
\filldraw[fill opacity=0.8,fill=gray!20,draw=none](-6.173,.453)--(-6.128,.453)--(-6.117,.461)--(-6.115,.49)--(-6.151,.49)--cycle;
\draw(-6.173,.453)--(-6.128,.453);
\draw(-6.117,.461)--(-6.115,.49)--(-6.151,.49);
\filldraw[fill opacity=0.8,fill=gray!20,draw=none](-6.103,.521)--(-6.114,.52)--(-6.126,.49)--(-6.115,.486)--(-6.094,.518)--cycle;
\draw(-6.115,.486)--(-6.094,.518);
\filldraw[fill opacity=0.8,fill=gray!20,draw=none](-6.118,.502)--(-6.164,.531)--(-6.172,.513)--(-6.133,.496)--cycle;
\draw(-6.172,.513)--(-6.133,.496);
\filldraw[fill opacity=0.8,fill=gray!20](-7.714,.544)--(-7.751,.567)--(-7.769,.563)--(-7.749,.537)--cycle;
\filldraw[fill opacity=0.8,fill=gray!20,draw=none](-8.266,2.824)--(-8.285,2.838)--(-8.293,2.828)--cycle;
\draw(-8.285,2.838)--(-8.293,2.828);
\filldraw[fill opacity=0.8,fill=gray!20](-7.091,.527)--(-7.05,.566)--(-7.061,.578)--(-7.107,.545)--cycle;
\filldraw[fill opacity=0.8,fill=gray!20,draw=none](-6.956,.462)--(-6.785,.46)--(-6.784,.482)--(-6.787,.497)--(-6.954,.499)--cycle;
\draw(-6.787,.497)--(-6.954,.499)--(-6.956,.462)--(-6.785,.46);
\filldraw[fill opacity=0.8,fill=gray!20,draw=none](-6.958,.414)--(-6.788,.412)--(-6.785,.419)--(-6.782,.442)--(-6.785,.46)--(-6.956,.462)--cycle;
\draw(-6.785,.46)--(-6.956,.462)--(-6.958,.414)--(-6.788,.412);
\filldraw[fill opacity=0.8,fill=gray!20](-6.74,.444)--(-6.762,.496)--(-6.793,.475)--(-6.775,.421)--cycle;
\filldraw[fill opacity=0.8,fill=gray!20,draw=none](-8.453,3.196)--(-8.477,3.198)--(-8.494,3.186)--cycle;
\filldraw[fill opacity=0.8,fill=gray!20,draw=none](-8.887,1.323)--(-8.877,1.314)--(-8.87,1.305)--cycle;
\draw(-8.887,1.323)--(-8.877,1.314)--(-8.87,1.305);
\filldraw[fill opacity=0.8,fill=gray!20,draw=none](-8.887,1.323)--(-8.877,1.314)--(-8.87,1.305)--cycle;
\draw(-8.887,1.323)--(-8.877,1.314)--(-8.87,1.305);
\filldraw[fill opacity=0.8,fill=gray!20,draw=none](-8.887,1.323)--(-8.877,1.314)--(-8.87,1.305)--cycle;
\draw(-8.887,1.323)--(-8.877,1.314)--(-8.87,1.305);
\filldraw[fill opacity=0.8,fill=gray!20,draw=none](-8.192,2.874)--(-8.179,2.883)--(-8.188,2.887)--cycle;
\draw(-8.179,2.883)--(-8.188,2.887);
\filldraw[fill opacity=0.8,fill=gray!20,draw=none](-8.192,2.888)--(-8.17,2.879)--(-8.142,2.928)--(-8.155,2.933)--cycle;
\draw(-8.192,2.888)--(-8.17,2.879)--(-8.142,2.928)--(-8.155,2.933);
\filldraw[fill opacity=0.8,fill=gray!20,draw=none](-7.488,4.63)--(-7.486,4.609)--(-7.483,4.595)--(-7.485,4.62)--cycle;
\draw(-7.488,4.63)--(-7.486,4.609);
\draw(-7.483,4.595)--(-7.485,4.62);
\filldraw[fill opacity=0.8,fill=gray!20,draw=none](-7.789,.324)--(-7.78,.302)--(-7.756,.33)--(-7.769,.335)--cycle;
\draw(-7.78,.302)--(-7.756,.33)--(-7.769,.335);
\filldraw[fill opacity=0.8,fill=gray!20](-7.891,.367)--(-7.888,.414)--(-7.948,.429)--(-7.954,.382)--cycle;
\filldraw[fill opacity=0.8,fill=gray!20](-7.888,.414)--(-7.878,.46)--(-7.932,.474)--(-7.948,.429)--cycle;
\filldraw[fill opacity=0.8,fill=gray!20,draw=none](-7.769,.335)--(-7.769,.371)--(-7.797,.371)--(-7.796,.324)--(-7.789,.324)--cycle;
\draw(-7.769,.371)--(-7.797,.371)--(-7.796,.324)--(-7.789,.324);
\filldraw[fill opacity=0.8,fill=gray!20,draw=none](-7.792,.345)--(-7.756,.33)--(-7.731,.374)--(-7.757,.385)--cycle;
\draw(-7.792,.345)--(-7.756,.33)--(-7.731,.374)--(-7.757,.385);
\filldraw[fill opacity=0.8,fill=gray!20,draw=none](-7.677,.326)--(-7.679,.37)--(-7.705,.37)--(-7.769,.335)--(-7.769,.324)--(-7.678,.323)--cycle;
\draw(-7.679,.37)--(-7.705,.37);
\draw(-7.769,.324)--(-7.678,.323);
\filldraw[fill opacity=0.8,fill=gray!20](-7.658,.292)--(-7.643,.333)--(-7.711,.319)--(-7.72,.28)--cycle;
\filldraw[fill opacity=0.8,fill=gray!20,draw=none](-7.645,4.301)--(-7.611,4.377)--(-7.624,4.405)--(-7.667,4.309)--cycle;
\draw(-7.645,4.301)--(-7.611,4.377);
\draw(-7.624,4.405)--(-7.667,4.309);
\filldraw[fill opacity=0.8,fill=gray!20,draw=none](-7.622,4.4)--(-7.611,4.377)--(-7.615,4.427)--cycle;
\filldraw[fill opacity=0.8,fill=gray!20,draw=none](-8.425,3.211)--(-8.429,3.221)--(-8.452,3.212)--cycle;
\draw(-8.429,3.221)--(-8.452,3.212);
\filldraw[fill opacity=0.8,fill=gray!20,draw=none](-8.45,3.219)--(-8.442,3.219)--(-8.434,3.215)--(-8.454,3.212)--(-8.457,3.213)--cycle;
\draw(-8.45,3.219)--(-8.442,3.219)--(-8.434,3.215);
\draw(-8.454,3.212)--(-8.457,3.213);
\filldraw[fill opacity=0.8,fill=gray!20,draw=none](-8.062,3.878)--(-8.129,3.727)--(-8.1,3.705)--(-8.091,3.7)--(-8.02,3.86)--cycle;
\draw(-8.091,3.7)--(-8.02,3.86)--(-8.062,3.878)--(-8.129,3.727);
\filldraw[fill opacity=0.8,fill=gray!20](-8.073,3.665)--(-8.092,3.703)--(-8.157,3.719)--(-8.126,3.678)--cycle;
\filldraw[fill opacity=0.8,fill=gray!20,draw=none](-7.8,3.954)--(-7.698,4.183)--(-7.703,4.228)--(-7.81,3.989)--cycle;
\draw(-7.8,3.954)--(-7.698,4.183);
\draw(-7.703,4.228)--(-7.81,3.989);
\filldraw[fill opacity=0.8,fill=gray!20,draw=none](-7.955,1.171)--(-7.946,1.207)--(-7.923,1.223)--(-7.938,1.182)--cycle;
\draw(-7.946,1.207)--(-7.923,1.223)--(-7.938,1.182)--(-7.955,1.171);
\filldraw[fill opacity=0.8,fill=gray!20,draw=none](-7.955,1.171)--(-7.953,1.163)--(-7.957,1.162)--cycle;
\draw(-7.953,1.163)--(-7.957,1.162);
\filldraw[fill opacity=0.8,fill=gray!20,draw=none](-7.956,1.128)--(-7.973,1.117)--(-7.967,1.163)--(-7.953,1.172)--cycle;
\draw(-7.956,1.128)--(-7.973,1.117)--(-7.967,1.163)--(-7.953,1.172);
\filldraw[fill opacity=0.8,fill=gray!20,draw=none](-7.96,1.195)--(-7.841,1.222)--(-7.85,1.186)--(-7.953,1.163)--cycle;
\draw(-7.96,1.195)--(-7.841,1.222)--(-7.85,1.186)--(-7.953,1.163);
\filldraw[fill opacity=0.8,fill=gray!20,draw=none](-7.445,4.792)--(-7.465,4.818)--(-7.481,4.808)--cycle;
\draw(-7.445,4.792)--(-7.465,4.818)--(-7.481,4.808);
\filldraw[fill opacity=0.8,fill=gray!20,draw=none](-7.735,.502)--(-7.742,.507)--(-7.793,.508)--(-7.795,.471)--(-7.753,.47)--cycle;
\draw(-7.742,.507)--(-7.793,.508)--(-7.795,.471)--(-7.753,.47);
\filldraw[fill opacity=0.8,fill=gray!20,draw=none](-7.814,.499)--(-7.811,.509)--(-7.821,.537)--(-7.841,.538)--(-7.862,.503)--cycle;
\draw(-7.821,.537)--(-7.841,.538)--(-7.862,.503)--(-7.814,.499);
\filldraw[fill opacity=0.8,fill=gray!20,draw=none](-8.688,.778)--(-8.489,.823)--(-8.562,.822)--(-8.694,.792)--cycle;
\draw(-8.688,.778)--(-8.489,.823);
\draw(-8.562,.822)--(-8.694,.792);
\filldraw[fill opacity=0.8,fill=gray!20,draw=none](-8.692,.806)--(-8.714,.83)--(-8.707,.886)--(-8.686,.864)--(-8.686,.848)--(-8.691,.808)--cycle;
\draw(-8.692,.806)--(-8.714,.83)--(-8.707,.886)--(-8.686,.864);
\draw(-8.686,.848)--(-8.691,.808);
\filldraw[fill opacity=0.8,fill=gray!20,draw=none](-8.783,.812)--(-8.734,.925)--(-8.749,.9)--(-8.787,.814)--cycle;
\draw(-8.749,.9)--(-8.787,.814)--(-8.783,.812)--(-8.734,.925);
\filldraw[fill opacity=0.8,fill=gray!20,draw=none](-8.825,.932)--(-7.746,.482)--(-7.758,.518)--(-8.833,.967)--cycle;
\draw(-7.758,.518)--(-8.833,.967)--(-8.825,.932)--(-7.746,.482);
\filldraw[fill opacity=0.8,fill=gray!20,draw=none](-7.487,4.632)--(-7.486,4.624)--(-7.483,4.631)--cycle;
\draw(-7.486,4.624)--(-7.483,4.631);
\filldraw[fill opacity=0.8,fill=gray!20,draw=none](-7.485,4.621)--(-7.482,4.63)--(-7.487,4.632)--cycle;
\draw(-7.485,4.621)--(-7.482,4.63)--(-7.487,4.632);
\filldraw[fill opacity=0.8,fill=gray!20](-8.085,4.019)--(-8.038,4.046)--(-8.044,4.052)--(-8.097,4.032)--cycle;
\filldraw[fill opacity=0.8,fill=gray!20,draw=none](-7.893,3.803)--(-7.825,3.956)--(-7.857,3.979)--(-7.928,3.819)--cycle;
\draw(-7.857,3.979)--(-7.928,3.819)--(-7.893,3.803)--(-7.825,3.956);
\filldraw[fill opacity=0.8,fill=gray!20](-7.797,3.949)--(-7.832,3.994)--(-7.858,3.977)--(-7.829,3.929)--cycle;
\filldraw[fill opacity=0.8,fill=gray!20,draw=none](-7.8,3.953)--(-7.8,3.954)--(-7.81,3.989)--(-7.816,3.975)--cycle;
\draw(-7.8,3.953)--(-7.8,3.954);
\draw(-7.81,3.989)--(-7.816,3.975);
\filldraw[fill opacity=0.8,fill=gray!20,draw=none](-7.8,3.953)--(-7.816,3.975)--(-7.825,3.956)--cycle;
\draw(-7.816,3.975)--(-7.825,3.956);
\filldraw[fill opacity=0.8,fill=gray!20,draw=none](-7.789,.294)--(-7.756,.324)--(-7.796,.324)--(-7.795,.295)--cycle;
\draw(-7.756,.324)--(-7.796,.324)--(-7.795,.295);
\filldraw[fill opacity=0.8,fill=gray!20,draw=none](-7.795,.29)--(-7.785,.296)--(-7.796,.3)--cycle;
\draw(-7.795,.29)--(-7.785,.296)--(-7.796,.3);
\filldraw[fill opacity=0.8,fill=gray!20,draw=none](-8.193,3.071)--(-8.192,3.07)--(-8.193,3.071)--cycle;
\draw(-8.192,3.07)--(-8.193,3.071);
\filldraw[fill opacity=0.8,fill=gray!20,draw=none](-7.444,4.765)--(-7.427,4.758)--(-7.425,4.763)--cycle;
\draw(-7.427,4.758)--(-7.425,4.763);
\filldraw[fill opacity=0.8,fill=gray!20,draw=none](-7.442,4.79)--(-7.452,4.768)--(-7.444,4.765)--(-7.425,4.763)--(-7.418,4.78)--cycle;
\draw(-7.425,4.763)--(-7.418,4.78)--(-7.442,4.79)--(-7.452,4.768);
\filldraw[fill opacity=0.8,fill=gray!20,draw=none](-6.784,.482)--(-6.762,.496)--(-6.778,.517)--(-6.787,.497)--cycle;
\draw(-6.784,.482)--(-6.762,.496)--(-6.778,.517);
\filldraw[fill opacity=0.8,fill=gray!20,draw=none](-6.954,.499)--(-6.761,.497)--(-6.779,.518)--(-6.952,.52)--cycle;
\draw(-6.779,.518)--(-6.952,.52)--(-6.954,.499)--(-6.761,.497);
\filldraw[fill opacity=0.8,fill=gray!20,draw=none](-8.945,1.151)--(-8.945,1.154)--(-8.953,1.154)--cycle;
\draw(-8.945,1.151)--(-8.945,1.154)--(-8.953,1.154);
\filldraw[fill opacity=0.8,fill=gray!20,draw=none](-8.804,1.474)--(-8.812,1.478)--(-8.806,1.472)--cycle;
\draw(-8.812,1.478)--(-8.806,1.472)--(-8.804,1.474);
\filldraw[fill opacity=0.8,fill=gray!20,draw=none](-7.944,3.635)--(-7.942,3.634)--(-7.942,3.635)--cycle;
\draw(-7.942,3.634)--(-7.942,3.635);
\filldraw[fill opacity=0.8,fill=gray!20,draw=none](-7.944,3.635)--(-7.969,3.632)--(-8.177,3.165)--(-8.167,3.13)--(-7.942,3.634)--cycle;
\draw(-7.969,3.632)--(-8.177,3.165);
\draw(-8.167,3.13)--(-7.942,3.634);
\filldraw[fill opacity=0.8,fill=gray!20,draw=none](-8.193,3.072)--(-8.197,3.065)--(-8.185,3.06)--cycle;
\draw(-8.197,3.065)--(-8.185,3.06);
\filldraw[fill opacity=0.8,fill=gray!20,draw=none](-5.9,.361)--(-5.954,.384)--(-5.957,.405)--(-5.952,.433)--(-5.91,.414)--cycle;
\draw(-5.952,.433)--(-5.91,.414)--(-5.9,.361)--(-5.954,.384);
\filldraw[fill opacity=0.8,fill=gray!20,draw=none](-7.669,.512)--(-7.671,.506)--(-7.645,.506)--cycle;
\draw(-7.671,.506)--(-7.645,.506);
\filldraw[fill opacity=0.8,fill=gray!20](-8.44,2.841)--(-8.459,2.879)--(-8.524,2.895)--(-8.493,2.854)--cycle;
\filldraw[fill opacity=0.8,fill=gray!20,draw=none](-6.784,.267)--(-6.787,.26)--(-6.783,.259)--cycle;
\draw(-6.787,.26)--(-6.783,.259);
\filldraw[fill opacity=0.8,fill=gray!20,draw=none](-7.789,.294)--(-7.795,.295)--(-7.795,.289)--(-7.794,.289)--cycle;
\draw(-7.795,.295)--(-7.795,.289)--(-7.794,.289);
\filldraw[fill opacity=0.8,fill=gray!20](-7.649,1.04)--(-7.632,1.085)--(-7.614,1.066)--(-7.632,1.023)--cycle;
\filldraw[fill opacity=0.8,fill=gray!20,draw=none](-6.784,.267)--(-6.783,.259)--(-6.263,.254)--(-6.268,.273)--(-6.779,.278)--cycle;
\draw(-6.783,.259)--(-6.263,.254);
\draw(-6.268,.273)--(-6.779,.278);
\filldraw[fill opacity=0.8,fill=gray!20,draw=none](-7.715,4.524)--(-7.714,4.51)--(-7.73,4.517)--(-7.735,4.525)--(-7.736,4.527)--(-7.737,4.53)--cycle;
\draw(-7.715,4.524)--(-7.714,4.51);
\draw(-7.736,4.527)--(-7.737,4.53);
\filldraw[fill opacity=0.8,fill=gray!20,draw=none](-7.621,4.487)--(-7.625,4.488)--(-7.666,4.503)--(-7.714,4.521)--(-7.738,4.531)--(-7.715,4.524)--(-7.624,4.488)--cycle;
\draw(-7.625,4.488)--(-7.666,4.503)--(-7.714,4.521)--(-7.738,4.531);
\filldraw[fill opacity=0.8,fill=gray!20,draw=none](-7.671,4.87)--(-7.641,4.877)--(-7.647,4.88)--(-7.677,4.876)--cycle;
\draw(-7.647,4.88)--(-7.677,4.876)--(-7.671,4.87)--(-7.641,4.877);
\filldraw[fill opacity=0.8,fill=gray!20,draw=none](-7.68,4.875)--(-7.672,4.87)--(-7.669,4.877)--cycle;
\draw(-7.672,4.87)--(-7.669,4.877);
\filldraw[fill opacity=0.8,fill=gray!20,draw=none](-7.661,.987)--(-7.663,1.002)--(-7.644,1.001)--cycle;
\draw(-7.663,1.002)--(-7.644,1.001);
\filldraw[fill opacity=0.8,fill=gray!20,draw=none](-7.666,.992)--(-7.65,1.038)--(-7.648,1.04)--(-7.632,1.023)--(-7.661,.986)--cycle;
\draw(-7.648,1.04)--(-7.632,1.023)--(-7.661,.986)--(-7.666,.992);
\filldraw[fill opacity=0.8,fill=gray!20,draw=none](-8.452,3.195)--(-8.425,3.211)--(-8.452,3.212)--(-8.464,3.207)--cycle;
\draw(-8.452,3.212)--(-8.464,3.207)--(-8.452,3.195)--(-8.425,3.211);
\filldraw[fill opacity=0.8,fill=gray!20,draw=none](-8.224,3.155)--(-8.235,3.131)--(-8.193,3.128)--(-8.192,3.132)--cycle;
\draw(-8.224,3.155)--(-8.235,3.131);
\draw(-8.193,3.128)--(-8.192,3.132);
\filldraw[fill opacity=0.8,fill=gray!20](-8.164,3.125)--(-8.199,3.17)--(-8.225,3.153)--(-8.196,3.105)--cycle;
\filldraw[fill opacity=0.8,fill=gray!20,draw=none](-8.172,3.149)--(-8.177,3.142)--(-8.167,3.129)--(-8.167,3.13)--cycle;
\draw(-8.167,3.129)--(-8.167,3.13);
\filldraw[fill opacity=0.8,fill=gray!20,draw=none](-8.177,3.142)--(-8.184,3.131)--(-8.167,3.129)--cycle;
\filldraw[fill opacity=0.8,fill=gray!20,draw=none](-7.481,4.808)--(-7.465,4.818)--(-7.511,4.853)--(-7.529,4.841)--(-7.508,4.819)--cycle;
\draw(-7.481,4.808)--(-7.465,4.818)--(-7.511,4.853)--(-7.529,4.841)--(-7.508,4.819);
\filldraw[fill opacity=0.8,fill=gray!20](-7.632,1.085)--(-7.627,1.132)--(-7.608,1.112)--(-7.614,1.066)--cycle;
\filldraw[fill opacity=0.8,fill=gray!20,draw=none](-8.452,3.212)--(-8.434,3.215)--(-8.425,3.211)--cycle;
\draw(-8.434,3.215)--(-8.425,3.211);
\filldraw[fill opacity=0.8,fill=gray!20,draw=none](-6.107,.571)--(-6.101,.568)--(-6.078,.568)--cycle;
\filldraw[fill opacity=0.8,fill=gray!20,draw=none](-7.952,3.637)--(-7.962,3.647)--(-7.996,3.657)--(-7.994,3.635)--cycle;
\draw(-7.996,3.657)--(-7.994,3.635)--(-7.952,3.637);
\filldraw[fill opacity=0.8,fill=gray!20,draw=none](-7.965,3.64)--(-7.969,3.632)--(-7.944,3.635)--cycle;
\draw(-7.965,3.64)--(-7.969,3.632);
\filldraw[fill opacity=0.8,fill=gray!20,draw=none](-8.806,1.472)--(-8.812,1.478)--(-8.878,1.485)--(-8.867,1.461)--cycle;
\draw(-8.878,1.485)--(-8.867,1.461)--(-8.806,1.472)--(-8.812,1.478);
\filldraw[fill opacity=0.8,fill=gray!20](-7.037,.211)--(-7.056,.249)--(-7.122,.265)--(-7.091,.224)--cycle;
\filldraw[fill opacity=0.8,fill=gray!20,draw=none](-7.641,4.877)--(-7.62,4.883)--(-7.62,4.883)--(-7.647,4.88)--cycle;
\draw(-7.641,4.877)--(-7.62,4.883)--(-7.62,4.883)--(-7.647,4.88);
\filldraw[fill opacity=0.8,fill=gray!20,draw=none](-7.822,4.717)--(-7.81,4.703)--(-7.801,4.727)--cycle;
\draw(-7.822,4.717)--(-7.81,4.703)--(-7.801,4.727);
\filldraw[fill opacity=0.8,fill=gray!20](-7.05,.566)--(-7.002,.593)--(-7.008,.599)--(-7.061,.578)--cycle;
\filldraw[fill opacity=0.8,fill=gray!20,draw=none](-6.267,.45)--(-6.271,.44)--(-6.259,.435)--cycle;
\draw(-6.271,.44)--(-6.259,.435);
\filldraw[fill opacity=0.8,fill=gray!20](-7.79,.936)--(-7.788,.952)--(-7.764,.95)--(-7.79,.936)--cycle;
\filldraw[fill opacity=0.8,fill=gray!20](-7.79,.936)--(-7.812,.951)--(-7.788,.952)--(-7.79,.936)--cycle;
\filldraw[fill opacity=0.8,fill=gray!20](-7.898,1.256)--(-7.867,1.28)--(-7.831,1.287)--(-7.848,1.266)--cycle;
\filldraw[fill opacity=0.8,fill=gray!20](-7.709,.969)--(-7.675,1.001)--(-7.661,.986)--(-7.699,.958)--cycle;
\filldraw[fill opacity=0.8,fill=gray!20,draw=none](-8.038,4.046)--(-8.022,4.05)--(-8.026,4.054)--(-8.044,4.052)--cycle;
\draw(-8.026,4.054)--(-8.044,4.052)--(-8.038,4.046)--(-8.022,4.05);
\filldraw[fill opacity=0.8,fill=gray!20,draw=none](-7.928,3.819)--(-7.857,3.979)--(-7.861,3.981)--(-7.904,3.994)--(-7.973,3.839)--cycle;
\draw(-7.904,3.994)--(-7.973,3.839)--(-7.928,3.819)--(-7.857,3.979);
\filldraw[fill opacity=0.8,fill=gray!20](-7.832,3.994)--(-7.878,4.029)--(-7.896,4.017)--(-7.858,3.977)--cycle;
\filldraw[fill opacity=0.8,fill=gray!20](-8.777,1.424)--(-8.806,1.472)--(-8.867,1.461)--(-8.851,1.409)--cycle;
\filldraw[fill opacity=0.8,fill=gray!20,draw=none](-7.919,.305)--(-7.89,.285)--(-7.878,.282)--(-7.886,.315)--(-7.91,.317)--cycle;
\draw(-7.89,.285)--(-7.878,.282)--(-7.886,.315);
\filldraw[fill opacity=0.8,fill=gray!20,draw=none](-5.976,.57)--(-5.972,.567)--(-5.969,.525)--cycle;
\draw(-5.976,.57)--(-5.972,.567);
\filldraw[fill opacity=0.8,fill=gray!20,draw=none](-5.973,.568)--(-5.976,.57)--(-6.002,.567)--cycle;
\draw(-5.973,.568)--(-5.976,.57);
\filldraw[fill opacity=0.8,fill=gray!20,draw=none](-7.673,.289)--(-7.673,.287)--(-7.672,.287)--cycle;
\draw(-7.673,.287)--(-7.672,.287);
\filldraw[fill opacity=0.8,fill=gray!20,draw=none](-7.673,.287)--(-7.673,.289)--(-7.674,.289)--cycle;
\draw(-7.673,.289)--(-7.674,.289);
\filldraw[fill opacity=0.8,fill=gray!20,draw=none](-7.129,.312)--(-7.141,.315)--(-7.131,.288)--cycle;
\draw(-7.129,.312)--(-7.141,.315)--(-7.131,.288);
\filldraw[fill opacity=0.8,fill=gray!20,draw=none](-7.129,.312)--(-7.126,.317)--(-7.126,.364)--(-7.148,.37)--(-7.141,.315)--cycle;
\draw(-7.126,.364)--(-7.148,.37)--(-7.141,.315)--(-7.129,.312);
\filldraw[fill opacity=0.8,fill=gray!20,draw=none](-7.14,.317)--(-7.658,.323)--(-7.673,.289)--(-7.672,.287)--(-7.129,.282)--cycle;
\draw(-7.14,.317)--(-7.658,.323);
\draw(-7.672,.287)--(-7.129,.282);
\filldraw[fill opacity=0.8,fill=gray!20,draw=none](-8.425,3.211)--(-8.405,3.222)--(-8.411,3.228)--(-8.429,3.221)--cycle;
\draw(-8.425,3.211)--(-8.405,3.222)--(-8.411,3.228)--(-8.429,3.221);
\filldraw[fill opacity=0.8,fill=gray!20,draw=none](-7.61,4.864)--(-7.595,4.864)--(-7.62,4.883)--(-7.62,4.883)--(-7.623,4.869)--cycle;
\draw(-7.61,4.864)--(-7.595,4.864)--(-7.62,4.883)--(-7.62,4.883)--(-7.623,4.869);
\filldraw[fill opacity=0.8,fill=gray!20,draw=none](-7.623,4.869)--(-7.62,4.883)--(-7.62,4.883)--(-7.635,4.875)--cycle;
\draw(-7.623,4.869)--(-7.62,4.883)--(-7.62,4.883)--(-7.635,4.875);
\filldraw[fill opacity=0.8,fill=gray!20](-7.563,4.875)--(-7.62,4.883)--(-7.62,4.883)--(-7.573,4.869)--cycle;
\filldraw[fill opacity=0.8,fill=gray!20](-7.573,4.869)--(-7.62,4.883)--(-7.62,4.883)--(-7.595,4.864)--cycle;
\filldraw[fill opacity=0.8,fill=gray!20,draw=none](-7.635,4.875)--(-7.62,4.883)--(-7.62,4.883)--(-7.641,4.877)--cycle;
\draw(-7.635,4.875)--(-7.62,4.883)--(-7.62,4.883)--(-7.641,4.877);
\filldraw[fill opacity=0.8,fill=gray!20](-7.511,4.853)--(-7.563,4.875)--(-7.573,4.869)--(-7.529,4.841)--cycle;
\filldraw[fill opacity=0.8,fill=gray!20,draw=none](-8.562,3.011)--(-8.554,2.95)--(-8.537,2.95)--cycle;
\draw(-8.554,2.95)--(-8.537,2.95);
\filldraw[fill opacity=0.8,fill=gray!20,draw=none](-6.761,.497)--(-6.263,.492)--(-6.259,.505)--(-6.264,.512)--(-6.779,.518)--cycle;
\draw(-6.761,.497)--(-6.263,.492);
\draw(-6.264,.512)--(-6.779,.518);
\filldraw[fill opacity=0.8,fill=gray!20,draw=none](-7.794,.555)--(-7.793,.562)--(-7.816,.564)--(-7.82,.56)--cycle;
\draw(-7.794,.555)--(-7.793,.562)--(-7.816,.564)--(-7.82,.56);
\filldraw[fill opacity=0.8,fill=gray!20](-7.798,.246)--(-7.8,.276)--(-7.878,.282)--(-7.862,.25)--cycle;
\filldraw[fill opacity=0.8,fill=gray!20,draw=none](-7.528,4.49)--(-7.527,4.487)--(-7.533,4.473)--(-7.557,4.455)--cycle;
\draw(-7.528,4.49)--(-7.527,4.487);
\filldraw[fill opacity=0.8,fill=gray!20,draw=none](-7.527,4.487)--(-7.527,4.478)--(-7.533,4.473)--cycle;
\draw(-7.527,4.487)--(-7.527,4.478);
\filldraw[fill opacity=0.8,fill=gray!20,draw=none](-7.537,4.479)--(-7.529,4.484)--(-7.557,4.479)--(-7.562,4.472)--cycle;
\draw(-7.557,4.479)--(-7.562,4.472);
\filldraw[fill opacity=0.8,fill=gray!20,draw=none](-7.528,4.484)--(-7.516,4.494)--(-7.551,4.487)--(-7.557,4.479)--cycle;
\draw(-7.516,4.494)--(-7.551,4.487)--(-7.557,4.479);
\filldraw[fill opacity=0.8,fill=gray!20,draw=none](-7.512,4.496)--(-7.509,4.472)--(-7.525,4.457)--(-7.527,4.484)--cycle;
\draw(-7.525,4.457)--(-7.527,4.484);
\filldraw[fill opacity=0.8,fill=gray!20,draw=none](-7.557,4.455)--(-7.527,4.478)--(-7.525,4.457)--(-7.57,4.435)--(-7.57,4.438)--cycle;
\draw(-7.527,4.478)--(-7.525,4.457);
\draw(-7.57,4.435)--(-7.57,4.438);
\filldraw[fill opacity=0.8,fill=gray!20,draw=none](-7.516,4.494)--(-7.512,4.498)--(-7.546,4.499)--(-7.551,4.487)--cycle;
\draw(-7.546,4.499)--(-7.551,4.487)--(-7.516,4.494);
\filldraw[fill opacity=0.8,fill=gray!20,draw=none](-7.512,4.498)--(-7.512,4.496)--(-7.516,4.494)--cycle;
\filldraw[fill opacity=0.8,fill=gray!20,draw=none](-7.525,4.457)--(-7.496,4.138)--(-7.541,4.115)--(-7.57,4.435)--cycle;
\draw(-7.525,4.457)--(-7.496,4.138);
\draw(-7.541,4.115)--(-7.57,4.435);
\filldraw[fill opacity=0.8,fill=gray!20,draw=none](-7.509,4.472)--(-7.474,4.152)--(-7.485,4.138)--(-7.495,4.129)--(-7.525,4.457)--cycle;
\draw(-7.495,4.129)--(-7.525,4.457);
\filldraw[fill opacity=0.8,fill=gray!20,draw=none](-7.583,4.497)--(-7.607,4.499)--(-7.611,4.49)--(-7.602,4.484)--(-7.589,4.481)--cycle;
\filldraw[fill opacity=0.8,fill=gray!20,draw=none](-7.608,4.496)--(-7.579,4.493)--(-7.521,4.644)--(-7.535,4.65)--cycle;
\draw(-7.521,4.644)--(-7.535,4.65);
\filldraw[fill opacity=0.8,fill=gray!20,draw=none](-7.608,4.496)--(-7.535,4.65)--(-7.547,4.654)--(-7.609,4.496)--cycle;
\draw(-7.535,4.65)--(-7.547,4.654)--(-7.609,4.496);
\filldraw[fill opacity=0.8,fill=gray!20,draw=none](-7.673,4.467)--(-7.663,4.464)--(-7.634,4.475)--(-7.658,4.485)--(-7.701,4.488)--(-7.704,4.487)--(-7.697,4.48)--cycle;
\draw(-7.658,4.485)--(-7.701,4.488);
\draw(-7.704,4.487)--(-7.697,4.48);
\filldraw[fill opacity=0.8,fill=gray!20,draw=none](-7.633,4.474)--(-7.662,4.456)--(-7.657,4.433)--(-7.647,4.443)--(-7.629,4.473)--cycle;
\draw(-7.657,4.433)--(-7.647,4.443)--(-7.629,4.473);
\filldraw[fill opacity=0.8,fill=gray!20,draw=none](-7.625,4.48)--(-7.633,4.474)--(-7.629,4.473)--(-7.626,4.477)--cycle;
\draw(-7.629,4.473)--(-7.626,4.477);
\filldraw[fill opacity=0.8,fill=gray!20,draw=none](-7.621,4.487)--(-7.624,4.488)--(-7.619,4.486)--cycle;
\filldraw[fill opacity=0.8,fill=gray!20,draw=none](-7.607,4.66)--(-7.666,4.503)--(-7.625,4.488)--(-7.561,4.643)--cycle;
\draw(-7.561,4.643)--(-7.607,4.66)--(-7.666,4.503)--(-7.625,4.488);
\filldraw[fill opacity=0.8,fill=gray!20,draw=none](-7.627,4.48)--(-7.625,4.48)--(-7.616,4.497)--(-7.629,4.491)--cycle;
\filldraw[fill opacity=0.8,fill=gray!20,draw=none](-7.625,4.488)--(-7.62,4.486)--(-7.561,4.643)--cycle;
\draw(-7.625,4.488)--(-7.62,4.486)--(-7.561,4.643);
\filldraw[fill opacity=0.8,fill=gray!20,draw=none](-7.623,4.494)--(-7.611,4.523)--(-7.615,4.523)--(-7.629,4.491)--cycle;
\draw(-7.615,4.523)--(-7.629,4.491);
\filldraw[fill opacity=0.8,fill=gray!20,draw=none](-7.611,4.523)--(-7.561,4.643)--(-7.615,4.523)--cycle;
\draw(-7.561,4.643)--(-7.615,4.523);
\filldraw[fill opacity=0.8,fill=gray!20,draw=none](-7.634,4.475)--(-7.629,4.477)--(-7.629,4.48)--(-7.636,4.484)--(-7.658,4.485)--cycle;
\draw(-7.629,4.477)--(-7.629,4.48);
\draw(-7.636,4.484)--(-7.658,4.485);
\filldraw[fill opacity=0.8,fill=gray!20](-7.655,4.678)--(-7.714,4.521)--(-7.666,4.503)--(-7.607,4.66)--cycle;
\filldraw[fill opacity=0.8,fill=gray!20,draw=none](-7.632,4.483)--(-7.561,4.643)--(-7.606,4.663)--(-7.674,4.51)--cycle;
\draw(-7.632,4.483)--(-7.561,4.643)--(-7.606,4.663)--(-7.674,4.51);
\filldraw[fill opacity=0.8,fill=gray!20,draw=none](-7.649,4.506)--(-7.63,4.493)--(-7.615,4.492)--(-7.547,4.654)--(-7.581,4.667)--cycle;
\draw(-7.547,4.654)--(-7.581,4.667);
\filldraw[fill opacity=0.8,fill=gray!20,draw=none](-7.532,4.691)--(-7.616,4.5)--(-7.612,4.499)--cycle;
\draw(-7.532,4.691)--(-7.616,4.5);
\filldraw[fill opacity=0.8,fill=gray!20,draw=none](-7.649,4.506)--(-7.581,4.667)--(-7.598,4.673)--(-7.66,4.513)--cycle;
\draw(-7.581,4.667)--(-7.598,4.673)--(-7.66,4.513);
\filldraw[fill opacity=0.8,fill=gray!20,draw=none](-7.553,4.706)--(-7.583,4.71)--(-7.656,4.512)--(-7.63,4.495)--(-7.618,4.494)--(-7.532,4.691)--cycle;
\draw(-7.618,4.494)--(-7.532,4.691);
\filldraw[fill opacity=0.8,fill=gray!20,draw=none](-7.607,4.499)--(-7.616,4.5)--(-7.618,4.494)--(-7.611,4.49)--cycle;
\draw(-7.616,4.5)--(-7.618,4.494);
\filldraw[fill opacity=0.8,fill=gray!20,draw=none](-7.615,4.492)--(-7.61,4.492)--(-7.547,4.654)--cycle;
\draw(-7.61,4.492)--(-7.547,4.654);
\filldraw[fill opacity=0.8,fill=gray!20](-7.535,4.524)--(-7.525,4.572)--(-7.633,4.567)--(-7.632,4.52)--cycle;
\filldraw[fill opacity=0.8,fill=gray!20,draw=none](-7.474,4.152)--(-7.505,4.432)--(-7.488,4.441)--(-7.463,4.166)--cycle;
\draw(-7.488,4.441)--(-7.463,4.166);
\filldraw[fill opacity=0.8,fill=gray!20,draw=none](-7.575,4.57)--(-7.525,4.572)--(-7.522,4.626)--(-7.534,4.625)--cycle;
\draw(-7.575,4.57)--(-7.525,4.572)--(-7.522,4.626)--(-7.534,4.625);
\filldraw[fill opacity=0.8,fill=gray!20,draw=none](-7.511,4.628)--(-7.522,4.632)--(-7.522,4.626)--cycle;
\draw(-7.522,4.632)--(-7.522,4.626)--(-7.511,4.628);
\filldraw[fill opacity=0.8,fill=gray!20,draw=none](-7.7,4.517)--(-7.674,4.51)--(-7.606,4.663)--(-7.653,4.684)--cycle;
\draw(-7.674,4.51)--(-7.606,4.663)--(-7.653,4.684);
\filldraw[fill opacity=0.8,fill=gray!20,draw=none](-7.698,4.695)--(-7.742,4.578)--(-7.737,4.53)--(-7.714,4.521)--(-7.655,4.678)--cycle;
\draw(-7.737,4.53)--(-7.714,4.521)--(-7.655,4.678)--(-7.698,4.695)--(-7.742,4.578);
\filldraw[fill opacity=0.8,fill=gray!20,draw=none](-7.582,4.569)--(-7.575,4.57)--(-7.534,4.625)--cycle;
\draw(-7.582,4.569)--(-7.575,4.57);
\filldraw[fill opacity=0.8,fill=gray!20,draw=none](-7.7,4.517)--(-7.653,4.684)--(-7.724,4.524)--cycle;
\draw(-7.653,4.684)--(-7.724,4.524);
\filldraw[fill opacity=0.8,fill=gray!20,draw=none](-7.703,4.524)--(-7.66,4.513)--(-7.598,4.673)--(-7.619,4.682)--cycle;
\draw(-7.66,4.513)--(-7.598,4.673)--(-7.619,4.682);
\filldraw[fill opacity=0.8,fill=gray!20,draw=none](-7.583,4.71)--(-7.667,4.519)--(-7.656,4.512)--cycle;
\draw(-7.583,4.71)--(-7.667,4.519);
\filldraw[fill opacity=0.8,fill=gray!20,draw=none](-7.6,4.718)--(-7.689,4.525)--(-7.667,4.519)--(-7.583,4.71)--cycle;
\draw(-7.667,4.519)--(-7.583,4.71);
\filldraw[fill opacity=0.8,fill=gray!20,draw=none](-7.582,4.569)--(-7.534,4.625)--(-7.633,4.621)--(-7.633,4.567)--cycle;
\draw(-7.534,4.625)--(-7.633,4.621)--(-7.633,4.567)--(-7.582,4.569);
\filldraw[fill opacity=0.8,fill=gray!20,draw=none](-7.534,4.625)--(-7.522,4.626)--(-7.522,4.632)--(-7.529,4.637)--cycle;
\draw(-7.534,4.625)--(-7.522,4.626)--(-7.522,4.632);
\filldraw[fill opacity=0.8,fill=gray!20,draw=none](-7.625,4.621)--(-7.534,4.625)--(-7.529,4.637)--(-7.548,4.653)--(-7.597,4.663)--(-7.611,4.659)--cycle;
\draw(-7.625,4.621)--(-7.534,4.625);
\filldraw[fill opacity=0.8,fill=gray!20,draw=none](-7.589,4.384)--(-7.583,4.411)--(-7.585,4.433)--(-7.588,4.43)--(-7.598,4.403)--cycle;
\draw(-7.588,4.43)--(-7.598,4.403);
\filldraw[fill opacity=0.8,fill=gray!20,draw=none](-7.585,4.433)--(-7.583,4.411)--(-7.577,4.442)--cycle;
\filldraw[fill opacity=0.8,fill=gray!20,draw=none](-7.622,4.4)--(-7.615,4.427)--(-7.624,4.405)--cycle;
\draw(-7.615,4.427)--(-7.624,4.405);
\filldraw[fill opacity=0.8,fill=gray!20,draw=none](-7.667,4.309)--(-7.626,4.401)--(-7.647,4.409)--(-7.672,4.394)--(-7.704,4.323)--cycle;
\draw(-7.667,4.309)--(-7.626,4.401);
\draw(-7.672,4.394)--(-7.704,4.323);
\filldraw[fill opacity=0.8,fill=gray!20,draw=none](-7.639,4.298)--(-7.607,4.381)--(-7.646,4.395)--(-7.649,4.392)--(-7.68,4.313)--cycle;
\draw(-7.649,4.392)--(-7.68,4.313)--(-7.639,4.298)--(-7.607,4.381);
\filldraw[fill opacity=0.8,fill=gray!20,draw=none](-7.626,4.401)--(-7.615,4.426)--(-7.621,4.425)--(-7.647,4.409)--cycle;
\draw(-7.626,4.401)--(-7.615,4.426);
\filldraw[fill opacity=0.8,fill=gray!20,draw=none](-7.607,4.381)--(-7.593,4.416)--(-7.619,4.411)--(-7.646,4.395)--cycle;
\draw(-7.607,4.381)--(-7.593,4.416);
\filldraw[fill opacity=0.8,fill=gray!20,draw=none](-7.593,4.416)--(-7.588,4.43)--(-7.619,4.411)--cycle;
\draw(-7.593,4.416)--(-7.588,4.43);
\filldraw[fill opacity=0.8,fill=gray!20,draw=none](-7.588,4.43)--(-7.583,4.433)--(-7.57,4.435)--(-7.541,4.115)--(-7.555,4.111)--(-7.592,4.116)--(-7.618,4.399)--cycle;
\draw(-7.57,4.435)--(-7.541,4.115);
\draw(-7.592,4.116)--(-7.618,4.399);
\filldraw[fill opacity=0.8,fill=gray!20,draw=none](-7.471,4.466)--(-7.449,4.222)--(-7.463,4.166)--(-7.488,4.441)--cycle;
\draw(-7.471,4.466)--(-7.449,4.222);
\draw(-7.463,4.166)--(-7.488,4.441);
\filldraw[fill opacity=0.8,fill=gray!20,draw=none](-7.555,4.086)--(-7.555,4.103)--(-7.466,4.137)--(-7.447,4.136)--(-7.451,4.079)--cycle;
\draw(-7.466,4.137)--(-7.447,4.136)--(-7.451,4.079)--(-7.555,4.086)--(-7.555,4.103);
\filldraw[fill opacity=0.8,fill=gray!20,draw=none](-7.556,4.038)--(-7.555,4.086)--(-7.528,4.084)--cycle;
\draw(-7.556,4.038)--(-7.555,4.086)--(-7.528,4.084);
\filldraw[fill opacity=0.8,fill=gray!20,draw=none](-7.556,4.03)--(-7.556,4.038)--(-7.528,4.084)--(-7.451,4.079)--(-7.463,4.023)--cycle;
\draw(-7.528,4.084)--(-7.451,4.079)--(-7.463,4.023)--(-7.556,4.03)--(-7.556,4.038);
\filldraw[fill opacity=0.8,fill=gray!20](-7.558,3.978)--(-7.556,4.03)--(-7.463,4.023)--(-7.482,3.972)--cycle;
\filldraw[fill opacity=0.8,fill=gray!20,draw=none](-7.496,4.138)--(-7.485,4.02)--(-7.53,4)--(-7.541,4.115)--cycle;
\draw(-7.496,4.138)--(-7.485,4.02)--(-7.53,4)--(-7.541,4.115);
\filldraw[fill opacity=0.8,fill=gray!20,draw=none](-7.541,4.115)--(-7.54,4.109)--(-7.555,4.111)--cycle;
\draw(-7.541,4.115)--(-7.54,4.109);
\filldraw[fill opacity=0.8,fill=gray!20,draw=none](-7.485,4.138)--(-7.495,4.126)--(-7.495,4.129)--cycle;
\draw(-7.495,4.126)--(-7.495,4.129);
\filldraw[fill opacity=0.8,fill=gray!20,draw=none](-7.555,4.103)--(-7.555,4.144)--(-7.466,4.137)--cycle;
\draw(-7.555,4.103)--(-7.555,4.144)--(-7.466,4.137);
\filldraw[fill opacity=0.8,fill=gray!20,draw=none](-7.462,4.161)--(-7.469,4.137)--(-7.485,4.138)--cycle;
\draw(-7.469,4.137)--(-7.485,4.138);
\filldraw[fill opacity=0.8,fill=gray!20,draw=none](-7.474,4.152)--(-7.462,4.043)--(-7.485,4.02)--(-7.495,4.126)--cycle;
\draw(-7.462,4.043)--(-7.485,4.02)--(-7.495,4.126);
\filldraw[fill opacity=0.8,fill=gray!20,draw=none](-7.474,4.149)--(-7.474,4.152)--(-7.463,4.166)--(-7.462,4.161)--cycle;
\draw(-7.463,4.166)--(-7.462,4.161);
\filldraw[fill opacity=0.8,fill=gray!20,draw=none](-7.469,4.137)--(-7.453,4.19)--(-7.451,4.19)--(-7.447,4.136)--cycle;
\draw(-7.453,4.19)--(-7.451,4.19)--(-7.447,4.136)--(-7.469,4.137);
\filldraw[fill opacity=0.8,fill=gray!20,draw=none](-7.433,4.074)--(-7.451,4.079)--(-7.447,4.136)--(-7.439,4.134)--cycle;
\draw(-7.433,4.074)--(-7.451,4.079)--(-7.447,4.136)--(-7.439,4.134);
\filldraw[fill opacity=0.8,fill=gray!20](-7.463,4.023)--(-7.451,4.079)--(-7.378,4.061)--(-7.398,4.007)--cycle;
\filldraw[fill opacity=0.8,fill=gray!20,draw=none](-7.439,4.134)--(-7.447,4.136)--(-7.451,4.19)--(-7.446,4.188)--cycle;
\draw(-7.439,4.134)--(-7.447,4.136)--(-7.451,4.19)--(-7.446,4.188);
\filldraw[fill opacity=0.8,fill=gray!20,draw=none](-7.453,4.19)--(-7.452,4.193)--(-7.451,4.19)--cycle;
\draw(-7.452,4.193)--(-7.451,4.19)--(-7.453,4.19);
\filldraw[fill opacity=0.8,fill=gray!20,draw=none](-7.459,4.181)--(-7.449,4.222)--(-7.447,4.203)--(-7.452,4.193)--cycle;
\draw(-7.449,4.222)--(-7.447,4.203);
\filldraw[fill opacity=0.8,fill=gray!20,draw=none](-7.452,4.193)--(-7.447,4.203)--(-7.447,4.2)--cycle;
\draw(-7.447,4.203)--(-7.447,4.2);
\filldraw[fill opacity=0.8,fill=gray!20,draw=none](-7.446,4.188)--(-7.451,4.19)--(-7.463,4.237)--(-7.451,4.235)--cycle;
\draw(-7.446,4.188)--(-7.451,4.19)--(-7.463,4.237)--(-7.451,4.235);
\filldraw[fill opacity=0.8,fill=gray!20,draw=none](-7.452,4.193)--(-7.447,4.2)--(-7.437,4.09)--(-7.452,4.052)--(-7.463,4.166)--cycle;
\draw(-7.447,4.2)--(-7.437,4.09)--(-7.452,4.052)--(-7.463,4.166);
\filldraw[fill opacity=0.8,fill=gray!20,draw=none](-7.459,4.181)--(-7.452,4.193)--(-7.463,4.166)--cycle;
\filldraw[fill opacity=0.8,fill=gray!20,draw=none](-7.462,4.161)--(-7.485,4.138)--(-7.555,4.144)--(-7.555,4.197)--(-7.453,4.19)--cycle;
\draw(-7.485,4.138)--(-7.555,4.144)--(-7.555,4.197)--(-7.453,4.19);
\filldraw[fill opacity=0.8,fill=gray!20,draw=none](-7.453,4.19)--(-7.555,4.197)--(-7.556,4.244)--(-7.463,4.237)--(-7.452,4.193)--cycle;
\draw(-7.453,4.19)--(-7.555,4.197)--(-7.556,4.244)--(-7.463,4.237)--(-7.452,4.193);
\filldraw[fill opacity=0.8,fill=gray!20](-7.556,4.244)--(-7.558,4.281)--(-7.482,4.276)--(-7.463,4.237)--cycle;
\filldraw[fill opacity=0.8,fill=gray!20,draw=none](-7.452,4.238)--(-7.451,4.235)--(-7.463,4.237)--(-7.482,4.276)--(-7.475,4.274)--cycle;
\draw(-7.451,4.235)--(-7.463,4.237)--(-7.482,4.276)--(-7.475,4.274);
\filldraw[fill opacity=0.8,fill=gray!20,draw=none](-7.558,4.281)--(-7.561,4.305)--(-7.536,4.303)--(-7.515,4.297)--(-7.491,4.285)--(-7.482,4.276)--cycle;
\draw(-7.491,4.285)--(-7.482,4.276)--(-7.558,4.281)--(-7.561,4.305)--(-7.536,4.303);
\filldraw[fill opacity=0.8,fill=gray!20,draw=none](-7.475,4.274)--(-7.482,4.276)--(-7.491,4.285)--cycle;
\draw(-7.475,4.274)--(-7.482,4.276)--(-7.491,4.285);
\filldraw[fill opacity=0.8,fill=gray!20](-7.558,4.76)--(-7.505,4.187)--(-7.466,4.163)--(-7.518,4.736)--cycle;
\filldraw[fill opacity=0.8,fill=gray!20,draw=none](-6.259,.325)--(-6.244,.329)--(-6.244,.368)--cycle;
\draw(-6.244,.329)--(-6.244,.368);
\filldraw[fill opacity=0.8,fill=gray!20,draw=none](-7.673,.289)--(-7.658,.323)--(-7.678,.323)--cycle;
\draw(-7.658,.323)--(-7.678,.323);
\filldraw[fill opacity=0.8,fill=gray!20,draw=none](-6.778,.517)--(-6.797,.541)--(-6.822,.524)--(-6.795,.478)--cycle;
\draw(-6.778,.517)--(-6.797,.541)--(-6.822,.524)--(-6.795,.478);
\filldraw[fill opacity=0.8,fill=gray!20,draw=none](-8.405,3.222)--(-8.389,3.226)--(-8.394,3.23)--(-8.411,3.228)--cycle;
\draw(-8.394,3.23)--(-8.411,3.228)--(-8.405,3.222)--(-8.389,3.226);
\filldraw[fill opacity=0.8,fill=gray!20,draw=none](-8.228,3.157)--(-8.271,3.17)--(-8.285,3.138)--(-8.235,3.131)--(-8.224,3.154)--cycle;
\draw(-8.271,3.17)--(-8.285,3.138);
\draw(-8.235,3.131)--(-8.224,3.154);
\filldraw[fill opacity=0.8,fill=gray!20](-8.199,3.17)--(-8.245,3.205)--(-8.263,3.193)--(-8.225,3.153)--cycle;
\filldraw[fill opacity=0.8,fill=gray!20,draw=none](-8.184,2.898)--(-8.155,2.933)--(-8.166,2.938)--cycle;
\draw(-8.155,2.933)--(-8.166,2.938);
\filldraw[fill opacity=0.8,fill=gray!20,draw=none](-8.184,2.898)--(-8.173,2.922)--(-8.195,2.891)--(-8.196,2.89)--cycle;
\draw(-8.195,2.891)--(-8.196,2.89)--(-8.184,2.898);
\filldraw[fill opacity=0.8,fill=gray!20,draw=none](-5.957,.405)--(-5.96,.436)--(-5.952,.433)--cycle;
\draw(-5.96,.436)--(-5.952,.433);
\filldraw[fill opacity=0.8,fill=gray!20](-7.79,.936)--(-7.83,.947)--(-7.812,.951)--(-7.79,.936)--cycle;
\filldraw[fill opacity=0.8,fill=gray!20,draw=none](-7.008,1.118)--(-7.032,1.16)--(-7.032,1.166)--(-6.985,1.175)--(-6.988,1.121)--cycle;
\draw(-7.032,1.166)--(-6.985,1.175)--(-6.988,1.121)--(-7.008,1.118);
\filldraw[fill opacity=0.8,fill=gray!20,draw=none](-7.006,1.171)--(-7.04,1.196)--(-7.042,1.21)--(-7.015,1.215)--(-6.977,1.215)--(-6.985,1.175)--cycle;
\draw(-7.042,1.21)--(-7.015,1.215);
\draw(-6.977,1.215)--(-6.985,1.175)--(-7.006,1.171);
\filldraw[fill opacity=0.8,fill=gray!20,draw=none](-7.006,1.171)--(-7.035,1.166)--(-7.04,1.196)--cycle;
\draw(-7.006,1.171)--(-7.035,1.166);
\filldraw[fill opacity=0.8,fill=gray!20,draw=none](-7.032,1.16)--(-7.035,1.166)--(-7.032,1.166)--cycle;
\draw(-7.035,1.166)--(-7.032,1.166);
\filldraw[fill opacity=0.8,fill=gray!20,draw=none](-6.868,1.154)--(-7.622,1.184)--(-7.637,1.204)--(-7.628,1.219)--(-6.874,1.189)--cycle;
\draw(-7.628,1.219)--(-6.874,1.189)--(-6.868,1.154)--(-7.622,1.184);
\filldraw[fill opacity=0.8,fill=gray!20](-7.627,1.132)--(-7.632,1.178)--(-7.614,1.158)--(-7.608,1.112)--cycle;
\filldraw[fill opacity=0.8,fill=gray!20](-7.749,.537)--(-7.769,.563)--(-7.793,.562)--(-7.796,.535)--cycle;
\filldraw[fill opacity=0.5,fill=gray!20](-9,-.705)--(-9,-.906)--(-9.43,-.892)--(-9.382,-.692)--cycle;
\filldraw[fill opacity=0.8,fill=gray!20,draw=none](-7.928,3.819)--(-7.999,3.659)--(-7.996,3.657)--(-7.962,3.647)--(-7.893,3.803)--cycle;
\draw(-7.962,3.647)--(-7.893,3.803)--(-7.928,3.819)--(-7.999,3.659);
\filldraw[fill opacity=0.8,fill=gray!20](-7.994,3.635)--(-7.997,3.659)--(-8.073,3.665)--(-8.048,3.639)--cycle;
\filldraw[fill opacity=0.8,fill=gray!20](-7.962,4.04)--(-7.987,4.059)--(-7.987,4.059)--(-7.991,4.039)--cycle;
\filldraw[fill opacity=0.8,fill=gray!20,draw=none](-8.062,3.878)--(-7.99,4.039)--(-8.009,4.047)--(-8.022,4.05)--(-8.092,3.892)--cycle;
\draw(-8.022,4.05)--(-8.092,3.892)--(-8.062,3.878)--(-7.99,4.039);
\filldraw[fill opacity=0.8,fill=gray!20,draw=none](-8.016,4.043)--(-7.987,4.059)--(-7.987,4.059)--(-8.022,4.05)--cycle;
\draw(-8.016,4.043)--(-7.987,4.059)--(-7.987,4.059)--(-8.022,4.05);
\filldraw[fill opacity=0.8,fill=gray!20](-7.931,4.051)--(-7.987,4.059)--(-7.987,4.059)--(-7.94,4.045)--cycle;
\filldraw[fill opacity=0.8,fill=gray!20](-7.94,4.045)--(-7.987,4.059)--(-7.987,4.059)--(-7.962,4.04)--cycle;
\filldraw[fill opacity=0.8,fill=gray!20](-7.878,4.029)--(-7.931,4.051)--(-7.94,4.045)--(-7.896,4.017)--cycle;
\filldraw[fill opacity=0.8,fill=gray!20](-7.991,4.039)--(-7.987,4.059)--(-7.987,4.059)--(-8.019,4.041)--cycle;
\filldraw[fill opacity=0.8,fill=gray!20,draw=none](-7.857,.364)--(-7.85,.369)--(-7.831,.41)--(-7.888,.414)--(-7.891,.367)--cycle;
\draw(-7.831,.41)--(-7.888,.414)--(-7.891,.367)--(-7.857,.364);
\filldraw[fill opacity=0.8,fill=gray!20,draw=none](-8.78,.732)--(-8.753,.746)--(-8.764,.728)--cycle;
\draw(-8.753,.746)--(-8.764,.728)--(-8.78,.732);
\filldraw[fill opacity=0.8,fill=gray!20,draw=none](-8.764,.728)--(-8.753,.746)--(-8.725,.767)--(-8.713,.755)--(-8.748,.711)--cycle;
\draw(-8.725,.767)--(-8.713,.755)--(-8.748,.711)--(-8.764,.728)--(-8.753,.746);
\filldraw[fill opacity=0.8,fill=gray!20,draw=none](-8.733,.738)--(-8.756,.705)--(-8.748,.711)--(-8.713,.755)--(-8.726,.746)--cycle;
\draw(-8.756,.705)--(-8.748,.711)--(-8.713,.755)--(-8.726,.746);
\filldraw[fill opacity=0.8,fill=gray!20,draw=none](-8.853,.757)--(-8.701,.791)--(-8.691,.808)--(-8.695,.826)--(-8.844,.792)--cycle;
\draw(-8.695,.826)--(-8.844,.792)--(-8.853,.757)--(-8.701,.791);
\filldraw[fill opacity=0.8,fill=gray!20,draw=none](-8.898,.76)--(-7.785,.296)--(-7.78,.302)--(-7.799,.348)--(-8.868,.794)--cycle;
\draw(-7.799,.348)--(-8.868,.794)--(-8.898,.76)--(-7.785,.296)--(-7.78,.302);
\filldraw[fill opacity=0.5,fill=gray!20](-9.5,-.955)--(-9.572,-.993)--(-10.005,-.844)--(-9.92,-.811)--cycle;
\filldraw[fill opacity=0.8,fill=gray!20,draw=none](-7.82,.544)--(-7.819,.56)--(-7.82,.56)--(-7.829,.551)--cycle;
\draw(-7.82,.56)--(-7.829,.551);
\filldraw[fill opacity=0.8,fill=gray!20,draw=none](-7.673,.469)--(-7.673,.467)--(-7.658,.47)--(-7.667,.485)--cycle;
\draw(-7.673,.467)--(-7.658,.47)--(-7.667,.485);
\filldraw[fill opacity=0.8,fill=gray!20,draw=none](-7.678,.421)--(-7.677,.418)--(-7.643,.425)--(-7.658,.47)--(-7.673,.467)--cycle;
\draw(-7.677,.418)--(-7.643,.425)--(-7.658,.47)--(-7.673,.467);
\filldraw[fill opacity=0.8,fill=gray!20,draw=none](-7.126,.501)--(-7.671,.506)--(-7.663,.469)--(-7.129,.464)--cycle;
\draw(-7.126,.501)--(-7.671,.506);
\draw(-7.663,.469)--(-7.129,.464);
\filldraw[fill opacity=0.8,fill=gray!20,draw=none](-7.82,.544)--(-7.81,.536)--(-7.796,.535)--(-7.794,.555)--(-7.819,.56)--cycle;
\draw(-7.81,.536)--(-7.796,.535)--(-7.794,.555);
\filldraw[fill opacity=0.8,fill=gray!20](-7.878,.46)--(-7.862,.503)--(-7.906,.514)--(-7.932,.474)--cycle;
\filldraw[fill opacity=0.8,fill=gray!20,draw=none](-7.749,.436)--(-7.753,.47)--(-7.795,.471)--(-7.796,.423)--(-7.758,.422)--cycle;
\draw(-7.753,.47)--(-7.795,.471)--(-7.796,.423)--(-7.758,.422);
\filldraw[fill opacity=0.8,fill=gray!20,draw=none](-7.818,.456)--(-7.814,.464)--(-7.814,.499)--(-7.862,.503)--(-7.878,.46)--cycle;
\draw(-7.814,.499)--(-7.862,.503)--(-7.878,.46)--(-7.818,.456);
\filldraw[fill opacity=0.8,fill=gray!20,draw=none](-8.701,.791)--(-8.562,.822)--(-8.68,.827)--cycle;
\draw(-8.701,.791)--(-8.562,.822);
\filldraw[fill opacity=0.8,fill=gray!20,draw=none](-8.694,.831)--(-8.694,.828)--(-8.696,.803)--(-8.692,.806)--(-8.686,.848)--cycle;
\draw(-8.696,.803)--(-8.692,.806)--(-8.686,.848);
\filldraw[fill opacity=0.8,fill=gray!20,draw=none](-8.691,.808)--(-8.68,.827)--(-8.687,.827)--(-8.695,.826)--cycle;
\draw(-8.687,.827)--(-8.695,.826);
\filldraw[fill opacity=0.8,fill=gray!20,draw=none](-8.787,.814)--(-8.749,.9)--(-8.788,.87)--(-8.808,.823)--cycle;
\draw(-8.788,.87)--(-8.808,.823)--(-8.787,.814)--(-8.749,.9);
\filldraw[fill opacity=0.8,fill=gray!20](-8.829,.887)--(-7.716,.423)--(-7.713,.468)--(-8.825,.932)--cycle;
\filldraw[fill opacity=0.8,fill=gray!20](-7.858,3.674)--(-7.829,3.715)--(-7.902,3.7)--(-7.918,3.663)--cycle;
\filldraw[fill opacity=0.8,fill=gray!20,draw=none](-8.545,2.938)--(-8.539,2.932)--(-8.544,2.943)--cycle;
\draw(-8.539,2.932)--(-8.544,2.943);
\filldraw[fill opacity=0.8,fill=gray!20,draw=none](-7.715,4.69)--(-7.737,4.685)--(-7.746,4.621)--(-7.742,4.578)--(-7.713,4.653)--cycle;
\draw(-7.742,4.578)--(-7.713,4.653);
\filldraw[fill opacity=0.8,fill=gray!20,draw=none](-7.745,4.606)--(-7.743,4.593)--(-7.716,4.655)--(-7.716,4.695)--(-7.735,4.693)--cycle;
\draw(-7.743,4.593)--(-7.716,4.655);
\filldraw[fill opacity=0.8,fill=gray!20,draw=none](-7.716,4.702)--(-7.715,4.69)--(-7.698,4.695)--cycle;
\draw(-7.698,4.695)--(-7.716,4.702);
\filldraw[fill opacity=0.8,fill=gray!20,draw=none](-7.716,4.695)--(-7.697,4.696)--(-7.695,4.702)--(-7.716,4.712)--cycle;
\draw(-7.697,4.696)--(-7.695,4.702)--(-7.716,4.712);
\filldraw[fill opacity=0.8,fill=gray!20,draw=none](-7.715,4.69)--(-7.713,4.653)--(-7.698,4.695)--cycle;
\draw(-7.713,4.653)--(-7.698,4.695);
\filldraw[fill opacity=0.8,fill=gray!20,draw=none](-7.716,4.655)--(-7.697,4.696)--(-7.716,4.695)--cycle;
\draw(-7.716,4.655)--(-7.697,4.696);
\filldraw[fill opacity=0.8,fill=gray!20,draw=none](-7.718,4.708)--(-7.724,4.681)--(-7.702,4.703)--(-7.7,4.706)--(-7.701,4.721)--cycle;
\draw(-7.7,4.706)--(-7.701,4.721);
\filldraw[fill opacity=0.8,fill=gray!20,draw=none](-7.731,4.727)--(-7.727,4.726)--(-7.7,4.714)--(-7.714,4.72)--cycle;
\filldraw[fill opacity=0.8,fill=gray!20](-7.79,.936)--(-7.764,.95)--(-7.748,.946)--(-7.79,.936)--cycle;
\filldraw[fill opacity=0.8,fill=gray!20](-7.719,1.264)--(-7.74,1.286)--(-7.709,1.278)--(-7.675,1.253)--cycle;
\filldraw[fill opacity=0.8,fill=gray!20,draw=none](-8.453,3.196)--(-8.426,3.202)--(-8.442,3.219)--(-8.45,3.219)--(-8.477,3.198)--cycle;
\draw(-8.426,3.202)--(-8.442,3.219)--(-8.45,3.219);
\filldraw[fill opacity=0.8,fill=gray!20](-6.797,.541)--(-6.842,.575)--(-6.86,.563)--(-6.822,.524)--cycle;
\filldraw[fill opacity=0.8,fill=gray!20](-7.002,.593)--(-6.952,.605)--(-6.952,.605)--(-7.008,.599)--cycle;
\filldraw[fill opacity=0.8,fill=gray!20,draw=none](-7.735,.502)--(-7.753,.47)--(-7.687,.469)--cycle;
\draw(-7.753,.47)--(-7.687,.469);
\filldraw[fill opacity=0.8,fill=gray!20](-7.682,.511)--(-7.714,.544)--(-7.749,.537)--(-7.733,.501)--cycle;
\filldraw[fill opacity=0.5,fill=gray!20](-10.439,-.157)--(-10.9,.044)--(-11.066,.464)--(-10.605,.262)--cycle;
\filldraw[fill opacity=0.5,fill=gray!20](-10.918,-.028)--(-10.9,.044)--(-11.066,.464)--(-11.092,.411)--cycle;
\filldraw[fill opacity=0.8,fill=gray!20](-8.361,2.811)--(-8.364,2.835)--(-8.44,2.841)--(-8.415,2.815)--cycle;
\filldraw[fill opacity=0.8,fill=gray!20,draw=none](-8.376,3.223)--(-8.389,3.226)--(-8.4,3.2)--(-8.365,3.196)--(-8.357,3.215)--cycle;
\draw(-8.389,3.226)--(-8.4,3.2);
\draw(-8.365,3.196)--(-8.357,3.215);
\filldraw[fill opacity=0.8,fill=gray!20,draw=none](-8.383,3.219)--(-8.354,3.235)--(-8.354,3.235)--(-8.389,3.226)--cycle;
\draw(-8.383,3.219)--(-8.354,3.235)--(-8.354,3.235)--(-8.389,3.226);
\filldraw[fill opacity=0.8,fill=gray!20](-8.358,3.215)--(-8.354,3.235)--(-8.354,3.235)--(-8.386,3.217)--cycle;
\filldraw[fill opacity=0.8,fill=gray!20](-8.245,3.205)--(-8.298,3.227)--(-8.307,3.221)--(-8.263,3.193)--cycle;
\filldraw[fill opacity=0.8,fill=gray!20](-8.298,3.227)--(-8.354,3.235)--(-8.354,3.235)--(-8.307,3.221)--cycle;
\filldraw[fill opacity=0.8,fill=gray!20](-8.307,3.221)--(-8.354,3.235)--(-8.354,3.235)--(-8.329,3.216)--cycle;
\filldraw[fill opacity=0.8,fill=gray!20](-8.329,3.216)--(-8.354,3.235)--(-8.354,3.235)--(-8.358,3.215)--cycle;
\filldraw[fill opacity=0.8,fill=gray!20](-7.643,.333)--(-7.638,.378)--(-7.709,.364)--(-7.711,.319)--cycle;
\filldraw[fill opacity=0.8,fill=gray!20](-7.733,.248)--(-7.72,.28)--(-7.8,.276)--(-7.798,.246)--cycle;
\filldraw[fill opacity=0.8,fill=gray!20,draw=none](-7.777,.988)--(-7.778,.988)--(-7.786,.995)--cycle;
\draw(-7.777,.988)--(-7.778,.988)--(-7.786,.995);
\filldraw[fill opacity=0.8,fill=gray!20,draw=none](-7.787,.986)--(-7.787,.986)--(-7.778,.988)--(-7.777,.988)--cycle;
\draw(-7.787,.986)--(-7.778,.988)--(-7.777,.988);
\filldraw[fill opacity=0.8,fill=gray!20](-7.748,.946)--(-7.709,.969)--(-7.699,.958)--(-7.743,.941)--cycle;
\filldraw[fill opacity=0.8,fill=gray!20](-7.743,.941)--(-7.699,.958)--(-7.714,.949)--(-7.751,.936)--cycle;
\filldraw[fill opacity=0.8,fill=gray!20,draw=none](-7.668,.981)--(-7.661,.986)--(-7.662,.986)--(-7.669,.981)--cycle;
\draw(-7.668,.981)--(-7.661,.986)--(-7.662,.986);
\filldraw[fill opacity=0.8,fill=gray!20,draw=none](-7.662,.986)--(-7.661,.986)--(-7.661,.987)--cycle;
\draw(-7.662,.986)--(-7.661,.986)--(-7.661,.987);
\filldraw[fill opacity=0.8,fill=gray!20,draw=none](-7.66,.981)--(-7.79,.986)--(-7.781,1.006)--(-7.663,1.002)--cycle;
\draw(-7.66,.981)--(-7.79,.986)--(-7.781,1.006)--(-7.663,1.002);
\filldraw[fill opacity=0.8,fill=gray!20,draw=none](-8.173,2.922)--(-8.164,2.944)--(-8.188,2.91)--(-8.195,2.891)--cycle;
\draw(-8.188,2.91)--(-8.195,2.891);
\filldraw[fill opacity=0.8,fill=gray!20,draw=none](-8.198,3.062)--(-8.196,3.06)--(-8.199,3.061)--cycle;
\draw(-8.196,3.06)--(-8.199,3.061);
\filldraw[fill opacity=0.8,fill=gray!20,draw=none](-8.198,3.062)--(-8.196,3.06)--(-8.199,3.061)--cycle;
\draw(-8.196,3.06)--(-8.199,3.061);
\filldraw[fill opacity=0.8,fill=gray!20,draw=none](-8.26,2.846)--(-8.25,2.845)--(-8.225,2.85)--(-8.196,2.89)--(-8.196,2.89)--cycle;
\draw(-8.25,2.845)--(-8.225,2.85)--(-8.196,2.89)--(-8.196,2.89);
\filldraw[fill opacity=0.8,fill=gray!20,draw=none](-8.946,1.163)--(-8.947,1.191)--(-8.994,1.194)--cycle;
\draw(-8.946,1.163)--(-8.947,1.191)--(-8.994,1.194);
\filldraw[fill opacity=0.8,fill=gray!20,draw=none](-7.651,4.865)--(-7.635,4.875)--(-7.641,4.877)--(-7.671,4.87)--cycle;
\draw(-7.641,4.877)--(-7.671,4.87)--(-7.651,4.865)--(-7.635,4.875);
\filldraw[fill opacity=0.8,fill=gray!20,draw=none](-7.727,4.716)--(-7.725,4.716)--(-7.719,4.713)--cycle;
\draw(-7.727,4.716)--(-7.725,4.716)--(-7.719,4.713);
\filldraw[fill opacity=0.8,fill=gray!20,draw=none](-7.731,4.701)--(-7.725,4.716)--(-7.727,4.716)--cycle;
\draw(-7.731,4.701)--(-7.725,4.716)--(-7.727,4.716);
\filldraw[fill opacity=0.8,fill=gray!20,draw=none](-8.198,3.062)--(-8.219,3.023)--(-8.177,3.004)--(-8.177,3.007)--cycle;
\draw(-8.219,3.023)--(-8.177,3.004);
\filldraw[fill opacity=0.8,fill=gray!20,draw=none](-7.952,3.637)--(-7.938,3.638)--(-7.918,3.663)--(-7.973,3.66)--cycle;
\draw(-7.952,3.637)--(-7.938,3.638)--(-7.918,3.663)--(-7.973,3.66);
\filldraw[fill opacity=0.8,fill=gray!20,draw=none](-8.203,3.067)--(-8.198,3.062)--(-8.199,3.062)--cycle;
\filldraw[fill opacity=0.8,fill=gray!20,draw=none](-8.203,3.067)--(-8.198,3.062)--(-8.199,3.062)--cycle;
\filldraw[fill opacity=0.8,fill=gray!20,draw=none](-7.739,4.554)--(-7.736,4.524)--(-7.741,4.564)--(-7.748,4.637)--cycle;
\draw(-7.739,4.554)--(-7.736,4.524);
\draw(-7.741,4.564)--(-7.748,4.637);
\filldraw[fill opacity=0.8,fill=gray!20,draw=none](-7.736,4.532)--(-7.757,4.538)--(-7.758,4.539)--(-7.761,4.54)--(-7.736,4.532)--cycle;
\draw(-7.757,4.538)--(-7.758,4.539);
\filldraw[fill opacity=0.8,fill=gray!20,draw=none](-7.737,4.53)--(-7.738,4.531)--(-7.757,4.538)--(-7.736,4.532)--cycle;
\draw(-7.738,4.531)--(-7.757,4.538);
\filldraw[fill opacity=0.8,fill=gray!20,draw=none](-7.713,4.734)--(-7.716,4.721)--(-7.706,4.717)--(-7.701,4.721)--(-7.703,4.744)--cycle;
\draw(-7.701,4.721)--(-7.703,4.744)--(-7.713,4.734);
\filldraw[fill opacity=0.8,fill=gray!20,draw=none](-7.713,4.734)--(-7.703,4.744)--(-7.701,4.745)--cycle;
\draw(-7.713,4.734)--(-7.703,4.744)--(-7.701,4.745);
\filldraw[fill opacity=0.8,fill=gray!20,draw=none](-7.764,4.531)--(-7.735,4.519)--(-7.737,4.53)--(-7.76,4.535)--cycle;
\draw(-7.737,4.53)--(-7.76,4.535);
\filldraw[fill opacity=0.8,fill=gray!20,draw=none](-7.721,4.714)--(-7.724,4.715)--(-7.731,4.701)--cycle;
\filldraw[fill opacity=0.8,fill=gray!20,draw=none](-7.719,4.713)--(-7.725,4.716)--(-7.731,4.701)--cycle;
\draw(-7.719,4.713)--(-7.725,4.716)--(-7.731,4.701);
\filldraw[fill opacity=0.8,fill=gray!20,draw=none](-7.717,4.712)--(-7.721,4.714)--(-7.731,4.701)--(-7.734,4.696)--(-7.718,4.708)--cycle;
\filldraw[fill opacity=0.8,fill=gray!20,draw=none](-7.742,4.578)--(-7.757,4.538)--(-7.737,4.53)--cycle;
\draw(-7.742,4.578)--(-7.757,4.538)--(-7.737,4.53);
\filldraw[fill opacity=0.8,fill=gray!20,draw=none](-7.738,4.544)--(-7.695,4.702)--(-7.743,4.593)--cycle;
\draw(-7.695,4.702)--(-7.743,4.593);
\filldraw[fill opacity=0.8,fill=gray!20,draw=none](-7.754,4.544)--(-7.76,4.535)--(-7.733,4.529)--cycle;
\draw(-7.76,4.535)--(-7.733,4.529);
\filldraw[fill opacity=0.8,fill=gray!20,draw=none](-7.732,4.673)--(-7.742,4.6)--(-7.702,4.703)--cycle;
\draw(-7.742,4.6)--(-7.702,4.703);
\filldraw[fill opacity=0.8,fill=gray!20,draw=none](-7.738,4.544)--(-7.737,4.532)--(-7.733,4.529)--(-7.724,4.524)--(-7.653,4.684)--(-7.695,4.702)--cycle;
\draw(-7.724,4.524)--(-7.653,4.684)--(-7.695,4.702);
\filldraw[fill opacity=0.8,fill=gray!20,draw=none](-7.735,4.519)--(-7.717,4.511)--(-7.723,4.524)--(-7.733,4.529)--(-7.737,4.53)--cycle;
\draw(-7.717,4.511)--(-7.723,4.524);
\draw(-7.733,4.529)--(-7.737,4.53);
\filldraw[fill opacity=0.8,fill=gray!20,draw=none](-7.731,4.663)--(-7.735,4.633)--(-7.712,4.685)--cycle;
\draw(-7.735,4.633)--(-7.712,4.685);
\filldraw[fill opacity=0.8,fill=gray!20,draw=none](-7.73,4.517)--(-7.736,4.519)--(-7.736,4.526)--cycle;
\draw(-7.736,4.519)--(-7.736,4.526);
\filldraw[fill opacity=0.8,fill=gray!20,draw=none](-7.754,4.544)--(-7.756,4.528)--(-7.736,4.519)--(-7.737,4.532)--cycle;
\filldraw[fill opacity=0.8,fill=gray!20,draw=none](-7.746,4.621)--(-7.758,4.539)--(-7.757,4.538)--(-7.742,4.578)--cycle;
\draw(-7.758,4.539)--(-7.757,4.538)--(-7.742,4.578);
\filldraw[fill opacity=0.8,fill=gray!20,draw=none](-7.754,4.544)--(-7.737,4.532)--(-7.743,4.593)--(-7.747,4.584)--cycle;
\draw(-7.743,4.593)--(-7.747,4.584);
\filldraw[fill opacity=0.8,fill=gray!20,draw=none](-7.752,4.548)--(-7.754,4.544)--(-7.733,4.529)--(-7.725,4.527)--(-7.725,4.528)--cycle;
\draw(-7.733,4.529)--(-7.725,4.527)--(-7.725,4.528);
\filldraw[fill opacity=0.8,fill=gray!20,draw=none](-7.735,4.525)--(-7.736,4.526)--(-7.736,4.527)--cycle;
\draw(-7.736,4.526)--(-7.736,4.527);
\filldraw[fill opacity=0.8,fill=gray!20,draw=none](-7.744,4.592)--(-7.738,4.537)--(-7.725,4.527)--(-7.715,4.529)--(-7.651,4.694)--(-7.689,4.71)--(-7.7,4.706)--(-7.702,4.703)--(-7.742,4.6)--cycle;
\draw(-7.715,4.529)--(-7.651,4.694)--(-7.689,4.71);
\draw(-7.702,4.703)--(-7.742,4.6);
\filldraw[fill opacity=0.8,fill=gray!20,draw=none](-7.752,4.548)--(-7.738,4.537)--(-7.744,4.592)--cycle;
\filldraw[fill opacity=0.8,fill=gray!20,draw=none](-7.748,4.553)--(-7.752,4.548)--(-7.725,4.528)--(-7.728,4.538)--cycle;
\draw(-7.725,4.528)--(-7.728,4.538);
\filldraw[fill opacity=0.8,fill=gray!20,draw=none](-7.739,4.546)--(-7.736,4.524)--(-7.736,4.519)--(-7.736,4.519)--cycle;
\draw(-7.736,4.524)--(-7.736,4.519);
\filldraw[fill opacity=0.8,fill=gray!20,draw=none](-7.698,4.705)--(-7.699,4.704)--(-7.712,4.685)--(-7.735,4.633)--(-7.743,4.585)--(-7.741,4.57)--cycle;
\draw(-7.712,4.685)--(-7.735,4.633);
\filldraw[fill opacity=0.8,fill=gray!20,draw=none](-7.752,4.548)--(-7.755,4.528)--(-7.736,4.519)--(-7.738,4.537)--cycle;
\filldraw[fill opacity=0.8,fill=gray!20,draw=none](-7.654,4.72)--(-7.698,4.705)--(-7.741,4.57)--(-7.739,4.546)--(-7.728,4.538)--(-7.72,4.538)--(-7.643,4.717)--cycle;
\draw(-7.72,4.538)--(-7.643,4.717);
\filldraw[fill opacity=0.8,fill=gray!20,draw=none](-7.737,4.532)--(-7.736,4.519)--(-7.727,4.516)--(-7.724,4.523)--cycle;
\draw(-7.727,4.516)--(-7.724,4.523);
\filldraw[fill opacity=0.8,fill=gray!20,draw=none](-7.733,4.529)--(-7.724,4.523)--(-7.724,4.524)--cycle;
\draw(-7.724,4.523)--(-7.724,4.524);
\filldraw[fill opacity=0.8,fill=gray!20,draw=none](-7.723,4.524)--(-7.725,4.527)--(-7.733,4.529)--cycle;
\draw(-7.723,4.524)--(-7.725,4.527)--(-7.733,4.529);
\filldraw[fill opacity=0.8,fill=gray!20,draw=none](-7.725,4.527)--(-7.738,4.537)--(-7.737,4.527)--cycle;
\filldraw[fill opacity=0.8,fill=gray!20,draw=none](-7.746,4.621)--(-7.749,4.652)--(-7.787,4.551)--(-7.758,4.539)--cycle;
\draw(-7.749,4.652)--(-7.787,4.551)--(-7.758,4.539);
\filldraw[fill opacity=0.8,fill=gray!20,draw=none](-7.762,4.551)--(-7.743,4.593)--(-7.75,4.659)--(-7.796,4.557)--cycle;
\draw(-7.762,4.551)--(-7.743,4.593);
\draw(-7.75,4.659)--(-7.796,4.557);
\filldraw[fill opacity=0.8,fill=gray!20,draw=none](-7.76,4.554)--(-7.744,4.596)--(-7.752,4.676)--(-7.797,4.561)--cycle;
\draw(-7.76,4.554)--(-7.744,4.596);
\draw(-7.752,4.676)--(-7.797,4.561);
\filldraw[fill opacity=0.8,fill=gray!20,draw=none](-7.754,4.544)--(-7.747,4.584)--(-7.762,4.551)--cycle;
\draw(-7.747,4.584)--(-7.762,4.551);
\filldraw[fill opacity=0.8,fill=gray!20,draw=none](-7.752,4.548)--(-7.744,4.592)--(-7.744,4.596)--(-7.76,4.554)--cycle;
\draw(-7.744,4.596)--(-7.76,4.554);
\filldraw[fill opacity=0.8,fill=gray!20,draw=none](-7.747,4.552)--(-7.741,4.57)--(-7.745,4.609)--(-7.764,4.565)--cycle;
\draw(-7.745,4.609)--(-7.764,4.565);
\filldraw[fill opacity=0.8,fill=gray!20,draw=none](-7.747,4.552)--(-7.739,4.546)--(-7.741,4.57)--cycle;
\filldraw[fill opacity=0.8,fill=gray!20,draw=none](-7.748,4.553)--(-7.747,4.552)--(-7.731,4.549)--(-7.736,4.572)--cycle;
\draw(-7.731,4.549)--(-7.736,4.572);
\filldraw[fill opacity=0.8,fill=gray!20,draw=none](-7.747,4.552)--(-7.739,4.546)--(-7.729,4.545)--(-7.731,4.549)--cycle;
\draw(-7.729,4.545)--(-7.731,4.549);
\filldraw[fill opacity=0.8,fill=gray!20,draw=none](-7.658,4.485)--(-7.633,4.483)--(-7.649,4.506)--(-7.66,4.513)--(-7.707,4.525)--(-7.725,4.527)--(-7.717,4.511)--cycle;
\draw(-7.658,4.485)--(-7.633,4.483);
\draw(-7.707,4.525)--(-7.725,4.527)--(-7.717,4.511);
\filldraw[fill opacity=0.8,fill=gray!20,draw=none](-7.703,4.507)--(-7.7,4.517)--(-7.724,4.524)--cycle;
\filldraw[fill opacity=0.8,fill=gray!20,draw=none](-7.697,4.503)--(-7.724,4.524)--(-7.727,4.516)--cycle;
\draw(-7.724,4.524)--(-7.727,4.516);
\filldraw[fill opacity=0.8,fill=gray!20,draw=none](-7.725,4.527)--(-7.737,4.527)--(-7.736,4.519)--(-7.722,4.513)--(-7.718,4.522)--cycle;
\draw(-7.722,4.513)--(-7.718,4.522);
\filldraw[fill opacity=0.8,fill=gray!20,draw=none](-7.739,4.546)--(-7.728,4.538)--(-7.729,4.545)--cycle;
\draw(-7.728,4.538)--(-7.729,4.545);
\filldraw[fill opacity=0.8,fill=gray!20,draw=none](-7.728,4.538)--(-7.739,4.546)--(-7.738,4.537)--cycle;
\filldraw[fill opacity=0.8,fill=gray!20,draw=none](-7.707,4.525)--(-7.727,4.534)--(-7.725,4.527)--cycle;
\draw(-7.727,4.534)--(-7.725,4.527)--(-7.707,4.525);
\filldraw[fill opacity=0.8,fill=gray!20,draw=none](-7.708,4.515)--(-7.619,4.682)--(-7.651,4.694)--(-7.718,4.523)--cycle;
\draw(-7.619,4.682)--(-7.651,4.694)--(-7.718,4.523);
\filldraw[fill opacity=0.8,fill=gray!20,draw=none](-7.707,4.525)--(-7.685,4.524)--(-7.729,4.542)--(-7.727,4.534)--cycle;
\draw(-7.707,4.525)--(-7.685,4.524);
\draw(-7.729,4.542)--(-7.727,4.534);
\filldraw[fill opacity=0.8,fill=gray!20,draw=none](-7.728,4.538)--(-7.738,4.537)--(-7.736,4.519)--(-7.73,4.517)--(-7.722,4.534)--cycle;
\draw(-7.736,4.519)--(-7.73,4.517)--(-7.722,4.534);
\filldraw[fill opacity=0.8,fill=gray!20,draw=none](-7.7,4.517)--(-7.703,4.507)--(-7.697,4.503)--(-7.681,4.495)--(-7.674,4.51)--cycle;
\draw(-7.681,4.495)--(-7.674,4.51);
\filldraw[fill opacity=0.8,fill=gray!20,draw=none](-7.725,4.527)--(-7.716,4.528)--(-7.715,4.529)--cycle;
\draw(-7.716,4.528)--(-7.715,4.529);
\filldraw[fill opacity=0.8,fill=gray!20,draw=none](-7.725,4.527)--(-7.718,4.522)--(-7.716,4.528)--cycle;
\draw(-7.718,4.522)--(-7.716,4.528);
\filldraw[fill opacity=0.8,fill=gray!20,draw=none](-7.689,4.525)--(-7.6,4.718)--(-7.602,4.719)--(-7.643,4.717)--(-7.71,4.562)--(-7.72,4.533)--cycle;
\draw(-7.643,4.717)--(-7.71,4.562);
\filldraw[fill opacity=0.8,fill=gray!20,draw=none](-7.71,4.562)--(-7.722,4.534)--(-7.721,4.531)--cycle;
\draw(-7.71,4.562)--(-7.722,4.534);
\filldraw[fill opacity=0.8,fill=gray!20,draw=none](-7.685,4.524)--(-7.632,4.52)--(-7.633,4.567)--(-7.735,4.574)--(-7.736,4.572)--(-7.729,4.542)--cycle;
\draw(-7.685,4.524)--(-7.632,4.52)--(-7.633,4.567)--(-7.735,4.574);
\draw(-7.736,4.572)--(-7.729,4.542);
\filldraw[fill opacity=0.8,fill=gray!20,draw=none](-7.726,4.604)--(-7.735,4.574)--(-7.633,4.567)--(-7.633,4.621)--(-7.703,4.626)--cycle;
\draw(-7.735,4.574)--(-7.633,4.567)--(-7.633,4.621)--(-7.703,4.626);
\filldraw[fill opacity=0.8,fill=gray!20,draw=none](-7.748,4.637)--(-7.737,4.52)--(-7.733,4.518)--(-7.734,4.696)--cycle;
\draw(-7.748,4.637)--(-7.737,4.52);
\filldraw[fill opacity=0.8,fill=gray!20,draw=none](-5.969,.525)--(-5.972,.567)--(-5.967,.563)--(-5.967,.515)--cycle;
\draw(-5.972,.567)--(-5.967,.563)--(-5.967,.515);
\filldraw[fill opacity=0.8,fill=gray!20,draw=none](-6.036,.534)--(-6.002,.54)--(-5.974,.551)--(-5.967,.563)--(-5.973,.568)--(-6.002,.567)--(-6.081,.557)--cycle;
\draw(-6.036,.534)--(-6.002,.54)--(-5.974,.551)--(-5.967,.563)--(-5.973,.568);
\filldraw[fill opacity=0.8,fill=gray!20](-6.955,.586)--(-6.952,.605)--(-6.952,.605)--(-6.983,.588)--cycle;
\filldraw[fill opacity=0.8,fill=gray!20](-6.926,.587)--(-6.952,.605)--(-6.952,.605)--(-6.955,.586)--cycle;
\filldraw[fill opacity=0.8,fill=gray!20](-6.904,.591)--(-6.952,.605)--(-6.952,.605)--(-6.926,.587)--cycle;
\filldraw[fill opacity=0.8,fill=gray!20](-6.983,.588)--(-6.952,.605)--(-6.952,.605)--(-7.002,.593)--cycle;
\filldraw[fill opacity=0.8,fill=gray!20](-6.958,.182)--(-6.961,.206)--(-7.037,.211)--(-7.012,.186)--cycle;
\filldraw[fill opacity=0.8,fill=gray!20](-6.895,.597)--(-6.952,.605)--(-6.952,.605)--(-6.904,.591)--cycle;
\filldraw[fill opacity=0.8,fill=gray!20](-6.842,.575)--(-6.895,.597)--(-6.904,.591)--(-6.86,.563)--cycle;
\filldraw[fill opacity=0.8,fill=gray!20,draw=none](-7.753,4.684)--(-7.802,4.572)--(-7.764,4.565)--(-7.745,4.609)--cycle;
\draw(-7.753,4.684)--(-7.802,4.572);
\draw(-7.764,4.565)--(-7.745,4.609);
\filldraw[fill opacity=0.8,fill=gray!20,draw=none](-7.758,4.539)--(-7.765,4.542)--(-7.761,4.54)--cycle;
\draw(-7.758,4.539)--(-7.765,4.542);
\filldraw[fill opacity=0.8,fill=gray!20,draw=none](-7.76,4.535)--(-7.736,4.572)--(-7.737,4.574)--(-7.81,4.592)--(-7.79,4.543)--cycle;
\draw(-7.736,4.572)--(-7.737,4.574)--(-7.81,4.592)--(-7.79,4.543)--(-7.76,4.535);
\filldraw[fill opacity=0.8,fill=gray!20,draw=none](-7.764,4.531)--(-7.76,4.535)--(-7.79,4.543)--cycle;
\draw(-7.76,4.535)--(-7.79,4.543);
\filldraw[fill opacity=0.8,fill=gray!20,draw=none](-6.078,.568)--(-6.101,.568)--(-6.095,.565)--(-6.062,.566)--cycle;
\filldraw[fill opacity=0.8,fill=gray!20,draw=none](-8.21,2.896)--(-8.2,2.903)--(-8.198,2.902)--cycle;
\draw(-8.2,2.903)--(-8.198,2.902);
\filldraw[fill opacity=0.8,fill=gray!20,draw=none](-8.194,2.905)--(-8.206,2.889)--(-8.196,2.89)--(-8.188,2.91)--cycle;
\draw(-8.206,2.889)--(-8.196,2.89)--(-8.188,2.91);
\filldraw[fill opacity=0.8,fill=gray!20,draw=none](-8.194,2.905)--(-8.216,2.886)--(-8.206,2.889)--cycle;
\draw(-8.216,2.886)--(-8.206,2.889);
\filldraw[fill opacity=0.8,fill=gray!20,draw=none](-8.206,2.91)--(-8.192,2.904)--(-8.198,2.896)--(-8.251,2.881)--(-8.258,2.884)--cycle;
\draw(-8.206,2.91)--(-8.192,2.904);
\draw(-8.251,2.881)--(-8.258,2.884);
\filldraw[fill opacity=0.8,fill=gray!20,draw=none](-8.189,2.895)--(-8.196,2.89)--(-8.198,2.888)--cycle;
\draw(-8.189,2.895)--(-8.196,2.89)--(-8.198,2.888);
\filldraw[fill opacity=0.8,fill=gray!20,draw=none](-8.173,2.931)--(-8.165,2.928)--(-8.189,2.898)--(-8.19,2.898)--cycle;
\draw(-8.189,2.898)--(-8.19,2.898);
\filldraw[fill opacity=0.8,fill=gray!20,draw=none](-8.196,2.89)--(-8.19,2.898)--(-8.189,2.898)--cycle;
\draw(-8.19,2.898)--(-8.189,2.898);
\filldraw[fill opacity=0.8,fill=gray!20,draw=none](-8.194,2.882)--(-8.201,2.885)--(-8.189,2.898)--(-8.18,2.894)--cycle;
\draw(-8.189,2.898)--(-8.18,2.894)--(-8.194,2.882);
\filldraw[fill opacity=0.8,fill=gray!20,draw=none](-8.165,2.928)--(-8.161,2.926)--(-8.18,2.894)--(-8.189,2.898)--cycle;
\draw(-8.161,2.926)--(-8.18,2.894)--(-8.189,2.898);
\filldraw[fill opacity=0.8,fill=gray!20,draw=none](-8.196,2.891)--(-8.196,2.89)--(-8.201,2.885)--(-8.203,2.885)--cycle;
\filldraw[fill opacity=0.8,fill=gray!20,draw=none](-8.274,2.846)--(-8.26,2.846)--(-8.196,2.89)--(-8.216,2.886)--cycle;
\draw(-8.196,2.89)--(-8.216,2.886);
\filldraw[fill opacity=0.8,fill=gray!20,draw=none](-8.203,2.885)--(-8.2,2.884)--(-8.213,2.866)--(-8.222,2.858)--(-8.235,2.864)--cycle;
\draw(-8.213,2.866)--(-8.222,2.858)--(-8.235,2.864);
\filldraw[fill opacity=0.8,fill=gray!20,draw=none](-8.2,2.884)--(-8.194,2.882)--(-8.213,2.866)--cycle;
\draw(-8.194,2.882)--(-8.213,2.866);
\filldraw[fill opacity=0.8,fill=gray!20,draw=none](-8.197,2.897)--(-8.227,2.871)--(-8.251,2.881)--cycle;
\draw(-8.227,2.871)--(-8.251,2.881);
\filldraw[fill opacity=0.8,fill=gray!20,draw=none](-8.24,2.853)--(-8.251,2.858)--(-8.235,2.864)--(-8.222,2.858)--cycle;
\draw(-8.235,2.864)--(-8.222,2.858)--(-8.24,2.853);
\filldraw[fill opacity=0.8,fill=gray!20,draw=none](-8.189,2.903)--(-8.208,2.87)--(-8.18,2.894)--(-8.161,2.926)--cycle;
\draw(-8.208,2.87)--(-8.18,2.894)--(-8.161,2.926);
\filldraw[fill opacity=0.8,fill=gray!20,draw=none](-8.251,2.858)--(-8.24,2.853)--(-8.244,2.851)--(-8.258,2.855)--cycle;
\draw(-8.24,2.853)--(-8.244,2.851);
\filldraw[fill opacity=0.8,fill=gray!20,draw=none](-8.192,2.904)--(-8.189,2.903)--(-8.197,2.897)--(-8.198,2.896)--cycle;
\draw(-8.192,2.904)--(-8.189,2.903);
\filldraw[fill opacity=0.8,fill=gray!20,draw=none](-8.266,2.852)--(-8.258,2.855)--(-8.244,2.851)--(-8.255,2.848)--cycle;
\draw(-8.244,2.851)--(-8.255,2.848);
\filldraw[fill opacity=0.8,fill=gray!20,draw=none](-8.203,2.879)--(-8.189,2.903)--(-8.227,2.871)--cycle;
\filldraw[fill opacity=0.8,fill=gray!20,draw=none](-8.189,2.903)--(-8.183,2.9)--(-8.197,2.897)--cycle;
\draw(-8.189,2.903)--(-8.183,2.9);
\filldraw[fill opacity=0.8,fill=gray!20,draw=none](-8.203,2.879)--(-8.227,2.871)--(-8.255,2.848)--(-8.222,2.858)--(-8.208,2.87)--cycle;
\draw(-8.255,2.848)--(-8.222,2.858)--(-8.208,2.87);
\filldraw[fill opacity=0.8,fill=gray!20,draw=none](-8.197,2.897)--(-8.183,2.9)--(-8.168,2.894)--(-8.21,2.863)--(-8.227,2.871)--cycle;
\draw(-8.183,2.9)--(-8.168,2.894);
\draw(-8.21,2.863)--(-8.227,2.871);
\filldraw[fill opacity=0.8,fill=gray!20,draw=none](-8.206,2.91)--(-8.168,2.894)--(-8.21,2.863)--(-8.258,2.884)--cycle;
\draw(-8.206,2.91)--(-8.168,2.894);
\draw(-8.21,2.863)--(-8.258,2.884);
\filldraw[fill opacity=0.8,fill=gray!20,draw=none](-6.179,.501)--(-6.174,.511)--(-6.183,.511)--cycle;
\draw(-6.174,.511)--(-6.183,.511);
\filldraw[fill opacity=0.8,fill=gray!20,draw=none](-6.179,.501)--(-6.172,.513)--(-6.186,.519)--cycle;
\draw(-6.172,.513)--(-6.186,.519);
\filldraw[fill opacity=0.8,fill=gray!20,draw=none](-7.714,4.51)--(-7.712,4.49)--(-7.735,4.514)--(-7.736,4.519)--cycle;
\draw(-7.714,4.51)--(-7.712,4.49);
\draw(-7.735,4.514)--(-7.736,4.519);
\filldraw[fill opacity=0.8,fill=gray!20,draw=none](-7.673,4.467)--(-7.697,4.48)--(-7.689,4.471)--cycle;
\draw(-7.697,4.48)--(-7.689,4.471);
\filldraw[fill opacity=0.8,fill=gray!20,draw=none](-7.673,4.467)--(-7.672,4.463)--(-7.697,4.48)--cycle;
\draw(-7.673,4.467)--(-7.672,4.463);
\filldraw[fill opacity=0.8,fill=gray!20,draw=none](-8.17,3.049)--(-8.171,3.055)--(-8.172,3.054)--cycle;
\draw(-8.171,3.055)--(-8.172,3.054);
\filldraw[fill opacity=0.8,fill=gray!20,draw=none](-8.222,3.12)--(-8.216,3.117)--(-8.195,3.104)--(-8.19,3.1)--(-8.206,3.107)--cycle;
\draw(-8.19,3.1)--(-8.206,3.107);
\filldraw[fill opacity=0.8,fill=gray!20,draw=none](-8.181,3.086)--(-8.19,3.09)--(-8.188,3.097)--cycle;
\filldraw[fill opacity=0.8,fill=gray!20,draw=none](-8.172,3.054)--(-8.171,3.055)--(-8.181,3.086)--(-8.19,3.1)--(-8.195,3.104)--(-8.186,3.076)--cycle;
\draw(-8.172,3.054)--(-8.171,3.055);
\draw(-8.195,3.104)--(-8.186,3.076);
\filldraw[fill opacity=0.8,fill=gray!20,draw=none](-8.152,3.052)--(-8.153,3.053)--(-8.152,3.053)--cycle;
\draw(-8.153,3.053)--(-8.152,3.053);
\filldraw[fill opacity=0.8,fill=gray!20,draw=none](-8.185,3.097)--(-8.189,3.109)--(-8.196,3.105)--(-8.195,3.104)--cycle;
\draw(-8.189,3.109)--(-8.196,3.105)--(-8.195,3.104);
\filldraw[fill opacity=0.8,fill=gray!20,draw=none](-8.181,3.086)--(-8.185,3.097)--(-8.19,3.1)--cycle;
\filldraw[fill opacity=0.8,fill=gray!20,draw=none](-8.17,3.081)--(-8.181,3.086)--(-8.188,3.097)--(-8.188,3.099)--(-8.18,3.096)--cycle;
\draw(-8.188,3.099)--(-8.18,3.096)--(-8.17,3.081);
\filldraw[fill opacity=0.8,fill=gray!20,draw=none](-8.188,3.097)--(-8.19,3.1)--(-8.188,3.099)--cycle;
\draw(-8.19,3.1)--(-8.188,3.099);
\filldraw[fill opacity=0.8,fill=gray!20,draw=none](-8.209,3.114)--(-8.208,3.113)--(-8.189,3.101)--(-8.188,3.099)--(-8.19,3.1)--cycle;
\draw(-8.208,3.113)--(-8.189,3.101);
\draw(-8.188,3.099)--(-8.19,3.1);
\filldraw[fill opacity=0.8,fill=gray!20,draw=none](-8.189,3.101)--(-8.18,3.096)--(-8.188,3.099)--cycle;
\draw(-8.189,3.101)--(-8.18,3.096)--(-8.188,3.099);
\filldraw[fill opacity=0.8,fill=gray!20,draw=none](-8.189,3.085)--(-8.164,3.046)--(-8.168,3.072)--cycle;
\filldraw[fill opacity=0.8,fill=gray!20,draw=none](-8.194,3.087)--(-8.188,3.083)--(-8.196,3.105)--(-8.206,3.103)--cycle;
\draw(-8.188,3.083)--(-8.196,3.105)--(-8.206,3.103);
\filldraw[fill opacity=0.8,fill=gray!20,draw=none](-8.216,3.117)--(-8.209,3.114)--(-8.195,3.104)--cycle;
\filldraw[fill opacity=0.8,fill=gray!20,draw=none](-8.237,3.112)--(-8.234,3.111)--(-8.217,3.101)--(-8.194,3.087)--(-8.214,3.095)--cycle;
\draw(-8.237,3.112)--(-8.234,3.111);
\draw(-8.194,3.087)--(-8.214,3.095);
\filldraw[fill opacity=0.8,fill=gray!20,draw=none](-8.171,3.073)--(-8.168,3.072)--(-8.169,3.073)--cycle;
\filldraw[fill opacity=0.8,fill=gray!20,draw=none](-8.234,3.111)--(-8.217,3.101)--(-8.196,3.105)--(-8.203,3.109)--cycle;
\draw(-8.217,3.101)--(-8.196,3.105);
\filldraw[fill opacity=0.8,fill=gray!20,draw=none](-8.203,3.106)--(-8.189,3.085)--(-8.171,3.073)--(-8.169,3.073)--(-8.17,3.081)--(-8.18,3.096)--(-8.194,3.105)--cycle;
\draw(-8.17,3.081)--(-8.18,3.096)--(-8.194,3.105);
\filldraw[fill opacity=0.8,fill=gray!20,draw=none](-8.194,3.087)--(-8.186,3.076)--(-8.188,3.083)--cycle;
\draw(-8.186,3.076)--(-8.188,3.083);
\filldraw[fill opacity=0.8,fill=gray!20,draw=none](-7.994,2.972)--(-7.999,2.978)--(-7.995,2.978)--cycle;
\draw(-7.999,2.978)--(-7.995,2.978);
\filldraw[fill opacity=0.8,fill=gray!20,draw=none](-7.995,2.996)--(-7.995,2.978)--(-7.999,2.978)--(-8.015,3.001)--(-8.015,3.008)--cycle;
\draw(-7.995,2.978)--(-7.999,2.978);
\draw(-8.015,3.001)--(-8.015,3.008);
\filldraw[fill opacity=0.8,fill=gray!20,draw=none](-8.021,3.011)--(-8.015,3.008)--(-8.015,3.001)--cycle;
\draw(-8.015,3.008)--(-8.015,3.001);
\filldraw[fill opacity=0.8,fill=gray!20,draw=none](-7.995,2.996)--(-8.015,3.008)--(-8.016,3.026)--(-7.996,3.025)--cycle;
\draw(-8.015,3.008)--(-8.016,3.026)--(-7.996,3.025);
\filldraw[fill opacity=0.8,fill=gray!20,draw=none](-7.971,2.986)--(-7.971,2.962)--(-7.997,3)--(-7.995,3)--cycle;
\draw(-7.997,3)--(-7.995,3);
\filldraw[fill opacity=0.8,fill=gray!20,draw=none](-7.971,2.986)--(-7.971,2.998)--(-7.94,2.996)--(-7.931,2.963)--cycle;
\draw(-7.971,2.998)--(-7.94,2.996)--(-7.931,2.963);
\filldraw[fill opacity=0.8,fill=gray!20,draw=none](-7.971,2.986)--(-7.995,3)--(-7.971,2.998)--cycle;
\draw(-7.995,3)--(-7.971,2.998);
\filldraw[fill opacity=0.8,fill=gray!20,draw=none](-8.198,3.062)--(-8.194,3.087)--(-7.944,2.978)--(-7.918,2.939)--(-8.196,3.06)--cycle;
\draw(-8.194,3.087)--(-7.944,2.978)--(-7.918,2.939)--(-8.196,3.06);
\filldraw[fill opacity=0.8,fill=gray!20,draw=none](-8.198,3.062)--(-8.194,3.087)--(-7.944,2.978)--(-7.918,2.939)--(-8.196,3.06)--cycle;
\draw(-8.194,3.087)--(-7.944,2.978)--(-7.918,2.939)--(-8.196,3.06);
\filldraw[fill opacity=0.8,fill=gray!20](-6.822,.221)--(-6.793,.261)--(-6.867,.247)--(-6.882,.209)--cycle;
\filldraw[fill opacity=0.8,fill=gray!20,draw=none](-7.962,3.647)--(-7.965,3.64)--(-7.952,3.637)--cycle;
\draw(-7.962,3.647)--(-7.965,3.64);
\filldraw[fill opacity=0.8,fill=gray!20,draw=none](-6.038,.563)--(-6.002,.567)--(-6.062,.566)--cycle;
\filldraw[fill opacity=0.8,fill=gray!20,draw=none](-6.784,.267)--(-6.779,.278)--(-6.785,.278)--cycle;
\draw(-6.779,.278)--(-6.785,.278);
\filldraw[fill opacity=0.8,fill=gray!20,draw=none](-7.663,1.239)--(-7.65,1.22)--(-7.664,1.24)--cycle;
\filldraw[fill opacity=0.8,fill=gray!20,draw=none](-7.648,1.22)--(-7.774,1.225)--(-7.781,1.244)--(-7.62,1.237)--cycle;
\draw(-7.648,1.22)--(-7.774,1.225)--(-7.781,1.244)--(-7.62,1.237);
\filldraw[fill opacity=0.8,fill=gray!20,draw=none](-8.305,2.813)--(-8.293,2.828)--(-8.323,2.837)--(-8.364,2.835)--(-8.361,2.811)--cycle;
\draw(-8.323,2.837)--(-8.364,2.835)--(-8.361,2.811)--(-8.305,2.813)--(-8.293,2.828);
\filldraw[fill opacity=0.8,fill=gray!20,draw=none](-7.746,4.621)--(-7.737,4.685)--(-7.749,4.652)--cycle;
\draw(-7.737,4.685)--(-7.749,4.652);
\filldraw[fill opacity=0.8,fill=gray!20,draw=none](-7.79,4.489)--(-7.775,4.521)--(-7.787,4.542)--(-7.8,4.547)--(-7.82,4.502)--cycle;
\draw(-7.79,4.489)--(-7.775,4.521);
\draw(-7.8,4.547)--(-7.82,4.502);
\filldraw[fill opacity=0.8,fill=gray!20,draw=none](-7.804,4.441)--(-7.768,4.426)--(-7.734,4.502)--(-7.735,4.516)--(-7.737,4.52)--(-7.769,4.534)--(-7.79,4.489)--cycle;
\draw(-7.768,4.426)--(-7.734,4.502);
\draw(-7.769,4.534)--(-7.79,4.489);
\filldraw[fill opacity=0.8,fill=gray!20,draw=none](-7.831,4.372)--(-7.774,4.519)--(-7.787,4.542)--(-7.802,4.548)--(-7.865,4.387)--cycle;
\draw(-7.802,4.548)--(-7.865,4.387)--(-7.831,4.372)--(-7.774,4.519);
\filldraw[fill opacity=0.8,fill=gray!20,draw=none](-8.092,3.892)--(-8.163,3.733)--(-8.129,3.727)--(-8.062,3.878)--cycle;
\draw(-8.129,3.727)--(-8.062,3.878)--(-8.092,3.892)--(-8.163,3.733);
\filldraw[fill opacity=0.8,fill=gray!20](-8.092,3.703)--(-8.104,3.75)--(-8.177,3.768)--(-8.157,3.719)--cycle;
\filldraw[fill opacity=0.8,fill=gray!20,draw=none](-7.637,1.204)--(-7.648,1.218)--(-7.648,1.22)--(-7.628,1.219)--cycle;
\draw(-7.648,1.22)--(-7.628,1.219);
\filldraw[fill opacity=0.8,fill=gray!20](-7.632,1.178)--(-7.649,1.219)--(-7.632,1.201)--(-7.614,1.158)--cycle;
\filldraw[fill opacity=0.8,fill=gray!20,draw=none](-7.745,4.606)--(-7.735,4.693)--(-7.75,4.659)--cycle;
\draw(-7.735,4.693)--(-7.75,4.659);
\filldraw[fill opacity=0.8,fill=gray!20,draw=none](-7.724,4.681)--(-7.718,4.708)--(-7.734,4.696)--(-7.732,4.673)--cycle;
\draw(-7.734,4.696)--(-7.732,4.673);
\filldraw[fill opacity=0.8,fill=gray!20,draw=none](-7.686,4.498)--(-7.718,4.523)--(-7.722,4.513)--cycle;
\draw(-7.718,4.523)--(-7.722,4.513);
\filldraw[fill opacity=0.8,fill=gray!20,draw=none](-7.728,4.538)--(-7.722,4.534)--(-7.72,4.538)--cycle;
\draw(-7.722,4.534)--(-7.72,4.538);
\filldraw[fill opacity=0.8,fill=gray!20,draw=none](-7.722,4.534)--(-7.724,4.531)--(-7.715,4.524)--cycle;
\draw(-7.722,4.534)--(-7.724,4.531);
\filldraw[fill opacity=0.8,fill=gray!20,draw=none](-7.71,4.512)--(-7.707,4.518)--(-7.724,4.531)--(-7.73,4.517)--(-7.727,4.515)--cycle;
\draw(-7.724,4.531)--(-7.73,4.517)--(-7.727,4.515);
\filldraw[fill opacity=0.8,fill=gray!20,draw=none](-7.66,4.513)--(-7.676,4.523)--(-7.707,4.525)--cycle;
\draw(-7.676,4.523)--(-7.707,4.525);
\filldraw[fill opacity=0.8,fill=gray!20,draw=none](-7.691,4.521)--(-7.703,4.524)--(-7.708,4.515)--(-7.698,4.508)--(-7.696,4.507)--cycle;
\filldraw[fill opacity=0.8,fill=gray!20,draw=none](-7.689,4.525)--(-7.72,4.533)--(-7.721,4.531)--(-7.715,4.524)--(-7.696,4.509)--cycle;
\filldraw[fill opacity=0.8,fill=gray!20,draw=none](-7.71,4.512)--(-7.727,4.515)--(-7.711,4.509)--cycle;
\draw(-7.727,4.515)--(-7.711,4.509);
\filldraw[fill opacity=0.8,fill=gray!20,draw=none](-7.733,4.518)--(-7.717,4.511)--(-7.734,4.696)--cycle;
\draw(-7.717,4.511)--(-7.734,4.696);
\filldraw[fill opacity=0.8,fill=gray!20,draw=none](-7.705,4.441)--(-7.703,4.444)--(-7.708,4.45)--cycle;
\draw(-7.705,4.441)--(-7.703,4.444);
\filldraw[fill opacity=0.8,fill=gray!20,draw=none](-7.672,4.463)--(-7.669,4.426)--(-7.708,4.447)--(-7.712,4.49)--cycle;
\draw(-7.672,4.463)--(-7.669,4.426);
\draw(-7.708,4.447)--(-7.712,4.49);
\filldraw[fill opacity=0.8,fill=gray!20,draw=none](-7.787,4.542)--(-7.769,4.534)--(-7.762,4.551)--(-7.796,4.557)--cycle;
\draw(-7.769,4.534)--(-7.762,4.551);
\filldraw[fill opacity=0.8,fill=gray!20,draw=none](-7.787,4.542)--(-7.796,4.557)--(-7.8,4.547)--cycle;
\draw(-7.796,4.557)--(-7.8,4.547);
\filldraw[fill opacity=0.8,fill=gray!20,draw=none](-7.475,4.52)--(-7.461,4.539)--(-7.477,4.536)--cycle;
\draw(-7.475,4.52)--(-7.461,4.539)--(-7.477,4.536);
\filldraw[fill opacity=0.8,fill=gray!20,draw=none](-7.569,4.476)--(-7.577,4.477)--(-7.575,4.474)--(-7.566,4.473)--cycle;
\filldraw[fill opacity=0.8,fill=gray!20,draw=none](-7.572,4.462)--(-7.546,4.506)--(-7.564,4.492)--(-7.568,4.48)--cycle;
\draw(-7.564,4.492)--(-7.568,4.48);
\filldraw[fill opacity=0.8,fill=gray!20,draw=none](-7.943,4.145)--(-7.79,4.489)--(-7.82,4.502)--cycle;
\draw(-7.943,4.145)--(-7.79,4.489);
\filldraw[fill opacity=0.8,fill=gray!20,draw=none](-8.009,4.047)--(-7.989,4.041)--(-7.943,4.145)--(-7.82,4.502)--(-8.021,4.052)--cycle;
\draw(-7.989,4.041)--(-7.943,4.145);
\draw(-7.82,4.502)--(-8.021,4.052);
\filldraw[fill opacity=0.8,fill=gray!20](-6.903,.184)--(-6.882,.209)--(-6.961,.206)--(-6.958,.182)--cycle;
\filldraw[fill opacity=0.8,fill=gray!20](-7.949,1.206)--(-7.92,1.242)--(-7.898,1.256)--(-7.923,1.223)--cycle;
\filldraw[fill opacity=0.8,fill=gray!20,draw=none](-5.957,.435)--(-5.96,.436)--(-5.966,.448)--cycle;
\draw(-5.957,.435)--(-5.96,.436);
\filldraw[fill opacity=0.8,fill=gray!20,draw=none](-5.969,.525)--(-5.967,.515)--(-5.967,.506)--cycle;
\draw(-5.967,.515)--(-5.967,.506);
\filldraw[fill opacity=0.8,fill=gray!20,draw=none](-7.787,4.542)--(-7.797,4.561)--(-7.802,4.548)--cycle;
\draw(-7.797,4.561)--(-7.802,4.548);
\filldraw[fill opacity=0.8,fill=gray!20,draw=none](-8.177,3.007)--(-8.177,3.004)--(-8.175,3.003)--cycle;
\draw(-8.177,3.004)--(-8.175,3.003);
\filldraw[fill opacity=0.8,fill=gray!20,draw=none](-8.177,3.007)--(-8.161,3)--(-8.17,2.95)--(-8.171,2.951)--cycle;
\draw(-8.177,3.007)--(-8.161,3);
\draw(-8.17,2.95)--(-8.171,2.951);
\filldraw[fill opacity=0.8,fill=gray!20,draw=none](-6.244,.329)--(-6.244,.263)--(-6.227,.272)--(-6.209,.299)--(-6.209,.303)--cycle;
\draw(-6.244,.329)--(-6.244,.263);
\draw(-6.209,.299)--(-6.209,.303);
\filldraw[fill opacity=0.8,fill=gray!20,draw=none](-6.263,.254)--(-6.25,.254)--(-6.238,.272)--(-6.268,.273)--cycle;
\draw(-6.263,.254)--(-6.25,.254);
\draw(-6.238,.272)--(-6.268,.273);
\filldraw[fill opacity=0.8,fill=gray!20,draw=none](-7.756,4.528)--(-7.754,4.544)--(-7.762,4.551)--(-7.769,4.534)--cycle;
\draw(-7.762,4.551)--(-7.769,4.534);
\filldraw[fill opacity=0.8,fill=gray!20,draw=none](-7.787,4.542)--(-7.768,4.534)--(-7.76,4.554)--(-7.797,4.561)--cycle;
\draw(-7.768,4.534)--(-7.76,4.554);
\filldraw[fill opacity=0.8,fill=gray!20,draw=none](-7.486,4.609)--(-7.479,4.542)--(-7.477,4.532)--(-7.483,4.595)--cycle;
\draw(-7.486,4.609)--(-7.479,4.542);
\draw(-7.477,4.532)--(-7.483,4.595);
\filldraw[fill opacity=0.5,fill=gray!20](-10.084,-.854)--(-10.154,-.838)--(-10.536,-.56)--(-10.464,-.577)--cycle;
\filldraw[fill opacity=0.8,fill=gray!20,draw=none](-8.161,2.959)--(-8.161,2.959)--(-8.161,2.96)--cycle;
\filldraw[fill opacity=0.8,fill=gray!20](-7.79,.936)--(-7.838,.942)--(-7.83,.947)--(-7.79,.936)--cycle;
\filldraw[fill opacity=0.8,fill=gray!20,draw=none](-8.172,2.943)--(-8.164,2.948)--(-8.161,2.959)--(-8.161,2.96)--(-8.161,2.976)--cycle;
\draw(-8.172,2.943)--(-8.164,2.948);
\filldraw[fill opacity=0.8,fill=gray!20,draw=none](-8.185,2.946)--(-8.202,2.908)--(-8.209,2.912)--cycle;
\draw(-8.202,2.908)--(-8.209,2.912);
\filldraw[fill opacity=0.8,fill=gray!20,draw=none](-8.185,2.946)--(-8.202,2.908)--(-8.209,2.912)--cycle;
\draw(-8.202,2.908)--(-8.209,2.912);
\filldraw[fill opacity=0.8,fill=gray!20,draw=none](-8.233,2.906)--(-8.223,2.902)--(-8.209,2.912)--(-8.185,2.946)--(-8.207,2.956)--cycle;
\draw(-8.233,2.906)--(-8.223,2.902);
\draw(-8.185,2.946)--(-8.207,2.956);
\filldraw[fill opacity=0.8,fill=gray!20,draw=none](-8.209,2.912)--(-8.206,2.91)--(-8.223,2.902)--cycle;
\draw(-8.209,2.912)--(-8.206,2.91);
\filldraw[fill opacity=0.8,fill=gray!20,draw=none](-8.209,2.912)--(-8.206,2.91)--(-8.223,2.902)--cycle;
\draw(-8.209,2.912)--(-8.206,2.91);
\filldraw[fill opacity=0.8,fill=gray!20,draw=none](-8.184,3.131)--(-8.172,3.149)--(-8.177,3.165)--(-8.192,3.132)--cycle;
\draw(-8.177,3.165)--(-8.192,3.132);
\filldraw[fill opacity=0.8,fill=gray!20,draw=none](-8.009,4.047)--(-8.021,4.052)--(-8.022,4.05)--cycle;
\draw(-8.021,4.052)--(-8.022,4.05);
\filldraw[fill opacity=0.8,fill=gray!20,draw=none](-8.019,4.041)--(-8.016,4.043)--(-8.022,4.05)--(-8.038,4.046)--cycle;
\draw(-8.022,4.05)--(-8.038,4.046)--(-8.019,4.041)--(-8.016,4.043);
\filldraw[fill opacity=0.8,fill=gray!20,draw=none](-5.971,.441)--(-5.973,.393)--(-5.97,.384)--cycle;
\filldraw[fill opacity=0.8,fill=gray!20,draw=none](-8.177,3.004)--(-8.171,2.951)--(-8.181,2.955)--cycle;
\draw(-8.171,2.951)--(-8.181,2.955);
\filldraw[fill opacity=0.8,fill=gray!20,draw=none](-8.176,3.007)--(-8.161,3)--(-8.17,2.95)--(-8.181,2.955)--cycle;
\draw(-8.176,3.007)--(-8.161,3);
\draw(-8.17,2.95)--(-8.181,2.955);
\filldraw[fill opacity=0.8,fill=gray!20,draw=none](-5.971,.471)--(-5.971,.441)--(-5.967,.506)--(-5.967,.515)--cycle;
\draw(-5.967,.506)--(-5.967,.515);
\filldraw[fill opacity=0.8,fill=gray!20,draw=none](-7.703,4.744)--(-7.7,4.705)--(-7.65,4.684)--(-7.658,4.764)--cycle;
\draw(-7.65,4.684)--(-7.658,4.764)--(-7.703,4.744)--(-7.7,4.705);
\filldraw[fill opacity=0.8,fill=gray!20,draw=none](-7.7,4.714)--(-7.729,4.726)--(-7.726,4.725)--(-7.698,4.713)--(-7.695,4.712)--cycle;
\draw(-7.726,4.725)--(-7.698,4.713)--(-7.695,4.712);
\filldraw[fill opacity=0.8,fill=gray!20,draw=none](-7.624,4.863)--(-7.623,4.869)--(-7.635,4.875)--(-7.651,4.865)--cycle;
\draw(-7.635,4.875)--(-7.651,4.865)--(-7.624,4.863)--(-7.623,4.869);
\filldraw[fill opacity=0.8,fill=gray!20,draw=none](-8.163,3.733)--(-8.142,3.697)--(-8.129,3.727)--cycle;
\draw(-8.142,3.697)--(-8.129,3.727);
\filldraw[fill opacity=0.8,fill=gray!20,draw=none](-8.163,3.733)--(-8.388,3.228)--(-8.376,3.223)--(-8.356,3.217)--(-8.142,3.697)--cycle;
\draw(-8.163,3.733)--(-8.388,3.228);
\draw(-8.356,3.217)--(-8.142,3.697);
\filldraw[fill opacity=0.8,fill=gray!20,draw=none](-6.081,.557)--(-6.038,.563)--(-6.062,.566)--(-6.095,.565)--cycle;
\filldraw[fill opacity=0.8,fill=gray!20,draw=none](-7.757,.385)--(-7.731,.374)--(-7.716,.423)--(-7.758,.44)--cycle;
\draw(-7.757,.385)--(-7.731,.374)--(-7.716,.423)--(-7.758,.44);
\filldraw[fill opacity=0.8,fill=gray!20,draw=none](-7.673,.469)--(-7.728,.47)--(-7.749,.436)--(-7.748,.422)--(-7.658,.421)--cycle;
\draw(-7.673,.469)--(-7.728,.47);
\draw(-7.748,.422)--(-7.658,.421);
\filldraw[fill opacity=0.8,fill=gray!20,draw=none](-7.674,.467)--(-7.667,.485)--(-7.682,.511)--(-7.733,.501)--(-7.72,.458)--cycle;
\draw(-7.667,.485)--(-7.682,.511)--(-7.733,.501)--(-7.72,.458)--(-7.674,.467);
\filldraw[fill opacity=0.8,fill=gray!20,draw=none](-7.744,4.596)--(-7.742,4.6)--(-7.725,4.725)--(-7.732,4.728)--(-7.752,4.676)--cycle;
\draw(-7.744,4.596)--(-7.742,4.6);
\draw(-7.725,4.725)--(-7.732,4.728)--(-7.752,4.676);
\filldraw[fill opacity=0.8,fill=gray!20,draw=none](-6.258,.467)--(-6.259,.435)--(-6.259,.46)--cycle;
\draw(-6.259,.435)--(-6.259,.46);
\filldraw[fill opacity=0.8,fill=gray!20,draw=none](-8.181,2.955)--(-8.17,2.95)--(-8.172,2.943)--(-8.186,2.917)--(-8.194,2.905)--(-8.202,2.908)--cycle;
\draw(-8.181,2.955)--(-8.17,2.95);
\draw(-8.194,2.905)--(-8.202,2.908);
\filldraw[fill opacity=0.8,fill=gray!20,draw=none](-8.181,2.955)--(-8.17,2.95)--(-8.172,2.943)--(-8.186,2.917)--(-8.194,2.905)--(-8.202,2.908)--cycle;
\draw(-8.181,2.955)--(-8.17,2.95);
\draw(-8.194,2.905)--(-8.202,2.908);
\filldraw[fill opacity=0.8,fill=gray!20,draw=none](-7.769,.335)--(-7.705,.37)--(-7.769,.371)--cycle;
\draw(-7.705,.37)--(-7.769,.371);
\filldraw[fill opacity=0.8,fill=gray!20,draw=none](-7.677,.418)--(-7.678,.421)--(-7.796,.423)--(-7.797,.371)--(-7.679,.37)--cycle;
\draw(-7.678,.421)--(-7.796,.423)--(-7.797,.371)--(-7.679,.37);
\filldraw[fill opacity=0.8,fill=gray!20](-7.638,.378)--(-7.643,.425)--(-7.711,.412)--(-7.709,.364)--cycle;
\filldraw[fill opacity=0.8,fill=gray!20,draw=none](-6.192,.253)--(-6.174,.253)--(-6.192,.265)--cycle;
\draw(-6.192,.253)--(-6.174,.253);
\filldraw[fill opacity=0.8,fill=gray!20,draw=none](-8.325,2.837)--(-8.364,2.85)--(-8.364,2.835)--cycle;
\draw(-8.364,2.85)--(-8.364,2.835)--(-8.325,2.837);
\filldraw[fill opacity=0.5,fill=gray!20,draw=none](-8.641,2.899)--(-8.661,2.924)--(-8.311,2.833)--(-8.25,2.813)--(-8.207,2.75)--cycle;
\draw(-8.311,2.833)--(-8.25,2.813)--(-8.207,2.75)--(-8.641,2.899)--(-8.661,2.924);
\filldraw[fill opacity=0.8,fill=gray!20,draw=none](-5.972,.555)--(-5.971,.471)--(-5.967,.515)--(-5.967,.563)--cycle;
\draw(-5.967,.515)--(-5.967,.563)--(-5.972,.555);
\filldraw[fill opacity=0.8,fill=gray!20,draw=none](-6.958,.315)--(-7.14,.317)--(-7.129,.282)--(-6.956,.28)--cycle;
\draw(-7.129,.282)--(-6.956,.28)--(-6.958,.315)--(-7.14,.317);
\filldraw[fill opacity=0.8,fill=gray!20,draw=none](-7.056,.249)--(-7.068,.297)--(-7.129,.312)--(-7.131,.288)--(-7.122,.265)--cycle;
\draw(-7.131,.288)--(-7.122,.265)--(-7.056,.249)--(-7.068,.297)--(-7.129,.312);
\filldraw[fill opacity=0.8,fill=gray!20,draw=none](-7.557,4.455)--(-7.57,4.438)--(-7.571,4.444)--cycle;
\draw(-7.57,4.438)--(-7.571,4.444);
\filldraw[fill opacity=0.5,fill=gray!20,draw=none](-8.791,2.58)--(-8.618,2.505)--(-8.354,2.414)--(-8.336,2.415)--(-8.434,2.458)--cycle;
\draw(-8.336,2.415)--(-8.434,2.458)--(-8.791,2.58)--(-8.618,2.505)--(-8.354,2.414);
\filldraw[fill opacity=0.8,fill=gray!20,draw=none](-7.716,4.721)--(-7.718,4.708)--(-7.706,4.717)--cycle;
\filldraw[fill opacity=0.8,fill=gray!20,draw=none](-7.732,4.673)--(-7.702,4.703)--(-7.698,4.713)--(-7.725,4.725)--cycle;
\draw(-7.702,4.703)--(-7.698,4.713)--(-7.725,4.725);
\filldraw[fill opacity=0.8,fill=gray!20,draw=none](-7.726,4.669)--(-7.724,4.681)--(-7.732,4.673)--(-7.731,4.663)--cycle;
\draw(-7.732,4.673)--(-7.731,4.663);
\filldraw[fill opacity=0.8,fill=gray!20,draw=none](-7.724,4.681)--(-7.728,4.661)--(-7.699,4.704)--(-7.7,4.705)--cycle;
\draw(-7.699,4.704)--(-7.7,4.705);
\filldraw[fill opacity=0.8,fill=gray!20,draw=none](-7.7,4.714)--(-7.695,4.712)--(-7.689,4.71)--cycle;
\draw(-7.695,4.712)--(-7.689,4.71);
\filldraw[fill opacity=0.8,fill=gray!20,draw=none](-7.653,4.864)--(-7.672,4.87)--(-7.72,4.759)--(-7.731,4.663)--(-7.712,4.685)--(-7.639,4.852)--cycle;
\draw(-7.672,4.87)--(-7.72,4.759);
\draw(-7.712,4.685)--(-7.639,4.852);
\filldraw[fill opacity=0.8,fill=gray!20](-7.681,4.834)--(-7.651,4.865)--(-7.671,4.87)--(-7.718,4.843)--cycle;
\filldraw[fill opacity=0.8,fill=gray!20,draw=none](-7.941,1.017)--(-7.929,1.019)--(-7.924,.995)--cycle;
\draw(-7.941,1.017)--(-7.929,1.019);
\filldraw[fill opacity=0.8,fill=gray!20,draw=none](-7.929,1.019)--(-7.924,.995)--(-7.949,1.027)--(-7.938,1.034)--cycle;
\draw(-7.924,.995)--(-7.949,1.027)--(-7.938,1.034);
\filldraw[fill opacity=0.8,fill=gray!20,draw=none](-8.388,3.228)--(-8.389,3.226)--(-8.376,3.223)--cycle;
\draw(-8.388,3.228)--(-8.389,3.226);
\filldraw[fill opacity=0.8,fill=gray!20,draw=none](-8.386,3.217)--(-8.383,3.219)--(-8.389,3.226)--(-8.405,3.222)--cycle;
\draw(-8.389,3.226)--(-8.405,3.222)--(-8.386,3.217)--(-8.383,3.219);
\filldraw[fill opacity=0.8,fill=gray!20,draw=none](-8.4,3.2)--(-8.374,3.178)--(-8.365,3.196)--cycle;
\draw(-8.374,3.178)--(-8.365,3.196);
\filldraw[fill opacity=0.8,fill=gray!20,draw=none](-8.17,2.93)--(-8.173,2.931)--(-8.164,2.949)--cycle;
\filldraw[fill opacity=0.8,fill=gray!20,draw=none](-8.161,2.926)--(-8.17,2.93)--(-8.164,2.949)--(-8.152,2.943)--cycle;
\draw(-8.164,2.949)--(-8.152,2.943)--(-8.161,2.926);
\filldraw[fill opacity=0.8,fill=gray!20,draw=none](-7.639,4.852)--(-7.625,4.849)--(-7.624,4.863)--(-7.651,4.865)--(-7.653,4.864)--cycle;
\draw(-7.625,4.849)--(-7.624,4.863)--(-7.651,4.865)--(-7.653,4.864);
\filldraw[fill opacity=0.8,fill=gray!20,draw=none](-7.664,4.887)--(-7.672,4.87)--(-7.636,4.858)--(-7.63,4.873)--cycle;
\draw(-7.636,4.858)--(-7.63,4.873)--(-7.664,4.887)--(-7.672,4.87);
\filldraw[fill opacity=0.8,fill=gray!20,draw=none](-7.72,4.759)--(-7.753,4.684)--(-7.745,4.609)--(-7.735,4.633)--cycle;
\draw(-7.72,4.759)--(-7.753,4.684);
\draw(-7.745,4.609)--(-7.735,4.633);
\filldraw[fill opacity=0.8,fill=gray!20,draw=none](-7.802,4.572)--(-7.784,4.54)--(-7.776,4.537)--(-7.764,4.565)--cycle;
\draw(-7.784,4.54)--(-7.776,4.537)--(-7.764,4.565);
\filldraw[fill opacity=0.8,fill=gray!20,draw=none](-7.802,4.572)--(-7.81,4.552)--(-7.784,4.54)--cycle;
\draw(-7.802,4.572)--(-7.81,4.552)--(-7.784,4.54);
\filldraw[fill opacity=0.8,fill=gray!20,draw=none](-8.151,2.983)--(-8.164,2.949)--(-8.159,2.978)--(-8.157,2.985)--cycle;
\filldraw[fill opacity=0.8,fill=gray!20,draw=none](-7.571,4.444)--(-7.57,4.435)--(-7.616,4.428)--cycle;
\draw(-7.571,4.444)--(-7.57,4.435);
\filldraw[fill opacity=0.8,fill=gray!20,draw=none](-7.585,4.433)--(-7.577,4.442)--(-7.572,4.462)--(-7.581,4.447)--(-7.586,4.436)--cycle;
\draw(-7.581,4.447)--(-7.586,4.436);
\filldraw[fill opacity=0.8,fill=gray!20,draw=none](-7.569,4.476)--(-7.566,4.473)--(-7.562,4.472)--(-7.56,4.476)--cycle;
\draw(-7.562,4.472)--(-7.56,4.476);
\filldraw[fill opacity=0.8,fill=gray!20,draw=none](-7.572,4.462)--(-7.568,4.48)--(-7.581,4.447)--cycle;
\draw(-7.568,4.48)--(-7.581,4.447);
\filldraw[fill opacity=0.8,fill=gray!20,draw=none](-7.58,4.466)--(-7.588,4.455)--(-7.583,4.453)--cycle;
\draw(-7.588,4.455)--(-7.583,4.453);
\filldraw[fill opacity=0.8,fill=gray!20,draw=none](-7.577,4.477)--(-7.605,4.479)--(-7.575,4.474)--cycle;
\filldraw[fill opacity=0.8,fill=gray!20,draw=none](-7.577,4.477)--(-7.58,4.466)--(-7.575,4.474)--cycle;
\filldraw[fill opacity=0.8,fill=gray!20,draw=none](-7.577,4.477)--(-7.575,4.484)--(-7.578,4.478)--cycle;
\draw(-7.575,4.484)--(-7.578,4.478);
\filldraw[fill opacity=0.8,fill=gray!20,draw=none](-7.58,4.48)--(-7.57,4.477)--(-7.568,4.48)--(-7.571,4.493)--(-7.574,4.493)--cycle;
\draw(-7.57,4.477)--(-7.568,4.48);
\filldraw[fill opacity=0.8,fill=gray!20,draw=none](-7.568,4.48)--(-7.564,4.492)--(-7.571,4.493)--cycle;
\draw(-7.568,4.48)--(-7.564,4.492);
\filldraw[fill opacity=0.8,fill=gray!20,draw=none](-7.57,4.477)--(-7.567,4.476)--(-7.56,4.476)--(-7.551,4.487)--(-7.578,4.486)--cycle;
\draw(-7.56,4.476)--(-7.551,4.487)--(-7.578,4.486);
\filldraw[fill opacity=0.8,fill=gray!20,draw=none](-7.57,4.477)--(-7.569,4.476)--(-7.567,4.476)--cycle;
\filldraw[fill opacity=0.8,fill=gray!20,draw=none](-7.58,4.48)--(-7.577,4.477)--(-7.569,4.476)--(-7.57,4.477)--cycle;
\filldraw[fill opacity=0.8,fill=gray!20,draw=none](-7.58,4.466)--(-7.577,4.477)--(-7.578,4.478)--(-7.588,4.455)--(-7.588,4.455)--cycle;
\draw(-7.578,4.478)--(-7.588,4.455)--(-7.588,4.455);
\filldraw[fill opacity=0.8,fill=gray!20,draw=none](-7.598,4.485)--(-7.58,4.48)--(-7.574,4.493)--(-7.609,4.496)--(-7.61,4.492)--cycle;
\draw(-7.609,4.496)--(-7.61,4.492);
\filldraw[fill opacity=0.8,fill=gray!20,draw=none](-7.621,4.487)--(-7.625,4.48)--(-7.624,4.48)--(-7.62,4.486)--cycle;
\draw(-7.624,4.48)--(-7.62,4.486);
\filldraw[fill opacity=0.8,fill=gray!20,draw=none](-7.629,4.481)--(-7.627,4.48)--(-7.629,4.491)--(-7.632,4.483)--cycle;
\draw(-7.629,4.491)--(-7.632,4.483);
\filldraw[fill opacity=0.8,fill=gray!20,draw=none](-7.605,4.479)--(-7.577,4.477)--(-7.585,4.485)--(-7.629,4.483)--(-7.629,4.482)--cycle;
\draw(-7.585,4.485)--(-7.629,4.483)--(-7.629,4.482);
\filldraw[fill opacity=0.8,fill=gray!20,draw=none](-7.605,4.479)--(-7.629,4.482)--(-7.629,4.48)--cycle;
\draw(-7.629,4.482)--(-7.629,4.48);
\filldraw[fill opacity=0.8,fill=gray!20,draw=none](-7.621,4.487)--(-7.619,4.486)--(-7.617,4.485)--(-7.62,4.486)--cycle;
\draw(-7.617,4.485)--(-7.62,4.486);
\filldraw[fill opacity=0.8,fill=gray!20,draw=none](-7.625,4.488)--(-7.621,4.487)--(-7.62,4.486)--cycle;
\draw(-7.62,4.486)--(-7.625,4.488);
\filldraw[fill opacity=0.8,fill=gray!20,draw=none](-7.551,4.487)--(-7.549,4.49)--(-7.63,4.493)--(-7.629,4.483)--cycle;
\draw(-7.63,4.493)--(-7.629,4.483)--(-7.551,4.487)--(-7.549,4.49);
\filldraw[fill opacity=0.8,fill=gray!20,draw=none](-7.602,4.484)--(-7.617,4.488)--(-7.624,4.471)--(-7.588,4.455)--(-7.581,4.471)--cycle;
\draw(-7.624,4.471)--(-7.588,4.455)--(-7.581,4.471);
\filldraw[fill opacity=0.8,fill=gray!20,draw=none](-7.602,4.484)--(-7.581,4.471)--(-7.578,4.478)--cycle;
\draw(-7.581,4.471)--(-7.578,4.478);
\filldraw[fill opacity=0.8,fill=gray!20,draw=none](-7.549,4.49)--(-7.535,4.524)--(-7.632,4.52)--(-7.63,4.493)--cycle;
\draw(-7.549,4.49)--(-7.535,4.524)--(-7.632,4.52)--(-7.63,4.493);
\filldraw[fill opacity=0.8,fill=gray!20,draw=none](-7.615,4.492)--(-7.617,4.488)--(-7.602,4.484)--cycle;
\filldraw[fill opacity=0.8,fill=gray!20,draw=none](-7.577,4.457)--(-7.573,4.469)--(-7.598,4.485)--(-7.612,4.488)--(-7.614,4.482)--cycle;
\draw(-7.577,4.457)--(-7.573,4.469);
\draw(-7.612,4.488)--(-7.614,4.482);
\filldraw[fill opacity=0.8,fill=gray!20,draw=none](-7.573,4.469)--(-7.57,4.477)--(-7.598,4.485)--cycle;
\draw(-7.573,4.469)--(-7.57,4.477);
\filldraw[fill opacity=0.8,fill=gray!20,draw=none](-7.58,4.48)--(-7.57,4.477)--(-7.578,4.486)--(-7.585,4.485)--cycle;
\draw(-7.578,4.486)--(-7.585,4.485);
\filldraw[fill opacity=0.8,fill=gray!20,draw=none](-7.82,3.965)--(-7.667,4.309)--(-7.704,4.323)--(-7.857,3.979)--cycle;
\draw(-7.82,3.965)--(-7.667,4.309);
\draw(-7.704,4.323)--(-7.857,3.979);
\filldraw[fill opacity=0.8,fill=gray!20,draw=none](-7.616,4.428)--(-7.615,4.427)--(-7.615,4.428)--cycle;
\filldraw[fill opacity=0.8,fill=gray!20,draw=none](-7.615,4.426)--(-7.615,4.427)--(-7.616,4.428)--(-7.621,4.425)--cycle;
\draw(-7.615,4.426)--(-7.615,4.427);
\filldraw[fill opacity=0.8,fill=gray!20,draw=none](-7.616,4.428)--(-7.585,4.433)--(-7.618,4.399)--(-7.621,4.426)--cycle;
\draw(-7.618,4.399)--(-7.621,4.426);
\filldraw[fill opacity=0.8,fill=gray!20,draw=none](-7.616,4.428)--(-7.621,4.426)--(-7.621,4.427)--cycle;
\draw(-7.621,4.426)--(-7.621,4.427);
\filldraw[fill opacity=0.8,fill=gray!20,draw=none](-7.621,4.427)--(-7.621,4.426)--(-7.669,4.429)--(-7.67,4.434)--cycle;
\draw(-7.621,4.427)--(-7.621,4.426);
\draw(-7.669,4.429)--(-7.67,4.434);
\filldraw[fill opacity=0.8,fill=gray!20,draw=none](-7.621,4.425)--(-7.662,4.417)--(-7.672,4.394)--cycle;
\draw(-7.662,4.417)--(-7.672,4.394);
\filldraw[fill opacity=0.8,fill=gray!20,draw=none](-7.715,4.327)--(-7.704,4.323)--(-7.663,4.415)--(-7.668,4.418)--(-7.712,4.425)--(-7.723,4.4)--cycle;
\draw(-7.704,4.323)--(-7.663,4.415);
\draw(-7.712,4.425)--(-7.723,4.4);
\filldraw[fill opacity=0.8,fill=gray!20,draw=none](-7.668,4.418)--(-7.703,4.444)--(-7.712,4.425)--cycle;
\draw(-7.703,4.444)--(-7.712,4.425);
\filldraw[fill opacity=0.8,fill=gray!20,draw=none](-7.715,4.326)--(-7.68,4.313)--(-7.663,4.356)--(-7.694,4.413)--(-7.699,4.414)--(-7.719,4.364)--cycle;
\draw(-7.715,4.326)--(-7.68,4.313)--(-7.663,4.356);
\draw(-7.699,4.414)--(-7.719,4.364);
\filldraw[fill opacity=0.8,fill=gray!20,draw=none](-7.663,4.356)--(-7.648,4.395)--(-7.667,4.41)--(-7.694,4.413)--cycle;
\draw(-7.663,4.356)--(-7.648,4.395);
\filldraw[fill opacity=0.8,fill=gray!20,draw=none](-7.667,4.41)--(-7.693,4.429)--(-7.699,4.414)--cycle;
\draw(-7.693,4.429)--(-7.699,4.414);
\filldraw[fill opacity=0.8,fill=gray!20,draw=none](-7.616,4.379)--(-7.591,4.101)--(-7.64,4.112)--(-7.666,4.398)--cycle;
\draw(-7.616,4.379)--(-7.591,4.101);
\draw(-7.64,4.112)--(-7.666,4.398);
\filldraw[fill opacity=0.8,fill=gray!20,draw=none](-7.723,4.4)--(-7.705,4.441)--(-7.708,4.45)--(-7.731,4.475)--cycle;
\draw(-7.723,4.4)--(-7.705,4.441);
\filldraw[fill opacity=0.8,fill=gray!20,draw=none](-7.719,4.364)--(-7.697,4.419)--(-7.708,4.445)--(-7.73,4.47)--cycle;
\draw(-7.719,4.364)--(-7.697,4.419);
\filldraw[fill opacity=0.8,fill=gray!20,draw=none](-7.663,4.358)--(-7.642,4.128)--(-7.668,4.134)--(-7.681,4.145)--(-7.706,4.429)--cycle;
\draw(-7.663,4.358)--(-7.642,4.128);
\draw(-7.681,4.145)--(-7.706,4.429);
\filldraw[fill opacity=0.8,fill=gray!20](-7.663,4.193)--(-7.653,4.24)--(-7.556,4.244)--(-7.555,4.197)--cycle;
\filldraw[fill opacity=0.8,fill=gray!20](-7.653,4.24)--(-7.637,4.278)--(-7.558,4.281)--(-7.556,4.244)--cycle;
\filldraw[fill opacity=0.8,fill=gray!20](-7.637,4.278)--(-7.617,4.303)--(-7.561,4.305)--(-7.558,4.281)--cycle;
\filldraw[fill opacity=0.8,fill=gray!20,draw=none](-7.536,4.303)--(-7.561,4.305)--(-7.562,4.308)--cycle;
\draw(-7.536,4.303)--(-7.561,4.305)--(-7.562,4.308);
\filldraw[fill opacity=0.8,fill=gray!20,draw=none](-7.617,4.303)--(-7.615,4.303)--(-7.564,4.308)--(-7.562,4.308)--(-7.561,4.305)--cycle;
\draw(-7.562,4.308)--(-7.561,4.305)--(-7.617,4.303)--(-7.615,4.303);
\filldraw[fill opacity=0.8,fill=gray!20](-7.607,4.77)--(-7.554,4.197)--(-7.505,4.187)--(-7.558,4.76)--cycle;
\filldraw[fill opacity=0.8,fill=gray!20,draw=none](-7.91,.317)--(-7.886,.315)--(-7.888,.322)--(-7.904,.326)--cycle;
\draw(-7.886,.315)--(-7.888,.322)--(-7.904,.326);
\filldraw[fill opacity=0.8,fill=gray!20](-7.79,.936)--(-7.748,.946)--(-7.743,.941)--(-7.79,.936)--cycle;
\filldraw[fill opacity=0.8,fill=gray!20,draw=none](-5.971,.458)--(-5.97,.454)--(-5.987,.479)--(-5.979,.475)--cycle;
\draw(-5.987,.479)--(-5.979,.475);
\filldraw[fill opacity=0.8,fill=gray!20,draw=none](-5.971,.458)--(-5.979,.475)--(-5.974,.473)--cycle;
\draw(-5.979,.475)--(-5.974,.473);
\filldraw[fill opacity=0.8,fill=gray!20,draw=none](-5.971,.441)--(-5.971,.471)--(-5.974,.439)--(-5.974,.394)--(-5.973,.393)--cycle;
\draw(-5.974,.439)--(-5.974,.394);
\filldraw[fill opacity=0.8,fill=gray!20,draw=none](-5.971,.47)--(-5.971,.471)--(-5.972,.555)--(-5.974,.551)--(-5.974,.473)--cycle;
\draw(-5.972,.555)--(-5.974,.551)--(-5.974,.473);
\filldraw[fill opacity=0.8,fill=gray!20,draw=none](-5.971,.47)--(-5.974,.473)--(-5.974,.439)--cycle;
\draw(-5.974,.473)--(-5.974,.439);
\filldraw[fill opacity=0.8,fill=gray!20,draw=none](-5.952,.433)--(-5.957,.435)--(-5.966,.448)--(-5.971,.458)--(-5.974,.473)--(-5.955,.465)--cycle;
\draw(-5.952,.433)--(-5.957,.435);
\draw(-5.974,.473)--(-5.955,.465);
\filldraw[fill opacity=0.8,fill=gray!20,draw=none](-7.49,4.657)--(-7.492,4.675)--(-7.498,4.662)--cycle;
\draw(-7.492,4.675)--(-7.498,4.662);
\filldraw[fill opacity=0.8,fill=gray!20,draw=none](-6.176,.507)--(-6.179,.501)--(-6.174,.491)--(-6.151,.49)--(-6.155,.5)--cycle;
\draw(-6.174,.491)--(-6.151,.49);
\filldraw[fill opacity=0.8,fill=gray!20,draw=none](-6.162,.508)--(-6.172,.513)--(-6.199,.47)--(-6.192,.467)--cycle;
\draw(-6.162,.508)--(-6.172,.513);
\draw(-6.199,.47)--(-6.192,.467);
\filldraw[fill opacity=0.8,fill=gray!20,draw=none](-6.259,.505)--(-6.25,.491)--(-6.184,.491)--(-6.179,.501)--(-6.183,.511)--(-6.257,.512)--cycle;
\draw(-6.25,.491)--(-6.184,.491);
\draw(-6.183,.511)--(-6.257,.512);
\filldraw[fill opacity=0.8,fill=gray!20,draw=none](-6.137,.55)--(-6.081,.557)--(-6.095,.565)--(-6.134,.564)--cycle;
\filldraw[fill opacity=0.8,fill=gray!20,draw=none](-6.244,.473)--(-6.227,.454)--(-6.212,.454)--(-6.227,.491)--(-6.25,.491)--cycle;
\draw(-6.227,.454)--(-6.212,.454);
\draw(-6.227,.491)--(-6.25,.491);
\filldraw[fill opacity=0.8,fill=gray!20,draw=none](-6.259,.512)--(-6.259,.435)--(-6.244,.368)--(-6.244,.51)--cycle;
\draw(-6.259,.512)--(-6.259,.435);
\draw(-6.244,.368)--(-6.244,.51);
\filldraw[fill opacity=0.8,fill=gray!20,draw=none](-6.002,.356)--(-6.002,.229)--(-5.974,.271)--(-5.974,.394)--cycle;
\draw(-6.002,.356)--(-6.002,.229);
\draw(-5.974,.271)--(-5.974,.394);
\filldraw[fill opacity=0.8,fill=gray!20,draw=none](-5.981,.22)--(-5.985,.222)--(-5.974,.271)--(-5.938,.256)--cycle;
\draw(-5.974,.271)--(-5.938,.256)--(-5.981,.22)--(-5.985,.222);
\filldraw[fill opacity=0.8,fill=gray!20,draw=none](-7.667,.485)--(-7.673,.469)--(-7.663,.469)--cycle;
\draw(-7.673,.469)--(-7.663,.469);
\filldraw[fill opacity=0.8,fill=gray!20,draw=none](-7.444,4.765)--(-7.452,4.768)--(-7.453,4.765)--cycle;
\draw(-7.452,4.768)--(-7.453,4.765);
\filldraw[fill opacity=0.5,fill=gray!20](-10.591,-.522)--(-10.626,-.465)--(-10.912,-.09)--(-10.882,-.139)--cycle;
\filldraw[fill opacity=0.8,fill=gray!20,draw=none](-9.04,1.033)--(-9.059,1.041)--(-9.086,1.078)--(-9.088,1.081)--(-9.065,1.071)--cycle;
\draw(-9.04,1.033)--(-9.059,1.041);
\draw(-9.088,1.081)--(-9.065,1.071);
\filldraw[fill opacity=0.8,fill=gray!20,draw=none](-9.08,1.068)--(-9.09,1.082)--(-9.088,1.081)--cycle;
\draw(-9.09,1.082)--(-9.088,1.081);
\filldraw[fill opacity=0.8,fill=gray!20,draw=none](-9.085,1.062)--(-9.08,1.068)--(-9.086,1.078)--(-9.092,1.072)--cycle;
\draw(-9.086,1.078)--(-9.092,1.072);
\filldraw[fill opacity=0.8,fill=gray!20,draw=none](-9.085,1.062)--(-9.08,1.068)--(-9.086,1.078)--(-9.092,1.072)--cycle;
\draw(-9.086,1.078)--(-9.092,1.072);
\filldraw[fill opacity=0.8,fill=gray!20,draw=none](-9.085,1.062)--(-9.08,1.068)--(-9.086,1.078)--(-9.092,1.072)--cycle;
\draw(-9.086,1.078)--(-9.092,1.072);
\filldraw[fill opacity=0.8,fill=gray!20,draw=none](-9.077,1.036)--(-9.095,1.079)--(-9.098,1.077)--(-9.083,1.032)--cycle;
\draw(-9.095,1.079)--(-9.098,1.077)--(-9.083,1.032)--(-9.077,1.036);
\filldraw[fill opacity=0.8,fill=gray!20,draw=none](-9.077,1.036)--(-9.095,1.079)--(-9.098,1.077)--(-9.083,1.032)--cycle;
\draw(-9.095,1.079)--(-9.098,1.077)--(-9.083,1.032)--(-9.077,1.036);
\filldraw[fill opacity=0.8,fill=gray!20,draw=none](-9.095,1.079)--(-9.092,1.072)--(-9.09,1.073)--cycle;
\draw(-9.092,1.072)--(-9.09,1.073);
\filldraw[fill opacity=0.8,fill=gray!20,draw=none](-9.065,1.071)--(-9.089,1.082)--(-9.097,1.086)--(-9.112,1.096)--(-9.125,1.105)--(-9.104,1.095)--cycle;
\draw(-9.065,1.071)--(-9.089,1.082);
\draw(-9.125,1.105)--(-9.104,1.095);
\filldraw[fill opacity=0.8,fill=gray!20,draw=none](-9.095,1.08)--(-9.106,1.091)--(-9.098,1.077)--cycle;
\draw(-9.106,1.091)--(-9.098,1.077)--(-9.095,1.08);
\filldraw[fill opacity=0.8,fill=gray!20,draw=none](-9.095,1.08)--(-9.106,1.091)--(-9.098,1.077)--cycle;
\draw(-9.106,1.091)--(-9.098,1.077)--(-9.095,1.08);
\filldraw[fill opacity=0.8,fill=gray!20,draw=none](-9.095,1.079)--(-9.098,1.085)--(-9.099,1.084)--cycle;
\filldraw[fill opacity=0.8,fill=gray!20,draw=none](-9.095,1.079)--(-9.098,1.085)--(-9.099,1.084)--cycle;
\filldraw[fill opacity=0.8,fill=gray!20,draw=none](-9.095,1.079)--(-9.098,1.085)--(-9.099,1.084)--cycle;
\filldraw[fill opacity=0.8,fill=gray!20,draw=none](-7.948,1.198)--(-7.95,1.202)--(-7.949,1.206)--(-7.946,1.207)--cycle;
\draw(-7.95,1.202)--(-7.949,1.206)--(-7.946,1.207);
\filldraw[fill opacity=0.8,fill=gray!20,draw=none](-7.948,1.205)--(-7.949,1.206)--(-7.95,1.202)--cycle;
\draw(-7.948,1.205)--(-7.949,1.206)--(-7.95,1.202);
\filldraw[fill opacity=0.8,fill=gray!20,draw=none](-7.948,1.205)--(-7.931,1.217)--(-7.939,1.218)--(-7.949,1.206)--cycle;
\draw(-7.939,1.218)--(-7.949,1.206)--(-7.948,1.205);
\filldraw[fill opacity=0.8,fill=gray!20,draw=none](-7.954,1.212)--(-7.824,1.241)--(-7.841,1.222)--(-7.948,1.198)--cycle;
\draw(-7.954,1.212)--(-7.824,1.241)--(-7.841,1.222)--(-7.948,1.198);
\filldraw[fill opacity=0.8,fill=gray!20,draw=none](-8.395,2.907)--(-8.381,2.887)--(-8.366,2.882)--(-8.366,2.89)--cycle;
\draw(-8.366,2.882)--(-8.366,2.89);
\filldraw[fill opacity=0.8,fill=gray!20,draw=none](-8.393,2.917)--(-8.434,2.934)--(-8.424,2.899)--(-8.365,2.874)--cycle;
\draw(-8.424,2.899)--(-8.365,2.874)--(-8.393,2.917)--(-8.434,2.934);
\filldraw[fill opacity=0.8,fill=gray!20](-8.048,4.01)--(-8.019,4.041)--(-8.038,4.046)--(-8.085,4.019)--cycle;
\filldraw[fill opacity=0.8,fill=gray!20,draw=none](-6.227,.272)--(-6.201,.272)--(-6.219,.283)--cycle;
\draw(-6.227,.272)--(-6.201,.272);
\filldraw[fill opacity=0.8,fill=gray!20,draw=none](-6.208,.3)--(-6.211,.278)--(-6.201,.272)--(-6.192,.272)--(-6.206,.301)--cycle;
\draw(-6.201,.272)--(-6.192,.272);
\filldraw[fill opacity=0.8,fill=gray!20,draw=none](-6.209,.299)--(-6.219,.283)--(-6.211,.278)--(-6.208,.3)--cycle;
\filldraw[fill opacity=0.8,fill=gray!20,draw=none](-6.244,.263)--(-6.244,.189)--(-6.209,.179)--(-6.209,.281)--cycle;
\draw(-6.244,.263)--(-6.244,.189)--(-6.209,.179)--(-6.209,.281);
\filldraw[fill opacity=0.8,fill=gray!20,draw=none](-8.995,1.194)--(-8.947,1.191)--(-8.949,1.238)--(-9.022,1.243)--cycle;
\draw(-8.995,1.194)--(-8.947,1.191)--(-8.949,1.238)--(-9.022,1.243);
\filldraw[fill opacity=0.8,fill=gray!20,draw=none](-8.997,1.198)--(-8.998,1.197)--(-8.994,1.194)--cycle;
\filldraw[fill opacity=0.8,fill=gray!20,draw=none](-8.997,1.198)--(-8.998,1.197)--(-8.994,1.194)--cycle;
\filldraw[fill opacity=0.8,fill=gray!20,draw=none](-8.997,1.198)--(-8.998,1.197)--(-8.994,1.194)--cycle;
\filldraw[fill opacity=0.8,fill=gray!20,draw=none](-7.821,.537)--(-7.81,.536)--(-7.82,.544)--cycle;
\draw(-7.821,.537)--(-7.81,.536);
\filldraw[fill opacity=0.8,fill=gray!20](-7.785,1.289)--(-7.788,1.297)--(-7.764,1.295)--(-7.74,1.286)--cycle;
\filldraw[fill opacity=0.8,fill=gray!20](-7.831,1.287)--(-7.812,1.296)--(-7.788,1.297)--(-7.785,1.289)--cycle;
\filldraw[fill opacity=0.8,fill=gray!20,draw=none](-7.549,4.837)--(-7.529,4.841)--(-7.573,4.869)--(-7.595,4.864)--(-7.587,4.854)--cycle;
\draw(-7.549,4.837)--(-7.529,4.841)--(-7.573,4.869)--(-7.595,4.864)--(-7.587,4.854);
\filldraw[fill opacity=0.8,fill=gray!20,draw=none](-6.137,.55)--(-6.134,.564)--(-6.258,.56)--(-6.259,.558)--(-6.244,.546)--(-6.22,.54)--cycle;
\draw(-6.258,.56)--(-6.259,.558)--(-6.244,.546)--(-6.22,.54);
\filldraw[fill opacity=0.5,fill=gray!20](-9.82,2.906)--(-9.862,2.871)--(-9.411,2.985)--(-9.355,3.023)--cycle;
\filldraw[fill opacity=0.8,fill=gray!20,draw=none](-7.472,4.581)--(-7.48,4.557)--(-7.477,4.536)--(-7.461,4.539)--(-7.453,4.562)--cycle;
\draw(-7.477,4.536)--(-7.461,4.539)--(-7.453,4.562);
\filldraw[fill opacity=0.8,fill=gray!20,draw=none](-5.971,.458)--(-5.966,.448)--(-5.97,.454)--cycle;
\filldraw[fill opacity=0.8,fill=gray!20,draw=none](-7.121,.501)--(-7.126,.501)--(-7.127,.485)--cycle;
\draw(-7.121,.501)--(-7.126,.501);
\filldraw[fill opacity=0.8,fill=gray!20,draw=none](-8.426,3.202)--(-8.419,3.208)--(-8.442,3.219)--cycle;
\draw(-8.419,3.208)--(-8.442,3.219)--(-8.426,3.202);
\filldraw[fill opacity=0.8,fill=gray!20,draw=none](-8.408,3.193)--(-8.386,3.217)--(-8.405,3.222)--(-8.425,3.211)--cycle;
\draw(-8.408,3.193)--(-8.386,3.217)--(-8.405,3.222)--(-8.425,3.211);
\filldraw[fill opacity=0.8,fill=gray!20,draw=none](-7.127,.485)--(-7.122,.48)--(-7.119,.484)--(-7.121,.501)--cycle;
\draw(-7.127,.485)--(-7.122,.48)--(-7.119,.484);
\filldraw[fill opacity=0.8,fill=gray!20,draw=none](-7.769,.335)--(-7.789,.324)--(-7.769,.324)--cycle;
\draw(-7.789,.324)--(-7.769,.324);
\filldraw[fill opacity=0.8,fill=gray!20,draw=none](-7.789,.324)--(-7.769,.335)--(-7.799,.348)--cycle;
\draw(-7.769,.335)--(-7.799,.348);
\filldraw[fill opacity=0.8,fill=gray!20,draw=none](-8.178,2.924)--(-8.171,2.944)--(-8.172,2.943)--(-8.186,2.917)--(-8.188,2.91)--cycle;
\draw(-8.171,2.944)--(-8.172,2.943);
\draw(-8.186,2.917)--(-8.188,2.91);
\filldraw[fill opacity=0.8,fill=gray!20,draw=none](-8.169,2.95)--(-8.164,2.947)--(-8.189,2.903)--(-8.192,2.904)--cycle;
\draw(-8.169,2.95)--(-8.164,2.947);
\draw(-8.189,2.903)--(-8.192,2.904);
\filldraw[fill opacity=0.8,fill=gray!20,draw=none](-8.169,2.95)--(-8.164,2.947)--(-8.189,2.903)--(-8.192,2.904)--cycle;
\draw(-8.169,2.95)--(-8.164,2.947);
\draw(-8.189,2.903)--(-8.192,2.904);
\filldraw[fill opacity=0.8,fill=gray!20,draw=none](-8.317,2.85)--(-8.282,2.846)--(-8.277,2.858)--cycle;
\draw(-8.282,2.846)--(-8.277,2.858);
\filldraw[fill opacity=0.8,fill=gray!20,draw=none](-8.324,2.864)--(-8.318,2.861)--(-8.356,2.885)--(-8.358,2.886)--cycle;
\draw(-8.324,2.864)--(-8.318,2.861);
\draw(-8.356,2.885)--(-8.358,2.886);
\filldraw[fill opacity=0.8,fill=gray!20,draw=none](-8.323,2.863)--(-8.318,2.861)--(-8.356,2.885)--(-8.361,2.887)--cycle;
\draw(-8.323,2.863)--(-8.318,2.861);
\draw(-8.356,2.885)--(-8.361,2.887);
\filldraw[fill opacity=0.8,fill=gray!20,draw=none](-8.338,2.873)--(-8.296,2.875)--(-8.367,2.917)--(-8.366,2.89)--cycle;
\draw(-8.338,2.873)--(-8.296,2.875);
\draw(-8.367,2.917)--(-8.366,2.89);
\filldraw[fill opacity=0.8,fill=gray!20,draw=none](-8.338,2.873)--(-8.358,2.886)--(-8.361,2.887)--cycle;
\draw(-8.358,2.886)--(-8.361,2.887);
\filldraw[fill opacity=0.8,fill=gray!20,draw=none](-8.366,2.897)--(-8.367,2.917)--(-8.368,2.919)--(-8.382,2.92)--cycle;
\draw(-8.366,2.897)--(-8.367,2.917);
\draw(-8.368,2.919)--(-8.382,2.92);
\filldraw[fill opacity=0.8,fill=gray!20,draw=none](-8.361,2.887)--(-8.358,2.886)--(-8.361,2.893)--(-8.381,2.924)--(-8.382,2.924)--cycle;
\draw(-8.361,2.887)--(-8.358,2.886);
\draw(-8.381,2.924)--(-8.382,2.924);
\filldraw[fill opacity=0.8,fill=gray!20,draw=none](-8.361,2.887)--(-8.356,2.885)--(-8.381,2.924)--(-8.382,2.924)--cycle;
\draw(-8.361,2.887)--(-8.356,2.885);
\draw(-8.381,2.924)--(-8.382,2.924);
\filldraw[fill opacity=0.8,fill=gray!20,draw=none](-8.324,3.102)--(-8.318,3.099)--(-8.272,3.113)--(-8.277,3.115)--cycle;
\draw(-8.324,3.102)--(-8.318,3.099);
\draw(-8.272,3.113)--(-8.277,3.115);
\filldraw[fill opacity=0.8,fill=gray!20,draw=none](-8.324,3.102)--(-8.318,3.099)--(-8.272,3.113)--(-8.277,3.115)--cycle;
\draw(-8.324,3.102)--(-8.318,3.099);
\draw(-8.272,3.113)--(-8.277,3.115);
\filldraw[fill opacity=0.8,fill=gray!20,draw=none](-8.388,2.926)--(-8.381,2.924)--(-8.39,2.972)--(-8.397,2.975)--cycle;
\draw(-8.388,2.926)--(-8.381,2.924);
\draw(-8.39,2.972)--(-8.397,2.975);
\filldraw[fill opacity=0.8,fill=gray!20,draw=none](-8.397,2.973)--(-8.388,2.926)--(-8.381,2.924)--(-8.39,2.972)--(-8.396,2.974)--cycle;
\draw(-8.388,2.926)--(-8.381,2.924);
\draw(-8.39,2.972)--(-8.396,2.974);
\filldraw[fill opacity=0.8,fill=gray!20,draw=none](-8.361,3.069)--(-8.359,3.068)--(-8.346,3.075)--(-8.318,3.099)--(-8.323,3.101)--cycle;
\draw(-8.361,3.069)--(-8.359,3.068);
\draw(-8.318,3.099)--(-8.323,3.101);
\filldraw[fill opacity=0.8,fill=gray!20,draw=none](-8.361,3.069)--(-8.359,3.068)--(-8.346,3.075)--(-8.318,3.099)--(-8.323,3.101)--cycle;
\draw(-8.361,3.069)--(-8.359,3.068);
\draw(-8.318,3.099)--(-8.323,3.101);
\filldraw[fill opacity=0.8,fill=gray!20,draw=none](-8.397,2.973)--(-8.396,2.974)--(-8.397,2.975)--cycle;
\draw(-8.396,2.974)--(-8.397,2.975);
\filldraw[fill opacity=0.8,fill=gray!20,draw=none](-8.397,2.975)--(-8.39,2.972)--(-8.381,3.022)--(-8.388,3.025)--cycle;
\draw(-8.397,2.975)--(-8.39,2.972);
\draw(-8.381,3.022)--(-8.388,3.025);
\filldraw[fill opacity=0.8,fill=gray!20,draw=none](-8.397,2.975)--(-8.39,2.972)--(-8.381,3.022)--(-8.388,3.025)--cycle;
\draw(-8.397,2.975)--(-8.39,2.972);
\draw(-8.381,3.022)--(-8.388,3.025);
\filldraw[fill opacity=0.8,fill=gray!20,draw=none](-8.402,2.961)--(-8.4,2.952)--(-8.342,2.864)--(-8.293,2.858)--(-8.25,2.88)--(-8.177,2.94)--(-8.168,2.996)--(-8.177,3.052)--(-8.249,3.095)--(-8.317,3.113)--(-8.323,3.111)--(-8.332,3.104)--cycle;
\draw(-8.402,2.961)--(-8.4,2.952);
\draw(-8.317,3.113)--(-8.323,3.111)--(-8.332,3.104);
\filldraw[fill opacity=0.8,fill=gray!20,draw=none](-8.172,2.943)--(-8.161,2.976)--(-8.161,3)--(-8.171,2.994)--(-8.177,2.94)--cycle;
\draw(-8.161,3)--(-8.171,2.994)--(-8.177,2.94)--(-8.172,2.943);
\filldraw[fill opacity=0.8,fill=gray!20,draw=none](-8.475,.897)--(-7.941,1.017)--(-7.924,.995)--(-7.921,.981)--(-8.049,.952)--cycle;
\draw(-8.475,.897)--(-7.941,1.017);
\draw(-7.921,.981)--(-8.049,.952);
\filldraw[fill opacity=0.8,fill=gray!20,draw=none](-7.744,4.592)--(-7.742,4.6)--(-7.744,4.596)--cycle;
\draw(-7.742,4.6)--(-7.744,4.596);
\filldraw[fill opacity=0.8,fill=gray!20,draw=none](-7.735,4.633)--(-7.745,4.609)--(-7.743,4.585)--cycle;
\draw(-7.735,4.633)--(-7.745,4.609);
\filldraw[fill opacity=0.8,fill=gray!20](-7.737,4.574)--(-7.741,4.629)--(-7.817,4.647)--(-7.81,4.592)--cycle;
\filldraw[fill opacity=0.8,fill=gray!20,draw=none](-6.965,.956)--(-6.96,.957)--(-6.959,.956)--cycle;
\draw(-6.965,.956)--(-6.96,.957)--(-6.959,.956);
\filldraw[fill opacity=0.8,fill=gray!20,draw=none](-6.973,.908)--(-6.987,.911)--(-7.02,.946)--(-6.965,.956)--(-6.959,.956)--(-6.939,.914)--cycle;
\draw(-6.987,.911)--(-7.02,.946)--(-6.965,.956);
\draw(-6.959,.956)--(-6.939,.914)--(-6.973,.908);
\filldraw[fill opacity=0.8,fill=gray!20,draw=none](-6.965,.956)--(-7.02,.946)--(-7.027,.957)--cycle;
\draw(-6.965,.956)--(-7.02,.946)--(-7.027,.957);
\filldraw[fill opacity=0.8,fill=gray!20,draw=none](-7.039,.933)--(-7.044,.957)--(-7.027,.957)--(-7.02,.946)--cycle;
\draw(-7.027,.957)--(-7.02,.946)--(-7.039,.933);
\filldraw[fill opacity=0.8,fill=gray!20,draw=none](-7.057,.957)--(-7.048,.977)--(-7.044,.957)--cycle;
\filldraw[fill opacity=0.8,fill=gray!20,draw=none](-6.89,.95)--(-7.66,.981)--(-7.661,.987)--(-7.644,1.001)--(-6.881,.971)--cycle;
\draw(-7.644,1.001)--(-6.881,.971)--(-6.89,.95)--(-7.66,.981);
\filldraw[fill opacity=0.8,fill=gray!20,draw=none](-7.8,.276)--(-7.8,.291)--(-7.886,.315)--(-7.878,.282)--cycle;
\draw(-7.886,.315)--(-7.878,.282)--(-7.8,.276)--(-7.8,.291);
\filldraw[fill opacity=0.8,fill=gray!20,draw=none](-8.867,1.461)--(-8.878,1.485)--(-8.945,1.463)--(-8.945,1.457)--cycle;
\draw(-8.945,1.463)--(-8.945,1.457)--(-8.867,1.461)--(-8.878,1.485);
\filldraw[fill opacity=0.8,fill=gray!20,draw=none](-7.973,3.839)--(-7.904,3.994)--(-7.927,4.011)--(-7.944,4.017)--(-7.949,4.018)--(-8.02,3.86)--cycle;
\draw(-7.949,4.018)--(-8.02,3.86)--(-7.973,3.839)--(-7.904,3.994);
\filldraw[fill opacity=0.8,fill=gray!20](-7.896,4.017)--(-7.94,4.045)--(-7.962,4.04)--(-7.938,4.009)--cycle;
\filldraw[fill opacity=0.8,fill=gray!20,draw=none](-8.326,3.151)--(-8.315,3.151)--(-8.323,3.154)--cycle;
\draw(-8.315,3.151)--(-8.323,3.154);
\filldraw[fill opacity=0.8,fill=gray!20,draw=none](-7.683,4.496)--(-7.69,4.499)--(-7.583,4.453)--(-7.588,4.455)--(-7.624,4.471)--cycle;
\draw(-7.683,4.496)--(-7.69,4.499);
\draw(-7.583,4.453)--(-7.588,4.455)--(-7.624,4.471);
\filldraw[fill opacity=0.8,fill=gray!20,draw=none](-7.787,4.542)--(-7.775,4.521)--(-7.769,4.534)--cycle;
\draw(-7.775,4.521)--(-7.769,4.534);
\filldraw[fill opacity=0.8,fill=gray!20,draw=none](-7.731,1.042)--(-7.74,1.007)--(-7.756,.988)--(-7.777,.988)--(-7.786,.995)--(-7.794,1.234)--(-7.779,1.221)--(-7.757,1.186)--(-7.741,1.139)--(-7.731,1.089)--cycle;
\draw(-7.794,1.234)--(-7.779,1.221)--(-7.757,1.186)--(-7.741,1.139)--(-7.731,1.089)--(-7.731,1.042)--(-7.74,1.007)--(-7.756,.988)--(-7.777,.988);
\filldraw[fill opacity=0.8,fill=gray!20](-7.829,3.715)--(-7.81,3.764)--(-7.892,3.748)--(-7.902,3.7)--cycle;
\filldraw[fill opacity=0.8,fill=gray!20](-7.012,.557)--(-6.983,.588)--(-7.002,.593)--(-7.05,.566)--cycle;
\filldraw[fill opacity=0.8,fill=gray!20,draw=none](-8.506,2.939)--(-8.445,2.912)--(-8.431,2.941)--(-8.477,2.961)--cycle;
\draw(-8.506,2.939)--(-8.445,2.912);
\draw(-8.431,2.941)--(-8.477,2.961);
\filldraw[fill opacity=0.8,fill=gray!20,draw=none](-8.468,2.913)--(-8.471,2.926)--(-8.538,2.943)--(-8.544,2.943)--(-8.539,2.932)--cycle;
\draw(-8.468,2.913)--(-8.471,2.926)--(-8.538,2.943);
\draw(-8.544,2.943)--(-8.539,2.932);
\filldraw[fill opacity=0.8,fill=gray!20,draw=none](-8.462,2.891)--(-8.468,2.913)--(-8.539,2.932)--cycle;
\draw(-8.462,2.891)--(-8.468,2.913);
\filldraw[fill opacity=0.8,fill=gray!20,draw=none](-5.91,.414)--(-5.952,.433)--(-5.955,.465)--(-5.938,.458)--cycle;
\draw(-5.955,.465)--(-5.938,.458)--(-5.91,.414)--(-5.952,.433);
\filldraw[fill opacity=0.8,fill=gray!20,draw=none](-7.588,4.43)--(-7.585,4.433)--(-7.583,4.433)--cycle;
\filldraw[fill opacity=0.8,fill=gray!20,draw=none](-7.99,4.039)--(-7.989,4.041)--(-8.009,4.047)--cycle;
\draw(-7.99,4.039)--(-7.989,4.041);
\filldraw[fill opacity=0.8,fill=gray!20,draw=none](-7.585,4.433)--(-7.586,4.436)--(-7.588,4.43)--cycle;
\draw(-7.586,4.436)--(-7.588,4.43);
\filldraw[fill opacity=0.8,fill=gray!20,draw=none](-7.648,1.219)--(-7.65,1.22)--(-7.666,1.244)--(-7.661,1.239)--(-7.632,1.201)--cycle;
\draw(-7.666,1.244)--(-7.661,1.239)--(-7.632,1.201)--(-7.648,1.219);
\filldraw[fill opacity=0.8,fill=gray!20,draw=none](-7.616,4.428)--(-7.615,4.428)--(-7.624,4.471)--(-7.636,4.476)--(-7.643,4.458)--cycle;
\draw(-7.636,4.476)--(-7.643,4.458);
\filldraw[fill opacity=0.8,fill=gray!20,draw=none](-7.621,4.426)--(-7.62,4.414)--(-7.669,4.426)--(-7.669,4.429)--cycle;
\draw(-7.621,4.426)--(-7.62,4.414);
\draw(-7.669,4.426)--(-7.669,4.429);
\filldraw[fill opacity=0.8,fill=gray!20,draw=none](-7.62,4.414)--(-7.619,4.412)--(-7.668,4.418)--(-7.669,4.426)--cycle;
\draw(-7.62,4.414)--(-7.619,4.412);
\draw(-7.668,4.418)--(-7.669,4.426);
\filldraw[fill opacity=0.8,fill=gray!20,draw=none](-7.621,4.425)--(-7.616,4.428)--(-7.643,4.458)--(-7.662,4.417)--cycle;
\draw(-7.643,4.458)--(-7.662,4.417);
\filldraw[fill opacity=0.8,fill=gray!20,draw=none](-7.658,4.485)--(-7.717,4.511)--(-7.706,4.489)--cycle;
\draw(-7.717,4.511)--(-7.706,4.489)--(-7.658,4.485);
\filldraw[fill opacity=0.8,fill=gray!20,draw=none](-7.617,4.412)--(-7.588,4.43)--(-7.586,4.436)--(-7.591,4.442)--(-7.609,4.432)--cycle;
\draw(-7.588,4.43)--(-7.586,4.436);
\filldraw[fill opacity=0.8,fill=gray!20,draw=none](-7.586,4.436)--(-7.581,4.447)--(-7.591,4.442)--cycle;
\draw(-7.586,4.436)--(-7.581,4.447);
\filldraw[fill opacity=0.8,fill=gray!20,draw=none](-7.614,4.466)--(-7.591,4.442)--(-7.581,4.447)--(-7.58,4.451)--cycle;
\draw(-7.581,4.447)--(-7.58,4.451);
\filldraw[fill opacity=0.8,fill=gray!20,draw=none](-7.614,4.466)--(-7.58,4.451)--(-7.577,4.457)--(-7.614,4.482)--(-7.619,4.471)--cycle;
\draw(-7.58,4.451)--(-7.577,4.457);
\draw(-7.614,4.482)--(-7.619,4.471);
\filldraw[fill opacity=0.8,fill=gray!20,draw=none](-7.591,4.442)--(-7.619,4.471)--(-7.641,4.415)--cycle;
\draw(-7.619,4.471)--(-7.641,4.415);
\filldraw[fill opacity=0.8,fill=gray!20,draw=none](-7.669,4.426)--(-7.668,4.416)--(-7.689,4.424)--(-7.708,4.445)--(-7.708,4.447)--cycle;
\draw(-7.669,4.426)--(-7.668,4.416);
\draw(-7.708,4.445)--(-7.708,4.447);
\filldraw[fill opacity=0.8,fill=gray!20,draw=none](-7.663,4.415)--(-7.636,4.476)--(-7.681,4.495)--(-7.703,4.444)--cycle;
\draw(-7.663,4.415)--(-7.636,4.476);
\draw(-7.681,4.495)--(-7.703,4.444);
\filldraw[fill opacity=0.8,fill=gray!20,draw=none](-7.619,4.412)--(-7.619,4.411)--(-7.644,4.406)--(-7.667,4.41)--(-7.668,4.418)--cycle;
\draw(-7.619,4.412)--(-7.619,4.411);
\draw(-7.667,4.41)--(-7.668,4.418);
\filldraw[fill opacity=0.8,fill=gray!20,draw=none](-7.619,4.411)--(-7.617,4.412)--(-7.609,4.432)--(-7.641,4.415)--(-7.644,4.406)--cycle;
\draw(-7.641,4.415)--(-7.644,4.406);
\filldraw[fill opacity=0.8,fill=gray!20,draw=none](-7.646,4.395)--(-7.619,4.411)--(-7.644,4.406)--(-7.648,4.395)--cycle;
\draw(-7.644,4.406)--(-7.648,4.395);
\filldraw[fill opacity=0.8,fill=gray!20,draw=none](-7.646,4.395)--(-7.648,4.395)--(-7.649,4.392)--cycle;
\draw(-7.648,4.395)--(-7.649,4.392);
\filldraw[fill opacity=0.8,fill=gray!20,draw=none](-7.644,4.406)--(-7.619,4.411)--(-7.619,4.403)--cycle;
\draw(-7.619,4.411)--(-7.619,4.403);
\filldraw[fill opacity=0.8,fill=gray!20,draw=none](-7.624,4.48)--(-7.629,4.48)--(-7.629,4.477)--cycle;
\draw(-7.629,4.48)--(-7.629,4.477);
\filldraw[fill opacity=0.8,fill=gray!20,draw=none](-7.624,4.471)--(-7.627,4.48)--(-7.633,4.482)--(-7.636,4.476)--cycle;
\draw(-7.633,4.482)--(-7.636,4.476);
\filldraw[fill opacity=0.8,fill=gray!20,draw=none](-7.627,4.48)--(-7.626,4.477)--(-7.625,4.48)--cycle;
\filldraw[fill opacity=0.8,fill=gray!20,draw=none](-7.625,4.48)--(-7.626,4.477)--(-7.624,4.48)--cycle;
\draw(-7.626,4.477)--(-7.624,4.48);
\filldraw[fill opacity=0.8,fill=gray!20,draw=none](-7.598,4.485)--(-7.61,4.492)--(-7.612,4.488)--cycle;
\draw(-7.61,4.492)--(-7.612,4.488);
\filldraw[fill opacity=0.8,fill=gray!20,draw=none](-7.629,4.48)--(-7.629,4.483)--(-7.636,4.484)--cycle;
\draw(-7.629,4.48)--(-7.629,4.483)--(-7.636,4.484);
\filldraw[fill opacity=0.8,fill=gray!20,draw=none](-7.629,4.481)--(-7.632,4.483)--(-7.633,4.482)--cycle;
\draw(-7.632,4.483)--(-7.633,4.482);
\filldraw[fill opacity=0.8,fill=gray!20,draw=none](-7.636,4.476)--(-7.632,4.483)--(-7.678,4.5)--(-7.681,4.495)--cycle;
\draw(-7.636,4.476)--(-7.632,4.483);
\draw(-7.678,4.5)--(-7.681,4.495);
\filldraw[fill opacity=0.8,fill=gray!20,draw=none](-7.632,4.483)--(-7.674,4.51)--(-7.678,4.5)--cycle;
\draw(-7.674,4.51)--(-7.678,4.5);
\filldraw[fill opacity=0.8,fill=gray!20,draw=none](-7.649,4.506)--(-7.633,4.483)--(-7.629,4.483)--(-7.63,4.493)--cycle;
\draw(-7.633,4.483)--(-7.629,4.483)--(-7.63,4.493);
\filldraw[fill opacity=0.8,fill=gray!20,draw=none](-7.684,4.422)--(-7.668,4.416)--(-7.667,4.41)--cycle;
\draw(-7.668,4.416)--(-7.667,4.41);
\filldraw[fill opacity=0.8,fill=gray!20,draw=none](-7.648,4.395)--(-7.614,4.482)--(-7.665,4.502)--(-7.693,4.429)--cycle;
\draw(-7.648,4.395)--(-7.614,4.482);
\draw(-7.665,4.502)--(-7.693,4.429);
\filldraw[fill opacity=0.8,fill=gray!20,draw=none](-7.644,4.406)--(-7.619,4.403)--(-7.616,4.379)--(-7.666,4.398)--(-7.667,4.402)--cycle;
\draw(-7.619,4.403)--(-7.616,4.379);
\draw(-7.666,4.398)--(-7.667,4.402);
\filldraw[fill opacity=0.8,fill=gray!20,draw=none](-7.63,4.488)--(-7.614,4.482)--(-7.612,4.488)--cycle;
\draw(-7.614,4.482)--(-7.612,4.488);
\filldraw[fill opacity=0.8,fill=gray!20,draw=none](-7.615,4.492)--(-7.617,4.488)--(-7.612,4.488)--(-7.61,4.492)--cycle;
\draw(-7.612,4.488)--(-7.61,4.492);
\filldraw[fill opacity=0.8,fill=gray!20,draw=none](-7.644,4.494)--(-7.63,4.488)--(-7.617,4.488)--(-7.615,4.492)--cycle;
\filldraw[fill opacity=0.8,fill=gray!20,draw=none](-7.617,4.488)--(-7.62,4.489)--(-7.628,4.472)--(-7.624,4.471)--cycle;
\draw(-7.62,4.489)--(-7.628,4.472)--(-7.624,4.471);
\filldraw[fill opacity=0.8,fill=gray!20,draw=none](-7.678,4.494)--(-7.683,4.496)--(-7.624,4.471)--(-7.628,4.472)--cycle;
\draw(-7.624,4.471)--(-7.628,4.472)--(-7.678,4.494)--(-7.683,4.496);
\filldraw[fill opacity=0.8,fill=gray!20,draw=none](-7.615,4.492)--(-7.618,4.494)--(-7.62,4.489)--(-7.617,4.488)--cycle;
\draw(-7.618,4.494)--(-7.62,4.489);
\filldraw[fill opacity=0.8,fill=gray!20,draw=none](-7.63,4.495)--(-7.622,4.489)--(-7.62,4.489)--(-7.618,4.494)--cycle;
\draw(-7.62,4.489)--(-7.618,4.494);
\filldraw[fill opacity=0.8,fill=gray!20,draw=none](-7.667,4.489)--(-7.628,4.472)--(-7.62,4.489)--cycle;
\draw(-7.667,4.489)--(-7.628,4.472)--(-7.62,4.489);
\filldraw[fill opacity=0.8,fill=gray!20,draw=none](-7.649,4.506)--(-7.63,4.493)--(-7.632,4.52)--(-7.661,4.522)--cycle;
\draw(-7.63,4.493)--(-7.632,4.52)--(-7.661,4.522);
\filldraw[fill opacity=0.8,fill=gray!20,draw=none](-7.649,4.506)--(-7.653,4.497)--(-7.644,4.494)--(-7.63,4.493)--cycle;
\filldraw[fill opacity=0.8,fill=gray!20,draw=none](-7.696,4.507)--(-7.697,4.506)--(-7.686,4.498)--(-7.669,4.49)--(-7.665,4.502)--cycle;
\draw(-7.669,4.49)--(-7.665,4.502);
\filldraw[fill opacity=0.8,fill=gray!20,draw=none](-7.63,4.495)--(-7.676,4.497)--(-7.678,4.494)--(-7.667,4.489)--(-7.622,4.489)--cycle;
\draw(-7.676,4.497)--(-7.678,4.494)--(-7.667,4.489);
\filldraw[fill opacity=0.8,fill=gray!20,draw=none](-7.653,4.497)--(-7.649,4.506)--(-7.66,4.513)--(-7.665,4.502)--cycle;
\draw(-7.66,4.513)--(-7.665,4.502);
\filldraw[fill opacity=0.8,fill=gray!20,draw=none](-7.649,4.506)--(-7.661,4.522)--(-7.676,4.523)--cycle;
\draw(-7.661,4.522)--(-7.676,4.523);
\filldraw[fill opacity=0.8,fill=gray!20,draw=none](-7.691,4.521)--(-7.696,4.507)--(-7.665,4.502)--(-7.66,4.513)--cycle;
\draw(-7.665,4.502)--(-7.66,4.513);
\filldraw[fill opacity=0.8,fill=gray!20,draw=none](-7.667,4.519)--(-7.676,4.497)--(-7.63,4.495)--cycle;
\draw(-7.667,4.519)--(-7.676,4.497);
\filldraw[fill opacity=0.8,fill=gray!20,draw=none](-7.648,4.656)--(-7.693,4.638)--(-7.703,4.626)--(-7.633,4.621)--(-7.633,4.653)--cycle;
\draw(-7.703,4.626)--(-7.633,4.621)--(-7.633,4.653);
\filldraw[fill opacity=0.8,fill=gray!20,draw=none](-7.625,4.621)--(-7.611,4.659)--(-7.633,4.653)--(-7.633,4.621)--cycle;
\draw(-7.633,4.653)--(-7.633,4.621)--(-7.625,4.621);
\filldraw[fill opacity=0.8,fill=gray!20](-7.666,4.139)--(-7.663,4.193)--(-7.555,4.197)--(-7.555,4.144)--cycle;
\filldraw[fill opacity=0.8,fill=gray!20](-7.658,4.764)--(-7.605,4.191)--(-7.554,4.197)--(-7.607,4.77)--cycle;
\filldraw[fill opacity=0.5,fill=gray!20](-9.231,3.021)--(-9.294,3.036)--(-8.819,3.019)--(-8.753,3.005)--cycle;
\filldraw[fill opacity=0.8,fill=gray!20,draw=none](-7.734,4.502)--(-7.732,4.506)--(-7.735,4.516)--cycle;
\draw(-7.734,4.502)--(-7.732,4.506);
\filldraw[fill opacity=0.8,fill=gray!20,draw=none](-7.731,4.475)--(-7.708,4.45)--(-7.732,4.506)--(-7.734,4.502)--cycle;
\draw(-7.732,4.506)--(-7.734,4.502);
\filldraw[fill opacity=0.8,fill=gray!20,draw=none](-7.73,4.47)--(-7.708,4.445)--(-7.729,4.494)--(-7.732,4.486)--cycle;
\draw(-7.729,4.494)--(-7.732,4.486);
\filldraw[fill opacity=0.8,fill=gray!20,draw=none](-7.712,4.49)--(-7.708,4.447)--(-7.716,4.456)--(-7.735,4.511)--(-7.735,4.514)--cycle;
\draw(-7.712,4.49)--(-7.708,4.447);
\draw(-7.735,4.511)--(-7.735,4.514);
\filldraw[fill opacity=0.8,fill=gray!20](-8.104,3.75)--(-8.108,3.805)--(-8.184,3.823)--(-8.177,3.768)--cycle;
\filldraw[fill opacity=0.8,fill=gray!20,draw=none](-7.015,1.215)--(-7.049,1.209)--(-7.044,1.215)--cycle;
\draw(-7.015,1.215)--(-7.049,1.209)--(-7.044,1.215);
\filldraw[fill opacity=0.8,fill=gray!20,draw=none](-7.04,1.196)--(-7.051,1.204)--(-7.049,1.209)--(-7.042,1.21)--cycle;
\draw(-7.051,1.204)--(-7.049,1.209)--(-7.042,1.21);
\filldraw[fill opacity=0.8,fill=gray!20,draw=none](-7.051,1.207)--(-7.049,1.209)--(-7.051,1.204)--cycle;
\draw(-7.051,1.207)--(-7.049,1.209)--(-7.051,1.204);
\filldraw[fill opacity=0.8,fill=gray!20,draw=none](-7.051,1.207)--(-7.048,1.215)--(-7.044,1.215)--(-7.049,1.209)--cycle;
\draw(-7.044,1.215)--(-7.049,1.209)--(-7.051,1.207);
\filldraw[fill opacity=0.8,fill=gray!20,draw=none](-6.874,1.189)--(-7.648,1.22)--(-7.62,1.237)--(-6.881,1.208)--cycle;
\draw(-7.62,1.237)--(-6.881,1.208)--(-6.874,1.189)--(-7.648,1.22);
\filldraw[fill opacity=0.8,fill=gray!20,draw=none](-7.728,.47)--(-7.753,.47)--(-7.749,.436)--cycle;
\draw(-7.728,.47)--(-7.753,.47);
\filldraw[fill opacity=0.8,fill=gray!20](-8.44,3.144)--(-8.415,3.186)--(-8.452,3.195)--(-8.493,3.157)--cycle;
\filldraw[fill opacity=0.8,fill=gray!20,draw=none](-8.353,3.152)--(-8.326,3.151)--(-8.323,3.154)--(-8.369,3.174)--cycle;
\draw(-8.323,3.154)--(-8.369,3.174);
\filldraw[fill opacity=0.8,fill=gray!20,draw=none](-8.342,3.135)--(-8.326,3.151)--(-8.353,3.152)--cycle;
\filldraw[fill opacity=0.8,fill=gray!20,draw=none](-8.294,3.187)--(-8.311,3.193)--(-8.316,3.194)--(-8.335,3.152)--(-8.324,3.145)--(-8.305,3.142)--(-8.282,3.145)--(-8.271,3.17)--cycle;
\draw(-8.316,3.194)--(-8.335,3.152);
\draw(-8.282,3.145)--(-8.271,3.17);
\filldraw[fill opacity=0.8,fill=gray!20](-8.263,3.193)--(-8.307,3.221)--(-8.329,3.216)--(-8.305,3.185)--cycle;
\filldraw[fill opacity=0.8,fill=gray!20,draw=none](-7.677,.326)--(-7.678,.323)--(-7.677,.323)--cycle;
\draw(-7.678,.323)--(-7.677,.323);
\filldraw[fill opacity=0.8,fill=gray!20,draw=none](-6.958,.362)--(-7.66,.37)--(-7.677,.326)--(-7.677,.323)--(-6.958,.315)--cycle;
\draw(-7.677,.323)--(-6.958,.315)--(-6.958,.362)--(-7.66,.37);
\filldraw[fill opacity=0.8,fill=gray!20,draw=none](-7.825,3.956)--(-7.82,3.965)--(-7.857,3.979)--cycle;
\draw(-7.825,3.956)--(-7.82,3.965);
\filldraw[fill opacity=0.8,fill=gray!20,draw=none](-7.713,4.49)--(-7.706,4.489)--(-7.717,4.511)--(-7.73,4.517)--cycle;
\draw(-7.713,4.49)--(-7.706,4.489)--(-7.717,4.511);
\filldraw[fill opacity=0.8,fill=gray!20,draw=none](-7.61,4.864)--(-7.623,4.869)--(-7.624,4.863)--cycle;
\draw(-7.623,4.869)--(-7.624,4.863)--(-7.61,4.864);
\filldraw[fill opacity=0.8,fill=gray!20,draw=none](-7.754,.5)--(-7.733,.501)--(-7.749,.537)--(-7.796,.535)--(-7.796,.526)--cycle;
\draw(-7.754,.5)--(-7.733,.501)--(-7.749,.537)--(-7.796,.535)--(-7.796,.526);
\filldraw[fill opacity=0.8,fill=gray!20](-7.867,1.28)--(-7.83,1.292)--(-7.812,1.296)--(-7.831,1.287)--cycle;
\filldraw[fill opacity=0.8,fill=gray!20,draw=none](-7.677,.326)--(-7.66,.37)--(-7.679,.37)--cycle;
\draw(-7.66,.37)--(-7.679,.37);
\filldraw[fill opacity=0.8,fill=gray!20,draw=none](-7.787,4.542)--(-7.774,4.519)--(-7.768,4.534)--cycle;
\draw(-7.774,4.519)--(-7.768,4.534);
\filldraw[fill opacity=0.8,fill=gray!20,draw=none](-7.962,3.647)--(-7.973,3.66)--(-7.997,3.659)--(-7.996,3.657)--cycle;
\draw(-7.973,3.66)--(-7.997,3.659)--(-7.996,3.657);
\filldraw[fill opacity=0.8,fill=gray!20,draw=none](-7.996,3.657)--(-7.967,3.637)--(-7.962,3.647)--cycle;
\draw(-7.967,3.637)--(-7.962,3.647);
\filldraw[fill opacity=0.8,fill=gray!20,draw=none](-7.999,3.659)--(-8.224,3.155)--(-8.187,3.141)--(-7.967,3.637)--cycle;
\draw(-7.999,3.659)--(-8.224,3.155);
\draw(-8.187,3.141)--(-7.967,3.637);
\filldraw[fill opacity=0.8,fill=gray!20,draw=none](-8.376,3.223)--(-8.357,3.215)--(-8.356,3.217)--cycle;
\draw(-8.357,3.215)--(-8.356,3.217);
\filldraw[fill opacity=0.8,fill=gray!20,draw=none](-7.811,.509)--(-7.802,.535)--(-7.821,.537)--cycle;
\draw(-7.802,.535)--(-7.821,.537);
\filldraw[fill opacity=0.8,fill=gray!20,draw=none](-7.904,.326)--(-7.888,.322)--(-7.889,.337)--cycle;
\draw(-7.904,.326)--(-7.888,.322)--(-7.889,.337);
\filldraw[fill opacity=0.8,fill=gray!20](-6.86,.563)--(-6.904,.591)--(-6.926,.587)--(-6.903,.555)--cycle;
\filldraw[fill opacity=0.8,fill=gray!20,draw=none](-7.673,.469)--(-7.674,.467)--(-7.673,.467)--cycle;
\draw(-7.674,.467)--(-7.673,.467);
\filldraw[fill opacity=0.8,fill=gray!20,draw=none](-7.654,.469)--(-7.673,.469)--(-7.666,.445)--cycle;
\draw(-7.654,.469)--(-7.673,.469);
\filldraw[fill opacity=0.8,fill=gray!20,draw=none](-8.224,3.155)--(-8.192,3.132)--(-8.187,3.141)--cycle;
\draw(-8.192,3.132)--(-8.187,3.141);
\filldraw[fill opacity=0.8,fill=gray!20,draw=none](-6.143,.518)--(-6.13,.51)--(-6.115,.51)--(-6.114,.52)--cycle;
\filldraw[fill opacity=0.8,fill=gray!20,draw=none](-7.679,.418)--(-7.673,.467)--(-7.674,.467)--(-7.713,.419)--(-7.711,.412)--cycle;
\draw(-7.673,.467)--(-7.674,.467);
\draw(-7.713,.419)--(-7.711,.412)--(-7.679,.418);
\filldraw[fill opacity=0.8,fill=gray!20,draw=none](-7.126,.364)--(-7.126,.416)--(-7.129,.423)--(-7.141,.426)--(-7.148,.37)--cycle;
\draw(-7.129,.423)--(-7.141,.426)--(-7.148,.37)--(-7.126,.364);
\filldraw[fill opacity=0.8,fill=gray!20,draw=none](-7.129,.423)--(-7.131,.455)--(-7.141,.426)--cycle;
\draw(-7.131,.455)--(-7.141,.426)--(-7.129,.423);
\filldraw[fill opacity=0.8,fill=gray!20,draw=none](-7.129,.464)--(-7.654,.469)--(-7.666,.445)--(-7.658,.421)--(-7.14,.416)--cycle;
\draw(-7.129,.464)--(-7.654,.469);
\draw(-7.658,.421)--(-7.14,.416);
\filldraw[fill opacity=0.8,fill=gray!20,draw=none](-6.259,.505)--(-6.263,.492)--(-6.25,.491)--cycle;
\draw(-6.263,.492)--(-6.25,.491);
\filldraw[fill opacity=0.8,fill=gray!20,draw=none](-7.701,4.488)--(-7.706,4.489)--(-7.704,4.487)--cycle;
\draw(-7.701,4.488)--(-7.706,4.489)--(-7.704,4.487);
\filldraw[fill opacity=0.8,fill=gray!20,draw=none](-7.472,4.581)--(-7.453,4.562)--(-7.443,4.588)--(-7.471,4.582)--cycle;
\draw(-7.453,4.562)--(-7.443,4.588)--(-7.471,4.582);
\filldraw[fill opacity=0.8,fill=gray!20](-6.793,.261)--(-6.775,.31)--(-6.857,.294)--(-6.867,.247)--cycle;
\filldraw[fill opacity=0.8,fill=gray!20](-7.706,4.792)--(-7.681,4.834)--(-7.718,4.843)--(-7.759,4.805)--cycle;
\filldraw[fill opacity=0.8,fill=gray!20,draw=none](-8.293,2.828)--(-8.285,2.838)--(-8.286,2.839)--(-8.323,2.837)--cycle;
\draw(-8.293,2.828)--(-8.285,2.838);
\draw(-8.286,2.839)--(-8.323,2.837);
\filldraw[fill opacity=0.8,fill=gray!20](-7.72,.28)--(-7.711,.319)--(-7.801,.315)--(-7.8,.276)--cycle;
\filldraw[fill opacity=0.8,fill=gray!20](-7.92,1.242)--(-7.882,1.27)--(-7.867,1.28)--(-7.898,1.256)--cycle;
\filldraw[fill opacity=0.8,fill=gray!20](-7.833,.937)--(-7.872,.951)--(-7.882,.961)--(-7.838,.942)--cycle;
\filldraw[fill opacity=0.8,fill=gray!20](-7.79,.936)--(-7.833,.937)--(-7.838,.942)--(-7.79,.936)--cycle;
\filldraw[fill opacity=0.8,fill=gray!20,draw=none](-6.782,.291)--(-6.785,.278)--(-6.779,.278)--cycle;
\draw(-6.785,.278)--(-6.779,.278);
\filldraw[fill opacity=0.8,fill=gray!20,draw=none](-7.481,4.808)--(-7.508,4.819)--(-7.491,4.801)--cycle;
\draw(-7.508,4.819)--(-7.491,4.801)--(-7.481,4.808);
\filldraw[fill opacity=0.8,fill=gray!20,draw=none](-6.788,.314)--(-6.785,.308)--(-6.775,.31)--(-6.768,.365)--(-6.79,.361)--cycle;
\draw(-6.785,.308)--(-6.775,.31)--(-6.768,.365)--(-6.79,.361);
\filldraw[fill opacity=0.8,fill=gray!20,draw=none](-6.782,.291)--(-6.779,.278)--(-6.755,.313)--(-6.776,.313)--cycle;
\draw(-6.755,.313)--(-6.776,.313);
\filldraw[fill opacity=0.8,fill=gray!20,draw=none](-6.779,.278)--(-6.227,.272)--(-6.219,.283)--(-6.26,.308)--(-6.755,.313)--cycle;
\draw(-6.779,.278)--(-6.227,.272);
\draw(-6.26,.308)--(-6.755,.313);
\filldraw[fill opacity=0.8,fill=gray!20,draw=none](-8.148,2.962)--(-8.152,2.943)--(-8.155,2.945)--cycle;
\draw(-8.148,2.962)--(-8.152,2.943)--(-8.155,2.945);
\filldraw[fill opacity=0.8,fill=gray!20,draw=none](-7.674,.467)--(-7.72,.458)--(-7.713,.419)--cycle;
\draw(-7.674,.467)--(-7.72,.458)--(-7.713,.419);
\filldraw[fill opacity=0.5,fill=gray!20](-10.2,2.423)--(-10.027,2.347)--(-9.671,2.547)--(-9.844,2.623)--cycle;
\filldraw[fill opacity=0.5,fill=gray!20](-10.292,2.587)--(-10.2,2.423)--(-9.844,2.623)--(-9.891,2.812)--cycle;
\filldraw[fill opacity=0.8,fill=gray!20,draw=none](-6.259,.558)--(-6.259,.512)--(-6.244,.51)--(-6.244,.546)--cycle;
\draw(-6.244,.51)--(-6.244,.546)--(-6.259,.558)--(-6.259,.512);
\filldraw[fill opacity=0.8,fill=gray!20,draw=none](-7.492,4.677)--(-7.513,4.684)--(-7.498,4.662)--(-7.492,4.675)--cycle;
\draw(-7.498,4.662)--(-7.492,4.675);
\filldraw[fill opacity=0.8,fill=gray!20,draw=none](-7.491,4.786)--(-7.481,4.786)--(-7.491,4.801)--(-7.511,4.797)--cycle;
\draw(-7.481,4.786)--(-7.491,4.801)--(-7.511,4.797);
\filldraw[fill opacity=0.8,fill=gray!20,draw=none](-7.48,4.807)--(-7.482,4.808)--(-7.489,4.791)--(-7.453,4.765)--(-7.452,4.768)--cycle;
\draw(-7.48,4.807)--(-7.482,4.808)--(-7.489,4.791);
\draw(-7.453,4.765)--(-7.452,4.768);
\filldraw[fill opacity=0.8,fill=gray!20,draw=none](-7.48,4.807)--(-7.452,4.768)--(-7.442,4.79)--cycle;
\draw(-7.452,4.768)--(-7.442,4.79)--(-7.48,4.807);
\filldraw[fill opacity=0.8,fill=gray!20](-7.74,1.286)--(-7.764,1.295)--(-7.748,1.291)--(-7.709,1.278)--cycle;
\filldraw[fill opacity=0.8,fill=gray!20](-7.79,.936)--(-7.751,.936)--(-7.769,.933)--(-7.79,.936)--cycle;
\filldraw[fill opacity=0.8,fill=gray!20](-7.79,.936)--(-7.743,.941)--(-7.751,.936)--(-7.79,.936)--cycle;
\filldraw[fill opacity=0.8,fill=gray!20](-7.79,.936)--(-7.769,.933)--(-7.793,.931)--(-7.79,.936)--cycle;
\filldraw[fill opacity=0.8,fill=gray!20](-7.79,.936)--(-7.793,.931)--(-7.816,.933)--(-7.79,.936)--cycle;
\filldraw[fill opacity=0.8,fill=gray!20](-7.79,.936)--(-7.816,.933)--(-7.833,.937)--(-7.79,.936)--cycle;
\filldraw[fill opacity=0.8,fill=gray!20,draw=none](-8.17,2.95)--(-8.169,2.95)--(-8.172,2.943)--cycle;
\draw(-8.17,2.95)--(-8.169,2.95);
\filldraw[fill opacity=0.8,fill=gray!20,draw=none](-8.17,2.95)--(-8.169,2.95)--(-8.172,2.943)--cycle;
\draw(-8.17,2.95)--(-8.169,2.95);
\filldraw[fill opacity=0.8,fill=gray!20,draw=none](-8.168,2.92)--(-8.161,2.926)--(-8.152,2.943)--(-8.148,2.962)--(-8.155,2.998)--cycle;
\draw(-8.161,2.926)--(-8.152,2.943)--(-8.148,2.962);
\filldraw[fill opacity=0.8,fill=gray!20,draw=none](-7.975,2.895)--(-7.965,2.865)--(-7.995,2.867)--(-7.994,2.873)--cycle;
\draw(-7.965,2.865)--(-7.995,2.867);
\filldraw[fill opacity=0.8,fill=gray!20,draw=none](-7.995,2.845)--(-7.995,2.867)--(-7.959,2.864)--cycle;
\draw(-7.995,2.867)--(-7.959,2.864);
\filldraw[fill opacity=0.8,fill=gray!20,draw=none](-7.975,2.895)--(-7.994,2.873)--(-7.984,2.923)--cycle;
\filldraw[fill opacity=0.8,fill=gray!20,draw=none](-7.971,2.863)--(-7.971,2.867)--(-7.969,2.866)--cycle;
\draw(-7.971,2.867)--(-7.969,2.866);
\filldraw[fill opacity=0.8,fill=gray!20,draw=none](-7.946,2.865)--(-7.971,2.867)--(-7.962,2.914)--cycle;
\draw(-7.946,2.865)--(-7.971,2.867);
\filldraw[fill opacity=0.8,fill=gray!20,draw=none](-7.971,2.819)--(-7.971,2.863)--(-7.969,2.866)--(-7.93,2.864)--(-7.94,2.817)--cycle;
\draw(-7.969,2.866)--(-7.93,2.864)--(-7.94,2.817)--(-7.971,2.819);
\filldraw[fill opacity=0.8,fill=gray!20](-7.94,2.817)--(-7.93,2.864)--(-7.869,2.849)--(-7.885,2.804)--cycle;
\filldraw[fill opacity=0.8,fill=gray!20,draw=none](-8.161,3)--(-7.909,2.89)--(-7.918,2.84)--(-8.17,2.95)--cycle;
\draw(-8.161,3)--(-7.909,2.89)--(-7.918,2.84)--(-8.17,2.95);
\filldraw[fill opacity=0.8,fill=gray!20,draw=none](-8.161,3)--(-8.151,2.996)--(-8.169,2.95)--(-8.17,2.95)--cycle;
\draw(-8.161,3)--(-8.151,2.996);
\draw(-8.169,2.95)--(-8.17,2.95);
\filldraw[fill opacity=0.8,fill=gray!20,draw=none](-6.244,.263)--(-6.227,.272)--(-6.238,.272)--cycle;
\draw(-6.227,.272)--(-6.238,.272);
\filldraw[fill opacity=0.8,fill=gray!20,draw=none](-7.492,4.677)--(-7.502,4.76)--(-7.527,4.704)--(-7.513,4.684)--cycle;
\draw(-7.502,4.76)--(-7.527,4.704);
\filldraw[fill opacity=0.8,fill=gray!20,draw=none](-5.985,.478)--(-5.987,.479)--(-5.996,.484)--cycle;
\draw(-5.985,.478)--(-5.987,.479);
\filldraw[fill opacity=0.8,fill=gray!20,draw=none](-7.811,.509)--(-7.808,.499)--(-7.798,.498)--(-7.796,.535)--(-7.802,.535)--cycle;
\draw(-7.808,.499)--(-7.798,.498)--(-7.796,.535)--(-7.802,.535);
\filldraw[fill opacity=0.8,fill=gray!20,draw=none](-7.04,1.084)--(-7.034,1.062)--(-7.048,1.063)--cycle;
\draw(-7.034,1.062)--(-7.048,1.063);
\filldraw[fill opacity=0.8,fill=gray!20,draw=none](-7.04,1.084)--(-7.05,1.114)--(-7.03,1.113)--cycle;
\draw(-7.05,1.114)--(-7.03,1.113);
\filldraw[fill opacity=0.8,fill=gray!20,draw=none](-7.032,1.062)--(-7.034,1.062)--(-7.04,1.084)--(-7.03,1.113)--cycle;
\draw(-7.032,1.062)--(-7.034,1.062);
\filldraw[fill opacity=0.8,fill=gray!20,draw=none](-7.035,1.055)--(-7.034,1.062)--(-7.032,1.062)--cycle;
\draw(-7.034,1.062)--(-7.032,1.062);
\filldraw[fill opacity=0.8,fill=gray!20](-7.067,1.049)--(-7.073,1.105)--(-6.988,1.121)--(-6.985,1.065)--cycle;
\filldraw[fill opacity=0.8,fill=gray!20](-8.073,3.968)--(-8.048,4.01)--(-8.085,4.019)--(-8.126,3.981)--cycle;
\filldraw[fill opacity=0.8,fill=gray!20,draw=none](-8.151,2.996)--(-7.909,2.89)--(-7.918,2.84)--(-8.169,2.95)--cycle;
\draw(-8.151,2.996)--(-7.909,2.89)--(-7.918,2.84)--(-8.169,2.95);
\filldraw[fill opacity=0.8,fill=gray!20,draw=none](-8.172,2.943)--(-8.177,2.94)--(-8.186,2.917)--cycle;
\draw(-8.172,2.943)--(-8.177,2.94)--(-8.186,2.917);
\filldraw[fill opacity=0.8,fill=gray!20,draw=none](-8.538,2.943)--(-8.471,2.926)--(-8.472,2.935)--cycle;
\draw(-8.538,2.943)--(-8.471,2.926)--(-8.472,2.935);
\filldraw[fill opacity=0.8,fill=gray!20,draw=none](-7.653,4.864)--(-7.639,4.852)--(-7.636,4.858)--cycle;
\draw(-7.639,4.852)--(-7.636,4.858);
\filldraw[fill opacity=0.8,fill=gray!20,draw=none](-7.625,4.849)--(-7.639,4.852)--(-7.691,4.733)--(-7.654,4.72)--(-7.639,4.726)--(-7.592,4.832)--cycle;
\draw(-7.639,4.852)--(-7.691,4.733);
\draw(-7.639,4.726)--(-7.592,4.832);
\filldraw[fill opacity=0.8,fill=gray!20,draw=none](-7.627,4.83)--(-7.626,4.84)--(-7.653,4.864)--(-7.681,4.834)--cycle;
\draw(-7.653,4.864)--(-7.681,4.834)--(-7.627,4.83)--(-7.626,4.84);
\filldraw[fill opacity=0.8,fill=gray!20,draw=none](-6.784,.482)--(-6.783,.497)--(-6.787,.497)--cycle;
\draw(-6.783,.497)--(-6.787,.497);
\filldraw[fill opacity=0.8,fill=gray!20,draw=none](-6.784,.482)--(-6.787,.497)--(-6.795,.478)--(-6.793,.475)--cycle;
\draw(-6.795,.478)--(-6.793,.475)--(-6.784,.482);
\filldraw[fill opacity=0.8,fill=gray!20,draw=none](-7.664,1.24)--(-7.665,1.24)--(-7.675,1.253)--(-7.673,1.251)--cycle;
\draw(-7.665,1.24)--(-7.675,1.253)--(-7.673,1.251);
\filldraw[fill opacity=0.8,fill=gray!20,draw=none](-7.918,.988)--(-7.911,.983)--(-7.913,.982)--cycle;
\draw(-7.911,.983)--(-7.913,.982);
\filldraw[fill opacity=0.8,fill=gray!20,draw=none](-7.904,.976)--(-8.049,.952)--(-7.911,.983)--cycle;
\draw(-8.049,.952)--(-7.911,.983);
\filldraw[fill opacity=0.8,fill=gray!20,draw=none](-7.889,.963)--(-7.906,.975)--(-7.92,.99)--(-7.917,.988)--cycle;
\draw(-7.889,.963)--(-7.906,.975)--(-7.92,.99)--(-7.917,.988);
\filldraw[fill opacity=0.8,fill=gray!20,draw=none](-6.227,.454)--(-6.244,.473)--(-6.244,.368)--(-6.209,.413)--(-6.209,.417)--cycle;
\draw(-6.244,.473)--(-6.244,.368);
\draw(-6.209,.413)--(-6.209,.417);
\filldraw[fill opacity=0.8,fill=gray!20,draw=none](-6.785,.46)--(-6.238,.454)--(-6.25,.491)--(-6.783,.497)--cycle;
\draw(-6.785,.46)--(-6.238,.454);
\draw(-6.25,.491)--(-6.783,.497);
\filldraw[fill opacity=0.8,fill=gray!20,draw=none](-8.216,2.886)--(-8.188,2.91)--(-8.177,2.94)--(-8.259,2.924)--(-8.269,2.876)--cycle;
\draw(-8.188,2.91)--(-8.177,2.94)--(-8.259,2.924)--(-8.269,2.876)--(-8.216,2.886);
\filldraw[fill opacity=0.8,fill=gray!20](-7.906,.975)--(-7.932,1.01)--(-7.949,1.027)--(-7.92,.99)--cycle;
\filldraw[fill opacity=0.8,fill=gray!20,draw=none](-6.071,.528)--(-6.083,.534)--(-6.091,.522)--cycle;
\draw(-6.083,.534)--(-6.091,.522);
\filldraw[fill opacity=0.8,fill=gray!20](-7.741,4.629)--(-7.737,4.685)--(-7.81,4.703)--(-7.817,4.647)--cycle;
\filldraw[fill opacity=0.8,fill=gray!20,draw=none](-7.587,4.854)--(-7.595,4.864)--(-7.61,4.864)--cycle;
\draw(-7.587,4.854)--(-7.595,4.864)--(-7.61,4.864);
\filldraw[fill opacity=0.8,fill=gray!20,draw=none](-8.453,3.196)--(-8.42,3.192)--(-8.422,3.197)--(-8.426,3.202)--cycle;
\draw(-8.42,3.192)--(-8.422,3.197)--(-8.426,3.202);
\filldraw[fill opacity=0.8,fill=gray!20,draw=none](-8.342,3.149)--(-8.382,3.159)--(-8.393,3.133)--(-8.375,3.134)--cycle;
\draw(-8.382,3.159)--(-8.393,3.133);
\filldraw[fill opacity=0.8,fill=gray!20,draw=none](-8.43,3.143)--(-8.396,3.141)--(-8.393,3.15)--(-8.405,3.185)--(-8.415,3.186)--(-8.431,3.158)--cycle;
\draw(-8.43,3.143)--(-8.396,3.141);
\draw(-8.405,3.185)--(-8.415,3.186)--(-8.431,3.158);
\filldraw[fill opacity=0.8,fill=gray!20,draw=none](-8.375,3.134)--(-8.393,3.133)--(-8.398,3.124)--cycle;
\draw(-8.393,3.133)--(-8.398,3.124);
\filldraw[fill opacity=0.8,fill=gray!20,draw=none](-8.342,3.149)--(-8.375,3.134)--(-8.342,3.135)--(-8.337,3.148)--cycle;
\draw(-8.342,3.135)--(-8.337,3.148);
\filldraw[fill opacity=0.8,fill=gray!20,draw=none](-8.437,3.1)--(-8.396,3.105)--(-8.396,3.141)--(-8.43,3.143)--cycle;
\draw(-8.396,3.141)--(-8.43,3.143);
\filldraw[fill opacity=0.8,fill=gray!20,draw=none](-8.416,3.13)--(-8.413,3.159)--(-8.414,3.162)--(-8.453,3.196)--cycle;
\draw(-8.416,3.13)--(-8.413,3.159)--(-8.414,3.162);
\filldraw[fill opacity=0.8,fill=gray!20,draw=none](-8.414,3.162)--(-8.42,3.192)--(-8.453,3.196)--cycle;
\draw(-8.414,3.162)--(-8.42,3.192);
\filldraw[fill opacity=0.8,fill=gray!20,draw=none](-9.098,1.077)--(-9.106,1.091)--(-9.173,1.108)--(-9.16,1.065)--cycle;
\draw(-9.173,1.108)--(-9.16,1.065)--(-9.098,1.077)--(-9.106,1.091);
\filldraw[fill opacity=0.8,fill=gray!20,draw=none](-9.098,1.077)--(-9.106,1.091)--(-9.173,1.108)--(-9.16,1.065)--cycle;
\draw(-9.173,1.108)--(-9.16,1.065)--(-9.098,1.077)--(-9.106,1.091);
\filldraw[fill opacity=0.8,fill=gray!20,draw=none](-9.112,1.096)--(-9.128,1.106)--(-9.125,1.105)--cycle;
\draw(-9.128,1.106)--(-9.125,1.105);
\filldraw[fill opacity=0.8,fill=gray!20,draw=none](-9.104,1.095)--(-9.128,1.106)--(-9.173,1.111)--(-9.149,1.1)--cycle;
\draw(-9.104,1.095)--(-9.128,1.106);
\draw(-9.173,1.111)--(-9.149,1.1);
\filldraw[fill opacity=0.8,fill=gray!20,draw=none](-9.155,1.098)--(-9.03,.982)--(-9.04,1.033)--(-9.065,1.071)--(-9.124,1.108)--cycle;
\filldraw[fill opacity=0.8,fill=gray!20,draw=none](-8.593,1.068)--(-7.954,1.212)--(-7.948,1.198)--(-8.665,1.037)--cycle;
\draw(-8.593,1.068)--(-7.954,1.212);
\draw(-7.948,1.198)--(-8.665,1.037);
\filldraw[fill opacity=0.8,fill=gray!20,draw=none](-7.948,1.198)--(-7.955,1.171)--(-7.967,1.163)--(-7.95,1.202)--cycle;
\draw(-7.955,1.171)--(-7.967,1.163)--(-7.95,1.202);
\filldraw[fill opacity=0.8,fill=gray!20,draw=none](-7.922,.988)--(-7.924,.995)--(-7.913,.982)--(-7.917,.981)--cycle;
\draw(-7.913,.982)--(-7.917,.981);
\filldraw[fill opacity=0.8,fill=gray!20,draw=none](-7.973,3.839)--(-8.041,3.686)--(-7.999,3.659)--(-7.928,3.819)--cycle;
\draw(-7.999,3.659)--(-7.928,3.819)--(-7.973,3.839)--(-8.041,3.686);
\filldraw[fill opacity=0.8,fill=gray!20,draw=none](-8.02,3.86)--(-8.091,3.7)--(-8.041,3.686)--(-7.973,3.839)--cycle;
\draw(-8.041,3.686)--(-7.973,3.839)--(-8.02,3.86)--(-8.091,3.7);
\filldraw[fill opacity=0.8,fill=gray!20](-7.997,3.659)--(-7.999,3.696)--(-8.092,3.703)--(-8.073,3.665)--cycle;
\filldraw[fill opacity=0.8,fill=gray!20,draw=none](-8.02,3.86)--(-7.949,4.018)--(-7.99,4.038)--(-8.062,3.878)--cycle;
\draw(-7.99,4.038)--(-8.062,3.878)--(-8.02,3.86)--(-7.949,4.018);
\filldraw[fill opacity=0.8,fill=gray!20](-7.994,4.006)--(-7.991,4.039)--(-8.019,4.041)--(-8.048,4.01)--cycle;
\filldraw[fill opacity=0.8,fill=gray!20,draw=none](-9.077,1.089)--(-9.08,1.093)--(-9.079,1.092)--cycle;
\draw(-9.08,1.093)--(-9.079,1.092);
\filldraw[fill opacity=0.8,fill=gray!20,draw=none](-8.186,2.917)--(-8.192,2.904)--(-8.194,2.905)--cycle;
\draw(-8.192,2.904)--(-8.194,2.905);
\filldraw[fill opacity=0.8,fill=gray!20,draw=none](-8.186,2.917)--(-8.192,2.904)--(-8.194,2.905)--cycle;
\draw(-8.192,2.904)--(-8.194,2.905);
\filldraw[fill opacity=0.8,fill=gray!20,draw=none](-8.512,2.99)--(-8.551,2.999)--(-8.548,2.976)--cycle;
\draw(-8.512,2.99)--(-8.551,2.999)--(-8.548,2.976);
\filldraw[fill opacity=0.8,fill=gray!20,draw=none](-8.512,2.99)--(-8.475,2.98)--(-8.473,3.018)--cycle;
\draw(-8.512,2.99)--(-8.475,2.98)--(-8.473,3.018);
\filldraw[fill opacity=0.8,fill=gray!20,draw=none](-8.438,2.996)--(-8.423,2.989)--(-8.411,3.034)--cycle;
\draw(-8.438,2.996)--(-8.423,2.989);
\filldraw[fill opacity=0.8,fill=gray!20,draw=none](-8.464,2.98)--(-8.438,2.996)--(-8.43,3.015)--(-8.436,3.035)--(-8.461,3.037)--(-8.473,3.018)--(-8.475,2.98)--cycle;
\draw(-8.436,3.035)--(-8.461,3.037);
\draw(-8.473,3.018)--(-8.475,2.98)--(-8.464,2.98);
\filldraw[fill opacity=0.8,fill=gray!20,draw=none](-8.461,3.009)--(-8.548,2.976)--(-8.537,2.95)--(-8.53,2.949)--(-8.496,2.969)--(-8.463,3.007)--cycle;
\draw(-8.537,2.95)--(-8.53,2.949)--(-8.496,2.969)--(-8.463,3.007)--(-8.461,3.009);
\filldraw[fill opacity=0.8,fill=gray!20,draw=none](-8.161,3)--(-8.17,3.049)--(-8.172,3.054)--(-8.177,3.05)--(-8.171,2.994)--cycle;
\draw(-8.172,3.054)--(-8.177,3.05)--(-8.171,2.994)--(-8.161,3);
\filldraw[fill opacity=0.8,fill=gray!20,draw=none](-5.979,.475)--(-5.985,.478)--(-5.996,.484)--(-6.027,.504)--(-6.008,.496)--cycle;
\draw(-5.979,.475)--(-5.985,.478);
\draw(-6.027,.504)--(-6.008,.496);
\filldraw[fill opacity=0.8,fill=gray!20,draw=none](-7.749,.436)--(-7.758,.422)--(-7.748,.422)--cycle;
\draw(-7.758,.422)--(-7.748,.422);
\filldraw[fill opacity=0.8,fill=gray!20,draw=none](-6.071,.528)--(-6.091,.522)--(-6.126,.471)--(-6.099,.458)--(-6.055,.518)--cycle;
\draw(-6.091,.522)--(-6.126,.471)--(-6.099,.458);
\filldraw[fill opacity=0.8,fill=gray!20,draw=none](-6.114,.52)--(-6.116,.503)--(-6.088,.514)--(-6.103,.521)--cycle;
\draw(-6.088,.514)--(-6.103,.521);
\filldraw[fill opacity=0.8,fill=gray!20,draw=none](-7.773,.984)--(-7.787,.986)--(-7.741,.984)--cycle;
\draw(-7.787,.986)--(-7.741,.984);
\filldraw[fill opacity=0.8,fill=gray!20,draw=none](-7.865,.949)--(-7.866,.963)--(-7.889,.963)--(-7.872,.951)--cycle;
\draw(-7.889,.963)--(-7.872,.951)--(-7.865,.949);
\filldraw[fill opacity=0.8,fill=gray!20,draw=none](-8.309,.868)--(-7.839,.974)--(-7.777,.988)--(-7.756,.988)--(-8.175,.894)--cycle;
\draw(-8.309,.868)--(-7.839,.974);
\draw(-7.777,.988)--(-7.756,.988)--(-8.175,.894);
\filldraw[fill opacity=0.8,fill=gray!20,draw=none](-7.571,4.833)--(-7.587,4.854)--(-7.61,4.864)--(-7.624,4.863)--(-7.627,4.83)--cycle;
\draw(-7.61,4.864)--(-7.624,4.863)--(-7.627,4.83)--(-7.571,4.833)--(-7.587,4.854);
\filldraw[fill opacity=0.8,fill=gray!20](-8.851,1.409)--(-8.867,1.461)--(-8.945,1.457)--(-8.947,1.405)--cycle;
\filldraw[fill opacity=0.8,fill=gray!20,draw=none](-6.179,.501)--(-6.184,.491)--(-6.174,.491)--cycle;
\draw(-6.184,.491)--(-6.174,.491);
\filldraw[fill opacity=0.8,fill=gray!20,draw=none](-8.237,3.112)--(-8.232,3.111)--(-8.194,3.087)--(-8.214,3.095)--cycle;
\draw(-8.237,3.112)--(-8.232,3.111);
\draw(-8.194,3.087)--(-8.214,3.095);
\filldraw[fill opacity=0.8,fill=gray!20,draw=none](-7.831,.41)--(-7.827,.414)--(-7.818,.456)--(-7.878,.46)--(-7.888,.414)--cycle;
\draw(-7.818,.456)--(-7.878,.46)--(-7.888,.414)--(-7.831,.41);
\filldraw[fill opacity=0.8,fill=gray!20,draw=none](-8.725,.767)--(-8.696,.796)--(-8.713,.755)--cycle;
\draw(-8.696,.796)--(-8.713,.755)--(-8.725,.767);
\filldraw[fill opacity=0.8,fill=gray!20,draw=none](-8.701,.791)--(-8.692,.806)--(-8.692,.806)--(-8.696,.796)--cycle;
\draw(-8.692,.806)--(-8.692,.806)--(-8.696,.796);
\filldraw[fill opacity=0.8,fill=gray!20,draw=none](-8.711,.783)--(-8.726,.746)--(-8.713,.755)--(-8.692,.806)--(-8.701,.8)--cycle;
\draw(-8.726,.746)--(-8.713,.755)--(-8.692,.806)--(-8.701,.8);
\filldraw[fill opacity=0.8,fill=gray!20,draw=none](-8.808,.823)--(-8.788,.87)--(-8.844,.839)--cycle;
\draw(-8.844,.839)--(-8.808,.823)--(-8.788,.87);
\filldraw[fill opacity=0.8,fill=gray!20,draw=none](-8.774,.808)--(-8.695,.826)--(-8.694,.828)--(-8.844,.839)--cycle;
\draw(-8.774,.808)--(-8.695,.826);
\filldraw[fill opacity=0.8,fill=gray!20,draw=none](-8.844,.792)--(-8.774,.808)--(-8.844,.839)--cycle;
\draw(-8.844,.839)--(-8.844,.792)--(-8.774,.808);
\filldraw[fill opacity=0.8,fill=gray!20,draw=none](-8.868,.794)--(-7.792,.345)--(-7.757,.385)--(-8.844,.839)--cycle;
\draw(-7.757,.385)--(-8.844,.839)--(-8.868,.794)--(-7.792,.345);
\filldraw[fill opacity=0.8,fill=gray!20,draw=none](-8.177,2.94)--(-8.182,2.909)--(-8.168,2.92)--(-8.158,2.978)--cycle;
\filldraw[fill opacity=0.8,fill=gray!20](-8.108,3.805)--(-8.104,3.861)--(-8.177,3.879)--(-8.184,3.823)--cycle;
\filldraw[fill opacity=0.8,fill=gray!20](-7.938,4.009)--(-7.962,4.04)--(-7.991,4.039)--(-7.994,4.006)--cycle;
\filldraw[fill opacity=0.8,fill=gray!20,draw=none](-5.996,.484)--(-6.036,.508)--(-6.027,.504)--cycle;
\draw(-6.036,.508)--(-6.027,.504);
\filldraw[fill opacity=0.8,fill=gray!20,draw=none](-9.022,1.243)--(-8.949,1.238)--(-8.949,1.291)--(-9.031,1.297)--cycle;
\draw(-9.022,1.243)--(-8.949,1.238)--(-8.949,1.291)--(-9.031,1.297);
\filldraw[fill opacity=0.8,fill=gray!20,draw=none](-8.357,3.214)--(-8.374,3.178)--(-8.335,3.152)--(-8.316,3.194)--cycle;
\draw(-8.357,3.214)--(-8.374,3.178);
\draw(-8.335,3.152)--(-8.316,3.194);
\filldraw[fill opacity=0.8,fill=gray!20,draw=none](-8.407,3.156)--(-8.399,3.153)--(-8.353,3.152)--(-8.369,3.174)--(-8.404,3.19)--cycle;
\draw(-8.407,3.156)--(-8.399,3.153);
\draw(-8.369,3.174)--(-8.404,3.19);
\filldraw[fill opacity=0.8,fill=gray!20,draw=none](-8.398,3.185)--(-8.361,3.182)--(-8.358,3.215)--(-8.386,3.217)--(-8.408,3.193)--cycle;
\draw(-8.398,3.185)--(-8.361,3.182)--(-8.358,3.215)--(-8.386,3.217)--(-8.408,3.193);
\filldraw[fill opacity=0.8,fill=gray!20,draw=none](-6.192,.265)--(-6.192,.272)--(-6.201,.272)--cycle;
\draw(-6.192,.272)--(-6.201,.272);
\filldraw[fill opacity=0.8,fill=gray!20](-7.037,.514)--(-7.012,.557)--(-7.05,.566)--(-7.091,.527)--cycle;
\filldraw[fill opacity=0.8,fill=gray!20,draw=none](-7.508,4.819)--(-7.529,4.841)--(-7.549,4.837)--cycle;
\draw(-7.508,4.819)--(-7.529,4.841)--(-7.549,4.837);
\filldraw[fill opacity=0.8,fill=gray!20,draw=none](-7.787,.986)--(-7.839,.974)--(-7.787,.986)--cycle;
\draw(-7.839,.974)--(-7.787,.986);
\filldraw[fill opacity=0.8,fill=gray!20,draw=none](-7.8,.291)--(-7.801,.315)--(-7.888,.322)--(-7.886,.315)--cycle;
\draw(-7.8,.291)--(-7.801,.315)--(-7.888,.322)--(-7.886,.315);
\filldraw[fill opacity=0.8,fill=gray!20,draw=none](-6.002,.388)--(-6.002,.356)--(-5.974,.394)--(-5.974,.439)--cycle;
\draw(-6.002,.388)--(-6.002,.356);
\draw(-5.974,.394)--(-5.974,.439);
\filldraw[fill opacity=0.8,fill=gray!20,draw=none](-6.103,.521)--(-6.094,.518)--(-6.091,.522)--cycle;
\draw(-6.094,.518)--(-6.091,.522);
\filldraw[fill opacity=0.8,fill=gray!20,draw=none](-6.094,.518)--(-6.103,.521)--(-6.1,.519)--cycle;
\draw(-6.103,.521)--(-6.1,.519);
\filldraw[fill opacity=0.8,fill=gray!20,draw=none](-7.663,1.239)--(-7.664,1.24)--(-7.673,1.251)--(-7.666,1.244)--cycle;
\draw(-7.673,1.251)--(-7.666,1.244);
\filldraw[fill opacity=0.8,fill=gray!20,draw=none](-7.471,4.582)--(-7.443,4.588)--(-7.437,4.642)--(-7.458,4.638)--cycle;
\draw(-7.471,4.582)--(-7.443,4.588)--(-7.437,4.642)--(-7.458,4.638);
\filldraw[fill opacity=0.8,fill=gray!20](-7.725,4.741)--(-7.706,4.792)--(-7.759,4.805)--(-7.79,4.757)--cycle;
\filldraw[fill opacity=0.8,fill=gray!20,draw=none](-7.701,.957)--(-7.699,.958)--(-7.668,.981)--(-7.669,.981)--(-7.685,.97)--cycle;
\draw(-7.701,.957)--(-7.699,.958)--(-7.668,.981);
\filldraw[fill opacity=0.8,fill=gray!20](-7.932,1.01)--(-7.948,1.051)--(-7.967,1.07)--(-7.949,1.027)--cycle;
\filldraw[fill opacity=0.8,fill=gray!20,draw=none](-8.692,.806)--(-8.691,.808)--(-8.692,.806)--cycle;
\draw(-8.691,.808)--(-8.692,.806)--(-8.692,.806);
\filldraw[fill opacity=0.8,fill=gray!20,draw=none](-8.694,.828)--(-8.701,.8)--(-8.696,.803)--cycle;
\draw(-8.701,.8)--(-8.696,.803);
\filldraw[fill opacity=0.8,fill=gray!20,draw=none](-8.695,.826)--(-8.687,.827)--(-8.694,.828)--cycle;
\draw(-8.695,.826)--(-8.687,.827);
\filldraw[fill opacity=0.8,fill=gray!20,draw=none](-8.844,.839)--(-7.757,.385)--(-7.758,.44)--(-8.829,.887)--cycle;
\draw(-7.758,.44)--(-8.829,.887)--(-8.844,.839)--(-7.757,.385);
\filldraw[fill opacity=0.8,fill=gray!20,draw=none](-7.771,.984)--(-7.799,.985)--(-7.79,.986)--(-7.787,.986)--cycle;
\draw(-7.771,.984)--(-7.799,.985)--(-7.79,.986)--(-7.787,.986);
\filldraw[fill opacity=0.8,fill=gray!20](-8.841,1.242)--(-8.838,1.296)--(-8.949,1.291)--(-8.949,1.238)--cycle;
\filldraw[fill opacity=0.8,fill=gray!20,draw=none](-7.754,.5)--(-7.796,.526)--(-7.798,.498)--cycle;
\draw(-7.796,.526)--(-7.798,.498)--(-7.754,.5);
\filldraw[fill opacity=0.8,fill=gray!20](-7.675,1.253)--(-7.709,1.278)--(-7.699,1.268)--(-7.661,1.239)--cycle;
\filldraw[fill opacity=0.8,fill=gray!20,draw=none](-7.922,.988)--(-7.917,.981)--(-7.921,.981)--cycle;
\draw(-7.917,.981)--(-7.921,.981);
\filldraw[fill opacity=0.8,fill=gray!20,draw=none](-8.189,2.903)--(-8.182,2.909)--(-8.181,2.918)--cycle;
\filldraw[fill opacity=0.8,fill=gray!20,draw=none](-8.021,2.829)--(-8.015,2.842)--(-8.015,2.834)--cycle;
\draw(-8.015,2.842)--(-8.015,2.834);
\filldraw[fill opacity=0.8,fill=gray!20,draw=none](-7.995,2.845)--(-8.015,2.834)--(-8.015,2.842)--(-7.999,2.867)--(-7.995,2.867)--cycle;
\draw(-8.015,2.834)--(-8.015,2.842);
\draw(-7.999,2.867)--(-7.995,2.867);
\filldraw[fill opacity=0.8,fill=gray!20,draw=none](-7.994,2.873)--(-7.995,2.867)--(-7.999,2.867)--cycle;
\draw(-7.995,2.867)--(-7.999,2.867);
\filldraw[fill opacity=0.8,fill=gray!20,draw=none](-7.995,2.845)--(-7.996,2.81)--(-8.016,2.812)--(-8.015,2.834)--cycle;
\draw(-7.996,2.81)--(-8.016,2.812)--(-8.015,2.834);
\filldraw[fill opacity=0.8,fill=gray!20,draw=none](-8.029,2.811)--(-8.021,2.829)--(-8.015,2.834)--(-8.016,2.812)--cycle;
\draw(-8.015,2.834)--(-8.016,2.812)--(-8.029,2.811);
\filldraw[fill opacity=0.8,fill=gray!20,draw=none](-8.018,2.76)--(-8.016,2.808)--(-7.995,2.81)--(-7.922,2.805)--(-7.941,2.754)--cycle;
\draw(-7.995,2.81)--(-7.922,2.805)--(-7.941,2.754)--(-8.018,2.76)--(-8.016,2.808);
\filldraw[fill opacity=0.8,fill=gray!20,draw=none](-7.998,2.82)--(-7.997,2.821)--(-7.995,2.821)--cycle;
\draw(-7.997,2.821)--(-7.995,2.821);
\filldraw[fill opacity=0.8,fill=gray!20,draw=none](-7.971,2.863)--(-7.971,2.819)--(-7.997,2.821)--cycle;
\draw(-7.971,2.819)--(-7.997,2.821);
\filldraw[fill opacity=0.8,fill=gray!20,draw=none](-8.016,2.808)--(-8.016,2.812)--(-7.995,2.81)--cycle;
\draw(-8.016,2.808)--(-8.016,2.812)--(-7.995,2.81);
\filldraw[fill opacity=0.8,fill=gray!20,draw=none](-8.014,2.779)--(-7.998,2.82)--(-7.995,2.821)--(-7.94,2.817)--(-7.956,2.775)--cycle;
\draw(-7.995,2.821)--(-7.94,2.817)--(-7.956,2.775)--(-8.014,2.779);
\filldraw[fill opacity=0.8,fill=gray!20](-7.941,2.754)--(-7.922,2.805)--(-7.857,2.789)--(-7.888,2.741)--cycle;
\filldraw[fill opacity=0.8,fill=gray!20](-7.956,2.775)--(-7.94,2.817)--(-7.885,2.804)--(-7.911,2.764)--cycle;
\filldraw[fill opacity=0.8,fill=gray!20,draw=none](-7.98,2.782)--(-7.979,2.784)--(-7.959,2.802)--(-7.933,2.804)--(-7.946,2.775)--cycle;
\draw(-7.98,2.782)--(-7.979,2.784);
\draw(-7.933,2.804)--(-7.946,2.775);
\filldraw[fill opacity=0.8,fill=gray!20,draw=none](-7.959,2.802)--(-7.926,2.834)--(-7.922,2.832)--(-7.933,2.804)--cycle;
\draw(-7.926,2.834)--(-7.922,2.832)--(-7.933,2.804);
\filldraw[fill opacity=0.8,fill=gray!20,draw=none](-7.923,2.832)--(-7.922,2.832)--(-7.926,2.834)--cycle;
\draw(-7.922,2.832)--(-7.926,2.834);
\filldraw[fill opacity=0.8,fill=gray!20,draw=none](-8.164,2.947)--(-7.918,2.84)--(-7.944,2.796)--(-8.189,2.903)--cycle;
\draw(-8.164,2.947)--(-7.918,2.84)--(-7.944,2.796)--(-8.189,2.903);
\filldraw[fill opacity=0.8,fill=gray!20,draw=none](-8.164,2.947)--(-8.156,2.944)--(-8.189,2.903)--(-8.189,2.903)--cycle;
\draw(-8.164,2.947)--(-8.156,2.944);
\draw(-8.189,2.903)--(-8.189,2.903);
\filldraw[fill opacity=0.8,fill=gray!20,draw=none](-8.512,2.99)--(-8.473,3.018)--(-8.471,3.037)--(-8.544,3.055)--(-8.551,2.999)--cycle;
\draw(-8.473,3.018)--(-8.471,3.037)--(-8.544,3.055)--(-8.551,2.999)--(-8.512,2.99);
\filldraw[fill opacity=0.8,fill=gray!20](-8.305,3.185)--(-8.329,3.216)--(-8.358,3.215)--(-8.361,3.182)--cycle;
\filldraw[fill opacity=0.8,fill=gray!20,draw=none](-6.155,.5)--(-6.151,.49)--(-6.126,.49)--cycle;
\draw(-6.151,.49)--(-6.126,.49);
\filldraw[fill opacity=0.8,fill=gray!20](-6.961,.206)--(-6.963,.243)--(-7.056,.249)--(-7.037,.211)--cycle;
\filldraw[fill opacity=0.8,fill=gray!20](-6.958,.553)--(-6.955,.586)--(-6.983,.588)--(-7.012,.557)--cycle;
\filldraw[fill opacity=0.8,fill=gray!20,draw=none](-6.782,.291)--(-6.776,.313)--(-6.786,.314)--cycle;
\draw(-6.776,.313)--(-6.786,.314);
\filldraw[fill opacity=0.8,fill=gray!20,draw=none](-8.274,2.846)--(-8.216,2.886)--(-8.269,2.876)--(-8.282,2.846)--cycle;
\draw(-8.216,2.886)--(-8.269,2.876)--(-8.282,2.846);
\filldraw[fill opacity=0.8,fill=gray!20](-7.858,3.977)--(-7.896,4.017)--(-7.938,4.009)--(-7.918,3.966)--cycle;
\filldraw[fill opacity=0.8,fill=gray!20,draw=none](-7.79,4.356)--(-7.75,4.34)--(-7.723,4.4)--(-7.731,4.475)--(-7.741,4.486)--(-7.768,4.426)--cycle;
\draw(-7.75,4.34)--(-7.723,4.4);
\draw(-7.741,4.486)--(-7.768,4.426);
\filldraw[fill opacity=0.8,fill=gray!20,draw=none](-7.784,4.353)--(-7.757,4.422)--(-7.804,4.441)--(-7.831,4.372)--cycle;
\draw(-7.804,4.441)--(-7.831,4.372)--(-7.784,4.353)--(-7.757,4.422);
\filldraw[fill opacity=0.8,fill=gray!20,draw=none](-7.949,4.019)--(-7.9,4.127)--(-7.79,4.489)--(-7.988,4.045)--cycle;
\draw(-7.949,4.019)--(-7.9,4.127);
\draw(-7.79,4.489)--(-7.988,4.045);
\filldraw[fill opacity=0.8,fill=gray!20,draw=none](-7.726,4.669)--(-7.731,4.663)--(-7.73,4.649)--cycle;
\draw(-7.731,4.663)--(-7.73,4.649);
\filldraw[fill opacity=0.8,fill=gray!20,draw=none](-7.801,.315)--(-7.801,.349)--(-7.884,.337)--(-7.889,.333)--(-7.888,.322)--cycle;
\draw(-7.889,.333)--(-7.888,.322)--(-7.801,.315)--(-7.801,.349);
\filldraw[fill opacity=0.8,fill=gray!20](-7.737,4.685)--(-7.725,4.741)--(-7.79,4.757)--(-7.81,4.703)--cycle;
\filldraw[fill opacity=0.8,fill=gray!20,draw=none](-7.126,.317)--(-7.129,.312)--(-7.126,.311)--cycle;
\draw(-7.129,.312)--(-7.126,.311);
\filldraw[fill opacity=0.8,fill=gray!20,draw=none](-8.266,2.852)--(-8.255,2.848)--(-8.272,2.842)--(-8.282,2.846)--cycle;
\draw(-8.255,2.848)--(-8.272,2.842)--(-8.282,2.846);
\filldraw[fill opacity=0.8,fill=gray!20,draw=none](-8.272,2.856)--(-8.248,2.854)--(-8.227,2.871)--cycle;
\filldraw[fill opacity=0.8,fill=gray!20,draw=none](-8.272,2.856)--(-8.282,2.853)--(-8.293,2.844)--(-8.272,2.842)--(-8.255,2.848)--(-8.248,2.854)--cycle;
\draw(-8.293,2.844)--(-8.272,2.842)--(-8.255,2.848);
\filldraw[fill opacity=0.8,fill=gray!20,draw=none](-8.075,2.778)--(-8.057,2.796)--(-8.016,2.808)--(-8.017,2.768)--cycle;
\draw(-8.016,2.808)--(-8.017,2.768);
\filldraw[fill opacity=0.8,fill=gray!20,draw=none](-8.075,2.778)--(-8.104,2.783)--(-8.057,2.796)--cycle;
\filldraw[fill opacity=0.8,fill=gray!20,draw=none](-8.096,2.756)--(-8.104,2.783)--(-8.017,2.768)--(-8.018,2.76)--cycle;
\draw(-8.017,2.768)--(-8.018,2.76)--(-8.096,2.756)--(-8.104,2.783);
\filldraw[fill opacity=0.8,fill=gray!20](-8.02,2.716)--(-8.018,2.76)--(-7.941,2.754)--(-7.966,2.712)--cycle;
\filldraw[fill opacity=0.8,fill=gray!20,draw=none](-8.023,2.716)--(-8.096,2.756)--(-8.018,2.76)--(-8.02,2.716)--cycle;
\draw(-8.096,2.756)--(-8.018,2.76)--(-8.02,2.716)--(-8.023,2.716);
\filldraw[fill opacity=0.8,fill=gray!20,draw=none](-8.045,2.778)--(-8.036,2.788)--(-8.019,2.793)--(-8.019,2.78)--cycle;
\draw(-8.019,2.793)--(-8.019,2.78)--(-8.045,2.778);
\filldraw[fill opacity=0.8,fill=gray!20,draw=none](-8.045,2.778)--(-8.073,2.777)--(-8.036,2.788)--cycle;
\draw(-8.045,2.778)--(-8.073,2.777);
\filldraw[fill opacity=0.8,fill=gray!20,draw=none](-8.083,2.774)--(-8.073,2.777)--(-8.019,2.78)--(-8.02,2.763)--cycle;
\draw(-8.073,2.777)--(-8.019,2.78)--(-8.02,2.763);
\filldraw[fill opacity=0.8,fill=gray!20](-8.021,2.743)--(-8.019,2.78)--(-7.956,2.775)--(-7.977,2.74)--cycle;
\filldraw[fill opacity=0.8,fill=gray!20,draw=none](-8.043,2.751)--(-8.062,2.77)--(-8.02,2.763)--(-8.021,2.745)--cycle;
\draw(-8.02,2.763)--(-8.021,2.745);
\filldraw[fill opacity=0.8,fill=gray!20,draw=none](-8.043,2.751)--(-8.078,2.761)--(-8.083,2.774)--(-8.062,2.77)--cycle;
\draw(-8.078,2.761)--(-8.083,2.774);
\filldraw[fill opacity=0.5,fill=gray!20](-8.261,2.383)--(-8.169,2.567)--(-7.825,2.317)--(-7.956,2.16)--cycle;
\filldraw[fill opacity=0.8,fill=gray!20,draw=none](-8.124,2.773)--(-8.104,2.783)--(-8.096,2.756)--cycle;
\draw(-8.104,2.783)--(-8.096,2.756);
\filldraw[fill opacity=0.8,fill=gray!20,draw=none](-8.089,2.77)--(-8.083,2.774)--(-8.08,2.767)--cycle;
\draw(-8.083,2.774)--(-8.08,2.767);
\filldraw[fill opacity=0.8,fill=gray!20,draw=none](-8.107,2.765)--(-8.091,2.77)--(-8.096,2.759)--cycle;
\draw(-8.091,2.77)--(-8.096,2.759);
\filldraw[fill opacity=0.8,fill=gray!20,draw=none](-8.096,2.759)--(-8.091,2.77)--(-8.09,2.77)--(-8.048,2.74)--(-8.051,2.733)--cycle;
\draw(-8.096,2.759)--(-8.091,2.77);
\draw(-8.048,2.74)--(-8.051,2.733);
\filldraw[fill opacity=0.8,fill=gray!20,draw=none](-8.101,2.763)--(-8.089,2.77)--(-8.08,2.767)--(-8.07,2.745)--cycle;
\draw(-8.08,2.767)--(-8.07,2.745);
\filldraw[fill opacity=0.8,fill=gray!20,draw=none](-8.09,2.77)--(-8.079,2.772)--(-8.037,2.764)--(-8.048,2.74)--cycle;
\draw(-8.037,2.764)--(-8.048,2.74);
\filldraw[fill opacity=0.8,fill=gray!20,draw=none](-8.079,2.772)--(-8.03,2.78)--(-8.037,2.764)--cycle;
\draw(-8.03,2.78)--(-8.037,2.764);
\filldraw[fill opacity=0.8,fill=gray!20,draw=none](-8.023,2.716)--(-8.076,2.713)--(-8.096,2.756)--cycle;
\draw(-8.023,2.716)--(-8.076,2.713)--(-8.096,2.756);
\filldraw[fill opacity=0.8,fill=gray!20,draw=none](-8.07,2.706)--(-8.076,2.713)--(-8.02,2.716)--(-8.022,2.699)--cycle;
\draw(-8.07,2.706)--(-8.076,2.713)--(-8.02,2.716)--(-8.022,2.699);
\filldraw[fill opacity=0.5,fill=gray!20](-8.57,2.704)--(-8.599,2.811)--(-8.18,2.667)--(-8.169,2.567)--cycle;
\filldraw[fill opacity=0.5,fill=gray!20,draw=none](-8.354,2.414)--(-8.27,2.385)--(-8.264,2.384)--(-8.336,2.415)--cycle;
\draw(-8.354,2.414)--(-8.27,2.385);
\draw(-8.264,2.384)--(-8.336,2.415);
\filldraw[fill opacity=0.5,fill=gray!20](-8.618,2.505)--(-8.57,2.704)--(-8.169,2.567)--(-8.261,2.383)--cycle;
\filldraw[fill opacity=0.8,fill=gray!20,draw=none](-8.034,2.742)--(-8.043,2.751)--(-8.021,2.745)--(-8.021,2.743)--cycle;
\draw(-8.021,2.745)--(-8.021,2.743)--(-8.034,2.742);
\filldraw[fill opacity=0.5,fill=gray!20](-8.599,2.811)--(-8.641,2.899)--(-8.207,2.75)--(-8.18,2.667)--cycle;
\filldraw[fill opacity=0.5,fill=gray!20,draw=none](-8.27,2.385)--(-8.261,2.383)--(-8.264,2.384)--cycle;
\draw(-8.27,2.385)--(-8.261,2.383)--(-8.264,2.384);
\filldraw[fill opacity=0.8,fill=gray!20,draw=none](-8.153,2.741)--(-8.157,2.745)--(-8.096,2.756)--(-8.079,2.719)--cycle;
\draw(-8.153,2.741)--(-8.157,2.745)--(-8.096,2.756)--(-8.079,2.719);
\filldraw[fill opacity=0.8,fill=gray!20,draw=none](-8.307,2.395)--(-8.19,2.665)--(-8.118,2.719)--(-8.114,2.717)--(-8.261,2.378)--cycle;
\draw(-8.307,2.395)--(-8.19,2.665);
\draw(-8.114,2.717)--(-8.261,2.378);
\filldraw[fill opacity=0.8,fill=gray!20,draw=none](-8.086,2.712)--(-8.109,2.728)--(-8.079,2.719)--(-8.076,2.713)--cycle;
\draw(-8.079,2.719)--(-8.076,2.713)--(-8.086,2.712);
\filldraw[fill opacity=0.8,fill=gray!20,draw=none](-8.138,2.552)--(-8.179,2.567)--(-8.112,2.721)--(-8.086,2.712)--(-8.066,2.697)--(-8.11,2.596)--cycle;
\draw(-8.179,2.567)--(-8.112,2.721);
\draw(-8.066,2.697)--(-8.11,2.596);
\filldraw[fill opacity=0.8,fill=gray!20,draw=none](-8.086,2.712)--(-8.064,2.703)--(-8.066,2.697)--cycle;
\draw(-8.064,2.703)--(-8.066,2.697);
\filldraw[fill opacity=0.8,fill=gray!20,draw=none](-8.034,2.742)--(-8.021,2.743)--(-8.022,2.736)--cycle;
\draw(-8.034,2.742)--(-8.021,2.743)--(-8.022,2.736);
\filldraw[fill opacity=0.8,fill=gray!20,draw=none](-8.001,2.732)--(-8.022,2.736)--(-8.021,2.743)--(-8.013,2.742)--cycle;
\draw(-8.022,2.736)--(-8.021,2.743)--(-8.013,2.742);
\filldraw[fill opacity=0.8,fill=gray!20,draw=none](-8.001,2.732)--(-8.013,2.742)--(-7.977,2.74)--(-7.987,2.729)--cycle;
\draw(-8.013,2.742)--(-7.977,2.74)--(-7.987,2.729);
\filldraw[fill opacity=0.8,fill=gray!20,draw=none](-7.999,2.754)--(-8.013,2.734)--(-8.022,2.736)--(-8.034,2.742)--(-8.043,2.751)--(-8.037,2.764)--cycle;
\draw(-8.043,2.751)--(-8.037,2.764);
\filldraw[fill opacity=0.8,fill=gray!20,draw=none](-8.156,2.944)--(-7.918,2.84)--(-7.944,2.796)--(-8.189,2.903)--cycle;
\draw(-8.156,2.944)--(-7.918,2.84)--(-7.944,2.796)--(-8.189,2.903);
\filldraw[fill opacity=0.8,fill=gray!20,draw=none](-7.998,2.82)--(-8.014,2.779)--(-8.019,2.78)--(-8.018,2.806)--cycle;
\draw(-8.014,2.779)--(-8.019,2.78)--(-8.018,2.806);
\filldraw[fill opacity=0.8,fill=gray!20,draw=none](-8.019,2.78)--(-7.98,2.782)--(-7.999,2.754)--(-8.037,2.764)--(-8.033,2.774)--cycle;
\draw(-8.037,2.764)--(-8.033,2.774);
\filldraw[fill opacity=0.8,fill=gray!20,draw=none](-8.019,2.78)--(-8.033,2.774)--(-8.03,2.78)--cycle;
\draw(-8.033,2.774)--(-8.03,2.78);
\filldraw[fill opacity=0.8,fill=gray!20,draw=none](-8.792,1.029)--(-8.11,2.596)--(-8.045,2.689)--(-8.022,2.685)--(-8.665,1.208)--cycle;
\draw(-8.792,1.029)--(-8.11,2.596);
\draw(-8.022,2.685)--(-8.665,1.208);
\filldraw[fill opacity=0.8,fill=gray!20,draw=none](-8.02,2.683)--(-8.023,2.686)--(-8.02,2.716)--(-7.966,2.712)--(-7.996,2.681)--cycle;
\draw(-8.023,2.686)--(-8.02,2.716)--(-7.966,2.712)--(-7.996,2.681)--(-8.02,2.683);
\filldraw[fill opacity=0.8,fill=gray!20,draw=none](-8.078,2.713)--(-8.076,2.713)--(-8.073,2.71)--cycle;
\draw(-8.078,2.713)--(-8.076,2.713)--(-8.073,2.71);
\filldraw[fill opacity=0.8,fill=gray!20,draw=none](-8.042,2.694)--(-8.11,2.596)--(-8.063,2.705)--cycle;
\draw(-8.11,2.596)--(-8.063,2.705);
\filldraw[fill opacity=0.8,fill=gray!20,draw=none](-8.074,2.677)--(-8.119,2.705)--(-8.078,2.713)--(-8.073,2.71)--(-8.052,2.682)--cycle;
\draw(-8.073,2.71)--(-8.052,2.682)--(-8.074,2.677)--(-8.119,2.705)--(-8.078,2.713);
\filldraw[fill opacity=0.8,fill=gray!20,draw=none](-8.086,2.712)--(-8.109,2.728)--(-8.096,2.759)--(-8.051,2.733)--(-8.064,2.703)--cycle;
\draw(-8.109,2.728)--(-8.096,2.759);
\draw(-8.051,2.733)--(-8.064,2.703);
\filldraw[fill opacity=0.8,fill=gray!20,draw=none](-8.045,2.689)--(-8.042,2.694)--(-8.022,2.685)--cycle;
\filldraw[fill opacity=0.8,fill=gray!20,draw=none](-8.052,2.682)--(-8.07,2.706)--(-8.022,2.699)--(-8.024,2.683)--cycle;
\draw(-8.022,2.699)--(-8.024,2.683)--(-8.052,2.682)--(-8.07,2.706);
\filldraw[fill opacity=0.8,fill=gray!20,draw=none](-8.042,2.694)--(-7.98,2.782)--(-7.98,2.782)--(-8.022,2.685)--cycle;
\draw(-7.98,2.782)--(-8.022,2.685);
\filldraw[fill opacity=0.8,fill=gray!20](-7.966,2.712)--(-7.941,2.754)--(-7.888,2.741)--(-7.929,2.703)--cycle;
\filldraw[fill opacity=0.8,fill=gray!20](-7.977,2.74)--(-7.956,2.775)--(-7.911,2.764)--(-7.945,2.732)--cycle;
\filldraw[fill opacity=0.8,fill=gray!20](-7.996,2.681)--(-7.966,2.712)--(-7.929,2.703)--(-7.976,2.676)--cycle;
\filldraw[fill opacity=0.8,fill=gray!20,draw=none](-8.019,2.726)--(-8.042,2.694)--(-8.063,2.705)--(-8.043,2.751)--cycle;
\draw(-8.063,2.705)--(-8.043,2.751);
\filldraw[fill opacity=0.8,fill=gray!20,draw=none](-8.015,2.732)--(-8.019,2.726)--(-8.034,2.742)--cycle;
\filldraw[fill opacity=0.8,fill=gray!20,draw=none](-8.013,2.734)--(-8.015,2.732)--(-8.022,2.736)--cycle;
\filldraw[fill opacity=0.8,fill=gray!20,draw=none](-7.977,2.739)--(-7.977,2.74)--(-7.945,2.732)--(-7.971,2.717)--cycle;
\draw(-7.977,2.739)--(-7.977,2.74)--(-7.945,2.732)--(-7.971,2.717);
\filldraw[fill opacity=0.8,fill=gray!20,draw=none](-7.992,2.724)--(-7.977,2.739)--(-7.971,2.717)--cycle;
\draw(-7.992,2.724)--(-7.977,2.739);
\filldraw[fill opacity=0.8,fill=gray!20,draw=none](-8.001,2.732)--(-7.987,2.729)--(-7.992,2.724)--cycle;
\draw(-7.987,2.729)--(-7.992,2.724);
\filldraw[fill opacity=0.8,fill=gray!20,draw=none](-7.992,2.724)--(-8.001,2.732)--(-7.98,2.782)--(-7.946,2.775)--(-7.971,2.717)--cycle;
\draw(-8.001,2.732)--(-7.98,2.782);
\draw(-7.946,2.775)--(-7.971,2.717);
\filldraw[fill opacity=0.8,fill=gray!20,draw=none](-8.234,2.874)--(-7.982,2.764)--(-8.027,2.749)--(-8.277,2.858)--cycle;
\draw(-8.234,2.874)--(-7.982,2.764)--(-8.027,2.749)--(-8.277,2.858);
\filldraw[fill opacity=0.8,fill=gray!20,draw=none](-8.234,2.874)--(-8.164,2.843)--(-8.201,2.825)--(-8.277,2.858)--cycle;
\draw(-8.234,2.874)--(-8.164,2.843);
\draw(-8.201,2.825)--(-8.277,2.858);
\filldraw[fill opacity=0.8,fill=gray!20,draw=none](-6.958,.414)--(-7.659,.421)--(-7.679,.37)--(-6.958,.362)--cycle;
\draw(-7.679,.37)--(-6.958,.362)--(-6.958,.414)--(-7.659,.421);
\filldraw[fill opacity=0.8,fill=gray!20,draw=none](-7.659,.421)--(-7.677,.421)--(-7.679,.37)--cycle;
\draw(-7.659,.421)--(-7.677,.421);
\filldraw[fill opacity=0.8,fill=gray!20,draw=none](-8.199,2.895)--(-8.189,2.903)--(-8.181,2.918)--(-8.177,2.94)--cycle;
\filldraw[fill opacity=0.8,fill=gray!20](-7.81,3.764)--(-7.804,3.818)--(-7.889,3.802)--(-7.892,3.748)--cycle;
\filldraw[fill opacity=0.8,fill=gray!20](-7.918,3.663)--(-7.902,3.7)--(-7.999,3.696)--(-7.997,3.659)--cycle;
\filldraw[fill opacity=0.8,fill=gray!20](-8.092,3.917)--(-8.073,3.968)--(-8.126,3.981)--(-8.157,3.933)--cycle;
\filldraw[fill opacity=0.5,fill=gray!20,draw=none](-8.661,2.924)--(-8.693,2.964)--(-8.311,2.833)--cycle;
\draw(-8.661,2.924)--(-8.693,2.964)--(-8.311,2.833);
\filldraw[fill opacity=0.8,fill=gray!20,draw=none](-6.773,1.062)--(-6.769,1.119)--(-6.719,1.106)--(-6.699,1.06)--(-6.701,1.044)--cycle;
\draw(-6.699,1.06)--(-6.701,1.044)--(-6.773,1.062)--(-6.769,1.119)--(-6.719,1.106);
\filldraw[fill opacity=0.8,fill=gray!20,draw=none](-6.244,.368)--(-6.244,.329)--(-6.209,.303)--cycle;
\draw(-6.244,.368)--(-6.244,.329);
\filldraw[fill opacity=0.8,fill=gray!20,draw=none](-7.974,4.035)--(-7.988,4.045)--(-7.99,4.038)--cycle;
\draw(-7.988,4.045)--(-7.99,4.038);
\filldraw[fill opacity=0.8,fill=gray!20,draw=none](-7.949,4.018)--(-7.949,4.019)--(-7.974,4.035)--(-7.99,4.038)--cycle;
\draw(-7.949,4.018)--(-7.949,4.019);
\filldraw[fill opacity=0.8,fill=gray!20,draw=none](-7.677,.418)--(-7.677,.421)--(-7.678,.421)--cycle;
\draw(-7.677,.421)--(-7.678,.421);
\filldraw[fill opacity=0.8,fill=gray!20,draw=none](-7.678,.421)--(-7.679,.418)--(-7.677,.418)--cycle;
\draw(-7.679,.418)--(-7.677,.418);
\filldraw[fill opacity=0.8,fill=gray!20,draw=none](-6.047,.532)--(-6.036,.534)--(-6.081,.557)--(-6.22,.54)--(-6.209,.537)--(-6.159,.53)--(-6.103,.529)--cycle;
\draw(-6.22,.54)--(-6.209,.537)--(-6.159,.53)--(-6.103,.529)--(-6.047,.532)--(-6.036,.534);
\filldraw[fill opacity=0.8,fill=gray!20](-6.903,.555)--(-6.926,.587)--(-6.955,.586)--(-6.958,.553)--cycle;
\filldraw[fill opacity=0.8,fill=gray!20,draw=none](-8.168,2.996)--(-8.163,2.968)--(-8.158,2.978)--(-8.155,2.998)--(-8.162,3.035)--cycle;
\filldraw[fill opacity=0.8,fill=gray!20,draw=none](-8.168,2.996)--(-8.177,2.94)--(-8.163,2.968)--cycle;
\filldraw[fill opacity=0.8,fill=gray!20,draw=none](-7.736,4.519)--(-7.736,4.519)--(-7.735,4.511)--cycle;
\draw(-7.736,4.519)--(-7.735,4.511);
\filldraw[fill opacity=0.8,fill=gray!20,draw=none](-7.732,4.506)--(-7.727,4.516)--(-7.737,4.52)--cycle;
\draw(-7.732,4.506)--(-7.727,4.516);
\filldraw[fill opacity=0.8,fill=gray!20,draw=none](-7.708,4.45)--(-7.703,4.444)--(-7.681,4.495)--(-7.727,4.516)--(-7.732,4.506)--cycle;
\draw(-7.703,4.444)--(-7.681,4.495);
\draw(-7.727,4.516)--(-7.732,4.506);
\filldraw[fill opacity=0.8,fill=gray!20,draw=none](-7.731,4.475)--(-7.734,4.502)--(-7.741,4.486)--cycle;
\draw(-7.734,4.502)--(-7.741,4.486);
\filldraw[fill opacity=0.8,fill=gray!20,draw=none](-7.757,4.422)--(-7.722,4.513)--(-7.768,4.534)--(-7.804,4.441)--cycle;
\draw(-7.757,4.422)--(-7.722,4.513);
\draw(-7.768,4.534)--(-7.804,4.441);
\filldraw[fill opacity=0.8,fill=gray!20,draw=none](-7.804,4.441)--(-7.9,4.127)--(-7.768,4.426)--cycle;
\draw(-7.9,4.127)--(-7.768,4.426);
\filldraw[fill opacity=0.8,fill=gray!20,draw=none](-6.212,.454)--(-6.173,.453)--(-6.151,.49)--(-6.227,.491)--cycle;
\draw(-6.212,.454)--(-6.173,.453);
\draw(-6.151,.49)--(-6.227,.491);
\filldraw[fill opacity=0.8,fill=gray!20,draw=none](-6.153,.504)--(-6.162,.508)--(-6.192,.467)--(-6.185,.464)--cycle;
\draw(-6.153,.504)--(-6.162,.508);
\draw(-6.192,.467)--(-6.185,.464);
\filldraw[fill opacity=0.8,fill=gray!20,draw=none](-7.701,.957)--(-7.685,.97)--(-7.695,.963)--(-7.714,.949)--cycle;
\draw(-7.695,.963)--(-7.714,.949)--(-7.701,.957);
\filldraw[fill opacity=0.8,fill=gray!20,draw=none](-8.129,3.727)--(-8.131,3.722)--(-8.1,3.705)--cycle;
\draw(-8.129,3.727)--(-8.131,3.722);
\filldraw[fill opacity=0.8,fill=gray!20,draw=none](-8.1,3.705)--(-8.131,3.722)--(-8.355,3.22)--(-8.316,3.195)--(-8.092,3.699)--cycle;
\draw(-8.131,3.722)--(-8.355,3.22);
\draw(-8.316,3.195)--(-8.092,3.699);
\filldraw[fill opacity=0.8,fill=gray!20,draw=none](-6.041,.508)--(-6.1,.519)--(-6.088,.514)--cycle;
\draw(-6.1,.519)--(-6.088,.514);
\filldraw[fill opacity=0.8,fill=gray!20,draw=none](-7.126,.317)--(-7.126,.311)--(-7.068,.297)--(-7.073,.351)--(-7.104,.359)--cycle;
\draw(-7.126,.311)--(-7.068,.297)--(-7.073,.351)--(-7.104,.359);
\filldraw[fill opacity=0.8,fill=gray!20,draw=none](-8.305,3.142)--(-8.285,3.138)--(-8.282,3.145)--cycle;
\draw(-8.285,3.138)--(-8.282,3.145);
\filldraw[fill opacity=0.8,fill=gray!20](-8.225,3.153)--(-8.263,3.193)--(-8.305,3.185)--(-8.285,3.142)--cycle;
\filldraw[fill opacity=0.8,fill=gray!20,draw=none](-8.25,2.88)--(-8.193,2.908)--(-8.177,2.94)--cycle;
\filldraw[fill opacity=0.8,fill=gray!20,draw=none](-6.035,.507)--(-6.036,.508)--(-6.041,.508)--cycle;
\draw(-6.035,.507)--(-6.036,.508);
\filldraw[fill opacity=0.8,fill=gray!20,draw=none](-8.285,2.838)--(-8.285,2.839)--(-8.286,2.839)--cycle;
\draw(-8.285,2.838)--(-8.285,2.839)--(-8.286,2.839);
\filldraw[fill opacity=0.8,fill=gray!20,draw=none](-6.002,.493)--(-5.974,.473)--(-5.974,.495)--cycle;
\draw(-5.974,.473)--(-5.974,.495);
\filldraw[fill opacity=0.8,fill=gray!20,draw=none](-5.938,.458)--(-5.979,.475)--(-6.008,.496)--(-5.981,.484)--cycle;
\draw(-6.008,.496)--(-5.981,.484)--(-5.938,.458)--(-5.979,.475);
\filldraw[fill opacity=0.8,fill=gray!20,draw=none](-7.884,.337)--(-7.889,.337)--(-7.889,.333)--cycle;
\draw(-7.889,.337)--(-7.889,.333);
\filldraw[fill opacity=0.8,fill=gray!20](-8.104,3.861)--(-8.092,3.917)--(-8.157,3.933)--(-8.177,3.879)--cycle;
\filldraw[fill opacity=0.8,fill=gray!20,draw=none](-7.663,1.239)--(-7.781,1.244)--(-7.79,1.243)--(-7.66,1.238)--cycle;
\draw(-7.663,1.239)--(-7.781,1.244)--(-7.79,1.243)--(-7.66,1.238);
\filldraw[fill opacity=0.8,fill=gray!20,draw=none](-8.293,2.858)--(-8.277,2.858)--(-8.272,2.856)--cycle;
\draw(-8.277,2.858)--(-8.272,2.856);
\filldraw[fill opacity=0.8,fill=gray!20,draw=none](-8.293,2.858)--(-8.277,2.858)--(-8.272,2.856)--cycle;
\draw(-8.277,2.858)--(-8.272,2.856);
\filldraw[fill opacity=0.8,fill=gray!20,draw=none](-8.25,2.88)--(-8.282,2.853)--(-8.227,2.871)--(-8.199,2.895)--(-8.193,2.908)--cycle;
\filldraw[fill opacity=0.8,fill=gray!20](-8.459,3.093)--(-8.44,3.144)--(-8.493,3.157)--(-8.524,3.109)--cycle;
\filldraw[fill opacity=0.8,fill=gray!20](-8.177,2.94)--(-8.171,2.994)--(-8.256,2.978)--(-8.259,2.924)--cycle;
\filldraw[fill opacity=0.8,fill=gray!20,draw=none](-8.953,1.458)--(-8.945,1.457)--(-8.945,1.463)--cycle;
\draw(-8.953,1.458)--(-8.945,1.457)--(-8.945,1.463);
\filldraw[fill opacity=0.8,fill=gray!20,draw=none](-7.755,4.528)--(-7.752,4.548)--(-7.76,4.554)--(-7.768,4.534)--cycle;
\draw(-7.76,4.554)--(-7.768,4.534);
\filldraw[fill opacity=0.8,fill=gray!20,draw=none](-7.511,4.797)--(-7.541,4.807)--(-7.559,4.765)--(-7.527,4.704)--(-7.491,4.786)--cycle;
\draw(-7.541,4.807)--(-7.559,4.765);
\draw(-7.527,4.704)--(-7.491,4.786);
\filldraw[fill opacity=0.8,fill=gray!20,draw=none](-7.568,4.825)--(-7.592,4.832)--(-7.635,4.734)--(-7.629,4.732)--(-7.552,4.783)--(-7.541,4.807)--cycle;
\draw(-7.592,4.832)--(-7.635,4.734);
\draw(-7.552,4.783)--(-7.541,4.807);
\filldraw[fill opacity=0.8,fill=gray!20,draw=none](-7.491,4.801)--(-7.508,4.819)--(-7.549,4.837)--(-7.571,4.833)--(-7.551,4.79)--cycle;
\draw(-7.549,4.837)--(-7.571,4.833)--(-7.551,4.79)--(-7.491,4.801)--(-7.508,4.819);
\filldraw[fill opacity=0.8,fill=gray!20,draw=none](-7.549,4.837)--(-7.587,4.854)--(-7.571,4.833)--cycle;
\draw(-7.587,4.854)--(-7.571,4.833)--(-7.549,4.837);
\filldraw[fill opacity=0.8,fill=gray!20,draw=none](-8.341,3.211)--(-8.355,3.22)--(-8.357,3.214)--cycle;
\draw(-8.355,3.22)--(-8.357,3.214);
\filldraw[fill opacity=0.8,fill=gray!20,draw=none](-8.341,3.211)--(-8.357,3.214)--(-8.316,3.194)--(-8.316,3.195)--cycle;
\draw(-8.316,3.194)--(-8.316,3.195);
\filldraw[fill opacity=0.8,fill=gray!20](-7.948,1.051)--(-7.954,1.096)--(-7.973,1.117)--(-7.967,1.07)--cycle;
\filldraw[fill opacity=0.8,fill=gray!20,draw=none](-7.67,.981)--(-7.677,.98)--(-7.771,.984)--(-7.773,.984)--(-7.741,.984)--(-7.669,.981)--cycle;
\draw(-7.677,.98)--(-7.771,.984);
\draw(-7.741,.984)--(-7.669,.981);
\filldraw[fill opacity=0.8,fill=gray!20,draw=none](-7.662,.986)--(-7.661,.987)--(-7.632,1.023)--(-7.658,1.006)--(-7.682,.972)--cycle;
\draw(-7.661,.987)--(-7.632,1.023)--(-7.658,1.006)--(-7.682,.972)--(-7.662,.986);
\filldraw[fill opacity=0.8,fill=gray!20,draw=none](-6.027,.504)--(-6.035,.507)--(-6.041,.508)--(-6.088,.514)--(-6.07,.506)--cycle;
\draw(-6.027,.504)--(-6.035,.507);
\draw(-6.088,.514)--(-6.07,.506);
\filldraw[fill opacity=0.8,fill=gray!20,draw=none](-7.513,4.684)--(-7.527,4.704)--(-7.532,4.691)--cycle;
\draw(-7.527,4.704)--(-7.532,4.691);
\filldraw[fill opacity=0.8,fill=gray!20,draw=none](-7.126,.317)--(-7.104,.359)--(-7.126,.364)--cycle;
\draw(-7.104,.359)--(-7.126,.364);
\filldraw[fill opacity=0.8,fill=gray!20,draw=none](-6.002,.493)--(-6.002,.388)--(-5.974,.439)--(-5.974,.473)--cycle;
\draw(-6.002,.493)--(-6.002,.388);
\draw(-5.974,.439)--(-5.974,.473);
\filldraw[fill opacity=0.8,fill=gray!20,draw=none](-7.035,1.055)--(-7.04,1.014)--(-7.052,1.015)--cycle;
\draw(-7.04,1.014)--(-7.052,1.015);
\filldraw[fill opacity=0.8,fill=gray!20,draw=none](-7.051,.999)--(-7.067,1.049)--(-7.006,1.06)--cycle;
\draw(-7.051,.999)--(-7.067,1.049)--(-7.006,1.06);
\filldraw[fill opacity=0.8,fill=gray!20](-6.822,.524)--(-6.86,.563)--(-6.903,.555)--(-6.882,.512)--cycle;
\filldraw[fill opacity=0.8,fill=gray!20](-6.769,1.119)--(-6.773,1.173)--(-6.701,1.155)--(-6.694,1.1)--cycle;
\filldraw[fill opacity=0.8,fill=gray!20,draw=none](-7.884,.337)--(-7.854,.341)--(-7.852,.364)--(-7.857,.364)--cycle;
\draw(-7.852,.364)--(-7.857,.364);
\filldraw[fill opacity=0.8,fill=gray!20,draw=none](-7.728,4.661)--(-7.73,4.649)--(-7.723,4.571)--(-7.688,4.574)--(-7.699,4.704)--cycle;
\draw(-7.73,4.649)--(-7.723,4.571);
\draw(-7.688,4.574)--(-7.699,4.704);
\filldraw[fill opacity=0.8,fill=gray!20,draw=none](-7.774,.456)--(-7.72,.458)--(-7.733,.501)--(-7.798,.498)--(-7.799,.484)--cycle;
\draw(-7.774,.456)--(-7.72,.458)--(-7.733,.501)--(-7.798,.498)--(-7.799,.484);
\filldraw[fill opacity=0.8,fill=gray!20,draw=none](-8.995,1.408)--(-8.947,1.405)--(-8.945,1.457)--(-8.953,1.458)--cycle;
\draw(-8.995,1.408)--(-8.947,1.405)--(-8.945,1.457)--(-8.953,1.458);
\filldraw[fill opacity=0.8,fill=gray!20,draw=none](-7.48,4.557)--(-7.472,4.581)--(-7.473,4.582)--(-7.483,4.58)--cycle;
\draw(-7.473,4.582)--(-7.483,4.58);
\filldraw[fill opacity=0.8,fill=gray!20](-8.471,3.037)--(-8.459,3.093)--(-8.524,3.109)--(-8.544,3.055)--cycle;
\filldraw[fill opacity=0.8,fill=gray!20](-7.056,.464)--(-7.037,.514)--(-7.091,.527)--(-7.122,.48)--cycle;
\filldraw[fill opacity=0.8,fill=gray!20](-6.882,.209)--(-6.867,.247)--(-6.963,.243)--(-6.961,.206)--cycle;
\filldraw[fill opacity=0.8,fill=gray!20,draw=none](-7.814,.499)--(-7.808,.499)--(-7.811,.509)--cycle;
\draw(-7.814,.499)--(-7.808,.499);
\filldraw[fill opacity=0.8,fill=gray!20](-7.816,.933)--(-7.841,.943)--(-7.872,.951)--(-7.833,.937)--cycle;
\filldraw[fill opacity=0.8,fill=gray!20,draw=none](-8.369,3.174)--(-8.353,3.152)--(-8.337,3.148)--(-8.335,3.152)--cycle;
\draw(-8.337,3.148)--(-8.335,3.152);
\filldraw[fill opacity=0.8,fill=gray!20,draw=none](-7.662,.986)--(-7.682,.972)--(-7.695,.963)--cycle;
\draw(-7.662,.986)--(-7.682,.972)--(-7.695,.963);
\filldraw[fill opacity=0.8,fill=gray!20,draw=none](-7.702,4.703)--(-7.7,4.705)--(-7.7,4.706)--cycle;
\draw(-7.7,4.705)--(-7.7,4.706);
\filldraw[fill opacity=0.8,fill=gray!20,draw=none](-7.7,4.706)--(-7.701,4.706)--(-7.702,4.703)--cycle;
\draw(-7.701,4.706)--(-7.702,4.703);
\filldraw[fill opacity=0.8,fill=gray!20,draw=none](-7.689,4.71)--(-7.698,4.713)--(-7.701,4.706)--cycle;
\draw(-7.689,4.71)--(-7.698,4.713)--(-7.701,4.706);
\filldraw[fill opacity=0.8,fill=gray!20,draw=none](-9.241,.86)--(-9.266,.871)--(-9.223,.845)--(-9.199,.834)--cycle;
\draw(-9.223,.845)--(-9.199,.834)--(-9.241,.86)--(-9.266,.871);
\filldraw[fill opacity=0.8,fill=gray!20,draw=none](-6.956,.462)--(-7.129,.464)--(-7.14,.416)--(-6.958,.414)--cycle;
\draw(-7.14,.416)--(-6.958,.414)--(-6.956,.462)--(-7.129,.464);
\filldraw[fill opacity=0.8,fill=gray!20,draw=none](-9.003,.772)--(-9.256,.882)--(-9.218,.858)--(-8.948,.741)--cycle;
\draw(-9.003,.772)--(-9.256,.882);
\draw(-9.218,.858)--(-8.948,.741);
\filldraw[fill opacity=0.8,fill=gray!20,draw=none](-8.325,2.837)--(-8.311,2.837)--(-8.335,2.853)--(-8.365,2.856)--(-8.364,2.85)--cycle;
\draw(-8.325,2.837)--(-8.311,2.837);
\draw(-8.365,2.856)--(-8.364,2.85);
\filldraw[fill opacity=0.8,fill=gray!20,draw=none](-7.63,4.873)--(-7.638,4.855)--(-7.625,4.849)--(-7.588,4.841)--(-7.583,4.852)--cycle;
\draw(-7.588,4.841)--(-7.583,4.852)--(-7.63,4.873)--(-7.638,4.855);
\filldraw[fill opacity=0.8,fill=gray!20,draw=none](-7.638,4.855)--(-7.639,4.852)--(-7.625,4.849)--cycle;
\draw(-7.638,4.855)--(-7.639,4.852);
\filldraw[fill opacity=0.8,fill=gray!20,draw=none](-7.639,4.852)--(-7.626,4.84)--(-7.625,4.849)--cycle;
\draw(-7.626,4.84)--(-7.625,4.849);
\filldraw[fill opacity=0.8,fill=gray!20,draw=none](-7.691,4.733)--(-7.699,4.714)--(-7.687,4.709)--(-7.654,4.72)--cycle;
\draw(-7.691,4.733)--(-7.699,4.714);
\filldraw[fill opacity=0.8,fill=gray!20,draw=none](-7.7,4.705)--(-7.699,4.7)--(-7.65,4.679)--(-7.65,4.684)--cycle;
\draw(-7.7,4.705)--(-7.699,4.7);
\draw(-7.65,4.679)--(-7.65,4.684);
\filldraw[fill opacity=0.8,fill=gray!20,draw=none](-7.699,4.714)--(-7.704,4.702)--(-7.687,4.709)--cycle;
\draw(-7.699,4.714)--(-7.704,4.702);
\filldraw[fill opacity=0.8,fill=gray!20,draw=none](-7.699,4.704)--(-7.704,4.702)--(-7.712,4.685)--cycle;
\draw(-7.704,4.702)--(-7.712,4.685);
\filldraw[fill opacity=0.8,fill=gray!20,draw=none](-7.739,4.546)--(-7.764,4.565)--(-7.776,4.537)--(-7.736,4.519)--cycle;
\draw(-7.764,4.565)--(-7.776,4.537)--(-7.736,4.519);
\filldraw[fill opacity=0.8,fill=gray!20,draw=none](-6.002,.493)--(-6.027,.504)--(-6.07,.506)--(-6.047,.496)--cycle;
\draw(-6.002,.493)--(-6.027,.504);
\draw(-6.07,.506)--(-6.047,.496);
\filldraw[fill opacity=0.8,fill=gray!20,draw=none](-6.047,.522)--(-6.047,.211)--(-6.002,.229)--(-6.002,.493)--cycle;
\draw(-6.047,.522)--(-6.047,.211);
\draw(-6.002,.229)--(-6.002,.493);
\filldraw[fill opacity=0.8,fill=gray!20,draw=none](-6.031,.204)--(-6.036,.206)--(-6.002,.229)--(-5.981,.22)--cycle;
\draw(-6.002,.229)--(-5.981,.22)--(-6.031,.204)--(-6.036,.206);
\filldraw[fill opacity=0.8,fill=gray!20](-7.711,.319)--(-7.709,.364)--(-7.801,.36)--(-7.801,.315)--cycle;
\filldraw[fill opacity=0.8,fill=gray!20,draw=none](-7.04,1.141)--(-7.03,1.113)--(-7.05,1.114)--cycle;
\draw(-7.03,1.113)--(-7.05,1.114);
\filldraw[fill opacity=0.8,fill=gray!20,draw=none](-7.03,1.113)--(-7.048,1.161)--(-7.032,1.16)--cycle;
\draw(-7.048,1.161)--(-7.032,1.16);
\filldraw[fill opacity=0.8,fill=gray!20,draw=none](-7.008,1.118)--(-7.073,1.105)--(-7.067,1.16)--(-7.035,1.166)--cycle;
\draw(-7.008,1.118)--(-7.073,1.105)--(-7.067,1.16)--(-7.035,1.166);
\filldraw[fill opacity=0.8,fill=gray!20,draw=none](-8.143,3.037)--(-8.12,3.027)--(-8.13,2.987)--(-8.151,2.996)--cycle;
\draw(-8.143,3.037)--(-8.12,3.027);
\draw(-8.13,2.987)--(-8.151,2.996);
\filldraw[fill opacity=0.8,fill=gray!20,draw=none](-6.227,.272)--(-6.209,.281)--(-6.209,.299)--cycle;
\draw(-6.209,.281)--(-6.209,.299);
\filldraw[fill opacity=0.8,fill=gray!20,draw=none](-8.177,3.051)--(-8.172,3.054)--(-8.186,3.076)--(-8.178,3.052)--cycle;
\draw(-8.177,3.051)--(-8.172,3.054);
\draw(-8.186,3.076)--(-8.178,3.052);
\filldraw[fill opacity=0.8,fill=gray!20,draw=none](-8.168,2.996)--(-8.162,3.035)--(-8.164,3.046)--(-8.181,3.072)--cycle;
\filldraw[fill opacity=0.8,fill=gray!20](-7.882,1.27)--(-7.838,1.287)--(-7.83,1.292)--(-7.867,1.28)--cycle;
\filldraw[fill opacity=0.8,fill=gray!20,draw=none](-7.861,3.981)--(-7.857,3.98)--(-7.704,4.323)--(-7.75,4.34)--(-7.899,4.005)--cycle;
\draw(-7.857,3.98)--(-7.704,4.323);
\draw(-7.75,4.34)--(-7.899,4.005);
\filldraw[fill opacity=0.8,fill=gray!20,draw=none](-9.031,1.297)--(-8.949,1.291)--(-8.949,1.349)--(-9.022,1.354)--cycle;
\draw(-9.031,1.297)--(-8.949,1.291)--(-8.949,1.349)--(-9.022,1.354);
\filldraw[fill opacity=0.8,fill=gray!20,draw=none](-7.129,.423)--(-7.126,.422)--(-7.106,.476)--(-7.122,.48)--(-7.131,.455)--cycle;
\draw(-7.129,.423)--(-7.126,.422);
\draw(-7.106,.476)--(-7.122,.48)--(-7.131,.455);
\filldraw[fill opacity=0.8,fill=gray!20](-7.751,.936)--(-7.714,.949)--(-7.749,.942)--(-7.769,.933)--cycle;
\filldraw[fill opacity=0.8,fill=gray!20,draw=none](-7.861,3.981)--(-7.857,3.979)--(-7.857,3.98)--cycle;
\draw(-7.857,3.979)--(-7.857,3.98);
\filldraw[fill opacity=0.8,fill=gray!20,draw=none](-7.509,4.628)--(-7.437,4.642)--(-7.443,4.699)--(-7.525,4.683)--(-7.523,4.651)--cycle;
\draw(-7.509,4.628)--(-7.437,4.642)--(-7.443,4.699)--(-7.525,4.683)--(-7.523,4.651);
\filldraw[fill opacity=0.8,fill=gray!20,draw=none](-9.16,1.065)--(-9.173,1.108)--(-9.24,1.076)--(-9.24,1.062)--cycle;
\draw(-9.24,1.076)--(-9.24,1.062)--(-9.16,1.065)--(-9.173,1.108);
\filldraw[fill opacity=0.8,fill=gray!20,draw=none](-9.16,1.065)--(-9.173,1.108)--(-9.24,1.076)--(-9.24,1.062)--cycle;
\draw(-9.24,1.076)--(-9.24,1.062)--(-9.16,1.065)--(-9.173,1.108);
\filldraw[fill opacity=0.8,fill=gray!20,draw=none](-9.149,1.1)--(-9.173,1.111)--(-9.218,1.096)--(-9.194,1.085)--cycle;
\draw(-9.149,1.1)--(-9.173,1.111);
\draw(-9.218,1.096)--(-9.194,1.085);
\filldraw[fill opacity=0.8,fill=gray!20,draw=none](-9.251,1.063)--(-9.24,1.062)--(-9.24,1.076)--cycle;
\draw(-9.251,1.063)--(-9.24,1.062)--(-9.24,1.076);
\filldraw[fill opacity=0.8,fill=gray!20,draw=none](-9.251,1.063)--(-9.24,1.062)--(-9.24,1.076)--cycle;
\draw(-9.251,1.063)--(-9.24,1.062)--(-9.24,1.076);
\filldraw[fill opacity=0.8,fill=gray!20,draw=none](-9.194,1.085)--(-9.218,1.096)--(-9.256,1.064)--(-9.232,1.053)--cycle;
\draw(-9.194,1.085)--(-9.218,1.096);
\draw(-9.256,1.064)--(-9.232,1.053);
\filldraw[fill opacity=0.8,fill=gray!20,draw=none](-9.232,1.053)--(-9.256,1.064)--(-9.282,1.019)--(-9.258,1.009)--cycle;
\draw(-9.232,1.053)--(-9.256,1.064);
\draw(-9.282,1.019)--(-9.258,1.009);
\filldraw[fill opacity=0.8,fill=gray!20,draw=none](-9.061,.959)--(-9.119,1.064)--(-9.167,1.108)--(-9.199,1.098)--(-9.241,1.062)--(-9.254,1.04)--cycle;
\draw(-9.167,1.108)--(-9.199,1.098)--(-9.241,1.062)--(-9.254,1.04);
\filldraw[fill opacity=0.8,fill=gray!20,draw=none](-6.958,.315)--(-6.692,.313)--(-6.651,.359)--(-6.958,.362)--cycle;
\draw(-6.651,.359)--(-6.958,.362)--(-6.958,.315)--(-6.692,.313);
\filldraw[fill opacity=0.8,fill=gray!20,draw=none](-6.977,1.215)--(-6.975,1.223)--(-6.879,1.227)--(-6.878,1.212)--cycle;
\draw(-6.977,1.215)--(-6.975,1.223)--(-6.879,1.227)--(-6.878,1.212);
\filldraw[fill opacity=0.8,fill=gray!20,draw=none](-7.931,1.217)--(-7.948,1.205)--(-7.932,1.188)--(-7.914,1.216)--cycle;
\draw(-7.948,1.205)--(-7.932,1.188)--(-7.914,1.216);
\filldraw[fill opacity=0.8,fill=gray!20,draw=none](-7.965,1.204)--(-7.803,1.241)--(-7.824,1.241)--(-7.947,1.213)--cycle;
\draw(-7.965,1.204)--(-7.803,1.241)--(-7.824,1.241)--(-7.947,1.213);
\filldraw[fill opacity=0.8,fill=gray!20,draw=none](-6.209,.278)--(-6.209,.179)--(-6.159,.173)--(-6.159,.251)--cycle;
\draw(-6.209,.278)--(-6.209,.179)--(-6.159,.173)--(-6.159,.251);
\filldraw[fill opacity=0.8,fill=gray!20,draw=none](-8.045,3.676)--(-8.266,3.181)--(-8.228,3.157)--(-8.224,3.156)--(-7.999,3.659)--cycle;
\draw(-8.045,3.676)--(-8.266,3.181);
\draw(-8.224,3.156)--(-7.999,3.659);
\filldraw[fill opacity=0.8,fill=gray!20,draw=none](-8.041,3.686)--(-8.045,3.676)--(-7.999,3.659)--cycle;
\draw(-8.041,3.686)--(-8.045,3.676);
\filldraw[fill opacity=0.8,fill=gray!20,draw=none](-7.861,3.981)--(-7.899,4.005)--(-7.904,3.994)--cycle;
\draw(-7.899,4.005)--(-7.904,3.994);
\filldraw[fill opacity=0.8,fill=gray!20](-8.841,1.353)--(-8.851,1.409)--(-8.947,1.405)--(-8.949,1.349)--cycle;
\filldraw[fill opacity=0.8,fill=gray!20,draw=none](-6.13,.51)--(-6.118,.502)--(-6.116,.503)--(-6.115,.51)--cycle;
\filldraw[fill opacity=0.8,fill=gray!20,draw=none](-7.715,4.327)--(-7.723,4.4)--(-7.75,4.34)--cycle;
\draw(-7.723,4.4)--(-7.75,4.34);
\filldraw[fill opacity=0.8,fill=gray!20,draw=none](-7.461,4.753)--(-7.481,4.786)--(-7.549,4.783)--(-7.535,4.739)--cycle;
\draw(-7.549,4.783)--(-7.535,4.739)--(-7.461,4.753)--(-7.481,4.786);
\filldraw[fill opacity=0.8,fill=gray!20,draw=none](-7.481,4.786)--(-7.489,4.791)--(-7.491,4.786)--cycle;
\draw(-7.489,4.791)--(-7.491,4.786);
\filldraw[fill opacity=0.8,fill=gray!20,draw=none](-7.814,.464)--(-7.798,.493)--(-7.798,.498)--(-7.814,.499)--cycle;
\draw(-7.798,.493)--(-7.798,.498)--(-7.814,.499);
\filldraw[fill opacity=0.8,fill=gray!20](-7.632,1.023)--(-7.614,1.066)--(-7.643,1.047)--(-7.658,1.006)--cycle;
\filldraw[fill opacity=0.8,fill=gray!20](-7.954,1.096)--(-7.948,1.143)--(-7.967,1.163)--(-7.973,1.117)--cycle;
\filldraw[fill opacity=0.8,fill=gray!20](-7.829,3.929)--(-7.858,3.977)--(-7.918,3.966)--(-7.902,3.915)--cycle;
\filldraw[fill opacity=0.8,fill=gray!20,draw=none](-7.644,4.406)--(-7.667,4.402)--(-7.667,4.41)--cycle;
\draw(-7.667,4.402)--(-7.667,4.41);
\filldraw[fill opacity=0.8,fill=gray!20,draw=none](-6.785,.308)--(-6.786,.314)--(-6.788,.314)--cycle;
\draw(-6.786,.314)--(-6.788,.314);
\filldraw[fill opacity=0.8,fill=gray!20,draw=none](-6.788,.308)--(-6.785,.308)--(-6.788,.314)--cycle;
\draw(-6.788,.308)--(-6.785,.308);
\filldraw[fill opacity=0.8,fill=gray!20](-8.838,1.296)--(-8.841,1.353)--(-8.949,1.349)--(-8.949,1.291)--cycle;
\filldraw[fill opacity=0.8,fill=gray!20,draw=none](-7.689,4.424)--(-7.684,4.422)--(-7.667,4.41)--(-7.666,4.399)--cycle;
\draw(-7.667,4.41)--(-7.666,4.399);
\filldraw[fill opacity=0.8,fill=gray!20,draw=none](-7.697,4.419)--(-7.669,4.49)--(-7.722,4.513)--(-7.729,4.494)--cycle;
\draw(-7.697,4.419)--(-7.669,4.49);
\draw(-7.722,4.513)--(-7.729,4.494);
\filldraw[fill opacity=0.8,fill=gray!20,draw=none](-7.689,4.424)--(-7.666,4.399)--(-7.663,4.358)--(-7.706,4.429)--(-7.707,4.43)--cycle;
\draw(-7.666,4.399)--(-7.663,4.358);
\draw(-7.706,4.429)--(-7.707,4.43);
\filldraw[fill opacity=0.8,fill=gray!20,draw=none](-7.696,4.507)--(-7.698,4.508)--(-7.697,4.506)--cycle;
\filldraw[fill opacity=0.8,fill=gray!20,draw=none](-7.697,4.51)--(-7.71,4.512)--(-7.711,4.509)--(-7.678,4.494)--cycle;
\draw(-7.711,4.509)--(-7.678,4.494);
\filldraw[fill opacity=0.8,fill=gray!20,draw=none](-7.689,4.525)--(-7.696,4.509)--(-7.678,4.494)--(-7.667,4.519)--cycle;
\draw(-7.678,4.494)--(-7.667,4.519);
\filldraw[fill opacity=0.8,fill=gray!20,draw=none](-7.715,4.326)--(-7.719,4.364)--(-7.731,4.333)--cycle;
\draw(-7.719,4.364)--(-7.731,4.333)--(-7.715,4.326);
\filldraw[fill opacity=0.8,fill=gray!20,draw=none](-7.731,4.333)--(-7.719,4.364)--(-7.73,4.47)--(-7.736,4.476)--(-7.784,4.353)--cycle;
\draw(-7.736,4.476)--(-7.784,4.353)--(-7.731,4.333)--(-7.719,4.364);
\filldraw[fill opacity=0.8,fill=gray!20,draw=none](-7.708,4.447)--(-7.68,4.138)--(-7.704,4.177)--(-7.731,4.473)--cycle;
\draw(-7.708,4.447)--(-7.68,4.138);
\draw(-7.704,4.177)--(-7.731,4.473);
\filldraw[fill opacity=0.8,fill=gray!20,draw=none](-7.718,4.322)--(-7.704,4.177)--(-7.711,4.237)--(-7.719,4.321)--cycle;
\draw(-7.718,4.322)--(-7.704,4.177);
\draw(-7.711,4.237)--(-7.719,4.321);
\filldraw[fill opacity=0.8,fill=gray!20,draw=none](-7.705,4.184)--(-7.708,4.208)--(-7.701,4.231)--(-7.653,4.24)--(-7.663,4.193)--cycle;
\draw(-7.701,4.231)--(-7.653,4.24)--(-7.663,4.193)--(-7.705,4.184);
\filldraw[fill opacity=0.8,fill=gray!20,draw=none](-7.701,4.231)--(-7.693,4.247)--(-7.672,4.271)--(-7.637,4.278)--(-7.653,4.24)--cycle;
\draw(-7.672,4.271)--(-7.637,4.278)--(-7.653,4.24)--(-7.701,4.231);
\filldraw[fill opacity=0.8,fill=gray!20,draw=none](-7.672,4.271)--(-7.661,4.28)--(-7.619,4.302)--(-7.617,4.303)--(-7.637,4.278)--cycle;
\draw(-7.619,4.302)--(-7.617,4.303)--(-7.637,4.278)--(-7.672,4.271);
\filldraw[fill opacity=0.8,fill=gray!20,draw=none](-7.619,4.302)--(-7.615,4.303)--(-7.617,4.303)--cycle;
\draw(-7.615,4.303)--(-7.617,4.303)--(-7.619,4.302);
\filldraw[fill opacity=0.8,fill=gray!20,draw=none](-7.699,4.7)--(-7.651,4.171)--(-7.605,4.191)--(-7.65,4.679)--cycle;
\draw(-7.699,4.7)--(-7.651,4.171)--(-7.605,4.191)--(-7.65,4.679);
\filldraw[fill opacity=0.8,fill=gray!20,draw=none](-8.164,3.046)--(-8.162,3.045)--(-8.162,3.035)--cycle;
\draw(-8.164,3.046)--(-8.162,3.045);
\filldraw[fill opacity=0.8,fill=gray!20,draw=none](-8.228,3.157)--(-8.224,3.154)--(-8.224,3.156)--cycle;
\draw(-8.224,3.154)--(-8.224,3.156);
\filldraw[fill opacity=0.8,fill=gray!20](-7.804,3.818)--(-7.81,3.874)--(-7.892,3.859)--(-7.889,3.802)--cycle;
\filldraw[fill opacity=0.8,fill=gray!20,draw=none](-7.625,4.849)--(-7.592,4.832)--(-7.588,4.841)--cycle;
\draw(-7.592,4.832)--(-7.588,4.841);
\filldraw[fill opacity=0.8,fill=gray!20,draw=none](-9.022,1.354)--(-8.949,1.349)--(-8.947,1.405)--(-8.995,1.408)--cycle;
\draw(-9.022,1.354)--(-8.949,1.349)--(-8.947,1.405)--(-8.995,1.408);
\filldraw[fill opacity=0.8,fill=gray!20,draw=none](-8.228,3.157)--(-8.266,3.181)--(-8.271,3.17)--cycle;
\draw(-8.266,3.181)--(-8.271,3.17);
\filldraw[fill opacity=0.8,fill=gray!20](-7.788,1.297)--(-7.79,1.293)--(-7.79,1.293)--(-7.764,1.295)--cycle;
\filldraw[fill opacity=0.8,fill=gray!20](-7.812,1.296)--(-7.79,1.293)--(-7.79,1.293)--(-7.788,1.297)--cycle;
\filldraw[fill opacity=0.8,fill=gray!20,draw=none](-6.002,.54)--(-6.002,.493)--(-5.974,.495)--(-5.974,.551)--cycle;
\draw(-5.974,.495)--(-5.974,.551)--(-6.002,.54)--(-6.002,.493);
\filldraw[fill opacity=0.8,fill=gray!20,draw=none](-7.637,.978)--(-7.673,.98)--(-7.669,.981)--(-7.661,.981)--cycle;
\draw(-7.637,.978)--(-7.673,.98);
\draw(-7.669,.981)--(-7.661,.981);
\filldraw[fill opacity=0.8,fill=gray!20,draw=none](-7.854,.341)--(-7.801,.349)--(-7.801,.36)--(-7.852,.364)--cycle;
\draw(-7.801,.349)--(-7.801,.36)--(-7.852,.364);
\filldraw[fill opacity=0.8,fill=gray!20,draw=none](-9.099,1.109)--(-9.123,1.119)--(-9.173,1.125)--(-9.149,1.114)--cycle;
\draw(-9.173,1.125)--(-9.149,1.114)--(-9.099,1.109)--(-9.123,1.119);
\filldraw[fill opacity=0.8,fill=gray!20,draw=none](-6.788,.308)--(-6.788,.314)--(-6.815,.356)--(-6.853,.348)--(-6.857,.294)--cycle;
\draw(-6.815,.356)--(-6.853,.348)--(-6.857,.294)--(-6.788,.308);
\filldraw[fill opacity=0.8,fill=gray!20,draw=none](-8.474,.856)--(-8.049,.952)--(-7.92,.974)--(-7.889,.963)--(-8.309,.868)--cycle;
\draw(-8.474,.856)--(-8.049,.952);
\draw(-7.889,.963)--(-8.309,.868);
\filldraw[fill opacity=0.8,fill=gray!20,draw=none](-6.126,.49)--(-6.133,.472)--(-6.126,.471)--(-6.115,.486)--cycle;
\draw(-6.133,.472)--(-6.126,.471)--(-6.115,.486);
\filldraw[fill opacity=0.8,fill=gray!20,draw=none](-6.117,.466)--(-6.117,.467)--(-6.126,.471)--(-6.133,.472)--cycle;
\draw(-6.117,.467)--(-6.126,.471)--(-6.133,.472);
\filldraw[fill opacity=0.8,fill=gray!20,draw=none](-6.128,.453)--(-6.117,.453)--(-6.117,.461)--cycle;
\draw(-6.128,.453)--(-6.117,.453)--(-6.117,.461);
\filldraw[fill opacity=0.8,fill=gray!20,draw=none](-6.159,.453)--(-6.159,.412)--(-6.103,.407)--(-6.103,.482)--cycle;
\draw(-6.159,.453)--(-6.159,.412);
\draw(-6.103,.407)--(-6.103,.482);
\filldraw[fill opacity=0.8,fill=gray!20,draw=none](-6.071,.491)--(-6.103,.482)--(-6.103,.423)--(-6.047,.452)--(-6.047,.465)--cycle;
\draw(-6.103,.482)--(-6.103,.423);
\draw(-6.047,.452)--(-6.047,.465);
\filldraw[fill opacity=0.8,fill=gray!20,draw=none](-6.081,.473)--(-6.153,.504)--(-6.185,.464)--(-6.124,.437)--cycle;
\draw(-6.185,.464)--(-6.124,.437)--(-6.081,.473)--(-6.153,.504);
\filldraw[fill opacity=0.8,fill=gray!20,draw=none](-8.208,3.125)--(-8.225,3.153)--(-8.285,3.142)--(-8.282,3.132)--cycle;
\draw(-8.208,3.125)--(-8.225,3.153)--(-8.285,3.142)--(-8.282,3.132);
\filldraw[fill opacity=0.8,fill=gray!20,draw=none](-7.472,4.581)--(-7.471,4.582)--(-7.473,4.582)--cycle;
\draw(-7.471,4.582)--(-7.473,4.582);
\filldraw[fill opacity=0.8,fill=gray!20,draw=none](-7.057,.957)--(-7.044,.957)--(-7.039,.933)--(-7.055,.942)--(-7.06,.948)--cycle;
\draw(-7.055,.942)--(-7.06,.948);
\filldraw[fill opacity=0.8,fill=gray!20,draw=none](-7.057,.957)--(-7.06,.948)--(-7.066,.956)--cycle;
\draw(-7.06,.948)--(-7.066,.956);
\filldraw[fill opacity=0.8,fill=gray!20,draw=none](-6.899,.949)--(-7.637,.978)--(-7.661,.981)--(-6.89,.95)--cycle;
\draw(-7.661,.981)--(-6.89,.95)--(-6.899,.949)--(-7.637,.978);
\filldraw[fill opacity=0.8,fill=gray!20,draw=none](-8.342,3.135)--(-8.353,3.152)--(-8.399,3.153)--(-8.347,3.13)--cycle;
\draw(-8.399,3.153)--(-8.347,3.13);
\filldraw[fill opacity=0.8,fill=gray!20](-7.709,1.278)--(-7.748,1.291)--(-7.743,1.286)--(-7.699,1.268)--cycle;
\filldraw[fill opacity=0.8,fill=gray!20](-8.171,2.994)--(-8.177,3.05)--(-8.259,3.035)--(-8.256,2.978)--cycle;
\filldraw[fill opacity=0.8,fill=gray!20,draw=none](-7.484,4.597)--(-7.473,4.582)--(-7.471,4.582)--(-7.458,4.638)--(-7.489,4.632)--cycle;
\draw(-7.473,4.582)--(-7.471,4.582);
\draw(-7.458,4.638)--(-7.489,4.632);
\filldraw[fill opacity=0.8,fill=gray!20,draw=none](-8.1,3.705)--(-8.092,3.699)--(-8.091,3.7)--cycle;
\draw(-8.092,3.699)--(-8.091,3.7);
\filldraw[fill opacity=0.8,fill=gray!20,draw=none](-7.559,4.765)--(-7.577,4.724)--(-7.553,4.706)--(-7.527,4.702)--(-7.527,4.704)--cycle;
\draw(-7.559,4.765)--(-7.577,4.724);
\draw(-7.527,4.702)--(-7.527,4.704);
\filldraw[fill opacity=0.8,fill=gray!20,draw=none](-7.553,4.706)--(-7.532,4.691)--(-7.527,4.702)--cycle;
\draw(-7.532,4.691)--(-7.527,4.702);
\filldraw[fill opacity=0.8,fill=gray!20](-7.443,4.699)--(-7.461,4.753)--(-7.535,4.739)--(-7.525,4.683)--cycle;
\filldraw[fill opacity=0.8,fill=gray!20,draw=none](-6.788,.314)--(-6.79,.361)--(-6.815,.356)--cycle;
\draw(-6.79,.361)--(-6.815,.356);
\filldraw[fill opacity=0.8,fill=gray!20,draw=none](-7.483,4.58)--(-7.473,4.582)--(-7.484,4.597)--cycle;
\draw(-7.483,4.58)--(-7.473,4.582);
\filldraw[fill opacity=0.8,fill=gray!20,draw=none](-7.491,4.786)--(-7.511,4.797)--(-7.551,4.79)--(-7.549,4.783)--cycle;
\draw(-7.511,4.797)--(-7.551,4.79)--(-7.549,4.783);
\filldraw[fill opacity=0.8,fill=gray!20,draw=none](-7.511,4.797)--(-7.491,4.786)--(-7.489,4.791)--cycle;
\draw(-7.491,4.786)--(-7.489,4.791);
\filldraw[fill opacity=0.8,fill=gray!20,draw=none](-7.531,4.829)--(-7.541,4.807)--(-7.489,4.791)--(-7.482,4.808)--cycle;
\draw(-7.489,4.791)--(-7.482,4.808)--(-7.531,4.829)--(-7.541,4.807);
\filldraw[fill opacity=0.8,fill=gray!20,draw=none](-7.736,4.52)--(-7.736,4.519)--(-7.735,4.511)--(-7.734,4.496)--cycle;
\draw(-7.735,4.511)--(-7.734,4.496);
\filldraw[fill opacity=0.8,fill=gray!20,draw=none](-7.79,4.356)--(-7.768,4.426)--(-7.797,4.359)--cycle;
\draw(-7.768,4.426)--(-7.797,4.359);
\filldraw[fill opacity=0.8,fill=gray!20,draw=none](-7.814,.464)--(-7.814,.456)--(-7.8,.455)--(-7.798,.493)--cycle;
\draw(-7.814,.456)--(-7.8,.455)--(-7.798,.493);
\filldraw[fill opacity=0.8,fill=gray!20,draw=none](-6.206,.301)--(-6.209,.303)--(-6.209,.299)--cycle;
\draw(-6.209,.303)--(-6.209,.299);
\filldraw[fill opacity=0.8,fill=gray!20,draw=none](-6.219,.283)--(-6.208,.301)--(-6.207,.303)--(-6.21,.307)--(-6.26,.308)--cycle;
\draw(-6.21,.307)--(-6.26,.308);
\filldraw[fill opacity=0.8,fill=gray!20](-7.83,1.292)--(-7.79,1.293)--(-7.79,1.293)--(-7.812,1.296)--cycle;
\filldraw[fill opacity=0.8,fill=gray!20,draw=none](-7.934,4.015)--(-7.897,4.01)--(-7.75,4.34)--(-7.797,4.359)--(-7.946,4.024)--cycle;
\draw(-7.897,4.01)--(-7.75,4.34);
\draw(-7.797,4.359)--(-7.946,4.024);
\filldraw[fill opacity=0.8,fill=gray!20](-6.793,.475)--(-6.822,.524)--(-6.882,.512)--(-6.867,.461)--cycle;
\filldraw[fill opacity=0.8,fill=gray!20,draw=none](-6.958,.362)--(-6.79,.361)--(-6.788,.412)--(-6.958,.414)--cycle;
\draw(-6.788,.412)--(-6.958,.414)--(-6.958,.362)--(-6.79,.361);
\filldraw[fill opacity=0.8,fill=gray!20,draw=none](-6.79,.361)--(-6.768,.365)--(-6.775,.421)--(-6.812,.414)--cycle;
\draw(-6.79,.361)--(-6.768,.365)--(-6.775,.421)--(-6.812,.414);
\filldraw[fill opacity=0.8,fill=gray!20](-6.879,1.227)--(-6.881,1.264)--(-6.805,1.258)--(-6.785,1.22)--cycle;
\filldraw[fill opacity=0.8,fill=gray!20](-6.975,1.223)--(-6.96,1.26)--(-6.881,1.264)--(-6.879,1.227)--cycle;
\filldraw[fill opacity=0.8,fill=gray!20](-7.711,.412)--(-7.72,.458)--(-7.8,.455)--(-7.801,.408)--cycle;
\filldraw[fill opacity=0.8,fill=gray!20,draw=none](-7.716,4.456)--(-7.731,4.473)--(-7.735,4.511)--cycle;
\draw(-7.731,4.473)--(-7.735,4.511);
\filldraw[fill opacity=0.8,fill=gray!20,draw=none](-7.944,4.017)--(-7.934,4.015)--(-7.946,4.024)--(-7.949,4.019)--cycle;
\draw(-7.946,4.024)--(-7.949,4.019);
\filldraw[fill opacity=0.8,fill=gray!20,draw=none](-7.126,.416)--(-7.126,.422)--(-7.129,.423)--cycle;
\draw(-7.126,.422)--(-7.129,.423);
\filldraw[fill opacity=0.8,fill=gray!20,draw=none](-7.944,4.017)--(-7.949,4.019)--(-7.949,4.018)--cycle;
\draw(-7.949,4.019)--(-7.949,4.018);
\filldraw[fill opacity=0.8,fill=gray!20](-7.81,3.874)--(-7.829,3.929)--(-7.902,3.915)--(-7.892,3.859)--cycle;
\filldraw[fill opacity=0.8,fill=gray!20,draw=none](-8.205,3.121)--(-8.208,3.125)--(-8.209,3.125)--cycle;
\draw(-8.205,3.121)--(-8.208,3.125);
\filldraw[fill opacity=0.8,fill=gray!20,draw=none](-7.057,1.214)--(-7.048,1.215)--(-7.051,1.207)--(-7.08,1.188)--(-7.066,1.205)--cycle;
\draw(-7.051,1.207)--(-7.08,1.188)--(-7.066,1.205);
\filldraw[fill opacity=0.8,fill=gray!20,draw=none](-6.881,1.208)--(-7.663,1.239)--(-7.66,1.238)--(-6.89,1.207)--cycle;
\draw(-7.66,1.238)--(-6.89,1.207)--(-6.881,1.208)--(-7.663,1.239);
\filldraw[fill opacity=0.8,fill=gray!20,draw=none](-7.934,4.015)--(-7.944,4.017)--(-7.927,4.011)--cycle;
\filldraw[fill opacity=0.8,fill=gray!20](-6.773,1.173)--(-6.785,1.22)--(-6.72,1.204)--(-6.701,1.155)--cycle;
\filldraw[fill opacity=0.8,fill=gray!20,draw=none](-6.785,.419)--(-6.775,.421)--(-6.782,.442)--cycle;
\draw(-6.785,.419)--(-6.775,.421)--(-6.782,.442);
\filldraw[fill opacity=0.8,fill=gray!20,draw=none](-6.78,.43)--(-6.765,.46)--(-6.785,.46)--cycle;
\draw(-6.765,.46)--(-6.785,.46);
\filldraw[fill opacity=0.8,fill=gray!20,draw=none](-7.739,4.546)--(-7.736,4.519)--(-7.737,4.52)--(-7.741,4.564)--cycle;
\draw(-7.737,4.52)--(-7.741,4.564);
\filldraw[fill opacity=0.8,fill=gray!20](-7.629,4.786)--(-7.627,4.83)--(-7.681,4.834)--(-7.706,4.792)--cycle;
\filldraw[fill opacity=0.8,fill=gray!20](-7.764,1.295)--(-7.79,1.293)--(-7.79,1.293)--(-7.748,1.291)--cycle;
\filldraw[fill opacity=0.8,fill=gray!20,draw=none](-7.948,1.143)--(-7.932,1.188)--(-7.948,1.205)--(-7.95,1.202)--(-7.967,1.163)--cycle;
\draw(-7.95,1.202)--(-7.967,1.163)--(-7.948,1.143)--(-7.932,1.188)--(-7.948,1.205);
\filldraw[fill opacity=0.8,fill=gray!20](-7.709,.364)--(-7.711,.412)--(-7.801,.408)--(-7.801,.36)--cycle;
\filldraw[fill opacity=0.8,fill=gray!20](-7.614,1.066)--(-7.608,1.112)--(-7.638,1.092)--(-7.643,1.047)--cycle;
\filldraw[fill opacity=0.8,fill=gray!20](-7.793,.931)--(-7.796,.94)--(-7.841,.943)--(-7.816,.933)--cycle;
\filldraw[fill opacity=0.8,fill=gray!20,draw=none](-7.934,4.015)--(-7.927,4.011)--(-7.901,4.001)--(-7.897,4.01)--cycle;
\draw(-7.901,4.001)--(-7.897,4.01);
\filldraw[fill opacity=0.8,fill=gray!20,draw=none](-7.85,.369)--(-7.857,.364)--(-7.852,.364)--cycle;
\draw(-7.857,.364)--(-7.852,.364);
\filldraw[fill opacity=0.8,fill=gray!20,draw=none](-8.091,3.7)--(-8.313,3.2)--(-8.301,3.191)--(-8.264,3.186)--(-8.05,3.667)--cycle;
\draw(-8.091,3.7)--(-8.313,3.2);
\draw(-8.264,3.186)--(-8.05,3.667);
\filldraw[fill opacity=0.8,fill=gray!20,draw=none](-8.091,3.7)--(-8.05,3.667)--(-8.041,3.686)--cycle;
\draw(-8.05,3.667)--(-8.041,3.686);
\filldraw[fill opacity=0.8,fill=gray!20,draw=none](-6.78,.43)--(-6.776,.412)--(-6.23,.406)--(-6.238,.454)--(-6.765,.46)--cycle;
\draw(-6.776,.412)--(-6.23,.406);
\draw(-6.238,.454)--(-6.765,.46);
\filldraw[fill opacity=0.8,fill=gray!20,draw=none](-8.507,2.94)--(-8.506,2.939)--(-8.477,2.961)--(-8.488,2.966)--cycle;
\draw(-8.507,2.94)--(-8.506,2.939);
\draw(-8.477,2.961)--(-8.488,2.966);
\filldraw[fill opacity=0.8,fill=gray!20,draw=none](-8.464,2.98)--(-8.474,2.969)--(-8.477,2.961)--(-8.446,2.947)--(-8.428,2.981)--(-8.427,2.991)--(-8.438,2.996)--cycle;
\draw(-8.477,2.961)--(-8.446,2.947);
\draw(-8.427,2.991)--(-8.438,2.996);
\filldraw[fill opacity=0.8,fill=gray!20,draw=none](-8.506,2.939)--(-8.472,2.935)--(-8.474,2.963)--cycle;
\draw(-8.472,2.935)--(-8.474,2.963);
\filldraw[fill opacity=0.8,fill=gray!20,draw=none](-6.692,.313)--(-6.21,.307)--(-6.243,.355)--(-6.651,.359)--cycle;
\draw(-6.692,.313)--(-6.21,.307);
\draw(-6.243,.355)--(-6.651,.359);
\filldraw[fill opacity=0.8,fill=gray!20,draw=none](-6.071,.491)--(-6.047,.465)--(-6.047,.496)--cycle;
\draw(-6.047,.465)--(-6.047,.496);
\filldraw[fill opacity=0.8,fill=gray!20,draw=none](-6.047,.496)--(-6.088,.514)--(-6.133,.496)--(-6.093,.478)--cycle;
\draw(-6.047,.496)--(-6.088,.514);
\draw(-6.133,.496)--(-6.093,.478);
\filldraw[fill opacity=0.8,fill=gray!20,draw=none](-7.116,.389)--(-7.126,.364)--(-7.104,.359)--cycle;
\draw(-7.126,.364)--(-7.104,.359);
\filldraw[fill opacity=0.8,fill=gray!20,draw=none](-7.126,.364)--(-7.116,.389)--(-7.126,.416)--cycle;
\filldraw[fill opacity=0.8,fill=gray!20,draw=none](-7.904,3.994)--(-7.901,4.001)--(-7.927,4.011)--cycle;
\draw(-7.904,3.994)--(-7.901,4.001);
\filldraw[fill opacity=0.8,fill=gray!20,draw=none](-8.53,2.949)--(-8.507,2.94)--(-8.488,2.966)--(-8.496,2.969)--cycle;
\draw(-8.488,2.966)--(-8.496,2.969)--(-8.53,2.949)--(-8.507,2.94);
\filldraw[fill opacity=0.8,fill=gray!20,draw=none](-8.317,2.847)--(-8.293,2.844)--(-8.25,2.88)--cycle;
\draw(-8.317,2.847)--(-8.293,2.844);
\filldraw[fill opacity=0.8,fill=gray!20,draw=none](-7.774,.456)--(-7.799,.484)--(-7.8,.455)--cycle;
\draw(-7.799,.484)--(-7.8,.455)--(-7.774,.456);
\filldraw[fill opacity=0.8,fill=gray!20,draw=none](-7.126,.416)--(-7.116,.389)--(-7.105,.417)--(-7.126,.422)--cycle;
\draw(-7.105,.417)--(-7.126,.422);
\filldraw[fill opacity=0.8,fill=gray!20,draw=none](-7.126,.422)--(-7.068,.408)--(-7.056,.464)--(-7.106,.476)--cycle;
\draw(-7.126,.422)--(-7.068,.408)--(-7.056,.464)--(-7.106,.476);
\filldraw[fill opacity=0.8,fill=gray!20,draw=none](-7.85,.369)--(-7.852,.364)--(-7.801,.36)--(-7.801,.401)--cycle;
\draw(-7.852,.364)--(-7.801,.36)--(-7.801,.401);
\filldraw[fill opacity=0.8,fill=gray!20,draw=none](-8.177,3.051)--(-8.178,3.052)--(-8.177,3.05)--cycle;
\draw(-8.178,3.052)--(-8.177,3.05)--(-8.177,3.051);
\filldraw[fill opacity=0.8,fill=gray!20,draw=none](-8.177,3.05)--(-8.186,3.076)--(-8.206,3.103)--(-8.269,3.09)--(-8.259,3.035)--cycle;
\draw(-8.206,3.103)--(-8.269,3.09)--(-8.259,3.035)--(-8.177,3.05)--(-8.186,3.076);
\filldraw[fill opacity=0.8,fill=gray!20,draw=none](-8.234,3.111)--(-8.203,3.109)--(-8.227,3.123)--(-8.239,3.114)--cycle;
\filldraw[fill opacity=0.8,fill=gray!20,draw=none](-8.234,3.111)--(-8.232,3.111)--(-8.217,3.101)--cycle;
\draw(-8.234,3.111)--(-8.232,3.111);
\filldraw[fill opacity=0.8,fill=gray!20,draw=none](-8.249,3.095)--(-8.217,3.101)--(-8.246,3.11)--(-8.261,3.098)--cycle;
\draw(-8.249,3.095)--(-8.217,3.101);
\filldraw[fill opacity=0.8,fill=gray!20,draw=none](-8.217,3.101)--(-8.234,3.111)--(-8.242,3.112)--(-8.246,3.11)--cycle;
\filldraw[fill opacity=0.8,fill=gray!20,draw=none](-8.273,3.114)--(-8.272,3.113)--(-8.227,3.108)--(-8.232,3.111)--cycle;
\draw(-8.273,3.114)--(-8.272,3.113);
\draw(-8.227,3.108)--(-8.232,3.111);
\filldraw[fill opacity=0.8,fill=gray!20,draw=none](-8.255,3.111)--(-8.227,3.108)--(-8.237,3.112)--cycle;
\draw(-8.227,3.108)--(-8.237,3.112);
\filldraw[fill opacity=0.8,fill=gray!20,draw=none](-8.277,3.115)--(-8.272,3.113)--(-8.255,3.111)--(-8.242,3.112)--cycle;
\draw(-8.277,3.115)--(-8.272,3.113);
\filldraw[fill opacity=0.8,fill=gray!20,draw=none](-8.277,3.115)--(-8.273,3.114)--(-8.232,3.111)--(-8.234,3.111)--cycle;
\draw(-8.277,3.115)--(-8.273,3.114);
\draw(-8.232,3.111)--(-8.234,3.111);
\filldraw[fill opacity=0.8,fill=gray!20,draw=none](-8.234,3.111)--(-8.239,3.114)--(-8.242,3.112)--cycle;
\filldraw[fill opacity=0.8,fill=gray!20,draw=none](-8.177,3.052)--(-8.181,3.072)--(-8.189,3.085)--(-8.255,3.126)--(-8.272,3.128)--(-8.293,3.121)--cycle;
\draw(-8.255,3.126)--(-8.272,3.128)--(-8.293,3.121);
\filldraw[fill opacity=0.8,fill=gray!20,draw=none](-8.313,3.2)--(-8.316,3.195)--(-8.311,3.193)--(-8.301,3.191)--cycle;
\draw(-8.313,3.2)--(-8.316,3.195);
\filldraw[fill opacity=0.8,fill=gray!20,draw=none](-8.311,3.193)--(-8.316,3.195)--(-8.316,3.194)--cycle;
\draw(-8.316,3.195)--(-8.316,3.194);
\filldraw[fill opacity=0.8,fill=gray!20](-7.769,.933)--(-7.749,.942)--(-7.796,.94)--(-7.793,.931)--cycle;
\filldraw[fill opacity=0.8,fill=gray!20,draw=none](-8.335,3.152)--(-8.337,3.148)--(-8.324,3.145)--cycle;
\draw(-8.335,3.152)--(-8.337,3.148);
\filldraw[fill opacity=0.8,fill=gray!20,draw=none](-6.874,1.008)--(-7.04,1.014)--(-7.035,1.055)--(-7.032,1.062)--(-6.868,1.056)--cycle;
\draw(-7.032,1.062)--(-6.868,1.056)--(-6.874,1.008)--(-7.04,1.014);
\filldraw[fill opacity=0.8,fill=gray!20,draw=none](-6.868,1.056)--(-7.032,1.062)--(-7.03,1.113)--(-6.867,1.107)--cycle;
\draw(-7.03,1.113)--(-6.867,1.107)--(-6.868,1.056)--(-7.032,1.062);
\filldraw[fill opacity=0.8,fill=gray!20,draw=none](-8.305,3.142)--(-8.337,3.148)--(-8.342,3.135)--cycle;
\draw(-8.337,3.148)--(-8.342,3.135);
\filldraw[fill opacity=0.8,fill=gray!20,draw=none](-8.301,3.191)--(-8.311,3.193)--(-8.294,3.187)--cycle;
\filldraw[fill opacity=0.8,fill=gray!20,draw=none](-7.85,.369)--(-7.826,.384)--(-7.828,.41)--(-7.831,.41)--cycle;
\draw(-7.828,.41)--(-7.831,.41);
\filldraw[fill opacity=0.8,fill=gray!20](-7.999,3.696)--(-8,3.743)--(-8.104,3.75)--(-8.092,3.703)--cycle;
\filldraw[fill opacity=0.8,fill=gray!20](-7.997,3.962)--(-7.994,4.006)--(-8.048,4.01)--(-8.073,3.968)--cycle;
\filldraw[fill opacity=0.8,fill=gray!20,draw=none](-8.301,3.191)--(-8.294,3.187)--(-8.268,3.177)--(-8.264,3.186)--cycle;
\draw(-8.268,3.177)--(-8.264,3.186);
\filldraw[fill opacity=0.8,fill=gray!20,draw=none](-7.73,4.47)--(-7.732,4.486)--(-7.736,4.476)--cycle;
\draw(-7.732,4.486)--(-7.736,4.476);
\filldraw[fill opacity=0.8,fill=gray!20,draw=none](-7.654,4.72)--(-7.643,4.717)--(-7.639,4.726)--cycle;
\draw(-7.643,4.717)--(-7.639,4.726);
\filldraw[fill opacity=0.8,fill=gray!20,draw=none](-6.867,1.107)--(-7.03,1.113)--(-7.032,1.16)--(-6.868,1.154)--cycle;
\draw(-7.032,1.16)--(-6.868,1.154)--(-6.867,1.107)--(-7.03,1.113);
\filldraw[fill opacity=0.8,fill=gray!20,draw=none](-8.405,3.185)--(-8.398,3.185)--(-8.404,3.19)--cycle;
\draw(-8.405,3.185)--(-8.398,3.185);
\filldraw[fill opacity=0.8,fill=gray!20,draw=none](-8.413,3.159)--(-8.407,3.156)--(-8.404,3.19)--(-8.422,3.197)--cycle;
\draw(-8.404,3.19)--(-8.422,3.197)--(-8.413,3.159)--(-8.407,3.156);
\filldraw[fill opacity=0.8,fill=gray!20,draw=none](-8.294,3.187)--(-8.271,3.17)--(-8.268,3.177)--cycle;
\draw(-8.271,3.17)--(-8.268,3.177);
\filldraw[fill opacity=0.8,fill=gray!20,draw=none](-7.553,4.706)--(-7.577,4.724)--(-7.583,4.71)--cycle;
\draw(-7.577,4.724)--(-7.583,4.71);
\filldraw[fill opacity=0.8,fill=gray!20,draw=none](-7.67,.981)--(-7.673,.98)--(-7.677,.98)--cycle;
\draw(-7.673,.98)--(-7.677,.98);
\filldraw[fill opacity=0.8,fill=gray!20,draw=none](-7.818,.456)--(-7.814,.456)--(-7.814,.464)--cycle;
\draw(-7.818,.456)--(-7.814,.456);
\filldraw[fill opacity=0.8,fill=gray!20,draw=none](-8.189,3.085)--(-8.208,3.113)--(-8.222,3.122)--(-8.255,3.126)--cycle;
\draw(-8.208,3.113)--(-8.222,3.122)--(-8.255,3.126);
\filldraw[fill opacity=0.8,fill=gray!20,draw=none](-8.232,3.111)--(-8.198,3.096)--(-8.194,3.087)--cycle;
\draw(-8.232,3.111)--(-8.198,3.096);
\filldraw[fill opacity=0.8,fill=gray!20,draw=none](-8.021,3.011)--(-8.029,3.026)--(-8.016,3.026)--(-8.015,3.008)--cycle;
\draw(-8.029,3.026)--(-8.016,3.026)--(-8.015,3.008);
\filldraw[fill opacity=0.8,fill=gray!20,draw=none](-8.043,3.025)--(-8.029,3.026)--(-8.021,3.011)--cycle;
\draw(-8.043,3.025)--(-8.029,3.026);
\filldraw[fill opacity=0.8,fill=gray!20,draw=none](-8.031,3.031)--(-8.029,3.026)--(-8.043,3.025)--(-8.057,3.034)--cycle;
\draw(-8.029,3.026)--(-8.043,3.025);
\filldraw[fill opacity=0.8,fill=gray!20,draw=none](-7.995,3)--(-7.998,3.001)--(-8.008,3.021)--(-7.946,3.008)--(-7.94,2.996)--cycle;
\draw(-7.946,3.008)--(-7.94,2.996)--(-7.995,3);
\filldraw[fill opacity=0.8,fill=gray!20,draw=none](-8.008,3.021)--(-7.998,3.001)--(-8.018,3.013)--(-8.019,3.023)--cycle;
\draw(-8.018,3.013)--(-8.019,3.023);
\filldraw[fill opacity=0.8,fill=gray!20,draw=none](-8.036,3.025)--(-8.019,3.023)--(-8.018,3.013)--cycle;
\draw(-8.019,3.023)--(-8.018,3.013);
\filldraw[fill opacity=0.8,fill=gray!20,draw=none](-8.198,3.096)--(-7.982,3.001)--(-7.944,2.978)--(-8.194,3.087)--cycle;
\draw(-8.198,3.096)--(-7.982,3.001)--(-7.944,2.978)--(-8.194,3.087);
\filldraw[fill opacity=0.8,fill=gray!20,draw=none](-8.232,3.111)--(-8.227,3.108)--(-8.199,3.091)--(-8.194,3.088)--(-8.194,3.087)--cycle;
\draw(-8.232,3.111)--(-8.227,3.108);
\filldraw[fill opacity=0.8,fill=gray!20,draw=none](-8.198,3.096)--(-7.982,3.001)--(-7.944,2.978)--(-8.194,3.087)--cycle;
\draw(-8.198,3.096)--(-7.982,3.001)--(-7.944,2.978)--(-8.194,3.087);
\filldraw[fill opacity=0.8,fill=gray!20,draw=none](-6.785,.419)--(-6.782,.442)--(-6.793,.475)--(-6.867,.461)--(-6.857,.405)--cycle;
\draw(-6.782,.442)--(-6.793,.475)--(-6.867,.461)--(-6.857,.405)--(-6.785,.419);
\filldraw[fill opacity=0.8,fill=gray!20](-7.551,4.79)--(-7.571,4.833)--(-7.627,4.83)--(-7.629,4.786)--cycle;
\filldraw[fill opacity=0.8,fill=gray!20,draw=none](-6.785,.419)--(-6.78,.43)--(-6.782,.442)--cycle;
\filldraw[fill opacity=0.8,fill=gray!20,draw=none](-7.689,4.424)--(-7.707,4.43)--(-7.708,4.445)--cycle;
\draw(-7.707,4.43)--(-7.708,4.445);
\filldraw[fill opacity=0.8,fill=gray!20,draw=none](-6.209,.417)--(-6.208,.416)--(-6.212,.454)--(-6.227,.454)--cycle;
\draw(-6.212,.454)--(-6.227,.454);
\filldraw[fill opacity=0.8,fill=gray!20,draw=none](-6.208,.416)--(-6.206,.415)--(-6.192,.454)--(-6.212,.454)--cycle;
\draw(-6.192,.454)--(-6.212,.454);
\filldraw[fill opacity=0.8,fill=gray!20,draw=none](-6.244,.546)--(-6.244,.473)--(-6.209,.435)--(-6.209,.537)--cycle;
\draw(-6.209,.435)--(-6.209,.537)--(-6.244,.546)--(-6.244,.473);
\filldraw[fill opacity=0.8,fill=gray!20,draw=none](-6.244,.473)--(-6.238,.454)--(-6.227,.454)--cycle;
\draw(-6.238,.454)--(-6.227,.454);
\filldraw[fill opacity=0.8,fill=gray!20,draw=none](-8.335,2.853)--(-8.365,2.872)--(-8.366,2.872)--(-8.365,2.856)--cycle;
\draw(-8.365,2.872)--(-8.366,2.872)--(-8.365,2.856);
\filldraw[fill opacity=0.8,fill=gray!20,draw=none](-8.35,2.865)--(-8.323,2.847)--(-8.317,2.847)--(-8.293,2.858)--cycle;
\draw(-8.35,2.865)--(-8.323,2.847)--(-8.317,2.847);
\filldraw[fill opacity=0.8,fill=gray!20,draw=none](-8.124,2.773)--(-8.147,2.787)--(-8.104,2.783)--cycle;
\filldraw[fill opacity=0.8,fill=gray!20,draw=none](-8.089,2.77)--(-8.096,2.774)--(-8.094,2.775)--(-8.083,2.774)--cycle;
\draw(-8.096,2.774)--(-8.094,2.775);
\filldraw[fill opacity=0.8,fill=gray!20,draw=none](-8.104,2.776)--(-8.094,2.775)--(-8.101,2.774)--cycle;
\draw(-8.094,2.775)--(-8.101,2.774);
\filldraw[fill opacity=0.8,fill=gray!20,draw=none](-8.089,2.77)--(-8.101,2.774)--(-8.096,2.774)--cycle;
\draw(-8.101,2.774)--(-8.096,2.774);
\filldraw[fill opacity=0.8,fill=gray!20,draw=none](-8.093,2.769)--(-8.104,2.776)--(-8.089,2.774)--(-8.091,2.77)--cycle;
\draw(-8.089,2.774)--(-8.091,2.77);
\filldraw[fill opacity=0.8,fill=gray!20,draw=none](-8.093,2.769)--(-8.107,2.765)--(-8.13,2.779)--(-8.104,2.776)--cycle;
\filldraw[fill opacity=0.8,fill=gray!20,draw=none](-8.089,2.77)--(-8.101,2.763)--(-8.115,2.771)--(-8.101,2.774)--cycle;
\draw(-8.115,2.771)--(-8.101,2.774);
\filldraw[fill opacity=0.8,fill=gray!20,draw=none](-8.079,2.772)--(-8.091,2.77)--(-8.089,2.774)--cycle;
\draw(-8.091,2.77)--(-8.089,2.774);
\filldraw[fill opacity=0.8,fill=gray!20,draw=none](-8.034,2.742)--(-8.063,2.741)--(-8.07,2.745)--(-8.078,2.761)--(-8.043,2.751)--cycle;
\draw(-8.034,2.742)--(-8.063,2.741);
\draw(-8.07,2.745)--(-8.078,2.761);
\filldraw[fill opacity=0.8,fill=gray!20,draw=none](-8.127,2.778)--(-8.104,2.776)--(-8.101,2.774)--(-8.115,2.771)--cycle;
\draw(-8.101,2.774)--(-8.115,2.771);
\filldraw[fill opacity=0.8,fill=gray!20,draw=none](-8.293,2.858)--(-8.272,2.856)--(-8.027,2.749)--(-8.072,2.754)--(-8.312,2.859)--cycle;
\draw(-8.272,2.856)--(-8.027,2.749)--(-8.072,2.754)--(-8.312,2.859);
\filldraw[fill opacity=0.8,fill=gray!20,draw=none](-8.293,2.858)--(-8.272,2.856)--(-8.027,2.749)--(-8.072,2.754)--(-8.312,2.859)--cycle;
\draw(-8.272,2.856)--(-8.027,2.749)--(-8.072,2.754)--(-8.312,2.859);
\filldraw[fill opacity=0.8,fill=gray!20,draw=none](-9.085,.831)--(-9.08,.834)--(-9.072,.809)--cycle;
\draw(-9.085,.831)--(-9.08,.834)--(-9.072,.809);
\filldraw[fill opacity=0.8,fill=gray!20,draw=none](-9.085,.831)--(-9.088,.836)--(-9.096,.884)--(-9.087,.891)--(-9.08,.834)--cycle;
\draw(-9.096,.884)--(-9.087,.891)--(-9.08,.834)--(-9.085,.831);
\filldraw[fill opacity=0.8,fill=gray!20,draw=none](-9.012,.803)--(-9.282,.921)--(-9.256,.882)--(-8.987,.764)--cycle;
\draw(-9.256,.882)--(-8.987,.764)--(-9.012,.803)--(-9.282,.921);
\filldraw[fill opacity=0.8,fill=gray!20](-6.162,.332)--(-6.152,.279)--(-6.124,.235)--(-6.081,.209)--(-6.031,.204)--(-5.981,.22)--(-5.938,.256)--(-5.91,.305)--(-5.9,.361)--(-5.91,.414)--(-5.938,.458)--(-5.981,.484)--(-6.031,.489)--(-6.081,.473)--(-6.124,.437)--(-6.152,.388)--cycle;
\filldraw[fill opacity=0.8,fill=gray!20,draw=none](-7.827,.414)--(-7.8,.447)--(-7.8,.455)--(-7.818,.456)--cycle;
\draw(-7.8,.447)--(-7.8,.455)--(-7.818,.456);
\filldraw[fill opacity=0.8,fill=gray!20,draw=none](-8.225,3.125)--(-8.223,3.127)--(-8.226,3.127)--cycle;
\filldraw[fill opacity=0.8,fill=gray!20,draw=none](-7.583,4.852)--(-7.589,4.84)--(-7.573,4.829)--(-7.534,4.823)--(-7.531,4.829)--cycle;
\draw(-7.534,4.823)--(-7.531,4.829)--(-7.583,4.852)--(-7.589,4.84);
\filldraw[fill opacity=0.8,fill=gray!20,draw=none](-7.589,4.84)--(-7.592,4.832)--(-7.573,4.829)--cycle;
\draw(-7.589,4.84)--(-7.592,4.832);
\filldraw[fill opacity=0.8,fill=gray!20,draw=none](-7.568,4.825)--(-7.573,4.829)--(-7.592,4.832)--cycle;
\filldraw[fill opacity=0.8,fill=gray!20,draw=none](-7.635,4.734)--(-7.639,4.726)--(-7.629,4.732)--cycle;
\draw(-7.635,4.734)--(-7.639,4.726);
\filldraw[fill opacity=0.8,fill=gray!20,draw=none](-7.602,4.719)--(-7.629,4.732)--(-7.639,4.726)--(-7.643,4.717)--cycle;
\draw(-7.639,4.726)--(-7.643,4.717);
\filldraw[fill opacity=0.8,fill=gray!20,draw=none](-6.206,.301)--(-6.209,.299)--(-6.209,.278)--(-6.159,.251)--cycle;
\draw(-6.209,.299)--(-6.209,.278);
\filldraw[fill opacity=0.8,fill=gray!20](-7.918,3.966)--(-7.938,4.009)--(-7.994,4.006)--(-7.997,3.962)--cycle;
\filldraw[fill opacity=0.8,fill=gray!20,draw=none](-8.272,2.868)--(-8.269,2.876)--(-8.296,2.875)--cycle;
\draw(-8.272,2.868)--(-8.269,2.876)--(-8.296,2.875);
\filldraw[fill opacity=0.8,fill=gray!20,draw=none](-7.629,4.732)--(-7.602,4.719)--(-7.579,4.721)--(-7.552,4.783)--cycle;
\draw(-7.579,4.721)--(-7.552,4.783);
\filldraw[fill opacity=0.8,fill=gray!20,draw=none](-9.27,.904)--(-9.294,.914)--(-9.266,.871)--(-9.241,.86)--cycle;
\draw(-9.266,.871)--(-9.241,.86)--(-9.27,.904)--(-9.294,.914);
\filldraw[fill opacity=0.8,fill=gray!20](-7.608,1.112)--(-7.614,1.158)--(-7.643,1.139)--(-7.638,1.092)--cycle;
\filldraw[fill opacity=0.8,fill=gray!20,draw=none](-8.396,3.141)--(-8.396,3.105)--(-8.356,3.126)--(-8.352,3.132)--(-8.393,3.15)--cycle;
\draw(-8.352,3.132)--(-8.393,3.15);
\filldraw[fill opacity=0.8,fill=gray!20,draw=none](-8.389,3.14)--(-8.364,3.138)--(-8.362,3.158)--(-8.398,3.185)--(-8.405,3.185)--cycle;
\draw(-8.389,3.14)--(-8.364,3.138)--(-8.362,3.158);
\draw(-8.398,3.185)--(-8.405,3.185);
\filldraw[fill opacity=0.8,fill=gray!20](-6.961,.509)--(-6.958,.553)--(-7.012,.557)--(-7.037,.514)--cycle;
\filldraw[fill opacity=0.8,fill=gray!20](-6.963,.243)--(-6.964,.289)--(-7.068,.297)--(-7.056,.249)--cycle;
\filldraw[fill opacity=0.8,fill=gray!20](-6.916,.883)--(-6.939,.914)--(-6.884,.917)--(-6.887,.884)--cycle;
\filldraw[fill opacity=0.8,fill=gray!20](-6.887,.884)--(-6.884,.917)--(-6.83,.913)--(-6.859,.882)--cycle;
\filldraw[fill opacity=0.8,fill=gray!20,draw=none](-6.788,.412)--(-6.776,.412)--(-6.78,.43)--cycle;
\draw(-6.788,.412)--(-6.776,.412);
\filldraw[fill opacity=0.8,fill=gray!20,draw=none](-7.573,4.829)--(-7.568,4.825)--(-7.537,4.816)--(-7.534,4.823)--cycle;
\draw(-7.537,4.816)--(-7.534,4.823);
\filldraw[fill opacity=0.8,fill=gray!20,draw=none](-7.479,4.542)--(-7.475,4.494)--(-7.471,4.466)--(-7.477,4.532)--cycle;
\draw(-7.479,4.542)--(-7.475,4.494);
\draw(-7.471,4.466)--(-7.477,4.532);
\filldraw[fill opacity=0.8,fill=gray!20,draw=none](-7.71,4.512)--(-7.697,4.51)--(-7.707,4.518)--cycle;
\filldraw[fill opacity=0.8,fill=gray!20,draw=none](-8.364,2.85)--(-8.364,2.851)--(-8.377,2.856)--cycle;
\draw(-8.364,2.85)--(-8.364,2.851);
\filldraw[fill opacity=0.8,fill=gray!20,draw=none](-7.568,4.825)--(-7.541,4.807)--(-7.537,4.816)--cycle;
\draw(-7.541,4.807)--(-7.537,4.816);
\filldraw[fill opacity=0.8,fill=gray!20,draw=none](-7.77,1)--(-7.773,1.002)--(-7.807,1.003)--(-7.799,.985)--(-7.778,.984)--cycle;
\draw(-7.773,1.002)--(-7.807,1.003)--(-7.799,.985)--(-7.778,.984);
\filldraw[fill opacity=0.8,fill=gray!20,draw=none](-7.773,.984)--(-7.756,.988)--(-7.74,1.007)--(-7.841,.984)--cycle;
\draw(-7.773,.984)--(-7.756,.988)--(-7.74,1.007)--(-7.841,.984);
\filldraw[fill opacity=0.8,fill=gray!20,draw=none](-7.116,.389)--(-7.104,.359)--(-7.073,.351)--(-7.068,.408)--(-7.105,.417)--cycle;
\draw(-7.104,.359)--(-7.073,.351)--(-7.068,.408)--(-7.105,.417);
\filldraw[fill opacity=0.8,fill=gray!20,draw=none](-7.826,.384)--(-7.801,.401)--(-7.801,.408)--(-7.828,.41)--cycle;
\draw(-7.801,.401)--(-7.801,.408)--(-7.828,.41);
\filldraw[fill opacity=0.8,fill=gray!20,draw=none](-9.173,.835)--(-9.182,.832)--(-9.166,.83)--cycle;
\draw(-9.182,.832)--(-9.166,.83);
\filldraw[fill opacity=0.8,fill=gray!20,draw=none](-7.015,1.215)--(-6.975,1.223)--(-6.977,1.215)--cycle;
\draw(-7.015,1.215)--(-6.975,1.223)--(-6.977,1.215);
\filldraw[fill opacity=0.8,fill=gray!20,draw=none](-7.602,4.719)--(-7.583,4.71)--(-7.579,4.721)--cycle;
\draw(-7.583,4.71)--(-7.579,4.721);
\filldraw[fill opacity=0.8,fill=gray!20,draw=none](-8.391,2.862)--(-8.364,2.851)--(-8.365,2.856)--cycle;
\draw(-8.364,2.851)--(-8.365,2.856);
\filldraw[fill opacity=0.8,fill=gray!20,draw=none](-8.362,3.158)--(-8.361,3.182)--(-8.398,3.185)--cycle;
\draw(-8.362,3.158)--(-8.361,3.182)--(-8.398,3.185);
\filldraw[fill opacity=0.8,fill=gray!20,draw=none](-7.827,.414)--(-7.831,.41)--(-7.828,.41)--cycle;
\draw(-7.831,.41)--(-7.828,.41);
\filldraw[fill opacity=0.8,fill=gray!20](-7.838,1.287)--(-7.79,1.293)--(-7.79,1.293)--(-7.83,1.292)--cycle;
\filldraw[fill opacity=0.8,fill=gray!20,draw=none](-7.827,.414)--(-7.828,.41)--(-7.801,.408)--(-7.8,.447)--cycle;
\draw(-7.828,.41)--(-7.801,.408)--(-7.8,.447);
\filldraw[fill opacity=0.8,fill=gray!20,draw=none](-7.484,4.597)--(-7.489,4.632)--(-7.509,4.628)--cycle;
\draw(-7.489,4.632)--(-7.509,4.628);
\filldraw[fill opacity=0.8,fill=gray!20,draw=none](-7.865,.949)--(-7.841,.943)--(-7.86,.963)--(-7.866,.963)--cycle;
\draw(-7.865,.949)--(-7.841,.943)--(-7.86,.963);
\filldraw[fill opacity=0.8,fill=gray!20,draw=none](-8.33,3.14)--(-8.285,3.142)--(-8.305,3.185)--(-8.361,3.182)--(-8.363,3.145)--cycle;
\draw(-8.33,3.14)--(-8.285,3.142)--(-8.305,3.185)--(-8.361,3.182)--(-8.363,3.145);
\filldraw[fill opacity=0.8,fill=gray!20,draw=none](-8.219,3.105)--(-8.198,3.096)--(-8.194,3.088)--cycle;
\draw(-8.219,3.105)--(-8.198,3.096);
\filldraw[fill opacity=0.8,fill=gray!20,draw=none](-7.931,1.217)--(-7.91,1.232)--(-7.92,1.242)--(-7.939,1.218)--cycle;
\draw(-7.91,1.232)--(-7.92,1.242)--(-7.939,1.218);
\filldraw[fill opacity=0.8,fill=gray!20](-7.714,.949)--(-7.682,.972)--(-7.733,.963)--(-7.749,.942)--cycle;
\filldraw[fill opacity=0.8,fill=gray!20,draw=none](-6.192,.416)--(-6.136,.426)--(-6.159,.453)--(-6.192,.467)--cycle;
\draw(-6.159,.453)--(-6.192,.467);
\filldraw[fill opacity=0.8,fill=gray!20,draw=none](-6.209,.299)--(-6.208,.3)--(-6.208,.301)--cycle;
\filldraw[fill opacity=0.8,fill=gray!20,draw=none](-8.411,3.045)--(-8.407,3.063)--(-8.457,3.036)--(-8.421,3.034)--cycle;
\draw(-8.457,3.036)--(-8.421,3.034);
\filldraw[fill opacity=0.8,fill=gray!20,draw=none](-8.512,2.99)--(-8.461,3.009)--(-8.446,3.037)--cycle;
\draw(-8.461,3.009)--(-8.446,3.037);
\filldraw[fill opacity=0.8,fill=gray!20,draw=none](-8.919,1.012)--(-8.914,1.009)--(-8.93,1.012)--(-9.065,1.071)--(-9.104,1.095)--(-8.935,1.022)--cycle;
\draw(-8.93,1.012)--(-9.065,1.071);
\draw(-9.104,1.095)--(-8.935,1.022);
\filldraw[fill opacity=0.8,fill=gray!20](-7.902,3.7)--(-7.892,3.748)--(-8,3.743)--(-7.999,3.696)--cycle;
\filldraw[fill opacity=0.8,fill=gray!20,draw=none](-8.391,2.862)--(-8.365,2.856)--(-8.366,2.872)--(-8.427,2.876)--cycle;
\draw(-8.365,2.856)--(-8.366,2.872)--(-8.427,2.876);
\filldraw[fill opacity=0.8,fill=gray!20,draw=none](-8.427,2.876)--(-8.371,2.872)--(-8.375,2.878)--(-8.401,2.893)--(-8.468,2.913)--(-8.462,2.891)--cycle;
\draw(-8.427,2.876)--(-8.371,2.872);
\draw(-8.468,2.913)--(-8.462,2.891);
\filldraw[fill opacity=0.8,fill=gray!20,draw=none](-6.208,.301)--(-6.208,.3)--(-6.206,.301)--(-6.207,.302)--cycle;
\filldraw[fill opacity=0.8,fill=gray!20,draw=none](-6.159,.251)--(-6.159,.173)--(-6.103,.172)--(-6.103,.218)--cycle;
\draw(-6.159,.251)--(-6.159,.173)--(-6.103,.172)--(-6.103,.218);
\filldraw[fill opacity=0.8,fill=gray!20,draw=none](-8.021,2.829)--(-8.029,2.811)--(-8.043,2.811)--cycle;
\draw(-8.029,2.811)--(-8.043,2.811);
\filldraw[fill opacity=0.8,fill=gray!20,draw=none](-8.057,2.796)--(-8.043,2.811)--(-8.016,2.812)--(-8.016,2.808)--cycle;
\draw(-8.043,2.811)--(-8.016,2.812)--(-8.016,2.808);
\filldraw[fill opacity=0.8,fill=gray!20,draw=none](-8.036,2.788)--(-8.018,2.806)--(-8.019,2.793)--cycle;
\draw(-8.018,2.806)--(-8.019,2.793);
\filldraw[fill opacity=0.8,fill=gray!20,draw=none](-7.98,2.782)--(-8.019,2.78)--(-7.971,2.802)--(-7.979,2.784)--cycle;
\draw(-7.971,2.802)--(-7.979,2.784);
\filldraw[fill opacity=0.8,fill=gray!20,draw=none](-7.98,2.782)--(-7.979,2.784)--(-7.98,2.782)--cycle;
\draw(-7.979,2.784)--(-7.98,2.782);
\filldraw[fill opacity=0.8,fill=gray!20,draw=none](-7.959,2.802)--(-7.979,2.784)--(-7.971,2.802)--cycle;
\draw(-7.979,2.784)--(-7.971,2.802);
\filldraw[fill opacity=0.8,fill=gray!20,draw=none](-8.168,2.894)--(-7.944,2.796)--(-7.982,2.764)--(-8.21,2.863)--cycle;
\draw(-8.168,2.894)--(-7.944,2.796)--(-7.982,2.764)--(-8.21,2.863);
\filldraw[fill opacity=0.8,fill=gray!20,draw=none](-8.168,2.894)--(-7.944,2.796)--(-7.982,2.764)--(-8.21,2.863)--cycle;
\draw(-8.168,2.894)--(-7.944,2.796)--(-7.982,2.764)--(-8.21,2.863);
\filldraw[fill opacity=0.8,fill=gray!20,draw=none](-8.356,3.126)--(-8.347,3.13)--(-8.352,3.132)--cycle;
\draw(-8.347,3.13)--(-8.352,3.132);
\filldraw[fill opacity=0.8,fill=gray!20,draw=none](-8.446,2.947)--(-8.431,2.941)--(-8.428,2.981)--cycle;
\draw(-8.446,2.947)--(-8.431,2.941);
\filldraw[fill opacity=0.8,fill=gray!20,draw=none](-7.736,4.52)--(-7.734,4.496)--(-7.718,4.322)--(-7.719,4.321)--(-7.737,4.52)--cycle;
\draw(-7.734,4.496)--(-7.718,4.322);
\draw(-7.719,4.321)--(-7.737,4.52);
\filldraw[fill opacity=0.8,fill=gray!20,draw=none](-6.79,.361)--(-6.213,.354)--(-6.23,.406)--(-6.788,.412)--cycle;
\draw(-6.79,.361)--(-6.213,.354);
\draw(-6.23,.406)--(-6.788,.412);
\filldraw[fill opacity=0.8,fill=gray!20,draw=none](-6.79,.361)--(-6.812,.414)--(-6.857,.405)--(-6.853,.348)--cycle;
\draw(-6.812,.414)--(-6.857,.405)--(-6.853,.348)--(-6.79,.361);
\filldraw[fill opacity=0.8,fill=gray!20,draw=none](-7.864,.969)--(-7.866,.966)--(-7.866,.963)--(-7.773,.984)--(-7.841,.984)--(-7.846,.983)--cycle;
\draw(-7.866,.963)--(-7.773,.984);
\draw(-7.841,.984)--(-7.846,.983);
\filldraw[fill opacity=0.8,fill=gray!20,draw=none](-7.86,.963)--(-7.862,.964)--(-7.906,.975)--(-7.889,.963)--cycle;
\draw(-7.86,.963)--(-7.862,.964)--(-7.906,.975)--(-7.889,.963);
\filldraw[fill opacity=0.8,fill=gray!20,draw=none](-6.208,.301)--(-6.207,.302)--(-6.207,.303)--cycle;
\filldraw[fill opacity=0.8,fill=gray!20,draw=none](-9.194,.848)--(-9.227,.851)--(-9.199,.834)--(-9.182,.832)--(-9.173,.835)--cycle;
\draw(-9.227,.851)--(-9.199,.834)--(-9.182,.832);
\filldraw[fill opacity=0.8,fill=gray!20](-6.882,.512)--(-6.903,.555)--(-6.958,.553)--(-6.961,.509)--cycle;
\filldraw[fill opacity=0.8,fill=gray!20,draw=none](-7.015,1.215)--(-7.044,1.215)--(-7.02,1.249)--(-6.96,1.26)--(-6.975,1.223)--cycle;
\draw(-7.044,1.215)--(-7.02,1.249)--(-6.96,1.26)--(-6.975,1.223)--(-7.015,1.215);
\filldraw[fill opacity=0.8,fill=gray!20](-7.748,1.291)--(-7.79,1.293)--(-7.79,1.293)--(-7.743,1.286)--cycle;
\filldraw[fill opacity=0.8,fill=gray!20,draw=none](-6.207,.303)--(-6.207,.307)--(-6.21,.307)--cycle;
\draw(-6.207,.307)--(-6.21,.307);
\filldraw[fill opacity=0.8,fill=gray!20,draw=none](-8.293,2.858)--(-8.312,2.859)--(-8.318,2.861)--cycle;
\draw(-8.312,2.859)--(-8.318,2.861);
\filldraw[fill opacity=0.8,fill=gray!20,draw=none](-8.293,2.858)--(-8.312,2.859)--(-8.318,2.861)--cycle;
\draw(-8.312,2.859)--(-8.318,2.861);
\filldraw[fill opacity=0.8,fill=gray!20,draw=none](-8.296,2.875)--(-8.269,2.876)--(-8.259,2.924)--(-8.367,2.919)--(-8.367,2.917)--cycle;
\draw(-8.296,2.875)--(-8.269,2.876)--(-8.259,2.924)--(-8.367,2.919)--(-8.367,2.917);
\filldraw[fill opacity=0.8,fill=gray!20,draw=none](-9.149,1.114)--(-9.173,1.125)--(-9.223,1.108)--(-9.199,1.098)--cycle;
\draw(-9.223,1.108)--(-9.199,1.098)--(-9.149,1.114)--(-9.173,1.125);
\filldraw[fill opacity=0.8,fill=gray!20](-7.614,1.158)--(-7.632,1.201)--(-7.658,1.185)--(-7.643,1.139)--cycle;
\filldraw[fill opacity=0.8,fill=gray!20](-7.906,1.228)--(-7.872,1.26)--(-7.882,1.27)--(-7.92,1.242)--cycle;
\filldraw[fill opacity=0.8,fill=gray!20,draw=none](-9.079,1.092)--(-9.071,1.092)--(-9.084,1.1)--cycle;
\draw(-9.071,1.092)--(-9.084,1.1);
\filldraw[fill opacity=0.8,fill=gray!20](-6.867,.247)--(-6.857,.294)--(-6.964,.289)--(-6.963,.243)--cycle;
\filldraw[fill opacity=0.8,fill=gray!20](-6.938,.878)--(-6.982,.906)--(-6.939,.914)--(-6.916,.883)--cycle;
\filldraw[fill opacity=0.8,fill=gray!20,draw=none](-6.103,.218)--(-6.103,.197)--(-6.077,.203)--cycle;
\draw(-6.103,.218)--(-6.103,.197);
\filldraw[fill opacity=0.8,fill=gray!20,draw=none](-8.227,3.108)--(-8.219,3.105)--(-8.199,3.091)--cycle;
\draw(-8.227,3.108)--(-8.219,3.105);
\filldraw[fill opacity=0.8,fill=gray!20,draw=none](-8.203,3.106)--(-8.194,3.105)--(-8.208,3.113)--cycle;
\draw(-8.194,3.105)--(-8.208,3.113);
\filldraw[fill opacity=0.8,fill=gray!20,draw=none](-6.21,.307)--(-6.207,.307)--(-6.213,.354)--(-6.243,.355)--cycle;
\draw(-6.21,.307)--(-6.207,.307);
\draw(-6.213,.354)--(-6.243,.355);
\filldraw[fill opacity=0.8,fill=gray!20,draw=none](-8.439,3.092)--(-8.402,3.089)--(-8.396,3.1)--(-8.396,3.105)--(-8.437,3.1)--cycle;
\draw(-8.439,3.092)--(-8.402,3.089);
\filldraw[fill opacity=0.8,fill=gray!20,draw=none](-8.356,3.126)--(-8.401,3.102)--(-8.378,3.092)--cycle;
\draw(-8.401,3.102)--(-8.378,3.092);
\filldraw[fill opacity=0.8,fill=gray!20,draw=none](-6.207,.413)--(-6.209,.413)--(-6.207,.412)--cycle;
\draw(-6.209,.413)--(-6.207,.412);
\filldraw[fill opacity=0.8,fill=gray!20,draw=none](-6.208,.416)--(-6.209,.417)--(-6.209,.413)--cycle;
\draw(-6.209,.417)--(-6.209,.413);
\filldraw[fill opacity=0.8,fill=gray!20,draw=none](-6.208,.416)--(-6.209,.413)--(-6.206,.415)--cycle;
\filldraw[fill opacity=0.8,fill=gray!20,draw=none](-6.23,.406)--(-6.207,.406)--(-6.208,.414)--(-6.227,.454)--(-6.238,.454)--cycle;
\draw(-6.23,.406)--(-6.207,.406);
\draw(-6.227,.454)--(-6.238,.454);
\filldraw[fill opacity=0.8,fill=gray!20,draw=none](-7.689,.999)--(-7.676,.981)--(-7.658,1.006)--(-7.689,1)--cycle;
\draw(-7.676,.981)--(-7.658,1.006)--(-7.689,1);
\filldraw[fill opacity=0.8,fill=gray!20,draw=none](-7.773,1.002)--(-7.77,1)--(-7.74,1.007)--(-7.731,1.042)--(-7.768,1.034)--cycle;
\draw(-7.77,1)--(-7.74,1.007)--(-7.731,1.042)--(-7.768,1.034);
\filldraw[fill opacity=0.8,fill=gray!20,draw=none](-7.67,.996)--(-7.669,.998)--(-7.769,1.002)--(-7.778,.984)--(-7.771,.984)--cycle;
\draw(-7.669,.998)--(-7.769,1.002);
\draw(-7.778,.984)--(-7.771,.984);
\filldraw[fill opacity=0.8,fill=gray!20,draw=none](-6.227,.454)--(-6.209,.417)--(-6.209,.435)--cycle;
\draw(-6.209,.417)--(-6.209,.435);
\filldraw[fill opacity=0.5,fill=gray!20](-9.831,2.386)--(-10.292,2.587)--(-9.891,2.812)--(-9.43,2.611)--cycle;
\filldraw[fill opacity=0.5,fill=gray!20](-10.281,2.635)--(-10.292,2.587)--(-9.891,2.812)--(-9.862,2.871)--cycle;
\filldraw[fill opacity=0.8,fill=gray!20,draw=none](-9.104,1.095)--(-9.079,1.092)--(-9.084,1.1)--(-9.099,1.109)--(-9.132,1.112)--cycle;
\draw(-9.084,1.1)--(-9.099,1.109)--(-9.132,1.112);
\filldraw[fill opacity=0.8,fill=gray!20,draw=none](-8.43,3.143)--(-8.431,3.158)--(-8.44,3.144)--cycle;
\draw(-8.431,3.158)--(-8.44,3.144)--(-8.43,3.143);
\filldraw[fill opacity=0.8,fill=gray!20,draw=none](-7.67,.996)--(-7.771,.984)--(-7.677,.98)--cycle;
\draw(-7.771,.984)--(-7.677,.98);
\filldraw[fill opacity=0.8,fill=gray!20,draw=none](-8.461,3.037)--(-8.457,3.036)--(-8.407,3.063)--(-8.405,3.072)--(-8.405,3.089)--(-8.439,3.092)--cycle;
\draw(-8.461,3.037)--(-8.457,3.036);
\draw(-8.405,3.089)--(-8.439,3.092);
\filldraw[fill opacity=0.8,fill=gray!20,draw=none](-8.424,3.089)--(-8.418,3.109)--(-8.416,3.13)--(-8.431,3.158)--cycle;
\draw(-8.424,3.089)--(-8.418,3.109)--(-8.416,3.13);
\filldraw[fill opacity=0.8,fill=gray!20,draw=none](-6.207,.413)--(-6.192,.416)--(-6.192,.454)--cycle;
\filldraw[fill opacity=0.8,fill=gray!20,draw=none](-7.735,4.574)--(-7.737,4.574)--(-7.736,4.572)--cycle;
\draw(-7.735,4.574)--(-7.737,4.574)--(-7.736,4.572);
\filldraw[fill opacity=0.8,fill=gray!20,draw=none](-8.12,3.027)--(-7.918,2.939)--(-7.909,2.89)--(-8.13,2.987)--cycle;
\draw(-8.12,3.027)--(-7.918,2.939)--(-7.909,2.89)--(-8.13,2.987);
\filldraw[fill opacity=0.5,fill=gray!20](-11.066,.464)--(-10.888,.535)--(-10.938,.947)--(-11.123,.927)--cycle;
\filldraw[fill opacity=0.5,fill=gray!20](-10.888,.535)--(-10.715,.46)--(-10.765,.871)--(-10.938,.947)--cycle;
\filldraw[fill opacity=0.8,fill=gray!20](-7.862,.964)--(-7.878,.996)--(-7.932,1.01)--(-7.906,.975)--cycle;
\filldraw[fill opacity=0.8,fill=gray!20,draw=none](-7.632,1.201)--(-7.661,1.238)--(-7.662,1.238)--(-7.67,1.233)--(-7.677,1.217)--(-7.676,1.214)--(-7.658,1.185)--cycle;
\draw(-7.662,1.238)--(-7.67,1.233);
\draw(-7.676,1.214)--(-7.658,1.185)--(-7.632,1.201)--(-7.661,1.238);
\filldraw[fill opacity=0.8,fill=gray!20,draw=none](-7.04,1.196)--(-7.035,1.166)--(-7.067,1.16)--(-7.051,1.204)--cycle;
\draw(-7.035,1.166)--(-7.067,1.16)--(-7.051,1.204);
\filldraw[fill opacity=0.8,fill=gray!20,draw=none](-7.67,1.233)--(-7.662,1.238)--(-7.669,1.238)--cycle;
\draw(-7.67,1.233)--(-7.662,1.238);
\filldraw[fill opacity=0.8,fill=gray!20,draw=none](-7.813,1.223)--(-7.786,1.22)--(-7.779,1.221)--(-7.792,1.232)--cycle;
\draw(-7.786,1.22)--(-7.779,1.221)--(-7.792,1.232);
\filldraw[fill opacity=0.8,fill=gray!20,draw=none](-7.102,1.137)--(-7.08,1.188)--(-7.051,1.207)--(-7.051,1.204)--(-7.067,1.16)--cycle;
\draw(-7.051,1.204)--(-7.067,1.16)--(-7.102,1.137)--(-7.08,1.188)--(-7.051,1.207);
\filldraw[fill opacity=0.8,fill=gray!20,draw=none](-7.07,1.178)--(-7.065,1.193)--(-7.066,1.205)--(-7.08,1.188)--cycle;
\draw(-7.066,1.205)--(-7.08,1.188)--(-7.07,1.178);
\filldraw[fill opacity=0.8,fill=gray!20](-6.89,1.207)--(-7.79,1.243)--(-7.799,1.222)--(-6.899,1.187)--cycle;
\filldraw[fill opacity=0.8,fill=gray!20](-6.859,.882)--(-6.83,.913)--(-6.792,.904)--(-6.839,.877)--cycle;
\filldraw[fill opacity=0.8,fill=gray!20](-6.785,1.22)--(-6.805,1.258)--(-6.751,1.245)--(-6.72,1.204)--cycle;
\filldraw[fill opacity=0.8,fill=gray!20,draw=none](-8.406,2.894)--(-8.406,2.922)--(-8.471,2.926)--(-8.468,2.913)--cycle;
\draw(-8.406,2.922)--(-8.471,2.926)--(-8.468,2.913);
\filldraw[fill opacity=0.8,fill=gray!20](-7.872,1.26)--(-7.833,1.282)--(-7.838,1.287)--(-7.882,1.27)--cycle;
\filldraw[fill opacity=0.8,fill=gray!20,draw=none](-7.735,4.574)--(-7.726,4.604)--(-7.738,4.592)--(-7.737,4.574)--cycle;
\draw(-7.738,4.592)--(-7.737,4.574)--(-7.735,4.574);
\filldraw[fill opacity=0.8,fill=gray!20](-7.632,4.734)--(-7.629,4.786)--(-7.706,4.792)--(-7.725,4.741)--cycle;
\filldraw[fill opacity=0.8,fill=gray!20,draw=none](-9.021,.851)--(-9.291,.969)--(-9.282,.921)--(-9.012,.803)--cycle;
\draw(-9.282,.921)--(-9.012,.803)--(-9.021,.851)--(-9.291,.969);
\filldraw[fill opacity=0.8,fill=gray!20,draw=none](-9.232,.871)--(-9.218,.85)--(-9.194,.848)--cycle;
\filldraw[fill opacity=0.8,fill=gray!20,draw=none](-6.722,1.049)--(-6.701,1.044)--(-6.707,1.027)--cycle;
\draw(-6.722,1.049)--(-6.701,1.044)--(-6.707,1.027);
\filldraw[fill opacity=0.8,fill=gray!20,draw=none](-8.318,2.861)--(-8.312,2.859)--(-8.336,2.876)--(-8.356,2.885)--cycle;
\draw(-8.318,2.861)--(-8.312,2.859);
\draw(-8.336,2.876)--(-8.356,2.885);
\filldraw[fill opacity=0.8,fill=gray!20,draw=none](-8.318,2.861)--(-8.312,2.859)--(-8.336,2.876)--(-8.356,2.885)--cycle;
\draw(-8.318,2.861)--(-8.312,2.859);
\draw(-8.336,2.876)--(-8.356,2.885);
\filldraw[fill opacity=0.8,fill=gray!20,draw=none](-6.719,1.106)--(-6.694,1.1)--(-6.699,1.06)--cycle;
\draw(-6.719,1.106)--(-6.694,1.1)--(-6.699,1.06);
\filldraw[fill opacity=0.8,fill=gray!20,draw=none](-6.709,1.001)--(-6.713,.983)--(-6.72,.99)--(-6.707,1.027)--cycle;
\draw(-6.713,.983)--(-6.72,.99)--(-6.707,1.027);
\filldraw[fill opacity=0.8,fill=gray!20,draw=none](-6.709,1.001)--(-6.707,1.027)--(-6.701,1.044)--(-6.698,1.042)--cycle;
\draw(-6.707,1.027)--(-6.701,1.044)--(-6.698,1.042);
\filldraw[fill opacity=0.8,fill=gray!20,draw=none](-6.698,1.049)--(-6.699,1.042)--(-6.701,1.044)--(-6.699,1.06)--cycle;
\draw(-6.699,1.042)--(-6.701,1.044)--(-6.699,1.06);
\filldraw[fill opacity=0.8,fill=gray!20,draw=none](-6.698,1.042)--(-6.699,1.042)--(-6.698,1.049)--cycle;
\draw(-6.698,1.042)--(-6.699,1.042);
\filldraw[fill opacity=0.8,fill=gray!20,draw=none](-6.868,1.056)--(-6.69,1.048)--(-6.687,1)--(-6.874,1.008)--cycle;
\draw(-6.687,1)--(-6.874,1.008)--(-6.868,1.056)--(-6.69,1.048);
\filldraw[fill opacity=0.8,fill=gray!20,draw=none](-6.698,1.049)--(-6.699,1.06)--(-6.694,1.1)--cycle;
\draw(-6.699,1.06)--(-6.694,1.1);
\filldraw[fill opacity=0.8,fill=gray!20,draw=none](-6.867,1.107)--(-6.694,1.1)--(-6.693,1.049)--(-6.868,1.056)--cycle;
\draw(-6.693,1.049)--(-6.868,1.056)--(-6.867,1.107)--(-6.694,1.1);
\filldraw[fill opacity=0.8,fill=gray!20](-6.914,1.05)--(-6.912,1.102)--(-6.907,1.149)--(-6.899,1.187)--(-6.89,1.207)--(-6.881,1.208)--(-6.874,1.189)--(-6.868,1.154)--(-6.867,1.107)--(-6.868,1.056)--(-6.874,1.008)--(-6.881,.971)--(-6.89,.95)--(-6.899,.949)--(-6.907,.968)--(-6.912,1.003)--cycle;
\filldraw[fill opacity=0.8,fill=gray!20](-6.914,1.05)--(-6.912,1.003)--(-6.907,.968)--(-6.899,.949)--(-6.89,.95)--(-6.881,.971)--(-6.874,1.008)--(-6.868,1.056)--(-6.867,1.107)--(-6.868,1.154)--(-6.874,1.189)--(-6.881,1.208)--(-6.89,1.207)--(-6.899,1.187)--(-6.907,1.149)--(-6.912,1.102)--cycle;
\filldraw[fill opacity=0.8,fill=gray!20,draw=none](-7.804,1.241)--(-7.803,1.241)--(-7.794,1.234)--cycle;
\draw(-7.804,1.241)--(-7.803,1.241)--(-7.794,1.234);
\filldraw[fill opacity=0.8,fill=gray!20](-8,3.743)--(-8,3.797)--(-8.108,3.805)--(-8.104,3.75)--cycle;
\filldraw[fill opacity=0.8,fill=gray!20](-7.999,3.91)--(-7.997,3.962)--(-8.073,3.968)--(-8.092,3.917)--cycle;
\filldraw[fill opacity=0.8,fill=gray!20,draw=none](-6.698,1.147)--(-6.694,1.1)--(-6.699,1.139)--cycle;
\draw(-6.694,1.1)--(-6.699,1.139);
\filldraw[fill opacity=0.8,fill=gray!20,draw=none](-6.698,1.147)--(-6.699,1.139)--(-6.701,1.155)--(-6.699,1.153)--cycle;
\draw(-6.699,1.139)--(-6.701,1.155)--(-6.699,1.153);
\filldraw[fill opacity=0.8,fill=gray!20,draw=none](-6.868,1.154)--(-6.673,1.146)--(-6.694,1.1)--(-6.867,1.107)--cycle;
\draw(-6.694,1.1)--(-6.867,1.107)--(-6.868,1.154)--(-6.673,1.146);
\filldraw[fill opacity=0.8,fill=gray!20,draw=none](-7.051,.993)--(-7.08,.974)--(-7.102,1.026)--(-7.067,1.049)--(-7.051,.999)--cycle;
\draw(-7.051,.993)--(-7.08,.974)--(-7.102,1.026)--(-7.067,1.049)--(-7.051,.999);
\filldraw[fill opacity=0.8,fill=gray!20](-7.102,1.026)--(-7.109,1.081)--(-7.073,1.105)--(-7.067,1.049)--cycle;
\filldraw[fill opacity=0.8,fill=gray!20,draw=none](-7.881,1.175)--(-7.886,1.197)--(-7.914,1.216)--(-7.932,1.188)--cycle;
\draw(-7.914,1.216)--(-7.932,1.188)--(-7.881,1.175);
\filldraw[fill opacity=0.8,fill=gray!20,draw=none](-7.888,1.128)--(-7.879,1.171)--(-7.881,1.175)--(-7.932,1.188)--(-7.948,1.143)--cycle;
\draw(-7.881,1.175)--(-7.932,1.188)--(-7.948,1.143)--(-7.888,1.128)--(-7.879,1.171);
\filldraw[fill opacity=0.8,fill=gray!20,draw=none](-8.74,1.035)--(-8.593,1.068)--(-8.665,1.037)--(-8.781,1.011)--cycle;
\draw(-8.74,1.035)--(-8.593,1.068);
\draw(-8.665,1.037)--(-8.781,1.011);
\filldraw[fill opacity=0.8,fill=gray!20,draw=none](-8.795,.996)--(-8.812,.998)--(-8.803,1.006)--(-8.798,1.007)--cycle;
\draw(-8.803,1.006)--(-8.798,1.007);
\filldraw[fill opacity=0.8,fill=gray!20,draw=none](-8.804,.99)--(-8.798,1.004)--(-8.793,.983)--cycle;
\draw(-8.798,1.004)--(-8.793,.983);
\filldraw[fill opacity=0.8,fill=gray!20,draw=none](-8.919,1.012)--(-8.933,1.021)--(-8.892,1.016)--(-8.892,1.013)--cycle;
\draw(-8.892,1.016)--(-8.892,1.013)--(-8.919,1.012);
\filldraw[fill opacity=0.8,fill=gray!20,draw=none](-8.914,1.008)--(-8.919,1.012)--(-8.892,1.013)--cycle;
\draw(-8.919,1.012)--(-8.892,1.013);
\filldraw[fill opacity=0.8,fill=gray!20,draw=none](-8.919,1.012)--(-8.914,1.008)--(-8.914,1.009)--cycle;
\filldraw[fill opacity=0.8,fill=gray!20,draw=none](-8.974,.997)--(-8.989,1.003)--(-8.988,1.009)--(-8.919,1.012)--(-8.914,1.008)--cycle;
\draw(-8.989,1.003)--(-8.988,1.009)--(-8.919,1.012);
\filldraw[fill opacity=0.8,fill=gray!20,draw=none](-8.794,.99)--(-8.798,1.004)--(-8.798,1.006)--(-8.788,1.003)--(-8.762,.984)--cycle;
\draw(-8.794,.99)--(-8.798,1.004);
\draw(-8.798,1.006)--(-8.788,1.003);
\filldraw[fill opacity=0.8,fill=gray!20,draw=none](-8.797,1.008)--(-8.74,1.035)--(-8.781,1.011)--(-8.798,1.007)--cycle;
\draw(-8.781,1.011)--(-8.798,1.007);
\filldraw[fill opacity=0.8,fill=gray!20,draw=none](-8.788,1.003)--(-8.799,1.006)--(-8.801,1.01)--cycle;
\draw(-8.788,1.003)--(-8.799,1.006)--(-8.801,1.01);
\filldraw[fill opacity=0.8,fill=gray!20,draw=none](-8.831,1.001)--(-8.858,1.006)--(-8.855,1.014)--cycle;
\draw(-8.858,1.006)--(-8.855,1.014);
\filldraw[fill opacity=0.8,fill=gray!20,draw=none](-8.836,1.02)--(-8.817,1.023)--(-8.806,1.015)--(-8.808,1.012)--cycle;
\filldraw[fill opacity=0.8,fill=gray!20,draw=none](-8.817,1.023)--(-8.794,1.028)--(-8.806,1.015)--cycle;
\filldraw[fill opacity=0.8,fill=gray!20,draw=none](-8.82,.999)--(-8.806,1.015)--(-8.801,1.01)--(-8.803,1.006)--cycle;
\draw(-8.801,1.01)--(-8.803,1.006);
\filldraw[fill opacity=0.8,fill=gray!20,draw=none](-8.806,1.015)--(-8.792,1.029)--(-8.801,1.01)--cycle;
\draw(-8.792,1.029)--(-8.801,1.01);
\filldraw[fill opacity=0.8,fill=gray!20,draw=none](-8.837,1.02)--(-8.807,1.024)--(-8.801,1.01)--cycle;
\draw(-8.807,1.024)--(-8.801,1.01);
\filldraw[fill opacity=0.8,fill=gray!20,draw=none](-8.812,.998)--(-8.815,.998)--(-8.803,1.006)--cycle;
\filldraw[fill opacity=0.8,fill=gray!20,draw=none](-8.888,1.01)--(-8.854,1.017)--(-8.858,1.006)--cycle;
\draw(-8.854,1.017)--(-8.858,1.006);
\filldraw[fill opacity=0.8,fill=gray!20,draw=none](-8.797,1.008)--(-8.798,1.007)--(-8.803,1.006)--cycle;
\draw(-8.798,1.007)--(-8.803,1.006);
\filldraw[fill opacity=0.8,fill=gray!20,draw=none](-8.82,.995)--(-8.803,1.006)--(-8.81,.989)--cycle;
\draw(-8.803,1.006)--(-8.81,.989);
\filldraw[fill opacity=0.8,fill=gray!20,draw=none](-8.815,.998)--(-8.833,.999)--(-8.803,1.006)--cycle;
\draw(-8.833,.999)--(-8.803,1.006);
\filldraw[fill opacity=0.8,fill=gray!20,draw=none](-8.779,.969)--(-8.81,.989)--(-8.792,1.029)--(-8.766,1.066)--(-8.73,1.057)--(-8.765,.977)--cycle;
\draw(-8.81,.989)--(-8.792,1.029);
\draw(-8.73,1.057)--(-8.765,.977);
\filldraw[fill opacity=0.8,fill=gray!20,draw=none](-8.797,1.008)--(-8.788,1.025)--(-8.74,1.035)--cycle;
\draw(-8.788,1.025)--(-8.74,1.035);
\filldraw[fill opacity=0.8,fill=gray!20,draw=none](-8.677,1.044)--(-7.965,1.204)--(-7.947,1.213)--(-8.74,1.035)--cycle;
\draw(-8.677,1.044)--(-7.965,1.204);
\draw(-7.947,1.213)--(-8.74,1.035);
\filldraw[fill opacity=0.8,fill=gray!20,draw=none](-8.758,.983)--(-8.762,.984)--(-8.74,1.035)--(-8.702,1.039)--(-8.704,1.033)--cycle;
\draw(-8.762,.984)--(-8.74,1.035);
\draw(-8.702,1.039)--(-8.704,1.033);
\filldraw[fill opacity=0.8,fill=gray!20,draw=none](-8.758,.983)--(-8.741,1)--(-8.728,.978)--cycle;
\filldraw[fill opacity=0.8,fill=gray!20,draw=none](-8.741,1)--(-8.704,1.033)--(-8.728,.978)--cycle;
\draw(-8.704,1.033)--(-8.728,.978);
\filldraw[fill opacity=0.8,fill=gray!20,draw=none](-8.788,1.003)--(-8.801,1.01)--(-8.807,1.024)--(-8.76,1.026)--(-8.733,.99)--cycle;
\draw(-8.801,1.01)--(-8.807,1.024);
\draw(-8.76,1.026)--(-8.733,.99)--(-8.788,1.003);
\filldraw[fill opacity=0.8,fill=gray!20,draw=none](-8.885,1)--(-8.888,1.002)--(-8.788,1.025)--(-8.797,1.008)--(-8.803,1.006)--(-8.833,.999)--cycle;
\draw(-8.888,1.002)--(-8.788,1.025);
\draw(-8.803,1.006)--(-8.833,.999);
\filldraw[fill opacity=0.8,fill=gray!20,draw=none](-8.742,.969)--(-8.788,1.003)--(-8.733,.99)--(-8.728,.978)--cycle;
\draw(-8.788,1.003)--(-8.733,.99)--(-8.728,.978);
\filldraw[fill opacity=0.8,fill=gray!20,draw=none](-8.844,.839)--(-8.829,.887)--(-8.825,.932)--(-8.833,.967)--(-8.852,.986)--(-8.879,.986)--(-8.909,.968)--(-8.938,.934)--(-8.962,.89)--cycle;
\draw(-8.844,.839)--(-8.829,.887)--(-8.825,.932)--(-8.833,.967)--(-8.852,.986)--(-8.879,.986)--(-8.909,.968)--(-8.938,.934)--(-8.962,.89);
\filldraw[fill opacity=0.8,fill=gray!20,draw=none](-8.905,1.006)--(-8.914,1.008)--(-8.935,1.022)--(-8.918,1.014)--cycle;
\draw(-8.935,1.022)--(-8.918,1.014);
\filldraw[fill opacity=0.8,fill=gray!20,draw=none](-8.919,1.012)--(-8.988,1.009)--(-8.981,1.027)--(-8.933,1.021)--cycle;
\draw(-8.919,1.012)--(-8.988,1.009)--(-8.981,1.027);
\filldraw[fill opacity=0.8,fill=gray!20,draw=none](-8.837,1.02)--(-8.892,1.013)--(-8.893,1.036)--cycle;
\draw(-8.892,1.013)--(-8.893,1.036);
\filldraw[fill opacity=0.8,fill=gray!20,draw=none](-8.945,1.026)--(-9.104,1.095)--(-9.149,1.1)--(-8.981,1.027)--cycle;
\draw(-8.945,1.026)--(-9.104,1.095);
\draw(-9.149,1.1)--(-8.981,1.027);
\filldraw[fill opacity=0.8,fill=gray!20,draw=none](-8.933,1.021)--(-8.945,1.026)--(-8.981,1.027)--cycle;
\draw(-8.933,1.021)--(-8.945,1.026);
\filldraw[fill opacity=0.8,fill=gray!20,draw=none](-8.981,1.027)--(-8.973,1.046)--(-8.894,1.05)--(-8.892,1.016)--cycle;
\draw(-8.981,1.027)--(-8.973,1.046)--(-8.894,1.05)--(-8.892,1.016);
\filldraw[fill opacity=0.8,fill=gray!20,draw=none](-8.794,1.028)--(-8.854,1.017)--(-8.179,2.567)--(-8.131,2.549)--(-8.792,1.029)--cycle;
\draw(-8.854,1.017)--(-8.179,2.567);
\draw(-8.131,2.549)--(-8.792,1.029);
\filldraw[fill opacity=0.8,fill=gray!20,draw=none](-8.851,1.024)--(-8.893,1.036)--(-8.894,1.05)--(-8.889,1.049)--cycle;
\draw(-8.893,1.036)--(-8.894,1.05)--(-8.889,1.049);
\filldraw[fill opacity=0.8,fill=gray!20,draw=none](-8.912,1.005)--(-8.892,1.05)--(-8.889,1.049)--(-8.851,1.024)--(-8.854,1.017)--cycle;
\draw(-8.912,1.005)--(-8.892,1.05);
\draw(-8.851,1.024)--(-8.854,1.017);
\filldraw[fill opacity=0.8,fill=gray!20,draw=none](-8.889,1.049)--(-8.892,1.051)--(-8.307,2.395)--(-8.261,2.378)--(-8.847,1.033)--cycle;
\draw(-8.892,1.051)--(-8.307,2.395);
\draw(-8.261,2.378)--(-8.847,1.033);
\filldraw[fill opacity=0.8,fill=gray!20,draw=none](-8.892,1.05)--(-8.892,1.051)--(-8.889,1.049)--cycle;
\draw(-8.892,1.05)--(-8.892,1.051);
\filldraw[fill opacity=0.8,fill=gray!20,draw=none](-8.891,1.049)--(-8.894,1.05)--(-8.894,1.052)--cycle;
\draw(-8.891,1.049)--(-8.894,1.05)--(-8.894,1.052);
\filldraw[fill opacity=0.8,fill=gray!20,draw=none](-8.894,1.052)--(-8.925,1.073)--(-8.653,1.699)--(-8.635,1.642)--(-8.892,1.051)--cycle;
\draw(-8.925,1.073)--(-8.653,1.699);
\draw(-8.635,1.642)--(-8.892,1.051);
\filldraw[fill opacity=0.8,fill=gray!20,draw=none](-8.93,1.062)--(-8.925,1.073)--(-8.894,1.052)--cycle;
\draw(-8.93,1.062)--(-8.925,1.073);
\filldraw[fill opacity=0.8,fill=gray!20,draw=none](-8.935,1.025)--(-8.897,1.04)--(-8.91,1.009)--cycle;
\draw(-8.897,1.04)--(-8.91,1.009);
\filldraw[fill opacity=0.8,fill=gray!20,draw=none](-8.905,1.006)--(-8.91,1.009)--(-8.916,1.013)--(-8.894,1.004)--cycle;
\draw(-8.916,1.013)--(-8.894,1.004);
\filldraw[fill opacity=0.8,fill=gray!20,draw=none](-8.96,.993)--(-8.95,1.016)--(-8.916,1.013)--(-8.91,1.009)--(-8.912,1.005)--cycle;
\draw(-8.96,.993)--(-8.95,1.016);
\draw(-8.91,1.009)--(-8.912,1.005);
\filldraw[fill opacity=0.8,fill=gray!20,draw=none](-8.91,1.009)--(-8.918,1.014)--(-8.916,1.013)--cycle;
\draw(-8.918,1.014)--(-8.916,1.013);
\filldraw[fill opacity=0.8,fill=gray!20,draw=none](-9.08,.945)--(-9.072,.968)--(-9.064,.98)--(-9.041,.998)--(-8.988,1.009)--(-8.998,.961)--cycle;
\draw(-9.041,.998)--(-8.988,1.009)--(-8.998,.961)--(-9.08,.945)--(-9.072,.968);
\filldraw[fill opacity=0.8,fill=gray!20,draw=none](-9.02,.915)--(-8.983,.999)--(-8.955,1.016)--(-8.95,1.016)--(-8.998,.906)--cycle;
\draw(-8.95,1.016)--(-8.998,.906)--(-9.02,.915)--(-8.983,.999);
\filldraw[fill opacity=0.8,fill=gray!20,draw=none](-8.858,.988)--(-8.894,1.004)--(-8.905,.994)--(-8.903,.993)--cycle;
\draw(-8.905,.994)--(-8.903,.993)--(-8.858,.988)--(-8.894,1.004);
\filldraw[fill opacity=0.8,fill=gray!20,draw=none](-8.742,.969)--(-8.728,.978)--(-8.718,.951)--cycle;
\draw(-8.728,.978)--(-8.718,.951);
\filldraw[fill opacity=0.8,fill=gray!20,draw=none](-8.724,.972)--(-8.728,.978)--(-8.702,1.039)--(-8.695,1.016)--(-8.707,.987)--cycle;
\draw(-8.728,.978)--(-8.702,1.039);
\draw(-8.695,1.016)--(-8.707,.987);
\filldraw[fill opacity=0.8,fill=gray!20,draw=none](-8.732,.942)--(-8.739,.947)--(-8.729,.944)--cycle;
\draw(-8.739,.947)--(-8.729,.944);
\filldraw[fill opacity=0.8,fill=gray!20,draw=none](-8.732,.942)--(-8.729,.944)--(-8.714,.94)--(-8.712,.927)--cycle;
\draw(-8.729,.944)--(-8.714,.94)--(-8.712,.927);
\filldraw[fill opacity=0.8,fill=gray!20,draw=none](-8.724,.972)--(-8.707,.987)--(-8.717,.964)--cycle;
\draw(-8.707,.987)--(-8.717,.964);
\filldraw[fill opacity=0.8,fill=gray!20,draw=none](-8.717,.972)--(-8.733,.99)--(-8.76,1.026)--(-8.743,1.005)--(-8.717,.973)--cycle;
\draw(-8.717,.972)--(-8.733,.99)--(-8.76,1.026);
\filldraw[fill opacity=0.8,fill=gray!20,draw=none](-8.855,.986)--(-8.882,.998)--(-8.885,1)--(-8.858,.988)--cycle;
\draw(-8.885,1)--(-8.858,.988);
\filldraw[fill opacity=0.8,fill=gray!20,draw=none](-8.853,.985)--(-8.855,.986)--(-8.858,.988)--cycle;
\draw(-8.858,.988)--(-8.853,.985);
\filldraw[fill opacity=0.8,fill=gray!20,draw=none](-8.903,.864)--(-8.853,.985)--(-8.858,.988)--(-8.903,.993)--(-8.916,.989)--cycle;
\draw(-8.853,.985)--(-8.858,.988)--(-8.903,.993)--(-8.916,.989);
\filldraw[fill opacity=0.8,fill=gray!20,draw=none](-8.844,.839)--(-8.854,.889)--(-8.87,.936)--(-8.892,.971)--(-8.911,.987)--(-8.92,.991)--(-8.937,.991)--(-8.954,.972)--(-8.962,.936)--(-8.962,.89)--cycle;
\draw(-8.844,.839)--(-8.854,.889)--(-8.87,.936)--(-8.892,.971)--(-8.911,.987);
\draw(-8.92,.991)--(-8.937,.991)--(-8.954,.972)--(-8.962,.936)--(-8.962,.89);
\filldraw[fill opacity=0.8,fill=gray!20,draw=none](-8.903,.993)--(-8.905,.994)--(-8.961,.983)--(-8.948,.978)--cycle;
\draw(-8.961,.983)--(-8.948,.978)--(-8.903,.993)--(-8.905,.994);
\filldraw[fill opacity=0.8,fill=gray!20,draw=none](-8.892,.971)--(-7.856,1.204)--(-7.792,1.232)--(-7.803,1.241)--(-8.916,.991)--cycle;
\draw(-7.792,1.232)--(-7.803,1.241)--(-8.916,.991)--(-8.892,.971)--(-7.856,1.204);
\filldraw[fill opacity=0.8,fill=gray!20,draw=none](-9.28,.957)--(-9.304,.968)--(-9.294,.914)--(-9.27,.904)--cycle;
\draw(-9.294,.914)--(-9.27,.904)--(-9.28,.957)--(-9.304,.968);
\filldraw[fill opacity=0.8,fill=gray!20,draw=none](-7.726,4.604)--(-7.719,4.627)--(-7.741,4.629)--(-7.738,4.592)--cycle;
\draw(-7.719,4.627)--(-7.741,4.629)--(-7.738,4.592);
\filldraw[fill opacity=0.8,fill=gray!20,draw=none](-7.661,1.238)--(-7.661,1.239)--(-7.662,1.238)--cycle;
\draw(-7.661,1.238)--(-7.661,1.239)--(-7.662,1.238);
\filldraw[fill opacity=0.8,fill=gray!20,draw=none](-7.669,1.238)--(-7.662,1.238)--(-7.661,1.239)--(-7.668,1.244)--cycle;
\draw(-7.662,1.238)--(-7.661,1.239)--(-7.668,1.244);
\filldraw[fill opacity=0.8,fill=gray!20](-7.833,1.282)--(-7.79,1.293)--(-7.79,1.293)--(-7.838,1.287)--cycle;
\filldraw[fill opacity=0.8,fill=gray!20,draw=none](-8.371,2.872)--(-8.366,2.872)--(-8.366,2.873)--(-8.375,2.878)--cycle;
\draw(-8.371,2.872)--(-8.366,2.872)--(-8.366,2.873);
\filldraw[fill opacity=0.5,fill=gray!20](-9.911,-.494)--(-9.738,-.57)--(-10.044,-.347)--(-10.217,-.271)--cycle;
\filldraw[fill opacity=0.8,fill=gray!20,draw=none](-7.669,1.238)--(-7.668,1.244)--(-7.699,1.268)--(-7.714,1.258)--(-7.695,1.238)--cycle;
\draw(-7.668,1.244)--(-7.699,1.268)--(-7.714,1.258)--(-7.695,1.238);
\filldraw[fill opacity=0.8,fill=gray!20,draw=none](-8.437,3.1)--(-8.43,3.143)--(-8.44,3.144)--(-8.457,3.098)--cycle;
\draw(-8.43,3.143)--(-8.44,3.144)--(-8.457,3.098);
\filldraw[fill opacity=0.8,fill=gray!20,draw=none](-8.406,2.922)--(-8.452,2.979)--(-8.475,2.98)--(-8.471,2.926)--cycle;
\draw(-8.452,2.979)--(-8.475,2.98)--(-8.471,2.926)--(-8.406,2.922);
\filldraw[fill opacity=0.8,fill=gray!20,draw=none](-7.931,1.217)--(-7.914,1.216)--(-7.906,1.228)--(-7.91,1.232)--cycle;
\draw(-7.914,1.216)--(-7.906,1.228)--(-7.91,1.232);
\filldraw[fill opacity=0.8,fill=gray!20,draw=none](-8.365,2.872)--(-8.366,2.873)--(-8.366,2.872)--cycle;
\draw(-8.366,2.873)--(-8.366,2.872)--(-8.365,2.872);
\filldraw[fill opacity=0.5,fill=gray!20](-9.43,-.892)--(-9.5,-.955)--(-9.92,-.811)--(-9.831,-.754)--cycle;
\filldraw[fill opacity=0.8,fill=gray!20](-7.535,4.739)--(-7.551,4.79)--(-7.629,4.786)--(-7.632,4.734)--cycle;
\filldraw[fill opacity=0.8,fill=gray!20,draw=none](-8.511,.857)--(-8.549,.88)--(-8.475,.897)--(-8.049,.952)--(-8.474,.856)--cycle;
\draw(-8.549,.88)--(-8.475,.897);
\draw(-8.049,.952)--(-8.474,.856);
\filldraw[fill opacity=0.8,fill=gray!20,draw=none](-8.375,2.878)--(-8.381,2.887)--(-8.401,2.893)--cycle;
\filldraw[fill opacity=0.8,fill=gray!20,draw=none](-8.395,2.907)--(-8.406,2.913)--(-8.406,2.894)--(-8.381,2.887)--cycle;
\filldraw[fill opacity=0.8,fill=gray!20,draw=none](-7.048,.977)--(-7.057,.957)--(-7.066,.956)--(-7.08,.974)--(-7.051,.993)--cycle;
\draw(-7.066,.956)--(-7.08,.974)--(-7.051,.993);
\filldraw[fill opacity=0.8,fill=gray!20](-7.699,1.268)--(-7.743,1.286)--(-7.751,1.281)--(-7.714,1.258)--cycle;
\filldraw[fill opacity=0.8,fill=gray!20](-7.743,1.286)--(-7.79,1.293)--(-7.79,1.293)--(-7.751,1.281)--cycle;
\filldraw[fill opacity=0.8,fill=gray!20](-7.751,1.281)--(-7.79,1.293)--(-7.79,1.293)--(-7.769,1.278)--cycle;
\filldraw[fill opacity=0.8,fill=gray!20](-7.769,1.278)--(-7.79,1.293)--(-7.79,1.293)--(-7.793,1.276)--cycle;
\filldraw[fill opacity=0.8,fill=gray!20](-7.793,1.276)--(-7.79,1.293)--(-7.79,1.293)--(-7.816,1.278)--cycle;
\filldraw[fill opacity=0.8,fill=gray!20](-7.796,.94)--(-7.798,.96)--(-7.862,.964)--(-7.841,.943)--cycle;
\filldraw[fill opacity=0.8,fill=gray!20](-7.816,1.278)--(-7.79,1.293)--(-7.79,1.293)--(-7.833,1.282)--cycle;
\filldraw[fill opacity=0.8,fill=gray!20,draw=none](-8.449,3.035)--(-8.446,3.037)--(-8.435,3.056)--(-8.424,3.089)--(-8.43,3.143)--cycle;
\draw(-8.446,3.037)--(-8.435,3.056)--(-8.424,3.089);
\filldraw[fill opacity=0.8,fill=gray!20](-6.963,.457)--(-6.961,.509)--(-7.037,.514)--(-7.056,.464)--cycle;
\filldraw[fill opacity=0.8,fill=gray!20](-6.964,.289)--(-6.965,.343)--(-7.073,.351)--(-7.068,.297)--cycle;
\filldraw[fill opacity=0.8,fill=gray!20,draw=none](-8.272,3.113)--(-8.21,3.086)--(-8.198,3.096)--(-8.227,3.108)--cycle;
\draw(-8.272,3.113)--(-8.21,3.086);
\draw(-8.198,3.096)--(-8.227,3.108);
\filldraw[fill opacity=0.8,fill=gray!20,draw=none](-8.255,3.111)--(-8.267,3.111)--(-8.21,3.086)--(-8.198,3.096)--(-8.227,3.108)--cycle;
\draw(-8.267,3.111)--(-8.21,3.086);
\draw(-8.198,3.096)--(-8.227,3.108);
\filldraw[fill opacity=0.8,fill=gray!20,draw=none](-6.208,.416)--(-6.199,.441)--(-6.209,.438)--(-6.209,.417)--cycle;
\draw(-6.209,.438)--(-6.209,.417);
\filldraw[fill opacity=0.8,fill=gray!20,draw=none](-6.209,.417)--(-6.208,.414)--(-6.208,.416)--cycle;
\filldraw[fill opacity=0.8,fill=gray!20,draw=none](-7.689,1)--(-7.694,1.034)--(-7.715,1.035)--(-7.769,1.002)--(-7.689,.999)--cycle;
\draw(-7.694,1.034)--(-7.715,1.035);
\draw(-7.769,1.002)--(-7.689,.999);
\filldraw[fill opacity=0.8,fill=gray!20,draw=none](-7.682,.972)--(-7.676,.981)--(-7.69,1)--(-7.72,.994)--(-7.733,.963)--cycle;
\draw(-7.69,1)--(-7.72,.994)--(-7.733,.963)--(-7.682,.972)--(-7.676,.981);
\filldraw[fill opacity=0.8,fill=gray!20,draw=none](-6.199,.441)--(-6.208,.416)--(-6.206,.415)--(-6.159,.453)--cycle;
\filldraw[fill opacity=0.8,fill=gray!20,draw=none](-6.208,.416)--(-6.208,.414)--(-6.207,.413)--(-6.206,.415)--cycle;
\filldraw[fill opacity=0.8,fill=gray!20](-7.109,1.081)--(-7.102,1.137)--(-7.067,1.16)--(-7.073,1.105)--cycle;
\filldraw[fill opacity=0.8,fill=gray!20,draw=none](-8.312,2.859)--(-8.255,2.834)--(-8.252,2.839)--(-8.336,2.876)--cycle;
\draw(-8.312,2.859)--(-8.255,2.834);
\draw(-8.252,2.839)--(-8.336,2.876);
\filldraw[fill opacity=0.8,fill=gray!20,draw=none](-8.164,2.843)--(-7.982,2.764)--(-8.027,2.749)--(-8.201,2.825)--cycle;
\draw(-8.164,2.843)--(-7.982,2.764)--(-8.027,2.749)--(-8.201,2.825);
\filldraw[fill opacity=0.8,fill=gray!20,draw=none](-8.165,2.797)--(-8.187,2.811)--(-8.195,2.822)--(-8.204,2.848)--(-8.122,2.864)--(-8.112,2.808)--cycle;
\draw(-8.195,2.822)--(-8.204,2.848)--(-8.122,2.864)--(-8.112,2.808)--(-8.165,2.797);
\filldraw[fill opacity=0.8,fill=gray!20,draw=none](-8.147,2.787)--(-8.165,2.797)--(-8.112,2.808)--(-8.104,2.783)--cycle;
\draw(-8.165,2.797)--(-8.112,2.808)--(-8.104,2.783);
\filldraw[fill opacity=0.8,fill=gray!20,draw=none](-8.021,2.829)--(-8.043,2.811)--(-8.112,2.808)--(-8.122,2.864)--(-8.014,2.868)--(-8.015,2.842)--cycle;
\draw(-8.043,2.811)--(-8.112,2.808)--(-8.122,2.864)--(-8.014,2.868)--(-8.015,2.842);
\filldraw[fill opacity=0.8,fill=gray!20,draw=none](-8.057,2.796)--(-8.104,2.783)--(-8.112,2.808)--(-8.043,2.811)--cycle;
\draw(-8.104,2.783)--(-8.112,2.808)--(-8.043,2.811);
\filldraw[fill opacity=0.8,fill=gray!20](-8.122,2.864)--(-8.125,2.92)--(-8.014,2.925)--(-8.014,2.868)--cycle;
\filldraw[fill opacity=0.8,fill=gray!20](-8.098,2.819)--(-8.106,2.866)--(-8.016,2.87)--(-8.017,2.823)--cycle;
\filldraw[fill opacity=0.8,fill=gray!20](-8.106,2.866)--(-8.109,2.913)--(-8.016,2.917)--(-8.016,2.87)--cycle;
\filldraw[fill opacity=0.8,fill=gray!20,draw=none](-8.036,2.788)--(-8.073,2.777)--(-8.085,2.777)--(-8.098,2.819)--(-8.017,2.823)--(-8.018,2.806)--cycle;
\draw(-8.073,2.777)--(-8.085,2.777)--(-8.098,2.819)--(-8.017,2.823)--(-8.018,2.806);
\filldraw[fill opacity=0.8,fill=gray!20](-8.159,2.807)--(-8.174,2.853)--(-8.106,2.866)--(-8.098,2.819)--cycle;
\filldraw[fill opacity=0.8,fill=gray!20,draw=none](-8.104,2.776)--(-8.13,2.779)--(-8.138,2.784)--(-8.134,2.793)--cycle;
\draw(-8.138,2.784)--(-8.134,2.793);
\filldraw[fill opacity=0.8,fill=gray!20](-8.204,2.848)--(-8.21,2.904)--(-8.125,2.92)--(-8.122,2.864)--cycle;
\filldraw[fill opacity=0.8,fill=gray!20](-8.174,2.853)--(-8.18,2.9)--(-8.109,2.913)--(-8.106,2.866)--cycle;
\filldraw[fill opacity=0.8,fill=gray!20](-8.125,2.92)--(-8.122,2.974)--(-8.014,2.979)--(-8.014,2.925)--cycle;
\filldraw[fill opacity=0.8,fill=gray!20](-8.21,2.904)--(-8.204,2.959)--(-8.122,2.974)--(-8.125,2.92)--cycle;
\filldraw[fill opacity=0.8,fill=gray!20](-8.18,2.9)--(-8.174,2.945)--(-8.106,2.958)--(-8.109,2.913)--cycle;
\filldraw[fill opacity=0.8,fill=gray!20,draw=none](-8.134,2.793)--(-8.159,2.807)--(-8.098,2.819)--(-8.09,2.795)--cycle;
\draw(-8.159,2.807)--(-8.098,2.819)--(-8.09,2.795);
\filldraw[fill opacity=0.8,fill=gray!20,draw=none](-8.134,2.793)--(-8.15,2.792)--(-8.159,2.807)--cycle;
\draw(-8.15,2.792)--(-8.159,2.807);
\filldraw[fill opacity=0.8,fill=gray!20,draw=none](-8.161,2.806)--(-8.17,2.809)--(-8.189,2.84)--(-8.19,2.843)--(-8.174,2.853)--(-8.159,2.807)--cycle;
\draw(-8.19,2.843)--(-8.174,2.853)--(-8.159,2.807)--(-8.161,2.806);
\filldraw[fill opacity=0.8,fill=gray!20,draw=none](-8.163,2.8)--(-8.166,2.803)--(-8.138,2.784)--cycle;
\filldraw[fill opacity=0.8,fill=gray!20,draw=none](-8.165,2.801)--(-8.166,2.803)--(-8.159,2.807)--(-8.15,2.792)--cycle;
\draw(-8.166,2.803)--(-8.159,2.807)--(-8.15,2.792);
\filldraw[fill opacity=0.8,fill=gray!20,draw=none](-8.161,2.806)--(-8.166,2.803)--(-8.17,2.809)--cycle;
\draw(-8.161,2.806)--(-8.166,2.803);
\filldraw[fill opacity=0.8,fill=gray!20,draw=none](-8.166,2.803)--(-8.17,2.809)--(-8.122,2.919)--(-8.086,2.903)--(-8.138,2.784)--cycle;
\draw(-8.17,2.809)--(-8.122,2.919)--(-8.086,2.903)--(-8.138,2.784);
\filldraw[fill opacity=0.8,fill=gray!20,draw=none](-8.134,2.793)--(-8.104,2.776)--(-8.127,2.778)--(-8.15,2.792)--(-8.15,2.792)--cycle;
\draw(-8.15,2.792)--(-8.15,2.792);
\filldraw[fill opacity=0.8,fill=gray!20,draw=none](-8.312,2.859)--(-8.072,2.754)--(-8.11,2.778)--(-8.336,2.876)--cycle;
\draw(-8.312,2.859)--(-8.072,2.754)--(-8.11,2.778)--(-8.336,2.876);
\filldraw[fill opacity=0.8,fill=gray!20,draw=none](-9.199,1.098)--(-9.223,1.108)--(-9.266,1.073)--(-9.241,1.062)--cycle;
\draw(-9.266,1.073)--(-9.241,1.062)--(-9.199,1.098)--(-9.223,1.108);
\filldraw[fill opacity=0.8,fill=gray!20,draw=none](-6.208,.414)--(-6.207,.412)--(-6.207,.413)--cycle;
\filldraw[fill opacity=0.8,fill=gray!20](-7.902,3.915)--(-7.918,3.966)--(-7.997,3.962)--(-7.999,3.91)--cycle;
\filldraw[fill opacity=0.8,fill=gray!20](-7,.894)--(-7.045,.929)--(-7.02,.946)--(-6.982,.906)--cycle;
\filldraw[fill opacity=0.8,fill=gray!20](-6.96,1.26)--(-6.939,1.286)--(-6.884,1.288)--(-6.881,1.264)--cycle;
\filldraw[fill opacity=0.8,fill=gray!20](-6.881,1.264)--(-6.884,1.288)--(-6.83,1.284)--(-6.805,1.258)--cycle;
\filldraw[fill opacity=0.8,fill=gray!20,draw=none](-6.132,.422)--(-6.136,.426)--(-6.207,.413)--(-6.207,.412)--(-6.152,.388)--cycle;
\draw(-6.207,.412)--(-6.152,.388)--(-6.132,.422);
\filldraw[fill opacity=0.8,fill=gray!20,draw=none](-6.047,.532)--(-6.047,.522)--(-6.002,.493)--(-6.002,.54)--cycle;
\draw(-6.002,.493)--(-6.002,.54)--(-6.047,.532)--(-6.047,.522);
\filldraw[fill opacity=0.8,fill=gray!20,draw=none](-5.981,.484)--(-6.002,.493)--(-6.047,.496)--(-6.031,.489)--cycle;
\draw(-6.047,.496)--(-6.031,.489)--(-5.981,.484)--(-6.002,.493);
\filldraw[fill opacity=0.8,fill=gray!20,draw=none](-7.065,.956)--(-7.07,.963)--(-7.08,.974)--(-7.066,.956)--cycle;
\draw(-7.07,.963)--(-7.08,.974)--(-7.066,.956);
\filldraw[fill opacity=0.8,fill=gray!20,draw=none](-7.07,.963)--(-7.072,.974)--(-7.082,1.005)--(-7.102,1.026)--(-7.08,.974)--cycle;
\draw(-7.082,1.005)--(-7.102,1.026)--(-7.08,.974)--(-7.07,.963);
\filldraw[fill opacity=0.8,fill=gray!20,draw=none](-6.907,.968)--(-7.661,.998)--(-7.67,.996)--(-7.677,.98)--(-6.899,.949)--cycle;
\draw(-7.677,.98)--(-6.899,.949)--(-6.907,.968)--(-7.661,.998);
\filldraw[fill opacity=0.8,fill=gray!20,draw=none](-9.232,.871)--(-9.26,.889)--(-9.241,.86)--(-9.227,.851)--(-9.218,.85)--cycle;
\draw(-9.26,.889)--(-9.241,.86)--(-9.227,.851);
\filldraw[fill opacity=0.8,fill=gray!20,draw=none](-8.418,3.109)--(-8.408,3.105)--(-8.393,3.15)--(-8.413,3.159)--cycle;
\draw(-8.393,3.15)--(-8.413,3.159)--(-8.418,3.109)--(-8.408,3.105);
\filldraw[fill opacity=0.8,fill=gray!20,draw=none](-8.396,3.141)--(-8.389,3.14)--(-8.393,3.15)--cycle;
\draw(-8.396,3.141)--(-8.389,3.14);
\filldraw[fill opacity=0.8,fill=gray!20,draw=none](-6.213,.354)--(-6.207,.406)--(-6.23,.406)--cycle;
\draw(-6.207,.406)--(-6.23,.406);
\filldraw[fill opacity=0.8,fill=gray!20,draw=none](-7.726,4.604)--(-7.703,4.626)--(-7.719,4.627)--cycle;
\draw(-7.703,4.626)--(-7.719,4.627);
\filldraw[fill opacity=0.8,fill=gray!20,draw=none](-8.481,2.963)--(-8.477,2.961)--(-8.474,2.969)--cycle;
\draw(-8.481,2.963)--(-8.477,2.961);
\filldraw[fill opacity=0.8,fill=gray!20,draw=none](-8.477,2.961)--(-8.474,2.963)--(-8.474,2.969)--cycle;
\draw(-8.474,2.963)--(-8.474,2.969);
\filldraw[fill opacity=0.8,fill=gray!20,draw=none](-8.356,2.885)--(-8.336,2.876)--(-8.343,2.907)--(-8.381,2.924)--cycle;
\draw(-8.356,2.885)--(-8.336,2.876);
\draw(-8.343,2.907)--(-8.381,2.924);
\filldraw[fill opacity=0.8,fill=gray!20,draw=none](-8.358,2.886)--(-8.336,2.876)--(-8.343,2.907)--(-8.372,2.92)--cycle;
\draw(-8.358,2.886)--(-8.336,2.876);
\draw(-8.343,2.907)--(-8.372,2.92);
\filldraw[fill opacity=0.8,fill=gray!20,draw=none](-8.366,2.897)--(-8.382,2.924)--(-8.386,2.926)--cycle;
\draw(-8.382,2.924)--(-8.386,2.926);
\filldraw[fill opacity=0.8,fill=gray!20,draw=none](-8.366,2.897)--(-8.382,2.924)--(-8.386,2.926)--cycle;
\draw(-8.382,2.924)--(-8.386,2.926);
\filldraw[fill opacity=0.8,fill=gray!20,draw=none](-8.382,2.92)--(-8.386,2.926)--(-8.388,2.926)--cycle;
\draw(-8.386,2.926)--(-8.388,2.926);
\filldraw[fill opacity=0.8,fill=gray!20,draw=none](-8.382,2.92)--(-8.386,2.926)--(-8.388,2.926)--cycle;
\draw(-8.386,2.926)--(-8.388,2.926);
\filldraw[fill opacity=0.8,fill=gray!20,draw=none](-8.4,2.952)--(-8.393,2.917)--(-8.365,2.874)--(-8.35,2.865)--(-8.342,2.864)--cycle;
\draw(-8.4,2.952)--(-8.393,2.917)--(-8.365,2.874)--(-8.35,2.865);
\filldraw[fill opacity=0.8,fill=gray!20](-7.749,.942)--(-7.733,.963)--(-7.798,.96)--(-7.796,.94)--cycle;
\filldraw[fill opacity=0.8,fill=gray!20,draw=none](-9.096,.884)--(-9.088,.935)--(-9.085,.942)--(-9.08,.945)--(-9.087,.891)--cycle;
\draw(-9.085,.942)--(-9.08,.945)--(-9.087,.891)--(-9.096,.884);
\filldraw[fill opacity=0.8,fill=gray!20,draw=none](-9.012,.902)--(-9.282,1.019)--(-9.291,.969)--(-9.021,.851)--cycle;
\draw(-9.291,.969)--(-9.021,.851)--(-9.012,.902)--(-9.282,1.019);
\filldraw[fill opacity=0.8,fill=gray!20](-7.892,3.748)--(-7.889,3.802)--(-8,3.797)--(-8,3.743)--cycle;
\filldraw[fill opacity=0.5,fill=gray!20](-10.912,-.09)--(-10.918,-.028)--(-11.092,.411)--(-11.092,.364)--cycle;
\filldraw[fill opacity=0.8,fill=gray!20,draw=none](-6.209,.475)--(-6.209,.438)--(-6.159,.453)--(-6.159,.475)--cycle;
\draw(-6.209,.475)--(-6.209,.438);
\draw(-6.159,.453)--(-6.159,.475);
\filldraw[fill opacity=0.8,fill=gray!20,draw=none](-8.282,3.132)--(-8.285,3.142)--(-8.33,3.14)--cycle;
\draw(-8.282,3.132)--(-8.285,3.142)--(-8.33,3.14);
\filldraw[fill opacity=0.8,fill=gray!20,draw=none](-8.496,2.969)--(-8.481,2.963)--(-8.474,2.969)--(-8.461,3.006)--(-8.463,3.007)--cycle;
\draw(-8.461,3.006)--(-8.463,3.007)--(-8.496,2.969)--(-8.481,2.963);
\filldraw[fill opacity=0.8,fill=gray!20,draw=none](-7.705,4.683)--(-7.696,4.739)--(-7.725,4.741)--(-7.737,4.685)--cycle;
\draw(-7.696,4.739)--(-7.725,4.741)--(-7.737,4.685)--(-7.705,4.683);
\filldraw[fill opacity=0.8,fill=gray!20,draw=none](-7.719,4.627)--(-7.705,4.683)--(-7.737,4.685)--(-7.741,4.629)--cycle;
\draw(-7.705,4.683)--(-7.737,4.685)--(-7.741,4.629)--(-7.719,4.627);
\filldraw[fill opacity=0.8,fill=gray!20,draw=none](-7.508,4.627)--(-7.509,4.628)--(-7.511,4.628)--cycle;
\draw(-7.509,4.628)--(-7.511,4.628);
\filldraw[fill opacity=0.8,fill=gray!20,draw=none](-8.249,3.095)--(-8.261,3.098)--(-8.27,3.092)--(-8.269,3.09)--cycle;
\draw(-8.27,3.092)--(-8.269,3.09)--(-8.249,3.095);
\filldraw[fill opacity=0.8,fill=gray!20,draw=none](-8.249,3.095)--(-8.293,3.121)--(-8.317,3.113)--cycle;
\draw(-8.293,3.121)--(-8.317,3.113);
\filldraw[fill opacity=0.8,fill=gray!20](-7.878,.996)--(-7.888,1.036)--(-7.948,1.051)--(-7.932,1.01)--cycle;
\filldraw[fill opacity=0.8,fill=gray!20](-8.259,2.924)--(-8.256,2.978)--(-8.367,2.973)--(-8.367,2.919)--cycle;
\filldraw[fill opacity=0.8,fill=gray!20,draw=none](-7.511,4.628)--(-7.509,4.628)--(-7.523,4.651)--(-7.522,4.632)--cycle;
\draw(-7.511,4.628)--(-7.509,4.628);
\draw(-7.523,4.651)--(-7.522,4.632);
\filldraw[fill opacity=0.8,fill=gray!20](-6.867,.461)--(-6.882,.512)--(-6.961,.509)--(-6.963,.457)--cycle;
\filldraw[fill opacity=0.8,fill=gray!20,draw=none](-6.71,.979)--(-6.703,1.001)--(-6.666,.999)--(-6.7,.963)--(-6.711,.964)--cycle;
\draw(-6.703,1.001)--(-6.666,.999);
\draw(-6.7,.963)--(-6.711,.964);
\filldraw[fill opacity=0.8,fill=gray!20,draw=none](-6.71,.979)--(-6.711,.964)--(-6.716,.964)--cycle;
\draw(-6.711,.964)--(-6.716,.964);
\filldraw[fill opacity=0.8,fill=gray!20,draw=none](-6.714,.952)--(-6.716,.964)--(-6.7,.963)--cycle;
\draw(-6.716,.964)--(-6.7,.963);
\filldraw[fill opacity=0.8,fill=gray!20](-6.751,.942)--(-6.72,.99)--(-6.7,.969)--(-6.735,.925)--cycle;
\filldraw[fill opacity=0.8,fill=gray!20,draw=none](-6.71,.979)--(-6.713,.983)--(-6.709,1.001)--cycle;
\draw(-6.71,.979)--(-6.713,.983);
\filldraw[fill opacity=0.8,fill=gray!20,draw=none](-6.874,1.008)--(-6.703,1.001)--(-6.716,.964)--(-6.881,.971)--cycle;
\draw(-6.716,.964)--(-6.881,.971)--(-6.874,1.008)--(-6.703,1.001);
\filldraw[fill opacity=0.8,fill=gray!20,draw=none](-8.335,3.117)--(-8.282,3.132)--(-8.315,3.137)--cycle;
\filldraw[fill opacity=0.8,fill=gray!20,draw=none](-7.67,1.233)--(-7.669,1.238)--(-7.695,1.238)--(-7.682,1.225)--cycle;
\draw(-7.695,1.238)--(-7.682,1.225)--(-7.67,1.233);
\filldraw[fill opacity=0.8,fill=gray!20,draw=none](-8.396,3.105)--(-8.365,3.108)--(-8.364,3.138)--(-8.396,3.141)--cycle;
\draw(-8.365,3.108)--(-8.364,3.138)--(-8.396,3.141);
\filldraw[fill opacity=0.8,fill=gray!20,draw=none](-8.396,3.141)--(-8.408,3.105)--(-8.401,3.102)--(-8.396,3.105)--cycle;
\draw(-8.408,3.105)--(-8.401,3.102);
\filldraw[fill opacity=0.8,fill=gray!20,draw=none](-8.474,2.969)--(-8.445,2.999)--(-8.461,3.006)--cycle;
\draw(-8.445,2.999)--(-8.461,3.006);
\filldraw[fill opacity=0.8,fill=gray!20](-8,3.854)--(-7.999,3.91)--(-8.092,3.917)--(-8.104,3.861)--cycle;
\filldraw[fill opacity=0.8,fill=gray!20](-8,3.797)--(-8,3.854)--(-8.104,3.861)--(-8.108,3.805)--cycle;
\filldraw[fill opacity=0.8,fill=gray!20,draw=none](-8.406,2.913)--(-8.395,2.907)--(-8.406,2.922)--cycle;
\filldraw[fill opacity=0.8,fill=gray!20,draw=none](-6.973,.908)--(-6.982,.906)--(-6.987,.911)--cycle;
\draw(-6.973,.908)--(-6.982,.906)--(-6.987,.911);
\filldraw[fill opacity=0.8,fill=gray!20,draw=none](-9.27,1.013)--(-9.294,1.023)--(-9.304,.968)--(-9.28,.957)--cycle;
\draw(-9.304,.968)--(-9.28,.957)--(-9.27,1.013)--(-9.294,1.023);
\filldraw[fill opacity=0.8,fill=gray!20](-6.857,.294)--(-6.853,.348)--(-6.965,.343)--(-6.964,.289)--cycle;
\filldraw[fill opacity=0.8,fill=gray!20,draw=none](-8.428,2.981)--(-8.423,2.989)--(-8.427,2.991)--cycle;
\draw(-8.423,2.989)--(-8.427,2.991);
\filldraw[fill opacity=0.8,fill=gray!20,draw=none](-9.155,1.098)--(-9.124,1.108)--(-9.132,1.112)--(-9.149,1.114)--(-9.167,1.108)--cycle;
\draw(-9.132,1.112)--(-9.149,1.114)--(-9.167,1.108);
\filldraw[fill opacity=0.8,fill=gray!20,draw=none](-8.335,3.117)--(-8.315,3.137)--(-8.33,3.14)--(-8.364,3.138)--(-8.365,3.108)--cycle;
\draw(-8.33,3.14)--(-8.364,3.138)--(-8.365,3.108);
\filldraw[fill opacity=0.8,fill=gray!20,draw=none](-8.439,3.092)--(-8.437,3.1)--(-8.457,3.098)--(-8.459,3.093)--cycle;
\draw(-8.457,3.098)--(-8.459,3.093)--(-8.439,3.092);
\filldraw[fill opacity=0.8,fill=gray!20,draw=none](-6.707,1.173)--(-6.699,1.153)--(-6.701,1.155)--(-6.707,1.17)--cycle;
\draw(-6.699,1.153)--(-6.701,1.155)--(-6.707,1.17);
\filldraw[fill opacity=0.8,fill=gray!20,draw=none](-6.698,1.147)--(-6.699,1.153)--(-6.698,1.152)--cycle;
\draw(-6.699,1.153)--(-6.698,1.152);
\filldraw[fill opacity=0.8,fill=gray!20,draw=none](-6.707,1.173)--(-6.707,1.17)--(-6.72,1.204)--(-6.719,1.203)--cycle;
\draw(-6.707,1.17)--(-6.72,1.204)--(-6.719,1.203);
\filldraw[fill opacity=0.8,fill=gray!20,draw=none](-6.709,1.183)--(-6.698,1.152)--(-6.699,1.153)--(-6.707,1.173)--cycle;
\draw(-6.698,1.152)--(-6.699,1.153);
\filldraw[fill opacity=0.8,fill=gray!20,draw=none](-6.709,1.183)--(-6.707,1.173)--(-6.719,1.203)--(-6.713,1.197)--cycle;
\draw(-6.719,1.203)--(-6.713,1.197);
\filldraw[fill opacity=0.8,fill=gray!20,draw=none](-6.874,1.189)--(-6.671,1.181)--(-6.693,1.147)--(-6.868,1.154)--cycle;
\draw(-6.693,1.147)--(-6.868,1.154)--(-6.874,1.189)--(-6.671,1.181);
\filldraw[fill opacity=0.8,fill=gray!20,draw=none](-6.209,.537)--(-6.209,.475)--(-6.159,.475)--(-6.159,.53)--cycle;
\draw(-6.159,.475)--(-6.159,.53)--(-6.209,.537)--(-6.209,.475);
\filldraw[fill opacity=0.8,fill=gray!20,draw=none](-8.33,3.14)--(-8.363,3.145)--(-8.364,3.138)--cycle;
\draw(-8.363,3.145)--(-8.364,3.138)--(-8.33,3.14);
\filldraw[fill opacity=0.8,fill=gray!20,draw=none](-7.693,4.638)--(-7.718,4.629)--(-7.719,4.627)--(-7.703,4.626)--cycle;
\draw(-7.719,4.627)--(-7.703,4.626);
\filldraw[fill opacity=0.8,fill=gray!20,draw=none](-8.461,3.037)--(-8.439,3.092)--(-8.459,3.093)--(-8.471,3.037)--cycle;
\draw(-8.439,3.092)--(-8.459,3.093)--(-8.471,3.037)--(-8.461,3.037);
\filldraw[fill opacity=0.8,fill=gray!20,draw=none](-9.258,.91)--(-9.253,.884)--(-9.232,.871)--cycle;
\filldraw[fill opacity=0.8,fill=gray!20,draw=none](-7.693,4.638)--(-7.676,4.66)--(-7.71,4.665)--(-7.718,4.629)--cycle;
\filldraw[fill opacity=0.8,fill=gray!20,draw=none](-9.241,1.062)--(-9.266,1.073)--(-9.294,1.023)--(-9.27,1.013)--cycle;
\draw(-9.294,1.023)--(-9.27,1.013)--(-9.241,1.062)--(-9.266,1.073);
\filldraw[fill opacity=0.8,fill=gray!20,draw=none](-8.272,3.113)--(-8.267,3.111)--(-8.255,3.111)--cycle;
\draw(-8.272,3.113)--(-8.267,3.111);
\filldraw[fill opacity=0.8,fill=gray!20,draw=none](-8.261,3.098)--(-8.272,3.101)--(-8.27,3.092)--cycle;
\draw(-8.272,3.101)--(-8.27,3.092);
\filldraw[fill opacity=0.8,fill=gray!20,draw=none](-7.705,4.683)--(-7.66,4.68)--(-7.632,4.727)--(-7.632,4.734)--(-7.696,4.739)--cycle;
\draw(-7.705,4.683)--(-7.66,4.68);
\draw(-7.632,4.727)--(-7.632,4.734)--(-7.696,4.739);
\filldraw[fill opacity=0.8,fill=gray!20](-7.02,1.249)--(-6.982,1.277)--(-6.939,1.286)--(-6.96,1.26)--cycle;
\filldraw[fill opacity=0.8,fill=gray!20](-6.792,.904)--(-6.751,.942)--(-6.735,.925)--(-6.781,.892)--cycle;
\filldraw[fill opacity=0.8,fill=gray!20](-6.89,.864)--(-6.916,.883)--(-6.887,.884)--(-6.89,.864)--cycle;
\filldraw[fill opacity=0.8,fill=gray!20](-6.89,.864)--(-6.887,.884)--(-6.859,.882)--(-6.89,.864)--cycle;
\filldraw[fill opacity=0.8,fill=gray!20,draw=none](-8.406,2.922)--(-8.417,2.976)--(-8.452,2.979)--cycle;
\draw(-8.417,2.976)--(-8.452,2.979);
\filldraw[fill opacity=0.5,fill=gray!20](-10.005,-.844)--(-10.084,-.854)--(-10.464,-.577)--(-10.378,-.573)--cycle;
\filldraw[fill opacity=0.8,fill=gray!20](-6.965,.343)--(-6.964,.4)--(-7.068,.408)--(-7.073,.351)--cycle;
\filldraw[fill opacity=0.8,fill=gray!20](-6.964,.4)--(-6.963,.457)--(-7.056,.464)--(-7.068,.408)--cycle;
\filldraw[fill opacity=0.8,fill=gray!20,draw=none](-7.581,4.68)--(-7.525,4.683)--(-7.535,4.739)--(-7.632,4.734)--(-7.632,4.732)--cycle;
\draw(-7.581,4.68)--(-7.525,4.683)--(-7.535,4.739)--(-7.632,4.734)--(-7.632,4.732);
\filldraw[fill opacity=0.8,fill=gray!20](-6.947,.872)--(-7,.894)--(-6.982,.906)--(-6.938,.878)--cycle;
\filldraw[fill opacity=0.8,fill=gray!20,draw=none](-8.464,2.98)--(-8.452,2.979)--(-8.435,2.998)--cycle;
\draw(-8.464,2.98)--(-8.452,2.979);
\filldraw[fill opacity=0.8,fill=gray!20,draw=none](-8.464,2.98)--(-8.438,2.996)--(-8.445,2.999)--cycle;
\draw(-8.438,2.996)--(-8.445,2.999);
\filldraw[fill opacity=0.8,fill=gray!20](-7.841,1.252)--(-7.816,1.278)--(-7.833,1.282)--(-7.872,1.26)--cycle;
\filldraw[fill opacity=0.8,fill=gray!20,draw=none](-8.461,3.037)--(-8.471,3.037)--(-8.473,3.018)--cycle;
\draw(-8.461,3.037)--(-8.471,3.037)--(-8.473,3.018);
\filldraw[fill opacity=0.8,fill=gray!20,draw=none](-8.473,3.018)--(-8.467,3.022)--(-8.461,3.037)--cycle;
\filldraw[fill opacity=0.8,fill=gray!20,draw=none](-8.386,3.085)--(-8.378,3.092)--(-8.382,3.093)--cycle;
\draw(-8.378,3.092)--(-8.382,3.093);
\filldraw[fill opacity=0.8,fill=gray!20,draw=none](-8.361,2.893)--(-8.372,2.92)--(-8.381,2.924)--cycle;
\draw(-8.372,2.92)--(-8.381,2.924);
\filldraw[fill opacity=0.8,fill=gray!20,draw=none](-7.548,4.653)--(-7.522,4.632)--(-7.523,4.647)--cycle;
\draw(-7.522,4.632)--(-7.523,4.647);
\filldraw[fill opacity=0.8,fill=gray!20,draw=none](-7.786,1.222)--(-7.799,1.222)--(-7.807,1.185)--(-7.785,1.184)--cycle;
\draw(-7.786,1.222)--(-7.799,1.222)--(-7.807,1.185)--(-7.785,1.184);
\filldraw[fill opacity=0.8,fill=gray!20,draw=none](-7.78,1.181)--(-7.757,1.186)--(-7.779,1.221)--(-7.824,1.211)--cycle;
\draw(-7.78,1.181)--(-7.757,1.186)--(-7.779,1.221)--(-7.824,1.211);
\filldraw[fill opacity=0.8,fill=gray!20](-7.888,1.036)--(-7.891,1.081)--(-7.954,1.096)--(-7.948,1.051)--cycle;
\filldraw[fill opacity=0.8,fill=gray!20](-7.891,1.081)--(-7.888,1.128)--(-7.948,1.143)--(-7.954,1.096)--cycle;
\filldraw[fill opacity=0.8,fill=gray!20,draw=none](-7.777,1.136)--(-7.773,1.132)--(-7.741,1.139)--(-7.757,1.186)--(-7.78,1.181)--cycle;
\draw(-7.773,1.132)--(-7.741,1.139)--(-7.757,1.186)--(-7.78,1.181);
\filldraw[fill opacity=0.8,fill=gray!20,draw=none](-7.676,1.214)--(-7.677,1.217)--(-7.741,1.22)--(-7.78,1.184)--(-7.689,1.181)--cycle;
\draw(-7.677,1.217)--(-7.741,1.22);
\draw(-7.78,1.184)--(-7.689,1.181);
\filldraw[fill opacity=0.8,fill=gray!20,draw=none](-7.67,1.233)--(-7.682,1.225)--(-7.677,1.216)--cycle;
\draw(-7.67,1.233)--(-7.682,1.225)--(-7.677,1.216);
\filldraw[fill opacity=0.8,fill=gray!20,draw=none](-7.71,4.665)--(-7.676,4.66)--(-7.66,4.68)--(-7.705,4.683)--cycle;
\draw(-7.66,4.68)--(-7.705,4.683);
\filldraw[fill opacity=0.8,fill=gray!20,draw=none](-8.405,3.072)--(-8.402,3.089)--(-8.405,3.089)--cycle;
\draw(-8.402,3.089)--(-8.405,3.089);
\filldraw[fill opacity=0.8,fill=gray!20,draw=none](-8.435,3.056)--(-8.421,3.05)--(-8.386,3.085)--(-8.382,3.093)--(-8.418,3.109)--cycle;
\draw(-8.382,3.093)--(-8.418,3.109)--(-8.435,3.056)--(-8.421,3.05);
\filldraw[fill opacity=0.8,fill=gray!20,draw=none](-8.467,3.022)--(-8.449,3.035)--(-8.439,3.092)--cycle;
\filldraw[fill opacity=0.8,fill=gray!20,draw=none](-6.698,1.042)--(-6.698,1.049)--(-6.694,1.1)--(-6.671,1.076)--(-6.679,1.02)--cycle;
\draw(-6.694,1.1)--(-6.671,1.076)--(-6.679,1.02)--(-6.698,1.042);
\filldraw[fill opacity=0.8,fill=gray!20,draw=none](-8.296,3.089)--(-8.269,3.09)--(-8.272,3.101)--cycle;
\draw(-8.296,3.089)--(-8.269,3.09)--(-8.272,3.101);
\filldraw[fill opacity=0.8,fill=gray!20,draw=none](-7.868,.962)--(-7.866,.963)--(-7.866,.966)--cycle;
\draw(-7.868,.962)--(-7.866,.963);
\filldraw[fill opacity=0.8,fill=gray!20](-7.892,3.859)--(-7.902,3.915)--(-7.999,3.91)--(-8,3.854)--cycle;
\filldraw[fill opacity=0.8,fill=gray!20,draw=none](-7.548,4.653)--(-7.523,4.647)--(-7.525,4.683)--(-7.581,4.68)--cycle;
\draw(-7.523,4.647)--(-7.525,4.683)--(-7.581,4.68);
\filldraw[fill opacity=0.8,fill=gray!20,draw=none](-8.427,3.007)--(-8.452,2.979)--(-8.417,2.976)--cycle;
\draw(-8.452,2.979)--(-8.417,2.976);
\filldraw[fill opacity=0.8,fill=gray!20,draw=none](-7.846,.983)--(-7.846,.994)--(-7.878,.996)--(-7.864,.969)--cycle;
\draw(-7.846,.994)--(-7.878,.996)--(-7.864,.969);
\filldraw[fill opacity=0.8,fill=gray!20,draw=none](-8.175,.894)--(-7.868,.962)--(-7.866,.966)--(-7.867,.978)--(-8.091,.928)--cycle;
\draw(-8.175,.894)--(-7.868,.962);
\draw(-7.867,.978)--(-8.091,.928);
\filldraw[fill opacity=0.8,fill=gray!20](-6.89,.864)--(-6.938,.878)--(-6.916,.883)--(-6.89,.864)--cycle;
\filldraw[fill opacity=0.8,fill=gray!20,draw=none](-9.041,.998)--(-9.023,1.011)--(-8.981,1.027)--(-8.988,1.009)--cycle;
\draw(-8.981,1.027)--(-8.988,1.009)--(-9.041,.998);
\filldraw[fill opacity=0.8,fill=gray!20,draw=none](-9.023,.916)--(-9.004,.96)--(-8.986,.995)--(-8.983,.999)--(-9.02,.915)--cycle;
\draw(-8.983,.999)--(-9.02,.915)--(-9.023,.916)--(-9.004,.96);
\filldraw[fill opacity=0.8,fill=gray!20,draw=none](-8.92,1)--(-9.149,1.1)--(-9.194,1.085)--(-8.968,.987)--cycle;
\draw(-8.92,1)--(-9.149,1.1);
\draw(-9.194,1.085)--(-8.968,.987);
\filldraw[fill opacity=0.8,fill=gray!20,draw=none](-6.694,1.1)--(-6.698,1.147)--(-6.698,1.152)--(-6.679,1.131)--(-6.671,1.076)--cycle;
\draw(-6.698,1.152)--(-6.679,1.131)--(-6.671,1.076)--(-6.694,1.1);
\filldraw[fill opacity=0.8,fill=gray!20](-7.714,1.258)--(-7.751,1.281)--(-7.769,1.278)--(-7.749,1.251)--cycle;
\filldraw[fill opacity=0.8,fill=gray!20,draw=none](-8.386,3.085)--(-8.421,3.05)--(-8.405,3.042)--cycle;
\draw(-8.421,3.05)--(-8.405,3.042);
\filldraw[fill opacity=0.8,fill=gray!20](-7.889,3.802)--(-7.892,3.859)--(-8,3.854)--(-8,3.797)--cycle;
\filldraw[fill opacity=0.8,fill=gray!20,draw=none](-7.648,4.656)--(-7.676,4.66)--(-7.693,4.638)--cycle;
\filldraw[fill opacity=0.8,fill=gray!20,draw=none](-6.71,.979)--(-6.709,1.001)--(-6.698,1.042)--(-6.679,1.02)--(-6.7,.969)--cycle;
\draw(-6.698,1.042)--(-6.679,1.02)--(-6.7,.969)--(-6.71,.979);
\filldraw[fill opacity=0.8,fill=gray!20,draw=none](-8.318,3.099)--(-8.3,3.091)--(-8.267,3.111)--(-8.272,3.113)--cycle;
\draw(-8.318,3.099)--(-8.3,3.091);
\draw(-8.267,3.111)--(-8.272,3.113);
\filldraw[fill opacity=0.8,fill=gray!20,draw=none](-8.259,3.035)--(-8.269,3.09)--(-8.296,3.089)--(-8.367,3.032)--(-8.367,3.03)--cycle;
\draw(-8.367,3.032)--(-8.367,3.03)--(-8.259,3.035)--(-8.269,3.09)--(-8.296,3.089);
\filldraw[fill opacity=0.8,fill=gray!20,draw=none](-8.318,3.099)--(-8.233,3.062)--(-8.21,3.086)--(-8.272,3.113)--cycle;
\draw(-8.318,3.099)--(-8.233,3.062);
\draw(-8.21,3.086)--(-8.272,3.113);
\filldraw[fill opacity=0.8,fill=gray!20,draw=none](-7.771,1.037)--(-7.768,1.034)--(-7.731,1.042)--(-7.731,1.089)--(-7.769,1.08)--cycle;
\draw(-7.768,1.034)--(-7.731,1.042)--(-7.731,1.089)--(-7.769,1.08);
\filldraw[fill opacity=0.8,fill=gray!20,draw=none](-7.769,1.08)--(-7.772,1.084)--(-7.814,1.086)--(-7.812,1.039)--(-7.771,1.037)--cycle;
\draw(-7.772,1.084)--(-7.814,1.086)--(-7.812,1.039)--(-7.771,1.037);
\filldraw[fill opacity=0.8,fill=gray!20,draw=none](-7.772,1.079)--(-7.731,1.089)--(-7.741,1.139)--(-7.773,1.132)--cycle;
\draw(-7.772,1.079)--(-7.731,1.089)--(-7.741,1.139)--(-7.773,1.132);
\filldraw[fill opacity=0.8,fill=gray!20,draw=none](-7.693,1.037)--(-7.695,1.081)--(-7.769,1.084)--(-7.769,1.08)--(-7.74,1.036)--(-7.694,1.034)--cycle;
\draw(-7.695,1.081)--(-7.769,1.084);
\draw(-7.74,1.036)--(-7.694,1.034);
\filldraw[fill opacity=0.8,fill=gray!20,draw=none](-7.69,1)--(-7.658,1.006)--(-7.643,1.047)--(-7.711,1.034)--(-7.715,1.018)--cycle;
\draw(-7.69,1)--(-7.658,1.006)--(-7.643,1.047)--(-7.711,1.034)--(-7.715,1.018);
\filldraw[fill opacity=0.8,fill=gray!20,draw=none](-8.365,3.086)--(-8.335,3.117)--(-8.365,3.108)--(-8.366,3.086)--cycle;
\draw(-8.365,3.108)--(-8.366,3.086)--(-8.365,3.086);
\filldraw[fill opacity=0.8,fill=gray!20](-6.805,1.258)--(-6.83,1.284)--(-6.792,1.275)--(-6.751,1.245)--cycle;
\filldraw[fill opacity=0.8,fill=gray!20](-6.89,.864)--(-6.859,.882)--(-6.839,.877)--(-6.89,.864)--cycle;
\filldraw[fill opacity=0.8,fill=gray!20,draw=none](-6.076,.208)--(-6.081,.209)--(-6.085,.211)--cycle;
\draw(-6.076,.208)--(-6.081,.209)--(-6.085,.211);
\filldraw[fill opacity=0.8,fill=gray!20,draw=none](-7.737,4.52)--(-7.732,4.468)--(-7.733,4.518)--cycle;
\draw(-7.737,4.52)--(-7.732,4.468);
\filldraw[fill opacity=0.8,fill=gray!20,draw=none](-8.396,3.1)--(-8.393,3.105)--(-8.396,3.105)--cycle;
\filldraw[fill opacity=0.8,fill=gray!20](-8.256,2.978)--(-8.259,3.035)--(-8.367,3.03)--(-8.367,2.973)--cycle;
\filldraw[fill opacity=0.5,fill=gray!20](-10.536,-.56)--(-10.591,-.522)--(-10.882,-.139)--(-10.829,-.174)--cycle;
\filldraw[fill opacity=0.8,fill=gray!20,draw=none](-7.597,4.663)--(-7.548,4.653)--(-7.57,4.671)--cycle;
\filldraw[fill opacity=0.8,fill=gray!20,draw=none](-8.43,3.015)--(-8.421,3.034)--(-8.436,3.035)--cycle;
\draw(-8.421,3.034)--(-8.436,3.035);
\filldraw[fill opacity=0.8,fill=gray!20,draw=none](-8.463,3.007)--(-8.438,2.996)--(-8.411,3.034)--(-8.408,3.044)--(-8.435,3.056)--cycle;
\draw(-8.408,3.044)--(-8.435,3.056)--(-8.463,3.007)--(-8.438,2.996);
\filldraw[fill opacity=0.8,fill=gray!20,draw=none](-9.258,.91)--(-9.276,.939)--(-9.27,.904)--(-9.26,.889)--(-9.253,.884)--cycle;
\draw(-9.276,.939)--(-9.27,.904)--(-9.26,.889);
\filldraw[fill opacity=0.8,fill=gray!20,draw=none](-8.396,3.1)--(-8.396,3.088)--(-8.371,3.087)--(-8.365,3.103)--(-8.365,3.108)--(-8.393,3.105)--cycle;
\draw(-8.396,3.088)--(-8.371,3.087);
\draw(-8.365,3.103)--(-8.365,3.108);
\filldraw[fill opacity=0.8,fill=gray!20,draw=none](-8.438,2.996)--(-8.435,2.998)--(-8.427,3.007)--(-8.43,3.015)--cycle;
\filldraw[fill opacity=0.8,fill=gray!20,draw=none](-7.715,1.035)--(-7.812,1.039)--(-7.807,1.003)--(-7.769,1.002)--cycle;
\draw(-7.715,1.035)--(-7.812,1.039)--(-7.807,1.003)--(-7.769,1.002);
\filldraw[fill opacity=0.8,fill=gray!20,draw=none](-8.367,2.917)--(-8.367,2.919)--(-8.368,2.919)--cycle;
\draw(-8.367,2.917)--(-8.367,2.919)--(-8.368,2.919);
\filldraw[fill opacity=0.8,fill=gray!20](-6.857,.405)--(-6.867,.461)--(-6.963,.457)--(-6.964,.4)--cycle;
\filldraw[fill opacity=0.8,fill=gray!20](-6.839,.877)--(-6.792,.904)--(-6.781,.892)--(-6.834,.871)--cycle;
\filldraw[fill opacity=0.8,fill=gray!20,draw=none](-8.368,2.919)--(-8.367,2.919)--(-8.367,2.973)--(-8.377,2.973)--cycle;
\draw(-8.368,2.919)--(-8.367,2.919)--(-8.367,2.973)--(-8.377,2.973);
\filldraw[fill opacity=0.8,fill=gray!20,draw=none](-7.77,1)--(-7.769,1.002)--(-7.773,1.002)--cycle;
\draw(-7.769,1.002)--(-7.773,1.002);
\filldraw[fill opacity=0.8,fill=gray!20,draw=none](-7.773,.999)--(-7.77,1)--(-7.773,1.002)--cycle;
\draw(-7.773,.999)--(-7.77,1);
\filldraw[fill opacity=0.8,fill=gray!20,draw=none](-8.402,3.089)--(-8.396,3.088)--(-8.396,3.1)--cycle;
\draw(-8.402,3.089)--(-8.396,3.088);
\filldraw[fill opacity=0.5,fill=gray!20](-9.769,2.916)--(-9.82,2.906)--(-9.355,3.023)--(-9.294,3.036)--cycle;
\filldraw[fill opacity=0.8,fill=gray!20](-6.853,.348)--(-6.857,.405)--(-6.964,.4)--(-6.965,.343)--cycle;
\filldraw[fill opacity=0.8,fill=gray!20,draw=none](-7.866,.966)--(-7.858,.98)--(-7.867,.978)--cycle;
\draw(-7.858,.98)--(-7.867,.978);
\filldraw[fill opacity=0.8,fill=gray!20,draw=none](-7.676,4.66)--(-7.648,4.656)--(-7.633,4.661)--(-7.633,4.678)--(-7.66,4.68)--cycle;
\draw(-7.633,4.661)--(-7.633,4.678)--(-7.66,4.68);
\filldraw[fill opacity=0.8,fill=gray!20,draw=none](-7.846,.994)--(-7.835,1.019)--(-7.839,1.032)--(-7.888,1.036)--(-7.878,.996)--cycle;
\draw(-7.839,1.032)--(-7.888,1.036)--(-7.878,.996)--(-7.846,.994);
\filldraw[fill opacity=0.8,fill=gray!20,draw=none](-8.091,.928)--(-7.773,.999)--(-7.773,1.002)--(-7.806,1.025)--(-8.072,.966)--cycle;
\draw(-8.091,.928)--(-7.773,.999);
\draw(-7.806,1.025)--(-8.072,.966);
\filldraw[fill opacity=0.8,fill=gray!20,draw=none](-7.813,1.223)--(-7.856,1.204)--(-7.786,1.22)--cycle;
\draw(-7.856,1.204)--(-7.786,1.22);
\filldraw[fill opacity=0.8,fill=gray!20,draw=none](-8.43,3.015)--(-8.427,3.007)--(-8.405,3.033)--(-8.421,3.034)--cycle;
\draw(-8.405,3.033)--(-8.421,3.034);
\filldraw[fill opacity=0.8,fill=gray!20,draw=none](-8.427,3.007)--(-8.417,2.976)--(-8.405,3.033)--cycle;
\filldraw[fill opacity=0.8,fill=gray!20,draw=none](-8.411,3.034)--(-8.405,3.042)--(-8.408,3.044)--cycle;
\draw(-8.405,3.042)--(-8.408,3.044);
\filldraw[fill opacity=0.8,fill=gray!20,draw=none](-8.405,3.072)--(-8.406,3.064)--(-8.375,3.08)--(-8.371,3.087)--(-8.402,3.089)--cycle;
\draw(-8.371,3.087)--(-8.402,3.089);
\filldraw[fill opacity=0.8,fill=gray!20,draw=none](-7.874,1.22)--(-7.859,1.222)--(-7.841,1.252)--(-7.872,1.26)--(-7.906,1.228)--cycle;
\draw(-7.859,1.222)--(-7.841,1.252)--(-7.872,1.26)--(-7.906,1.228)--(-7.874,1.22);
\filldraw[fill opacity=0.8,fill=gray!20,draw=none](-7.597,4.663)--(-7.57,4.671)--(-7.581,4.68)--(-7.633,4.678)--(-7.633,4.671)--cycle;
\draw(-7.581,4.68)--(-7.633,4.678)--(-7.633,4.671);
\filldraw[fill opacity=0.8,fill=gray!20,draw=none](-7.773,1.002)--(-7.768,1.034)--(-7.806,1.025)--cycle;
\draw(-7.768,1.034)--(-7.806,1.025);
\filldraw[fill opacity=0.8,fill=gray!20,draw=none](-8.407,3.063)--(-8.406,3.064)--(-8.405,3.072)--cycle;
\filldraw[fill opacity=0.8,fill=gray!20,draw=none](-8.122,2.974)--(-8.112,3.022)--(-8.043,3.025)--(-8.021,3.011)--(-8.015,3.001)--(-8.014,2.979)--cycle;
\draw(-8.015,3.001)--(-8.014,2.979)--(-8.122,2.974)--(-8.112,3.022)--(-8.043,3.025);
\filldraw[fill opacity=0.8,fill=gray!20,draw=none](-8.233,3.062)--(-8.202,3.048)--(-8.177,3.072)--(-8.21,3.086)--cycle;
\draw(-8.233,3.062)--(-8.202,3.048);
\draw(-8.177,3.072)--(-8.21,3.086);
\filldraw[fill opacity=0.8,fill=gray!20,draw=none](-8.3,3.091)--(-8.228,3.06)--(-8.21,3.086)--(-8.267,3.111)--cycle;
\draw(-8.3,3.091)--(-8.228,3.06);
\draw(-8.21,3.086)--(-8.267,3.111);
\filldraw[fill opacity=0.8,fill=gray!20,draw=none](-8.406,3.052)--(-8.406,3.064)--(-8.407,3.063)--(-8.411,3.045)--cycle;
\filldraw[fill opacity=0.8,fill=gray!20](-7.796,1.249)--(-7.793,1.276)--(-7.816,1.278)--(-7.841,1.252)--cycle;
\filldraw[fill opacity=0.8,fill=gray!20,draw=none](-7.853,.964)--(-7.846,.965)--(-7.846,.983)--(-7.864,.969)--(-7.862,.964)--cycle;
\draw(-7.864,.969)--(-7.862,.964)--(-7.853,.964);
\filldraw[fill opacity=0.8,fill=gray!20,draw=none](-9.267,.959)--(-9.271,.931)--(-9.258,.91)--cycle;
\filldraw[fill opacity=0.8,fill=gray!20,draw=none](-7.741,1.22)--(-7.786,1.222)--(-7.785,1.184)--(-7.78,1.184)--cycle;
\draw(-7.741,1.22)--(-7.786,1.222);
\draw(-7.785,1.184)--(-7.78,1.184);
\filldraw[fill opacity=0.5,fill=gray!20](-9.167,2.981)--(-9.231,3.021)--(-8.753,3.005)--(-8.693,2.964)--cycle;
\filldraw[fill opacity=0.8,fill=gray!20,draw=none](-8.375,3.08)--(-8.406,3.064)--(-8.406,3.033)--cycle;
\filldraw[fill opacity=0.8,fill=gray!20,draw=none](-7.581,4.68)--(-7.632,4.732)--(-7.633,4.678)--cycle;
\draw(-7.632,4.732)--(-7.633,4.678)--(-7.581,4.68);
\filldraw[fill opacity=0.8,fill=gray!20,draw=none](-8.371,3.087)--(-8.366,3.086)--(-8.365,3.103)--cycle;
\draw(-8.371,3.087)--(-8.366,3.086)--(-8.365,3.103);
\filldraw[fill opacity=0.8,fill=gray!20,draw=none](-8.411,3.045)--(-8.421,3.034)--(-8.414,3.033)--cycle;
\draw(-8.421,3.034)--(-8.414,3.033);
\filldraw[fill opacity=0.8,fill=gray!20,draw=none](-6.159,.475)--(-6.159,.453)--(-6.103,.482)--cycle;
\draw(-6.159,.475)--(-6.159,.453);
\filldraw[fill opacity=0.8,fill=gray!20,draw=none](-8.381,2.924)--(-8.358,2.913)--(-8.275,2.922)--(-8.39,2.972)--cycle;
\draw(-8.381,2.924)--(-8.358,2.913);
\draw(-8.275,2.922)--(-8.39,2.972);
\filldraw[fill opacity=0.8,fill=gray!20,draw=none](-8.381,2.924)--(-8.343,2.907)--(-8.334,2.947)--(-8.39,2.972)--cycle;
\draw(-8.381,2.924)--(-8.343,2.907);
\draw(-8.334,2.947)--(-8.39,2.972);
\filldraw[fill opacity=0.8,fill=gray!20,draw=none](-8.406,3.052)--(-8.411,3.045)--(-8.414,3.033)--(-8.406,3.033)--cycle;
\draw(-8.414,3.033)--(-8.406,3.033);
\filldraw[fill opacity=0.8,fill=gray!20,draw=none](-6.132,.422)--(-6.124,.437)--(-6.159,.453)--cycle;
\draw(-6.132,.422)--(-6.124,.437)--(-6.159,.453);
\filldraw[fill opacity=0.8,fill=gray!20,draw=none](-7.66,4.68)--(-7.633,4.678)--(-7.632,4.727)--cycle;
\draw(-7.66,4.68)--(-7.633,4.678)--(-7.632,4.727);
\filldraw[fill opacity=0.8,fill=gray!20,draw=none](-7.597,4.663)--(-7.633,4.671)--(-7.633,4.653)--cycle;
\draw(-7.633,4.671)--(-7.633,4.653);
\filldraw[fill opacity=0.8,fill=gray!20,draw=none](-6.031,.489)--(-6.047,.496)--(-6.093,.478)--(-6.081,.473)--cycle;
\draw(-6.093,.478)--(-6.081,.473)--(-6.031,.489)--(-6.047,.496);
\filldraw[fill opacity=0.8,fill=gray!20,draw=none](-6.103,.529)--(-6.103,.526)--(-6.071,.491)--(-6.047,.496)--(-6.047,.532)--cycle;
\draw(-6.047,.496)--(-6.047,.532)--(-6.103,.529)--(-6.103,.526);
\filldraw[fill opacity=0.8,fill=gray!20](-7.749,1.251)--(-7.769,1.278)--(-7.793,1.276)--(-7.796,1.249)--cycle;
\filldraw[fill opacity=0.8,fill=gray!20,draw=none](-7.853,.964)--(-7.798,.96)--(-7.799,.971)--cycle;
\draw(-7.853,.964)--(-7.798,.96)--(-7.799,.971);
\filldraw[fill opacity=0.8,fill=gray!20,draw=none](-7.648,4.656)--(-7.633,4.653)--(-7.633,4.661)--cycle;
\draw(-7.633,4.653)--(-7.633,4.661);
\filldraw[fill opacity=0.8,fill=gray!20,draw=none](-7.864,.969)--(-7.846,.983)--(-7.858,.98)--cycle;
\draw(-7.846,.983)--(-7.858,.98);
\filldraw[fill opacity=0.8,fill=gray!20,draw=none](-7.663,4.082)--(-7.555,4.111)--(-7.555,4.086)--cycle;
\draw(-7.555,4.111)--(-7.555,4.086)--(-7.663,4.082);
\filldraw[fill opacity=0.8,fill=gray!20](-7.653,4.026)--(-7.663,4.082)--(-7.555,4.086)--(-7.556,4.03)--cycle;
\filldraw[fill opacity=0.8,fill=gray!20](-7.637,3.974)--(-7.653,4.026)--(-7.556,4.03)--(-7.558,3.978)--cycle;
\filldraw[fill opacity=0.8,fill=gray!20,draw=none](-7.54,4.109)--(-7.53,4)--(-7.581,3.995)--(-7.592,4.116)--cycle;
\draw(-7.54,4.109)--(-7.53,4)--(-7.581,3.995)--(-7.592,4.116);
\filldraw[fill opacity=0.8,fill=gray!20,draw=none](-7.642,4.128)--(-7.641,4.124)--(-7.666,4.133)--(-7.668,4.134)--cycle;
\draw(-7.642,4.128)--(-7.641,4.124);
\filldraw[fill opacity=0.8,fill=gray!20,draw=none](-7.641,4.124)--(-7.64,4.112)--(-7.666,4.133)--cycle;
\draw(-7.641,4.124)--(-7.64,4.112);
\filldraw[fill opacity=0.8,fill=gray!20,draw=none](-7.663,4.082)--(-7.666,4.139)--(-7.555,4.144)--(-7.555,4.111)--cycle;
\draw(-7.663,4.082)--(-7.666,4.139)--(-7.555,4.144)--(-7.555,4.111);
\filldraw[fill opacity=0.8,fill=gray!20,draw=none](-7.668,4.134)--(-7.68,4.138)--(-7.681,4.145)--cycle;
\draw(-7.68,4.138)--(-7.681,4.145);
\filldraw[fill opacity=0.8,fill=gray!20,draw=none](-7.741,4.067)--(-7.73,4.126)--(-7.676,4.137)--(-7.666,4.133)--(-7.663,4.082)--cycle;
\draw(-7.73,4.126)--(-7.676,4.137);
\draw(-7.666,4.133)--(-7.663,4.082)--(-7.741,4.067);
\filldraw[fill opacity=0.8,fill=gray!20](-7.726,4.011)--(-7.745,4.066)--(-7.663,4.082)--(-7.653,4.026)--cycle;
\filldraw[fill opacity=0.8,fill=gray!20,draw=none](-7.68,4.138)--(-7.703,4.168)--(-7.704,4.177)--cycle;
\draw(-7.703,4.168)--(-7.704,4.177);
\filldraw[fill opacity=0.8,fill=gray!20,draw=none](-7.676,4.137)--(-7.68,4.137)--(-7.703,4.168)--(-7.705,4.184)--(-7.663,4.193)--(-7.666,4.139)--cycle;
\draw(-7.705,4.184)--(-7.663,4.193)--(-7.666,4.139)--(-7.676,4.137);
\filldraw[fill opacity=0.8,fill=gray!20,draw=none](-7.703,4.168)--(-7.679,4.136)--(-7.699,4.132)--cycle;
\draw(-7.679,4.136)--(-7.699,4.132);
\filldraw[fill opacity=0.8,fill=gray!20,draw=none](-7.68,4.138)--(-7.67,4.028)--(-7.694,4.062)--(-7.703,4.168)--cycle;
\draw(-7.68,4.138)--(-7.67,4.028)--(-7.694,4.062)--(-7.703,4.168);
\filldraw[fill opacity=0.8,fill=gray!20,draw=none](-7.723,4.571)--(-7.683,4.139)--(-7.651,4.171)--(-7.688,4.574)--cycle;
\draw(-7.723,4.571)--(-7.683,4.139)--(-7.651,4.171)--(-7.688,4.574);
\filldraw[fill opacity=0.8,fill=gray!20](-6.89,.864)--(-6.947,.872)--(-6.938,.878)--(-6.89,.864)--cycle;
\filldraw[fill opacity=0.8,fill=gray!20,draw=none](-6.107,.529)--(-6.159,.53)--(-6.159,.475)--(-6.103,.482)--(-6.103,.526)--cycle;
\draw(-6.107,.529)--(-6.159,.53)--(-6.159,.475);
\draw(-6.103,.482)--(-6.103,.526);
\filldraw[fill opacity=0.8,fill=gray!20,draw=none](-9.011,.911)--(-9.002,.953)--(-9.004,.96)--(-9.023,.916)--cycle;
\draw(-9.004,.96)--(-9.023,.916)--(-9.011,.911);
\filldraw[fill opacity=0.8,fill=gray!20,draw=none](-8.948,.978)--(-9.194,1.085)--(-9.232,1.053)--(-8.987,.946)--cycle;
\draw(-9.232,1.053)--(-8.987,.946)--(-8.948,.978)--(-9.194,1.085);
\filldraw[fill opacity=0.8,fill=gray!20](-7.682,1.225)--(-7.714,1.258)--(-7.749,1.251)--(-7.733,1.215)--cycle;
\filldraw[fill opacity=0.8,fill=gray!20,draw=none](-8.375,3.08)--(-8.366,3.085)--(-8.366,3.086)--(-8.371,3.087)--cycle;
\draw(-8.366,3.085)--(-8.366,3.086)--(-8.371,3.087);
\filldraw[fill opacity=0.8,fill=gray!20,draw=none](-7.846,.965)--(-7.799,.971)--(-7.8,.99)--(-7.833,.993)--(-7.846,.983)--cycle;
\draw(-7.799,.971)--(-7.8,.99)--(-7.833,.993);
\filldraw[fill opacity=0.8,fill=gray!20](-7.643,1.047)--(-7.638,1.092)--(-7.709,1.079)--(-7.711,1.034)--cycle;
\filldraw[fill opacity=0.8,fill=gray!20,draw=none](-7.874,1.22)--(-7.906,1.228)--(-7.914,1.216)--cycle;
\draw(-7.874,1.22)--(-7.906,1.228)--(-7.914,1.216);
\filldraw[fill opacity=0.8,fill=gray!20,draw=none](-7.733,.963)--(-7.723,.985)--(-7.799,.971)--(-7.798,.96)--cycle;
\draw(-7.799,.971)--(-7.798,.96)--(-7.733,.963)--(-7.723,.985);
\filldraw[fill opacity=0.8,fill=gray!20,draw=none](-6.72,.944)--(-6.714,.952)--(-6.724,.943)--cycle;
\draw(-6.72,.944)--(-6.714,.952);
\filldraw[fill opacity=0.8,fill=gray!20,draw=none](-6.881,.971)--(-6.716,.964)--(-6.713,.943)--(-6.89,.95)--cycle;
\draw(-6.713,.943)--(-6.89,.95)--(-6.881,.971)--(-6.716,.964);
\filldraw[fill opacity=0.8,fill=gray!20,draw=none](-6.071,.491)--(-6.079,.5)--(-6.103,.482)--cycle;
\filldraw[fill opacity=0.8,fill=gray!20,draw=none](-8.382,3.032)--(-8.382,3.031)--(-8.367,3.03)--(-8.367,3.039)--cycle;
\draw(-8.382,3.031)--(-8.367,3.03)--(-8.367,3.039);
\filldraw[fill opacity=0.8,fill=gray!20,draw=none](-8.382,3.032)--(-8.367,3.039)--(-8.366,3.056)--cycle;
\draw(-8.367,3.039)--(-8.366,3.056);
\filldraw[fill opacity=0.8,fill=gray!20,draw=none](-8.359,3.068)--(-8.356,3.067)--(-8.346,3.075)--cycle;
\draw(-8.359,3.068)--(-8.356,3.067);
\filldraw[fill opacity=0.8,fill=gray!20,draw=none](-8.359,3.068)--(-8.356,3.067)--(-8.346,3.075)--cycle;
\draw(-8.359,3.068)--(-8.356,3.067);
\filldraw[fill opacity=0.8,fill=gray!20,draw=none](-8.4,2.99)--(-8.403,2.97)--(-8.402,2.961)--(-8.332,3.104)--(-8.337,3.099)--cycle;
\draw(-8.4,2.99)--(-8.403,2.97)--(-8.402,2.961);
\draw(-8.332,3.104)--(-8.337,3.099);
\filldraw[fill opacity=0.8,fill=gray!20,draw=none](-8.254,3.022)--(-8.228,3.011)--(-8.202,3.048)--(-8.233,3.062)--cycle;
\draw(-8.254,3.022)--(-8.228,3.011);
\draw(-8.202,3.048)--(-8.233,3.062);
\filldraw[fill opacity=0.8,fill=gray!20,draw=none](-8.356,3.067)--(-8.25,3.02)--(-8.228,3.06)--(-8.318,3.099)--cycle;
\draw(-8.356,3.067)--(-8.25,3.02);
\draw(-8.228,3.06)--(-8.318,3.099);
\filldraw[fill opacity=0.8,fill=gray!20,draw=none](-8.356,3.067)--(-8.254,3.022)--(-8.233,3.062)--(-8.318,3.099)--cycle;
\draw(-8.356,3.067)--(-8.254,3.022);
\draw(-8.233,3.062)--(-8.318,3.099);
\filldraw[fill opacity=0.8,fill=gray!20,draw=none](-8.365,3.086)--(-8.366,3.086)--(-8.366,3.085)--cycle;
\draw(-8.365,3.086)--(-8.366,3.086)--(-8.366,3.085);
\filldraw[fill opacity=0.8,fill=gray!20,draw=none](-7.689,1.179)--(-7.689,1.181)--(-7.807,1.185)--(-7.812,1.137)--(-7.694,1.133)--cycle;
\draw(-7.689,1.181)--(-7.807,1.185)--(-7.812,1.137)--(-7.694,1.133);
\filldraw[fill opacity=0.8,fill=gray!20,draw=none](-7.88,1.172)--(-7.865,1.161)--(-7.78,1.181)--(-7.824,1.211)--(-7.886,1.197)--cycle;
\draw(-7.865,1.161)--(-7.78,1.181);
\draw(-7.824,1.211)--(-7.886,1.197);
\filldraw[fill opacity=0.8,fill=gray!20,draw=none](-7.872,1.189)--(-7.862,1.217)--(-7.874,1.22)--(-7.914,1.216)--cycle;
\draw(-7.872,1.189)--(-7.862,1.217)--(-7.874,1.22);
\filldraw[fill opacity=0.8,fill=gray!20,draw=none](-7.048,1.215)--(-7.039,1.236)--(-7.023,1.247)--(-7.036,1.226)--(-7.044,1.215)--cycle;
\draw(-7.039,1.236)--(-7.023,1.247);
\draw(-7.036,1.226)--(-7.044,1.215);
\filldraw[fill opacity=0.8,fill=gray!20,draw=none](-7.733,4.518)--(-7.732,4.468)--(-7.719,4.321)--(-7.699,4.311)--(-7.717,4.511)--cycle;
\draw(-7.732,4.468)--(-7.719,4.321);
\draw(-7.699,4.311)--(-7.717,4.511);
\filldraw[fill opacity=0.8,fill=gray!20,draw=none](-6.666,.999)--(-6.178,.98)--(-6.167,.971)--(-6.215,.944)--(-6.7,.963)--cycle;
\draw(-6.666,.999)--(-6.178,.98);
\draw(-6.215,.944)--(-6.7,.963);
\filldraw[fill opacity=0.8,fill=gray!20,draw=none](-8.377,2.973)--(-8.367,2.973)--(-8.367,3.03)--(-8.368,3.03)--cycle;
\draw(-8.377,2.973)--(-8.367,2.973)--(-8.367,3.03)--(-8.368,3.03);
\filldraw[fill opacity=0.8,fill=gray!20,draw=none](-8.388,3.025)--(-8.381,3.022)--(-8.364,3.052)--cycle;
\draw(-8.388,3.025)--(-8.381,3.022);
\filldraw[fill opacity=0.8,fill=gray!20,draw=none](-8.382,3.031)--(-8.388,3.025)--(-8.386,3.024)--cycle;
\draw(-8.388,3.025)--(-8.386,3.024);
\filldraw[fill opacity=0.8,fill=gray!20](-6.89,.864)--(-6.839,.877)--(-6.834,.871)--(-6.89,.864)--cycle;
\filldraw[fill opacity=0.8,fill=gray!20,draw=none](-9.267,.959)--(-9.273,.994)--(-9.28,.957)--(-9.276,.939)--(-9.271,.931)--cycle;
\draw(-9.273,.994)--(-9.28,.957)--(-9.276,.939);
\filldraw[fill opacity=0.8,fill=gray!20,draw=none](-6.079,.5)--(-6.103,.526)--(-6.103,.482)--cycle;
\draw(-6.103,.526)--(-6.103,.482);
\filldraw[fill opacity=0.8,fill=gray!20,draw=none](-7.475,4.494)--(-7.442,4.129)--(-7.437,4.09)--(-7.471,4.466)--cycle;
\draw(-7.475,4.494)--(-7.442,4.129)--(-7.437,4.09)--(-7.471,4.466);
\filldraw[fill opacity=0.8,fill=gray!20,draw=none](-7.67,.996)--(-7.661,.998)--(-7.669,.998)--cycle;
\draw(-7.661,.998)--(-7.669,.998);
\filldraw[fill opacity=0.8,fill=gray!20,draw=none](-6.71,1.194)--(-6.703,1.183)--(-6.709,1.183)--cycle;
\draw(-6.703,1.183)--(-6.709,1.183);
\filldraw[fill opacity=0.8,fill=gray!20,draw=none](-6.71,1.194)--(-6.711,1.201)--(-6.659,1.199)--(-6.687,1.182)--(-6.703,1.183)--cycle;
\draw(-6.711,1.201)--(-6.659,1.199);
\draw(-6.687,1.182)--(-6.703,1.183);
\filldraw[fill opacity=0.8,fill=gray!20,draw=none](-6.698,1.152)--(-6.713,1.197)--(-6.7,1.183)--(-6.679,1.131)--cycle;
\draw(-6.713,1.197)--(-6.7,1.183)--(-6.679,1.131)--(-6.698,1.152);
\filldraw[fill opacity=0.8,fill=gray!20,draw=none](-7.689,.999)--(-7.689,1)--(-7.69,1)--cycle;
\draw(-7.689,1)--(-7.69,1);
\filldraw[fill opacity=0.8,fill=gray!20,draw=none](-7.689,1)--(-7.689,.999)--(-7.688,.999)--cycle;
\draw(-7.689,.999)--(-7.688,.999);
\filldraw[fill opacity=0.8,fill=gray!20,draw=none](-7.082,1.005)--(-7.082,1.01)--(-7.087,1.057)--(-7.109,1.081)--(-7.102,1.026)--cycle;
\draw(-7.087,1.057)--(-7.109,1.081)--(-7.102,1.026)--(-7.082,1.005);
\filldraw[fill opacity=0.8,fill=gray!20,draw=none](-6.912,1.003)--(-7.677,1.034)--(-7.689,1)--(-7.688,.999)--(-6.907,.968)--cycle;
\draw(-7.688,.999)--(-6.907,.968)--(-6.912,1.003)--(-7.677,1.034);
\filldraw[fill opacity=0.8,fill=gray!20,draw=none](-8.382,3.032)--(-8.384,3.031)--(-8.382,3.031)--cycle;
\draw(-8.384,3.031)--(-8.382,3.031);
\filldraw[fill opacity=0.8,fill=gray!20,draw=none](-8.382,3.031)--(-8.386,3.024)--(-8.381,3.022)--(-8.364,3.052)--cycle;
\draw(-8.386,3.024)--(-8.381,3.022);
\filldraw[fill opacity=0.8,fill=gray!20,draw=none](-6.716,1.202)--(-6.71,1.194)--(-6.719,1.203)--cycle;
\draw(-6.71,1.194)--(-6.719,1.203);
\filldraw[fill opacity=0.8,fill=gray!20,draw=none](-6.881,1.208)--(-6.711,1.201)--(-6.709,1.183)--(-6.874,1.189)--cycle;
\draw(-6.709,1.183)--(-6.874,1.189)--(-6.881,1.208)--(-6.711,1.201);
\filldraw[fill opacity=0.8,fill=gray!20,draw=none](-8.382,3.031)--(-8.364,3.052)--(-8.356,3.067)--(-8.359,3.068)--cycle;
\draw(-8.356,3.067)--(-8.359,3.068);
\filldraw[fill opacity=0.8,fill=gray!20,draw=none](-8.382,3.031)--(-8.364,3.052)--(-8.356,3.067)--(-8.359,3.068)--cycle;
\draw(-8.356,3.067)--(-8.359,3.068);
\filldraw[fill opacity=0.8,fill=gray!20,draw=none](-7.689,1)--(-7.677,1.034)--(-7.694,1.034)--cycle;
\draw(-7.677,1.034)--(-7.694,1.034);
\filldraw[fill opacity=0.8,fill=gray!20,draw=none](-9.26,.994)--(-9.272,.999)--(-9.273,.994)--(-9.267,.959)--cycle;
\draw(-9.272,.999)--(-9.273,.994);
\filldraw[fill opacity=0.8,fill=gray!20,draw=none](-7.69,1)--(-7.715,1.018)--(-7.72,.994)--cycle;
\draw(-7.715,1.018)--(-7.72,.994)--(-7.69,1);
\filldraw[fill opacity=0.8,fill=gray!20,draw=none](-8.39,2.972)--(-8.29,2.928)--(-8.271,2.974)--(-8.381,3.022)--cycle;
\draw(-8.39,2.972)--(-8.29,2.928);
\draw(-8.271,2.974)--(-8.381,3.022);
\filldraw[fill opacity=0.8,fill=gray!20,draw=none](-8.39,2.972)--(-8.29,2.928)--(-8.276,2.963)--(-8.278,2.977)--(-8.381,3.022)--cycle;
\draw(-8.39,2.972)--(-8.29,2.928);
\draw(-8.278,2.977)--(-8.381,3.022);
\filldraw[fill opacity=0.8,fill=gray!20,draw=none](-7.023,1.247)--(-7.02,1.249)--(-7.036,1.226)--cycle;
\draw(-7.023,1.247)--(-7.02,1.249)--(-7.036,1.226);
\filldraw[fill opacity=0.8,fill=gray!20](-6.884,1.288)--(-6.887,1.298)--(-6.859,1.296)--(-6.83,1.284)--cycle;
\filldraw[fill opacity=0.8,fill=gray!20](-6.939,1.286)--(-6.916,1.297)--(-6.887,1.298)--(-6.884,1.288)--cycle;
\filldraw[fill opacity=0.8,fill=gray!20,draw=none](-8.4,2.99)--(-8.337,3.099)--(-8.365,3.076)--(-8.393,3.026)--cycle;
\draw(-8.337,3.099)--(-8.365,3.076)--(-8.393,3.026)--(-8.4,2.99);
\filldraw[fill opacity=0.8,fill=gray!20,draw=none](-8.336,2.876)--(-8.276,2.85)--(-8.278,2.878)--(-8.343,2.907)--cycle;
\draw(-8.336,2.876)--(-8.276,2.85);
\draw(-8.278,2.878)--(-8.343,2.907);
\filldraw[fill opacity=0.8,fill=gray!20,draw=none](-8.094,2.775)--(-8.104,2.776)--(-8.134,2.793)--(-8.09,2.795)--(-8.085,2.777)--cycle;
\draw(-8.09,2.795)--(-8.085,2.777)--(-8.094,2.775);
\filldraw[fill opacity=0.8,fill=gray!20,draw=none](-8.094,2.775)--(-8.085,2.777)--(-8.083,2.774)--cycle;
\draw(-8.094,2.775)--(-8.085,2.777)--(-8.083,2.774);
\filldraw[fill opacity=0.8,fill=gray!20,draw=none](-8.104,2.776)--(-8.134,2.793)--(-8.086,2.903)--(-8.041,2.884)--(-8.089,2.774)--cycle;
\draw(-8.134,2.793)--(-8.086,2.903)--(-8.041,2.884)--(-8.089,2.774);
\filldraw[fill opacity=0.8,fill=gray!20,draw=none](-8.209,2.844)--(-8.211,2.85)--(-8.22,2.898)--(-8.21,2.904)--(-8.204,2.848)--cycle;
\draw(-8.22,2.898)--(-8.21,2.904)--(-8.204,2.848)--(-8.209,2.844);
\filldraw[fill opacity=0.8,fill=gray!20,draw=none](-8.209,2.844)--(-8.204,2.848)--(-8.195,2.822)--cycle;
\draw(-8.209,2.844)--(-8.204,2.848)--(-8.195,2.822);
\filldraw[fill opacity=0.8,fill=gray!20,draw=none](-8.163,2.8)--(-8.165,2.802)--(-8.17,2.809)--cycle;
\filldraw[fill opacity=0.8,fill=gray!20,draw=none](-8.19,2.843)--(-8.198,2.888)--(-8.18,2.9)--(-8.174,2.853)--cycle;
\draw(-8.198,2.888)--(-8.18,2.9)--(-8.174,2.853)--(-8.19,2.843);
\filldraw[fill opacity=0.8,fill=gray!20,draw=none](-8.18,2.836)--(-8.179,2.845)--(-8.143,2.928)--(-8.122,2.919)--(-8.17,2.809)--cycle;
\draw(-8.179,2.845)--(-8.143,2.928)--(-8.122,2.919)--(-8.17,2.809);
\filldraw[fill opacity=0.8,fill=gray!20,draw=none](-8.336,2.876)--(-8.11,2.778)--(-8.136,2.817)--(-8.343,2.907)--cycle;
\draw(-8.336,2.876)--(-8.11,2.778)--(-8.136,2.817)--(-8.343,2.907);
\filldraw[fill opacity=0.8,fill=gray!20,draw=none](-8.844,.839)--(-8.889,.858)--(-8.936,.879)--(-8.978,.897)--(-9.004,.908)--(-9.011,.911)--(-9.023,.916)--(-9.02,.915)--(-8.998,.906)--(-8.962,.89)--(-8.917,.87)--(-8.87,.85)--(-8.828,.832)--(-8.798,.818)--(-8.783,.812)--(-8.787,.814)--(-8.808,.823)--cycle;
\draw(-9.011,.911)--(-9.023,.916)--(-9.02,.915)--(-8.998,.906)--(-8.962,.89)--(-8.917,.87)--(-8.87,.85)--(-8.828,.832)--(-8.798,.818)--(-8.783,.812)--(-8.787,.814)--(-8.808,.823)--(-8.844,.839)--(-8.889,.858)--(-8.936,.879)--(-8.978,.897)--(-9.004,.908);
\filldraw[fill opacity=0.8,fill=gray!20,draw=none](-9.085,.942)--(-9.072,.968)--(-9.08,.945)--cycle;
\draw(-9.072,.968)--(-9.08,.945)--(-9.085,.942);
\filldraw[fill opacity=0.8,fill=gray!20,draw=none](-8.987,.946)--(-9.232,1.053)--(-9.258,1.009)--(-9.012,.902)--cycle;
\draw(-9.258,1.009)--(-9.012,.902)--(-8.987,.946)--(-9.232,1.053);
\filldraw[fill opacity=0.8,fill=gray!20,draw=none](-7.677,1.217)--(-7.677,1.216)--(-7.676,1.214)--cycle;
\draw(-7.677,1.216)--(-7.676,1.214);
\filldraw[fill opacity=0.8,fill=gray!20,draw=none](-7.69,1.178)--(-7.658,1.185)--(-7.676,1.214)--cycle;
\draw(-7.69,1.178)--(-7.658,1.185)--(-7.676,1.214);
\filldraw[fill opacity=0.8,fill=gray!20,draw=none](-7.694,1.133)--(-7.693,1.13)--(-7.643,1.139)--(-7.658,1.185)--(-7.689,1.179)--cycle;
\draw(-7.693,1.13)--(-7.643,1.139)--(-7.658,1.185)--(-7.689,1.179);
\filldraw[fill opacity=0.8,fill=gray!20,draw=none](-7.082,1.116)--(-7.072,1.156)--(-7.07,1.178)--(-7.08,1.188)--(-7.102,1.137)--cycle;
\draw(-7.07,1.178)--(-7.08,1.188)--(-7.102,1.137)--(-7.082,1.116);
\filldraw[fill opacity=0.8,fill=gray!20,draw=none](-6.899,1.187)--(-7.677,1.217)--(-7.669,1.18)--(-6.907,1.149)--cycle;
\draw(-7.669,1.18)--(-6.907,1.149)--(-6.899,1.187)--(-7.677,1.217);
\filldraw[fill opacity=0.8,fill=gray!20,draw=none](-9.272,.999)--(-9.26,.994)--(-9.252,1.039)--(-9.254,1.04)--(-9.27,1.013)--cycle;
\draw(-9.254,1.04)--(-9.27,1.013)--(-9.272,.999);
\filldraw[fill opacity=0.8,fill=gray!20,draw=none](-8.381,3.022)--(-8.271,2.974)--(-8.25,3.02)--(-8.356,3.067)--cycle;
\draw(-8.381,3.022)--(-8.271,2.974);
\draw(-8.25,3.02)--(-8.356,3.067);
\filldraw[fill opacity=0.8,fill=gray!20,draw=none](-8.381,3.022)--(-8.271,2.974)--(-8.25,3.02)--(-8.356,3.067)--cycle;
\draw(-8.381,3.022)--(-8.271,2.974);
\draw(-8.25,3.02)--(-8.356,3.067);
\filldraw[fill opacity=0.5,fill=gray!20](-10.506,2.133)--(-10.333,2.058)--(-10.027,2.347)--(-10.2,2.423)--cycle;
\filldraw[fill opacity=0.5,fill=gray!20](-10.636,2.262)--(-10.506,2.133)--(-10.2,2.423)--(-10.292,2.587)--cycle;
\filldraw[fill opacity=0.8,fill=gray!20,draw=none](-9.258,1.009)--(-9.243,1.035)--(-9.252,1.039)--cycle;
\filldraw[fill opacity=0.8,fill=gray!20,draw=none](-7.677,1.216)--(-7.682,1.225)--(-7.733,1.215)--(-7.725,1.192)--cycle;
\draw(-7.677,1.216)--(-7.682,1.225)--(-7.733,1.215)--(-7.725,1.192);
\filldraw[fill opacity=0.8,fill=gray!20,draw=none](-8.057,3.034)--(-8.043,3.025)--(-8.112,3.022)--(-8.104,3.04)--cycle;
\draw(-8.043,3.025)--(-8.112,3.022)--(-8.104,3.04);
\filldraw[fill opacity=0.8,fill=gray!20,draw=none](-8.098,2.998)--(-8.085,3.029)--(-8.073,3.03)--(-8.036,3.025)--(-8.018,3.013)--(-8.017,3.001)--cycle;
\draw(-8.018,3.013)--(-8.017,3.001)--(-8.098,2.998)--(-8.085,3.029)--(-8.073,3.03);
\filldraw[fill opacity=0.8,fill=gray!20,draw=none](-7.998,3.001)--(-7.997,3)--(-8.017,3.001)--(-8.018,3.013)--cycle;
\draw(-7.997,3)--(-8.017,3.001)--(-8.018,3.013);
\filldraw[fill opacity=0.8,fill=gray!20,draw=none](-8.073,3.03)--(-8.085,3.029)--(-8.083,3.031)--cycle;
\draw(-8.073,3.03)--(-8.085,3.029)--(-8.083,3.031);
\filldraw[fill opacity=0.8,fill=gray!20,draw=none](-8.21,3.086)--(-8.027,3.006)--(-7.982,3.001)--(-8.198,3.096)--cycle;
\draw(-8.21,3.086)--(-8.027,3.006)--(-7.982,3.001)--(-8.198,3.096);
\filldraw[fill opacity=0.8,fill=gray!20,draw=none](-8.21,3.086)--(-8.027,3.006)--(-7.982,3.001)--(-8.198,3.096)--cycle;
\draw(-8.21,3.086)--(-8.027,3.006)--(-7.982,3.001)--(-8.198,3.096);
\filldraw[fill opacity=0.8,fill=gray!20,draw=none](-7.693,1.133)--(-7.777,1.136)--(-7.734,1.083)--(-7.695,1.081)--cycle;
\draw(-7.693,1.133)--(-7.777,1.136);
\draw(-7.734,1.083)--(-7.695,1.081);
\filldraw[fill opacity=0.8,fill=gray!20](-7.638,1.092)--(-7.643,1.139)--(-7.711,1.126)--(-7.709,1.079)--cycle;
\filldraw[fill opacity=0.8,fill=gray!20,draw=none](-7.039,.933)--(-7.045,.929)--(-7.055,.942)--cycle;
\draw(-7.039,.933)--(-7.045,.929)--(-7.055,.942);
\filldraw[fill opacity=0.8,fill=gray!20,draw=none](-7.69,1.178)--(-7.676,1.214)--(-7.677,1.216)--(-7.725,1.192)--(-7.72,1.173)--cycle;
\draw(-7.676,1.214)--(-7.677,1.216);
\draw(-7.725,1.192)--(-7.72,1.173)--(-7.69,1.178);
\filldraw[fill opacity=0.8,fill=gray!20,draw=none](-7.676,1.214)--(-7.689,1.181)--(-7.669,1.18)--cycle;
\draw(-7.689,1.181)--(-7.669,1.18);
\filldraw[fill opacity=0.8,fill=gray!20,draw=none](-6.716,1.202)--(-6.719,1.203)--(-6.72,1.204)--(-6.751,1.245)--(-6.735,1.228)--(-6.725,1.215)--cycle;
\draw(-6.719,1.203)--(-6.72,1.204)--(-6.751,1.245)--(-6.735,1.228)--(-6.725,1.215);
\filldraw[fill opacity=0.8,fill=gray!20,draw=none](-6.107,.529)--(-6.103,.526)--(-6.103,.529)--cycle;
\draw(-6.103,.526)--(-6.103,.529)--(-6.107,.529);
\filldraw[fill opacity=0.5,fill=gray!20](-10.605,.262)--(-11.066,.464)--(-11.123,.927)--(-10.661,.725)--cycle;
\filldraw[fill opacity=0.5,fill=gray!20](-11.092,.411)--(-11.066,.464)--(-11.123,.927)--(-11.151,.896)--cycle;
\filldraw[fill opacity=0.8,fill=gray!20](-6.982,1.277)--(-6.938,1.292)--(-6.916,1.297)--(-6.939,1.286)--cycle;
\filldraw[fill opacity=0.8,fill=gray!20,draw=none](-7.846,.983)--(-7.833,.993)--(-7.846,.994)--cycle;
\draw(-7.833,.993)--(-7.846,.994);
\filldraw[fill opacity=0.8,fill=gray!20,draw=none](-7.048,1.215)--(-7.06,1.213)--(-7.045,1.232)--(-7.039,1.236)--cycle;
\draw(-7.06,1.213)--(-7.045,1.232)--(-7.039,1.236);
\filldraw[fill opacity=0.8,fill=gray!20,draw=none](-7.723,.985)--(-7.72,.994)--(-7.8,.99)--(-7.799,.971)--cycle;
\draw(-7.723,.985)--(-7.72,.994)--(-7.8,.99)--(-7.799,.971);
\filldraw[fill opacity=0.8,fill=gray!20](-7.798,1.212)--(-7.796,1.249)--(-7.841,1.252)--(-7.862,1.217)--cycle;
\filldraw[fill opacity=0.8,fill=gray!20,draw=none](-7.839,1.032)--(-7.835,1.065)--(-7.841,1.077)--(-7.891,1.081)--(-7.888,1.036)--cycle;
\draw(-7.841,1.077)--(-7.891,1.081)--(-7.888,1.036)--(-7.839,1.032);
\filldraw[fill opacity=0.8,fill=gray!20,draw=none](-8.569,.89)--(-8.475,.897)--(-8.549,.88)--cycle;
\draw(-8.475,.897)--(-8.549,.88);
\filldraw[fill opacity=0.8,fill=gray!20,draw=none](-8.59,.849)--(-7.768,1.034)--(-7.798,1.074)--(-8.666,.879)--cycle;
\draw(-8.59,.849)--(-7.768,1.034);
\draw(-7.798,1.074)--(-8.666,.879);
\filldraw[fill opacity=0.8,fill=gray!20](-7.045,1.232)--(-7,1.266)--(-6.982,1.277)--(-7.02,1.249)--cycle;
\filldraw[fill opacity=0.8,fill=gray!20](-6.941,.866)--(-6.988,.882)--(-7,.894)--(-6.947,.872)--cycle;
\filldraw[fill opacity=0.8,fill=gray!20](-6.89,.864)--(-6.941,.866)--(-6.947,.872)--(-6.89,.864)--cycle;
\filldraw[fill opacity=0.8,fill=gray!20](-6.83,1.284)--(-6.859,1.296)--(-6.839,1.291)--(-6.792,1.275)--cycle;
\filldraw[fill opacity=0.8,fill=gray!20](-6.89,.864)--(-6.843,.865)--(-6.865,.86)--(-6.89,.864)--cycle;
\filldraw[fill opacity=0.8,fill=gray!20](-6.89,.864)--(-6.834,.871)--(-6.843,.865)--(-6.89,.864)--cycle;
\filldraw[fill opacity=0.8,fill=gray!20](-6.89,.864)--(-6.922,.861)--(-6.941,.866)--(-6.89,.864)--cycle;
\filldraw[fill opacity=0.8,fill=gray!20](-6.988,.882)--(-7.029,.912)--(-7.045,.929)--(-7,.894)--cycle;
\filldraw[fill opacity=0.8,fill=gray!20](-6.89,.864)--(-6.894,.859)--(-6.922,.861)--(-6.89,.864)--cycle;
\filldraw[fill opacity=0.8,fill=gray!20](-6.89,.864)--(-6.865,.86)--(-6.894,.859)--(-6.89,.864)--cycle;
\filldraw[fill opacity=0.8,fill=gray!20,draw=none](-7.769,1.08)--(-7.771,1.037)--(-7.74,1.036)--cycle;
\draw(-7.771,1.037)--(-7.74,1.036);
\filldraw[fill opacity=0.8,fill=gray!20,draw=none](-7.771,1.037)--(-7.769,1.08)--(-7.798,1.074)--cycle;
\draw(-7.769,1.08)--(-7.798,1.074);
\filldraw[fill opacity=0.8,fill=gray!20](-7.733,1.215)--(-7.749,1.251)--(-7.796,1.249)--(-7.798,1.212)--cycle;
\filldraw[fill opacity=0.8,fill=gray!20,draw=none](-7.777,1.136)--(-7.812,1.137)--(-7.814,1.086)--(-7.734,1.083)--cycle;
\draw(-7.777,1.136)--(-7.812,1.137)--(-7.814,1.086)--(-7.734,1.083);
\filldraw[fill opacity=0.8,fill=gray!20,draw=none](-7.841,1.077)--(-7.845,1.116)--(-7.852,1.126)--(-7.888,1.128)--(-7.891,1.081)--cycle;
\draw(-7.852,1.126)--(-7.888,1.128)--(-7.891,1.081)--(-7.841,1.077);
\filldraw[fill opacity=0.8,fill=gray!20,draw=none](-7.852,1.126)--(-7.865,1.161)--(-7.879,1.171)--(-7.888,1.128)--cycle;
\draw(-7.879,1.171)--(-7.888,1.128)--(-7.852,1.126);
\filldraw[fill opacity=0.8,fill=gray!20,draw=none](-8.759,.953)--(-8.771,.963)--(-8.765,.977)--(-8.758,.983)--(-8.728,.978)--(-8.735,.963)--cycle;
\draw(-8.771,.963)--(-8.765,.977);
\draw(-8.728,.978)--(-8.735,.963);
\filldraw[fill opacity=0.8,fill=gray!20,draw=none](-8.759,.953)--(-8.735,.963)--(-8.745,.941)--cycle;
\draw(-8.735,.963)--(-8.745,.941);
\filldraw[fill opacity=0.8,fill=gray!20,draw=none](-8.701,.911)--(-8.712,.927)--(-8.714,.94)--(-8.709,.936)--cycle;
\draw(-8.712,.927)--(-8.714,.94)--(-8.709,.936);
\filldraw[fill opacity=0.8,fill=gray!20,draw=none](-8.718,.951)--(-8.733,.99)--(-8.717,.972)--cycle;
\draw(-8.718,.951)--(-8.733,.99)--(-8.717,.972);
\filldraw[fill opacity=0.8,fill=gray!20,draw=none](-8.745,.941)--(-8.735,.963)--(-8.724,.972)--(-8.717,.964)--(-8.734,.925)--cycle;
\draw(-8.745,.941)--(-8.735,.963);
\draw(-8.717,.964)--(-8.734,.925);
\filldraw[fill opacity=0.8,fill=gray!20,draw=none](-8.734,.925)--(-8.715,.969)--(-8.717,.973)--(-8.719,.97)--(-8.741,.919)--cycle;
\draw(-8.734,.925)--(-8.715,.969);
\draw(-8.719,.97)--(-8.741,.919);
\filldraw[fill opacity=0.8,fill=gray!20,draw=none](-8.715,.969)--(-8.695,1.016)--(-8.717,.973)--cycle;
\draw(-8.715,.969)--(-8.695,1.016);
\filldraw[fill opacity=0.8,fill=gray!20,draw=none](-8.714,.94)--(-8.718,.951)--(-8.717,.972)--(-8.714,.969)--(-8.713,.967)--(-8.692,.917)--cycle;
\draw(-8.717,.972)--(-8.714,.969);
\draw(-8.713,.967)--(-8.692,.917)--(-8.714,.94)--(-8.718,.951);
\filldraw[fill opacity=0.8,fill=gray!20,draw=none](-8.724,.972)--(-8.735,.963)--(-8.728,.978)--cycle;
\draw(-8.735,.963)--(-8.728,.978);
\filldraw[fill opacity=0.8,fill=gray!20,draw=none](-8.701,.911)--(-8.709,.936)--(-8.692,.917)--(-8.688,.893)--cycle;
\draw(-8.709,.936)--(-8.692,.917)--(-8.688,.893);
\filldraw[fill opacity=0.8,fill=gray!20,draw=none](-8.734,.925)--(-8.741,.919)--(-8.749,.9)--cycle;
\draw(-8.741,.919)--(-8.749,.9);
\filldraw[fill opacity=0.8,fill=gray!20,draw=none](-8.714,.969)--(-8.717,.972)--(-8.717,.973)--(-8.715,.97)--cycle;
\draw(-8.714,.969)--(-8.717,.972);
\filldraw[fill opacity=0.8,fill=gray!20,draw=none](-8.704,.923)--(-8.7,.912)--(-8.692,.917)--(-8.713,.967)--cycle;
\draw(-8.7,.912)--(-8.692,.917)--(-8.713,.967);
\filldraw[fill opacity=0.8,fill=gray!20,draw=none](-8.749,.9)--(-8.741,.919)--(-8.76,.921)--(-8.772,.907)--(-8.779,.892)--cycle;
\draw(-8.749,.9)--(-8.741,.919);
\draw(-8.772,.907)--(-8.779,.892);
\filldraw[fill opacity=0.8,fill=gray!20,draw=none](-8.76,.921)--(-8.741,.919)--(-8.719,.97)--cycle;
\draw(-8.741,.919)--(-8.719,.97);
\filldraw[fill opacity=0.8,fill=gray!20,draw=none](-8.854,.889)--(-7.777,1.131)--(-7.78,1.181)--(-8.87,.936)--cycle;
\draw(-7.78,1.181)--(-8.87,.936)--(-8.854,.889)--(-7.777,1.131);
\filldraw[fill opacity=0.8,fill=gray!20,draw=none](-7.029,.912)--(-7.032,.916)--(-7.06,.948)--(-7.045,.929)--cycle;
\draw(-7.06,.948)--(-7.045,.929)--(-7.029,.912)--(-7.032,.916);
\filldraw[fill opacity=0.8,fill=gray!20](-6.834,.871)--(-6.781,.892)--(-6.799,.88)--(-6.843,.865)--cycle;
\filldraw[fill opacity=0.8,fill=gray!20,draw=none](-6.714,.952)--(-6.7,.963)--(-6.326,.949)--(-6.355,.929)--(-6.713,.943)--cycle;
\draw(-6.7,.963)--(-6.326,.949);
\draw(-6.355,.929)--(-6.713,.943);
\filldraw[fill opacity=0.8,fill=gray!20](-7.72,.994)--(-7.711,1.034)--(-7.801,1.03)--(-7.8,.99)--cycle;
\filldraw[fill opacity=0.8,fill=gray!20,draw=none](-7.835,1.019)--(-7.846,.994)--(-7.829,.993)--cycle;
\draw(-7.846,.994)--(-7.829,.993);
\filldraw[fill opacity=0.8,fill=gray!20,draw=none](-7.874,1.22)--(-7.862,1.217)--(-7.859,1.222)--cycle;
\draw(-7.874,1.22)--(-7.862,1.217)--(-7.859,1.222);
\filldraw[fill opacity=0.8,fill=gray!20,draw=none](-7.032,.916)--(-7.06,.953)--(-7.061,.954)--(-7.066,.956)--(-7.06,.948)--cycle;
\draw(-7.032,.916)--(-7.06,.953)--(-7.061,.954);
\draw(-7.066,.956)--(-7.06,.948);
\filldraw[fill opacity=0.8,fill=gray!20,draw=none](-6.659,1.199)--(-6.215,1.182)--(-6.167,1.166)--(-6.178,1.162)--(-6.687,1.182)--cycle;
\draw(-6.659,1.199)--(-6.215,1.182);
\draw(-6.178,1.162)--(-6.687,1.182);
\filldraw[fill opacity=0.8,fill=gray!20,draw=none](-8.358,2.913)--(-8.23,2.858)--(-8.195,2.887)--(-8.275,2.922)--cycle;
\draw(-8.358,2.913)--(-8.23,2.858);
\draw(-8.195,2.887)--(-8.275,2.922);
\filldraw[fill opacity=0.8,fill=gray!20](-6.781,.892)--(-6.735,.925)--(-6.761,.908)--(-6.799,.88)--cycle;
\filldraw[fill opacity=0.8,fill=gray!20,draw=none](-7.835,1.019)--(-7.829,.993)--(-7.8,.99)--(-7.801,1.03)--(-7.83,1.032)--cycle;
\draw(-7.829,.993)--(-7.8,.99)--(-7.801,1.03)--(-7.83,1.032);
\filldraw[fill opacity=0.8,fill=gray!20,draw=none](-7.881,1.175)--(-7.878,1.175)--(-7.872,1.189)--(-7.886,1.197)--cycle;
\draw(-7.881,1.175)--(-7.878,1.175)--(-7.872,1.189);
\filldraw[fill opacity=0.8,fill=gray!20,draw=none](-8.779,.969)--(-8.765,.977)--(-8.771,.963)--cycle;
\draw(-8.765,.977)--(-8.771,.963);
\filldraw[fill opacity=0.8,fill=gray!20,draw=none](-8.758,.983)--(-8.765,.977)--(-8.762,.984)--cycle;
\draw(-8.765,.977)--(-8.762,.984);
\filldraw[fill opacity=0.8,fill=gray!20,draw=none](-8.87,.936)--(-7.877,1.159)--(-7.886,1.197)--(-8.892,.971)--cycle;
\draw(-7.886,1.197)--(-8.892,.971)--(-8.87,.936)--(-7.877,1.159);
\filldraw[fill opacity=0.8,fill=gray!20](-6.751,1.245)--(-6.792,1.275)--(-6.781,1.263)--(-6.735,1.228)--cycle;
\filldraw[fill opacity=0.8,fill=gray!20,draw=none](-7.693,1.037)--(-7.694,1.034)--(-7.693,1.034)--cycle;
\draw(-7.694,1.034)--(-7.693,1.034);
\filldraw[fill opacity=0.8,fill=gray!20,draw=none](-6.914,1.05)--(-7.678,1.081)--(-7.693,1.037)--(-7.693,1.034)--(-6.912,1.003)--cycle;
\draw(-7.693,1.034)--(-6.912,1.003)--(-6.914,1.05)--(-7.678,1.081);
\filldraw[fill opacity=0.8,fill=gray!20,draw=none](-7.693,1.037)--(-7.678,1.081)--(-7.695,1.081)--cycle;
\draw(-7.678,1.081)--(-7.695,1.081);
\filldraw[fill opacity=0.8,fill=gray!20,draw=none](-7.683,1.156)--(-7.673,1.18)--(-7.689,1.181)--cycle;
\draw(-7.673,1.18)--(-7.689,1.181);
\filldraw[fill opacity=0.8,fill=gray!20,draw=none](-6.741,.921)--(-6.735,.925)--(-6.72,.944)--(-6.724,.943)--(-6.737,.933)--cycle;
\draw(-6.741,.921)--(-6.735,.925)--(-6.72,.944);
\filldraw[fill opacity=0.8,fill=gray!20,draw=none](-6.737,.933)--(-6.724,.943)--(-6.733,.942)--cycle;
\filldraw[fill opacity=0.8,fill=gray!20,draw=none](-6.89,.95)--(-6.713,.943)--(-6.694,.941)--(-6.899,.949)--cycle;
\draw(-6.694,.941)--(-6.899,.949)--(-6.89,.95)--(-6.713,.943);
\filldraw[fill opacity=0.8,fill=gray!20,draw=none](-7.683,1.156)--(-7.689,1.179)--(-7.694,1.133)--(-7.693,1.133)--cycle;
\draw(-7.694,1.133)--(-7.693,1.133);
\filldraw[fill opacity=0.8,fill=gray!20,draw=none](-7.694,1.133)--(-7.689,1.179)--(-7.69,1.178)--(-7.703,1.161)--cycle;
\draw(-7.689,1.179)--(-7.69,1.178);
\filldraw[fill opacity=0.8,fill=gray!20,draw=none](-7.769,1.08)--(-7.769,1.084)--(-7.772,1.084)--cycle;
\draw(-7.769,1.084)--(-7.772,1.084);
\filldraw[fill opacity=0.8,fill=gray!20,draw=none](-7.69,1.178)--(-7.72,1.173)--(-7.715,1.144)--cycle;
\draw(-7.69,1.178)--(-7.72,1.173)--(-7.715,1.144);
\filldraw[fill opacity=0.8,fill=gray!20,draw=none](-7.087,1.057)--(-7.082,1.108)--(-7.082,1.116)--(-7.102,1.137)--(-7.109,1.081)--cycle;
\draw(-7.082,1.116)--(-7.102,1.137)--(-7.109,1.081)--(-7.087,1.057);
\filldraw[fill opacity=0.8,fill=gray!20,draw=none](-6.907,1.149)--(-7.673,1.18)--(-7.693,1.133)--(-6.912,1.102)--cycle;
\draw(-7.693,1.133)--(-6.912,1.102)--(-6.907,1.149)--(-7.673,1.18);
\filldraw[fill opacity=0.8,fill=gray!20,draw=none](-7.8,1.176)--(-7.798,1.212)--(-7.862,1.217)--(-7.872,1.189)--cycle;
\draw(-7.8,1.176)--(-7.798,1.212)--(-7.862,1.217)--(-7.872,1.189);
\filldraw[fill opacity=0.8,fill=gray!20,draw=none](-7.816,1.07)--(-7.772,1.079)--(-7.773,1.132)--(-7.777,1.131)--cycle;
\draw(-7.816,1.07)--(-7.772,1.079);
\draw(-7.773,1.132)--(-7.777,1.131);
\filldraw[fill opacity=0.8,fill=gray!20,draw=none](-7.777,1.136)--(-7.777,1.131)--(-7.773,1.132)--cycle;
\draw(-7.777,1.131)--(-7.773,1.132);
\filldraw[fill opacity=0.8,fill=gray!20,draw=none](-8.689,.876)--(-8.701,.911)--(-8.695,.903)--cycle;
\filldraw[fill opacity=0.8,fill=gray!20,draw=none](-8.689,.876)--(-8.695,.903)--(-8.688,.893)--(-8.684,.861)--cycle;
\draw(-8.688,.893)--(-8.684,.861);
\filldraw[fill opacity=0.8,fill=gray!20,draw=none](-8.686,.864)--(-8.689,.876)--(-8.684,.861)--cycle;
\draw(-8.684,.861)--(-8.686,.864);
\filldraw[fill opacity=0.8,fill=gray!20,draw=none](-8.686,.864)--(-8.684,.861)--(-8.686,.848)--cycle;
\draw(-8.686,.864)--(-8.684,.861)--(-8.686,.848);
\filldraw[fill opacity=0.8,fill=gray!20,draw=none](-8.687,.86)--(-8.684,.861)--(-8.688,.893)--(-8.689,.893)--cycle;
\draw(-8.687,.86)--(-8.684,.861)--(-8.688,.893);
\filldraw[fill opacity=0.8,fill=gray!20,draw=none](-8.698,.893)--(-8.688,.893)--(-8.692,.917)--(-8.7,.912)--cycle;
\draw(-8.688,.893)--(-8.692,.917)--(-8.7,.912);
\filldraw[fill opacity=0.8,fill=gray!20,draw=none](-8.697,.872)--(-8.692,.856)--(-8.687,.86)--(-8.689,.893)--(-8.698,.893)--cycle;
\draw(-8.692,.856)--(-8.687,.86);
\filldraw[fill opacity=0.8,fill=gray!20,draw=none](-8.749,.9)--(-8.779,.892)--(-8.788,.87)--cycle;
\draw(-8.779,.892)--(-8.788,.87);
\filldraw[fill opacity=0.8,fill=gray!20,draw=none](-8.811,.87)--(-8.788,.87)--(-8.779,.892)--(-8.787,.893)--cycle;
\draw(-8.788,.87)--(-8.779,.892);
\filldraw[fill opacity=0.8,fill=gray!20,draw=none](-8.788,.87)--(-8.811,.87)--(-8.844,.839)--cycle;
\filldraw[fill opacity=0.8,fill=gray!20,draw=none](-8.787,.893)--(-8.779,.892)--(-8.772,.907)--cycle;
\draw(-8.779,.892)--(-8.772,.907);
\filldraw[fill opacity=0.8,fill=gray!20,draw=none](-8.844,.839)--(-7.816,1.07)--(-7.777,1.131)--(-8.854,.889)--cycle;
\draw(-7.777,1.131)--(-8.854,.889)--(-8.844,.839)--(-7.816,1.07);
\filldraw[fill opacity=0.8,fill=gray!20,draw=none](-7.474,4.149)--(-7.462,4.161)--(-7.452,4.052)--(-7.462,4.043)--cycle;
\draw(-7.462,4.161)--(-7.452,4.052)--(-7.462,4.043);
\filldraw[fill opacity=0.8,fill=gray!20,draw=none](-7.496,4.493)--(-7.466,4.163)--(-7.442,4.129)--(-7.475,4.493)--cycle;
\draw(-7.496,4.493)--(-7.466,4.163)--(-7.442,4.129)--(-7.475,4.493);
\filldraw[fill opacity=0.8,fill=gray!20,draw=none](-6.71,1.194)--(-6.716,1.202)--(-6.714,1.201)--(-6.7,1.183)--cycle;
\draw(-6.714,1.201)--(-6.7,1.183)--(-6.71,1.194);
\filldraw[fill opacity=0.8,fill=gray!20,draw=none](-6.052,1.177)--(-6.056,1.174)--(-6.056,1.177)--cycle;
\draw(-6.056,1.174)--(-6.056,1.177);
\filldraw[fill opacity=0.8,fill=gray!20,draw=none](-6.09,1.178)--(-6.056,1.177)--(-6.056,1.174)--(-6.073,1.155)--(-6.101,1.167)--(-6.101,1.169)--cycle;
\draw(-6.056,1.177)--(-6.056,1.174);
\draw(-6.101,1.167)--(-6.101,1.169);
\filldraw[fill opacity=0.8,fill=gray!20,draw=none](-6.137,1.179)--(-6.151,1.18)--(-6.163,1.171)--cycle;
\draw(-6.151,1.18)--(-6.163,1.171);
\filldraw[fill opacity=0.8,fill=gray!20,draw=none](-6.05,1.176)--(-6.095,1.122)--(-6.067,1.146)--(-6.043,1.174)--cycle;
\draw(-6.05,1.176)--(-6.095,1.122)--(-6.067,1.146)--(-6.043,1.174);
\filldraw[fill opacity=0.8,fill=gray!20,draw=none](-6.076,1.074)--(-6.102,1.044)--(-6.107,1.086)--(-6.068,1.133)--cycle;
\draw(-6.076,1.074)--(-6.102,1.044)--(-6.107,1.086)--(-6.068,1.133);
\filldraw[fill opacity=0.8,fill=gray!20,draw=none](-6.148,1.158)--(-6.171,1.151)--(-6.19,1.162)--(-6.151,1.161)--cycle;
\draw(-6.19,1.162)--(-6.151,1.161);
\filldraw[fill opacity=0.8,fill=gray!20,draw=none](-6.167,1.166)--(-6.151,1.161)--(-6.178,1.162)--cycle;
\draw(-6.151,1.161)--(-6.178,1.162);
\filldraw[fill opacity=0.8,fill=gray!20,draw=none](-6.146,1.159)--(-6.186,1.132)--(-6.167,1.166)--cycle;
\draw(-6.186,1.132)--(-6.167,1.166);
\filldraw[fill opacity=0.8,fill=gray!20,draw=none](-6.148,1.158)--(-6.151,1.161)--(-6.14,1.16)--cycle;
\draw(-6.151,1.161)--(-6.14,1.16);
\filldraw[fill opacity=0.8,fill=gray!20,draw=none](-6.167,1.166)--(-6.161,1.168)--(-6.124,1.175)--(-6.121,1.164)--(-6.14,1.16)--(-6.151,1.161)--cycle;
\draw(-6.14,1.16)--(-6.151,1.161);
\filldraw[fill opacity=0.8,fill=gray!20,draw=none](-6.121,1.164)--(-6.12,1.16)--(-6.14,1.16)--cycle;
\draw(-6.12,1.16)--(-6.14,1.16);
\filldraw[fill opacity=0.8,fill=gray!20,draw=none](-6.121,1.164)--(-6.102,1.168)--(-6.091,1.158)--(-6.12,1.16)--cycle;
\draw(-6.091,1.158)--(-6.12,1.16);
\filldraw[fill opacity=0.8,fill=gray!20,draw=none](-6.124,1.178)--(-6.137,1.179)--(-6.163,1.171)--(-6.164,1.169)--(-6.129,1.154)--cycle;
\draw(-6.163,1.171)--(-6.164,1.169)--(-6.129,1.154);
\filldraw[fill opacity=0.8,fill=gray!20,draw=none](-6.144,1.111)--(-6.129,1.154)--(-6.164,1.169)--(-6.17,1.159)--cycle;
\draw(-6.129,1.154)--(-6.164,1.169)--(-6.17,1.159);
\filldraw[fill opacity=0.8,fill=gray!20,draw=none](-6.161,1.168)--(-6.168,1.166)--(-6.17,1.167)--(-6.162,1.18)--(-6.131,1.178)--cycle;
\draw(-6.162,1.18)--(-6.131,1.178);
\filldraw[fill opacity=0.8,fill=gray!20,draw=none](-6.113,1.178)--(-6.101,1.178)--(-6.101,1.167)--(-6.113,1.177)--cycle;
\draw(-6.101,1.178)--(-6.101,1.167);
\filldraw[fill opacity=0.8,fill=gray!20,draw=none](-6.102,1.168)--(-6.062,1.176)--(-6.045,1.175)--(-6.069,1.158)--(-6.091,1.158)--cycle;
\draw(-6.062,1.176)--(-6.045,1.175);
\draw(-6.069,1.158)--(-6.091,1.158);
\filldraw[fill opacity=0.8,fill=gray!20,draw=none](-6.17,1.167)--(-6.215,1.182)--(-6.162,1.18)--cycle;
\draw(-6.215,1.182)--(-6.162,1.18);
\filldraw[fill opacity=0.8,fill=gray!20,draw=none](-6.124,1.175)--(-6.121,1.176)--(-6.103,1.168)--(-6.121,1.164)--cycle;
\filldraw[fill opacity=0.8,fill=gray!20,draw=none](-6.09,1.178)--(-6.101,1.169)--(-6.101,1.178)--cycle;
\draw(-6.101,1.169)--(-6.101,1.178);
\filldraw[fill opacity=0.8,fill=gray!20,draw=none](-6.121,1.176)--(-6.112,1.178)--(-6.062,1.176)--(-6.103,1.168)--cycle;
\draw(-6.112,1.178)--(-6.062,1.176);
\filldraw[fill opacity=0.8,fill=gray!20,draw=none](-6.113,1.178)--(-6.124,1.178)--(-6.129,1.154)--cycle;
\filldraw[fill opacity=0.8,fill=gray!20,draw=none](-6.161,1.168)--(-6.131,1.178)--(-6.112,1.178)--cycle;
\draw(-6.131,1.178)--(-6.112,1.178);
\filldraw[fill opacity=0.8,fill=gray!20,draw=none](-6.113,1.177)--(-6.101,1.167)--(-6.101,1.126)--(-6.129,1.092)--(-6.129,1.154)--cycle;
\draw(-6.101,1.167)--(-6.101,1.126);
\draw(-6.129,1.092)--(-6.129,1.154);
\filldraw[fill opacity=0.8,fill=gray!20,draw=none](-6.073,1.155)--(-6.101,1.126)--(-6.101,1.167)--cycle;
\draw(-6.101,1.126)--(-6.101,1.167);
\filldraw[fill opacity=0.8,fill=gray!20,draw=none](-6.081,1.158)--(-6.069,1.158)--(-6.076,1.152)--cycle;
\draw(-6.081,1.158)--(-6.069,1.158);
\filldraw[fill opacity=0.8,fill=gray!20,draw=none](-5.964,1.057)--(-5.944,1.058)--(-5.944,.995)--(-5.954,.994)--(-5.964,.998)--cycle;
\draw(-5.944,1.058)--(-5.944,.995);
\filldraw[fill opacity=0.8,fill=gray!20,draw=none](-5.888,.97)--(-5.913,.94)--(-5.954,.931)--(-5.916,.976)--cycle;
\draw(-5.888,.97)--(-5.913,.94)--(-5.954,.931)--(-5.916,.976);
\filldraw[fill opacity=0.8,fill=gray!20,draw=none](-6.06,1.162)--(-6.065,1.174)--(-6.113,1.178)--(-6.129,1.154)--(-6.118,1.149)--cycle;
\draw(-6.129,1.154)--(-6.118,1.149);
\filldraw[fill opacity=0.8,fill=gray!20,draw=none](-6.121,1.178)--(-6.113,1.178)--(-6.113,1.177)--(-6.129,1.154)--cycle;
\filldraw[fill opacity=0.8,fill=gray!20,draw=none](-6.121,1.178)--(-6.129,1.154)--(-6.129,1.178)--cycle;
\draw(-6.129,1.154)--(-6.129,1.178);
\filldraw[fill opacity=0.8,fill=gray!20,draw=none](-6.89,1.207)--(-6.021,1.173)--(-6.062,1.176)--(-6.881,1.208)--cycle;
\draw(-6.062,1.176)--(-6.881,1.208)--(-6.89,1.207)--(-6.021,1.173);
\filldraw[fill opacity=0.8,fill=gray!20,draw=none](-6.125,.853)--(-6.136,.862)--(-6.129,.874)--(-6.101,.885)--(-6.056,.893)--(-6.004,.896)--cycle;
\draw(-6.125,.853)--(-6.136,.862)--(-6.129,.874)--(-6.101,.885)--(-6.056,.893)--(-6.004,.896);
\filldraw[fill opacity=0.8,fill=gray!20](-6.922,.861)--(-6.951,.873)--(-6.988,.882)--(-6.941,.866)--cycle;
\filldraw[fill opacity=0.8,fill=gray!20,draw=none](-7.835,1.019)--(-7.83,1.032)--(-7.839,1.032)--cycle;
\draw(-7.83,1.032)--(-7.839,1.032);
\filldraw[fill opacity=0.8,fill=gray!20](-7.72,1.173)--(-7.733,1.215)--(-7.798,1.212)--(-7.8,1.169)--cycle;
\filldraw[fill opacity=0.8,fill=gray!20](-7,1.266)--(-6.947,1.286)--(-6.938,1.292)--(-6.982,1.277)--cycle;
\filldraw[fill opacity=0.8,fill=gray!20,draw=none](-6.899,.949)--(-6.733,.942)--(-6.729,.95)--(-6.73,.953)--(-6.741,.961)--(-6.907,.968)--cycle;
\draw(-6.741,.961)--(-6.907,.968)--(-6.899,.949)--(-6.733,.942);
\filldraw[fill opacity=0.8,fill=gray!20,draw=none](-6.724,.943)--(-6.714,.952)--(-6.7,.969)--(-6.732,.949)--(-6.736,.942)--cycle;
\draw(-6.714,.952)--(-6.7,.969)--(-6.732,.949)--(-6.736,.942);
\filldraw[fill opacity=0.8,fill=gray!20](-7.711,1.034)--(-7.709,1.079)--(-7.801,1.074)--(-7.801,1.03)--cycle;
\filldraw[fill opacity=0.8,fill=gray!20,draw=none](-7.869,1.174)--(-7.868,1.188)--(-7.872,1.189)--(-7.878,1.175)--cycle;
\draw(-7.872,1.189)--(-7.878,1.175)--(-7.869,1.174);
\filldraw[fill opacity=0.8,fill=gray!20](-6.843,.865)--(-6.799,.88)--(-6.841,.872)--(-6.865,.86)--cycle;
\filldraw[fill opacity=0.5,fill=gray!20](-9.382,-.692)--(-9.43,-.892)--(-9.831,-.754)--(-9.738,-.57)--cycle;
\filldraw[fill opacity=0.8,fill=gray!20,draw=none](-7.869,1.174)--(-7.8,1.169)--(-7.8,1.176)--(-7.868,1.188)--cycle;
\draw(-7.869,1.174)--(-7.8,1.169)--(-7.8,1.176);
\filldraw[fill opacity=0.8,fill=gray!20,draw=none](-7.879,1.171)--(-7.878,1.175)--(-7.881,1.175)--cycle;
\draw(-7.879,1.171)--(-7.878,1.175)--(-7.881,1.175);
\filldraw[fill opacity=0.5,fill=gray!20](-11.123,.927)--(-10.938,.947)--(-10.888,1.369)--(-11.066,1.402)--cycle;
\filldraw[fill opacity=0.5,fill=gray!20](-10.938,.947)--(-10.765,.871)--(-10.715,1.294)--(-10.888,1.369)--cycle;
\filldraw[fill opacity=0.8,fill=gray!20](-6.7,.969)--(-6.679,1.02)--(-6.713,.998)--(-6.732,.949)--cycle;
\filldraw[fill opacity=0.8,fill=gray!20,draw=none](-6.912,1.102)--(-7.677,1.132)--(-7.695,1.081)--(-6.914,1.05)--cycle;
\draw(-7.695,1.081)--(-6.914,1.05)--(-6.912,1.102)--(-7.677,1.132);
\filldraw[fill opacity=0.8,fill=gray!20,draw=none](-7.677,1.132)--(-7.693,1.133)--(-7.695,1.081)--cycle;
\draw(-7.677,1.132)--(-7.693,1.133);
\filldraw[fill opacity=0.8,fill=gray!20,draw=none](-6.04,.894)--(-6.056,.893)--(-6.056,.903)--cycle;
\draw(-6.04,.894)--(-6.056,.893)--(-6.056,.903);
\filldraw[fill opacity=0.8,fill=gray!20,draw=none](-7.693,1.13)--(-7.703,1.161)--(-7.715,1.144)--(-7.711,1.126)--cycle;
\draw(-7.715,1.144)--(-7.711,1.126)--(-7.693,1.13);
\filldraw[fill opacity=0.8,fill=gray!20,draw=none](-7.88,1.172)--(-7.877,1.159)--(-7.865,1.161)--cycle;
\draw(-7.877,1.159)--(-7.865,1.161);
\filldraw[fill opacity=0.8,fill=gray!20,draw=none](-7.865,1.161)--(-7.869,1.174)--(-7.878,1.175)--(-7.879,1.171)--cycle;
\draw(-7.869,1.174)--(-7.878,1.175)--(-7.879,1.171);
\filldraw[fill opacity=0.8,fill=gray!20,draw=none](-7.835,1.065)--(-7.839,1.032)--(-7.82,1.031)--cycle;
\draw(-7.839,1.032)--(-7.82,1.031);
\filldraw[fill opacity=0.5,fill=gray!20](-9.173,2.593)--(-9,2.518)--(-8.618,2.505)--(-8.791,2.58)--cycle;
\filldraw[fill opacity=0.8,fill=gray!20](-6.916,1.297)--(-6.89,1.293)--(-6.89,1.293)--(-6.887,1.298)--cycle;
\filldraw[fill opacity=0.8,fill=gray!20](-6.887,1.298)--(-6.89,1.293)--(-6.89,1.293)--(-6.859,1.296)--cycle;
\filldraw[fill opacity=0.8,fill=gray!20,draw=none](-7.852,1.126)--(-7.846,1.125)--(-7.834,1.172)--(-7.869,1.174)--cycle;
\draw(-7.852,1.126)--(-7.846,1.125);
\draw(-7.834,1.172)--(-7.869,1.174);
\filldraw[fill opacity=0.8,fill=gray!20,draw=none](-7.065,.956)--(-7.061,.954)--(-7.07,.963)--cycle;
\draw(-7.061,.954)--(-7.07,.963);
\filldraw[fill opacity=0.5,fill=gray!20](-10.254,2.662)--(-10.281,2.635)--(-9.862,2.871)--(-9.82,2.906)--cycle;
\filldraw[fill opacity=0.8,fill=gray!20,draw=none](-6.741,.921)--(-6.737,.933)--(-6.754,.918)--(-6.761,.908)--cycle;
\draw(-6.754,.918)--(-6.761,.908)--(-6.741,.921);
\filldraw[fill opacity=0.8,fill=gray!20,draw=none](-8.343,2.907)--(-8.23,2.858)--(-8.195,2.887)--(-8.334,2.947)--cycle;
\draw(-8.343,2.907)--(-8.23,2.858);
\draw(-8.195,2.887)--(-8.334,2.947);
\filldraw[fill opacity=0.8,fill=gray!20,draw=none](-5.98,.832)--(-6.036,.834)--(-6.086,.84)--(-6.121,.85)--(-6.125,.853)--(-6.013,.893)--(-5.895,.841)--(-5.925,.836)--cycle;
\draw(-5.895,.841)--(-5.925,.836)--(-5.98,.832)--(-6.036,.834)--(-6.086,.84)--(-6.121,.85)--(-6.125,.853);
\filldraw[fill opacity=0.8,fill=gray!20](-6.792,1.275)--(-6.839,1.291)--(-6.834,1.285)--(-6.781,1.263)--cycle;
\filldraw[fill opacity=0.8,fill=gray!20,draw=none](-8.255,2.834)--(-8.072,2.754)--(-8.11,2.778)--(-8.252,2.839)--cycle;
\draw(-8.255,2.834)--(-8.072,2.754)--(-8.11,2.778)--(-8.252,2.839);
\filldraw[fill opacity=0.8,fill=gray!20,draw=none](-6.065,.917)--(-5.99,.914)--(-5.999,.913)--(-6.056,.915)--cycle;
\draw(-6.065,.917)--(-5.99,.914)--(-5.999,.913)--(-6.056,.915);
\filldraw[fill opacity=0.8,fill=gray!20,draw=none](-5.98,.879)--(-6.049,.909)--(-5.949,.882)--(-5.925,.871)--cycle;
\draw(-5.98,.879)--(-6.049,.909);
\draw(-5.949,.882)--(-5.925,.871);
\filldraw[fill opacity=0.8,fill=gray!20,draw=none](-6.032,.905)--(-5.983,.897)--(-5.949,.882)--cycle;
\draw(-5.983,.897)--(-5.949,.882);
\filldraw[fill opacity=0.8,fill=gray!20,draw=none](-6.074,.928)--(-6.056,.915)--(-5.999,.913)--(-6.007,.932)--(-6.074,.935)--cycle;
\draw(-6.056,.915)--(-5.999,.913)--(-6.007,.932)--(-6.074,.935);
\filldraw[fill opacity=0.8,fill=gray!20,draw=none](-5.983,.93)--(-5.981,.935)--(-5.979,.943)--cycle;
\draw(-5.983,.93)--(-5.981,.935)--(-5.979,.943);
\filldraw[fill opacity=0.8,fill=gray!20,draw=none](-5.901,1.085)--(-5.937,1.121)--(-5.981,1.145)--(-5.999,1.149)--(-5.981,.937)--(-5.954,.931)--(-5.913,.94)--(-5.885,.964)--(-5.873,1)--(-5.878,1.042)--cycle;
\draw(-5.981,.937)--(-5.954,.931)--(-5.913,.94)--(-5.885,.964)--(-5.873,1)--(-5.878,1.042)--(-5.901,1.085)--(-5.937,1.121)--(-5.981,1.145)--(-5.999,1.149);
\filldraw[fill opacity=0.8,fill=gray!20,draw=none](-5.901,.881)--(-5.928,.893)--(-5.915,.905)--(-5.879,.89)--cycle;
\draw(-5.915,.905)--(-5.879,.89);
\filldraw[fill opacity=0.8,fill=gray!20,draw=none](-5.953,1.178)--(-5.997,1.189)--(-5.987,1.184)--(-5.984,1.183)--cycle;
\draw(-5.987,1.184)--(-5.984,1.183);
\filldraw[fill opacity=0.8,fill=gray!20,draw=none](-5.903,1.161)--(-5.926,1.174)--(-5.92,1.172)--cycle;
\draw(-5.926,1.174)--(-5.92,1.172);
\filldraw[fill opacity=0.8,fill=gray!20,draw=none](-5.903,1.164)--(-5.879,1.148)--(-5.903,1.161)--(-5.92,1.172)--(-5.91,1.167)--cycle;
\draw(-5.92,1.172)--(-5.91,1.167);
\filldraw[fill opacity=0.8,fill=gray!20,draw=none](-5.953,1.178)--(-5.91,1.167)--(-5.926,1.174)--cycle;
\draw(-5.91,1.167)--(-5.926,1.174);
\filldraw[fill opacity=0.8,fill=gray!20,draw=none](-5.93,1.172)--(-5.953,1.178)--(-5.984,1.183)--(-5.966,1.175)--cycle;
\draw(-5.984,1.183)--(-5.966,1.175);
\filldraw[fill opacity=0.8,fill=gray!20,draw=none](-5.98,1.143)--(-5.98,.879)--(-5.954,.864)--(-5.925,.871)--(-5.925,1.157)--cycle;
\draw(-5.98,1.143)--(-5.98,.879);
\draw(-5.925,.871)--(-5.925,1.157);
\filldraw[fill opacity=0.8,fill=gray!20,draw=none](-5.97,.875)--(-5.98,.879)--(-5.925,.871)--cycle;
\draw(-5.97,.875)--(-5.98,.879);
\filldraw[fill opacity=0.8,fill=gray!20,draw=none](-5.859,.976)--(-5.859,.96)--(-5.864,.964)--cycle;
\draw(-5.859,.976)--(-5.859,.96);
\filldraw[fill opacity=0.8,fill=gray!20,draw=none](-5.849,1.012)--(-5.844,.99)--(-5.844,.986)--(-5.846,.933)--(-5.859,.96)--(-5.859,.976)--cycle;
\draw(-5.844,.99)--(-5.844,.986);
\draw(-5.859,.96)--(-5.859,.976);
\filldraw[fill opacity=0.8,fill=gray!20,draw=none](-5.864,.937)--(-5.884,.946)--(-5.856,.996)--(-5.844,.99)--cycle;
\draw(-5.864,.937)--(-5.884,.946);
\draw(-5.856,.996)--(-5.844,.99);
\filldraw[fill opacity=0.8,fill=gray!20,draw=none](-5.879,1.148)--(-5.869,1.142)--(-5.873,1.143)--(-5.903,1.161)--cycle;
\draw(-5.869,1.142)--(-5.873,1.143);
\filldraw[fill opacity=0.8,fill=gray!20,draw=none](-5.847,1.149)--(-5.844,1.133)--(-5.844,1.117)--cycle;
\draw(-5.844,1.133)--(-5.844,1.117);
\filldraw[fill opacity=0.8,fill=gray!20,draw=none](-5.879,1.148)--(-5.903,1.164)--(-5.869,1.146)--(-5.851,1.134)--(-5.862,1.138)--cycle;
\draw(-5.851,1.134)--(-5.862,1.138);
\filldraw[fill opacity=0.8,fill=gray!20,draw=none](-5.847,1.149)--(-5.85,1.182)--(-5.844,1.162)--(-5.844,1.133)--cycle;
\draw(-5.844,1.162)--(-5.844,1.133);
\filldraw[fill opacity=0.8,fill=gray!20,draw=none](-5.869,1.146)--(-5.903,1.164)--(-5.906,1.165)--(-5.879,1.154)--cycle;
\draw(-5.906,1.165)--(-5.879,1.154);
\filldraw[fill opacity=0.8,fill=gray!20,draw=none](-5.925,1.157)--(-5.925,.836)--(-5.884,.843)--(-5.879,.852)--(-5.879,1.154)--cycle;
\draw(-5.925,1.157)--(-5.925,.836)--(-5.884,.843);
\draw(-5.879,.852)--(-5.879,1.154);
\filldraw[fill opacity=0.8,fill=gray!20,draw=none](-5.954,.864)--(-5.925,.846)--(-5.925,.871)--cycle;
\draw(-5.925,.846)--(-5.925,.871);
\filldraw[fill opacity=0.8,fill=gray!20,draw=none](-5.922,.866)--(-5.958,.87)--(-5.97,.875)--(-5.925,.871)--(-5.913,.866)--cycle;
\draw(-5.922,.866)--(-5.958,.87)--(-5.97,.875);
\draw(-5.925,.871)--(-5.913,.866);
\filldraw[fill opacity=0.8,fill=gray!20,draw=none](-5.999,1.149)--(-6.027,1.154)--(-6.067,1.146)--(-6.095,1.122)--(-6.107,1.086)--(-6.102,1.044)--(-6.079,1.001)--(-6.043,.965)--(-5.999,.94)--(-5.981,.937)--cycle;
\draw(-5.999,1.149)--(-6.027,1.154)--(-6.067,1.146)--(-6.095,1.122)--(-6.107,1.086)--(-6.102,1.044)--(-6.079,1.001)--(-6.043,.965)--(-5.999,.94)--(-5.981,.937);
\filldraw[fill opacity=0.8,fill=gray!20,draw=none](-6.012,.968)--(-6.007,.932)--(-5.999,.913)--(-5.99,.914)--(-5.983,.93)--(-5.981,.937)--(-5.999,1.149)--(-6.001,1.143)--(-6.007,1.114)--(-6.012,1.066)--(-6.014,1.015)--cycle;
\draw(-6.001,1.143)--(-6.007,1.114)--(-6.012,1.066)--(-6.014,1.015)--(-6.012,.968)--(-6.007,.932)--(-5.999,.913)--(-5.99,.914)--(-5.983,.93);
\filldraw[fill opacity=0.8,fill=gray!20](-6.679,1.02)--(-6.671,1.076)--(-6.707,1.052)--(-6.713,.998)--cycle;
\filldraw[fill opacity=0.8,fill=gray!20,draw=none](-6.133,.999)--(-6.145,.979)--(-6.151,.979)--cycle;
\draw(-6.145,.979)--(-6.151,.979);
\filldraw[fill opacity=0.8,fill=gray!20](-6.671,1.076)--(-6.679,1.131)--(-6.713,1.109)--(-6.707,1.052)--cycle;
\filldraw[fill opacity=0.8,fill=gray!20,draw=none](-6.679,1.131)--(-6.7,1.183)--(-6.729,1.164)--(-6.73,1.158)--(-6.713,1.109)--cycle;
\draw(-6.73,1.158)--(-6.713,1.109)--(-6.679,1.131)--(-6.7,1.183)--(-6.729,1.164);
\filldraw[fill opacity=0.8,fill=gray!20,draw=none](-6.09,.966)--(-6.074,.935)--(-6.007,.932)--(-6.012,.968)--(-6.09,.971)--cycle;
\draw(-6.074,.935)--(-6.007,.932)--(-6.012,.968)--(-6.09,.971);
\filldraw[fill opacity=0.8,fill=gray!20,draw=none](-6.036,.948)--(-6.036,.912)--(-5.98,.879)--(-5.98,.962)--cycle;
\draw(-6.036,.948)--(-6.036,.912);
\draw(-5.98,.879)--(-5.98,.962);
\filldraw[fill opacity=0.8,fill=gray!20](-6.001,.896)--(-6.164,.967)--(-6.122,.941)--(-5.958,.87)--cycle;
\filldraw[fill opacity=0.8,fill=gray!20,draw=none](-7.835,1.065)--(-7.82,1.031)--(-7.801,1.03)--(-7.801,1.074)--(-7.834,1.077)--cycle;
\draw(-7.82,1.031)--(-7.801,1.03)--(-7.801,1.074)--(-7.834,1.077);
\filldraw[fill opacity=0.8,fill=gray!20](-6.938,1.292)--(-6.89,1.293)--(-6.89,1.293)--(-6.916,1.297)--cycle;
\filldraw[fill opacity=0.8,fill=gray!20](-7.711,1.126)--(-7.72,1.173)--(-7.8,1.169)--(-7.801,1.122)--cycle;
\filldraw[fill opacity=0.8,fill=gray!20,draw=none](-7.057,1.214)--(-7.066,1.205)--(-7.06,1.213)--cycle;
\draw(-7.066,1.205)--(-7.06,1.213);
\filldraw[fill opacity=0.8,fill=gray!20](-7.709,1.079)--(-7.711,1.126)--(-7.801,1.122)--(-7.801,1.074)--cycle;
\filldraw[fill opacity=0.8,fill=gray!20](-6.859,1.296)--(-6.89,1.293)--(-6.89,1.293)--(-6.839,1.291)--cycle;
\filldraw[fill opacity=0.8,fill=gray!20](-6.894,.859)--(-6.897,.869)--(-6.951,.873)--(-6.922,.861)--cycle;
\filldraw[fill opacity=0.8,fill=gray!20,draw=none](-6.713,.943)--(-6.18,.922)--(-6.233,.923)--(-6.694,.941)--cycle;
\draw(-6.713,.943)--(-6.18,.922);
\draw(-6.233,.923)--(-6.694,.941);
\filldraw[fill opacity=0.8,fill=gray!20,draw=none](-7.835,1.065)--(-7.834,1.077)--(-7.841,1.077)--cycle;
\draw(-7.834,1.077)--(-7.841,1.077);
\filldraw[fill opacity=0.8,fill=gray!20](-6.865,.86)--(-6.841,.872)--(-6.897,.869)--(-6.894,.859)--cycle;
\filldraw[fill opacity=0.8,fill=gray!20,draw=none](-7.846,1.125)--(-7.801,1.122)--(-7.8,1.169)--(-7.834,1.172)--cycle;
\draw(-7.846,1.125)--(-7.801,1.122)--(-7.8,1.169)--(-7.834,1.172);
\filldraw[fill opacity=0.8,fill=gray!20,draw=none](-6.056,1.178)--(-6.116,1.18)--(-6.057,1.22)--(-6.022,1.216)--(-6.02,1.215)--cycle;
\draw(-6.057,1.22)--(-6.022,1.216)--(-6.02,1.215);
\filldraw[fill opacity=0.8,fill=gray!20,draw=none](-6.02,1.215)--(-6.022,1.216)--(-6.019,1.215)--cycle;
\draw(-6.02,1.215)--(-6.022,1.216)--(-6.019,1.215);
\filldraw[fill opacity=0.8,fill=gray!20,draw=none](-6.042,1.212)--(-6.02,1.209)--(-6.052,1.177)--(-6.056,1.177)--(-6.056,1.205)--cycle;
\draw(-6.056,1.177)--(-6.056,1.205);
\filldraw[fill opacity=0.8,fill=gray!20,draw=none](-6.019,1.215)--(-6.022,1.216)--(-6.057,1.22)--(-6.02,1.201)--cycle;
\draw(-6.019,1.215)--(-6.022,1.216)--(-6.057,1.22);
\filldraw[fill opacity=0.8,fill=gray!20,draw=none](-7.841,1.077)--(-7.815,1.075)--(-7.845,1.116)--cycle;
\draw(-7.841,1.077)--(-7.815,1.075);
\filldraw[fill opacity=0.8,fill=gray!20,draw=none](-7.845,1.116)--(-7.846,1.125)--(-7.852,1.126)--cycle;
\draw(-7.846,1.125)--(-7.852,1.126);
\filldraw[fill opacity=0.8,fill=gray!20](-6.951,.873)--(-6.976,.899)--(-7.029,.912)--(-6.988,.882)--cycle;
\filldraw[fill opacity=0.8,fill=gray!20,draw=none](-6.716,1.202)--(-6.725,1.215)--(-6.714,1.201)--cycle;
\draw(-6.725,1.215)--(-6.714,1.201);
\filldraw[fill opacity=0.8,fill=gray!20,draw=none](-7.845,1.116)--(-7.815,1.075)--(-7.801,1.074)--(-7.801,1.122)--(-7.846,1.125)--cycle;
\draw(-7.815,1.075)--(-7.801,1.074)--(-7.801,1.122)--(-7.846,1.125);
\filldraw[fill opacity=0.8,fill=gray!20,draw=none](-6.018,1.214)--(-6.019,1.215)--(-6.019,1.215)--cycle;
\draw(-6.018,1.214)--(-6.019,1.215);
\filldraw[fill opacity=0.8,fill=gray!20,draw=none](-7.072,.974)--(-7.07,.963)--(-7.067,.96)--cycle;
\draw(-7.07,.963)--(-7.067,.96);
\filldraw[fill opacity=0.5,fill=gray!20](-10.882,-.139)--(-10.912,-.09)--(-11.092,.364)--(-11.065,.324)--cycle;
\filldraw[fill opacity=0.8,fill=gray!20](-6.947,1.286)--(-6.89,1.293)--(-6.89,1.293)--(-6.938,1.292)--cycle;
\filldraw[fill opacity=0.5,fill=gray!20](-9.92,-.811)--(-10.005,-.844)--(-10.378,-.573)--(-10.28,-.548)--cycle;
\filldraw[fill opacity=0.8,fill=gray!20,draw=none](-7.065,1.193)--(-7.058,1.213)--(-7.06,1.213)--(-7.066,1.205)--cycle;
\draw(-7.06,1.213)--(-7.066,1.205);
\filldraw[fill opacity=0.8,fill=gray!20](-6.799,.88)--(-6.761,.908)--(-6.821,.897)--(-6.841,.872)--cycle;
\filldraw[fill opacity=0.8,fill=gray!20,draw=none](-7.032,1.21)--(-7.029,1.215)--(-7.045,1.232)--(-7.06,1.213)--cycle;
\draw(-7.032,1.21)--(-7.029,1.215)--(-7.045,1.232)--(-7.06,1.213);
\filldraw[fill opacity=0.8,fill=gray!20,draw=none](-6.083,.92)--(-6.101,.921)--(-6.101,.932)--cycle;
\draw(-6.101,.921)--(-6.101,.932);
\filldraw[fill opacity=0.8,fill=gray!20](-6.839,1.291)--(-6.89,1.293)--(-6.89,1.293)--(-6.834,1.285)--cycle;
\filldraw[fill opacity=0.5,fill=gray!20](-10.175,2.06)--(-10.636,2.262)--(-10.292,2.587)--(-9.831,2.386)--cycle;
\filldraw[fill opacity=0.5,fill=gray!20](-10.642,2.294)--(-10.636,2.262)--(-10.292,2.587)--(-10.281,2.635)--cycle;
\filldraw[fill opacity=0.8,fill=gray!20,draw=none](-6.129,1.154)--(-6.129,.952)--(-6.136,.986)--(-6.136,1.095)--cycle;
\draw(-6.129,1.154)--(-6.129,.952);
\draw(-6.136,.986)--(-6.136,1.095);
\filldraw[fill opacity=0.8,fill=gray!20,draw=none](-6.116,1.18)--(-6.151,1.18)--(-6.122,1.205)--(-6.072,1.221)--(-6.057,1.22)--cycle;
\draw(-6.151,1.18)--(-6.122,1.205)--(-6.072,1.221)--(-6.057,1.22);
\filldraw[fill opacity=0.8,fill=gray!20,draw=none](-6.09,1.178)--(-6.056,1.205)--(-6.056,1.177)--cycle;
\draw(-6.056,1.205)--(-6.056,1.177);
\filldraw[fill opacity=0.8,fill=gray!20,draw=none](-6.042,1.212)--(-6.056,1.205)--(-6.056,1.214)--cycle;
\draw(-6.056,1.205)--(-6.056,1.214);
\filldraw[fill opacity=0.8,fill=gray!20,draw=none](-6.086,.991)--(-6.09,.971)--(-6.012,.968)--(-6.014,1.015)--(-6.071,1.017)--cycle;
\draw(-6.09,.971)--(-6.012,.968)--(-6.014,1.015)--(-6.071,1.017);
\filldraw[fill opacity=0.8,fill=gray!20,draw=none](-6.071,1.017)--(-6.086,.99)--(-6.086,.964)--(-6.036,.912)--(-6.036,.948)--cycle;
\draw(-6.086,.99)--(-6.086,.964);
\draw(-6.036,.912)--(-6.036,.948);
\filldraw[fill opacity=0.8,fill=gray!20](-6.029,.939)--(-6.193,1.011)--(-6.164,.967)--(-6.001,.896)--cycle;
\filldraw[fill opacity=0.8,fill=gray!20,draw=none](-6.101,.921)--(-6.101,.885)--(-6.118,.879)--(-6.129,.888)--(-6.129,.92)--cycle;
\draw(-6.101,.921)--(-6.101,.885)--(-6.118,.879);
\draw(-6.129,.888)--(-6.129,.92);
\filldraw[fill opacity=0.8,fill=gray!20,draw=none](-6.113,.941)--(-6.101,.932)--(-6.101,.921)--(-6.129,.92)--(-6.129,.928)--cycle;
\draw(-6.101,.932)--(-6.101,.921);
\draw(-6.129,.92)--(-6.129,.928);
\filldraw[fill opacity=0.8,fill=gray!20,draw=none](-6.132,.92)--(-6.129,.928)--(-6.129,.92)--cycle;
\draw(-6.129,.928)--(-6.129,.92);
\filldraw[fill opacity=0.8,fill=gray!20,draw=none](-6.326,.949)--(-6.112,.94)--(-6.065,.917)--(-6.355,.929)--cycle;
\draw(-6.326,.949)--(-6.112,.94);
\draw(-6.065,.917)--(-6.355,.929);
\filldraw[fill opacity=0.8,fill=gray!20,draw=none](-6.02,1.201)--(-6.057,1.22)--(-6.072,1.221)--(-6.02,1.199)--cycle;
\draw(-6.057,1.22)--(-6.072,1.221)--(-6.02,1.199);
\filldraw[fill opacity=0.8,fill=gray!20,draw=none](-6.161,.966)--(-6.168,.97)--(-6.167,.971)--cycle;
\filldraw[fill opacity=0.8,fill=gray!20,draw=none](-6.737,.933)--(-6.733,.942)--(-6.736,.942)--(-6.754,.918)--cycle;
\draw(-6.736,.942)--(-6.754,.918);
\filldraw[fill opacity=0.8,fill=gray!20](-7.029,1.215)--(-6.988,1.253)--(-7,1.266)--(-7.045,1.232)--cycle;
\filldraw[fill opacity=0.8,fill=gray!20,draw=none](-6.204,.95)--(-6.168,.97)--(-6.121,.945)--(-6.126,.941)--(-6.171,.942)--cycle;
\draw(-6.126,.941)--(-6.171,.942);
\filldraw[fill opacity=0.8,fill=gray!20,draw=none](-7.072,.974)--(-7.067,.96)--(-7.061,.954)--(-7.061,.954)--(-7.074,.987)--cycle;
\draw(-7.067,.96)--(-7.061,.954);
\draw(-7.061,.954)--(-7.074,.987);
\filldraw[fill opacity=0.8,fill=gray!20,draw=none](-8.228,3.06)--(-8.149,3.025)--(-8.108,3.042)--(-8.21,3.086)--cycle;
\draw(-8.228,3.06)--(-8.149,3.025);
\draw(-8.108,3.042)--(-8.21,3.086);
\filldraw[fill opacity=0.8,fill=gray!20,draw=none](-7.065,1.193)--(-7.061,1.168)--(-7.06,1.167)--(-7.032,1.21)--(-7.058,1.213)--cycle;
\draw(-7.061,1.168)--(-7.06,1.167)--(-7.032,1.21);
\filldraw[fill opacity=0.8,fill=gray!20,draw=none](-6.729,1.164)--(-6.7,1.183)--(-6.714,1.201)--(-6.724,1.201)--(-6.733,1.18)--cycle;
\draw(-6.729,1.164)--(-6.7,1.183)--(-6.714,1.201);
\filldraw[fill opacity=0.8,fill=gray!20,draw=none](-6.899,1.187)--(-6.729,1.18)--(-6.716,1.2)--(-6.89,1.207)--cycle;
\draw(-6.716,1.2)--(-6.89,1.207)--(-6.899,1.187)--(-6.729,1.18);
\filldraw[fill opacity=0.5,fill=gray!20](-10.74,1.774)--(-10.567,1.698)--(-10.333,2.058)--(-10.506,2.133)--cycle;
\filldraw[fill opacity=0.5,fill=gray!20](-10.9,1.857)--(-10.74,1.774)--(-10.506,2.133)--(-10.636,2.262)--cycle;
\filldraw[fill opacity=0.8,fill=gray!20,draw=none](-6.056,1.214)--(-6.072,1.221)--(-6.122,1.205)--(-6.101,1.196)--cycle;
\draw(-6.056,1.214)--(-6.072,1.221)--(-6.122,1.205)--(-6.101,1.196);
\filldraw[fill opacity=0.8,fill=gray!20](-6.976,.899)--(-6.995,.937)--(-7.06,.953)--(-7.029,.912)--cycle;
\filldraw[fill opacity=0.5,fill=gray!20](-10.464,-.577)--(-10.536,-.56)--(-10.829,-.174)--(-10.756,-.194)--cycle;
\filldraw[fill opacity=0.8,fill=gray!20](-6.988,1.253)--(-6.941,1.28)--(-6.947,1.286)--(-7,1.266)--cycle;
\filldraw[fill opacity=0.8,fill=gray!20,draw=none](-6.724,1.201)--(-6.714,1.201)--(-6.72,1.209)--cycle;
\draw(-6.714,1.201)--(-6.72,1.209);
\filldraw[fill opacity=0.8,fill=gray!20,draw=none](-6.204,.95)--(-6.171,.942)--(-6.215,.944)--cycle;
\draw(-6.171,.942)--(-6.215,.944);
\filldraw[fill opacity=0.8,fill=gray!20,draw=none](-6.724,1.201)--(-6.72,1.209)--(-6.735,1.228)--(-6.761,1.211)--(-6.754,1.2)--cycle;
\draw(-6.72,1.209)--(-6.735,1.228)--(-6.761,1.211)--(-6.754,1.2);
\filldraw[fill opacity=0.5,fill=gray!20](-9.708,2.901)--(-9.769,2.916)--(-9.294,3.036)--(-9.231,3.021)--cycle;
\filldraw[fill opacity=0.5,fill=gray!20](-9.106,2.915)--(-9.167,2.981)--(-8.693,2.964)--(-8.641,2.899)--cycle;
\filldraw[fill opacity=0.8,fill=gray!20,draw=none](-7.072,.974)--(-7.074,.987)--(-7.08,1.002)--(-7.082,1.005)--cycle;
\draw(-7.074,.987)--(-7.08,1.002)--(-7.082,1.005);
\filldraw[fill opacity=0.8,fill=gray!20](-7.899,3.326)--(-8.014,3.368)--(-7.792,3.955)--cycle;
\filldraw[fill opacity=0.8,fill=gray!20](-7.792,3.955)--(-7.678,3.913)--(-7.899,3.326)--cycle;
\filldraw[fill opacity=0.8,fill=gray!20](-7.779,3.281)--(-7.899,3.326)--(-7.678,3.913)--cycle;
\filldraw[fill opacity=0.8,fill=gray!20,draw=none](-7.929,3.336)--(-7.977,3.101)--(-8.072,3.137)--(-7.989,3.357)--cycle;
\draw(-7.929,3.336)--(-7.977,3.101)--(-8.072,3.137)--(-7.989,3.357);
\filldraw[fill opacity=0.8,fill=gray!20,draw=none](-7.929,3.336)--(-7.893,3.322)--(-7.977,3.101)--cycle;
\draw(-7.893,3.322)--(-7.977,3.101)--(-7.929,3.336);
\filldraw[fill opacity=0.8,fill=gray!20,draw=none](-7.831,3.299)--(-7.877,3.064)--(-7.977,3.101)--(-7.893,3.322)--cycle;
\draw(-7.831,3.299)--(-7.877,3.064)--(-7.977,3.101)--(-7.893,3.322);
\filldraw[fill opacity=0.8,fill=gray!20,draw=none](-7.831,3.299)--(-7.793,3.285)--(-7.877,3.064)--cycle;
\draw(-7.793,3.285)--(-7.877,3.064)--(-7.831,3.299);
\filldraw[fill opacity=0.8,fill=gray!20,draw=none](-7.738,3.263)--(-7.788,3.029)--(-7.877,3.064)--(-7.793,3.285)--cycle;
\draw(-7.738,3.263)--(-7.788,3.029)--(-7.877,3.064)--(-7.793,3.285);
\filldraw[fill opacity=0.8,fill=gray!20,draw=none](-8.018,3.368)--(-8.072,3.137)--(-8.149,3.163)--(-8.066,3.384)--cycle;
\draw(-8.018,3.368)--(-8.072,3.137)--(-8.149,3.163)--(-8.066,3.384);
\filldraw[fill opacity=0.8,fill=gray!20,draw=none](-8.018,3.368)--(-7.989,3.357)--(-8.072,3.137)--cycle;
\draw(-7.989,3.357)--(-8.072,3.137)--(-8.018,3.368);
\filldraw[fill opacity=0.8,fill=gray!20,draw=none](-7.738,3.263)--(-7.705,3.25)--(-7.788,3.029)--cycle;
\draw(-7.705,3.25)--(-7.788,3.029)--(-7.738,3.263);
\filldraw[fill opacity=0.8,fill=gray!20,draw=none](-7.665,3.233)--(-7.724,3.002)--(-7.788,3.029)--(-7.705,3.25)--cycle;
\draw(-7.665,3.233)--(-7.724,3.002)--(-7.788,3.029)--(-7.705,3.25);
\filldraw[fill opacity=0.8,fill=gray!20,draw=none](-8.084,3.39)--(-8.149,3.163)--(-8.196,3.178)--(-8.113,3.399)--cycle;
\draw(-8.084,3.39)--(-8.149,3.163)--(-8.196,3.178)--(-8.113,3.399);
\filldraw[fill opacity=0.8,fill=gray!20,draw=none](-8.084,3.39)--(-8.066,3.384)--(-8.149,3.163)--cycle;
\draw(-8.066,3.384)--(-8.149,3.163)--(-8.084,3.39);
\filldraw[fill opacity=0.8,fill=gray!20,draw=none](-7.665,3.233)--(-7.641,3.223)--(-7.724,3.002)--cycle;
\draw(-7.641,3.223)--(-7.724,3.002)--(-7.665,3.233);
\filldraw[fill opacity=0.8,fill=gray!20,draw=none](-7.622,3.214)--(-7.694,2.988)--(-7.724,3.002)--(-7.641,3.223)--cycle;
\draw(-7.622,3.214)--(-7.694,2.988)--(-7.724,3.002)--(-7.641,3.223);
\filldraw[fill opacity=0.8,fill=gray!20,draw=none](-8.116,3.399)--(-8.196,3.178)--(-8.205,3.178)--(-8.122,3.399)--cycle;
\draw(-8.116,3.399)--(-8.196,3.178)--(-8.205,3.178)--(-8.122,3.399);
\filldraw[fill opacity=0.8,fill=gray!20,draw=none](-8.116,3.399)--(-8.113,3.399)--(-8.196,3.178)--cycle;
\draw(-8.113,3.399)--(-8.196,3.178)--(-8.116,3.399);
\filldraw[fill opacity=0.8,fill=gray!20,draw=none](-7.622,3.214)--(-7.611,3.209)--(-7.694,2.988)--cycle;
\draw(-7.611,3.209)--(-7.694,2.988)--(-7.622,3.214);
\filldraw[fill opacity=0.8,fill=gray!20](-8.018,3.063)--(-8.02,3.087)--(-7.966,3.083)--(-7.941,3.057)--cycle;
\filldraw[fill opacity=0.8,fill=gray!20](-7.941,3.057)--(-7.966,3.083)--(-7.929,3.074)--(-7.888,3.044)--cycle;
\filldraw[fill opacity=0.8,fill=gray!20](-8.02,3.087)--(-8.024,3.097)--(-7.996,3.095)--(-7.966,3.083)--cycle;
\filldraw[fill opacity=0.8,fill=gray!20](-7.966,3.083)--(-7.996,3.095)--(-7.976,3.09)--(-7.929,3.074)--cycle;
\filldraw[fill opacity=0.8,fill=gray!20](-7.922,3.019)--(-7.941,3.057)--(-7.888,3.044)--(-7.857,3.003)--cycle;
\filldraw[fill opacity=0.8,fill=gray!20](-7.857,3.003)--(-7.888,3.044)--(-7.872,3.027)--(-7.837,2.982)--cycle;
\filldraw[fill opacity=0.8,fill=gray!20](-7.888,3.044)--(-7.929,3.074)--(-7.917,3.062)--(-7.872,3.027)--cycle;
\filldraw[fill opacity=0.8,fill=gray!20](-8.076,3.085)--(-8.052,3.096)--(-8.024,3.097)--(-8.02,3.087)--cycle;
\filldraw[fill opacity=0.8,fill=gray!20](-8.024,3.097)--(-8.027,3.092)--(-8.027,3.092)--(-7.996,3.095)--cycle;
\filldraw[fill opacity=0.8,fill=gray!20](-7.929,3.074)--(-7.976,3.09)--(-7.97,3.084)--(-7.917,3.062)--cycle;
\filldraw[fill opacity=0.8,fill=gray!20](-7.996,3.095)--(-8.027,3.092)--(-8.027,3.092)--(-7.976,3.09)--cycle;
\filldraw[fill opacity=0.8,fill=gray!20](-7.976,3.09)--(-8.027,3.092)--(-8.027,3.092)--(-7.97,3.084)--cycle;
\filldraw[fill opacity=0.8,fill=gray!20](-8.016,3.026)--(-8.018,3.063)--(-7.941,3.057)--(-7.922,3.019)--cycle;
\filldraw[fill opacity=0.8,fill=gray!20](-7.917,3.062)--(-7.97,3.084)--(-7.98,3.078)--(-7.935,3.05)--cycle;
\filldraw[fill opacity=0.8,fill=gray!20](-7.97,3.084)--(-8.027,3.092)--(-8.027,3.092)--(-7.98,3.078)--cycle;
\filldraw[fill opacity=0.8,fill=gray!20](-7.922,3.064)--(-8.022,3.102)--(-8.111,3.137)--(-8.175,3.164)--(-8.205,3.178)--(-8.196,3.178)--(-8.149,3.163)--(-8.072,3.137)--(-7.977,3.101)--(-7.877,3.064)--(-7.788,3.029)--(-7.724,3.002)--(-7.694,2.988)--(-7.703,2.988)--(-7.75,3.003)--(-7.827,3.029)--cycle;
\filldraw[fill opacity=0.8,fill=gray!20,draw=none](-7.616,3.209)--(-7.703,2.988)--(-7.694,2.988)--(-7.611,3.209)--cycle;
\draw(-7.616,3.209)--(-7.703,2.988)--(-7.694,2.988)--(-7.611,3.209);
\filldraw[fill opacity=0.8,fill=gray!20,draw=none](-8.11,3.394)--(-8.205,3.178)--(-8.175,3.164)--(-8.092,3.385)--cycle;
\draw(-8.11,3.394)--(-8.205,3.178)--(-8.175,3.164)--(-8.092,3.385);
\filldraw[fill opacity=0.8,fill=gray!20,draw=none](-8.11,3.394)--(-8.122,3.399)--(-8.205,3.178)--cycle;
\draw(-8.122,3.399)--(-8.205,3.178)--(-8.11,3.394);
\filldraw[fill opacity=0.8,fill=gray!20,draw=none](-7.616,3.209)--(-7.62,3.209)--(-7.703,2.988)--cycle;
\draw(-7.62,3.209)--(-7.703,2.988)--(-7.616,3.209);
\filldraw[fill opacity=0.8,fill=gray!20,draw=none](-7.649,3.218)--(-7.75,3.003)--(-7.703,2.988)--(-7.62,3.209)--cycle;
\draw(-7.649,3.218)--(-7.75,3.003)--(-7.703,2.988)--(-7.62,3.209);
\filldraw[fill opacity=0.8,fill=gray!20,draw=none](-8.068,3.375)--(-8.175,3.164)--(-8.111,3.137)--(-8.028,3.358)--cycle;
\draw(-8.068,3.375)--(-8.175,3.164)--(-8.111,3.137)--(-8.028,3.358);
\filldraw[fill opacity=0.8,fill=gray!20,draw=none](-8.068,3.375)--(-8.092,3.385)--(-8.175,3.164)--cycle;
\draw(-8.092,3.385)--(-8.175,3.164)--(-8.068,3.375);
\filldraw[fill opacity=0.8,fill=gray!20,draw=none](-7.649,3.218)--(-7.667,3.223)--(-7.75,3.003)--cycle;
\draw(-7.667,3.223)--(-7.75,3.003)--(-7.649,3.218);
\filldraw[fill opacity=0.8,fill=gray!20,draw=none](-7.714,3.24)--(-7.827,3.029)--(-7.75,3.003)--(-7.667,3.223)--cycle;
\draw(-7.714,3.24)--(-7.827,3.029)--(-7.75,3.003)--(-7.667,3.223);
\filldraw[fill opacity=0.8,fill=gray!20,draw=none](-7.994,3.345)--(-8.111,3.137)--(-8.022,3.102)--(-7.939,3.323)--cycle;
\draw(-7.994,3.345)--(-8.111,3.137)--(-8.022,3.102)--(-7.939,3.323);
\filldraw[fill opacity=0.8,fill=gray!20,draw=none](-7.994,3.345)--(-8.028,3.358)--(-8.111,3.137)--cycle;
\draw(-8.028,3.358)--(-8.111,3.137)--(-7.994,3.345);
\filldraw[fill opacity=0.8,fill=gray!20,draw=none](-7.714,3.24)--(-7.744,3.25)--(-7.827,3.029)--cycle;
\draw(-7.744,3.25)--(-7.827,3.029)--(-7.714,3.24);
\filldraw[fill opacity=0.8,fill=gray!20,draw=none](-7.803,3.272)--(-7.922,3.064)--(-7.827,3.029)--(-7.744,3.25)--cycle;
\draw(-7.803,3.272)--(-7.922,3.064)--(-7.827,3.029)--(-7.744,3.25);
\filldraw[fill opacity=0.8,fill=gray!20,draw=none](-7.901,3.309)--(-8.022,3.102)--(-7.922,3.064)--(-7.839,3.285)--cycle;
\draw(-7.901,3.309)--(-8.022,3.102)--(-7.922,3.064)--(-7.839,3.285);
\filldraw[fill opacity=0.8,fill=gray!20,draw=none](-7.901,3.309)--(-7.939,3.323)--(-8.022,3.102)--cycle;
\draw(-7.939,3.323)--(-8.022,3.102)--(-7.901,3.309);
\filldraw[fill opacity=0.8,fill=gray!20,draw=none](-7.803,3.272)--(-7.839,3.285)--(-7.922,3.064)--cycle;
\draw(-7.839,3.285)--(-7.922,3.064)--(-7.803,3.272);
\filldraw[fill opacity=0.8,fill=gray!20](-7.834,3.282)--(-7.954,3.327)--(-8.06,3.369)--(-8.137,3.401)--(-8.173,3.418)--(-8.162,3.418)--(-8.106,3.401)--(-8.014,3.368)--(-7.899,3.326)--(-7.779,3.281)--(-7.672,3.239)--(-7.596,3.207)--(-7.56,3.19)--(-7.571,3.19)--(-7.627,3.207)--(-7.719,3.24)--cycle;
\filldraw[fill opacity=0.8,fill=gray!20](-6.941,1.28)--(-6.89,1.293)--(-6.89,1.293)--(-6.947,1.286)--cycle;
\filldraw[fill opacity=0.8,fill=gray!20](-6.735,1.228)--(-6.781,1.263)--(-6.799,1.251)--(-6.761,1.211)--cycle;
\filldraw[fill opacity=0.8,fill=gray!20,draw=none](-8.439,.839)--(-8.309,.868)--(-8.175,.894)--(-8.375,.849)--cycle;
\draw(-8.439,.839)--(-8.309,.868);
\draw(-8.175,.894)--(-8.375,.849);
\filldraw[fill opacity=0.8,fill=gray!20,draw=none](-8.491,.841)--(-8.5,.85)--(-8.474,.856)--(-8.309,.868)--(-8.439,.839)--cycle;
\draw(-8.5,.85)--(-8.474,.856);
\draw(-8.309,.868)--(-8.439,.839);
\filldraw[fill opacity=0.8,fill=gray!20,draw=none](-8.375,.849)--(-8.175,.894)--(-8.091,.928)--(-8.352,.869)--cycle;
\draw(-8.375,.849)--(-8.175,.894);
\draw(-8.091,.928)--(-8.352,.869);
\filldraw[fill opacity=0.8,fill=gray!20,draw=none](-7.07,1.178)--(-7.061,1.168)--(-7.065,1.193)--cycle;
\draw(-7.07,1.178)--(-7.061,1.168);
\filldraw[fill opacity=0.8,fill=gray!20,draw=none](-6.136,1.095)--(-6.136,.986)--(-6.121,1.029)--cycle;
\draw(-6.136,1.095)--(-6.136,.986);
\filldraw[fill opacity=0.8,fill=gray!20,draw=none](-6.09,1.075)--(-6.09,1.069)--(-6.012,1.066)--(-6.007,1.114)--(-6.074,1.116)--cycle;
\draw(-6.09,1.069)--(-6.012,1.066)--(-6.007,1.114)--(-6.074,1.116);
\filldraw[fill opacity=0.8,fill=gray!20,draw=none](-6.09,1.069)--(-6.121,1.029)--(-6.086,.964)--(-6.086,1.047)--cycle;
\draw(-6.086,.964)--(-6.086,1.047);
\filldraw[fill opacity=0.8,fill=gray!20,draw=none](-6.091,.981)--(-6.09,.971)--(-6.086,.991)--cycle;
\filldraw[fill opacity=0.8,fill=gray!20,draw=none](-6.071,1.017)--(-6.086,1.047)--(-6.086,.99)--cycle;
\draw(-6.086,1.047)--(-6.086,.99);
\filldraw[fill opacity=0.8,fill=gray!20,draw=none](-6.091,.981)--(-6.086,.991)--(-6.081,1.017)--(-6.096,1.018)--cycle;
\draw(-6.081,1.017)--(-6.096,1.018);
\filldraw[fill opacity=0.8,fill=gray!20,draw=none](-6.086,.991)--(-6.071,1.017)--(-6.081,1.017)--cycle;
\draw(-6.071,1.017)--(-6.081,1.017);
\filldraw[fill opacity=0.8,fill=gray!20,draw=none](-6.096,1.018)--(-6.014,1.015)--(-6.012,1.066)--(-6.09,1.069)--cycle;
\draw(-6.096,1.018)--(-6.014,1.015)--(-6.012,1.066)--(-6.09,1.069);
\filldraw[fill opacity=0.8,fill=gray!20,draw=none](-6.057,1.042)--(-6.071,1.017)--(-6.036,.948)--(-6.036,1.028)--cycle;
\draw(-6.036,.948)--(-6.036,1.028);
\filldraw[fill opacity=0.8,fill=gray!20,draw=none](-6.036,1.028)--(-6.036,.948)--(-5.98,.962)--(-5.98,1.031)--cycle;
\draw(-6.036,1.028)--(-6.036,.948);
\draw(-5.98,.962)--(-5.98,1.031);
\filldraw[fill opacity=0.8,fill=gray!20](-6.039,.993)--(-6.203,1.064)--(-6.193,1.011)--(-6.029,.939)--cycle;
\filldraw[fill opacity=0.8,fill=gray!20](-6.781,1.263)--(-6.834,1.285)--(-6.843,1.279)--(-6.799,1.251)--cycle;
\filldraw[fill opacity=0.8,fill=gray!20](-6.834,1.285)--(-6.89,1.293)--(-6.89,1.293)--(-6.843,1.279)--cycle;
\filldraw[fill opacity=0.8,fill=gray!20](-6.922,1.275)--(-6.89,1.293)--(-6.89,1.293)--(-6.941,1.28)--cycle;
\filldraw[fill opacity=0.8,fill=gray!20](-6.897,.869)--(-6.9,.893)--(-6.976,.899)--(-6.951,.873)--cycle;
\filldraw[fill opacity=0.8,fill=gray!20](-6.843,1.279)--(-6.89,1.293)--(-6.89,1.293)--(-6.865,1.274)--cycle;
\filldraw[fill opacity=0.8,fill=gray!20](-6.894,1.273)--(-6.89,1.293)--(-6.89,1.293)--(-6.922,1.275)--cycle;
\filldraw[fill opacity=0.8,fill=gray!20](-6.865,1.274)--(-6.89,1.293)--(-6.89,1.293)--(-6.894,1.273)--cycle;
\filldraw[fill opacity=0.8,fill=gray!20,draw=none](-6.761,.908)--(-6.736,.942)--(-6.765,.942)--(-6.805,.934)--(-6.821,.897)--cycle;
\draw(-6.765,.942)--(-6.805,.934)--(-6.821,.897)--(-6.761,.908)--(-6.736,.942);
\filldraw[fill opacity=0.8,fill=gray!20,draw=none](-6.729,.95)--(-6.733,.942)--(-6.729,.942)--cycle;
\draw(-6.733,.942)--(-6.729,.942);
\filldraw[fill opacity=0.8,fill=gray!20,draw=none](-6.042,1.212)--(-6,1.237)--(-6,1.229)--(-6.02,1.209)--cycle;
\draw(-6,1.237)--(-6,1.229);
\filldraw[fill opacity=0.8,fill=gray!20,draw=none](-5.984,1.244)--(-6,1.229)--(-6,1.237)--cycle;
\draw(-6,1.229)--(-6,1.237);
\filldraw[fill opacity=0.8,fill=gray!20](-7.678,3.913)--(-7.558,3.867)--(-7.779,3.281)--cycle;
\filldraw[fill opacity=0.8,fill=gray!20](-7.672,3.239)--(-7.779,3.281)--(-7.558,3.867)--cycle;
\filldraw[fill opacity=0.8,fill=gray!20,draw=none](-6.121,.945)--(-6.112,.94)--(-6.126,.941)--cycle;
\draw(-6.112,.94)--(-6.126,.941);
\filldraw[fill opacity=0.8,fill=gray!20,draw=none](-6.113,.941)--(-6.129,.928)--(-6.129,.952)--cycle;
\draw(-6.129,.928)--(-6.129,.952);
\filldraw[fill opacity=0.8,fill=gray!20,draw=none](-6.135,.909)--(-6.132,.92)--(-6.129,.92)--(-6.129,.888)--cycle;
\draw(-6.129,.92)--(-6.129,.888);
\filldraw[fill opacity=0.8,fill=gray!20,draw=none](-6.132,.92)--(-6.135,.909)--(-6.136,.913)--(-6.136,.919)--cycle;
\draw(-6.136,.913)--(-6.136,.919);
\filldraw[fill opacity=0.8,fill=gray!20,draw=none](-6.147,.921)--(-6.065,.917)--(-6.056,.915)--(-6.132,.919)--cycle;
\draw(-6.147,.921)--(-6.065,.917);
\draw(-6.056,.915)--(-6.132,.919);
\filldraw[fill opacity=0.8,fill=gray!20,draw=none](-6.129,.952)--(-6.129,.928)--(-6.132,.92)--(-6.136,.919)--(-6.136,.96)--cycle;
\draw(-6.129,.952)--(-6.129,.928);
\draw(-6.136,.919)--(-6.136,.96);
\filldraw[fill opacity=0.8,fill=gray!20,draw=none](-6.129,.952)--(-6.136,.96)--(-6.136,.986)--cycle;
\draw(-6.136,.96)--(-6.136,.986);
\filldraw[fill opacity=0.8,fill=gray!20,draw=none](-6.136,.986)--(-6.136,.913)--(-6.121,.934)--(-6.121,.99)--cycle;
\draw(-6.136,.986)--(-6.136,.913);
\draw(-6.121,.934)--(-6.121,.99);
\filldraw[fill opacity=0.8,fill=gray!20,draw=none](-6.18,.922)--(-6.147,.921)--(-6.132,.919)--(-6.233,.923)--cycle;
\draw(-6.18,.922)--(-6.147,.921);
\draw(-6.132,.919)--(-6.233,.923);
\filldraw[fill opacity=0.8,fill=gray!20,draw=none](-6.092,.968)--(-6.121,.99)--(-6.121,.934)--(-6.116,.936)--cycle;
\draw(-6.121,.99)--(-6.121,.934);
\filldraw[fill opacity=0.8,fill=gray!20,draw=none](-6.729,.95)--(-6.729,.942)--(-6.074,.916)--(-6.074,.928)--(-6.084,.935)--(-6.725,.96)--cycle;
\draw(-6.729,.942)--(-6.074,.916);
\draw(-6.084,.935)--(-6.725,.96);
\filldraw[fill opacity=0.8,fill=gray!20,draw=none](-8.276,2.85)--(-8.11,2.778)--(-8.136,2.817)--(-8.278,2.878)--cycle;
\draw(-8.276,2.85)--(-8.11,2.778)--(-8.136,2.817)--(-8.278,2.878);
\filldraw[fill opacity=0.8,fill=gray!20,draw=none](-6,1.253)--(-6,1.237)--(-6.042,1.212)--(-6.056,1.214)--(-6.056,1.25)--cycle;
\draw(-6.056,1.214)--(-6.056,1.25)--(-6,1.253)--(-6,1.237);
\filldraw[fill opacity=0.8,fill=gray!20,draw=none](-5.975,1.253)--(-5.984,1.244)--(-6,1.237)--(-6,1.253)--cycle;
\draw(-6,1.237)--(-6,1.253)--(-5.975,1.253);
\filldraw[fill opacity=0.8,fill=gray!20,draw=none](-6.119,1.178)--(-6.101,1.196)--(-6.122,1.205)--(-6.151,1.18)--cycle;
\draw(-6.101,1.196)--(-6.122,1.205)--(-6.151,1.18);
\filldraw[fill opacity=0.8,fill=gray!20](-6.841,.872)--(-6.821,.897)--(-6.9,.893)--(-6.897,.869)--cycle;
\filldraw[fill opacity=0.5,fill=gray!20](-10.888,1.369)--(-10.715,1.294)--(-10.567,1.698)--(-10.74,1.774)--cycle;
\filldraw[fill opacity=0.5,fill=gray!20](-11.066,1.402)--(-10.888,1.369)--(-10.74,1.774)--(-10.9,1.857)--cycle;
\filldraw[fill opacity=0.8,fill=gray!20,draw=none](-7.061,.954)--(-7.06,.953)--(-7.061,.954)--cycle;
\draw(-7.061,.954)--(-7.06,.953)--(-7.061,.954);
\filldraw[fill opacity=0.8,fill=gray!20,draw=none](-6.118,.879)--(-6.129,.874)--(-6.129,.888)--cycle;
\draw(-6.118,.879)--(-6.129,.874)--(-6.129,.888);
\filldraw[fill opacity=0.8,fill=gray!20,draw=none](-5.975,1.253)--(-5.965,1.252)--(-5.984,1.244)--cycle;
\draw(-5.975,1.253)--(-5.965,1.252);
\filldraw[fill opacity=0.8,fill=gray!20,draw=none](-5.979,1.246)--(-5.975,1.253)--(-6,1.253)--(-6.051,1.25)--cycle;
\draw(-5.975,1.253)--(-6,1.253)--(-6.051,1.25);
\filldraw[fill opacity=0.8,fill=gray!20,draw=none](-6.907,1.149)--(-6.741,1.143)--(-6.729,1.16)--(-6.733,1.18)--(-6.899,1.187)--cycle;
\draw(-6.733,1.18)--(-6.899,1.187)--(-6.907,1.149)--(-6.741,1.143);
\filldraw[fill opacity=0.8,fill=gray!20,draw=none](-6.724,1.201)--(-6.754,1.2)--(-6.736,1.171)--cycle;
\draw(-6.754,1.2)--(-6.736,1.171);
\filldraw[fill opacity=0.8,fill=gray!20,draw=none](-6.995,.937)--(-7.007,.984)--(-7.058,.997)--(-7.074,.987)--(-7.06,.953)--cycle;
\draw(-7.074,.987)--(-7.06,.953)--(-6.995,.937)--(-7.007,.984)--(-7.058,.997);
\filldraw[fill opacity=0.8,fill=gray!20,draw=none](-8.352,.869)--(-8.091,.928)--(-8.072,.966)--(-8.372,.898)--cycle;
\draw(-8.352,.869)--(-8.091,.928);
\draw(-8.072,.966)--(-8.372,.898);
\filldraw[fill opacity=0.8,fill=gray!20,draw=none](-6.167,1.166)--(-6.168,1.166)--(-6.161,1.168)--cycle;
\filldraw[fill opacity=0.8,fill=gray!20,draw=none](-6.144,1.111)--(-6.17,1.159)--(-6.193,1.12)--(-6.148,1.1)--cycle;
\draw(-6.17,1.159)--(-6.193,1.12)--(-6.148,1.1);
\filldraw[fill opacity=0.8,fill=gray!20,draw=none](-6.074,1.116)--(-6.074,1.13)--(-6.129,1.154)--(-6.144,1.111)--(-6.136,1.095)--(-6.09,1.075)--cycle;
\draw(-6.074,1.13)--(-6.129,1.154);
\draw(-6.136,1.095)--(-6.09,1.075);
\filldraw[fill opacity=0.8,fill=gray!20,draw=none](-6.133,1.173)--(-6.132,1.177)--(-6.129,1.178)--(-6.129,1.154)--(-6.136,1.095)--cycle;
\draw(-6.129,1.178)--(-6.129,1.154);
\filldraw[fill opacity=0.8,fill=gray!20,draw=none](-6.056,1.153)--(-6.06,1.162)--(-6.118,1.149)--(-6.074,1.13)--cycle;
\draw(-6.118,1.149)--(-6.074,1.13);
\filldraw[fill opacity=0.8,fill=gray!20,draw=none](-6.133,1.173)--(-6.136,1.095)--(-6.136,1.162)--cycle;
\draw(-6.136,1.095)--(-6.136,1.162);
\filldraw[fill opacity=0.8,fill=gray!20,draw=none](-6.09,1.069)--(-6.086,1.047)--(-6.086,1.073)--cycle;
\draw(-6.086,1.047)--(-6.086,1.073);
\filldraw[fill opacity=0.8,fill=gray!20,draw=none](-6.071,1.017)--(-6.057,1.042)--(-6.086,1.062)--(-6.086,1.047)--cycle;
\draw(-6.086,1.062)--(-6.086,1.047);
\filldraw[fill opacity=0.8,fill=gray!20,draw=none](-6.057,1.042)--(-6.04,1.071)--(-6.073,1.084)--(-6.086,1.073)--(-6.086,1.062)--cycle;
\draw(-6.086,1.073)--(-6.086,1.062);
\filldraw[fill opacity=0.8,fill=gray!20,draw=none](-6.04,1.071)--(-6.057,1.042)--(-6.036,1.028)--(-6.036,1.07)--cycle;
\draw(-6.036,1.028)--(-6.036,1.07);
\filldraw[fill opacity=0.8,fill=gray!20,draw=none](-6.012,1.07)--(-6.036,1.07)--(-6.036,1.028)--(-5.98,1.031)--(-5.98,1.06)--cycle;
\draw(-6.036,1.07)--(-6.036,1.028);
\draw(-5.98,1.031)--(-5.98,1.06);
\filldraw[fill opacity=0.8,fill=gray!20](-6.029,1.049)--(-6.193,1.12)--(-6.203,1.064)--(-6.039,.993)--cycle;
\filldraw[fill opacity=0.8,fill=gray!20,draw=none](-6.132,1.156)--(-6.136,1.162)--(-6.136,1.095)--(-6.121,1.029)--(-6.121,1.123)--cycle;
\draw(-6.136,1.162)--(-6.136,1.095);
\draw(-6.121,1.029)--(-6.121,1.123);
\filldraw[fill opacity=0.8,fill=gray!20,draw=none](-6.729,1.18)--(-6.023,1.152)--(-6.022,1.16)--(-6.156,1.178)--(-6.716,1.2)--cycle;
\draw(-6.729,1.18)--(-6.023,1.152);
\draw(-6.156,1.178)--(-6.716,1.2);
\filldraw[fill opacity=0.8,fill=gray!20](-7.558,3.867)--(-7.451,3.825)--(-7.672,3.239)--cycle;
\filldraw[fill opacity=0.8,fill=gray!20](-8.014,3.368)--(-8.106,3.401)--(-7.885,3.987)--cycle;
\filldraw[fill opacity=0.8,fill=gray!20](-7.885,3.987)--(-7.792,3.955)--(-8.014,3.368)--cycle;
\filldraw[fill opacity=0.8,fill=gray!20](-7.596,3.207)--(-7.672,3.239)--(-7.451,3.825)--cycle;
\filldraw[fill opacity=0.8,fill=gray!20](-7.451,3.825)--(-7.374,3.793)--(-7.596,3.207)--cycle;
\filldraw[fill opacity=0.8,fill=gray!20](-8.106,3.401)--(-8.162,3.418)--(-7.941,4.004)--cycle;
\filldraw[fill opacity=0.8,fill=gray!20](-7.941,4.004)--(-7.885,3.987)--(-8.106,3.401)--cycle;
\filldraw[fill opacity=0.8,fill=gray!20](-7.56,3.19)--(-7.596,3.207)--(-7.374,3.793)--cycle;
\filldraw[fill opacity=0.8,fill=gray!20](-7.374,3.793)--(-7.339,3.776)--(-7.56,3.19)--cycle;
\filldraw[fill opacity=0.8,fill=gray!20](-7.952,4.005)--(-7.941,4.004)--(-8.162,3.418)--cycle;
\filldraw[fill opacity=0.8,fill=gray!20](-7.498,3.826)--(-7.406,3.794)--(-7.35,3.776)--(-7.339,3.776)--(-7.374,3.793)--(-7.451,3.825)--(-7.558,3.867)--(-7.678,3.913)--(-7.792,3.955)--(-7.885,3.987)--(-7.941,4.004)--(-7.952,4.005)--(-7.916,3.988)--(-7.839,3.956)--(-7.733,3.914)--(-7.613,3.868)--cycle;
\filldraw[fill opacity=0.8,fill=gray!20,draw=none](-7.072,1.156)--(-7.067,1.175)--(-7.07,1.178)--cycle;
\draw(-7.067,1.175)--(-7.07,1.178);
\filldraw[fill opacity=0.8,fill=gray!20,draw=none](-5.979,1.246)--(-5.951,1.244)--(-5.965,1.252)--(-5.975,1.253)--cycle;
\draw(-5.965,1.252)--(-5.975,1.253);
\filldraw[fill opacity=0.8,fill=gray!20,draw=none](-6.056,1.214)--(-6.056,1.205)--(-6.09,1.178)--(-6.101,1.178)--(-6.101,1.218)--cycle;
\draw(-6.056,1.214)--(-6.056,1.205);
\draw(-6.101,1.178)--(-6.101,1.218);
\filldraw[fill opacity=0.8,fill=gray!20,draw=none](-6.031,1.183)--(-6.02,1.199)--(-6.056,1.214)--(-6.101,1.196)--(-6.058,1.177)--cycle;
\draw(-6.02,1.199)--(-6.056,1.214);
\draw(-6.101,1.196)--(-6.058,1.177);
\filldraw[fill opacity=0.8,fill=gray!20,draw=none](-6.056,1.25)--(-6.056,1.214)--(-6.101,1.218)--(-6.101,1.242)--cycle;
\draw(-6.101,1.218)--(-6.101,1.242)--(-6.056,1.25)--(-6.056,1.214);
\filldraw[fill opacity=0.8,fill=gray!20,draw=none](-6.741,.961)--(-6.73,.953)--(-6.713,.998)--(-6.746,.991)--cycle;
\draw(-6.73,.953)--(-6.713,.998)--(-6.746,.991);
\filldraw[fill opacity=0.8,fill=gray!20,draw=none](-6.749,.997)--(-6.746,.991)--(-6.713,.998)--(-6.707,1.052)--(-6.75,1.044)--cycle;
\draw(-6.746,.991)--(-6.713,.998)--(-6.707,1.052)--(-6.75,1.044);
\filldraw[fill opacity=0.8,fill=gray!20,draw=none](-6.907,.968)--(-6.084,.935)--(-6.147,.973)--(-6.912,1.003)--cycle;
\draw(-6.147,.973)--(-6.912,1.003)--(-6.907,.968)--(-6.084,.935);
\filldraw[fill opacity=0.8,fill=gray!20,draw=none](-7.082,1.01)--(-7.082,1.005)--(-7.081,1.004)--cycle;
\draw(-7.082,1.005)--(-7.081,1.004);
\filldraw[fill opacity=0.8,fill=gray!20,draw=none](-6.101,1.242)--(-6.101,1.196)--(-6.119,1.183)--(-6.129,1.191)--(-6.129,1.232)--cycle;
\draw(-6.129,1.191)--(-6.129,1.232)--(-6.101,1.242)--(-6.101,1.196);
\filldraw[fill opacity=0.8,fill=gray!20,draw=none](-5.997,1.189)--(-6.02,1.201)--(-6.02,1.199)--(-6.001,1.191)--cycle;
\draw(-6.02,1.199)--(-6.001,1.191);
\filldraw[fill opacity=0.8,fill=gray!20,draw=none](-6.031,1.183)--(-6.001,1.191)--(-6.02,1.199)--cycle;
\draw(-6.001,1.191)--(-6.02,1.199);
\filldraw[fill opacity=0.8,fill=gray!20,draw=none](-6.022,1.16)--(-6.021,1.173)--(-6.156,1.178)--cycle;
\draw(-6.021,1.173)--(-6.156,1.178);
\filldraw[fill opacity=0.8,fill=gray!20,draw=none](-6.129,1.232)--(-6.129,1.186)--(-6.132,1.177)--(-6.136,1.176)--(-6.136,1.219)--cycle;
\draw(-6.136,1.176)--(-6.136,1.219)--(-6.129,1.232)--(-6.129,1.186);
\filldraw[fill opacity=0.8,fill=gray!20,draw=none](-5.979,1.246)--(-6.051,1.25)--(-6.056,1.25)--(-6.101,1.242)--(-6.129,1.232)--(-6.136,1.219)--(-6.121,1.207)--(-6.086,1.197)--(-6.036,1.191)--(-6.004,1.19)--cycle;
\draw(-6.051,1.25)--(-6.056,1.25)--(-6.101,1.242)--(-6.129,1.232)--(-6.136,1.219)--(-6.121,1.207)--(-6.086,1.197)--(-6.036,1.191)--(-6.004,1.19);
\filldraw[fill opacity=0.8,fill=gray!20](-6.951,1.244)--(-6.922,1.275)--(-6.941,1.28)--(-6.988,1.253)--cycle;
\filldraw[fill opacity=0.8,fill=gray!20,draw=none](-6.729,.95)--(-6.725,.96)--(-6.731,.961)--cycle;
\draw(-6.725,.96)--(-6.731,.961);
\filldraw[fill opacity=0.5,fill=gray!20](-11.092,.364)--(-11.092,.411)--(-11.151,.896)--(-11.153,.865)--cycle;
\filldraw[fill opacity=0.8,fill=gray!20,draw=none](-7.082,1.01)--(-7.081,1.004)--(-7.08,1.002)--(-7.082,1.018)--cycle;
\draw(-7.081,1.004)--(-7.08,1.002)--(-7.082,1.018);
\filldraw[fill opacity=0.8,fill=gray!20,draw=none](-6.736,.942)--(-6.732,.949)--(-6.765,.942)--cycle;
\draw(-6.736,.942)--(-6.732,.949)--(-6.765,.942);
\filldraw[fill opacity=0.8,fill=gray!20,draw=none](-7.058,.997)--(-7.08,1.002)--(-7.074,.987)--cycle;
\draw(-7.058,.997)--(-7.08,1.002)--(-7.074,.987);
\filldraw[fill opacity=0.8,fill=gray!20,draw=none](-7.072,1.156)--(-7.074,1.13)--(-7.06,1.167)--(-7.067,1.175)--cycle;
\draw(-7.074,1.13)--(-7.06,1.167)--(-7.067,1.175);
\filldraw[fill opacity=0.8,fill=gray!20,draw=none](-6.135,.909)--(-6.129,.888)--(-6.129,.874)--(-6.136,.862)--(-6.136,.905)--cycle;
\draw(-6.129,.888)--(-6.129,.874)--(-6.136,.862)--(-6.136,.905);
\filldraw[fill opacity=0.8,fill=gray!20,draw=none](-8.511,.857)--(-8.474,.856)--(-8.5,.85)--cycle;
\draw(-8.474,.856)--(-8.5,.85);
\filldraw[fill opacity=0.8,fill=gray!20,draw=none](-7.082,1.01)--(-7.082,1.018)--(-7.087,1.057)--cycle;
\draw(-7.082,1.018)--(-7.087,1.057);
\filldraw[fill opacity=0.5,fill=gray!20](-10.661,.725)--(-11.123,.927)--(-11.066,1.402)--(-10.605,1.201)--cycle;
\filldraw[fill opacity=0.5,fill=gray!20](-11.151,.896)--(-11.123,.927)--(-11.066,1.402)--(-11.092,1.394)--cycle;
\filldraw[fill opacity=0.8,fill=gray!20](-6.799,1.251)--(-6.843,1.279)--(-6.865,1.274)--(-6.841,1.243)--cycle;
\filldraw[fill opacity=0.8,fill=gray!20,draw=none](-6.741,.961)--(-6.739,.947)--(-6.732,.949)--(-6.73,.953)--cycle;
\draw(-6.739,.947)--(-6.732,.949)--(-6.73,.953);
\filldraw[fill opacity=0.8,fill=gray!20,draw=none](-7.007,.984)--(-7.011,1.039)--(-7.061,1.051)--(-7.082,1.018)--(-7.08,1.002)--cycle;
\draw(-7.082,1.018)--(-7.08,1.002)--(-7.007,.984)--(-7.011,1.039)--(-7.061,1.051);
\filldraw[fill opacity=0.8,fill=gray!20,draw=none](-6.73,.953)--(-6.731,.961)--(-6.741,.961)--cycle;
\draw(-6.731,.961)--(-6.741,.961);
\filldraw[fill opacity=0.8,fill=gray!20,draw=none](-7.082,1.116)--(-7.08,1.113)--(-7.074,1.13)--(-7.072,1.156)--cycle;
\draw(-7.082,1.116)--(-7.08,1.113)--(-7.074,1.13);
\filldraw[fill opacity=0.5,fill=gray!20](-10.217,-.271)--(-10.044,-.347)--(-10.279,-.039)--(-10.452,.037)--cycle;
\filldraw[fill opacity=0.8,fill=gray!20,draw=none](-6.739,.947)--(-6.746,.991)--(-6.795,.982)--(-6.805,.934)--cycle;
\draw(-6.746,.991)--(-6.795,.982)--(-6.805,.934)--(-6.739,.947);
\filldraw[fill opacity=0.8,fill=gray!20,draw=none](-7.061,1.051)--(-7.087,1.057)--(-7.082,1.018)--cycle;
\draw(-7.061,1.051)--(-7.087,1.057)--(-7.082,1.018);
\filldraw[fill opacity=0.8,fill=gray!20](-6.976,1.202)--(-6.951,1.244)--(-6.988,1.253)--(-7.029,1.215)--cycle;
\filldraw[fill opacity=0.8,fill=gray!20,draw=none](-6.119,1.178)--(-6.113,1.178)--(-6.101,1.196)--cycle;
\filldraw[fill opacity=0.8,fill=gray!20,draw=none](-6.101,1.196)--(-6.101,1.178)--(-6.113,1.178)--(-6.119,1.183)--cycle;
\draw(-6.101,1.196)--(-6.101,1.178);
\filldraw[fill opacity=0.8,fill=gray!20,draw=none](-8.204,2.959)--(-8.195,2.981)--(-8.187,2.993)--(-8.165,3.012)--(-8.112,3.022)--(-8.122,2.974)--cycle;
\draw(-8.165,3.012)--(-8.112,3.022)--(-8.122,2.974)--(-8.204,2.959)--(-8.195,2.981);
\filldraw[fill opacity=0.8,fill=gray!20](-8.106,2.958)--(-8.098,2.998)--(-8.017,3.001)--(-8.016,2.962)--cycle;
\filldraw[fill opacity=0.8,fill=gray!20,draw=none](-8.165,3.012)--(-8.147,3.024)--(-8.104,3.04)--(-8.112,3.022)--cycle;
\draw(-8.104,3.04)--(-8.112,3.022)--(-8.165,3.012);
\filldraw[fill opacity=0.8,fill=gray!20,draw=none](-8.159,2.986)--(-8.15,2.999)--(-8.127,3.015)--(-8.094,3.027)--(-8.085,3.029)--(-8.098,2.998)--cycle;
\draw(-8.094,3.027)--(-8.085,3.029)--(-8.098,2.998)--(-8.159,2.986)--(-8.15,2.999);
\filldraw[fill opacity=0.8,fill=gray!20,draw=none](-8.094,3.027)--(-8.083,3.031)--(-8.085,3.029)--cycle;
\draw(-8.083,3.031)--(-8.085,3.029)--(-8.094,3.027);
\filldraw[fill opacity=0.8,fill=gray!20,draw=none](-8.202,3.048)--(-8.072,2.991)--(-8.027,3.006)--(-8.177,3.072)--cycle;
\draw(-8.202,3.048)--(-8.072,2.991)--(-8.027,3.006)--(-8.177,3.072);
\filldraw[fill opacity=0.8,fill=gray!20,draw=none](-5.94,1.236)--(-5.951,1.244)--(-5.979,1.246)--(-5.985,1.232)--cycle;
\filldraw[fill opacity=0.8,fill=gray!20](-6.9,.893)--(-6.902,.93)--(-6.995,.937)--(-6.976,.899)--cycle;
\filldraw[fill opacity=0.8,fill=gray!20](-6.897,1.24)--(-6.894,1.273)--(-6.922,1.275)--(-6.951,1.244)--cycle;
\filldraw[fill opacity=0.8,fill=gray!20,draw=none](-7.082,1.108)--(-7.081,1.115)--(-7.082,1.116)--cycle;
\draw(-7.081,1.115)--(-7.082,1.116);
\filldraw[fill opacity=0.8,fill=gray!20,draw=none](-6.043,1.174)--(-6.045,1.175)--(-6.062,1.176)--cycle;
\draw(-6.045,1.175)--(-6.062,1.176);
\filldraw[fill opacity=0.8,fill=gray!20,draw=none](-6.113,1.178)--(-6.049,1.173)--(-6.101,1.196)--cycle;
\draw(-6.049,1.173)--(-6.101,1.196);
\filldraw[fill opacity=0.8,fill=gray!20,draw=none](-7.075,1.082)--(-7.082,1.097)--(-7.087,1.057)--cycle;
\draw(-7.082,1.097)--(-7.087,1.057);
\filldraw[fill opacity=0.8,fill=gray!20,draw=none](-7.061,1.051)--(-7.075,1.082)--(-7.087,1.057)--cycle;
\draw(-7.087,1.057)--(-7.061,1.051);
\filldraw[fill opacity=0.8,fill=gray!20,draw=none](-7.087,1.057)--(-7.082,1.097)--(-7.082,1.108)--cycle;
\draw(-7.087,1.057)--(-7.082,1.097);
\filldraw[fill opacity=0.8,fill=gray!20](-6.841,1.243)--(-6.865,1.274)--(-6.894,1.273)--(-6.897,1.24)--cycle;
\filldraw[fill opacity=0.8,fill=gray!20,draw=none](-7.831,3.299)--(-7.893,3.322)--(-7.756,3.688)--cycle;
\draw(-7.893,3.322)--(-7.756,3.688)--(-7.831,3.299);
\filldraw[fill opacity=0.8,fill=gray!20,draw=none](-7.929,3.336)--(-7.851,3.723)--(-7.756,3.688)--(-7.893,3.322)--cycle;
\draw(-7.929,3.336)--(-7.851,3.723)--(-7.756,3.688)--(-7.893,3.322);
\filldraw[fill opacity=0.8,fill=gray!20,draw=none](-7.082,1.108)--(-7.082,1.097)--(-7.08,1.113)--(-7.081,1.115)--cycle;
\draw(-7.082,1.097)--(-7.08,1.113)--(-7.081,1.115);
\filldraw[fill opacity=0.8,fill=gray!20,draw=none](-8.25,3.02)--(-8.187,2.993)--(-8.149,3.025)--(-8.228,3.06)--cycle;
\draw(-8.25,3.02)--(-8.187,2.993);
\draw(-8.149,3.025)--(-8.228,3.06);
\filldraw[fill opacity=0.8,fill=gray!20,draw=none](-5.901,1.241)--(-5.951,1.244)--(-5.94,1.236)--cycle;
\filldraw[fill opacity=0.8,fill=gray!20,draw=none](-6.815,1.201)--(-6.761,1.211)--(-6.799,1.251)--(-6.841,1.243)--(-6.821,1.201)--cycle;
\draw(-6.815,1.201)--(-6.761,1.211)--(-6.799,1.251)--(-6.841,1.243)--(-6.821,1.201);
\filldraw[fill opacity=0.8,fill=gray!20,draw=none](-7.831,3.299)--(-7.756,3.688)--(-7.656,3.65)--(-7.793,3.285)--cycle;
\draw(-7.831,3.299)--(-7.756,3.688)--(-7.656,3.65)--(-7.793,3.285);
\filldraw[fill opacity=0.8,fill=gray!20](-6.995,1.151)--(-6.976,1.202)--(-7.029,1.215)--(-7.06,1.167)--cycle;
\filldraw[fill opacity=0.8,fill=gray!20](-6.821,.897)--(-6.805,.934)--(-6.902,.93)--(-6.9,.893)--cycle;
\filldraw[fill opacity=0.8,fill=gray!20,draw=none](-7.075,1.082)--(-7.063,1.109)--(-7.08,1.113)--(-7.082,1.097)--cycle;
\draw(-7.063,1.109)--(-7.08,1.113)--(-7.082,1.097);
\filldraw[fill opacity=0.8,fill=gray!20,draw=none](-6.116,1.118)--(-6.121,1.123)--(-6.121,1.102)--(-6.107,1.101)--cycle;
\draw(-6.121,1.123)--(-6.121,1.102);
\filldraw[fill opacity=0.8,fill=gray!20,draw=none](-6.725,1.142)--(-6.007,1.114)--(-5.999,1.151)--(-6.733,1.18)--cycle;
\draw(-6.725,1.142)--(-6.007,1.114)--(-5.999,1.151)--(-6.733,1.18);
\filldraw[fill opacity=0.8,fill=gray!20,draw=none](-6.729,1.164)--(-6.733,1.18)--(-6.736,1.171)--(-6.732,1.163)--cycle;
\draw(-6.736,1.171)--(-6.732,1.163)--(-6.729,1.164);
\filldraw[fill opacity=0.8,fill=gray!20,draw=none](-7.738,3.263)--(-7.793,3.285)--(-7.656,3.65)--cycle;
\draw(-7.793,3.285)--(-7.656,3.65)--(-7.738,3.263);
\filldraw[fill opacity=0.8,fill=gray!20,draw=none](-7.058,1.108)--(-7.074,1.13)--(-7.08,1.113)--cycle;
\draw(-7.074,1.13)--(-7.08,1.113)--(-7.058,1.108);
\filldraw[fill opacity=0.8,fill=gray!20,draw=none](-7.058,1.108)--(-7.007,1.095)--(-6.995,1.151)--(-7.06,1.167)--(-7.074,1.13)--cycle;
\draw(-7.058,1.108)--(-7.007,1.095)--(-6.995,1.151)--(-7.06,1.167)--(-7.074,1.13);
\filldraw[fill opacity=0.8,fill=gray!20,draw=none](-5.883,1.242)--(-5.872,1.239)--(-5.86,1.212)--cycle;
\draw(-5.883,1.242)--(-5.872,1.239);
\filldraw[fill opacity=0.8,fill=gray!20,draw=none](-5.901,1.241)--(-5.869,1.238)--(-5.883,1.242)--cycle;
\draw(-5.869,1.238)--(-5.883,1.242);
\filldraw[fill opacity=0.8,fill=gray!20,draw=none](-6.136,.986)--(-6.121,.99)--(-6.121,1.029)--cycle;
\draw(-6.121,.99)--(-6.121,1.029);
\filldraw[fill opacity=0.8,fill=gray!20,draw=none](-6.09,.971)--(-6.121,1.029)--(-6.121,.99)--(-6.092,.968)--cycle;
\draw(-6.121,1.029)--(-6.121,.99);
\filldraw[fill opacity=0.8,fill=gray!20,draw=none](-6.912,1.003)--(-6.093,.971)--(-6.129,1.019)--(-6.914,1.05)--cycle;
\draw(-6.129,1.019)--(-6.914,1.05)--(-6.912,1.003)--(-6.093,.971);
\filldraw[fill opacity=0.8,fill=gray!20,draw=none](-8.491,.841)--(-8.439,.839)--(-8.479,.83)--cycle;
\draw(-8.439,.839)--(-8.479,.83);
\filldraw[fill opacity=0.8,fill=gray!20,draw=none](-7.929,3.336)--(-7.989,3.357)--(-7.851,3.723)--cycle;
\draw(-7.989,3.357)--(-7.851,3.723)--(-7.929,3.336);
\filldraw[fill opacity=0.8,fill=gray!20,draw=none](-7.738,3.263)--(-7.656,3.65)--(-7.567,3.615)--(-7.705,3.25)--cycle;
\draw(-7.738,3.263)--(-7.656,3.65)--(-7.567,3.615)--(-7.705,3.25);
\filldraw[fill opacity=0.8,fill=gray!20,draw=none](-8.018,3.368)--(-7.928,3.75)--(-7.851,3.723)--(-7.989,3.357)--cycle;
\draw(-8.018,3.368)--(-7.928,3.75)--(-7.851,3.723)--(-7.989,3.357);
\filldraw[fill opacity=0.8,fill=gray!20,draw=none](-7.665,3.233)--(-7.705,3.25)--(-7.567,3.615)--cycle;
\draw(-7.705,3.25)--(-7.567,3.615)--(-7.665,3.233);
\filldraw[fill opacity=0.8,fill=gray!20,draw=none](-8.018,3.368)--(-8.066,3.384)--(-7.928,3.75)--cycle;
\draw(-8.066,3.384)--(-7.928,3.75)--(-8.018,3.368);
\filldraw[fill opacity=0.8,fill=gray!20,draw=none](-7.665,3.233)--(-7.567,3.615)--(-7.503,3.588)--(-7.641,3.223)--cycle;
\draw(-7.665,3.233)--(-7.567,3.615)--(-7.503,3.588)--(-7.641,3.223);
\filldraw[fill opacity=0.8,fill=gray!20,draw=none](-8.084,3.39)--(-7.975,3.765)--(-7.928,3.75)--(-8.066,3.384)--cycle;
\draw(-8.084,3.39)--(-7.975,3.765)--(-7.928,3.75)--(-8.066,3.384);
\filldraw[fill opacity=0.8,fill=gray!20,draw=none](-8.084,3.39)--(-8.113,3.399)--(-7.975,3.765)--cycle;
\draw(-8.113,3.399)--(-7.975,3.765)--(-8.084,3.39);
\filldraw[fill opacity=0.8,fill=gray!20,draw=none](-7.622,3.214)--(-7.641,3.223)--(-7.503,3.588)--cycle;
\draw(-7.641,3.223)--(-7.503,3.588)--(-7.622,3.214);
\filldraw[fill opacity=0.8,fill=gray!20,draw=none](-7.622,3.214)--(-7.503,3.588)--(-7.473,3.574)--(-7.611,3.209)--cycle;
\draw(-7.622,3.214)--(-7.503,3.588)--(-7.473,3.574)--(-7.611,3.209);
\filldraw[fill opacity=0.8,fill=gray!20,draw=none](-8.116,3.399)--(-7.984,3.765)--(-7.975,3.765)--(-8.113,3.399)--cycle;
\draw(-8.116,3.399)--(-7.984,3.765)--(-7.975,3.765)--(-8.113,3.399);
\filldraw[fill opacity=0.8,fill=gray!20](-7.606,3.616)--(-7.529,3.589)--(-7.482,3.574)--(-7.473,3.574)--(-7.503,3.588)--(-7.567,3.615)--(-7.656,3.65)--(-7.756,3.688)--(-7.851,3.723)--(-7.928,3.75)--(-7.975,3.765)--(-7.984,3.765)--(-7.954,3.75)--(-7.89,3.724)--(-7.801,3.689)--(-7.701,3.651)--cycle;
\filldraw[fill opacity=0.8,fill=gray!20,draw=none](-7.061,1.051)--(-7.011,1.039)--(-7.007,1.095)--(-7.063,1.109)--(-7.075,1.082)--cycle;
\draw(-7.061,1.051)--(-7.011,1.039)--(-7.007,1.095)--(-7.063,1.109);
\filldraw[fill opacity=0.8,fill=gray!20,draw=none](-5.842,1.028)--(-5.844,.99)--(-5.856,.996)--cycle;
\draw(-5.844,.99)--(-5.856,.996);
\filldraw[fill opacity=0.8,fill=gray!20,draw=none](-5.844,.986)--(-5.844,.93)--(-5.846,.933)--cycle;
\draw(-5.844,.986)--(-5.844,.93);
\filldraw[fill opacity=0.8,fill=gray!20,draw=none](-5.849,1.012)--(-5.844,1.03)--(-5.844,.99)--cycle;
\draw(-5.844,1.03)--(-5.844,.99);
\filldraw[fill opacity=0.8,fill=gray!20,draw=none](-5.851,.932)--(-5.864,.937)--(-5.844,.99)--cycle;
\draw(-5.851,.932)--(-5.864,.937);
\filldraw[fill opacity=0.8,fill=gray!20,draw=none](-5.845,1.032)--(-5.85,1.038)--(-5.851,1.017)--(-5.851,.854)--(-5.844,.867)--(-5.844,.986)--cycle;
\draw(-5.851,1.017)--(-5.851,.854)--(-5.844,.867)--(-5.844,.986);
\filldraw[fill opacity=0.8,fill=gray!20,draw=none](-6.749,1.095)--(-6.75,1.044)--(-6.707,1.052)--(-6.713,1.109)--(-6.746,1.102)--cycle;
\draw(-6.75,1.044)--(-6.707,1.052)--(-6.713,1.109)--(-6.746,1.102);
\filldraw[fill opacity=0.8,fill=gray!20,draw=none](-6.746,1.102)--(-6.713,1.109)--(-6.732,1.163)--(-6.739,1.161)--cycle;
\draw(-6.746,1.102)--(-6.713,1.109)--(-6.732,1.163)--(-6.739,1.161);
\filldraw[fill opacity=0.8,fill=gray!20,draw=none](-6.144,1.111)--(-6.148,1.1)--(-6.136,1.095)--cycle;
\draw(-6.148,1.1)--(-6.136,1.095);
\filldraw[fill opacity=0.8,fill=gray!20,draw=none](-6.107,1.101)--(-6.121,1.102)--(-6.121,1.029)--(-6.09,1.069)--cycle;
\draw(-6.121,1.102)--(-6.121,1.029);
\filldraw[fill opacity=0.8,fill=gray!20,draw=none](-6.912,1.102)--(-6.09,1.069)--(-6.095,1.117)--(-6.907,1.149)--cycle;
\draw(-6.095,1.117)--(-6.907,1.149)--(-6.912,1.102)--(-6.09,1.069);
\filldraw[fill opacity=0.8,fill=gray!20,draw=none](-8.479,.83)--(-8.439,.839)--(-8.375,.849)--(-8.489,.823)--cycle;
\draw(-8.479,.83)--(-8.439,.839);
\draw(-8.375,.849)--(-8.489,.823);
\filldraw[fill opacity=0.8,fill=gray!20,draw=none](-6.113,1.178)--(-6.113,1.177)--(-6.113,1.178)--cycle;
\filldraw[fill opacity=0.8,fill=gray!20,draw=none](-5.874,1.239)--(-5.901,1.241)--(-5.94,1.236)--(-5.926,1.228)--cycle;
\filldraw[fill opacity=0.8,fill=gray!20,draw=none](-6.113,1.178)--(-6.129,1.178)--(-6.129,1.191)--cycle;
\draw(-6.129,1.178)--(-6.129,1.191);
\filldraw[fill opacity=0.8,fill=gray!20,draw=none](-6.754,1.2)--(-6.761,1.211)--(-6.815,1.201)--cycle;
\draw(-6.754,1.2)--(-6.761,1.211)--(-6.815,1.201);
\filldraw[fill opacity=0.8,fill=gray!20,draw=none](-6.732,1.163)--(-6.754,1.2)--(-6.815,1.201)--(-6.821,1.2)--(-6.805,1.149)--cycle;
\draw(-6.815,1.201)--(-6.821,1.2)--(-6.805,1.149)--(-6.732,1.163)--(-6.754,1.2);
\filldraw[fill opacity=0.8,fill=gray!20,draw=none](-5.86,1.212)--(-5.872,1.239)--(-5.859,1.236)--(-5.859,1.211)--cycle;
\draw(-5.872,1.239)--(-5.859,1.236)--(-5.859,1.211);
\filldraw[fill opacity=0.8,fill=gray!20,draw=none](-6.729,1.164)--(-6.732,1.163)--(-6.73,1.158)--cycle;
\draw(-6.729,1.164)--(-6.732,1.163)--(-6.73,1.158);
\filldraw[fill opacity=0.8,fill=gray!20,draw=none](-5.86,1.212)--(-5.859,1.211)--(-5.859,1.21)--cycle;
\draw(-5.859,1.211)--(-5.859,1.21);
\filldraw[fill opacity=0.8,fill=gray!20,draw=none](-5.85,1.182)--(-5.847,1.149)--(-5.859,1.21)--(-5.859,1.211)--cycle;
\draw(-5.859,1.21)--(-5.859,1.211);
\filldraw[fill opacity=0.8,fill=gray!20,draw=none](-5.855,1.233)--(-5.85,1.182)--(-5.859,1.211)--(-5.859,1.236)--cycle;
\draw(-5.859,1.211)--(-5.859,1.236)--(-5.855,1.233);
\filldraw[fill opacity=0.8,fill=gray!20,draw=none](-5.85,1.182)--(-5.855,1.233)--(-5.844,1.224)--(-5.844,1.162)--cycle;
\draw(-5.855,1.233)--(-5.844,1.224)--(-5.844,1.162);
\filldraw[fill opacity=0.8,fill=gray!20,draw=none](-5.874,1.239)--(-5.926,1.228)--(-5.881,1.2)--(-5.879,1.2)--(-5.851,1.211)--(-5.844,1.224)--(-5.859,1.236)--(-5.869,1.238)--cycle;
\draw(-5.881,1.2)--(-5.879,1.2)--(-5.851,1.211)--(-5.844,1.224)--(-5.859,1.236)--(-5.869,1.238);
\filldraw[fill opacity=0.5,fill=gray!20](-10.211,2.668)--(-10.254,2.662)--(-9.82,2.906)--(-9.769,2.916)--cycle;
\filldraw[fill opacity=0.8,fill=gray!20,draw=none](-6.741,1.143)--(-6.725,1.142)--(-6.729,1.16)--cycle;
\draw(-6.741,1.143)--(-6.725,1.142);
\filldraw[fill opacity=0.5,fill=gray!20](-10.217,2.007)--(-10.044,1.932)--(-9.738,2.221)--(-9.911,2.297)--cycle;
\filldraw[fill opacity=0.5,fill=gray!20](-10.452,1.648)--(-10.279,1.573)--(-10.044,1.932)--(-10.217,2.007)--cycle;
\filldraw[fill opacity=0.5,fill=gray!20](-10.439,1.656)--(-10.9,1.857)--(-10.636,2.262)--(-10.175,2.06)--cycle;
\filldraw[fill opacity=0.5,fill=gray!20](-10.918,1.871)--(-10.9,1.857)--(-10.636,2.262)--(-10.642,2.294)--cycle;
\filldraw[fill opacity=0.5,fill=gray!20](-9.831,-.754)--(-9.92,-.811)--(-10.28,-.548)--(-10.175,-.504)--cycle;
\filldraw[fill opacity=0.8,fill=gray!20,draw=none](-6.914,1.05)--(-6.096,1.018)--(-6.114,1.07)--(-6.912,1.102)--cycle;
\draw(-6.114,1.07)--(-6.912,1.102)--(-6.914,1.05)--(-6.096,1.018);
\filldraw[fill opacity=0.8,fill=gray!20,draw=none](-6.749,.991)--(-6.746,.991)--(-6.749,.997)--cycle;
\draw(-6.749,.991)--(-6.746,.991);
\filldraw[fill opacity=0.8,fill=gray!20,draw=none](-6.135,.909)--(-6.136,.905)--(-6.136,.913)--cycle;
\draw(-6.136,.905)--(-6.136,.913);
\filldraw[fill opacity=0.8,fill=gray!20,draw=none](-6.136,.913)--(-6.136,.862)--(-6.121,.85)--(-6.121,.934)--cycle;
\draw(-6.136,.913)--(-6.136,.862)--(-6.121,.85)--(-6.121,.934);
\filldraw[fill opacity=0.8,fill=gray!20,draw=none](-8.29,2.928)--(-8.269,2.919)--(-8.276,2.963)--cycle;
\draw(-8.29,2.928)--(-8.269,2.919);
\filldraw[fill opacity=0.8,fill=gray!20,draw=none](-8.29,2.928)--(-8.246,2.909)--(-8.217,2.951)--(-8.271,2.974)--cycle;
\draw(-8.29,2.928)--(-8.246,2.909);
\draw(-8.217,2.951)--(-8.271,2.974);
\filldraw[fill opacity=0.8,fill=gray!20](-6.902,.93)--(-6.903,.977)--(-7.007,.984)--(-6.995,.937)--cycle;
\filldraw[fill opacity=0.8,fill=gray!20](-6.9,1.196)--(-6.897,1.24)--(-6.951,1.244)--(-6.976,1.202)--cycle;
\filldraw[fill opacity=0.8,fill=gray!20,draw=none](-6.749,.991)--(-6.749,.997)--(-6.773,1.039)--(-6.792,1.036)--(-6.795,.982)--cycle;
\draw(-6.773,1.039)--(-6.792,1.036)--(-6.795,.982)--(-6.749,.991);
\filldraw[fill opacity=0.8,fill=gray!20,draw=none](-6.749,1.102)--(-6.746,1.102)--(-6.739,1.161)--(-6.765,1.156)--cycle;
\draw(-6.749,1.102)--(-6.746,1.102);
\draw(-6.739,1.161)--(-6.765,1.156);
\filldraw[fill opacity=0.8,fill=gray!20,draw=none](-5.925,1.193)--(-5.881,1.2)--(-5.94,1.236)--(-5.985,1.232)--(-6.004,1.19)--(-5.98,1.19)--cycle;
\draw(-6.004,1.19)--(-5.98,1.19)--(-5.925,1.193)--(-5.881,1.2);
\filldraw[fill opacity=0.8,fill=gray!20,draw=none](-5.842,1.028)--(-5.836,1.043)--(-5.844,.99)--cycle;
\filldraw[fill opacity=0.8,fill=gray!20,draw=none](-6.749,.997)--(-6.75,1.044)--(-6.773,1.039)--cycle;
\draw(-6.75,1.044)--(-6.773,1.039);
\filldraw[fill opacity=0.8,fill=gray!20,draw=none](-5.845,1.032)--(-5.844,.986)--(-5.844,1.03)--cycle;
\draw(-5.844,.986)--(-5.844,1.03);
\filldraw[fill opacity=0.8,fill=gray!20,draw=none](-8.015,2.842)--(-8.014,2.868)--(-7.999,2.867)--cycle;
\draw(-8.015,2.842)--(-8.014,2.868)--(-7.999,2.867);
\filldraw[fill opacity=0.8,fill=gray!20,draw=none](-7.994,2.873)--(-7.999,2.867)--(-8.014,2.868)--(-8.014,2.925)--(-7.984,2.923)--cycle;
\draw(-7.999,2.867)--(-8.014,2.868)--(-8.014,2.925)--(-7.984,2.923);
\filldraw[fill opacity=0.8,fill=gray!20,draw=none](-6.821,1.201)--(-6.841,1.243)--(-6.897,1.24)--(-6.899,1.203)--cycle;
\draw(-6.821,1.201)--(-6.841,1.243)--(-6.897,1.24)--(-6.899,1.203);
\filldraw[fill opacity=0.8,fill=gray!20,draw=none](-5.816,.916)--(-5.851,.932)--(-5.844,.99)--(-5.787,.966)--cycle;
\draw(-5.844,.99)--(-5.787,.966)--(-5.816,.916)--(-5.851,.932);
\filldraw[fill opacity=0.8,fill=gray!20,draw=none](-5.824,.982)--(-5.844,.99)--(-5.836,1.043)--(-5.835,1.046)--(-5.826,1.042)--cycle;
\draw(-5.824,.982)--(-5.844,.99);
\draw(-5.835,1.046)--(-5.826,1.042);
\filldraw[fill opacity=0.8,fill=gray!20,draw=none](-8.489,.823)--(-8.375,.849)--(-8.352,.869)--(-8.528,.83)--cycle;
\draw(-8.489,.823)--(-8.375,.849);
\draw(-8.352,.869)--(-8.528,.83);
\filldraw[fill opacity=0.8,fill=gray!20,draw=none](-7.994,2.972)--(-7.984,2.923)--(-8.014,2.925)--(-8.014,2.979)--(-7.999,2.978)--cycle;
\draw(-7.984,2.923)--(-8.014,2.925)--(-8.014,2.979)--(-7.999,2.978);
\filldraw[fill opacity=0.8,fill=gray!20](-6.805,.934)--(-6.795,.982)--(-6.903,.977)--(-6.902,.93)--cycle;
\filldraw[fill opacity=0.8,fill=gray!20,draw=none](-5.836,1.043)--(-5.835,1.047)--(-5.835,1.046)--cycle;
\draw(-5.835,1.047)--(-5.835,1.046);
\filldraw[fill opacity=0.8,fill=gray!20,draw=none](-5.839,1.089)--(-5.831,1.061)--(-5.835,1.046)--(-5.835,1.047)--cycle;
\draw(-5.835,1.046)--(-5.835,1.047);
\filldraw[fill opacity=0.8,fill=gray!20,draw=none](-6.129,1.186)--(-6.129,1.178)--(-6.132,1.177)--cycle;
\draw(-6.129,1.186)--(-6.129,1.178);
\filldraw[fill opacity=0.8,fill=gray!20,draw=none](-8.276,2.963)--(-8.271,2.974)--(-8.278,2.977)--cycle;
\draw(-8.271,2.974)--(-8.278,2.977);
\filldraw[fill opacity=0.5,fill=gray!20](-10.829,-.174)--(-10.882,-.139)--(-11.065,.324)--(-11.014,.291)--cycle;
\filldraw[fill opacity=0.8,fill=gray!20,draw=none](-5.831,1.061)--(-5.826,1.042)--(-5.835,1.046)--cycle;
\draw(-5.826,1.042)--(-5.835,1.046);
\filldraw[fill opacity=0.8,fill=gray!20](-8.174,2.945)--(-8.159,2.986)--(-8.098,2.998)--(-8.106,2.958)--cycle;
\filldraw[fill opacity=0.8,fill=gray!20,draw=none](-7.971,2.863)--(-7.997,2.821)--(-8.017,2.823)--(-8.016,2.87)--(-7.971,2.867)--cycle;
\draw(-7.997,2.821)--(-8.017,2.823)--(-8.016,2.87)--(-7.971,2.867);
\filldraw[fill opacity=0.8,fill=gray!20,draw=none](-7.971,2.867)--(-8.016,2.87)--(-8.016,2.917)--(-7.962,2.914)--cycle;
\draw(-7.971,2.867)--(-8.016,2.87)--(-8.016,2.917)--(-7.962,2.914);
\filldraw[fill opacity=0.8,fill=gray!20,draw=none](-7.923,2.832)--(-7.926,2.834)--(-7.952,2.845)--(-7.994,2.863)--(-8.041,2.884)--(-8.086,2.903)--(-8.122,2.919)--(-8.143,2.928)--(-8.147,2.93)--(-8.135,2.924)--(-8.131,2.923)--cycle;
\draw(-7.926,2.834)--(-7.952,2.845)--(-7.994,2.863)--(-8.041,2.884)--(-8.086,2.903)--(-8.122,2.919)--(-8.143,2.928)--(-8.147,2.93)--(-8.135,2.924);
\filldraw[fill opacity=0.8,fill=gray!20,draw=none](-8.22,2.898)--(-8.211,2.948)--(-8.209,2.955)--(-8.204,2.959)--(-8.21,2.904)--cycle;
\draw(-8.209,2.955)--(-8.204,2.959)--(-8.21,2.904)--(-8.22,2.898);
\filldraw[fill opacity=0.8,fill=gray!20,draw=none](-8.209,2.955)--(-8.195,2.981)--(-8.204,2.959)--cycle;
\draw(-8.195,2.981)--(-8.204,2.959)--(-8.209,2.955);
\filldraw[fill opacity=0.8,fill=gray!20,draw=none](-8.198,2.888)--(-8.19,2.935)--(-8.174,2.945)--(-8.18,2.9)--cycle;
\draw(-8.19,2.935)--(-8.174,2.945)--(-8.18,2.9)--(-8.198,2.888);
\filldraw[fill opacity=0.8,fill=gray!20,draw=none](-8.19,2.935)--(-8.189,2.938)--(-8.166,2.982)--(-8.159,2.986)--(-8.174,2.945)--cycle;
\draw(-8.166,2.982)--(-8.159,2.986)--(-8.174,2.945)--(-8.19,2.935);
\filldraw[fill opacity=0.8,fill=gray!20,draw=none](-8.166,2.982)--(-8.165,2.983)--(-8.15,2.999)--(-8.159,2.986)--cycle;
\draw(-8.15,2.999)--(-8.159,2.986)--(-8.166,2.982);
\filldraw[fill opacity=0.8,fill=gray!20,draw=none](-8.17,2.876)--(-8.165,2.889)--(-8.147,2.93)--(-8.143,2.928)--(-8.179,2.845)--cycle;
\draw(-8.165,2.889)--(-8.147,2.93)--(-8.143,2.928)--(-8.179,2.845);
\filldraw[fill opacity=0.8,fill=gray!20,draw=none](-8.137,2.915)--(-8.135,2.924)--(-8.147,2.93)--(-8.165,2.889)--cycle;
\draw(-8.135,2.924)--(-8.147,2.93)--(-8.165,2.889);
\filldraw[fill opacity=0.8,fill=gray!20,draw=none](-8.271,2.974)--(-8.136,2.915)--(-8.11,2.959)--(-8.25,3.02)--cycle;
\draw(-8.271,2.974)--(-8.136,2.915)--(-8.11,2.959)--(-8.25,3.02);
\filldraw[fill opacity=0.8,fill=gray!20,draw=none](-8.271,2.974)--(-8.136,2.915)--(-8.11,2.959)--(-8.25,3.02)--cycle;
\draw(-8.271,2.974)--(-8.136,2.915)--(-8.11,2.959)--(-8.25,3.02);
\filldraw[fill opacity=0.8,fill=gray!20,draw=none](-7.703,4.168)--(-7.699,4.132)--(-7.73,4.126)--(-7.717,4.182)--(-7.715,4.182)--cycle;
\draw(-7.699,4.132)--(-7.73,4.126);
\draw(-7.717,4.182)--(-7.715,4.182);
\filldraw[fill opacity=0.8,fill=gray!20,draw=none](-7.703,4.168)--(-7.715,4.182)--(-7.705,4.184)--cycle;
\draw(-7.715,4.182)--(-7.705,4.184);
\filldraw[fill opacity=0.8,fill=gray!20,draw=none](-7.706,4.193)--(-7.704,4.177)--(-7.694,4.062)--(-7.699,4.101)--(-7.709,4.213)--cycle;
\draw(-7.704,4.177)--(-7.694,4.062)--(-7.699,4.101)--(-7.709,4.213);
\filldraw[fill opacity=0.8,fill=gray!20,draw=none](-7.719,4.321)--(-7.699,4.101)--(-7.683,4.139)--(-7.699,4.311)--cycle;
\draw(-7.719,4.321)--(-7.699,4.101)--(-7.683,4.139)--(-7.699,4.311);
\filldraw[fill opacity=0.8,fill=gray!20,draw=none](-8.276,2.963)--(-8.269,2.919)--(-8.246,2.909)--(-8.217,2.951)--(-8.271,2.974)--cycle;
\draw(-8.269,2.919)--(-8.246,2.909);
\draw(-8.217,2.951)--(-8.271,2.974);
\filldraw[fill opacity=0.8,fill=gray!20,draw=none](-8.228,3.011)--(-8.187,2.993)--(-8.149,3.025)--(-8.202,3.048)--cycle;
\draw(-8.228,3.011)--(-8.187,2.993);
\draw(-8.149,3.025)--(-8.202,3.048);
\filldraw[fill opacity=0.5,fill=gray!20](-10.65,.821)--(-10.477,.745)--(-10.426,1.168)--(-10.599,1.243)--cycle;
\filldraw[fill opacity=0.5,fill=gray!20](-10.599,1.243)--(-10.426,1.168)--(-10.279,1.573)--(-10.452,1.648)--cycle;
\filldraw[fill opacity=0.5,fill=gray!20](-10.605,1.201)--(-11.066,1.402)--(-10.9,1.857)--(-10.439,1.656)--cycle;
\filldraw[fill opacity=0.5,fill=gray!20](-11.092,1.394)--(-11.066,1.402)--(-10.9,1.857)--(-10.918,1.871)--cycle;
\filldraw[fill opacity=0.8,fill=gray!20,draw=none](-6.749,1.095)--(-6.746,1.102)--(-6.749,1.102)--cycle;
\draw(-6.746,1.102)--(-6.749,1.102);
\filldraw[fill opacity=0.5,fill=gray!20](-9.05,2.827)--(-9.106,2.915)--(-8.641,2.899)--(-8.599,2.811)--cycle;
\filldraw[fill opacity=0.5,fill=gray!20](-10.626,2.31)--(-10.642,2.294)--(-10.281,2.635)--(-10.254,2.662)--cycle;
\filldraw[fill opacity=0.8,fill=gray!20,draw=none](-6.749,1.102)--(-6.765,1.156)--(-6.805,1.149)--(-6.795,1.093)--cycle;
\draw(-6.765,1.156)--(-6.805,1.149)--(-6.795,1.093)--(-6.749,1.102);
\filldraw[fill opacity=0.8,fill=gray!20,draw=none](-6.773,1.039)--(-6.75,1.044)--(-6.76,1.07)--cycle;
\draw(-6.773,1.039)--(-6.75,1.044);
\filldraw[fill opacity=0.8,fill=gray!20,draw=none](-6.749,1.095)--(-6.76,1.07)--(-6.75,1.044)--cycle;
\filldraw[fill opacity=0.5,fill=gray!20](-10.378,-.573)--(-10.464,-.577)--(-10.756,-.194)--(-10.664,-.198)--cycle;
\filldraw[fill opacity=0.8,fill=gray!20,draw=none](-6.76,1.07)--(-6.749,1.095)--(-6.749,1.102)--(-6.77,1.098)--cycle;
\draw(-6.749,1.102)--(-6.77,1.098);
\filldraw[fill opacity=0.8,fill=gray!20,draw=none](-5.845,1.032)--(-5.847,1.079)--(-5.85,1.038)--cycle;
\filldraw[fill opacity=0.8,fill=gray!20](-6.902,1.144)--(-6.9,1.196)--(-6.976,1.202)--(-6.995,1.151)--cycle;
\filldraw[fill opacity=0.8,fill=gray!20](-6.903,.977)--(-6.903,1.031)--(-7.011,1.039)--(-7.007,.984)--cycle;
\filldraw[fill opacity=0.8,fill=gray!20,draw=none](-5.777,1.021)--(-5.826,1.042)--(-5.831,1.061)--(-5.823,1.091)--(-5.787,1.075)--cycle;
\draw(-5.823,1.091)--(-5.787,1.075)--(-5.777,1.021)--(-5.826,1.042);
\filldraw[fill opacity=0.8,fill=gray!20,draw=none](-5.787,.966)--(-5.824,.982)--(-5.826,1.042)--(-5.777,1.021)--cycle;
\draw(-5.826,1.042)--(-5.777,1.021)--(-5.787,.966)--(-5.824,.982);
\filldraw[fill opacity=0.8,fill=gray!20,draw=none](-7.999,2.978)--(-8.014,2.979)--(-8.015,3.001)--cycle;
\draw(-7.999,2.978)--(-8.014,2.979)--(-8.015,3.001);
\filldraw[fill opacity=0.8,fill=gray!20,draw=none](-5.831,1.061)--(-5.839,1.089)--(-5.839,1.098)--(-5.823,1.091)--cycle;
\draw(-5.839,1.098)--(-5.823,1.091);
\filldraw[fill opacity=0.8,fill=gray!20,draw=none](-6.132,1.177)--(-6.136,1.162)--(-6.136,1.176)--cycle;
\draw(-6.136,1.162)--(-6.136,1.176);
\filldraw[fill opacity=0.8,fill=gray!20,draw=none](-6.815,1.201)--(-6.821,1.201)--(-6.821,1.2)--cycle;
\draw(-6.821,1.201)--(-6.821,1.2)--(-6.815,1.201);
\filldraw[fill opacity=0.8,fill=gray!20,draw=none](-5.847,1.112)--(-5.845,1.104)--(-5.862,1.131)--cycle;
\filldraw[fill opacity=0.8,fill=gray!20,draw=none](-5.862,1.131)--(-5.873,1.143)--(-5.869,1.142)--cycle;
\draw(-5.873,1.143)--(-5.869,1.142);
\filldraw[fill opacity=0.8,fill=gray!20,draw=none](-5.848,1.118)--(-5.847,1.112)--(-5.862,1.131)--(-5.869,1.142)--(-5.855,1.136)--cycle;
\draw(-5.869,1.142)--(-5.855,1.136);
\filldraw[fill opacity=0.8,fill=gray!20,draw=none](-5.879,1.148)--(-5.862,1.138)--(-5.869,1.142)--cycle;
\draw(-5.862,1.138)--(-5.869,1.142);
\filldraw[fill opacity=0.8,fill=gray!20,draw=none](-5.848,1.118)--(-5.855,1.136)--(-5.851,1.134)--cycle;
\draw(-5.855,1.136)--(-5.851,1.134);
\filldraw[fill opacity=0.8,fill=gray!20](-5.879,1.2)--(-5.879,.843)--(-5.851,.854)--(-5.851,1.211)--cycle;
\filldraw[fill opacity=0.8,fill=gray!20,draw=none](-6.773,1.039)--(-6.76,1.07)--(-6.77,1.098)--(-6.795,1.093)--(-6.792,1.036)--cycle;
\draw(-6.77,1.098)--(-6.795,1.093)--(-6.792,1.036)--(-6.773,1.039);
\filldraw[fill opacity=0.8,fill=gray!20,draw=none](-5.847,1.105)--(-5.847,1.079)--(-5.844,1.117)--(-5.844,1.133)--cycle;
\draw(-5.844,1.117)--(-5.844,1.133);
\filldraw[fill opacity=0.5,fill=gray!20](-9.642,2.861)--(-9.708,2.901)--(-9.231,3.021)--(-9.167,2.981)--cycle;
\filldraw[fill opacity=0.8,fill=gray!20,draw=none](-5.839,1.089)--(-5.841,1.099)--(-5.839,1.098)--cycle;
\draw(-5.841,1.099)--(-5.839,1.098);
\filldraw[fill opacity=0.8,fill=gray!20,draw=none](-5.847,1.112)--(-5.844,1.109)--(-5.839,1.098)--(-5.841,1.099)--(-5.845,1.104)--cycle;
\draw(-5.839,1.098)--(-5.841,1.099);
\filldraw[fill opacity=0.8,fill=gray!20,draw=none](-6.074,.916)--(-6.056,.915)--(-6.074,.928)--cycle;
\draw(-6.074,.916)--(-6.056,.915);
\filldraw[fill opacity=0.8,fill=gray!20,draw=none](-6.821,1.2)--(-6.821,1.201)--(-6.899,1.203)--(-6.9,1.196)--cycle;
\draw(-6.899,1.203)--(-6.9,1.196)--(-6.821,1.2)--(-6.821,1.201);
\filldraw[fill opacity=0.8,fill=gray!20,draw=none](-5.851,1.212)--(-5.847,1.105)--(-5.844,1.133)--(-5.844,1.224)--cycle;
\draw(-5.844,1.133)--(-5.844,1.224)--(-5.851,1.212);
\filldraw[fill opacity=0.8,fill=gray!20](-6.805,1.149)--(-6.821,1.2)--(-6.9,1.196)--(-6.902,1.144)--cycle;
\filldraw[fill opacity=0.8,fill=gray!20,draw=none](-5.997,1.189)--(-6.001,1.191)--(-5.987,1.184)--cycle;
\draw(-6.001,1.191)--(-5.987,1.184);
\filldraw[fill opacity=0.8,fill=gray!20,draw=none](-6.01,1.171)--(-5.985,1.182)--(-5.987,1.184)--(-6.001,1.191)--(-6.058,1.177)--(-6.049,1.173)--cycle;
\draw(-5.987,1.184)--(-6.001,1.191);
\draw(-6.058,1.177)--(-6.049,1.173);
\filldraw[fill opacity=0.8,fill=gray!20,draw=none](-5.847,1.105)--(-5.851,1.212)--(-5.851,1.211)--(-5.851,1.075)--cycle;
\draw(-5.851,1.212)--(-5.851,1.211)--(-5.851,1.075);
\filldraw[fill opacity=0.8,fill=gray!20,draw=none](-5.832,1.095)--(-5.839,1.098)--(-5.848,1.118)--(-5.851,1.134)--cycle;
\draw(-5.832,1.095)--(-5.839,1.098);
\filldraw[fill opacity=0.8,fill=gray!20](-6.795,.982)--(-6.792,1.036)--(-6.903,1.031)--(-6.903,.977)--cycle;
\filldraw[fill opacity=0.8,fill=gray!20,draw=none](-6.042,1.17)--(-6.049,1.173)--(-6.065,1.174)--(-6.056,1.153)--cycle;
\draw(-6.042,1.17)--(-6.049,1.173);
\filldraw[fill opacity=0.8,fill=gray!20,draw=none](-7.962,2.914)--(-8.016,2.917)--(-8.016,2.962)--(-7.971,2.959)--cycle;
\draw(-7.962,2.914)--(-8.016,2.917)--(-8.016,2.962)--(-7.971,2.959);
\filldraw[fill opacity=0.8,fill=gray!20](-8.109,2.913)--(-8.106,2.958)--(-8.016,2.962)--(-8.016,2.917)--cycle;
\filldraw[fill opacity=0.8,fill=gray!20](-6.903,1.088)--(-6.902,1.144)--(-6.995,1.151)--(-7.007,1.095)--cycle;
\filldraw[fill opacity=0.8,fill=gray!20](-6.903,1.031)--(-6.903,1.088)--(-7.007,1.095)--(-7.011,1.039)--cycle;
\filldraw[fill opacity=0.8,fill=gray!20,draw=none](-5.847,1.105)--(-5.851,1.075)--(-5.851,1.041)--(-5.85,1.038)--(-5.847,1.079)--cycle;
\draw(-5.851,1.075)--(-5.851,1.041);
\filldraw[fill opacity=0.8,fill=gray!20,draw=none](-8.528,.83)--(-8.352,.869)--(-8.372,.898)--(-8.59,.849)--cycle;
\draw(-8.528,.83)--(-8.352,.869);
\draw(-8.372,.898)--(-8.59,.849);
\filldraw[fill opacity=0.8,fill=gray!20,draw=none](-7.998,2.82)--(-8.018,2.806)--(-8.017,2.823)--(-7.997,2.821)--cycle;
\draw(-8.018,2.806)--(-8.017,2.823)--(-7.997,2.821);
\filldraw[fill opacity=0.8,fill=gray!20,draw=none](-5.862,.882)--(-5.879,.89)--(-5.851,.932)--cycle;
\draw(-5.862,.882)--(-5.879,.89);
\filldraw[fill opacity=0.8,fill=gray!20,draw=none](-5.858,.881)--(-5.862,.882)--(-5.851,.932)--(-5.816,.916)--cycle;
\draw(-5.851,.932)--(-5.816,.916)--(-5.858,.881)--(-5.862,.882);
\filldraw[fill opacity=0.8,fill=gray!20,draw=none](-7.591,4.101)--(-7.581,3.995)--(-7.63,4.005)--(-7.64,4.112)--cycle;
\draw(-7.591,4.101)--(-7.581,3.995)--(-7.63,4.005)--(-7.64,4.112);
\filldraw[fill opacity=0.8,fill=gray!20,draw=none](-6.094,.941)--(-6.084,.935)--(-6.074,.935)--(-6.09,.966)--cycle;
\draw(-6.084,.935)--(-6.074,.935);
\filldraw[fill opacity=0.8,fill=gray!20,draw=none](-6.094,.941)--(-6.09,.966)--(-6.093,.971)--(-6.147,.973)--cycle;
\draw(-6.093,.971)--(-6.147,.973);
\filldraw[fill opacity=0.8,fill=gray!20,draw=none](-6.116,.936)--(-6.121,.93)--(-6.121,.85)--(-6.086,.84)--(-6.086,.949)--cycle;
\draw(-6.121,.93)--(-6.121,.85)--(-6.086,.84)--(-6.086,.949);
\filldraw[fill opacity=0.8,fill=gray!20,draw=none](-5.823,1.091)--(-5.832,1.095)--(-5.851,1.134)--(-5.823,1.121)--cycle;
\draw(-5.823,1.091)--(-5.832,1.095);
\draw(-5.851,1.134)--(-5.823,1.121);
\filldraw[fill opacity=0.8,fill=gray!20,draw=none](-9.033,.731)--(-9.062,.78)--(-9.041,.784)--(-8.973,.743)--cycle;
\draw(-8.973,.743)--(-9.033,.731)--(-9.062,.78)--(-9.041,.784);
\filldraw[fill opacity=0.8,fill=gray!20,draw=none](-8.9,.702)--(-8.905,.702)--(-8.961,.719)--(-8.973,.743)--cycle;
\draw(-8.9,.702)--(-8.905,.702);
\draw(-8.961,.719)--(-8.973,.743);
\filldraw[fill opacity=0.8,fill=gray!20](-8.995,.692)--(-9.033,.731)--(-8.973,.743)--(-8.952,.7)--cycle;
\filldraw[fill opacity=0.8,fill=gray!20,draw=none](-5.848,1.118)--(-5.844,1.109)--(-5.847,1.112)--cycle;
\filldraw[fill opacity=0.8,fill=gray!20,draw=none](-8.157,2.745)--(-8.161,2.753)--(-8.124,2.773)--(-8.096,2.756)--cycle;
\draw(-8.096,2.756)--(-8.157,2.745)--(-8.161,2.753);
\filldraw[fill opacity=0.5,fill=gray!20](-9.555,2.497)--(-9.382,2.421)--(-9,2.518)--(-9.173,2.593)--cycle;
\filldraw[fill opacity=0.8,fill=gray!20](-6.795,1.093)--(-6.805,1.149)--(-6.902,1.144)--(-6.903,1.088)--cycle;
\filldraw[fill opacity=0.8,fill=gray!20,draw=none](-8.489,.823)--(-8.528,.83)--(-8.562,.822)--cycle;
\draw(-8.528,.83)--(-8.562,.822);
\filldraw[fill opacity=0.8,fill=gray!20](-6.792,1.036)--(-6.795,1.093)--(-6.903,1.088)--(-6.903,1.031)--cycle;
\filldraw[fill opacity=0.8,fill=gray!20](-7.697,3.963)--(-7.726,4.011)--(-7.653,4.026)--(-7.637,3.974)--cycle;
\filldraw[fill opacity=0.8,fill=gray!20,draw=none](-7.676,4.137)--(-7.679,4.136)--(-7.68,4.137)--cycle;
\draw(-7.676,4.137)--(-7.679,4.136);
\filldraw[fill opacity=0.8,fill=gray!20,draw=none](-7.666,4.133)--(-7.64,4.112)--(-7.63,4.005)--(-7.67,4.028)--(-7.68,4.138)--cycle;
\draw(-7.64,4.112)--(-7.63,4.005)--(-7.67,4.028)--(-7.68,4.138);
\filldraw[fill opacity=0.8,fill=gray!20,draw=none](-5.851,1.041)--(-5.851,1.017)--(-5.85,1.038)--cycle;
\draw(-5.851,1.041)--(-5.851,1.017);
\filldraw[fill opacity=0.8,fill=gray!20,draw=none](-7.971,2.962)--(-7.971,2.959)--(-8.016,2.962)--(-8.017,3.001)--(-7.997,3)--cycle;
\draw(-7.971,2.959)--(-8.016,2.962)--(-8.017,3.001)--(-7.997,3);
\filldraw[fill opacity=0.8,fill=gray!20,draw=none](-8.029,3.026)--(-8.044,3.062)--(-8.018,3.063)--(-8.016,3.026)--cycle;
\draw(-8.044,3.062)--(-8.018,3.063)--(-8.016,3.026)--(-8.029,3.026);
\filldraw[fill opacity=0.8,fill=gray!20](-7.91,2.972)--(-7.922,3.019)--(-7.857,3.003)--(-7.837,2.954)--cycle;
\filldraw[fill opacity=0.8,fill=gray!20,draw=none](-6.01,1.171)--(-6.049,1.173)--(-6.027,1.164)--cycle;
\draw(-6.049,1.173)--(-6.027,1.164);
\filldraw[fill opacity=0.8,fill=gray!20,draw=none](-8.124,2.773)--(-8.161,2.753)--(-8.184,2.79)--(-8.147,2.787)--cycle;
\draw(-8.161,2.753)--(-8.184,2.79);
\filldraw[fill opacity=0.8,fill=gray!20,draw=none](-6.136,1.169)--(-6.136,1.162)--(-6.132,1.156)--cycle;
\draw(-6.136,1.169)--(-6.136,1.162);
\filldraw[fill opacity=0.8,fill=gray!20,draw=none](-5.973,1.171)--(-5.966,1.175)--(-5.987,1.184)--cycle;
\draw(-5.966,1.175)--(-5.987,1.184);
\filldraw[fill opacity=0.8,fill=gray!20,draw=none](-6.136,1.176)--(-6.136,1.169)--(-6.132,1.156)--(-6.121,1.14)--(-6.121,1.174)--cycle;
\draw(-6.136,1.176)--(-6.136,1.169);
\draw(-6.121,1.14)--(-6.121,1.174);
\filldraw[fill opacity=0.8,fill=gray!20,draw=none](-6.074,.928)--(-6.074,.935)--(-6.084,.935)--cycle;
\draw(-6.074,.935)--(-6.084,.935);
\filldraw[fill opacity=0.8,fill=gray!20,draw=none](-6.136,1.219)--(-6.136,1.176)--(-6.121,1.174)--(-6.121,1.207)--cycle;
\draw(-6.121,1.174)--(-6.121,1.207)--(-6.136,1.219)--(-6.136,1.176);
\filldraw[fill opacity=0.8,fill=gray!20,draw=none](-7.676,4.137)--(-7.666,4.139)--(-7.666,4.133)--cycle;
\draw(-7.676,4.137)--(-7.666,4.139)--(-7.666,4.133);
\filldraw[fill opacity=0.8,fill=gray!20,draw=none](-7.668,4.134)--(-7.666,4.133)--(-7.68,4.138)--cycle;
\filldraw[fill opacity=0.8,fill=gray!20,draw=none](-7.996,3.025)--(-7.922,3.019)--(-7.92,3.009)--cycle;
\draw(-7.996,3.025)--(-7.922,3.019)--(-7.92,3.009);
\filldraw[fill opacity=0.8,fill=gray!20,draw=none](-6.116,.936)--(-6.121,.934)--(-6.121,.93)--cycle;
\draw(-6.121,.934)--(-6.121,.93);
\filldraw[fill opacity=0.8,fill=gray!20,draw=none](-5.787,1.075)--(-5.823,1.091)--(-5.823,1.121)--(-5.816,1.118)--cycle;
\draw(-5.823,1.121)--(-5.816,1.118)--(-5.787,1.075)--(-5.823,1.091);
\filldraw[fill opacity=0.8,fill=gray!20,draw=none](-8.905,.702)--(-8.952,.7)--(-8.961,.719)--cycle;
\draw(-8.905,.702)--(-8.952,.7)--(-8.961,.719);
\filldraw[fill opacity=0.8,fill=gray!20,draw=none](-8.031,3.031)--(-8.057,3.034)--(-8.096,3.059)--(-8.044,3.062)--cycle;
\draw(-8.096,3.059)--(-8.044,3.062);
\filldraw[fill opacity=0.8,fill=gray!20,draw=none](-6.022,1.16)--(-6.023,1.152)--(-5.999,1.151)--(-5.997,1.156)--cycle;
\draw(-6.023,1.152)--(-5.999,1.151)--(-5.997,1.156);
\filldraw[fill opacity=0.8,fill=gray!20,draw=none](-5.997,1.156)--(-5.999,1.151)--(-6.001,1.143)--cycle;
\draw(-5.997,1.156)--(-5.999,1.151)--(-6.001,1.143);
\filldraw[fill opacity=0.8,fill=gray!20,draw=none](-5.94,1.14)--(-5.985,1.182)--(-6.027,1.164)--(-5.958,1.134)--cycle;
\draw(-6.027,1.164)--(-5.958,1.134)--(-5.94,1.14);
\filldraw[fill opacity=0.8,fill=gray!20,draw=none](-8.147,2.787)--(-8.184,2.79)--(-8.186,2.793)--(-8.165,2.797)--cycle;
\draw(-8.184,2.79)--(-8.186,2.793)--(-8.165,2.797);
\filldraw[fill opacity=0.8,fill=gray!20](-8.929,.668)--(-8.952,.7)--(-8.897,.703)--(-8.9,.67)--cycle;
\filldraw[fill opacity=0.5,fill=gray!20](-11.065,.324)--(-11.092,.364)--(-11.153,.865)--(-11.128,.835)--cycle;
\filldraw[fill opacity=0.8,fill=gray!20](-7.561,3.934)--(-7.558,3.978)--(-7.482,3.972)--(-7.507,3.93)--cycle;
\filldraw[fill opacity=0.8,fill=gray!20](-7.617,3.932)--(-7.637,3.974)--(-7.558,3.978)--(-7.561,3.934)--cycle;
\filldraw[fill opacity=0.8,fill=gray!20,draw=none](-6.09,.966)--(-6.09,.971)--(-6.093,.971)--cycle;
\draw(-6.09,.971)--(-6.093,.971);
\filldraw[fill opacity=0.8,fill=gray!20,draw=none](-6.092,.968)--(-6.116,.936)--(-6.086,.949)--(-6.086,.964)--cycle;
\draw(-6.086,.949)--(-6.086,.964);
\filldraw[fill opacity=0.8,fill=gray!20,draw=none](-8.9,.67)--(-8.897,.7)--(-8.851,.69)--(-8.872,.668)--cycle;
\draw(-8.851,.69)--(-8.872,.668)--(-8.9,.67)--(-8.897,.7);
\filldraw[fill opacity=0.8,fill=gray!20,draw=none](-6.036,1.114)--(-6.036,1.07)--(-5.98,1.07)--(-5.98,1.143)--cycle;
\draw(-6.036,1.114)--(-6.036,1.07);
\draw(-5.98,1.07)--(-5.98,1.143);
\filldraw[fill opacity=0.8,fill=gray!20,draw=none](-5.958,1.134)--(-6.042,1.17)--(-6.056,1.153)--(-6.038,1.114)--(-6.001,1.098)--cycle;
\draw(-6.038,1.114)--(-6.001,1.098)--(-5.958,1.134)--(-6.042,1.17);
\filldraw[fill opacity=0.8,fill=gray!20,draw=none](-9.041,.784)--(-9.062,.78)--(-9.069,.801)--cycle;
\draw(-9.041,.784)--(-9.062,.78)--(-9.069,.801);
\filldraw[fill opacity=0.8,fill=gray!20,draw=none](-6.132,1.156)--(-6.121,1.123)--(-6.121,1.14)--cycle;
\draw(-6.121,1.123)--(-6.121,1.14);
\filldraw[fill opacity=0.8,fill=gray!20,draw=none](-5.879,1.154)--(-5.91,1.167)--(-5.93,1.172)--(-5.966,1.175)--(-5.913,1.152)--cycle;
\draw(-5.879,1.154)--(-5.91,1.167);
\draw(-5.966,1.175)--(-5.913,1.152);
\filldraw[fill opacity=0.8,fill=gray!20,draw=none](-8.165,2.797)--(-8.186,2.793)--(-8.193,2.815)--cycle;
\draw(-8.165,2.797)--(-8.186,2.793)--(-8.193,2.815);
\filldraw[fill opacity=0.5,fill=gray!20](-11.153,.865)--(-11.151,.896)--(-11.092,1.394)--(-11.092,1.38)--cycle;
\filldraw[fill opacity=0.8,fill=gray!20,draw=none](-8.083,2.774)--(-8.085,2.777)--(-8.073,2.777)--cycle;
\draw(-8.083,2.774)--(-8.085,2.777)--(-8.073,2.777);
\filldraw[fill opacity=0.8,fill=gray!20,draw=none](-8.918,1.014)--(-8.933,1.021)--(-8.981,1.027)--(-8.967,1.021)--cycle;
\draw(-8.918,1.014)--(-8.933,1.021);
\draw(-8.981,1.027)--(-8.967,1.021);
\filldraw[fill opacity=0.8,fill=gray!20,draw=none](-8.23,2.858)--(-8.136,2.817)--(-8.145,2.865)--(-8.195,2.887)--cycle;
\draw(-8.23,2.858)--(-8.136,2.817)--(-8.145,2.865)--(-8.195,2.887);
\filldraw[fill opacity=0.8,fill=gray!20,draw=none](-8.23,2.858)--(-8.136,2.817)--(-8.145,2.865)--(-8.195,2.887)--cycle;
\draw(-8.23,2.858)--(-8.136,2.817)--(-8.145,2.865)--(-8.195,2.887);
\filldraw[fill opacity=0.8,fill=gray!20,draw=none](-5.884,.843)--(-5.879,.843)--(-5.879,.852)--cycle;
\draw(-5.884,.843)--(-5.879,.843)--(-5.879,.852);
\filldraw[fill opacity=0.8,fill=gray!20](-7.482,3.972)--(-7.463,4.023)--(-7.398,4.007)--(-7.429,3.959)--cycle;
\filldraw[fill opacity=0.8,fill=gray!20,draw=none](-8.057,3.034)--(-8.104,3.04)--(-8.096,3.059)--cycle;
\draw(-8.104,3.04)--(-8.096,3.059);
\filldraw[fill opacity=0.5,fill=gray!20](-9.738,-.57)--(-9.831,-.754)--(-10.175,-.504)--(-10.044,-.347)--cycle;
\filldraw[fill opacity=0.8,fill=gray!20,draw=none](-8.246,2.909)--(-8.145,2.865)--(-8.136,2.915)--(-8.217,2.951)--cycle;
\draw(-8.246,2.909)--(-8.145,2.865)--(-8.136,2.915)--(-8.217,2.951);
\filldraw[fill opacity=0.8,fill=gray!20,draw=none](-8.246,2.909)--(-8.145,2.865)--(-8.136,2.915)--(-8.217,2.951)--cycle;
\draw(-8.246,2.909)--(-8.145,2.865)--(-8.136,2.915)--(-8.217,2.951);
\filldraw[fill opacity=0.8,fill=gray!20](-8.951,.664)--(-8.995,.692)--(-8.952,.7)--(-8.929,.668)--cycle;
\filldraw[fill opacity=0.8,fill=gray!20](-7.93,2.956)--(-7.94,2.996)--(-7.885,2.982)--(-7.869,2.941)--cycle;
\filldraw[fill opacity=0.8,fill=gray!20,draw=none](-7.439,4.134)--(-7.446,4.188)--(-7.378,4.172)--(-7.371,4.117)--cycle;
\draw(-7.446,4.188)--(-7.378,4.172)--(-7.371,4.117)--(-7.439,4.134);
\filldraw[fill opacity=0.8,fill=gray!20,draw=none](-7.433,4.074)--(-7.439,4.134)--(-7.371,4.117)--(-7.378,4.061)--cycle;
\draw(-7.439,4.134)--(-7.371,4.117)--(-7.378,4.061)--(-7.433,4.074);
\filldraw[fill opacity=0.8,fill=gray!20,draw=none](-5.903,1.164)--(-5.91,1.167)--(-5.906,1.165)--cycle;
\draw(-5.91,1.167)--(-5.906,1.165);
\filldraw[fill opacity=0.8,fill=gray!20](-7.659,3.923)--(-7.697,3.963)--(-7.637,3.974)--(-7.617,3.932)--cycle;
\filldraw[fill opacity=0.8,fill=gray!20,draw=none](-5.896,.878)--(-5.901,.881)--(-5.879,.89)--cycle;
\filldraw[fill opacity=0.8,fill=gray!20,draw=none](-5.883,.873)--(-5.896,.878)--(-5.879,.89)--(-5.858,.881)--cycle;
\draw(-5.879,.89)--(-5.858,.881)--(-5.883,.873);
\filldraw[fill opacity=0.8,fill=gray!20,draw=none](-8.889,1.049)--(-8.847,1.033)--(-8.851,1.024)--cycle;
\draw(-8.847,1.033)--(-8.851,1.024);
\filldraw[fill opacity=0.8,fill=gray!20,draw=none](-8.837,1.02)--(-8.851,1.024)--(-8.889,1.049)--(-8.818,1.044)--(-8.807,1.024)--cycle;
\draw(-8.889,1.049)--(-8.818,1.044)--(-8.807,1.024);
\filldraw[fill opacity=0.8,fill=gray!20,draw=none](-6.056,1.153)--(-6.074,1.13)--(-6.038,1.114)--cycle;
\draw(-6.074,1.13)--(-6.038,1.114);
\filldraw[fill opacity=0.8,fill=gray!20,draw=none](-9.064,.798)--(-9.069,.801)--(-9.072,.809)--cycle;
\draw(-9.069,.801)--(-9.072,.809);
\filldraw[fill opacity=0.8,fill=gray!20,draw=none](-8.118,2.719)--(-8.11,2.726)--(-8.114,2.717)--cycle;
\draw(-8.11,2.726)--(-8.114,2.717);
\filldraw[fill opacity=0.8,fill=gray!20,draw=none](-8.112,2.721)--(-8.109,2.728)--(-8.086,2.712)--cycle;
\draw(-8.112,2.721)--(-8.109,2.728);
\filldraw[fill opacity=0.8,fill=gray!20,draw=none](-8.156,2.743)--(-8.158,2.744)--(-8.157,2.745)--(-8.154,2.742)--cycle;
\draw(-8.158,2.744)--(-8.157,2.745)--(-8.154,2.742);
\filldraw[fill opacity=0.8,fill=gray!20,draw=none](-8.156,2.743)--(-8.154,2.742)--(-8.153,2.741)--cycle;
\draw(-8.154,2.742)--(-8.153,2.741);
\filldraw[fill opacity=0.8,fill=gray!20,draw=none](-8.19,2.665)--(-8.151,2.753)--(-8.107,2.765)--(-8.096,2.759)--(-8.11,2.726)--cycle;
\draw(-8.19,2.665)--(-8.151,2.753);
\draw(-8.096,2.759)--(-8.11,2.726);
\filldraw[fill opacity=0.8,fill=gray!20,draw=none](-8.086,2.712)--(-8.119,2.705)--(-8.153,2.741)--(-8.109,2.728)--cycle;
\draw(-8.086,2.712)--(-8.119,2.705)--(-8.153,2.741);
\filldraw[fill opacity=0.8,fill=gray!20,draw=none](-8.079,2.772)--(-8.089,2.774)--(-8.041,2.884)--(-7.994,2.863)--(-8.03,2.78)--cycle;
\draw(-8.089,2.774)--(-8.041,2.884)--(-7.994,2.863)--(-8.03,2.78);
\filldraw[fill opacity=0.8,fill=gray!20,draw=none](-7.907,2.858)--(-7.868,2.86)--(-7.869,2.849)--cycle;
\draw(-7.868,2.86)--(-7.869,2.849)--(-7.907,2.858);
\filldraw[fill opacity=0.8,fill=gray!20,draw=none](-8.187,2.811)--(-8.193,2.815)--(-8.195,2.822)--cycle;
\draw(-8.193,2.815)--(-8.195,2.822);
\filldraw[fill opacity=0.8,fill=gray!20,draw=none](-8.149,3.025)--(-8.072,2.991)--(-8.027,3.006)--(-8.108,3.042)--cycle;
\draw(-8.149,3.025)--(-8.072,2.991)--(-8.027,3.006)--(-8.108,3.042);
\filldraw[fill opacity=0.8,fill=gray!20,draw=none](-5.857,1.14)--(-5.826,1.125)--(-5.816,1.118)--(-5.851,1.134)--cycle;
\draw(-5.826,1.125)--(-5.816,1.118)--(-5.851,1.134);
\filldraw[fill opacity=0.8,fill=gray!20,draw=none](-5.869,1.146)--(-5.857,1.14)--(-5.851,1.134)--cycle;
\filldraw[fill opacity=0.8,fill=gray!20,draw=none](-6.086,.938)--(-6.086,.84)--(-6.036,.834)--(-6.036,.912)--cycle;
\draw(-6.086,.938)--(-6.086,.84)--(-6.036,.834)--(-6.036,.912);
\filldraw[fill opacity=0.8,fill=gray!20,draw=none](-7.995,3)--(-7.997,3)--(-7.998,3.001)--cycle;
\draw(-7.995,3)--(-7.997,3);
\filldraw[fill opacity=0.8,fill=gray!20,draw=none](-8.138,2.552)--(-8.11,2.596)--(-8.131,2.549)--cycle;
\draw(-8.11,2.596)--(-8.131,2.549);
\filldraw[fill opacity=0.8,fill=gray!20,draw=none](-9.099,.822)--(-9.085,.831)--(-9.072,.809)--(-9.069,.801)--cycle;
\draw(-9.099,.822)--(-9.085,.831);
\draw(-9.072,.809)--(-9.069,.801);
\filldraw[fill opacity=0.8,fill=gray!20,draw=none](-9.093,.76)--(-9.115,.812)--(-9.099,.822)--(-9.069,.801)--(-9.062,.78)--cycle;
\draw(-9.069,.801)--(-9.062,.78)--(-9.093,.76)--(-9.115,.812)--(-9.099,.822);
\filldraw[fill opacity=0.8,fill=gray!20](-9.058,.715)--(-9.093,.76)--(-9.062,.78)--(-9.033,.731)--cycle;
\filldraw[fill opacity=0.8,fill=gray!20,draw=none](-8.019,2.78)--(-8.03,2.78)--(-7.994,2.863)--(-7.952,2.845)--(-7.971,2.802)--cycle;
\draw(-8.03,2.78)--(-7.994,2.863)--(-7.952,2.845)--(-7.971,2.802);
\filldraw[fill opacity=0.8,fill=gray!20,draw=none](-8.107,2.765)--(-8.151,2.753)--(-8.139,2.781)--(-8.13,2.779)--cycle;
\draw(-8.151,2.753)--(-8.139,2.781);
\filldraw[fill opacity=0.8,fill=gray!20,draw=none](-8.223,2.836)--(-8.209,2.844)--(-8.195,2.822)--(-8.193,2.815)--cycle;
\draw(-8.223,2.836)--(-8.209,2.844);
\draw(-8.195,2.822)--(-8.193,2.815);
\filldraw[fill opacity=0.8,fill=gray!20,draw=none](-8.193,2.789)--(-8.222,2.784)--(-8.239,2.825)--(-8.223,2.836)--(-8.193,2.815)--(-8.186,2.793)--cycle;
\draw(-8.222,2.784)--(-8.239,2.825)--(-8.223,2.836);
\draw(-8.193,2.815)--(-8.186,2.793)--(-8.193,2.789);
\filldraw[fill opacity=0.8,fill=gray!20,draw=none](-8.261,2.599)--(-8.203,2.733)--(-8.167,2.747)--(-8.158,2.744)--(-8.156,2.742)--(-8.19,2.665)--cycle;
\draw(-8.261,2.599)--(-8.203,2.733);
\draw(-8.156,2.742)--(-8.19,2.665);
\filldraw[fill opacity=0.8,fill=gray!20,draw=none](-8.167,2.747)--(-8.164,2.748)--(-8.158,2.744)--cycle;
\filldraw[fill opacity=0.8,fill=gray!20,draw=none](-8.164,2.748)--(-8.203,2.733)--(-8.189,2.766)--cycle;
\draw(-8.203,2.733)--(-8.189,2.766);
\filldraw[fill opacity=0.8,fill=gray!20,draw=none](-8.213,2.768)--(-8.189,2.766)--(-8.197,2.746)--cycle;
\draw(-8.189,2.766)--(-8.197,2.746);
\filldraw[fill opacity=0.8,fill=gray!20](-8.182,2.728)--(-8.217,2.773)--(-8.186,2.793)--(-8.157,2.745)--cycle;
\filldraw[fill opacity=0.8,fill=gray!20,draw=none](-9.023,1.011)--(-8.973,1.046)--(-8.981,1.027)--cycle;
\draw(-8.973,1.046)--(-8.981,1.027);
\filldraw[fill opacity=0.8,fill=gray!20](-7.507,3.93)--(-7.482,3.972)--(-7.429,3.959)--(-7.47,3.921)--cycle;
\filldraw[fill opacity=0.8,fill=gray!20,draw=none](-8.766,1.066)--(-8.665,1.208)--(-8.73,1.057)--cycle;
\draw(-8.665,1.208)--(-8.73,1.057);
\filldraw[fill opacity=0.8,fill=gray!20,draw=none](-8.147,3.024)--(-8.124,3.04)--(-8.104,3.04)--cycle;
\filldraw[fill opacity=0.8,fill=gray!20,draw=none](-8.161,3.041)--(-8.157,3.048)--(-8.096,3.059)--(-8.104,3.04)--cycle;
\draw(-8.161,3.041)--(-8.157,3.048)--(-8.096,3.059)--(-8.104,3.04);
\filldraw[fill opacity=0.8,fill=gray!20,draw=none](-8.872,.668)--(-8.851,.69)--(-8.814,.684)--(-8.852,.663)--cycle;
\draw(-8.814,.684)--(-8.852,.663)--(-8.872,.668)--(-8.851,.69);
\filldraw[fill opacity=0.8,fill=gray!20,draw=none](-8.048,2.714)--(-8.068,2.741)--(-8.034,2.742)--(-8.022,2.736)--(-8.024,2.715)--cycle;
\draw(-8.022,2.736)--(-8.024,2.715)--(-8.048,2.714)--(-8.068,2.741)--(-8.034,2.742);
\filldraw[fill opacity=0.8,fill=gray!20](-9.013,.68)--(-9.058,.715)--(-9.033,.731)--(-8.995,.692)--cycle;
\filldraw[fill opacity=0.8,fill=gray!20,draw=none](-8.919,1.049)--(-8.973,1.046)--(-8.952,1.071)--(-8.939,1.072)--cycle;
\draw(-8.919,1.049)--(-8.973,1.046)--(-8.952,1.071)--(-8.939,1.072);
\filldraw[fill opacity=0.8,fill=gray!20,draw=none](-8.916,1.013)--(-8.918,1.014)--(-8.967,1.021)--(-8.957,1.016)--cycle;
\draw(-8.916,1.013)--(-8.918,1.014);
\draw(-8.967,1.021)--(-8.957,1.016);
\filldraw[fill opacity=0.8,fill=gray!20,draw=none](-8.939,1.072)--(-8.952,1.071)--(-8.947,1.074)--cycle;
\draw(-8.939,1.072)--(-8.952,1.071)--(-8.947,1.074);
\filldraw[fill opacity=0.8,fill=gray!20,draw=none](-8.947,1.074)--(-8.95,1.075)--(-8.285,2.602)--(-8.263,2.595)--(-8.924,1.077)--cycle;
\draw(-8.95,1.075)--(-8.285,2.602);
\draw(-8.263,2.595)--(-8.924,1.077);
\filldraw[fill opacity=0.8,fill=gray!20,draw=none](-8.95,1.073)--(-8.95,1.075)--(-8.947,1.074)--cycle;
\draw(-8.95,1.073)--(-8.95,1.075);
\filldraw[fill opacity=0.8,fill=gray!20,draw=none](-8.995,1.063)--(-8.953,1.077)--(-8.95,1.073)--(-8.952,1.071)--cycle;
\draw(-8.95,1.073)--(-8.952,1.071)--(-8.995,1.063)--(-8.953,1.077);
\filldraw[fill opacity=0.8,fill=gray!20](-9.033,1.034)--(-8.995,1.063)--(-8.952,1.071)--(-8.973,1.046)--cycle;
\filldraw[fill opacity=0.8,fill=gray!20,draw=none](-8.986,.995)--(-8.996,.98)--(-8.953,1.077)--(-8.946,1.085)--(-8.981,1.004)--cycle;
\draw(-8.996,.98)--(-8.953,1.077);
\draw(-8.946,1.085)--(-8.981,1.004);
\filldraw[fill opacity=0.8,fill=gray!20,draw=none](-8.953,1.077)--(-8.951,1.078)--(-8.929,1.082)--(-8.95,1.073)--cycle;
\draw(-8.953,1.077)--(-8.951,1.078)--(-8.929,1.082)--(-8.95,1.073);
\filldraw[fill opacity=0.8,fill=gray!20,draw=none](-8.955,1.016)--(-8.976,1.014)--(-8.95,1.073)--(-8.947,1.074)--(-8.927,1.068)--(-8.948,1.02)--cycle;
\draw(-8.976,1.014)--(-8.95,1.073);
\draw(-8.927,1.068)--(-8.948,1.02);
\filldraw[fill opacity=0.8,fill=gray!20,draw=none](-8.947,1.074)--(-8.924,1.077)--(-8.927,1.068)--cycle;
\draw(-8.924,1.077)--(-8.927,1.068);
\filldraw[fill opacity=0.8,fill=gray!20,draw=none](-8.919,1.049)--(-8.939,1.072)--(-8.897,1.074)--(-8.894,1.05)--cycle;
\draw(-8.939,1.072)--(-8.897,1.074)--(-8.894,1.05)--(-8.919,1.049);
\filldraw[fill opacity=0.8,fill=gray!20,draw=none](-8.13,2.779)--(-8.139,2.781)--(-8.138,2.784)--cycle;
\draw(-8.139,2.781)--(-8.138,2.784);
\filldraw[fill opacity=0.8,fill=gray!20,draw=none](-8.891,1.049)--(-8.894,1.052)--(-8.897,1.074)--(-8.843,1.07)--(-8.818,1.044)--cycle;
\draw(-8.894,1.052)--(-8.897,1.074)--(-8.843,1.07)--(-8.818,1.044)--(-8.891,1.049);
\filldraw[fill opacity=0.8,fill=gray!20,draw=none](-5.973,1.171)--(-5.94,1.14)--(-5.908,1.15)--(-5.966,1.175)--cycle;
\draw(-5.94,1.14)--(-5.908,1.15)--(-5.966,1.175);
\filldraw[fill opacity=0.8,fill=gray!20](-8.096,3.059)--(-8.076,3.085)--(-8.02,3.087)--(-8.018,3.063)--cycle;
\filldraw[fill opacity=0.8,fill=gray!20,draw=none](-7.446,4.188)--(-7.451,4.235)--(-7.398,4.221)--(-7.378,4.172)--cycle;
\draw(-7.451,4.235)--(-7.398,4.221)--(-7.378,4.172)--(-7.446,4.188);
\filldraw[fill opacity=0.8,fill=gray!20,draw=none](-8.103,2.734)--(-8.12,2.752)--(-8.101,2.763)--(-8.07,2.745)--(-8.068,2.741)--cycle;
\draw(-8.07,2.745)--(-8.068,2.741)--(-8.103,2.734)--(-8.12,2.752);
\filldraw[fill opacity=0.8,fill=gray!20,draw=none](-8.063,2.741)--(-8.068,2.741)--(-8.07,2.745)--cycle;
\draw(-8.063,2.741)--(-8.068,2.741)--(-8.07,2.745);
\filldraw[fill opacity=0.8,fill=gray!20,draw=none](-7.706,4.193)--(-7.709,4.213)--(-7.711,4.237)--cycle;
\draw(-7.709,4.213)--(-7.711,4.237);
\filldraw[fill opacity=0.8,fill=gray!20,draw=none](-5.901,.881)--(-5.883,.873)--(-5.908,.864)--(-5.925,.871)--cycle;
\draw(-5.883,.873)--(-5.908,.864)--(-5.925,.871);
\filldraw[fill opacity=0.8,fill=gray!20,draw=none](-8.844,.839)--(-8.962,.89)--(-8.977,.842)--(-8.981,.797)--(-8.973,.762)--(-8.954,.743)--(-8.928,.742)--(-8.898,.76)--(-8.868,.794)--cycle;
\draw(-8.962,.89)--(-8.977,.842)--(-8.981,.797)--(-8.973,.762)--(-8.954,.743)--(-8.928,.742)--(-8.898,.76)--(-8.868,.794)--(-8.844,.839);
\filldraw[fill opacity=0.8,fill=gray!20,draw=none](-8.935,1.025)--(-8.944,1.031)--(-8.93,1.062)--(-8.894,1.052)--(-8.892,1.05)--(-8.897,1.04)--cycle;
\draw(-8.944,1.031)--(-8.93,1.062);
\draw(-8.892,1.05)--(-8.897,1.04);
\filldraw[fill opacity=0.8,fill=gray!20,draw=none](-8.894,1.052)--(-8.892,1.051)--(-8.892,1.05)--cycle;
\draw(-8.892,1.051)--(-8.892,1.05);
\filldraw[fill opacity=0.8,fill=gray!20,draw=none](-8.137,2.693)--(-8.182,2.728)--(-8.158,2.744)--(-8.153,2.741)--(-8.119,2.705)--cycle;
\draw(-8.153,2.741)--(-8.119,2.705)--(-8.137,2.693)--(-8.182,2.728)--(-8.158,2.744);
\filldraw[fill opacity=0.8,fill=gray!20,draw=none](-8.653,1.699)--(-8.515,2.015)--(-8.49,1.975)--(-8.635,1.642)--cycle;
\draw(-8.653,1.699)--(-8.515,2.015);
\draw(-8.49,1.975)--(-8.635,1.642);
\filldraw[fill opacity=0.8,fill=gray!20,draw=none](-8.101,2.763)--(-8.12,2.752)--(-8.135,2.767)--(-8.115,2.771)--cycle;
\draw(-8.12,2.752)--(-8.135,2.767)--(-8.115,2.771);
\filldraw[fill opacity=0.8,fill=gray!20,draw=none](-8.515,2.015)--(-8.345,2.407)--(-8.307,2.395)--(-8.49,1.975)--cycle;
\draw(-8.515,2.015)--(-8.345,2.407);
\draw(-8.307,2.395)--(-8.49,1.975);
\filldraw[fill opacity=0.8,fill=gray!20,draw=none](-8.345,2.407)--(-8.261,2.599)--(-8.19,2.665)--(-8.307,2.395)--cycle;
\draw(-8.345,2.407)--(-8.261,2.599);
\draw(-8.19,2.665)--(-8.307,2.395);
\filldraw[fill opacity=0.8,fill=gray!20,draw=none](-8.164,2.748)--(-8.151,2.753)--(-8.156,2.742)--cycle;
\draw(-8.151,2.753)--(-8.156,2.742);
\filldraw[fill opacity=0.8,fill=gray!20,draw=none](-8.164,2.748)--(-8.189,2.766)--(-8.181,2.784)--(-8.139,2.781)--(-8.151,2.753)--cycle;
\draw(-8.189,2.766)--(-8.181,2.784);
\draw(-8.139,2.781)--(-8.151,2.753);
\filldraw[fill opacity=0.8,fill=gray!20,draw=none](-8.154,2.782)--(-8.165,2.802)--(-8.138,2.784)--(-8.139,2.781)--cycle;
\draw(-8.138,2.784)--(-8.139,2.781);
\filldraw[fill opacity=0.8,fill=gray!20,draw=none](-8.127,2.778)--(-8.115,2.771)--(-8.135,2.767)--(-8.142,2.779)--cycle;
\draw(-8.115,2.771)--(-8.135,2.767)--(-8.142,2.779);
\filldraw[fill opacity=0.8,fill=gray!20](-8.066,2.711)--(-8.103,2.734)--(-8.068,2.741)--(-8.048,2.714)--cycle;
\filldraw[fill opacity=0.8,fill=gray!20,draw=none](-8.76,1.026)--(-8.702,1.039)--(-8.711,1.04)--(-8.74,1.035)--(-8.788,1.025)--cycle;
\draw(-8.76,1.026)--(-8.702,1.039);
\draw(-8.74,1.035)--(-8.788,1.025);
\filldraw[fill opacity=0.8,fill=gray!20,draw=none](-8.807,1.024)--(-8.818,1.044)--(-8.79,1.037)--(-8.76,1.026)--cycle;
\draw(-8.807,1.024)--(-8.818,1.044)--(-8.79,1.037);
\filldraw[fill opacity=0.8,fill=gray!20,draw=none](-8.02,2.683)--(-7.996,2.681)--(-8.004,2.676)--cycle;
\draw(-8.02,2.683)--(-7.996,2.681)--(-8.004,2.676);
\filldraw[fill opacity=0.8,fill=gray!20,draw=none](-7.997,2.68)--(-7.996,2.681)--(-7.976,2.676)--(-7.991,2.672)--cycle;
\draw(-7.997,2.68)--(-7.996,2.681)--(-7.976,2.676)--(-7.991,2.672);
\filldraw[fill opacity=0.8,fill=gray!20,draw=none](-8.004,2.676)--(-7.997,2.68)--(-7.991,2.672)--cycle;
\draw(-8.004,2.676)--(-7.997,2.68);
\filldraw[fill opacity=0.8,fill=gray!20,draw=none](-8.01,2.713)--(-8.003,2.727)--(-7.992,2.724)--(-7.983,2.716)--cycle;
\draw(-8.01,2.713)--(-8.003,2.727);
\filldraw[fill opacity=0.8,fill=gray!20,draw=none](-8.003,2.727)--(-8.001,2.732)--(-7.992,2.724)--cycle;
\draw(-8.003,2.727)--(-8.001,2.732);
\filldraw[fill opacity=0.8,fill=gray!20,draw=none](-8.024,2.715)--(-8.022,2.736)--(-8.001,2.732)--(-7.992,2.724)--(-8.001,2.714)--cycle;
\draw(-7.992,2.724)--(-8.001,2.714)--(-8.024,2.715)--(-8.022,2.736);
\filldraw[fill opacity=0.8,fill=gray!20,draw=none](-9.088,.836)--(-9.085,.831)--(-9.087,.83)--cycle;
\draw(-9.085,.831)--(-9.087,.83);
\filldraw[fill opacity=0.8,fill=gray!20,draw=none](-8.211,2.85)--(-8.209,2.844)--(-8.21,2.843)--cycle;
\draw(-8.209,2.844)--(-8.21,2.843);
\filldraw[fill opacity=0.8,fill=gray!20,draw=none](-6.073,1.084)--(-6.036,1.07)--(-6.036,1.114)--cycle;
\draw(-6.036,1.07)--(-6.036,1.114);
\filldraw[fill opacity=0.8,fill=gray!20,draw=none](-6.009,1.083)--(-6.038,1.114)--(-6.074,1.13)--(-6.073,1.068)--(-6.029,1.049)--cycle;
\draw(-6.038,1.114)--(-6.074,1.13);
\draw(-6.073,1.068)--(-6.029,1.049)--(-6.009,1.083);
\filldraw[fill opacity=0.8,fill=gray!20](-7.564,3.901)--(-7.561,3.934)--(-7.507,3.93)--(-7.536,3.899)--cycle;
\filldraw[fill opacity=0.8,fill=gray!20](-7.593,3.9)--(-7.617,3.932)--(-7.561,3.934)--(-7.564,3.901)--cycle;
\filldraw[fill opacity=0.8,fill=gray!20,draw=none](-6.121,1.127)--(-6.121,1.123)--(-6.116,1.118)--cycle;
\draw(-6.121,1.127)--(-6.121,1.123);
\filldraw[fill opacity=0.8,fill=gray!20,draw=none](-8.127,2.778)--(-8.142,2.779)--(-8.15,2.792)--cycle;
\draw(-8.142,2.779)--(-8.15,2.792);
\filldraw[fill opacity=0.8,fill=gray!20,draw=none](-6.09,1.075)--(-6.074,1.116)--(-6.095,1.117)--cycle;
\draw(-6.074,1.116)--(-6.095,1.117);
\filldraw[fill opacity=0.8,fill=gray!20,draw=none](-6.121,1.207)--(-6.121,1.127)--(-6.116,1.118)--(-6.098,1.1)--(-6.086,1.099)--(-6.086,1.197)--cycle;
\draw(-6.086,1.099)--(-6.086,1.197)--(-6.121,1.207)--(-6.121,1.127);
\filldraw[fill opacity=0.8,fill=gray!20,draw=none](-8.885,1)--(-8.898,1)--(-8.888,1.002)--cycle;
\draw(-8.898,1)--(-8.888,1.002);
\filldraw[fill opacity=0.8,fill=gray!20,draw=none](-8.92,.991)--(-8.896,.995)--(-8.76,1.026)--(-8.788,1.025)--(-8.937,.991)--cycle;
\draw(-8.896,.995)--(-8.76,1.026);
\draw(-8.788,1.025)--(-8.937,.991)--(-8.92,.991);
\filldraw[fill opacity=0.8,fill=gray!20,draw=none](-8.894,1.004)--(-8.916,1.013)--(-8.95,1.016)--(-8.955,1.016)--(-8.905,.994)--cycle;
\draw(-8.894,1.004)--(-8.916,1.013);
\draw(-8.955,1.016)--(-8.905,.994);
\filldraw[fill opacity=0.8,fill=gray!20,draw=none](-8.95,1.016)--(-8.944,1.031)--(-8.916,1.013)--cycle;
\draw(-8.95,1.016)--(-8.944,1.031);
\filldraw[fill opacity=0.8,fill=gray!20](-7.94,2.996)--(-7.956,3.027)--(-7.911,3.017)--(-7.885,2.982)--cycle;
\filldraw[fill opacity=0.8,fill=gray!20,draw=none](-9.088,.836)--(-9.087,.83)--(-9.099,.822)--(-9.118,.837)--(-9.122,.867)--(-9.107,.877)--cycle;
\draw(-9.087,.83)--(-9.099,.822);
\draw(-9.118,.837)--(-9.122,.867)--(-9.107,.877);
\filldraw[fill opacity=0.8,fill=gray!20,draw=none](-8.036,3.025)--(-8.045,3.031)--(-8.019,3.032)--(-8.019,3.023)--cycle;
\draw(-8.045,3.031)--(-8.019,3.032)--(-8.019,3.023);
\filldraw[fill opacity=0.8,fill=gray!20,draw=none](-8.008,3.021)--(-8.019,3.023)--(-8.019,3.032)--(-8.014,3.032)--cycle;
\draw(-8.019,3.023)--(-8.019,3.032)--(-8.014,3.032);
\filldraw[fill opacity=0.8,fill=gray!20,draw=none](-5.98,.879)--(-5.98,.832)--(-5.925,.836)--(-5.925,.846)--cycle;
\draw(-5.98,.879)--(-5.98,.832)--(-5.925,.836)--(-5.925,.846);
\filldraw[fill opacity=0.8,fill=gray!20,draw=none](-8.211,2.85)--(-8.21,2.843)--(-8.223,2.836)--(-8.242,2.85)--(-8.246,2.88)--(-8.231,2.891)--cycle;
\draw(-8.21,2.843)--(-8.223,2.836);
\draw(-8.242,2.85)--(-8.246,2.88)--(-8.231,2.891);
\filldraw[fill opacity=0.8,fill=gray!20,draw=none](-5.826,1.125)--(-5.869,1.146)--(-5.879,1.154)--(-5.858,1.145)--cycle;
\draw(-5.879,1.154)--(-5.858,1.145)--(-5.826,1.125);
\filldraw[fill opacity=0.8,fill=gray!20,draw=none](-5.922,.866)--(-5.913,.866)--(-5.908,.864)--cycle;
\draw(-5.913,.866)--(-5.908,.864)--(-5.922,.866);
\filldraw[fill opacity=0.8,fill=gray!20,draw=none](-6.012,1.07)--(-5.98,1.06)--(-5.98,1.07)--cycle;
\draw(-5.98,1.06)--(-5.98,1.07);
\filldraw[fill opacity=0.8,fill=gray!20](-6.039,.993)--(-6.029,.939)--(-6.001,.896)--(-5.958,.87)--(-5.908,.864)--(-5.858,.881)--(-5.816,.916)--(-5.787,.966)--(-5.777,1.021)--(-5.787,1.075)--(-5.816,1.118)--(-5.858,1.145)--(-5.908,1.15)--(-5.958,1.134)--(-6.001,1.098)--(-6.029,1.049)--cycle;
\filldraw[fill opacity=0.8,fill=gray!20,draw=none](-8.008,3.021)--(-8.014,3.032)--(-7.956,3.027)--(-7.946,3.008)--cycle;
\draw(-8.014,3.032)--(-7.956,3.027)--(-7.946,3.008);
\filldraw[fill opacity=0.8,fill=gray!20](-7.888,2.741)--(-7.857,2.789)--(-7.837,2.768)--(-7.872,2.724)--cycle;
\filldraw[fill opacity=0.8,fill=gray!20](-7.857,2.789)--(-7.837,2.843)--(-7.815,2.819)--(-7.837,2.768)--cycle;
\filldraw[fill opacity=0.8,fill=gray!20,draw=none](-6.09,.971)--(-6.092,.968)--(-6.086,.964)--cycle;
\filldraw[fill opacity=0.8,fill=gray!20,draw=none](-9.088,.836)--(-9.107,.877)--(-9.096,.884)--cycle;
\draw(-9.107,.877)--(-9.096,.884);
\filldraw[fill opacity=0.8,fill=gray!20,draw=none](-8.193,2.789)--(-8.217,2.773)--(-8.222,2.784)--cycle;
\draw(-8.193,2.789)--(-8.217,2.773)--(-8.222,2.784);
\filldraw[fill opacity=0.8,fill=gray!20,draw=none](-6.086,.964)--(-6.086,.938)--(-6.036,.912)--cycle;
\draw(-6.086,.964)--(-6.086,.938);
\filldraw[fill opacity=0.8,fill=gray!20,draw=none](-6.116,1.118)--(-6.107,1.101)--(-6.098,1.1)--cycle;
\filldraw[fill opacity=0.8,fill=gray!20,draw=none](-8.211,2.85)--(-8.231,2.891)--(-8.22,2.898)--cycle;
\draw(-8.231,2.891)--(-8.22,2.898);
\filldraw[fill opacity=0.8,fill=gray!20,draw=none](-8.02,2.683)--(-8.024,2.683)--(-8.023,2.686)--cycle;
\draw(-8.02,2.683)--(-8.024,2.683)--(-8.023,2.686);
\filldraw[fill opacity=0.8,fill=gray!20](-7.837,2.843)--(-7.831,2.899)--(-7.808,2.875)--(-7.815,2.819)--cycle;
\filldraw[fill opacity=0.8,fill=gray!20,draw=none](-6.093,.971)--(-6.09,.971)--(-6.096,1.018)--(-6.129,1.019)--cycle;
\draw(-6.093,.971)--(-6.09,.971);
\draw(-6.096,1.018)--(-6.129,1.019);
\filldraw[fill opacity=0.8,fill=gray!20,draw=none](-8.702,1.039)--(-8.677,1.044)--(-8.711,1.04)--cycle;
\draw(-8.702,1.039)--(-8.677,1.044);
\filldraw[fill opacity=0.8,fill=gray!20,draw=none](-8.74,1.035)--(-8.73,1.057)--(-8.702,1.039)--cycle;
\draw(-8.74,1.035)--(-8.73,1.057);
\filldraw[fill opacity=0.8,fill=gray!20,draw=none](-8.73,1.057)--(-8.726,1.068)--(-8.692,1.062)--(-8.702,1.039)--cycle;
\draw(-8.73,1.057)--(-8.726,1.068);
\draw(-8.692,1.062)--(-8.702,1.039);
\filldraw[fill opacity=0.8,fill=gray!20,draw=none](-8.726,1.068)--(-8.025,2.678)--(-7.998,2.673)--(-7.992,2.671)--(-8.692,1.062)--cycle;
\draw(-8.726,1.068)--(-8.025,2.678);
\draw(-7.992,2.671)--(-8.692,1.062);
\filldraw[fill opacity=0.8,fill=gray!20,draw=none](-8.025,2.678)--(-8.023,2.682)--(-8.004,2.676)--(-7.998,2.673)--cycle;
\draw(-8.025,2.678)--(-8.023,2.682);
\filldraw[fill opacity=0.8,fill=gray!20,draw=none](-8.023,2.682)--(-8.022,2.684)--(-8.004,2.676)--cycle;
\draw(-8.023,2.682)--(-8.022,2.684);
\filldraw[fill opacity=0.8,fill=gray!20,draw=none](-8.004,2.676)--(-8.022,2.684)--(-8.01,2.713)--(-7.983,2.716)--(-7.975,2.709)--(-7.991,2.672)--cycle;
\draw(-8.022,2.684)--(-8.01,2.713);
\draw(-7.975,2.709)--(-7.991,2.672);
\filldraw[fill opacity=0.8,fill=gray!20,draw=none](-7.959,2.802)--(-7.971,2.802)--(-7.952,2.845)--(-7.926,2.834)--cycle;
\draw(-7.971,2.802)--(-7.952,2.845)--(-7.926,2.834);
\filldraw[fill opacity=0.8,fill=gray!20,draw=none](-8.154,2.782)--(-8.181,2.784)--(-8.171,2.805)--(-8.165,2.802)--cycle;
\draw(-8.181,2.784)--(-8.171,2.805);
\filldraw[fill opacity=0.8,fill=gray!20,draw=none](-9.023,1.011)--(-9.06,.997)--(-9.033,1.034)--(-8.973,1.046)--cycle;
\draw(-9.06,.997)--(-9.033,1.034)--(-8.973,1.046);
\filldraw[fill opacity=0.8,fill=gray!20](-8.903,.65)--(-8.929,.668)--(-8.9,.67)--(-8.903,.65)--cycle;
\filldraw[fill opacity=0.8,fill=gray!20](-8.903,.65)--(-8.9,.67)--(-8.872,.668)--(-8.903,.65)--cycle;
\filldraw[fill opacity=0.8,fill=gray!20,draw=none](-8.207,2.782)--(-8.191,2.817)--(-8.183,2.814)--(-8.171,2.805)--(-8.181,2.784)--cycle;
\draw(-8.207,2.782)--(-8.191,2.817);
\draw(-8.171,2.805)--(-8.181,2.784);
\filldraw[fill opacity=0.8,fill=gray!20,draw=none](-8.174,2.776)--(-8.185,2.79)--(-8.167,2.803)--(-8.15,2.792)--(-8.142,2.779)--cycle;
\draw(-8.174,2.776)--(-8.185,2.79)--(-8.167,2.803);
\draw(-8.15,2.792)--(-8.142,2.779);
\filldraw[fill opacity=0.8,fill=gray!20](-8.96,.658)--(-9.013,.68)--(-8.995,.692)--(-8.951,.664)--cycle;
\filldraw[fill opacity=0.8,fill=gray!20,draw=none](-8.073,3.03)--(-8.045,3.031)--(-8.036,3.025)--cycle;
\draw(-8.073,3.03)--(-8.045,3.031);
\filldraw[fill opacity=0.8,fill=gray!20,draw=none](-8.157,3.048)--(-8.119,3.076)--(-8.116,3.077)--(-8.096,3.059)--cycle;
\draw(-8.096,3.059)--(-8.157,3.048)--(-8.119,3.076)--(-8.116,3.077);
\filldraw[fill opacity=0.8,fill=gray!20,draw=none](-8.116,3.077)--(-8.076,3.085)--(-8.096,3.059)--cycle;
\draw(-8.116,3.077)--(-8.076,3.085)--(-8.096,3.059);
\filldraw[fill opacity=0.8,fill=gray!20,draw=none](-8.027,2.663)--(-8.024,2.683)--(-8.02,2.683)--(-8.004,2.676)--(-8.027,2.663)--cycle;
\draw(-8.004,2.676)--(-8.027,2.663)--(-8.027,2.663)--(-8.024,2.683)--(-8.02,2.683);
\filldraw[fill opacity=0.8,fill=gray!20](-8.027,2.663)--(-8.052,2.682)--(-8.024,2.683)--(-8.027,2.663)--cycle;
\filldraw[fill opacity=0.8,fill=gray!20](-7.929,2.703)--(-7.888,2.741)--(-7.872,2.724)--(-7.917,2.691)--cycle;
\filldraw[fill opacity=0.8,fill=gray!20,draw=none](-8.187,2.993)--(-8.11,2.959)--(-8.072,2.991)--(-8.149,3.025)--cycle;
\draw(-8.187,2.993)--(-8.11,2.959)--(-8.072,2.991)--(-8.149,3.025);
\filldraw[fill opacity=0.8,fill=gray!20,draw=none](-8.187,2.993)--(-8.11,2.959)--(-8.072,2.991)--(-8.149,3.025)--cycle;
\draw(-8.187,2.993)--(-8.11,2.959)--(-8.072,2.991)--(-8.149,3.025);
\filldraw[fill opacity=0.8,fill=gray!20](-8.084,2.671)--(-8.137,2.693)--(-8.119,2.705)--(-8.074,2.677)--cycle;
\filldraw[fill opacity=0.8,fill=gray!20,draw=none](-6.036,.912)--(-6.036,.834)--(-5.98,.832)--(-5.98,.879)--cycle;
\draw(-6.036,.912)--(-6.036,.834)--(-5.98,.832)--(-5.98,.879);
\filldraw[fill opacity=0.8,fill=gray!20](-7.615,3.896)--(-7.659,3.923)--(-7.617,3.932)--(-7.593,3.9)--cycle;
\filldraw[fill opacity=0.8,fill=gray!20,draw=none](-8.156,2.753)--(-8.174,2.776)--(-8.142,2.779)--(-8.135,2.767)--cycle;
\draw(-8.142,2.779)--(-8.135,2.767)--(-8.156,2.753)--(-8.174,2.776);
\filldraw[fill opacity=0.5,fill=gray!20](-10.154,2.651)--(-10.211,2.668)--(-9.769,2.916)--(-9.708,2.901)--cycle;
\filldraw[fill opacity=0.8,fill=gray!20,draw=none](-8.147,3.024)--(-8.184,3.011)--(-8.161,3.041)--(-8.124,3.04)--cycle;
\draw(-8.184,3.011)--(-8.161,3.041);
\filldraw[fill opacity=0.8,fill=gray!20,draw=none](-7.741,4.067)--(-7.745,4.066)--(-7.751,4.122)--(-7.73,4.126)--cycle;
\draw(-7.741,4.067)--(-7.745,4.066)--(-7.751,4.122)--(-7.73,4.126);
\filldraw[fill opacity=0.8,fill=gray!20,draw=none](-7.73,4.126)--(-7.751,4.122)--(-7.745,4.177)--(-7.717,4.182)--cycle;
\draw(-7.73,4.126)--(-7.751,4.122)--(-7.745,4.177)--(-7.717,4.182);
\filldraw[fill opacity=0.8,fill=gray!20,draw=none](-9.041,.998)--(-9.062,.994)--(-9.06,.997)--(-9.023,1.011)--cycle;
\draw(-9.041,.998)--(-9.062,.994)--(-9.06,.997);
\filldraw[fill opacity=0.8,fill=gray!20,draw=none](-7.705,4.184)--(-7.715,4.182)--(-7.731,4.198)--(-7.711,4.227)--cycle;
\draw(-7.705,4.184)--(-7.715,4.182);
\filldraw[fill opacity=0.8,fill=gray!20,draw=none](-6.074,1.116)--(-6.09,1.075)--(-6.073,1.068)--cycle;
\draw(-6.09,1.075)--(-6.073,1.068);
\filldraw[fill opacity=0.8,fill=gray!20](-8.118,2.724)--(-8.156,2.753)--(-8.135,2.767)--(-8.103,2.734)--cycle;
\filldraw[fill opacity=0.8,fill=gray!20,draw=none](-8.045,3.031)--(-8.07,3.048)--(-8.068,3.05)--(-8.021,3.052)--(-8.019,3.032)--cycle;
\draw(-8.07,3.048)--(-8.068,3.05)--(-8.021,3.052)--(-8.019,3.032)--(-8.045,3.031);
\filldraw[fill opacity=0.8,fill=gray!20](-8.019,3.032)--(-8.021,3.052)--(-7.977,3.049)--(-7.956,3.027)--cycle;
\filldraw[fill opacity=0.8,fill=gray!20,draw=none](-8.165,3.012)--(-8.186,3.008)--(-8.184,3.011)--(-8.147,3.024)--cycle;
\draw(-8.165,3.012)--(-8.186,3.008)--(-8.184,3.011);
\filldraw[fill opacity=0.8,fill=gray!20,draw=none](-9.064,.98)--(-9.052,.996)--(-9.041,.998)--cycle;
\draw(-9.052,.996)--(-9.041,.998);
\filldraw[fill opacity=0.8,fill=gray!20,draw=none](-6.107,1.101)--(-6.092,1.072)--(-6.086,1.073)--(-6.086,1.099)--cycle;
\draw(-6.086,1.073)--(-6.086,1.099);
\filldraw[fill opacity=0.8,fill=gray!20](-8.903,.65)--(-8.951,.664)--(-8.929,.668)--(-8.903,.65)--cycle;
\filldraw[fill opacity=0.5,fill=gray!20](-9,2.719)--(-9.05,2.827)--(-8.599,2.811)--(-8.57,2.704)--cycle;
\filldraw[fill opacity=0.8,fill=gray!20,draw=none](-8.187,2.993)--(-8.176,3.01)--(-8.165,3.012)--cycle;
\draw(-8.176,3.01)--(-8.165,3.012);
\filldraw[fill opacity=0.8,fill=gray!20](-8.027,2.663)--(-8.074,2.677)--(-8.052,2.682)--(-8.027,2.663)--cycle;
\filldraw[fill opacity=0.8,fill=gray!20](-7.831,2.899)--(-7.837,2.954)--(-7.815,2.93)--(-7.808,2.875)--cycle;
\filldraw[fill opacity=0.8,fill=gray!20,draw=none](-8.562,.822)--(-8.528,.83)--(-8.59,.849)--(-8.647,.836)--cycle;
\draw(-8.562,.822)--(-8.528,.83);
\draw(-8.59,.849)--(-8.647,.836);
\filldraw[fill opacity=0.8,fill=gray!20](-7.536,3.899)--(-7.507,3.93)--(-7.47,3.921)--(-7.517,3.894)--cycle;
\filldraw[fill opacity=0.8,fill=gray!20](-8.818,1.044)--(-8.843,1.07)--(-8.805,1.061)--(-8.764,1.031)--cycle;
\filldraw[fill opacity=0.8,fill=gray!20](-8.903,.65)--(-8.872,.668)--(-8.852,.663)--(-8.903,.65)--cycle;
\filldraw[fill opacity=0.8,fill=gray!20,draw=none](-8.045,3.031)--(-8.073,3.03)--(-8.083,3.031)--(-8.07,3.048)--cycle;
\draw(-8.045,3.031)--(-8.073,3.03);
\draw(-8.083,3.031)--(-8.07,3.048);
\filldraw[fill opacity=0.5,fill=gray!20](-10.912,1.872)--(-10.918,1.871)--(-10.642,2.294)--(-10.626,2.31)--cycle;
\filldraw[fill opacity=0.8,fill=gray!20,draw=none](-9.085,.942)--(-9.087,.941)--(-9.08,.96)--(-9.069,.975)--(-9.072,.968)--cycle;
\draw(-9.085,.942)--(-9.087,.941);
\draw(-9.069,.975)--(-9.072,.968);
\filldraw[fill opacity=0.8,fill=gray!20,draw=none](-9.072,.968)--(-9.069,.975)--(-9.064,.98)--cycle;
\draw(-9.072,.968)--(-9.069,.975);
\filldraw[fill opacity=0.8,fill=gray!20,draw=none](-8.195,2.981)--(-8.193,2.988)--(-8.187,2.993)--cycle;
\draw(-8.195,2.981)--(-8.193,2.988);
\filldraw[fill opacity=0.8,fill=gray!20,draw=none](-8.209,2.955)--(-8.223,2.946)--(-8.193,2.988)--(-8.195,2.981)--cycle;
\draw(-8.209,2.955)--(-8.223,2.946);
\draw(-8.193,2.988)--(-8.195,2.981);
\filldraw[fill opacity=0.8,fill=gray!20,draw=none](-9.088,.935)--(-9.096,.884)--(-9.097,.908)--cycle;
\filldraw[fill opacity=0.8,fill=gray!20,draw=none](-9.096,.884)--(-9.122,.867)--(-9.115,.923)--(-9.097,.934)--cycle;
\draw(-9.096,.884)--(-9.122,.867)--(-9.115,.923)--(-9.097,.934);
\filldraw[fill opacity=0.8,fill=gray!20,draw=none](-9.064,.98)--(-9.069,.975)--(-9.062,.994)--(-9.052,.996)--cycle;
\draw(-9.069,.975)--(-9.062,.994)--(-9.052,.996);
\filldraw[fill opacity=0.8,fill=gray!20,draw=none](-7.715,4.182)--(-7.745,4.177)--(-7.735,4.202)--cycle;
\draw(-7.715,4.182)--(-7.745,4.177)--(-7.735,4.202);
\filldraw[fill opacity=0.8,fill=gray!20](-7.976,2.676)--(-7.929,2.703)--(-7.917,2.691)--(-7.97,2.67)--cycle;
\filldraw[fill opacity=0.8,fill=gray!20](-7.758,3.991)--(-7.78,4.043)--(-7.745,4.066)--(-7.726,4.011)--cycle;
\filldraw[fill opacity=0.8,fill=gray!20](-7.78,4.043)--(-7.787,4.099)--(-7.751,4.122)--(-7.745,4.066)--cycle;
\filldraw[fill opacity=0.8,fill=gray!20,draw=none](-8.22,2.898)--(-8.22,2.922)--(-8.211,2.948)--cycle;
\filldraw[fill opacity=0.8,fill=gray!20,draw=none](-8.22,2.898)--(-8.246,2.88)--(-8.239,2.936)--(-8.221,2.948)--cycle;
\draw(-8.22,2.898)--(-8.246,2.88)--(-8.239,2.936)--(-8.221,2.948);
\filldraw[fill opacity=0.8,fill=gray!20,draw=none](-8.187,2.993)--(-8.193,2.988)--(-8.186,3.008)--(-8.176,3.01)--cycle;
\draw(-8.193,2.988)--(-8.186,3.008)--(-8.176,3.01);
\filldraw[fill opacity=0.8,fill=gray!20,draw=none](-7.983,2.716)--(-7.992,2.724)--(-7.971,2.717)--cycle;
\filldraw[fill opacity=0.8,fill=gray!20,draw=none](-7.983,2.716)--(-7.971,2.717)--(-7.975,2.709)--cycle;
\draw(-7.971,2.717)--(-7.975,2.709);
\filldraw[fill opacity=0.8,fill=gray!20,draw=none](-8.001,2.714)--(-7.992,2.724)--(-7.971,2.717)--(-7.985,2.71)--cycle;
\draw(-7.971,2.717)--(-7.985,2.71)--(-8.001,2.714)--(-7.992,2.724);
\filldraw[fill opacity=0.8,fill=gray!20](-7.885,2.804)--(-7.869,2.849)--(-7.851,2.829)--(-7.869,2.786)--cycle;
\filldraw[fill opacity=0.8,fill=gray!20](-7.911,2.764)--(-7.885,2.804)--(-7.869,2.786)--(-7.898,2.75)--cycle;
\filldraw[fill opacity=0.8,fill=gray!20](-7.723,3.946)--(-7.758,3.991)--(-7.726,4.011)--(-7.697,3.963)--cycle;
\filldraw[fill opacity=0.8,fill=gray!20,draw=none](-9.08,.96)--(-9.069,.99)--(-9.062,.994)--(-9.069,.975)--cycle;
\draw(-9.069,.99)--(-9.062,.994)--(-9.069,.975);
\filldraw[fill opacity=0.8,fill=gray!20,draw=none](-9.088,.935)--(-9.087,.941)--(-9.085,.942)--cycle;
\draw(-9.087,.941)--(-9.085,.942);
\filldraw[fill opacity=0.8,fill=gray!20](-7.869,2.849)--(-7.863,2.895)--(-7.844,2.875)--(-7.851,2.829)--cycle;
\filldraw[fill opacity=0.8,fill=gray!20,draw=none](-8.223,2.946)--(-8.239,2.936)--(-8.217,2.987)--(-8.186,3.008)--(-8.193,2.988)--cycle;
\draw(-8.223,2.946)--(-8.239,2.936)--(-8.217,2.987)--(-8.186,3.008)--(-8.193,2.988);
\filldraw[fill opacity=0.8,fill=gray!20,draw=none](-8.165,2.802)--(-8.171,2.805)--(-8.17,2.809)--cycle;
\draw(-8.171,2.805)--(-8.17,2.809);
\filldraw[fill opacity=0.5,fill=gray!20](-10.28,-.548)--(-10.378,-.573)--(-10.664,-.198)--(-10.557,-.185)--cycle;
\filldraw[fill opacity=0.8,fill=gray!20,draw=none](-8.814,.684)--(-8.8,.684)--(-8.794,.677)--(-8.809,.671)--cycle;
\draw(-8.8,.684)--(-8.794,.677)--(-8.809,.671);
\filldraw[fill opacity=0.8,fill=gray!20,draw=none](-8.852,.663)--(-8.814,.684)--(-8.809,.671)--(-8.847,.657)--cycle;
\draw(-8.809,.671)--(-8.847,.657)--(-8.852,.663)--(-8.814,.684);
\filldraw[fill opacity=0.8,fill=gray!20,draw=none](-8.211,2.948)--(-8.21,2.954)--(-8.209,2.955)--cycle;
\draw(-8.21,2.954)--(-8.209,2.955);
\filldraw[fill opacity=0.8,fill=gray!20,draw=none](-6.096,1.018)--(-6.09,1.069)--(-6.114,1.07)--cycle;
\draw(-6.09,1.069)--(-6.114,1.07);
\filldraw[fill opacity=0.8,fill=gray!20,draw=none](-8.165,2.801)--(-8.167,2.803)--(-8.166,2.803)--cycle;
\draw(-8.167,2.803)--(-8.166,2.803);
\filldraw[fill opacity=0.8,fill=gray!20,draw=none](-8.18,2.812)--(-8.18,2.813)--(-8.17,2.809)--(-8.171,2.805)--cycle;
\draw(-8.17,2.809)--(-8.171,2.805);
\filldraw[fill opacity=0.8,fill=gray!20,draw=none](-8.189,2.84)--(-8.166,2.803)--(-8.167,2.803)--(-8.179,2.812)--cycle;
\draw(-8.166,2.803)--(-8.167,2.803);
\filldraw[fill opacity=0.5,fill=gray!20](-10.756,-.194)--(-10.829,-.174)--(-11.014,.291)--(-10.939,.268)--cycle;
\filldraw[fill opacity=0.8,fill=gray!20,draw=none](-9.088,.935)--(-9.097,.908)--(-9.097,.934)--(-9.087,.941)--cycle;
\draw(-9.097,.934)--(-9.087,.941);
\filldraw[fill opacity=0.8,fill=gray!20](-8.027,2.699)--(-8.048,2.714)--(-8.024,2.715)--(-8.027,2.699)--cycle;
\filldraw[fill opacity=0.8,fill=gray!20](-8.027,2.699)--(-8.024,2.715)--(-8.001,2.714)--(-8.027,2.699)--cycle;
\filldraw[fill opacity=0.8,fill=gray!20](-7.945,2.732)--(-7.911,2.764)--(-7.898,2.75)--(-7.936,2.722)--cycle;
\filldraw[fill opacity=0.8,fill=gray!20,draw=none](-9.08,.96)--(-9.087,.941)--(-9.099,.933)--cycle;
\draw(-9.087,.941)--(-9.099,.933);
\filldraw[fill opacity=0.8,fill=gray!20](-8.074,2.706)--(-8.118,2.724)--(-8.103,2.734)--(-8.066,2.711)--cycle;
\filldraw[fill opacity=0.8,fill=gray!20,draw=none](-8.194,2.813)--(-8.191,2.817)--(-8.195,2.81)--cycle;
\draw(-8.191,2.817)--(-8.195,2.81);
\filldraw[fill opacity=0.8,fill=gray!20,draw=none](-8.167,2.803)--(-8.185,2.79)--(-8.2,2.826)--cycle;
\draw(-8.167,2.803)--(-8.185,2.79)--(-8.2,2.826);
\filldraw[fill opacity=0.8,fill=gray!20,draw=none](-6.092,1.072)--(-6.09,1.069)--(-6.086,1.073)--cycle;
\filldraw[fill opacity=0.5,fill=gray!20](-10.452,.037)--(-10.279,-.039)--(-10.426,.334)--(-10.599,.409)--cycle;
\filldraw[fill opacity=0.8,fill=gray!20,draw=none](-8.211,2.948)--(-8.22,2.922)--(-8.221,2.948)--(-8.21,2.954)--cycle;
\draw(-8.221,2.948)--(-8.21,2.954);
\filldraw[fill opacity=0.8,fill=gray!20,draw=none](-8.95,1.016)--(-8.957,1.016)--(-8.955,1.016)--cycle;
\draw(-8.957,1.016)--(-8.955,1.016);
\filldraw[fill opacity=0.8,fill=gray!20,draw=none](-8.955,1.016)--(-8.948,1.02)--(-8.95,1.016)--cycle;
\draw(-8.948,1.02)--(-8.95,1.016);
\filldraw[fill opacity=0.8,fill=gray!20,draw=none](-8.285,2.602)--(-8.237,2.712)--(-8.203,2.733)--(-8.263,2.595)--cycle;
\draw(-8.285,2.602)--(-8.237,2.712);
\draw(-8.203,2.733)--(-8.263,2.595);
\filldraw[fill opacity=0.8,fill=gray!20,draw=none](-8.237,2.712)--(-8.213,2.768)--(-8.197,2.746)--(-8.203,2.733)--cycle;
\draw(-8.237,2.712)--(-8.213,2.768);
\draw(-8.197,2.746)--(-8.203,2.733);
\filldraw[fill opacity=0.8,fill=gray!20,draw=none](-8.213,2.768)--(-8.207,2.782)--(-8.181,2.784)--(-8.189,2.766)--cycle;
\draw(-8.213,2.768)--(-8.207,2.782);
\draw(-8.181,2.784)--(-8.189,2.766);
\filldraw[fill opacity=0.8,fill=gray!20,draw=none](-8.18,2.813)--(-8.18,2.836)--(-8.17,2.809)--cycle;
\filldraw[fill opacity=0.8,fill=gray!20,draw=none](-8.094,3.027)--(-8.135,3.019)--(-8.12,3.03)--(-8.083,3.031)--cycle;
\draw(-8.094,3.027)--(-8.135,3.019)--(-8.12,3.03);
\filldraw[fill opacity=0.8,fill=gray!20,draw=none](-8.12,3.03)--(-8.103,3.043)--(-8.071,3.049)--(-8.07,3.048)--(-8.083,3.031)--cycle;
\draw(-8.12,3.03)--(-8.103,3.043)--(-8.071,3.049);
\draw(-8.07,3.048)--(-8.083,3.031);
\filldraw[fill opacity=0.8,fill=gray!20](-7.787,4.099)--(-7.78,4.154)--(-7.745,4.177)--(-7.751,4.122)--cycle;
\filldraw[fill opacity=0.8,fill=gray!20,draw=none](-7.454,4.254)--(-7.427,4.229)--(-7.451,4.235)--cycle;
\draw(-7.427,4.229)--(-7.451,4.235);
\filldraw[fill opacity=0.8,fill=gray!20,draw=none](-7.998,2.673)--(-7.991,2.672)--(-7.992,2.671)--cycle;
\draw(-7.991,2.672)--(-7.992,2.671);
\filldraw[fill opacity=0.8,fill=gray!20,draw=none](-8.027,2.663)--(-8.004,2.676)--(-7.991,2.672)--(-8.027,2.663)--cycle;
\draw(-7.991,2.672)--(-8.027,2.663)--(-8.027,2.663)--(-8.004,2.676);
\filldraw[fill opacity=0.8,fill=gray!20,draw=none](-7.998,2.673)--(-8.004,2.676)--(-7.991,2.672)--cycle;
\filldraw[fill opacity=0.8,fill=gray!20](-7.677,3.912)--(-7.723,3.946)--(-7.697,3.963)--(-7.659,3.923)--cycle;
\filldraw[fill opacity=0.8,fill=gray!20](-7.837,2.954)--(-7.857,3.003)--(-7.837,2.982)--(-7.815,2.93)--cycle;
\filldraw[fill opacity=0.8,fill=gray!20](-8.027,2.699)--(-8.066,2.711)--(-8.048,2.714)--(-8.027,2.699)--cycle;
\filldraw[fill opacity=0.8,fill=gray!20](-7.863,2.895)--(-7.869,2.941)--(-7.851,2.922)--(-7.844,2.875)--cycle;
\filldraw[fill opacity=0.8,fill=gray!20,draw=none](-9.069,.99)--(-9.06,.997)--(-9.062,.994)--cycle;
\draw(-9.06,.997)--(-9.062,.994)--(-9.069,.99);
\filldraw[fill opacity=0.8,fill=gray!20,draw=none](-8.562,.822)--(-8.647,.836)--(-8.687,.827)--cycle;
\draw(-8.647,.836)--(-8.687,.827);
\filldraw[fill opacity=0.8,fill=gray!20,draw=none](-8.193,3.003)--(-8.184,3.011)--(-8.186,3.008)--cycle;
\draw(-8.184,3.011)--(-8.186,3.008)--(-8.193,3.003);
\filldraw[fill opacity=0.8,fill=gray!20](-8.903,.65)--(-8.96,.658)--(-8.951,.664)--(-8.903,.65)--cycle;
\filldraw[fill opacity=0.8,fill=gray!20,draw=none](-9.069,.99)--(-9.093,.974)--(-9.058,1.018)--(-9.033,1.034)--(-9.06,.997)--cycle;
\draw(-9.069,.99)--(-9.093,.974)--(-9.058,1.018)--(-9.033,1.034)--(-9.06,.997);
\filldraw[fill opacity=0.8,fill=gray!20](-7.956,3.027)--(-7.977,3.049)--(-7.945,3.041)--(-7.911,3.017)--cycle;
\filldraw[fill opacity=0.8,fill=gray!20](-8.027,2.699)--(-8.001,2.714)--(-7.985,2.71)--(-8.027,2.699)--cycle;
\filldraw[fill opacity=0.8,fill=gray!20](-8.162,3.418)--(-8.173,3.418)--(-7.952,4.005)--cycle;
\filldraw[fill opacity=0.8,fill=gray!20](-8.027,2.663)--(-8.084,2.671)--(-8.074,2.677)--(-8.027,2.663)--cycle;
\filldraw[fill opacity=0.8,fill=gray!20,draw=none](-8.127,3.015)--(-8.115,3.023)--(-8.094,3.027)--cycle;
\draw(-8.115,3.023)--(-8.094,3.027);
\filldraw[fill opacity=0.8,fill=gray!20,draw=none](-8.193,3.003)--(-8.217,2.987)--(-8.182,3.031)--(-8.157,3.048)--(-8.184,3.011)--cycle;
\draw(-8.193,3.003)--(-8.217,2.987)--(-8.182,3.031)--(-8.157,3.048)--(-8.184,3.011);
\filldraw[fill opacity=0.8,fill=gray!20,draw=none](-6.073,1.084)--(-6.082,1.087)--(-6.086,1.073)--cycle;
\filldraw[fill opacity=0.8,fill=gray!20,draw=none](-6.065,1.135)--(-6.086,1.135)--(-6.086,1.073)--cycle;
\draw(-6.086,1.135)--(-6.086,1.073);
\filldraw[fill opacity=0.8,fill=gray!20,draw=none](-7.708,4.208)--(-7.711,4.227)--(-7.71,4.229)--(-7.701,4.231)--cycle;
\draw(-7.71,4.229)--(-7.701,4.231);
\filldraw[fill opacity=0.8,fill=gray!20,draw=none](-7.454,4.254)--(-7.452,4.238)--(-7.475,4.274)--(-7.475,4.274)--cycle;
\draw(-7.475,4.274)--(-7.475,4.274);
\filldraw[fill opacity=0.8,fill=gray!20,draw=none](-9.08,.96)--(-9.099,.933)--(-9.115,.923)--(-9.093,.974)--(-9.069,.99)--cycle;
\draw(-9.099,.933)--(-9.115,.923)--(-9.093,.974)--(-9.069,.99);
\filldraw[fill opacity=0.8,fill=gray!20,draw=none](-7.991,2.672)--(-7.976,2.676)--(-7.97,2.67)--(-7.986,2.668)--cycle;
\draw(-7.991,2.672)--(-7.976,2.676)--(-7.97,2.67)--(-7.986,2.668);
\filldraw[fill opacity=0.8,fill=gray!20,draw=none](-7.986,2.668)--(-7.97,2.67)--(-7.978,2.665)--cycle;
\draw(-7.986,2.668)--(-7.97,2.67)--(-7.978,2.665);
\filldraw[fill opacity=0.8,fill=gray!20,draw=none](-7.978,2.665)--(-7.986,2.668)--(-7.991,2.672)--(-7.975,2.709)--(-7.963,2.711)--(-7.958,2.707)--(-7.977,2.665)--cycle;
\draw(-7.991,2.672)--(-7.975,2.709);
\draw(-7.958,2.707)--(-7.977,2.665);
\filldraw[fill opacity=0.8,fill=gray!20,draw=none](-7.975,2.709)--(-7.971,2.717)--(-7.963,2.711)--cycle;
\draw(-7.975,2.709)--(-7.971,2.717);
\filldraw[fill opacity=0.8,fill=gray!20](-7.985,2.71)--(-7.945,2.732)--(-7.936,2.722)--(-7.98,2.705)--cycle;
\filldraw[fill opacity=0.8,fill=gray!20,draw=none](-8.955,1.016)--(-8.983,.999)--(-8.976,1.014)--cycle;
\draw(-8.983,.999)--(-8.976,1.014);
\filldraw[fill opacity=0.8,fill=gray!20,draw=none](-9.099,.822)--(-9.115,.812)--(-9.118,.837)--cycle;
\draw(-9.099,.822)--(-9.115,.812)--(-9.118,.837);
\filldraw[fill opacity=0.8,fill=gray!20,draw=none](-5.858,1.145)--(-5.879,1.154)--(-5.913,1.152)--(-5.908,1.15)--cycle;
\draw(-5.913,1.152)--(-5.908,1.15)--(-5.858,1.145)--(-5.879,1.154);
\filldraw[fill opacity=0.8,fill=gray!20,draw=none](-5.925,1.183)--(-5.925,1.157)--(-5.879,1.154)--cycle;
\draw(-5.925,1.183)--(-5.925,1.157);
\filldraw[fill opacity=0.8,fill=gray!20,draw=none](-5.925,1.193)--(-5.925,1.183)--(-5.879,1.154)--(-5.879,1.2)--cycle;
\draw(-5.879,1.154)--(-5.879,1.2)--(-5.925,1.193)--(-5.925,1.183);
\filldraw[fill opacity=0.8,fill=gray!20](-8.903,.65)--(-8.852,.663)--(-8.847,.657)--(-8.903,.65)--cycle;
\filldraw[fill opacity=0.8,fill=gray!20,draw=none](-8.223,2.836)--(-8.239,2.825)--(-8.242,2.85)--cycle;
\draw(-8.223,2.836)--(-8.239,2.825)--(-8.242,2.85);
\filldraw[fill opacity=0.8,fill=gray!20,draw=none](-6.077,1.102)--(-6.082,1.087)--(-6.073,1.084)--(-6.036,1.114)--cycle;
\filldraw[fill opacity=0.8,fill=gray!20](-7.398,4.007)--(-7.378,4.061)--(-7.356,4.037)--(-7.378,3.986)--cycle;
\filldraw[fill opacity=0.8,fill=gray!20](-7.429,3.959)--(-7.398,4.007)--(-7.378,3.986)--(-7.413,3.942)--cycle;
\filldraw[fill opacity=0.5,fill=gray!20](-9.572,2.797)--(-9.642,2.861)--(-9.167,2.981)--(-9.106,2.915)--cycle;
\filldraw[fill opacity=0.8,fill=gray!20](-7.378,4.061)--(-7.371,4.117)--(-7.349,4.093)--(-7.356,4.037)--cycle;
\filldraw[fill opacity=0.8,fill=gray!20](-7.869,2.941)--(-7.885,2.982)--(-7.869,2.965)--(-7.851,2.922)--cycle;
\filldraw[fill opacity=0.8,fill=gray!20,draw=none](-8.939,1.072)--(-8.947,1.074)--(-8.929,1.082)--(-8.9,1.084)--(-8.897,1.074)--cycle;
\draw(-8.947,1.074)--(-8.929,1.082)--(-8.9,1.084)--(-8.897,1.074)--(-8.939,1.072);
\filldraw[fill opacity=0.8,fill=gray!20](-8.897,1.074)--(-8.9,1.084)--(-8.872,1.082)--(-8.843,1.07)--cycle;
\filldraw[fill opacity=0.8,fill=gray!20,draw=none](-8.702,1.039)--(-7.999,2.654)--(-7.98,2.663)--(-7.978,2.662)--(-8.695,1.016)--cycle;
\draw(-8.702,1.039)--(-7.999,2.654);
\draw(-7.978,2.662)--(-8.695,1.016);
\filldraw[fill opacity=0.8,fill=gray!20,draw=none](-8.797,.68)--(-8.8,.684)--(-8.788,.684)--cycle;
\draw(-8.797,.68)--(-8.8,.684);
\filldraw[fill opacity=0.8,fill=gray!20](-7.78,4.154)--(-7.758,4.205)--(-7.726,4.226)--(-7.745,4.177)--cycle;
\filldraw[fill opacity=0.8,fill=gray!20](-7.568,3.881)--(-7.564,3.901)--(-7.536,3.899)--(-7.568,3.881)--cycle;
\filldraw[fill opacity=0.8,fill=gray!20](-7.47,3.921)--(-7.429,3.959)--(-7.413,3.942)--(-7.458,3.909)--cycle;
\filldraw[fill opacity=0.8,fill=gray!20](-7.568,3.881)--(-7.593,3.9)--(-7.564,3.901)--(-7.568,3.881)--cycle;
\filldraw[fill opacity=0.8,fill=gray!20,draw=none](-7.999,2.654)--(-7.992,2.671)--(-7.986,2.668)--(-7.98,2.663)--cycle;
\draw(-7.999,2.654)--(-7.992,2.671);
\filldraw[fill opacity=0.8,fill=gray!20,draw=none](-7.992,2.671)--(-7.991,2.672)--(-7.986,2.668)--cycle;
\draw(-7.992,2.671)--(-7.991,2.672);
\filldraw[fill opacity=0.8,fill=gray!20,draw=none](-8.027,2.663)--(-7.991,2.672)--(-7.986,2.668)--(-8.027,2.663)--cycle;
\draw(-7.986,2.668)--(-8.027,2.663)--(-8.027,2.663)--(-7.991,2.672);
\filldraw[fill opacity=0.8,fill=gray!20,draw=none](-7.956,2.712)--(-7.963,2.711)--(-7.954,2.715)--cycle;
\draw(-7.963,2.711)--(-7.954,2.715);
\filldraw[fill opacity=0.8,fill=gray!20,draw=none](-7.971,2.717)--(-7.922,2.832)--(-7.907,2.825)--(-7.958,2.707)--cycle;
\draw(-7.971,2.717)--(-7.922,2.832)--(-7.907,2.825)--(-7.958,2.707);
\filldraw[fill opacity=0.8,fill=gray!20,draw=none](-7.923,2.832)--(-7.919,2.831)--(-7.922,2.832)--cycle;
\draw(-7.919,2.831)--(-7.922,2.832);
\filldraw[fill opacity=0.8,fill=gray!20](-7.918,2.939)--(-7.944,2.978)--(-7.982,3.001)--(-8.027,3.006)--(-8.072,2.991)--(-8.11,2.959)--(-8.136,2.915)--(-8.145,2.865)--(-8.136,2.817)--(-8.11,2.778)--(-8.072,2.754)--(-8.027,2.749)--(-7.982,2.764)--(-7.944,2.796)--(-7.918,2.84)--(-7.909,2.89)--cycle;
\filldraw[fill opacity=0.8,fill=gray!20](-7.918,2.939)--(-7.944,2.978)--(-7.982,3.001)--(-8.027,3.006)--(-8.072,2.991)--(-8.11,2.959)--(-8.136,2.915)--(-8.145,2.865)--(-8.136,2.817)--(-8.11,2.778)--(-8.072,2.754)--(-8.027,2.749)--(-7.982,2.764)--(-7.944,2.796)--(-7.918,2.84)--(-7.909,2.89)--cycle;
\filldraw[fill opacity=0.8,fill=gray!20,draw=none](-6.009,1.083)--(-6.001,1.098)--(-6.038,1.114)--cycle;
\draw(-6.009,1.083)--(-6.001,1.098)--(-6.038,1.114);
\filldraw[fill opacity=0.8,fill=gray!20,draw=none](-6.065,1.135)--(-6.077,1.102)--(-6.036,1.114)--(-6.036,1.135)--cycle;
\draw(-6.036,1.114)--(-6.036,1.135);
\filldraw[fill opacity=0.8,fill=gray!20](-8.027,2.699)--(-8.074,2.706)--(-8.066,2.711)--(-8.027,2.699)--cycle;
\filldraw[fill opacity=0.5,fill=gray!20](-10.591,2.308)--(-10.626,2.31)--(-10.254,2.662)--(-10.211,2.668)--cycle;
\filldraw[fill opacity=0.8,fill=gray!20,draw=none](-7.932,2.836)--(-7.911,2.827)--(-7.907,2.825)--(-7.919,2.831)--(-7.923,2.832)--(-8.131,2.923)--(-8.128,2.921)--(-8.102,2.91)--(-8.06,2.892)--(-8.013,2.871)--(-7.968,2.852)--cycle;
\draw(-8.128,2.921)--(-8.102,2.91)--(-8.06,2.892)--(-8.013,2.871)--(-7.968,2.852)--(-7.932,2.836)--(-7.911,2.827)--(-7.907,2.825)--(-7.919,2.831);
\filldraw[fill opacity=0.8,fill=gray!20](-7.625,3.889)--(-7.677,3.912)--(-7.659,3.923)--(-7.615,3.896)--cycle;
\filldraw[fill opacity=0.8,fill=gray!20,draw=none](-7.701,4.231)--(-7.71,4.229)--(-7.693,4.247)--cycle;
\draw(-7.701,4.231)--(-7.71,4.229);
\filldraw[fill opacity=0.8,fill=gray!20,draw=none](-8.79,1.037)--(-8.764,1.031)--(-8.76,1.026)--cycle;
\draw(-8.79,1.037)--(-8.764,1.031)--(-8.76,1.026);
\filldraw[fill opacity=0.5,fill=gray!20](-11.092,1.38)--(-11.092,1.394)--(-10.918,1.871)--(-10.912,1.872)--cycle;
\filldraw[fill opacity=0.8,fill=gray!20,draw=none](-6.086,1.197)--(-6.086,1.135)--(-6.036,1.135)--(-6.036,1.191)--cycle;
\draw(-6.036,1.135)--(-6.036,1.191)--(-6.086,1.197)--(-6.086,1.135);
\filldraw[fill opacity=0.8,fill=gray!20,draw=none](-8.18,2.813)--(-8.18,2.812)--(-8.183,2.814)--cycle;
\filldraw[fill opacity=0.8,fill=gray!20,draw=none](-8.19,2.843)--(-8.19,2.842)--(-8.208,2.881)--(-8.198,2.888)--cycle;
\draw(-8.19,2.843)--(-8.19,2.842);
\draw(-8.208,2.881)--(-8.198,2.888);
\filldraw[fill opacity=0.8,fill=gray!20,draw=none](-8.189,2.84)--(-8.19,2.842)--(-8.19,2.843)--cycle;
\draw(-8.19,2.842)--(-8.19,2.843);
\filldraw[fill opacity=0.8,fill=gray!20,draw=none](-8.189,2.84)--(-8.179,2.812)--(-8.2,2.826)--(-8.203,2.834)--(-8.19,2.842)--cycle;
\draw(-8.2,2.826)--(-8.203,2.834)--(-8.19,2.842);
\filldraw[fill opacity=0.8,fill=gray!20,draw=none](-8.18,2.813)--(-8.183,2.814)--(-8.19,2.82)--(-8.181,2.841)--(-8.18,2.836)--cycle;
\draw(-8.19,2.82)--(-8.181,2.841);
\filldraw[fill opacity=0.8,fill=gray!20,draw=none](-7.731,4.198)--(-7.735,4.202)--(-7.726,4.226)--(-7.71,4.229)--cycle;
\draw(-7.735,4.202)--(-7.726,4.226)--(-7.71,4.229);
\filldraw[fill opacity=0.8,fill=gray!20](-8.027,2.699)--(-7.985,2.71)--(-7.98,2.705)--(-8.027,2.699)--cycle;
\filldraw[fill opacity=0.8,fill=gray!20,draw=none](-8.19,2.842)--(-8.203,2.834)--(-8.21,2.88)--(-8.208,2.881)--cycle;
\draw(-8.19,2.842)--(-8.203,2.834)--(-8.21,2.88)--(-8.208,2.881);
\filldraw[fill opacity=0.8,fill=gray!20](-7.568,3.881)--(-7.615,3.896)--(-7.593,3.9)--(-7.568,3.881)--cycle;
\filldraw[fill opacity=0.8,fill=gray!20](-7.371,4.117)--(-7.378,4.172)--(-7.356,4.148)--(-7.349,4.093)--cycle;
\filldraw[fill opacity=0.8,fill=gray!20,draw=none](-8.127,3.015)--(-8.142,3.009)--(-8.135,3.019)--(-8.115,3.023)--cycle;
\draw(-8.142,3.009)--(-8.135,3.019)--(-8.115,3.023);
\filldraw[fill opacity=0.8,fill=gray!20,draw=none](-8.15,2.999)--(-8.142,3.009)--(-8.127,3.015)--cycle;
\draw(-8.15,2.999)--(-8.142,3.009);
\filldraw[fill opacity=0.8,fill=gray!20,draw=none](-8.165,2.983)--(-8.137,3.018)--(-8.135,3.019)--(-8.15,2.999)--cycle;
\draw(-8.137,3.018)--(-8.135,3.019)--(-8.15,2.999);
\filldraw[fill opacity=0.8,fill=gray!20,draw=none](-7.693,4.247)--(-7.71,4.229)--(-7.726,4.226)--(-7.711,4.247)--(-7.685,4.266)--cycle;
\draw(-7.71,4.229)--(-7.726,4.226)--(-7.711,4.247);
\filldraw[fill opacity=0.8,fill=gray!20](-7.568,3.881)--(-7.536,3.899)--(-7.517,3.894)--(-7.568,3.881)--cycle;
\filldraw[fill opacity=0.8,fill=gray!20](-8.068,3.05)--(-8.048,3.059)--(-8.024,3.06)--(-8.021,3.052)--cycle;
\filldraw[fill opacity=0.8,fill=gray!20](-8.021,3.052)--(-8.024,3.06)--(-8.001,3.059)--(-7.977,3.049)--cycle;
\filldraw[fill opacity=0.8,fill=gray!20](-8.119,3.076)--(-8.074,3.091)--(-8.052,3.096)--(-8.076,3.085)--cycle;
\filldraw[fill opacity=0.8,fill=gray!20,draw=none](-8.191,2.817)--(-8.19,2.82)--(-8.183,2.814)--cycle;
\draw(-8.191,2.817)--(-8.19,2.82);
\filldraw[fill opacity=0.8,fill=gray!20,draw=none](-7.427,4.229)--(-7.454,4.254)--(-7.456,4.269)--(-7.429,4.262)--(-7.398,4.221)--cycle;
\draw(-7.456,4.269)--(-7.429,4.262)--(-7.398,4.221)--(-7.427,4.229);
\filldraw[fill opacity=0.8,fill=gray!20](-7.517,3.894)--(-7.47,3.921)--(-7.458,3.909)--(-7.511,3.888)--cycle;
\filldraw[fill opacity=0.8,fill=gray!20](-9.058,1.018)--(-9.013,1.051)--(-8.995,1.063)--(-9.033,1.034)--cycle;
\filldraw[fill opacity=0.8,fill=gray!20](-8.903,.65)--(-8.954,.652)--(-8.96,.658)--(-8.903,.65)--cycle;
\filldraw[fill opacity=0.8,fill=gray!20](-8.954,.652)--(-9.001,.668)--(-9.013,.68)--(-8.96,.658)--cycle;
\filldraw[fill opacity=0.8,fill=gray!20,draw=none](-8.797,.68)--(-8.788,.684)--(-8.784,.684)--(-8.794,.677)--cycle;
\draw(-8.784,.684)--(-8.794,.677)--(-8.797,.68);
\filldraw[fill opacity=0.8,fill=gray!20,draw=none](-7.454,4.254)--(-7.475,4.274)--(-7.456,4.269)--cycle;
\draw(-7.475,4.274)--(-7.456,4.269);
\filldraw[fill opacity=0.8,fill=gray!20](-8.027,2.663)--(-8.078,2.665)--(-8.084,2.671)--(-8.027,2.663)--cycle;
\filldraw[fill opacity=0.8,fill=gray!20](-8.078,2.665)--(-8.125,2.681)--(-8.137,2.693)--(-8.084,2.671)--cycle;
\filldraw[fill opacity=0.8,fill=gray!20,draw=none](-8.175,3.036)--(-8.179,3.034)--(-8.137,3.065)--(-8.119,3.076)--(-8.157,3.048)--cycle;
\draw(-8.179,3.034)--(-8.137,3.065)--(-8.119,3.076)--(-8.157,3.048)--(-8.175,3.036);
\filldraw[fill opacity=0.8,fill=gray!20,draw=none](-8.071,3.049)--(-8.068,3.05)--(-8.07,3.048)--cycle;
\draw(-8.071,3.049)--(-8.068,3.05)--(-8.07,3.048);
\filldraw[fill opacity=0.8,fill=gray!20](-7.885,2.982)--(-7.911,3.017)--(-7.898,3.002)--(-7.869,2.965)--cycle;
\filldraw[fill opacity=0.8,fill=gray!20](-9.001,.668)--(-9.042,.698)--(-9.058,.715)--(-9.013,.68)--cycle;
\filldraw[fill opacity=0.8,fill=gray!20](-8.843,1.07)--(-8.872,1.082)--(-8.852,1.077)--(-8.805,1.061)--cycle;
\filldraw[fill opacity=0.8,fill=gray!20](-8.903,.65)--(-8.935,.647)--(-8.954,.652)--(-8.903,.65)--cycle;
\filldraw[fill opacity=0.8,fill=gray!20](-8.903,.65)--(-8.907,.645)--(-8.935,.647)--(-8.903,.65)--cycle;
\filldraw[fill opacity=0.8,fill=gray!20](-8.903,.65)--(-8.878,.646)--(-8.907,.645)--(-8.903,.65)--cycle;
\filldraw[fill opacity=0.8,fill=gray!20](-8.903,.65)--(-8.856,.65)--(-8.878,.646)--(-8.903,.65)--cycle;
\filldraw[fill opacity=0.8,fill=gray!20](-8.903,.65)--(-8.847,.657)--(-8.856,.65)--(-8.903,.65)--cycle;
\filldraw[fill opacity=0.8,fill=gray!20](-8.125,2.681)--(-8.166,2.711)--(-8.182,2.728)--(-8.137,2.693)--cycle;
\filldraw[fill opacity=0.8,fill=gray!20](-8.027,2.663)--(-8.03,2.658)--(-8.058,2.66)--(-8.027,2.663)--cycle;
\filldraw[fill opacity=0.8,fill=gray!20](-8.027,2.663)--(-8.058,2.66)--(-8.078,2.665)--(-8.027,2.663)--cycle;
\filldraw[fill opacity=0.8,fill=gray!20,draw=none](-7.981,2.664)--(-7.986,2.668)--(-7.978,2.665)--cycle;
\filldraw[fill opacity=0.8,fill=gray!20,draw=none](-8.027,2.663)--(-7.986,2.668)--(-7.978,2.665)--(-7.98,2.664)--(-8.027,2.663)--cycle;
\draw(-7.978,2.665)--(-7.98,2.664)--(-8.027,2.663)--(-8.027,2.663)--(-7.986,2.668);
\filldraw[fill opacity=0.8,fill=gray!20](-8.027,2.663)--(-7.98,2.664)--(-8.002,2.66)--(-8.027,2.663)--cycle;
\filldraw[fill opacity=0.8,fill=gray!20](-8.027,2.663)--(-8.002,2.66)--(-8.03,2.658)--(-8.027,2.663)--cycle;
\filldraw[fill opacity=0.8,fill=gray!20,draw=none](-7.456,4.269)--(-7.475,4.274)--(-7.476,4.274)--(-7.5,4.299)--(-7.47,4.292)--(-7.458,4.284)--cycle;
\draw(-7.456,4.269)--(-7.475,4.274);
\draw(-7.5,4.299)--(-7.47,4.292)--(-7.458,4.284);
\filldraw[fill opacity=0.8,fill=gray!20,draw=none](-7.476,4.274)--(-7.491,4.285)--(-7.507,4.301)--(-7.5,4.299)--cycle;
\draw(-7.491,4.285)--(-7.507,4.301)--(-7.5,4.299);
\filldraw[fill opacity=0.8,fill=gray!20](-7.466,4.163)--(-7.505,4.187)--(-7.554,4.197)--(-7.605,4.191)--(-7.651,4.171)--(-7.683,4.139)--(-7.699,4.101)--(-7.694,4.062)--(-7.67,4.028)--(-7.63,4.005)--(-7.581,3.995)--(-7.53,4)--(-7.485,4.02)--(-7.452,4.052)--(-7.437,4.09)--(-7.442,4.129)--cycle;
\filldraw[fill opacity=0.8,fill=gray!20,draw=none](-7.693,4.247)--(-7.685,4.266)--(-7.68,4.269)--(-7.672,4.271)--cycle;
\draw(-7.68,4.269)--(-7.672,4.271);
\filldraw[fill opacity=0.8,fill=gray!20,draw=none](-8.198,2.888)--(-8.198,2.912)--(-8.19,2.935)--(-8.19,2.935)--cycle;
\draw(-8.19,2.935)--(-8.19,2.935);
\filldraw[fill opacity=0.8,fill=gray!20,draw=none](-8.198,2.888)--(-8.21,2.88)--(-8.203,2.926)--(-8.198,2.93)--cycle;
\draw(-8.198,2.888)--(-8.21,2.88)--(-8.203,2.926)--(-8.198,2.93);
\filldraw[fill opacity=0.8,fill=gray!20,draw=none](-8.165,2.983)--(-8.166,2.982)--(-8.167,2.981)--cycle;
\draw(-8.166,2.982)--(-8.167,2.981);
\filldraw[fill opacity=0.8,fill=gray!20,draw=none](-8.189,2.938)--(-8.175,2.976)--(-8.166,2.982)--cycle;
\draw(-8.175,2.976)--(-8.166,2.982);
\filldraw[fill opacity=0.8,fill=gray!20,draw=none](-8.071,3.049)--(-8.073,3.054)--(-8.066,3.056)--(-8.048,3.059)--(-8.068,3.05)--cycle;
\draw(-8.073,3.054)--(-8.066,3.056)--(-8.048,3.059)--(-8.068,3.05)--(-8.071,3.049);
\filldraw[fill opacity=0.8,fill=gray!20,draw=none](-8.165,2.983)--(-8.167,2.981)--(-8.185,2.969)--(-8.156,3.006)--(-8.137,3.018)--cycle;
\draw(-8.167,2.981)--(-8.185,2.969)--(-8.156,3.006)--(-8.137,3.018);
\filldraw[fill opacity=0.8,fill=gray!20](-9.042,.698)--(-9.073,.739)--(-9.093,.76)--(-9.058,.715)--cycle;
\filldraw[fill opacity=0.8,fill=gray!20,draw=none](-8.847,.657)--(-8.809,.671)--(-8.815,.664)--(-8.856,.65)--cycle;
\draw(-8.815,.664)--(-8.856,.65)--(-8.847,.657)--(-8.809,.671);
\filldraw[fill opacity=0.8,fill=gray!20,draw=none](-8.18,2.836)--(-8.181,2.841)--(-8.179,2.845)--cycle;
\draw(-8.181,2.841)--(-8.179,2.845);
\filldraw[fill opacity=0.8,fill=gray!20](-7.378,4.172)--(-7.398,4.221)--(-7.378,4.2)--(-7.356,4.148)--cycle;
\filldraw[fill opacity=0.8,fill=gray!20](-8.166,2.711)--(-8.197,2.752)--(-8.217,2.773)--(-8.182,2.728)--cycle;
\filldraw[fill opacity=0.8,fill=gray!20](-7.97,2.67)--(-7.917,2.691)--(-7.935,2.679)--(-7.98,2.664)--cycle;
\filldraw[fill opacity=0.8,fill=gray!20,draw=none](-8.788,.684)--(-8.778,.689)--(-8.784,.684)--cycle;
\draw(-8.778,.689)--(-8.784,.684);
\filldraw[fill opacity=0.8,fill=gray!20,draw=none](-8.071,3.049)--(-8.103,3.043)--(-8.073,3.054)--cycle;
\draw(-8.071,3.049)--(-8.103,3.043)--(-8.073,3.054);
\filldraw[fill opacity=0.8,fill=gray!20](-8.069,2.701)--(-8.109,2.714)--(-8.118,2.724)--(-8.074,2.706)--cycle;
\filldraw[fill opacity=0.8,fill=gray!20](-8.027,2.699)--(-8.069,2.701)--(-8.074,2.706)--(-8.027,2.699)--cycle;
\filldraw[fill opacity=0.8,fill=gray!20,draw=none](-8.156,3.006)--(-8.13,3.025)--(-8.12,3.03)--(-8.135,3.019)--cycle;
\draw(-8.12,3.03)--(-8.135,3.019)--(-8.156,3.006)--(-8.13,3.025);
\filldraw[fill opacity=0.8,fill=gray!20](-7.758,4.205)--(-7.723,4.249)--(-7.697,4.266)--(-7.726,4.226)--cycle;
\filldraw[fill opacity=0.8,fill=gray!20,draw=none](-8.19,2.935)--(-8.19,2.935)--(-8.189,2.938)--cycle;
\draw(-8.19,2.935)--(-8.19,2.935);
\filldraw[fill opacity=0.8,fill=gray!20](-8.027,2.699)--(-8.03,2.695)--(-8.053,2.697)--(-8.027,2.699)--cycle;
\filldraw[fill opacity=0.8,fill=gray!20](-8.027,2.699)--(-8.053,2.697)--(-8.069,2.701)--(-8.027,2.699)--cycle;
\filldraw[fill opacity=0.8,fill=gray!20](-8.109,2.714)--(-8.143,2.739)--(-8.156,2.753)--(-8.118,2.724)--cycle;
\filldraw[fill opacity=0.8,fill=gray!20](-8.027,2.699)--(-7.98,2.705)--(-7.987,2.699)--(-8.027,2.699)--cycle;
\filldraw[fill opacity=0.8,fill=gray!20](-8.027,2.699)--(-7.987,2.699)--(-8.006,2.696)--(-8.027,2.699)--cycle;
\filldraw[fill opacity=0.8,fill=gray!20](-8.027,2.699)--(-8.006,2.696)--(-8.03,2.695)--(-8.027,2.699)--cycle;
\filldraw[fill opacity=0.8,fill=gray!20](-7.977,3.049)--(-8.001,3.059)--(-7.985,3.055)--(-7.945,3.041)--cycle;
\filldraw[fill opacity=0.8,fill=gray!20](-7.568,3.881)--(-7.625,3.889)--(-7.615,3.896)--(-7.568,3.881)--cycle;
\filldraw[fill opacity=0.8,fill=gray!20,draw=none](-7.515,4.297)--(-7.499,4.293)--(-7.491,4.285)--cycle;
\draw(-7.499,4.293)--(-7.491,4.285);
\filldraw[fill opacity=0.8,fill=gray!20,draw=none](-8.189,2.938)--(-8.19,2.935)--(-8.203,2.926)--(-8.185,2.969)--(-8.175,2.976)--cycle;
\draw(-8.19,2.935)--(-8.203,2.926)--(-8.185,2.969)--(-8.175,2.976);
\filldraw[fill opacity=0.8,fill=gray!20,draw=none](-8.809,.671)--(-8.794,.677)--(-8.812,.666)--(-8.815,.664)--cycle;
\draw(-8.809,.671)--(-8.794,.677)--(-8.812,.666)--(-8.815,.664);
\filldraw[fill opacity=0.8,fill=gray!20,draw=none](-8.198,2.912)--(-8.198,2.93)--(-8.19,2.935)--cycle;
\draw(-8.198,2.93)--(-8.19,2.935);
\filldraw[fill opacity=0.8,fill=gray!20,draw=none](-8.743,1.005)--(-8.76,1.026)--(-8.764,1.031)--(-8.759,1.025)--cycle;
\draw(-8.76,1.026)--(-8.764,1.031)--(-8.759,1.025);
\filldraw[fill opacity=0.8,fill=gray!20,draw=none](-6.036,1.135)--(-6.036,1.114)--(-5.98,1.143)--cycle;
\draw(-6.036,1.135)--(-6.036,1.114);
\filldraw[fill opacity=0.8,fill=gray!20,draw=none](-8.986,.995)--(-8.981,1.004)--(-8.983,.999)--cycle;
\draw(-8.981,1.004)--(-8.983,.999);
\filldraw[fill opacity=0.8,fill=gray!20,draw=none](-8.953,1.077)--(-8.29,2.602)--(-8.285,2.602)--(-8.946,1.085)--cycle;
\draw(-8.953,1.077)--(-8.29,2.602);
\draw(-8.285,2.602)--(-8.946,1.085);
\filldraw[fill opacity=0.8,fill=gray!20,draw=none](-5.954,1.161)--(-5.925,1.157)--(-5.925,1.183)--cycle;
\draw(-5.925,1.157)--(-5.925,1.183);
\filldraw[fill opacity=0.8,fill=gray!20,draw=none](-5.954,1.161)--(-5.98,1.143)--(-5.925,1.157)--cycle;
\filldraw[fill opacity=0.8,fill=gray!20](-9.073,.739)--(-9.093,.788)--(-9.115,.812)--(-9.093,.76)--cycle;
\filldraw[fill opacity=0.8,fill=gray!20,draw=none](-8.794,.677)--(-8.784,.684)--(-8.773,.695)--(-8.774,.694)--(-8.812,.666)--cycle;
\draw(-8.773,.695)--(-8.774,.694)--(-8.812,.666)--(-8.794,.677)--(-8.784,.684);
\filldraw[fill opacity=0.8,fill=gray!20,draw=none](-8.29,2.602)--(-8.25,2.694)--(-8.237,2.712)--(-8.285,2.602)--cycle;
\draw(-8.29,2.602)--(-8.25,2.694);
\draw(-8.237,2.712)--(-8.285,2.602);
\filldraw[fill opacity=0.8,fill=gray!20,draw=none](-8.25,2.694)--(-8.225,2.749)--(-8.213,2.768)--(-8.237,2.712)--cycle;
\draw(-8.25,2.694)--(-8.225,2.749);
\draw(-8.213,2.768)--(-8.237,2.712);
\filldraw[fill opacity=0.8,fill=gray!20,draw=none](-8.214,2.77)--(-8.213,2.778)--(-8.224,2.79)--(-8.217,2.773)--cycle;
\draw(-8.224,2.79)--(-8.217,2.773)--(-8.214,2.77);
\filldraw[fill opacity=0.8,fill=gray!20,draw=none](-8.225,2.749)--(-8.215,2.773)--(-8.213,2.768)--cycle;
\draw(-8.225,2.749)--(-8.215,2.773);
\filldraw[fill opacity=0.8,fill=gray!20,draw=none](-8.215,2.773)--(-8.214,2.777)--(-8.207,2.782)--(-8.213,2.768)--cycle;
\draw(-8.215,2.773)--(-8.214,2.777);
\draw(-8.207,2.782)--(-8.213,2.768);
\filldraw[fill opacity=0.8,fill=gray!20,draw=none](-8.213,2.777)--(-8.213,2.778)--(-8.205,2.795)--(-8.194,2.813)--(-8.195,2.81)--(-8.207,2.782)--cycle;
\draw(-8.213,2.778)--(-8.205,2.795);
\draw(-8.195,2.81)--(-8.207,2.782);
\filldraw[fill opacity=0.8,fill=gray!20,draw=none](-8.194,2.822)--(-8.198,2.826)--(-8.2,2.826)--(-8.194,2.813)--cycle;
\draw(-8.2,2.826)--(-8.194,2.813);
\filldraw[fill opacity=0.8,fill=gray!20,draw=none](-8.205,2.795)--(-8.193,2.825)--(-8.19,2.82)--(-8.191,2.817)--cycle;
\draw(-8.205,2.795)--(-8.193,2.825);
\draw(-8.19,2.82)--(-8.191,2.817);
\filldraw[fill opacity=0.8,fill=gray!20,draw=none](-8.193,2.825)--(-8.179,2.857)--(-8.181,2.841)--(-8.19,2.82)--cycle;
\draw(-8.193,2.825)--(-8.179,2.857);
\draw(-8.181,2.841)--(-8.19,2.82);
\filldraw[fill opacity=0.8,fill=gray!20,draw=none](-8.18,2.851)--(-8.17,2.876)--(-8.179,2.845)--(-8.181,2.841)--cycle;
\draw(-8.179,2.845)--(-8.181,2.841);
\filldraw[fill opacity=0.8,fill=gray!20](-8.143,2.739)--(-8.169,2.773)--(-8.185,2.79)--(-8.156,2.753)--cycle;
\filldraw[fill opacity=0.8,fill=gray!20,draw=none](-7.981,2.664)--(-7.978,2.665)--(-7.977,2.664)--(-7.978,2.662)--cycle;
\draw(-7.977,2.664)--(-7.978,2.662);
\filldraw[fill opacity=0.8,fill=gray!20,draw=none](-7.958,2.707)--(-7.956,2.712)--(-7.96,2.709)--cycle;
\draw(-7.958,2.707)--(-7.956,2.712);
\filldraw[fill opacity=0.8,fill=gray!20,draw=none](-7.98,2.705)--(-7.963,2.711)--(-7.948,2.714)--(-7.951,2.712)--(-7.987,2.699)--cycle;
\draw(-7.948,2.714)--(-7.951,2.712)--(-7.987,2.699)--(-7.98,2.705)--(-7.963,2.711);
\filldraw[fill opacity=0.8,fill=gray!20,draw=none](-8.743,1.005)--(-8.759,1.025)--(-8.748,1.014)--(-8.73,.99)--cycle;
\draw(-8.759,1.025)--(-8.748,1.014)--(-8.73,.99);
\filldraw[fill opacity=0.8,fill=gray!20](-8.764,1.031)--(-8.805,1.061)--(-8.794,1.048)--(-8.748,1.014)--cycle;
\filldraw[fill opacity=0.8,fill=gray!20](-7.917,2.691)--(-7.872,2.724)--(-7.897,2.707)--(-7.935,2.679)--cycle;
\filldraw[fill opacity=0.8,fill=gray!20,draw=none](-8.781,.687)--(-8.784,.684)--(-8.778,.689)--(-8.76,.703)--cycle;
\draw(-8.784,.684)--(-8.778,.689);
\filldraw[fill opacity=0.8,fill=gray!20,draw=none](-7.672,4.271)--(-7.68,4.269)--(-7.661,4.28)--cycle;
\draw(-7.672,4.271)--(-7.68,4.269);
\filldraw[fill opacity=0.8,fill=gray!20](-7.568,3.881)--(-7.517,3.894)--(-7.511,3.888)--(-7.568,3.881)--cycle;
\filldraw[fill opacity=0.8,fill=gray!20,draw=none](-6.036,1.191)--(-6.036,1.135)--(-5.98,1.143)--(-5.98,1.19)--cycle;
\draw(-5.98,1.143)--(-5.98,1.19)--(-6.036,1.191)--(-6.036,1.135);
\filldraw[fill opacity=0.8,fill=gray!20,draw=none](-7.956,2.712)--(-7.954,2.715)--(-7.936,2.722)--(-7.948,2.714)--cycle;
\draw(-7.954,2.715)--(-7.936,2.722)--(-7.948,2.714);
\filldraw[fill opacity=0.8,fill=gray!20,draw=none](-8.213,2.778)--(-8.213,2.791)--(-8.217,2.801)--(-8.239,2.825)--(-8.224,2.79)--cycle;
\draw(-8.213,2.791)--(-8.217,2.801)--(-8.239,2.825)--(-8.224,2.79);
\filldraw[fill opacity=0.8,fill=gray!20,draw=none](-5.98,1.19)--(-5.98,1.165)--(-5.954,1.161)--(-5.925,1.183)--(-5.925,1.193)--cycle;
\draw(-5.925,1.183)--(-5.925,1.193)--(-5.98,1.19)--(-5.98,1.165);
\filldraw[fill opacity=0.8,fill=gray!20,draw=none](-8.183,2.808)--(-8.178,2.819)--(-8.193,2.825)--(-8.194,2.822)--cycle;
\draw(-8.183,2.808)--(-8.178,2.819);
\draw(-8.193,2.825)--(-8.194,2.822);
\filldraw[fill opacity=0.8,fill=gray!20,draw=none](-8.169,2.773)--(-8.185,2.814)--(-8.194,2.824)--(-8.194,2.813)--(-8.185,2.79)--cycle;
\draw(-8.194,2.813)--(-8.185,2.79)--(-8.169,2.773)--(-8.185,2.814)--(-8.194,2.824);
\filldraw[fill opacity=0.8,fill=gray!20](-7.936,2.722)--(-7.898,2.75)--(-7.919,2.736)--(-7.951,2.712)--cycle;
\filldraw[fill opacity=0.8,fill=gray!20,draw=none](-8.13,3.025)--(-8.118,3.033)--(-8.103,3.043)--(-8.12,3.03)--cycle;
\draw(-8.13,3.025)--(-8.118,3.033)--(-8.103,3.043)--(-8.12,3.03);
\filldraw[fill opacity=0.8,fill=gray!20](-7.911,3.017)--(-7.945,3.041)--(-7.936,3.031)--(-7.898,3.002)--cycle;
\filldraw[fill opacity=0.5,fill=gray!20](-9,2.518)--(-9,2.719)--(-8.57,2.704)--(-8.618,2.505)--cycle;
\filldraw[fill opacity=0.8,fill=gray!20,draw=none](-7.685,4.266)--(-7.684,4.269)--(-7.68,4.269)--cycle;
\draw(-7.684,4.269)--(-7.68,4.269);
\filldraw[fill opacity=0.8,fill=gray!20,draw=none](-7.661,4.28)--(-7.68,4.269)--(-7.684,4.269)--(-7.673,4.285)--(-7.659,4.294)--(-7.636,4.299)--cycle;
\draw(-7.68,4.269)--(-7.684,4.269);
\draw(-7.673,4.285)--(-7.659,4.294)--(-7.636,4.299);
\filldraw[fill opacity=0.8,fill=gray!20,draw=none](-5.98,1.165)--(-5.98,1.143)--(-5.954,1.161)--cycle;
\draw(-5.98,1.165)--(-5.98,1.143);
\filldraw[fill opacity=0.8,fill=gray!20,draw=none](-7.685,4.266)--(-7.711,4.247)--(-7.697,4.266)--(-7.684,4.269)--cycle;
\draw(-7.711,4.247)--(-7.697,4.266)--(-7.684,4.269);
\filldraw[fill opacity=0.8,fill=gray!20,draw=none](-7.456,4.269)--(-7.458,4.284)--(-7.429,4.262)--cycle;
\draw(-7.458,4.284)--(-7.429,4.262)--(-7.456,4.269);
\filldraw[fill opacity=0.8,fill=gray!20,draw=none](-9.093,.788)--(-9.099,.834)--(-9.118,.837)--(-9.115,.812)--cycle;
\draw(-9.118,.837)--(-9.115,.812)--(-9.093,.788)--(-9.099,.834);
\filldraw[fill opacity=0.8,fill=gray!20,draw=none](-9.004,.96)--(-8.996,.98)--(-8.986,.995)--cycle;
\draw(-9.004,.96)--(-8.996,.98);
\filldraw[fill opacity=0.8,fill=gray!20,draw=none](-8.694,.831)--(-8.686,.848)--(-8.684,.861)--(-8.687,.86)--cycle;
\draw(-8.686,.848)--(-8.684,.861)--(-8.687,.86);
\filldraw[fill opacity=0.8,fill=gray!20,draw=none](-8.694,.829)--(-8.694,.831)--(-8.687,.86)--(-8.692,.856)--cycle;
\draw(-8.687,.86)--(-8.692,.856);
\filldraw[fill opacity=0.8,fill=gray!20,draw=none](-8.647,.836)--(-8.59,.849)--(-8.666,.879)--(-8.731,.864)--cycle;
\draw(-8.647,.836)--(-8.59,.849);
\draw(-8.666,.879)--(-8.731,.864);
\filldraw[fill opacity=0.8,fill=gray!20,draw=none](-7.515,4.297)--(-7.525,4.303)--(-7.507,4.301)--(-7.499,4.293)--cycle;
\draw(-7.525,4.303)--(-7.507,4.301)--(-7.499,4.293);
\filldraw[fill opacity=0.8,fill=gray!20](-7.571,3.19)--(-7.56,3.19)--(-7.339,3.776)--cycle;
\filldraw[fill opacity=0.8,fill=gray!20,draw=none](-8.217,2.801)--(-8.222,2.847)--(-8.242,2.85)--(-8.239,2.825)--cycle;
\draw(-8.242,2.85)--(-8.239,2.825)--(-8.217,2.801)--(-8.222,2.847);
\filldraw[fill opacity=0.8,fill=gray!20](-7.872,2.724)--(-7.837,2.768)--(-7.868,2.748)--(-7.897,2.707)--cycle;
\filldraw[fill opacity=0.8,fill=gray!20,draw=none](-7.684,4.269)--(-7.697,4.266)--(-7.673,4.285)--cycle;
\draw(-7.684,4.269)--(-7.697,4.266)--(-7.673,4.285);
\filldraw[fill opacity=0.8,fill=gray!20,draw=none](-7.661,4.28)--(-7.636,4.299)--(-7.619,4.302)--cycle;
\draw(-7.636,4.299)--(-7.619,4.302);
\filldraw[fill opacity=0.8,fill=gray!20,draw=none](-7.536,4.303)--(-7.525,4.303)--(-7.515,4.297)--cycle;
\draw(-7.536,4.303)--(-7.525,4.303);
\filldraw[fill opacity=0.8,fill=gray!20](-8.935,.647)--(-8.964,.659)--(-9.001,.668)--(-8.954,.652)--cycle;
\filldraw[fill opacity=0.8,fill=gray!20,draw=none](-8.197,2.827)--(-8.203,2.834)--(-8.2,2.826)--cycle;
\draw(-8.197,2.827)--(-8.203,2.834)--(-8.2,2.826);
\filldraw[fill opacity=0.8,fill=gray!20](-7.398,4.221)--(-7.429,4.262)--(-7.413,4.245)--(-7.378,4.2)--cycle;
\filldraw[fill opacity=0.8,fill=gray!20,draw=none](-8.116,3.399)--(-8.122,3.399)--(-7.984,3.765)--cycle;
\draw(-8.122,3.399)--(-7.984,3.765)--(-8.116,3.399);
\filldraw[fill opacity=0.8,fill=gray!20](-8.058,2.66)--(-8.087,2.672)--(-8.125,2.681)--(-8.078,2.665)--cycle;
\filldraw[fill opacity=0.8,fill=gray!20,draw=none](-8.194,2.824)--(-8.189,2.845)--(-8.191,2.86)--(-8.21,2.88)--(-8.203,2.834)--cycle;
\draw(-8.189,2.845)--(-8.191,2.86)--(-8.21,2.88)--(-8.203,2.834)--(-8.194,2.824);
\filldraw[fill opacity=0.8,fill=gray!20](-7.898,2.75)--(-7.869,2.786)--(-7.895,2.769)--(-7.919,2.736)--cycle;
\filldraw[fill opacity=0.8,fill=gray!20,draw=none](-9.099,.834)--(-9.1,.843)--(-9.122,.867)--(-9.118,.837)--cycle;
\draw(-9.099,.834)--(-9.1,.843)--(-9.122,.867)--(-9.118,.837);
\filldraw[fill opacity=0.8,fill=gray!20,draw=none](-7.525,4.303)--(-7.536,4.303)--(-7.562,4.308)--(-7.564,4.314)--cycle;
\draw(-7.525,4.303)--(-7.536,4.303);
\draw(-7.562,4.308)--(-7.564,4.314);
\filldraw[fill opacity=0.8,fill=gray!20,draw=none](-8.202,2.764)--(-8.183,2.808)--(-8.194,2.822)--(-8.213,2.778)--cycle;
\draw(-8.202,2.764)--(-8.183,2.808);
\draw(-8.194,2.822)--(-8.213,2.778);
\filldraw[fill opacity=0.8,fill=gray!20,draw=none](-8.214,2.77)--(-8.197,2.752)--(-8.213,2.791)--cycle;
\draw(-8.214,2.77)--(-8.197,2.752)--(-8.213,2.791);
\filldraw[fill opacity=0.8,fill=gray!20,draw=none](-8.222,2.847)--(-8.223,2.856)--(-8.246,2.88)--(-8.242,2.85)--cycle;
\draw(-8.222,2.847)--(-8.223,2.856)--(-8.246,2.88)--(-8.242,2.85);
\filldraw[fill opacity=0.8,fill=gray!20,draw=none](-7.978,2.665)--(-7.977,2.665)--(-7.977,2.664)--cycle;
\draw(-7.977,2.665)--(-7.977,2.664);
\filldraw[fill opacity=0.8,fill=gray!20,draw=none](-7.616,3.209)--(-7.611,3.209)--(-7.473,3.574)--cycle;
\draw(-7.611,3.209)--(-7.473,3.574)--(-7.616,3.209);
\filldraw[fill opacity=0.8,fill=gray!20](-9.013,1.051)--(-8.96,1.072)--(-8.951,1.078)--(-8.995,1.063)--cycle;
\filldraw[fill opacity=0.8,fill=gray!20](-8.053,2.697)--(-8.077,2.706)--(-8.109,2.714)--(-8.069,2.701)--cycle;
\filldraw[fill opacity=0.8,fill=gray!20,draw=none](-8.137,3.065)--(-8.108,3.076)--(-8.116,3.077)--(-8.119,3.076)--cycle;
\draw(-8.116,3.077)--(-8.119,3.076)--(-8.137,3.065)--(-8.108,3.076);
\filldraw[fill opacity=0.8,fill=gray!20](-8.856,.65)--(-8.812,.666)--(-8.854,.657)--(-8.878,.646)--cycle;
\filldraw[fill opacity=0.8,fill=gray!20,draw=none](-7.525,4.303)--(-7.564,4.314)--(-7.564,4.315)--(-7.536,4.313)--(-7.507,4.301)--cycle;
\draw(-7.564,4.314)--(-7.564,4.315)--(-7.536,4.313)--(-7.507,4.301)--(-7.525,4.303);
\filldraw[fill opacity=0.8,fill=gray!20,draw=none](-8.695,1.016)--(-7.977,2.665)--(-7.985,2.657)--(-8.71,.991)--cycle;
\draw(-8.695,1.016)--(-7.977,2.665);
\draw(-7.985,2.657)--(-8.71,.991);
\filldraw[fill opacity=0.8,fill=gray!20,draw=none](-8.695,1.016)--(-8.704,1)--(-8.719,.97)--cycle;
\filldraw[fill opacity=0.8,fill=gray!20,draw=none](-8.108,3.076)--(-8.084,3.085)--(-8.074,3.091)--(-8.116,3.077)--cycle;
\draw(-8.108,3.076)--(-8.084,3.085)--(-8.074,3.091)--(-8.116,3.077);
\filldraw[fill opacity=0.8,fill=gray!20,draw=none](-7.977,2.665)--(-7.958,2.707)--(-7.96,2.709)--(-7.964,2.705)--(-7.98,2.668)--cycle;
\draw(-7.977,2.665)--(-7.958,2.707);
\draw(-7.964,2.705)--(-7.98,2.668);
\filldraw[fill opacity=0.8,fill=gray!20,draw=none](-7.987,2.699)--(-7.96,2.709)--(-7.961,2.71)--(-7.986,2.705)--(-8.006,2.696)--cycle;
\draw(-7.961,2.71)--(-7.986,2.705)--(-8.006,2.696)--(-7.987,2.699)--(-7.96,2.709);
\filldraw[fill opacity=0.8,fill=gray!20](-8.006,2.696)--(-7.986,2.705)--(-8.032,2.703)--(-8.03,2.695)--cycle;
\filldraw[fill opacity=0.8,fill=gray!20,draw=none](-7.961,2.71)--(-7.96,2.715)--(-7.976,2.717)--(-7.986,2.705)--cycle;
\draw(-7.976,2.717)--(-7.986,2.705)--(-7.961,2.71);
\filldraw[fill opacity=0.8,fill=gray!20,draw=none](-7.986,2.705)--(-7.976,2.717)--(-7.982,2.723)--(-7.99,2.725)--(-8.035,2.723)--(-8.032,2.703)--cycle;
\draw(-7.99,2.725)--(-8.035,2.723)--(-8.032,2.703)--(-7.986,2.705)--(-7.976,2.717);
\filldraw[fill opacity=0.8,fill=gray!20,draw=none](-7.991,2.67)--(-7.984,2.668)--(-7.98,2.668)--(-7.964,2.704)--(-7.97,2.71)--(-7.988,2.708)--(-8,2.68)--cycle;
\draw(-7.98,2.668)--(-7.964,2.704);
\draw(-7.988,2.708)--(-8,2.68);
\filldraw[fill opacity=0.8,fill=gray!20](-7.98,2.664)--(-7.935,2.679)--(-7.978,2.671)--(-8.002,2.66)--cycle;
\filldraw[fill opacity=0.8,fill=gray!20,draw=none](-7.619,4.302)--(-7.622,4.302)--(-7.615,4.303)--cycle;
\draw(-7.619,4.302)--(-7.622,4.302);
\filldraw[fill opacity=0.8,fill=gray!20,draw=none](-7.977,2.665)--(-7.98,2.668)--(-7.985,2.657)--cycle;
\draw(-7.98,2.668)--(-7.985,2.657);
\filldraw[fill opacity=0.8,fill=gray!20,draw=none](-7.956,2.712)--(-7.907,2.825)--(-7.911,2.827)--(-7.964,2.705)--cycle;
\draw(-7.956,2.712)--(-7.907,2.825)--(-7.911,2.827)--(-7.964,2.705);
\filldraw[fill opacity=0.5,fill=gray!20](-11.014,.291)--(-11.065,.324)--(-11.128,.835)--(-11.077,.806)--cycle;
\filldraw[fill opacity=0.8,fill=gray!20,draw=none](-8.118,3.033)--(-8.074,3.051)--(-8.07,3.053)--(-8.073,3.054)--(-8.103,3.043)--cycle;
\draw(-8.073,3.054)--(-8.103,3.043)--(-8.118,3.033)--(-8.074,3.051)--(-8.07,3.053);
\filldraw[fill opacity=0.8,fill=gray!20,draw=none](-8.781,.687)--(-8.76,.703)--(-8.757,.705)--(-8.773,.695)--cycle;
\draw(-8.757,.705)--(-8.773,.695);
\filldraw[fill opacity=0.8,fill=gray!20,draw=none](-7.622,4.302)--(-7.659,4.294)--(-7.615,4.31)--(-7.593,4.314)--(-7.615,4.303)--cycle;
\draw(-7.622,4.302)--(-7.659,4.294)--(-7.615,4.31)--(-7.593,4.314)--(-7.615,4.303);
\filldraw[fill opacity=0.8,fill=gray!20,draw=none](-7.615,4.303)--(-7.605,4.308)--(-7.564,4.308)--cycle;
\draw(-7.615,4.303)--(-7.605,4.308);
\filldraw[fill opacity=0.8,fill=gray!20](-9.1,.843)--(-9.093,.899)--(-9.115,.923)--(-9.122,.867)--cycle;
\filldraw[fill opacity=0.8,fill=gray!20,draw=none](-8.954,.744)--(-8.987,.764)--(-9.003,.772)--(-8.956,.745)--cycle;
\draw(-8.954,.744)--(-8.987,.764)--(-9.003,.772);
\filldraw[fill opacity=0.8,fill=gray!20,draw=none](-7.564,4.308)--(-7.562,4.308)--(-7.562,4.308)--cycle;
\draw(-7.562,4.308)--(-7.562,4.308);
\filldraw[fill opacity=0.8,fill=gray!20](-8.223,2.856)--(-8.217,2.912)--(-8.239,2.936)--(-8.246,2.88)--cycle;
\filldraw[fill opacity=0.8,fill=gray!20](-7.837,2.768)--(-7.815,2.819)--(-7.85,2.797)--(-7.868,2.748)--cycle;
\filldraw[fill opacity=0.8,fill=gray!20,draw=none](-7.723,4.249)--(-7.677,4.283)--(-7.665,4.291)--(-7.673,4.285)--(-7.697,4.266)--cycle;
\draw(-7.673,4.285)--(-7.697,4.266)--(-7.723,4.249)--(-7.677,4.283)--(-7.665,4.291);
\filldraw[fill opacity=0.8,fill=gray!20](-7.568,3.881)--(-7.619,3.883)--(-7.625,3.889)--(-7.568,3.881)--cycle;
\filldraw[fill opacity=0.8,fill=gray!20](-7.619,3.883)--(-7.666,3.899)--(-7.677,3.912)--(-7.625,3.889)--cycle;
\filldraw[fill opacity=0.8,fill=gray!20,draw=none](-7.564,4.308)--(-7.605,4.308)--(-7.593,4.314)--(-7.564,4.315)--(-7.562,4.308)--cycle;
\draw(-7.605,4.308)--(-7.593,4.314)--(-7.564,4.315)--(-7.562,4.308);
\filldraw[fill opacity=0.8,fill=gray!20,draw=none](-7.5,4.299)--(-7.507,4.301)--(-7.536,4.313)--(-7.521,4.309)--cycle;
\draw(-7.5,4.299)--(-7.507,4.301)--(-7.536,4.313)--(-7.521,4.309);
\filldraw[fill opacity=0.8,fill=gray!20](-7.666,3.899)--(-7.707,3.929)--(-7.723,3.946)--(-7.677,3.912)--cycle;
\filldraw[fill opacity=0.8,fill=gray!20](-7.568,3.881)--(-7.599,3.878)--(-7.619,3.883)--(-7.568,3.881)--cycle;
\filldraw[fill opacity=0.8,fill=gray!20](-7.568,3.881)--(-7.571,3.876)--(-7.599,3.878)--(-7.568,3.881)--cycle;
\filldraw[fill opacity=0.8,fill=gray!20](-7.568,3.881)--(-7.542,3.878)--(-7.571,3.876)--(-7.568,3.881)--cycle;
\filldraw[fill opacity=0.8,fill=gray!20](-7.568,3.881)--(-7.52,3.882)--(-7.542,3.878)--(-7.568,3.881)--cycle;
\filldraw[fill opacity=0.8,fill=gray!20](-7.568,3.881)--(-7.511,3.888)--(-7.52,3.882)--(-7.568,3.881)--cycle;
\filldraw[fill opacity=0.8,fill=gray!20](-8.191,2.86)--(-8.185,2.907)--(-8.203,2.926)--(-8.21,2.88)--cycle;
\filldraw[fill opacity=0.8,fill=gray!20](-7.869,2.786)--(-7.851,2.829)--(-7.879,2.81)--(-7.895,2.769)--cycle;
\filldraw[fill opacity=0.8,fill=gray!20](-8.929,1.082)--(-8.903,1.079)--(-8.903,1.079)--(-8.9,1.084)--cycle;
\filldraw[fill opacity=0.8,fill=gray!20](-8.9,1.084)--(-8.903,1.079)--(-8.903,1.079)--(-8.872,1.082)--cycle;
\filldraw[fill opacity=0.8,fill=gray!20](-8.052,3.096)--(-8.027,3.092)--(-8.027,3.092)--(-8.024,3.097)--cycle;
\filldraw[fill opacity=0.8,fill=gray!20,draw=none](-7.616,3.209)--(-7.473,3.574)--(-7.482,3.574)--(-7.62,3.209)--cycle;
\draw(-7.616,3.209)--(-7.473,3.574)--(-7.482,3.574)--(-7.62,3.209);
\filldraw[fill opacity=0.8,fill=gray!20](-8.805,1.061)--(-8.852,1.077)--(-8.847,1.071)--(-8.794,1.048)--cycle;
\filldraw[fill opacity=0.8,fill=gray!20](-8.024,3.06)--(-8.027,3.056)--(-8.027,3.056)--(-8.001,3.059)--cycle;
\filldraw[fill opacity=0.8,fill=gray!20](-8.048,3.059)--(-8.027,3.056)--(-8.027,3.056)--(-8.024,3.06)--cycle;
\filldraw[fill opacity=0.8,fill=gray!20](-7.339,3.776)--(-7.35,3.776)--(-7.571,3.19)--cycle;
\filldraw[fill opacity=0.8,fill=gray!20](-7.707,3.929)--(-7.738,3.97)--(-7.758,3.991)--(-7.723,3.946)--cycle;
\filldraw[fill opacity=0.8,fill=gray!20](-7.511,3.888)--(-7.458,3.909)--(-7.476,3.897)--(-7.52,3.882)--cycle;
\filldraw[fill opacity=0.8,fill=gray!20](-7.945,3.041)--(-7.985,3.055)--(-7.98,3.049)--(-7.936,3.031)--cycle;
\filldraw[fill opacity=0.5,fill=gray!20](-11.128,.835)--(-11.153,.865)--(-11.092,1.38)--(-11.065,1.359)--cycle;
\filldraw[fill opacity=0.8,fill=gray!20,draw=none](-8.18,2.851)--(-8.179,2.857)--(-8.168,2.88)--(-8.17,2.876)--cycle;
\draw(-8.179,2.857)--(-8.168,2.88);
\filldraw[fill opacity=0.8,fill=gray!20](-8.951,1.078)--(-8.903,1.079)--(-8.903,1.079)--(-8.929,1.082)--cycle;
\filldraw[fill opacity=0.8,fill=gray!20,draw=none](-8.905,.994)--(-8.92,1)--(-8.968,.987)--(-8.961,.983)--cycle;
\draw(-8.905,.994)--(-8.92,1);
\draw(-8.968,.987)--(-8.961,.983);
\filldraw[fill opacity=0.8,fill=gray!20,draw=none](-8.76,.703)--(-8.756,.705)--(-8.757,.705)--cycle;
\draw(-8.756,.705)--(-8.757,.705);
\filldraw[fill opacity=0.8,fill=gray!20](-8.074,3.091)--(-8.027,3.092)--(-8.027,3.092)--(-8.052,3.096)--cycle;
\filldraw[fill opacity=0.8,fill=gray!20](-8.066,3.056)--(-8.027,3.056)--(-8.027,3.056)--(-8.048,3.059)--cycle;
\filldraw[fill opacity=0.8,fill=gray!20,draw=none](-7.5,4.299)--(-7.521,4.309)--(-7.517,4.308)--(-7.47,4.292)--cycle;
\draw(-7.521,4.309)--(-7.517,4.308)--(-7.47,4.292)--(-7.5,4.299);
\filldraw[fill opacity=0.8,fill=gray!20,draw=none](-8.751,.716)--(-8.757,.705)--(-8.756,.705)--(-8.733,.738)--cycle;
\draw(-8.757,.705)--(-8.756,.705);
\filldraw[fill opacity=0.8,fill=gray!20,draw=none](-8.194,2.822)--(-8.194,2.824)--(-8.197,2.827)--(-8.198,2.826)--cycle;
\draw(-8.194,2.824)--(-8.197,2.827);
\filldraw[fill opacity=0.8,fill=gray!20,draw=none](-9.021,.851)--(-9.012,.803)--(-8.987,.764)--(-8.954,.744)--(-8.903,.864)--(-9.008,.91)--(-9.012,.902)--cycle;
\draw(-9.008,.91)--(-9.012,.902)--(-9.021,.851)--(-9.012,.803)--(-8.987,.764)--(-8.954,.744);
\filldraw[fill opacity=0.8,fill=gray!20,draw=none](-8.903,.864)--(-8.916,.989)--(-8.948,.978)--(-8.987,.946)--(-9.008,.91)--cycle;
\draw(-8.916,.989)--(-8.948,.978)--(-8.987,.946)--(-9.008,.91);
\filldraw[fill opacity=0.8,fill=gray!20](-8.872,1.082)--(-8.903,1.079)--(-8.903,1.079)--(-8.852,1.077)--cycle;
\filldraw[fill opacity=0.8,fill=gray!20](-8.907,.645)--(-8.91,.655)--(-8.964,.659)--(-8.935,.647)--cycle;
\filldraw[fill opacity=0.8,fill=gray!20](-9.093,.899)--(-9.073,.953)--(-9.093,.974)--(-9.115,.923)--cycle;
\filldraw[fill opacity=0.8,fill=gray!20](-8.185,2.907)--(-8.169,2.951)--(-8.185,2.969)--(-8.203,2.926)--cycle;
\filldraw[fill opacity=0.8,fill=gray!20](-8.03,2.695)--(-8.032,2.703)--(-8.077,2.706)--(-8.053,2.697)--cycle;
\filldraw[fill opacity=0.8,fill=gray!20](-8.217,2.912)--(-8.197,2.966)--(-8.217,2.987)--(-8.239,2.936)--cycle;
\filldraw[fill opacity=0.8,fill=gray!20](-8.03,2.658)--(-8.034,2.668)--(-8.087,2.672)--(-8.058,2.66)--cycle;
\filldraw[fill opacity=0.8,fill=gray!20](-8.001,3.059)--(-8.027,3.056)--(-8.027,3.056)--(-7.985,3.055)--cycle;
\filldraw[fill opacity=0.8,fill=gray!20](-7.851,2.829)--(-7.844,2.875)--(-7.874,2.856)--(-7.879,2.81)--cycle;
\filldraw[fill opacity=0.8,fill=gray!20](-7.815,2.819)--(-7.808,2.875)--(-7.844,2.851)--(-7.85,2.797)--cycle;
\filldraw[fill opacity=0.8,fill=gray!20,draw=none](-8.757,.705)--(-8.74,.738)--(-8.745,.734)--(-8.774,.694)--cycle;
\draw(-8.74,.738)--(-8.745,.734)--(-8.774,.694)--(-8.757,.705);
\filldraw[fill opacity=0.8,fill=gray!20,draw=none](-8.694,.828)--(-8.694,.831)--(-8.694,.829)--cycle;
\filldraw[fill opacity=0.8,fill=gray!20,draw=none](-8.687,.827)--(-8.647,.836)--(-8.731,.864)--(-8.785,.852)--cycle;
\draw(-8.687,.827)--(-8.647,.836);
\draw(-8.731,.864)--(-8.785,.852);
\filldraw[fill opacity=0.8,fill=gray!20](-7.738,3.97)--(-7.758,4.02)--(-7.78,4.043)--(-7.758,3.991)--cycle;
\filldraw[fill opacity=0.8,fill=gray!20](-7.458,3.909)--(-7.413,3.942)--(-7.438,3.926)--(-7.476,3.897)--cycle;
\filldraw[fill opacity=0.8,fill=gray!20](-8.878,.646)--(-8.854,.657)--(-8.91,.655)--(-8.907,.645)--cycle;
\filldraw[fill opacity=0.8,fill=gray!20,draw=none](-7.429,4.262)--(-7.458,4.284)--(-7.458,4.28)--(-7.413,4.245)--cycle;
\draw(-7.458,4.28)--(-7.413,4.245)--(-7.429,4.262)--(-7.458,4.284);
\filldraw[fill opacity=0.8,fill=gray!20](-8.002,2.66)--(-7.978,2.671)--(-8.034,2.668)--(-8.03,2.658)--cycle;
\filldraw[fill opacity=0.8,fill=gray!20,draw=none](-8.175,3.036)--(-8.182,3.031)--(-8.179,3.034)--cycle;
\draw(-8.175,3.036)--(-8.182,3.031)--(-8.179,3.034);
\filldraw[fill opacity=0.8,fill=gray!20](-8.964,.659)--(-8.989,.684)--(-9.042,.698)--(-9.001,.668)--cycle;
\filldraw[fill opacity=0.8,fill=gray!20](-8.077,2.706)--(-8.098,2.728)--(-8.143,2.739)--(-8.109,2.714)--cycle;
\filldraw[fill opacity=0.8,fill=gray!20,draw=none](-8.717,.974)--(-8.717,.973)--(-8.743,1.005)--(-8.733,.994)--cycle;
\filldraw[fill opacity=0.8,fill=gray!20](-8.087,2.672)--(-8.113,2.698)--(-8.166,2.711)--(-8.125,2.681)--cycle;
\filldraw[fill opacity=0.5,fill=gray!20](-9.911,2.297)--(-9.738,2.221)--(-9.382,2.421)--(-9.555,2.497)--cycle;
\filldraw[fill opacity=0.8,fill=gray!20,draw=none](-7.96,2.709)--(-7.951,2.712)--(-7.961,2.71)--cycle;
\draw(-7.96,2.709)--(-7.951,2.712)--(-7.961,2.71);
\filldraw[fill opacity=0.8,fill=gray!20,draw=none](-7.665,4.291)--(-7.659,4.294)--(-7.673,4.285)--cycle;
\draw(-7.665,4.291)--(-7.659,4.294)--(-7.673,4.285);
\filldraw[fill opacity=0.8,fill=gray!20,draw=none](-8.07,3.053)--(-8.066,3.056)--(-8.073,3.054)--cycle;
\draw(-8.07,3.053)--(-8.066,3.056)--(-8.073,3.054);
\filldraw[fill opacity=0.8,fill=gray!20,draw=none](-8.17,2.876)--(-8.168,2.88)--(-8.165,2.889)--cycle;
\draw(-8.168,2.88)--(-8.165,2.889);
\filldraw[fill opacity=0.8,fill=gray!20](-8.169,2.951)--(-8.143,2.991)--(-8.156,3.006)--(-8.185,2.969)--cycle;
\filldraw[fill opacity=0.8,fill=gray!20](-7.844,2.875)--(-7.851,2.922)--(-7.879,2.903)--(-7.874,2.856)--cycle;
\filldraw[fill opacity=0.8,fill=gray!20](-9.073,.953)--(-9.042,1.001)--(-9.058,1.018)--(-9.093,.974)--cycle;
\filldraw[fill opacity=0.8,fill=gray!20](-7.758,4.02)--(-7.764,4.074)--(-7.787,4.099)--(-7.78,4.043)--cycle;
\filldraw[fill opacity=0.8,fill=gray!20](-7.413,3.942)--(-7.378,3.986)--(-7.409,3.966)--(-7.438,3.926)--cycle;
\filldraw[fill opacity=0.8,fill=gray!20,draw=none](-8.213,2.777)--(-8.214,2.777)--(-8.213,2.778)--cycle;
\draw(-8.214,2.777)--(-8.213,2.778);
\filldraw[fill opacity=0.8,fill=gray!20](-8.197,2.966)--(-8.166,3.014)--(-8.182,3.031)--(-8.217,2.987)--cycle;
\filldraw[fill opacity=0.8,fill=gray!20](-7.808,2.875)--(-7.815,2.93)--(-7.85,2.908)--(-7.844,2.851)--cycle;
\filldraw[fill opacity=0.5,fill=gray!20](-10.175,-.504)--(-10.28,-.548)--(-10.557,-.185)--(-10.439,-.157)--cycle;
\filldraw[fill opacity=0.8,fill=gray!20](-8.074,3.051)--(-8.027,3.056)--(-8.027,3.056)--(-8.066,3.056)--cycle;
\filldraw[fill opacity=0.8,fill=gray!20,draw=none](-9.002,.953)--(-8.996,.978)--(-9.004,.96)--cycle;
\draw(-8.996,.978)--(-9.004,.96);
\filldraw[fill opacity=0.8,fill=gray!20,draw=none](-8.952,1.077)--(-8.945,1.074)--(-8.903,1.079)--(-8.903,1.079)--(-8.951,1.078)--cycle;
\draw(-8.945,1.074)--(-8.903,1.079)--(-8.903,1.079)--(-8.951,1.078)--(-8.952,1.077);
\filldraw[fill opacity=0.8,fill=gray!20,draw=none](-8.939,1.069)--(-8.903,1.079)--(-8.903,1.079)--(-8.945,1.074)--cycle;
\draw(-8.939,1.069)--(-8.903,1.079)--(-8.903,1.079)--(-8.945,1.074);
\filldraw[fill opacity=0.8,fill=gray!20,draw=none](-9.002,.953)--(-8.997,.938)--(-8.935,1.08)--(-8.953,1.077)--(-8.996,.978)--cycle;
\draw(-8.997,.938)--(-8.935,1.08);
\draw(-8.953,1.077)--(-8.996,.978);
\filldraw[fill opacity=0.8,fill=gray!20,draw=none](-8.935,1.08)--(-8.202,2.764)--(-8.213,2.778)--(-8.953,1.077)--cycle;
\draw(-8.935,1.08)--(-8.202,2.764);
\draw(-8.213,2.778)--(-8.953,1.077);
\filldraw[fill opacity=0.8,fill=gray!20,draw=none](-7.96,2.715)--(-7.911,2.827)--(-7.932,2.836)--(-7.984,2.718)--cycle;
\draw(-7.96,2.715)--(-7.911,2.827)--(-7.932,2.836)--(-7.984,2.718);
\filldraw[fill opacity=0.8,fill=gray!20,draw=none](-7.961,2.71)--(-7.951,2.712)--(-7.919,2.736)--(-7.957,2.728)--cycle;
\draw(-7.961,2.71)--(-7.951,2.712)--(-7.919,2.736)--(-7.957,2.728);
\filldraw[fill opacity=0.8,fill=gray!20,draw=none](-8.178,2.819)--(-8.162,2.856)--(-8.179,2.857)--(-8.193,2.825)--cycle;
\draw(-8.178,2.819)--(-8.162,2.856);
\draw(-8.179,2.857)--(-8.193,2.825);
\filldraw[fill opacity=0.8,fill=gray!20,draw=none](-8.162,2.856)--(-8.149,2.886)--(-8.168,2.88)--(-8.179,2.857)--cycle;
\draw(-8.162,2.856)--(-8.149,2.886);
\draw(-8.168,2.88)--(-8.179,2.857);
\filldraw[fill opacity=0.8,fill=gray!20,draw=none](-8.149,2.886)--(-8.137,2.913)--(-8.137,2.915)--(-8.165,2.889)--(-8.168,2.88)--cycle;
\draw(-8.149,2.886)--(-8.137,2.913);
\draw(-8.165,2.889)--(-8.168,2.88);
\filldraw[fill opacity=0.8,fill=gray!20](-7.599,3.878)--(-7.628,3.89)--(-7.666,3.899)--(-7.619,3.883)--cycle;
\filldraw[fill opacity=0.8,fill=gray!20](-8.084,3.085)--(-8.027,3.092)--(-8.027,3.092)--(-8.074,3.091)--cycle;
\filldraw[fill opacity=0.8,fill=gray!20,draw=none](-8.704,1)--(-8.71,.991)--(-8.719,.97)--cycle;
\draw(-8.71,.991)--(-8.719,.97);
\filldraw[fill opacity=0.8,fill=gray!20](-8.812,.666)--(-8.774,.694)--(-8.834,.682)--(-8.854,.657)--cycle;
\filldraw[fill opacity=0.8,fill=gray!20](-7.935,2.679)--(-7.897,2.707)--(-7.958,2.696)--(-7.978,2.671)--cycle;
\filldraw[fill opacity=0.8,fill=gray!20](-7.985,3.055)--(-8.027,3.056)--(-8.027,3.056)--(-7.98,3.049)--cycle;
\filldraw[fill opacity=0.8,fill=gray!20,draw=none](-8.952,1.077)--(-8.96,1.072)--(-8.945,1.074)--cycle;
\draw(-8.952,1.077)--(-8.96,1.072)--(-8.945,1.074);
\filldraw[fill opacity=0.8,fill=gray!20,draw=none](-8.687,.827)--(-8.785,.852)--(-8.844,.839)--cycle;
\draw(-8.785,.852)--(-8.844,.839);
\filldraw[fill opacity=0.8,fill=gray!20](-7.677,4.283)--(-7.625,4.303)--(-7.615,4.31)--(-7.659,4.294)--cycle;
\filldraw[fill opacity=0.8,fill=gray!20](-8.852,1.077)--(-8.903,1.079)--(-8.903,1.079)--(-8.847,1.071)--cycle;
\filldraw[fill opacity=0.8,fill=gray!20](-8.143,2.991)--(-8.109,3.023)--(-8.118,3.033)--(-8.156,3.006)--cycle;
\filldraw[fill opacity=0.8,fill=gray!20](-7.851,2.922)--(-7.869,2.965)--(-7.895,2.948)--(-7.879,2.903)--cycle;
\filldraw[fill opacity=0.8,fill=gray!20,draw=none](-8.751,.716)--(-8.733,.738)--(-8.728,.745)--(-8.74,.738)--cycle;
\draw(-8.728,.745)--(-8.74,.738);
\filldraw[fill opacity=0.8,fill=gray!20](-7.52,3.882)--(-7.476,3.897)--(-7.519,3.889)--(-7.542,3.878)--cycle;
\filldraw[fill opacity=0.8,fill=gray!20,draw=none](-8.733,.738)--(-8.726,.746)--(-8.728,.745)--cycle;
\draw(-8.726,.746)--(-8.728,.745);
\filldraw[fill opacity=0.8,fill=gray!20,draw=none](-7.458,4.284)--(-7.47,4.292)--(-7.458,4.28)--cycle;
\draw(-7.458,4.284)--(-7.47,4.292)--(-7.458,4.28);
\filldraw[fill opacity=0.8,fill=gray!20,draw=none](-8.724,.759)--(-8.728,.745)--(-8.726,.746)--(-8.711,.783)--cycle;
\draw(-8.728,.745)--(-8.726,.746);
\filldraw[fill opacity=0.8,fill=gray!20](-9.042,1.001)--(-9.001,1.039)--(-9.013,1.051)--(-9.058,1.018)--cycle;
\filldraw[fill opacity=0.8,fill=gray!20,draw=none](-8.694,.828)--(-8.694,.829)--(-8.711,.794)--(-8.701,.8)--cycle;
\draw(-8.711,.794)--(-8.701,.8);
\filldraw[fill opacity=0.8,fill=gray!20,draw=none](-8.92,.991)--(-8.916,.991)--(-8.896,.995)--cycle;
\draw(-8.92,.991)--(-8.916,.991)--(-8.896,.995);
\filldraw[fill opacity=0.8,fill=gray!20,draw=none](-8.911,.987)--(-8.916,.991)--(-8.92,.991)--cycle;
\draw(-8.911,.987)--(-8.916,.991)--(-8.92,.991);
\filldraw[fill opacity=0.8,fill=gray!20](-7.764,4.074)--(-7.758,4.13)--(-7.78,4.154)--(-7.787,4.099)--cycle;
\filldraw[fill opacity=0.8,fill=gray!20](-7.378,3.986)--(-7.356,4.037)--(-7.391,4.015)--(-7.409,3.966)--cycle;
\filldraw[fill opacity=0.8,fill=gray!20,draw=none](-8.703,.809)--(-8.694,.829)--(-8.692,.856)--(-8.7,.851)--cycle;
\draw(-8.692,.856)--(-8.7,.851);
\filldraw[fill opacity=0.8,fill=gray!20,draw=none](-8.194,2.824)--(-8.185,2.814)--(-8.189,2.845)--cycle;
\draw(-8.194,2.824)--(-8.185,2.814)--(-8.189,2.845);
\filldraw[fill opacity=0.8,fill=gray!20](-8.166,3.014)--(-8.125,3.052)--(-8.137,3.065)--(-8.182,3.031)--cycle;
\filldraw[fill opacity=0.8,fill=gray!20](-7.815,2.93)--(-7.837,2.982)--(-7.868,2.962)--(-7.85,2.908)--cycle;
\filldraw[fill opacity=0.8,fill=gray!20,draw=none](-8.711,.783)--(-8.701,.8)--(-8.704,.798)--cycle;
\draw(-8.701,.8)--(-8.704,.798);
\filldraw[fill opacity=0.8,fill=gray!20](-7.593,4.314)--(-7.568,4.31)--(-7.568,4.31)--(-7.564,4.315)--cycle;
\filldraw[fill opacity=0.8,fill=gray!20](-7.564,4.315)--(-7.568,4.31)--(-7.568,4.31)--(-7.536,4.313)--cycle;
\filldraw[fill opacity=0.5,fill=gray!20](-10.084,2.612)--(-10.154,2.651)--(-9.708,2.901)--(-9.642,2.861)--cycle;
\filldraw[fill opacity=0.8,fill=gray!20](-8.098,2.728)--(-8.114,2.759)--(-8.169,2.773)--(-8.143,2.739)--cycle;
\filldraw[fill opacity=0.8,fill=gray!20,draw=none](-8.734,.741)--(-8.728,.745)--(-8.724,.759)--cycle;
\draw(-8.734,.741)--(-8.728,.745);
\filldraw[fill opacity=0.8,fill=gray!20](-8.109,3.023)--(-8.069,3.045)--(-8.074,3.051)--(-8.118,3.033)--cycle;
\filldraw[fill opacity=0.8,fill=gray!20](-7.869,2.965)--(-7.898,3.002)--(-7.919,2.988)--(-7.895,2.948)--cycle;
\filldraw[fill opacity=0.8,fill=gray!20,draw=none](-8.717,.974)--(-8.733,.994)--(-8.73,.99)--(-8.717,.974)--cycle;
\draw(-8.73,.99)--(-8.717,.974);
\filldraw[fill opacity=0.8,fill=gray!20,draw=none](-9.011,.911)--(-9.009,.91)--(-8.997,.938)--(-9.002,.953)--cycle;
\draw(-9.011,.911)--(-9.009,.91)--(-8.997,.938);
\filldraw[fill opacity=0.8,fill=gray!20,draw=none](-8.717,.974)--(-8.715,.97)--(-8.717,.973)--cycle;
\filldraw[fill opacity=0.8,fill=gray!20](-7.47,4.292)--(-7.517,4.308)--(-7.511,4.302)--(-7.458,4.28)--cycle;
\filldraw[fill opacity=0.8,fill=gray!20,draw=none](-8.715,.97)--(-8.717,.974)--(-8.717,.974)--(-8.713,.969)--cycle;
\draw(-8.717,.974)--(-8.713,.969);
\filldraw[fill opacity=0.8,fill=gray!20](-8.989,.684)--(-9.008,.723)--(-9.073,.739)--(-9.042,.698)--cycle;
\filldraw[fill opacity=0.8,fill=gray!20,draw=none](-8.714,.969)--(-8.715,.97)--(-8.713,.969)--cycle;
\draw(-8.713,.969)--(-8.714,.969);
\filldraw[fill opacity=0.8,fill=gray!20,draw=none](-8.714,.969)--(-8.713,.969)--(-8.713,.967)--cycle;
\draw(-8.714,.969)--(-8.713,.969)--(-8.713,.967);
\filldraw[fill opacity=0.8,fill=gray!20,draw=none](-8.724,.759)--(-8.711,.783)--(-8.704,.798)--(-8.715,.791)--cycle;
\draw(-8.704,.798)--(-8.715,.791);
\filldraw[fill opacity=0.8,fill=gray!20,draw=none](-8.704,.923)--(-8.713,.967)--(-8.713,.969)--(-8.722,.964)--cycle;
\draw(-8.713,.967)--(-8.713,.969)--(-8.722,.964);
\filldraw[fill opacity=0.8,fill=gray!20](-7.615,4.31)--(-7.568,4.31)--(-7.568,4.31)--(-7.593,4.314)--cycle;
\filldraw[fill opacity=0.8,fill=gray!20](-9.001,1.039)--(-8.954,1.066)--(-8.96,1.072)--(-9.013,1.051)--cycle;
\filldraw[fill opacity=0.8,fill=gray!20,draw=none](-8.726,.961)--(-8.713,.969)--(-8.717,.974)--cycle;
\draw(-8.726,.961)--(-8.713,.969)--(-8.717,.974);
\filldraw[fill opacity=0.8,fill=gray!20,draw=none](-8.114,2.759)--(-8.119,2.78)--(-8.149,2.802)--(-8.183,2.808)--(-8.169,2.773)--cycle;
\draw(-8.183,2.808)--(-8.169,2.773)--(-8.114,2.759)--(-8.119,2.78);
\filldraw[fill opacity=0.8,fill=gray!20,draw=none](-8.119,2.78)--(-8.108,2.781)--(-8.06,2.892)--(-8.102,2.91)--(-8.149,2.802)--cycle;
\draw(-8.108,2.781)--(-8.06,2.892)--(-8.102,2.91)--(-8.149,2.802);
\filldraw[fill opacity=0.8,fill=gray!20,draw=none](-8.055,2.755)--(-8.061,2.762)--(-8.121,2.786)--(-8.114,2.759)--cycle;
\draw(-8.121,2.786)--(-8.114,2.759)--(-8.055,2.755);
\filldraw[fill opacity=0.8,fill=gray!20,draw=none](-8.142,2.738)--(-8.13,2.732)--(-8.111,2.774)--(-8.119,2.78)--(-8.16,2.776)--(-8.168,2.758)--cycle;
\draw(-8.13,2.732)--(-8.111,2.774);
\draw(-8.16,2.776)--(-8.168,2.758);
\filldraw[fill opacity=0.8,fill=gray!20](-8.113,2.698)--(-8.132,2.736)--(-8.197,2.752)--(-8.166,2.711)--cycle;
\filldraw[fill opacity=0.5,fill=gray!20](-10.664,-.198)--(-10.756,-.194)--(-10.939,.268)--(-10.843,.256)--cycle;
\filldraw[fill opacity=0.8,fill=gray!20,draw=none](-8.719,.97)--(-8.717,.974)--(-8.735,.976)--(-8.745,.969)--(-8.761,.933)--cycle;
\draw(-8.719,.97)--(-8.717,.974);
\draw(-8.745,.969)--(-8.761,.933);
\filldraw[fill opacity=0.8,fill=gray!20,draw=none](-8.76,.921)--(-8.719,.97)--(-8.761,.933)--(-8.765,.922)--cycle;
\draw(-8.761,.933)--(-8.765,.922);
\filldraw[fill opacity=0.8,fill=gray!20,draw=none](-8.726,.961)--(-8.717,.974)--(-8.73,.99)--(-8.745,.949)--cycle;
\draw(-8.717,.974)--(-8.73,.99);
\draw(-8.745,.949)--(-8.726,.961);
\filldraw[fill opacity=0.8,fill=gray!20,draw=none](-8.717,.974)--(-8.71,.991)--(-8.735,.976)--cycle;
\draw(-8.717,.974)--(-8.71,.991);
\filldraw[fill opacity=0.8,fill=gray!20,draw=none](-8.735,.976)--(-8.742,.976)--(-8.745,.969)--cycle;
\draw(-8.742,.976)--(-8.745,.969);
\filldraw[fill opacity=0.8,fill=gray!20,draw=none](-8.736,.972)--(-8.73,.99)--(-8.748,1.014)--(-8.772,.998)--cycle;
\draw(-8.73,.99)--(-8.748,1.014)--(-8.772,.998);
\filldraw[fill opacity=0.8,fill=gray!20,draw=none](-8.735,.976)--(-8.71,.991)--(-7.98,2.667)--(-7.984,2.668)--(-8.007,2.665)--(-8.742,.976)--cycle;
\draw(-8.71,.991)--(-7.98,2.667);
\draw(-8.007,2.665)--(-8.742,.976);
\filldraw[fill opacity=0.8,fill=gray!20](-8.125,3.052)--(-8.078,3.079)--(-8.084,3.085)--(-8.137,3.065)--cycle;
\filldraw[fill opacity=0.8,fill=gray!20](-7.837,2.982)--(-7.872,3.027)--(-7.897,3.01)--(-7.868,2.962)--cycle;
\filldraw[fill opacity=0.8,fill=gray!20,draw=none](-8.697,.872)--(-8.695,.854)--(-8.692,.856)--cycle;
\draw(-8.695,.854)--(-8.692,.856);
\filldraw[fill opacity=0.8,fill=gray!20,draw=none](-7.98,2.667)--(-7.98,2.668)--(-7.984,2.668)--cycle;
\draw(-7.98,2.667)--(-7.98,2.668);
\filldraw[fill opacity=0.8,fill=gray!20,draw=none](-7.964,2.704)--(-7.961,2.711)--(-7.97,2.71)--cycle;
\draw(-7.964,2.704)--(-7.961,2.711);
\filldraw[fill opacity=0.8,fill=gray!20,draw=none](-7.961,2.711)--(-7.96,2.715)--(-7.984,2.718)--(-7.988,2.708)--cycle;
\draw(-7.961,2.711)--(-7.96,2.715);
\draw(-7.984,2.718)--(-7.988,2.708);
\filldraw[fill opacity=0.8,fill=gray!20,draw=none](-7.96,2.715)--(-7.957,2.728)--(-7.969,2.726)--(-7.976,2.717)--cycle;
\draw(-7.957,2.728)--(-7.969,2.726)--(-7.976,2.717);
\filldraw[fill opacity=0.8,fill=gray!20](-8.069,3.045)--(-8.027,3.056)--(-8.027,3.056)--(-8.074,3.051)--cycle;
\filldraw[fill opacity=0.8,fill=gray!20](-7.898,3.002)--(-7.936,3.031)--(-7.951,3.021)--(-7.919,2.988)--cycle;
\filldraw[fill opacity=0.8,fill=gray!20](-7.536,4.313)--(-7.568,4.31)--(-7.568,4.31)--(-7.517,4.308)--cycle;
\filldraw[fill opacity=0.8,fill=gray!20](-7.758,4.13)--(-7.738,4.184)--(-7.758,4.205)--(-7.78,4.154)--cycle;
\filldraw[fill opacity=0.8,fill=gray!20](-7.571,3.876)--(-7.574,3.886)--(-7.628,3.89)--(-7.599,3.878)--cycle;
\filldraw[fill opacity=0.8,fill=gray!20](-7.356,4.037)--(-7.349,4.093)--(-7.385,4.069)--(-7.391,4.015)--cycle;
\filldraw[fill opacity=0.8,fill=gray!20](-8.032,2.703)--(-8.035,2.723)--(-8.098,2.728)--(-8.077,2.706)--cycle;
\filldraw[fill opacity=0.8,fill=gray!20](-8.053,3.041)--(-8.027,3.056)--(-8.027,3.056)--(-8.069,3.045)--cycle;
\filldraw[fill opacity=0.8,fill=gray!20](-8.03,3.04)--(-8.027,3.056)--(-8.027,3.056)--(-8.053,3.041)--cycle;
\filldraw[fill opacity=0.8,fill=gray!20](-8.006,3.041)--(-8.027,3.056)--(-8.027,3.056)--(-8.03,3.04)--cycle;
\filldraw[fill opacity=0.8,fill=gray!20](-7.987,3.044)--(-8.027,3.056)--(-8.027,3.056)--(-8.006,3.041)--cycle;
\filldraw[fill opacity=0.8,fill=gray!20](-7.98,3.049)--(-8.027,3.056)--(-8.027,3.056)--(-7.987,3.044)--cycle;
\filldraw[fill opacity=0.8,fill=gray!20](-7.936,3.031)--(-7.98,3.049)--(-7.987,3.044)--(-7.951,3.021)--cycle;
\filldraw[fill opacity=0.8,fill=gray!20](-7.542,3.878)--(-7.519,3.889)--(-7.574,3.886)--(-7.571,3.876)--cycle;
\filldraw[fill opacity=0.8,fill=gray!20,draw=none](-8.734,.741)--(-8.724,.759)--(-8.715,.791)--(-8.726,.783)--(-8.745,.734)--cycle;
\draw(-8.715,.791)--(-8.726,.783)--(-8.745,.734)--(-8.734,.741);
\filldraw[fill opacity=0.5,fill=gray!20](-9.5,2.713)--(-9.572,2.797)--(-9.106,2.915)--(-9.05,2.827)--cycle;
\filldraw[fill opacity=0.8,fill=gray!20,draw=none](-8.711,.794)--(-8.703,.809)--(-8.7,.851)--(-8.72,.838)--(-8.726,.783)--cycle;
\draw(-8.7,.851)--(-8.72,.838)--(-8.726,.783)--(-8.711,.794);
\filldraw[fill opacity=0.8,fill=gray!20](-7.919,2.736)--(-7.895,2.769)--(-7.956,2.757)--(-7.969,2.726)--cycle;
\filldraw[fill opacity=0.8,fill=gray!20,draw=none](-8.954,1.066)--(-8.939,1.069)--(-8.945,1.074)--(-8.96,1.072)--cycle;
\draw(-8.945,1.074)--(-8.96,1.072)--(-8.954,1.066)--(-8.939,1.069);
\filldraw[fill opacity=0.8,fill=gray!20,draw=none](-8.761,.933)--(-8.738,.986)--(-8.775,.999)--(-8.812,.912)--cycle;
\draw(-8.761,.933)--(-8.738,.986);
\draw(-8.775,.999)--(-8.812,.912);
\filldraw[fill opacity=0.8,fill=gray!20](-8.748,1.014)--(-8.794,1.048)--(-8.812,1.037)--(-8.774,.997)--cycle;
\filldraw[fill opacity=0.8,fill=gray!20,draw=none](-7.991,2.67)--(-8,2.68)--(-8.036,2.69)--(-8.034,2.668)--cycle;
\draw(-8.036,2.69)--(-8.034,2.668)--(-7.991,2.67);
\filldraw[fill opacity=0.8,fill=gray!20,draw=none](-7.984,2.668)--(-8.003,2.673)--(-8.007,2.665)--cycle;
\draw(-8.003,2.673)--(-8.007,2.665);
\filldraw[fill opacity=0.8,fill=gray!20](-8.078,3.079)--(-8.027,3.092)--(-8.027,3.092)--(-8.084,3.085)--cycle;
\filldraw[fill opacity=0.8,fill=gray!20](-7.872,3.027)--(-7.917,3.062)--(-7.935,3.05)--(-7.897,3.01)--cycle;
\filldraw[fill opacity=0.8,fill=gray!20](-7.628,3.89)--(-7.653,3.916)--(-7.707,3.929)--(-7.666,3.899)--cycle;
\filldraw[fill opacity=0.8,fill=gray!20,draw=none](-8.11,3.394)--(-7.954,3.75)--(-7.984,3.765)--(-8.122,3.399)--cycle;
\draw(-8.11,3.394)--(-7.954,3.75)--(-7.984,3.765)--(-8.122,3.399);
\filldraw[fill opacity=0.8,fill=gray!20,draw=none](-8.695,.854)--(-8.697,.872)--(-8.703,.893)--(-8.726,.892)--(-8.72,.838)--cycle;
\draw(-8.726,.892)--(-8.72,.838)--(-8.695,.854);
\filldraw[fill opacity=0.8,fill=gray!20,draw=none](-8.697,.872)--(-8.698,.893)--(-8.703,.893)--cycle;
\filldraw[fill opacity=0.8,fill=gray!20,draw=none](-8.698,.893)--(-8.7,.912)--(-8.726,.894)--(-8.726,.892)--cycle;
\draw(-8.7,.912)--(-8.726,.894)--(-8.726,.892);
\filldraw[fill opacity=0.8,fill=gray!20](-8.91,.655)--(-8.913,.679)--(-8.989,.684)--(-8.964,.659)--cycle;
\filldraw[fill opacity=0.8,fill=gray!20,draw=none](-8.978,.897)--(-8.908,1.057)--(-8.91,1.059)--(-8.926,1.066)--(-8.939,1.069)--(-9.009,.91)--cycle;
\draw(-8.939,1.069)--(-9.009,.91)--(-8.978,.897)--(-8.908,1.057);
\filldraw[fill opacity=0.8,fill=gray!20](-8.907,1.059)--(-8.903,1.079)--(-8.903,1.079)--(-8.935,1.061)--cycle;
\filldraw[fill opacity=0.8,fill=gray!20,draw=none](-8.933,1.062)--(-8.903,1.079)--(-8.903,1.079)--(-8.939,1.069)--cycle;
\draw(-8.933,1.062)--(-8.903,1.079)--(-8.903,1.079)--(-8.939,1.069);
\filldraw[fill opacity=0.8,fill=gray!20](-8.847,1.071)--(-8.903,1.079)--(-8.903,1.079)--(-8.856,1.064)--cycle;
\filldraw[fill opacity=0.8,fill=gray!20](-8.878,1.06)--(-8.903,1.079)--(-8.903,1.079)--(-8.907,1.059)--cycle;
\filldraw[fill opacity=0.8,fill=gray!20](-8.856,1.064)--(-8.903,1.079)--(-8.903,1.079)--(-8.878,1.06)--cycle;
\filldraw[fill opacity=0.8,fill=gray!20](-8.794,1.048)--(-8.847,1.071)--(-8.856,1.064)--(-8.812,1.037)--cycle;
\filldraw[fill opacity=0.8,fill=gray!20,draw=none](-8.704,.923)--(-8.702,.91)--(-8.7,.912)--cycle;
\draw(-8.702,.91)--(-8.7,.912);
\filldraw[fill opacity=0.8,fill=gray!20,draw=none](-7.99,2.725)--(-8.017,2.742)--(-8.036,2.743)--(-8.035,2.723)--cycle;
\draw(-8.036,2.743)--(-8.035,2.723)--(-7.99,2.725);
\filldraw[fill opacity=0.8,fill=gray!20,draw=none](-8.036,2.69)--(-8,2.68)--(-7.983,2.721)--(-7.99,2.725)--(-8.02,2.733)--(-8.038,2.692)--cycle;
\draw(-8,2.68)--(-7.983,2.721);
\draw(-8.02,2.733)--(-8.038,2.692);
\filldraw[fill opacity=0.8,fill=gray!20](-8.034,2.668)--(-8.036,2.692)--(-8.113,2.698)--(-8.087,2.672)--cycle;
\filldraw[fill opacity=0.8,fill=gray!20](-8.058,3.074)--(-8.027,3.092)--(-8.027,3.092)--(-8.078,3.079)--cycle;
\filldraw[fill opacity=0.8,fill=gray!20](-8.03,3.072)--(-8.027,3.092)--(-8.027,3.092)--(-8.058,3.074)--cycle;
\filldraw[fill opacity=0.8,fill=gray!20](-8.002,3.074)--(-8.027,3.092)--(-8.027,3.092)--(-8.03,3.072)--cycle;
\filldraw[fill opacity=0.8,fill=gray!20](-7.98,3.078)--(-8.027,3.092)--(-8.027,3.092)--(-8.002,3.074)--cycle;
\filldraw[fill opacity=0.5,fill=gray!20](-10.882,1.862)--(-10.912,1.872)--(-10.626,2.31)--(-10.591,2.308)--cycle;
\filldraw[fill opacity=0.8,fill=gray!20](-8.774,.694)--(-8.745,.734)--(-8.818,.72)--(-8.834,.682)--cycle;
\filldraw[fill opacity=0.8,fill=gray!20,draw=none](-8.702,.91)--(-8.704,.923)--(-8.722,.964)--(-8.745,.949)--(-8.726,.894)--cycle;
\draw(-8.722,.964)--(-8.745,.949)--(-8.726,.894)--(-8.702,.91);
\filldraw[fill opacity=0.8,fill=gray!20](-7.897,2.707)--(-7.868,2.748)--(-7.942,2.733)--(-7.958,2.696)--cycle;
\filldraw[fill opacity=0.8,fill=gray!20](-7.738,4.184)--(-7.707,4.232)--(-7.723,4.249)--(-7.758,4.205)--cycle;
\filldraw[fill opacity=0.8,fill=gray!20](-7.349,4.093)--(-7.356,4.148)--(-7.391,4.126)--(-7.385,4.069)--cycle;
\filldraw[fill opacity=0.8,fill=gray!20](-8.173,3.418)--(-8.137,3.401)--(-7.916,3.988)--cycle;
\filldraw[fill opacity=0.8,fill=gray!20](-7.916,3.988)--(-7.952,4.005)--(-8.173,3.418)--cycle;
\filldraw[fill opacity=0.8,fill=gray!20](-7.625,4.303)--(-7.568,4.31)--(-7.568,4.31)--(-7.615,4.31)--cycle;
\filldraw[fill opacity=0.8,fill=gray!20](-8.854,.657)--(-8.834,.682)--(-8.913,.679)--(-8.91,.655)--cycle;
\filldraw[fill opacity=0.8,fill=gray!20,draw=none](-7.991,2.67)--(-7.978,2.671)--(-7.958,2.696)--(-8.012,2.693)--cycle;
\draw(-7.991,2.67)--(-7.978,2.671)--(-7.958,2.696)--(-8.012,2.693);
\filldraw[fill opacity=0.8,fill=gray!20](-7.476,3.897)--(-7.438,3.926)--(-7.498,3.914)--(-7.519,3.889)--cycle;
\filldraw[fill opacity=0.8,fill=gray!20,draw=none](-9.004,.908)--(-9.009,.91)--(-9.011,.911)--cycle;
\draw(-9.004,.908)--(-9.009,.91)--(-9.011,.911);
\filldraw[fill opacity=0.8,fill=gray!20,draw=none](-7.991,2.67)--(-8,2.68)--(-8.003,2.673)--cycle;
\draw(-8,2.68)--(-8.003,2.673);
\filldraw[fill opacity=0.8,fill=gray!20,draw=none](-8.131,2.923)--(-8.135,2.924)--(-8.132,2.923)--cycle;
\draw(-8.135,2.924)--(-8.132,2.923);
\filldraw[fill opacity=0.8,fill=gray!20,draw=none](-8.137,2.913)--(-8.132,2.923)--(-8.135,2.924)--cycle;
\draw(-8.137,2.913)--(-8.132,2.923)--(-8.135,2.924);
\filldraw[fill opacity=0.8,fill=gray!20](-9.008,.723)--(-9.02,.77)--(-9.093,.788)--(-9.073,.739)--cycle;
\filldraw[fill opacity=0.8,fill=gray!20](-7.517,4.308)--(-7.568,4.31)--(-7.568,4.31)--(-7.511,4.302)--cycle;
\filldraw[fill opacity=0.8,fill=gray!20,draw=none](-8.168,2.758)--(-8.162,2.772)--(-8.183,2.808)--(-8.202,2.764)--cycle;
\draw(-8.168,2.758)--(-8.162,2.772);
\draw(-8.183,2.808)--(-8.202,2.764);
\filldraw[fill opacity=0.8,fill=gray!20,draw=none](-8.162,2.772)--(-8.102,2.91)--(-8.132,2.923)--(-8.183,2.808)--cycle;
\draw(-8.162,2.772)--(-8.102,2.91)--(-8.132,2.923)--(-8.183,2.808);
\filldraw[fill opacity=0.8,fill=gray!20,draw=none](-8.119,2.78)--(-8.149,2.802)--(-8.16,2.776)--cycle;
\draw(-8.149,2.802)--(-8.16,2.776);
\filldraw[fill opacity=0.8,fill=gray!20](-8.132,2.736)--(-8.144,2.784)--(-8.217,2.801)--(-8.197,2.752)--cycle;
\filldraw[fill opacity=0.8,fill=gray!20](-8.077,3.016)--(-8.053,3.041)--(-8.069,3.045)--(-8.109,3.023)--cycle;
\filldraw[fill opacity=0.8,fill=gray!20](-7.707,4.232)--(-7.666,4.271)--(-7.677,4.283)--(-7.723,4.249)--cycle;
\filldraw[fill opacity=0.8,fill=gray!20](-7.356,4.148)--(-7.378,4.2)--(-7.409,4.18)--(-7.391,4.126)--cycle;
\filldraw[fill opacity=0.8,fill=gray!20,draw=none](-8.935,1.061)--(-8.933,1.062)--(-8.939,1.069)--(-8.954,1.066)--cycle;
\draw(-8.939,1.069)--(-8.954,1.066)--(-8.935,1.061)--(-8.933,1.062);
\filldraw[fill opacity=0.8,fill=gray!20,draw=none](-8.926,1.066)--(-8.939,1.071)--(-8.939,1.069)--cycle;
\draw(-8.939,1.071)--(-8.939,1.069);
\filldraw[fill opacity=0.8,fill=gray!20,draw=none](-8.926,1.066)--(-8.907,1.06)--(-8.181,2.728)--(-8.202,2.764)--(-8.939,1.071)--cycle;
\draw(-8.907,1.06)--(-8.181,2.728);
\draw(-8.202,2.764)--(-8.939,1.071);
\filldraw[fill opacity=0.8,fill=gray!20,draw=none](-8.131,2.923)--(-8.132,2.923)--(-8.128,2.921)--cycle;
\draw(-8.132,2.923)--(-8.128,2.921);
\filldraw[fill opacity=0.8,fill=gray!20](-7.951,3.021)--(-7.987,3.044)--(-8.006,3.041)--(-7.986,3.014)--cycle;
\filldraw[fill opacity=0.8,fill=gray!20,draw=none](-8.181,2.728)--(-8.168,2.758)--(-8.202,2.764)--cycle;
\draw(-8.181,2.728)--(-8.168,2.758);
\filldraw[fill opacity=0.8,fill=gray!20,draw=none](-8.124,2.797)--(-8.124,2.799)--(-8.185,2.814)--(-8.183,2.808)--cycle;
\draw(-8.124,2.797)--(-8.124,2.799)--(-8.185,2.814)--(-8.183,2.808);
\filldraw[fill opacity=0.5,fill=gray!20](-10.044,-.347)--(-10.175,-.504)--(-10.439,-.157)--(-10.279,-.039)--cycle;
\filldraw[fill opacity=0.8,fill=gray!20](-7.895,2.769)--(-7.879,2.81)--(-7.948,2.797)--(-7.956,2.757)--cycle;
\filldraw[fill opacity=0.8,fill=gray!20](-8.124,2.799)--(-8.128,2.844)--(-8.191,2.86)--(-8.185,2.814)--cycle;
\filldraw[fill opacity=0.8,fill=gray!20,draw=none](-7.982,2.723)--(-7.932,2.836)--(-7.968,2.852)--(-8.02,2.733)--cycle;
\draw(-7.982,2.723)--(-7.932,2.836)--(-7.968,2.852)--(-8.02,2.733);
\filldraw[fill opacity=0.8,fill=gray!20,draw=none](-7.976,2.717)--(-7.969,2.726)--(-7.984,2.725)--cycle;
\draw(-7.976,2.717)--(-7.969,2.726)--(-7.984,2.725);
\filldraw[fill opacity=0.8,fill=gray!20](-8.964,1.03)--(-8.935,1.061)--(-8.954,1.066)--(-9.001,1.039)--cycle;
\filldraw[fill opacity=0.8,fill=gray!20,draw=none](-8.76,.921)--(-8.765,.922)--(-8.772,.907)--cycle;
\draw(-8.765,.922)--(-8.772,.907);
\filldraw[fill opacity=0.8,fill=gray!20](-7.653,3.916)--(-7.673,3.954)--(-7.738,3.97)--(-7.707,3.929)--cycle;
\filldraw[fill opacity=0.8,fill=gray!20](-8.087,3.043)--(-8.058,3.074)--(-8.078,3.079)--(-8.125,3.052)--cycle;
\filldraw[fill opacity=0.5,fill=gray!20](-10.536,2.289)--(-10.591,2.308)--(-10.211,2.668)--(-10.154,2.651)--cycle;
\filldraw[fill opacity=0.8,fill=gray!20](-7.666,4.271)--(-7.619,4.297)--(-7.625,4.303)--(-7.677,4.283)--cycle;
\filldraw[fill opacity=0.8,fill=gray!20](-7.378,4.2)--(-7.413,4.245)--(-7.438,4.229)--(-7.409,4.18)--cycle;
\filldraw[fill opacity=0.8,fill=gray!20](-8.098,2.98)--(-8.077,3.016)--(-8.109,3.023)--(-8.143,2.991)--cycle;
\filldraw[fill opacity=0.8,fill=gray!20,draw=none](-8.736,.972)--(-8.772,.998)--(-8.774,.997)--(-8.745,.949)--cycle;
\draw(-8.772,.998)--(-8.774,.997)--(-8.745,.949);
\filldraw[fill opacity=0.5,fill=gray!20](-10.599,.409)--(-10.426,.334)--(-10.477,.745)--(-10.65,.821)--cycle;
\filldraw[fill opacity=0.8,fill=gray!20,draw=none](-8.889,.858)--(-8.821,1.014)--(-8.844,1.03)--(-8.86,1.036)--(-8.867,1.037)--(-8.936,.879)--cycle;
\draw(-8.867,1.037)--(-8.936,.879)--(-8.889,.858)--(-8.821,1.014);
\filldraw[fill opacity=0.8,fill=gray!20](-8.812,1.037)--(-8.856,1.064)--(-8.878,1.06)--(-8.854,1.028)--cycle;
\filldraw[fill opacity=0.8,fill=gray!20](-7.935,3.05)--(-7.98,3.078)--(-8.002,3.074)--(-7.978,3.042)--cycle;
\filldraw[fill opacity=0.8,fill=gray!20,draw=none](-8.745,.734)--(-8.739,.749)--(-8.761,.777)--(-8.808,.767)--(-8.818,.72)--cycle;
\draw(-8.761,.777)--(-8.808,.767)--(-8.818,.72)--(-8.745,.734)--(-8.739,.749);
\filldraw[fill opacity=0.8,fill=gray!20,draw=none](-8.036,2.743)--(-8.016,2.742)--(-7.968,2.852)--(-8.013,2.871)--(-8.062,2.759)--cycle;
\draw(-8.016,2.742)--(-7.968,2.852)--(-8.013,2.871)--(-8.062,2.759);
\filldraw[fill opacity=0.8,fill=gray!20](-8.035,2.723)--(-8.036,2.754)--(-8.114,2.759)--(-8.098,2.728)--cycle;
\filldraw[fill opacity=0.8,fill=gray!20](-8.032,3.012)--(-8.03,3.04)--(-8.053,3.041)--(-8.077,3.016)--cycle;
\filldraw[fill opacity=0.8,fill=gray!20](-7.868,2.748)--(-7.85,2.797)--(-7.932,2.781)--(-7.942,2.733)--cycle;
\filldraw[fill opacity=0.5,fill=gray!20](-11.065,1.359)--(-11.092,1.38)--(-10.912,1.872)--(-10.882,1.862)--cycle;
\filldraw[fill opacity=0.8,fill=gray!20](-9.02,.77)--(-9.024,.824)--(-9.1,.843)--(-9.093,.788)--cycle;
\filldraw[fill opacity=0.8,fill=gray!20,draw=none](-8.908,1.058)--(-8.907,1.06)--(-8.926,1.066)--cycle;
\draw(-8.908,1.058)--(-8.907,1.06);
\filldraw[fill opacity=0.8,fill=gray!20](-8.128,2.844)--(-8.124,2.892)--(-8.185,2.907)--(-8.191,2.86)--cycle;
\filldraw[fill opacity=0.8,fill=gray!20](-7.986,3.014)--(-8.006,3.041)--(-8.03,3.04)--(-8.032,3.012)--cycle;
\filldraw[fill opacity=0.8,fill=gray!20](-7.619,4.297)--(-7.568,4.31)--(-7.568,4.31)--(-7.625,4.303)--cycle;
\filldraw[fill opacity=0.8,fill=gray!20](-7.413,4.245)--(-7.458,4.28)--(-7.476,4.268)--(-7.438,4.229)--cycle;
\filldraw[fill opacity=0.8,fill=gray!20,draw=none](-8.119,2.78)--(-8.124,2.797)--(-8.149,2.802)--cycle;
\draw(-8.119,2.78)--(-8.124,2.797);
\filldraw[fill opacity=0.8,fill=gray!20](-8.144,2.784)--(-8.148,2.838)--(-8.223,2.856)--(-8.217,2.801)--cycle;
\filldraw[fill opacity=0.8,fill=gray!20,draw=none](-8.794,.895)--(-8.787,.893)--(-8.772,.907)--cycle;
\filldraw[fill opacity=0.8,fill=gray!20,draw=none](-8.794,.895)--(-8.772,.907)--(-8.761,.933)--(-8.812,.912)--(-8.817,.9)--cycle;
\draw(-8.772,.907)--(-8.761,.933);
\draw(-8.812,.912)--(-8.817,.9);
\filldraw[fill opacity=0.8,fill=gray!20,draw=none](-8.738,.986)--(-8.005,2.669)--(-8.038,2.692)--(-8.775,.999)--cycle;
\draw(-8.738,.986)--(-8.005,2.669);
\draw(-8.038,2.692)--(-8.775,.999);
\filldraw[fill opacity=0.8,fill=gray!20,draw=none](-8.036,2.69)--(-8.005,2.669)--(-8,2.68)--cycle;
\draw(-8.005,2.669)--(-8,2.68);
\filldraw[fill opacity=0.8,fill=gray!20,draw=none](-8,2.68)--(-8.012,2.693)--(-8.036,2.692)--(-8.036,2.69)--cycle;
\draw(-8.012,2.693)--(-8.036,2.692)--(-8.036,2.69);
\filldraw[fill opacity=0.8,fill=gray!20,draw=none](-7.983,2.721)--(-7.982,2.723)--(-7.99,2.725)--cycle;
\draw(-7.983,2.721)--(-7.982,2.723);
\filldraw[fill opacity=0.8,fill=gray!20,draw=none](-7.982,2.723)--(-7.984,2.725)--(-7.99,2.725)--cycle;
\draw(-7.984,2.725)--(-7.99,2.725);
\filldraw[fill opacity=0.8,fill=gray!20,draw=none](-8.11,3.394)--(-8.092,3.385)--(-7.954,3.75)--cycle;
\draw(-8.092,3.385)--(-7.954,3.75)--(-8.11,3.394);
\filldraw[fill opacity=0.8,fill=gray!20](-7.919,2.988)--(-7.951,3.021)--(-7.986,3.014)--(-7.969,2.979)--cycle;
\filldraw[fill opacity=0.8,fill=gray!20,draw=none](-8.739,.749)--(-8.726,.783)--(-8.761,.777)--cycle;
\draw(-8.739,.749)--(-8.726,.783)--(-8.761,.777);
\filldraw[fill opacity=0.8,fill=gray!20](-8.114,2.938)--(-8.098,2.98)--(-8.143,2.991)--(-8.169,2.951)--cycle;
\filldraw[fill opacity=0.8,fill=gray!20,draw=none](-7.99,2.725)--(-7.969,2.726)--(-7.956,2.757)--(-8.036,2.754)--cycle;
\draw(-7.99,2.725)--(-7.969,2.726)--(-7.956,2.757)--(-8.036,2.754);
\filldraw[fill opacity=0.8,fill=gray!20](-7.879,2.81)--(-7.874,2.856)--(-7.945,2.842)--(-7.948,2.797)--cycle;
\filldraw[fill opacity=0.8,fill=gray!20](-7.574,3.886)--(-7.577,3.91)--(-7.653,3.916)--(-7.628,3.89)--cycle;
\filldraw[fill opacity=0.8,fill=gray!20](-7.511,4.302)--(-7.568,4.31)--(-7.568,4.31)--(-7.52,4.296)--cycle;
\filldraw[fill opacity=0.8,fill=gray!20](-7.458,4.28)--(-7.511,4.302)--(-7.52,4.296)--(-7.476,4.268)--cycle;
\filldraw[fill opacity=0.8,fill=gray!20](-7.571,4.29)--(-7.568,4.31)--(-7.568,4.31)--(-7.599,4.292)--cycle;
\filldraw[fill opacity=0.8,fill=gray!20](-7.599,4.292)--(-7.568,4.31)--(-7.568,4.31)--(-7.619,4.297)--cycle;
\filldraw[fill opacity=0.8,fill=gray!20](-7.542,4.292)--(-7.568,4.31)--(-7.568,4.31)--(-7.571,4.29)--cycle;
\filldraw[fill opacity=0.8,fill=gray!20](-7.52,4.296)--(-7.568,4.31)--(-7.568,4.31)--(-7.542,4.292)--cycle;
\filldraw[fill opacity=0.8,fill=gray!20](-7.438,3.926)--(-7.409,3.966)--(-7.483,3.951)--(-7.498,3.914)--cycle;
\filldraw[fill opacity=0.8,fill=gray!20,draw=none](-8.811,.87)--(-8.787,.893)--(-8.794,.895)--(-8.828,.877)--(-8.83,.871)--cycle;
\draw(-8.828,.877)--(-8.83,.871);
\filldraw[fill opacity=0.8,fill=gray!20](-8.124,2.892)--(-8.114,2.938)--(-8.169,2.951)--(-8.185,2.907)--cycle;
\filldraw[fill opacity=0.8,fill=gray!20,draw=none](-7.649,3.218)--(-7.62,3.209)--(-7.482,3.574)--cycle;
\draw(-7.62,3.209)--(-7.482,3.574)--(-7.649,3.218);
\filldraw[fill opacity=0.8,fill=gray!20](-8.989,.988)--(-8.964,1.03)--(-9.001,1.039)--(-9.042,1.001)--cycle;
\filldraw[fill opacity=0.8,fill=gray!20](-8.113,3.001)--(-8.087,3.043)--(-8.125,3.052)--(-8.166,3.014)--cycle;
\filldraw[fill opacity=0.8,fill=gray!20,draw=none](-8.936,.879)--(-8.867,1.037)--(-8.908,1.057)--(-8.978,.897)--cycle;
\draw(-8.908,1.057)--(-8.978,.897)--(-8.936,.879)--(-8.867,1.037);
\filldraw[fill opacity=0.8,fill=gray!20,draw=none](-8.86,1.036)--(-8.878,1.06)--(-8.907,1.059)--(-8.908,1.043)--cycle;
\draw(-8.86,1.036)--(-8.878,1.06)--(-8.907,1.059)--(-8.908,1.043);
\filldraw[fill opacity=0.8,fill=gray!20,draw=none](-8.867,1.037)--(-8.867,1.038)--(-8.889,1.054)--(-8.908,1.057)--cycle;
\draw(-8.867,1.037)--(-8.867,1.038);
\filldraw[fill opacity=0.8,fill=gray!20](-8.91,1.026)--(-8.907,1.059)--(-8.935,1.061)--(-8.964,1.03)--cycle;
\filldraw[fill opacity=0.8,fill=gray!20](-8.913,.679)--(-8.915,.716)--(-9.008,.723)--(-8.989,.684)--cycle;
\filldraw[fill opacity=0.8,fill=gray!20](-7.519,3.889)--(-7.498,3.914)--(-7.577,3.91)--(-7.574,3.886)--cycle;
\filldraw[fill opacity=0.8,fill=gray!20,draw=none](-8.041,2.693)--(-8.038,2.692)--(-8.02,2.732)--(-8.062,2.759)--(-8.08,2.718)--cycle;
\draw(-8.038,2.692)--(-8.02,2.732);
\draw(-8.062,2.759)--(-8.08,2.718);
\filldraw[fill opacity=0.8,fill=gray!20,draw=none](-8.08,2.718)--(-8.062,2.759)--(-8.112,2.773)--(-8.13,2.732)--cycle;
\draw(-8.08,2.718)--(-8.062,2.759);
\draw(-8.112,2.773)--(-8.13,2.732);
\filldraw[fill opacity=0.8,fill=gray!20](-8.036,2.692)--(-8.038,2.729)--(-8.132,2.736)--(-8.113,2.698)--cycle;
\filldraw[fill opacity=0.8,fill=gray!20](-8.034,3.039)--(-8.03,3.072)--(-8.058,3.074)--(-8.087,3.043)--cycle;
\filldraw[fill opacity=0.8,fill=gray!20,draw=none](-8.794,.895)--(-8.817,.9)--(-8.828,.877)--cycle;
\draw(-8.817,.9)--(-8.828,.877);
\filldraw[fill opacity=0.8,fill=gray!20](-9.024,.824)--(-9.02,.881)--(-9.093,.899)--(-9.1,.843)--cycle;
\filldraw[fill opacity=0.8,fill=gray!20](-8.148,2.838)--(-8.144,2.894)--(-8.217,2.912)--(-8.223,2.856)--cycle;
\filldraw[fill opacity=0.8,fill=gray!20](-7.978,3.042)--(-8.002,3.074)--(-8.03,3.072)--(-8.034,3.039)--cycle;
\filldraw[fill opacity=0.8,fill=gray!20](-7.673,3.954)--(-7.685,4.002)--(-7.758,4.02)--(-7.738,3.97)--cycle;
\filldraw[fill opacity=0.8,fill=gray!20](-7.895,2.948)--(-7.919,2.988)--(-7.969,2.979)--(-7.956,2.936)--cycle;
\filldraw[fill opacity=0.8,fill=gray!20,draw=none](-8.811,.87)--(-8.83,.871)--(-8.844,.839)--cycle;
\draw(-8.83,.871)--(-8.844,.839);
\filldraw[fill opacity=0.8,fill=gray!20](-7.874,2.856)--(-7.879,2.903)--(-7.948,2.889)--(-7.945,2.842)--cycle;
\filldraw[fill opacity=0.8,fill=gray!20,draw=none](-7.649,3.218)--(-7.482,3.574)--(-7.529,3.589)--(-7.667,3.223)--cycle;
\draw(-7.649,3.218)--(-7.482,3.574)--(-7.529,3.589)--(-7.667,3.223);
\filldraw[fill opacity=0.8,fill=gray!20,draw=none](-8.908,1.057)--(-8.908,1.058)--(-8.91,1.059)--cycle;
\draw(-8.908,1.057)--(-8.908,1.058);
\filldraw[fill opacity=0.8,fill=gray!20,draw=none](-8.844,.839)--(-8.775,.999)--(-8.777,1)--(-8.821,1.014)--(-8.889,.858)--cycle;
\draw(-8.821,1.014)--(-8.889,.858)--(-8.844,.839)--(-8.775,.999);
\filldraw[fill opacity=0.8,fill=gray!20](-8.774,.997)--(-8.812,1.037)--(-8.854,1.028)--(-8.834,.986)--cycle;
\filldraw[fill opacity=0.8,fill=gray!20](-9.008,.937)--(-8.989,.988)--(-9.042,1.001)--(-9.073,.953)--cycle;
\filldraw[fill opacity=0.8,fill=gray!20](-8.726,.783)--(-8.72,.838)--(-8.805,.821)--(-8.808,.767)--cycle;
\filldraw[fill opacity=0.8,fill=gray!20](-8.834,.682)--(-8.818,.72)--(-8.915,.716)--(-8.913,.679)--cycle;
\filldraw[fill opacity=0.8,fill=gray!20](-7.897,3.01)--(-7.935,3.05)--(-7.978,3.042)--(-7.958,2.999)--cycle;
\filldraw[fill opacity=0.8,fill=gray!20](-8.132,2.95)--(-8.113,3.001)--(-8.166,3.014)--(-8.197,2.966)--cycle;
\filldraw[fill opacity=0.8,fill=gray!20](-7.958,2.696)--(-7.942,2.733)--(-8.038,2.729)--(-8.036,2.692)--cycle;
\filldraw[fill opacity=0.8,fill=gray!20](-7.85,2.797)--(-7.844,2.851)--(-7.929,2.835)--(-7.932,2.781)--cycle;
\filldraw[fill opacity=0.8,fill=gray!20,draw=none](-8.889,1.054)--(-8.905,1.064)--(-8.908,1.057)--cycle;
\draw(-8.905,1.064)--(-8.908,1.057);
\filldraw[fill opacity=0.8,fill=gray!20,draw=none](-8.867,1.038)--(-8.13,2.73)--(-8.142,2.738)--(-8.17,2.753)--(-8.905,1.064)--cycle;
\draw(-8.867,1.038)--(-8.13,2.73);
\draw(-8.17,2.753)--(-8.905,1.064);
\filldraw[fill opacity=0.8,fill=gray!20](-7.627,3.207)--(-7.571,3.19)--(-7.35,3.776)--cycle;
\filldraw[fill opacity=0.8,fill=gray!20,draw=none](-8.142,2.738)--(-8.168,2.758)--(-8.17,2.753)--cycle;
\draw(-8.168,2.758)--(-8.17,2.753);
\filldraw[fill opacity=0.8,fill=gray!20](-9.02,.881)--(-9.008,.937)--(-9.073,.953)--(-9.093,.899)--cycle;
\filldraw[fill opacity=0.8,fill=gray!20](-8.144,2.894)--(-8.132,2.95)--(-8.197,2.966)--(-8.217,2.912)--cycle;
\filldraw[fill opacity=0.8,fill=gray!20](-7.879,2.903)--(-7.895,2.948)--(-7.956,2.936)--(-7.948,2.889)--cycle;
\filldraw[fill opacity=0.8,fill=gray!20](-7.628,4.261)--(-7.599,4.292)--(-7.619,4.297)--(-7.666,4.271)--cycle;
\filldraw[fill opacity=0.8,fill=gray!20](-8.035,2.976)--(-8.032,3.012)--(-8.077,3.016)--(-8.098,2.98)--cycle;
\filldraw[fill opacity=0.5,fill=gray!20](-10.939,.268)--(-11.014,.291)--(-11.077,.806)--(-11.001,.779)--cycle;
\filldraw[fill opacity=0.8,fill=gray!20,draw=none](-8.777,1)--(-8.775,.999)--(-8.774,1)--cycle;
\draw(-8.775,.999)--(-8.774,1);
\filldraw[fill opacity=0.8,fill=gray!20,draw=none](-8.777,1)--(-8.774,1)--(-8.038,2.691)--(-8.041,2.693)--(-8.084,2.709)--(-8.816,1.025)--cycle;
\draw(-8.774,1)--(-8.038,2.691);
\draw(-8.084,2.709)--(-8.816,1.025);
\filldraw[fill opacity=0.8,fill=gray!20](-7.476,4.268)--(-7.52,4.296)--(-7.542,4.292)--(-7.519,4.26)--cycle;
\filldraw[fill opacity=0.8,fill=gray!20](-7.969,2.979)--(-7.986,3.014)--(-8.032,3.012)--(-8.035,2.976)--cycle;
\filldraw[fill opacity=0.8,fill=gray!20,draw=none](-8.777,1)--(-8.816,1.025)--(-8.821,1.014)--cycle;
\draw(-8.816,1.025)--(-8.821,1.014);
\filldraw[fill opacity=0.8,fill=gray!20,draw=none](-8.038,2.691)--(-8.038,2.692)--(-8.041,2.693)--cycle;
\draw(-8.038,2.691)--(-8.038,2.692);
\filldraw[fill opacity=0.8,fill=gray!20,draw=none](-8.036,2.743)--(-8.02,2.732)--(-8.016,2.742)--cycle;
\draw(-8.02,2.732)--(-8.016,2.742);
\filldraw[fill opacity=0.8,fill=gray!20,draw=none](-8.017,2.742)--(-8.036,2.754)--(-8.036,2.743)--cycle;
\draw(-8.036,2.754)--(-8.036,2.743);
\filldraw[fill opacity=0.8,fill=gray!20](-7.409,3.966)--(-7.391,4.015)--(-7.473,3.999)--(-7.483,3.951)--cycle;
\filldraw[fill opacity=0.8,fill=gray!20,draw=none](-8.041,2.693)--(-8.08,2.718)--(-8.084,2.709)--cycle;
\draw(-8.08,2.718)--(-8.084,2.709);
\filldraw[fill opacity=0.8,fill=gray!20](-8.745,.949)--(-8.774,.997)--(-8.834,.986)--(-8.818,.934)--cycle;
\filldraw[fill opacity=0.8,fill=gray!20](-8.72,.838)--(-8.726,.894)--(-8.808,.878)--(-8.805,.821)--cycle;
\filldraw[fill opacity=0.8,fill=gray!20](-7.685,4.002)--(-7.689,4.056)--(-7.764,4.074)--(-7.758,4.02)--cycle;
\filldraw[fill opacity=0.8,fill=gray!20](-7.868,2.962)--(-7.897,3.01)--(-7.958,2.999)--(-7.942,2.948)--cycle;
\filldraw[fill opacity=0.8,fill=gray!20](-7.844,2.851)--(-7.85,2.908)--(-7.932,2.892)--(-7.929,2.835)--cycle;
\filldraw[fill opacity=0.8,fill=gray!20](-7.956,2.757)--(-7.948,2.797)--(-8.038,2.793)--(-8.036,2.754)--cycle;
\filldraw[fill opacity=0.8,fill=gray!20,draw=none](-8.13,2.73)--(-8.13,2.732)--(-8.142,2.738)--cycle;
\draw(-8.13,2.73)--(-8.13,2.732);
\filldraw[fill opacity=0.8,fill=gray!20](-7.35,3.776)--(-7.406,3.794)--(-7.627,3.207)--cycle;
\filldraw[fill opacity=0.8,fill=gray!20,draw=none](-8.119,2.78)--(-8.111,2.774)--(-8.108,2.781)--cycle;
\draw(-8.111,2.774)--(-8.108,2.781);
\filldraw[fill opacity=0.8,fill=gray!20,draw=none](-8.854,1.028)--(-8.86,1.036)--(-8.908,1.043)--(-8.91,1.026)--cycle;
\draw(-8.908,1.043)--(-8.91,1.026)--(-8.854,1.028)--(-8.86,1.036);
\filldraw[fill opacity=0.8,fill=gray!20](-8.726,.894)--(-8.745,.949)--(-8.818,.934)--(-8.808,.878)--cycle;
\filldraw[fill opacity=0.8,fill=gray!20,draw=none](-8.061,2.762)--(-8.013,2.871)--(-8.06,2.892)--(-8.108,2.781)--cycle;
\draw(-8.061,2.762)--(-8.013,2.871)--(-8.06,2.892)--(-8.108,2.781);
\filldraw[fill opacity=0.8,fill=gray!20,draw=none](-8.061,2.762)--(-8.091,2.797)--(-8.124,2.799)--(-8.121,2.786)--cycle;
\draw(-8.091,2.797)--(-8.124,2.799)--(-8.121,2.786);
\filldraw[fill opacity=0.8,fill=gray!20,draw=none](-8.86,1.036)--(-8.867,1.038)--(-8.867,1.037)--cycle;
\draw(-8.867,1.038)--(-8.867,1.037);
\filldraw[fill opacity=0.8,fill=gray!20,draw=none](-8.844,1.03)--(-8.864,1.044)--(-8.867,1.038)--cycle;
\draw(-8.864,1.044)--(-8.867,1.038);
\filldraw[fill opacity=0.8,fill=gray!20,draw=none](-8.844,1.03)--(-8.818,1.021)--(-8.088,2.698)--(-8.13,2.732)--(-8.864,1.044)--cycle;
\draw(-8.818,1.021)--(-8.088,2.698);
\draw(-8.13,2.732)--(-8.864,1.044);
\filldraw[fill opacity=0.8,fill=gray!20](-7.85,2.908)--(-7.868,2.962)--(-7.942,2.948)--(-7.932,2.892)--cycle;
\filldraw[fill opacity=0.8,fill=gray!20](-7.653,4.219)--(-7.628,4.261)--(-7.666,4.271)--(-7.707,4.232)--cycle;
\filldraw[fill opacity=0.8,fill=gray!20,draw=none](-8.088,2.698)--(-8.08,2.718)--(-8.13,2.732)--cycle;
\draw(-8.088,2.698)--(-8.08,2.718);
\filldraw[fill opacity=0.8,fill=gray!20,draw=none](-8.062,2.759)--(-8.061,2.762)--(-8.108,2.781)--(-8.112,2.773)--cycle;
\draw(-8.062,2.759)--(-8.061,2.762);
\draw(-8.108,2.781)--(-8.112,2.773);
\filldraw[fill opacity=0.8,fill=gray!20,draw=none](-8.821,1.014)--(-8.818,1.021)--(-8.844,1.03)--cycle;
\draw(-8.821,1.014)--(-8.818,1.021);
\filldraw[fill opacity=0.8,fill=gray!20,draw=none](-8.055,2.755)--(-8.036,2.754)--(-8.038,2.793)--(-8.091,2.797)--cycle;
\draw(-8.055,2.755)--(-8.036,2.754)--(-8.038,2.793)--(-8.091,2.797);
\filldraw[fill opacity=0.8,fill=gray!20](-7.577,3.91)--(-7.579,3.947)--(-7.673,3.954)--(-7.653,3.916)--cycle;
\filldraw[fill opacity=0.8,fill=gray!20](-7.574,4.257)--(-7.571,4.29)--(-7.599,4.292)--(-7.628,4.261)--cycle;
\filldraw[fill opacity=0.8,fill=gray!20](-8.913,.982)--(-8.91,1.026)--(-8.964,1.03)--(-8.989,.988)--cycle;
\filldraw[fill opacity=0.8,fill=gray!20](-8.915,.716)--(-8.916,.763)--(-9.02,.77)--(-9.008,.723)--cycle;
\filldraw[fill opacity=0.8,fill=gray!20](-8.038,2.729)--(-8.04,2.776)--(-8.144,2.784)--(-8.132,2.736)--cycle;
\filldraw[fill opacity=0.8,fill=gray!20](-8.036,2.995)--(-8.034,3.039)--(-8.087,3.043)--(-8.113,3.001)--cycle;
\filldraw[fill opacity=0.5,fill=gray!20](-11.077,.806)--(-11.128,.835)--(-11.065,1.359)--(-11.014,1.334)--cycle;
\filldraw[fill opacity=0.8,fill=gray!20](-7.689,4.056)--(-7.685,4.113)--(-7.758,4.13)--(-7.764,4.074)--cycle;
\filldraw[fill opacity=0.8,fill=gray!20](-7.519,4.26)--(-7.542,4.292)--(-7.571,4.29)--(-7.574,4.257)--cycle;
\filldraw[fill opacity=0.8,fill=gray!20](-8.036,2.932)--(-8.035,2.976)--(-8.098,2.98)--(-8.114,2.938)--cycle;
\filldraw[fill opacity=0.8,fill=gray!20](-8.038,2.793)--(-8.038,2.838)--(-8.128,2.844)--(-8.124,2.799)--cycle;
\filldraw[fill opacity=0.8,fill=gray!20](-8.834,.986)--(-8.854,1.028)--(-8.91,1.026)--(-8.913,.982)--cycle;
\filldraw[fill opacity=0.5,fill=gray!20](-9.43,2.611)--(-9.5,2.713)--(-9.05,2.827)--(-9,2.719)--cycle;
\filldraw[fill opacity=0.8,fill=gray!20](-7.958,2.999)--(-7.978,3.042)--(-8.034,3.039)--(-8.036,2.995)--cycle;
\filldraw[fill opacity=0.8,fill=gray!20](-7.438,4.229)--(-7.476,4.268)--(-7.519,4.26)--(-7.498,4.217)--cycle;
\filldraw[fill opacity=0.8,fill=gray!20](-7.498,3.914)--(-7.483,3.951)--(-7.579,3.947)--(-7.577,3.91)--cycle;
\filldraw[fill opacity=0.8,fill=gray!20](-7.673,4.168)--(-7.653,4.219)--(-7.707,4.232)--(-7.738,4.184)--cycle;
\filldraw[fill opacity=0.8,fill=gray!20](-7.391,4.015)--(-7.385,4.069)--(-7.47,4.053)--(-7.473,3.999)--cycle;
\filldraw[fill opacity=0.8,fill=gray!20](-8.818,.72)--(-8.808,.767)--(-8.916,.763)--(-8.915,.716)--cycle;
\filldraw[fill opacity=0.8,fill=gray!20](-7.956,2.936)--(-7.969,2.979)--(-8.035,2.976)--(-8.036,2.932)--cycle;
\filldraw[fill opacity=0.8,fill=gray!20](-7.942,2.733)--(-7.932,2.781)--(-8.04,2.776)--(-8.038,2.729)--cycle;
\filldraw[fill opacity=0.8,fill=gray!20](-7.685,4.113)--(-7.673,4.168)--(-7.738,4.184)--(-7.758,4.13)--cycle;
\filldraw[fill opacity=0.5,fill=gray!20](-10.557,-.185)--(-10.664,-.198)--(-10.843,.256)--(-10.731,.253)--cycle;
\filldraw[fill opacity=0.8,fill=gray!20](-7.948,2.797)--(-7.945,2.842)--(-8.038,2.838)--(-8.038,2.793)--cycle;
\filldraw[fill opacity=0.8,fill=gray!20,draw=none](-8.068,3.375)--(-7.89,3.724)--(-7.954,3.75)--(-8.092,3.385)--cycle;
\draw(-8.068,3.375)--(-7.89,3.724)--(-7.954,3.75)--(-8.092,3.385);
\filldraw[fill opacity=0.5,fill=gray!20](-10.005,2.554)--(-10.084,2.612)--(-9.642,2.861)--(-9.572,2.797)--cycle;
\filldraw[fill opacity=0.8,fill=gray!20](-8.038,2.885)--(-8.036,2.932)--(-8.114,2.938)--(-8.124,2.892)--cycle;
\filldraw[fill opacity=0.8,fill=gray!20](-8.038,2.838)--(-8.038,2.885)--(-8.124,2.892)--(-8.128,2.844)--cycle;
\filldraw[fill opacity=0.8,fill=gray!20](-7.409,4.18)--(-7.438,4.229)--(-7.498,4.217)--(-7.483,4.166)--cycle;
\filldraw[fill opacity=0.8,fill=gray!20](-7.385,4.069)--(-7.391,4.126)--(-7.473,4.11)--(-7.47,4.053)--cycle;
\filldraw[fill opacity=0.8,fill=gray!20](-7.948,2.889)--(-7.956,2.936)--(-8.036,2.932)--(-8.038,2.885)--cycle;
\filldraw[fill opacity=0.8,fill=gray!20](-8.915,.93)--(-8.913,.982)--(-8.989,.988)--(-9.008,.937)--cycle;
\filldraw[fill opacity=0.8,fill=gray!20](-8.916,.763)--(-8.916,.816)--(-9.024,.824)--(-9.02,.77)--cycle;
\filldraw[fill opacity=0.8,fill=gray!20](-7.945,2.842)--(-7.948,2.889)--(-8.038,2.885)--(-8.038,2.838)--cycle;
\filldraw[fill opacity=0.8,fill=gray!20](-8.038,2.943)--(-8.036,2.995)--(-8.113,3.001)--(-8.132,2.95)--cycle;
\filldraw[fill opacity=0.8,fill=gray!20](-8.04,2.776)--(-8.04,2.83)--(-8.148,2.838)--(-8.144,2.784)--cycle;
\filldraw[fill opacity=0.8,fill=gray!20](-7.391,4.126)--(-7.409,4.18)--(-7.483,4.166)--(-7.473,4.11)--cycle;
\filldraw[fill opacity=0.8,fill=gray!20](-7.579,3.947)--(-7.581,3.994)--(-7.685,4.002)--(-7.673,3.954)--cycle;
\filldraw[fill opacity=0.8,fill=gray!20](-7.577,4.213)--(-7.574,4.257)--(-7.628,4.261)--(-7.653,4.219)--cycle;
\filldraw[fill opacity=0.8,fill=gray!20](-8.818,.934)--(-8.834,.986)--(-8.913,.982)--(-8.915,.93)--cycle;
\filldraw[fill opacity=0.8,fill=gray!20](-7.942,2.948)--(-7.958,2.999)--(-8.036,2.995)--(-8.038,2.943)--cycle;
\filldraw[fill opacity=0.5,fill=gray!20](-9.382,2.421)--(-9.43,2.611)--(-9,2.719)--(-9,2.518)--cycle;
\filldraw[fill opacity=0.8,fill=gray!20](-8.808,.767)--(-8.805,.821)--(-8.916,.816)--(-8.916,.763)--cycle;
\filldraw[fill opacity=0.8,fill=gray!20](-7.932,2.781)--(-7.929,2.835)--(-8.04,2.83)--(-8.04,2.776)--cycle;
\filldraw[fill opacity=0.8,fill=gray!20,draw=none](-8.068,3.375)--(-8.028,3.358)--(-7.89,3.724)--cycle;
\draw(-8.028,3.358)--(-7.89,3.724)--(-8.068,3.375);
\filldraw[fill opacity=0.8,fill=gray!20](-8.137,3.401)--(-8.06,3.369)--(-7.839,3.956)--cycle;
\filldraw[fill opacity=0.8,fill=gray!20](-7.839,3.956)--(-7.916,3.988)--(-8.137,3.401)--cycle;
\filldraw[fill opacity=0.8,fill=gray!20](-7.498,4.217)--(-7.519,4.26)--(-7.574,4.257)--(-7.577,4.213)--cycle;
\filldraw[fill opacity=0.8,fill=gray!20](-8.916,.874)--(-8.915,.93)--(-9.008,.937)--(-9.02,.881)--cycle;
\filldraw[fill opacity=0.8,fill=gray!20](-8.916,.816)--(-8.916,.874)--(-9.02,.881)--(-9.024,.824)--cycle;
\filldraw[fill opacity=0.8,fill=gray!20](-8.04,2.887)--(-8.038,2.943)--(-8.132,2.95)--(-8.144,2.894)--cycle;
\filldraw[fill opacity=0.8,fill=gray!20](-8.04,2.83)--(-8.04,2.887)--(-8.144,2.894)--(-8.148,2.838)--cycle;
\filldraw[fill opacity=0.8,fill=gray!20](-7.483,3.951)--(-7.473,3.999)--(-7.581,3.994)--(-7.579,3.947)--cycle;
\filldraw[fill opacity=0.8,fill=gray!20,draw=none](-7.714,3.24)--(-7.667,3.223)--(-7.529,3.589)--cycle;
\draw(-7.667,3.223)--(-7.529,3.589)--(-7.714,3.24);
\filldraw[fill opacity=0.5,fill=gray!20](-10.829,1.84)--(-10.882,1.862)--(-10.591,2.308)--(-10.536,2.289)--cycle;
\filldraw[fill opacity=0.8,fill=gray!20](-8.808,.878)--(-8.818,.934)--(-8.915,.93)--(-8.916,.874)--cycle;
\filldraw[fill opacity=0.8,fill=gray!20](-7.932,2.892)--(-7.942,2.948)--(-8.038,2.943)--(-8.04,2.887)--cycle;
\filldraw[fill opacity=0.8,fill=gray!20](-8.805,.821)--(-8.808,.878)--(-8.916,.874)--(-8.916,.816)--cycle;
\filldraw[fill opacity=0.8,fill=gray!20](-7.929,2.835)--(-7.932,2.892)--(-8.04,2.887)--(-8.04,2.83)--cycle;
\filldraw[fill opacity=0.8,fill=gray!20,draw=none](-7.714,3.24)--(-7.529,3.589)--(-7.606,3.616)--(-7.744,3.25)--cycle;
\draw(-7.714,3.24)--(-7.529,3.589)--(-7.606,3.616)--(-7.744,3.25);
\filldraw[fill opacity=0.8,fill=gray!20](-7.579,4.161)--(-7.577,4.213)--(-7.653,4.219)--(-7.673,4.168)--cycle;
\filldraw[fill opacity=0.8,fill=gray!20](-7.581,3.994)--(-7.581,4.048)--(-7.689,4.056)--(-7.685,4.002)--cycle;
\filldraw[fill opacity=0.5,fill=gray!20](-10.464,2.253)--(-10.536,2.289)--(-10.154,2.651)--(-10.084,2.612)--cycle;
\filldraw[fill opacity=0.8,fill=gray!20](-7.483,4.166)--(-7.498,4.217)--(-7.577,4.213)--(-7.579,4.161)--cycle;
\filldraw[fill opacity=0.8,fill=gray!20](-7.473,3.999)--(-7.47,4.053)--(-7.581,4.048)--(-7.581,3.994)--cycle;
\filldraw[fill opacity=0.8,fill=gray!20](-7.581,4.105)--(-7.579,4.161)--(-7.673,4.168)--(-7.685,4.113)--cycle;
\filldraw[fill opacity=0.8,fill=gray!20](-7.581,4.048)--(-7.581,4.105)--(-7.685,4.113)--(-7.689,4.056)--cycle;
\filldraw[fill opacity=0.5,fill=gray!20](-11.014,1.334)--(-11.065,1.359)--(-10.882,1.862)--(-10.829,1.84)--cycle;
\filldraw[fill opacity=0.8,fill=gray!20](-7.719,3.24)--(-7.627,3.207)--(-7.406,3.794)--cycle;
\filldraw[fill opacity=0.8,fill=gray!20](-7.473,4.11)--(-7.483,4.166)--(-7.579,4.161)--(-7.581,4.105)--cycle;
\filldraw[fill opacity=0.8,fill=gray!20,draw=none](-7.994,3.345)--(-7.801,3.689)--(-7.89,3.724)--(-8.028,3.358)--cycle;
\draw(-7.994,3.345)--(-7.801,3.689)--(-7.89,3.724)--(-8.028,3.358);
\filldraw[fill opacity=0.8,fill=gray!20](-7.47,4.053)--(-7.473,4.11)--(-7.581,4.105)--(-7.581,4.048)--cycle;
\filldraw[fill opacity=0.5,fill=gray!20](-10.843,.256)--(-10.939,.268)--(-11.001,.779)--(-10.905,.757)--cycle;
\filldraw[fill opacity=0.8,fill=gray!20](-7.406,3.794)--(-7.498,3.826)--(-7.719,3.24)--cycle;
\filldraw[fill opacity=0.5,fill=gray!20](-10.439,-.157)--(-10.557,-.185)--(-10.731,.253)--(-10.605,.262)--cycle;
\filldraw[fill opacity=0.8,fill=gray!20,draw=none](-7.994,3.345)--(-7.939,3.323)--(-7.801,3.689)--cycle;
\draw(-7.939,3.323)--(-7.801,3.689)--(-7.994,3.345);
\filldraw[fill opacity=0.8,fill=gray!20,draw=none](-7.803,3.272)--(-7.744,3.25)--(-7.606,3.616)--cycle;
\draw(-7.744,3.25)--(-7.606,3.616)--(-7.803,3.272);
\filldraw[fill opacity=0.5,fill=gray!20](-9.92,2.478)--(-10.005,2.554)--(-9.572,2.797)--(-9.5,2.713)--cycle;
\filldraw[fill opacity=0.5,fill=gray!20](-10.279,-.039)--(-10.439,-.157)--(-10.605,.262)--(-10.426,.334)--cycle;
\filldraw[fill opacity=0.5,fill=gray!20](-11.001,.779)--(-11.077,.806)--(-11.014,1.334)--(-10.939,1.304)--cycle;
\filldraw[fill opacity=0.8,fill=gray!20,draw=none](-7.803,3.272)--(-7.606,3.616)--(-7.701,3.651)--(-7.839,3.285)--cycle;
\draw(-7.803,3.272)--(-7.606,3.616)--(-7.701,3.651)--(-7.839,3.285);
\filldraw[fill opacity=0.8,fill=gray!20](-8.06,3.369)--(-7.954,3.327)--(-7.733,3.914)--cycle;
\filldraw[fill opacity=0.8,fill=gray!20](-7.733,3.914)--(-7.839,3.956)--(-8.06,3.369)--cycle;
\filldraw[fill opacity=0.8,fill=gray!20,draw=none](-7.901,3.309)--(-7.701,3.651)--(-7.801,3.689)--(-7.939,3.323)--cycle;
\draw(-7.901,3.309)--(-7.701,3.651)--(-7.801,3.689)--(-7.939,3.323);
\filldraw[fill opacity=0.8,fill=gray!20,draw=none](-7.901,3.309)--(-7.839,3.285)--(-7.701,3.651)--cycle;
\draw(-7.839,3.285)--(-7.701,3.651)--(-7.901,3.309);
\filldraw[fill opacity=0.8,fill=gray!20](-7.834,3.282)--(-7.719,3.24)--(-7.498,3.826)--cycle;
\filldraw[fill opacity=0.5,fill=gray!20](-10.756,1.807)--(-10.829,1.84)--(-10.536,2.289)--(-10.464,2.253)--cycle;
\filldraw[fill opacity=0.8,fill=gray!20](-7.498,3.826)--(-7.613,3.868)--(-7.834,3.282)--cycle;
\filldraw[fill opacity=0.5,fill=gray!20](-10.378,2.202)--(-10.464,2.253)--(-10.084,2.612)--(-10.005,2.554)--cycle;
\filldraw[fill opacity=0.8,fill=gray!20](-7.954,3.327)--(-7.834,3.282)--(-7.613,3.868)--cycle;
\filldraw[fill opacity=0.8,fill=gray!20](-7.613,3.868)--(-7.733,3.914)--(-7.954,3.327)--cycle;
\filldraw[fill opacity=0.5,fill=gray!20](-10.731,.253)--(-10.843,.256)--(-10.905,.757)--(-10.79,.738)--cycle;
\filldraw[fill opacity=0.5,fill=gray!20](-9.831,2.386)--(-9.92,2.478)--(-9.5,2.713)--(-9.43,2.611)--cycle;
\filldraw[fill opacity=0.5,fill=gray!20](-9.738,2.221)--(-9.831,2.386)--(-9.43,2.611)--(-9.382,2.421)--cycle;
\filldraw[fill opacity=0.5,fill=gray!20](-10.939,1.304)--(-11.014,1.334)--(-10.829,1.84)--(-10.756,1.807)--cycle;
\filldraw[fill opacity=0.5,fill=gray!20](-10.905,.757)--(-11.001,.779)--(-10.939,1.304)--(-10.843,1.271)--cycle;
\filldraw[fill opacity=0.5,fill=gray!20](-10.426,.334)--(-10.605,.262)--(-10.661,.725)--(-10.477,.745)--cycle;
\filldraw[fill opacity=0.5,fill=gray!20](-10.28,2.136)--(-10.378,2.202)--(-10.005,2.554)--(-9.92,2.478)--cycle;
\filldraw[fill opacity=0.5,fill=gray!20](-10.605,.262)--(-10.731,.253)--(-10.79,.738)--(-10.661,.725)--cycle;
\filldraw[fill opacity=0.5,fill=gray!20](-10.664,1.764)--(-10.756,1.807)--(-10.464,2.253)--(-10.378,2.202)--cycle;
\filldraw[fill opacity=0.5,fill=gray!20](-10.843,1.271)--(-10.939,1.304)--(-10.756,1.807)--(-10.664,1.764)--cycle;
\filldraw[fill opacity=0.5,fill=gray!20](-10.79,.738)--(-10.905,.757)--(-10.843,1.271)--(-10.731,1.236)--cycle;
\filldraw[fill opacity=0.5,fill=gray!20](-10.044,1.932)--(-10.175,2.06)--(-9.831,2.386)--(-9.738,2.221)--cycle;
\filldraw[fill opacity=0.5,fill=gray!20](-10.175,2.06)--(-10.28,2.136)--(-9.92,2.478)--(-9.831,2.386)--cycle;
\filldraw[fill opacity=0.5,fill=gray!20](-10.477,.745)--(-10.661,.725)--(-10.605,1.201)--(-10.426,1.168)--cycle;
\filldraw[fill opacity=0.5,fill=gray!20](-10.557,1.713)--(-10.664,1.764)--(-10.378,2.202)--(-10.28,2.136)--cycle;
\filldraw[fill opacity=0.5,fill=gray!20](-10.661,.725)--(-10.79,.738)--(-10.731,1.236)--(-10.605,1.201)--cycle;
\filldraw[fill opacity=0.5,fill=gray!20](-10.731,1.236)--(-10.843,1.271)--(-10.664,1.764)--(-10.557,1.713)--cycle;
\filldraw[fill opacity=0.5,fill=gray!20](-10.279,1.573)--(-10.439,1.656)--(-10.175,2.06)--(-10.044,1.932)--cycle;
\filldraw[fill opacity=0.5,fill=gray!20](-10.426,1.168)--(-10.605,1.201)--(-10.439,1.656)--(-10.279,1.573)--cycle;
\filldraw[fill opacity=0.5,fill=gray!20](-10.439,1.656)--(-10.557,1.713)--(-10.28,2.136)--(-10.175,2.06)--cycle;
\filldraw[fill opacity=0.5,fill=gray!20](-10.605,1.201)--(-10.731,1.236)--(-10.557,1.713)--(-10.439,1.656)--cycle;
\end{tikzpicture}% End sketch output

\caption{Geometry}
\label{geometry}
\end{center}
\end{figure}


\section{System Description}
The properties of the bodies are given in Tables~\ref{bodycoords}~and~\ref{bodyinertia}.
The properties of the connecctions are given in Tables~\ref{pointcoords},~\ref{linecoords},~and~\ref{connectcoords}.
\begin{table}[ht]
\begin{center}
\begin{threeparttable}
\begin{footnotesize}
\caption{Body CG Locations and Mass}
\label{bodycoords}
\pgfplotstabletypeset[%
every head row/.style={output empty row,
before row={\toprule No. & Body Name &
\multicolumn{6}{c}{Location [m]} &
\multicolumn{2}{c}{Mass [kg]}\\},
after row=\midrule},
columns={num,name,rx,ry,rz,mass}]{bodydata.out}
\end{footnotesize}
\end{threeparttable}
\end{center}
\end{table}
\begin{table}[ht]
\begin{center}
\begin{threeparttable}
\begin{footnotesize}
\caption{Body Inertia Properties}
\label{bodyinertia}
\pgfplotstabletypeset[%
every head row/.style={output empty row, before row={\toprule No. & Body Name &
\multicolumn{12}{c}{Inertia [kg$\cdot$m$^2$] ($I_{\textrm{xx}}$, $I_{\textrm{yy}}$, $I_{\textrm{zz}}$;  $I_{\textrm{xy}}$, $I_{\textrm{yz}}$, $I_{\textrm{zx}}$)}\\},
after row=\midrule}, columns={num,name,ixx,iyy,izz,ixy,iyz,ixz}]{bodydata.out}
\begin{tablenotes}
\item Note: inertias are defined as the positive integral over the body, e.g., $I_{\textrm{xy}}=+\! \int \! r_{\textrm{x}}r_{\textrm{y}} \,\text{d}m $.
\end{tablenotes}
\end{footnotesize}
\end{threeparttable}
\end{center}
\end{table}
\begin{table}[ht]
\begin{center}
\begin{threeparttable}
\begin{footnotesize}
\caption{Connection Location and Direction}
\label{pointcoords}
\pgfplotstabletypeset[%
every head row/.style={output empty row, before row={\toprule No. & Connection Name &
\multicolumn{6}{c}{Location [m]} & \multicolumn{6}{c}{Unit Axis}\\},
after row=\midrule}, columns={num,name,rx,ry,rz,ux,uy,uz}]{pointdata.out}
\end{footnotesize}
\end{threeparttable}
\end{center}
\end{table}
\begin{table}[ht]
\begin{center}
\begin{threeparttable}
\begin{footnotesize}
\caption{Connection Locations}
\label{linecoords}
\pgfplotstabletypeset[%
every head row/.style={output empty row, before row={\toprule No. & Connection Name &
\multicolumn{6}{c}{Location [m]} & \multicolumn{6}{c}{Location [m]}\\},
after row=\midrule}, columns={num,name,rx,ry,rz,ux,uy,uz}]{linedata.out}
\end{footnotesize}
\end{threeparttable}
\end{center}
\end{table}
\begin{table}[ht]
\begin{center}
\begin{threeparttable}
\begin{footnotesize}
\caption{Connection Properties}
\label{connectcoords}
\pgfplotstabletypeset[%
every head row/.style={output empty row, before row={\toprule No. & Connection Name &
\multicolumn{1}{c}{Stiffness [N/m]} & \multicolumn{1}{c}{Damping [Ns/m]}\\},
after row=\midrule}, columns={num,name,stiffness,damping}]{stiffnessdata.out}
\end{footnotesize}
\end{threeparttable}
\end{center}
\end{table}
