\chapter{Equations of Motion}
The equations of motion are of the form
\[
\begin{bmatrix}
{\mathbf I} & {\bm 0} & {\bm 0} \\ {\bm 0} & {\mathbf M} & -{\mathbf G} \\ {\bm 0} & {\bm 0} & {\bm 0}
\end{bmatrix}
\begin{Bmatrix}
\dot{\bm p}\\ \dot{\bm w} \\ \dot{\bm u}
\end{Bmatrix}
+\begin{bmatrix}
{\mathbf V} & -{\mathbf I} & {\bm 0} \\ {\mathbf K} & {\mathbf L} & -{\mathbf F} \\ {\bm 0} & {\bm 0} & {\mathbf I}
\end{bmatrix}
\begin{Bmatrix}
{\bm p}\\ {\bm w}\\ {\bm u}
\end{Bmatrix}
=\begin{bmatrix}
{\bm 0}\\ {\bm 0}\\ {\mathbf I}
\end{bmatrix}
\begin{Bmatrix}
{\bm u}
\end{Bmatrix}
\]
The mass matrix of the system is
\begin{footnotesize}
\begin{center}
\pgfplotstabletypeset[
begin table=\begin{longtable},
end table=\end{longtable},
columns={[index]0,[index]1,[index]2,[index]0,[index]1,[index]2},
display columns/0/.style={select equal part entry of={0}{2}},
display columns/1/.style={select equal part entry of={0}{2}},
display columns/2/.style={select equal part entry of={0}{2}},
display columns/3/.style={select equal part entry of={1}{2}},
display columns/4/.style={select equal part entry of={1}{2}},
display columns/5/.style={select equal part entry of={1}{2}},
columns/0/.style={int detect, column type=c,column name={Row}},
columns/1/.style={int detect, column type=c,column name={Column}},
columns/2/.style={sci,sci zerofill,precision=8,dec sep align,column name={Value}}]
{unformatted/mass_matrix.out}
\end{center}
\end{footnotesize}
The damping matrix is
\begin{footnotesize}
\begin{center}
\pgfplotstabletypeset[
begin table=\begin{longtable},
end table=\end{longtable},
columns={[index]0,[index]1,[index]2,[index]0,[index]1,[index]2},
display columns/0/.style={select equal part entry of={0}{2}},
display columns/1/.style={select equal part entry of={0}{2}},
display columns/2/.style={select equal part entry of={0}{2}},
display columns/3/.style={select equal part entry of={1}{2}},
display columns/4/.style={select equal part entry of={1}{2}},
display columns/5/.style={select equal part entry of={1}{2}},
columns/0/.style={int detect, column type=c,column name={Row}},
columns/1/.style={int detect, column type=c,column name={Column}},
columns/2/.style={sci,sci zerofill,precision=8,dec sep align,column name={Value}}]
{unformatted/damping_matrix.out}
\end{center}
\end{footnotesize}
The stiffness matrix is
\begin{footnotesize}
\begin{center}
\pgfplotstabletypeset[
begin table=\begin{longtable},
end table=\end{longtable},
columns={[index]0,[index]1,[index]2,[index]0,[index]1,[index]2},
display columns/0/.style={select equal part entry of={0}{2}},
display columns/1/.style={select equal part entry of={0}{2}},
display columns/2/.style={select equal part entry of={0}{2}},
display columns/3/.style={select equal part entry of={1}{2}},
display columns/4/.style={select equal part entry of={1}{2}},
display columns/5/.style={select equal part entry of={1}{2}},
columns/0/.style={int detect, column type=c,column name={Row}},
columns/1/.style={int detect, column type=c,column name={Column}},
columns/2/.style={sci,sci zerofill,precision=8,dec sep align,column name={Value}}]
{unformatted/stiffness_matrix.out}
\end{center}
\end{footnotesize}
The velocity matrix is
\begin{footnotesize}
\begin{center}
\pgfplotstabletypeset[
begin table=\begin{longtable},
end table=\end{longtable},
columns={[index]0,[index]1,[index]2,[index]0,[index]1,[index]2},
display columns/0/.style={select equal part entry of={0}{2}},
display columns/1/.style={select equal part entry of={0}{2}},
display columns/2/.style={select equal part entry of={0}{2}},
display columns/3/.style={select equal part entry of={1}{2}},
display columns/4/.style={select equal part entry of={1}{2}},
display columns/5/.style={select equal part entry of={1}{2}},
columns/0/.style={int detect, column type=c,column name={Row}},
columns/1/.style={int detect, column type=c,column name={Column}},
columns/2/.style={sci,sci zerofill,precision=8,dec sep align,column name={Value}}]
{unformatted/velocity_matrix.out}
\end{center}
\end{footnotesize}
The input force matrix is
\begin{footnotesize}
\begin{center}
\pgfplotstabletypeset[
begin table=\begin{longtable},
end table=\end{longtable},
columns={[index]0,[index]1,[index]2,[index]0,[index]1,[index]2},
display columns/0/.style={select equal part entry of={0}{2}},
display columns/1/.style={select equal part entry of={0}{2}},
display columns/2/.style={select equal part entry of={0}{2}},
display columns/3/.style={select equal part entry of={1}{2}},
display columns/4/.style={select equal part entry of={1}{2}},
display columns/5/.style={select equal part entry of={1}{2}},
columns/0/.style={int detect, column type=c,column name={Row}},
columns/1/.style={int detect, column type=c,column name={Column}},
columns/2/.style={sci,sci zerofill,precision=8,dec sep align,column name={Value}}]
{unformatted/forcing_matrix.out}
\end{center}
\end{footnotesize}
The input force rate matrix is
\begin{footnotesize}
\begin{center}
\pgfplotstabletypeset[
begin table=\begin{longtable},
end table=\end{longtable},
columns={[index]0,[index]1,[index]2,[index]0,[index]1,[index]2},
display columns/0/.style={select equal part entry of={0}{2}},
display columns/1/.style={select equal part entry of={0}{2}},
display columns/2/.style={select equal part entry of={0}{2}},
display columns/3/.style={select equal part entry of={1}{2}},
display columns/4/.style={select equal part entry of={1}{2}},
display columns/5/.style={select equal part entry of={1}{2}},
columns/0/.style={int detect, column type=c,column name={Row}},
columns/1/.style={int detect, column type=c,column name={Column}},
columns/2/.style={sci,sci zerofill,precision=8,dec sep align,column name={Value}}]
{unformatted/forcing_rate_matrix.out}
\end{center}
\end{footnotesize}
The system is subject to constraints
\[
\begin{bmatrix}
\mathbf{J}_\textrm{h} & {\bm 0} & {\bm 0}\\-{\mathbf J}_\textrm{h}{\mathbf V} & {\mathbf J}_\textrm{h} & {\bm 0}\\ {\bm 0} & {\mathbf J}_\textrm{nh} & {\bm 0}
\end{bmatrix}
\begin{bmatrix}
\dot{\bm p} & {\bm p} \\ \dot{\bm w} & {\bm w} \\ \dot{\bm u} & {\bm u}
\end{bmatrix}
=\begin{bmatrix}
{\bm 0} & {\bm 0}\\{\bm 0} & {\bm 0}\\{\bm 0} & {\bm 0}
\end{bmatrix}
\]
\begin{footnotesize}
\begin{center}
\pgfplotstabletypeset[
begin table=\begin{longtable},
end table=\end{longtable},
columns={[index]0,[index]1,[index]2,[index]0,[index]1,[index]2},
display columns/0/.style={select equal part entry of={0}{2}},
display columns/1/.style={select equal part entry of={0}{2}},
display columns/2/.style={select equal part entry of={0}{2}},
display columns/3/.style={select equal part entry of={1}{2}},
display columns/4/.style={select equal part entry of={1}{2}},
display columns/5/.style={select equal part entry of={1}{2}},
columns/0/.style={int detect, column type=c,column name={Row}},
columns/1/.style={int detect, column type=c,column name={Column}},
columns/2/.style={sci,sci zerofill,precision=8,dec sep align,column name={Value}}]
{unformatted/jacobian_matrix.out}
\end{center}
\end{footnotesize}
\noindent The full state space equations:
\[
\begin{bmatrix}
{\mathbf E} & {\bm 0} \\ {\bm 0} & {\mathbf I}
\end{bmatrix}
\begin{Bmatrix}
\dot{\bm x}\\ {\bm y}
\end{Bmatrix}
=\begin{bmatrix}
\mathbf{A} & \mathbf{B} \\ \mathbf{C} & \mathbf{D}
\end{bmatrix}
\begin{Bmatrix}
{\bm x}\\ {\bm u}\end{Bmatrix}
\]
\pgfplotstabletypeset[
every head row/.style={output empty row},
every last row/.style={after row={}},
every row no 22/.style={after row=\midrule},
columns/0/.style={sci,sci zerofill,precision=8,dec sep align},
columns/1/.style={sci,sci zerofill,precision=8,dec sep align},
columns/2/.style={sci,sci zerofill,precision=8,dec sep align},
columns/3/.style={sci,sci zerofill,precision=8,dec sep align},
columns/4/.style={sci,sci zerofill,precision=8,dec sep align},
columns/5/.style={sci,sci zerofill,precision=8,dec sep align},
columns/6/.style={sci,sci zerofill,precision=8,dec sep align},
columns/7/.style={sci,sci zerofill,precision=8,dec sep align},
columns/8/.style={sci,sci zerofill,precision=8,dec sep align},
columns/9/.style={sci,sci zerofill,precision=8,dec sep align},
columns/10/.style={sci,sci zerofill,precision=8,dec sep align},
columns/11/.style={sci,sci zerofill,precision=8,dec sep align},
columns/12/.style={sci,sci zerofill,precision=8,dec sep align},
columns/13/.style={sci,sci zerofill,precision=8,dec sep align},
columns/14/.style={sci,sci zerofill,precision=8,dec sep align},
columns/15/.style={sci,sci zerofill,precision=8,dec sep align},
columns/16/.style={sci,sci zerofill,precision=8,dec sep align},
columns/17/.style={sci,sci zerofill,precision=8,dec sep align},
columns/18/.style={sci,sci zerofill,precision=8,dec sep align},
columns/19/.style={sci,sci zerofill,precision=8,dec sep align},
columns/20/.style={sci,sci zerofill,precision=8,dec sep align},
columns/21/.style={sci,sci zerofill,precision=8,dec sep align},
columns/22/.style={sci,sci zerofill,precision=8,dec sep align,column type/.add={}{|}},
columns/23/.style={sci,sci zerofill,precision=8,dec sep align},
write to macro=\ABCD,
typeset=false]
{unformatted/ABCD.out}
\pgfplotstabletypeset[
every head row/.style={output empty row},
every last row/.style={after row={}},
columns/0/.style={sci,sci zerofill,precision=8,dec sep align},
columns/1/.style={sci,sci zerofill,precision=8,dec sep align},
columns/2/.style={sci,sci zerofill,precision=8,dec sep align},
columns/3/.style={sci,sci zerofill,precision=8,dec sep align},
columns/4/.style={sci,sci zerofill,precision=8,dec sep align},
columns/5/.style={sci,sci zerofill,precision=8,dec sep align},
columns/6/.style={sci,sci zerofill,precision=8,dec sep align},
columns/7/.style={sci,sci zerofill,precision=8,dec sep align},
columns/8/.style={sci,sci zerofill,precision=8,dec sep align},
columns/9/.style={sci,sci zerofill,precision=8,dec sep align},
columns/10/.style={sci,sci zerofill,precision=8,dec sep align},
columns/11/.style={sci,sci zerofill,precision=8,dec sep align},
columns/12/.style={sci,sci zerofill,precision=8,dec sep align},
columns/13/.style={sci,sci zerofill,precision=8,dec sep align},
columns/14/.style={sci,sci zerofill,precision=8,dec sep align},
columns/15/.style={sci,sci zerofill,precision=8,dec sep align},
columns/16/.style={sci,sci zerofill,precision=8,dec sep align},
columns/17/.style={sci,sci zerofill,precision=8,dec sep align},
columns/18/.style={sci,sci zerofill,precision=8,dec sep align},
columns/19/.style={sci,sci zerofill,precision=8,dec sep align},
columns/20/.style={sci,sci zerofill,precision=8,dec sep align},
columns/21/.style={sci,sci zerofill,precision=8,dec sep align},
columns/22/.style={sci,sci zerofill,precision=8,dec sep align},
columns/23/.style={sci,sci zerofill,precision=8,dec sep align},
write to macro=\E,
typeset=false]
{unformatted/E.out}
\begin{tiny}
\[
\begin{bmatrix}
{\mathbf A} & {\mathbf B} \\ {\mathbf C} & {\mathbf D}
\end{bmatrix}=
\left[
\ABCD
\right]
\]
\[
{\mathbf E}=
\left[
\E
\right]
\]
\end{tiny}
\noindent The reduced state space equations:
\pgfplotstabletypeset[
every head row/.style={output empty row},
every last row/.style={after row={}},
every row no 5/.style={after row=\midrule},
columns/0/.style={sci,sci zerofill,precision=8,dec sep align},
columns/1/.style={sci,sci zerofill,precision=8,dec sep align},
columns/2/.style={sci,sci zerofill,precision=8,dec sep align},
columns/3/.style={sci,sci zerofill,precision=8,dec sep align},
columns/4/.style={sci,sci zerofill,precision=8,dec sep align},
columns/5/.style={sci,sci zerofill,precision=8,dec sep align,column type/.add={}{|}},
columns/6/.style={sci,sci zerofill,precision=8,dec sep align},
write to macro=\ABCD,
typeset=false]
{unformatted/ABCDmin.out}
\pgfplotstabletypeset[
every head row/.style={output empty row},
every last row/.style={after row={}},
columns/0/.style={sci,sci zerofill,precision=8,dec sep align},
columns/1/.style={sci,sci zerofill,precision=8,dec sep align},
columns/2/.style={sci,sci zerofill,precision=8,dec sep align},
columns/3/.style={sci,sci zerofill,precision=8,dec sep align},
columns/4/.style={sci,sci zerofill,precision=8,dec sep align},
columns/5/.style={sci,sci zerofill,precision=8,dec sep align},
columns/6/.style={sci,sci zerofill,precision=8,dec sep align},
write to macro=\E,
typeset=false]
{unformatted/E.out}
\begin{tiny}
\[
\begin{bmatrix}
{\mathbf A} & {\mathbf B} \\ {\mathbf C} & {\mathbf D}
\end{bmatrix}=
\left[
\ABCD
\right]
\]
\end{tiny}
