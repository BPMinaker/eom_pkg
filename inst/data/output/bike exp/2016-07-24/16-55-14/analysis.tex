\chapter{Analysis}
Replace this text with the body of your report.  Add sections or subsections as appropriate.
\section{Eigenvalue Analysis}
The eigenvalue properties are given in Tables~\ref{evals}~and~\ref{evals-b}.

\begin{table}[ht]
\begin{center}
\begin{threeparttable}
\begin{footnotesize}
\caption{Eigenvalues}
\label{evals}
\pgfplotstabletypeset[columns={num,real,imag,realhz,imaghz}]{eigen.out}
\begin{tablenotes}
\item Note: oscillatory roots appear as complex conjugates.
\end{tablenotes}
\end{footnotesize}
\end{threeparttable}
\end{center}
\end{table}
\begin{table}[ht]
\begin{center}
\begin{threeparttable}
\begin{footnotesize}
\caption{Eigenvalue Analysis}
\label{evals-b}
\pgfplotstabletypeset[columns={num,nfreq,zeta,tau,lambda}]{freq.out}
\begin{tablenotes}
\item Notes: a) oscillatory roots are listed twice, b) negative time constants denote unstable roots.
\end{tablenotes}
\end{footnotesize}
\end{threeparttable}
\end{center}
\end{table}
There are 11 degrees of freedom.\\

There are 6 oscillatory modes, 13 damped modes, 3 unstable modes, and 0 rigid body modes.
\pagebreak
\section{Frequency Response Plots}
\begin{figure}[hbtp]
\begin{center}
\begin{footnotesize}
\begin{tikzpicture}
\begin{semilogxaxis}[height=2in,width=5in,tick style={thin,black},extra y ticks={0},extra y tick style={grid=major},major y grid style={dotted,black},xlabel={Frequency [Hz]},ylabel={Transfer Function [dB]},enlarge x limits=false,legend style={at={(1.0,1.03)},anchor=south east},legend columns=1,legend cell align=left,cycle list name=linestyles*]
\addplot+[black,line width=1pt,mark=none] table[x=frequency,y=m1]{bode.out};
\addlegendentry{yaw rate/steer torque}
\addplot+[black,line width=1pt,mark=none] table[x=frequency,y=m2]{bode.out};
\addlegendentry{roll angle/steer torque}
\end{semilogxaxis}
\end{tikzpicture}
\begin{tikzpicture}
\begin{semilogxaxis}[height=2in,width=5in,ymin=-180,ymax=180,ytick={-180,-90,0,90,180},tick style={thin,black},extra y ticks={0},extra y tick style={grid=major},major y grid style={dotted,black},xlabel={Frequency [Hz]},ylabel={Phase Angle  [$^\circ$]},enlargelimits=false,cycle list name=linestyles*,restrict y to domain= -180:180,unbounded coords=jump]
\addplot+[black,line width=1pt,mark=none] table[x=frequency,y=p1]{bode.out};
\addplot+[black,line width=1pt,mark=none] table[x=frequency,y=p2]{bode.out};
\end{semilogxaxis}
\end{tikzpicture}
\end{footnotesize}
\caption{Frequency response: steer torque}
\label{bode_plot_1}
\end{center}
\end{figure}

\clearpage

\section{Steady State Gains}
The steady state gains are given in Table~\ref{sstf}.
\begin{table}[ht]
\begin{center}
\begin{threeparttable}
\begin{footnotesize}
\caption{Steady State Gains}
\label{sstf}
\pgfplotstabletypeset{sstf.out}
\end{footnotesize}
\end{threeparttable}
\end{center}
\end{table}
\section{Equilibrium Analysis}
The results of the equlibrium load analysis are given in Tables~\ref{deflns}~and~\ref{preloads}.
\begin{center}
\begin{footnotesize}
\pgfplotstabletypeset[
begin table=\begin{longtable},
end table=\end{longtable},
every head row/.style={output empty row,
before row={\caption{System Static Deflections}\label{deflns}\\
\toprule No. & Body Name & Type &
\multicolumn{6}{c}{Deflection [m] or [rad]}\\},
after row=\midrule}, columns={num,name,type,x,y,z}]{defln.out}
\end{footnotesize}
\end{center}
\begin{center}
\begin{footnotesize}
\pgfplotstabletypeset[
begin table=\begin{longtable},
end table=\end{longtable},
every head row/.style={output empty row,
before row={\caption{System Preloads}\label{preloads}\\
\toprule No. & Connector Name & Type &
\multicolumn{8}{c}{Load [N] or [Nm] (Components; Magnitude)}\\},
after row=\midrule}, columns={num,name,type,fx,fy,fz,fxyz}]{preload.out}
\end{footnotesize}
\end{center}
