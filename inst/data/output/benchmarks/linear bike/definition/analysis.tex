\section*{Analysis}
From the system parameters, the equations of motion can be formed.  If small angle linearizing assumptions are used, they can be cast in linear second form, as follows.

\begin{equation}
{\mathbf M}\ddot{\bm q} + v{\mathbf C}_1\dot{\bm q} + [g{\mathbf K}_0 + v^2{\mathbf K_2}]{\bm q}={\mathbf 0}
\end{equation}

where

\begin{equation}
{\bm q}=\begin{Bmatrix}
\phi\\\delta
\end{Bmatrix}
\end{equation}


\begin{equation}
{\mathbf M}=
\begin{bmatrix}
80.81722 & 2.31941332208709\\
2.31941332208709 & 0.29784188199686
\end{bmatrix}
\end{equation}

\begin{equation}
{\mathbf K}_0=
\begin{bmatrix}
-80.95 & -2.59951685249872\\
-2.59951685249872 & -0.80329488458618
\end{bmatrix}
\end{equation}

\begin{equation}
{\mathbf K_2}=
\begin{bmatrix}
0 & 76.5973459573222\\
0 & 2.65431523794604
\end{bmatrix}
\end{equation}

\begin{equation}
{\mathbf C_1}=
\begin{bmatrix}
0 & 33.86641391492494\\
-0.85035641456978 & 1.68540397397560
\end{bmatrix}
\end{equation}

If an exponential solution is assumed, the solution can be cast as a second order eigenvalue problem.

\begin{equation}
\det [{\mathbf M}s^2 + v{\mathbf C}_1s + g{\mathbf K}_0 + v^2{\mathbf K_2}]=0
\end{equation}

