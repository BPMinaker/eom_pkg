\chapter{Introduction}
This report provides the results of the analysis of the IFToMM linear bicycle benchmark problem using the EoM software produced by the University of Windsor Vehicle Dynamics and Control research group.  The problem consists of four rigid bodies connected by three revolute joints, and two rolling contacts.  The properties are summarized below.  The problem is modified slightly by defining a steering torque between the fork and frame as the system input and the steer and lean angles as the system output.  Note that there may be some round-off in the data describing the system properties in this document, but that the full precision values were used to define the system.  Please see the problem definition document for the more precise values.

\section{System Description}
The properties of the bodies are given in Tables~\ref{bodycoords}~and~\ref{bodyinertia}.
The properties of the connecctions are given in Table~\ref{pointcoords}.
\begin{table}[ht]
\begin{center}
\begin{threeparttable}
\begin{footnotesize}
\caption{Body CG Locations and Mass}
\label{bodycoords}
\pgfplotstabletypeset[%
every head row/.style={output empty row,
before row={\toprule No. & Body Name &
\multicolumn{6}{c}{Location [m]} &
\multicolumn{2}{c}{Mass [kg]}\\},
after row=\midrule},
columns={num,name,rx,ry,rz,mass}]{bodydata.out}
\end{footnotesize}
\end{threeparttable}
\end{center}
\end{table}
\begin{table}[ht]
\begin{center}
\begin{threeparttable}
\begin{footnotesize}
\caption{Body Inertia Properties}
\label{bodyinertia}
\pgfplotstabletypeset[%
every head row/.style={output empty row, before row={\toprule No. & Body Name &
\multicolumn{12}{c}{Inertia [kg$\cdot$m$^2$] ($I_{\textrm{xx}}$, $I_{\textrm{yy}}$, $I_{\textrm{zz}}$;  $I_{\textrm{xy}}$, $I_{\textrm{yz}}$, $I_{\textrm{zx}}$)}\\},
after row=\midrule}, columns={num,name,ixx,iyy,izz,ixy,iyz,ixz}]{bodydata.out}
\begin{tablenotes}
\item Note: inertias are defined as the positive integral over the body, e.g., $I_{\textrm{xy}}=+\! \int \! r_{\textrm{x}}r_{\textrm{y}} \,\text{d}m $.
\end{tablenotes}
\end{footnotesize}
\end{threeparttable}
\end{center}
\end{table}
\begin{table}[ht]
\begin{center}
\begin{threeparttable}
\begin{footnotesize}
\caption{Connection Location and Direction}
\label{pointcoords}
\pgfplotstabletypeset[%
every head row/.style={output empty row, before row={\toprule No. & Connection Name &
\multicolumn{6}{c}{Location [m]} & \multicolumn{6}{c}{Unit Axis}\\},
after row=\midrule}, columns={num,name,rx,ry,rz,ux,uy,uz}]{pointdata.out}
\end{footnotesize}
\end{threeparttable}
\end{center}
\end{table}

