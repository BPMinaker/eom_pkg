\chapter{Equations of Motion}
The equations of motion are prepared in first order form.
\[
\begin{bmatrix}
{\mathbf I} & {\bm 0} & {\bm 0} \\ {\bm 0} & {\mathbf M} & -{\mathbf G} \\ {\bm 0} & {\bm 0} & {\bm 0}
\end{bmatrix}
\begin{Bmatrix}
\dot{\bm p}\\ \dot{\bm w} \\ \dot{\bm u}
\end{Bmatrix}
+\begin{bmatrix}
{\mathbf V} & -{\mathbf I} & {\bm 0} \\ {\mathbf K} & {\mathbf L} & -{\mathbf F} \\ {\bm 0} & {\bm 0} & {\mathbf I}
\end{bmatrix}
\begin{Bmatrix}
{\bm p}\\ {\bm w}\\ {\bm u}
\end{Bmatrix}
=\begin{bmatrix}
{\bm 0}\\ {\bm 0}\\ {\mathbf I}
\end{bmatrix}
\begin{Bmatrix}
{\bm u}
\end{Bmatrix}
\]

The state vector consists of six global position coordinates for each body~${\bm p}$, and six body fixed velocity coordinates~${\bm w}$ for each body.  The input vector~${\bm u}$ is appended to the state.  The first row of the system are the linearized kinematic differential equations.  The matrix ${\mathbf V}$ is determined only by the velocity around which the linearization occurs.  The second row of the system are the linearized Newton Euler equations.  The matrix ${\mathbf M}$ represents the mass and inertia terms, the matrix ${\mathbf G}$ allows the inclusion of systems that have dependency on the rate of input. The ${\mathbf K}$ is the stiffness matrix, and includes terms that depend on the physical stiffnes, the geometry, and the static preload carried in the connections.  The matrix ${\mathbf L}$ is the damping matrix, and depends on the physical damping coefficients, the geometry, and the velocity of linearization, in order to include the centripetal and gyroscopic terms.  The ${\mathbf F}$ matrix represents the senitvity to the various inputs.  The bottom row appends the inputs to the state vector.

The system is subject to constraints.

\[
\begin{bmatrix}
\mathbf{J}_\textrm{h} & {\bm 0} & {\bm 0}\\-{\mathbf J}_\textrm{h}{\mathbf V} & {\mathbf J}_\textrm{h} & {\bm 0}\\ {\bm 0} & {\mathbf J}_\textrm{nh} & {\bm 0}
\end{bmatrix}
\begin{bmatrix}
\dot{\bm p} & {\bm p} \\ \dot{\bm w} & {\bm w} \\ \dot{\bm u} & {\bm u}
\end{bmatrix}
=\begin{bmatrix}
{\bm 0} & {\bm 0}\\{\bm 0} & {\bm 0}\\{\bm 0} & {\bm 0}
\end{bmatrix}
\]

The first row of the system are the holonomic contraints, applied to position and rate of change of position.  The $\mathbf{J}_\text{h}$ matrix is the constraint Jacobian.  The second row of the system are also the holonomic contraints, applied to velocity and rate of change of velocity.  There is redundancy between the first entry in the first row and the second entry in the second row.  The third row of the system are the nonholonomic contraints, applied to velocity and rate of change of velocity.

A null space of the constraint matrix is found, and used to reduce the system to a smaller set of coordinates, in the vector ${\bm x}$, and giving the matrices ${\mathbf E}$, ${\mathbf A}$, and ${\mathbf B}$.  The result is combined with a set of output equations, also transformed to minimal coordinates, to give ${\mathbf C}$ and ${\mathbf D}$.  Note that the resulting system may still define a set of DAEs, depending on the condition of the resulting ${\mathbf E}$ matrix.  If ${\mathbf E}$ is singular, then the system of equations can be further reduced to a minimal realization.






The full state space equations:
\[
\begin{bmatrix}
{\mathbf E} & {\bm 0} \\ {\bm 0} & {\mathbf I}
\end{bmatrix}
\begin{Bmatrix}
\dot{\bm x}\\ {\bm y}
\end{Bmatrix}
=\begin{bmatrix}
\mathbf{A} & \mathbf{B} \\ \mathbf{C} & \mathbf{D}
\end{bmatrix}
\begin{Bmatrix}
{\bm x}\\ {\bm u}\end{Bmatrix}
\]

