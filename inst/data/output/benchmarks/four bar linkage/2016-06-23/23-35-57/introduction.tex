\chapter{Introduction}
This report provides the results of the analysis of the IFToMM four bar linkage benchmark problem using the EoM software produced by the University of Windsor Vehicle Dynamics and Control research group.  The problem consists of three uniform slender rods, of unit mass and length, with both gravity and a diagonal spring acting.  The properties are summarized below.  The problem is modified slightly by defining a torque on body three as the system input and the angular motion of body three as the system output.  Note that there may be some round-off in the data describing the system properties.  Please see the problem definition document for more precise values.

\section{System Description}
The properties of the bodies are given in Tables~\ref{bodycoords}~and~\ref{bodyinertia}.
The properties of the connecctions are given in Tables~\ref{pointcoords},~\ref{linecoords},~and~\ref{connectcoords}.
\begin{table}[ht]
\begin{center}
\begin{threeparttable}
\begin{footnotesize}
\caption{Body CG Locations and Mass}
\label{bodycoords}
\pgfplotstabletypeset[%
every head row/.style={output empty row,
before row={\toprule No. & Body Name &
\multicolumn{6}{c}{Location [m]} &
\multicolumn{2}{c}{Mass [kg]}\\},
after row=\midrule},
columns={num,name,rx,ry,rz,mass}]{bodydata.out}
\end{footnotesize}
\end{threeparttable}
\end{center}
\end{table}
\begin{table}[ht]
\begin{center}
\begin{threeparttable}
\begin{footnotesize}
\caption{Body Inertia Properties}
\label{bodyinertia}
\pgfplotstabletypeset[%
every head row/.style={output empty row, before row={\toprule No. & Body Name &
\multicolumn{12}{c}{Inertia [kg$\cdot$m$^2$] ($I_{\textrm{xx}}$, $I_{\textrm{yy}}$, $I_{\textrm{zz}}$;  $I_{\textrm{xy}}$, $I_{\textrm{yz}}$, $I_{\textrm{zx}}$)}\\},
after row=\midrule}, columns={num,name,ixx,iyy,izz,ixy,iyz,ixz}]{bodydata.out}
\begin{tablenotes}
\item Note: inertias are defined as the positive integral over the body, e.g., $I_{\textrm{xy}}=+\! \int \! r_{\textrm{x}}r_{\textrm{y}} \,\text{d}m $.
\end{tablenotes}
\end{footnotesize}
\end{threeparttable}
\end{center}
\end{table}
\begin{table}[ht]
\begin{center}
\begin{threeparttable}
\begin{footnotesize}
\caption{Connection Location and Direction}
\label{pointcoords}
\pgfplotstabletypeset[%
every head row/.style={output empty row, before row={\toprule No. & Connection Name &
\multicolumn{6}{c}{Location [m]} & \multicolumn{6}{c}{Unit Axis}\\},
after row=\midrule}, columns={num,name,rx,ry,rz,ux,uy,uz}]{pointdata.out}
\end{footnotesize}
\end{threeparttable}
\end{center}
\end{table}
\begin{table}[ht]
\begin{center}
\begin{threeparttable}
\begin{footnotesize}
\caption{Connection Locations}
\label{linecoords}
\pgfplotstabletypeset[%
every head row/.style={output empty row, before row={\toprule No. & Connection Name &
\multicolumn{6}{c}{Location [m]} & \multicolumn{6}{c}{Location [m]}\\},
after row=\midrule}, columns={num,name,rx,ry,rz,ux,uy,uz}]{linedata.out}
\end{footnotesize}
\end{threeparttable}
\end{center}
\end{table}
\begin{table}[ht]
\begin{center}
\begin{threeparttable}
\begin{footnotesize}
\caption{Connection Properties}
\label{connectcoords}
\pgfplotstabletypeset[%
every head row/.style={output empty row, before row={\toprule No. & Connection Name &
\multicolumn{1}{c}{Stiffness [N/m]} & \multicolumn{1}{c}{Damping [Ns/m]}\\},
after row=\midrule}, columns={num,name,stiffness,damping}]{stiffnessdata.out}
\end{footnotesize}
\end{threeparttable}
\end{center}
\end{table}
