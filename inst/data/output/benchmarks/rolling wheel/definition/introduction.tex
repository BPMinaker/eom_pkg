\chapter*{IFToMM Benchmark Problem\\ Wheel on Tipping Table}
\section*{Problem Description}
\setcounter{chapter}{1}
The system considered is made up of two bodies, a rigid wheel that rests on a rigid table, where the table is held in place by a revolute joint at its mass centre.  The table has one degree of freedom, which is a rotation around a horizontal axis.  The wheel also has one degree of freedom; it can roll without slip on the table.  To simulate the no-slip condition, it is constrained to the table in both directions at the point of contact, which is initially the mass centre of the table.  However, the point of contact is allowed to move as the wheel rolls.  The surface on which the wheel rolls is also assumed to pass through the table's centre of mass.  A schematic diagram is shown in Figure~\ref{wheel}.

Both bodies are assumed to have equal mass, in this case chosen as $m=1$~\si{\kg}, the wheel has a radius $r=0.25$~\si{\m} and the table is treated a thin square plate with side length $l=1$~\si{\m}.  The inertias are calculated using standard expressions for thin laminar bodies.

\begin{figure}[hbtp]
\begin{center}

\definecolor{cffffff}{RGB}{255,255,255}


\begin{tikzpicture}[y=0.80pt, x=0.80pt, yscale=-1.000000, xscale=1.000000, inner sep=0pt, outer sep=0pt,>=stealth']
\begin{scope}[shift={(0,-62.36218)}]
  \path[shift={(0,62.36218)},draw=black,line join=miter,line cap=butt,miter
    limit=4.00,line width=1.000pt] (416.2500,810.0000) -- (376.8750,753.7500) ..
    controls (366.5476,738.9966) and (353.4524,738.9966) .. (343.1250,753.7500) --
    (303.7500,810.0000);
  \path[draw=black,fill=cffffff,line join=miter,line cap=round,miter
    limit=4.00,line width=1.000pt] (427.5000,827.3622) .. controls
    (433.1428,827.3622) and (433.1428,844.2456) .. (427.5178,844.2456) --
    (292.4822,844.2456) .. controls (286.8572,844.2456) and (286.8572,827.3637) ..
    (292.5000,827.3622) .. controls (317.4972,827.3552) and (315.0024,827.3622) ..
    (348.7500,827.3622) .. controls (348.7500,816.1122) and (348.7500,816.1122) ..
    (360.0000,816.1122) .. controls (371.2500,816.1122) and (371.2500,816.1122) ..
    (371.2500,827.3622) -- cycle;
  \path[xscale=1.000,yscale=1.000,fill=black] (337.5122,866.7058) node[above
    right] (text3818-0) {$\theta_1$};
  \path[xscale=-1.000,yscale=1.000,draw=black,->,line join=miter,line cap=rect,miter
    limit=4.00,line width=0.800pt]
    (-336.3223,747.6332)arc(270.000:334.616:45.000);
  \path[draw=black,line join=miter,line cap=butt,miter limit=4.00,line
    width=1.000pt] (270.0000,872.3622) -- (450.0000,872.3622);
  \path[xscale=1.000,yscale=1.000,fill=black] (382.5138,838.5819) node[above
    right] (text3818-0-9-1-3) {$m$};
  \path[xscale=1.000,yscale=1.000,fill=black] (388.1390,889.2050) node[above
    right] (text3818-0-9-1-7) {$l$};
  \path[draw=black,line join=miter,line cap=butt,miter limit=4.00,even odd
    rule,line width=0.800pt] (286.8750,844.2372) -- (286.8750,894.8622);
  \path[draw=black,line join=miter,line cap=butt,even odd rule,line width=0.800pt]
    (433.1250,844.2372) -- (433.1250,894.8622);
  \path[draw=black,<-,line join=miter,line cap=rect,miter limit=4.00,line
    width=0.800pt] (331.7073,862.3553)arc(128.956:180.000:45.000);
  \path[draw=black,<-,line join=miter,line cap=butt,miter limit=4.00,even odd
    rule,line width=0.800pt] (286.8750,889.2372) -- (382.5000,889.2372);
  \path[draw=black,<-,line join=miter,line cap=butt,miter limit=4.00,even odd
    rule,line width=0.800pt] (433.1250,889.2372) -- (405.0000,889.2372);
  \path[draw=black,line join=miter,line cap=butt,miter limit=4.00,even odd
    rule,line width=1.000pt] (348.7500,827.3622) -- (371.2500,827.3622);
  \path[draw=black,fill=cffffff,line join=miter,line cap=rect,miter
    limit=4.00,line width=1.000pt] (337.5000,793.5937) circle (0.9525cm);
  \path[draw=black,->,line join=miter,line cap=butt,miter limit=4.00,line
    width=0.800pt] (337.5000,793.5937) -- (320.6250,765.4687);
  \path[xscale=1.000,yscale=1.000,fill=black] (315.0114,787.9587) node[above
    right] (text3818-0-9-1-3-7-1) {$r$};
  \path[draw=black,line join=miter,line cap=butt,miter limit=4.00,line
    width=0.800pt] (331.8750,793.5937) -- (343.1250,793.5937);
  \path[xscale=1.000,yscale=1.000,fill=black] (343.1374,804.8331) node[above
    right] (text3818-0-9-1) {$m$};
  \path[xscale=1.000,yscale=1.000,fill=black] (286.8854,793.5834) node[above
    right] (text3818) {$\theta_2$};
  \path[draw=black,line join=miter,line cap=butt,miter limit=4.00,line
    width=0.800pt] (337.5000,787.9687) -- (337.5000,799.2187);
  \path[xscale=1.000,yscale=-1.000,draw=black,fill=cffffff,dash pattern=on 2.00pt
    off 2.00pt,line join=miter,line cap=rect,miter limit=4.00,line width=1.000pt]
    (354.3750,-827.3807)arc(180.000:270.000:5.625000 and
    5.607)arc(-90.000:0.000:5.625000 and 5.607) -- (360.0000,-827.3807) -- cycle;
  \path[shift={(0,-27.65631)},draw=black,fill=cffffff,line join=miter,line
    cap=rect,miter limit=4.00,line width=1.000pt]
    (354.3750,855.0000)arc(180.000:270.000:5.625000 and
    5.607)arc(-90.000:0.000:5.625000 and 5.607) -- (360.0000,855.0000) -- cycle;
\end{scope}

\end{tikzpicture}


\caption{Wheel on tipping table.}
\label{wheel}
\end{center}
\end{figure}

